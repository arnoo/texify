  
  
    
    \bbook{AMOS}{AMOS}
      
         
      \bchapter{}
      \begin{verse}
${}^{1}Paroles d’Amos, qui fut l’un des éleveurs de troupeaux à Teqoa, – ce qu’il a vu sur Israël au temps d’Ozias, roi de Juda, et de Jéroboam, fils de Josias, roi d’Israël, deux ans avant le tremblement de terre.
      
         
       
${}^{2}Il a dit :
        \\Le Seigneur rugit depuis Sion,
        depuis Jérusalem il donne de la voix ;
        \\les pâturages des bergers sont désolés,
        le sommet du Carmel est desséché.
${}^{3}Ainsi parle le Seigneur :
        \\À cause de trois crimes de Damas, et même de quatre,
        je l’ai décidé sans retour !
        \\Parce qu’ils ont broyé le Galaad
        avec des herses de fer,
${}^{4}j’enverrai un feu dans la maison d’Hazaël,
        et il dévorera les palais de Ben-Hadad ;
${}^{5}je briserai les verrous de Damas ;
        je supprimerai de la Vallée-du-Mal tout habitant,
        \\et de la Maison-d’Éden, celui qui tient le sceptre ;
        et le peuple d’Aram sera déporté à Qir.
        \\Le Seigneur a parlé.
${}^{6}Ainsi parle le Seigneur :
        \\À cause de trois crimes de Gaza, et même de quatre,
        je l’ai décidé sans retour !
        \\Parce qu’ils ont mené en déportation des masses de déportés,
        pour les livrer à Édom,
${}^{7}j’enverrai un feu dans la muraille de Gaza,
        et il dévorera ses palais ;
${}^{8}je supprimerai d’Ashdod tout habitant,
        et d’Ascalon, celui qui tient le sceptre ;
        \\je retournerai ma main contre Eqrone ;
        le reste des Philistins périra.
        \\Le Seigneur Dieu a parlé.
${}^{9}Ainsi parle le Seigneur :
        \\À cause de trois crimes de Tyr, et même de quatre,
        je l’ai décidé sans retour !
        \\Parce qu’ils ont livré à Édom
        des masses de déportés,
        sans garder mémoire de l’alliance entre frères,
${}^{10}j’enverrai un feu dans la muraille de Tyr,
        et il dévorera ses palais.
${}^{11}Ainsi parle le Seigneur :
        \\À cause de trois crimes d’Édom, et même de quatre,
        je l’ai décidé sans retour !
        \\Parce qu’il a poursuivi de l’épée son frère,
        étouffant sa pitié,
        \\et entretenu sans fin sa fureur,
        gardant à jamais sa rancune,
${}^{12}j’enverrai un feu dans Témane,
        et il dévorera les palais de Bosra.
${}^{13}Ainsi parle le Seigneur :
        \\À cause de trois crimes des fils d’Ammone, et même de quatre,
        je l’ai décidé sans retour !
        \\Parce qu’ils ont éventré les femmes enceintes du Galaad,
        pour élargir leur territoire,
${}^{14}j’allumerai un feu dans la muraille de Rabba,
        et il dévorera ses palais,
        \\au cri de guerre, un jour de bataille,
        dans la tempête, un jour d’ouragan ;
${}^{15}son roi partira en déportation,
        lui et ses princes, tous ensemble.
        \\Le Seigneur a parlé.
      
         
      \bchapter{}
${}^{1}Ainsi parle le Seigneur :
        \\À cause de trois crimes de Moab, et même de quatre,
        je l’ai décidé sans retour !
        \\Parce qu’il a brûlé à la chaux vive
        les os du roi d’Édom,
${}^{2}j’enverrai un feu dans Moab,
        et il dévorera les palais de Qeriyoth ;
        \\Moab mourra dans le vacarme,
        au cri de guerre, au son du cor ;
${}^{3}de chez lui je supprimerai le juge,
        et tuerai tous ses princes avec lui.
        \\Le Seigneur a parlé.
        
           
${}^{4}Ainsi parle le Seigneur :
        \\À cause de trois crimes de Juda, et même de quatre,
        je l’ai décidé sans retour !
        \\Parce qu’ils ont rejeté la Loi du Seigneur,
        et n’ont pas gardé ses décrets,
        \\parce que leurs idoles les ont égarés,
        celles que leurs pères avaient suivies,
${}^{5}j’enverrai un feu en Juda,
        et il dévorera les palais de Jérusalem.
        ${}^{6}Ainsi parle le Seigneur :
        \\À cause de trois crimes d’Israël, et même de quatre,
        je l’ai décidé sans retour !
        \\Ils vendent le juste pour de l’argent,
        le malheureux pour une paire de sandales.
        ${}^{7}Ils écrasent la tête des faibles dans la poussière,
        aux humbles ils ferment la route.
        \\Le fils et le père vont vers la même fille
        et profanent ainsi mon saint nom.
        ${}^{8}Auprès des autels, ils se couchent
        sur les vêtements qu’ils ont pris en gage.
        \\Dans la maison de leur Dieu,
        ils boivent le vin de ceux qu’ils ont frappés d’amende.
        ${}^{9}Moi, pourtant, j’avais détruit devant eux l’Amorite,
        dont la stature égalait celle des cèdres
        et la vigueur, celle des chênes !
        \\Je l’avais anéanti de haut en bas,
        depuis les fruits jusqu’aux racines\\.
        ${}^{10}Moi, je vous avais fait monter du pays d’Égypte
        et je vous avais, pendant quarante ans,
        conduits à travers le désert,
        \\pour vous donner en héritage le pays de l’Amorite.
${}^{11}J’avais suscité des prophètes parmi vos fils
        et, parmi vos jeunes gens, des naziréens (c’est-à-dire des hommes voués à Dieu).
        \\Oui ou non, est-ce vrai, fils d’Israël ?
        – oracle du Seigneur.
${}^{12}Mais vous faites boire du vin aux naziréens,
        et aux prophètes vous donnez cet ordre :
        « Ne prophétisez pas ! »
        ${}^{13}Eh bien, moi, maintenant, je vous écraserai sur place,
        comme un char plein de gerbes
        écrase tout sur son passage.
        ${}^{14}L’homme le plus rapide ne pourra pas fuir,
        le plus fort ne pourra pas montrer sa vigueur,
        même le héros ne sauvera pas sa vie.
        ${}^{15}L’archer ne tiendra pas,
        le coureur n’échappera pas,
        le cavalier ne sauvera pas sa vie.
        ${}^{16}Le plus brave\\s’enfuira tout nu, ce jour-là,
        – oracle du Seigneur.
      
         
      \bchapter{}
        ${}^{1}Écoutez cette parole que le Seigneur prononce
        contre vous, fils d’Israël,
        contre tout le peuple qu’il a fait monter\\du pays d’Égypte :
        ${}^{2}Je vous ai distingués\\, vous seuls,
        parmi tous les peuples\\de la terre ;
        \\aussi je vous demanderai compte
        de tous vos crimes.
        
           
         
        ${}^{3}Deux hommes font-ils route ensemble
        sans s’être mis d’accord ?
        ${}^{4}Est-ce que le lion rugit dans la forêt
        sans avoir de proie ?
        \\Le lionceau va-t-il crier du fond de sa tanière
        sans avoir rien pris ?
        ${}^{5}L’oiseau tombe-t-il dans le filet posé à terre
        sans y être attiré par un appât ?
        \\Le piège se relève-t-il du sol
        sans avoir rien attrapé ?
        ${}^{6}Va-t-on sonner du cor dans une ville
        sans que le peuple tremble ?
        \\Un malheur arrive-t-il dans une ville
        sans qu’il soit l’œuvre du Seigneur ?
        ${}^{7}– Car le Seigneur Dieu ne fait rien
        sans en révéler le secret
        à ses serviteurs les prophètes.
        ${}^{8}Quand le lion a rugi,
        qui peut échapper à la peur ?
        \\Quand le Seigneur Dieu a parlé,
        qui refuserait d’être prophète ?
        
           
${}^{9}Proclamez près des palais d’Ashdod,
        près des palais du pays d’Égypte.
        \\Dites :
        \\Assemblez-vous sur les montagnes de Samarie,
        voyez les grands désordres au milieu d’elle ;
        en elle, que d’oppressions !
${}^{10}Ils n’ont pas su agir avec droiture
        – oracle du Seigneur –,
        \\ceux qui entassent dans leurs palais
        violence et rapine.
${}^{11}C’est pourquoi – ainsi parle le Seigneur Dieu –
        l’ennemi encerclera le pays ;
        \\il te dépouillera de ta force
        et tes palais seront pillés.
${}^{12}Ainsi parle le Seigneur :
        \\Comme un berger sauve de la gueule du lion
        deux pattes ou un bout d’oreille,
        \\voilà comment seront sauvés les fils d’Israël,
        ceux qui sont, dans Samarie, assis au bord de leur couche,
        sur leur divan de Damas !
${}^{13}Écoutez et témoignez dans la maison de Jacob
        – oracle du Seigneur Dieu, le Dieu de l’univers :
${}^{14}car au jour de mon intervention, les crimes d’Israël seront sur lui,
        et j’interviendrai contre les autels de Béthel :
        \\les cornes de l’autel seront brisées,
        elles tomberont à terre ;
${}^{15}je ferai crouler maison d’hiver sur maison d’été,
        les maisons d’ivoire seront détruites
        \\et les vastes maisons disparaîtront
        – oracle du Seigneur.
       
      
         
      \bchapter{}
${}^{1}Écoutez cette parole, vaches du Bashane,
        \\sur la montagne de Samarie,
        \\vous qui exploitez les faibles,
        qui maltraitez les malheureux,
        \\qui dites à vos seigneurs :
        « Apportez-nous à boire ! »
${}^{2}Le Seigneur Dieu le jure par sa sainteté :
        \\Oui, voici venir sur vous des jours
        où l’on vous enlèvera avec des crocs,
        où votre progéniture sera prise au harpon ;
${}^{3}vous sortirez par les brèches,
        l’une devant l’autre,
        \\et vous serez poussées vers l’Hermon
        – oracle du Seigneur.
        
           
         
${}^{4}Allez à Béthel et commettez vos crimes ;
        à Guilgal, multipliez-les !
        \\Apportez dès le matin vos sacrifices,
        et le troisième jour, vos dîmes ;
${}^{5}faites fumer en action de grâce du pain sans levain,
        proclamez en public des offrandes volontaires,
        \\car c’est cela que vous aimez, fils d’Israël !
        – oracle du Seigneur Dieu.
        
           
         
${}^{6}Quant à moi, voici ce que je vous ai donné :
        rien à vous mettre sous la dent en toutes vos villes,
        plus de pain en aucun lieu.
        \\Et vous n’êtes pas revenus à moi !
        – oracle du Seigneur.
        
           
         
${}^{7}C’est moi aussi qui vous ai refusé la pluie
        à trois mois de la récolte :
        \\j’ai fait pleuvoir sur une ville,
        et sur une autre ville je n’ai pas fait pleuvoir ;
        \\une parcelle a reçu la pluie,
        et une autre, sans pluie, s’est desséchée ;
${}^{8}deux ou trois villes se traînaient vers une autre ville
        pour boire de l’eau,
        mais sans être désaltérées.
        \\Et vous n’êtes pas revenus à moi !
        – oracle du Seigneur.
        
           
         
${}^{9}J’ai frappé votre blé de rouille et de nielle ;
        \\et tous vos jardins et vos vignes,
        vos figuiers et vos oliviers,
        \\la chenille les a dévorés.
        \\Et vous n’êtes pas revenus à moi !
        – oracle du Seigneur.
        
           
         
${}^{10}J’ai jeté la peste chez vous, comme en Égypte ;
        \\j’ai tué par l’épée vos jeunes gens
        quand on a capturé vos chevaux ;
        \\la puanteur de votre camp,
        je l’ai fait monter à vos narines.
        \\Et vous n’êtes pas revenus à moi !
        – oracle du Seigneur.
        
           
         
        ${}^{11}J’ai tout détruit\\chez vous,
        comme Dieu a détruit Sodome et Gomorrhe ;
        \\vous étiez comme un tison
        sauvé de l’incendie.
        \\Et vous n’êtes pas revenus à moi !
        – oracle du Seigneur.
        
           
         
        ${}^{12}C’est pourquoi, voici comment je vais te traiter, Israël !
        \\Et puisque c’est ainsi que je vais te traiter,
        prépare-toi, Israël, à rencontrer ton Dieu.
${}^{13}Car c’est lui qui façonne les montagnes
        et crée le vent ;
        \\il révèle aux hommes sa pensée,
        
           
         
        il fait l’aurore et les ténèbres,
        \\il marche sur les hauteurs de la terre :
        son nom est « Le Seigneur, Dieu de l’univers ».
        
           
       
      
         
      \bchapter{}
${}^{1}Écoutez cette parole que je profère contre vous,
        une lamentation, maison d’Israël :
${}^{2}Elle est tombée, la vierge d’Israël,
        elle ne se relèvera plus ;
        \\elle est abandonnée sur sa terre,
        sans personne qui la relève.
${}^{3}Car ainsi parle le Seigneur Dieu :
        \\La ville d’où sortait un millier d’hommes
        n’en aura plus qu’une centaine,
        \\et celle d’où sortait une centaine
        n’en aura plus que dix
        pour la maison d’Israël.
        
           
         
        ${}^{4}Car ainsi parle le Seigneur à la maison d’Israël :
        \\Cherchez-moi et vous vivrez.
${}^{5}Mais ne cherchez pas à Béthel,
        \\n’entrez pas à Guilgal,
        ne passez pas à Bershéba ;
        \\car Guilgal sera entièrement déporté
        et Béthel, réduit à néant.
${}^{6}Cherchez le Seigneur et vous vivrez,
        de peur qu’il ne tombe comme le feu sur la maison de Joseph,
        \\un feu qui dévore,
        et personne, à Béthel, pour l’éteindre !
        
           
         
${}^{7}On change le droit en poison,
        on jette à terre la justice.
${}^{8}L’auteur des Pléiades et d’Orion,
        \\qui change en matin l’ombre de la mort
        et rend le jour obscur comme la nuit,
        \\qui convoque les eaux de la mer
        et les répand à la surface de la terre,
        \\son nom est « Le Seigneur ».
${}^{9}Sur le puissant il déchaîne la dévastation,
        et la dévastation arrive sur la citadelle.
${}^{10}Au tribunal, on déteste celui qui réprimande,
        \\et l’homme intègre dans sa parole, on le hait.
${}^{11}C’est pourquoi, vous qui avez pressuré le faible
        et prélevé sur lui un tribut de blé,
        \\les maisons que vous avez bâties en pierre de taille,
        vous n’y habiterez pas,
        \\les vignes exquises que vous avez plantées,
        vous n’en boirez pas le vin.
${}^{12}Oui, je connais vos nombreux crimes,
        vos énormes péchés,
        \\oppresseurs du juste, preneurs de pots-de-vin ;
        au tribunal les malheureux sont écartés.
${}^{13}C’est pourquoi, en ce temps-ci,
        \\l’homme avisé se tait,
        car c’est un temps de malheur.
        
           
         
        ${}^{14}Cherchez le bien et non le mal,
        afin de vivre.
        \\Ainsi le Seigneur, Dieu de l’univers, sera avec vous,
        comme vous le déclarez.
        ${}^{15}Détestez le mal, aimez le bien,
        faites régner le droit au tribunal\\ ;
        \\peut-être alors le Seigneur, Dieu de l’univers,
        fera-t-il grâce à ce qui reste d’Israël\\.
        
           
         
${}^{16}C’est pourquoi, ainsi parle le Seigneur,
        le Dieu de l’univers, mon Seigneur :
        \\Sur toutes les places, on fera des lamentations,
        dans toutes les rues, on dira : « Hélas ! hélas ! »
        \\On convoquera le paysan au deuil,
        et aux lamentations, les pleureurs ;
${}^{17}dans toutes les vignes, on fera des lamentations,
        car je passerai au milieu de toi.
        \\Le Seigneur a parlé.
        
           
         
${}^{18}Malheur à ceux qui aspirent au jour du Seigneur !
        Que sera-t-il pour vous, le jour du Seigneur ?
        \\Jour de ténèbre et non de lumière !
${}^{19}C’est comme un homme qui fuit devant un lion
        et qui tombe sur un ours ;
        \\il arrive à la maison, il pose la main sur le mur,
        et le serpent le mord.
${}^{20}Ainsi sera-t-il ténèbre, le jour du Seigneur,
        et non lumière,
        \\obscur, sans aucune clarté !
        
           
         
        ${}^{21}Je déteste, je méprise vos fêtes,
        je n’ai aucun goût pour vos assemblées.
        ${}^{22}Quand vous me présentez des holocaustes et des offrandes,
        je ne les accueille pas ;
        \\vos sacrifices de bêtes grasses,
        je ne les regarde même pas.
        ${}^{23}Éloignez de moi le tapage de vos cantiques ;
        que je n’entende pas la musique de vos harpes.
        ${}^{24}Mais que le droit jaillisse comme une source ;
        la justice, comme un torrent qui ne tarit jamais !
${}^{25}Des sacrifices et des offrandes, m’en avez-vous apporté
        pendant quarante ans au désert, maison d’Israël ?
${}^{26}Vous portiez les statues de Sikkouth, votre roi, et de Kiyyoun,
        avec l’étoile de vos dieux que vous aviez fabriqués.
${}^{27}Je vous déporterai au-delà de Damas.
        Le Seigneur a parlé.
        \\Son nom est « Dieu de l’univers ».
        
           
       
      
         
      \bchapter{}
        ${}^{1}Malheur à ceux qui vivent bien tranquilles
        \\dans Sion,
        \\et à ceux qui se croient en sécurité
        sur la montagne de Samarie,
        \\ces notables de la première des nations,
        vers qui se rend la maison d’Israël !
${}^{2}Passez par Kalné, et voyez ;
        de là, partez pour Hamath, la grande,
        puis descendez à Gath des Philistins.
        \\Ces villes sont-elles les meilleures des royaumes ?
        Leur territoire, plus grand que votre territoire ?
${}^{3}En croyant repousser le jour du malheur,
        vous rapprochez le règne de la violence !
        ${}^{4}Couchés sur des lits d’ivoire,
        vautrés sur leurs divans,
        \\ils mangent les agneaux du troupeau,
        les veaux les plus tendres de l’étable\\ ;
        ${}^{5}ils improvisent au son de la harpe,
        ils inventent, comme David, des instruments de musique ;
        ${}^{6}ils boivent le vin à même les amphores,
        ils se frottent avec des parfums de luxe,
        \\mais ils ne se tourmentent guère du désastre d’Israël\\ !
        ${}^{7}C’est pourquoi maintenant ils vont être déportés,
        ils seront les premiers des déportés ;
        \\et la bande des vautrés n’existera plus.
        
           
         
${}^{8}Le Seigneur Dieu le jure par lui-même
        – oracle du Seigneur, Dieu de l’univers :
        \\Moi, j’ai en horreur l’orgueil de Jacob,
        et je déteste ses palais ;
        aussi, je livrerai la ville et ce qu’elle renferme.
${}^{9}Et s’il reste dix hommes dans une seule maison,
        ils mourront.
${}^{10}Un parent, celui qui sortira les ossements
        hors de la maison, pour les brûler,
        \\dira à celui qui est au fond de la maison :
        « Y en a-t-il encore avec toi ? »
        \\Celui-ci répondra : « Plus rien ! »
        Il dira : « Silence !
        Car on n’a plus à faire mémoire du nom du Seigneur ! »
${}^{11}Voici que le Seigneur commande ;
        \\il frappe la grande maison : elle s’écroule,
        la petite maison : elle se fissure.
        
           
         
${}^{12}Est-ce qu’on fait galoper des chevaux sur des rochers,
        est-ce qu’on laboure la mer avec des bœufs,
        \\pour que vous changiez le droit en venin
        et le fruit de la justice en poison ?
        
           
         
${}^{13}Il en est qui se réjouissent pour Lo-Debar
        (c’est-à-dire Rien-du-tout),
        \\et qui disent : « N’est-ce pas par notre force
        que nous avons conquis Qarnaïm
        (c’est-à-dire Les-Deux-Cornes) ? »
${}^{14}Eh bien ! maison d’Israël, voici que moi
        – oracle du Seigneur, Dieu de l’univers –,
        \\je dresse contre vous une nation ;
        \\elle vous opprimera depuis l’Entrée-de-Hamath
        jusqu’au torrent de la Araba.
        
           
      
         
      \bchapter{}
${}^{1}Le Seigneur Dieu me donna cette vision :
        voici qu’il formait des sauterelles
        \\au temps où le regain commence à pousser,
        le regain après la fenaison pour le roi.
${}^{2}Comme elles achevaient de dévorer l’herbe du pays,
        je dis : « Seigneur Dieu, je t’en prie, pardonne !
        Jacob est si petit ! Qui le relèverait ? »
${}^{3}Le Seigneur s’en repentit.
        « Cela n’arrivera pas », dit le Seigneur.
        
           
${}^{4}Le Seigneur Dieu me donna cette vision :
        voici que le Seigneur Dieu en appelait au procès par le feu ;
        \\celui-ci avait dévoré les eaux profondes
        et déjà il dévorait la campagne.
${}^{5}Je dis : « Seigneur Dieu, je t’en prie, arrête !
        Jacob est si petit ! Qui le relèverait ? »
${}^{6}Le Seigneur s’en repentit.
        « Cela non plus n’arrivera pas », dit le Seigneur.
${}^{7}Il me donna cette vision :
        voici que le Seigneur se tenait
        sur un rempart monté d’aplomb,
        il avait en main un fil à plomb.
${}^{8}Le Seigneur me dit :
        \\« Que vois-tu, Amos ? »
        Je répondis : « Un fil à plomb ».
        \\Le Seigneur me dit : « Voici que je tiens le fil à plomb
        au milieu de mon peuple Israël ;
        j’en ai fini de passer outre en sa faveur.
${}^{9}Les lieux sacrés d’Isaac seront dévastés,
        et les sanctuaires d’Israël, rasés ;
        \\je me dresserai avec l’épée
        contre la maison de Jéroboam. »
${}^{10}Amazias, le prêtre de Béthel, envoya dire à Jéroboam, roi d’Israël : « Amos prêche la révolte contre toi, en plein royaume\\d’Israël ; le pays ne peut plus supporter tous ses discours, 
${}^{11} car voici ce que dit Amos : “Le roi Jéroboam périra par l’épée, et Israël sera déporté loin de sa terre.” »
${}^{12}Puis Amazias dit à Amos : « Toi, le voyant, va-t’en d’ici, fuis au pays de Juda ; c’est là-bas que tu pourras gagner ta vie en faisant ton métier de prophète. 
${}^{13} Mais ici, à Béthel, arrête de prophétiser ; car c’est un sanctuaire royal, un temple\\du royaume. »
${}^{14}Amos répondit à Amazias : « Je n’étais pas prophète ni fils de prophète ; j’étais bouvier, et je soignais les sycomores. 
${}^{15} Mais le Seigneur m’a saisi quand j’étais derrière le troupeau, et c’est lui qui m’a dit : “Va, tu seras prophète pour mon peuple Israël.” 
${}^{16} Écoute maintenant la parole du Seigneur, toi qui me dis : “Ne prophétise pas contre Israël, ne parle pas contre la maison d’Isaac.” 
${}^{17} Eh bien, voici ce que le Seigneur a dit :
        \\Ta femme devra se prostituer en pleine ville,
        tes fils et tes filles tomberont par l’épée,
        \\la terre qui t’appartient sera partagée au cordeau,
        toi, tu mourras sur une terre impure,
        \\et Israël sera déporté loin de sa terre. »
      
         
      \bchapter{}
${}^{1}Le Seigneur Dieu me donna cette vision :
        c’était une corbeille de fruits mûrs.
${}^{2}Il dit : « Que vois-tu, Amos ? »
        Je répondis : « Une corbeille de fruits mûrs. »
        \\Le Seigneur me dit :
        \\« Mon peuple Israël est mûr, sa fin est arrivée ;
        j’en ai fini de passer outre en sa faveur.
${}^{3}Ce jour-là, les chants du palais hurleront,
        – oracle du Seigneur Dieu.
        \\Nombreux seront les cadavres,
        en tout lieu on les jettera. Silence ! »
        
           
        ${}^{4}Écoutez ceci, vous qui écrasez le malheureux
        pour anéantir les humbles du pays,
        ${}^{5}car vous dites :
        \\« Quand donc la fête de la nouvelle lune sera-t-elle passée,
        pour que nous puissions vendre notre blé ?
        \\Quand donc le sabbat sera-t-il fini,
        pour que nous puissions écouler notre froment ?
        \\Nous allons diminuer les mesures\\,
        augmenter les prix\\et fausser les balances.
        ${}^{6}Nous pourrons acheter le faible pour un peu\\d’argent,
        le malheureux pour une paire de sandales.
        \\Nous vendrons jusqu’aux déchets du froment ! »
        ${}^{7}Le Seigneur le jure par la Fierté de Jacob :
        Non, jamais je n’oublierai aucun de leurs méfaits\\.
${}^{8}À cause de cela, la terre ne va-t-elle pas trembler,
        et toute sa population, prendre le deuil ?
        \\Ne va-t-elle pas monter, tout entière, comme le Nil,
        déborder, inonder, comme le fleuve d’Égypte ?
         
        ${}^{9}Ce jour-là
        – oracle du Seigneur Dieu –,
        \\je ferai disparaître le soleil en plein midi,
        en plein jour, j’obscurcirai la lumière sur la terre.
        ${}^{10}Je changerai vos fêtes en deuil,
        tous vos chants en lamentations ;
        \\je vous obligerai tous à vous vêtir de toile à sac,
        à vous raser la tête.
        \\Je mettrai ce pays en deuil comme pour un fils unique,
        et, dans la suite des jours, il connaîtra l’amertume.
        ${}^{11}Voici venir des jours – oracle du Seigneur Dieu –,
        où j’enverrai la famine sur la terre ;
        \\ce ne sera pas une faim de pain ni une soif d’eau,
        mais la faim et la soif d’entendre les paroles du Seigneur.
        ${}^{12}On se traînera d’une mer à l’autre,
        marchant à l’aventure du nord au levant,
        \\pour chercher en tout lieu\\la parole du Seigneur,
        mais on ne la trouvera pas.
         
${}^{13}Ce jour-là, les jeunes filles en leur beauté se faneront
        et les jeunes hommes souffriront de la soif.
${}^{14}Ceux qui juraient par l’idole sacrilège de Samarie,
        et qui disaient : « Vive ton dieu, Dane ! »
        et : « Vive ton chemin, Bershéba ! »,
        \\ils tomberont et ne se lèveront plus.
      
         
      \bchapter{}
${}^{1}Je vis le Seigneur debout près de l’autel.
        \\Il dit : Frappe les chapiteaux,
        et que tremblent les seuils !
        \\Brise tous ceux qui sont en tête,
        et les suivants, je les tuerai par l’épée ;
        \\pas un d’entre eux ne pourra s’enfuir,
        pas un d’entre eux ne pourra s’échapper.
${}^{2}S’ils forcent le séjour des morts,
        de là, ma main les extirpera ;
        \\s’ils escaladent les cieux,
        de là, je les ferai descendre ;
${}^{3}s’ils se cachent au sommet du Carmel,
        là, je les chercherai et les prendrai ;
        \\s’ils se dérobent à mes yeux au fond de la mer,
        là, je commanderai au Serpent de les mordre ;
${}^{4}s’ils s’en vont en captivité, poussés par l’ennemi,
        là-bas, je commanderai à l’épée de les tuer ;
        \\j’aurai l’œil sur eux,
        pour le malheur, non pour le bonheur.
        
           
         
${}^{5}Le Seigneur, Dieu de l’univers,
        \\qu’il touche la terre, elle s’effondre,
        et tous ses habitants sont en deuil ;
        \\elle monte, tout entière, comme le Nil,
        elle baisse comme le fleuve d’Égypte.
${}^{6}Lui qui bâtit son escalier dans le ciel
        et fonde sa voûte sur la terre,
        \\lui qui convoque les eaux de la mer
        et les répand à la surface de la terre,
        \\son nom est « Le Seigneur ».
        
           
         
${}^{7}N’êtes-vous pas pour moi, fils d’Israël,
        comme des fils d’Éthiopiens ?
        – oracle du Seigneur.
        \\N’ai-je pas fait monter Israël du pays d’Égypte ?
        De Kaftor, les Philistins ? Et de Qir, les Araméens ?
        
           
         
${}^{8}Voici les yeux du Seigneur Dieu sur le royaume pécheur :
        je vais le supprimer de la surface de la terre ;
        \\toutefois, je ne supprimerai pas entièrement la maison de Jacob
        – oracle du Seigneur.
${}^{9}Car voici que, moi, je commande ;
        \\je vais secouer la maison d’Israël
        parmi toutes les nations,
        \\comme on secoue dans un crible,
        et pas un caillou n’échappe.
${}^{10}Tous les pécheurs de mon peuple
        périront par l’épée,
        \\eux qui disaient :
        \\« Le malheur n’approchera pas,
        il ne nous atteindra pas ! »
        
           
        ${}^{11}Ce jour-là, je relèverai la hutte de David, qui s’écroule ;
        je réparerai ses brèches, je relèverai ses ruines,
        \\je la rebâtirai telle qu’aux jours d’autrefois,
        ${}^{12}afin que ses habitants prennent possession
        du reste d’Édom et de toutes les nations
        \\sur lesquelles mon nom fut jadis invoqué,
        – oracle du Seigneur, qui fera tout cela.
        ${}^{13}Voici venir des jours
        – oracle du Seigneur –
        \\où se suivront de près laboureur et moissonneur,
        le fouleur de raisins et celui qui jette la semence.
        \\Les montagnes laisseront couler le vin nouveau,
        toutes les collines en seront ruisselantes.
        ${}^{14}Je ramènerai les captifs de mon peuple Israël ;
        ils rebâtiront les villes dévastées et les habiteront ;
        \\ils planteront des vignes et en boiront le vin ;
        ils cultiveront des jardins et en mangeront les fruits.
        ${}^{15}Je les planterai sur leur sol,
        \\et jamais plus ils ne seront arrachés
        du sol que je leur ai donné.
        \\Le Seigneur ton Dieu a parlé.
