  
  
    
    \bbook{OSÉE}{OSÉE}
      
         
      \bchapter{}
      \begin{verse}
${}^{1}Parole du Seigneur adressée à Osée, fils de Beéri, au temps d’Ozias, de Yotam, d’Acaz, d’Ézékias, rois de Juda, et au temps de Jéroboam, fils de Joas, roi d’Israël.
      
         
${}^{2}Commencement de la parole que le Seigneur a dite par la bouche d’Osée.
        \\Le Seigneur dit à Osée :
        \\« Va, prends-toi pour femme une prostituée
        et des enfants de prostitution,
        \\car vraiment le pays se prostitue
        en se détournant du Seigneur. »
         
${}^{3}Il alla donc et prit Gomer, fille de Diblaïm ;
        elle devint enceinte et lui enfanta un fils.
${}^{4}Et le Seigneur dit à Osée :
        \\« Donne-lui le nom de Yizréel,
        car encore un peu de temps
        \\et je sévis contre la maison de Jéhu
        à cause du sang versé à Yizréel,
        \\et je mets fin à la royauté de la maison d’Israël :
${}^{5}il adviendra, en ce jour-là,
        que je briserai l’arc d’Israël
        dans la vallée de Yizréel. »
         
${}^{6}Elle devint encore enceinte et enfanta une fille.
        \\Et le Seigneur dit à Osée :
        \\« Donne-lui le nom de Lô-Rouhama (c’est-à-dire “Pas-Aimée”),
        car je n’aime plus la maison d’Israël
        et ne veux plus lui pardonner.
${}^{7}C’est la maison de Juda que j’aime :
        je vais les sauver, par le Seigneur leur Dieu,
        \\et non par l’arc, l’épée ou la guerre,
        ni par les chevaux ni par les cavaliers. »
         
${}^{8}Quand elle eut sevré Lô-Rouhama,
        Gomer devint enceinte et enfanta un fils.
${}^{9}Et le Seigneur dit :
        \\« Donne-lui le nom de Lô-Ammi (c’est-à-dire “Pas-mon-Peuple”),
        car vous n’êtes pas mon peuple,
        \\et moi, je ne suis pas pour vous. »
      
         
      \bchapter{}
${}^{1}Comme le sable de la mer
        que l’on ne peut ni compter ni mesurer,
        ainsi sera le nombre des fils d’Israël.
        \\Au lieu de leur dire : « Vous n’êtes pas mon peuple »,
        on leur dira : « Fils du Dieu vivant ».
${}^{2}Les fils de Juda et les fils d’Israël se réuniront,
        ils se donneront un seul chef
        \\et ils sortiront du pays ;
        oui, il est grand, le jour de Yizréel !
${}^{3}Dites à vos frères : « Mon-Peuple »,
        et à vos sœurs : « Tendrement-Aimée ».
        
           
${}^{4}Accusez votre mère, accusez-la,
        car elle n’est plus ma femme,
        et moi, je ne suis plus son mari !
        \\Qu’elle écarte de son visage ses prostitutions,
        et d’entre ses seins, ses adultères ;
${}^{5}sinon, je la déshabille toute nue,
        je l’expose comme au jour de sa naissance,
        \\je la rends pareille au désert,
        je la réduis en terre aride
        et je la fais mourir de soif.
         
${}^{6}Pour ses fils je n’aurai pas de tendresse,
        car ils sont des fils de prostitution.
${}^{7}Oui, leur mère s’est prostituée,
        celle qui les conçut s’est déshonorée
        \\quand elle disait :
        \\« Je veux courir après mes amants
        qui me donnent mon pain et mon eau,
        \\ma laine et mon lin,
        mon huile et ma boisson. »
${}^{8}C’est pourquoi
        \\je vais obstruer son chemin avec des ronces,
        le barrer d’une barrière :
        elle ne trouvera plus ses sentiers.
${}^{9}Elle poursuivra ses amants sans les atteindre,
        elle les cherchera sans les trouver.
        \\Alors elle dira :
        \\« Je vais revenir à mon premier mari,
        car j’étais autrefois plus heureuse que maintenant. »
${}^{10}Elle ne savait donc pas
        \\que c’est moi qui lui avais donné
        le froment, le vin nouveau et l’huile fraîche,
        \\moi qui lui avais prodigué de l’argent,
        et l’or utilisé pour Baal !
${}^{11}C’est pourquoi je reviendrai,
        \\je reprendrai mon froment en sa saison
        et mon vin nouveau en son temps ;
        \\j’arracherai ma laine et mon lin
        dont elle couvrait sa nudité.
${}^{12}Alors je dévoilerai sa honte
        aux yeux de ses amants,
        \\et nul ne la délivrera de ma main.
${}^{13}Je mettrai fin à toute sa gaieté,
        à ses fêtes, ses nouvelles lunes, ses sabbats,
        et à toutes ses solennités.
${}^{14}Je dévasterai sa vigne et son figuier
        dont elle disait :
        \\« Ils sont à moi, c’est le salaire
        que m’ont donné mes amants. »
        \\Je les changerai en friche
        et les bêtes sauvages les dévoreront.
${}^{15}Je sévirai contre elle
        à cause des jours des Baals,
        \\quand elle brûlait pour eux de l’encens,
        se parait de ses anneaux et de son collier,
        et courait après ses amants.
        \\Et moi, elle m’oubliait !
        \\– oracle du Seigneur.
        ${}^{16}C’est pourquoi, mon épouse infidèle\\,
        je vais la séduire,
        \\je vais l’entraîner jusqu’au désert,
        et je lui parlerai cœur à cœur.
        ${}^{17}Et là, je lui rendrai ses vignobles,
        \\et je ferai du Val d’Akor (c’est-à-dire « de la Déroute »)
        la porte de l’espérance.
        \\Là, elle me répondra
        comme au temps de sa jeunesse,
        au jour où elle est sortie du pays d’Égypte.
        ${}^{18}En ce jour-là – oracle du Seigneur –,
        voici ce qui arrivera :
        \\Tu m’appelleras : « Mon époux »
        et non plus : « Mon Baal » (c’est-à-dire « mon maître »)\\.
        ${}^{19}J’éloignerai de ses lèvres les noms des Baals,
        on ne prononcera plus leurs noms.
        ${}^{20}En ce jour-là je conclurai à leur profit
        une alliance avec les bêtes sauvages,
        avec les oiseaux du ciel et les bestioles de la terre ;
        \\l’arc, l’épée et la guerre, je les briserai
        pour en délivrer le pays ;
        \\et ses habitants, je les ferai reposer en sécurité.
        ${}^{21}Je ferai de toi mon épouse pour toujours,
        je ferai de toi mon épouse
        \\dans la justice et le droit,
        dans la fidélité et la tendresse ;
        ${}^{22}je ferai de toi mon épouse dans la loyauté,
        et tu connaîtras le Seigneur.
        ${}^{23}En ce jour-là je répondrai – oracle du Seigneur ;
        oui, je répondrai aux cieux,
        \\eux, ils répondront à l’appel de la terre ;
        ${}^{24}la terre répondra au froment,
        au vin nouveau et à l’huile fraîche,
        \\eux, ils répondront à la « Vallée-de-la-fertilité\\ ».
        ${}^{25}Je m’en ferai une terre ensemencée,
        \\J’aimerai celle qu’on appelait « Pas-Aimée »
        et à celui qu’on appelait « Pas-mon-Peuple »,
        \\je dirai : « Tu es mon peuple »,
        et il dira : « Tu es mon Dieu ! »
      
         
      \bchapter{}
${}^{1}Le Seigneur me dit :
        \\« Va de nouveau,
        aime une femme aimée d’un compagnon
        et qui commet l’adultère.
        \\Car tel est l’amour du Seigneur
        pour les fils d’Israël,
        \\eux qui se tournent vers d’autres dieux
        et qui aiment les gâteaux de raisins. »
${}^{2}Je m’achetai une femme pour quinze pièces d’argent
        et une mesure et demie d’orge.
${}^{3}Et je lui dis :
        \\« Tu resteras chez moi de nombreux jours,
        sans te prostituer et sans appartenir à un homme ;
        \\et moi, je ferai de même à ton égard. »
${}^{4}Car, pendant de nombreux jours,
        \\les fils d’Israël resteront
        sans roi ni prince,
        \\sans sacrifice ni stèle,
        sans éphod ni terafim.
${}^{5}Après quoi, les fils d’Israël reviendront,
        \\ils rechercheront le Seigneur, leur Dieu,
        et David, leur roi ;
        \\ils iront en tremblant vers le Seigneur et sa bonté,
        pour la suite des temps.
        
           
      
         
      \bchapter{}
${}^{1}Écoutez la parole du Seigneur, fils d’Israël,
        car le Seigneur est en procès avec les habitants du pays :
        \\il n’y a, dans le pays, ni vérité ni fidélité,
        ni connaissance de Dieu,
${}^{2}mais parjure et mensonge,
        assassinat et vol ;
        \\on commet l’adultère, on se déchire :
        le sang appelle le sang.
${}^{3}C’est pourquoi le pays est en deuil,
        tous ses habitants dépérissent,
        \\ainsi que les bêtes sauvages et les oiseaux du ciel ;
        même les poissons de la mer disparaissent.
        
           
${}^{4}Mais que nul n’accuse, que nul ne réprimande :
        Prêtre, c’est avec toi que je suis en procès !
${}^{5}Tu trébuches le jour,
        le prophète aussi trébuche avec toi la nuit ;
        \\je réduirai ta mère au silence,
${}^{6}et mon peuple, faute de connaissance,
        sera, lui aussi, réduit au silence.
        \\Puisque tu as rejeté la connaissance,
        je te rejetterai et tu ne seras plus mon prêtre ;
        \\puisque tu as oublié la loi de ton Dieu,
        à mon tour, j’oublierai tes fils.
${}^{7}Tous, tant qu’ils sont, ils ont péché contre moi :
        je vais changer leur gloire en infamie.
${}^{8}Ils se repaissent du péché de mon peuple
        et vers leur faute ils portent leur désir.
${}^{9}Il en sera du prêtre comme du peuple :
        \\je sévirai contre lui à cause de sa conduite
        et je lui revaudrai ses actions.
${}^{10}Ils se repaissent, et ne sont pas rassasiés,
        ils se prostituent, et ne s’accroissent pas,
        car ils ont cessé de respecter le Seigneur.
${}^{11}Prostitution, vin et vin nouveau
        emprisonnent le cœur.
${}^{12}Mon peuple consulte son idole de bois,
        c’est son bâton qui le renseigne ;
        \\car un esprit de prostitution l’égare :
        ils se sont prostitués, éloignés de leur Dieu.
${}^{13}Sur les sommets des montagnes, ils sacrifient,
        sur les collines, ils brûlent des offrandes,
        \\sous le chêne, le peuplier, le térébinthe,
        dont l’ombrage est agréable !
        \\C’est pourquoi vos filles se prostituent
        et vos belles-filles sont adultères.
${}^{14}Je ne sévirai pas contre vos filles
        à cause de leurs prostitutions,
        \\ni contre vos belles-filles
        à cause de leurs adultères,
        \\puisqu’eux-mêmes vont à l’écart avec les prostituées
        et sacrifient avec les courtisanes sacrées.
        \\Un peuple qui ne comprend pas court à sa perte.
${}^{15}Si tu te prostitues, toi, Israël,
        que Juda ne se rende pas coupable !
        \\N’allez donc pas à Guilgal,
        ne montez pas à Beth-Awen
        et ne jurez pas par la vie du Seigneur.
${}^{16}Puisqu’Israël a été rétif
        comme une vache rétive,
        \\le Seigneur le conduirait-il maintenant
        comme un agneau dans une vaste prairie ?
${}^{17}Éphraïm est l’allié des idoles :
        laisse-le !
${}^{18}Après les beuveries, c’est la prostitution ;
        ils préfèrent l’ignominie à leur gloire.
${}^{19}Le vent les enveloppera de ses ailes
        et ils rougiront de leurs sacrifices.
      
         
      \bchapter{}
${}^{1}Écoutez ceci, vous les prêtres ;
        \\ sois attentive, maison d’Israël ;
        maison du roi, prête l’oreille !
        \\Oui, le jugement est contre vous,
        car vous avez été un piège à Mispa,
        un filet tendu sur le Tabor.
${}^{2}Des infidèles ont creusé une fosse profonde,
        et moi, je serai pour eux tous un châtiment.
${}^{3}Moi, je connais Éphraïm,
        et Israël ne m’est pas caché.
        \\Oui, maintenant, tu t’es prostitué, Éphraïm,
        tu t’es souillé, Israël.
${}^{4}Leurs actions ne leur permettent pas
        de retourner vers leur Dieu,
        \\car un esprit de prostitution les habite,
        et ils ne connaissent pas le Seigneur.
${}^{5}L’orgueil d’Israël témoigne contre lui.
        \\Israël et Éphraïm trébuchent à cause de leur faute,
        et Juda, lui aussi, trébuche avec eux.
${}^{6}Avec leurs brebis et leurs bœufs,
        ils iront chercher le Seigneur,
        \\mais ils ne le trouveront pas :
        il s’est éloigné d’eux !
${}^{7}Ils ont trahi le Seigneur :
        ils ont engendré des bâtards ;
        \\maintenant la nouvelle lune
        va les dévorer, avec leur héritage.
        
           
${}^{8}Sonnez du cor à Guibéa,
        de la trompette à Rama ;
        \\poussez des cris à Beth-Awen,
        on te poursuit, Benjamin !
${}^{9}Éphraïm sera une ruine
        au jour du reproche ;
        \\j’annonce une chose certaine
        aux tribus d’Israël.
${}^{10}Les chefs de Juda
        sont comme des gens qui déplacent des bornes ;
        \\sur eux, je déverserai,
        comme des flots, ma colère.
${}^{11}Éphraïm est exploité,
        le droit est malmené,
        \\car on se plaît
        à poursuivre le néant.
${}^{12}Et moi, je serai comme un abcès pour Éphraïm,
        comme une carie pour la maison de Juda.
${}^{13}Éphraïm a vu sa maladie,
        et Juda, son ulcère ;
        \\Éphraïm est allé vers Assour,
        il a envoyé des messagers au Grand Roi ;
        \\mais lui ne peut vous guérir
        ni cicatriser votre ulcère.
${}^{14}Car moi, je serai comme un lion pour Éphraïm,
        comme un lionceau pour la maison de Juda.
        \\Oui, moi, je déchire et je m’en vais,
        j’emporte, et personne qui délivre.
${}^{15}Je m’en irai, je retournerai en ma demeure,
        jusqu’à ce qu’ils s’avouent coupables
        \\et recherchent ma face,
        et que dans leur détresse ils me cherchent.
      <p class="cantique" id="bib_ct-at_42"><span class="cantique_label">Cantique AT 42</span> = <span class="cantique_ref"><a class="unitex_link" href="#bib_os_6_1">Os 6, 1-6</a></span>
      
         
      \bchapter{}
        ${}^{1}Venez, retournons vers le Seigneur !
        \\il a blessé, mais il nous guérira ;
        \\il a frappé, mais il nous soignera.
        
           
         
        ${}^{2}Après deux jours, il nous rendra la vie ;
        \\il nous relèvera le troisième jour :
        \\alors, nous vivrons devant sa face.
        
           
         
        ${}^{3}Efforçons-nous de connaître le Seigneur :
        \\son lever est aussi sûr que l’aurore ;
        \\il nous viendra comme la pluie,
        \\l’ondée qui arrose la terre.
        
           
         
        ${}^{4}– Que ferai-je de toi, Éphraïm ?
        \\Que ferai-je de toi, Juda ?
        \\Votre fidélité, une brume du matin,
        \\une rosée d’aurore qui s’en va.
        
           
         
        ${}^{5}Voilà pourquoi j’ai frappé par mes prophètes,
        \\donné la mort par les paroles de ma bouche :
        \\mon jugement jaillit comme la lumière\\.
        
           
         
        ${}^{6}Je veux la fidélité, non le sacrifice,
        \\la connaissance de Dieu plus que les holocaustes.
        
           
${}^{7}Mais, dans la ville d’Adame, eux, ils ont transgressé l’alliance,
        et là, ils m’ont trahi.
${}^{8}Galaad, cité de malfaiteurs,
        est tachée de sang.
${}^{9}Sur la route de Sichem,
        une bande de prêtres assassinent
        \\comme des brigands en embuscade :
        voilà les horreurs qu’ils commettent !
${}^{10}Dans la maison d’Israël,
        j’ai vu des choses monstrueuses,
        \\là où se prostitue Éphraïm,
        où Israël se rend impur.
${}^{11}Pour toi aussi, Juda, je prépare une moisson :
        je changerai le sort de mon peuple.
       
      
         
      \bchapter{}
${}^{1}Quand je voulais guérir Israël,
        \\alors s’est dévoilée la faute d’Éphraïm,
        et les méfaits de Samarie,
        car ils pratiquent le mensonge.
        \\Le voleur s’introduit dans les maisons,
        et au-dehors sévit le brigand.
${}^{2}Ils ne disent pas dans leur cœur
        que je me souviens de tous leurs méfaits ;
        \\à présent leurs œuvres les encerclent,
        elles sont là devant moi.
        
           
${}^{3}Dans leur malice, ils amusent le roi,
        et par leurs mensonges, les princes.
${}^{4}Tous, ils sont adultères,
        comme un four brûlant,
        \\que le boulanger cesse d’attiser
        depuis qu’il a pétri la pâte jusqu’à ce qu’elle ait levé.
${}^{5}Au jour de notre roi,
        \\les princes, embrasés par le vin,
        se rendent malades ;
        \\on tend la main aux railleurs.
${}^{6}Par leur complot,
        ils ont rendu leur cœur pareil au four
        \\dont le boulanger sommeille toute la nuit
        et qu’un feu violent fait brûler au matin.
${}^{7}Tous, comme un four, ils sont embrasés :
        ils dévorent leurs juges ;
        \\tous leurs rois sont tombés,
        pas un d’entre eux ne crie vers moi !
${}^{8}Éphraïm se mêle aux autres peuples,
        Éphraïm est une galette qu’on n’a pas retournée !
${}^{9}Des étrangers dévorent sa force,
        et lui n’en sait rien ;
        \\ses cheveux ont blanchi,
        et lui n’en sait rien.
${}^{10}L’orgueil d’Israël témoigne contre lui,
        ils ne reviennent pas au Seigneur, leur Dieu ;
        malgré tout cela, ils ne le cherchent pas.
${}^{11}Voici Éphraïm,
        colombe naïve et sans cœur :
        \\ils appellent l’Égypte,
        ils s’en vont à Assour.
${}^{12}Où qu’ils aillent, je jette sur eux mon filet,
        je les fais descendre comme les oiseaux du ciel,
        je les attrape dès que j’entends qu’ils se rassemblent.
${}^{13}Malheur sur eux, car ils ont fui loin de moi !
        Ruine sur eux, car ils se sont révoltés contre moi !
        \\Moi, je veux les racheter ;
        eux, ils disent des mensonges contre moi.
${}^{14}Ce n’est pas du fond du cœur qu’ils crient vers moi,
        quand ils se lamentent sur leurs couches ;
        \\ils s’inquiètent pour du froment et du vin nouveau,
        c’est de moi qu’ils s’éloignent.
${}^{15}Alors que j’avais dirigé,
        que j’avais fortifié leur bras,
        \\ils méditent le mal envers moi.
${}^{16}Ils reviennent, ils n’ont pas le dessus,
        ils sont comme un arc trompeur ;
        \\leurs chefs tomberont sous l’épée
        pour l’insolence de leur langage :
        \\on en rira au pays d’Égypte.
      
         
      \bchapter{}
${}^{1}Sonne du cor !
        \\Un aigle plane sur la maison du Seigneur !
        \\Car ils ont transgressé mon alliance,
        et contre ma loi ils se sont révoltés.
${}^{2}Ils crient vers moi : « Mon Dieu,
        nous t’avons connu, nous, Israël ! »
${}^{3}Israël a rejeté le bien :
        l’ennemi le poursuit !
        
           
        ${}^{4}Les fils d’Israël\\ont établi des rois sans me consulter,
        ils ont nommé des princes sans mon accord ;
        \\avec leur argent et leur or, ils se sont fabriqué des idoles.
        Ils seront anéantis.
        ${}^{5}Je le rejette, ton veau, Samarie !
        Ma colère s’est enflammée contre tes enfants.
        \\Refuseront-ils toujours de retrouver l’innocence ?
        ${}^{6}Ce veau est l’œuvre d’Israël,
        un artisan l’a fabriqué,
        \\ce n’est pas un dieu ;
        ce veau de Samarie sera mis en pièces.
        ${}^{7}Ils ont semé le vent,
        ils récolteront la tempête.
        \\L’épi ne donnera pas de grain ;
        \\s’il y avait du grain,
        il ne donnerait pas de farine ;
        \\et, s’il en donnait,
        elle serait dévorée par les étrangers.
         
${}^{8}Israël est dévoré ;
        les voici maintenant parmi les nations,
        comme un objet de rebut.
${}^{9}Quand ils montent vers Assour,
        alors que l’âne sauvage vit à l’écart,
        Éphraïm s’offre des amants.
${}^{10}Même s’ils offrent des dons parmi les nations,
        je vais maintenant les regrouper ;
        \\ils trembleront bientôt
        sous le joug du roi et des princes.
        ${}^{11}Éphraïm a multiplié les autels pour expier le péché ;
        et ces autels ne lui servent qu’à pécher.
        ${}^{12}J’ai beau lui mettre par écrit tous les articles\\de ma loi,
        il n’y voit qu’une loi étrangère.
        ${}^{13}Ils offrent des sacrifices pour me plaire
        et ils en mangent la viande,
        mais le Seigneur n’y prend pas de plaisir.
        \\Au contraire, il y trouve le rappel de toutes leurs fautes,
        il fait le compte de leurs péchés.
        \\Qu’ils retournent donc en Égypte !
${}^{14}Israël a oublié Celui qui le fait,
        il s’est construit des palais.
        \\Quant à Juda, il a multiplié ses villes fortes,
        \\mais j’enverrai le feu dans ses villes,
        il en dévorera les citadelles.
      
         
      \bchapter{}
${}^{1}Ne te réjouis pas, Israël,
        ne jubile pas comme les peuples,
        \\car tu as pratiqué la prostitution loin de ton Dieu,
        et tu en aimes le salaire
        sur toutes les aires à blé.
${}^{2}L’aire à blé et le pressoir
        ne nourriront pas tes fils,
        et le vin nouveau les décevra.
${}^{3}Ils n’habiteront plus le pays du Seigneur :
        Éphraïm retournera en Égypte,
        et en Assour ils mangeront des aliments impurs.
${}^{4}Ils ne verseront plus de vin en libation pour le Seigneur,
        leurs sacrifices ne lui plairont plus ;
        \\ce sera pour eux comme un pain de deuil :
        tous ceux qui le mangent deviendront impurs ;
        \\ce pain-là ne sera que pour eux,
        il n’entrera pas dans la maison du Seigneur.
${}^{5}Que ferez-vous alors au jour de la Rencontre,
        au jour de la fête du Seigneur ?
${}^{6}Car voilà qu’ils ont fui devant la destruction ;
        \\l’Égypte les rassemble,
        Memphis les enterre ;
        \\l’ortie héritera de leurs trésors d’argent,
        les ronces envahiront leurs tentes.
${}^{7}Ils sont arrivés, les jours du châtiment,
        ils sont arrivés, les jours de la rétribution :
        qu’Israël le sache !
        \\Le prophète devient fou,
        l’homme inspiré délire ;
        \\à cause de la grandeur de ta faute,
        grande est l’hostilité contre toi.
${}^{8}La sentinelle d’Éphraïm, le prophète, est avec mon Dieu ;
        un filet d’oiseleur est sur tous ses chemins,
        \\l’hostilité atteint la maison de Dieu.
${}^{9}Ils ont touché le fond de la corruption,
        comme aux jours de Guibéa.
        \\Dieu se souviendra de leur crime,
        il fera le compte de leurs péchés.
        
           
${}^{10}Comme des raisins au désert,
        j’avais trouvé Israël ;
        \\comme un premier fruit sur un jeune figuier,
        j’avais vu vos pères.
        \\Mais eux, arrivés à Baal-Péor,
        ils se sont voués à la Honte,
        \\ils sont devenus aussi horribles
        que l’objet de leur amour.
${}^{11}Éphraïm ! Comme un oiseau s’envolera ta gloire,
        dès la naissance, dès la grossesse et la conception.
${}^{12}Même s’ils élèvent des fils,
        je les en priverai avant qu’ils aient l’âge d’homme.
        \\Oui, malheur à eux, quand je m’en éloignerai !
${}^{13}Éphraïm, je le vois comme une autre Tyr
        plantée dans un pâturage,
        \\mais il fait partir ses fils au-devant du tueur.
${}^{14}Donne-leur, Seigneur,
        – et que vas-tu donner ? – 
        \\donne-leur ventre stérile
        et seins desséchés.
${}^{15}Toute leur malice est à Guilgal :
        c’est là que je les ai pris en haine.
        \\À cause de la méchanceté de leurs actes,
        je les chasserai de ma maison,
        \\je cesserai de les aimer :
        ils sont rebelles, tous leurs princes.
${}^{16}Éphraïm a été frappé,
        \\leur racine s’est desséchée,
        ils ne feront pas de fruit !
        \\Même s’ils enfantent,
        je ferai mourir les trésors de leur ventre.
${}^{17}Mon Dieu les rejettera,
        car ils ne l’ont pas écouté :
        \\ils s’en iront, errant parmi les nations.
      
         
      \bchapter{}
        ${}^{1}Israël était une vigne luxuriante,
        \\qui portait beaucoup de fruit.
        \\Mais plus ses fruits se multipliaient,
        plus Israël multipliait les autels ;
        \\plus sa terre devenait belle,
        plus il embellissait les stèles des faux dieux\\.
        ${}^{2}Son cœur est partagé ;
        maintenant il va expier :
        \\le Seigneur renversera ses autels ;
        les stèles, il les détruira.
        ${}^{3}Maintenant Israël va dire :
        \\« Nous sommes privés de roi,
        car nous n’avons pas craint le Seigneur.
        \\Et si nous avions un roi,
        que pourrait-il faire pour nous ? »
${}^{4}On parle, on parle,
        on fait de faux serments,
        on conclut des alliances ;
        \\le jugement fleurit comme l’herbe vénéneuse
        sur les sillons des champs.
${}^{5}Les habitants de Samarie tremblent
        pour le veau de Beth-Awen :
        \\pour lui son peuple est en deuil,
        avec ses desservants qui pour lui exultaient,
        \\car sa gloire a été déportée au loin.
${}^{6}Lui-même sera transporté en Assour
        comme offrande au Grand Roi ;
        \\Éphraïm en aura de la honte,
        et Israël rougira de son idole.
        ${}^{7}Ils ont disparu, Samarie et son roi,
        comme de l’écume à la surface de l’eau.
        ${}^{8}Les lieux sacrés seront détruits,
        ils sont le crime, le péché d’Israël ;
        épines et ronces recouvriront leurs autels.
        \\Alors on dira aux montagnes : « Cachez-nous ! »
        et aux collines : « Tombez sur nous ! »
        
           
         
${}^{9}Depuis les jours de Guibéa,
        tu as péché, Israël !
        \\C’est là qu’ils en sont restés !
        \\Et le combat contre les fils du crime
        ne les atteindrait pas à Guibéa ?
${}^{10}Au gré de mes désirs, je vais les corriger :
        les peuples se ligueront contre eux
        et les corrigeront de leur double faute.
${}^{11}Éphraïm était une génisse bien dressée,
        qui aimait fouler le grain.
        \\Et moi, j’ai passé le joug
        sur la beauté de son encolure,
        \\j’attellerai Éphraïm,
        Juda labourera
        et Jacob hersera.
        
           
         
        ${}^{12}Faites des semailles de justice,
        récoltez une moisson de fidélité,
        défrichez vos terres en friche.
        \\Il est temps de chercher le Seigneur,
        jusqu’à ce qu’il vienne répandre sur vous
        une pluie de justice.
${}^{13}Vous avez labouré la méchanceté,
        vous avez moissonné la perfidie,
        vous avez mangé le fruit du mensonge.
        \\Parce que tu as mis ta confiance dans tes chars,
        dans tes nombreux guerriers,
${}^{14}il s’élèvera du vacarme parmi ton peuple,
        et toutes tes villes fortes seront dévastées,
        \\comme Shalmane dévasta Beth-Arbel au jour du combat,
        quand la mère fut écrasée sur ses enfants.
${}^{15}Ainsi vous fera Béthel,
        face à votre abominable méfait ;
        à l’aurore, il sera vraiment perdu, le roi d’Israël.
        
           
      
         
      \bchapter{}
        ${}^{1}Oui, j’ai aimé Israël dès son enfance\\,
        \\et, pour le faire sortir\\d’Égypte, j’ai appelé mon fils.
        ${}^{2}Quand je l’ai appelé\\,
        il s’est éloigné\\pour sacrifier aux Baals
        et brûler des offrandes aux idoles.
        ${}^{3}C’est moi qui lui\\apprenais à marcher,
        en le soutenant de mes bras,
        et il n’a pas compris que je venais à son secours.
        ${}^{4}Je le guidais avec humanité,
        par des liens d’amour\\ ;
        \\je le traitais comme un nourrisson
        qu’on soulève tout contre sa joue ;
        je me penchais vers lui pour le faire manger.
        \\Mais ils ont refusé de revenir à moi :
        vais-je les livrer au châtiment ?
        ${}^{5}Il ne retournera pas au pays d’Égypte ;
        Assour deviendra son roi,
        car ils ont refusé de revenir à moi.
        ${}^{6}L’épée frappera dans ses villes,
        elle brisera les verrous de ses portes,
        elle les dévorera à cause de leurs intrigues.
        ${}^{7}Mon peuple s’accroche à son infidélité ;
        on l’appelle vers le haut ;
        aucun ne s’élève.
        
           
         
        ${}^{8}Vais-je t’abandonner, Éphraïm,
        et te livrer, Israël ?
        \\Vais-je t’abandonner comme Adma,
        et te rendre comme Seboïm ?
        \\Non ! Mon cœur se retourne contre moi ;
        en même temps, mes entrailles frémissent\\.
        ${}^{9}Je n’agirai pas selon l’ardeur de ma colère,
        je ne détruirai plus Israël,
        \\car moi, je suis Dieu, et non pas homme :
        au milieu de vous je suis le Dieu\\saint,
        \\et je ne viens pas pour exterminer.
        
           
         
        ${}^{10}Ils marcheront à la suite du Seigneur ;
        comme un lion il rugira,
        \\oui, il rugira, lui,
        et, tout tremblants, ses fils viendront de l’Occident.
        ${}^{11}Comme un oiseau, tout tremblants, ils viendront de l’Égypte,
        et comme une colombe, du pays d’Assour ;
        \\je les ferai habiter dans leurs maisons,
        – oracle du Seigneur.
        
           
      
         
      \bchapter{}
${}^{1}Éphraïm m’encercle de mensonge
        et la maison d’Israël, de tromperie.
        \\Mais Juda marche encore avec Dieu
        et reste fidèle au Très-Saint.
${}^{2}Éphraïm se repaît de vent
        et poursuit le vent d’est ;
        \\tout le jour, il multiplie mensonge et ruine :
        \\il conclut une alliance avec Assour,
        il porte de l’huile à l’Égypte.
        
           
${}^{3}Le Seigneur est en procès avec Juda,
        il va sévir contre Jacob en raison de sa conduite,
        lui rendre selon ses œuvres.
${}^{4}Dans le sein de sa mère, il a supplanté son frère,
        et, à l’âge mûr, il a lutté avec Dieu.
${}^{5}Il a lutté avec un ange et il a eu le dessus ;
        il a pleuré et l’a supplié.
        \\À Béthel, il le trouva,
        – c’est là que Dieu a parlé avec nous.
${}^{6}« Seigneur, Dieu de l’univers » :
        c’est ainsi qu’on fait mémoire du Seigneur.
${}^{7}Toi, reviens à ton Dieu :
        garde la fidélité et le droit,
        et mets ton espoir en ton Dieu, toujours.
${}^{8}Canaan, une balance trompeuse à la main,
        aime frauder.
${}^{9}Éphraïm dit :
        \\« Oui, je me suis enrichi,
        j’ai édifié une fortune. 
        \\Tout cela est fruit de mon travail ;
        on ne trouvera pas chez moi
        de faute qui soit péché. »
${}^{10}Mais moi, je suis le Seigneur ton Dieu
        depuis le pays d’Égypte.
        \\Je te ferai de nouveau habiter sous les tentes
        comme au jour de la Rencontre.
${}^{11}Je parlerai aux prophètes ;
        moi, je multiplierai les visions,
        \\et, par le moyen des prophètes,
        je dirai des paraboles.
${}^{12}Si Galaad est fausseté,
        \\ils ne sont que néant,
        ceux qui ont sacrifié des taureaux à Guilgal ;
        \\aussi leurs autels seront-ils comme des galets
        entassés sur les sillons des champs.
${}^{13}Jacob a fui dans le champ d’Aram,
        \\Israël a servi pour une femme,
        pour une femme il a gardé les troupeaux.
${}^{14}Mais par un prophète
        \\le Seigneur fit monter Israël d’Égypte,
        et, par un prophète, Israël a été gardé.
${}^{15}Éphraïm a irrité le Seigneur jusqu’à l’amertume :
        \\sur lui sera rejeté le sang versé,
        et son Maître lui rendra ses outrages.
      
         
      \bchapter{}
${}^{1}Quand Éphraïm parlait, c’était la terreur,
        car lui, il était chef en Israël.
        \\Mais il s’est compromis avec Baal
        et il en est mort.
${}^{2}À présent, ils continuent de pécher,
        ils se font des images de métal fondu,
        \\des idoles avec leur argent et par habileté :
        œuvre d’artisans que tout cela !
        \\C’est à leur propos que l’on dit :
        \\« Eux qui sacrifient des hommes,
        ils vénèrent des veaux. »
${}^{3}C’est pourquoi ils seront comme la brume du matin
        et comme la rosée d’aurore qui s’en va,
        \\comme la paille emportée loin de l’aire à grain
        et comme la fumée qui sort de la cheminée.
${}^{4}Et moi, je suis le Seigneur, ton Dieu
        depuis la sortie du pays d’Égypte ;
        \\tu ne connaîtras pas d’autre dieu que moi,
        hors moi, pas de sauveur.
${}^{5}Moi, je t’ai connu au désert,
        au pays de l’aridité.
${}^{6}Arrivés au pâturage, ils se sont rassasiés ;
        ils se sont rassasiés, et leur cœur s’est enorgueilli ;
        aussi m’ont-ils oublié.
${}^{7}Je serai pour eux comme un lion ;
        comme un léopard sur le chemin, je serai à l’affût.
${}^{8}Je vais les attaquer comme une ourse
        à qui l’on a ravi ses petits,
        \\je vais déchirer l’enveloppe de leur cœur,
        \\comme une lionne je vais les dévorer sur place,
        une bête sauvage les mettra en pièces.
${}^{9}Te voilà détruit, Israël,
        alors que ton secours est en moi !
${}^{10}Où donc est ton roi,
        pour qu’il te sauve dans toutes tes villes ?
        \\Où sont tes juges, à qui tu as dit :
        « Donne-moi un roi et des princes » ?
${}^{11}Je te donne un roi dans ma colère
        et, dans ma fureur, je le reprends.
${}^{12}La faute d’Éphraïm est tenue en lieu sûr,
        son péché est mis en réserve.
${}^{13}Les douleurs de celle qui enfante lui viendront ;
        mais lui, c’est un fils stupide :
        le moment venu, il ne quitte pas le sein maternel !
${}^{14}Vais-je les libérer de l’emprise des enfers,
        les racheter de la mort ?
        \\Mort, où est ta pestilence ?
        Enfers, où est votre fléau ?
        \\Toute consolation se dérobe à mes yeux.
${}^{15}Éphraïm a beau prospérer au milieu de ses frères,
        un vent d’est viendra,
        souffle du Seigneur montant du désert.
        \\Il tarira sa source, desséchera sa fontaine,
        il pillera le trésor, tous les objets précieux.
        
           
       
      
         
      \bchapter{}
${}^{1}Elle s’est rendue coupable, Samarie,
        \\car elle s’est rebellée contre son Dieu.
        \\Ils tomberont par l’épée,
        leurs nourrissons seront écrasés,
        et leurs femmes enceintes, éventrées.
        
           
        ${}^{2}Reviens, Israël, au Seigneur ton Dieu ;
        car tu t’es effondré par suite de tes fautes.
        ${}^{3}Revenez au Seigneur
        en lui présentant ces paroles :
        \\« Enlève toutes les fautes, et accepte ce qui est bon.
        \\Au lieu de taureaux, nous t’offrons en sacrifice
        les paroles de nos lèvres.
        ${}^{4}Puisque les Assyriens ne peuvent pas nous sauver,
        nous ne monterons plus sur des chevaux,
        \\et nous ne dirons plus à l’ouvrage de nos mains :
        “Tu es notre Dieu”,
        car de toi seul l’orphelin reçoit de la tendresse. »
         
        ${}^{5}Voici la réponse du Seigneur\\ :
        \\Je les guérirai de leur infidélité,
        je les aimerai d’un amour gratuit,
        car ma colère s’est détournée d’Israël\\.
        ${}^{6}Je serai pour Israël comme la rosée,
        il fleurira comme le lis,
        il étendra ses racines comme les arbres du Liban.
        ${}^{7}Ses jeunes pousses vont grandir,
        sa parure sera comme celle de l’olivier,
        son parfum, comme celui de la forêt du Liban.
        ${}^{8}Ils reviendront s’asseoir à son ombre,
        ils feront revivre le froment,
        \\ils fleuriront comme la vigne,
        ils seront renommés comme le vin du Liban.
        ${}^{9}Éphraïm ! Peux-tu me confondre avec les idoles ?
        C’est moi qui te réponds et qui te regarde.
        \\Je suis comme le cyprès toujours vert,
        c’est moi qui te donne ton fruit.
        ${}^{10}Qui donc est assez sage
        pour comprendre ces choses,
        assez pénétrant pour les saisir ?
        \\Oui, les chemins du Seigneur sont droits :
        les justes y avancent,
        \\mais les pécheurs\\y trébuchent.
