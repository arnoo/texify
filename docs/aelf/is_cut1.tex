  
  
    
    \bbook{ISAÏE}{ISAÏE}
      
         
      \bchapter{}
      \begin{verse}
${}^{1}Vision d’Isaïe, fils d’Amots, – ce qu’il a vu au sujet de Juda et de Jérusalem, au temps d’Ozias, de Yotam, d’Acaz et d’Ézékias, rois de Juda.
      
         
        ${}^{2}Cieux, écoutez ;
        \\terre, prête l’oreille,
        car le Seigneur a parlé.
        \\J’ai fait grandir des enfants, je les ai élevés,
        mais ils se sont révoltés contre moi.
        ${}^{3}Le bœuf connaît son propriétaire,
        et l’âne, la crèche de son maître.
        \\Israël ne le connaît pas,
        mon peuple ne comprend pas.
        ${}^{4}Malheur à vous, nation pécheresse, peuple chargé de fautes,
        engeance de malfaiteurs, fils pervertis !
        \\Ils abandonnent le Seigneur,
        ils méprisent le Saint d’Israël,
        ils lui tournent le dos.
        ${}^{5}Où donc faut-il vous frapper encore,
        vous qui multipliez les reniements ?
        \\Toute la tête est malade,
        tout le cœur est atteint ;
        ${}^{6}de la plante des pieds à la tête,
        plus rien n’est intact :
        \\partout blessures, contusions, plaies ouvertes,
        qui ne sont ni pansées, ni bandées,
        ni soignées avec de l’huile.
${}^{7}Votre pays n’est que désolation,
        vos villes sont consumées par le feu ;
        \\votre terre, des étrangers la dévorent sous vos yeux,
        c’est une désolation, comme un désastre venu des étrangers.
         
${}^{8}Ce qui reste de la fille de Sion
        est comme une hutte dans une vigne,
        \\comme un abri dans un potager,
        comme une ville assiégée.
${}^{9}Si le Seigneur de l’univers ne nous avait laissé un petit reste,
        nous serions comme Sodome,
        nous ressemblerions à Gomorrhe.
        ${}^{10}Écoutez la parole du Seigneur,
        vous qui êtes pareils\\aux chefs de Sodome\\ !
        \\Prêtez l’oreille à l’enseignement de notre Dieu,
        vous, peuple de Gomorrhe !
        ${}^{11}Que m’importe le nombre de vos sacrifices ?
        – dit le Seigneur.
        \\Les holocaustes de béliers, la graisse des veaux,
        j’en suis rassasié.
        \\Le sang des taureaux, des agneaux et des boucs,
        je n’y prends pas plaisir.
        ${}^{12}Quand vous venez vous présenter devant ma face,
        qui vous demande de fouler mes parvis ?
        ${}^{13}Cessez d’apporter de vaines offrandes ;
        j’ai horreur de votre encens\\.
        \\Les nouvelles lunes, les sabbats, les assemblées,
        je n’en peux plus de ces crimes et de ces fêtes.
        ${}^{14}Vos nouvelles lunes et vos solennités,
        moi, je les déteste :
        \\elles me sont un fardeau,
        je suis fatigué de le porter.
        ${}^{15}Quand vous étendez les mains,
        je détourne les yeux.
        \\Vous avez beau multiplier les prières,
        je n’écoute pas :
        vos mains sont pleines de sang.
        ${}^{16}Lavez-vous, purifiez-vous,
        ôtez de ma vue vos actions mauvaises,
        cessez de faire le mal.
        ${}^{17}Apprenez à faire le bien :
        recherchez le droit,
        \\mettez au pas l’oppresseur,
        rendez justice\\à l’orphelin,
        \\défendez la cause de la veuve.
         
        ${}^{18}Venez, et discutons – dit le Seigneur.
        \\Si vos péchés sont comme l’écarlate,
        ils deviendront aussi blancs que neige.
        \\S’ils sont rouges comme le vermillon,
        ils deviendront comme de la laine.
        ${}^{19}Si vous consentez à m’obéir,
        les bonnes choses du pays, vous les mangerez ;
        ${}^{20}mais si vous refusez, si vous vous obstinez,
        c’est l’épée qui vous mangera.
        – Oui, la bouche du Seigneur a parlé.
${}^{21}Comment ! Elle s’est prostituée,
        la cité fidèle !
        \\Le droit y régnait, la justice l’habitait,
        et maintenant, ce sont les meurtriers.
${}^{22}Ton argent n’est plus que scories,
        ton meilleur vin est mêlé d’eau.
${}^{23}Tes princes sont des rebelles,
        complices de voleurs,
        \\tous avides de cadeaux,
        courant les pots-de-vin ;
        \\ils ne rendent pas justice à l’orphelin,
        la cause de la veuve ne les touche pas.
${}^{24}Voilà pourquoi
        – oracle du Maître et Seigneur de l’univers,
        Force d’Israël – :
        \\Malheur ! Je prendrai ma revanche sur mes adversaires,
        je me vengerai de mes ennemis.
${}^{25}Je ramènerai ma main sur toi ;
        comme le fait la potasse, j’ôterai tes scories,
        j’enlèverai tous tes déchets.
${}^{26}Je rendrai tes juges tels que jadis,
        tes conseillers comme autrefois.
        \\Alors on t’appellera « Ville de justice »,
        « Cité fidèle ».
${}^{27}Par le droit, Sion sera délivrée ;
        ils le seront par la justice,
        ceux des siens qui se convertiront.
${}^{28}Mais rebelles et pécheurs, ensemble, seront brisés !
        Ceux qui abandonnent le Seigneur périront.
${}^{29}Oui, vous aurez honte des térébinthes,
        ces bosquets sacrés que vous chérissez,
        vous rougirez des jardins que vous préférez,
${}^{30}car vous serez comme un térébinthe au feuillage flétri,
        comme un jardin sans eau.
${}^{31}Le colosse deviendra comme de l’étoupe,
        et son ouvrage, une étincelle :
        \\les deux flamberont ensemble,
        et personne pour éteindre.
      <h2 class="intertitle hmbot" id="d85e237151">1. Sur Juda et Jérusalem (2 – 5)</h2>
      <p class="cantique" id="bib_ct-at_17"><span class="cantique_label">Cantique AT 17</span> = <span class="cantique_ref"><a class="unitex_link" href="#bib_is_2_2">Is 2, 2-5</a></span>
      
         
      \bchapter{}
      \begin{verse}
${}^{1}Parole d’Isaïe, fils d’Amots, – ce qu’il a vu au sujet de Juda et de Jérusalem.
      
         
        ${}^{2}Il arrivera dans les derniers jours
        \\que la montagne de la maison du Seigneur
        se tiendra plus haut que les monts\\, *
        s’élèvera au-dessus des collines.
        \\Vers elle afflueront toutes les nations
        ${}^{3}et viendront des peuples nombreux.
         
        \\Ils diront : « Venez !
        montons à la montagne du Seigneur, *
        à la maison du Dieu de Jacob !
        \\Qu’il nous enseigne ses chemins,
        et nous irons par ses sentiers. »
        \\Oui, la loi sortira de Sion,
        et de Jérusalem, la parole du Seigneur.
         
        ${}^{4}Il sera juge entre les nations
        et l’arbitre de peuples nombreux.
        \\De leurs épées, ils forgeront des socs,
        et de leurs lances, des faucilles.
        \\Jamais nation contre nation
        ne lèvera l’épée ;
        \\ils n’apprendront plus la guerre.
         
        ${}^{5}Venez, maison de Jacob !
        \\Marchons à la lumière du Seigneur.
${}^{6}Oui, tu as délaissé ton peuple,
        la maison de Jacob,
        \\car ils sont remplis des superstitions de l’Orient,
        ils exercent la divination comme les Philistins,
        ils applaudissent aux pratiques étrangères.
${}^{7}Le pays est rempli d’or et d’argent,
        on ne peut compter ses trésors !
        \\Le pays est rempli de chevaux,
        on ne peut compter ses chars !
${}^{8}Le pays est rempli de faux dieux :
        les gens se prosternent devant l’ouvrage de leurs mains,
        devant ce que leurs doigts ont fabriqué.
${}^{9}L’être humain sera humilié,
        l’homme sera abaissé,
        tu ne saurais lui pardonner.
         
${}^{10}Entre dans les rochers,
        cache-toi dans la poussière,
        \\épouvanté, loin du Seigneur,
        loin de l’éclat de sa majesté.
${}^{11}Les regards arrogants des humains seront abaissés,
        et la prétention des hommes sera humiliée.
        \\Seul le Seigneur sera exalté
        en ce jour-là.
${}^{12}Oui, pour le Seigneur de l’univers, il y aura un jour
        contre tout orgueil et toute prétention,
        contre tout ce qui s’élève et sera abaissé,
${}^{13}contre tous les cèdres du Liban, prétentieux et altiers,
        contre tous les chênes du Bashane,
${}^{14}contre toute haute montagne,
        et toute colline élevée,
${}^{15}contre toutes les tours arrogantes,
        et tout rempart fortifié,
${}^{16}contre tout vaisseau de Tarsis,
        et tout navire de grand prix.
${}^{17}L’arrogance des humains sera humiliée ;
        la prétention des hommes sera abaissée.
        \\Seul le Seigneur sera exalté
        en ce jour-là.
${}^{18}Et les faux dieux, tous à la fois, disparaîtront.
         
${}^{19}Entrez dans les cavernes des rochers,
        dans les grottes souterraines,
        \\épouvantés, loin du Seigneur,
        loin de l’éclat de sa majesté,
        quand il se dressera pour terrifier la terre.
         
${}^{20}Ce jour-là, les hommes jetteront
        les faux dieux d’or et d’argent
        \\qu’ils s’étaient fabriqués pour les adorer ;
        ils les jetteront aux taupes et aux chauves-souris.
${}^{21}Eux, ils entreront dans les creux des rochers
        et dans les fentes des falaises,
        \\épouvantés, loin du Seigneur,
        loin de l’éclat de sa majesté,
        quand il se dressera pour terrifier la terre.
${}^{22}Cessez de vous appuyer sur l’être humain :
        sa vie tient à un souffle ;
        et quelle est sa valeur ?
      
         
      \bchapter{}
${}^{1}Voici que le Maître et Seigneur de l’univers
        \\va retirer de Jérusalem et de Juda
        réserves et ressources,
        \\toute réserve de pain,
        toute réserve d’eau,
${}^{2}le héros et l’homme de guerre,
        le juge et le prophète,
        \\le devin et l’ancien,
${}^{3}l’officier, le notable,
        \\le conseiller, l’expert en magie,
        et le charmeur habile.
${}^{4}Je leur donne pour princes des gamins
        dont le caprice les gouvernera.
${}^{5}Les gens seront des tyrans les uns pour les autres,
        chacun pour son prochain ;
        \\le gamin s’en prendra à l’ancien,
        et le vaurien, au vénérable.
${}^{6}Un individu se saisira de son frère
        dans la maison paternelle, en disant :
        \\« Tu as un manteau : tu seras notre chef !
        Ce pays en ruine, gouverne-le ! » ;
${}^{7}ce jour-là, l’autre répliquera :
        « Je ne suis pas un guérisseur !
        \\Il n’y a, dans ma maison, ni pain ni manteau,
        ne me faites pas chef du peuple ! »
${}^{8}Oui, Jérusalem trébuche
        et Juda s’écroule,
        \\parce que leurs paroles et leurs actes
        envers le Seigneur
        sont des insultes au regard de sa gloire.
${}^{9}Leur partialité témoigne contre eux ;
        comme Sodome, ils étalent leur péché,
        ils n’en cachent rien.
        \\Hélas pour eux !
        Ils font leur propre malheur.
        
           
         
${}^{10}Dites : « Qu’il est bon pour le juste
        de se nourrir du fruit de ses actes !
${}^{11}Quel malheur, hélas, pour le méchant
        d’être traité selon l’œuvre de ses mains ! »
        
           
         
${}^{12}Ô mon peuple ! Un nourrisson le tyrannise !
        Des femmes le gouvernent !
        \\Ô mon peuple, tes guides te fourvoient ;
        ils brouillent le tracé de tes chemins.
${}^{13}Le Seigneur s’est levé pour accuser,
        il est debout pour juger les peuples.
        
           
${}^{14}Le Seigneur entre en jugement
        avec les anciens du peuple et ses princes :
        \\« C’est vous qui dévastez la vigne ;
        les biens volés au pauvre emplissent vos maisons.
${}^{15}De quel droit écrasez-vous mon peuple,
        piétinez-vous le visage du pauvre ? »
        – oracle du Seigneur, Dieu de l’univers.
${}^{16}Le Seigneur le déclare :
        \\Parce que les filles de Sion sont orgueilleuses,
        qu’elles marchent la poitrine haute
        et les yeux provocants,
        \\qu’elles avancent à petits pas
        en faisant sonner les anneaux de leurs pieds,
${}^{17}le Seigneur rendra chauve
        le crâne des filles de Sion,
        le Seigneur dénudera leur front.
${}^{18}Ce jour-là, le Seigneur ôtera leurs parures :
        grelots et médaillons,
        <p class="verset_anchor" id="para_bib_is_3_19">pendentifs
${}^{19}et boucles d’oreilles,
        bracelets et voilettes,
${}^{20}diadèmes et chaînes de chevilles,
        cordelières, porte-bonheur et amulettes,
${}^{21}bagues et anneaux de narines,
${}^{22}robes de fête et mantilles,
        \\châles et petits sacs,
${}^{23}miroirs, linges fins, turbans et capes légères.
${}^{24}Au lieu de parfum, la puanteur ;
        au lieu de ceinture, une corde ;
        \\au lieu de coiffure bouclée, un crâne tondu ;
        au lieu d’une robe d’apparat, un pagne en toile à sac ;
        \\une marque au fer rouge au lieu de la beauté.
         
${}^{25}Tes hommes tomberont sous l’épée,
        et ton élite, dans le combat.
${}^{26}Elles gémiront, elles se lamenteront, les portes de la ville,
        qui sera désertée, assise par terre.
       
      
         
      \bchapter{}
${}^{1}Sept femmes saisiront un même homme,
        ce jour-là, et lui diront :
        \\« Nous mangerons de notre pain,
        nous mettrons nos propres habits ;
        \\laisse-nous seulement porter ton nom :
        enlève notre déshonneur. »
        
           
        ${}^{2}Ce jour-là,
        le Germe que fera grandir le Seigneur
        \\sera l’honneur et la gloire des rescapés d’Israël,
        le Fruit de la terre sera leur fierté et leur splendeur.
        ${}^{3}Alors, ceux qui seront restés dans Sion,
        les survivants de Jérusalem,
        seront appelés saints :
        \\tous seront inscrits à Jérusalem
        pour y vivre.
        ${}^{4}Quand le Seigneur aura lavé la souillure des filles de Sion,
        purifié Jérusalem du sang répandu,
        \\en y faisant passer le souffle\\du jugement,
        un souffle d’incendie,
        ${}^{5}alors, sur toute la montagne\\de Sion,
        sur les assemblées qui s’y tiennent,
        \\le Seigneur créera
        une nuée pendant le jour
        \\et, pendant la nuit, une fumée
        avec un feu de flammes éclatantes.
        \\Et au-dessus de tout,
        comme un dais, la gloire du Seigneur\\ :
        ${}^{6}elle sera, contre la chaleur du jour, l’ombre d’une hutte,
        un refuge, un abri contre l’orage et la pluie.
      
         
      \bchapter{}
        ${}^{1}Je veux chanter pour mon ami
        le chant du bien-aimé\\à sa vigne.
        \\Mon ami avait une vigne
        sur un coteau fertile.
        ${}^{2}Il en retourna la terre, en retira les pierres,
        pour y mettre un plant de qualité.
        \\Au milieu, il bâtit une tour de garde
        et creusa aussi un pressoir.
        \\Il en attendait de beaux raisins,
        mais elle en donna de mauvais.
        
           
         
        ${}^{3}Et maintenant, habitants de Jérusalem, hommes de Juda,
        soyez donc juges entre moi et ma vigne !
        ${}^{4}Pouvais-je faire pour ma vigne
        plus que je n’ai fait ?
        \\J’attendais de beaux raisins,
        pourquoi en a-t-elle donné de mauvais ?
        ${}^{5}Eh bien, je vais vous apprendre
        ce que je ferai de ma vigne :
        \\enlever sa clôture
        pour qu’elle soit dévorée par les animaux,
        \\ouvrir une brèche dans son mur
        pour qu’elle soit piétinée.
        ${}^{6}J’en ferai une pente désolée ;
        elle ne sera ni taillée ni sarclée,
        il y poussera des épines et des ronces ;
        \\j’interdirai aux nuages
        d’y faire tomber la pluie.
        ${}^{7}La vigne du Seigneur de l’univers,
        c’est la maison d’Israël.
        \\Le plant qu’il chérissait,
        ce sont les hommes de Juda.
        \\Il en attendait le droit,
        et voici le crime ;
        \\il en attendait la justice,
        et voici les cris.
        
           
${}^{8}Malheureux, vous qui ajoutez maison à maison,
        qui joignez champ à champ,
        \\jusqu’à occuper toute la place
        et habiter, seuls, au milieu du pays !
${}^{9}J’ai entendu le serment du Seigneur de l’univers :
        \\De nombreuses maisons seront ruinées,
        belles ou grandes, elles seront inhabitées.
${}^{10}Dix arpents de vignes produiront un seul tonneau,
        et dix boisseaux de semence, un seul boisseau.
         
${}^{11}Malheureux, ceux qui, dès le petit matin,
        courent après la boisson forte
        \\et que le vin échauffe encore,
        tard dans la soirée !
${}^{12}Ce ne sont que cithares et harpes,
        tambourins et flûtes,
        et vin pour leurs beuveries.
        \\Mais sur l’œuvre du Seigneur
        ils n’ont pas un regard ;
        \\ce qu’il fait de ses mains,
        ils ne le voient pas.
${}^{13}Voilà pourquoi mon peuple est en exil,
        faute de n’avoir rien compris ;
        \\son élite meurt de faim,
        ses foules sont dévorées de soif.
${}^{14}Voilà pourquoi la Mort dilate sa gorge
        et ouvre démesurément la gueule :
        \\la noblesse et la foule y descendent
        dans le vacarme de la fête.
${}^{15}L’être humain devra plier,
        l’homme sera abaissé,
        \\les arrogants baisseront les yeux.
${}^{16}Le Seigneur de l’univers
        est exalté par ce jugement,
        \\et le Dieu saint,
        sanctifié par cet acte de justice.
${}^{17}Des agneaux viendront brouter
        comme à leur pâturage
        \\et, au milieu des ruines,
        des chevreaux s’engraisseront.
${}^{18}Malheureux, ceux qui traînent le péché
        avec la corde du mensonge,
        \\et la faute
        comme avec des rênes de chariot !
${}^{19}Ils disent : « Que Dieu se dépêche !
        Qu’il fasse vite son œuvre,
        que nous puissions voir !
        \\Le projet du Saint d’Israël,
        qu’il avance et se réalise,
        alors nous comprendrons ! »
${}^{20}Malheureux, ces gens qui déclarent bien ce qui est mal,
        et mal ce qui est bien,
        \\qui font des ténèbres la lumière
        et de la lumière les ténèbres,
        \\qui rendent amer ce qui est doux
        et doux ce qui est amer !
${}^{21}Malheureux, ceux qui se prennent pour des sages,
        ceux qui se croient intelligents !
${}^{22}Malheureux, ceux qui sont champions pour boire du vin,
        experts en mélange des boissons fortes :
${}^{23}ils acquittent le coupable contre un cadeau,
        ils privent les innocents de leur justice !
${}^{24}Voilà pourquoi, comme la paille est dévorée par le feu,
        et le foin disparaît dans les flammes,
        \\leurs racines ne seront plus que puanteur,
        et leurs fleurs partiront en poussière :
        \\oui, ils ont rejeté la loi du Seigneur de l’univers,
        méprisé la parole du Saint d’Israël.
${}^{25}Voilà pourquoi la colère du Seigneur
        s’est enflammée contre son peuple :
        \\sa main s’est levée contre lui,
        et il l’a frappé,
        \\et les montagnes ont tremblé,
        et ses cadavres sont comme des ordures au milieu des rues.
        \\Et avec tout cela, sa colère ne s’est pas détournée,
        sa main reste levée.
         
${}^{26}Il dresse l’étendard vers une nation lointaine,
        il siffle pour l’appeler des extrémités de la terre,
        \\et la voici, rapide et alerte,
        qui vient.
${}^{27}Aucun ne flanche, aucun ne trébuche,
        aucun ne somnole, aucun ne s’endort ;
        \\pas une ceinture qui tombe des reins,
        pas une courroie de sandale qui se rompe.
${}^{28}Ses flèches sont aiguisées,
        tous ses arcs, tendus ;
        \\les sabots de ses chevaux, comme du silex,
        et les roues de ses chars, un cyclone.
${}^{29}Elle rugit comme une lionne,
        elle rugit comme les fauves,
        \\elle gronde et saisit sa proie,
        elle l’emporte, et personne qui délivre.
${}^{30}Ce jour-là, Dieu grondera contre son peuple
        comme gronde la mer.
        \\Il regardera la terre : ténèbres de détresse,
        lumière que la brume obscurcit.
