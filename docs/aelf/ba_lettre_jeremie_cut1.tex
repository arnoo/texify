  
  
    
    \bbook{LETTRE DE JÉRÉMIE}{LETTRE DE JÉRÉMIE}
      Copie de la lettre que Jérémie envoya à ceux qui allaient être emmenés captifs à Babylone par le roi des Babyloniens, pour leur annoncer ce que Dieu lui avait prescrit.
${}^{1}À cause des péchés que vous avez commis devant Dieu, vous allez être emmenés captifs à Babylone par Nabucodonosor, roi des Babyloniens. 
${}^{2}Une fois arrivés à Babylone, vous y resterez bien des années, un temps très long, jusqu’à la septième génération. Mais après cela, je vous en ferai sortir en paix. 
${}^{3}À Babylone, vous verrez des dieux d’argent, d’or et de bois, que l’on porte sur les épaules et qui inspirent la crainte aux nations païennes. 
${}^{4}Soyez donc sur vos gardes : ne devenez pas, vous aussi, pareils aux étrangers ; ne vous laissez pas envahir par la crainte envers ces dieux, 
${}^{5}lorsque vous verrez la foule se prosterner devant et derrière eux. Dites-vous plutôt en pensée : « C’est devant toi, ô Maître, qu’il faut se prosterner. » 
${}^{6}Car mon ange est avec vous, lui qui demandera compte de vos âmes.
${}^{7}La langue de ces dieux est poncée par un artisan, elle est recouverte d’or et d’argent, mais ils ne sont que mensonge, ils ne peuvent parler. 
${}^{8}Comme pour une jeune fille qui aime à se parer, ces gens prennent de l’or 
${}^{9}dont ils fabriquent des couronnes pour les têtes de leurs dieux. Il arrive même parfois que les prêtres dérobent à leurs dieux de l’or et de l’argent pour leurs propres dépenses, 
${}^{10}et qu’ils en donnent aussi aux prostituées sacrées. Ces dieux d’argent, d’or et de bois, on les pare de vêtements comme des hommes, 
${}^{11}eux qui ne peuvent se défendre ni de la rouille, ni des mites. Ils sont revêtus d’un habit de pourpre, 
${}^{12}mais on leur essuie le visage, à cause de la poussière du temple qui s’accumule sur eux. 
${}^{13}L’un porte un sceptre, comme le juge d’un pays, mais il ne peut faire périr celui qui l’offense. 
${}^{14}Un autre tient un poignard dans la main droite et une hache, mais il ne peut se défendre ni de la guerre ni des brigands. Voilà pourquoi il est clair que ce ne sont pas des dieux : ne les craignez donc pas !
${}^{15}Comme l’outil d’un homme devient inutile quand il se brise, 
${}^{16}ainsi en va-t-il de leurs dieux : à peine sont-ils installés dans leurs temples, que leurs yeux se remplissent de la poussière soulevée par les pieds de ceux qui entrent. 
${}^{17}Comme on verrouille les portes de toutes parts sur un homme qui a offensé le roi, en vue de le conduire à la mort, ainsi les prêtres font-ils de leurs temples des forteresses, avec des portes munies de verrous et de barres, par crainte d’un pillage de voleurs. 
${}^{18}Ils allument des lampes, bien plus que pour eux-mêmes, alors que ces dieux sont incapables d’en voir une seule. 
${}^{19}Ils sont comme l’une des poutres du temple dont on dit que le cœur est rongé : ces dieux ne remarquent pas les insectes qui, sortant de terre, les dévorent, ainsi que leurs vêtements. 
${}^{20}Leurs visages sont noircis par la fumée, qui s’élève dans le temple. 
${}^{21}Au-dessus de leurs corps et de leurs têtes volent des chauves-souris, des hirondelles et d’autres oiseaux ; il y a même des chats. 
${}^{22}Voilà pourquoi vous comprendrez que ce ne sont pas des dieux : ne les craignez donc pas !
${}^{23}L’or dont on les recouvre pour les rendre beaux, si personne n’en essuie la surface ternie, ce n’est certes pas eux qui le feront briller, car même quand on les fondait, ils ne sentaient rien. 
${}^{24}On les achète à un prix exorbitant, eux qui n’ont pas le moindre souffle. 
${}^{25}Comme ils n’ont pas de pieds, on les porte sur les épaules. Ils montrent ainsi aux hommes ce qui fait leur déshonneur. Même ceux qui les servent en éprouvent de la honte, 
${}^{26}car pour les empêcher de tomber à terre, ils les soutiennent. Si quelqu’un les met debout, ils ne se déplaceront pas d’eux-mêmes ; s’ils penchent, ils ne peuvent pas davantage se redresser. Mais c’est comme à des morts qu’on leur présente des offrandes. 
${}^{27}Les victimes qui leur sont offertes, les prêtres les revendent et en tirent profit ; de même, leurs femmes aussi en prennent une partie pour les saler, sans rien distribuer au pauvre ou à l’infirme. 
${}^{28}La femme en état d’impureté et la femme qui vient d’accoucher ont un contact avec ces offrandes ! Sachant donc par là que ce ne sont pas des dieux, ne les craignez pas !
${}^{29}D’où vient qu’on les appelle des dieux, alors que des femmes présentent des offrandes à ces dieux d’argent, d’or et de bois ? 
${}^{30}Les prêtres siègent dans leurs temples, avec des habits déchirés, les cheveux et la barbe rasés, la tête découverte. 
${}^{31}Ils poussent des hurlements et crient devant leurs dieux, comme des gens qui prennent part à un repas funèbre. 
${}^{32}Ils enlèvent les vêtements des dieux pour en habiller leurs femmes et leurs enfants. 
${}^{33}Quelqu’un fait-il du mal ou du bien à ces dieux, ils sont incapables de le rendre ; incapables aussi d’établir ou de destituer un roi. 
${}^{34}De même, ils ne pourront donner ni richesse ni argent. Si quelqu’un ne s’acquitte pas d’un vœu qu’il leur a fait, jamais ils ne le réclameront. 
${}^{35}Ils n’arracheront pas un homme à la mort, ils ne délivreront pas le faible des mains d’un puissant. 
${}^{36}À l’aveugle, ils ne pourront rendre la vue, ni délivrer l’homme en détresse. 
${}^{37}Ils n’éprouveront aucune compassion pour la veuve, ils ne seront d’aucun secours à l’orphelin. 
${}^{38}Ils sont pareils aux pierres extraites de la montagne, ces objets de bois, recouverts d’or et d’argent ; et ceux qui les servent en seront pleins de honte. 
${}^{39}Comment donc peut-on penser ou proclamer que ce sont des dieux ?
${}^{40}Bien plus, les Chaldéens eux-mêmes les déshonorent, car lorsqu’ils voient un muet qui ne peut parler, ils le présentent à Bel et demandent qu’il recouvre la voix, comme si le dieu était capable de percevoir quoi que ce soit. 
${}^{41}Eux-mêmes sont incapables, même quand ils réfléchissent, d’abandonner ces dieux, tant le bon sens leur manque ! 
${}^{42}Quant aux femmes, elles s’entourent d’une corde et s’installent sur les chemins pour brûler du son comme une fumée d’encens. 
${}^{43}Quand l’une d’entre elles, entraînée par quelque passant, a couché avec lui, elle se moque de sa voisine, qui n’a pas obtenu le même honneur, et dont la corde n’a pas été rompue. 
${}^{44}Ce n’est que mensonge, tout ce qui se fait pour eux. Comment donc peut-on aller jusqu’à penser ou proclamer que ce sont des dieux ?
${}^{45}Fabriqués par des menuisiers et des orfèvres, ils ne sauraient devenir autre chose que ce que veulent ces artisans. 
${}^{46}Et ceux qui les fabriquent, eux-mêmes ne vivront pas longtemps. 
${}^{47}Comment donc les objets de leur fabrication seraient-ils des dieux ? Oui, ces hommes n’auront laissé que mensonge et dérision à leurs descendants. 
${}^{48}Quand une guerre ou des malheurs s’abattent sur ces dieux, les prêtres délibèrent pour savoir où se cacher avec eux. 
${}^{49}Comment donc ne pas percevoir qu’ils ne sont pas des dieux, alors qu’ils ne se sauvent eux-mêmes ni de la guerre ni des malheurs ? 
${}^{50}Ces objets de bois, recouverts d’or ou d’argent, on comprendra plus tard qu’ils ne sont que mensonge. Aux yeux de tous, peuples et rois, il sera évident que ce ne sont pas des dieux, mais des ouvrages de main d’homme, et qu’en eux, rien n’est l’œuvre d’un dieu. 
${}^{51}Pour qui donc n’est-il pas clair que ce ne sont pas des dieux ?
${}^{52}En effet, ils ne sauraient instituer un roi dans un pays, ni donner la pluie aux hommes, 
${}^{53}ni rendre un jugement concernant leur propre cause, ni défendre un opprimé. Ils ne sont bons à rien, 
${}^{54}pareils à des corneilles entre ciel et terre. Oui, si le feu tombait sur le temple de ces dieux de bois, recouverts d’or et d’argent, leurs prêtres prendraient la fuite et seraient sauvés, mais eux seraient entièrement consumés, comme des poutres au milieu des flammes. 
${}^{55}Ils ne peuvent opposer de résistance ni à un roi ni à des ennemis. 
${}^{56}Comment donc admettre ou penser que ce sont des dieux ?
      Ils ne sauraient se protéger ni des voleurs ni des brigands, les dieux de bois, recouverts d’argent et d’or ; 
${}^{57}ces hommes forts leur arracheront l’or et l’argent, et s’en iront en emportant les habits qui les couvraient, sans que ces dieux ne puissent se porter secours à eux-mêmes. 
${}^{58}Aussi vaut-il mieux être un roi déployant son courage ou, dans une maison, un outil efficace dont fait usage son propriétaire, que d’être ces faux dieux. Mieux vaut être une porte dans une maison, protégeant ce qui s’y trouve, un pilier de bois dans un palais royal, que d’être ces faux dieux. 
${}^{59}Car le soleil, la lune et les étoiles, qui brillent et ont mission de servir, obéissent. 
${}^{60}De même, l’éclair, beau à voir quand il jaillit, et le vent lui-même qui souffle en tout pays, 
${}^{61}et les nuages, quand Dieu leur ordonne de parcourir toute la terre, tous ils accomplissent son ordre. 
${}^{62}Et le feu, envoyé d’en haut pour dévaster montagnes et forêts, fait, lui aussi, ce qui est commandé. Or, ces dieux ne leur sont comparables ni en beauté, ni en puissance. 
${}^{63}Voilà pourquoi il ne faut pas penser ni proclamer que ce sont des dieux, eux qui ne sont pas capables de rendre un jugement ni de faire du bien aux hommes. 
${}^{64}Sachant donc que ce ne sont pas des dieux, ne les craignez pas !
${}^{65}Ils ne peuvent en effet ni maudire ni bénir les rois. 
${}^{66}Ils ne peuvent montrer parmi les nations des signes dans le ciel, ni resplendir comme le soleil ou briller comme la lune. 
${}^{67}Les bêtes sauvages leur sont supérieures, elles qui peuvent fuir vers une tanière pour se mettre elles-mêmes à l’abri. 
${}^{68}D’aucune manière donc il n’est évident pour nous qu’ils sont des dieux. Aussi, ne les craignez pas !
${}^{69}Comme, dans un champ de concombres, un épouvantail qui ne garde rien, ainsi en est-il de leurs dieux de bois, recouverts d’or et d’argent. 
${}^{70}Un buisson d’épines dans un jardin où viennent se poser tous les oiseaux, un cadavre jeté dans un endroit ténébreux : voilà à quoi ressemblent leurs dieux de bois, recouverts d’or et d’argent. 
${}^{71}À voir se ternir sur eux la pourpre et l’éclat, vous comprendrez que ce ne sont pas des dieux. Ils finiront par être eux-mêmes dévorés et deviendront la risée du pays. 
${}^{72}Mieux vaut l’homme juste qui n’a pas d’idoles : il sera à l’abri de la dérision.
