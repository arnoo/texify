  
  
    
    \bbook{GENÈSE}{GENÈSE}
      <h2 class="intertitle" id="d85e4003">1. La création et la chute (1 – 5)</h2>
      
         
      \bchapter{}
        ${}^{1}Au commencement,
        Dieu créa le ciel et la terre.
        ${}^{2}La terre était informe et vide,
        les ténèbres étaient au-dessus de l’abîme
        et le souffle de Dieu planait au-dessus des eaux\\.
        
           
         
        ${}^{3}Dieu dit :
        « Que la lumière soit. »
        Et la lumière fut.
        ${}^{4}Dieu vit que la lumière était bonne,
        et Dieu sépara la lumière des ténèbres.
        ${}^{5}Dieu appela la lumière « jour »,
        il appela les ténèbres « nuit ».
        \\Il y eut un soir, il y eut un matin : premier jour.
        ${}^{6}Et Dieu dit :
        « Qu’il y ait un firmament au milieu des eaux,
        et qu’il sépare les eaux. »
        ${}^{7}Dieu fit le firmament,
        il sépara les eaux qui sont au-dessous du firmament
        et les eaux qui sont au-dessus.
        Et ce fut ainsi.
        ${}^{8}Dieu appela le firmament « ciel ».
        \\Il y eut un soir, il y eut un matin : deuxième jour.
        ${}^{9}Et Dieu dit :
        « Les eaux qui sont au-dessous du ciel,
        qu’elles se rassemblent en un seul lieu,
        et que paraisse la terre ferme\\. »
        Et ce fut ainsi.
        ${}^{10}Dieu appela la terre ferme « terre »,
        et il appela la masse des eaux « mer ».
        Et Dieu vit que cela était bon.
        ${}^{11}Dieu dit :
        « Que la terre produise l’herbe,
        la plante qui porte sa semence,
        et que, sur la terre, l’arbre à fruit donne,
        selon son espèce,
        le fruit qui porte sa semence. »
        Et ce fut ainsi.
        ${}^{12}La terre produisit l’herbe,
        la plante qui porte sa semence, selon son espèce,
        et l’arbre qui donne, selon son espèce,
        le fruit qui porte sa semence.
        Et Dieu vit que cela était bon.
        ${}^{13}Il y eut un soir, il y eut un matin : troisième jour.
        ${}^{14}Et Dieu dit :
        « Qu’il y ait des luminaires au firmament du ciel,
        pour séparer le jour de la nuit ;
        qu’ils servent de signes
        pour marquer les fêtes, les jours et les années ;
        ${}^{15}et qu’ils soient, au firmament du ciel,
        des luminaires pour éclairer la terre. »
        Et ce fut ainsi.
        ${}^{16}Dieu fit les deux grands luminaires :
        le plus grand pour commander au jour,
        le plus petit pour commander à la nuit ;
        \\il fit aussi les étoiles.
        ${}^{17}Dieu les plaça au firmament du ciel
        pour éclairer la terre,
        ${}^{18}pour commander au jour et à la nuit,
        pour séparer la lumière des ténèbres.
        Et Dieu vit que cela était bon.
        ${}^{19}Il y eut un soir, il y eut un matin : quatrième jour.
        ${}^{20}Et Dieu dit :
        « Que les eaux foisonnent
        d’une profusion d’êtres vivants,
        et que les oiseaux volent au-dessus de la terre,
        sous le\\firmament du ciel. »
        ${}^{21}Dieu créa, selon leur espèce,
        les grands monstres marins,
        tous les êtres vivants qui vont et viennent
        et foisonnent dans les eaux,
        et aussi, selon leur espèce,
        tous les oiseaux qui volent.
        Et Dieu vit que cela était bon.
        ${}^{22}Dieu les bénit par ces paroles :
        « Soyez féconds et multipliez-vous,
        remplissez les mers,
        que les oiseaux se multiplient sur la terre. »
        ${}^{23}Il y eut un soir, il y eut un matin : cinquième jour.
        ${}^{24}Et Dieu dit :
        « Que la terre produise des êtres vivants
        selon leur espèce,
        bestiaux, bestioles et bêtes sauvages
        selon leur espèce. »
        Et ce fut ainsi.
        ${}^{25}Dieu fit les bêtes sauvages selon leur espèce,
        les bestiaux selon leur espèce,
        et toutes les bestioles de la terre selon leur espèce.
        Et Dieu vit que cela était bon.
        ${}^{26}Dieu dit :
        « Faisons l’homme à notre image,
        selon notre ressemblance.
        Qu’il soit le maître
        des poissons de la mer, des oiseaux du ciel,
        des bestiaux, de toutes les bêtes sauvages\\,
        et de toutes les bestioles
        qui vont et viennent sur la terre. »
        
           
         
        ${}^{27}Dieu créa l’homme à son image,
        à l’image de Dieu il le créa,
        il les créa homme et femme.
        
           
         
        ${}^{28}Dieu les bénit et leur dit :
        « Soyez féconds et multipliez-vous,
        remplissez la terre et soumettez-la.
        Soyez les maîtres
        des poissons de la mer, des oiseaux du ciel,
        et de tous les animaux qui vont et viennent sur la terre. »
        ${}^{29}Dieu dit encore :
        « Je vous donne toute plante qui porte sa semence
        sur toute la surface de la terre,
        et tout arbre dont le fruit porte sa semence :
        telle sera votre nourriture.
        ${}^{30}À tous les animaux de la terre,
        à tous les oiseaux du ciel,
        à tout ce qui va et vient sur la terre
        et qui a souffle de vie,
        je donne\\comme nourriture toute herbe verte. »
        Et ce fut ainsi.
        ${}^{31}Et Dieu vit tout ce qu’il avait fait ;
        et voici : cela était très bon.
        \\Il y eut un soir, il y eut un matin : sixième jour.
        
           
       
      
         
      \bchapter{}
        ${}^{1}Ainsi furent achevés le ciel et la terre,
        et tout leur déploiement\\.
        ${}^{2}Le septième jour, Dieu avait achevé l’œuvre qu’il avait faite.
        \\Il se reposa, le septième jour,
        de toute l’œuvre qu’il avait faite.
        ${}^{3}Et Dieu bénit le septième jour : il le sanctifia
        puisque, ce jour-là, il se reposa de toute l’œuvre de création qu’il avait faite.
        ${}^{4}Telle fut l’origine du ciel et de la terre
        lorsqu’ils furent créés.
        
           
      Lorsque le Seigneur Dieu fit la terre et le ciel, 
${}^{5} aucun buisson n’était encore sur la terre, aucune herbe n’avait poussé, parce que le Seigneur Dieu n’avait pas encore fait pleuvoir sur la terre, et il n’y avait pas d’homme pour travailler le sol. 
${}^{6} Mais une source\\montait de la terre et irriguait toute la surface du sol. 
${}^{7} Alors le Seigneur Dieu modela l’homme avec la poussière tirée du sol ; il insuffla dans ses narines le souffle de vie\\, et l’homme devint un être vivant. 
${}^{8} Le Seigneur Dieu planta un jardin en Éden, à l’orient, et y plaça l’homme qu’il avait modelé. 
${}^{9} Le Seigneur Dieu fit pousser du sol toutes sortes d’arbres à l’aspect désirable et aux fruits savoureux\\ ; il y avait aussi l’arbre de vie au milieu du jardin, et l’arbre de la connaissance du bien et du mal.
${}^{10}Un fleuve sortait d’Éden pour irriguer le jardin ; puis il se divisait en quatre bras : 
${}^{11}le premier s’appelle le Pishone, il contourne tout le pays de Havila où l’on trouve de l’or 
${}^{12}– et l’or de ce pays est bon – ainsi que de l’ambre jaune et de la cornaline ; 
${}^{13}le deuxième fleuve s’appelle le Guihone, il contourne tout le pays de Koush ; 
${}^{14}le troisième fleuve s’appelle le Tigre, il coule à l’est d’Assour ; le quatrième fleuve est l’Euphrate.
${}^{15}Le Seigneur Dieu prit l’homme et le conduisit dans le jardin d’Éden pour qu’il le travaille et le garde. 
${}^{16} Le Seigneur Dieu donna à l’homme cet ordre : « Tu peux manger les fruits\\de tous les arbres du jardin ; 
${}^{17} mais l’arbre de la connaissance du bien et du mal, tu n’en mangeras pas ; car, le jour où tu en mangeras, tu mourras. »
${}^{18}Le Seigneur Dieu dit : « Il n’est pas bon que l’homme soit seul. Je vais lui faire une aide qui lui correspondra\\. » 
${}^{19} Avec de la terre, le Seigneur Dieu modela toutes les bêtes des champs et tous les oiseaux du ciel, et il les amena vers l’homme pour voir quels noms il leur donnerait. C’étaient des êtres vivants, et l’homme donna un nom à chacun. 
${}^{20} L’homme donna donc leurs noms à tous les animaux, aux oiseaux du ciel et à toutes les bêtes des champs. Mais il ne trouva aucune aide qui lui corresponde. 
${}^{21} Alors le Seigneur Dieu fit tomber sur lui un sommeil mystérieux\\, et l’homme s’endormit. Le Seigneur Dieu prit une de ses côtes, puis il referma la chair à sa place. 
${}^{22} Avec la côte qu’il avait prise à l’homme, il façonna une femme et il l’amena vers l’homme.
        ${}^{23}L’homme dit alors :
        \\« Cette fois-ci, voilà l’os de mes os
        \\et la chair de ma chair !
        \\On l’appellera femme – Ishsha –,
        \\elle qui fut tirée de l’homme – Ish. »
${}^{24}À cause de cela, l’homme quittera son père et sa mère, il s’attachera à sa femme, et tous deux ne feront plus qu’un\\.
${}^{25}Tous les deux, l’homme et sa femme, étaient nus, et ils n’en éprouvaient aucune honte l’un devant l’autre\\.
      
         
      \bchapter{}
      \begin{verse}
${}^{1}Le serpent était le plus rusé de tous les animaux des champs que le Seigneur Dieu avait faits. Il dit à la femme : « Alors, Dieu vous a vraiment dit : “Vous ne mangerez d’aucun arbre du jardin” ? » 
${}^{2} La femme répondit au serpent : « Nous mangeons les fruits des arbres du jardin. 
${}^{3} Mais, pour le fruit de l’arbre qui est au milieu du jardin, Dieu a dit : “Vous n’en mangerez pas, vous n’y toucherez pas, sinon vous mourrez.” » 
${}^{4} Le serpent dit à la femme : « Pas du tout ! Vous ne mourrez pas ! 
${}^{5} Mais Dieu sait que, le jour où vous en mangerez, vos yeux s’ouvriront, et vous serez comme des dieux, connaissant le bien et le mal. »
${}^{6}La femme s’aperçut que le fruit de l’arbre devait être savoureux\\, qu’il était agréable à regarder\\et qu’il était désirable, cet arbre, puisqu’il donnait l’intelligence. Elle prit de son fruit, et en mangea. Elle en donna aussi à son mari\\, et il en mangea. 
${}^{7} Alors leurs yeux à tous deux s’ouvrirent et ils se rendirent compte qu’ils étaient nus. Ils attachèrent les unes aux autres des feuilles de figuier, et ils s’en firent des pagnes.
${}^{8}Ils entendirent la voix\\du Seigneur Dieu qui se promenait dans le jardin à la brise du jour. L’homme et sa femme allèrent se cacher aux regards du Seigneur Dieu parmi les arbres du jardin. 
${}^{9} Le Seigneur Dieu appela l’homme et lui dit : « Où es-tu donc ? » 
${}^{10} Il répondit : « J’ai entendu ta voix dans le jardin, j’ai pris peur parce que je suis nu, et je me suis caché. » 
${}^{11} Le Seigneur\\reprit : « Qui donc t’a dit que tu étais nu ? Aurais-tu mangé de l’arbre dont je t’avais interdit de manger ? » 
${}^{12} L’homme répondit : « La femme que tu m’as donnée, c’est elle qui m’a donné du fruit\\de l’arbre, et j’en ai mangé. »
${}^{13}Le Seigneur Dieu dit à la femme : « Qu’as-tu fait là ? » La femme répondit : « Le serpent m’a trompée, et j’ai mangé. »
${}^{14}Alors le Seigneur Dieu dit au serpent : « Parce que tu as fait cela, tu seras maudit parmi tous les animaux et toutes les bêtes des champs. Tu ramperas sur le ventre et tu mangeras de la poussière tous les jours de ta vie. 
${}^{15} Je mettrai une hostilité entre toi et la femme, entre ta descendance et sa descendance : celle-ci\\te meurtrira la tête, et toi, tu lui meurtriras le talon. »
${}^{16}Le Seigneur Dieu\\dit ensuite à la femme : « Je multiplierai\\la peine de tes grossesses\\ ; c’est dans la peine que tu enfanteras des fils. Ton désir te portera vers ton mari, et celui-ci dominera sur toi. »
${}^{17}Il dit enfin à l’homme : « Parce que tu as écouté la voix de ta femme, et que tu as mangé le fruit\\de l’arbre que je t’avais interdit de manger : maudit soit le sol à cause de toi ! C’est dans la peine que tu en tireras ta nourriture, tous les jours de ta vie. 
${}^{18} De lui-même, il te donnera épines et chardons, mais tu auras ta nourriture en cultivant les champs\\. 
${}^{19} C’est à la sueur de ton visage\\que tu gagneras ton pain\\, jusqu’à ce que tu retournes à la terre dont tu proviens ; car tu es poussière, et à la poussière tu retourneras. »
${}^{20}L’homme appela sa femme Ève (c’est-à-dire : la vivante)\\, parce qu’elle fut la mère de tous les vivants. 
${}^{21} Le Seigneur Dieu fit à l’homme et à sa femme des tuniques de peau et les en revêtit. 
${}^{22} Puis le Seigneur Dieu déclara : « Voilà que l’homme est devenu comme l’un de nous par la connaissance du bien et du mal ! Maintenant, ne permettons pas qu’il avance la main, qu’il cueille aussi le fruit\\de l’arbre de vie, qu’il en mange et vive éternellement ! »
${}^{23}Alors le Seigneur Dieu le renvoya du jardin d’Éden, pour qu’il travaille la terre d’où il avait été tiré. 
${}^{24} Il expulsa l’homme, et il posta, à l’orient du jardin d’Éden, les Kéroubim, armés d’un glaive fulgurant, pour garder l’accès de l’arbre de vie.
      
         
      \bchapter{}
      \begin{verse}
${}^{1}L’homme s’unit à\\Ève, sa femme : elle devint enceinte, et elle mit au monde Caïn. Elle dit alors : « J’ai acquis\\un homme avec l’aide du Seigneur ! » 
${}^{2} Dans la suite, elle mit au monde Abel, frère de Caïn. Abel devint berger, et Caïn cultivait la terre\\.
${}^{3}Au temps fixé\\, Caïn présenta des produits de la terre en offrande au Seigneur. 
${}^{4} De son côté, Abel présenta les premiers-nés de son troupeau, en offrant les morceaux les meilleurs. Le Seigneur tourna son regard vers Abel et son offrande, 
${}^{5} mais vers Caïn et son offrande, il ne le tourna pas.
      Caïn en fut très irrité et montra\\un visage abattu. 
${}^{6} Le Seigneur dit à Caïn : « Pourquoi es-tu irrité, pourquoi ce visage abattu ? 
${}^{7} Si tu agis bien, ne relèveras-tu pas ton visage ? Mais si tu n’agis pas bien…, le péché est accroupi à ta porte. Il est à l’affût\\, mais tu dois le dominer. »
${}^{8}Caïn dit à son frère Abel : « Sortons dans les champs\\. » Et, quand ils furent dans la campagne, Caïn se jeta sur son frère Abel et le tua.
${}^{9}Le Seigneur dit à Caïn : « Où est ton frère Abel ? » Caïn répondit : « Je ne sais pas. Est-ce que je suis, moi, le gardien de mon frère ? » 
${}^{10} Le Seigneur reprit : « Qu’as-tu fait ? La voix du sang de ton frère crie de la terre\\vers moi ! 
${}^{11} Maintenant donc, sois maudit et chassé loin de cette terre qui a ouvert la bouche pour boire le sang de ton frère, versé par ta main. 
${}^{12} Tu auras beau cultiver la terre, elle ne produira plus rien pour toi. Tu seras un errant, un vagabond sur la terre. »
${}^{13}Alors Caïn dit au Seigneur : « Mon châtiment est trop lourd à porter\\ ! 
${}^{14} Voici qu’aujourd’hui tu m’as chassé de cette terre\\. Je dois me cacher loin de toi, je serai un errant, un vagabond sur la terre, et le premier venu qui me trouvera me tuera. » 
${}^{15} Le Seigneur lui répondit : « Si quelqu’un tue Caïn, Caïn sera vengé sept fois. » Et le Seigneur mit un signe sur Caïn pour le préserver d’être tué par le premier venu qui le trouverait.
${}^{16}Caïn s’éloigna de la face du Seigneur et s’en vint habiter au pays de Nod, à l’est d’Éden. 
${}^{17}Il s’unit à sa femme, elle devint enceinte et mit au monde Hénok. Il construisit une ville et l’appela du nom de son fils : Hénok. 
${}^{18}À Hénok naquit Irad, Irad engendra Mehouyaël, Mehouyaël engendra Metoushaël, et Metoushaël engendra Lamek.
${}^{19}Lamek prit deux femmes : l’une s’appelait Ada et l’autre, Silla. 
${}^{20}Ada mit au monde Yabal : celui-ci fut le père de ceux qui habitent sous la tente et parmi les troupeaux. 
${}^{21}Son frère s’appelait Youbal ; il fut le père de tous ceux qui jouent de la cithare et de la flûte. 
${}^{22}Silla, quant à elle, mit au monde Toubal-Caïn qui aiguisait les socs de bronze et de fer. La sœur de Toubal-Caïn était Naama.
${}^{23}Lamek dit à ses femmes :
        \\« Ada et Silla, entendez ma voix,
        \\épouses de Lamek, écoutez ma parole :
        \\Pour une blessure, j’ai tué un homme ;
        \\pour une meurtrissure, un enfant.
${}^{24}Caïn sera vengé sept fois,
        \\et Lamek, soixante-dix-sept fois ! »
${}^{25}Adam s’unit\\encore à sa femme, et elle mit au monde un fils. Elle lui donna le nom de Seth (ce qui veut dire : accordé)\\, car elle dit : « Dieu m’a accordé une nouvelle descendance à la place d’Abel, tué par Caïn. » 
${}^{26}Seth, lui aussi, eut un fils. Il l’appela du nom d’Énosh. Alors on commença à invoquer le nom du Seigneur.
      
         
      \bchapter{}
      \begin{verse}
${}^{1}Voici le livre de la descendance d’Adam. Le jour où Dieu créa l’homme, il le fit à la ressemblance de Dieu ; 
${}^{2}il les créa homme et femme ; il les bénit et il leur donna le nom d’« Homme », le jour où ils furent créés.
${}^{3}Adam vécut cent trente ans, puis il engendra un fils à sa ressemblance et selon son image ; il l’appela du nom de Seth. 
${}^{4}Après qu’Adam eut engendré Seth, la durée de sa vie fut encore de huit cents ans, et il engendra des fils et des filles. 
${}^{5}Adam vécut en tout neuf cent trente ans ; puis il mourut.
${}^{6}Seth vécut cent cinq ans, puis il engendra Énosh. 
${}^{7}Après avoir engendré Énosh, Seth vécut encore huit cent sept ans et engendra des fils et des filles. 
${}^{8}Seth vécut en tout neuf cent douze ans ; puis il mourut.
${}^{9}Énosh vécut quatre-vingt-dix ans, puis il engendra Qénane. 
${}^{10}Après avoir engendré Qénane, Énosh vécut encore huit cent quinze ans et engendra des fils et des filles. 
${}^{11}Énosh vécut en tout neuf cent cinq ans ; puis il mourut.
${}^{12}Qénane vécut soixante-dix ans, puis il engendra Mahalalel. 
${}^{13}Après avoir engendré Mahalalel, Qénane vécut encore huit cent quarante ans et engendra des fils et des filles. 
${}^{14}Qénane vécut en tout neuf cent dix ans ; puis il mourut.
${}^{15}Mahalalel vécut soixante-cinq ans, puis il engendra Yèred. 
${}^{16}Après avoir engendré Yèred, Mahalalel vécut encore huit cent trente ans et engendra des fils et des filles. 
${}^{17}Mahalalel vécut en tout huit cent quatre-vingt-quinze ans ; puis il mourut.
${}^{18}Yèred vécut cent soixante-deux ans, puis il engendra Hénok. 
${}^{19}Après avoir engendré Hénok, Yèred vécut encore huit cents ans et engendra des fils et des filles. 
${}^{20}Yèred vécut en tout neuf cent soixante-deux ans ; puis il mourut.
${}^{21}Hénok vécut soixante-cinq ans, puis il engendra Mathusalem. 
${}^{22}Après avoir engendré Mathusalem, Hénok marcha encore avec Dieu pendant trois cents ans et engendra des fils et des filles. 
${}^{23}Hénok vécut en tout trois cent soixante-cinq ans. 
${}^{24}Il avait marché avec Dieu, puis il disparut car Dieu l’avait enlevé.
${}^{25}Mathusalem vécut cent quatre-vingt-sept ans, puis il engendra Lamek. 
${}^{26}Après avoir engendré Lamek, Mathusalem vécut encore sept cent quatre-vingt-deux ans et engendra des fils et des filles. 
${}^{27}Mathusalem vécut en tout neuf cent soixante-neuf ans ; puis il mourut.
${}^{28}Lamek vécut cent quatre-vingt-deux ans, puis il engendra un fils. 
${}^{29}Il l’appela du nom de Noé, en disant : « Celui-ci nous soulagera de nos labeurs et de la peine qu’impose à nos mains un sol maudit par le Seigneur. » 
${}^{30}Après avoir engendré Noé, Lamek vécut encore cinq cent quatre-vingt-quinze ans et engendra des fils et des filles. 
${}^{31}Lamek vécut en tout sept cent soixante-dix-sept ans ; puis il mourut.
${}^{32}Noé était âgé de cinq cents ans quand il engendra Sem, Cham et Japhet.
      <h2 class="intertitle" id="d85e6137">2. Le déluge (6 – 9,17)</h2>
      
         
      \bchapter{}
      \begin{verse}
${}^{1}Quand les hommes commencèrent à se multiplier sur la terre et qu’ils eurent des filles, 
${}^{2}les fils des dieux s’aperçurent que les filles des hommes étaient belles. Ils prirent pour eux des femmes parmi toutes celles qu’ils avaient distinguées. 
${}^{3}Alors le Seigneur dit : « Mon souffle n’habitera pas indéfiniment dans l’homme : celui-ci s’égare, il n’est qu’un être de chair, sa vie ne durera que cent vingt ans. »
${}^{4}En ces jours-là, et même plus tard, il y avait des géants sur la terre. Les fils des dieux s’approchaient des filles des hommes et elles en avaient des enfants : ce sont les héros du temps jadis, des hommes de renom.
${}^{5}Le Seigneur vit que la méchanceté de l’homme était grande sur la terre, et que toutes les pensées de son cœur se portaient uniquement vers le mal à longueur de journée. 
${}^{6} Le Seigneur se repentit d’avoir fait l’homme sur la terre ; il s’irrita en son cœur et il dit : 
${}^{7} « Je vais effacer de la surface du sol les hommes que j’ai créés – et non seulement les hommes mais aussi les bestiaux, les bestioles et les oiseaux du ciel – car je me repens de les avoir faits. » 
${}^{8} Mais Noé trouva grâce aux yeux du Seigneur.
${}^{9}Voici l’histoire de Noé. Parmi ses contemporains, Noé fut un homme juste, parfait. Noé marchait avec Dieu. 
${}^{10}Il engendra trois fils : Sem, Cham et Japhet. 
${}^{11}Mais la terre s’était corrompue devant la face de Dieu, la terre était remplie de violence. 
${}^{12}Dieu regarda la terre, et voici qu’elle était corrompue car, sur la terre, tout être de chair avait une conduite corrompue.
${}^{13}Dieu dit à Noé : « Je l’ai décidé, c’est la fin de tout être de chair ! À cause des hommes, la terre est remplie de violence. Eh bien ! je vais les détruire et la terre avec eux. 
${}^{14}Fais-toi une arche en bois de cyprès. Tu la diviseras en cellules et tu l’enduiras de bitume à l’intérieur et à l’extérieur. 
${}^{15}Tu la feras ainsi : trois cents coudées de long, cinquante de large et trente de haut. 
${}^{16}Tu feras à l’arche un toit à pignon que tu fixeras une coudée au-dessus d’elle. Tu mettras l’entrée de l’arche sur le côté, puis tu lui feras un étage inférieur, un deuxième étage et un troisième.
${}^{17}Et voici que moi je fais venir le déluge, les eaux recouvriront la terre ; ainsi je détruirai, sous les cieux, tout être de chair animé d’un souffle de vie. Tout ce qui vit sur la terre expirera. 
${}^{18}Mais, avec toi, j’établirai mon alliance. Toi, tu entreras dans l’arche et, avec toi, tes fils, ta femme et les femmes de tes fils. 
${}^{19}De tout ce qui vit, tout ce qui est de chair, tu feras entrer dans l’arche un mâle et une femelle, pour qu’ils restent en vie avec toi. 
${}^{20}De chaque espèce d’oiseaux, de chaque espèce d’animaux domestiques, de chaque espèce de reptiles du sol, un couple t’accompagnera pour rester en vie. 
${}^{21}Et toi, procure-toi de quoi manger ; fais-en provision. Ce sera ta nourriture et la leur. »
${}^{22}Noé fit ainsi. Tout ce que Dieu lui avait ordonné, il le fit.
      
         
      \bchapter{}
      \begin{verse}
${}^{1}Le Seigneur dit à Noé : « Entre dans l’arche, toi et toute ta famille\\, car j’ai vu qu’au sein de cette génération, devant moi, tu es juste. 
${}^{2} De tous les animaux purs, tu prendras sept mâles et sept femelles\\ ; des animaux qui ne sont pas purs, tu en prendras deux, un mâle et une femelle ; 
${}^{3} et de même des oiseaux du ciel, sept mâles et sept femelles\\, pour que leur race continue à vivre à la surface de la terre. 
${}^{4} Encore sept jours, en effet, et je vais faire tomber la pluie sur la terre, pendant quarante jours et quarante nuits ; j’effacerai de la surface du sol tous les êtres que j’ai faits. »
${}^{5}Noé fit tout ce que le Seigneur lui avait ordonné. 
${}^{6}Noé avait six cents ans quand eut lieu le déluge, c’est-à-dire les eaux sur la terre. 
${}^{7}Noé entra dans l’arche avec ses fils, sa femme et les femmes de ses fils, à cause des eaux du déluge. 
${}^{8}Des animaux purs et des animaux impurs, des oiseaux et de tout ce qui va et vient sur le sol, 
${}^{9}un couple – un mâle et une femelle – entra dans l’arche avec Noé, comme Dieu l’avait ordonné à Noé.
${}^{10}Sept jours plus tard, les eaux du déluge étaient sur la terre.
${}^{11}L’an six cent de la vie de Noé, le deuxième mois, le dix-septième jour du mois, ce jour-là, les réservoirs du grand abîme se fendirent ; les vannes des cieux s’ouvrirent. 
${}^{12}Et la pluie tomba sur la terre pendant quarante jours et quarante nuits. 
${}^{13}En ce jour même, Noé entra dans l’arche avec ses fils Sem, Cham et Japhet, avec sa femme et les trois femmes de ses fils. 
${}^{14}Y entrèrent aussi tous les animaux selon leur espèce, tous les bestiaux selon leur espèce, tous les reptiles qui rampent sur la terre selon leur espèce, et tous les oiseaux selon leur espèce, tout ce qui vole, tout ce qui a des ailes. 
${}^{15}Couple par couple, tous les êtres de chair animés d’un souffle de vie entrèrent dans l’arche avec Noé. 
${}^{16}Ceux qui entraient, c’était un mâle et une femelle de tous les êtres de chair, comme Dieu l’avait ordonné à Noé.
      Alors le Seigneur ferma la porte sur Noé.
${}^{17}Et ce fut le déluge sur la terre pendant quarante jours. Les eaux grossirent et soulevèrent l’arche qui s’éleva au-dessus de la terre. 
${}^{18}Les eaux montèrent et grossirent beaucoup sur la terre, et l’arche flottait à la surface des eaux. 
${}^{19}Les eaux montèrent encore beaucoup, beaucoup sur la terre ; sous tous les cieux, toutes les hautes montagnes furent recouvertes. 
${}^{20}Les eaux étaient montées de quinze coudées au-dessus des montagnes qu’elles recouvraient. 
${}^{21}Alors expira tout être de chair, tout ce qui va et vient sur la terre : oiseaux, bestiaux, bêtes sauvages, tout ce qui foisonne sur la terre, et tous les hommes. 
${}^{22}Parmi tout ce qui existait sur la terre ferme, tout ce qui avait en ses narines un souffle de vie mourut. 
${}^{23}Ainsi furent effacés de la surface du sol tous les êtres qui s’y trouvaient, non seulement les hommes mais aussi les bestiaux, les bestioles et les oiseaux du ciel ; ils furent effacés de la terre : il ne resta que Noé et ceux qui étaient avec lui dans l’arche.
${}^{24}Et les eaux montèrent au-dessus de la terre pendant cent cinquante jours.
      
         
      \bchapter{}
      \begin{verse}
${}^{1}Dieu se souvint de Noé, de toutes les bêtes sauvages et de tous les bestiaux qui étaient avec lui dans l’arche ; il fit passer un souffle sur la terre : les eaux se calmèrent. 
${}^{2}Les sources de l’abîme et les vannes du ciel se fermèrent, la pluie des cieux s’arrêta. 
${}^{3}Par un mouvement de flux et de reflux, les eaux se retirèrent de la surface de la terre. Au bout de cent cinquante jours, les eaux avaient baissé 
${}^{4}et, le dix-septième jour du septième mois, l’arche se posa sur les monts d’Ararat. 
${}^{5}Les eaux continuèrent à baisser jusqu’au dixième mois ; le premier jour du dixième mois, les sommets des montagnes apparurent. 
${}^{6}Au bout de quarante jours, Noé ouvrit la fenêtre de l’arche qu’il avait construite, 
${}^{7}et il lâcha le corbeau ; celui-ci fit des allers et retours, jusqu’à ce que les eaux se soient retirées, laissant la terre à sec. 
${}^{8}Noé lâcha aussi\\la colombe pour voir si les eaux avaient baissé à la surface du sol. 
${}^{9}La colombe ne trouva pas d’endroit où se poser\\, et elle revint vers l’arche auprès de lui, parce que les eaux étaient sur toute la surface de la terre ; Noé tendit la main, prit la colombe, et la fit rentrer auprès de lui dans l’arche. 
${}^{10}Il attendit encore sept jours, et lâcha de nouveau la colombe hors de l’arche. 
${}^{11}Vers le soir, la colombe revint, et voici qu’il y avait dans son bec un rameau d’olivier tout frais ! Noé comprit ainsi que les eaux avaient baissé sur la terre. 
${}^{12}Il attendit encore sept autres jours et lâcha la colombe, qui, cette fois-ci, ne revint plus vers lui.
${}^{13}C’est en l’an six cent un de la vie de Noé\\, au premier mois, le premier jour\\du mois, que les eaux s’étaient retirées, laissant la terre à sec. Noé enleva le toit de l’arche, et regarda : et voici que la surface du sol était sèche. 
${}^{14}Au deuxième mois, le vingt-septième jour du mois, la terre était sèche.
${}^{15}Dieu parla à Noé et lui dit : 
${}^{16}« Sors de l’arche, toi et, avec toi, ta femme, tes fils et les femmes de tes fils. 
${}^{17}Tous les animaux qui sont avec toi, tous ces êtres de chair, oiseaux, bestiaux, reptiles qui rampent sur la terre, fais-les sortir avec toi ; qu’ils foisonnent sur la terre, qu’ils soient féconds et se multiplient sur la terre. » 
${}^{18}Noé sortit donc avec ses fils, sa femme et les femmes de ses fils. 
${}^{19}Tous les animaux, tous les reptiles, tous les oiseaux, tout ce qui va et vient sur la terre, sortirent de l’arche, par familles.
${}^{20}Noé bâtit un autel au Seigneur ; il prit, parmi tous les animaux purs et tous les oiseaux purs, des victimes\\qu’il offrit en holocauste sur l’autel. 
${}^{21} Le Seigneur respira l’agréable odeur, et il se dit en lui-même : « Jamais plus je ne maudirai le sol à cause de l’homme : le cœur de l’homme est enclin au mal dès sa jeunesse, mais jamais plus je ne frapperai tous les vivants comme je l’ai fait.
        ${}^{22}Tant que la terre durera,
        \\semailles et moissons,
        \\froidure et chaleur,
        \\été et hiver,
        \\jour et nuit
        \\jamais ne cesseront. »
      
         
      \bchapter{}
      \begin{verse}
${}^{1}Dieu bénit Noé et ses fils. Il leur dit : « Soyez féconds, multipliez-vous, remplissez la terre. 
${}^{2} Vous serez la crainte et la terreur de tous les animaux de la terre, de tous les oiseaux du ciel, de tout ce qui va et vient sur le sol, et de tous les poissons de la mer : ils sont livrés entre vos mains. 
${}^{3} Tout ce qui va et vient, tout ce qui vit sera votre nourriture ; comme je vous avais donné\\l’herbe verte, je vous donne tout cela. 
${}^{4} Mais, avec la chair, vous ne mangerez pas le principe de vie, c’est-à-dire le sang.
${}^{5}Quant au sang, votre principe de vie, j’en demanderai compte à tout animal et j’en demanderai compte à tout homme ; à chacun, je demanderai compte de la vie de l’homme, son frère.
        ${}^{6}Si quelqu’un verse le sang de l’homme,
        \\par l’homme son sang sera versé.
        \\Car Dieu a fait l’homme à son image.
${}^{7}Et vous, soyez féconds, multipliez-vous, devenez très nombreux\\sur la terre ; oui, multipliez-vous\\ ! »
${}^{8}Dieu dit encore à Noé et à ses fils\\ : 
${}^{9} « Voici que moi, j’établis mon alliance avec vous, avec votre descendance après vous, 
${}^{10} et avec tous les êtres vivants qui sont avec vous : les oiseaux, le bétail, toutes les bêtes de la terre\\, tout ce qui est sorti de l’arche\\. 
${}^{11} Oui, j’établis mon alliance avec vous : aucun être de chair ne sera plus détruit par les eaux du déluge, il n’y aura plus de déluge pour ravager la terre. »
${}^{12}Dieu dit encore : « Voici le signe de l’alliance que j’établis entre moi et vous, et avec tous les êtres vivants qui sont avec vous, pour les générations à jamais : 
${}^{13}je mets mon arc au milieu des nuages, pour qu’il soit le signe de l’alliance entre moi et la terre. 
${}^{14}Lorsque je rassemblerai les nuages au-dessus de la terre, et que l’arc apparaîtra au milieu des nuages, 
${}^{15}je me souviendrai de mon alliance qui est entre moi et vous, et tous les êtres vivants\\ : les eaux ne se changeront plus en déluge pour détruire tout être de chair. 
${}^{16}L’arc sera au milieu des nuages, je le verrai et, alors, je me souviendrai de l’alliance éternelle entre Dieu et tout être vivant qui est sur la terre. »
${}^{17}Dieu dit à Noé : « Voilà le signe de l’alliance que j’ai établie entre moi et tout être de chair qui est sur la terre. »
      <h2 class="intertitle" id="d85e6916">3. Du déluge à Abraham (9,18 – 11,26)</h2>
${}^{18}Les fils de Noé qui sortirent de l’arche sont Sem, Cham et Japhet. Cham est le père de Canaan. 
${}^{19}Tels sont les trois fils de Noé, et à partir d’eux toute la terre fut repeuplée.
${}^{20}Noé, homme de la terre, fut le premier à planter la vigne. 
${}^{21}Il en but le vin, s’enivra et se retrouva nu au milieu de sa tente. 
${}^{22}Cham, le père de Canaan, vit que son père était nu et il en informa ses deux frères qui étaient dehors. 
${}^{23}Sem et Japhet prirent le manteau, le placèrent sur leurs épaules à tous deux et, marchant à reculons, ils en couvrirent leur père qui était nu. Comme leurs visages étaient détournés, ils ne virent pas la nudité de leur père.
${}^{24}Noé, ayant cuvé son vin, se réveilla et apprit ce qu’avait fait son plus jeune fils. 
${}^{25}Il dit :
        \\« Maudit soit Canaan !
        \\Il sera pour ses frères
        \\l’esclave des esclaves. »
${}^{26}Et il ajouta :
        \\« Béni soit le Seigneur, le Dieu de Sem !
        \\Que Canaan soit son esclave !
${}^{27}Que Dieu mette Japhet au large !
        \\Qu’il demeure dans les tentes de Sem,
        \\et que Canaan soit son esclave. »
${}^{28}Après le déluge, Noé vécut encore trois cent cinquante ans. 
${}^{29}En tout, il vécut neuf cent cinquante ans, puis il mourut.
      
         
      \bchapter{}
      \begin{verse}
${}^{1}Voici la descendance des fils de Noé, Sem, Cham et Japhet. Il leur naquit des fils après le déluge.
${}^{2}Fils de Japhet : Gomer, Magog, Madaï, Yavane, Toubal, Mèshek et Tirâs. 
${}^{3}Fils de Gomer : Ashkenaz, Rifath et Togarma. 
${}^{4}Fils de Yavane : Élisha, Tarsis, Kittim et Rodanim. 
${}^{5}C’est à partir d’eux que se fit la dispersion dans les îles des nations ; chacun s’installa, selon son clan et sa langue, sur sa terre parmi les nations.
${}^{6}Fils de Cham : Koush, Misraïm, Pouth et Canaan. 
${}^{7}Fils de Koush : Séba, Havila, Sabta, Raéma, Sabteka. Fils de Raéma : Saba et Dedane.
${}^{8}Koush engendra Nemrod. Il fut le premier héros sur la terre. 
${}^{9}C’était un vaillant chasseur devant le Seigneur. C’est pourquoi on dit : « Être, tel Nemrod, vaillant chasseur devant le Seigneur. » 
${}^{10}Les capitales de son royaume furent Babel, Érek, Akkad, Kalné, au pays de Shinéar. 
${}^{11}De ce pays sortit Assour qui construisit Ninive, Rehoboth-Ir, Kalah, 
${}^{12}et Rèsèn entre Ninive et Kalah : c’est la grande ville.
${}^{13}Misraïm engendra les gens de Loud, d’Einame, de Lehab, de Naftouah, 
${}^{14}de Patrous et de Kaslouah d’où sortirent les Philistins et les gens de Kaftor.
${}^{15}Canaan engendra Sidon, son premier-né, et Heth, 
${}^{16}puis le Jébuséen, l’Amorite, le Guirgashite, 
${}^{17}le Hivvite, l’Arqite, le Sinite, 
${}^{18}l’Arvadite, le Semarite, le Hamatite. Les clans des Cananéens se dispersèrent ensuite 
${}^{19}et le territoire cananéen s’étendit de Sidon vers Guérar jusqu’à Gaza, vers Sodome et Gomorrhe, Adma et Seboïm jusqu’à Lèsha.
${}^{20}Tels furent les fils de Cham installés selon leurs clans et leurs langues, sur leurs terres parmi les nations.
${}^{21}De Sem, le frère aîné de Japhet, naquit aussi le père de tous les fils d’Éber.
${}^{22}Fils de Sem : Élam, Assour, Arpaxad, Loud et Aram. 
${}^{23}Fils d’Aram : Ouç, Houl, Guèter et Mash.
${}^{24}Arpaxad engendra Shèlah, et Shèlah engendra Éber. 
${}^{25}À Éber naquirent deux fils. Le premier s’appelait Pèleg, ce qui signifie « diviser », car en son temps la terre fut divisée, et son frère s’appelait Yoqtane. 
${}^{26}Yoqtane engendra Almodad, Shèlef, Haçarmaveth, Yèrah, 
${}^{27}Hadoram, Ouzal, Diqla, 
${}^{28}Obal, Abimaël, Saba, 
${}^{29}Ofir, Havila, Yobab. Tous ceux-là sont les fils de Yoqtane ; 
${}^{30}leur lieu d’habitation s’étendait depuis Mésha en direction de Sefar, jusqu’à la montagne de l’orient.
${}^{31}Tels furent les fils de Sem installés selon leurs clans et leurs langues, sur leurs terres parmi les nations. 
${}^{32}Tels furent les clans des fils de Noé, selon leur descendance, d’après leurs nations. C’est à partir d’eux que se fit la dispersion des nations sur la terre après le déluge.
      
         
      \bchapter{}
      \begin{verse}
${}^{1}Toute la terre avait alors la même langue et les mêmes mots. 
${}^{2} Au cours de leurs déplacements du côté de l’orient, les hommes\\découvrirent une plaine en Mésopotamie\\, et s’y établirent. 
${}^{3} Ils se dirent l’un à l’autre : « Allons ! fabriquons des briques et mettons-les à cuire ! » Les briques leur servaient de pierres, et le bitume, de mortier. 
${}^{4} Ils dirent : « Allons ! bâtissons-nous une ville, avec une tour dont le sommet soit dans les cieux. Faisons-nous un nom, pour ne pas être disséminés sur toute la surface de la terre. »
${}^{5}Le Seigneur descendit pour voir la ville et la tour que les hommes\\avaient bâties. 
${}^{6} Et le Seigneur dit : « Ils sont un seul peuple, ils ont tous la même langue : s’ils commencent ainsi, rien ne les empêchera désormais de faire tout ce qu’ils décideront. 
${}^{7} Allons ! descendons, et là, embrouillons leur langue : qu’ils ne se comprennent plus les uns les autres. » 
${}^{8} De là, le Seigneur les dispersa sur toute la surface de la terre. Ils cessèrent donc de bâtir la ville. 
${}^{9} C’est pourquoi on l’appela Babel\\, car c’est là que le Seigneur embrouilla la langue des habitants de toute la terre ; et c’est de là qu’il les dispersa sur toute la surface de la terre.
${}^{10}Voici la descendance de Sem.
      Sem était âgé de cent ans quand, deux ans après le déluge, il engendra Arpaxad. 
${}^{11}Après avoir engendré Arpaxad, Sem vécut encore cinq cents ans et engendra des fils et des filles.
${}^{12}Arpaxad vécut trente-cinq ans, puis il engendra Shèlah. 
${}^{13}Après avoir engendré Shèlah, Arpaxad vécut encore quatre cent trois ans et engendra des fils et des filles.
${}^{14}Shèlah vécut trente ans, puis il engendra Éber. 
${}^{15}Après avoir engendré Éber, Shèlah vécut encore quatre cent trois ans et engendra des fils et des filles.
${}^{16}Éber vécut trente-quatre ans, puis il engendra Pèleg. 
${}^{17}Après avoir engendré Pèleg, Éber vécut encore quatre cent trente ans et engendra des fils et des filles.
${}^{18}Pèleg vécut trente ans, puis il engendra Réou. 
${}^{19}Après avoir engendré Réou, Pèleg vécut encore deux cent neuf ans et engendra des fils et des filles.
${}^{20}Réou vécut trente-deux ans, puis il engendra Seroug. 
${}^{21}Après avoir engendré Seroug, Réou vécut encore deux cent sept ans et engendra des fils et des filles.
${}^{22}Seroug vécut trente ans, puis il engendra Nahor. 
${}^{23}Après avoir engendré Nahor, Seroug vécut encore deux cents ans et engendra des fils et des filles.
${}^{24}Nahor vécut vingt-neuf ans, puis il engendra Tèrah. 
${}^{25}Après avoir engendré Tèrah, Nahor vécut encore cent dix-neuf ans et engendra des fils et des filles.
${}^{26}Tèrah vécut soixante-dix ans, puis il engendra Abram, Nahor et Harane.
