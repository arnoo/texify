  
  
${}^{22}Moïse fit partir les fils d’Israël de la mer des Roseaux, et ils sortirent en direction du désert de Shour. Ils marchèrent trois jours à travers le désert sans trouver d’eau. 
${}^{23}Ils arrivèrent à Mara mais ne purent boire l’eau de Mara car elle était amère ; d’où son nom de « Mara ». 
${}^{24}Et le peuple récrimina contre Moïse en disant : « Que boirons-nous ? » 
${}^{25}Alors Moïse cria vers le Seigneur, et le Seigneur lui montra un morceau de bois. Moïse le jeta dans l’eau, et l’eau devint douce. C’est là que le Seigneur leur fixa un statut et un droit, là où il les mit à l’épreuve.
${}^{26}Il dit : « Si tu écoutes bien la voix du Seigneur ton Dieu, si tu fais ce qui est droit à ses yeux, si tu prêtes l’oreille à ses commandements, si tu observes tous ses décrets, je ne t’infligerai aucune des maladies que j’ai infligées aux Égyptiens, car je suis le Seigneur, celui qui te guérit. »
${}^{27}Les fils d’Israël atteignirent ensuite Élim, où il y a douze sources et soixante-dix palmiers. Et là, ils campèrent près de l’eau.
      
         
      \bchapter{}
      \begin{verse}
${}^{1}Toute la communauté des fils d’Israël partit d’Élim et atteignit le désert de Sine, entre Élim et le Sinaï, le quinzième jour du deuxième mois après sa sortie du pays d’Égypte. 
${}^{2} Dans le désert, toute la communauté des fils d’Israël récriminait contre Moïse et Aaron. 
${}^{3} Les fils d’Israël leur dirent : « Ah ! Il aurait mieux valu mourir de la main du Seigneur, au pays d’Égypte, quand nous étions assis près des marmites de viande, quand nous mangions du pain à satiété ! Vous nous avez fait sortir dans ce désert pour faire mourir de faim tout ce peuple assemblé\\ ! »
${}^{4}Le Seigneur dit à Moïse : « Voici que, du ciel, je vais faire pleuvoir du pain pour vous. Le peuple sortira pour recueillir chaque jour sa ration quotidienne, et ainsi je vais le mettre à l’épreuve : je verrai s’il marchera, ou non, selon ma loi. 
${}^{5} Mais, le sixième jour, quand ils feront le compte de leur récolte, ils trouveront le double de la ration quotidienne. »
${}^{6}Moïse et Aaron dirent alors aux fils d’Israël : « Ce soir, vous saurez que le Seigneur vous a fait sortir du pays d’Égypte ; 
${}^{7}et, demain matin, vous verrez la gloire du Seigneur, parce qu’il a entendu vos récriminations contre lui. Nous, que sommes-nous pour que vous récriminiez contre nous ? » 
${}^{8}Par là, Moïse voulait dire : « Vous verrez la gloire du Seigneur quand, le soir, il vous donnera de la viande en nourriture et, le matin, du pain à satiété. En effet, le Seigneur a entendu vos récriminations. Car ce n’est pas contre nous que vous récriminez mais bien contre le Seigneur. »
${}^{9}Moïse dit ensuite à Aaron : « Ordonne\\à toute la communauté des fils d’Israël : “Présentez-vous devant le Seigneur, car il a entendu vos récriminations.” » 
${}^{10}Aaron parla à toute la communauté des fils d’Israël ; puis ils se tournèrent du côté du désert, et voici que la gloire du Seigneur apparut dans la nuée.
${}^{11}Le Seigneur dit alors à Moïse : 
${}^{12}« J’ai entendu les récriminations des fils d’Israël. Tu leur diras : “Au coucher du soleil\\, vous mangerez de la viande et, le lendemain matin, vous aurez du pain à satiété. Alors vous saurez que moi, le Seigneur, je suis votre Dieu.” » 
${}^{13}Le soir même, surgit un vol de cailles qui recouvrirent le camp ; et, le lendemain matin, il y avait une couche de rosée autour du camp. 
${}^{14}Lorsque la couche de rosée s’évapora\\, il y avait, à la surface du désert, une fine croûte, quelque chose de fin comme du givre, sur le sol. 
${}^{15}Quand ils virent cela, les fils d’Israël se dirent l’un à l’autre : « Mann hou ? » (ce qui veut dire : Qu’est-ce que c’est ?)\\, car ils ne savaient pas ce que c’était. Moïse leur dit : « C’est le pain que le Seigneur vous donne à manger. 
${}^{16}Voici ce que le Seigneur a ordonné : Recueillez-en autant que chacun peut en manger : une mesure par personne. Chacun de vous en prendra selon le nombre d’habitants de sa tente. »
${}^{17}Les fils d’Israël firent ainsi : certains en recueillirent beaucoup, d’autres peu. 
${}^{18}Celui qui en avait ramassé beaucoup n’eut rien de trop ; celui qui en avait ramassé peu ne manqua de rien. Ainsi, chacun en avait recueilli autant qu’il pouvait en manger.
${}^{19}Moïse leur dit encore : « Que personne n’en garde jusqu’au matin ! » 
${}^{20}Ils n’écoutèrent pas Moïse et certains en gardèrent jusqu’au matin. Mais le surplus fut infesté de vers et se mit à sentir mauvais. Alors Moïse s’irrita contre eux. 
${}^{21}Matin après matin, ils en recueillaient autant que chacun pouvait en manger. À la chaleur du soleil, tout était fondu.
${}^{22}Or, le sixième jour, ils recueillirent le double de ce pain : deux mesures par personne. Et tous les chefs de la communauté vinrent en informer Moïse. 
${}^{23}Moïse leur dit : « Oui, c’est bien ce que le Seigneur avait dit. Demain est un grand sabbat, un sabbat consacré au Seigneur. Cuisez ce qui doit cuire, faites bouillir ce qui est à bouillir. Et gardez le surplus en réserve jusqu’au matin. » 
${}^{24}Ils le gardèrent, comme Moïse l’avait ordonné. Et il n’y eut ni mauvaise odeur ni vermine. 
${}^{25}Moïse leur dit : « Mangez-le aujourd’hui. Aujourd’hui, c’est le sabbat du Seigneur. Aujourd’hui, vous n’en trouverez pas dehors. 
${}^{26}Pendant six jours, vous en ramasserez, mais, le septième jour, c’est le sabbat : il n’y en aura pas. » 
${}^{27}Or, le septième jour, des gens sortirent pour en recueillir, mais ils n’en trouvèrent pas. 
${}^{28}Le Seigneur dit à Moïse : « Combien de temps encore refuserez-vous de garder mes commandements et mes lois ? 
${}^{29}Voyez : le Seigneur vous a donné le sabbat ; aussi, le sixième jour, vous donne-t-il du pain pour deux jours. Restez donc chacun chez vous. Que personne ne sorte de chez lui le septième jour. » 
${}^{30}Et, le septième jour, le peuple cessa toute activité.
${}^{31}La maison d’Israël donna à ce pain le nom de « manne ». C’était comme de la graine de coriandre, de couleur blanche, au goût de beignet au miel.
${}^{32}Moïse dit : « Voici ce que le Seigneur a ordonné : Qu’on en garde une pleine mesure en réserve pour les générations futures. Ainsi pourront-ils voir le pain dont je vous ai nourri au désert, quand je vous ai fait sortir du pays d’Égypte. » 
${}^{33}Moïse dit à Aaron : « Prends un vase, tu y mettras une pleine mesure de manne et tu le déposeras devant le Seigneur, en réserve pour les générations futures. » 
${}^{34}Comme le Seigneur l’avait ordonné à Moïse, Aaron déposa le vase, en réserve, devant le Témoignage.
${}^{35}Les fils d’Israël mangèrent de la manne pendant quarante ans, jusqu’à leur arrivée en pays habité ; ils mangèrent de la manne jusqu’à leur arrivée aux confins du pays de Canaan. 
${}^{36}La mesure utilisée, l’omèr, est un dixième de l’épha.
      
         
      \bchapter{}
      \begin{verse}
${}^{1}Toute la communauté des fils d’Israël partit du désert de Sine, en observant les étapes prescrites par le Seigneur. Ils campèrent à Rephidim. Comme il n’y avait pas d’eau à boire\\, 
${}^{2} le peuple chercha querelle à Moïse : « Donne-nous de l’eau à boire ! » Moïse leur répondit : « Pourquoi me cherchez-vous querelle ? Pourquoi mettez-vous le Seigneur à l’épreuve ? »
${}^{3}Là, le peuple souffrit de la soif\\. Il récrimina contre Moïse et dit : « Pourquoi nous as-tu fait monter d’Égypte ? Était-ce pour nous faire mourir de soif avec nos fils et nos troupeaux\\ ? » 
${}^{4} Moïse cria vers le Seigneur : « Que vais-je faire de ce peuple ? Encore un peu, et ils me lapideront ! » 
${}^{5} Le Seigneur dit à Moïse : « Passe devant le peuple, emmène\\avec toi plusieurs des anciens d’Israël, prends en main le bâton avec lequel tu as frappé le Nil, et va ! 
${}^{6} Moi, je serai là, devant toi, sur le rocher du mont\\Horeb. Tu frapperas le rocher, il en sortira de l’eau, et le peuple boira ! » Et Moïse fit ainsi sous les yeux des anciens d’Israël.
${}^{7}Il donna à ce lieu le nom de Massa (c’est-à-dire : Épreuve)\\et Mériba (c’est-à-dire : Querelle)\\, parce que les fils d’Israël avaient cherché querelle au Seigneur, et parce qu’ils l’avaient mis à l’épreuve, en disant : « Le Seigneur est-il au milieu de nous, oui ou non ? »
${}^{8}Les Amalécites\\survinrent et attaquèrent Israël à Rephidim. 
${}^{9} Moïse dit alors à Josué : « Choisis\\des hommes, et va combattre les Amalécites. Moi, demain, je me tiendrai sur le sommet de la colline, le bâton de Dieu à la main. » 
${}^{10} Josué fit ce que Moïse avait dit : il mena le combat contre les Amalécites. Moïse, Aaron et Hour étaient montés au sommet de la colline. 
${}^{11} Quand Moïse tenait la main levée, Israël était le plus fort. Quand il la laissait retomber, Amalec était le plus fort. 
${}^{12} Mais les mains de Moïse s’alourdissaient ; on prit une pierre, on la plaça derrière lui, et il s’assit dessus. Aaron et Hour lui soutenaient les mains, l’un d’un côté, l’autre de l’autre. Ainsi les mains de Moïse restèrent fermes jusqu’au coucher du soleil. 
${}^{13} Et Josué triompha des Amalécites\\au fil de l’épée.
${}^{14}Alors le Seigneur dit à Moïse : « Écris cela dans le Livre pour en faire mémoire et déclare à Josué que j’effacerai complètement le souvenir d’Amalec de dessous les cieux ! » 
${}^{15}Moïse bâtit un autel et l’appela : « Le-Seigneur-est-mon-étendard. » 
${}^{16}Et il dit : « Puisqu’une main s’est levée contre le trône du Seigneur, le Seigneur est en guerre contre Amalec, de génération en génération. »
      
         
      \bchapter{}
      \begin{verse}
${}^{1}Jéthro, prêtre de Madiane, le beau-père de Moïse, entendit parler de tout ce que Dieu avait fait pour Moïse et pour Israël, son peuple : le Seigneur avait fait sortir Israël d’Égypte ! 
${}^{2}Jéthro, le beau-père de Moïse, prit Cippora, la femme de Moïse, qu’il avait répudiée ; 
${}^{3}il prit aussi ses deux fils. L’un s’appelait Guershom (ce qui signifie : Immigré en ce lieu) car, avait dit Moïse, « Je suis devenu un immigré en terre étrangère ». 
${}^{4}L’autre s’appelait Élièzer (ce qui signifie : Mon Dieu est mon secours) « car, avait-il dit, le Dieu de mon père est venu à mon secours, il m’a délivré de l’épée de Pharaon ».
${}^{5}Jéthro, beau-père de Moïse, prit donc Cippora et ses deux fils ; il s’en alla rejoindre Moïse, au désert, là où il campait, à la montagne de Dieu. 
${}^{6}Il fit dire à Moïse : « C’est moi Jéthro, ton beau-père, qui viens vers toi avec ta femme et tes deux fils. » 
${}^{7}Moïse sortit à la rencontre de son beau-père, se prosterna et l’embrassa ; ils se saluèrent et entrèrent dans la tente. 
${}^{8}Moïse raconta à son beau-père tout ce que le Seigneur avait fait à Pharaon et à l’Égypte à cause d’Israël, toutes les difficultés survenues en chemin et dont le Seigneur les avait délivrés. 
${}^{9}Jéthro se réjouit de tout le bien que le Seigneur avait fait à Israël, en le délivrant de la main des Égyptiens. 
${}^{10}Et Jéthro dit : « Béni soit le Seigneur qui vous a délivrés de la main des Égyptiens et de la main de Pharaon ! Béni soit le Seigneur qui a délivré le peuple de la main des Égyptiens ! 
${}^{11}Je reconnais maintenant que le Seigneur est plus grand que tous les dieux, comme il l’a bien montré au temps de leur oppression. »
${}^{12}Jéthro, beau-père de Moïse, offrit un holocauste et des sacrifices à Dieu. Aaron et tous les anciens d’Israël vinrent participer au repas devant Dieu avec le beau-père de Moïse.
${}^{13}Or, le lendemain, Moïse siégea pour rendre la justice au peuple, et le peuple resta devant Moïse du matin jusqu’au soir. 
${}^{14}Le beau-père de Moïse vit tout ce que celui-ci faisait pour le peuple. Il lui dit : « Que fais-tu là pour le peuple ? Pourquoi es-tu seul à siéger, tandis que tout le peuple est debout devant toi du matin jusqu’au soir ? » 
${}^{15}Moïse dit à son beau-père : « C’est que le peuple vient à moi pour consulter Dieu. 
${}^{16}S’ils ont un litige, ils viennent me trouver ; je leur rends justice, et je fais connaître les décrets de Dieu et ses lois. » 
${}^{17}Le beau-père de Moïse lui dit : « Ta façon de faire n’est pas la bonne. 
${}^{18}Tu vas t’épuiser complètement, ainsi que ce peuple qui est avec toi. La tâche est trop lourde pour toi, tu ne peux l’accomplir seul. 
${}^{19}Maintenant, écoute-moi ! Je vais te donner un conseil, et que Dieu soit avec toi ! Tiens-toi face à Dieu au nom du peuple : tu présenteras les litiges devant Dieu, 
${}^{20}tu informeras les gens des décrets et des lois, tu leur feras connaître le chemin à suivre et la conduite à tenir. 
${}^{21}Toi, tu distingueras, dans tout le peuple, des hommes de valeur, craignant Dieu, dignes de confiance, incorruptibles, et tu les institueras officiers de millier, officiers de centaine, officiers de cinquantaine et officiers de dizaine. 
${}^{22}Ils auront à juger le peuple en tout temps. Les affaires importantes, ils te les présenteront, mais les affaires mineures, ils les jugeront eux-mêmes. Allège ainsi ta charge. Qu’ils la portent avec toi ! 
${}^{23}Si tu fais cela, et que Dieu te l’ordonne, tu pourras tenir et, de plus, tout ce peuple rentrera chez lui en paix. »
${}^{24}Moïse écouta la voix de son beau-père et fit tout ce qu’il avait dit. 
${}^{25}Parmi tout Israël, Moïse choisit des hommes de valeur et les plaça à la tête du peuple : officiers de millier, officiers de centaine, officiers de cinquantaine et officiers de dizaine. 
${}^{26}Ils jugeaient le peuple en tout temps. Les affaires difficiles, ils les présentaient à Moïse, et les affaires de moindre importance, ils les jugeaient eux-mêmes.
${}^{27}Et Moïse laissa partir son beau-père, qui s’en retourna dans son pays.
