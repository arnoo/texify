  
  
    
    \bbook{PREMIER LIVRE DES CHRONIQUES}{PREMIER LIVRE DES CHRONIQUES}
      
         
      \bchapter{}
      \begin{verse}
${}^{1}Adam, Seth, Énosh, 
${}^{2}Qénane, Mahalaléel, Yéred, 
${}^{3}Hénok, Mathusalem, Lamek, 
${}^{4}Noé, Sem, Cham et Japhet.
${}^{5}Fils de Japhet : Gomer, Magog, les Mèdes, Yavane, Toubal, Mèshek et Tirâs.
${}^{6}Fils de Gomer : Ashkenaz, Difath et Togarma.
${}^{7}Fils de Yavane : Élisha, Tarsis, les Kittim et les Rodanim.
${}^{8}Fils de Cham : Koush, Misraïm, Pouth et Canaan.
${}^{9}Fils de Koush : Séba, Havila, Sabta, Rama et Sabteka. Fils de Rama : Saba et Dedane. 
${}^{10}Koush engendra Nemrod qui fut sur la terre le premier héros.
${}^{11}Misraïm engendra les habitants de Loud, de Anam, de Lehab, de Naftouh, 
${}^{12}de Patros, de Kaslouah, d’où sont sortis les Philistins, et ceux de Kaftor. 
${}^{13}Canaan engendra Sidon, son premier-né, puis Heth, 
${}^{14}et le Jébuséen, l’Amorite, le Guirgashite, 
${}^{15}le Hivvite, l’Arqite, le Sinite, 
${}^{16}l’Arvadite, le Cemarite et le Hamatite.
${}^{17}Fils de Sem : Élam, Assour, Arpaxad, Loud et Aram. Fils d’Aram : Ouç, Houl, Guéter et Mèshek. 
${}^{18}Arpaxad engendra Shèlah, et Shèlah engendra Éber. 
${}^{19}À Éber naquirent deux fils : le nom du premier était Pèleg (c’est-à-dire : Division), car en son temps la terre fut divisée, et le nom de son frère était Yoqtane. 
${}^{20}Yoqtane engendra Almodad, Shélef, Haçarmaveth, Yérah, 
${}^{21}Hadoram, Ouzal, Diqla, 
${}^{22}Ébal, Abimaël, Saba, 
${}^{23}Ophir, Havila, Yobab ; tous ceux-là sont fils de Yoqtane.
${}^{24}Sem, Arpaxad, Shèlah, 
${}^{25}Éber, Pèleg, Réou, 
${}^{26}Seroug, Nahor, Tèrah, 
${}^{27}Abram – c’est Abraham.
${}^{28}Fils d’Abraham : Isaac et Ismaël. 
${}^{29}Voici leur descendance : le premier-né d’Ismaël, Nebayoth, puis Qédar, Adbéel, Mibsam, 
${}^{30}Mishma, Douma, Massa, Hadad, Téma, 
${}^{31}Yetour, Nafish et Qédma : tels sont les fils d’Ismaël.
${}^{32}Fils de Qetoura, concubine d’Abraham : elle enfanta Zimrane, Yoqshane, Medane, Madiane, Yishbaq et Shouah. Fils de Yoqshane : Saba et Dedane. 
${}^{33}Fils de Madiane : Éfa, Éfer, Hanok, Abida, Eldaa. Tous ceux-là sont fils de Qetoura.
${}^{34}Abraham engendra Isaac. Fils d’Isaac : Ésaü et Israël. 
${}^{35}Fils d’Ésaü : Élifaz, Réouël, Yéoush, Yaélam et Qorah. 
${}^{36}Fils d’Élifaz : Témane, Omar, Cefi, Gahetam, Qenaz, Timna et Amalec. 
${}^{37}Fils de Réouël : Nahath, Zèrah, Shamma et Mizza. 
${}^{38}Fils de Séïr : Lotane, Shobal, Cibéone, Ana, Dishone, Éçer et Dishane. 
${}^{39}Fils de Lotane : Hori et Homam. Sœur de Lotane : Timna. 
${}^{40}Fils de Shobal : Alyane, Manahath, Ébal, Shefi et Onam. Fils de Cibéone : Ayya et Ana. 
${}^{41}Fils d’Ana : Dishone. Fils de Dishone : Hamrane, Éshbane, Yitrane et Kerane. 
${}^{42}Fils d’Écer : Bilhane, Zaavane et Yaaqane. Fils de Dishane : Ouç et Arane.
${}^{43}Voici les rois qui ont régné au pays d’Édom avant que règne un roi des fils d’Israël : Bèla, fils de Béor ; le nom de sa ville était Dinhaba. 
${}^{44}Bèla mourut ; Yobab, fils de Zèrah, de Bosra, régna à sa place. 
${}^{45}Yobab mourut ; Housham, du pays des Témanites, régna à sa place. 
${}^{46}Housham mourut ; Hadad, fils de Bedad, régna à sa place ; il vainquit Madiane dans la campagne de Moab. Le nom de sa ville était Avith. 
${}^{47}Hadad mourut ; Samla, de Masréqa, régna à sa place. 
${}^{48}Samla mourut ; Saül, de Rehoboth-sur-le-Fleuve, régna à sa place. 
${}^{49}Saül mourut ; Baal-Hanane, fils d’Akbor, régna à sa place. 
${}^{50}Baal-Hanane mourut ; Hadad régna à sa place. Le nom de sa ville était Paï ; le nom de sa femme était Mehétabéel, fille de Matred, fille de Mé-Zahab. 
${}^{51}Hadad mourut, et il y eut alors des chefs en Édom : le chef Timna, le chef Alya, le chef Yeteth, 
${}^{52}le chef Oholibama, le chef Éla, le chef Pinone, 
${}^{53}le chef Qenaz, le chef Témane, le chef Mibsar, 
${}^{54}le chef Magdiël, le chef Iram. Tels sont les chefs d’Édom.
      
         
      \bchapter{}
      \begin{verse}
${}^{1}Voici les fils d’Israël : Roubène, Siméon, Lévi, Juda, Issakar et Zabulon, 
${}^{2}Dane, Joseph et Benjamin, Nephtali, Gad et Asher.
      
         
${}^{3}Fils de Juda : Er, Onane et Shéla. Tous trois lui sont nés de Bath-Shoua, la Cananéenne. Er, premier-né de Juda, déplut au Seigneur, qui le fit mourir. 
${}^{4}Tamar, la belle-fille de Juda, lui enfanta Pérès et Zérah. Il y eut en tout cinq fils de Juda.
${}^{5}Fils de Pérès : Hesrone et Hamoul.
${}^{6}Fils de Zérah : Zimri, Étane, Hémane, Kalkol et Dara, cinq en tout.
${}^{7}Fils de Karmi : Akar ; transgressant l’anathème, il fit le malheur d’Israël. 
${}^{8}Fils d’Étane : Azarias. 
${}^{9}Fils qui sont nés à Hesrone : Yerahméel, Ram et Keloubaï.
${}^{10}Ram engendra Amminadab, Amminadab engendra Nashone, chef des fils de Juda, 
${}^{11}Nashone engendra Salma, et Salma engendra Booz. 
${}^{12}Booz engendra Obed, et Obed engendra Jessé.
${}^{13}Jessé engendra Éliab son premier-né, Abinadab le deuxième, Shiméa le troisième, 
${}^{14}Netanel le quatrième, Raddaï le cinquième, 
${}^{15}Ocem le sixième, David le septième. 
${}^{16}Ils eurent pour sœurs Cerouya et Abigaïl. Fils de Cerouya : Abishaï, Joab et Asaël : ils étaient trois. 
${}^{17}Abigaïl enfanta Amasa ; le père d’Amasa était Yéter l’Ismaélite.
${}^{18}Caleb, fils de Hesrone, engendra Yerioth de sa femme Azouba ; en voici les fils : Yésher, Shobab et Ardone. 
${}^{19}Azouba mourut, et Caleb épousa Éphrath, qui lui enfanta Hour. 
${}^{20}Hour engendra Ouri, et Ouri engendra Beçaléel.
${}^{21}Puis Hesrone s’unit à la fille de Makir, père de Galaad. Il l’épousa alors qu’il avait soixante ans, et elle lui enfanta Segoub. 
${}^{22}Segoub engendra Yaïr qui posséda vingt-trois villes dans le pays de Galaad. 
${}^{23}Puis Gueshour et Aram leur prirent les campements de Yaïr, Qenath et ses dépendances, soixante villes. Tout cela appartenait aux fils de Makir, père de Galaad. 
${}^{24}Après la mort de Hesrone, Caleb s’unit à Éphrata, femme de son père Hesrone, qui lui enfanta Ashehour, père de Teqoa.
${}^{25}Yeraméel, fils aîné de Hesrone, eut des fils : Ram son premier-né, Bouna, Orèn, Ocem et Ahiyya. 
${}^{26}Yeraméel eut une autre femme du nom d’Atara ; elle fut la mère d’Onam.
${}^{27}Les fils de Ram, premier-né de Yeraméel, furent Maas, Yamine et Éqer.
${}^{28}Les fils d’Onam furent Shammaï et Yada. Fils de Shammaï : Nadab et Abishour. 
${}^{29}Le nom de la femme d’Abishour était Abihaïl ; elle lui enfanta Abane et Molid. 
${}^{30}Fils de Nadab : Séled et Appaïm. Séled mourut sans avoir eu de fils. 
${}^{31}Fils d’Appaïm : Yishéï ; fils de Yishéï : Shéshane ; fils de Shéshane : Ahlaï. 
${}^{32}Fils de Yada, frère de Shammaï : Yéter et Jonathan. Yéter mourut sans avoir eu de fils. 
${}^{33}Fils de Jonathan : Péleth et Zaza. Tels furent les fils de Yeraméel.
${}^{34}Shéshane n’eut pas de fils, mais des filles. Il avait un serviteur égyptien du nom de Yarha. 
${}^{35}Shéshane donna sa fille pour femme à son serviteur Yarha ; elle lui enfanta Attaï. 
${}^{36}Attaï engendra Nathan, Nathan engendra Zabad, 
${}^{37}Zabad engendra Éflal, Éflal engendra Obed, 
${}^{38}Obed engendra Jéhu, Jéhu engendra Azarias, 
${}^{39}Azarias engendra Hélès, Hélès engendra Éléasa, 
${}^{40}Éléasa engendra Sismaï, Sismaï engendra Shalloum, 
${}^{41}Shalloum engendra Yeqamya, Yeqamya engendra Élishama.
${}^{42}Fils de Caleb, frère de Yeraméel : Mésha, son premier-né, le père de Zif, et Marésha, son autre fils, père de Hébrone. 
${}^{43}Fils de Hébrone : Qorah, Tappouah, Rèqem et Shèma. 
${}^{44}Shèma engendra Raham, père de Yorqéam. Rèqem engendra Shammaï. 
${}^{45}Le fils de Shammaï fut Maone, et Maone fut le père de Beth-Sour. 
${}^{46}Éfa, concubine de Caleb, enfanta Harane, Moça et Gazèz. Harane engendra Gazèz. 
${}^{47}Fils de Yahdaï : Règuem, Yotam, Guéshane, Pèleth, Éfa et Shaaf. 
${}^{48}Maaka, concubine de Caleb, enfanta Shèber et Tirhana. 
${}^{49}Elle enfanta Shaaf, père de Madmanna, et Sheva, père de Makbena et père de Guibéa. La fille de Caleb était Aksa. 
${}^{50}Tels furent les descendants de Caleb.
      Fils de Hour, premier-né d’Éphrata : Shobal, père de Qiryath-Yearim, 
${}^{51}Salma, père de Bethléem, Haref, père de Beth-Gader. 
${}^{52}Shobal, père de Qiryath-Yearim, eut des fils : Haroé, une moitié des Manahatites, 
${}^{53}et les clans de Qiryath-Yearim : les Yitrites, les Poutites, les Shoumatites et les Mishraïtes. Les gens de Soréa et d’Eshtaol en sont issus.
${}^{54}Fils de Salma : Bethléem, les Netofatites, Atroth-Beth-Yoab, l’autre moitié des Manahatites, les Soréatites, 
${}^{55}les clans Sofrites habitant Yabès, les Tiréatites, les Shiméatites et les Soukatites. Ce sont les Qénites, qui viennent de Hammath, père de la maison de Récab.
      
         
      \bchapter{}
      \begin{verse}
${}^{1}Voici les fils de David qui lui sont nés à Hébron : Amnone l’aîné, de sa femme Ahinoam, originaire de Yizréel ; Daniel le deuxième, de sa femme Abigaïl, originaire de Carmel ; 
${}^{2}Absalom le troisième, fils de Maaka, fille de Talmaï, roi de Gueshour ; Adonias le quatrième, fils de sa femme Hagguith ; 
${}^{3}Shefatya le cinquième, de sa femme Abital ; et Yitréam le sixième, de sa femme Égla. 
${}^{4}Six fils lui sont nés à Hébron, où il régna sept ans et six mois. Puis il régna trente-trois ans à Jérusalem.
${}^{5}Voici les fils qui lui sont nés à Jérusalem : Shiméa, Shobab, Nathan et Salomon ; ces quatre fils étaient de Bath-Shoua, fille d’Ammiël ; 
${}^{6}puis, Yibhar, Èlishama, Éliféleth, 
${}^{7}Nogah, Néfeg, Yafia, 
${}^{8}Èlishama, Èlyada et Éliféleth : neuf autres fils.
${}^{9}Tels furent tous les fils de David, sans compter les fils des concubines et leur sœur Tamar. 
${}^{10}Fils de Salomon : Roboam. Puis successivement, son fils Abiya, son fils Asa, son fils Josaphat, 
${}^{11}son fils Joram, son fils Ocozias, son fils Joas, 
${}^{12}son fils Amasias, son fils Azarias, son fils Yotam, 
${}^{13}son fils Acaz, son fils Ézékias, son fils Manassé, 
${}^{14}son fils Amone, son fils Josias. 
${}^{15}Fils de Josias : Yohanane l’aîné, Joakim le deuxième, Sédécias le troisième et Shalloum le quatrième. 
${}^{16}Fils de Joakim : son fils Jékonias et son fils Sédécias.
${}^{17}Fils de Jékonias le captif : Salathiel son fils, 
${}^{18}ainsi que Malkiram, Pedaya, Shènaçar, Yeqamya, Hoshama et Nedabya. 
${}^{19}Fils de Pedaya : Zorobabel et Shiméï. Fils de Zorobabel : Meshoullam et Hananya ; Shelomith était leur sœur. 
${}^{20}Fils de Meshoullam : Hashouba, Ohel, Bèrèkya, Hasadya et Youshab-Hèsed ; ils étaient cinq. 
${}^{21}Fils de Hananya : Pelatya et Isaïe ; les fils de Refaya, les fils d’Arnane, les fils de Obadya et les fils de Shekanya. 
${}^{22}Fils de Shekanya : Shemaya, et les fils de Shemaya : Hattoush, Yiguéal, Bariah, Néarya et Shafath ; ils étaient six en tout. 
${}^{23}Fils de Néarya : Élyoénaï, Ézékias et Azriqam ; ils étaient trois. 
${}^{24}Fils d’Élyoénaï : Hodaïvahou, Élyashib, Pelaya, Aqqoub, Yohanane, Delaya et Anani ; ils étaient sept.
      
         
      \bchapter{}
      \begin{verse}
${}^{1}Fils de Juda : Pérès, Hesrone, Karmi, Hour et Shobal. 
${}^{2}Reaya, fils de Shobal, engendra Yahath, et Yahath engendra Ahoumaï et Lahad. Ce sont les clans des Soréatites. 
${}^{3}Voici les fils de Hour : Abi-Étam, Yizréel, Yishma et Yidbash ; le nom de leur sœur était Haslelponi ; 
${}^{4}puis Penouël, père de Guedor ; et Ézer, père de Housha. Tels furent les fils de Hour, premier-né d’Éphrata, père de Bethléem.
${}^{5}Ashehour, père de Teqoa, eut deux femmes : Héléa et Naara. 
${}^{6}Naara lui enfanta Ahouzam, Héfer, Tèmeni et l’Ahashtarite. Tels furent les fils de Naara. 
${}^{7}Fils de Héléa : Céreth, Sohar et Étnane. 
${}^{8}Qos engendra Anoub, Hassobéba et les clans d’Aharhel, fils de Haroum. 
${}^{9}Yabés fut plus honoré que ses frères. Sa mère lui donna le nom de Yabés (c’est-à-dire : Dans la douleur) en disant : « J’ai enfanté dans la détresse. » 
${}^{10}Yabés invoqua le Dieu d’Israël en disant : « Si vraiment tu me bénis, tu agrandiras mon territoire, ta main sera avec moi, et tu éloigneras de moi le malheur, en sorte que ma détresse prenne fin. » Et Dieu lui accorda ce qu’il avait demandé.
${}^{11}Keloub, frère de Shouha, engendra Mehir, qui fut le père d’Eshtone. 
${}^{12}Eshtone engendra Beth-Rafa, Paséah, Tehinna, père d’Ir-Nahash. Tels furent les hommes de Réca. 
${}^{13}Fils de Qenaz : Otniël et Seraya. Fils d’Otniël : Hatath et Meonotaï ; 
${}^{14}Meonotaï engendra Ofra. Seraya engendra Joab, père de Gué-Harashim, car c’étaient des Harashim (c’est-à-dire des artisans).
${}^{15}Fils de Caleb, lui-même fils de Yefounnè : Irou, Éla et Naam. Fils d’Éla : Qenaz.
${}^{16}Fils de Yehallélel : Zif, Zifa, Tirya et Asaréel.
${}^{17}Fils d’Ezra : Yéter, Méred, Éfer et Yalone. Bitya, femme de Méred, conçut Miryam, Shammaï et Yishba, père d’Eshtemoa, 
${}^{18}dont la femme judéenne enfanta Yéred, père de Guedor, Héber, père de Soko, et Yeqoutiël, père de Zanoah. Tels furent les fils de Bitya, la fille de Pharaon qu’avait épousée Méred.
${}^{19}Fils de la femme de Hodiya, sœur de Naham : le père de Qéïla le Garmite et Eshtemoa le Maakatite.
${}^{20}Fils de Shimone : Amnone, Rinna, Ben-Hanane et Tilone. Fils de Yishéï : Zoheth et Ben-Zoheth.
${}^{21}Fils de Shéla, lui-même fils de Juda : Er, père de Léka, Lada, père de Marésha, et les clans des producteurs de lin à Beth-Ashbéa, 
${}^{22}Yoqim, les hommes de Kozéba, Yoash et Saraf qui furent maîtres de Moab et revinrent à Bethléem – ces événements sont anciens ! 
${}^{23}Ils étaient potiers et habitaient Netaïm et Guedéra ; ils habitaient là avec le roi, à son service.
${}^{24}Fils de Siméon : Nemouël, Yamine, Yarib, Zèrah et Saül. 
${}^{25}Celui-ci eut pour fils Shalloum, qui eut pour fils Mibsam, qui eut pour fils Mishma. 
${}^{26}Descendants de Mishma : son fils Hammouël, qui eut pour fils Zakkour, qui eut pour fils Shiméï.
${}^{27}Shiméï eut seize fils et six filles, mais ses frères n’eurent pas beaucoup de fils. L’ensemble de leurs clans n’atteignit pas le nombre des fils de Juda.
${}^{28}Ils habitèrent Bershéba, Molada, Haçar-Shoual, 
${}^{29}Bilha, Ècem, Tolad, 
${}^{30}Betouel, Horma, Ciqlag, 
${}^{31}Beth-Markaboth, Haçar-Soussim, Beth-Biréï et Shaaraïm. Telles furent leurs villes jusqu’au règne de David. 
${}^{32}Ils eurent pour villages : Étam, Aïn, Rimmone, Tokèn et Ashane, donc cinq bourgs, 
${}^{33}ainsi que tous les villages qui entouraient ces bourgs jusqu’à Baalath. C’est là qu’habitèrent et furent enregistrés : 
${}^{34}Meshobab, Yamlek, Yosha, le fils d’Amasias, 
${}^{35}Joël, Jéhu, le fils de Yoshibya, fils de Seraya, fils d’Asiël, 
${}^{36}Élyoénaï, Yaaqoba, Yeshohaya, Asaya, Adiël, Yesimiël, Benaya, 
${}^{37}et Ziza, le fils de Shiféï, fils d’Allone, fils de Yedaya, fils de Shimri, fils de Shemaya. 
${}^{38}Ces hommes, recensés nominativement, étaient chefs dans leurs clans, et leurs familles s’agrandirent beaucoup. 
${}^{39}Ils allèrent à l’entrée de Guedor jusqu’à l’est de la vallée, cherchant des pâturages pour leur petit bétail. 
${}^{40}Ils trouvèrent de bons et gras pâturages ; le pays était vaste, tranquille et paisible. En effet, autrefois y habitaient les descendants de Cham.
${}^{41}Ces hommes donc, inscrits nominativement, arrivèrent au temps d’Ézékias, roi de Juda ; ils détruisirent les tentes et les abris qui se trouvaient là. Ils les vouèrent à l’anathème jusqu’à ce jour, et ils s’établirent à leur place, car il y avait là des pâturages pour leur petit bétail. 
${}^{42}Certains d’entre eux, de la descendance de Siméon, gagnèrent la montagne de Séïr : cinq cents hommes, ayant à leur tête Pelatya, Nearya, Refaya, Ouzziël, les fils de Yishéï. 
${}^{43}Ils vainquirent le reste des rescapés d’Amalec, et ils habitèrent là jusqu’à ce jour.
      
         
      \bchapter{}
      \begin{verse}
${}^{1}Voici les fils de Roubène, premier-né d’Israël. Il était en effet le premier-né. Mais, quand il eut profané la couche de son père, son droit d’aînesse fut donné aux fils de Joseph, fils d’Israël, sans que ce droit soit enregistré, 
${}^{2}car Juda prévalut sur ses frères, et de lui fut engendré un prince ; mais le droit d’aînesse appartient à Joseph.
${}^{3}Fils de Roubène, premier-né d’Israël : Hénok, Pallou, Hesrone et Karmi.
${}^{4}Fils de Joël : son fils Shemaya, son fils Gog, son fils Shiméï, 
${}^{5}son fils Mika, son fils Reaya, son fils Baal, 
${}^{6}et son fils Beéra que Téglath-Phalasar, roi d’Assour, emmena en déportation. Il était chef des Roubénites. 
${}^{7}Ses frères, par clans, enregistrés selon leur descendance : en tête Yeïel, puis Zacharie, 
${}^{8}et Bèla, fils d’Azaz, fils de Shèma, fils de Joël. Son clan habitait Aroër, jusqu’à Nébo et Baal-Meone. 
${}^{9}Ils habitaient à l’est, jusqu’à l’entrée du désert que limite l’Euphrate – le fleuve –, car leurs troupeaux s’étaient multipliés au pays de Galaad. 
${}^{10}Au temps de Saül, ils firent la guerre aux Hagrites qui tombèrent entre leurs mains ; ils s’établirent dans leurs tentes sur tout le côté oriental du Galaad.
${}^{11}En face d’eux, les fils de Gad habitaient le pays du Bashane jusqu’à Salka : 
${}^{12}Joël en tête, Shafam le deuxième, puis Yanaï et Shafath en Bashane. 
${}^{13}Leurs frères, par familles : Mikaël, Meshoullam, Shèba, Yoraï, Yakane, Zia et Éber ; ils étaient sept. 
${}^{14}Voici les fils d’Abihaïl, fils de Houri, fils de Yaroah, fils de Galaad, fils de Mikaël, fils de Yeshishaï, fils de Yahdo, fils de Bouz. 
${}^{15}Ahi, fils d’Abdiël, fils de Gouni, était le chef de leur famille. 
${}^{16}Ils habitaient en Galaad, en Bashane et dans leurs dépendances, ainsi que dans tous les pâturages du Sarone jusqu’à leurs limites extrêmes.
${}^{17}Tous ces fils de Gad furent enregistrés à l’époque de Yotam, roi de Juda, et de Jéroboam, roi d’Israël.
${}^{18}Les fils de Roubène, les fils de Gad et la demi-tribu de Manassé étaient des guerriers ; les hommes armés du bouclier et de l’épée, tirant de l’arc et exercés à la guerre étaient quarante-quatre mille sept cent soixante, capables d’aller au combat. 
${}^{19}Ils firent la guerre aux Hagrites, à Yetour, à Nafish et à Nodab, 
${}^{20}et reçurent de l’aide contre les Hagrites et tous leurs alliés, qui furent livrés entre leurs mains, car ils avaient fait appel à Dieu dans le combat : il les exauça puisqu’ils avaient mis en lui leur confiance. 
${}^{21}Ils enlevèrent leurs troupeaux, cinquante mille chameaux, deux cent cinquante mille têtes de petit bétail, deux mille ânes, et cent mille personnes. 
${}^{22}Nombreuses furent les victimes qui tombèrent, parce que Dieu lui-même avait mené le combat. Et ils habitèrent à leur place jusqu’à la déportation.
${}^{23}Les fils de la demi-tribu de Manassé habitèrent dans le pays entre Bashane et Baal-Hermon, le Senir et le mont Hermon. Ils étaient nombreux.
${}^{24}Voici les chefs de leurs familles : Éfer, Yishéï, Èliël, Azriël, Jérémie, Hodavya et Yahdiël. C’étaient de vaillants guerriers, des hommes renommés, des chefs de famille. 
${}^{25}Mais ils furent infidèles au Dieu de leurs pères, et se prostituèrent aux dieux des autres peuples du pays, que Dieu avait exterminés devant eux.
${}^{26}Le Dieu d’Israël excita l’esprit de Poul, roi d’Assour, et l’esprit de Téglath-Phalasar, roi d’Assour. Celui-ci déporta Roubène, Gad et la demi-tribu de Manassé, et les emmena à Halah, sur le Habor, à Hara, sur le fleuve de Gozane. Ils y sont restés jusqu’à ce jour.
${}^{27}Fils de Lévi : Guershone, Qehath et Merari. 
${}^{28}Fils de Qehath : Amram, Yicehar, Hébrone et Ouzziël. 
${}^{29}Fils d’Amram : Aaron et Moïse, ainsi que Miryam. Fils d’Aaron : Nadab et Abihou, Éléazar et Itamar. 
${}^{30}Éléazar engendra Pinhas, Pinhas engendra Abishoua, 
${}^{31}Abishoua engendra Bouqqi, Bouqqi engendra Ouzzi, 
${}^{32}Ouzzi engendra Zerahya, Zerahya engendra Merayoth, 
${}^{33}Merayoth engendra Amarya, Amarya engendra Ahitoub, 
${}^{34}Ahitoub engendra Sadoc, Sadoc engendra Ahimaas, 
${}^{35}Ahimaas engendra Azarias, Azarias engendra Yohanane, 
${}^{36}Yohanane engendra Azarias. Celui-ci exerça le sacerdoce dans la Maison qu’avait bâtie Salomon à Jérusalem. 
${}^{37}Azarias engendra Amarya, Amarya engendra Ahitoub, 
${}^{38}Ahitoub engendra Sadoc, Sadoc engendra Shalloum, 
${}^{39}Shalloum engendra Helcias, Helcias engendra Azarias, 
${}^{40}Azarias engendra Seraya, Seraya engendra Yehosadaq. 
${}^{41}Yehosadaq partit quand le Seigneur déporta Juda et Jérusalem par la main de Nabucodonosor.
      
         
      \bchapter{}
      \begin{verse}
${}^{1}Fils de Lévi : Guershom, Qehath et Merari. 
${}^{2}Voici les noms des fils de Guershom : Libni et Shiméï. 
${}^{3}Fils de Qehath : Amram, Yicehar, Hébrone, Ouzziël. 
${}^{4}Fils de Merari : Mahli et Moushi. Tels sont les clans de Lévi selon leurs ancêtres. 
${}^{5}Pour Guershom : son fils Libni, son fils Yahath, son fils Zimma, 
${}^{6}son fils Yoah, son fils Iddo, son fils Zèrah et son fils Yéotraï.
${}^{7}Fils de Qehath : son fils Amminadab, son fils Coré, son fils Assir, 
${}^{8}son fils Elcana, son fils Ébiasaf, son fils Assir, 
${}^{9}son fils Tahath, son fils Ouriël, son fils Ouzziya, et son fils Saül. 
${}^{10}Fils d’Elcana : Amasaï et Ahimoth. 
${}^{11}Son fils Elcana, son fils Sofaï, son fils Nahath, 
${}^{12}son fils Éliab, son fils Yeroham, son fils Elcana, et son fils Samuel. 
${}^{13}Fils de Samuel : Joël l’aîné et Abiya le second.
${}^{14}Fils de Merari : Mahli, son fils Libni, son fils Shiméï, son fils Ouzza, 
${}^{15}son fils Shiméa, son fils Hagguiya, et son fils Asaya.
${}^{16}Voici ceux que David chargea de diriger le chant dans la maison du Seigneur, dès que l’Arche eut un lieu de repos. 
${}^{17}Ils furent préposés au chant devant la demeure de la tente de la Rencontre, jusqu’à ce que Salomon eût bâti à Jérusalem la maison du Seigneur. Ils accomplissaient leur service selon ce qui leur était ordonné.
${}^{18}Voici donc ceux qui accomplissaient le service, ainsi que leurs fils. Parmi les fils des Qehatites : Hémane le chantre, fils de Joël, fils de Samuel, 
${}^{19}fils d’Elcana, fils de Yeroham, fils d’Éliël, fils de Toah, 
${}^{20}fils de Souf, fils d’Elcana, fils de Mahath, fils d’Amasaï, 
${}^{21}fils d’Elcana, fils de Joël, fils d’Azarias, fils de Sophonie, 
${}^{22}fils de Tahath, fils d’Assir, fils d’Ébiasaf, fils de Coré, 
${}^{23}fils de Yicehar, fils de Qehath, fils de Lévi, fils d’Israël.
${}^{24}Asaf, frère d’Hémane le chantre, se tenait à sa droite : c’était Asaf, fils de Bérékyahou, fils de Shiméa, 
${}^{25}fils de Mikaël, fils de Baaséya, fils de Malkiya, 
${}^{26}fils d’Étni, fils de Zèrah, fils d’Adaya, 
${}^{27}fils d’Étane, fils de Zimma, fils de Shiméï, 
${}^{28}fils de Yahath, fils de Guershom, fils de Lévi.
${}^{29}À gauche, leurs frères, fils de Merari : Étane, fils de Qishi, fils d’Abdi, fils de Mallouk, 
${}^{30}fils de Hashabya, fils d’Amasias, fils d’Helcias, 
${}^{31}fils d’Amsi, fils de Bani, fils de Shèmer, 
${}^{32}fils de Mahli, fils de Moushi, fils de Merari, fils de Lévi.
${}^{33}Leurs frères lévites étaient chargés de tout le service de la Demeure de la maison de Dieu. 
${}^{34}Aaron et ses fils faisaient fumer les sacrifices sur l’autel des holocaustes et sur l’autel de l’encens ; ils s’occupaient exclusivement des choses très saintes et du rite d’expiation pour Israël, selon tout ce qu’avait ordonné Moïse, serviteur de Dieu.
${}^{35}Voici les fils d’Aaron : son fils Éléazar, son fils Pinhas, son fils Abishoua, 
${}^{36}son fils Bouqqi, son fils Ouzzi, son fils Zerahya, 
${}^{37}son fils Merayoth, son fils Amarya, son fils Ahitoub, 
${}^{38}son fils Sadoc, et son fils Ahimaas.
${}^{39}Voici leurs lieux d’habitation, selon les limites de leurs campements. Aux fils d’Aaron, du clan de Qehath, désignés les premiers par le sort, 
${}^{40}on donna Hébron, dans le pays de Juda, avec les pâturages qui l’entourent. 
${}^{41}Et les champs de la ville et ses villages, on les donna à Caleb, fils de Yefounnè. 
${}^{42}On donna aux fils d’Aaron comme villes de refuge : Hébron, Libna et ses pâturages, Yattir, Eshtemoa et ses pâturages, 
${}^{43}Hilaz et ses pâturages, Debir et ses pâturages, 
${}^{44}Ashane et ses pâturages, Beth-Shèmesh et ses pâturages. 
${}^{45}Sur la tribu de Benjamin, on leur donna Guèba et ses pâturages, Alèmeth et ses pâturages, Anatoth et ses pâturages. Total de leurs villes : treize villes, réparties entre les clans.
${}^{46}Les autres fils de Qehath reçurent par tirage au sort dix villes prises sur les clans de la tribu d’Éphraïm, de la tribu de Dane et de la demi-tribu de Manassé. 
${}^{47}Les fils de Guershom reçurent pour leurs clans treize villes prises sur la tribu d’Issakar, sur la tribu d’Asher, sur la tribu de Nephtali et sur la tribu de Manassé, en Bashane. 
${}^{48}Les fils de Merari reçurent par tirage au sort, pour leurs clans, douze villes prises sur la tribu de Roubène, sur la tribu de Gad et sur la tribu de Zabulon.
${}^{49}Les fils d’Israël donnèrent aux Lévites ces villes avec leurs pâturages. 
${}^{50}Sur la tribu des fils de Juda, la tribu des fils de Siméon et la tribu des fils de Benjamin, on donna aussi par tirage au sort ces villes, qu’ils appelèrent de leurs noms.
${}^{51}Les autres clans des fils de Qehath reçurent pour domaine des villes prises sur la tribu d’Éphraïm. 
${}^{52}On leur donna comme villes de refuge : Sichem et ses pâturages dans la montagne d’Éphraïm, Guèzer et ses pâturages, 
${}^{53}Yoqméam et ses pâturages, Beth-Horone et ses pâturages, 
${}^{54}Ayyalone et ses pâturages, Gath-Rimmone et ses pâturages, 
${}^{55}ainsi que, pris sur la demi-tribu de Manassé : Anèr et ses pâturages, Biléam et ses pâturages. Ceci pour le clan des autres fils de Qehath.
${}^{56}Aux fils de Guershom, on donna, pris sur le clan de la demi-tribu de Manassé, Golane en Bashane et ses pâturages, Ashtaroth et ses pâturages ; 
${}^{57}pris sur la tribu d’Issakar, Qèdesh et ses pâturages, Daberath et ses pâturages, 
${}^{58}Ramoth et ses pâturages, Anem et ses pâturages ; 
${}^{59}pris sur la tribu d’Asher, Mashal et ses pâturages, Abdone et ses pâturages, 
${}^{60}Houqoq et ses pâturages, Rehob et ses pâturages ; 
${}^{61}pris sur la tribu de Nephtali, Qèdesh en Galilée et ses pâturages, Hammone et ses pâturages, Qiryataïm et ses pâturages.
${}^{62}Pour les autres fils de Merari, on donna, pris sur la tribu de Zabulon, Rimmone et ses pâturages, Tabor et ses pâturages ; 
${}^{63}au-delà du Jourdain vers Jéricho, à l’est du Jourdain, pris sur la tribu de Roubène, Bècèr dans le désert et ses pâturages, Yahça et ses pâturages, 
${}^{64}Qedémoth et ses pâturages, Méfaath et ses pâturages ; 
${}^{65}pris sur la tribu de Gad : Ramoth-de-Galaad et ses pâturages, Mahanaïm et ses pâturages, 
${}^{66}Heshbone et ses pâturages, Yazèr et ses pâturages.
      
         
      \bchapter{}
      \begin{verse}
${}^{1}Issakar eut quatre fils : Tola, Poua, Yashoub et Shimrone.
      \begin{verse}
${}^{2}Fils de Tola : Ouzzi, Refaya, Yeriël, Yahmaï, Yibsam et Samuel ; c’étaient les chefs des familles de Tola, qui comptaient, selon leur descendance, au temps de David, vingt-deux mille six cents vaillants guerriers. 
${}^{3}Fils d’Ouzzi : Yizrahya. Fils de Yizrahya : Mikaël, Obadya, Joël, Yishiya. En tout, cinq chefs. 
${}^{4}Ils devaient fournir, selon leur descendance, par familles, des troupes armées pour la guerre, qui comptaient trente-six mille hommes ; ils avaient en effet beaucoup de femmes et de fils. 
${}^{5}Ils avaient aussi des frères dans tous les clans d’Issakar, de vaillants guerriers, au nombre de quatre-vingt-sept mille hommes, suivant le total de l’enregistrement.
${}^{6}Fils de Benjamin : Bèla, Bèker et Yediaël ; ils étaient trois. 
${}^{7}Fils de Bèla : Esbone, Ouzzi, Ouzziël, Yerimoth et Iri ; ils étaient cinq, chefs de famille, vaillants guerriers ; l’enregistrement selon leur descendance comptait vingt-deux mille trente-quatre hommes. 
${}^{8}Fils de Bèker : Zemira, Yoash, Èlièzer, Élyoénaï, Omri, Yerémoth, Abiya, Anatoth et Alèmeth : tous ceux-là étaient les fils de Bèker ; 
${}^{9}l’enregistrement, selon leur descendance par chefs de famille, comptait vingt mille deux cents vaillants guerriers. 
${}^{10}Fils de Yediaël : Bilhane. Fils de Bilhane : Yéoush, Benjamin, Éhoud, Kenaana, Zétane, Tarsis et Ahishahar. 
${}^{11}Tous ces fils de Yediaël étaient des chefs de famille, vaillants guerriers, au nombre de dix-sept mille deux cents hommes, capables d’aller au combat pour la guerre.
${}^{12}Shouppim et Houppim étaient fils de Ir ; Houshim, fils d’Aher.
${}^{13}Fils de Nephtali : Yahaciël, Gouni, Yécer et Shalloum. Ils étaient fils de Bilha.
${}^{14}Fils de Manassé : Asriël, qu’avait enfanté sa concubine araméenne ; elle enfanta aussi Makir, père de Galaad. 
${}^{15}Makir prit une femme pour Houppim et une pour Shouppim. Le nom de sa sœur était Maaka. Le nom du second était Celofehad. Celofehad eut des filles. 
${}^{16}Maaka, femme de Makir, enfanta un fils qu’elle appela Pèresh. Le nom de son frère était Shèresh, dont les fils furent Oulam et Rèqem. 
${}^{17}Fils d’Oulam : Bedane. Tels furent les fils de Galaad, fils de Makir, fils de Manassé. 
${}^{18}Galaad avait pour sœur Hammolèketh, qui enfanta Ishehod, Abièzer et Mahla. 
${}^{19}Les fils de Shemida étaient Ahyane, Shèkem, Liqhi et Aniam.
${}^{20}Fils d’Éphraïm : Shoutèlah. Son fils Bèred, son fils Tahath, son fils Éléada, son fils Tahath, 
${}^{21}son fils Zabad, son fils Shoutèlah, ainsi qu’Ézer et Éléad. Des gens de Gath, natifs du pays, les tuèrent, car ils étaient descendus pour prendre leurs troupeaux. 
${}^{22}Leur père Éphraïm fut dans le deuil pendant de nombreux jours, et ses frères vinrent le consoler. 
${}^{23}Il alla vers sa femme. Elle devint enceinte et enfanta un fils qu’elle nomma Beria (c’est-à-dire : Dans le malheur), car chez lui elle était dans le malheur. 
${}^{24}Sa fille était Shèèra et elle construisit Beth-Horone-le-Bas et Beth-Horone-le-Haut, et Ouzzèn-Shèèra. 
${}^{25}Son fils Rèfah, et Rèchef, son fils Tèlah, son fils Tahane, 
${}^{26}son fils Ladane, son fils Ammihoud, son fils Élishama, 
${}^{27}son fils None et son fils Josué.
${}^{28}Leur propriété et leurs habitations étaient : Béthel et ses dépendances, Naarane à l’est, Guèzer et ses dépendances à l’ouest, Sichem et ses dépendances, jusqu’à Ayya et ses dépendances. 
${}^{29}Aux mains des fils de Manassé il y avait encore Beth-Shéane et ses dépendances, Taanak et ses dépendances, Meguiddo et ses dépendances, Dor et ses dépendances. C’est là qu’habitaient les fils de Joseph, fils d’Israël.
${}^{30}Fils d’Asher : Yimna, Yishwa, Yishwi, Beria et leur sœur Sèrah. 
${}^{31}Fils de Beria : Hèber et Malkiël qui fut le père de Birzaïth. 
${}^{32}Hèber engendra Yafleth, Shomer, Hotam et leur sœur Shoua. 
${}^{33}Fils de Yafleth : Pasak, Bimhal et Ashwath. Tels sont les fils de Yafleth. 
${}^{34}Fils de son frère Shèmer : Rohga, Houbba et Aram. 
${}^{35}Fils de son frère Hélem : Sofah, Yimna, Shélesh et Amal. 
${}^{36}Fils de Sofah : Souah, Harnèfer, Shoual, Béri, Yimra, 
${}^{37}Bècèr, Hod, Shamma, Shilsha, Yitrane et Beéra. 
${}^{38}Fils de Yèter : Yefounnè, Pispa, Ara. 
${}^{39}Fils d’Oulla : Arah, Hanniël et Ricia. 
${}^{40}Tous ceux-là étaient fils d’Asher, chefs de famille, hommes d’élite, vaillants guerriers, chefs des princes, enregistrés dans l’armée pour la guerre : ils étaient au nombre de vingt-six mille hommes.
      
         
      \bchapter{}
      \begin{verse}
${}^{1}Benjamin engendra Bèla, son premier-né, Ashbel, le deuxième, Ahiram, le troisième, 
${}^{2}Noha, le quatrième, et Rapha, le cinquième. 
${}^{3}Bèla eut des fils : Addar, Guéra, père d’Éhoud, 
${}^{4}Abishoua, Naaman et Ahoah, 
${}^{5}Guéra, Shefoufane et Houram.
${}^{6}Voici les fils d’Éhoud – ce sont eux qui furent les chefs de famille des habitants de Guèba et qui les déportèrent à Manahath – : 
${}^{7}Naaman, Ahiyya et Guéra – celui-ci les déporta ; il engendra Ouzza et Ahihoud. 
${}^{8}Shaharaïm répudia deux de ses femmes, Houshim et Baara ; puis il engendra des fils dans la région appelée « Champs-de-Moab ». 
${}^{9}De Hodesh, sa femme, il engendra Yobab, Cibya, Mésha, Malcam, 
${}^{10}Yéous, Sakya et Mirma. Tels furent ses fils, des chefs de famille. 
${}^{11}De Houshim il avait engendré Abitoub et Elpaal. 
${}^{12}Fils d’Elpaal : Éber, Mishéam et Shèmed ; celui-ci construisit Ono, et Lod avec ses dépendances.
${}^{13}Beria et Shèma étaient chefs de famille des habitants d’Ayyalone ; ce sont eux qui mirent en fuite les habitants de Gath.
${}^{14}Ahyo, Shashaq, Yerémoth, 
${}^{15}Zebadya, Arad, Éder, 
${}^{16}Mikaël, Yishpa et Yoha étaient fils de Beria.
${}^{17}Zebadya, Meshoullam, Hizqi, Hèber, 
${}^{18}Yishmeraï, Yizlia et Yobab étaient fils d’Elpaal.
${}^{19}Yaqim, Zikri, Zabdi, 
${}^{20}Elyénaï, Cilletaï, Èliël, 
${}^{21}Adaya, Beraya et Shimrath étaient fils de Shiméï.
${}^{22}Yishpane, Éber, Èliël, 
${}^{23}Abdone, Zikri, Hanane, 
${}^{24}Hananya, Élam, Antotiya, 
${}^{25}Yifdeya et Penouël étaient fils de Shashaq.
${}^{26}Shamsheraï, Sheharya, Athalie, 
${}^{27}Yaarèshya, Éliya et Zikri étaient fils de Yeroham.
${}^{28}Tels étaient les chefs de famille, selon leur descendance ; ils étaient des chefs. Ils habitaient Jérusalem.
${}^{29}À Gabaon habitaient Yeïel, père de Gabaon, dont la femme s’appelait Maaka, 
${}^{30}son fils premier-né Abdone, ainsi que Sour, Qish, Baal, Ner, Nadab, 
${}^{31}Guedor, Ahyo, Zèker et Miqloth. 
${}^{32}Miqloth engendra Shiméa ; ceux-ci, à la différence de leurs frères qui vivaient à Gabaon, habitaient Jérusalem avec leurs autres frères.
${}^{33}Ner engendra Qish, Qish engendra Saül. Saül engendra Jonathan, Malki-Shoua, Abinadab et Eshbaal. 
${}^{34}Fils de Jonathan : Merib-Baal. Merib-Baal engendra Mika. 
${}^{35}Fils de Mika : Pitone, Mèlek, Taréa et Ahaz. 
${}^{36}Ahaz engendra Yehoadda, Yehoadda engendra Alèmeth, Azmaweth et Zimri. Zimri engendra Moça, 
${}^{37}Moça engendra Binéa, qui eut pour fils Rapha, lequel eut pour fils Éléasa, qui eut pour fils Acel. 
${}^{38}Acel eut six fils dont voici les noms : Azriqam, Bocrou, puis Yishmaël, Shéarya, Obadya et Hanane. Ce sont là tous les fils d’Acel.
${}^{39}Fils d’Ésheq, frère d’Acel : Oulam, son premier-né, Yéoush, le deuxième, et Élifèleth, le troisième. 
${}^{40}Oulam eut des fils, de vaillants guerriers, tirant à l’arc. Ils eurent beaucoup de fils et de petits-fils, soit cent cinquante. Tous ceux-là étaient fils de Benjamin.
      
         
      \bchapter{}
      \begin{verse}
${}^{1}Tous les gens d’Israël furent enregistrés, selon leur descendance, et ils sont inscrits sur le Livre des rois d’Israël. Après que ceux de Juda furent déportés à Babylone à cause de leur infidélité, 
${}^{2}les premiers qui réoccupèrent leurs propriétés et leurs villes furent les gens d’Israël : les prêtres, les Lévites et les servants.
${}^{3}À Jérusalem habitèrent des fils de Juda, de Benjamin, d’Éphraïm et de Manassé.
${}^{4}Outaï, fils d’Ammihoud, fils d’Omri, fils d’Imri, fils de Bani, l’un des fils de Pèrès, fils de Juda. 
${}^{5}Parmi les gens de Shilo : Asaya, le premier-né, et ses fils. 
${}^{6}Parmi les fils de Zèrah : Yéouël, et leurs frères, soit six cent quatre-vingt-dix.
${}^{7}Parmi les fils de Benjamin : Sallou, fils de Meshoullam, fils de Hodawya, fils de Hassenoua ; 
${}^{8}Yibneya, fils de Yeroham ; Éla, fils d’Ouzzi, fils de Mikri ; Meshoullam, fils de Shefatya, fils de Réouël, fils de Yibniya. 
${}^{9}Leurs frères, selon leur descendance, étaient neuf cent cinquante-six. Tous ces hommes étaient chefs de famille, chacun dans sa propre famille.
${}^{10}Parmi les prêtres : Yedaya, Yehoyarib, Yakine, 
${}^{11}Azarya, fils de Hilqiya, fils de Meshoullam, fils de Sadoc, fils de Merayoth, fils d’Ahitoub, recteur de la maison de Dieu ; 
${}^{12}Adaya, fils de Yeroham, fils de Pashehour, fils de Malkiya, et Maasaï, fils d’Adiël, fils de Yahzéra, fils de Meshoullam, fils de Meshillémith, fils d’Immer.
${}^{13}Leurs frères, chefs de famille, étaient mille sept cent soixante : de vaillants guerriers, chargés du service de la maison de Dieu.
${}^{14}Parmi les Lévites : Shemaya, fils de Hashshoub, fils d’Azriqam, fils de Hashabya – ils étaient fils de Merari – ; 
${}^{15}Baqbaqar, Hèresh, Galal et Mattanya, fils de Mika, fils de Zikri, fils d’Asaf ; 
${}^{16}Obadya, fils de Shemaya, fils de Galal, fils de Yedoutoune, et Bèrèkya, fils d’Asa, fils d’Elcana, qui habitait dans les villages des Netofatites.
${}^{17}Les portiers : Shalloum, Aqqoub, Talmone, Ahimane et leurs frères. Shalloum était le chef. 
${}^{18}Jusqu’à présent, à la porte du roi, à l’Orient, ce sont eux les portiers des camps des Lévites : 
${}^{19}Shalloum, fils de Coré, fils d’Èbyasaf, fils de Coré, et ses frères les Coréites, de la même famille, étaient préposés au service du culte pour garder le seuil de la Tente, de même que leurs pères avaient été préposés au camp du Seigneur pour en garder les accès. 
${}^{20}Pinhas, fils d’Éléazar, avait été autrefois leur chef ; le Seigneur était avec lui. 
${}^{21}Zacharie, fils de Meshèlèmya, était portier à l’entrée de la tente de la Rencontre. 
${}^{22}Tous ces hommes avaient été choisis comme portiers des seuils ; ils étaient deux cent douze. Ils étaient enregistrés dans leurs villages. Ce sont eux que David et Samuel le Voyant avaient établis dans leur fonction permanente. 
${}^{23}Eux et leurs fils étaient préposés aux portes de la maison du Seigneur, de la maison de la Tente, selon leur tour de garde. 
${}^{24}Les portiers se tenaient aux quatre points cardinaux, à l’est, à l’ouest, au nord et au sud. 
${}^{25}Leurs frères, qui résidaient dans leurs villages, venaient de temps en temps se joindre à eux pour une semaine. 
${}^{26}Les quatre chefs des portiers, eux, exerçaient leur fonction en permanence. C’étaient les Lévites qui étaient préposés aux salles et aux trésors de la maison de Dieu. 
${}^{27}Ils passaient la nuit aux alentours de la maison de Dieu, car ils en avaient la garde et en assuraient l’ouverture chaque matin. 
${}^{28}Certains d’entre eux étaient préposés aux objets du culte ; ils en faisaient le compte lorsqu’ils les entraient, et ils en faisaient le compte lorsqu’ils les sortaient. 
${}^{29}D’autres étaient désignés comme responsables des objets, de tous les objets du sanctuaire, ainsi que de la fleur de farine, du vin, de l’huile, de l’encens et des aromates. 
${}^{30}C’étaient des fils de prêtres qui préparaient le mélange pour les aromates.
${}^{31}L’un des Lévites, Mattitya – le premier-né de Shalloum le Coréite – était chargé en permanence de la confection des galettes cuites à la plaque. 
${}^{32}Parmi leurs frères, c’était des fils de Qehatites qui étaient chargés des pains à disposer en rangées, chaque sabbat.
${}^{33}Il y avait aussi les chantres, chefs de famille lévitiques, logés dans les salles de la Maison, et exempts de tout service, car ils étaient à leur office jour et nuit. 
${}^{34}Tels étaient les chefs des familles lévitiques, selon leur descendance ; ils étaient des chefs. Ils habitaient Jérusalem.
${}^{35}À Gabaon habitaient le père de Gabaon, Yeïel, dont la femme s’appelait Maaka, 
${}^{36}son fils premier-né Abdone, ainsi que Sour, Qish, Baal, Ner, Nadab, 
${}^{37}Guedor, Ahyo, Zacharie et Miqloth. 
${}^{38}Miqloth engendra Shiméam ; ceux-ci, contrairement à leurs frères habitant Gabaon, habitaient Jérusalem avec leurs autres frères.
${}^{39}Ner engendra Qish, Qish engendra Saül, Saül engendra Jonathan, Malki-Shoua, Abinadab et Eshbaal. 
${}^{40}Fils de Jonathan : Merib-Baal. Merib-Baal engendra Mika. 
${}^{41}Fils de Mika : Pitone, Mèlek, Tahréa, Ahaz. 
${}^{42}Ahaz engendra Yara, Yara engendra Alèmeth, Azmaweth et Zimri. Zimri engendra Moça. 
${}^{43}Moça engendra Binéa, qui eut pour fils Refaya, lequel eut pour fils Éléasa, qui eut pour fils Acel. 
${}^{44}Acel eut six fils dont voici les noms : Azriqam, Bocrou, puis Yishmaël, Shéarya, Obadya et Hanane. Ce sont là les fils d’Acel.
      
         
      \bchapter{}
      \begin{verse}
${}^{1}Les Philistins livrèrent bataille à Israël. Les hommes d’Israël s’enfuirent devant les Philistins et tombèrent, frappés à mort, sur le mont Gelboé. 
${}^{2}Les Philistins, dans leur poursuite, serrèrent de près Saül et ses fils. Ils frappèrent Jonathan, Abinadab et Malki-Shoua, les fils de Saül. 
${}^{3}Le poids du combat se porta sur Saül. Les tireurs d’arc le surprirent, et il fut blessé par les tireurs. 
${}^{4}Saül dit à son écuyer : « Tire ton épée et transperce-moi, de peur que ces incirconcis ne viennent se jouer de moi. » Mais son écuyer refusa, tant il avait peur. Alors Saül prit son épée et se jeta sur elle. 
${}^{5}Quand l’écuyer vit que Saül était mort, il se jeta lui aussi sur son épée et mourut. 
${}^{6}Ainsi mourut Saül, avec ses trois fils ; et en même temps moururent tous les gens de sa maison. 
${}^{7}Tous les hommes d’Israël qui étaient dans la vallée, virent que leurs troupes avaient pris la fuite et que Saül et ses fils étaient morts. Ils abandonnèrent leurs villes et s’enfuirent. Alors les Philistins vinrent s’y installer.
${}^{8}Le lendemain, les Philistins, venus pour dépouiller les morts, trouvèrent Saül et ses fils, gisant sur le mont Gelboé. 
${}^{9}Ils le dépouillèrent, emportèrent sa tête et ses armes. Puis ils envoyèrent, à la ronde, dans le pays des Philistins, porter la bonne nouvelle à leurs idoles et au peuple. 
${}^{10}Ils déposèrent ses armes dans la maison de leur dieu, et son crâne, ils le clouèrent dans la maison de Dagone.
${}^{11}Tous les gens de Yabesh de Galaad apprirent tout ce que les Philistins avaient fait à Saül. 
${}^{12}Alors tous les hommes de valeur se mirent en route. Ils enlevèrent le corps de Saül et ceux de ses fils, et les apportèrent à Yabesh. Puis ils ensevelirent leurs ossements sous le térébinthe de Yabesh et jeûnèrent pendant sept jours.
${}^{13}Ainsi mourut Saül parce qu’il s’était montré infidèle envers le Seigneur : il n’avait pas gardé la parole du Seigneur, il avait même interrogé la nécromancienne et l’avait consultée. 
${}^{14}Il n’avait pas consulté le Seigneur, qui le fit mourir et qui transféra la royauté à David, fils de Jessé.
      
         
      \bchapter{}
      \begin{verse}
${}^{1}Alors tous les gens d’Israël se rassemblèrent auprès de David, à Hébron et lui dirent : « Vois ! Nous sommes de tes os et de ta chair. 
${}^{2}Dans le passé déjà, même quand Saül était roi, tu dirigeais les allées et venues de l’armée d’Israël, et le Seigneur ton Dieu t’a dit : “Tu seras le pasteur d’Israël mon peuple, tu seras le chef d’Israël mon peuple”. » 
${}^{3}Ainsi, tous les anciens d’Israël vinrent trouver le roi à Hébron. David fit alliance avec eux, à Hébron, devant le Seigneur. Ils donnèrent l’onction à David comme roi sur Israël, selon la parole du Seigneur transmise par Samuel.
      
         
${}^{4}David, avec tout Israël, marcha sur Jérusalem – c’est-à-dire Jébous. Les habitants du pays étaient alors les Jébuséens. 
${}^{5}Les habitants de Jébous dirent à David : « Tu n’entreras pas ici. » Mais David s’empara de la forteresse de Sion. – C’est la Cité de David. 
${}^{6}Et David déclara : « Quiconque frappera le premier un Jébuséen deviendra chef et prince. » Joab, fils de Cerouya, monta à l’assaut le premier et devint chef. 
${}^{7}David s’établit sur le rocher fortifié qui, pour cette raison, fut appelé Cité de David. 
${}^{8}Puis il construisit la ville tout autour, depuis le Terre-Plein jusqu’aux alentours. Et Joab restaura le reste de la ville. 
${}^{9}David devint de plus en plus puissant, et le Seigneur de l’univers était avec lui.
${}^{10}Voici les chefs des guerriers de David, ceux qui, sous son règne, lui prêtèrent main-forte et, avec tout Israël, l’ont fait roi, selon la parole du Seigneur sur Israël.
${}^{11}Voici la liste des guerriers de David : Yashobéam, fils de Hakmoni, le chef des Trois : c’est lui qui brandit sa lance et frappa à mort trois cents hommes en une seule fois. 
${}^{12}Après lui, Éléazar, fils de Dodo l’Ahohite. C’était l’un des Trois Guerriers. 
${}^{13}Il était avec David à Pas-Dammim, quand les Philistins s’y rassemblèrent pour le combat. Or il y avait un champ tout en orge, et quand l’armée prit la fuite devant les Philistins, 
${}^{14}tous deux se postèrent au milieu du champ, le défendirent et frappèrent les Philistins. Le Seigneur remporta une grande victoire.
${}^{15}Trois guerriers, l’élite des Trente, descendirent vers David, près du rocher, à la grotte d’Adoullam. Une troupe de Philistins campait dans le Val des Refaïtes. 
${}^{16}David était alors dans son refuge fortifié, et il y avait encore un poste de Philistins à Bethléem. 
${}^{17}David exprima un désir : « Qui me fera boire l’eau de la citerne qui est à la porte de Bethléem ? » 
${}^{18}Les Trois s’ouvrirent un passage à travers le camp des Philistins, tirèrent de l’eau de la citerne qui est à la porte de Bethléem, puis ils l’emportèrent pour l’offrir à David. Mais il refusa d’en boire et la répandit en libation devant le Seigneur 
${}^{19}en disant : « Dieu me garde de faire cela ! Boirais-je le sang de ces hommes qui ont risqué leur vie ? Car c’est en risquant leur vie qu’ils l’ont apportée ! » Il refusa donc de boire. Voilà ce que firent les Trois Guerriers.
${}^{20}Abishaï, frère de Joab, était le chef des Trente : c’est lui qui brandit sa lance et frappa à mort trois cents hommes ; il se fit un nom parmi les Trente. 
${}^{21}Il fut doublement honoré, plus que les Trente, et devint leur chef, mais il ne parvint pas au rang des Trois.
${}^{22}Benaya, fils d’un homme de valeur, Joad, fut prodigue en exploits. Il était originaire de Qabcéel. C’est lui qui frappa les deux Ariel de Moab, et c’est lui qui descendit tuer le lion dans la citerne, un jour de neige. 
${}^{23}C’est lui aussi qui frappa l’Égyptien, un géant de cinq coudées qui avait en main une lance aussi grosse que le rouleau d’un métier à tisser. Il descendit contre l’Égyptien avec un bâton, lui arracha la lance de la main et le tua avec sa propre lance. 
${}^{24}Voilà ce qu’accomplit Benaya, fils de Joad, et il se fit un nom parmi les Trente Guerriers. 
${}^{25}Il fut plus honoré que les Trente, mais ne parvint pas au rang des Trois. David le mit à la tête de sa garde personnelle.
${}^{26}Les autres vaillants guerriers étaient : Asaël, frère de Joab, Elhanane, fils de Dodo, de Bethléem, 
${}^{27}Shammoth de Haror, Hèlès de Pelone, 
${}^{28}Ira, fils d’Iqqesh, de Teqoa, Abièzer d’Anatoth, 
${}^{29}Sibbekaï de Housha, Ilaï d’Ahoh, 
${}^{30}Mahraï de Netofa, Hèled fils de Baana, de Netofa, 
${}^{31}Itaï fils de Ribaï, de Guibéa des fils de Benjamin, Benaya de Piréatone, 
${}^{32}Houraï, des Torrents de Gaash, Abiël d’Araba, 
${}^{33}Azmaweth de Baharoum, Élyahba de Shaalbone, 
${}^{34}Bené-Hashem de Guizone, Jonathan fils de Shagué, de Harar, 
${}^{35}Ahiam fils de Sakar, de Harar, Èlifal fils d’Our, 
${}^{36}Héfer de Mekéra, Ahiyya de Pelone, 
${}^{37}Hèsro de Carmel, Naaraï fils d’Ezbaï, 
${}^{38}Joël frère de Nathan, Mibhar fils de Hagri, 
${}^{39}Cèleq d’Ammone, Nahraï de Beéroth, écuyer de Joab fils de Cerouya, 
${}^{40}Ira de Yattir, Gareb de Yattir, 
${}^{41}Ourias le Hittite, Zabad fils d’Ahlaï, 
${}^{42}Adina fils de Shiza de Roubène, chef des Roubénites et responsable de ces trente guerriers ; 
${}^{43}Hanane fils de Maaka, Josaphat de Métène, 
${}^{44}Ouziyya d’Ashtaroth, Shama et Yéiël, les fils de Hotam d’Aroër, 
${}^{45}Yediaël fils de Shimri, et son frère Yoha de Tits, 
${}^{46}Èliël des Mahawim, Yeribaï et Yoshawya, les fils d’Èlnaâm, Yitma de Moab, 
${}^{47}Èliël, Obed et Yaasiël, de Soba.
      
         
      \bchapter{}
      \begin{verse}
${}^{1}Voici ceux qui rejoignirent David à Ciqlag, alors qu’il devait encore se tenir loin de Saül, fils de Qish. C’étaient des guerriers, des combattants d’élite. 
${}^{2}Ils étaient armés d’un arc, se servaient de la main droite comme de la main gauche, pour lancer des pierres et des flèches avec l’arc.
      Parmi les frères de Saül de Benjamin, il y avait : 
${}^{3}à leur tête, Ahièzer, puis Yoash, qui étaient les fils de Shemaa de Guibéa ; Yeziël et Pèleth, qui étaient les fils d’Azmaweth ; Beraka et Jéhu d’Anatoth, 
${}^{4}Yishmaya de Gabaon, guerrier parmi les Trente et chef des Trente ; 
${}^{5}Jérémie, Yahaziël, Yohanane et Yozabad de Guedéra ; 
${}^{6}Èlouzaï, Yerimoth, Béalya, Shemaryahou, Shefatyahou de Harouf, 
${}^{7}Elcana, Yishiyahou, Azarel, Yoèzer, Yashobéam, les Coréites, 
${}^{8}Yoéla, Zebadya, qui étaient les fils de Yeroham de Guedor.
${}^{9}Des gens de Gad firent sécession pour rejoindre David sur son rocher fortifié dans le désert. C’étaient de vaillants guerriers, des hommes d’armes exercés au combat, sachant manier le bouclier et la lance. Ils ressemblaient à des lions, et ils étaient aussi rapides que des gazelles sur les montagnes. 
${}^{10}Ézer était le chef, Abdias le deuxième, Èliab le troisième, 
${}^{11}Mishmanna le quatrième, Jérémie le cinquième, 
${}^{12}Attaï le sixième, Èliël le septième, 
${}^{13}Yohanane le huitième, Èlzabad le neuvième, 
${}^{14}Jérémie le dixième, Makbannaï le onzième. 
${}^{15}Tels étaient, parmi les fils de Gad, les chefs de l’armée ; le plus petit en valait cent, le plus grand en valait mille. 
${}^{16}Ce sont eux qui passèrent le Jourdain, au premier mois, alors qu’il déborde partout sur ses rives, et ils mirent en fuite tous les habitants des vallées, tant à l’orient qu’à l’occident.
${}^{17}Quelques-uns des fils de Benjamin et de Juda s’en vinrent aussi trouver David sur son rocher fortifié.
${}^{18}David sortit au-devant d’eux, prit la parole et leur dit : « Si c’est pour la paix que vous venez à moi, afin de m’aider, je serai de tout cœur avec vous. Mais si c’est pour me tromper au profit de mes adversaires alors qu’il n’y a aucune violence en mes mains, que le Dieu de nos pères le voie, et qu’il juge ! »
${}^{19}L’esprit revêtit alors Amasaï, chef des Trente :
        \\« Nous sommes à toi, David,
        \\avec toi, fils de Jessé !
        \\Paix, oui, paix à toi,
        \\et paix à celui qui t’aide,
        \\car ton Dieu t’a aidé. »
      David les accueillit et les plaça parmi les chefs de la troupe.
${}^{20}Quelques-uns de Manassé se rallièrent à David, alors que celui-ci venait avec les Philistins combattre Saül. Mais ils n’eurent pas à les aider, car les princes des Philistins, s’étant consultés, renvoyèrent David. Ils disaient : « David irait se rallier à son maître Saül au prix de nos têtes ! »
${}^{21}Lorsque David partit pour Ciqlag, quelques-uns de Manassé se rallièrent à lui : Adnah, Yozabad, Yediaël, Mikaël, Yozabad, Élihou et Cilletaï, chefs des milliers de Manassé.
${}^{22}Ils furent une aide pour David à la tête de la troupe, car ils étaient tous de vaillants guerriers ; ils devinrent officiers dans l’armée. 
${}^{23}Jour après jour, en effet, on venait auprès de David pour l’aider, si bien que son camp devint immense comme un camp de Dieu.
${}^{24}Voici le dénombrement des groupes d’hommes équipés pour l’armée, qui rejoignirent David à Hébron afin de lui transférer la royauté de Saül, selon l’ordre du Seigneur.
${}^{25}Fils de Juda portant le bouclier et la lance : 6 800 guerriers équipés pour l’armée.
${}^{26}Parmi les fils de Siméon : 7 100 vaillants guerriers.
${}^{27}Parmi les fils de Lévi : 4 600 hommes,
${}^{28}ainsi que Joad, commandant aux gens d’Aaron, accompagné de 3 700 hommes ;
${}^{29}il y avait aussi le jeune Sadoc, vaillant guerrier, et 22 officiers de la maison de son père.
${}^{30}Parmi les fils de Benjamin, les frères de Saül : 3 000 hommes ; la plupart d’entre eux étaient jusque-là au service de la maison de Saül.
${}^{31}Parmi les fils d’Éphraïm : 20 800 vaillants guerriers, hommes de renom dans leur famille.
${}^{32}De la demi-tribu de Manassé : 18 000 hommes désignés par leurs noms pour aller faire roi David.
${}^{33}Parmi les fils d’Issakar, capables de discerner les moments où Israël devait agir et comment le faire : 200 chefs, ayant tous leurs frères sous leurs ordres.
${}^{34}Parmi les gens de Zabulon : 50 000 hommes aptes à rejoindre l’armée, prêts pour le combat, avec tout leur matériel de combat, et résolus à prêter main-forte d’un cœur sans partage.
${}^{35}Parmi les gens de Nephtali : 1 000 officiers, et avec eux 37 000 hommes portant le bouclier et la lance.
${}^{36}Parmi les gens de Dane : 28 600 hommes prêts pour le combat.
${}^{37}Parmi les gens d’Asher : 40 000 hommes aptes à rejoindre l’armée pour se préparer au combat.
${}^{38}De Transjordanie : 120 000 hommes de Roubène, de Gad, de la demi-tribu de Manassé, avec tout le matériel d’une armée au combat.
${}^{39}Tous ceux-là étaient des hommes de guerre, préparés à se ranger en bataille ; d’un cœur unanime, ils se rendirent à Hébron pour faire David roi sur tout Israël. Et de même tout le reste d’Israël n’avait qu’un seul cœur pour faire David roi. 
${}^{40}Ils passèrent là trois jours avec David à manger et à boire, car leurs frères avaient tout apprêté pour eux. 
${}^{41}De plus, les gens de la région, jusqu’à Issakar, Zabulon et Nephtali, faisaient parvenir des vivres, sur des ânes, des chameaux, des mulets et des bœufs. Il y avait comme provisions de la farine, des gâteaux de figues et des gâteaux de raisins secs, du vin et de l’huile, ainsi que du gros et du petit bétail en quantité, car c’était grande joie en Israël.
      
         
      \bchapter{}
      \begin{verse}
${}^{1}David tint conseil avec les officiers de millier et de centaine, avec tous les commandants. 
${}^{2}Puis il dit à toute l’assemblée d’Israël : « Si cela vous semble bon et si cela provient du Seigneur notre Dieu, envoyons des messagers à nos frères qui sont restés dans tous les territoires d’Israël, ainsi qu’aux prêtres et aux Lévites dans les villes où ils possèdent des pâturages. Qu’ils viennent se joindre à nous ! 
${}^{3}Et nous ramènerons chez nous l’arche de notre Dieu, car nous l’avons négligée au temps de Saül. » 
${}^{4}Toute l’assemblée fut d’accord pour agir de la sorte, car cela paraissait juste aux yeux de tout le peuple.
${}^{5}David rassembla tout Israël, depuis le Torrent d’Égypte jusqu’à l’Entrée-de-Hamath, pour ramener de Qiryath-Yearim l’arche de Dieu. 
${}^{6}Puis David et tout Israël gagnèrent Baala, à Qiryath-Yearim en Juda, pour en faire monter l’arche de Dieu, sur laquelle est invoqué le nom du Seigneur, lui qui siège sur les Kéroubim. 
${}^{7}On chargea l’arche de Dieu sur un chariot neuf, pour la transporter depuis la maison d’Abinadab. Ouzza et Ahyo conduisaient le chariot. 
${}^{8}David et tout Israël dansaient de toutes leurs forces devant Dieu, accompagnés de chants, de cithares, de harpes, de tambourins, de cymbales et de trompettes.
${}^{9}Comme on arrivait au lieu-dit « Aire du Javelot », Ouzza étendit la main pour retenir l’arche, car les bœufs la faisaient verser. 
${}^{10}Alors la colère du Seigneur s’enflamma contre Ouzza et le frappa pour avoir porté la main sur l’Arche. Ouzza mourut là, devant Dieu. 
${}^{11}David fut irrité de ce que le Seigneur avait ouvert une brèche parmi les siens en frappant Ouzza, et on appela ce lieu Pèrès-Ouzza (c’est-à-dire : Brèche-d’Ouzza), nom qu’il a gardé jusqu’à ce jour.
${}^{12}David eut peur de Dieu, ce jour-là, et dit : « Comment ferais-je entrer chez moi l’arche de Dieu ? » 
${}^{13}Et David ne transféra pas l’Arche chez lui, dans la Cité de David, mais il la dévia vers la maison d’Obed-Édom, de Gath. 
${}^{14}L’arche de Dieu resta pendant trois mois chez Obed-Édom, dans sa maison. Et le Seigneur bénit la maison d’Obed-Édom et tout ce qui lui appartenait.
      
         
      \bchapter{}
      \begin{verse}
${}^{1}Hiram, le roi de Tyr, envoya des messagers à David, avec du bois de cèdre, des tailleurs de pierre et des charpentiers, pour lui bâtir une maison. 
${}^{2}Alors David comprit que le Seigneur l’avait établi comme roi sur Israël et qu’il avait hautement exalté sa royauté, à cause d’Israël son peuple.
${}^{3}À Jérusalem, David prit d’autres femmes et il engendra encore des fils et des filles. 
${}^{4}Voici les noms des enfants qui lui sont nés à Jérusalem : Shammoua et Shobab, Nathan et Salomon, 
${}^{5}Yibhar, Èlishoua, Èlpèleth, 
${}^{6}Nogah, Nèfeg, Yafia, 
${}^{7}Èlishama, Beèlyada et Élifèleth.
${}^{8}Les Philistins apprirent que David avait reçu l’onction comme roi sur tout Israël, et ils montèrent tous à sa recherche. Mais David l’apprit et sortit à leur rencontre. 
${}^{9}Les Philistins arrivèrent et envahirent la vallée des Refaïtes. 
${}^{10}Alors David consulta Dieu ; il demanda : « Dois-je monter pour attaquer les Philistins ? Les livreras-tu entre mes mains ? » Le Seigneur lui répondit : « Monte ! Et je les livrerai entre tes mains. » 
${}^{11}Ils montèrent à Baal-Peracim, où David les battit. Et David déclara :
        \\« C’est une brèche que Dieu a ouverte
        \\par ma main chez mes ennemis
        \\comme une brèche ouverte par les eaux. »
      C’est pourquoi on a donné à ce lieu le nom de Baal-Peracim (c’est-à-dire : Maître des brèches). 
${}^{12}Les Philistins avaient abandonné sur place leurs dieux. Et David ordonna de les brûler par le feu.
${}^{13}Une nouvelle fois, les Philistins envahirent la vallée. 
${}^{14}De nouveau, David consulta Dieu. Et Dieu lui répondit : « Ne monte pas à leur poursuite. Fais un large détour : tu les aborderas devant les micocouliers. 
${}^{15}Quand tu entendras un bruit de pas à la cime des micocouliers, alors tu sortiras pour le combat, car Dieu sort devant toi pour frapper le camp des Philistins. » 
${}^{16}David agit comme Dieu le lui avait ordonné, et ils frappèrent le camp des Philistins depuis Gabaon jusqu’à Guèzer. 
${}^{17}La renommée de David se répandit dans tous les pays, et le Seigneur le fit redouter de toutes les nations.
      
         
      \bchapter{}
      \begin{verse}
${}^{1}David se bâtit des maisons dans la Cité de David. Il prépara un emplacement pour l’arche de Dieu et dressa pour elle une tente. 
${}^{2}Il dit alors : « L’arche de Dieu ne peut être portée que par des Lévites, car le Seigneur les a choisis pour porter l’arche du Seigneur et en assurer le service à jamais. » 
${}^{3}David rassembla tout Israël à Jérusalem pour faire monter l’arche du Seigneur jusqu’à l’emplacement préparé pour elle. 
${}^{4}Il réunit les fils d’Aaron et les Lévites. 
${}^{5}Pour les fils de Qehath, il y avait Ouriël, qui était le chef, et ses cent vingt frères ; 
${}^{6}pour les fils de Merari : Asaya, le chef, et ses deux cent vingt frères ; 
${}^{7}pour les fils de Guershom : Joël, le chef, et ses cent trente frères ; 
${}^{8}pour les fils d’Èliçafane : Shemaya, le chef, et ses deux cents frères ; 
${}^{9}pour les fils d’Hébrone : Èliël, le chef, et ses quatre-vingts frères ; 
${}^{10}pour les fils d’Ouzziël : Amminadab, le chef, et ses cent douze frères.
${}^{11}David appela les prêtres Sadoc et Abiatar ainsi que les Lévites : Ouriël, Asaya, Joël, Shemaya, Èliël et Amminadab, 
${}^{12}et il leur dit : « Vous êtes les chefs des familles lévitiques ; purifiez-vous, vous et vos frères, et faites monter l’arche du Seigneur, Dieu d’Israël, à l’endroit que j’ai préparé pour elle. 
${}^{13}C’est parce que vous n’étiez pas là, la première fois, que le Seigneur a fait une brèche parmi nous, car nous ne l’avions pas consulté selon le droit. » 
${}^{14}Alors prêtres et Lévites se purifièrent, pour faire monter l’arche du Seigneur, Dieu d’Israël. 
${}^{15}Puis les Lévites\\transportèrent l’arche de Dieu, au moyen de barres placées sur leurs épaules, comme l’avait ordonné Moïse, selon la parole du Seigneur.
${}^{16}David dit aux chefs des Lévites de mettre en place leurs frères, les chantres, avec leurs instruments, harpes, cithares, cymbales, pour les faire retentir avec force en signe de joie. 
${}^{17}Les Lévites mirent en place Hémane fils de Joël, puis, parmi ses frères, Asaf fils de Bèrèkyahou, et, parmi leurs frères, fils de Merari, Étane fils de Qoushayahou. 
${}^{18}Ils avaient avec eux, en second, leurs frères : Zacharie le fils, Yaaziël, Shemiramoth, Yeïel, Ounni, Èliab, Benaya, Maaséyahou, Mattityahou, Èlifléhou, Miqnéyahou, Obed-Édom, Yeïel, qui étaient portiers. 
${}^{19}Hémane, Asaf et Étane, les chantres, faisaient retentir des cymbales de bronze. 
${}^{20}Zacharie, Aziël, Shemiramoth, Yeïel, Ounni, Èliab, Maaséyahou, Benaya jouaient de la harpe pour voix de soprano. 
${}^{21}Mattityahou, Èlifléhou, Miqnéyahou, Obed-Édom, Yeïel et Azazyahou accompagnaient avec des cithares à l’octave, pour diriger. 
${}^{22}Kenanyahou, chef des Lévites chargés du transport, commandait le transport, car il y était expert.
${}^{23}Bérékya et Elcana faisaient fonction de portiers auprès de l’Arche.
${}^{24}Les prêtres Shebanyahou, Josaphat, Netanel, Amasaï, Zacharie, Benaya et Èlièzer sonnaient de la trompette devant l’arche de Dieu. Obed-Édom et Yehïya étaient portiers auprès de l’Arche.
${}^{25}David, les anciens d’Israël et les officiers de millier partirent de la maison d’Obed-Édom pour faire monter l’arche de l’Alliance du Seigneur au milieu des cris de joie. 
${}^{26}Avec l’aide de Dieu, les Lévites transportèrent l’arche de l’Alliance du Seigneur, et l’on offrit en sacrifice sept taureaux et sept béliers. 
${}^{27}David était revêtu d’un manteau précieux, ainsi que tous les Lévites qui portaient l’Arche et ceux qui chantaient sous la direction de Kenanya, l’officier chargé du transport. David avait aussi sur lui le pagne de lin des prêtres. 
${}^{28}Tout Israël fit monter l’arche de l’Alliance du Seigneur parmi les ovations au son du cor, des trompettes et des cymbales, en faisant retentir des harpes et des cithares.
${}^{29}Or, comme l’arche de l’Alliance du Seigneur atteignait la Cité de David, Mikal, fille de Saül, se pencha par la fenêtre : elle vit le roi David bondir et danser. Dans son cœur, elle le méprisa.
      
         
      \bchapter{}
      \begin{verse}
${}^{1}Ils amenèrent donc l’arche de Dieu et l’installèrent au milieu de la tente que David avait dressée pour elle. Puis on présenta devant Dieu des holocaustes et des sacrifices de paix. 
${}^{2}Quand David eut achevé d’offrir les holocaustes\\et les sacrifices de paix, il bénit le peuple au nom du Seigneur. 
${}^{3}Puis il fit une distribution à tous les gens d’Israël, hommes et femmes : à chacun une couronne de pain, un gâteau de dattes et un gâteau de raisins.
${}^{4}Devant l’arche du Seigneur, David plaça des Lévites qui faisaient le service, pour célébrer le mémorial, l’action de grâce et la louange du Seigneur, Dieu d’Israël. 
${}^{5}Il y avait Asaf comme chef, et Zacharie en second, puis Yeïel, Shemiramoth, Yehiel, Mattitya, Èliab, Benaya, Obed-Édom et Yeïel, avec leurs instruments de musique, harpes et cithares ; Asaf faisait retentir des cymbales. 
${}^{6}Les prêtres Benaya et Yahaziël jouaient constamment de la trompette devant l’arche de l’Alliance de Dieu.
${}^{7}Ce jour-là, pour la première fois, David chargea Asaf et ses frères de célébrer l’action de grâce du Seigneur :
       
${}^{8}Rendez grâce au Seigneur, proclamez son nom,
        \\annoncez parmi les peuples ses hauts faits ;
${}^{9}chantez et jouez pour lui,
        \\redites sans fin ses merveilles ;
${}^{10}glorifiez-vous de son nom très saint :
        \\joie pour les cœurs qui cherchent Dieu !
         
${}^{11}Cherchez le Seigneur et sa puissance,
        \\recherchez sans trêve sa face ;
${}^{12}souvenez-vous des merveilles qu’il a faites,
        \\de ses prodiges, des jugements qu’il prononça,
${}^{13}vous, la race d’Israël son serviteur,
        \\les fils de Jacob, qu’il a choisis.
         
${}^{14}Le Seigneur, c’est lui notre Dieu :
        \\ses jugements font loi pour l’univers.
${}^{15}Souvenez-vous toujours de son alliance,
        \\parole édictée pour mille générations :
         
${}^{16}promesse faite à Abraham,
        \\garantie par serment à Isaac,
${}^{17}érigée en loi avec Jacob,
        \\alliance éternelle pour Israël.
${}^{18}Il a dit : « Je vous donne le pays de Canaan,
        \\ce sera votre part d’héritage. »
         
${}^{19}En ces temps-là, on pouvait les compter :
        \\c’était une poignée d’immigrants ;
${}^{20}ils allaient de nation en nation,
        \\d’un royaume vers un autre royaume.
         
${}^{21}Mais Dieu ne souffrait pas qu’on les opprime ;
        \\à cause d’eux, il châtiait des rois :
${}^{22}« Ne touchez pas à qui m’est consacré,
        \\ne maltraitez pas mes prophètes ! »
       
${}^{23}Chantez au Seigneur, terre entière,
        \\de jour en jour, proclamez son salut,
${}^{24}racontez à tous les peuples sa gloire,
        \\à toutes les nations ses merveilles !
         
${}^{25}Il est grand, le Seigneur, hautement loué,
        \\redoutable au-dessus de tous les dieux :
${}^{26}néant, tous les dieux des nations !
         
        \\Lui, le Seigneur, a fait les cieux :
${}^{27}devant lui, splendeur et majesté,
        \\dans le lieu où il demeure, puissance et allégresse.
         
${}^{28}Rendez au Seigneur, familles des peuples
        \\rendez au Seigneur la gloire et la puissance,
${}^{29}rendez au Seigneur la gloire de son nom.
         
        \\Apportez votre offrande, entrez devant lui,
        \\adorez le Seigneur, éblouissant de sainteté :
${}^{30}tremblez devant lui, terre entière.
        \\Le monde, inébranlable, tient bon.
         
${}^{31}Joie au ciel ! Exulte la terre !
        \\Que l’on dise aux nations : « Le Seigneur est roi ! »
${}^{32}Les masses de la mer mugissent,
        \\la campagne tout entière est en fête.
         
${}^{33}Les arbres des forêts dansent de joie
        \\devant la face du Seigneur,
        \\car il vient pour juger la terre.
       
${}^{34}Rendez grâce au Seigneur : il est bon !
        \\Éternel est son amour !
         
${}^{35}Dites : Sauve-nous, Dieu de notre salut,
        \\rassemble-nous et délivre-nous des païens,
        \\que nous rendions grâce à ton saint nom,
        \\fiers de chanter ta louange !
         
${}^{36}Béni soit le Seigneur, le Dieu d’Israël,
        \\depuis toujours et pour la suite des temps !
         
        \\Et tout le peuple dit : « Amen ! Louange au Seigneur ! »
${}^{37}Là, devant l’arche de l’Alliance du Seigneur, David laissa Asaf et ses frères, pour qu’ils assurent un service permanent devant l’Arche, selon ce qui est prescrit pour chaque jour. 
${}^{38}Il laissa aussi Obed-Édom et ses soixante-huit frères ; Obed-Édom, fils de Yedoutoune, et Hosa étaient portiers.
${}^{39}Quant au prêtre Sadoc et à ses frères les prêtres, il les laissa devant la Demeure du Seigneur, sur le lieu sacré de Gabaon. 
${}^{40}Ils devaient offrir en permanence des holocaustes au Seigneur sur l’autel des holocaustes, matin et soir, et faire tout ce qui est écrit dans la loi du Seigneur, qu’il a prescrite à Israël. 
${}^{41}Avec eux, il y avait Hémane, Yedoutoune et le reste de ceux qui avaient été choisis et désignés par leurs noms pour rendre grâce au Seigneur, « car éternel est son amour ». 
${}^{42}Avec eux – Hémane et Yedoutoune –, il y avait de retentissantes trompettes et cymbales, des instruments pour les chants adressés à Dieu, ainsi que les fils de Yedoutoune, préposés à la porte.
${}^{43}Puis tout le peuple s’en alla, chacun chez soi, et David s’en retourna bénir les gens de sa maison.
      
         
      \bchapter{}
      \begin{verse}
${}^{1}Lorsque David habita dans sa maison, il dit au prophète Nathan : « Voici que j’habite dans une maison de cèdre, et l’arche de l’Alliance du Seigneur est sous un abri de toile ! » 
${}^{2}Nathan répondit à David : « Tout ce que tu as l’intention de faire, fais-le, car Dieu est avec toi. » 
${}^{3}Mais, cette nuit-là, la parole de Dieu fut adressée à Nathan : 
${}^{4}« Va dire à David, mon serviteur : Ainsi parle le Seigneur : Ce n’est pas toi qui me bâtiras une maison pour que j’y habite. 
${}^{5}Depuis le jour où j’ai fait monter Israël et jusqu’à ce jour, je n’ai jamais habité dans une maison ; j’allais d’une tente à une autre et d’une demeure à une autre. 
${}^{6}Pendant tout le temps où j’étais comme un voyageur avec tout Israël, ai-je demandé à un seul des juges que j’avais institués pasteurs de mon peuple : “Pourquoi ne m’avez-vous pas bâti une maison de cèdre ?” 
${}^{7}Tu diras donc à mon serviteur David : Ainsi parle le Seigneur de l’univers : C’est moi qui t’ai pris au pâturage, derrière le troupeau, pour que tu sois le chef de mon peuple Israël. 
${}^{8}J’ai été avec toi partout où tu es allé, j’ai abattu devant toi tous tes ennemis. Je te ferai un nom égal à celui des plus grands de la terre. 
${}^{9}Je fixerai en ce lieu mon peuple Israël, je l’y planterai, il s’y établira et ne tremblera plus, et les méchants ne viendront plus le maltraiter, comme ils l’ont fait autrefois, 
${}^{10}depuis le temps où j’ai institué des juges pour conduire mon peuple Israël. J’ai humilié tous tes ennemis. Je t’annonce que le Seigneur te bâtira une maison. 
${}^{11}Quand tes jours seront accomplis et que tu rejoindras tes pères, je te susciterai dans ta descendance un successeur, qui sera l’un de tes fils, et je rendrai stable sa royauté. 
${}^{12}C’est lui qui me bâtira une maison, et je rendrai stable pour toujours son trône. 
${}^{13}Moi, je serai pour lui un père ; et lui sera pour moi un fils ; je ne lui retirerai pas ma fidélité comme je l’ai retirée à celui qui t’a précédé. 
${}^{14}Je l’établirai pour toujours dans ma maison et dans mon royaume, et son trône sera stable pour toujours. » 
${}^{15}Toutes ces paroles, toute cette vision, Nathan les rapporta fidèlement à David.
      
         
${}^{16}Le roi David vint s’asseoir en présence du Seigneur. Il dit : « Qui suis-je donc, Seigneur Dieu, et qu’est-ce que ma maison, pour que tu m’aies conduit jusqu’ici ? 
${}^{17}Mais cela ne te paraît pas suffisant, à toi Dieu, et tu adresses une parole à la maison de ton serviteur pour un avenir lointain. Tu m’as considéré comme un homme de haut rang, Seigneur Dieu. 
${}^{18}Qu’est-ce que David pourrait ajouter à la gloire que tu lui as donnée ? Toi, tu connais ton serviteur. 
${}^{19}Seigneur, à cause de ton serviteur, et selon ton cœur, tu as accompli toute cette grande action, pour faire connaître toutes tes grandeurs. 
${}^{20}Seigneur, personne n’est comme toi, et il n’y a pas de Dieu en dehors de toi, d’après tout ce que nous avons entendu de nos oreilles. 
${}^{21}Est-il sur la terre une seule nation comme ton peuple Israël ? Ce peuple, Dieu est allé le libérer pour qu’il devienne son peuple. Pour te faire un nom, par de grandes actions et des choses redoutables, tu as chassé des nations devant ton peuple, que tu as libéré d’Égypte. 
${}^{22}Tu t’es donné ton peuple Israël afin qu’il soit ton peuple à jamais, et toi, Seigneur, tu es devenu son Dieu. 
${}^{23}Maintenant donc, Seigneur, la parole que tu as dite à ton serviteur et à sa maison, qu’elle tienne pour toujours, et agis selon ce que tu as dit ! 
${}^{24}Que cette parole soit tenue et que ton nom soit exalté pour toujours ! Que l’on dise : “Le Seigneur de l’univers est le Dieu d’Israël, il est Dieu pour Israël”, et que la maison de David ton serviteur subsiste en ta présence. 
${}^{25}Oui, c’est toi, mon Dieu, qui as fait cette révélation à ton serviteur, à savoir lui bâtir une maison. C’est pourquoi ton serviteur se tient en prière devant toi. 
${}^{26}Seigneur, c’est toi qui es Dieu, et tu as fait cette magnifique promesse à ton serviteur. 
${}^{27}Daigne bénir la maison de ton serviteur, afin qu’elle soit pour toujours en ta présence. Car toi, Seigneur, tu bénis, et la maison est bénie pour toujours ! »
      
         
      \bchapter{}
      \begin{verse}
${}^{1}Après cela, David battit les Philistins et les soumit. Il prit de la main des Philistins Gath et ses dépendances. 
${}^{2}Puis il battit les gens de Moab qui furent asservis à David et payèrent tribut.
${}^{3}David battit Hadadèzer, roi de Soba, à Hamath, alors qu’il allait établir son pouvoir sur l’Euphrate. 
${}^{4}David lui prit mille chars, sept mille cavaliers et vingt mille fantassins. Il fit couper les jarrets de tous les attelages et n’en laissa qu’une centaine. 
${}^{5}Les Araméens de Damas vinrent au secours de Hadadèzer, roi de Soba, mais David battit vingt-deux mille hommes parmi les Araméens. 
${}^{6}David établit des postes de garde chez les Araméens de Damas. Aram fut asservi à David et paya tribut. En tout lieu où allait David, le Seigneur lui donnait la victoire. 
${}^{7}David prit les carquois en or appartenant aux serviteurs de Hadadèzer et les emporta à Jérusalem. 
${}^{8}À Tibhath et à Koun, villes de Hadadèzer, David prit du bronze en grande quantité ; Salomon en fit la Mer de bronze, les colonnes et les objets de bronze.
${}^{9}Lorsque Tôou, roi de Hamath, apprit que David avait battu toute l’armée de Hadadèzer, roi de Soba, 
${}^{10}il envoya son fils Hadoram auprès du roi David, pour le saluer et le féliciter d’avoir fait la guerre à Hadadèzer et de l’avoir battu – en effet, Hadadèzer était constamment en guerre avec Tôou. Hadoram apporta toutes sortes d’objets en or, en argent et en bronze. 
${}^{11}Le roi David consacra ces objets au Seigneur, en plus de l’argent et de l’or qu’il avait enlevés à toutes les nations : Édom, Moab, les fils d’Ammone, les Philistins et Amalec. 
${}^{12}Abishaï, fils de Cerouya, battit les gens d’Édom dans la vallée du Sel, au nombre de dix-huit mille. 
${}^{13}Il établit des postes de garde en Édom, et tous les gens d’Édom furent asservis à David.
      En tout lieu où allait David, le Seigneur lui donnait la victoire.
${}^{14}David régna sur tout Israël, faisant droit et justice à tout son peuple.
${}^{15}Joab, fils de Cerouya, commandait l’armée ; Josaphat, fils d’Ahiloud, était archiviste ; 
${}^{16}Sadoc, fils d’Ahitoub, et Abimèlek, fils d’Abiatar, étaient prêtres ; Shawsha était secrétaire ; 
${}^{17}Benaya, fils de Joad, commandait les Kerétiens et les Pelétiens. Les fils de David étaient les premiers aux côtés du roi.
      
         
      \bchapter{}
      \begin{verse}
${}^{1}Après cela, Nahash, roi des fils d’Ammone, mourut, et son fils régna à sa place. 
${}^{2}David se dit : « Je traiterai Hanoune, fils de Nahash, avec fidélité, parce que son père m’a traité avec fidélité. » Et David envoya des messagers lui présenter des condoléances au sujet de son père. Les serviteurs de David arrivèrent au pays des fils d’Ammone, auprès de Hanoun, pour lui faire part de ces condoléances. 
${}^{3}Mais les princes des fils d’Ammone dirent à Hanoun : « Penses-tu que c’est bien pour honorer ton père que David t’envoie des porteurs de condoléances ? N’est-ce pas plutôt pour explorer, détruire, espionner le pays que ses serviteurs sont venus auprès de toi ? » 
${}^{4}Alors Hanoun se saisit des serviteurs de David, les rasa et coupa leurs vêtements à mi-hauteur jusqu’aux hanches, puis il les renvoya. 
${}^{5}On vint informer David de ce qui était arrivé à ces hommes : il envoya quelqu’un à leur rencontre, car ils étaient pleins de confusion. Le roi leur fit dire : « Restez à Jéricho jusqu’à ce que votre barbe ait repoussé. Ensuite, vous reviendrez. »
${}^{6}Les fils d’Ammone virent bien qu’ils s’étaient rendus odieux à David. Hanoun et les fils d’Ammone envoyèrent mille talents d’argent pour prendre à leur solde des Araméens de Mésopotamie, des Araméens de Maaka et de Soba, des chars et des cavaliers. 
${}^{7}Ils prirent à leur solde le roi de Maaka, ses troupes, et trente-deux mille chars ; ils vinrent camper devant Médeba, tandis que les fils d’Ammone, après avoir quitté leurs villes et s’être rassemblés, arrivaient pour la bataille. 
${}^{8}À cette nouvelle, David envoya Joab avec toute une armée, les guerriers d’élite. 
${}^{9}Les fils d’Ammone firent une sortie et se rangèrent en ordre de bataille à l’entrée de la ville, tandis que les rois qui étaient arrivés se tenaient à l’écart, en rase campagne. 
${}^{10}Lorsque Joab vit qu’il avait un front de combat à la fois devant et derrière lui, il choisit, parmi toute l’élite d’Israël, des hommes qu’il rangea face aux Araméens. 
${}^{11}Il confia le reste de la troupe à son frère Abishaï, et ils se rangèrent face aux fils d’Ammone. 
${}^{12}Joab dit alors à son frère : « Si les Araméens sont plus forts que moi, tu viendras à mon secours. Et si les fils d’Ammone sont plus forts que toi, je viendrai à ton secours. 
${}^{13}Sois fort, montrons-nous forts pour notre peuple, pour les villes de notre Dieu ! Que le Seigneur fasse ce qui est bon à ses yeux ! »
${}^{14}Joab, avec la troupe qui l’accompagnait, s’avança pour combattre les Araméens, qui s’enfuirent devant lui. 
${}^{15}Quand les fils d’Ammone virent que les Araméens s’étaient enfuis, à leur tour ils prirent la fuite devant Abishaï, le frère de Joab, et rentrèrent dans la ville. Alors Joab rentra à Jérusalem.
${}^{16}Les Araméens, se voyant battus par Israël, envoyèrent des messagers pour faire venir les Araméens d’au-delà de l’Euphrate ; Shofak, chef de l’armée de Hadadèzer, était à leur tête. 
${}^{17}David en fut informé. Il rassembla tout Israël, passa le Jourdain, les rejoignit et se rangea devant eux. David se rangea donc en ordre de combat en face des Araméens, qui engagèrent la bataille contre lui. 
${}^{18}Mais les Araméens s’enfuirent devant Israël. David massacra parmi les Araméens sept mille attelages et quarante mille fantassins. Il fit aussi périr Shofak, le chef de l’armée. 
${}^{19}Quand les serviteurs de Hadadèzer se virent battus par Israël, ils firent la paix avec David et passèrent à son service. Et désormais, les Araméens ne voulurent plus porter secours aux fils d’Ammone.
      
         
      \bchapter{}
      \begin{verse}
${}^{1}À l’époque du retour du printemps, époque où les rois se mettent en campagne, Joab emmena une forte armée et dévasta le pays des fils d’Ammone. Puis il vint mettre le siège devant Rabba, tandis que David restait à Jérusalem. Joab fit tomber Rabba et la détruisit. 
${}^{2}David enleva la couronne qui était sur la tête de leur roi. Il remarqua qu’elle pesait un talent d’or et qu’elle était ornée d’une pierre précieuse. On la mit sur la tête de David. Puis celui-ci emporta de la ville une très grande quantité de butin. 
${}^{3}Quant à sa population, il l’emmena, la condamna à la scie, aux pics de fer et aux haches. Il traita ainsi toutes les villes des fils d’Ammone. Puis David et toute l’armée revinrent à Jérusalem.
      
         
${}^{4}Après cela, une bataille s’engagea contre les Philistins à Guèzer. C’est alors que Sibbekaï de Housha abattit Sippaï, un descendant des géants Refaïtes. Et les Philistins furent soumis. 
${}^{5}Il y eut encore une bataille contre les Philistins. Elhanane, fils de Yaïr, abattit un nommé Lahmi, frère de Goliath de Gath, celui dont le bois de la lance était comme le rouleau d’un métier à tisser. 
${}^{6}Il y eut encore une bataille à Gath. Là se trouvait un homme de haute taille qui avait six doigts à chaque main, six à chaque pied, vingt-quatre en tout. Lui aussi était un descendant du géant Rafa. 
${}^{7}Comme il défiait Israël, Jonathan, fils de Shiméa frère de David, l’abattit. 
${}^{8}Ces hommes étaient de la descendance de Rafa, à Gath. Ils tombèrent sous les coups de David et de ses serviteurs.
      
         
      \bchapter{}
      \begin{verse}
${}^{1}Satan se dressa contre Israël et il incita David à dénombrer Israël. 
${}^{2}David dit à Joab et aux chefs du peuple : « Allez compter les gens d’Israël, de Bershéba à Dane, puis revenez-m’en faire connaître le chiffre. » 
${}^{3}Joab répondit : « Que le Seigneur fasse grandir son peuple cent fois plus ! Mon seigneur le roi, tous ne sont-ils pas les serviteurs de mon seigneur ? Pourquoi mon seigneur fait-il cette enquête ? Pourquoi Israël deviendrait-il coupable ? » 
${}^{4}Mais l’ordre du roi s’imposa à Joab. Joab partit donc, il parcourut tout Israël, puis revint à Jérusalem. 
${}^{5}Joab remit à David les chiffres du recensement du peuple : Israël, en tout, comptait onze cent mille hommes sachant tirer l’épée, et Juda quatre cent soixante-dix mille hommes sachant tirer l’épée. 
${}^{6}Mais parmi eux il n’avait recensé ni Lévi ni Benjamin, parce que l’ordre du roi avait paru abominable à Joab.
${}^{7}Cette affaire déplut à Dieu, et il frappa Israël. 
${}^{8}David dit alors à Dieu : « C’est un grand péché que j’ai commis en faisant cela ! Maintenant, daigne passer sur la faute de ton serviteur, car je me suis vraiment conduit comme un insensé ! » 
${}^{9}Alors le Seigneur adressa la parole à Gad, qui était le voyant attaché à David : 
${}^{10}« Va dire à David : Ainsi parle le Seigneur : Je te propose trois choses ; choisis l’une d’entre elles, et je te l’infligerai. » 
${}^{11}Gad se rendit alors chez David et lui dit : « Ainsi parle le Seigneur. Il faut que tu acceptes 
${}^{12}ou bien trois années de famine ; ou bien trois mois de déroute devant tes adversaires, au cours desquels tu seras atteint par l’épée de tes ennemis ; ou bien trois jours pendant lesquels l’épée du Seigneur et la peste seront dans le pays, l’ange du Seigneur ravageant tout le territoire d’Israël ! Maintenant vois ce que je dois répondre à celui qui m’a envoyé. » 
${}^{13}David répondit au prophète Gad : « Je suis dans une grande angoisse… Eh bien ! que je tombe entre les mains du Seigneur, car sa compassion est immense, mais que je ne tombe pas entre les mains des hommes ! »
${}^{14}Le Seigneur envoya donc la peste en Israël, et soixante-dix mille hommes d’Israël tombèrent. 
${}^{15}Puis Dieu envoya l’ange vers Jérusalem pour l’exterminer. Mais au moment d’exterminer, le Seigneur regarda, et il renonça à ce mal. Il dit à l’ange exterminateur : « Assez ! Maintenant, retire ta main ! » L’ange du Seigneur se tenait alors près de l’aire à grain d’Ornane le Jébuséen. 
${}^{16}Levant les yeux, David vit l’ange du Seigneur qui se tenait entre terre et ciel, l’épée dégainée à la main, tendue contre Jérusalem. Alors David et les anciens, revêtus de toile à sac, tombèrent face contre terre. 
${}^{17}David dit à Dieu : « N’est-ce pas moi qui ai ordonné de dénombrer le peuple ? N’est-ce pas moi qui ai péché et fait le mal ? Mais ceux-là, le troupeau, qu’ont-ils fait ? Seigneur mon Dieu, que ta main s’appesantisse donc sur moi et sur la maison de mon père, mais que ton peuple échappe au fléau ! »
${}^{18}L’ange du Seigneur dit alors à Gad : « Que David monte et qu’il élève un autel au Seigneur sur l’aire d’Ornane le Jébuséen. » 
${}^{19}David monta donc selon la parole que Gad lui avait dite au nom du Seigneur. 
${}^{20}Ornane était en train de battre le blé. S’étant retourné, il avait vu l’ange et s’était caché avec ses quatre fils. 
${}^{21}Lorsque David se rendit auprès de lui, Ornane regarda et aperçut David ; puis il sortit de l’aire et se prosterna devant David, face contre terre. 
${}^{22}David dit alors à Ornane : « Cède-moi l’emplacement de cette aire afin que j’y construise un autel pour le Seigneur. Cède-le-moi contre sa valeur en argent, et que le fléau s’écarte du peuple ! » 
${}^{23}Ornane dit à David : « Prends-le, et que mon seigneur le roi fasse ce qui lui semble bon ! Vois : je donne les bœufs pour les holocaustes, les traîneaux à battre le grain pour le bois du feu ainsi que le blé pour l’offrande. Je donne tout. » 
${}^{24}Le roi David répondit à Ornane : « Non ! je veux l’acheter pour sa valeur en argent, car je n’apporterai pas au Seigneur ce qui t’appartient et je n’offrirai pas un holocauste qui ne me coûterait rien. » 
${}^{25}Pour cet emplacement, David donna à Ornane une somme de six cents pièces d’or. 
${}^{26}David construisit là un autel pour le Seigneur, puis il offrit des holocaustes et des sacrifices de paix. Il invoqua le Seigneur, et le Seigneur lui répondit par le feu venu du ciel sur l’autel de l’holocauste. 
${}^{27}Alors le Seigneur ordonna à l’ange de remettre l’épée au fourreau. 
${}^{28}À ce moment-là, David, voyant que le Seigneur lui avait répondu sur l’aire d’Ornane le Jébuséen, y offrit des sacrifices. 
${}^{29}La Demeure du Seigneur que Moïse avait faite dans le désert et l’autel de l’holocauste se trouvaient à ce moment-là sur le lieu sacré de Gabaon. 
${}^{30}Mais David n’avait pu s’y présenter pour consulter Dieu, car il avait été effrayé par l’épée de l’ange du Seigneur.
      
         
      \bchapter{}
      \begin{verse}
${}^{1}Puis David déclara : « Voici la maison du Seigneur Dieu, voici l’autel de l’holocauste pour Israël. »
      
         
${}^{2}David ordonna de rassembler les immigrés qui étaient dans le pays d’Israël, et il chargea des ouvriers pour les carrières d’extraire des pierres de taille pour la construction de la maison de Dieu. 
${}^{3}De plus, David entreposa une grande quantité de fer, pour les clous des battants de porte et pour les assemblages, ainsi que du bronze en quantité impossible à peser, 
${}^{4}et des troncs de cèdre en nombre incalculable, car les habitants de Sidon et de Tyr avaient apporté à David des troncs de cèdre en abondance. 
${}^{5}David disait : « Mon fils Salomon est jeune et faible, et la Maison qu’il faut bâtir pour le Seigneur doit être absolument grandiose, pour que son nom et sa splendeur soient connus dans tous les pays. Je veux donc en faire pour lui les préparatifs. » Ainsi, avant sa mort, David fit de grands préparatifs.
${}^{6}Puis il appela son fils Salomon et lui donna l’ordre de bâtir une maison pour le Seigneur, Dieu d’Israël.
${}^{7}David dit à Salomon : « Mon fils, j’avais à cœur de bâtir une maison pour le nom du Seigneur mon Dieu. 
${}^{8}Mais la parole du Seigneur me fut adressée : “Tu as répandu beaucoup de sang et livré de grandes batailles ; ce n’est donc pas toi qui bâtiras une maison pour mon nom car tu as répandu devant moi beaucoup de sang sur la terre. 
${}^{9}Voici qu’un fils te naîtra, lui sera un homme tranquille, et je lui procurerai la tranquillité du côté de tous les ennemis qui l’entourent. Son nom sera Salomon (c’est-à-dire : Pacifique), et, pendant sa vie, je donnerai la paix et le calme à Israël. 
${}^{10}C’est lui qui bâtira une maison pour mon nom. Il sera pour moi un fils et je serai pour lui un père, et je rendrai stable pour toujours le trône de sa royauté sur Israël.”
${}^{11}Maintenant, mon fils, que le Seigneur soit avec toi, pour que tu réussisses à bâtir la maison du Seigneur ton Dieu, comme il l’a promis à ton sujet. 
${}^{12}Qu’il daigne t’accorder prudence et intelligence, quand il te donnera autorité sur Israël, pour que tu observes la Loi du Seigneur ton Dieu ! 
${}^{13}Alors, tu réussiras si tu observes et mets en pratique les décrets et les ordonnances que le Seigneur a prescrits à Moïse pour Israël. Sois fort et courageux ! Ne crains pas, ne t’effraie pas !
${}^{14}Voici que, dans ma peine, j’ai pu mettre de côté pour la maison du Seigneur cent mille talents d’or, un million de talents d’argent, et une telle quantité de bronze et de fer qu’on ne peut les peser. J’ai aussi entreposé du bois et des pierres, et tu en ajouteras encore. 
${}^{15}Tu as avec toi de nombreux artisans, ouvriers pour les carrières, ouvriers de la pierre et du bois, tous habiles pour tous les ouvrages. 
${}^{16}Quant à l’or, à l’argent, au bronze et au fer, on ne saurait les compter. Lève-toi ! Au travail ! et que le Seigneur soit avec toi. »
       
${}^{17}Alors David donna ordre à tous les princes d’Israël de prêter main-forte à son fils Salomon : 
${}^{18}« Le Seigneur, votre Dieu, n’est-il pas avec vous ? Il vous a donné partout la tranquillité, puisqu’il a livré entre mes mains les habitants du pays, et que le pays a été soumis au Seigneur et à son peuple. 
${}^{19}Maintenant, appliquez votre cœur et votre âme à chercher le Seigneur, votre Dieu. Levez-vous, bâtissez le sanctuaire du Seigneur Dieu, et faites entrer l’arche de l’Alliance du Seigneur et les objets sacrés de Dieu dans la Maison bâtie au nom du Seigneur. »
      
         
      \bchapter{}
      \begin{verse}
${}^{1}Devenu vieux et rassasié de jours, David établit son fils Salomon comme roi sur Israël. 
${}^{2}Il réunit tous les princes d’Israël, ainsi que les prêtres et les Lévites.
${}^{3}On compta les Lévites âgés de trente ans et plus. On les compta un par un, et leur nombre fut de trente-huit mille hommes. 
${}^{4}« Vingt-quatre mille d’entre eux, dit David, dirigeront les travaux de la maison du Seigneur, six mille seront scribes et juges, 
${}^{5}quatre mille seront portiers, et quatre mille loueront le Seigneur avec les instruments que j’ai faits à cette intention. »
${}^{6}David les répartit par classes selon les fils de Lévi : Guershone, Qehath et Merari.
${}^{7}Pour les Guershonites : Ladane et Shiméï. 
${}^{8}Fils de Ladane : Yehiël, l’aîné, Zétam et Joël ; ils étaient trois. 
${}^{9}Fils de Shiméï : Shelomith, Haziël et Harane ; ils étaient trois ; c’étaient les chefs des familles de Ladane. 
${}^{10}Fils de Shiméï : Yahath, Ziza, Yéoush, Beria ; c’étaient les fils de Shiméï, ils étaient quatre. 
${}^{11}Yahath était l’aîné, Ziza, le deuxième ; puis Yéoush et Beria qui n’eurent pas beaucoup de fils et ne formèrent qu’une seule famille pour une charge unique. 
${}^{12}Fils de Qehath : Amram, Yicehar, Hébrone et Ouzziël ; ils étaient quatre.
${}^{13}Fils d’Amram : Aaron et Moïse. Aaron fut mis à part pour être consacré aux choses très saintes, lui et ses fils à jamais, pour brûler l’encens devant le Seigneur, pour le servir et bénir son nom à jamais. 
${}^{14}Quant à Moïse, l’homme de Dieu, ses fils furent comptés dans la tribu de Lévi. 
${}^{15}Fils de Moïse : Guershom et Èlièzer. 
${}^{16}Fils de Guershom : Shebouël, l’aîné. 
${}^{17}Fils d’Èlièzer : Rehabya, l’aîné ; Èlièzer n’eut pas d’autres fils, mais les fils de Rehabya furent extrêmement nombreux.
${}^{18}Fils de Yicehar : Shelomith, l’aîné.
${}^{19}Fils de Hébrone : Yeriyahou, l’aîné, Amarya, le deuxième, Yahaziël, le troisième, et Yeqaméam, le quatrième.
${}^{20}Fils d’Ouzziël : Mika, l’aîné, et Yishiya, le second.
${}^{21}Fils de Merari : Mahli et Moushi.
      Fils de Mahli : Éléazar et Qish. 
${}^{22}Éléazar mourut sans avoir de fils, mais il eut des filles qu’enlevèrent leurs cousins, les fils de Qish. 
${}^{23}Fils de Moushi : Mahli, Éder et Yerémoth ; ils étaient trois.
${}^{24}Tels étaient les fils de Lévi selon leurs familles, les chefs de famille selon leurs charges, d’après le dénombrement nominatif, un par un ; âgés de vingt ans et plus, ils accomplissaient leur travail au service de la maison du Seigneur.
${}^{25}Car David avait dit : « Le Seigneur, Dieu d’Israël, a donné la tranquillité à son peuple et il demeure à Jérusalem pour toujours. 
${}^{26}Les Lévites n’auront plus à transporter la Tente et tous les objets destinés à son service. » 
${}^{27}C’est d’après les dernières paroles de David que les Lévites qui furent comptés étaient âgés de vingt ans et plus. 
${}^{28}Ils doivent se tenir aux côtés des fils d’Aaron pour le service de la maison du Seigneur en ce qui concerne les cours, les salles, la purification de toute chose sainte, en un mot, l’accomplissement du service de la maison de Dieu. 
${}^{29}Ils ont aussi à prévoir le pain à disposer en rangées, la fleur de farine destinée à l’offrande, les galettes sans levain, les gâteaux cuits à la plaque ou mélangés, et tous les instruments de capacité et de mesure. 
${}^{30}Ils doivent se tenir prêts chaque matin pour célébrer et louer le Seigneur, et de même le soir, 
${}^{31}ainsi que pour tous les holocaustes au Seigneur lors des sabbats, des nouvelles lunes et des solennités, selon le nombre qui leur a été prescrit pour toujours, devant le Seigneur. 
${}^{32}Ils gardent les observances de la tente de la Rencontre, les observances du sanctuaire et les observances de leurs frères, les fils d’Aaron, pour le service de la maison du Seigneur.
      
         
      \bchapter{}
      \begin{verse}
${}^{1}Voici la répartition des fils d’Aaron par classes. Fils d’Aaron : Nadab, Abihou, Éléazar et Itamar. 
${}^{2}Nadab et Abihou moururent avant leur père sans laisser de fils, et c’est Éléazar et Itamar qui devinrent prêtres. 
${}^{3}David, avec Sadoc, l’un des descendants d’Éléazar, et avec Ahimélek, l’un des descendants d’Itamar, les répartit en classes, selon leur charge dans les services. 
${}^{4}Parmi les fils d’Éléazar, les chefs de groupes d’hommes se trouvèrent plus nombreux que parmi les fils d’Itamar ; on forma donc seize classes avec les chefs de famille des fils d’Éléazar, et huit avec les chefs de famille des fils d’Itamar. 
${}^{5}Les uns comme les autres, on les répartit en tirant au sort. Il y eut des officiers pour le sanctuaire, des officiers pour le service de Dieu, parmi les fils d’Éléazar comme parmi les fils d’Itamar. 
${}^{6}L’un des Lévites, le scribe Shemaya, fils de Netanel, les inscrivit en présence du roi, des officiers, du prêtre Sadoc, d’Ahimélek fils d’Abiatar, des chefs de familles sacerdotales et lévitiques : une famille était tirée au sort pour Éléazar, puis une autre, tandis qu’une seule était tirée pour Itamar. 
${}^{7}Le premier tirage au sort désigna Yehoyarib ; le deuxième désigna Yedaya ; 
${}^{8}le troisième, Harim ; le quatrième, Séorim ; 
${}^{9}le cinquième, Malkiya ; le sixième, Miyamine ; 
${}^{10}le septième, Haqqos ; le huitième, Abiya ; 
${}^{11}le neuvième, Yéshoua ; le dixième, Shekanyahou ; 
${}^{12}le onzième, Élyashib ; le douzième, Yaqim ; 
${}^{13}le treizième, Houppa ; le quatorzième, Yèshèbéab ; 
${}^{14}le quinzième, Bilga ; le seizième, Immer ; 
${}^{15}le dix-septième, Hézir ; le dix-huitième, Happicès ; 
${}^{16}le dix-neuvième, Petahya ; le vingtième, Ézékiel ; 
${}^{17}le vingt-et-unième, Yakine ; le vingt-deuxième, Gamoul ; 
${}^{18}le vingt-troisième, Delayahou ; le vingt-quatrième, Maazyahou.
${}^{19}Tels étaient ceux qui avaient pour charge, dans leur service, d’entrer dans la maison du Seigneur, selon leur règlement, transmis par Aaron, leur ancêtre, comme le lui avait ordonné le Seigneur, Dieu d’Israël.
${}^{20}Quant aux fils de Lévi qui restaient : pour les fils d’Amram, il y avait Shoubaël ; pour les fils de Shoubaël, il y avait Yèhdeyahou ; 
${}^{21}pour Rehabyahou et pour les fils de Rehabyahou, il y avait l’aîné Yishiya ; 
${}^{22}pour les Yiceharites, il y avait Shelomoth ; pour les fils de Shelomoth, il y avait Yahath. 
${}^{23}Les fils de Hébrone : Yeriya, le premier, Amaryahou, le deuxième, Yahaziel, le troisième, Yeqaméam, le quatrième. 
${}^{24}Pour les fils d’Ouzziël, il y avait Mika ; pour les fils de Mika, il y avait Shamir ; 
${}^{25}le frère de Mika était Yishiya ; pour les fils de Yishiya, il y avait Zacharie. 
${}^{26}Pour les fils de Merari, il y avait Mahli et Moushi. Pour les fils de Yaaziyahou, son fils, 
${}^{27}c’est-à-dire les descendants de Merari, par son fils Yaaziyahou, il y avait Shoham, Zakkour et Ibri. 
${}^{28}Pour Mahli, il y avait Éléazar qui n’eut pas de fils ; 
${}^{29}pour Qish et les fils de Qish, il y avait Yerahméel. 
${}^{30}Les fils de Moushi étaient Mahli, Éder et Yerimoth. Tels étaient les fils de Lévi, répartis par familles. 
${}^{31}Eux aussi, comme leurs frères, les fils d’Aaron, tirèrent au sort en présence du roi David, de Sadoc, d’Ahimélek, et des chefs de familles sacerdotales et lévitiques. Dans chaque famille, le chef était sur le même pied que son plus jeune frère.
      
         
      \bchapter{}
      \begin{verse}
${}^{1}Pour le service, David et les officiers de l’armée mirent à part les fils d’Asaf, de Hémane et de Yedoutoune, qui prophétisaient au son des cithares, des harpes et des cymbales. Voici le dénombrement des hommes qui accomplissaient ce service.
${}^{2}Pour les fils d’Asaf, il y avait Zakkour, Joseph, Netanya et Asaréla ; les fils d’Asaf étaient sous la direction d’Asaf lui-même, lequel prophétisait sous la direction du roi.
${}^{3}Pour Yedoutoune, il y avait les fils de Yedoutoune : Guedalyahou, Ceri, Isaïe, Shiméï, Hashabyahou et Mattityahou ; ils étaient six sous la direction de leur père Yedoutoune, lequel prophétisait au son de la cithare, pour célébrer et louer le Seigneur.
${}^{4}Pour Hémane, il y avait les fils de Hémane : Bouqqiyahou, Mattanyahou, Ouzziël, Shebouël, Yerimoth, Hananya, Hanani, Éliata, Guiddalti, Romamti-Ézer, Yoshbeqasha, Malloti, Hotir, Mahazioth. 
${}^{5}Tous ceux-là étaient fils de Hémane, le voyant du roi, qui interprétait les paroles de Dieu pour exalter sa puissance ; Dieu donna à Hémane quatorze fils et trois filles. 
${}^{6}Ils étaient tous sous la direction de leur père pour chanter dans la maison du Seigneur, au son des cymbales, des harpes et des cithares, pour le service de la maison de Dieu, sous la direction du roi, d’Asaf, de Yedoutoune et de Hémane. 
${}^{7}Ceux-là, y compris leurs frères exercés au chant pour le Seigneur, en avaient tous acquis la maîtrise ; leur nombre était de 288.
${}^{8}On tira au sort l’organisation du service, pour le petit comme pour le grand, pour le maître comme pour le disciple.
      <p class="retrait1char">
${}^{9}Le premier tirage au sort désigna, pour Asaf, Joseph.
      <p class="retrait1">Le deuxième désigna Guedalyahou ; avec ses fils et ses frères, ils étaient douze.
${}^{10}Le troisième désigna Zakkour ; avec ses fils et ses frères, ils étaient douze.
${}^{11}Le quatrième désigna Yiceri ; avec ses fils et ses frères, ils étaient douze.
${}^{12}Le cinquième désigna Netanyahou ; avec ses fils et ses frères, ils étaient douze.
${}^{13}Le sixième désigna Bouqqiyahou ; avec ses fils et ses frères, ils étaient douze.
${}^{14}Le septième désigna Yesaréla ; avec ses fils et ses frères, ils étaient douze.
${}^{15}Le huitième désigna Isaïe ; avec ses fils et ses frères, ils étaient douze.
${}^{16}Le neuvième désigna Mattanyahou ; avec ses fils et ses frères, ils étaient douze.
${}^{17}Le dixième désigna Shiméï ; avec ses fils et ses frères, ils étaient douze.
${}^{18}Le onzième désigna Azarel ; avec ses fils et ses frères, ils étaient douze.
${}^{19}Le douzième désigna Hashabya ; avec ses fils et ses frères, ils étaient douze.
${}^{20}Le treizième désigna Shoubaël ; avec ses fils et ses frères, ils étaient douze.
${}^{21}Le quatorzième désigna Mattityahou ; avec ses fils et ses frères, ils étaient douze.
${}^{22}Le quinzième désigna Yerémoth ; avec ses fils et ses frères, ils étaient douze.
${}^{23}Le seizième désigna Hananyahou ; avec ses fils et ses frères, ils étaient douze.
${}^{24}Le dix-septième désigna Yoshbeqasha ; avec ses fils et ses frères, ils étaient douze.
${}^{25}Le dix-huitième désigna Hanani ; avec ses fils et ses frères, ils étaient douze.
${}^{26}Le dix-neuvième désigna Malloti ; avec ses fils et ses frères, ils étaient douze.
${}^{27}Le vingtième désigna Élyata ; avec ses fils et ses frères, ils étaient douze.
${}^{28}Le vingt-et-unième désigna Hotir ; avec ses fils et ses frères, ils étaient douze.
${}^{29}Le vingt-deuxième désigna Guiddalti ; avec ses fils et ses frères, ils étaient douze.
${}^{30}Le vingt-troisième désigna Mahazioth ; avec ses fils et ses frères, ils étaient douze.
${}^{31}Le vingt-quatrième désigna Romamti-Ézer ; avec ses fils et ses frères, ils étaient douze.
      
         
      \bchapter{}
      \begin{verse}
${}^{1}Voici la répartition des classes des portiers.
      \begin{verse}Pour les Coréites : Meshélémyahou, fils de Coré, l’un des fils d’Èbyasaf. 
${}^{2}Meshélémyahou eut des fils : d’abord Zacharie, l’aîné, puis Yediaël, le deuxième, Zebadyahou, le troisième, Yatniel, le quatrième, 
${}^{3}Élam, le cinquième, Yehohanane, le sixième, et Élyehoénaï, le septième.
${}^{4}Obed-Édom eut des fils : Shemaya, l’aîné, Yehozabad, le deuxième, Yoah, le troisième, Sakar, le quatrième, Netanel, le cinquième, 
${}^{5}Ammiel, le sixième, Issakar, le septième, et Péoulletaï, le huitième, car Dieu avait béni Obed-Édom. 
${}^{6}À son fils Shemaya naquirent des fils qui eurent autorité sur leurs familles : c’étaient des hommes de grande valeur ; 
${}^{7}les fils de Shemaya étaient Otni, Raphaël, Obed, Élzabad, et ses frères – gens de valeur – Élihou et Semakyahou. 
${}^{8}Tous ceux-là étaient fils d’Obed-Édom. Eux, leurs fils et leurs frères étaient des hommes de valeur par l’énergie qu’ils montraient dans leur service ; ils étaient soixante-deux, issus d’Obed-Édom.
${}^{9}Meshélémyahou eut des fils et des frères – gens de valeur – au nombre de dix-huit.
${}^{10}Hosa, l’un des fils de Merari, eut des fils : Shimri était l’aîné, car, sans qu’il fût le premier-né, son père en avait fait l’aîné ; 
${}^{11}Hilqiyahou était le deuxième, Tebalyahou, le troisième, Zacharie, le quatrième. Treize en tout, fils et frères de Hosa.
${}^{12}Ceux-ci constituaient les classes de portiers. Les chefs de ces hommes avaient pour tâche, comme leurs frères, d’officier dans la maison du Seigneur. 
${}^{13}On tira au sort selon les familles, pour le petit comme pour le grand, afin d’organiser le service de chaque porte.
${}^{14}Pour la porte de l’est, le sort tomba sur Shélèmyahou. Pour son fils Zacharie, qui était un conseiller prudent, on tira au sort et le sort le désigna pour le nord. 
${}^{15}Obed-Édom eut la porte du sud. Et ses fils eurent les magasins. 
${}^{16}Shouppim et Hosa eurent l’ouest avec la porte Shallèketh sur la route montante. On avait proportionné les postes de garde de la manière suivante : 
${}^{17}six Lévites à l’est, mais quatre par jour au nord, quatre par jour au sud, et deux groupes de deux aux magasins ; 
${}^{18}pour les dépendances, à l’ouest, ils étaient quatre pour la route et deux pour les dépendances.
${}^{19}Telles étaient les classes des portiers chez les Coréites et les Merarites.
${}^{20}Les autres Lévites, leurs frères, étaient préposés aux trésors de la maison de Dieu et aux trésors des choses saintes.
${}^{21}Les fils de Ladane, fils de Guershonites par Ladane, avaient les Yehiélites pour chefs de famille. 
${}^{22}Les fils de Yehiel, Zétam et Joël son frère, étaient préposés aux trésors de la maison du Seigneur.
${}^{23}Pour les Amramites, les Yiceharites, les Hébronites et les Ouzziélites, 
${}^{24}il y avait Shebouël, descendant de Guershom, fils de Moïse, comme surintendant des trésors.
${}^{25}Il avait pour frères, par Èlièzer : le fils de ce dernier, Rehabyahou, qui eut pour fils Isaïe, qui eut pour fils Yoram, qui eut pour fils Zikri, lequel eut pour fils Shelomith. 
${}^{26}Ce Shelomith et ses frères furent préposés à tous les trésors des choses saintes, consacrées par le roi David, par les chefs de famille, les officiers de millier ou de centaine, et par les officiers de l’armée. 
${}^{27}Ces choses, prises sur le butin de guerre, avaient été consacrées pour entretenir la maison du Seigneur. 
${}^{28}Shelomith et ses frères étaient préposés à tout ce qu’avaient consacré Samuel le Voyant, Saül, fils de Qish, Abner, fils de Ner, et Joab, fils de Cerouya, en un mot tous ceux qui avaient consacré.
${}^{29}Pour les Yiceharites, il y avait pour les affaires extérieures en Israël Kenanyahou et ses fils comme secrétaires et juges.
${}^{30}Pour les Hébronites, il y avait Hashabyahou et ses frères, hommes de grande valeur, au nombre de 1 700, préposés à la surveillance d’Israël en deçà du Jourdain, à l’ouest, pour toutes les affaires du Seigneur et le service du roi.
${}^{31}Les Hébronites avaient pour chef Yeriya, d’après la généalogie de leurs pères. Ceux-ci avaient fait l’objet de recherches en la quarantième année du règne de David, et l’on avait trouvé parmi eux des hommes de grande valeur à Yazèr, en Galaad. 
${}^{32}Quant aux frères de Yeriya, gens de valeur, ils étaient 2 700 chefs de famille. Le roi David les nomma inspecteurs des Roubénites, des Gadites et de la demi-tribu de Manassé, en toute affaire divine ou royale.
      
         
      \bchapter{}
      \begin{verse}
${}^{1}Voici les fils d’Israël, d’après leur nombre. Les chefs de famille, les officiers de millier et de centaine, et leurs scribes étaient au service du roi, en toute affaire concernant les classes, qui se succédaient mois par mois, tous les mois de l’année ; chaque classe comptait 24 000 hommes.
${}^{2}Préposé à la première classe, pour le premier mois : Yashobéam, fils de Zabdiël ; sa classe comptait 24 000 hommes ; 
${}^{3}descendant de Pérès, il était chef de tous les officiers de l’armée, pour le premier mois.
${}^{4}Préposé à la classe du deuxième mois : Dodaï l’Ahohite ; sa classe avait pour commandant Miqloth ; elle comptait 24 000 hommes.
${}^{5}L’officier de la troisième armée, pour le troisième mois, était Benaya, fils de Joad, le chef des prêtres ; sa classe comptait 24 000 hommes. 
${}^{6}Benaya fut le plus valeureux des Trente guerriers, qu’il surpassait. Son fils Ammizabad était préposé à sa classe.
${}^{7}Le quatrième, pour le quatrième mois, était Asahel, frère de Joab ; son fils Zebadya lui succéda ; sa classe comptait 24 000 hommes.
${}^{8}Le cinquième, pour le cinquième mois, était l’officier Shamhouth, le Yizrahite ; sa classe comptait 24 000 hommes.
${}^{9}Le sixième, pour le sixième mois, était Ira, fils d’Iqqesh, de Teqoa ; sa classe comptait 24 000 hommes.
${}^{10}Le septième, pour le septième mois, était Hélès, le Pelonite, descendant d’Éphraïm ; sa classe comptait 24 000 hommes.
${}^{11}Le huitième, pour le huitième mois, était Sibbekaï, de Housha, un Zarhite ; sa classe comptait 24 000 hommes.
${}^{12}Le neuvième, pour le neuvième mois, était Abièzer, d’Anatoth, un Benjaminite ; sa classe comptait 24 000 hommes.
${}^{13}Le dixième, pour le dixième mois, était Mahraï, de Netofa, un Zarhite ; sa classe comptait 24 000 hommes.
${}^{14}Le onzième, pour le onzième mois, était Benaya, de Piréatone, descendant d’Éphraïm ; sa classe comptait 24 000 hommes.
${}^{15}Le douzième, pour le douzième mois, était Heldaï, de Netofa, du clan d’Otniël ; sa classe comptait 24 000 hommes.
${}^{16}Voici les préposés aux tribus d’Israël. Pour les gens de Roubène, le commandant était Èlièzer, fils de Zikri ; pour les gens de Siméon : Shefatyahou, fils de Maaka ; 
${}^{17}pour Lévi : Hashabya, fils de Qemouël ; pour Aaron : Sadoc ; 
${}^{18}pour Juda : Élihou, l’un des frères de David ; pour Issakar : Omri, fils de Mikaël ; 
${}^{19}pour Zabulon : Yishmayahou, fils d’Obadyahou ; pour Nephtali : Yerimoth, fils d’Azriël ; 
${}^{20}pour les fils d’Éphraïm : Osée, fils d’Azazyahou ; pour la demi-tribu de Manassé : Joël, fils de Pedayahou ; 
${}^{21}pour la demi-tribu de Manassé en Galaad : Yiddo, fils de Zacharie ; pour Benjamin : Yaasiël, fils d’Abner ; 
${}^{22}pour Dane : Azarel, fils de Yeroham. Tels étaient les princes des tribus d’Israël.
${}^{23}David n’avait pas relevé le nombre de ceux qui avaient vingt ans et au-dessous, parce que Dieu avait dit qu’il multiplierait Israël comme les étoiles du ciel. 
${}^{24}Joab, fils de Cerouya, avait commençé à les dénombrer, mais il n’avait pas achevé. Car, à cause du dénombrement, la Colère avait éclaté contre Israël, et le chiffre ne fut pas porté dans les comptes qui figurent dans les Annales du roi David.
${}^{25}Aux provisions du roi fut préposé Azmaweth, fils d’Adiël ; aux provisions dans la campagne, les villes, les villages et les tours : Jonathan, fils d’Ozias ; 
${}^{26}aux travailleurs des champs qui cultivaient le sol : Ezri, fils de Keloub ; 
${}^{27}aux vignobles : Shiméï, de Rama ; à ceux qui, dans les vignobles, s’occupaient des réserves de vin : Zabdi, de Shefam ; 
${}^{28}aux oliviers et aux sycomores dans le Bas-Pays : Baal-Hanane, de Guéder ; aux réserves d’huile : Yoash ; 
${}^{29}au gros bétail pâturant en Sarone : Shitraï, de Sarone ; au gros bétail dans les vallées : Shafath, fils d’Adlaï ; 
${}^{30}aux chameaux : Obil, l’Ismaélite ; aux ânesses : Yèhdeyahou, de Méronoth ; 
${}^{31}et au petit bétail : Yaziz, le Hagrite. Tous ceux-là étaient les responsables des biens appartenant au roi David.
${}^{32}Jonathan, oncle de David, était conseiller ; c’était un homme intelligent et il était scribe. Yehiël, fils de Hakmoni, s’occupait des fils du roi. 
${}^{33}Ahitofel était conseiller du roi. Houshaï, l’Arkite, était ami du roi. 
${}^{34}Joad, fils de Benaya, et Abiatar succédèrent à Ahitofel. Quant à Joab, il était chef de l’armée du roi.
      
         
      \bchapter{}
      \begin{verse}
${}^{1}David rassembla à Jérusalem tous les chefs d’Israël, ceux des tribus et ceux des classes qui étaient au service du roi, les officiers de millier et les officiers de centaine, et ceux qui étaient chargés de tous les biens et des troupeaux appartenant au roi et à ses fils, ainsi que les dignitaires et les guerriers, et tous les hommes de valeur. 
${}^{2}Le roi David se leva et, debout, déclara : « Écoutez-moi, mes frères et mon peuple. J’avais à cœur de bâtir une maison où reposeraient l’arche de l’Alliance du Seigneur et le piédestal de notre Dieu. J’avais fait les préparatifs en vue de la bâtir. 
${}^{3}Mais Dieu m’a dit : “Tu ne bâtiras pas une Maison pour mon nom, car tu es un homme de guerre et tu as répandu le sang.” 
${}^{4}De toute la maison de mon père, c’est moi que le Seigneur, Dieu d’Israël, a choisi pour être à jamais roi sur Israël. C’est en effet Juda qu’il a choisi pour en être le guide ; dans la maison de Juda, c’est ma famille qu’il a choisie ; et parmi les fils de mon père, c’est moi qu’il lui a plu de faire roi sur tout Israël. 
${}^{5}Parmi tous mes fils – car le Seigneur m’en a donné beaucoup –, c’est mon fils Salomon qu’il a choisi pour siéger sur le trône de la royauté du Seigneur sur Israël. 
${}^{6}Il m’a dit : “C’est ton fils Salomon qui bâtira ma Maison et mes cours, car je l’ai choisi pour fils et je serai pour lui un père. 
${}^{7}J’établirai fermement sa royauté à jamais, si, comme aujourd’hui, il persévère dans la pratique de mes commandements et de mes ordonnances.” 
${}^{8}Et maintenant, devant tout Israël qui nous voit, devant l’assemblée du Seigneur, devant notre Dieu qui nous entend, gardez, scrutez les commandements du Seigneur votre Dieu, afin que vous possédiez ce bon pays et le transmettiez après vous en héritage à vos fils, pour toujours. 
${}^{9}Toi, mon fils Salomon, connais le Dieu de ton père ; tu le serviras d’un cœur sans partage et d’une âme bien disposée, car le Seigneur scrute tous les cœurs et discerne toute forme de pensée. Si tu le recherches, il se laissera trouver par toi ; si tu l’abandonnes, il te rejettera pour toujours. 
${}^{10}Considère maintenant que le Seigneur t’a choisi pour lui bâtir une Maison comme sanctuaire. Sois fort ! Au travail ! »
      
         
       
${}^{11}David donna à son fils Salomon le plan du Vestibule, des bâtiments, des entrepôts, des chambres hautes, des chambres intérieures et de la salle du propitiatoire. 
${}^{12}Il lui donna aussi le plan de tout ce qu’il avait l’intention de faire concernant les cours de la maison du Seigneur, toutes les pièces alentour, les trésors de la maison de Dieu et les trésors des choses saintes. 
${}^{13}Il fit de même pour les classes des prêtres et des Lévites, toutes les charges du service de la maison du Seigneur, tous les objets destinés au service de la maison du Seigneur. 
${}^{14}Il fit de même pour l’or, donnant le poids en or de tous les objets de chaque service, et de même pour tous les objets d’argent, donnant le poids de tous les objets de chaque service. 
${}^{15}Il fit de même pour le poids des chandeliers d’or et de leurs lampes en or, donnant le poids de chacun des chandeliers et de ses lampes, et de même pour les chandeliers d’argent, donnant le poids du chandelier et de ses lampes selon l’utilisation de chacun des chandeliers.
${}^{16}Il fixa le poids en or destiné à chaque table des pains à disposer en rangées, puis l’argent destiné aux tables d’argent, 
${}^{17}puis les fourchettes, les bols pour l’aspersion, les aiguières en or pur ; il fit de même pour les coupes en or, donnant le poids pour chacune des coupes, et pour les coupes en argent, donnant le poids pour chacune des coupes. 
${}^{18}Il fit de même pour l’autel de l’encens en or purifié, dont il donna le poids. Il remit enfin le plan du char et des kéroubim en or aux ailes déployées qui protégeaient l’arche de l’Alliance du Seigneur. 
${}^{19}« Tout cela, disait David, se trouve dans un écrit de la main du Seigneur, qui m’a fait comprendre tout le travail dont il donnait le plan. »
${}^{20}David dit alors à son fils Salomon : « Sois ferme et courageux ! Au travail ! Ne crains pas, ne t’effraie pas, car le Seigneur Dieu, mon Dieu, est avec toi. Il ne te délaissera pas, il ne t’abandonnera pas, jusqu’à ce que tu aies achevé tout le travail pour le service de la maison du Seigneur. 
${}^{21}Voici les classes des prêtres et des Lévites pour tout le service de la maison de Dieu. Tu auras avec toi, pour tous ces travaux, des gens qui seront tous volontaires et habiles en toutes choses. Les chefs et tout le peuple seront à tes ordres. »
      <p class="cantique" id="bib_ct-at_4"><span class="cantique_label">Cantique AT 4</span> = <span class="cantique_ref"><a class="unitex_link" href="#bib_1ch_29_10">1 Ch 29, 10b-13</a></span>
      
         
      \bchapter{}
      \begin{verse}
${}^{1}Le roi David s’adressa à toute l’assemblée : « Mon fils Salomon, le seul que Dieu a choisi, est jeune et faible, alors que l’ouvrage à accomplir est considérable, car cette citadelle n’est pas destinée à un homme, mais au Seigneur Dieu. 
${}^{2}De toutes mes forces, j’ai préparé pour la maison de mon Dieu : de l’or pour ce qui doit être en or, de l’argent pour ce qui doit être en argent, du bronze pour ce qui doit être en bronze, du fer pour ce qui doit être en fer, du bois pour ce qui doit être en bois, des pierres de cornaline, des pierres d’ornement, des pierres noires et des pierres multicolores, toutes sortes de pierres précieuses, et de l’albâtre en abondance. 
${}^{3}Plus encore. Ce que je possède personnellement en or et en argent, je le donne à la maison de mon Dieu par amour pour elle, en plus de ce que j’ai préparé pour cette sainte Maison : 
${}^{4}trois mille talents d’or, de l’or d’Ophir, sept mille talents d’argent purifié, pour en recouvrir les murs des bâtiments ; 
${}^{5}l’or pour ce qui doit être en or, l’argent pour ce qui doit être en argent, et pour tout ouvrage qui sort de la main des artisans. Qui d’entre vous aujourd’hui est disposé à faire un don volontaire au Seigneur ? »
${}^{6}Alors les chefs de famille, les chefs des tribus d’Israël, les officiers de millier et de centaine, et les chefs des travaux du roi furent volontaires. 
${}^{7}Ils donnèrent pour le service de la maison de Dieu cinq mille talents d’or, dix mille pièces d’or, dix mille talents d’argent, dix-huit mille talents de bronze et cent mille talents de fer. 
${}^{8}Ceux qui possédaient des pierres précieuses les remirent, pour le trésor de la maison du Seigneur, entre les mains de Yehiël le Guershonite. 
${}^{9}Le peuple se réjouit de leurs dons volontaires, car c’est d’un cœur sans partage qu’ils avaient ainsi fait des dons volontaires au Seigneur. Le roi David en éprouva aussi une grande joie.
       
${}^{10}Alors David bénit le Seigneur sous les yeux de toute l’assemblée. Il dit :
       
        \\Béni\\sois-tu, Seigneur,
        \\Dieu de notre père Israël,
        \\depuis les siècles et pour les siècles !
         
        ${}^{11}À toi, Seigneur, force et grandeur,
        \\éclat, victoire, majesté,
        \\tout, dans les cieux et sur la terre !
         
        \\À toi, Seigneur, le règne,
        \\la primauté sur l’univers :
        ${}^{12}la richesse et la gloire viennent de ta face !
         
        \\C’est toi, le Maître de tout :
        \\dans ta main, force et puissance ;
        \\tout, par ta main, grandit et s’affermit.
         
        ${}^{13}Et maintenant, ô notre Dieu,
        \\nous voici pour te rendre grâce,
        \\pour célébrer l’éclat de ton nom !
         
${}^{14}Qui suis-je, en effet, et qui est mon peuple,
        \\pour être en mesure de faire de tels dons volontaires ?
        \\Tout nous vient de toi, et c’est de ta main
        \\que nous avons reçu ce que nous te donnons.
         
${}^{15}Car nous ne sommes devant toi que des immigrés,
        \\des hôtes comme tous nos pères ;
        \\nos jours sur la terre sont comme l’ombre,
        \\et il n’est pas d’espoir.
         
${}^{16}Seigneur, notre Dieu, toutes ces richesses
        \\que nous avons préparées à profusion
        \\pour te bâtir une Maison à ton saint nom
        \\viennent de ta main, et tout t’appartient.
         
${}^{17}Je le sais, mon Dieu, tu scrutes les cœurs
        \\et tu aimes la droiture.
        \\Moi, c’est d’un cœur droit
        \\que j’ai fait tous ces dons volontaires.
         
        \\Et maintenant, ton peuple ici présent,
        \\je le vois avec joie
        \\te faire ces dons volontaires.
         
${}^{18}Seigneur, Dieu d’Abraham, d’Isaac et d’Israël, nos pères,
        \\garde à jamais ces dispositions
        \\dans le cœur de ton peuple,
        \\et que son cœur s’attache à toi.
         
${}^{19}À mon fils Salomon, donne un cœur sans partage,
        \\pour qu’il garde tes commandements,
        \\tes exigences et tes décrets.
        \\Qu’il les mette tous en pratique
        \\et bâtisse la citadelle que je t’ai préparée. »
         
${}^{20}David dit ensuite à toute l’assemblée :
        \\« Bénissez le Seigneur votre Dieu ! »
       
      Et toute l’assemblée bénit le Seigneur, Dieu de leurs pères. Ils s’inclinèrent et se prosternèrent devant Dieu et devant le roi.
       
${}^{21}Le lendemain de ce jour, ils offrirent au Seigneur des sacrifices et des holocaustes : mille taureaux, mille béliers, mille agneaux, avec les libations habituelles, ainsi que des sacrifices en grand nombre pour tout Israël. 
${}^{22}Ce jour-là, ils mangèrent et burent en grande joie devant le Seigneur. Ils firent roi, pour la seconde fois, Salomon, fils de David ; ils donnèrent, pour le Seigneur, l’onction à Salomon comme chef et à Sadoc comme prêtre. 
${}^{23}Salomon siégea sur le trône du Seigneur comme roi à la place de David son père, et il réussit. Tout Israël lui obéit. 
${}^{24}Tous les officiers, les guerriers et même tous les fils du roi David se soumirent au roi Salomon. 
${}^{25}Sous les yeux de tout Israël, le Seigneur éleva Salomon au sommet de la grandeur. Il lui donna une majesté royale telle que n’en avait eue avant lui aucun roi en Israël.
${}^{26}David, fils de Jessé, régna donc sur tout Israël.
${}^{27}Son règne sur Israël fut de quarante ans : il avait régné sept ans à Hébron et trente-trois ans à Jérusalem.
${}^{28}Il mourut dans une heureuse vieillesse, rassasié de jours, de richesses et de gloire. Son fils Salomon régna à sa place.
${}^{29}Les actions du roi David, des premières aux dernières,
        \\cela est écrit dans les Actes de Samuel le Voyant,
        \\dans les Actes de Nathan le prophète,
        \\et dans les Actes de Gad le Voyant,
${}^{30}avec ce qui concerne tout son règne,
        \\sa bravoure, et les événements survenus dans sa vie, en Israël
        \\et dans tous les royaumes des autres pays.
