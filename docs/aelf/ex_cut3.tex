  
  
      <h2 class="intertitle" id="d85e22043">1. La manifestation de Dieu (19)</h2>
      
         
      \bchapter{}
      \begin{verse}
${}^{1}Le troisième mois qui suivit la sortie d’Égypte, jour pour jour\\, les fils d’Israël arrivèrent dans le désert du Sinaï. 
${}^{2} C’est en partant de Rephidim qu’ils arrivèrent dans ce désert, et ils y établirent leur camp juste en face de la montagne.
${}^{3}Moïse monta vers Dieu. Le Seigneur l’appela du haut de la montagne : « Tu diras à la maison de Jacob, et tu annonceras aux fils d’Israël : 
${}^{4} “Vous avez vu ce que j’ai fait à l’Égypte, comment je vous ai portés comme sur les ailes d’un aigle et vous ai amenés jusqu’à moi. 
${}^{5} Maintenant donc, si vous écoutez ma voix et gardez mon alliance, vous serez mon domaine particulier\\parmi tous les peuples, car toute la terre m’appartient ; 
${}^{6} mais vous, vous serez pour moi un royaume de prêtres, une nation sainte.” Voilà ce que tu diras aux fils d’Israël. » 
${}^{7} Moïse revint et convoqua les anciens du peuple, il leur exposa tout\\ce que le Seigneur avait ordonné. 
${}^{8} Le peuple tout entier répondit, unanime : « Tout ce que le Seigneur a dit, nous le mettrons en pratique\\. » Et Moïse rapporta au Seigneur les paroles du peuple.
${}^{9}Le Seigneur dit à Moïse : « Je vais venir vers toi dans l’épaisseur de la nuée, pour que le peuple, qui m’entendra te parler, mette sa foi en toi\\, pour toujours. » Puis Moïse transmit au Seigneur les paroles du peuple.
${}^{10}Le Seigneur dit encore à Moïse : « Va vers le peuple ; sanctifie-le, aujourd’hui et demain ; qu’ils lavent leurs vêtements, 
${}^{11}pour être prêts le troisième jour ; car, ce troisième jour, en présence de tout le peuple, le Seigneur descendra sur la montagne du Sinaï. 
${}^{12}Fixe des limites au peuple, en leur disant : Gardez-vous de gravir la montagne et d’en toucher le bord ! Quiconque touchera la montagne sera mis à mort ! 
${}^{13}Le condamné, tu ne le toucheras pas de la main, il sera lapidé ou percé de flèches. Qu’il s’agisse d’un animal ou d’un homme, il ne vivra pas. Quand la trompe retentira, quelques-uns monteront sur la montagne. »
${}^{14}Moïse descendit de la montagne vers le peuple. Il sanctifia le peuple ; tous lavèrent leurs vêtements, 
${}^{15}et Moïse dit au peuple : « Soyez prêts dans trois jours. N’approchez aucune femme. »
${}^{16}Le troisième jour, dès le matin, il y eut des coups de tonnerre, des éclairs, une lourde nuée sur la montagne, et une puissante sonnerie de cor ; dans le camp, tout le peuple trembla. 
${}^{17}Moïse fit sortir le peuple hors du camp, à la rencontre de Dieu, et ils restèrent debout au pied de la montagne. 
${}^{18}La montagne du Sinaï était toute fumante, car le Seigneur y était descendu dans le feu ; la fumée montait, comme la fumée d’une fournaise, et toute la montagne tremblait violemment. 
${}^{19}La sonnerie du cor était de plus en plus puissante. Moïse parlait, et la voix de Dieu lui répondait\\. 
${}^{20}Le Seigneur descendit sur le sommet du Sinaï, il appela Moïse sur le sommet de la montagne, et Moïse monta vers lui. 
${}^{21}Le Seigneur dit à Moïse : « Descends et avertis le peuple de ne pas se précipiter pour voir le Seigneur, car beaucoup d’entre eux périraient. 
${}^{22}Même les prêtres qui s’approchent du Seigneur doivent se sanctifier, de peur que le Seigneur ne s’emporte contre eux. » 
${}^{23}Moïse répondit au Seigneur : « Le peuple ne peut pas monter sur la montagne du Sinaï, puisque, toi-même, tu nous as avertis en ces termes : “Délimite la montagne et sanctifie-la.” » 
${}^{24}Puis le Seigneur lui dit : « Va, descends. Ensuite, tu remonteras, toi et Aaron avec toi. Quant aux prêtres et au peuple, qu’ils ne se précipitent pas pour monter vers le Seigneur, de peur que le Seigneur ne s’emporte contre eux. » 
${}^{25}Moïse descendit vers le peuple et leur en fit part.
      <h2 class="intertitle" id="d85e22285">2. Le Décalogue (20,1-21)</h2>
      
         
      \bchapter{}
      \begin{verse}
${}^{1}Alors Dieu prononça toutes les paroles que voici : 
${}^{2} « Je suis le Seigneur ton Dieu, qui t’ai fait sortir du pays d’Égypte, de la maison d’esclavage.
${}^{3}Tu n’auras pas d’autres dieux en face de moi.
${}^{4}Tu ne feras aucune idole, aucune image de ce qui est là-haut dans les cieux, ou en bas sur la terre, ou dans les eaux par-dessous la terre. 
${}^{5} Tu ne te prosterneras pas devant ces dieux\\, pour leur rendre un culte. Car moi, le Seigneur ton Dieu, je suis un Dieu jaloux : chez ceux qui me haïssent, je punis la faute des pères sur les fils, jusqu’à la troisième et la quatrième génération ; 
${}^{6} mais ceux qui m’aiment et observent mes commandements, je leur montre ma fidélité\\jusqu’à la millième génération.
${}^{7}Tu n’invoqueras pas en vain le nom du Seigneur ton Dieu, car le Seigneur ne laissera pas impuni celui qui invoque en vain son nom.
${}^{8}Souviens-toi du jour du sabbat pour le sanctifier. 
${}^{9} Pendant six jours tu travailleras et tu feras tout ton ouvrage ; 
${}^{10} mais le septième jour est le jour du repos, sabbat en l’honneur du Seigneur ton Dieu : tu ne feras aucun ouvrage, ni toi, ni ton fils, ni ta fille, ni ton serviteur, ni ta servante, ni tes bêtes, ni l’immigré qui est dans ta ville. 
${}^{11} Car en six jours le Seigneur a fait le ciel, la terre, la mer et tout ce qu’ils contiennent, mais il s’est reposé le septième jour. C’est pourquoi le Seigneur a béni le jour du sabbat et l’a sanctifié.
${}^{12}Honore ton père et ta mère, afin d’avoir longue vie sur la terre que te donne le Seigneur ton Dieu.
${}^{13}Tu ne commettras pas de meurtre.
${}^{14}Tu ne commettras pas d’adultère.
${}^{15}Tu ne commettras pas de vol.
${}^{16}Tu ne porteras pas de faux témoignage contre ton prochain.
${}^{17}Tu ne convoiteras pas la maison de ton prochain ; tu ne convoiteras pas la femme de ton prochain, ni son serviteur, ni sa servante, ni son bœuf, ni son âne : rien de ce qui lui appartient. »
${}^{18}Tout le peuple voyait les éclairs, les coups de tonnerre, la sonnerie du cor et la montagne fumante. Le peuple voyait : ils frémirent et se tinrent à distance. 
${}^{19} Ils dirent à Moïse : « Toi, parle-nous, et nous écouterons ; mais que Dieu ne nous parle pas, car ce serait notre mort. » 
${}^{20} Moïse répondit au peuple : « N’ayez pas peur. Dieu est venu pour vous mettre à l’épreuve\\, pour que vous soyez saisis de crainte en face de lui, et que vous ne péchiez pas. » 
${}^{21} Le peuple se tint à distance, mais Moïse s’approcha de la nuée obscure où Dieu était.
      <h2 class="intertitle" id="d85e22474">3. Le Code de l’Alliance (20,22 – 24)</h2>
${}^{22}Le Seigneur dit à Moïse : « Tu parleras ainsi aux fils d’Israël : “Vous avez vu que je vous ai parlé du haut des cieux. 
${}^{23}Vous ne ferez pas, à côté de moi, des dieux d’argent ou d’or ; vous n’en ferez pas pour moi. 
${}^{24}Tu me feras un autel de terre pour offrir tes holocaustes et tes sacrifices de paix, ton petit et ton gros bétail ; en tout lieu où je ferai rappeler mon nom, je viendrai vers toi et je te bénirai. 
${}^{25}Mais si tu me fais un autel de pierres, tu ne le bâtiras pas en pierres de taille car, en y passant ton ciseau, tu les profanerais. 
${}^{26}Et tu ne monteras pas à mon autel par des marches, afin que ta nudité n’y soit pas découverte.” »
      
         
      \bchapter{}
      \begin{verse}
${}^{1}« Voici les règles que tu leur exposeras.
      \begin{verse}
${}^{2}Quand tu achèteras un esclave hébreu, il servira durant six ans ; la septième année, il pourra s’en aller, libre, sans rien payer. 
${}^{3}S’il est arrivé seul, il s’en ira seul. S’il est déjà marié, sa femme s’en ira avec lui. 
${}^{4}Si son maître lui donne une femme et qu’elle lui enfante des fils ou des filles, la femme et les enfants appartiendront au maître, et lui s’en ira seul. 
${}^{5}Mais si l’esclave déclare : “J’aime mon maître, ma femme et mes fils, je ne veux pas être libéré”, 
${}^{6}son maître le fera approcher de Dieu, il le fera approcher du battant ou du montant de la porte, et lui percera l’oreille au poinçon. Alors l’esclave le servira pour toujours.
${}^{7}Et quand un homme vendra sa fille comme servante, elle ne s’en ira pas comme s’en vont les esclaves. 
${}^{8}Si elle déplaît à son maître, qui se l’était destinée, et qu’il la fasse racheter, il n’aura pas le droit de la vendre à un peuple étranger, car ce serait la trahir. 
${}^{9}S’il la destine à son fils, il agira pour elle selon la règle concernant les filles. 
${}^{10}S’il prend pour lui une autre femme, il ne diminuera en rien la nourriture, le vêtement, le logement de la première. 
${}^{11}Et s’il ne lui procure pas ces trois choses, elle pourra s’en aller, sans rien payer, sans verser d’argent.
${}^{12}« Qui frappe un homme à mort sera mis à mort. 
${}^{13}Mais s’il n’a pas traqué sa victime, si Dieu l’a mise à portée de sa main, je te fixerai un lieu où il pourra se réfugier. 
${}^{14}Mais quand un homme est en rage contre son prochain au point de le tuer par ruse, tu l’arracheras même de mon autel pour qu’il meure.
${}^{15}Celui qui frappe son père ou sa mère sera mis à mort.
${}^{16}Celui qui commet un rapt – qu’il ait vendu l’homme ou qu’on le trouve entre ses mains – sera mis à mort.
${}^{17}Celui qui maudit son père ou sa mère sera mis à mort.
${}^{18}« Quand des hommes se querellent et que l’un d’eux frappe son prochain avec une pierre ou avec le poing, sans le tuer mais en l’obligeant à garder le lit, 
${}^{19}si la victime peut se lever et se promener au-dehors avec sa canne, l’agresseur sera acquitté. Il devra seulement l’indemniser pour son arrêt de travail, jusqu’à complète guérison.
${}^{20}Si quelqu’un frappe avec un bâton et fait mourir de sa main son serviteur ou sa servante, la victime devra être vengée. 
${}^{21}Mais si elle survit un jour ou deux, elle ne sera pas vengée, car elle a été achetée avec l’argent du maître.
${}^{22}Si des hommes, en se battant, heurtent une femme enceinte et que celle-ci accouche prématurément sans qu’un autre malheur n’arrive, le coupable paiera l’indemnité imposée par le mari, avec l’accord des juges. 
${}^{23}Mais s’il arrive malheur, tu paieras vie pour vie, 
${}^{24}œil pour œil, dent pour dent, main pour main, pied pour pied, 
${}^{25}brûlure pour brûlure, blessure pour blessure, meurtrissure pour meurtrissure.
${}^{26}Si un homme blesse l’œil de son serviteur ou de sa servante, et que l’œil soit perdu, il rendra la liberté à la victime en compensation. 
${}^{27}Et s’il fait tomber une dent de son serviteur ou de sa servante, il rendra la liberté à la victime en compensation.
${}^{28}Si un bœuf tue d’un coup de corne un homme ou une femme, il sera lapidé et on ne mangera pas la viande. Mais le propriétaire sera tenu pour innocent. 
${}^{29}Par contre, quand le bœuf a déjà, plus d’une fois, donné des coups de corne et que son propriétaire, averti, l’a laissé sans surveillance, si l’animal a causé la mort d’un homme ou d’une femme, il sera lapidé, et le propriétaire lui-même sera mis à mort. 
${}^{30}Et si on lui impose une rançon, il donnera, pour racheter sa vie, tout ce qu’on lui imposera. 
${}^{31}Si c’est un fils que le bœuf frappe d’un coup de corne, ou si c’est une fille, on appliquera cette règle-là. 
${}^{32}Si c’est un serviteur que le bœuf frappe, ou si c’est une servante, on donnera au maître trente pièces d’argent, et le bœuf sera lapidé.
${}^{33}Si un homme laisse une citerne ouverte ou qu’il creuse une citerne sans la recouvrir, et qu’un bœuf ou un âne y tombe, 
${}^{34}le propriétaire de la citerne indemnisera le propriétaire de la bête morte ; celui-ci recevra une certaine somme d’argent et celui-là, le cadavre de la bête.
${}^{35}Si le bœuf d’un homme blesse le bœuf de son prochain et cause sa mort, les propriétaires vendront le bœuf vivant et se partageront l’argent ; quant à la bête morte, ils se la partageront également. 
${}^{36}Mais s’il est notoire que ce bœuf a déjà, plus d’une fois, donné des coups de corne et que son propriétaire l’a laissé sans surveillance, celui-ci fournira un bœuf en compensation de la bête morte qui, elle, lui reviendra.
${}^{37}« Si un homme vole un bœuf ou un mouton, et qu’il abatte ou vende la bête, il fournira en compensation cinq têtes de gros bétail pour un bœuf ou quatre têtes de petit bétail pour un mouton.
      
         
      \bchapter{}
      \begin{verse}
${}^{1}Si un voleur, surpris de nuit en délit d’effraction, est frappé à mort, les siens ne pourront pas le venger. 
${}^{2}Mais si le soleil était levé, la vengeance du sang s’exercera. Un voleur devra rembourser : s’il n’a pas de quoi, il sera vendu pour ce qu’il a volé. 
${}^{3}Si la bête volée – bœuf, âne ou mouton – est retrouvée vivante entre ses mains, il fournira en compensation le double de sa valeur.
${}^{4}Quand un homme fait brouter un champ ou une vigne, s’il envoie ses bêtes brouter le champ de quelqu’un d’autre, il fournira une compensation avec le meilleur de son champ, le meilleur de sa vigne.
${}^{5}Si un feu éclate, se propage dans des buissons d’épines et consume les meules, les moissons ou les champs, le responsable devra rembourser ce qui a brûlé.
${}^{6}Si un homme confie à son prochain de l’argent ou des objets pour qu’il les garde, et qu’on les vole dans la maison de celui-ci, le voleur, s’il est découvert, devra fournir en compensation le double de ce qu’il a pris. 
${}^{7}Mais si le voleur n’est pas découvert, le maître de la maison s’approchera de Dieu pour jurer qu’il n’a pas porté la main sur le bien de son prochain.
${}^{8}Pour toute affaire frauduleuse portant sur un bœuf, un âne, un mouton, un vêtement ou tout objet perdu dont chacun dira : « C’est bien à moi ! », l’affaire des deux parties sera portée devant Dieu. Et celui que Dieu déclarera coupable devra fournir le double en compensation à son prochain.
${}^{9}Si un homme confie à son prochain un âne, un bœuf, un mouton, ou toute sorte de bête pour qu’il la garde, et que la bête crève, se blesse ou soit enlevée sans témoin, 
${}^{10}les deux parties prêteront serment au nom du Seigneur : le gardien jurera qu’il n’a pas porté la main sur le bien du prochain. Alors le propriétaire de l’animal acceptera, et le gardien n’aura pas à fournir de compensation. 
${}^{11}Mais si l’animal a été volé à proximité du gardien, celui-ci fournira une compensation au propriétaire. 
${}^{12}Si l’animal a été déchiré, le gardien en produira une preuve et n’aura pas à fournir de compensation.
${}^{13}Et si un homme emprunte un animal à son prochain, et que la bête se blesse ou crève en l’absence de son propriétaire, il devra fournir une compensation à ce dernier. 
${}^{14}Mais il ne devra rien, si le propriétaire est présent. S’il s’agit d’un animal loué, le montant de la location reste dû.
${}^{15}Si un homme séduit une jeune fille vierge qui n’est pas fiancée, et qu’il couche avec elle, il devra verser le prix pour en faire sa femme. 
${}^{16}Si le père refuse de lui donner sa fille, l’homme versera néanmoins la somme d’argent fixée pour le prix d’une vierge.
${}^{17}« Une sorcière, tu ne la laisseras pas vivre.
${}^{18}Celui qui couche avec une bête sera mis à mort.
${}^{19}Celui qui sacrifie aux dieux sera voué à l’anathème, sauf s’il sacrifie au Seigneur, à lui seul.
${}^{20}Tu n’exploiteras pas l’immigré, tu ne l’opprimeras pas, car vous étiez vous-mêmes des immigrés au pays d’Égypte. 
${}^{21} Vous n’accablerez pas la veuve et l’orphelin. 
${}^{22} Si tu les accables et qu’ils crient vers moi, j’écouterai leur cri. 
${}^{23} Ma colère s’enflammera et je vous ferai périr par l’épée : vos femmes deviendront veuves, et vos fils, orphelins.
${}^{24}Si tu prêtes de l’argent à quelqu’un de mon peuple, à un pauvre parmi tes frères\\, tu n’agiras pas envers lui comme un usurier : tu ne lui imposeras pas d’intérêts. 
${}^{25} Si tu prends en gage le manteau de ton prochain, tu le lui rendras avant le coucher du soleil. 
${}^{26} C’est tout ce qu’il a pour se couvrir ; c’est le manteau dont il s’enveloppe, la seule couverture qu’il ait pour dormir. S’il crie vers moi, je l’écouterai, car moi, je suis compatissant\\ !
${}^{27}Dieu, tu ne le maudiras pas, et tu ne prononceras pas de malédiction contre un chef de ton peuple.
${}^{28}Tu ne tarderas pas à offrir le fruit de tes champs et de ton pressoir.
      Le premier-né de tes fils, tu me le donneras. 
${}^{29}Tu feras de même pour ton bœuf et ton petit bétail : le premier-né restera sept jours avec sa mère ; le huitième jour, tu me le donneras.
${}^{30}Vous serez pour moi des hommes de sainteté. Vous ne mangerez pas la viande d’une bête déchirée par un fauve dans la campagne ; vous la jetterez aux chiens.
      
         
      \bchapter{}
      \begin{verse}
${}^{1}« Tu ne répandras pas de vaines rumeurs. Tu ne prêteras pas main forte au méchant en lui servant de témoin à charge.
${}^{2}Tu ne suivras pas la foule pour faire le mal ; et quand tu déposeras dans un procès, tu ne t’aligneras pas sur son opinion pour faire dévier le droit. 
${}^{3}Tu ne favoriseras pas un faible dans son procès. 
${}^{4}Quand tu rencontreras, égaré, le bœuf ou l’âne de ton ennemi, tu devras le lui ramener. 
${}^{5}Si tu vois l’âne de celui qui te déteste crouler sous la charge, tu ne le laisseras pas à l’abandon mais tu lui viendras en aide. 
${}^{6}Tu ne feras pas dévier le droit du malheureux qui s’adresse à toi lors de son procès. 
${}^{7}Tu te tiendras éloigné d’une cause mensongère. Ne tue pas l’innocent ni le juste, car je ne justifie pas le méchant. 
${}^{8}Tu n’accepteras pas de présent, car le présent aveugle les clairvoyants et compromet la cause des justes. 
${}^{9}Tu n’opprimeras pas l’immigré : vous savez bien ce qu’est sa vie, car vous avez été, vous aussi, des immigrés au pays d’Égypte.
${}^{10}« Pendant six ans, tu ensemenceras la terre et tu récolteras son produit. 
${}^{11}Mais, la septième année, tu la laisseras en jachère et tu abandonneras son produit : les malheureux de ton peuple le mangeront et, ce qu’ils auront laissé, les bêtes sauvages le mangeront. Tu feras de même pour ta vigne et ton olivier.
${}^{12}Pendant six jours, tu feras ce que tu as à faire, mais, le septième jour, tu chômeras, afin que ton bœuf et ton âne se reposent, et que le fils de ta servante et l’immigré reprennent souffle.
${}^{13}Vous prendrez bien garde à tout ce que je vous ai dit. Vous ne prononcerez pas le nom d’autres dieux : on ne l’entendra pas sortir de ta bouche.
${}^{14}« Tu me fêteras trois fois par an. 
${}^{15}Tu observeras la fête des Pains sans levain. Comme je te l’ai ordonné, tu mangeras des pains sans levain pendant sept jours, au temps fixé du mois des Épis, car c’est alors que tu es sorti d’Égypte. On ne paraîtra pas devant ma face les mains vides. 
${}^{16}Tu observeras aussi la fête de la Moisson, celle des premiers fruits de ton travail, de ce que tu auras semé dans les champs. Et tu observeras la fête de la Récolte en fin d’année, quand tu récoltes dans les champs le fruit de ton travail. 
${}^{17}Trois fois par an, tous les hommes paraîtront devant la face du Maître, le Seigneur. 
${}^{18}Tu ne présenteras pas le sacrifice sanglant avec du pain levé, et tu ne laisseras pas jusqu’au lendemain matin la graisse offerte pour me fêter. 
${}^{19}Tu apporteras les tout premiers fruits de ton sol à la maison du Seigneur ton Dieu. Tu ne feras pas cuire un chevreau dans le lait de sa mère.
${}^{20}« Je vais envoyer un ange devant toi pour te garder en chemin et te faire parvenir au lieu que je t’ai préparé. 
${}^{21}Respecte sa présence, écoute sa voix. Ne lui résiste pas : il ne te pardonnerait pas ta révolte, car mon nom est en lui. 
${}^{22}Mais si tu écoutes parfaitement sa voix, si tu fais tout ce que je dirai, je serai l’ennemi de tes ennemis, et l’adversaire de tes adversaires. 
${}^{23}Mon ange marchera devant toi. Il te fera rencontrer de nombreux peuples : l’Amorite, le Hittite, le Perizzite et le Cananéen, le Hivvite et le Jébuséen. Je vais tous les anéantir. 
${}^{24}Tu ne te prosterneras pas devant leurs dieux. Tu ne les serviras pas. Tu ne te conduiras pas comme ces peuples, mais tu détruiras leurs dieux et tu briseras leurs stèles. 
${}^{25}Vous servirez le Seigneur votre Dieu : il bénira ton pain et ton eau, et j’écarterai de toi la maladie. 
${}^{26}Aucune femme de ton pays n’aura de fausse couche ou ne sera stérile, et je laisserai s’accomplir le nombre de tes jours. 
${}^{27}Devant toi, j’enverrai ma terreur ; je frapperai de panique tout peuple chez qui tu entreras ; devant toi, je ferai tourner le dos à tous tes ennemis. 
${}^{28}Devant toi, j’enverrai des frelons ; devant toi, ils chasseront le Hivvite, le Cananéen et le Hittite. 
${}^{29}Je ne les chasserai pas devant toi en une seule année, car le pays deviendrait une terre désolée où les bêtes sauvages se multiplieraient à tes dépens. 
${}^{30}Je les chasserai devant toi peu à peu, jusqu’à ce que ton peuple soit assez nombreux pour hériter du pays. 
${}^{31}Je fixerai tes frontières ainsi : de la mer des Roseaux à la Méditerranée, et du désert au Fleuve. Je livrerai entre vos mains les habitants du pays, et tu les chasseras devant toi. 
${}^{32}Tu ne concluras pas d’alliance avec eux ni avec leurs dieux. 
${}^{33}Ils n’habiteront pas dans ton pays, de peur qu’ils ne te fassent pécher contre moi : tu pourrais alors servir leurs dieux et ce serait pour toi un piège. »
      
         
      \bchapter{}
      \begin{verse}
${}^{1}Le Seigneur avait dit à Moïse : « Monte vers le Seigneur et prends avec toi Aaron, ses deux fils Nadab et Abihou, et soixante-dix des anciens d’Israël. Vous vous prosternerez à distance. 
${}^{2}Moïse, seul, s’approchera du Seigneur. Les autres ne s’approcheront pas et le peuple ne montera pas avec lui. »
${}^{3}Moïse vint rapporter au peuple toutes les paroles du Seigneur et toutes ses ordonnances. Tout le peuple répondit d’une seule voix : « Toutes ces paroles que le Seigneur a dites, nous les mettrons en pratique. »
${}^{4}Moïse écrivit toutes les paroles du Seigneur. Il se leva de bon matin et il bâtit un autel au pied de la montagne, et il dressa douze pierres pour les douze tribus d’Israël. 
${}^{5}Puis il chargea quelques jeunes garçons parmi les fils d’Israël d’offrir des holocaustes, et d’immoler au Seigneur des taureaux en sacrifice de paix. 
${}^{6}Moïse prit la moitié du sang et le mit dans des coupes\\ ; puis il aspergea l’autel avec le reste du sang. 
${}^{7}Il prit le livre de l’Alliance et en fit la lecture au peuple. Celui-ci répondit : « Tout ce que le Seigneur a dit, nous le mettrons en pratique\\, nous y obéirons. » 
${}^{8}Moïse prit le sang, en aspergea le peuple, et dit : « Voici le sang de l’Alliance que, sur la base de toutes ces paroles, le Seigneur a conclue avec vous. »
${}^{9}Et Moïse gravit la montagne avec Aaron, Nadab et Abihou, et soixante-dix des anciens d’Israël. 
${}^{10}Ils virent le Dieu d’Israël : il avait sous les pieds comme un pavement de saphir, limpide comme le fond du ciel. 
${}^{11}Sur ces privilégiés parmi les fils d’Israël, il ne porta pas la main. Ils contemplèrent Dieu, puis ils mangèrent et ils burent.
${}^{12}Le Seigneur dit à Moïse : « Monte vers moi sur la montagne et reste là ; je vais te donner les tables de pierre, la loi et les commandements que j’ai écrits pour qu’on les enseigne. » 
${}^{13}Moïse se leva avec Josué, son auxiliaire, et il gravit la montagne de Dieu. 
${}^{14}Auparavant il avait dit aux anciens : « Attendez-nous ici jusqu’à notre retour. Aaron et Hour sont avec vous : celui qui a une affaire à régler, qu’il s’adresse à eux. »
${}^{15}Moïse gravit donc la montagne, et la nuée recouvrit la montagne, 
${}^{16}la gloire du Seigneur demeura sur la montagne du Sinaï, que la nuée recouvrit pendant six jours. Le septième jour, le Seigneur appela Moïse du milieu de la nuée. 
${}^{17}La gloire du Seigneur apparaissait aux fils d’Israël comme un feu dévorant, au sommet de la montagne. 
${}^{18}Moïse entra dans la nuée et gravit la montagne. Moïse resta sur la montagne quarante jours et quarante nuits.
      <h2 class="intertitle" id="d85e23273">4. Directives pour la construction de la Demeure (25 – 31)</h2>
      
         
      \bchapter{}
      \begin{verse}
${}^{1}Le Seigneur parla à Moïse. Il dit : 
${}^{2}« Dis aux fils d’Israël de prélever pour moi une contribution. Vous la recevrez de tout homme que son cœur y incitera. 
${}^{3}Voici la contribution que vous recevrez d’eux : de l’or, de l’argent et du bronze, 
${}^{4}de la pourpre violette et de la pourpre rouge, du cramoisi éclatant, du lin et du poil de chèvre, 
${}^{5}des peaux de bélier teintes en rouge, du cuir fin et du bois d’acacia, 
${}^{6}de l’huile pour le luminaire, du baume pour l’huile de l’onction et de l’encens aromatique, 
${}^{7}des pierres de cornaline et des pierres pour orner l’éphod et le pectoral. 
${}^{8}Ils me feront un sanctuaire et je demeurerai au milieu d’eux. 
${}^{9}Je vais te montrer le modèle de la Demeure et le modèle de tous ses objets : vous les reproduirez exactement.
      
         
${}^{10}« On fera une arche en bois d’acacia de deux coudées et demie de long sur une coudée et demie de large et une coudée et demie de haut. 
${}^{11}Tu la plaqueras d’or pur à l’intérieur et à l’extérieur, et tu l’entoureras d’une moulure en or. 
${}^{12}Tu couleras quatre anneaux d’or que tu attacheras aux quatre pieds de l’arche : deux anneaux sur un côté, deux anneaux sur l’autre. 
${}^{13}Tu feras des barres en bois d’acacia, tu les plaqueras d’or 
${}^{14}et tu les introduiras dans les anneaux des côtés de l’arche pour pouvoir la porter. 
${}^{15}Les barres resteront dans les anneaux de l’arche ; elles n’en seront pas retirées. 
${}^{16}Tu placeras dans l’arche le Témoignage que je te donnerai. 
${}^{17}Puis tu feras en or pur un couvercle, le propitiatoire, long de deux coudées et demie et large d’une coudée et demie. 
${}^{18}Ensuite tu forgeras deux kéroubim en or à placer aux deux extrémités du propitiatoire. 
${}^{19}Fais un kéroub à une extrémité, et l’autre kéroub à l’autre extrémité ; vous ferez donc les kéroubim aux deux extrémités du propitiatoire. 
${}^{20}Les kéroubim auront les ailes déployées vers le haut et protégeront le propitiatoire de leurs ailes. Ils se feront face, le regard tourné vers le propitiatoire. 
${}^{21}Tu placeras le propitiatoire sur le dessus de l’arche et, dans l’arche, tu placeras le Témoignage que je te donnerai. 
${}^{22}C’est là que je te laisserai me rencontrer ; je parlerai avec toi d’au-dessus du propitiatoire entre les deux kéroubim situés sur l’arche du Témoignage ; là, je te donnerai mes ordres pour les fils d’Israël.
${}^{23}« Puis tu feras une table en bois d’acacia, longue de deux coudées, large d’une coudée et haute d’une coudée et demie. 
${}^{24}Tu la plaqueras d’or pur et tu l’entoureras d’une moulure en or. 
${}^{25}Tu feras des entretoises de la largeur d’une main et tu les entoureras d’une moulure en or. 
${}^{26}Tu feras quatre anneaux d’or que tu mettras aux quatre angles formés par les quatre pieds. 
${}^{27}Ces anneaux seront placés près des entretoises, pour loger les barres servant à porter la table. 
${}^{28}Tu feras des barres en bois d’acacia et tu les plaqueras d’or ; elles serviront à porter la table. 
${}^{29}Tu feras des plats, des gobelets, des aiguières et des timbales pour les libations. Tu les feras en or pur. 
${}^{30}Et sur la table, tu placeras face à moi le pain qui m’est destiné, perpétuellement.
${}^{31}« Puis tu feras un chandelier en or pur. Le chandelier sera forgé : base, tige, coupes, boutons et fleurs feront corps avec lui. 
${}^{32}Six branches s’en détacheront sur les côtés : trois d’un côté et trois de l’autre. 
${}^{33}Sur une branche, trois coupes en forme d’amande avec bouton et fleur et, sur une autre branche, trois coupes en forme d’amande avec bouton et fleur ; de même pour les six branches sortant du chandelier. 
${}^{34}Le chandelier lui-même portera quatre coupes en forme d’amande avec boutons et fleurs : 
${}^{35}un bouton sous les deux premières branches issues du chandelier, un bouton sous les deux suivantes et un bouton sous les deux dernières ; ainsi donc pour les six branches qui sortent du chandelier. 
${}^{36}Boutons et branches feront corps avec le chandelier qui sera tout entier forgé d’une seule pièce, en or pur. 
${}^{37}Ensuite, tu lui feras sept lampes. On allumera les lampes de manière à éclairer l’espace qui est devant lui. 
${}^{38}Ses pincettes et ses porte-lampes seront en or pur. 
${}^{39}Il te faudra un lingot d’or pur pour le chandelier et tous ses accessoires. 
${}^{40}Regarde et exécute selon le modèle qui t’a été montré sur la montagne.
      
         
      \bchapter{}
      \begin{verse}
${}^{1}« Pour construire la Demeure, tu feras dix tentures de lin retors, pourpre violette, pourpre rouge et cramoisi éclatant ; tu y broderas des kéroubim : ce sera une œuvre d’artiste. 
${}^{2}Chaque tenture mesurera vingt-huit coudées de long et quatre de large. Toutes les tentures auront les mêmes dimensions. 
${}^{3}Cinq tentures seront assemblées l’une à l’autre, et les cinq autres également. 
${}^{4}Tu feras des lacets de pourpre violette au bord de la première tenture, à l’extrémité de l’assemblage, et tu feras de même au bord de la dernière tenture du deuxième assemblage. 
${}^{5}Tu mettras cinquante lacets à la première tenture et cinquante lacets à l’extrémité de la tenture du deuxième assemblage, les lacets s’attachant l’un à l’autre. 
${}^{6}Tu feras cinquante agrafes en or, tu assembleras les tentures l’une à l’autre par les agrafes. Ainsi, la Demeure sera d’un seul tenant.
${}^{7}Ensuite, pour former une tente au-dessus de la Demeure, tu feras onze tentures en poil de chèvre. 
${}^{8}Chaque tenture mesurera trente coudées de long et quatre coudées de large. Les onze tentures auront les mêmes dimensions. 
${}^{9}Tu assembleras cinq tentures à part, puis six tentures à part, et tu replieras la sixième tenture sur le devant de la tente. 
${}^{10}Tu feras cinquante lacets au bord d’une première tenture, la dernière de l’assemblage, et cinquante lacets au bord de la même tenture du deuxième assemblage. 
${}^{11}Tu feras cinquante agrafes de bronze, tu introduiras les agrafes dans les lacets pour assembler la tente d’un seul tenant. 
${}^{12}De ce qui retombe en surplus des tentures, une moitié de la tenture en surplus retombera sur l’arrière de la Demeure. 
${}^{13}Et, dans le sens de la longueur des tentures, une coudée en surplus retombera, de part et d’autre, sur les côtés de la Demeure pour la couvrir.
${}^{14}Enfin tu feras pour la tente une couverture en peaux de béliers teintes en rouge, et une autre en cuir fin à mettre par-dessus.
${}^{15}« Puis tu feras pour la Demeure des cadres en bois d’acacia, dressés debout. 
${}^{16}Ils mesureront dix coudées de long et une coudée et demie de large. 
${}^{17}Un cadre sera assemblé par deux tenons jumelés : ainsi feras-tu pour tous les cadres de la Demeure. 
${}^{18}Tu disposeras les cadres pour la Demeure comme suit : vingt en direction du Néguev, au sud ; 
${}^{19}et tu feras quarante socles en argent sous les vingt cadres : deux socles sous un cadre pour ses deux tenons, puis deux socles sous un autre cadre pour ses deux tenons. 
${}^{20}Pour le deuxième côté de la Demeure, tu disposeras, en direction du nord, vingt cadres 
${}^{21}avec leurs quarante socles en argent : deux socles sous un cadre et deux socles sous un autre cadre. 
${}^{22}Et pour le fond de la Demeure, vers l’ouest, tu feras six cadres ; 
${}^{23}tu feras aussi deux cadres comme contreforts de la Demeure, au fond ; 
${}^{24}ils seront jumelés à leur base et le seront également au sommet, à la hauteur du premier anneau : ainsi en sera-t-il pour eux deux, ils seront comme deux contreforts. 
${}^{25}Il y aura donc huit cadres, avec leurs socles en argent, soit seize socles : deux socles sous un cadre et deux socles sous un autre cadre.
${}^{26}Puis tu feras des traverses en bois d’acacia : cinq pour les cadres du premier côté de la Demeure, 
${}^{27}cinq pour les cadres du deuxième côté de la Demeure, cinq pour les cadres qui forment le fond de la Demeure vers l’ouest ; 
${}^{28}tu feras aussi la traverse médiane, à mi-hauteur des cadres, traversant la Demeure d’un bout à l’autre. 
${}^{29}Les cadres, tu les plaqueras d’or, tu feras en or leurs anneaux pour loger les traverses, et les traverses, tu les plaqueras d’or. 
${}^{30}Tu dresseras la Demeure d’après la règle qui t’a été montrée sur la montagne.
${}^{31}« Puis tu feras un rideau de pourpre violette, pourpre rouge, cramoisi éclatant et lin retors ; ce sera une œuvre d’artiste : on y brodera des kéroubim. 
${}^{32}Tu le fixeras à quatre colonnes en acacia et tu les plaqueras d’or, munies de crochets en or et posées sur quatre socles en argent. 
${}^{33}Tu fixeras le rideau sous les agrafes et là, derrière le rideau, tu introduiras l’arche du Témoignage. Le rideau marquera pour vous la séparation entre le Sanctuaire et le Saint des Saints. 
${}^{34}Tu placeras le propitiatoire sur l’arche du Témoignage dans le Saint des Saints. 
${}^{35}À l’extérieur du rideau, tu poseras la table et, en face d’elle, le chandelier : la table côté nord de la Demeure, et le chandelier côté sud.
${}^{36}« Enfin, pour l’entrée de la tente, tu feras un voile en pourpre violette, pourpre rouge, cramoisi éclatant et lin retors : ce sera une œuvre d’artisan brocheur. 
${}^{37}Tu feras, pour le voile, cinq colonnes en acacia et tu les plaqueras d’or, tu les muniras de crochets en or, et tu couleras pour elles cinq socles en bronze.
      
         
      \bchapter{}
      \begin{verse}
${}^{1}« Puis tu feras l’autel en bois d’acacia. L’autel aura cinq coudées de long, cinq coudées de large – sa base sera donc carrée – et trois coudées de haut. 
${}^{2}Tu feras des cornes aux quatre angles de l’autel, et ses cornes feront corps avec lui. Tu le plaqueras de bronze. 
${}^{3}Tu feras les vases pour recueillir les cendres grasses, les pelles, les bols pour l’aspersion, les fourchettes et les brûle-parfums : tous ces accessoires, tu les feras en bronze. 
${}^{4}Tu lui feras une grille de bronze en forme de filet, munie de quatre anneaux de bronze aux quatre extrémités. 
${}^{5}Tu la mettras sous la bordure de l’autel, en bas ; la grille sera à mi-hauteur de l’autel. 
${}^{6}Tu feras pour l’autel des barres en bois d’acacia et tu les plaqueras de bronze. 
${}^{7}On les engagera dans les anneaux et elles seront placées sur les deux côtés de l’autel pour le porter. 
${}^{8}Tu le feras creux, en planches. Comme il te fut montré sur la montagne, c’est ainsi que l’on fera.
      
         
${}^{9}« Tu feras le parvis de la Demeure. Du côté du Néguev, au sud, le parvis aura des toiles en lin retors, sur une longueur de cent coudées pour un seul côté. 
${}^{10}Ses vingt colonnes et leurs vingt socles seront en bronze ; les crochets des colonnes et leurs tringles, en argent. 
${}^{11}De même, du côté nord, sur toute sa longueur, le parvis aura des toiles longues de cent coudées, vingt colonnes et leurs vingt socles en bronze ; les crochets des colonnes et leurs tringles seront en argent. 
${}^{12}En largeur, du côté ouest, le parvis aura des toiles sur cinquante coudées, avec leurs dix colonnes et leurs dix socles. 
${}^{13}La largeur du parvis du côté de l’est, vers le levant, sera de cinquante coudées ; 
${}^{14}il y aura quinze coudées de toiles sur une aile, avec leurs trois colonnes et leurs trois socles, 
${}^{15}et, sur la deuxième aile, quinze coudées de toiles, avec leurs trois colonnes et leurs trois socles. 
${}^{16}Pour la porte du parvis, il y aura un voile de vingt coudées, en pourpre violette, pourpre rouge, cramoisi éclatant et lin retors – œuvre d’artisan brocheur –, avec leurs quatre colonnes et leurs quatre socles. 
${}^{17}Toutes les colonnes du parvis seront réunies par des tringles en argent ; leurs crochets seront en argent et leurs socles en bronze. 
${}^{18}La longueur du parvis sera de cent coudées, sa largeur de cinquante, et sa hauteur de cinq – les socles seront en bronze.
${}^{19}Tous les accessoires utilisés pour le service de la Demeure, tous ses piquets et les piquets du parvis seront en bronze.
${}^{20}« Tu ordonneras également aux fils d’Israël de te procurer, pour le luminaire, de l’huile d’olive limpide et vierge, pour que, perpétuellement, monte la flamme d’une lampe. 
${}^{21}C’est dans la tente de la Rencontre, à l’extérieur du rideau qui abrite le Témoignage, que la disposeront Aaron et ses fils, pour qu’elle soit du soir au matin devant le Seigneur : c’est un décret perpétuel, de génération en génération, pour les fils d’Israël.
      
         
      \bchapter{}
      \begin{verse}
${}^{1}« Et toi, fais approcher, du milieu des fils d’Israël, ton frère Aaron avec ses fils, afin qu’il exerce pour moi le sacerdoce. Il y avait donc : Aaron et ses fils Nadab et Abihou, Éléazar et Itamar. 
${}^{2}Tu feras pour Aaron ton frère des vêtements sacrés, en signe de gloire et de majesté. 
${}^{3}Toi, tu t’adresseras à tous les artisans habiles, ceux que j’ai remplis d’un esprit de sagesse : ils feront les vêtements d’Aaron, afin que celui-ci soit consacré et qu’il exerce pour moi le sacerdoce. 
${}^{4}Voici les vêtements qu’ils feront : un pectoral, un éphod, un manteau, une tunique brodée, un turban et une ceinture. Ils feront donc des vêtements sacrés pour Aaron ton frère – et pour ses fils – afin qu’il exerce pour moi le sacerdoce. 
${}^{5}Ils utiliseront l’or, la pourpre violette, la pourpre rouge, le cramoisi éclatant et le lin.
${}^{6}Ils feront l’éphod en or, pourpre violette et pourpre rouge, cramoisi éclatant et lin retors. Ce sera une œuvre d’artiste. 
${}^{7}L’éphod sera fixé aux deux extrémités par deux brides. 
${}^{8}L’écharpe portée au-dessus de l’éphod et faisant corps avec lui sera travaillée de la même manière : en or, pourpre violette, pourpre rouge, cramoisi éclatant, lin retors. 
${}^{9}Tu prendras ensuite deux pierres de cornaline et tu y graveras les noms des fils d’Israël : 
${}^{10}six sur la première pierre, six sur la seconde, selon l’ordre de naissance. 
${}^{11}On taillera les deux pierres et tu les graveras aux noms des fils d’Israël, comme on grave un sceau ; et tu les enchâsseras dans des chatons en or. 
${}^{12}Tu placeras les deux pierres sur les brides de l’éphod. Ces pierres seront un mémorial pour les fils d’Israël. Ainsi, devant le Seigneur, Aaron portera leurs noms sur ses deux épaules, en mémorial. 
${}^{13}Tu feras des chatons en or 
${}^{14}et deux chaînettes torsadées, en or pur, que tu placeras sur les chatons.
${}^{15}Ensuite, tu feras le pectoral du jugement. Ce sera une œuvre d’artiste. Tu le feras de la même manière que l’éphod, en or, pourpre violette, pourpre rouge, cramoisi éclatant et lin retors. 
${}^{16}Il sera carré. On le doublera. Il aura un empan de côté. 
${}^{17}Tu le garniras de quatre rangées de pierres : la première, de sardoine, topaze et émeraude ; 
${}^{18}la deuxième, d’escarboucle, saphir et jaspe ; 
${}^{19}la troisième, de béryl, agate, et améthyste ; 
${}^{20}et la quatrième, de chrysolithe, cornaline et onyx. Elles seront serties dans l’or. 
${}^{21}Les pierres seront aux noms des fils d’Israël ; comme leurs noms, elles seront douze, gravées à la manière d’un sceau ; chacune portera le nom de l’une des douze tribus. 
${}^{22}Tu feras au pectoral des chaînettes tressées et torsadées, en or pur. 
${}^{23}Tu feras au pectoral deux anneaux d’or et tu fixeras les deux anneaux à deux extrémités du pectoral. 
${}^{24}Tu fixeras les deux torsades d’or aux deux anneaux, aux extrémités du pectoral, 
${}^{25}tandis que tu fixeras les deux extrémités des deux torsades aux deux chatons ; tu les fixeras aux brides de l’éphod par-devant. 
${}^{26}Tu feras deux anneaux d’or et tu les mettras à deux des extrémités du pectoral, du côté tourné vers l’éphod, en dedans. 
${}^{27}Tu feras deux anneaux d’or et tu les fixeras aux deux brides de l’éphod, à leur base, par-devant, près de leur point d’attache, au-dessus de l’écharpe de l’éphod. 
${}^{28}On reliera le pectoral par ses anneaux aux anneaux de l’éphod avec un cordon de pourpre violette : le pectoral sera sur l’écharpe de l’éphod et ne pourra pas s’en détacher. 
${}^{29}Ainsi, quand Aaron entrera dans le sanctuaire, il portera sur son cœur, avec le pectoral du jugement, les noms des fils d’Israël, en mémorial devant le Seigneur, perpétuellement. 
${}^{30}Tu placeras dans le pectoral du jugement les Ourim et les Toummim. Ces objets seront sur le cœur d’Aaron quand il se présentera devant le Seigneur. Aaron portera sur son cœur le jugement des fils d’Israël, devant le Seigneur, perpétuellement.
${}^{31}Puis tu feras le manteau de l’éphod, tout entier de pourpre violette. 
${}^{32}Il aura en son milieu une ouverture pour la tête, bordée comme celle d’une cuirasse, et donc indéchirable. Ce sera l’œuvre d’un ouvrier tisserand. 
${}^{33}Sur les pans du manteau, tout autour, tu feras des grenades de pourpre violette, de pourpre rouge et de cramoisi éclatant, alternant avec des clochettes d’or, tout autour : 
${}^{34}clochette d’or et grenade, clochette d’or et grenade, sur les pans du manteau, tout autour. 
${}^{35}Aaron portera ce manteau quand il officiera. On entendra le son des clochettes, quand il entrera dans le sanctuaire, devant le Seigneur, ou qu’il en sortira. Et ainsi, il ne mourra pas.
${}^{36}Puis tu feras un fleuron d’or pur. Comme sur un sceau, tu y graveras l’inscription : “Consacré au Seigneur”. 
${}^{37}Tu attacheras le fleuron à un cordon de pourpre violette et tu le placeras sur le devant du turban. 
${}^{38}Il se trouvera sur le front d’Aaron, et Aaron portera ainsi les fautes commises envers les choses saintes que consacreront les fils d’Israël, quelles que soient les choses saintes qu’ils donnent. Le fleuron restera toujours sur son front, pour que ces dons trouvent grâce devant le Seigneur.
${}^{39}Enfin, tu broderas une tunique de lin, tu feras un turban de lin et une ceinture. Ce sera l’œuvre d’un artisan brocheur. 
${}^{40}Pour les fils d’Aaron, tu feras des tuniques, des ceintures et des tiares, en signe de gloire et de majesté. 
${}^{41}Tu en revêtiras ton frère Aaron ainsi que ses fils ; tu leur donneras l’onction, tu leur conféreras l’investiture, tu les consacreras, et ils exerceront pour moi le sacerdoce. 
${}^{42}Fais-leur aussi des caleçons de lin pour couvrir leur nudité, des reins jusqu’aux cuisses. 
${}^{43}Aaron et ses fils les porteront quand ils entreront dans la tente de la Rencontre ou s’approcheront de l’autel pour officier dans le sanctuaire ; ainsi, ils ne se chargeront pas d’une faute qui entraînerait leur mort. C’est là un décret perpétuel pour Aaron et pour sa descendance.
      
         
      \bchapter{}
      \begin{verse}
${}^{1}« Et voici le rite que tu accompliras pour les consacrer, afin qu’ils exercent pour moi le sacerdoce : prends un taureau et deux béliers sans défaut, 
${}^{2}ainsi que du pain sans levain, des gâteaux sans levain pétris à l’huile et des galettes sans levain frottées d’huile. Tu les feras avec de la farine de blé. 
${}^{3}Tu les mettras dans une corbeille et tu les présenteras en même temps que le taureau et les deux béliers.
${}^{4}Tu feras approcher Aaron et ses fils à l’entrée de la tente de la Rencontre, et tu leur feras prendre un bain. 
${}^{5}Ensuite, tu prendras les vêtements et tu revêtiras Aaron de la tunique, du manteau, de l’éphod et du pectoral. Tu le draperas dans l’écharpe de l’éphod 
${}^{6}et tu poseras le turban sur sa tête. Sur le turban, tu mettras le saint diadème. 
${}^{7}Puis tu prendras l’huile de l’onction : tu lui en verseras sur la tête et tu lui donneras l’onction. 
${}^{8}Alors tu feras approcher les fils d’Aaron, tu les revêtiras de tuniques. 
${}^{9}Tu leur mettras des ceintures, tu les coifferas de tiares, et le sacerdoce leur appartiendra en vertu d’un décret perpétuel. Tu conféreras l’investiture à Aaron et à ses fils.
${}^{10}Tu feras approcher le taureau devant la tente de la Rencontre ; Aaron et ses fils imposeront la main sur sa tête, 
${}^{11}et tu l’immoleras devant le Seigneur, à l’entrée de la tente de la Rencontre. 
${}^{12}Tu prendras le sang du taureau et tu en mettras avec ton doigt sur les cornes de l’autel. Puis tu répandras le sang à la base de l’autel. 
${}^{13}Tu prendras toute la graisse qui enveloppe les entrailles ainsi que le lobe du foie, les deux rognons et la graisse qui les entoure, et tu les feras fumer sur l’autel. 
${}^{14}Mais tu brûleras hors du camp la chair du taureau, la peau et les excréments. C’est un sacrifice pour la faute.
${}^{15}Tu prendras le premier bélier. Aaron et ses fils imposeront la main sur sa tête. 
${}^{16}Puis tu l’immoleras, tu prendras son sang et tu en aspergeras chaque côté de l’autel. 
${}^{17}Tu couperas le bélier en quartiers, tu laveras ses entrailles et ses pattes, et tu les poseras sur les quartiers et la tête. 
${}^{18}Tu feras fumer entièrement le bélier sur l’autel. C’est un holocauste pour le Seigneur, une nourriture offerte, en agréable odeur pour le Seigneur.
${}^{19}Puis tu prendras le second bélier : Aaron et ses fils imposeront la main sur sa tête. 
${}^{20}Tu immoleras le bélier, tu prendras de son sang et tu en marqueras le lobe de l’oreille droite d’Aaron, le lobe de l’oreille droite de ses fils, le pouce de leur main droite et le gros orteil de leur pied droit. Et avec le sang tu aspergeras chaque côté de l’autel. 
${}^{21}Tu prendras du sang sur l’autel et de l’huile de l’onction, et tu feras l’aspersion sur Aaron et sur ses vêtements, sur ses fils et sur leurs vêtements également : ainsi seront consacrés Aaron et ses vêtements, ses fils et leurs vêtements. 
${}^{22}Tu prendras la graisse du bélier, la queue, la graisse qui recouvre les entrailles, le lobe du foie, les deux rognons et la graisse qui les recouvre, ainsi que la cuisse droite, car c’est un bélier d’investiture. 
${}^{23}Tu prendras une couronne de pain, un gâteau à l’huile et une galette, dans la corbeille des pains sans levain placée devant le Seigneur. 
${}^{24}Tu poseras le tout sur les paumes d’Aaron et de ses fils, et tu le leur feras présenter avec le geste d’élévation devant le Seigneur. 
${}^{25}Ensuite, tu le reprendras de leurs mains et tu le feras fumer sur l’autel de l’holocauste ; c’est une nourriture offerte, en agréable odeur pour le Seigneur. 
${}^{26}Tu prendras la poitrine du bélier d’investiture d’Aaron et tu feras avec elle le geste d’élévation devant le Seigneur : cette part sera la tienne. 
${}^{27}Tu consacreras la poitrine présentée et la cuisse prélevée du bélier d’investiture d’Aaron et de ses fils. 
${}^{28}Ce sera, selon un décret perpétuel, ce qu’Aaron et ses fils recevront des fils d’Israël. Car c’est une contribution des fils d’Israël – et cela le restera –, une contribution prise sur leurs sacrifices de paix, une contribution pour le Seigneur.
${}^{29}Les vêtements sacrés d’Aaron passeront après lui à ses fils qui les porteront pour leur onction et leur investiture. 
${}^{30}Pendant sept jours, le fils d’Aaron qui lui succédera comme prêtre portera ces vêtements. Il entrera dans la tente de la Rencontre pour officier dans le sanctuaire.
${}^{31}Tu prendras le bélier d’investiture et tu feras cuire sa chair dans un lieu saint. 
${}^{32}Aaron mangera – et ses fils avec lui – la chair du bélier et le pain qui est dans la corbeille, à l’entrée de la tente de la Rencontre. 
${}^{33}Ils mangeront ce qui a servi au rite de l’expiation, lors de leur investiture et de leur consécration. Aucun profane n’en mangera, car c’est chose sainte. 
${}^{34}Au matin, s’il reste de la viande et du pain, tu brûleras ce reste au feu. On ne le mangera pas, car c’est chose sainte. 
${}^{35}Tu feras donc ainsi pour Aaron et ses fils, comme je te l’ai ordonné. Pendant sept jours, tu accompliras le rite d’investiture.
${}^{36}« Chaque jour, tu apprêteras pour le rite d’expiation un taureau, en vue du sacrifice pour la faute ; puis tu accompliras le rite d’expiation sur l’autel en sacrifice pour la faute, et tu lui feras une onction pour le consacrer. 
${}^{37}Pendant sept jours, tu accompliras ce rite d’expiation sur l’autel et tu le consacreras ; ainsi, l’autel sera très saint, et tout ce qui touche à l’autel sera sanctifié.
${}^{38}Voici ce que tu mettras sur l’autel : des agneaux de l’année, deux par jour, perpétuellement. 
${}^{39}Le premier agneau, tu le mettras le matin ; et le second agneau, au coucher du soleil. 
${}^{40}Avec le premier agneau, tu mettras dix livres de fleur de farine, pétrie dans un quart de setier d’huile vierge ; et, de plus, une libation d’un quart de setier de vin. 
${}^{41}Avec le second agneau, que tu mettras au coucher du soleil, tu feras la même offrande que le matin, et la même libation : ce sera une nourriture offerte, en agréable odeur au Seigneur. 
${}^{42}Tel sera l’holocauste perpétuel que vous ferez d’âge en âge, à l’entrée de la tente de la Rencontre, en présence du Seigneur ; ce sera pour vous le lieu de rencontre, où je te parlerai.
${}^{43}Là, je me laisserai rencontrer par les fils d’Israël et ce lieu sera consacré par ma gloire. 
${}^{44}Je consacrerai la tente de la Rencontre ainsi que l’autel. Je consacrerai Aaron et ses fils, afin qu’ils exercent pour moi le sacerdoce. 
${}^{45}Je demeurerai au milieu des fils d’Israël, et je serai leur Dieu. 
${}^{46}Ils sauront que je suis le Seigneur, leur Dieu, qui les a fait sortir du pays d’Égypte pour demeurer au milieu d’eux. Je suis le Seigneur, leur Dieu.
      
         
      \bchapter{}
      \begin{verse}
${}^{1}« Tu feras encore un autel en bois d’acacia pour brûler de l’encens. 
${}^{2}Il aura une coudée de long, une coudée de large – sa base sera donc carrée – et de deux coudées et demie de haut. Ses cornes feront corps avec lui. 
${}^{3}Tu le plaqueras d’or pur : le dessus, les parois tout autour et les cornes ; tu l’entoureras d’une moulure en or. 
${}^{4}Sous la moulure, sur les deux côtés, tu placeras des anneaux d’or pour loger les barres servant à le porter. 
${}^{5}Tu feras les barres en acacia et tu les plaqueras d’or. 
${}^{6}Tu placeras l’autel devant le rideau qui abrite l’arche du Témoignage, au lieu où tu pourras me rencontrer. 
${}^{7}Quand, chaque matin, Aaron viendra entretenir les lampes, il y brûlera de l’encens aromatique. 
${}^{8}Et quand, au coucher du soleil, il viendra allumer les lampes, il y brûlera à nouveau de l’encens. De génération en génération, l’encens montera perpétuellement devant le Seigneur. 
${}^{9}Sur cet autel, vous n’offrirez pas d’encens profane, ni d’holocauste, ni d’offrande de céréales ; vous n’y verserez pas de libation. 
${}^{10}Aaron accomplira le rite d’expiation sur les cornes de l’autel, une fois par an. Il le fera avec le sang du sacrifice pour la faute, une fois par an, lors de la fête du Grand Pardon, de génération en génération. Ce sera une chose très sainte pour le Seigneur. »
      
         
${}^{11}Le Seigneur parla à Moïse. Il dit : 
${}^{12}« Quand tu dénombreras les fils d’Israël pour le recensement, chacun d’eux donnera au Seigneur le prix de la rançon pour sa vie : ainsi, aucun fléau ne les frappera lors du recensement. 
${}^{13}Voici ce que donnera tout homme soumis au recensement : un demi-sicle, selon le sicle du sanctuaire à vingt guéras par sicle, comme contribution pour le Seigneur. 
${}^{14}Tout homme de vingt ans et plus qui viendra se faire recenser s’acquittera de la contribution pour le Seigneur. 
${}^{15}Pour la payer, en rançon pour sa vie, le riche ne versera pas plus d’un demi-sicle et l’indigent, pas moins. 
${}^{16}Tu recevras, des fils d’Israël, l’argent de la rançon, et tu le donneras pour le service de la tente de la Rencontre. Pour les fils d’Israël, ce sera, en présence du Seigneur, un mémorial de la rançon pour vos vies. »
${}^{17}Le Seigneur parla à Moïse. Il dit : 
${}^{18}« Pour les ablutions, tu feras une cuve en bronze sur un support en bronze. Tu placeras la cuve entre la tente de la Rencontre et l’autel, et tu y verseras de l’eau. 
${}^{19}Aaron et ses fils s’y laveront les mains et les pieds. 
${}^{20}Quand ils entreront dans la tente de la Rencontre, ils se laveront avec l’eau, et ainsi ils ne mourront pas ; quand ils s’approcheront de l’autel pour officier, faire fumer une nourriture offerte pour le Seigneur, 
${}^{21}ils se laveront les mains et les pieds, et ainsi ils ne mourront pas. C’est là un décret perpétuel pour Aaron et sa descendance, de génération en génération. »
${}^{22}Le Seigneur parla à Moïse. Il dit : 
${}^{23}« Procure-toi aussi du baume de première qualité ; de la myrrhe fluide, cinq cents sicles ; du cinnamome aromatique, la moitié, soit deux cent cinquante ; du roseau aromatique, deux cent cinquante ; 
${}^{24}de la casse, cinq cents sicles – en sicles du sanctuaire –, et un setier d’huile d’olive. 
${}^{25}Tu en feras une huile d’onction sainte, un mélange parfumé, œuvre de parfumeur : ce sera l’huile d’onction sainte. 
${}^{26}Avec ce mélange, tu feras une onction sur la tente de la Rencontre, l’arche du Témoignage, 
${}^{27}la table et les accessoires, le chandelier et ses accessoires, l’autel de l’encens, 
${}^{28}l’autel de l’holocauste et ses accessoires, la cuve et son support. 
${}^{29}Tu les consacreras et ils seront très saints ; tout ce qui les touchera sera sanctifié. 
${}^{30}Tu donneras l’onction à Aaron et à ses fils, et tu les consacreras afin qu’ils exercent pour moi le sacerdoce. 
${}^{31}Puis tu t’adresseras aux fils d’Israël et tu leur diras : “Ceci est, pour moi, l’huile d’onction sainte, de génération en génération. 
${}^{32}On n’en répandra sur le corps d’aucune autre personne ; vous n’imiterez pas sa recette, car cette huile est sainte et elle restera sainte pour vous. 
${}^{33}Celui qui copiera ce mélange et en mettra sur un profane sera retranché de sa parenté.” »
${}^{34}Le Seigneur dit à Moïse : « Procure-toi des aromates : storax, ambre, galbanum aromatique et encens pur, en parties égales. 
${}^{35}Tu en feras un encens parfumé qui soit salé, pur et saint. C’est une œuvre de parfumeur. 
${}^{36}Tu en réduiras une partie en poudre que tu mettras devant l’arche du Témoignage, dans la tente de la Rencontre ; là je te laisserai me rencontrer. Pour vous, ce sera chose très sainte. 
${}^{37}L’encens composé selon cette recette, vous ne l’utiliserez pas pour votre propre usage : il sera saint, réservé au Seigneur. 
${}^{38}Celui qui en fera une imitation pour jouir de son odeur sera retranché de sa parenté. »
      
         
      \bchapter{}
      \begin{verse}
${}^{1}Le Seigneur parla à Moïse. Il dit : 
${}^{2}« Vois : j’ai appelé par son nom Beçalel, fils d’Ouri, fils de Hour, de la tribu de Juda. 
${}^{3}Je l’ai rempli de l’esprit de Dieu pour qu’il possède la sagesse, l’intelligence, la connaissance et le savoir-faire pour toutes sortes de travaux : 
${}^{4}la création artistique, le travail de l’or, de l’argent, du bronze, 
${}^{5}la taille des pierres précieuses, la sculpture sur bois et toutes sortes de travaux. 
${}^{6}Et c’est moi qui lui donne comme adjoint Oholiab, fils d’Ahisamak, de la tribu de Dane. C’est moi qui donne la sagesse à tout artisan habile, pour qu’il fasse tout ce que je t’ai ordonné, c’est-à-dire : 
${}^{7}la tente de la Rencontre, l’arche du Témoignage, le propitiatoire qui la couvre, tous les accessoires de la Tente, 
${}^{8}la table et ses accessoires, le chandelier d’or pur et tous ses accessoires, l’autel de l’encens, 
${}^{9}l’autel de l’holocauste et tous ses accessoires, la cuve et son support, 
${}^{10}les vêtements liturgiques, les vêtements sacrés pour le prêtre Aaron, les vêtements que porteront ses fils pour exercer le sacerdoce, 
${}^{11}l’huile de l’onction, l’encens aromatique pour le sanctuaire. Ils feront exactement comme je te l’ai ordonné. »
      
         
${}^{12}Le Seigneur dit à Moïse : 
${}^{13}« Toi, tu parleras ainsi aux fils d’Israël : Surtout, vous observerez mes sabbats, car c’est un signe entre moi et vous, de génération en génération, pour qu’on reconnaisse que je suis le Seigneur, celui qui vous sanctifie. 
${}^{14}Vous observerez le sabbat, car il est saint pour vous. Qui le profanera sera mis à mort : Oui, quiconque fera, en ce jour, quelque ouvrage, cette personne-là sera retranchée du milieu de sa parenté. 
${}^{15}Pendant six jours, on travaillera, mais, le septième jour, c’est un sabbat, un sabbat solennel consacré au Seigneur. Quiconque travaillera le jour du sabbat sera mis à mort. 
${}^{16}Les fils d’Israël observeront le sabbat en le célébrant de génération en génération : c’est une alliance éternelle. 
${}^{17}À jamais, il est un signe entre moi et les fils d’Israël, car le Seigneur a fait le ciel et la terre en six jours mais, le septième jour, il a chômé et repris souffle. »
${}^{18}Quand le Seigneur eut fini de parler avec Moïse sur le mont Sinaï, il lui donna les deux tables du Témoignage, les tables de pierre écrites du doigt de Dieu.
      <h2 class="intertitle" id="d85e24709">5. Rupture et renouvellement de l’Alliance (32 – 34)</h2>
      
         
      \bchapter{}
      \begin{verse}
${}^{1}Le peuple vit que Moïse tardait à descendre de la montagne. Il se rassembla contre Aaron et lui dit : « Debout ! Fais-nous des dieux qui marchent devant nous. Car ce Moïse, l’homme qui nous a fait monter du pays d’Égypte, nous ne savons pas ce qui lui est arrivé. » 
${}^{2}Aaron leur répondit : « Enlevez les boucles d’or qui sont aux oreilles de vos femmes, de vos fils, de vos filles, et apportez-les moi. » 
${}^{3}Tout le peuple se dépouilla des boucles d’or qu’ils avaient aux oreilles et ils les apportèrent à Aaron. 
${}^{4}Il reçut l’or de leurs mains, le façonna au burin et en fit un veau en métal fondu. Ils dirent alors : « Israël, voici tes dieux, qui t’ont fait monter du pays d’Égypte. » 
${}^{5}Ce que voyant, Aaron bâtit un autel en face du veau en métal fondu et il proclama : « Demain, fête pour le Seigneur ! »
${}^{6}Le lendemain, levés de bon matin, ils offrirent des holocaustes et présentèrent des sacrifices de paix ; le peuple s’assit pour manger et boire ; puis il se leva pour se divertir.
${}^{7}Le Seigneur parla à Moïse : « Va, descends, car ton peuple s’est corrompu, lui que tu as fait monter du pays d’Égypte. 
${}^{8}Ils n’auront pas mis longtemps à s’écarter du chemin que je leur avais ordonné de suivre\\ ! Ils se sont fait un veau en métal fondu et se sont prosternés devant lui. Ils lui ont offert des sacrifices en proclamant : “Israël, voici tes dieux\\, qui t’ont fait monter du pays d’Égypte.” » 
${}^{9}Le Seigneur dit encore à Moïse : « Je vois que ce peuple est un peuple à la nuque raide. 
${}^{10}Maintenant, laisse-moi faire ; ma colère va s’enflammer contre eux et je vais les exterminer ! Mais, de toi, je ferai une grande nation. »
${}^{11}Moïse apaisa le visage du Seigneur son Dieu en disant : « Pourquoi, Seigneur, ta colère s’enflammerait-elle contre ton peuple, que tu as fait sortir du pays d’Égypte par ta grande force et ta main puissante ? 
${}^{12}Pourquoi donner aux Égyptiens l’occasion de dire : “C’est par méchanceté qu’il les a fait sortir ; il voulait les tuer dans les montagnes et les exterminer à la surface de la terre” ? Reviens de l’ardeur de ta colère, renonce au mal que tu veux faire à ton peuple. 
${}^{13}Souviens-toi de tes serviteurs, Abraham, Isaac et Israël, à qui tu as juré par toi-même : “Je multiplierai votre descendance comme\\les étoiles du ciel ; je donnerai, comme je l’ai dit, tout ce pays à vos descendants, et il sera pour toujours leur héritage\\.” » 
${}^{14}Le Seigneur renonça au mal qu’il avait voulu faire à son peuple.
${}^{15}Moïse redescendit\\de la montagne. Il portait les deux tables du Témoignage ; ces tables étaient écrites sur les deux faces\\ ; 
${}^{16} elles étaient l’œuvre de Dieu, et l’écriture, c’était l’écriture de Dieu, gravée sur ces tables.
${}^{17}Josué entendit le bruit et le tumulte du peuple et dit à Moïse : « Bruit de bataille dans le camp. » 
${}^{18} Moïse répliqua : « Ces bruits, ce ne sont pas des chants de victoire ni de défaite ; ce que j’entends, ce sont des cantiques qui se répondent. »
${}^{19}Comme il approchait du camp, il aperçut le veau et les danses. Il s’enflamma de colère, il jeta les tables qu’il portait, et les brisa au bas de la montagne. 
${}^{20} Il se saisit du veau qu’ils avaient fait, le brûla, le réduisit en poussière\\, qu’il répandit à la surface de l’eau. Et cette eau, il la fit boire aux fils d’Israël.
${}^{21}Moïse dit à Aaron : « Qu’est-ce que ce peuple t’avait donc fait, pour que tu l’aies entraîné dans un si grand péché ? » 
${}^{22}Aaron répondit : « Que mon seigneur ne s’enflamme pas de colère ! Tu sais bien que ce peuple est porté au mal\\ ! 
${}^{23}C’est eux qui m’ont dit : “Fais-nous des dieux qui marchent devant nous. Car ce Moïse, l’homme qui nous a fait monter du pays d’Égypte, nous ne savons pas ce qui lui est arrivé.” 
${}^{24}Je leur ai dit : “Ceux d’entre vous qui ont de l’or, qu’ils s’en dépouillent.” Ils me l’ont donné, je l’ai jeté au feu, et il en est sorti ce veau. »
${}^{25}Moïse vit que le peuple était débridé, car Aaron leur avait laissé la bride sur le cou, les exposant aux moqueries de leurs adversaires. 
${}^{26}Alors, Moïse vint à la porte du camp et dit : « À moi, les partisans du Seigneur ! » Et tous les fils de Lévi se groupèrent autour de lui. 
${}^{27}Il leur dit : « Ainsi parle le Seigneur, le Dieu d’Israël : Mettez l’épée au côté, parcourez le camp de porte en porte, et tuez qui son frère, qui son ami, qui son proche ! » 
${}^{28}Les fils de Lévi exécutèrent la parole de Moïse et, parmi le peuple, il tomba, ce jour-là, environ trois mille hommes. 
${}^{29}Puis Moïse dit : « Recevez aujourd’hui l’investiture pour le Seigneur ; vous l’avez mérité, l’un au prix de son fils, l’autre au prix de son frère, et que le Seigneur vous accorde aujourd’hui sa bénédiction. »
${}^{30}Le lendemain, Moïse dit au peuple : « Vous avez commis un grand péché. Maintenant, je vais monter vers le Seigneur. Peut-être obtiendrai-je la rémission de votre péché. » 
${}^{31}Moïse retourna vers le Seigneur et lui dit : « Hélas ! Ce peuple a commis un grand péché : ils se sont fait des dieux en or. 
${}^{32}Ah, si\\tu voulais enlever leur péché ! Ou alors, efface-moi de ton livre, celui que tu as écrit. » 
${}^{33}Le Seigneur répondit à Moïse : « Celui que j’effacerai de mon livre, c’est celui qui a péché contre moi. 
${}^{34}Va donc, conduis le peuple vers le lieu que je t’ai indiqué, et mon ange ira devant toi. Le jour où j’interviendrai, je les punirai\\de leur péché. » 
${}^{35}Le Seigneur frappa le peuple, car ils avaient fait le veau, celui qu’avait fait Aaron.
      
         
      \bchapter{}
      \begin{verse}
${}^{1}Le Seigneur parla à Moïse : « Va, toi et le peuple que tu as fait monter du pays d’Égypte, monte d’ici vers la terre que j’ai juré de donner à Abraham, à Isaac et à Jacob, en leur disant : “C’est à ta descendance que je la donnerai.” 
${}^{2}J’enverrai devant toi un ange et je chasserai le Cananéen, l’Amorite, le Hittite, le Perizzite, le Hivvite et le Jébuséen. 
${}^{3}Monte vers une terre ruisselant de lait et de miel. Quant à moi, je ne monterai pas au milieu de toi, car tu es un peuple à la nuque raide, et je t’exterminerais en chemin. » 
${}^{4}À cette parole de malheur, le peuple prit le deuil, et personne ne porta plus ses habits de fête.
${}^{5}Le Seigneur dit à Moïse : « Répète aux fils d’Israël : Vous êtes un peuple à la nuque raide. Si je montais un seul instant au milieu de toi, je t’exterminerais. Maintenant débarrasse-toi de tes habits de fête, et je saurai comment te traiter. » 
${}^{6}À partir de la montagne de l’Horeb, les fils d’Israël se défirent de leurs habits de fête.
${}^{7}Moïse prenait la Tente et la plantait\\hors du camp, à bonne distance\\. On l’appelait : tente de la Rencontre, et quiconque voulait consulter le Seigneur devait sortir hors du camp pour gagner la tente de la Rencontre. 
${}^{8} Quand Moïse sortait pour aller à la Tente, tout le peuple se levait. Chacun se tenait à l’entrée de sa tente et suivait Moïse du regard jusqu’à ce qu’il soit entré. 
${}^{9} Au moment où Moïse entrait dans la Tente, la colonne de nuée descendait, se tenait à l’entrée de la Tente, et Dieu\\parlait avec Moïse. 
${}^{10} Tout le peuple voyait la colonne de nuée qui se tenait à l’entrée de la Tente, tous se levaient et se prosternaient, chacun devant sa tente. 
${}^{11} Le Seigneur parlait avec Moïse face à face, comme on parle d’homme à homme\\. Puis Moïse retournait dans le camp, mais son auxiliaire\\, le jeune Josué, fils de Noun, ne quittait pas l’intérieur de la Tente.
${}^{12}Moïse dit au Seigneur : « Vois ! Tu me dis toi-même : “Fais monter ce peuple”, mais tu ne m’as pas fait connaître celui que tu enverras avec moi. Pourtant, c’est toi qui avais dit : “Je te connais par ton nom ; tu as trouvé grâce à mes yeux.” 
${}^{13}Maintenant, si j’ai vraiment trouvé grâce à tes yeux, fais-moi connaître ton chemin, et je te connaîtrai, je saurai que j’ai trouvé grâce à tes yeux. Considère aussi que cette nation est ton peuple. » 
${}^{14}Le Seigneur dit : « J’irai en personne te donner le repos. » 
${}^{15}Et Moïse répondit : « Si tu ne viens pas en personne, ne nous fais pas monter d’ici. 
${}^{16}À quoi donc reconnaître que moi j’ai trouvé grâce à tes yeux – et ton peuple également ? N’est-ce pas au fait que tu marcheras avec nous ? Ainsi, moi et ton peuple, nous serons différents de tous les peuples de la terre. » 
${}^{17}Le Seigneur dit à Moïse : « Même ce que tu viens de dire, je le ferai, car tu as trouvé grâce à mes yeux et je te connais par ton nom. »
${}^{18}Moïse dit : « Je t’en prie, laisse-moi contempler ta gloire. » 
${}^{19}Le Seigneur dit : « Je vais passer devant toi avec toute ma splendeur, et je proclamerai devant toi mon nom qui est : Le Seigneur. Je fais grâce à qui je veux, je montre ma tendresse à qui je veux. » 
${}^{20}Il dit encore : « Tu ne pourras pas voir mon visage, car un être humain ne peut pas me voir et rester en vie. » 
${}^{21}Le Seigneur dit enfin : « Voici une place près de moi, tu te tiendras sur le rocher ; 
${}^{22}quand passera ma gloire, je te mettrai dans le creux du rocher et je t’abriterai de ma main jusqu’à ce que j’aie passé. 
${}^{23}Puis je retirerai ma main, et tu me verras de dos, mais mon visage, personne ne peut le voir. »
      
         
      \bchapter{}
      \begin{verse}
${}^{1}Le Seigneur dit à Moïse : « Taille deux tables de pierre, semblables aux premières : j’écrirai sur ces tables les paroles qui étaient sur les premières, celles que tu as brisées. 
${}^{2}Sois prêt pour demain et monte dès le matin sur la montagne du Sinaï. Tu te placeras là pour moi, au sommet de la montagne. 
${}^{3}Que personne ne monte avec toi ; que personne même ne paraisse sur toute la montagne. Que même le petit et le gros bétail ne soient pas conduits au pâturage devant cette montagne. » 
${}^{4}Moïse tailla deux tables de pierre semblables aux premières.<a class="anchor verset_lettre" id="bib_ex_34_4_b"/> Il se leva de bon matin, et il gravit la montagne du Sinaï comme le Seigneur le lui avait ordonné. Il emportait\\les deux tables de pierre.
      
         
${}^{5}Le Seigneur descendit dans la nuée et vint se placer là, auprès de Moïse. Il proclama\\son nom qui est : Le Seigneur\\.
${}^{6}Il passa devant Moïse et proclama : « Le Seigneur, Le Seigneur, Dieu tendre et miséricordieux\\, lent à la colère, plein d’amour\\et de vérité\\, 
${}^{7} qui garde sa fidélité jusqu’à la millième génération, supporte faute, transgression et péché, mais ne laisse rien passer, car il punit la faute des pères sur les fils et les petits-fils, jusqu’à la troisième et la quatrième génération\\. »
${}^{8}Aussitôt Moïse s’inclina jusqu’à terre et se prosterna. 
${}^{9} Il dit : « S’il est vrai, mon Seigneur\\, que j’ai trouvé grâce à tes yeux, daigne marcher au milieu de nous. Oui, c’est un peuple à la nuque raide ; mais tu pardonneras\\nos fautes et nos péchés, et tu feras de nous ton héritage. »
${}^{10}Le Seigneur dit : « Voici que je vais conclure une alliance. Devant tout ton peuple, je vais faire des merveilles qui n’ont été créées nulle part, dans aucune nation. Tout le peuple qui t’entoure verra l’œuvre du Seigneur, car je vais réaliser avec toi quelque chose d’extraordinaire. 
${}^{11}Observe donc bien ce que je t’ordonne aujourd’hui. Je vais chasser devant toi l’Amorite, le Cananéen, le Hittite, le Perizzite, le Hivvite et le Jébuséen. 
${}^{12}Garde-toi de conclure une alliance avec l’habitant du pays où tu vas entrer, de peur qu’il ne devienne un piège au milieu de toi. 
${}^{13}Bien plus, leurs autels, vous les démolirez ; leurs stèles, vous les briserez ; leurs poteaux sacrés, vous les couperez. 
${}^{14}Car tu ne te prosterneras pas devant un autre dieu. Le Seigneur, en effet, a pour nom : “Jaloux” ; il est un Dieu jaloux. 
${}^{15}Ne fais pas alliance avec les habitants du pays, car lorsqu’ils se prostituent avec leurs dieux et leur offrent des sacrifices, ils t’inviteraient et tu mangerais de leurs sacrifices, 
${}^{16}tu prendrais leurs filles comme épouses pour tes fils, leurs filles se prostitueraient avec leurs dieux et amèneraient tes fils à se prostituer avec leurs dieux.
${}^{17}Tu ne te feras pas des dieux en métal fondu.
${}^{18}Tu observeras la fête des Pains sans levain. Comme je te l’ai ordonné, tu mangeras des pains sans levain pendant sept jours, au temps fixé du mois des Épis, car c’est alors que tu es sorti d’Égypte.
${}^{19}Tout premier-né m’appartient : tout premier-né mâle de ton troupeau, gros ou petit bétail. 
${}^{20}Le premier-né des ânes, tu le rachèteras par un mouton, et si tu ne le rachètes pas, tu lui rompras la nuque. Tout premier-né de tes fils, tu le rachèteras. On ne se présentera pas devant moi les mains vides.
${}^{21}Pendant six jours, tu travailleras, mais, le septième jour, tu chômeras ; même au temps des labours et de la moisson, tu chômeras.
${}^{22}Tu célébreras la fête des Semaines, des premiers fruits, de la moisson des blés, et aussi la fête de la Récolte, à la fin de l’année.
${}^{23}Trois fois par an, tous les hommes paraîtront devant la face du Maître, le Seigneur, le Dieu d’Israël. 
${}^{24}En effet, lorsque j’aurai dépossédé les nations devant toi et que j’aurai élargi ton territoire, nul ne convoitera la terre qui t’appartient, quand, trois fois par an, tu monteras pour voir la face du Seigneur ton Dieu.
${}^{25}Tu n’immoleras pas le sacrifice sanglant en l’accompagnant de pain levé, et tu ne laisseras pas jusqu’au lendemain matin la victime sacrifiée pour la fête de la Pâque.
${}^{26}Tu apporteras les tout premiers fruits de ton sol à la maison du Seigneur ton Dieu. Tu ne feras pas cuire un chevreau dans le lait de sa mère. »
${}^{27}Le Seigneur dit encore à Moïse :
      « Mets par écrit ces paroles car, sur la base de celles-ci, je conclus une alliance avec toi et avec Israël. »
${}^{28}Moïse demeura sur le Sinaï\\avec le Seigneur quarante jours et quarante nuits ; il ne mangea pas de pain et ne but pas d’eau. Sur les tables de pierre\\, il écrivit les paroles de l’Alliance, les Dix Paroles.
${}^{29}Lorsque Moïse descendit de la montagne du Sinaï, ayant en mains les deux tables du Témoignage, il ne savait pas que son visage\\rayonnait de lumière\\depuis qu’il avait parlé avec le Seigneur\\. 
${}^{30} Aaron et tous les fils d’Israël virent arriver Moïse : son visage rayonnait. 
${}^{31} Comme ils n’osaient pas s’approcher, Moïse les appela. Aaron et tous les chefs de la communauté vinrent alors vers lui, et il leur adressa la parole.
${}^{32}Ensuite, tous les fils d’Israël s’approchèrent, et il leur transmit tous les ordres que le Seigneur lui avait donnés sur la montagne du Sinaï. 
${}^{33} Quand il eut fini de leur parler, il mit un voile sur son visage. 
${}^{34} Et, lorsqu’il se présentait devant le Seigneur pour parler avec lui, il enlevait son voile jusqu’à ce qu’il soit sorti. Alors, il transmettait aux fils d’Israël les ordres qu’il avait reçus, 
${}^{35} et les fils d’Israël voyaient rayonner son visage. Puis il remettait le voile sur son visage jusqu’à ce qu’il rentre pour parler avec le Seigneur\\.
      <h2 class="intertitle" id="d85e25721">6. Exécution des directives de Dieu (35 – 40)</h2>
      
         
      \bchapter{}
      \begin{verse}
${}^{1}Moïse rassembla toute la communauté des fils d’Israël. Il leur dit : « Voici ce que le Seigneur a ordonné : 
${}^{2}Pendant six jours, on travaillera, mais le septième jour sera pour vous un jour saint, un sabbat, un sabbat solennel pour le Seigneur. Quiconque travaillera ce jour-là sera mis à mort. 
${}^{3}Vous n’allumerez aucun feu dans vos maisons, le jour du sabbat. »
      
         
${}^{4}Moïse s’adressa à toute la communauté des fils d’Israël. Il dit : « Voici ce que le Seigneur a ordonné : 
${}^{5}Prélevez parmi vous une contribution pour le Seigneur. Tous les hommes que leur cœur y incitera apporteront cette contribution : de l’or, de l’argent et du bronze, 
${}^{6}de la pourpre violette et de la pourpre rouge, du cramoisi éclatant, du lin fin et du poil de chèvre, 
${}^{7}des peaux de bélier teintes en rouge, du cuir fin et du bois d’acacia, 
${}^{8}de l’huile pour le luminaire, du baume pour l’huile de l’onction et pour l’encens aromatique, 
${}^{9}des pierres de cornaline et des pierres pour orner l’éphod et le pectoral. 
${}^{10}Et que, parmi vous, tous les artisans habiles viennent et exécutent tout ce que le Seigneur a ordonné : 
${}^{11}la Demeure avec sa tente, sa couverture, ses agrafes, ses cadres, ses traverses, ses colonnes et ses socles ; 
${}^{12}l’arche avec ses barres, le propitiatoire, le rideau ; 
${}^{13}la table avec ses barres, tous ses accessoires et le pain de l’offrande ; 
${}^{14}le chandelier du luminaire avec ses accessoires et ses lampes ; l’huile du luminaire ; 
${}^{15}l’autel de l’encens avec ses barres, l’huile de l’onction, l’encens aromatique et le voile de l’entrée de la Demeure ; 
${}^{16}l’autel de l’holocauste avec sa grille de bronze, ses barres et tous ses accessoires ; la cuve avec son support ; 
${}^{17}les toiles du parvis, ses colonnes, ses socles et le voile de la porte du parvis ; 
${}^{18}les piquets de la Demeure, les piquets du parvis et leurs cordes ; 
${}^{19}les vêtements liturgiques pour officier dans le sanctuaire, les vêtements sacrés pour Aaron, le prêtre, et les vêtements que porteront ses fils pour exercer le sacerdoce. »
${}^{20}Toute la communauté des fils d’Israël se retira de devant Moïse. 
${}^{21}Alors tous les hommes que leur cœur y portait et que leur esprit y incitait vinrent apporter la contribution du Seigneur, pour les travaux de la tente de la Rencontre, pour tout son service et pour les vêtements sacrés. 
${}^{22}Les hommes vinrent aussi bien que les femmes ; tous ceux que leur cœur y incitait apportèrent broches, boucles, anneaux, breloques – tous objets d’or que chacun offrait au Seigneur avec le geste d’élévation. 
${}^{23}Tous ceux qui possédaient de la pourpre violette, de la pourpre rouge, du cramoisi éclatant, du lin, du poil de chèvre, des peaux de béliers teintes en rouge ou du cuir fin, tous ceux-là en apportèrent. 
${}^{24}Tous ceux qui offraient une contribution d’argent et de bronze apportèrent la contribution du Seigneur. Tous ceux qui possédaient du bois d’acacia l’apportèrent pour les travaux du service. 
${}^{25}Toutes les femmes habiles filèrent de leurs mains et apportèrent ce qu’elles avaient filé : la pourpre violette et la pourpre rouge, le cramoisi éclatant et le lin ; 
${}^{26}toutes les femmes que leur cœur y portait et qui étaient habiles filèrent le poil de chèvre. 
${}^{27}Les chefs de la communauté apportèrent les pierres de cornaline et les pierres pour orner l’éphod et le pectoral, 
${}^{28}ainsi que le baume et l’huile pour le luminaire, l’huile de l’onction et l’encens aromatique. 
${}^{29}Hommes et femmes, tous ceux que leur cœur y incitait apportèrent leur part à tout l’ouvrage que le Seigneur avait commandé par l’intermédiaire de Moïse ; ainsi, les fils d’Israël apportèrent une offrande volontaire au Seigneur.
${}^{30}Moïse dit aux fils d’Israël : « Voyez : Le Seigneur a appelé par son nom Beçalel, fils d’Ouri, fils de Hour, de la tribu de Juda. 
${}^{31}Il l’a rempli de l’esprit de Dieu : sagesse, intelligence, savoir, en toute sorte d’ouvrages, 
${}^{32}pour concevoir des œuvres d’art et les réaliser avec l’or, l’argent, le bronze, 
${}^{33}pour tailler les pierres à sertir, sculpter sur bois et pour exécuter toute œuvre d’art. 
${}^{34}Il a mis en son cœur le don de transmettre le savoir, comme en celui d’Oholiab, fils d’Ahisamak, de la tribu de Dane. 
${}^{35}Il a rempli leur cœur de sagesse pour exécuter tout le travail du ciseleur, du brodeur, du brocheur de pourpre violette et pourpre rouge, cramoisi éclatant et lin, ainsi que le travail du tisserand. Ce sont des artisans de toute sorte, de véritables artistes.
      
         
      \bchapter{}
      \begin{verse}
${}^{1}Beçalel, Oholiab et tout artisan habile à qui le Seigneur a donné sagesse et intelligence pour concevoir et exécuter les travaux au service du sanctuaire, tous exécuteront ce que le Seigneur a ordonné. »
      
         
${}^{2}Moïse appela donc, pour se mettre à l’ouvrage et l’exécuter, Beçalel, Oholiab et tout artisan habile à qui le Seigneur avait donné la sagesse, tous ceux que leur cœur y portait. 
${}^{3}Ils reçurent de Moïse la contribution que les fils d’Israël avaient apportée pour exécuter ces travaux au service du sanctuaire. Chaque matin, on apportait encore des offrandes volontaires. 
${}^{4}Alors, tous les artisans occupés aux divers travaux du sanctuaire quittèrent chacun l’ouvrage qu’ils étaient en train de faire 
${}^{5}et vinrent dire à Moïse : « Le peuple apporte plus qu’il n’en faut pour le travail que le Seigneur a ordonné d’exécuter. » 
${}^{6}Moïse donna donc cet ordre que l’on fit passer dans le camp : « Que personne, ni homme ni femme, n’apporte plus rien en contribution pour le sanctuaire. » Le peuple cessa d’apporter quoi que ce soit. 
${}^{7}Il y avait suffisamment de matériaux pour faire tout le travail ; il y en avait même en surplus.
${}^{8}Les ouvriers, artisans habiles, construisirent la Demeure ; ils firent dix tentures de lin retors, pourpre violette, pourpre rouge et cramoisi éclatant, et on y broda des kéroubim : c’est une œuvre d’artiste. 
${}^{9}Chaque tenture mesurait vingt-huit coudées de long et quatre de large. Toutes les tentures avaient les mêmes dimensions. 
${}^{10}On assembla cinq tentures l’une à l’autre, et les cinq autres également. 
${}^{11}On fit des lacets de pourpre violette au bord de la première tenture, à l’extrémité de l’assemblage, et on fit de même au bord de la dernière tenture du deuxième assemblage. 
${}^{12}On mit cinquante lacets à la première tenture et cinquante lacets à l’extrémité de la tenture du deuxième assemblage, les lacets s’attachant l’un à l’autre. 
${}^{13}On fit cinquante agrafes en or, on assembla les tentures l’une à l’autre par les agrafes. Ainsi, la Demeure fut d’un seul tenant.
${}^{14}Ensuite, pour former une tente au-dessus de la Demeure, on fit onze tentures en poil de chèvre. 
${}^{15}Chaque tenture mesurait trente coudées de long et quatre de large. Les onze tentures avaient les mêmes dimensions. 
${}^{16}On assembla cinq tentures à part, puis six tentures à part. 
${}^{17}On fit cinquante lacets au bord d’une première tenture, la dernière de l’assemblage, et cinquante lacets au bord de la même tenture du deuxième assemblage. 
${}^{18}On fit cinquante agrafes de bronze pour assembler la tente d’un seul tenant. 
${}^{19}Et on fit pour la tente une couverture en peaux de béliers teintes en rouge, et une autre en cuir fin mise par-dessus.
${}^{20}On fit pour la Demeure des cadres en bois d’acacia, dressés debout. 
${}^{21}Chaque cadre mesurait dix coudées de long et une coudée et demie de large. 
${}^{22}Il était assemblé par deux tenons jumelés. Ainsi fut-il fait pour tous les cadres de la Demeure. 
${}^{23}On en disposa vingt en direction du Néguev, au sud ; 
${}^{24}et on fit quarante socles en argent sous les vingt cadres : deux socles sous un cadre pour ses deux tenons, puis deux socles sous un autre cadre pour ses deux tenons. 
${}^{25}Pour le deuxième côté de la Demeure, on disposa, en direction du nord, vingt cadres 
${}^{26}avec leurs quarante socles en argent : deux socles sous un cadre et deux socles sous un autre cadre. 
${}^{27}Et pour le fond de la Demeure, vers l’ouest, on fit six cadres ; 
${}^{28}on fit aussi deux cadres comme contreforts de la Demeure, au fond ; 
${}^{29}ils étaient jumelés à leur base et l’étaient également à leur sommet, à la hauteur du premier anneau : ainsi fut-il fait pour eux deux, pour les deux contreforts. 
${}^{30}Il y eut donc huit cadres, avec leurs socles en argent, soit seize socles : deux socles sous un cadre, deux socles sous un autre cadre.
${}^{31}On fit les traverses en bois d’acacia : cinq pour les cadres du premier côté de la Demeure, 
${}^{32}cinq pour les cadres du deuxième côté de la Demeure, cinq pour les cadres qui forment le fond de la Demeure vers l’ouest. 
${}^{33}On fit aussi la traverse médiane, à mi-hauteur des cadres, traversant la Demeure d’un bout à l’autre. 
${}^{34}Les cadres, on les plaqua d’or, on fit en or leurs anneaux pour loger les traverses, et les traverses, on les plaqua d’or.
${}^{35}On fit un rideau de pourpre violette, pourpre rouge, cramoisi éclatant et lin retors ; c’est une œuvre d’artiste : on y broda des kéroubim. 
${}^{36}On le fixa à quatre colonnes en acacia, plaquées d’or et munies de crochets en or. On coula pour elles quatre socles en argent.
${}^{37}Pour l’entrée de la tente, on fit un voile en pourpre violette, pourpre rouge, cramoisi éclatant et lin retors : c’est une œuvre d’artisan brocheur. 
${}^{38}On fit pour le voile cinq colonnes avec leurs crochets ; leurs chapiteaux et leurs tringles furent plaqués d’or ; leurs cinq socles étaient en bronze.
      
         
      \bchapter{}
      \begin{verse}
${}^{1}Beçalel fit l’arche en bois d’acacia de deux coudées et demie de long sur une coudée et demie de large et une coudée et demie de haut. 
${}^{2}Il la plaqua d’or pur à l’intérieur et à l’extérieur, et il l’entoura d’une moulure en or. 
${}^{3}Il coula quatre anneaux d’or, qu’il attacha aux quatre pieds de l’arche : deux anneaux sur un côté, deux anneaux sur l’autre. 
${}^{4}Il fit des barres en bois d’acacia, il les plaqua d’or 
${}^{5}et il les introduisit dans les anneaux des côtés de l’arche pour qu’on puisse la porter.
${}^{6}Il fit un propitiatoire en or pur, long de deux coudées et demie et large d’une coudée et demie. 
${}^{7}Et il forgea deux kéroubim en or, qu’il plaça aux deux extrémités du propitiatoire, 
${}^{8}un kéroub à une extrémité, et l’autre kéroub à l’autre extrémité ; il fit donc les kéroubim aux deux extrémités du propitiatoire. 
${}^{9}Les kéroubim avaient les ailes déployées vers le haut et protégeaient le propitiatoire de leurs ailes ; ils se faisaient face, le regard tourné vers le propitiatoire.
${}^{10}Il fit la table en bois d’acacia, longue de deux coudées, large d’une coudée et haute d’une coudée et demie. 
${}^{11}Il la plaqua d’or pur et il l’entoura d’une moulure en or. 
${}^{12}Il lui fit des entretoises d’un palme et il les entoura d’une moulure en or. 
${}^{13}Il coula quatre anneaux d’or qu’il mit aux quatre angles formés par les quatre pieds. 
${}^{14}Ces anneaux furent placés près des entretoises, pour loger les barres servant à porter la table. 
${}^{15}Il fit les barres en bois d’acacia et il les plaqua d’or : elles servaient à porter la table. 
${}^{16}Il fit les accessoires de la table : plats, gobelets, timbales et aiguières pour les libations ; il les fit en or pur.
${}^{17}Il fit le chandelier en or pur ; il forgea le chandelier : base, tige, coupes, boutons et fleurs faisaient corps avec lui. 
${}^{18}Six branches s’en détachaient sur les côtés : trois d’un côté et trois de l’autre. 
${}^{19}Sur une branche, trois coupes en forme d’amande avec boutons et fleurs et, sur une autre branche, trois coupes en forme d’amande avec boutons et fleurs ; de même pour les six branches sortant du chandelier. 
${}^{20}Le chandelier lui-même portait quatre coupes en forme d’amande avec boutons et fleurs : 
${}^{21}un bouton sous les deux premières branches issues du chandelier, un bouton sous les deux suivantes, et un bouton sous les deux dernières ; ainsi donc pour les six branches qui en sortaient. 
${}^{22}Boutons et branches faisaient corps avec le chandelier qui était tout entier forgé d’une seule pièce, en or pur. 
${}^{23}Il lui fit des lampes au nombre de sept, des pincettes et des porte-lampes en or pur. 
${}^{24}Il lui fallut un lingot d’or pur pour le chandelier et tous ses accessoires.
${}^{25}Il fit l’autel de l’encens en bois d’acacia. L’autel avait une coudée de long, une coudée de large – sa base était donc carrée – et deux coudées et demie de haut. Ses cornes faisaient corps avec lui. 
${}^{26}Il le plaqua d’or pur : le dessus, les parois tout autour et les cornes ; et il l’entoura d’une moulure en or. 
${}^{27}Sous la moulure, sur les deux côtés, il plaça des anneaux d’or pour loger les barres servant à le porter. 
${}^{28}Il fit les barres en bois d’acacia et les plaqua d’or. 
${}^{29}Il fit l’huile d’onction sainte et l’encens aromatique pur : c’est une œuvre de parfumeur.
      
         
      \bchapter{}
      \begin{verse}
${}^{1}Il fit l’autel de l’holocauste en bois d’acacia. L’autel avait cinq coudées de long, cinq coudées de large – sa base était donc carrée – et trois coudées de haut. 
${}^{2}Il fit, aux quatre angles de l’autel, des cornes qui faisaient corps avec lui. Il le plaqua de bronze. 
${}^{3}Il fit tous les accessoires de l’autel : les vases, les pelles, les bols pour l’aspersion, les fourchettes et les brûle-parfums, le tout en bronze. 
${}^{4}Il fit pour l’autel une grille de bronze en forme de filet et il la mit sous la bordure de l’autel, en bas, à mi-hauteur. 
${}^{5}Il coula quatre anneaux aux quatre extrémités de la grille de bronze, pour loger les barres. 
${}^{6}Il fit les barres en bois d’acacia et il les plaqua de bronze. 
${}^{7}Il les engagea dans les anneaux placés sur les côtés de l’autel pour qu’elles servent à le porter. L’autel était creux, en planches.
      
         
${}^{8}Il fit la cuve en bronze sur un support en bronze, avec les miroirs des femmes qui étaient de service à l’entrée de la tente de la Rencontre.
${}^{9}Il fit le parvis. Du côté du Néguev, au sud, le parvis avait des toiles en lin retors, sur une longueur de cent coudées. 
${}^{10}Leurs vingt colonnes et leurs vingt socles étaient en bronze ; les crochets des colonnes et leurs tringles, en argent. 
${}^{11}De même, du côté nord, cent coudées, avec leurs vingt colonnes et leurs vingt socles en bronze ; les crochets des colonnes et leurs tringles étaient en argent. 
${}^{12}Du côté ouest, des toiles sur cinquante coudées, avec leurs dix colonnes et leurs dix socles ; les crochets des colonnes et leurs tringles étaient en argent. 
${}^{13}Du côté de l’est, vers le levant, cinquante coudées ; 
${}^{14}quinze coudées de toiles sur une aile, avec leurs trois colonnes et leurs trois socles, 
${}^{15}et, sur la deuxième aile, de part et d’autre de la porte du parvis, quinze coudées de toiles, avec leurs trois colonnes et leurs trois socles. 
${}^{16}Toutes les toiles de l’enceinte du parvis étaient en lin retors. 
${}^{17}Les socles des colonnes étaient en bronze ; leurs crochets et leurs tringles, en argent ; leurs chapiteaux étaient plaqués d’argent. Toutes les colonnes de l’enceinte du parvis étaient réunies par des tringles en argent.
${}^{18}Œuvre d’artisan brocheur, le rideau de la porte du parvis était en pourpre violette, pourpre rouge, cramoisi éclatant et lin retors ; il avait vingt coudées de long et cinq coudées de haut – prises dans la largeur –, comme les toiles du parvis. 
${}^{19}Les quatre colonnes et leurs quatre socles étaient en bronze, leurs crochets étaient en argent, leurs chapiteaux et leurs tringles étaient plaqués d’argent. 
${}^{20}Tous les piquets pour la Demeure et l’enceinte du parvis étaient en bronze.
${}^{21}Voici l’état des comptes de la Demeure – la Demeure du Témoignage – qui fut établi sur l’ordre de Moïse ; ce fut le service des Lévites, accompli par l’intermédiaire d’Itamar, fils d’Aaron, le prêtre. 
${}^{22}Beçalel, fils d’Ouri, fils de Hour, de la tribu de Juda, avait exécuté tout ce que le Seigneur avait ordonné à Moïse. 
${}^{23}Avec lui, Oholiab, fils d’Ahisamak, de la tribu de Dane : ciseleur et artiste, brocheur sur pourpre violette, pourpre rouge, cramoisi éclatant et lin.
${}^{24}Total de l’or utilisé pour les travaux, tous les travaux du sanctuaire – et c’était l’or provenant de l’offrande – : vingt-neuf talents et sept cent trente sicles, en sicles du sanctuaire.
${}^{25}Argent provenant des personnes recensées de la communauté : cent talents et mille sept cent soixante-quinze sicles, en sicles du sanctuaire, 
${}^{26}soit un béqua par tête ou un demi-sicle, en sicles du sanctuaire, pour tout homme passant au recensement, âgé de vingt ans et plus, soit six cent trois mille cinq cent cinquante hommes. 
${}^{27}Cent talents d’argent furent utilisés pour couler les socles du sanctuaire et les socles du rideau : cent socles avec les cent talents, un talent par socle. 
${}^{28}Avec les mille sept cent soixante-quinze sicles, on avait fait les crochets des colonnes, on avait plaqué leurs chapiteaux et on les avait reliées par des tringles.
${}^{29}Bronze provenant de l’offrande : soixante-dix talents et deux mille quatre cents sicles. 
${}^{30}On en avait fait les socles de l’entrée de la tente de la Rencontre, l’autel de bronze et sa grille de bronze, tous les accessoires de l’autel, 
${}^{31}les socles de l’enceinte du parvis, les socles de la porte du parvis, tous les piquets de la Demeure et tous les piquets de l’enceinte du parvis.
      
         
      \bchapter{}
      \begin{verse}
${}^{1}Avec la pourpre violette, la pourpre rouge et le cramoisi éclatant, on fit les vêtements liturgiques pour officier dans le sanctuaire, et les vêtements sacrés destinés à Aaron, comme le Seigneur l’avait ordonné à Moïse. 
${}^{2}On fit l’éphod en or, pourpre violette, pourpre rouge, cramoisi éclatant et lin retors. 
${}^{3}On lamina des plaques d’or et on y découpa des rubans pour les entrelacer avec la pourpre violette, la pourpre rouge, le cramoisi éclatant et le lin : c’était une œuvre d’artiste. 
${}^{4}On attacha l’éphod par des brides fixées à ses deux extrémités. 
${}^{5}L’écharpe portée au-dessus de l’éphod et faisant corps avec lui fut travaillée de la même manière : en or, pourpre violette, pourpre rouge, cramoisi éclatant et lin retors, comme le Seigneur l’avait ordonné à Moïse. 
${}^{6}On apprêta ensuite les pierres de cornaline : elles étaient enchâssées dans des chatons en or, et on les grava aux noms des fils d’Israël, comme on grave un sceau. 
${}^{7}On plaça sur les brides de l’éphod ces pierres qui sont un mémorial pour les fils d’Israël, comme le Seigneur l’avait ordonné à Moïse.
${}^{8}On fit le pectoral – œuvre d’artiste – de la même manière que l’éphod : en or, pourpre violette, pourpre rouge, cramoisi éclatant et lin retors. 
${}^{9}Il était carré. On lui fit une doublure. Il avait un empan de côté. 
${}^{10}On le garnit de quatre rangées de pierres : la première, de sardoine, topaze et émeraude ; 
${}^{11}la deuxième, d’escarboucle, saphir et jaspe ; 
${}^{12}la troisième, de béryl, agate et améthyste ; 
${}^{13}et la quatrième, de chrysolithe, cornaline et onyx. Elles étaient garnies de chatons d’or. 
${}^{14}Les pierres étaient aux noms des fils d’Israël ; comme leurs noms, elles étaient douze, gravées à la manière d’un sceau ; chacune portait le nom de l’une des douze tribus. 
${}^{15}On fit au pectoral des chaînettes tressées et torsadées, en or pur. 
${}^{16}On fit deux chatons d’or et deux anneaux d’or, et on fixa les deux anneaux à deux extrémités du pectoral. 
${}^{17}On fixa les deux torsades d’or aux deux anneaux, aux extrémités du pectoral ; 
${}^{18}on fixa les deux extrémités des deux torsades aux deux chatons ; on les fixa aux brides de l’éphod par-devant. 
${}^{19}On fit deux anneaux d’or et on les mit à deux extrémités du pectoral, du côté tourné vers l’éphod, en dedans. 
${}^{20}On fit deux anneaux d’or et on les fixa aux deux brides de l’éphod, à leur base, par-devant, près de leur point d’attache, au-dessus de l’écharpe de l’éphod. 
${}^{21}On relia le pectoral par ses anneaux aux anneaux de l’éphod avec un cordon de pourpre violette : le pectoral était sur l’écharpe de l’éphod et il ne pouvait s’en détacher. On exécuta cela comme le Seigneur l’avait ordonné à Moïse.
${}^{22}On fit le manteau de l’éphod, tout entier de pourpre violette : c’était l’œuvre d’un ouvrier tisserand. 
${}^{23}Il avait en son milieu une ouverture, bordée comme celle d’une cuirasse, donc indéchirable. 
${}^{24}Sur les pans du manteau, on fit des grenades de pourpre violette, de pourpre rouge, de cramoisi éclatant et de lin retors, 
${}^{25}alternant avec des clochettes d’or pur, tout autour : 
${}^{26}clochette et grenade, clochette et grenade, sur les pans du manteau, tout autour, pour officier, comme le Seigneur l’avait ordonné à Moïse.
${}^{27}On fit les tuniques de lin pour Aaron et pour ses fils : c’était l’œuvre d’un ouvrier tisserand. 
${}^{28}On fit aussi le turban de lin, les garnitures des tiares de lin, les caleçons de lin, en lin retors ; 
${}^{29}les ceintures en lin retors, pourpre violette, pourpre rouge et cramoisi éclatant : c’était l’œuvre d’un artisan brocheur. On exécuta cela comme le Seigneur l’avait ordonné à Moïse.
${}^{30}On fit en or pur le fleuron, le saint diadème. Comme sur un sceau, on y grava l’inscription : « Consacré au Seigneur ». 
${}^{31}On attacha le fleuron à un cordon de pourpre violette et on le plaça sur le devant du turban, comme le Seigneur l’avait ordonné à Moïse.
${}^{32}Ainsi fut achevé tout le service de la Demeure de la tente de la Rencontre. Les fils d’Israël s’étaient mis à l’œuvre : comme le Seigneur l’avait ordonné à Moïse, ainsi firent-ils.
${}^{33}Alors ils présentèrent la Demeure à Moïse : la Tente et tous ses accessoires, ses agrafes, ses cadres, ses traverses, ses colonnes et ses socles, 
${}^{34}la couverture en peaux de béliers teintes en rouge et la couverture en cuir fin, le rideau de séparation ; 
${}^{35}l’arche du Témoignage, ses barres et le propitiatoire ; 
${}^{36}la table, tous ses accessoires et le pain de l’offrande ; 
${}^{37}le chandelier d’or pur, ses lampes, tous ses accessoires, l’huile du luminaire ; 
${}^{38}l’autel d’or, l’huile de l’onction, l’encens aromatique et le rideau pour l’entrée de la Tente ; 
${}^{39}l’autel de bronze, sa grille de bronze, ses barres et tous ses accessoires ; la cuve et son support ; 
${}^{40}les toiles du parvis, ses colonnes, ses socles, le rideau pour la porte du parvis, ses cordes, ses piquets et tous les accessoires du service de la Demeure, pour la tente de la Rencontre ; 
${}^{41}les vêtements liturgiques pour officier dans le sanctuaire, les vêtements sacrés pour Aaron, le prêtre, et les vêtements que portent ses fils pour exercer le sacerdoce.
${}^{42}Les fils d’Israël avaient exécuté tout l’ouvrage, comme le Seigneur l’avait ordonné à Moïse. 
${}^{43}Moïse vit tout ce travail : voici qu’ils l’avaient fait ! Comme le Seigneur l’avait ordonné, ainsi avaient-ils fait. Alors Moïse les bénit.
      
         
      \bchapter{}
      \begin{verse}
${}^{1}Le Seigneur parla à Moïse. Il dit : 
${}^{2}« Au premier mois, le premier jour du mois, tu dresseras la Demeure de la tente de la Rencontre. 
${}^{3}Tu y mettras l’arche du Témoignage et tu protégeras l’arche avec le rideau. 
${}^{4}Tu apporteras la table et tu la disposeras avec soin. Tu apporteras le chandelier et tu allumeras ses lampes. 
${}^{5}Tu placeras l’autel d’or pour l’encens devant l’arche du Témoignage et tu mettras le rideau à l’entrée de la Demeure. 
${}^{6}Tu placeras l’autel de l’holocauste devant l’entrée de la Demeure de la tente de la Rencontre. 
${}^{7}Tu placeras la cuve entre la tente de la Rencontre et l’autel, et tu y verseras de l’eau. 
${}^{8}Tu installeras l’enceinte du parvis et tu placeras le rideau de la porte du parvis. 
${}^{9}Tu prendras l’huile de l’onction, et tu feras l’onction sur la Demeure et tout ce qu’elle contient ; tu la consacreras, ainsi que tous ses accessoires, et elle sera sainte. 
${}^{10}Tu feras l’onction sur l’autel de l’holocauste et tous ses accessoires, tu le consacreras, et l’autel sera très saint. 
${}^{11}Tu feras l’onction sur la cuve et son support, et tu la consacreras. 
${}^{12}Tu feras approcher Aaron et ses fils de l’entrée de la tente de la Rencontre, tu les baigneras dans l’eau, 
${}^{13}tu revêtiras Aaron des vêtements sacrés, tu lui donneras l’onction et tu le consacreras afin qu’il exerce pour moi le sacerdoce. 
${}^{14}Tu feras approcher ses fils, tu les revêtiras de tuniques, 
${}^{15}tu leur donneras l’onction comme tu l’as donnée à leur père, afin qu’ils exercent pour moi le sacerdoce. Ainsi, l’onction reçue leur conférera un sacerdoce perpétuel, de génération en génération. »
${}^{16}Moïse exécuta tout ce que le Seigneur lui avait ordonné\\. 
${}^{17} La demeure de Dieu\\fut érigée la deuxième année après la sortie d’Égypte\\, le premier jour du premier mois. 
${}^{18} Moïse érigea ainsi la Demeure : il en posa les bases, les poutres et les traverses, et il dressa les colonnes. 
${}^{19} Au-dessus de la Demeure, il déploya la Tente et la recouvrit comme le Seigneur le lui avait ordonné\\. 
${}^{20} Il prit le Témoignage et le déposa dans l’arche. Il mit à l’arche ses barres et la recouvrit de la plaque d’or appelée\\propitiatoire. 
${}^{21} Il introduisit l’arche dans la Demeure, et posa le rideau pour voiler l’arche du Témoignage comme le Seigneur le lui avait ordonné\\.
${}^{22}Moïse plaça la table dans la tente de la Rencontre, sur le côté nord de la Demeure, à l’extérieur du rideau. 
${}^{23}Il y disposa une rangée de pains devant le Seigneur, comme le Seigneur le lui avait ordonné. 
${}^{24}Il mit le chandelier dans la tente de la Rencontre en face de la table, sur le côté sud de la Demeure. 
${}^{25}Il alluma les lampes devant le Seigneur, comme le Seigneur le lui avait ordonné. 
${}^{26}Il mit l’autel d’or dans la tente de la Rencontre, en face du rideau, 
${}^{27}et il y brûla de l’encens aromatique, comme le Seigneur le lui avait ordonné. 
${}^{28}Il mit le rideau à l’entrée de la Demeure. 
${}^{29}Il dressa l’autel de l’holocauste à l’entrée de la tente de la Rencontre et y présenta l’holocauste et l’offrande de céréales, comme le Seigneur le lui avait ordonné. 
${}^{30}Il posa la cuve entre la tente de la Rencontre et l’autel, et y versa de l’eau pour les ablutions. 
${}^{31}Moïse, Aaron et ses fils s’y lavaient les mains et les pieds ; 
${}^{32}quand ils entraient dans la tente de la Rencontre et qu’ils s’approchaient de l’autel, ils se lavaient, comme le Seigneur l’avait ordonné à Moïse. 
${}^{33}Moïse installa le parvis autour de la Demeure et de l’autel, et il plaça le voile de la porte du parvis. Ainsi Moïse acheva le travail.
${}^{34}La nuée couvrit la tente de la Rencontre, et la gloire du Seigneur remplit la Demeure. 
${}^{35} Moïse ne pouvait pas entrer dans la tente de la Rencontre, car la nuée y demeurait et la gloire du Seigneur remplissait la Demeure.
${}^{36}À chaque étape, lorsque la nuée s’élevait et quittait la Demeure, les fils d’Israël levaient le camp. 
${}^{37} Si la nuée ne s’élevait pas, ils campaient\\jusqu’au jour où elle s’élevait. 
${}^{38} Dans la journée, la nuée du Seigneur reposait sur la Demeure, et la nuit, un feu brillait dans la nuée aux yeux de tout Israël\\. Et il en fut ainsi\\à toutes leurs étapes.
