  
  
    
    \bbook{ÉVANGILE SELON SAINT JEAN}{ÉVANGILE SELON SAINT JEAN}
      
         
      \bchapter{}
        ${}^{1}Au commencement était le Verbe,
        \\et le Verbe était auprès de Dieu,
        \\et le Verbe était Dieu.
        ${}^{2}Il était au commencement auprès de Dieu.
        ${}^{3}C’est par lui que tout est venu à l’existence,
        \\et rien de ce qui s’est fait ne s’est fait sans lui.
        ${}^{4}En lui était la vie,
        \\et la vie était la lumière des hommes ;
        ${}^{5}la lumière brille dans les ténèbres,
        \\et les ténèbres ne l’ont pas arrêtée.
        
           
         
        ${}^{6}Il y eut un homme envoyé par Dieu ;
        \\son nom était Jean.
        ${}^{7}Il est venu comme témoin,
        \\pour rendre témoignage à la Lumière,
        \\afin que tous croient par lui.
        ${}^{8}Cet homme n’était pas la Lumière,
        \\mais il était là pour rendre témoignage à la Lumière.
        
           
         
        ${}^{9}Le Verbe était la vraie Lumière,
        \\qui éclaire tout homme
        \\en venant dans le monde.
        ${}^{10}Il était dans le monde,
        \\et le monde était venu par lui à l’existence,
        \\mais le monde ne l’a pas reconnu.
        ${}^{11}Il est venu chez lui,
        \\et les siens ne l’ont pas reçu.
        ${}^{12}Mais à tous ceux qui l’ont reçu,
        \\il a donné de pouvoir devenir enfants de Dieu,
        \\eux qui croient en son nom.
        ${}^{13}Ils ne sont pas nés du sang,
        \\ni d’une volonté charnelle, ni d’une volonté d’homme :
        \\ils sont nés de Dieu.
        ${}^{14}Et le Verbe s’est fait chair,
        \\il a habité parmi nous,
        \\et nous avons vu sa gloire,
        \\la gloire qu’il tient de son Père
        \\comme Fils unique,
        \\plein de grâce et de vérité.
        
           
         
        ${}^{15}Jean le Baptiste lui rend témoignage en proclamant :
        \\« C’est de lui que j’ai dit :
        \\Celui qui vient derrière moi
        \\est passé devant moi,
        \\car avant moi il était. »
        ${}^{16}Tous nous avons eu part à sa plénitude,
        \\nous avons reçu grâce après grâce ;
        ${}^{17}car la Loi fut donnée par Moïse,
        \\la grâce et la vérité sont venues par Jésus Christ.
        ${}^{18}Dieu, personne ne l’a jamais vu ;
        \\le Fils unique, lui qui est Dieu,
        \\lui qui est dans le sein du Père,
        \\c’est lui qui l’a fait connaître.
        
           
      <h2 class="intertitle" id="d85e360131">1. Premiers jours de la révélation de Jésus (1,19-51)</h2>
${}^{19}Voici le témoignage de Jean, quand les Juifs lui envoyèrent de Jérusalem des prêtres et des lévites pour lui demander : « Qui es-tu ? » 
${}^{20}Il ne refusa pas de répondre, il déclara ouvertement : « Je ne suis pas le Christ. » 
${}^{21}Ils lui demandèrent : « Alors qu’en est-il ? Es-tu le prophète Élie ? » Il répondit : « Je ne le suis pas. – Es-tu le Prophète annoncé ? » Il répondit : « Non. » 
${}^{22}Alors ils lui dirent : « Qui es-tu ? Il faut que nous donnions une réponse à ceux qui nous ont envoyés. Que dis-tu sur toi-même ? » 
${}^{23}Il répondit :
      « Je suis la voix de celui qui crie dans le désert : Redressez le chemin du Seigneur, comme a dit le prophète Isaïe. »
${}^{24}Or, ils avaient été envoyés de la part des pharisiens. 
${}^{25}Ils lui posèrent encore cette question : « Pourquoi donc baptises-tu, si tu n’es ni le Christ, ni Élie, ni le Prophète ? » 
${}^{26}Jean leur répondit : « Moi, je baptise dans l’eau. Mais au milieu de vous se tient celui que vous ne connaissez pas ; 
${}^{27}c’est lui qui vient derrière moi, et je ne suis pas digne de délier la courroie de sa sandale. » 
${}^{28}Cela s’est passé à Béthanie, de l’autre côté du Jourdain, à l’endroit où Jean baptisait.
${}^{29}Le lendemain, voyant Jésus venir vers lui, Jean déclara : « Voici l’Agneau de Dieu, qui enlève le péché du monde ; 
${}^{30}c’est de lui que j’ai dit : L’homme qui vient derrière moi est passé devant moi, car avant moi il était. 
${}^{31}Et moi, je ne le connaissais pas ; mais, si je suis venu baptiser dans l’eau, c’est pour qu’il soit manifesté à Israël. » 
${}^{32}Alors Jean rendit ce témoignage : « J’ai vu l’Esprit descendre du ciel comme une colombe et il demeura sur lui. 
${}^{33}Et moi, je ne le connaissais pas, mais celui qui m’a envoyé baptiser dans l’eau m’a dit : “Celui sur qui tu verras l’Esprit descendre et demeurer, celui-là baptise dans l’Esprit Saint.” 
${}^{34}Moi, j’ai vu, et je rends témoignage : c’est lui le Fils de Dieu. »
${}^{35}Le lendemain encore, Jean se trouvait là avec deux de ses disciples. 
${}^{36}Posant son regard sur Jésus qui allait et venait, il dit : « Voici l’Agneau de Dieu. » 
${}^{37}Les deux disciples entendirent ce qu’il disait, et ils suivirent Jésus. 
${}^{38}Se retournant, Jésus vit qu’ils le suivaient, et leur dit : « Que cherchez-vous ? » Ils lui répondirent : « Rabbi – ce qui veut dire : Maître –, où demeures-tu ? » 
${}^{39}Il leur dit : « Venez, et vous verrez. » Ils allèrent donc, ils virent où il demeurait, et ils restèrent auprès de lui ce jour-là. C’était vers la dixième heure, (environ quatre heures de l’après-midi).
${}^{40}André, le frère de Simon-Pierre, était l’un des deux disciples qui avaient entendu la parole de Jean et qui avaient suivi Jésus. 
${}^{41}Il trouve d’abord Simon, son propre frère, et lui dit : « Nous avons trouvé le Messie » – ce qui veut dire : Christ. 
${}^{42}André amena son frère à Jésus. Jésus posa son regard sur lui et dit : « Tu es Simon, fils de Jean ; tu t’appelleras Kèphas » – ce qui veut dire : Pierre.
${}^{43}Le lendemain, Jésus décida de partir pour la Galilée. Il trouve Philippe, et lui dit : « Suis-moi. » 
${}^{44}Philippe était de Bethsaïde, le village d’André et de Pierre. 
${}^{45}Philippe trouve Nathanaël et lui dit : « Celui dont il est écrit dans la loi de Moïse et chez les Prophètes, nous l’avons trouvé : c’est Jésus fils de Joseph, de Nazareth. » 
${}^{46}Nathanaël répliqua : « De Nazareth peut-il sortir quelque chose de bon ? » Philippe répond : « Viens, et vois. » 
${}^{47}Lorsque Jésus voit Nathanaël venir à lui, il déclare à son sujet : « Voici vraiment un Israélite : il n’y a pas de ruse en lui. » 
${}^{48}Nathanaël lui demande : « D’où me connais-tu ? » Jésus lui répond : « Avant que Philippe t’appelle, quand tu étais sous le figuier, je t’ai vu. » 
${}^{49}Nathanaël lui dit : « Rabbi, c’est toi le Fils de Dieu ! C’est toi le roi d’Israël ! » 
${}^{50}Jésus reprend : « Je te dis que je t’ai vu sous le figuier, et c’est pour cela que tu crois ! Tu verras des choses plus grandes encore. » 
${}^{51}Et il ajoute : « Amen, amen, je vous le dis : vous verrez le ciel ouvert, et les anges de Dieu monter et descendre au-dessus du Fils de l’homme. »
      <h2 class="intertitle" id="d85e360353">2. Du premier au second signe de Cana (2 – 4)</h2>
      
         
      \bchapter{}
      \begin{verse}
${}^{1}Le troisième jour, il y eut un mariage à Cana de Galilée. La mère de Jésus était là. 
${}^{2}Jésus aussi avait été invité au mariage avec ses disciples.
${}^{3}Or, on manqua de vin. La mère de Jésus lui dit : « Ils n’ont pas de vin. » 
${}^{4}Jésus lui répond : « Femme, que me veux-tu ? Mon heure n’est pas encore venue. » 
${}^{5}Sa mère dit à ceux qui servaient : « Tout ce qu’il vous dira, faites-le. » 
${}^{6}Or, il y avait là six jarres de pierre pour les purifications rituelles des Juifs ; chacune contenait deux à trois mesures, (c’est-à-dire environ cent litres). 
${}^{7}Jésus dit à ceux qui servaient : « Remplissez d’eau les jarres. » Et ils les remplirent jusqu’au bord. 
${}^{8}Il leur dit : « Maintenant, puisez, et portez-en au maître du repas. » Ils lui en portèrent. 
${}^{9}Et celui-ci goûta l’eau changée en vin. Il ne savait pas d’où venait ce vin, mais ceux qui servaient le savaient bien, eux qui avaient puisé l’eau. Alors le maître du repas appelle le marié 
${}^{10}et lui dit : « Tout le monde sert le bon vin en premier et, lorsque les gens ont bien bu, on apporte le moins bon. Mais toi, tu as gardé le bon vin jusqu’à maintenant. »
${}^{11}Tel fut le commencement des signes que Jésus accomplit. C’était à Cana de Galilée. Il manifesta sa gloire, et ses disciples crurent en lui.
${}^{12}Après cela, il descendit à Capharnaüm avec sa mère, ses frères et ses disciples, et ils demeurèrent là-bas quelques jours.
${}^{13}Comme la Pâque juive était proche, Jésus monta à Jérusalem. 
${}^{14}Dans le Temple, il trouva installés les marchands de bœufs, de brebis et de colombes, et les changeurs. 
${}^{15}Il fit un fouet avec des cordes, et les chassa tous du Temple, ainsi que les brebis et les bœufs ; il jeta par terre la monnaie des changeurs, renversa leurs comptoirs, 
${}^{16}et dit aux marchands de colombes : « Enlevez cela d’ici. Cessez de faire de la maison de mon Père une maison de commerce. » 
${}^{17}Ses disciples se rappelèrent qu’il est écrit : L’amour de ta maison fera mon tourment. 
${}^{18}Des Juifs l’interpellèrent : « Quel signe peux-tu nous donner pour agir ainsi ? » 
${}^{19}Jésus leur répondit : « Détruisez ce sanctuaire, et en trois jours je le relèverai. » 
${}^{20}Les Juifs lui répliquèrent : « Il a fallu quarante-six ans pour bâtir ce sanctuaire, et toi, en trois jours tu le relèverais ! » 
${}^{21}Mais lui parlait du sanctuaire de son corps. 
${}^{22}Aussi, quand il se réveilla d’entre les morts, ses disciples se rappelèrent qu’il avait dit cela ; ils crurent à l’Écriture et à la parole que Jésus avait dite.
${}^{23}Pendant qu’il était à Jérusalem pour la fête de la Pâque, beaucoup crurent en son nom, à la vue des signes qu’il accomplissait. 
${}^{24}Jésus, lui, ne se fiait pas à eux, parce qu’il les connaissait tous 
${}^{25}et n’avait besoin d’aucun témoignage sur l’homme ; lui-même, en effet, connaissait ce qu’il y a dans l’homme.
      
         
      \bchapter{}
      \begin{verse}
${}^{1}Il y avait un homme, un pharisien nommé Nicodème ; c’était un notable parmi les Juifs. 
${}^{2}Il vint trouver Jésus pendant la nuit. Il lui dit : « Rabbi, nous le savons, c’est de la part de Dieu que tu es venu comme un maître qui enseigne, car personne ne peut accomplir les signes que toi, tu accomplis, si Dieu n’est pas avec lui. » 
${}^{3}Jésus lui répondit : « Amen, amen, je te le dis : à moins de naître d’en haut, on ne peut voir le royaume de Dieu. » 
${}^{4}Nicodème lui répliqua : « Comment un homme peut-il naître quand il est vieux ? Peut-il entrer une deuxième fois dans le sein de sa mère et renaître ? » 
${}^{5}Jésus répondit : « Amen, amen, je te le dis : personne, à moins de naître de l’eau et de l’Esprit, ne peut entrer dans le royaume de Dieu. 
${}^{6}Ce qui est né de la chair est chair ; ce qui est né de l’Esprit est esprit. 
${}^{7}Ne sois pas étonné si je t’ai dit : il vous faut naître d’en haut. 
${}^{8}Le vent souffle où il veut : tu entends sa voix, mais tu ne sais ni d’où il vient ni où il va. Il en est ainsi pour qui est né du souffle de l’Esprit. » 
${}^{9}Nicodème reprit : « Comment cela peut-il se faire ? » 
${}^{10}Jésus lui répondit : « Tu es un maître qui enseigne Israël et tu ne connais pas ces choses-là ?
${}^{11}Amen, amen, je te le dis : nous parlons de ce que nous savons, nous témoignons de ce que nous avons vu, et vous ne recevez pas notre témoignage. 
${}^{12}Si vous ne croyez pas lorsque je vous parle des choses de la terre, comment croirez-vous quand je vous parlerai des choses du ciel ? 
${}^{13}Car nul n’est monté au ciel sinon celui qui est descendu du ciel, le Fils de l’homme. 
${}^{14}De même que le serpent de bronze fut élevé par Moïse dans le désert, ainsi faut-il que le Fils de l’homme soit élevé, 
${}^{15}afin qu’en lui tout homme qui croit ait la vie éternelle.
${}^{16}Car Dieu a tellement aimé le monde qu’il a donné son Fils unique, afin que quiconque croit en lui ne se perde pas, mais obtienne la vie éternelle. 
${}^{17}Car Dieu a envoyé son Fils dans le monde, non pas pour juger le monde, mais pour que, par lui, le monde soit sauvé. 
${}^{18}Celui qui croit en lui échappe au Jugement ; celui qui ne croit pas est déjà jugé, du fait qu’il n’a pas cru au nom du Fils unique de Dieu. 
${}^{19}Et le Jugement, le voici : la lumière est venue dans le monde, et les hommes ont préféré les ténèbres à la lumière, parce que leurs œuvres étaient mauvaises. 
${}^{20}Celui qui fait le mal déteste la lumière : il ne vient pas à la lumière, de peur que ses œuvres ne soient dénoncées ; 
${}^{21}mais celui qui fait la vérité vient à la lumière, pour qu’il soit manifeste que ses œuvres ont été accomplies en union avec Dieu. »
${}^{22}Après cela, Jésus se rendit en Judée, ainsi que ses disciples ; il y séjourna avec eux, et il baptisait. 
${}^{23}Jean, quant à lui, baptisait à Aïnone, près de Salim, où l’eau était abondante. On venait là pour se faire baptiser. 
${}^{24}En effet, Jean n’avait pas encore été mis en prison.
${}^{25}Or, il y eut une discussion entre les disciples de Jean et un Juif au sujet des bains de purification. 
${}^{26}Ils allèrent trouver Jean et lui dirent : « Rabbi, celui qui était avec toi de l’autre côté du Jourdain, celui à qui tu as rendu témoignage, le voilà qui baptise, et tous vont à lui ! » 
${}^{27}Jean répondit : « Un homme ne peut rien s’attribuer, sinon ce qui lui est donné du Ciel. 
${}^{28}Vous-mêmes pouvez témoigner que j’ai dit : Moi, je ne suis pas le Christ, mais j’ai été envoyé devant lui. 
${}^{29}Celui à qui l’épouse appartient, c’est l’époux ; quant à l’ami de l’époux, il se tient là, il entend la voix de l’époux, et il en est tout joyeux. Telle est ma joie : elle est parfaite. 
${}^{30}Lui, il faut qu’il grandisse ; et moi, que je diminue.
${}^{31}Celui qui vient d’en haut est au-dessus de tous. Celui qui est de la terre est terrestre, et il parle de façon terrestre. Celui qui vient du ciel est au-dessus de tous, 
${}^{32}il témoigne de ce qu’il a vu et entendu, et personne ne reçoit son témoignage. 
${}^{33}Mais celui qui reçoit son témoignage certifie par là que Dieu est vrai. 
${}^{34}En effet, celui que Dieu a envoyé dit les paroles de Dieu, car Dieu lui donne l’Esprit sans mesure. 
${}^{35}Le Père aime le Fils et il a tout remis dans sa main. 
${}^{36}Celui qui croit au Fils a la vie éternelle ; celui qui refuse de croire le Fils ne verra pas la vie, mais la colère de Dieu demeure sur lui. »
      
         
      \bchapter{}
      \begin{verse}
${}^{1}Les pharisiens avaient entendu dire que Jésus faisait plus de disciples que Jean et qu’il en baptisait davantage. Jésus lui-même en eut connaissance. 
${}^{2}– À vrai dire, ce n’était pas Jésus en personne qui baptisait, mais ses disciples. 
${}^{3}Dès lors, il quitta la Judée pour retourner en Galilée.
${}^{4}Or, il lui fallait traverser la Samarie. 
${}^{5}Il arrive donc à une ville de Samarie, appelée Sykar, près du terrain que Jacob avait donné à son fils Joseph. 
${}^{6}Là se trouvait le puits de Jacob. Jésus, fatigué par la route, s’était donc assis près de la source. C’était la sixième heure, environ midi.
${}^{7}Arrive une femme de Samarie, qui venait puiser de l’eau. Jésus lui dit : « Donne-moi à boire. » 
${}^{8}– En effet, ses disciples étaient partis à la ville pour acheter des provisions. 
${}^{9}La Samaritaine lui dit : « Comment ! Toi, un Juif, tu me demandes à boire, à moi, une Samaritaine ? » – En effet, les Juifs ne fréquentent pas les Samaritains.
${}^{10}Jésus lui répondit : « Si tu savais le don de Dieu et qui est celui qui te dit : “Donne-moi à boire”, c’est toi qui lui aurais demandé, et il t’aurait donné de l’eau vive. » 
${}^{11}Elle lui dit : « Seigneur, tu n’as rien pour puiser, et le puits est profond. D’où as-tu donc cette eau vive ? 
${}^{12}Serais-tu plus grand que notre père Jacob qui nous a donné ce puits, et qui en a bu lui-même, avec ses fils et ses bêtes ? » 
${}^{13}Jésus lui répondit : « Quiconque boit de cette eau aura de nouveau soif ; 
${}^{14}mais celui qui boira de l’eau que moi je lui donnerai n’aura plus jamais soif ; et l’eau que je lui donnerai deviendra en lui une source d’eau jaillissant pour la vie éternelle. » 
${}^{15}La femme lui dit : « Seigneur, donne-moi de cette eau, que je n’aie plus soif, et que je n’aie plus à venir ici pour puiser. »
${}^{16}Jésus lui dit : « Va, appelle ton mari, et reviens. » 
${}^{17}La femme répliqua : « Je n’ai pas de mari. » Jésus reprit : « Tu as raison de dire que tu n’as pas de mari : 
${}^{18}des maris, tu en as eu cinq, et celui que tu as maintenant n’est pas ton mari ; là, tu dis vrai. » 
${}^{19}La femme lui dit : « Seigneur, je vois que tu es un prophète !... 
${}^{20}Eh bien ! Nos pères ont adoré sur la montagne qui est là, et vous, les Juifs, vous dites que le lieu où il faut adorer est à Jérusalem. » 
${}^{21}Jésus lui dit : « Femme, crois-moi : l’heure vient où vous n’irez plus ni sur cette montagne ni à Jérusalem pour adorer le Père. 
${}^{22}Vous, vous adorez ce que vous ne connaissez pas ; nous, nous adorons ce que nous connaissons, car le salut vient des Juifs. 
${}^{23}Mais l’heure vient – et c’est maintenant – où les vrais adorateurs adoreront le Père en esprit et vérité : tels sont les adorateurs que recherche le Père. 
${}^{24}Dieu est esprit, et ceux qui l’adorent, c’est en esprit et vérité qu’ils doivent l’adorer. » 
${}^{25}La femme lui dit : « Je sais qu’il vient, le Messie, celui qu’on appelle Christ. Quand il viendra, c’est lui qui nous fera connaître toutes choses. » 
${}^{26}Jésus lui dit : « Je le suis, moi qui te parle. »
${}^{27}À ce moment-là, ses disciples arrivèrent ; ils étaient surpris de le voir parler avec une femme. Pourtant, aucun ne lui dit : « Que cherches-tu ? » ou bien : « Pourquoi parles-tu avec elle ? » 
${}^{28}La femme, laissant là sa cruche, revint à la ville et dit aux gens : 
${}^{29}« Venez voir un homme qui m’a dit tout ce que j’ai fait. Ne serait-il pas le Christ ? » 
${}^{30}Ils sortirent de la ville, et ils se dirigeaient vers lui.
${}^{31}Entre-temps, les disciples l’appelaient : « Rabbi, viens manger. » 
${}^{32}Mais il répondit : « Pour moi, j’ai de quoi manger : c’est une nourriture que vous ne connaissez pas. » 
${}^{33}Les disciples se disaient entre eux : « Quelqu’un lui aurait-il apporté à manger ? » 
${}^{34}Jésus leur dit : « Ma nourriture, c’est de faire la volonté de Celui qui m’a envoyé et d’accomplir son œuvre. 
${}^{35}Ne dites-vous pas : “Encore quatre mois et ce sera la moisson” ? Et moi, je vous dis : Levez les yeux et regardez les champs déjà dorés pour la moisson. Dès maintenant, 
${}^{36}le moissonneur reçoit son salaire : il récolte du fruit pour la vie éternelle, si bien que le semeur se réjouit en même temps que le moissonneur. 
${}^{37}Il est bien vrai, le dicton : “L’un sème, l’autre moissonne.” 
${}^{38}Je vous ai envoyés moissonner ce qui ne vous a coûté aucun effort ; d’autres ont fait l’effort, et vous en avez bénéficié. »
${}^{39}Beaucoup de Samaritains de cette ville crurent en Jésus, à cause de la parole de la femme qui rendait ce témoignage : « Il m’a dit tout ce que j’ai fait. » 
${}^{40}Lorsqu’ils arrivèrent auprès de lui, ils l’invitèrent à demeurer chez eux. Il y demeura deux jours. 
${}^{41}Ils furent encore beaucoup plus nombreux à croire à cause de sa parole à lui, 
${}^{42}et ils disaient à la femme : « Ce n’est plus à cause de ce que tu nous as dit que nous croyons : nous-mêmes, nous l’avons entendu, et nous savons que c’est vraiment lui le Sauveur du monde. »
${}^{43}Deux jours après, Jésus partit de là pour la Galilée. 
${}^{44}– Lui-même avait témoigné qu’un prophète n’est pas considéré dans son propre pays. 
${}^{45}Il arriva donc en Galilée ; les Galiléens lui firent bon accueil, car ils avaient vu tout ce qu’il avait fait à Jérusalem pendant la fête de la Pâque, puisqu’ils étaient allés eux aussi à cette fête.
${}^{46}Ainsi donc Jésus revint à Cana de Galilée, où il avait changé l’eau en vin. Or, il y avait un fonctionnaire royal, dont le fils était malade à Capharnaüm. 
${}^{47}Ayant appris que Jésus arrivait de Judée en Galilée, il alla le trouver ; il lui demandait de descendre à Capharnaüm pour guérir son fils qui était mourant. 
${}^{48}Jésus lui dit : « Si vous ne voyez pas de signes et de prodiges, vous ne croirez donc pas ! » 
${}^{49}Le fonctionnaire royal lui dit : « Seigneur, descends, avant que mon enfant ne meure ! » 
${}^{50}Jésus lui répond : « Va, ton fils est vivant. » L’homme crut à la parole que Jésus lui avait dite et il partit. 
${}^{51}Pendant qu’il descendait, ses serviteurs arrivèrent à sa rencontre et lui dirent que son enfant était vivant. 
${}^{52}Il voulut savoir à quelle heure il s’était trouvé mieux. Ils lui dirent : « C’est hier, à la septième heure, (au début de l’après-midi), que la fièvre l’a quitté. » 
${}^{53}Le père se rendit compte que c’était justement l’heure où Jésus lui avait dit : « Ton fils est vivant. » Alors il crut, lui, ainsi que tous les gens de sa maison.
${}^{54}Tel fut le second signe que Jésus accomplit lorsqu’il revint de Judée en Galilée.
      <h2 class="intertitle" id="d85e360989">3. Jésus et les grandes fêtes juives (5 – 10)</h2>
      
         
      \bchapter{}
      \begin{verse}
${}^{1}Après cela, il y eut une fête juive, et Jésus monta à Jérusalem. 
${}^{2}Or, à Jérusalem, près de la porte des Brebis, il existe une piscine qu’on appelle en hébreu Bethzatha. Elle a cinq colonnades, 
${}^{3}sous lesquelles étaient couchés une foule de malades, aveugles, boiteux et impotents. [3b-4] 
${}^{5}Il y avait là un homme qui était malade depuis trente-huit ans. 
${}^{6}Jésus, le voyant couché là, et apprenant qu’il était dans cet état depuis longtemps, lui dit : « Veux-tu être guéri ? » 
${}^{7}Le malade lui répondit : « Seigneur, je n’ai personne pour me plonger dans la piscine au moment où l’eau bouillonne ; et pendant que j’y vais, un autre descend avant moi. » 
${}^{8}Jésus lui dit : « Lève-toi, prends ton brancard, et marche. » 
${}^{9}Et aussitôt l’homme fut guéri. Il prit son brancard : il marchait !
      Or, ce jour-là était un jour de sabbat. 
${}^{10}Les Juifs dirent donc à cet homme que Jésus avait remis sur pieds : « C’est le sabbat ! Il ne t’est pas permis de porter ton brancard. » 
${}^{11}Il leur répliqua : « Celui qui m’a guéri, c’est lui qui m’a dit : “Prends ton brancard, et marche !” » 
${}^{12}Ils l’interrogèrent : « Quel est l’homme qui t’a dit : “Prends ton brancard, et marche” ? » 
${}^{13}Mais celui qui avait été rétabli ne savait pas qui c’était ; en effet, Jésus s’était éloigné, car il y avait foule à cet endroit.
${}^{14}Plus tard, Jésus le retrouve dans le Temple et lui dit : « Te voilà guéri. Ne pèche plus, il pourrait t’arriver quelque chose de pire. » 
${}^{15}L’homme partit annoncer aux Juifs que c’était Jésus qui l’avait guéri. 
${}^{16}Et ceux-ci persécutaient Jésus parce qu’il avait fait cela le jour du sabbat.
${}^{17}Jésus leur déclara : « Mon Père est toujours à l’œuvre, et moi aussi, je suis à l’œuvre. » 
${}^{18}C’est pourquoi, de plus en plus, les Juifs cherchaient à le tuer, car non seulement il ne respectait pas le sabbat, mais encore il disait que Dieu était son propre Père, et il se faisait ainsi l’égal de Dieu.
${}^{19}Jésus reprit donc la parole. Il leur déclarait : « Amen, amen, je vous le dis : le Fils ne peut rien faire de lui-même, il fait seulement ce qu’il voit faire par le Père ; ce que fait celui-ci, le Fils le fait pareillement. 
${}^{20}Car le Père aime le Fils et lui montre tout ce qu’il fait. Il lui montrera des œuvres plus grandes encore, si bien que vous serez dans l’étonnement. 
${}^{21}Comme le Père, en effet, relève les morts et les fait vivre, ainsi le Fils, lui aussi, fait vivre qui il veut. 
${}^{22}Car le Père ne juge personne : il a donné au Fils tout pouvoir pour juger, 
${}^{23}afin que tous honorent le Fils comme ils honorent le Père. Celui qui ne rend pas honneur au Fils ne rend pas non plus honneur au Père, qui l’a envoyé. 
${}^{24}Amen, amen, je vous le dis : qui écoute ma parole et croit en Celui qui m’a envoyé, obtient la vie éternelle et il échappe au jugement, car déjà il passe de la mort à la vie.
${}^{25}Amen, amen, je vous le dis : l’heure vient – et c’est maintenant – où les morts entendront la voix du Fils de Dieu, et ceux qui l’auront entendue vivront. 
${}^{26}Comme le Père, en effet, a la vie en lui-même, ainsi a-t-il donné au Fils d’avoir, lui aussi, la vie en lui-même ; 
${}^{27}et il lui a donné pouvoir d’exercer le jugement, parce qu’il est le Fils de l’homme. 
${}^{28}Ne soyez pas étonnés ; l’heure vient où tous ceux qui sont dans les tombeaux entendront sa voix ; 
${}^{29}alors, ceux qui ont fait le bien sortiront pour ressusciter et vivre, ceux qui ont fait le mal, pour ressusciter et être jugés.
${}^{30}Moi, je ne peux rien faire de moi-même ; je rends mon jugement d’après ce que j’entends, et mon jugement est juste, parce que je ne cherche pas à faire ma volonté, mais la volonté de Celui qui m’a envoyé.
${}^{31}Si c’est moi qui me rends témoignage, mon témoignage n’est pas vrai ; 
${}^{32}c’est un autre qui me rend témoignage, et je sais que le témoignage qu’il me rend est vrai. 
${}^{33}Vous avez envoyé une délégation auprès de Jean le Baptiste, et il a rendu témoignage à la vérité. 
${}^{34}Moi, ce n’est pas d’un homme que je reçois le témoignage, mais je parle ainsi pour que vous soyez sauvés. 
${}^{35}Jean était la lampe qui brûle et qui brille, et vous avez voulu vous réjouir un moment à sa lumière. 
${}^{36}Mais j’ai pour moi un témoignage plus grand que celui de Jean : ce sont les œuvres que le Père m’a donné d’accomplir ; les œuvres mêmes que je fais témoignent que le Père m’a envoyé. 
${}^{37}Et le Père qui m’a envoyé, lui, m’a rendu témoignage. Vous n’avez jamais entendu sa voix, vous n’avez jamais vu sa face, 
${}^{38}et vous ne laissez pas sa parole demeurer en vous, puisque vous ne croyez pas en celui que le Père a envoyé. 
${}^{39}Vous scrutez les Écritures parce que vous pensez y trouver la vie éternelle ; or, ce sont les Écritures qui me rendent témoignage, 
${}^{40}et vous ne voulez pas venir à moi pour avoir la vie !
${}^{41}La gloire, je ne la reçois pas des hommes ; 
${}^{42}d’ailleurs je vous connais : vous n’avez pas en vous l’amour de Dieu. 
${}^{43}Moi, je suis venu au nom de mon Père, et vous ne me recevez pas ; qu’un autre vienne en son propre nom, celui-là, vous le recevrez ! 
${}^{44}Comment pourriez-vous croire, vous qui recevez votre gloire les uns des autres, et qui ne cherchez pas la gloire qui vient du Dieu unique ?
${}^{45}Ne pensez pas que c’est moi qui vous accuserai devant le Père. Votre accusateur, c’est Moïse, en qui vous avez mis votre espérance. 
${}^{46}Si vous croyiez Moïse, vous me croiriez aussi, car c’est à mon sujet qu’il a écrit. 
${}^{47}Mais si vous ne croyez pas ses écrits, comment croirez-vous mes paroles ? »
      
         
      \bchapter{}
      \begin{verse}
${}^{1}Après cela, Jésus passa de l’autre côté de la mer de Galilée, le lac de Tibériade. 
${}^{2}Une grande foule le suivait, parce qu’elle avait vu les signes qu’il accomplissait sur les malades. 
${}^{3}Jésus gravit la montagne, et là, il était assis avec ses disciples. 
${}^{4}Or, la Pâque, la fête des Juifs, était proche.
${}^{5}Jésus leva les yeux et vit qu’une foule nombreuse venait à lui. Il dit à Philippe : « Où pourrions-nous acheter du pain pour qu’ils aient à manger ? » 
${}^{6}Il disait cela pour le mettre à l’épreuve, car il savait bien, lui, ce qu’il allait faire. 
${}^{7}Philippe lui répondit : « Le salaire de deux cents journées ne suffirait pas pour que chacun reçoive un peu de pain. » 
${}^{8}Un de ses disciples, André, le frère de Simon-Pierre, lui dit : 
${}^{9}« Il y a là un jeune garçon qui a cinq pains d’orge et deux poissons, mais qu’est-ce que cela pour tant de monde ! » 
${}^{10}Jésus dit : « Faites asseoir les gens. » Il y avait beaucoup d’herbe à cet endroit. Ils s’assirent donc, au nombre d’environ cinq mille hommes.
${}^{11}Alors Jésus prit les pains et, après avoir rendu grâce, il les distribua aux convives ; il leur donna aussi du poisson, autant qu’ils en voulaient. 
${}^{12}Quand ils eurent mangé à leur faim, il dit à ses disciples : « Rassemblez les morceaux en surplus, pour que rien ne se perde. » 
${}^{13}Ils les rassemblèrent, et ils remplirent douze paniers avec les morceaux des cinq pains d’orge, restés en surplus pour ceux qui prenaient cette nourriture. 
${}^{14}À la vue du signe que Jésus avait accompli, les gens disaient : « C’est vraiment lui le Prophète annoncé, celui qui vient dans le monde. » 
${}^{15}Mais Jésus savait qu’ils allaient venir l’enlever pour faire de lui leur roi ; alors de nouveau il se retira dans la montagne, lui seul.
${}^{16}Le soir venu, ses disciples descendirent jusqu’à la mer. 
${}^{17}Ils s’embarquèrent pour gagner Capharnaüm, sur l’autre rive. C’était déjà les ténèbres, et Jésus n’avait pas encore rejoint les disciples. 
${}^{18}Un grand vent soufflait, et la mer était agitée. 
${}^{19}Les disciples avaient ramé sur une distance de vingt-cinq ou trente stades (c’est-à-dire environ cinq mille mètres), lorsqu’ils virent Jésus qui marchait sur la mer et se rapprochait de la barque. Alors, ils furent saisis de peur. 
${}^{20}Mais il leur dit : « C’est moi. N’ayez plus peur. » 
${}^{21}Les disciples voulaient le prendre dans la barque ; aussitôt, la barque toucha terre là où ils se rendaient.
${}^{22}Le lendemain, la foule restée sur l’autre rive se rendit compte qu’il n’y avait eu là qu’une seule barque, et que Jésus n’y était pas monté avec ses disciples, qui étaient partis sans lui. 
${}^{23}Cependant, d’autres barques, venant de Tibériade, étaient arrivées près de l’endroit où l’on avait mangé le pain après que le Seigneur eut rendu grâce. 
${}^{24}Quand la foule vit que Jésus n’était pas là, ni ses disciples, les gens montèrent dans les barques et se dirigèrent vers Capharnaüm à la recherche de Jésus. 
${}^{25}L’ayant trouvé sur l’autre rive, ils lui dirent : « Rabbi, quand es-tu arrivé ici ? »
${}^{26}Jésus leur répondit : « Amen, amen, je vous le dis : vous me cherchez, non parce que vous avez vu des signes, mais parce que vous avez mangé de ces pains et que vous avez été rassasiés. 
${}^{27}Travaillez non pas pour la nourriture qui se perd, mais pour la nourriture qui demeure jusque dans la vie éternelle, celle que vous donnera le Fils de l’homme, lui que Dieu, le Père, a marqué de son sceau. » 
${}^{28}Ils lui dirent alors : « Que devons-nous faire pour travailler aux œuvres de Dieu ? » 
${}^{29}Jésus leur répondit : « L’œuvre de Dieu, c’est que vous croyiez en celui qu’il a envoyé. »
${}^{30}Ils lui dirent alors : « Quel signe vas-tu accomplir pour que nous puissions le voir, et te croire ? Quelle œuvre vas-tu faire ? 
${}^{31}Au désert, nos pères ont mangé la manne ; comme dit l’Écriture : Il leur a donné à manger le pain venu du ciel. » 
${}^{32}Jésus leur répondit : « Amen, amen, je vous le dis : ce n’est pas Moïse qui vous a donné le pain venu du ciel ; c’est mon Père qui vous donne le vrai pain venu du ciel. 
${}^{33}Car le pain de Dieu, c’est celui qui descend du ciel et qui donne la vie au monde. » 
${}^{34}Ils lui dirent alors : « Seigneur, donne-nous toujours de ce pain-là. » 
${}^{35}Jésus leur répondit : « Moi, je suis le pain de la vie. Celui qui vient à moi n’aura jamais faim ; celui qui croit en moi n’aura jamais soif. 
${}^{36}Mais je vous l’ai déjà dit : vous avez vu, et pourtant vous ne croyez pas. 
${}^{37}Tous ceux que me donne le Père viendront jusqu’à moi ; et celui qui vient à moi, je ne vais pas le jeter dehors. 
${}^{38}Car je suis descendu du ciel pour faire non pas ma volonté, mais la volonté de Celui qui m’a envoyé. 
${}^{39}Or, telle est la volonté de Celui qui m’a envoyé : que je ne perde aucun de ceux qu’il m’a donnés, mais que je les ressuscite au dernier jour. 
${}^{40}Telle est la volonté de mon Père : que celui qui voit le Fils et croit en lui ait la vie éternelle ; et moi, je le ressusciterai au dernier jour. »
${}^{41}Les Juifs récriminaient contre Jésus parce qu’il avait déclaré : « Moi, je suis le pain qui est descendu du ciel. » 
${}^{42}Ils disaient : « Celui-là n’est-il pas Jésus, fils de Joseph ? Nous connaissons bien son père et sa mère. Alors comment peut-il dire maintenant : “Je suis descendu du ciel” ? » 
${}^{43}Jésus reprit la parole : « Ne récriminez pas entre vous. 
${}^{44}Personne ne peut venir à moi, si le Père qui m’a envoyé ne l’attire, et moi, je le ressusciterai au dernier jour. 
${}^{45}Il est écrit dans les prophètes : Ils seront tous instruits par Dieu lui-même. Quiconque a entendu le Père et reçu son enseignement vient à moi. 
${}^{46}Certes, personne n’a jamais vu le Père, sinon celui qui vient de Dieu : celui-là seula vu le Père. 
${}^{47}Amen, amen, je vous le dis : il a la vie éternelle, celui qui croit.
${}^{48}Moi, je suis le pain de la vie. 
${}^{49}Au désert, vos pères ont mangé la manne, et ils sont morts ; 
${}^{50}mais le pain qui descend du ciel est tel que celui qui en mange ne mourra pas.
${}^{51}Moi, je suis le pain vivant, qui est descendu du ciel : si quelqu’un mange de ce pain, il vivra éternellement. Le pain que je donnerai, c’est ma chair, donnée pour la vie du monde. »
${}^{52}Les Juifs se querellaient entre eux : « Comment celui-là peut-il nous donner sa chair à manger ? » 
${}^{53}Jésus leur dit alors : « Amen, amen, je vous le dis : si vous ne mangez pas la chair du Fils de l’homme, et si vous ne buvez pas son sang, vous n’avez pas la vie en vous. 
${}^{54}Celui qui mange ma chair et boit mon sang a la vie éternelle ; et moi, je le ressusciterai au dernier jour. 
${}^{55}En effet, ma chair est la vraie nourriture, et mon sang est la vraie boisson. 
${}^{56}Celui qui mange ma chair et boit mon sang demeure en moi, et moi, je demeure en lui. 
${}^{57}De même que le Père, qui est vivant, m’a envoyé, et que moi je vis par le Père, de même celui qui me mange, lui aussi vivra par moi. 
${}^{58}Tel est le pain qui est descendu du ciel : il n’est pas comme celui que les pères ont mangé. Eux, ils sont morts ; celui qui mange ce pain vivra éternellement. »
${}^{59}Voilà ce que Jésus a dit, alors qu’il enseignait à la synagogue de Capharnaüm.
${}^{60}Beaucoup de ses disciples, qui avaient entendu, déclarèrent : « Cette parole est rude ! Qui peut l’entendre ? » 
${}^{61}Jésus savait en lui-même que ses disciples récriminaient à son sujet. Il leur dit : « Cela vous scandalise ? 
${}^{62}Et quand vous verrez le Fils de l’homme monter là où il était auparavant !... 
${}^{63}C’est l’esprit qui fait vivre, la chair n’est capable de rien. Les paroles que je vous ai dites sont esprit et elles sont vie. 
${}^{64}Mais il y en a parmi vous qui ne croient pas. » Jésus savait en effet depuis le commencement quels étaient ceux qui ne croyaient pas, et qui était celui qui le livrerait. 
${}^{65}Il ajouta : « Voilà pourquoi je vous ai dit que personne ne peut venir à moi si cela ne lui est pas donné par le Père. »
${}^{66}À partir de ce moment, beaucoup de ses disciples s’en retournèrent et cessèrent de l’accompagner. 
${}^{67}Alors Jésus dit aux Douze : « Voulez-vous partir, vous aussi ? » 
${}^{68}Simon-Pierre lui répondit : « Seigneur, à qui irions-nous ? Tu as les paroles de la vie éternelle. 
${}^{69}Quant à nous, nous croyons, et nous savons que tu es le Saint de Dieu. » 
${}^{70}Jésus leur dit : « N’est-ce pas moi qui vous ai choisis, vous, les Douze ? Et l’un de vous est un diable ! » 
${}^{71}Il parlait de Judas, fils de Simon Iscariote ; celui-ci, en effet, l’un des Douze, allait le livrer.
      
         
      \bchapter{}
      \begin{verse}
${}^{1}Après cela, Jésus parcourait la Galilée : il ne voulait pas parcourir la Judée car les Juifs cherchaient à le tuer. 
${}^{2}La fête juive des Tentes était proche.
${}^{3}Alors les frères de Jésus lui dirent : « Ne reste pas ici, va en Judée pour que tes disciples aussi voient les œuvres que tu fais. 
${}^{4}On n’agit pas en secret quand on veut être un personnage public. Puisque tu fais de telles choses, il faut te manifester au monde. » 
${}^{5}En effet, les frères de Jésus eux-mêmes ne croyaient pas en lui. 
${}^{6}Jésus leur dit alors : « Pour moi, le moment n’est pas encore venu, mais pour vous, c’est toujours le bon moment. 
${}^{7}Le monde ne peut pas vous haïr, mais il a de la haine contre moi parce que je témoigne que ses œuvres sont mauvaises. 
${}^{8}Vous autres, montez à la fête ; moi, je ne monte pas à cette fête parce que mon temps n’est pas encore accompli. » 
${}^{9}Cela dit, il demeura en Galilée. 
${}^{10}Lorsque ses frères furent montés à Jérusalem pour la fête, il y monta lui aussi, non pas ostensiblement, mais en secret.
${}^{11}Les Juifs le cherchaient pendant la fête, en disant : « Où donc est-il ? » 
${}^{12}On discutait beaucoup à son sujet dans la foule. Tandis que les uns disaient : « C’est un homme de bien », d’autres répliquaient : « Mais non, il égare la foule. » 
${}^{13}Toutefois, personne ne parlait ouvertement de lui, par crainte des Juifs.
${}^{14}On était déjà au milieu de la semaine de la fête quand Jésus monta au Temple ; et là il enseignait. 
${}^{15}Les Juifs s’étonnaient et disaient : « Comment est-il instruit sans avoir étudié ? » 
${}^{16}Jésus leur répondit : « Mon enseignement n’est pas de moi, mais de Celui qui m’a envoyé. 
${}^{17}Quelqu’un veut-il faire la volonté de Dieu, il saura si cet enseignement vient de Dieu, ou si je parle de ma propre initiative. 
${}^{18}Si quelqu’un parle de sa propre initiative, il cherche sa gloire personnelle ; mais si quelqu’un cherche la gloire de celui qui l’a envoyé, celui-là est vrai et il n’y a pas d’imposture en lui. 
${}^{19}Moïse ne vous a-t-il pas donné la Loi ? Et aucun de vous ne met la Loi en pratique. Pourquoi cherchez-vous à me tuer ? »
${}^{20}La foule répondit : « Tu as un démon. Qui donc cherche à te tuer ? » 
${}^{21}Jésus leur répondit : « Pour une seule œuvre que j’ai faite, vous voilà tous dans l’étonnement. 
${}^{22}Moïse vous a donné la circoncision – en fait elle ne vient pas de Moïse, mais des patriarches –, et vous la pratiquez même le jour du sabbat. 
${}^{23}Eh bien ! Si, le jour du sabbat, un homme peut recevoir la circoncision afin que la loi de Moïse soit respectée, pourquoi vous emporter contre moi parce que j’ai guéri un homme tout entier le jour du sabbat ? 
${}^{24}Ne jugez pas d’après l’apparence, mais jugez selon la justice. »
${}^{25}Quelques habitants de Jérusalem disaient alors : « N’est-ce pas celui qu’on cherche à tuer ? 
${}^{26}Le voilà qui parle ouvertement, et personne ne lui dit rien ! Nos chefs auraient-ils vraiment reconnu que c’est lui le Christ ? 
${}^{27}Mais lui, nous savons d’où il est. Or, le Christ, quand il viendra, personne ne saura d’où il est. » 
${}^{28}Jésus, qui enseignait dans le Temple, s’écria : « Vous me connaissez ? Et vous savez d’où je suis ? Je ne suis pas venu de moi-même : mais il est véridique, Celui qui m’a envoyé, lui que vous ne connaissez pas. 
${}^{29}Moi, je le connais parce que je viens d’auprès de lui, et c’est lui qui m’a envoyé. »
${}^{30}On cherchait à l’arrêter, mais personne ne mit la main sur lui parce que son heure n’était pas encore venue. 
${}^{31}Dans la foule beaucoup crurent en lui, et ils disaient : « Le Christ, quand il viendra, accomplira-t-il plus de signes que celui-ci n’en a fait ? » 
${}^{32}Les pharisiens entendirent la foule discuter ainsi à son propos. Alors les grands prêtres et les pharisiens envoyèrent des gardes pour l’arrêter. 
${}^{33}Jésus déclara : « Pour un peu de temps encore, je suis avec vous ; puis je m’en vais auprès de Celui qui m’a envoyé. 
${}^{34}Vous me chercherez, et vous ne me trouverez pas ; et là où je suis, vous ne pouvez pas venir. » 
${}^{35}Les Juifs se dirent alors entre eux : « Où va-t-il bien partir pour que nous ne le trouvions pas ? Va-t-il partir chez les nôtres dispersés dans le monde grec, afin d’instruire les Grecs ? 
${}^{36}Que signifie cette parole qu’il a dite : “Vous me chercherez, et vous ne me trouverez pas, et là où je suis, vous ne pouvez pas venir” ? »
${}^{37}Au jour solennel où se terminait la fête, Jésus, debout, s’écria : « Si quelqu’un a soif, qu’il vienne à moi, et qu’il boive, 
${}^{38}celui qui croit en moi ! Comme dit l’Écriture : De son cœur couleront des fleuves d’eau vive. » 
${}^{39}En disant cela, il parlait de l’Esprit Saint qu’allaient recevoir ceux qui croiraient en lui. En effet, il ne pouvait y avoir l’Esprit, puisque Jésus n’avait pas encore été glorifié. 
${}^{40}Dans la foule, on avait entendu ses paroles, et les uns disaient : « C’est vraiment lui, le Prophète annoncé ! » 
${}^{41}D’autres disaient : « C’est lui le Christ ! » Mais d’autres encore demandaient : « Le Christ peut-il venir de Galilée ? 
${}^{42}L’Écriture ne dit-elle pas que c’est de la descendance de David et de Bethléem, le village de David, que vient le Christ ? » 
${}^{43}C’est ainsi que la foule se divisa à cause de lui. 
${}^{44}Quelques-uns d’entre eux voulaient l’arrêter, mais personne ne mit la main sur lui.
${}^{45}Les gardes revinrent auprès des grands prêtres et des pharisiens, qui leur demandèrent : « Pourquoi ne l’avez-vous pas amené ? » 
${}^{46}Les gardes répondirent : « Jamais un homme n’a parlé de la sorte ! » 
${}^{47}Les pharisiens leur répliquèrent : « Alors, vous aussi, vous vous êtes laissé égarer ? 
${}^{48}Parmi les chefs du peuple et les pharisiens, y en a-t-il un seul qui ait cru en lui ? 
${}^{49}Quant à cette foule qui ne sait rien de la Loi, ce sont des maudits ! »
${}^{50}Nicodème, l’un d’entre eux, celui qui était allé précédemment trouver Jésus, leur dit : 
${}^{51}« Notre Loi permet-elle de juger un homme sans l’entendre d’abord pour savoir ce qu’il a fait ? » 
${}^{52}Ils lui répondirent : « Serais-tu, toi aussi, de Galilée ? Cherche bien, et tu verras que jamais aucun prophète ne surgit de Galilée ! »
${}^{53}Puis ils s’en allèrent chacun chez soi.
      
         
      \bchapter{}
      \begin{verse}
${}^{1}Quant à Jésus, il s’en alla au mont des Oliviers.
      
         
${}^{2}Dès l’aurore, il retourna au Temple. Comme tout le peuple venait à lui, il s’assit et se mit à enseigner. 
${}^{3}Les scribes et les pharisiens lui amènent une femme qu’on avait surprise en situation d’adultère. Ils la mettent au milieu, 
${}^{4}et disent à Jésus : « Maître, cette femme a été surprise en flagrant délit d’adultère. 
${}^{5}Or, dans la Loi, Moïse nous a ordonné de lapider ces femmes-là. Et toi, que dis-tu ? » 
${}^{6}Ils parlaient ainsi pour le mettre à l’épreuve, afin de pouvoir l’accuser. Mais Jésus s’était baissé et, du doigt, il écrivait sur la terre. 
${}^{7}Comme on persistait à l’interroger, il se redressa et leur dit : « Celui d’entre vous qui est sans péché, qu’il soit le premier à lui jeter une pierre. » 
${}^{8}Il se baissa de nouveau et il écrivait sur la terre. 
${}^{9}Eux, après avoir entendu cela, s’en allaient un par un, en commençant par les plus âgés. Jésus resta seul avec la femme toujours là au milieu. 
${}^{10}Il se redressa et lui demanda : « Femme, où sont-ils donc ? Personne ne t’a condamnée ? » 
${}^{11}Elle répondit : « Personne, Seigneur. » Et Jésus lui dit : « Moi non plus, je ne te condamne pas. Va, et désormais ne pèche plus. »
${}^{12}De nouveau, Jésus leur parla : « Moi, je suis la lumière du monde. Celui qui me suit ne marchera pas dans les ténèbres, il aura la lumière de la vie. » 
${}^{13}Les pharisiens lui dirent alors : « Tu te rends témoignage à toi-même, ce n’est donc pas un vrai témoignage » 
${}^{14}Jésus leur répondit : « Oui, moi, je me rends témoignage à moi-même, et pourtant mon témoignage est vrai, car je sais d’où je suis venu, et où je vais ; mais vous, vous ne savez ni d’où je viens, ni où je vais. 
${}^{15}Vous, vous jugez de façon purement humaine. Moi, je ne juge personne. 
${}^{16}Et, s’il m’arrive de juger, mon jugement est vrai parce que je ne suis pas seul : j’ai avec moi le Père, qui m’a envoyé. 
${}^{17}Or, il est écrit dans votre Loi que, s’il y a deux témoins, c’est un vrai témoignage. 
${}^{18}Moi, je suis à moi-même mon propre témoin, et le Père, qui m’a envoyé, témoigne aussi pour moi. » 
${}^{19}Les pharisiens lui disaient : « Où est-il, ton père ? » Jésus répondit : « Vous ne connaissez ni moi ni mon Père ; si vous me connaissiez, vous connaîtriez aussi mon Père. » 
${}^{20}Il prononça ces paroles alors qu’il enseignait dans le Temple, à la salle du Trésor. Et personne ne l’arrêta, parce que son heure n’était pas encore venue.
${}^{21}Jésus leur dit encore : « Je m’en vais ; vous me chercherez, et vous mourrez dans votre péché. Là où moi je vais, vous ne pouvez pas aller. » 
${}^{22}Les Juifs disaient : « Veut-il donc se donner la mort, puisqu’il dit : “Là où moi je vais, vous ne pouvez pas aller” ? » 
${}^{23}Il leur répondit : « Vous, vous êtes d’en bas ; moi, je suis d’en haut. Vous, vous êtes de ce monde ; moi, je ne suis pas de ce monde. 
${}^{24}C’est pourquoi je vous ai dit que vous mourrez dans vos péchés. En effet, si vous ne croyez pas que moi, Je suis, vous mourrez dans vos péchés. »
${}^{25}Alors, ils lui demandaient : « Toi, qui es-tu ? » Jésus leur répondit : « Je n’ai pas cessé de vous le dire. 
${}^{26}À votre sujet, j’ai beaucoup à dire et à juger. D’ailleurs Celui qui m’a envoyé dit la vérité, et ce que j’ai entendu de lui, je le dis pour le monde. » 
${}^{27}Ils ne comprirent pas qu’il leur parlait du Père. 
${}^{28}Jésus leur déclara : « Quand vous aurez élevé le Fils de l’homme, alors vous comprendrez que moi, Je suis, et que je ne fais rien de moi-même ; ce que je dis là, je le dis comme le Père me l’a enseigné. 
${}^{29}Celui qui m’a envoyé est avec moi ; il ne m’a pas laissé seul, parce que je fais toujours ce qui lui est agréable. »
${}^{30}Sur ces paroles de Jésus, beaucoup crurent en lui.
${}^{31}Jésus disait à ceux des Juifs qui croyaient en lui : « Si vous demeurez fidèles à ma parole, vous êtes vraiment mes disciples ; 
${}^{32}alors vous connaîtrez la vérité, et la vérité vous rendra libres. » 
${}^{33}Ils lui répliquèrent : « Nous sommes la descendance d’Abraham, et nous n’avons jamais été les esclaves de personne. Comment peux-tu dire : “Vous deviendrez libres” ? »
${}^{34}Jésus leur répondit : « Amen, amen, je vous le dis : qui commet le péché est esclave du péché. 
${}^{35}L’esclave ne demeure pas pour toujours dans la maison ; le fils, lui, y demeure pour toujours. 
${}^{36}Si donc le Fils vous rend libres, réellement vous serez libres. 
${}^{37}Je sais bien que vous êtes la descendance d’Abraham, et pourtant vous cherchez à me tuer, parce que ma parole ne trouve pas sa place en vous. 
${}^{38}Je dis ce que moi, j’ai vu auprès de mon Père, et vous aussi, vous faites ce que vous avez entendu chez votre père. »
${}^{39}Ils lui répliquèrent : « Notre père, c’est Abraham. » Jésus leur dit : « Si vous étiez les enfants d’Abraham, vous feriez les œuvres d’Abraham. 
${}^{40}Mais maintenant, vous cherchez à me tuer, moi, un homme qui vous ai dit la vérité que j’ai entendue de Dieu. Cela, Abraham ne l’a pas fait. 
${}^{41}Vous, vous faites les œuvres de votre père. »
      Ils lui dirent : « Nous ne sommes pas nés de la prostitution ! Nous n’avons qu’un seul Père : c’est Dieu. » 
${}^{42}Jésus leur dit : « Si Dieu était votre Père, vous m’aimeriez, car moi, c’est de Dieu que je suis sorti et que je viens. Je ne suis pas venu de moi-même ; c’est lui qui m’a envoyé. 
${}^{43}Pourquoi ne comprenez-vous pas mon langage ? – C’est que vous n’êtes pas capables d’entendre ma parole. 
${}^{44}Vous, vous êtes du diable, c’est lui votre père, et vous cherchez à réaliser les convoitises de votre père. Depuis le commencement, il a été un meurtrier. Il ne s’est pas tenu dans la vérité, parce qu’il n’y a pas en lui de vérité. Quand il dit le mensonge, il le tire de lui-même, parce qu’il est menteur et père du mensonge. 
${}^{45}Mais moi, parce que je dis la vérité, vous ne me croyez pas. 
${}^{46}Qui d’entre vous pourrait faire la preuve que j’ai péché ? Si je dis la vérité, pourquoi ne me croyez-vous pas ? 
${}^{47}Celui qui est de Dieu écoute les paroles de Dieu. Et vous, si vous n’écoutez pas, c’est que vous n’êtes pas de Dieu. »
${}^{48}Les Juifs répliquèrent : « N’avons-nous pas raison de dire que tu es un Samaritain et que tu as un démon ? » 
${}^{49}Jésus répondit : « Non, je n’ai pas de démon. Au contraire, j’honore mon Père, et vous, vous refusez de m’honorer. 
${}^{50}Ce n’est pas moi qui recherche ma gloire, il y en a un qui la recherche, et qui juge. 
${}^{51}Amen, amen, je vous le dis : si quelqu’un garde ma parole, jamais il ne verra la mort. »
${}^{52}Les Juifs lui dirent : « Maintenant nous savons bien que tu as un démon. Abraham est mort, les prophètes aussi, et toi, tu dis : “Si quelqu’un garde ma parole, il ne connaîtra jamais la mort.” 
${}^{53}Es-tu donc plus grand que notre père Abraham ? Il est mort, et les prophètes aussi sont morts. Pour qui te prends-tu ? » 
${}^{54}Jésus répondit : « Si je me glorifie moi-même, ma gloire n’est rien ; c’est mon Père qui me glorifie, lui dont vous dites : “Il est notre Dieu”, 
${}^{55}alors que vous ne le connaissez pas. Moi, je le connais et, si je dis que je ne le connais pas, je serai comme vous, un menteur. Mais je le connais, et sa parole, je la garde. 
${}^{56}Abraham votre père a exulté, sachant qu’il verrait mon Jour. Il l’a vu, et il s’est réjoui. »
${}^{57}Les Juifs lui dirent alors : « Toi qui n’as pas encore cinquante ans, tu as vu Abraham ! » 
${}^{58}Jésus leur répondit : « Amen, amen, je vous le dis : avant qu’Abraham fût, moi, Je suis. »
${}^{59}Alors ils ramassèrent des pierres pour les lui jeter. Mais Jésus, en se cachant, sortit du Temple.
      
         
      \bchapter{}
      \begin{verse}
${}^{1}En passant, Jésus vit un homme aveugle de naissance.
      \begin{verse}
${}^{2}Ses disciples l’interrogèrent : « Rabbi, qui a péché, lui ou ses parents, pour qu’il soit né aveugle ? » 
${}^{3}Jésus répondit : « Ni lui, ni ses parents n’ont péché. Mais c’était pour que les œuvres de Dieu se manifestent en lui. 
${}^{4}Il nous faut travailler aux œuvres de Celui qui m’a envoyé, tant qu’il fait jour ; la nuit vient où personne ne pourra plus y travailler. 
${}^{5}Aussi longtemps que je suis dans le monde, je suis la lumière du monde. » 
${}^{6}Cela dit, il cracha à terre et, avec la salive, il fit de la boue ; puis il appliqua la boue sur les yeux de l’aveugle, 
${}^{7}et lui dit : « Va te laver à la piscine de Siloé » – ce nom se traduit : Envoyé. L’aveugle y alla donc, et il se lava ; quand il revint, il voyait.
${}^{8}Ses voisins, et ceux qui l’avaient observé auparavant – car il était mendiant – dirent alors : « N’est-ce pas celui qui se tenait là pour mendier ? » 
${}^{9}Les uns disaient : « C’est lui. » Les autres disaient : « Pas du tout, c’est quelqu’un qui lui ressemble. » Mais lui disait : « C’est bien moi. » 
${}^{10}Et on lui demandait : « Alors, comment tes yeux se sont-ils ouverts ? » 
${}^{11}Il répondit : « L’homme qu’on appelle Jésus a fait de la boue, il me l’a appliquée sur les yeux et il m’a dit : “Va à Siloé et lave-toi.” J’y suis donc allé et je me suis lavé ; alors, j’ai vu. » 
${}^{12}Ils lui dirent : « Et lui, où est-il ? » Il répondit : « Je ne sais pas. »
${}^{13}On l’amène aux pharisiens, lui, l’ancien aveugle. 
${}^{14}Or, c’était un jour de sabbat que Jésus avait fait de la boue et lui avait ouvert les yeux. 
${}^{15}À leur tour, les pharisiens lui demandaient comment il pouvait voir. Il leur répondit : « Il m’a mis de la boue sur les yeux, je me suis lavé, et je vois. » 
${}^{16}Parmi les pharisiens, certains disaient : « Cet homme-là n’est pas de Dieu, puisqu’il n’observe pas le repos du sabbat. » D’autres disaient : « Comment un homme pécheur peut-il accomplir des signes pareils ? » Ainsi donc ils étaient divisés. 
${}^{17}Alors ils s’adressent de nouveau à l’aveugle : « Et toi, que dis-tu de lui, puisqu’il t’a ouvert les yeux ? » Il dit : « C’est un prophète. »
${}^{18}Or, les Juifs ne voulaient pas croire que cet homme avait été aveugle et que maintenant il pouvait voir. C’est pourquoi ils convoquèrent ses parents 
${}^{19}et leur demandèrent : « Cet homme est bien votre fils, et vous dites qu’il est né aveugle ? Comment se fait-il qu’à présent il voie ? » 
${}^{20}Les parents répondirent : « Nous savons bien que c’est notre fils, et qu’il est né aveugle. 
${}^{21}Mais comment peut-il voir maintenant, nous ne le savons pas ; et qui lui a ouvert les yeux, nous ne le savons pas non plus. Interrogez-le, il est assez grand pour s’expliquer. »
${}^{22}Ses parents parlaient ainsi parce qu’ils avaient peur des Juifs. En effet, ceux-ci s’étaient déjà mis d’accord pour exclure de leurs assemblées tous ceux qui déclareraient publiquement que Jésus est le Christ. 
${}^{23}Voilà pourquoi les parents avaient dit : « Il est assez grand, interrogez-le ! »
${}^{24}Pour la seconde fois, les pharisiens convoquèrent l’homme qui avait été aveugle, et ils lui dirent : « Rends gloire à Dieu ! Nous savons, nous, que cet homme est un pécheur. » 
${}^{25}Il répondit : « Est-ce un pécheur ? Je n’en sais rien. Mais il y a une chose que je sais : j’étais aveugle, et à présent je vois. » 
${}^{26}Ils lui dirent alors : « Comment a-t-il fait pour t’ouvrir les yeux ? » 
${}^{27}Il leur répondit : « Je vous l’ai déjà dit, et vous n’avez pas écouté. Pourquoi voulez-vous m’entendre encore une fois ? Serait-ce que vous voulez, vous aussi, devenir ses disciples ? » 
${}^{28}Ils se mirent à l’injurier : « C’est toi qui es son disciple ; nous, c’est de Moïse que nous sommes les disciples. 
${}^{29}Nous savons que Dieu a parlé à Moïse ; mais celui-là, nous ne savons pas d’où il est. » 
${}^{30}L’homme leur répondit : « Voilà bien ce qui est étonnant ! Vous ne savez pas d’où il est, et pourtant il m’a ouvert les yeux. 
${}^{31}Dieu, nous le savons, n’exauce pas les pécheurs, mais si quelqu’un l’honore et fait sa volonté, il l’exauce. 
${}^{32}Jamais encore on n’avait entendu dire que quelqu’un ait ouvert les yeux à un aveugle de naissance. 
${}^{33}Si lui n’était pas de Dieu, il ne pourrait rien faire. » 
${}^{34}Ils répliquèrent : « Tu es tout entier dans le péché depuis ta naissance, et tu nous fais la leçon ? » Et ils le jetèrent dehors.
${}^{35}Jésus apprit qu’ils l’avaient jeté dehors. Il le retrouva et lui dit : « Crois-tu au Fils de l’homme ? » 
${}^{36}Il répondit : « Et qui est-il, Seigneur, pour que je croie en lui ? » 
${}^{37}Jésus lui dit : « Tu le vois, et c’est lui qui te parle. » 
${}^{38}Il dit : « Je crois, Seigneur ! » Et il se prosterna devant lui.
${}^{39}Jésus dit alors : « Je suis venu en ce monde pour rendre un jugement : que ceux qui ne voient pas puissent voir, et que ceux qui voient deviennent aveugles. » 
${}^{40}Parmi les pharisiens, ceux qui étaient avec lui entendirent ces paroles et lui dirent : « Serions-nous aveugles, nous aussi ? » 
${}^{41}Jésus leur répondit : « Si vous étiez aveugles, vous n’auriez pas de péché ; mais du moment que vous dites : “Nous voyons !”, votre péché demeure.
      
         
      \bchapter{}
      \begin{verse}
${}^{1}« Amen, amen, je vous le dis : celui qui entre dans l’enclos des brebis sans passer par la porte, mais qui escalade par un autre endroit, celui-là est un voleur et un bandit. 
${}^{2}Celui qui entre par la porte, c’est le pasteur, le berger des brebis. 
${}^{3}Le portier lui ouvre, et les brebis écoutent sa voix. Ses brebis à lui, il les appelle chacune par son nom, et il les fait sortir. 
${}^{4}Quand il a poussé dehors toutes les siennes, il marche à leur tête, et les brebis le suivent, car elles connaissent sa voix. 
${}^{5}Jamais elles ne suivront un étranger, mais elles s’enfuiront loin de lui, car elles ne connaissent pas la voix des étrangers. »
${}^{6}Jésus employa cette image pour s’adresser à eux, mais eux ne comprirent pas de quoi il leur parlait. 
${}^{7}C’est pourquoi Jésus reprit la parole : « Amen, amen, je vous le dis : Moi, je suis la porte des brebis. 
${}^{8}Tous ceux qui sont venus avant moi sont des voleurs et des bandits ; mais les brebis ne les ont pas écoutés. 
${}^{9}Moi, je suis la porte. Si quelqu’un entre en passant par moi, il sera sauvé ; il pourra entrer ; il pourra sortir et trouver un pâturage. 
${}^{10}Le voleur ne vient que pour voler, égorger, faire périr. Moi, je suis venu pour que les brebis aient la vie, la vie en abondance.
${}^{11}Moi, je suis le bon pasteur, le vrai berger, qui donne sa vie pour ses brebis. 
${}^{12}Le berger mercenaire n’est pas le pasteur, les brebis ne sont pas à lui : s’il voit venir le loup, il abandonne les brebis et s’enfuit ; le loup s’en empare et les disperse. 
${}^{13}Ce berger n’est qu’un mercenaire, et les brebis ne comptent pas vraiment pour lui. 
${}^{14}Moi, je suis le bon pasteur ; je connais mes brebis, et mes brebis me connaissent, 
${}^{15}comme le Père me connaît, et que je connais le Père ; et je donne ma vie pour mes brebis.
${}^{16}J’ai encore d’autres brebis, qui ne sont pas de cet enclos : celles-là aussi, il faut que je les conduise. Elles écouteront ma voix : il y aura un seul troupeau et un seul pasteur. 
${}^{17}Voici pourquoi le Père m’aime : parce que je donne ma vie, pour la recevoir de nouveau. 
${}^{18}Nul ne peut me l’enlever : je la donne de moi-même. J’ai le pouvoir de la donner, j’ai aussi le pouvoir de la recevoir de nouveau : voilà le commandement que j’ai reçu de mon Père. »
${}^{19}De nouveau les Juifs se divisèrent à cause de ces paroles. 
${}^{20}Beaucoup d’entre eux disaient : « Il a un démon, il délire. Pourquoi l’écoutez-vous ? » 
${}^{21}D’autres disaient : « Ces paroles ne sont pas celles d’un possédé… Un démon pourrait-il ouvrir les yeux des aveugles ? »
${}^{22}Alors arriva la fête de la dédicace du Temple à Jérusalem. C’était l’hiver. 
${}^{23}Jésus allait et venait dans le Temple, sous la colonnade de Salomon. 
${}^{24}Les Juifs firent cercle autour de lui ; ils lui disaient : « Combien de temps vas-tu nous tenir en haleine ? Si c’est toi le Christ, dis-le nous ouvertement ! » 
${}^{25}Jésus leur répondit : « Je vous l’ai dit, et vous ne croyez pas. Les œuvres que je fais, moi, au nom de mon Père, voilà ce qui me rend témoignage. 
${}^{26}Mais vous, vous ne croyez pas, parce que vous n’êtes pas de mes brebis. 
${}^{27}Mes brebis écoutent ma voix ; moi, je les connais, et elles me suivent. 
${}^{28}Je leur donne la vie éternelle : jamais elles ne périront, et personne ne les arrachera de ma main. 
${}^{29}Mon Père, qui me les a données, est plus grand que tout, et personne ne peut les arracher de la main du Père. 
${}^{30}Le Père et moi, nous sommes un. »
${}^{31}De nouveau, des Juifs prirent des pierres pour lapider Jésus. 
${}^{32}Celui-ci reprit la parole : « J’ai multiplié sous vos yeux les œuvres bonnes qui viennent du Père. Pour laquelle de ces œuvres voulez-vous me lapider ? » 
${}^{33}Ils lui répondirent : « Ce n’est pas pour une œuvre bonne que nous voulons te lapider, mais c’est pour un blasphème : tu n’es qu’un homme, et tu te fais Dieu. » 
${}^{34}Jésus leur répliqua : « N’est-il pas écrit dans votre Loi : J’ai dit : Vous êtes des dieux  ? 
${}^{35}Elle les appelle donc des dieux, ceux à qui la parole de Dieu s’adressait, et l’Écriture ne peut pas être abolie. 
${}^{36}Or, celui que le Père a consacré et envoyé dans le monde, vous lui dites : “Tu blasphèmes”, parce que j’ai dit : “Je suis le Fils de Dieu”. 
${}^{37}Si je ne fais pas les œuvres de mon Père, continuez à ne pas me croire. 
${}^{38}Mais si je les fais, même si vous ne me croyez pas, croyez les œuvres. Ainsi vous reconnaîtrez, et de plus en plus, que le Père est en moi, et moi dans le Père. » 
${}^{39}Eux cherchaient de nouveau à l’arrêter, mais il échappa à leurs mains.
${}^{40}Il repartit de l’autre côté du Jourdain, à l’endroit où, au début, Jean baptisait ; et il y demeura. 
${}^{41}Beaucoup vinrent à lui en déclarant : « Jean n’a pas accompli de signe ; mais tout ce que Jean a dit de celui-ci était vrai. » 
${}^{42}Et là, beaucoup crurent en lui.
      <h2 class="intertitle" id="d85e362900">4. Jésus devant l’heure de sa mort et de sa gloire (11 – 12,36)</h2>
      
         
      \bchapter{}
      \begin{verse}
${}^{1}Il y avait quelqu’un de malade, Lazare, de Béthanie, le village de Marie et de Marthe, sa sœur. 
${}^{2}Or Marie était celle qui répandit du parfum sur le Seigneur et lui essuya les pieds avec ses cheveux. C’était son frère Lazare qui était malade. 
${}^{3}Donc, les deux sœurs envoyèrent dire à Jésus : « Seigneur, celui que tu aimes est malade. » 
${}^{4}En apprenant cela, Jésus dit : « Cette maladie ne conduit pas à la mort, elle est pour la gloire de Dieu, afin que par elle le Fils de Dieu soit glorifié. » 
${}^{5}Jésus aimait Marthe et sa sœur, ainsi que Lazare. 
${}^{6}Quand il apprit que celui-ci était malade, il demeura deux jours encore à l’endroit où il se trouvait. 
${}^{7}Puis, après cela, il dit aux disciples : « Revenons en Judée. » 
${}^{8}Les disciples lui dirent : « Rabbi, tout récemment, les Juifs, là-bas, cherchaient à te lapider, et tu y retournes ? » 
${}^{9}Jésus répondit : « N’y a-t-il pas douze heures dans une journée ? Celui qui marche pendant le jour ne trébuche pas, parce qu’il voit la lumière de ce monde ; 
${}^{10}mais celui qui marche pendant la nuit trébuche, parce que la lumière n’est pas en lui. »
${}^{11}Après ces paroles, il ajouta : « Lazare, notre ami, s’est endormi ; mais je vais aller le tirer de ce sommeil. » 
${}^{12}Les disciples lui dirent alors : « Seigneur, s’il s’est endormi, il sera sauvé. » 
${}^{13}Jésus avait parlé de la mort ; eux pensaient qu’il parlait du repos du sommeil. 
${}^{14}Alors il leur dit ouvertement : « Lazare est mort, 
${}^{15}et je me réjouis de n’avoir pas été là, à cause de vous, pour que vous croyiez. Mais allons auprès de lui ! » 
${}^{16}Thomas, appelé Didyme (c’est-à-dire Jumeau), dit aux autres disciples : « Allons-y, nous aussi, pour mourir avec lui ! »
${}^{17}À son arrivée, Jésus trouva Lazare au tombeau depuis quatre jours déjà. 
${}^{18}Comme Béthanie était tout près de Jérusalem – à une distance de quinze stades (c’est-à-dire une demi-heure de marche environ) –,
${}^{19}beaucoup de Juifs étaient venus réconforter Marthe et Marie au sujet de leur frère. 
${}^{20}Lorsque Marthe apprit l’arrivée de Jésus, elle partit à sa rencontre, tandis que Marie restait assise à la maison. 
${}^{21}Marthe dit à Jésus : « Seigneur, si tu avais été ici, mon frère ne serait pas mort. 
${}^{22}Mais maintenant encore, je le sais, tout ce que tu demanderas à Dieu, Dieu te l’accordera. » 
${}^{23}Jésus lui dit : « Ton frère ressuscitera. » 
${}^{24}Marthe reprit : « Je sais qu’il ressuscitera à la résurrection, au dernier jour. » 
${}^{25}Jésus lui dit : « Moi, je suis la résurrection et la vie. Celui qui croit en moi, même s’il meurt, vivra ; 
${}^{26}quiconque vit et croit en moi ne mourra jamais. Crois-tu cela ? » 
${}^{27}Elle répondit : « Oui, Seigneur, je le crois : tu es le Christ, le Fils de Dieu, tu es celui qui vient dans le monde. »
${}^{28}Ayant dit cela, elle partit appeler sa sœur Marie, et lui dit tout bas : « Le Maître est là, il t’appelle. » 
${}^{29}Marie, dès qu’elle l’entendit, se leva rapidement et alla rejoindre Jésus. 
${}^{30}Il n’était pas encore entré dans le village, mais il se trouvait toujours à l’endroit où Marthe l’avait rencontré. 
${}^{31}Les Juifs qui étaient à la maison avec Marie et la réconfortaient, la voyant se lever et sortir si vite, la suivirent ; ils pensaient qu’elle allait au tombeau pour y pleurer. 
${}^{32}Marie arriva à l’endroit où se trouvait Jésus. Dès qu’elle le vit, elle se jeta à ses pieds et lui dit : « Seigneur, si tu avais été ici, mon frère ne serait pas mort. »
${}^{33}Quand il vit qu’elle pleurait, et que les Juifs venus avec elle pleuraient aussi, Jésus, en son esprit, fut saisi d’émotion, il fut bouleversé, 
${}^{34}et il demanda : « Où l’avez-vous déposé ? » Ils lui répondirent : « Seigneur, viens, et vois. » 
${}^{35}Alors Jésus se mit à pleurer. 
${}^{36}Les Juifs disaient : « Voyez comme il l’aimait ! » 
${}^{37}Mais certains d’entre eux dirent : « Lui qui a ouvert les yeux de l’aveugle, ne pouvait-il pas empêcher Lazare de mourir ? »
${}^{38}Jésus, repris par l’émotion, arriva au tombeau. C’était une grotte fermée par une pierre. 
${}^{39}Jésus dit : « Enlevez la pierre. » Marthe, la sœur du défunt, lui dit : « Seigneur, il sent déjà ; c’est le quatrième jour qu’il est là. » 
${}^{40}Alors Jésus dit à Marthe : « Ne te l’ai-je pas dit ? Si tu crois, tu verras la gloire de Dieu. » 
${}^{41}On enleva donc la pierre. Alors Jésus leva les yeux au ciel et dit : « Père, je te rends grâce parce que tu m’as exaucé. 
${}^{42}Je le savais bien, moi, que tu m’exauces toujours ; mais je le dis à cause de la foule qui m’entoure, afin qu’ils croient que c’est toi qui m’as envoyé. » 
${}^{43}Après cela, il cria d’une voix forte : « Lazare, viens dehors ! » 
${}^{44}Et le mort sortit, les pieds et les mains liés par des bandelettes, le visage enveloppé d’un suaire. Jésus leur dit : « Déliez-le, et laissez-le aller. »
${}^{45}Beaucoup de Juifs, qui étaient venus auprès de Marie et avaient donc vu ce que Jésus avait fait, crurent en lui.
${}^{46}Mais quelques-uns allèrent trouver les pharisiens pour leur raconter ce qu’il avait fait. 
${}^{47}Les grands prêtres et les pharisiens réunirent donc le Conseil suprême ; ils disaient : « Qu’allons-nous faire ? Cet homme accomplit un grand nombre de signes. 
${}^{48}Si nous le laissons faire, tout le monde va croire en lui, et les Romains viendront détruire notre Lieu saint et notre nation. » 
${}^{49}Alors, l’un d’entre eux, Caïphe, qui était grand prêtre cette année-là, leur dit : « Vous n’y comprenez rien ; 
${}^{50}vous ne voyez pas quel est votre intérêt : il vaut mieux qu’un seul homme meure pour le peuple, et que l’ensemble de la nation ne périsse pas. »
${}^{51}Ce qu’il disait là ne venait pas de lui-même ; mais, étant grand prêtre cette année-là, il prophétisa que Jésus allait mourir pour la nation ; 
${}^{52}et ce n’était pas seulement pour la nation, c’était afin de rassembler dans l’unité les enfants de Dieu dispersés.
${}^{53}À partir de ce jour-là, ils décidèrent de le tuer. 
${}^{54}C’est pourquoi Jésus ne se déplaçait plus ouvertement parmi les Juifs ; il partit pour la région proche du désert, dans la ville d’Éphraïm où il séjourna avec ses disciples.
${}^{55}Or, la Pâque juive était proche, et beaucoup montèrent de la campagne à Jérusalem pour se purifier avant la Pâque. 
${}^{56}Ils cherchaient Jésus et, dans le Temple, ils se disaient entre eux : « Qu’en pensez-vous ? Il ne viendra sûrement pas à la fête ! » 
${}^{57}Les grands prêtres et les pharisiens avaient donné des ordres : quiconque saurait où il était devait le dénoncer, pour qu’on puisse l’arrêter.
      
         
      \bchapter{}
      \begin{verse}
${}^{1}Six jours avant la Pâque, Jésus vint à Béthanie où habitait Lazare, qu’il avait réveillé d’entre les morts. 
${}^{2}On donna un repas en l’honneur de Jésus. Marthe faisait le service, Lazare était parmi les convives avec Jésus.
${}^{3}Or, Marie avait pris une livre d’un parfum très pur et de très grande valeur ; elle versa le parfum sur les pieds de Jésus, qu’elle essuya avec ses cheveux ; la maison fut remplie de l’odeur du parfum. 
${}^{4}Judas Iscariote, l’un de ses disciples, celui qui allait le livrer, dit alors : 
${}^{5}« Pourquoi n’a-t-on pas vendu ce parfum pour trois cents pièces d’argent, que l’on aurait données à des pauvres ? » 
${}^{6}Il parla ainsi, non par souci des pauvres, mais parce que c’était un voleur : comme il tenait la bourse commune, il prenait ce que l’on y mettait. 
${}^{7}Jésus lui dit : « Laisse-la observer cet usage en vue du jour de mon ensevelissement ! 
${}^{8}Des pauvres, vous en aurez toujours avec vous, mais moi, vous ne m’aurez pas toujours. »
${}^{9}Or, une grande foule de Juifs apprit que Jésus était là, et ils arrivèrent, non seulement à cause de Jésus, mais aussi pour voir ce Lazare qu’il avait réveillé d’entre les morts. 
${}^{10}Les grands prêtres décidèrent alors de tuer aussi Lazare, 
${}^{11}parce que beaucoup de Juifs, à cause de lui, s’en allaient, et croyaient en Jésus.
${}^{12}Le lendemain, la grande foule venue pour la fête apprit que Jésus arrivait à Jérusalem. 
${}^{13}Les gens prirent des branches de palmiers et sortirent à sa rencontre. Ils criaient :
        \\« Hosanna !
        \\Béni soit celui qui vient au nom du Seigneur !
        \\Béni soit le roi d’Israël ! »
${}^{14}Jésus, trouvant un petit âne, s’assit dessus, comme il est écrit :
        ${}^{15}Ne crains pas, fille de Sion.
        \\Voici ton roi qui vient,
        \\assis sur le petit d’une ânesse.
${}^{16}Cela, ses disciples ne le comprirent pas sur le moment ; mais, quand Jésus fut glorifié, ils se rappelèrent que l’Écriture disait cela de lui : c’était bien ce qu’on lui avait fait.
${}^{17}La foule rendait témoignage, elle qui était avec lui quand il avait appelé Lazare hors du tombeau et l’avait réveillé d’entre les morts. 
${}^{18}C’est pourquoi la foule vint à sa rencontre ; elle avait entendu dire qu’il avait accompli ce signe. 
${}^{19}Les pharisiens se dirent alors entre eux : « Vous voyez bien que vous n’arrivez à rien : voilà que tout le monde marche derrière lui ! »
${}^{20}Il y avait quelques Grecs parmi ceux qui étaient montés à Jérusalem pour adorer Dieu pendant la fête de la Pâque. 
${}^{21}Ils abordèrent Philippe, qui était de Bethsaïde en Galilée, et lui firent cette demande : « Nous voudrions voir Jésus. » 
${}^{22}Philippe va le dire à André, et tous deux vont le dire à Jésus. 
${}^{23}Alors Jésus leur déclare : « L’heure est venue où le Fils de l’homme doit être glorifié. 
${}^{24}Amen, amen, je vous le dis : si le grain de blé tombé en terre ne meurt pas, il reste seul ; mais s’il meurt, il porte beaucoup de fruit. 
${}^{25}Qui aime sa vie la perd ; qui s’en détache en ce monde la gardera pour la vie éternelle. 
${}^{26}Si quelqu’un veut me servir, qu’il me suive ; et là où moi je suis, là aussi sera mon serviteur. Si quelqu’un me sert, mon Père l’honorera.
${}^{27}Maintenant mon âme est bouleversée. Que vais-je dire ? “Père, sauve-moi de cette heure” ? – Mais non ! C’est pour cela que je suis parvenu à cette heure-ci ! 
${}^{28}Père, glorifie ton nom ! »
      Alors, du ciel vint une voix qui disait : « Je l’ai glorifié et je le glorifierai encore. » 
${}^{29}En l’entendant, la foule qui se tenait là disait que c’était un coup de tonnerre. D’autres disaient : « C’est un ange qui lui a parlé. » 
${}^{30}Mais Jésus leur répondit : « Ce n’est pas pour moi qu’il y a eu cette voix, mais pour vous. 
${}^{31}Maintenant a lieu le jugement de ce monde ; maintenant le prince de ce monde va être jeté dehors ; 
${}^{32}et moi, quand j’aurai été élevé de terre, j’attirerai à moi tous les hommes. » 
${}^{33}Il signifiait par là de quel genre de mort il allait mourir.
${}^{34}La foule lui répliqua : « Nous, nous avons appris dans la Loi que le Christ demeure pour toujours. Alors toi, comment peux-tu dire : “Il faut que le Fils de l’homme soit élevé” ? Qui est donc ce Fils de l’homme ? » 
${}^{35}Jésus leur déclara : « Pour peu de temps encore, la lumière est parmi vous ; marchez, tant que vous avez la lumière, afin que les ténèbres ne vous arrêtent pas ; celui qui marche dans les ténèbres ne sait pas où il va. 
${}^{36}Pendant que vous avez la lumière, croyez en la lumière : vous serez alors des fils de lumière. » Ainsi parla Jésus. Puis il les quitta et se cacha loin d’eux.
      <h2 class="intertitle" id="d85e363534">5. Conclusion du livre des signes (12,37-50)</h2>
${}^{37}Alors qu’il avait fait tant de signes devant eux, certains ne croyaient pas en lui. 
${}^{38}Ainsi s’accomplissait la parole dite par le prophète Isaïe :
        \\Seigneur, qui a cru ce que nous avons entendu ?
        \\À qui la puissance du Seigneur a-t-elle été révélée ?
${}^{39}Ils ne pouvaient pas croire, puisqu’Isaïe dit encore :
${}^{40}Il a rendu aveugles leurs yeux,
        \\il a endurci leur cœur,
        \\de peur qu’ils ne voient de leurs yeux,
        \\qu’ils ne comprennent dans leur cœur,
        \\et qu’ils ne se convertissent,
        \\– et moi, je les guérirai.
${}^{41}Ces paroles, Isaïe les a prononcées parce qu’il avait vu la gloire de Jésus, et c’est de lui qu’il a parlé. 
${}^{42}Cependant, même parmi les chefs du peuple, beaucoup crurent en lui ; mais, à cause des pharisiens, ils ne le déclaraient pas publiquement, de peur d’être exclus des assemblées. 
${}^{43}En effet, ils aimaient la gloire qui vient des hommes plus que la gloire qui vient de Dieu.
${}^{44}Alors, Jésus s’écria : « Celui qui croit en moi, ce n’est pas en moi qu’il croit, mais en Celui qui m’a envoyé ; 
${}^{45}et celui qui me voit voit Celui qui m’a envoyé. 
${}^{46}Moi qui suis la lumière, je suis venu dans le monde pour que celui qui croit en moi ne demeure pas dans les ténèbres. 
${}^{47}Si quelqu’un entend mes paroles et n’y reste pas fidèle, moi, je ne le juge pas, car je ne suis pas venu juger le monde, mais le sauver. 
${}^{48}Celui qui me rejette et n’accueille pas mes paroles aura, pour le juger, la parole que j’ai prononcée : c’est elle qui le jugera au dernier jour. 
${}^{49}Car ce n’est pas de ma propre initiative que j’ai parlé : le Père lui-même, qui m’a envoyé, m’a donné son commandement sur ce que je dois dire et déclarer ; 
${}^{50}et je sais que son commandement est vie éternelle. Donc, ce que je déclare, je le déclare comme le Père me l’a dit. »
      <h2 class="intertitle" id="d85e363707">1. Le dernier repas et le dernier discours (13 – 17)</h2>
      
         
      \bchapter{}
      \begin{verse}
${}^{1}Avant la fête de la Pâque, sachant que l’heure était venue pour lui de passer de ce monde à son Père, Jésus, ayant aimé les siens qui étaient dans le monde, les aima jusqu’au bout. 
${}^{2}Au cours du repas, alors que le diable a déjà mis dans le cœur de Judas, fils de Simon l’Iscariote, l’intention de le livrer, 
${}^{3}Jésus, sachant que le Père a tout remis entre ses mains, qu’il est sorti de Dieu et qu’il s’en va vers Dieu, 
${}^{4}se lève de table, dépose son vêtement, et prend un linge qu’il se noue à la ceinture ; 
${}^{5}puis il verse de l’eau dans un bassin. Alors il se mit à laver les pieds des disciples et à les essuyer avec le linge qu’il avait à la ceinture.
${}^{6}Il arrive donc à Simon-Pierre, qui lui dit : « C’est toi, Seigneur, qui me laves les pieds ? » 
${}^{7}Jésus lui répondit : « Ce que je veux faire, tu ne le sais pas maintenant ; plus tard tu comprendras. » 
${}^{8}Pierre lui dit : « Tu ne me laveras pas les pieds ; non, jamais ! » Jésus lui répondit : « Si je ne te lave pas, tu n’auras pas de part avec moi. » 
${}^{9}Simon-Pierre lui dit : « Alors, Seigneur, pas seulement les pieds, mais aussi les mains et la tête ! » 
${}^{10}Jésus lui dit : « Quand on vient de prendre un bain, on n’a pas besoin de se laver, sinon les pieds : on est pur tout entier. Vous-mêmes, vous êtes purs, mais non pas tous. » 
${}^{11}Il savait bien qui allait le livrer ; et c’est pourquoi il disait : « Vous n’êtes pas tous purs. »
${}^{12}Quand il leur eut lavé les pieds, il reprit son vêtement, se remit à table et leur dit : « Comprenez-vous ce que je viens de faire pour vous ? 
${}^{13}Vous m’appelez “Maître” et “Seigneur”, et vous avez raison, car vraiment je le suis. 
${}^{14}Si donc moi, le Seigneur et le Maître, je vous ai lavé les pieds, vous aussi, vous devez vous laver les pieds les uns aux autres. 
${}^{15}C’est un exemple que je vous ai donné afin que vous fassiez, vous aussi, comme j’ai fait pour vous. 
${}^{16}Amen, amen, je vous le dis : un serviteur n’est pas plus grand que son maître, ni un envoyé plus grand que celui qui l’envoie. 
${}^{17}Sachant cela, heureux êtes-vous, si vous le faites. 
${}^{18}Ce n’est pas de vous tous que je parle. Moi, je sais quels sont ceux que j’ai choisis, mais il faut que s’accomplisse l’Écriture : Celui qui mange le pain avec moi m’a frappé du talon. 
${}^{19}Je vous dis ces choses dès maintenant, avant qu’elles n’arrivent ; ainsi, lorsqu’elles arriveront, vous croirez que moi, Je suis. 
${}^{20}Amen, amen, je vous le dis : si quelqu’un reçoit celui que j’envoie, il me reçoit moi-même ; et celui qui me reçoit, reçoit Celui qui m’a envoyé. »
${}^{21}Après avoir ainsi parlé, Jésus fut bouleversé en son esprit, et il rendit ce témoignage : « Amen, amen, je vous le dis : l’un de vous me livrera. » 
${}^{22}Les disciples se regardaient les uns les autres avec embarras, ne sachant pas de qui Jésus parlait. 
${}^{23}Il y avait à table, appuyé contre Jésus, l’un de ses disciples, celui que Jésus aimait. 
${}^{24}Simon-Pierre lui fait signe de demander à Jésus de qui il veut parler. 
${}^{25}Le disciple se penche donc sur la poitrine de Jésus et lui dit : « Seigneur, qui est-ce ? » 
${}^{26}Jésus lui répond : « C’est celui à qui je donnerai la bouchée que je vais tremper dans le plat. » Il trempe la bouchée, et la donne à Judas, fils de Simon l’Iscariote. 
${}^{27}Et, quand Judas eut pris la bouchée, Satan entra en lui. Jésus lui dit alors : « Ce que tu fais, fais-le vite. » 
${}^{28}Mais aucun des convives ne comprit pourquoi il lui avait dit cela. 
${}^{29}Comme Judas tenait la bourse commune, certains pensèrent que Jésus voulait lui dire d’acheter ce qu’il fallait pour la fête, ou de donner quelque chose aux pauvres. 
${}^{30}Judas prit donc la bouchée, et sortit aussitôt. Or il faisait nuit.
${}^{31}Quand il fut sorti, Jésus déclara : « Maintenant le Fils de l’homme est glorifié, et Dieu est glorifié en lui. 
${}^{32}Si Dieu est glorifié en lui, Dieu aussi le glorifiera ; et il le glorifiera bientôt. 
${}^{33}Petits enfants, c’est pour peu de temps encore que je suis avec vous. Vous me chercherez, et, comme je l’ai dit aux Juifs : “Là où je vais, vous ne pouvez pas aller”, je vous le dis maintenant à vous aussi. 
${}^{34}Je vous donne un commandement nouveau : c’est de vous aimer les uns les autres. Comme je vous ai aimés, vous aussi aimez-vous les uns les autres. 
${}^{35}À ceci, tous reconnaîtront que vous êtes mes disciples : si vous avez de l’amour les uns pour les autres. »
${}^{36}Simon-Pierre lui dit : « Seigneur, où vas-tu ? » Jésus lui répondit : « Là où je vais, tu ne peux pas me suivre maintenant ; tu me suivras plus tard. » 
${}^{37}Pierre lui dit : « Seigneur, pourquoi ne puis-je pas te suivre à présent ? Je donnerai ma vie pour toi ! » 
${}^{38}Jésus réplique : « Tu donneras ta vie pour moi ? Amen, amen, je te le dis : le coq ne chantera pas avant que tu m’aies renié trois fois.
      
         
      \bchapter{}
      \begin{verse}
${}^{1}Que votre cœur ne soit pas bouleversé : vous croyez en Dieu, croyez aussi en moi. 
${}^{2}Dans la maison de mon Père, il y a de nombreuses demeures ; sinon, vous aurais-je dit : “Je pars vous préparer une place” ? 
${}^{3}Quand je serai parti vous préparer une place, je reviendrai et je vous emmènerai auprès de moi, afin que là où je suis, vous soyez, vous aussi. 
${}^{4}Pour aller où je vais, vous savez le chemin. »
${}^{5}Thomas lui dit : « Seigneur, nous ne savons pas où tu vas. Comment pourrions-nous savoir le chemin ? » 
${}^{6}Jésus lui répond : « Moi, je suis le Chemin, la Vérité et la Vie ; personne ne va vers le Père sans passer par moi. 
${}^{7}Puisque vous me connaissez, vous connaîtrez aussi mon Père. Dès maintenant vous le connaissez, et vous l’avez vu. »
${}^{8}Philippe lui dit : « Seigneur, montre-nous le Père ; cela nous suffit. » 
${}^{9}Jésus lui répond : « Il y a si longtemps que je suis avec vous, et tu ne me connais pas, Philippe ! Celui qui m’a vu a vu le Père. Comment peux-tu dire : “Montre-nous le Père” ? 
${}^{10}Tu ne crois donc pas que je suis dans le Père et que le Père est en moi ! Les paroles que je vous dis, je ne les dis pas de moi-même ; le Père qui demeure en moi fait ses propres œuvres. 
${}^{11}Croyez-moi : je suis dans le Père, et le Père est en moi ; si vous ne me croyez pas, croyez du moins à cause des œuvres elles-mêmes.
${}^{12}Amen, amen, je vous le dis : celui qui croit en moi fera les œuvres que je fais. Il en fera même de plus grandes, parce que je pars vers le Père, 
${}^{13}et tout ce que vous demanderez en mon nom, je le ferai, afin que le Père soit glorifié dans le Fils. 
${}^{14}Quand vous me demanderez quelque chose en mon nom, moi, je le ferai.
${}^{15}Si vous m’aimez, vous garderez mes commandements. 
${}^{16}Moi, je prierai le Père, et il vous donnera un autre Défenseur qui sera pour toujours avec vous : 
${}^{17}l’Esprit de vérité, lui que le monde ne peut recevoir, car il ne le voit pas et ne le connaît pas ; vous, vous le connaissez, car il demeure auprès de vous, et il sera en vous. 
${}^{18}Je ne vous laisserai pas orphelins, je reviens vers vous. 
${}^{19}D’ici peu de temps, le monde ne me verra plus, mais vous, vous me verrez vivant, et vous vivrez aussi. 
${}^{20}En ce jour-là, vous reconnaîtrez que je suis en mon Père, que vous êtes en moi, et moi en vous. 
${}^{21}Celui qui reçoit mes commandements et les garde, c’est celui-là qui m’aime ; et celui qui m’aime sera aimé de mon Père ; moi aussi, je l’aimerai, et je me manifesterai à lui. »
${}^{22}Jude – non pas Judas l’Iscariote – lui demanda : « Seigneur, que se passe-t-il ? Est-ce à nous que tu vas te manifester, et non pas au monde ? » 
${}^{23}Jésus lui répondit : « Si quelqu’un m’aime, il gardera ma parole ; mon Père l’aimera, nous viendrons vers lui et, chez lui, nous nous ferons une demeure. 
${}^{24}Celui qui ne m’aime pas ne garde pas mes paroles. Or, la parole que vous entendez n’est pas de moi : elle est du Père, qui m’a envoyé.
${}^{25}Je vous parle ainsi, tant que je demeure avec vous ; 
${}^{26}mais le Défenseur, l’Esprit Saint que le Père enverra en mon nom, lui, vous enseignera tout, et il vous fera souvenir de tout ce que je vous ai dit.
${}^{27}Je vous laisse la paix, je vous donne ma paix ; ce n’est pas à la manière du monde que je vous la donne. Que votre cœur ne soit pas bouleversé ni effrayé. 
${}^{28}Vous avez entendu ce que je vous ai dit : Je m’en vais, et je reviens vers vous. Si vous m’aimiez, vous seriez dans la joie puisque je pars vers le Père, car le Père est plus grand que moi. 
${}^{29}Je vous ai dit ces choses maintenant, avant qu’elles n’arrivent ; ainsi, lorsqu’elles arriveront, vous croirez.
${}^{30}Désormais, je ne parlerai plus beaucoup avec vous, car il vient, le prince du monde. Certes, sur moi il n’a aucune prise, 
${}^{31}mais il faut que le monde sache que j’aime le Père, et que je fais comme le Père me l’a commandé.
      Levez-vous, partons d’ici.
      
         
      \bchapter{}
      \begin{verse}
${}^{1}Moi, je suis la vraie vigne, et mon Père est le vigneron. 
${}^{2}Tout sarment qui est en moi, mais qui ne porte pas de fruit, mon Père l’enlève ; tout sarment qui porte du fruit, il le purifie en le taillant, pour qu’il en porte davantage. 
${}^{3}Mais vous, déjà vous voici purifiés grâce à la parole que je vous ai dite. 
${}^{4}Demeurez en moi, comme moi en vous. De même que le sarment ne peut pas porter de fruit par lui-même s’il ne demeure pas sur la vigne, de même vous non plus, si vous ne demeurez pas en moi.
${}^{5}Moi, je suis la vigne, et vous, les sarments. Celui qui demeure en moi et en qui je demeure, celui-là porte beaucoup de fruit, car, en dehors de moi, vous ne pouvez rien faire. 
${}^{6}Si quelqu’un ne demeure pas en moi, il est, comme le sarment, jeté dehors, et il se dessèche. Les sarments secs, on les ramasse, on les jette au feu, et ils brûlent. 
${}^{7}Si vous demeurez en moi, et que mes paroles demeurent en vous, demandez tout ce que vous voulez, et cela se réalisera pour vous. 
${}^{8}Ce qui fait la gloire de mon Père, c’est que vous portiez beaucoup de fruit et que vous soyez pour moi des disciples.
${}^{9}Comme le Père m’a aimé, moi aussi je vous ai aimés. Demeurez dans mon amour. 
${}^{10}Si vous gardez mes commandements, vous demeurerez dans mon amour, comme moi, j’ai gardé les commandements de mon Père, et je demeure dans son amour. 
${}^{11}Je vous ai dit cela pour que ma joie soit en vous, et que votre joie soit parfaite.
${}^{12}Mon commandement, le voici : Aimez-vous les uns les autres comme je vous ai aimés. 
${}^{13}Il n’y a pas de plus grand amour que de donner sa vie pour ceux qu’on aime. 
${}^{14}Vous êtes mes amis si vous faites ce que je vous commande. 
${}^{15}Je ne vous appelle plus serviteurs, car le serviteur ne sait pas ce que fait son maître ; je vous appelle mes amis, car tout ce que j’ai entendu de mon Père, je vous l’ai fait connaître.
${}^{16}Ce n’est pas vous qui m’avez choisi, c’est moi qui vous ai choisis et établis, afin que vous alliez, que vous portiez du fruit, et que votre fruit demeure. Alors, tout ce que vous demanderez au Père en mon nom, il vous le donnera. 
${}^{17}Voici ce que je vous commande : c’est de vous aimer les uns les autres.
${}^{18}Si le monde a de la haine contre vous, sachez qu’il en a eu d’abord contre moi. 
${}^{19}Si vous apparteniez au monde, le monde aimerait ce qui est à lui. Mais vous n’appartenez pas au monde, puisque je vous ai choisis en vous prenant dans le monde ; voilà pourquoi le monde a de la haine contre vous. 
${}^{20}Rappelez-vous la parole que je vous ai dite : un serviteur n’est pas plus grand que son maître. Si l’on m’a persécuté, on vous persécutera, vous aussi. Si l’on a gardé ma parole, on gardera aussi la vôtre. 
${}^{21}Les gens vous traiteront ainsi à cause de mon nom, parce qu’ils ne connaissent pas Celui qui m’a envoyé.
${}^{22}Si je n’étais pas venu, si je ne leur avais pas parlé, ils n’auraient pas de péché ; mais à présent ils sont sans excuse pour leur péché. 
${}^{23}Celui qui a de la haine contre moi a de la haine aussi contre mon Père. 
${}^{24}Si je n’avais pas fait parmi eux ces œuvres que personne d’autre n’a faites, ils n’auraient pas de péché. Mais à présent, ils ont vu, et ils sont remplis de haine contre moi et contre mon Père. 
${}^{25}Ainsi s’est accomplie cette parole écrite dans leur Loi : Ils m’ont haï sans raison.
${}^{26}Quand viendra le Défenseur, que je vous enverrai d’auprès du Père, lui, l’Esprit de vérité qui procède du Père, il rendra témoignage en ma faveur. 
${}^{27}Et vous aussi, vous allez rendre témoignage, car vous êtes avec moi depuis le commencement.
      
         
      \bchapter{}
      \begin{verse}
${}^{1}Je vous parle ainsi, pour que vous ne soyez pas scandalisés. 
${}^{2}On vous exclura des assemblées. Bien plus, l’heure vient où tous ceux qui vous tueront s’imagineront qu’ils rendent un culte à Dieu. 
${}^{3}Ils feront cela, parce qu’ils n’ont connu ni le Père ni moi. 
${}^{4}Eh bien, voici pourquoi je vous dis cela : quand l’heure sera venue, vous vous souviendrez que je vous l’avais dit.
      Je ne vous l’ai pas dit dès le commencement, parce que j’étais avec vous.
${}^{5}Je m’en vais maintenant auprès de Celui qui m’a envoyé, et aucun de vous ne me demande : “Où vas-tu ?” 
${}^{6}Mais, parce que je vous dis cela, la tristesse remplit votre cœur. 
${}^{7}Pourtant, je vous dis la vérité : il vaut mieux pour vous que je m’en aille, car, si je ne m’en vais pas, le Défenseur ne viendra pas à vous ; mais si je pars, je vous l’enverrai. 
${}^{8}Quand il viendra, il établira la culpabilité du monde en matière de péché, de justice et de jugement. 
${}^{9}En matière de péché, puisqu’on ne croit pas en moi. 
${}^{10}En matière de justice, puisque je m’en vais auprès du Père, et que vous ne me verrez plus. 
${}^{11}En matière de jugement, puisque déjà le prince de ce monde est jugé.
${}^{12}J’ai encore beaucoup de choses à vous dire, mais pour l’instant vous ne pouvez pas les porter. 
${}^{13}Quand il viendra, lui, l’Esprit de vérité, il vous conduira dans la vérité tout entière. En effet, ce qu’il dira ne viendra pas de lui-même : mais ce qu’il aura entendu, il le dira ; et ce qui va venir, il vous le fera connaître. 
${}^{14}Lui me glorifiera, car il recevra ce qui vient de moi pour vous le faire connaître. 
${}^{15}Tout ce que possède le Père est à moi ; voilà pourquoi je vous ai dit : L’Esprit reçoit ce qui vient de moi pour vous le faire connaître.
${}^{16}Encore un peu de temps, et vous ne me verrez plus ; encore un peu de temps, et vous me reverrez. » 
${}^{17}Alors, certains de ses disciples se dirent entre eux : « Que veut-il nous dire par là : “Encore un peu de temps, et vous ne me verrez plus ; encore un peu de temps, et vous me reverrez”. Et puis : “Je m’en vais auprès du Père” ? » 
${}^{18}Ils disaient donc : « Que veut dire : un peu de temps ? Nous ne savons pas de quoi il parle. »
${}^{19}Jésus comprit qu’ils voulaient l’interroger, et il leur dit : « Vous discutez entre vous parce que j’ai dit : “Encore un peu de temps, et vous ne me verrez plus ; encore un peu de temps, et vous me reverrez.” 
${}^{20}Amen, amen, je vous le dis : vous allez pleurer et vous lamenter, tandis que le monde se réjouira ; vous serez dans la peine, mais votre peine se changera en joie. 
${}^{21}La femme qui enfante est dans la peine parce que son heure est arrivée. Mais, quand l’enfant est né, elle ne se souvient plus de sa souffrance, tout heureuse qu’un être humain soit venu au monde. 
${}^{22}Vous aussi, maintenant, vous êtes dans la peine, mais je vous reverrai, et votre cœur se réjouira ; et votre joie, personne ne vous l’enlèvera. 
${}^{23}En ce jour-là, vous ne me poserez plus de questions.
      <a class="anchor verset_lettre" id="bib_jn_16_23_b"/>Amen, amen, je vous le dis : ce que vous demanderez au Père en mon nom, il vous le donnera. 
${}^{24}Jusqu’à présent vous n’avez rien demandé en mon nom ; demandez, et vous recevrez : ainsi votre joie sera parfaite.
${}^{25}En disant cela, je vous ai parlé en images. L’heure vient où je vous parlerai sans images, et vous annoncerai ouvertement ce qui concerne le Père. 
${}^{26}Ce jour-là, vous demanderez en mon nom ; or, je ne vous dis pas que moi, je prierai le Père pour vous, 
${}^{27}car le Père lui-même vous aime, parce que vous m’avez aimé et vous avez cru que c’est de Dieu que je suis sorti. 
${}^{28}Je suis sorti du Père, et je suis venu dans le monde ; maintenant, je quitte le monde, et je pars vers le Père. »
${}^{29}Ses disciples lui disent : « Voici que tu parles ouvertement et non plus en images. 
${}^{30}Maintenant nous savons que tu sais toutes choses, et tu n’as pas besoin qu’on t’interroge : voilà pourquoi nous croyons que tu es sorti de Dieu. » 
${}^{31}Jésus leur répondit : « Maintenant vous croyez ! 
${}^{32}Voici que l’heure vient – déjà elle est venue – où vous serez dispersés chacun de son côté, et vous me laisserez seul ; mais je ne suis pas seul, puisque le Père est avec moi. 
${}^{33}Je vous ai parlé ainsi, afin qu’en moi vous ayez la paix. Dans le monde, vous avez à souffrir, mais courage ! Moi, je suis vainqueur du monde. »
      
         
      \bchapter{}
      \begin{verse}
${}^{1}Ainsi parla Jésus. Puis il leva les yeux au ciel et dit : « Père, l’heure est venue. Glorifie ton Fils afin que le Fils te glorifie. 
${}^{2}Ainsi, comme tu lui as donné pouvoir sur tout être de chair, il donnera la vie éternelle à tous ceux que tu lui as donnés. 
${}^{3}Or, la vie éternelle, c’est qu’ils te connaissent, toi le seul vrai Dieu, et celui que tu as envoyé, Jésus Christ.
${}^{4}Moi, je t’ai glorifié sur la terre en accomplissant l’œuvre que tu m’avais donnée à faire. 
${}^{5}Et maintenant, glorifie-moi auprès de toi, Père, de la gloire que j’avais auprès de toi avant que le monde existe. 
${}^{6}J’ai manifesté ton nom aux hommes que tu as pris dans le monde pour me les donner. Ils étaient à toi, tu me les as donnés, et ils ont gardé ta parole. 
${}^{7}Maintenant, ils ont reconnu que tout ce que tu m’as donné vient de toi, 
${}^{8}car je leur ai donné les paroles que tu m’avais données : ils les ont reçues, ils ont vraiment reconnu que je suis sorti de toi, et ils ont cru que tu m’as envoyé.
${}^{9}Moi, je prie pour eux ; ce n’est pas pour le monde que je prie, mais pour ceux que tu m’as donnés, car ils sont à toi. 
${}^{10}Tout ce qui est à moi est à toi, et ce qui est à toi est à moi ; et je suis glorifié en eux. 
${}^{11}Désormais, je ne suis plus dans le monde ; eux, ils sont dans le monde, et moi, je viens vers toi.
      <a class="anchor verset_lettre" id="bib_jn_17_11_b"/>Père saint, garde-les unis dans ton nom, le nom que tu m’as donné, pour qu’ils soient un, comme nous-mêmes. 
${}^{12}Quand j’étais avec eux, je les gardais unis dans ton nom, le nom que tu m’as donné. J’ai veillé sur eux, et aucun ne s’est perdu, sauf celui qui s’en va à sa perte de sorte que l’Écriture soit accomplie. 
${}^{13}Et maintenant que je viens à toi, je parle ainsi, dans le monde, pour qu’ils aient en eux ma joie, et qu’ils en soient comblés. 
${}^{14}Moi, je leur ai donné ta parole, et le monde les a pris en haine parce qu’ils n’appartiennent pas au monde, de même que moi je n’appartiens pas au monde. 
${}^{15}Je ne prie pas pour que tu les retires du monde, mais pour que tu les gardes du Mauvais. 
${}^{16}Ils n’appartiennent pas au monde, de même que moi, je n’appartiens pas au monde. 
${}^{17}Sanctifie-les dans la vérité : ta parole est vérité. 
${}^{18}De même que tu m’as envoyé dans le monde, moi aussi, je les ai envoyés dans le monde. 
${}^{19}Et pour eux je me sanctifie moi-même, afin qu’ils soient, eux aussi, sanctifiés dans la vérité.
${}^{20}Je ne prie pas seulement pour ceux qui sont là, mais encore pour ceux qui, grâce à leur parole, croiront en moi.
${}^{21}Que tous soient un, comme toi, Père, tu es en moi, et moi en toi. Qu’ils soient un en nous, eux aussi, pour que le monde croie que tu m’as envoyé. 
${}^{22}Et moi, je leur ai donné la gloire que tu m’as donnée, pour qu’ils soient un comme nous sommes un : 
${}^{23}moi en eux, et toi en moi. Qu’ils deviennent ainsi parfaitement un, afin que le monde sache que tu m’as envoyé, et que tu les as aimés comme tu m’as aimé.
${}^{24}Père, ceux que tu m’as donnés, je veux que là où je suis, ils soient eux aussi avec moi, et qu’ils contemplent ma gloire, celle que tu m’as donnée parce que tu m’as aimé avant la fondation du monde. 
${}^{25}Père juste, le monde ne t’a pas connu, mais moi je t’ai connu, et ceux-ci ont reconnu que tu m’as envoyé. 
${}^{26}Je leur ai fait connaître ton nom, et je le ferai connaître, pour que l’amour dont tu m’as aimé soit en eux, et que moi aussi, je sois en eux. »
      <h2 class="intertitle" id="d85e364657">2. La Passion (18 – 19)</h2>
      
         
      \bchapter{}
      \begin{verse}
${}^{1}Ayant ainsi parlé, Jésus sortit avec ses disciples et traversa le torrent du Cédron ; il y avait là un jardin, dans lequel il entra avec ses disciples. 
${}^{2}Judas, qui le livrait, connaissait l’endroit, lui aussi, car Jésus et ses disciples s’y étaient souvent réunis. 
${}^{3}Judas, avec un détachement de soldats ainsi que des gardes envoyés par les grands prêtres et les pharisiens, arrive à cet endroit. Ils avaient des lanternes, des torches et des armes.
${}^{4}Alors Jésus, sachant tout ce qui allait lui arriver, s’avança et leur dit : « Qui cherchez-vous ? » 
${}^{5}Ils lui répondirent : « Jésus le Nazaréen. » Il leur dit : « C’est moi, je le suis. » Judas, qui le livrait, se tenait avec eux. 
${}^{6}Quand Jésus leur répondit : « C’est moi, je le suis », ils reculèrent, et ils tombèrent à terre. 
${}^{7}Il leur demanda de nouveau : « Qui cherchez-vous ? » Ils dirent : « Jésus le Nazaréen. » 
${}^{8}Jésus répondit : « Je vous l’ai dit : c’est moi, je le suis. Si c’est bien moi que vous cherchez, ceux-là, laissez-les partir. » 
${}^{9}Ainsi s’accomplissait la parole qu’il avait dite : « Je n’ai perdu aucun de ceux que tu m’as donnés. » 
${}^{10}Or Simon-Pierre avait une épée ; il la tira, frappa le serviteur du grand prêtre et lui coupa l’oreille droite. Le nom de ce serviteur était Malcus. 
${}^{11}Jésus dit à Pierre : « Remets ton épée au fourreau. La coupe que m’a donnée le Père, vais-je refuser de la boire ? »
${}^{12}Alors la troupe, le commandant et les gardes juifs se saisirent de Jésus et le ligotèrent. 
${}^{13}Ils l’emmenèrent d’abord chez Hanne, beau-père de Caïphe qui était grand prêtre cette année-là. 
${}^{14}Caïphe était celui qui avait donné aux Juifs ce conseil : « Il vaut mieux qu’un seul homme meure pour le peuple. »
${}^{15}Or Simon-Pierre, ainsi qu’un autre disciple, suivait Jésus. Comme ce disciple était connu du grand prêtre, il entra avec Jésus dans le palais du grand prêtre. 
${}^{16}Pierre se tenait près de la porte, dehors. Alors l’autre disciple – celui qui était connu du grand prêtre – sortit, dit un mot à la servante qui gardait la porte, et fit entrer Pierre. 
${}^{17}Cette jeune servante dit alors à Pierre : « N’es-tu pas, toi aussi, l’un des disciples de cet homme ? » Il répondit : « Non, je ne le suis pas ! » 
${}^{18}Les serviteurs et les gardes se tenaient là ; comme il faisait froid, ils avaient fait un feu de braise pour se réchauffer. Pierre était avec eux, en train de se chauffer.
${}^{19}Le grand prêtre interrogea Jésus sur ses disciples et sur son enseignement. 
${}^{20}Jésus lui répondit : « Moi, j’ai parlé au monde ouvertement. J’ai toujours enseigné à la synagogue et dans le Temple, là où tous les Juifs se réunissent, et je n’ai jamais parlé en cachette. 
${}^{21}Pourquoi m’interroges-tu ? Ce que je leur ai dit, demande-le à ceux qui m’ont entendu. Eux savent ce que j’ai dit. » 
${}^{22}À ces mots, un des gardes, qui était à côté de Jésus, lui donna une gifle en disant : « C’est ainsi que tu réponds au grand prêtre ! » 
${}^{23}Jésus lui répliqua : « Si j’ai mal parlé, montre ce que j’ai dit de mal ? Mais si j’ai bien parlé, pourquoi me frappes-tu ? » 
${}^{24}Hanne l’envoya, toujours ligoté, au grand prêtre Caïphe.
${}^{25}Simon-Pierre était donc en train de se chauffer. On lui dit : « N’es-tu pas, toi aussi, l’un de ses disciples ? » Pierre le nia et dit : « Non, je ne le suis pas ! » 
${}^{26}Un des serviteurs du grand prêtre, parent de celui à qui Pierre avait coupé l’oreille, insista : « Est-ce que moi, je ne t’ai pas vu dans le jardin avec lui ? » 
${}^{27}Encore une fois, Pierre le nia. Et aussitôt un coq chanta.
${}^{28}Alors on emmène Jésus de chez Caïphe au Prétoire. C’était le matin. Ceux qui l’avaient amené n’entrèrent pas dans le Prétoire, pour éviter une souillure et pouvoir manger l’agneau pascal.
${}^{29}Pilate sortit donc à leur rencontre et demanda : « Quelle accusation portez-vous contre cet homme ? » 
${}^{30}Ils lui répondirent : « S’il n’était pas un malfaiteur, nous ne t’aurions pas livré cet homme. » 
${}^{31}Pilate leur dit : « Prenez-le vous-mêmes et jugez-le suivant votre loi. » Les Juifs lui dirent : « Nous n’avons pas le droit de mettre quelqu’un à mort. » 
${}^{32}Ainsi s’accomplissait la parole que Jésus avait dite pour signifier de quel genre de mort il allait mourir.
${}^{33}Alors Pilate rentra dans le Prétoire ; il appela Jésus et lui dit : « Es-tu le roi des Juifs ? » 
${}^{34}Jésus lui demanda : « Dis-tu cela de toi-même, ou bien d’autres te l’ont dit à mon sujet ? » 
${}^{35}Pilate répondit : « Est-ce que je suis juif, moi ? Ta nation et les grands prêtres t’ont livré à moi : qu’as-tu donc fait ? » 
${}^{36}Jésus déclara : « Ma royauté n’est pas de ce monde ; si ma royauté était de ce monde, j’aurais des gardes qui se seraient battus pour que je ne sois pas livré aux Juifs. En fait, ma royauté n’est pas d’ici. » 
${}^{37}Pilate lui dit : « Alors, tu es roi ? » Jésus répondit : « C’est toi-même qui dis que je suis roi. Moi, je suis né, je suis venu dans le monde pour ceci : rendre témoignage à la vérité. Quiconque appartient à la vérité écoute ma voix. » 
${}^{38}Pilate lui dit : « Qu’est-ce que la vérité ? »
      Ayant dit cela, il sortit de nouveau à la rencontre des Juifs, et il leur déclara : « Moi, je ne trouve en lui aucun motif de condamnation. 
${}^{39}Mais, chez vous, c’est la coutume que je vous relâche quelqu’un pour la Pâque : voulez-vous donc que je vous relâche le roi des Juifs ? » 
${}^{40}Alors ils répliquèrent en criant : « Pas lui ! Mais Barabbas ! » Or ce Barabbas était un bandit.
      
         
      \bchapter{}
      \begin{verse}
${}^{1}Alors Pilate fit saisir Jésus pour qu’il soit flagellé. 
${}^{2}Les soldats tressèrent avec des épines une couronne qu’ils lui posèrent sur la tête ; puis ils le revêtirent d’un manteau pourpre. 
${}^{3}Ils s’avançaient vers lui et ils disaient : « Salut à toi, roi des Juifs ! » Et ils le giflaient.
${}^{4}Pilate, de nouveau, sortit dehors et leur dit : « Voyez, je vous l’amène dehors pour que vous sachiez que je ne trouve en lui aucun motif de condamnation. » 
${}^{5}Jésus donc sortit dehors, portant la couronne d’épines et le manteau pourpre. Et Pilate leur déclara : « Voici l’homme. »
${}^{6}Quand ils le virent, les grands prêtres et les gardes se mirent à crier : « Crucifie-le ! Crucifie-le ! » Pilate leur dit : « Prenez-le vous-mêmes, et crucifiez-le ; moi, je ne trouve en lui aucun motif de condamnation. » 
${}^{7}Ils lui répondirent : « Nous avons une Loi, et suivant la Loi il doit mourir, parce qu’il s’est fait Fils de Dieu. »
${}^{8}Quand Pilate entendit ces paroles, il redoubla de crainte. 
${}^{9}Il rentra dans le Prétoire, et dit à Jésus : « D’où es-tu ? » Jésus ne lui fit aucune réponse. 
${}^{10}Pilate lui dit alors : « Tu refuses de me parler, à moi ? Ne sais-tu pas que j’ai pouvoir de te relâcher, et pouvoir de te crucifier ? » 
${}^{11}Jésus répondit : « Tu n’aurais aucun pouvoir sur moi si tu ne l’avais reçu d’en haut ; c’est pourquoi celui qui m’a livré à toi porte un péché plus grand. »
${}^{12}Dès lors, Pilate cherchait à le relâcher ; mais des Juifs se mirent à crier : « Si tu le relâches, tu n’es pas un ami de l’empereur. Quiconque se fait roi s’oppose à l’empereur. » 
${}^{13}En entendant ces paroles, Pilate amena Jésus au-dehors ; il le fit asseoir sur une estrade au lieu dit le Dallage – en hébreu : Gabbatha. 
${}^{14}C’était le jour de la Préparation de la Pâque, vers la sixième heure, environ midi. Pilate dit aux Juifs : « Voici votre roi. » 
${}^{15}Alors ils crièrent : « À mort ! À mort ! Crucifie-le ! » Pilate leur dit : « Vais-je crucifier votre roi ? » Les grands prêtres répondirent : « Nous n’avons pas d’autre roi que l’empereur. » 
${}^{16}Alors, il leur livra Jésus pour qu’il soit crucifié.
      Ils se saisirent de Jésus. 
${}^{17}Et lui-même, portant sa croix, sortit en direction du lieu dit Le Crâne (ou Calvaire), qui se dit en hébreu Golgotha. 
${}^{18}C’est là qu’ils le crucifièrent, et deux autres avec lui, un de chaque côté, et Jésus au milieu.
${}^{19}Pilate avait rédigé un écriteau qu’il fit placer sur la croix ; il était écrit : « Jésus le Nazaréen, roi des Juifs. » 
${}^{20}Beaucoup de Juifs lurent cet écriteau, parce que l’endroit où l’on avait crucifié Jésus était proche de la ville, et que c’était écrit en hébreu, en latin et en grec. 
${}^{21}Alors les grands prêtres des Juifs dirent à Pilate : « N’écris pas : “Roi des Juifs” ; mais : “Cet homme a dit : Je suis le roi des Juifs”. » 
${}^{22}Pilate répondit : « Ce que j’ai écrit, je l’ai écrit. »
${}^{23}Quand les soldats eurent crucifié Jésus, ils prirent ses habits ; ils en firent quatre parts, une pour chaque soldat. Ils prirent aussi la tunique ; c’était une tunique sans couture, tissée tout d’une pièce de haut en bas. 
${}^{24}Alors ils se dirent entre eux : « Ne la déchirons pas, désignons par le sort celui qui l’aura. » Ainsi s’accomplissait la parole de l’Écriture :
        \\Ils se sont partagé mes habits ;
        \\ils ont tiré au sort mon vêtement .
      C’est bien ce que firent les soldats.
${}^{25}Or, près de la croix de Jésus se tenaient sa mère et la sœur de sa mère, Marie, femme de Cléophas, et Marie Madeleine. 
${}^{26}Jésus, voyant sa mère, et près d’elle le disciple qu’il aimait, dit à sa mère : « Femme, voici ton fils. » 
${}^{27}Puis il dit au disciple : « Voici ta mère. » Et à partir de cette heure-là, le disciple la prit chez lui.
${}^{28}Après cela, sachant que tout, désormais, était achevé pour que l’Écriture s’accomplisse jusqu’au bout, Jésus dit : « J’ai soif. » 
${}^{29}Il y avait là un récipient plein d’une boisson vinaigrée. On fixa donc une éponge remplie de ce vinaigre à une branche d’hysope, et on l’approcha de sa bouche. 
${}^{30}Quand il eut pris le vinaigre, Jésus dit : « Tout est accompli. » Puis, inclinant la tête, il remit l’esprit.
${}^{31}Comme c’était le jour de la Préparation (c’est-à-dire le vendredi), il ne fallait pas laisser les corps en croix durant le sabbat, d’autant plus que ce sabbat était le grand jour de la Pâque. Aussi les Juifs demandèrent à Pilate qu’on enlève les corps après leur avoir brisé les jambes. 
${}^{32}Les soldats allèrent donc briser les jambes du premier, puis de l’autre homme crucifié avec Jésus. 
${}^{33}Quand ils arrivèrent à Jésus, voyant qu’il était déjà mort, ils ne lui brisèrent pas les jambes, 
${}^{34}mais un des soldats avec sa lance lui perça le côté ; et aussitôt, il en sortit du sang et de l’eau.
${}^{35}Celui qui a vu rend témoignage, et son témoignage est véridique ; et celui-là sait qu’il dit vrai afin que vous aussi, vous croyiez.
${}^{36}Cela, en effet, arriva pour que s’accomplisse l’Écriture :
        \\Aucun de ses os ne sera brisé .
${}^{37}Un autre passage de l’Écriture dit encore :
        \\Ils lèveront les yeux vers celui qu’ils ont transpercé.
${}^{38}Après cela, Joseph d’Arimathie, qui était disciple de Jésus, mais en secret par crainte des Juifs, demanda à Pilate de pouvoir enlever le corps de Jésus. Et Pilate le permit. Joseph vint donc enlever le corps de Jésus. 
${}^{39}Nicodème – celui qui, au début, était venu trouver Jésus pendant la nuit – vint lui aussi ; il apportait un mélange de myrrhe et d’aloès pesant environ cent livres. 
${}^{40}Ils prirent donc le corps de Jésus, qu’ils lièrent de linges, en employant les aromates selon la coutume juive d’ensevelir les morts.
${}^{41}À l’endroit où Jésus avait été crucifié, il y avait un jardin et, dans ce jardin, un tombeau neuf dans lequel on n’avait encore déposé personne. 
${}^{42}À cause de la Préparation de la Pâque juive, et comme ce tombeau était proche, c’est là qu’ils déposèrent Jésus.
      <h2 class="intertitle" id="d85e365428">3. La Résurrection (20,1-29)</h2>
      
         
      \bchapter{}
      \begin{verse}
${}^{1}Le premier jour de la semaine, Marie Madeleine se rend au tombeau de grand matin ; c’était encore les ténèbres. Elle s’aperçoit que la pierre a été enlevée du tombeau. 
${}^{2}Elle court donc trouver Simon-Pierre et l’autre disciple, celui que Jésus aimait, et elle leur dit : « On a enlevé le Seigneur de son tombeau, et nous ne savons pas où on l’a déposé. »
${}^{3}Pierre partit donc avec l’autre disciple pour se rendre au tombeau. 
${}^{4}Ils couraient tous les deux ensemble, mais l’autre disciple courut plus vite que Pierre et arriva le premier au tombeau. 
${}^{5}En se penchant, il s’aperçoit que les linges sont posés à plat ; cependant il n’entre pas. 
${}^{6}Simon-Pierre, qui le suivait, arrive à son tour. Il entre dans le tombeau ; il aperçoit les linges, posés à plat, 
${}^{7}ainsi que le suaire qui avait entouré la tête de Jésus, non pas posé avec les linges, mais roulé à part à sa place.
${}^{8}C’est alors qu’entra l’autre disciple, lui qui était arrivé le premier au tombeau. Il vit, et il crut. 
${}^{9}Jusque-là, en effet, les disciples n’avaient pas compris que, selon l’Écriture, il fallait que Jésus ressuscite d’entre les morts. 
${}^{10}Ensuite, les disciples retournèrent chez eux.
${}^{11}Marie Madeleine se tenait près du tombeau, au-dehors, tout en pleurs. Et en pleurant, elle se pencha vers le tombeau. 
${}^{12}Elle aperçoit deux anges vêtus de blanc, assis l’un à la tête et l’autre aux pieds, à l’endroit où avait reposé le corps de Jésus. 
${}^{13}Ils lui demandent : « Femme, pourquoi pleures-tu ? » Elle leur répond : « On a enlevé mon Seigneur, et je ne sais pas où on l’a déposé. » 
${}^{14}Ayant dit cela, elle se retourna ; elle aperçoit Jésus qui se tenait là, mais elle ne savait pas que c’était Jésus. 
${}^{15}Jésus lui dit : « Femme, pourquoi pleures-tu ? Qui cherches-tu ? » Le prenant pour le jardinier, elle lui répond : « Si c’est toi qui l’as emporté, dis-moi où tu l’as déposé, et moi, j’irai le prendre. » 
${}^{16}Jésus lui dit alors : « Marie ! » S’étant retournée, elle lui dit en hébreu : « Rabbouni ! », c’est-à-dire : Maître. 
${}^{17}Jésus reprend : « Ne me retiens pas, car je ne suis pas encore monté vers le Père. Va trouver mes frères pour leur dire que je monte vers mon Père et votre Père, vers mon Dieu et votre Dieu. » 
${}^{18}Marie Madeleine s’en va donc annoncer aux disciples : « J’ai vu le Seigneur ! », et elle raconta ce qu’il lui avait dit.
${}^{19}Le soir venu, en ce premier jour de la semaine, alors que les portes du lieu où se trouvaient les disciples étaient verrouillées par crainte des Juifs, Jésus vint, et il était là au milieu d’eux. Il leur dit : « La paix soit avec vous ! » 
${}^{20}Après cette parole, il leur montra ses mains et son côté. Les disciples furent remplis de joie en voyant le Seigneur. 
${}^{21}Jésus leur dit de nouveau : « La paix soit avec vous ! De même que le Père m’a envoyé, moi aussi, je vous envoie. » 
${}^{22}Ayant ainsi parlé, il souffla sur eux et il leur dit : « Recevez l’Esprit Saint. 
${}^{23}À qui vous remettrez ses péchés, ils seront remis ; à qui vous maintiendrez ses péchés, ils seront maintenus. »
${}^{24}Or, l’un des Douze, Thomas, appelé Didyme (c’est-à-dire Jumeau), n’était pas avec eux quand Jésus était venu. 
${}^{25}Les autres disciples lui disaient : « Nous avons vu le Seigneur ! » Mais il leur déclara : « Si je ne vois pas dans ses mains la marque des clous, si je ne mets pas mon doigt dans la marque des clous, si je ne mets pas la main dans son côté, non, je ne croirai pas ! »
${}^{26}Huit jours plus tard, les disciples se trouvaient de nouveau dans la maison, et Thomas était avec eux. Jésus vient, alors que les portes étaient verrouillées, et il était là au milieu d’eux. Il dit : « La paix soit avec vous ! » 
${}^{27}Puis il dit à Thomas : « Avance ton doigt ici, et vois mes mains ; avance ta main, et mets-la dans mon côté : cesse d’être incrédule, sois croyant. » 
${}^{28}Alors Thomas lui dit : « Mon Seigneur et mon Dieu ! » 
${}^{29}Jésus lui dit : « Parce que tu m’as vu, tu crois. Heureux ceux qui croient sans avoir vu. »
      <h2 class="intertitle" id="d85e365675">4. Conclusion du livre de la gloire (20,30-31)</h2>
${}^{30}Il y a encore beaucoup d’autres signes que Jésus a faits en présence des disciples et qui ne sont pas écrits dans ce livre. 
${}^{31}Mais ceux-là ont été écrits pour que vous croyiez que Jésus est le Christ, le Fils de Dieu, et pour qu’en croyant, vous ayez la vie en son nom.
      
         
      \bchapter{}
      \begin{verse}
${}^{1}Après cela, Jésus se manifesta encore aux disciples sur le bord de la mer de Tibériade, et voici comment. 
${}^{2}Il y avait là, ensemble, Simon-Pierre, avec Thomas, appelé Didyme (c’est-à-dire Jumeau), Nathanaël, de Cana de Galilée, les fils de Zébédée, et deux autres de ses disciples. 
${}^{3}Simon-Pierre leur dit : « Je m’en vais à la pêche. » Ils lui répondent : « Nous aussi, nous allons avec toi. » Ils partirent et montèrent dans la barque ; or, cette nuit-là, ils ne prirent rien.
${}^{4}Au lever du jour, Jésus se tenait sur le rivage, mais les disciples ne savaient pas que c’était lui. 
${}^{5}Jésus leur dit : « Les enfants, auriez-vous quelque chose à manger ? » Ils lui répondirent : « Non. » 
${}^{6}Il leur dit : « Jetez le filet à droite de la barque, et vous trouverez. » Ils jetèrent donc le filet, et cette fois ils n’arrivaient pas à le tirer, tellement il y avait de poissons. 
${}^{7}Alors, le disciple que Jésus aimait dit à Pierre : « C’est le Seigneur ! » Quand Simon-Pierre entendit que c’était le Seigneur, il passa un vêtement, car il n’avait rien sur lui, et il se jeta à l’eau. 
${}^{8}Les autres disciples arrivèrent en barque, traînant le filet plein de poissons ; la terre n’était qu’à une centaine de mètres.
${}^{9}Une fois descendus à terre, ils aperçoivent, disposé là, un feu de braise avec du poisson posé dessus, et du pain. 
${}^{10}Jésus leur dit : « Apportez donc de ces poissons que vous venez de prendre. » 
${}^{11}Simon-Pierre remonta et tira jusqu’à terre le filet plein de gros poissons : il y en avait cent cinquante-trois. Et, malgré cette quantité, le filet ne s’était pas déchiré. 
${}^{12}Jésus leur dit alors : « Venez manger. » Aucun des disciples n’osait lui demander : « Qui es-tu ? » Ils savaient que c’était le Seigneur. 
${}^{13}Jésus s’approche ; il prend le pain et le leur donne ; et de même pour le poisson. 
${}^{14}C’était la troisième fois que Jésus ressuscité d’entre les morts se manifestait à ses disciples.
${}^{15}Quand ils eurent mangé, Jésus dit à Simon-Pierre : « Simon, fils de Jean, m’aimes-tu vraiment, plus que ceux-ci ? » Il lui répond : « Oui, Seigneur ! Toi, tu le sais : je t’aime. » Jésus lui dit : « Sois le berger de mes agneaux. »
${}^{16}Il lui dit une deuxième fois : « Simon, fils de Jean, m’aimes-tu vraiment ? » Il lui répond : « Oui, Seigneur ! Toi, tu le sais : je t’aime. » Jésus lui dit : « Sois le pasteur de mes brebis. »
${}^{17}Il lui dit, pour la troisième fois : « Simon, fils de Jean, m’aimes-tu ? » Pierre fut peiné parce que, la troisième fois, Jésus lui demandait : « M’aimes-tu ? » Il lui répond : « Seigneur, toi, tu sais tout : tu sais bien que je t’aime. » Jésus lui dit : « Sois le berger de mes brebis. 
${}^{18}Amen, amen, je te le dis : quand tu étais jeune, tu mettais ta ceinture toi-même pour aller là où tu voulais ; quand tu seras vieux, tu étendras les mains, et c’est un autre qui te mettra ta ceinture, pour t’emmener là où tu ne voudrais pas aller. »
${}^{19}Jésus disait cela pour signifier par quel genre de mort Pierre rendrait gloire à Dieu. Sur ces mots, il lui dit : « Suis-moi. »
${}^{20}S’étant retourné, Pierre aperçoit, marchant à leur suite, le disciple que Jésus aimait. C’est lui qui, pendant le repas, s’était penché sur la poitrine de Jésus pour lui dire : « Seigneur, quel est celui qui va te livrer ? » 
${}^{21}Pierre, voyant donc ce disciple, dit à Jésus : « Et lui, Seigneur, que lui arrivera-t-il ? » 
${}^{22}Jésus lui répond : « Si je veux qu’il demeure jusqu’à ce que je vienne, que t’importe ? Toi, suis-moi. » 
${}^{23}Le bruit courut donc parmi les frères que ce disciple ne mourrait pas. Or, Jésus n’avait pas dit à Pierre qu’il ne mourrait pas, mais : « Si je veux qu’il demeure jusqu’à ce que je vienne, que t’importe ? »
${}^{24}C’est ce disciple qui témoigne de ces choses et qui les a écrites, et nous savons que son témoignage est vrai. 
${}^{25}Il y a encore beaucoup d’autres choses que Jésus a faites ; et s’il fallait écrire chacune d’elles, je pense que le monde entier ne suffirait pas pour contenir les livres que l’on écrirait.
