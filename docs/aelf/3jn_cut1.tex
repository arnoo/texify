  
  
    
    \bbook{TROISIÈME LETTRE DE SAINT JEAN}{TROISIÈME LETTRE DE SAINT JEAN}
      <a class="bib_chap hidden" id="bib_3jn_1"/>
${}^{1}Moi, l’Ancien,
        \\à Gaïos, le bien-aimé,
        que j’aime en vérité.
${}^{2}Bien-aimé, je prie pour qu’en toutes choses tu ailles bien et que tu sois en bonne santé, comme c’est déjà le cas pour ton âme. 
${}^{3}J’ai eu beaucoup de joie quand des frères sont venus et qu’ils ont rendu témoignage à la vérité qui est en toi : ils ont dit comment tu marches dans la vérité. 
${}^{4}Rien ne me donne plus de joie que d’apprendre que mes enfants marchent dans la vérité.
${}^{5}Bien-aimé, tu agis fidèlement dans ce que tu fais pour les frères, et particulièrement pour des étrangers. 
${}^{6}En présence de l’Église, ils ont rendu témoignage à ta charité ; tu feras bien de faciliter leur voyage d’une manière digne de Dieu. 
${}^{7}Car c’est pour son nom qu’ils se sont mis en route sans rien recevoir des païens. 
${}^{8}Nous devons donc apporter notre soutien à de tels hommes pour être des collaborateurs de la vérité.
${}^{9}J’ai écrit une lettre à l’Église ; mais Diotréphès, qui aime tant être le premier d’entre eux, ne nous accueille pas. 
${}^{10}Alors si je viens, je dénoncerai les œuvres qu’il accomplit : il se répand en paroles méchantes contre nous ; non content de cela, il n’accueille pas les frères ; et ceux qui voudraient le faire, il les en empêche et les chasse de l’Église.
${}^{11}Bien-aimé, n’imite pas le mal, mais le bien. Celui qui fait le bien vient de Dieu ; celui qui fait le mal n’a pas vu Dieu.
${}^{12}Quant à Démétrios, il fait l’objet d’un bon témoignage de la part de tous et de la vérité elle-même ; nous aussi, nous lui rendons témoignage, et tu sais que notre témoignage est vrai.
${}^{13}J’aurais bien des choses à t’écrire, mais je ne veux pas le faire avec l’encre et la plume. 
${}^{14}J’espère te voir bientôt, et nous nous parlerons de vive voix.
${}^{15}La paix soit avec toi ! Les amis te saluent. Et toi, salue les amis, chacun par son nom.
