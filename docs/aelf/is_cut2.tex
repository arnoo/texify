  
  
      <h2 class="intertitle" id="d85e238987">2. Livre de l’Emmanuel (6 – 12)</h2>
      
         
      \bchapter{}
      \begin{verse}
${}^{1}L’année de la mort du roi Ozias, je vis le Seigneur qui siégeait sur un trône très élevé\\ ; les pans de son manteau remplissaient le Temple. 
${}^{2} Des séraphins se tenaient au-dessus de lui. Ils avaient chacun six ailes : deux pour se couvrir le visage, deux pour se couvrir les pieds, et deux pour voler. 
${}^{3} Ils se criaient l’un à l’autre : « Saint ! Saint ! Saint, le Seigneur de l’univers ! Toute la terre est remplie de sa gloire. » 
${}^{4} Les pivots des portes se mirent à trembler à la voix de celui qui criait, et le Temple\\se remplissait de fumée. 
${}^{5} Je dis alors : « Malheur à moi ! je suis perdu\\, car je suis un homme aux lèvres impures, j’habite au milieu d’un peuple aux lèvres impures : et mes yeux ont vu le Roi, le Seigneur de l’univers ! » 
${}^{6} L’un des séraphins vola vers moi, tenant un charbon brûlant qu’il avait pris avec des pinces sur l’autel. 
${}^{7} Il l’approcha de ma bouche et dit : « Ceci a touché tes lèvres, et maintenant ta faute est enlevée, ton péché est pardonné. »
${}^{8}J’entendis alors la voix du Seigneur qui disait : « Qui enverrai-je ? qui sera notre messager\\ ? » Et j’ai répondu : « Me voici : envoie-moi ! »
${}^{9}Il me dit :
        \\« Va dire à ce peuple :
        \\Écoutez bien, mais sans comprendre ;
        regardez bien, mais sans reconnaître.
${}^{10}Alourdis le cœur de ce peuple,
        rends-le dur d’oreille,
        aveugle ses yeux,
        \\de peur que ses yeux ne voient,
        que ses oreilles n’entendent,
        que son cœur ne comprenne,
        \\qu’il ne se convertisse
        et ne soit guéri. »
${}^{11}Et je dis :
        « Jusqu’à quand, Seigneur ? »
        \\Il répondit :
        \\« Jusqu’à ce que les villes
        soient ravagées, dépeuplées,
        \\les maisons, sans habitants,
        et la terre, désolée, ravagée,
${}^{12}jusqu’à ce que le Seigneur
        en ait éloigné les habitants,
        \\et que se multiplient dans le pays
        les terres abandonnées.
${}^{13}Et s’il en reste un dixième,
        à son tour, il sera détruit,
        \\comme le chêne et le térébinthe abattus
        dont il ne reste que la souche.
        \\– Cette souche est une semence sainte. »
      
         
      \bchapter{}
      \begin{verse}
${}^{1}Au temps d’Acaz, fils de Yotam, fils d’Ozias, roi de Juda, Recine, roi d’Aram, et Pékah, fils de Remalyahou, roi d’Israël, montèrent contre Jérusalem pour l’attaquer, mais ils ne purent lui donner l’assaut. 
${}^{2}On informa la maison de David que les Araméens avaient pris position en Éphraïm. Alors le cœur du roi\\et le cœur de son peuple furent secoués comme les arbres de la forêt sont secoués par le vent.
${}^{3}Le Seigneur dit alors à Isaïe : « Avec ton fils Shear-Yashoub (c’est-à-dire : Un-reste-reviendra)\\, va trouver Acaz, au bout du canal du réservoir supérieur, sur la route du Champ-du-Foulon.
        ${}^{4}Tu lui diras :
        \\“Garde ton calme, ne crains pas,
        ne va pas perdre cœur
        devant ces deux bouts de tisons fumants,
        \\à cause de la colère brûlante du roi d’Aram
        et du roi d’Israël\\,
        ${}^{5}Oui, Aram a décidé ta perte,
        en accord avec Éphraïm et son roi\\.
        \\Ils se sont dit :
        ${}^{6}Marchons contre le royaume\\de Juda,
        pour l’intimider\\,
        \\et nous le forcerons à se rendre ;
        alors, nous lui imposerons comme roi le fils de Tabéel.
        ${}^{7}Ainsi parle le Seigneur Dieu :
        \\Cela ne durera pas, ne sera pas,
        ${}^{8}que la capitale d’Aram soit Damas,
        et Recine, le chef de Damas,
        ${}^{9}que la capitale d’Éphraïm soit Samarie,
        et le fils de Remalyahou, chef de Samarie.
        \\– Dans soixante-cinq ans, Éphraïm, écrasé,
        cessera d’être un peuple\\.
        \\Mais vous, si vous ne croyez pas,
        vous ne pourrez pas tenir\\.” »
${}^{10}Le Seigneur parla encore ainsi au roi\\Acaz : 
${}^{11} « Demande pour toi un signe de la part du Seigneur ton Dieu, au fond du séjour des morts ou sur les sommets, là-haut. » 
${}^{12} Acaz répondit : « Non, je n’en demanderai pas, je ne mettrai pas le Seigneur à l’épreuve. »
${}^{13}Isaïe\\dit alors :
        \\« Écoutez, maison de David !
        \\Il ne vous suffit donc pas de fatiguer les hommes :
        \\il faut encore que vous fatiguiez mon Dieu !
        ${}^{14}C’est pourquoi le Seigneur lui-même
        \\vous donnera un signe :
        \\Voici que la vierge\\est enceinte,
        \\elle enfantera un fils,
        \\qu’elle appellera Emmanuel
        \\(c’est-à-dire : Dieu-avec-nous)\\.
        ${}^{15}De crème et de miel il se nourrira,
        \\jusqu’à ce qu’il sache rejeter le mal et choisir le bien.
        ${}^{16}Avant que cet enfant sache rejeter le mal
        et choisir le bien,
        \\la terre dont les deux rois te font trembler
        sera laissée à l’abandon.
${}^{17}Le Seigneur fera venir sur toi,
        sur ton peuple et la maison de ton père,
        \\des jours tels qu’il n’en est pas venu
        depuis la séparation d’Éphraïm et de Juda.
${}^{18}Il arrivera, en ce jour-là,
        que le Seigneur sifflera les mouches
        \\depuis les embouchures des fleuves d’Égypte
        et les guêpes du pays d’Assour.
${}^{19}Elles viendront et toutes se poseront
        dans le fond des ravins et les fentes des rochers,
        \\sur toutes les broussailles et tous les pacages.
${}^{20}Ce jour-là, le Seigneur rasera
        avec un rasoir loué au-delà de l’Euphrate,
        – c’est le roi d’Assour –,
        \\il rasera de la tête aux pieds ;
        il coupera même la barbe.
         
${}^{21}Il arrivera, en ce jour-là,
        que chacun élèvera une vache et deux chèvres ;
${}^{22}il y aura tant de lait
        qu’on en mangera la crème ;
        \\tous ceux qui resteront au cœur du pays
        se nourriront de crème et de miel.
         
${}^{23}Il arrivera, en ce jour-là, que tout lieu
        planté de mille vignes
        \\et valant mille pièces d’argent
        ne sera que des épines et des ronces.
${}^{24}On y viendra avec un arc et des flèches ;
        oui, tout le pays ne sera que des épines et des ronces.
${}^{25}Sur tous les coteaux bêchés et sarclés,
        on ne viendra plus par crainte des épines et des ronces ;
        \\on lâchera sur eux le gros bétail,
        et les moutons viendront les piétiner.
      
         
      \bchapter{}
${}^{1}Et le Seigneur me dit :
        « Prends une grande tablette ;
        \\écris dessus avec un simple stylet :
        Pour Maher-Shalal-Hash-Baz,
        (c’est-à-dire : Cours-au-butin-Vite-au-pillage). »
${}^{2}Je choisis des témoins dignes de foi,
        le prêtre Ourias et Zacharie, fils de Barakie.
${}^{3}Alors j’approchai la prophétesse, ma femme :
        elle devint enceinte et enfanta un fils.
        \\Et le Seigneur me dit :
        « Appelle-le : Maher-Shalal-Hash-Baz
${}^{4}car avant que l’enfant sache dire
        “papa” et “maman”,
        \\on aura porté chez le roi d’Assour
        les richesses de Damas et le butin de Samarie. »
        
           
         
${}^{5}Le Seigneur me parla encore une fois ;
        il me dit :
${}^{6}« Puisque ce peuple a dédaigné les eaux de Siloé
        qui s’écoulent paisiblement,
        \\puisqu’il perd courage devant Recine
        et le fils de Remalyahou,
${}^{7}eh bien ! voici que le Seigneur
        va faire monter contre eux
        \\les eaux puissantes et abondantes de l’Euphrate
        – c’est le roi d’Assour et toute sa gloire.
        \\Partout le fleuve sortira de son lit,
        il franchira toutes ses digues.
${}^{8}Il forcera l’entrée de Juda,
        il l’inondera, le submergera,
        montera jusqu’à son cou.
        \\Ses ailes déployées
        couvriront l’étendue de ton pays,
        ô Emmanuel (c’est-à-dire : Dieu-avec-nous).
${}^{9}Peuples, poussez le cri de guerre : vous tomberez !
        Écoutez bien, tous les pays lointains :
        \\Armez-vous, vous tomberez !
        Armez-vous, vous tomberez !
${}^{10}Dressez vos plans, ils s’effondreront !
        Dites une parole, elle ne tiendra pas,
        car Dieu est avec nous. »
        
           
${}^{11}Ainsi m’a parlé le Seigneur
        lorsque sa main me saisissait
        \\et qu’il m’enjoignait de ne pas suivre
        le chemin de ce peuple.
        Il disait :
${}^{12}« Ne nommez pas “complot”
        tout ce que ce peuple nomme “complot” ;
        \\ce qu’il craint, ne le craignez pas,
        ne le redoutez pas.
${}^{13}C’est le Seigneur de l’univers
        que vous tiendrez pour saint ;
        \\c’est lui que vous craindrez,
        lui que vous redouterez.
${}^{14}Il deviendra un lieu saint,
        qui sera une pierre d’achoppement,
        \\un roc faisant trébucher les deux maisons d’Israël,
        piège et filet pour l’habitant de Jérusalem.
${}^{15}Beaucoup trébucheront,
        ils tomberont, se briseront,
        piégés et capturés. »
         
${}^{16}Conserve ce témoignage,
        appose un sceau sur cet enseignement
        pour mes disciples.
${}^{17}J’attends le Seigneur qui cache sa face à la maison de Jacob,
        j’espère en lui.
${}^{18}Moi et les enfants que le Seigneur m’a donnés,
        nous sommes des signes et des présages pour Israël
        \\de la part du Seigneur de l’univers
        qui habite le mont Sion.
         
${}^{19}On vous dira peut-être :
        « Consultez les devins et les nécromanciens,
        qui chuchotent et marmonnent !
        \\Un peuple ne doit-il pas consulter ses dieux
        au sujet des vivants,
        et consulter aussi ses morts ? »
${}^{20}Non ! Revenez à l’enseignement et au témoignage.
        \\Malheur à qui ne s’en tient pas à cette parole :
        on ne pourra en conjurer la malédiction.
${}^{21}On traversera le pays,
        accablé, affamé,
        \\rendu furieux par la faim,
        on maudira son roi et son Dieu,
        \\on se tournera vers le ciel,
${}^{22}on regardera vers la terre :
        \\il n’y aura que détresse et ténèbres,
        nuit d’angoisse,
        obscurité inéluctable.
        ${}^{23}Pas la moindre lueur
        pour celui qui sera dans l’angoisse.
        \\Dans un premier temps, le Seigneur\\a couvert de honte
        le pays de Zabulon et le pays de Nephtali ;
        \\mais ensuite, il a couvert de gloire
        la route de la mer, le pays au-delà du Jourdain,
        et la Galilée des nations.
       
      <p class="cantique" id="bib_ct-at_18"><span class="cantique_label">Cantique AT 18</span> = <span class="cantique_ref"><a class="unitex_link" href="#bib_is_9_1">Is 9, 1-6</a></span>
      
         
      \bchapter{}
        ${}^{1}Le peuple qui marchait dans les ténèbres
        a vu se lever\\une grande lumière ;
        \\et sur les habitants du pays de l’ombre\\,
        une lumière a resplendi.
        
           
         
        ${}^{2}Tu as prodigué la joie\\,
        \\tu as fait grandir l’allégresse :
        \\ils se réjouissent devant toi,
        \\comme on se réjouit de la moisson,
        \\comme on exulte au partage du butin.
        
           
         
        ${}^{3}Car le joug qui pesait sur lui,
        \\la barre qui meurtrissait\\son épaule,
        le bâton du tyran,
        \\tu les as brisés comme au jour de Madiane.
        
           
         
        ${}^{4}Et les bottes qui frappaient le sol\\,
        \\et les manteaux couverts de sang\\,
        les voilà tous brûlés :
        \\le feu les a dévorés.
        
           
         
        ${}^{5}Oui, un enfant nous est né,
        \\un fils nous a été donné !
        \\Sur son épaule est le signe du pouvoir ;
        \\son nom est proclamé :
        « Conseiller-merveilleux, Dieu-Fort,
        Père-à-jamais, Prince-de-la-Paix ».
        
           
         
        ${}^{6}Et le pouvoir s’étendra,
        \\et la paix sera sans fin
        \\pour le trône de David et pour son règne
        \\qu’il établira, qu’il affermira
        sur le droit et la justice
        \\dès maintenant et pour toujours.
        
           
         
        \\Il fera cela, l’amour jaloux du Seigneur de l’univers\\ !
        
           
${}^{7}Le Seigneur a lancé une parole dans le pays de Jacob :
        en Israël, elle est tombée.
${}^{8}Tout le peuple la connaîtra,
        Éphraïm et l’habitant de Samarie ;
        \\dans leur orgueil et leur superbe, ils disent :
${}^{9}« Les briques sont tombées :
        nous rebâtirons en pierres de taille !
        \\Le sycomore s’est abattu :
        nous le remplacerons par du cèdre. »
${}^{10}Le Seigneur a dressé contre Israël des adversaires – Recine –,
        il a excité ses ennemis :
${}^{11}Aram à l’est, les Philistins à l’ouest ;
        ils ont dévoré Israël à pleine bouche.
        \\Et avec tout cela, sa colère ne s’est pas détournée,
        sa main reste levée.
         
${}^{12}Le peuple ne s’est pas retourné vers celui qui le frappait,
        il n’a pas recherché le Seigneur de l’univers.
${}^{13}D’Israël le Seigneur a tranché la tête et la queue,
        le palmier et le roseau, en un seul jour,
${}^{14}– la tête, c’est l’ancien et le notable ;
        la queue, c’est le prophète, maître de mensonge.
${}^{15}Ceux qui guidaient ce peuple l’ont fourvoyé,
        la piste de ceux qu’ils guidaient a été brouillée.
${}^{16}Voilà pourquoi le Seigneur ne ménage pas leurs jeunes gens,
        pour leurs orphelins et leurs veuves il n’a plus de compassion :
        \\tout le peuple est impie, malfaisant,
        toute bouche profère l’infamie.
        \\Et avec tout cela, sa colère ne s’est pas détournée,
        sa main reste levée.
         
${}^{17}Car la méchanceté brûle comme un feu,
        dévorant épines et ronces,
        \\enflammant les taillis de la forêt
        qui tourbillonnent en colonnes de fumée.
${}^{18}Par la fureur du Seigneur de l’univers,
        le pays est en flammes.
        \\Le peuple devient comme la proie du feu :
        nul n’épargne son frère.
${}^{19}On découpe à droite, et l’on reste affamé ;
        on dévore à gauche, on n’est pas rassasié !
        \\Chacun dévore la chair de son prochain.
${}^{20}Manassé dévore Éphraïm, et Éphraïm, Manassé,
        tous deux unis contre Juda.
        \\Et avec tout cela, sa colère ne s’est pas détournée,
        sa main reste levée.
      
         
      \bchapter{}
${}^{1}Malheureux ! Ils rédigent des décrets malfaisants,
        ils inscrivent des écrits d’oppression !
${}^{2}Ils refusent de rendre justice aux faibles,
        et privent de leurs droits les pauvres de mon peuple ;
        \\les veuves deviennent leurs proies,
        et ils dépouillent les orphelins !
${}^{3}Que ferez-vous au jour du châtiment,
        au jour d’une tourmente venue de loin ?
        \\Vers qui fuirez-vous pour demander secours ?
        Où laisserez-vous vos richesses ?
${}^{4}Il ne restera qu’à se courber parmi les prisonniers,
        on succombera parmi les cadavres.
        \\Et avec tout cela, sa colère ne s’est pas détournée,
        sa main reste levée.
        
           
        ${}^{5}Malheureux ! Assour, l’instrument de ma colère,
        le bâton\\de mon courroux\\.
        ${}^{6}Je l’envoie contre une nation impie,
        je lui donne mission contre un peuple qui excite ma fureur,
        \\pour le mettre au pillage et emporter le butin,
        pour le piétiner comme la boue des chemins.
        ${}^{7}Mais Assour ne l’entend pas ainsi,
        ce n’est pas du tout ce qu’il pense :
        \\ce qu’il veut, c’est détruire,
        exterminer quantité de nations.
${}^{8}Car il dit : « Tous mes officiers ne valent-ils pas des rois ?
        N’ai-je pas pris Kalno comme Karkémish,
${}^{9}Hamath comme Arpad,
        et Samarie comme Damas ?
${}^{10}Comme j’ai mis la main sur les royaumes des faux dieux
        et leurs idoles plus nombreuses qu’à Jérusalem et à Samarie,
${}^{11}n’ai-je pas traité ainsi Samarie et ses faux dieux ?
        ne puis-je pas traiter ainsi Jérusalem et ses images ? »
         
${}^{12}Mais quand le Seigneur aura fini son œuvre
        au mont Sion et à Jérusalem,
        \\il châtiera le roi d’Assour
        pour les fruits de son cœur arrogant,
        pour l’orgueil de ses regards hautains.
        ${}^{13}Car le roi d’Assour a dit :
        \\« C’est par la vigueur de ma main que j’ai agi,
        et par ma sagesse, car j’ai l’intelligence.
        \\J’ai déplacé les frontières des peuples,
        j’ai pillé leurs réserves ;
        fort entre les forts, j’ai détrôné des puissants\\.
        ${}^{14}J’ai mis la main sur les richesses des peuples,
        comme sur un nid.
        \\Comme on ramasse des œufs abandonnés,
        j’ai ramassé toute la terre,
        \\et il n’y a pas eu un battement d’aile,
        pas un bec ouvert,
        pas un cri. »
        ${}^{15}Mais le ciseau se glorifie-t-il
        aux dépens de celui qui s’en sert pour tailler ?
        \\La scie va-t-elle s’enfler d’orgueil
        aux dépens de celui qui la tient ?
        \\Comme si le bâton faisait mouvoir la main\\qui le brandit,
        comme si c’était le bois qui brandissait l’homme\\ !
        ${}^{16}C’est pourquoi le Seigneur Dieu de l’univers
        fera dépérir les soldats bien nourris du roi d’Assour,
        \\et au lieu de sa gloire s’allumera un brasier,
        le brasier d’un incendie.
${}^{17}La Lumière d’Israël deviendra un incendie
        et son Dieu Saint, une flamme
        \\qui brûle et dévore épines et ronces
        en un seul jour !
${}^{18}Les forêts et les vergers, gloire du roi,
        il les détruira de fond en comble ;
        \\ce sera comme un moribond qui s’éteint.
${}^{19}Le reste des arbres de sa forêt, on pourra les compter ;
        un enfant en écrirait le nombre.
${}^{20}En ces jours-là, le reste d’Israël
        et les rescapés de la maison de Jacob
        cesseront de s’appuyer sur celui qui les frappait :
        \\ils s’appuieront vraiment sur le Seigneur,
        le Saint d’Israël.
${}^{21}Un reste reviendra :
        le reste de Jacob,
        vers le Dieu-Fort.
${}^{22}Israël, même si ton peuple est comme le sable de la mer,
        seul un reste en reviendra.
        \\La destruction est décidée :
        la justice déferle.
${}^{23}Oui, c’est une destruction décidée,
        que le Seigneur, Dieu de l’univers,
        accomplit dans tout le pays.
${}^{24}C’est pourquoi ainsi parle le Seigneur,
        Dieu de l’univers :
        \\« Ô mon peuple, qui habites Sion,
        n’aie pas peur d’Assour :
        \\il te frappe de son gourdin,
        il abat sur toi son bâton
        comme l’a fait l’Égypte.
${}^{25}Mais encore un peu, très peu de temps,
        et mon indignation contre toi prendra fin,
        ma colère tournera à leur perte. »
${}^{26}Contre eux, le Seigneur de l’univers va brandir un fouet,
        comme il frappa Madiane au Rocher d’Oreb ;
        \\il lèvera son bâton sur la mer,
        comme il l’a fait en Égypte.
${}^{27}Il arrivera, en ce jour-là,
        que le fardeau sera retiré de tes épaules,
        et de ta nuque, le joug.
        \\Le dévastateur monte de la région de Rimmone.
${}^{28}Il vient contre Ayath, il passe à Migrone ;
        à Mikmas, il installe son matériel.
${}^{29}Ils franchissent le défilé :
        « À Guéba, notre étape ! »
        \\Rama tremble,
        Guibéa de Saül prend la fuite.
${}^{30}« Hurle, Bath-Gallim !
        Écoute, Laïsha !
        Réponds, Anatoth !
${}^{31}Madména s’enfuit ;
        ils se terrent, les habitants de Guébim.
${}^{32}Aujourd’hui même, à Nob, l’ennemi prend position ;
        il tend le poing
        vers la montagne de la fille de Sion,
        vers la colline de Jérusalem.
${}^{33}Voici que le Seigneur, Dieu de l’univers,
        d’un coup terrible abat les branches ;
        \\les arbres les plus grands sont coupés,
        et les plus fiers, jetés à terre.
${}^{34}Les taillis de la forêt s’abattent sous la hache,
        et le Liban dans sa splendeur s’effondre.
      
         
      \bchapter{}
        ${}^{1}Un rameau sortira de la souche de Jessé, père de David\\,
        un rejeton jaillira de ses racines.
        ${}^{2}Sur lui reposera l’esprit du Seigneur :
        esprit de sagesse et de discernement,
        \\esprit de conseil et de force,
        esprit de connaissance et de crainte du Seigneur
        ${}^{3}– qui lui inspirera la crainte du Seigneur\\.
        \\Il ne jugera pas sur l’apparence ;
        il ne se prononcera pas sur des rumeurs.
        ${}^{4}Il jugera les petits avec justice ;
        avec droiture, il se prononcera
        en faveur des humbles du pays.
        \\Du bâton de sa parole, il frappera le pays ;
        du souffle de ses lèvres, il fera mourir le méchant.
        ${}^{5}La justice est la ceinture de ses hanches ;
        la fidélité est la ceinture de ses reins.
        
           
         
        ${}^{6}Le loup habitera avec l’agneau,
        le léopard se couchera près du chevreau,
        \\le veau et le lionceau seront nourris ensemble,
        un petit garçon les conduira.
        ${}^{7}La vache et l’ourse auront même pâture\\,
        leurs petits auront même gîte.
        \\Le lion, comme le bœuf, mangera du fourrage\\.
        ${}^{8}Le nourrisson s’amusera sur le nid du cobra ;
        sur le trou de la vipère, l’enfant étendra la main.
        ${}^{9}Il n’y aura plus de mal ni de corruption
        sur toute ma montagne sainte ;
        \\car la connaissance du Seigneur remplira le pays
        comme les eaux recouvrent le fond de la mer.
        
           
        ${}^{10}Ce jour-là, la racine de Jessé, père de David\\,
        sera dressée comme un étendard pour les peuples,
        \\les nations la chercheront,
        et la gloire sera sa demeure.
${}^{11}Ce jour-là, une fois encore,
        le Seigneur étendra la main
        \\pour reprendre le reste de son peuple,
        ce reste qui reviendra d’Assour et d’Égypte,
        de Patros, d’Éthiopie et d’Élam,
        de Shinéar, de Hamath et des îles de la mer.
${}^{12}Il lèvera un étendard pour les nations ;
        il rassemblera les exilés d’Israël ;
        \\il réunira les dispersés de Juda
        des quatre coins de la terre.
${}^{13}Alors la jalousie d’Éphraïm cessera,
        et les adversaires de Juda seront retranchés.
        \\Éphraïm ne jalousera plus Juda,
        et Juda ne sera plus l’adversaire d’Éphraïm.
${}^{14}Ils fonceront sur le flanc des Philistins à l’Occident ;
        ensemble ils pilleront les fils de l’Orient.
        \\Ils mettront la main sur Édom et Moab,
        et les fils d’Ammone leur obéiront.
${}^{15}Le Seigneur asséchera la lagune de l’Égypte,
        il lèvera la main contre l’Euphrate,
        dans l’ardeur de son souffle ;
        \\il le divisera en sept ruisseaux
        où l’on marchera en sandales.
${}^{16}Il y aura une route pour le reste de son peuple,
        ce reste qui reviendra d’Assour,
        \\comme il y eut une route pour Israël,
        le jour où il monta du pays d’Égypte.
      <p class="cantique" id="bib_ct-at_19"><span class="cantique_label">Cantique AT 19</span> = <span class="cantique_ref"><a class="unitex_link" href="#bib_is_12_1">Is 12, 1b-6</a></span>
      
         
      \bchapter{}
        ${}^{1}Ce jour-là, tu diras :
        
           
       
        Seigneur, je te rends grâce :
        \\ta colère pesait sur moi,
        \\mais tu reviens de ta fureur
        et tu me consoles.
         
        ${}^{2}Voici le Dieu qui me sauve :
        \\j’ai confiance, je n’ai plus de crainte.
        \\Ma force et mon chant\\, c’est le Seigneur ;
        \\il est pour moi le salut.
         
        ${}^{3}Exultant de joie,
        vous puiserez les eaux
        \\aux sources du salut.
         
        ${}^{4}Ce jour-là, vous direz :
        « Rendez grâce au Seigneur,
        \\proclamez son nom,
        annoncez parmi les peuples ses hauts faits ! »
         
        \\Redites-le\\ : « Sublime est son nom ! »
        ${}^{5}Jouez pour le Seigneur,
        \\il montre sa magnificence,
        et toute la terre le sait.
         
        ${}^{6}Jubilez, criez de joie,
        habitants de Sion\\,
        \\car il est grand au milieu de toi,
        le Saint d’Israël !
      <h2 class="intertitle" id="d85e242134">3. Sur les nations (13 – 23)</h2>
      
         
      \bchapter{}
       
${}^{1}Proclamation sur Babylone – ce qu’Isaïe, fils d’Amots, a vu.
         
${}^{2}Sur un mont dénudé,
        dressez un étendard.
        \\Élevez la voix, faites signe de la main,
        qu’ils entrent par les portes des nobles ;
${}^{3}moi, je commande à qui m’est consacré ;
        j’ai convoqué les guerriers de ma colère,
        les passionnés de mon honneur.
${}^{4}Voix qui gronde sur les montagnes,
        comme d’un peuple immense ;
        \\voix et vacarme de royaumes,
        de nations rassemblées :
        \\le Seigneur de l’univers
        inspecte les troupes de combat.
${}^{5}D’une terre lointaine, des extrémités du ciel,
        ils viennent,
        \\le Seigneur et les instruments de son indignation,
        pour ravager toute la terre.
         
${}^{6}Hurlez ! Car le jour du Seigneur est proche :
        il vient, envoyé par le Puissant,
        comme une dévastation.
${}^{7}C’est pourquoi toute main défaille,
        le cœur manque à tout mortel.
${}^{8}Ils sont épouvantés,
        spasmes et souffrances les saisissent ;
        \\ils se tordent de douleur
        comme la femme qui accouche ;
        \\ils se regardent l’un l’autre avec stupeur,
        le visage en feu.
${}^{9}Voici venir, implacable, le jour du Seigneur,
        la fureur et l’ardente colère,
        \\pour faire de la terre un lieu désolé,
        pour en supprimer les pécheurs.
${}^{10}Les étoiles du ciel et ses constellations
        ne brilleront plus de leur lumière ;
        \\le soleil, dès son lever, s’obscurcira
        et la lune ne donnera plus sa clarté.
${}^{11}Je châtierai le monde pour sa méchanceté,
        et les impies pour leur faute.
        \\Je mettrai fin à l’orgueil des insolents
        et rabattrai l’arrogance des tyrans.
${}^{12}Je rendrai les mortels plus rares que l’or fin,
        les humains, plus rares que l’or d’Ophir.
${}^{13}C’est pourquoi j’ébranlerai les cieux,
        et la terre tremblera sur ses bases,
        \\dans la fureur du Seigneur de l’univers,
        le jour de son ardente colère.
${}^{14}Alors, comme une gazelle pourchassée,
        comme un troupeau que nul ne rassemble,
        \\chacun se tournera vers son peuple,
        chacun fuira vers son propre pays.
${}^{15}Tous ceux qu’on trouvera seront transpercés,
        tous ceux qu’on prendra tomberont par l’épée.
${}^{16}Leurs petits enfants seront écrasés sous leurs yeux,
        leurs maisons, pillées, et leurs femmes, violées.
         
${}^{17}Voici que j’excite contre eux les Mèdes,
        qui n’estiment pas l’argent et n’aiment pas l’or.
${}^{18}Mais leurs arcs abattent les jeunes gens ;
        ils n’ont pas de compassion pour les nouveau-nés ;
        aucun regard de pitié pour les enfants.
${}^{19}Alors Babylone, la perle des royaumes,
        l’orgueilleuse parure des Chaldéens,
        \\subira le bouleversement infligé par Dieu
        à Sodome et à Gomorrhe.
${}^{20}Elle sera inhabitée à jamais ;
        nul, d’âge en âge, n’y demeurera ;
        \\l’Arabe n’y dressera plus sa tente,
        les bergers n’y feront plus reposer leurs troupeaux.
${}^{21}Mais les chats sauvages y auront leur gîte ;
        les hiboux rempliront les maisons ;
        \\les autruches auront là leur demeure
        et les boucs y danseront.
${}^{22}Les hyènes hurleront dans ses châteaux,
        et les chacals dans ses palais d’agrément.
        \\Son temps est proche
        et ses jours ne tarderont pas.
      
         
      \bchapter{}
${}^{1}Oui, le Seigneur aura de la tendresse pour Jacob,
        il choisira encore Israël,
        il installera les siens sur leur terre
        \\où des immigrés se joindront à eux
        et s’uniront à la maison de Jacob.
${}^{2}Des peuples étrangers viendront les prendre
        pour les ramener chez eux.
        \\Sur la terre du Seigneur,
        la maison d’Israël s’attribuera ces peuples
        comme serviteurs et comme servantes.
        \\Ils tiendront captifs ceux qui les avaient capturés ;
        ils domineront ceux qui les avaient opprimés.
        
           
${}^{3}Le jour où le Seigneur t’aura fait reposer,
        après tant de peines et de tourments,
        après le dur esclavage qui fut le tien,
${}^{4}tu entonneras cette chanson contre le roi de Babylone,
        tu diras :
         
        \\« Comment ! Il est fini, l’oppresseur !
        Elle est finie, la dictature !
${}^{5}Le Seigneur a brisé le bâton des impies,
        le sceptre des tyrans
${}^{6}qui frappait les peuples avec fureur,
        les frappait sans relâche,
        \\qui dominait les nations avec colère
        et les persécutait sans retenue.
${}^{7}Toute la terre repose, tranquille.
        On éclate en cris de joie !
${}^{8}Même les cyprès et les cèdres du Liban
        se réjouissent à tes dépens :
        \\“Depuis que tu es tombé,
        plus personne ne monte pour nous abattre.”
         
${}^{9}Le tréfonds des enfers s’agite pour toi
        à l’annonce de ta venue.
        \\Pour toi, il réveille les ombres,
        tous les grands de la terre ;
        \\il fait lever de leur trône
        tous les rois des nations.
${}^{10}Tous, ils prennent la parole et te disent :
        “Toi aussi, comme nous, te voilà sans force,
        devenu pareil à nous.”
${}^{11}Elle est jetée aux enfers, ta majesté,
        avec la musique de tes harpes.
        \\Tu as pour couche la vermine,
        et des vers pour te couvrir.
         
${}^{12}Comment ! Tu es tombé du ciel,
        astre brillant, fils de l’aurore !
        \\Tu es renversé à terre,
        toi qui faisais ployer les nations,
${}^{13}toi qui te disais :
        “J’escaladerai les cieux ;
        \\plus haut que les étoiles de Dieu
        j’élèverai mon trône ;
        \\j’irai siéger à la montagne de l’assemblée des dieux
        au plus haut du mont Safone,
${}^{14}j’escaladerai les hauteurs des nuages,
        je serai semblable au Très-Haut !”
${}^{15}Mais te voilà jeté aux enfers,
        au plus profond de l’abîme.
${}^{16}Ceux qui te voient te dévisagent,
        ils s’interrogent sur toi :
        \\“Est-ce bien l’homme qui faisait trembler la terre,
        qui ébranlait les royaumes,
${}^{17}changeait le monde en désert,
        et rasait les villes
        sans renvoyer le prisonnier dans sa maison ?”
         
${}^{18}Tous les rois des nations
        reposent avec honneur, tous sans exception,
        chacun dans sa demeure.
${}^{19} Mais toi, tu es jeté dehors,
        loin de ton sépulcre,
        comme un rejeton réprouvé ;
        \\tu es couvert par les victimes de l’épée
        qui descendent sur les pierres de la fosse ;
        tu es comme un cadavre qu’on piétine.
${}^{20}Tu ne les rejoindras pas dans la tombe,
        car tu as ruiné ton pays,
        assassiné ton peuple.
        \\Plus jamais on n’évoquera
        l’engeance des méchants :
${}^{21}préparez pour les fils un lieu d’exécution
        à cause de la faute de leurs pères,
        \\de peur qu’ils ne se relèvent pour conquérir la terre
        et couvrir de villes la face du monde. »
         
${}^{22}Je me lèverai contre eux,
        – oracle du Seigneur de l’univers – ;
        \\de Babylone, je retrancherai
        le nom et le reste, lignée et postérité,
        – oracle du Seigneur.
${}^{23}J’en ferai le domaine du hérisson,
        un marécage ;
        \\je la balaierai, je la détruirai
        – oracle du Seigneur de l’univers.
${}^{24}Le Seigneur de l’univers en a fait le serment :
        \\« Ce que j’ai prévu arrivera,
        ce que j’ai décidé s’accomplira :
${}^{25}Sur ma terre, je briserai Assour,
        je le piétinerai sur mes montagnes ;
        \\son joug sera écarté de la nuque des miens,
        et le fardeau, écarté de leurs épaules. »
         
${}^{26}C’est la décision
        prise contre toute la terre ;
        \\c’est la main
        étendue contre toutes les nations.
${}^{27}Le Seigneur de l’univers l’a décidé.
        Qui donc l’arrêterait ?
        \\Et sa main étendue,
        qui la détournerait ?
${}^{28}L’année de la mort du roi Acaz,
        il y eut cette proclamation :
${}^{29}Ne te réjouis pas, Philistie tout entière,
        si le bâton qui te frappait est brisé.
        \\De la racine du serpent va sortir une vipère,
        et son fruit sera un dragon ailé.
${}^{30}Les premiers-nés des faibles auront un pâturage,
        les malheureux reposeront en confiance,
        \\mais ta race, je la ferai mourir de faim,
        et ce qui restera de toi, je le tuerai.
         
${}^{31}Hurle, Porte de la ville ! Citadelle, pousse des cris !
        Tremble, Philistie tout entière !
        \\Du Nord arrive une fumée :
        ce sont des troupes sans déserteur !
${}^{32}Que répondre aux envoyés de cette nation ?
        Ceci : Le Seigneur a fondé Sion ;
        en elle se réfugient les pauvres de son peuple.
      
         
      \bchapter{}
${}^{1}Proclamation sur Moab.
        
           
         
        \\Depuis la nuit où Ar fut dévastée,
        Moab est réduite au silence.
        \\Depuis la nuit où Qir fut dévastée,
        Moab est réduite au silence.
${}^{2}On monte au sanctuaire, à Dibone,
        aux lieux sacrés pour y pleurer.
        \\Sur Nébo et sur Madaba,
        Moab hurle de douleur.
        \\Ils ont tous la tête rasée,
        ils ont tous la barbe coupée.
${}^{3}Dans les rues,
        ils se revêtent de toile à sac ;
        \\sur les toits et sur les places,
        tout le monde hurle et fond en larmes.
${}^{4}Heshbone et Élalé poussent des cris,
        leurs voix s’entendent jusqu’à Yahaç.
        \\C’est pourquoi les guerriers de Moab poussent des clameurs,
        l’âme de Moab frémit.
${}^{5}Mon cœur crie sur Moab ;
        ses fugitifs sont déjà à Soar,
        vers Églath-Shelishiya.
        \\Sur la montée de Louhith,
        on monte en pleurant ;
        \\sur le chemin de Horonaïm,
        ce sont des cris déchirants.
${}^{6}Les eaux de Nimrim sont un endroit désolé :
        l’herbe a séché,
        \\la végétation a disparu,
        plus aucune verdure.
${}^{7}Voilà pourquoi ses réserves et ses provisions
        sont transportées au-delà du torrent des Saules.
${}^{8}Car les cris font le tour
        du territoire de Moab :
        \\hurlements, jusqu’à Églaïm,
        hurlements, jusqu’à Beér-Élim.
${}^{9}Les eaux de Dimone sont rouges de sang,
        car j’imposerai plus encore à Dimone :
        \\le lion, pour les rescapés de Moab,
        pour ceux qui resteront dans la campagne.
        
           
       
      
         
      \bchapter{}
${}^{1}Envoyez au maître du pays un agneau,
        depuis La Roche au désert,
        \\vers la montagne de la fille de Sion.
${}^{2}Des oiseaux qui s’enfuient,
        une nichée dispersée,
        \\telles seront les filles de Moab,
        aux gués de l’Arnon.
${}^{3}Moab dit à Juda :
        \\« Fais des plans ! Prends une décision !
        \\En plein midi, fais-nous une ombre comme la nuit,
        cache les expulsés, ne trahis pas les fugitifs !
${}^{4}Que les expulsés de Moab trouvent chez toi un asile,
        sois un abri pour eux face au dévastateur.
        \\Quand l’oppression aura disparu,
        quand la dévastation aura pris fin,
        \\quand sera parti du pays celui qui le foulait,
${}^{5}un trône s’établira sur la fidélité ;
        \\et, pour la maison de David,
        siégera sur ce trône avec loyauté
        \\le juge qui cherche le droit
        et fait prompte justice. »
        
           
         
${}^{6}Nous avons appris l’orgueil de Moab,
        son immense orgueil,
        \\son orgueil arrogant, démesuré,
        ses bavardages qui ne sont rien.
${}^{7}C’est pourquoi Moab hurle sur Moab,
        il n’est que hurlement.
        \\Pour les gâteaux de raisin de Qir-Harèseth,
        vous gémissez, tout abattus.
${}^{8}Car ils dépérissent, les vignobles de Heshbone,
        la vigne de Sibma
        \\dont les maîtres des nations
        ont saccagé les grappes rouges ;
        \\ses sarments allaient jusqu’à Yazèr,
        ils se perdaient dans le désert ;
        \\ses rejetons s’étendaient
        au-delà de la mer.
${}^{9}C’est pourquoi avec Yazèr je pleure
        la vigne de Sibma ;
        \\je t’arrose de mes larmes, Heshbone,
        et toi, Élalé,
        \\car sur tes récoltes et tes moissons,
        une clameur s’est abattue.
${}^{10}Joie et liesse
        disparaissent des vergers ;
        \\dans les vignes,
        plus de chants de jubilation ;
        \\au pressoir le fouleur ne foule plus le raisin :
        j’ai fait cesser les clameurs.
${}^{11}C’est pourquoi, comme une cithare,
        mes entrailles gémissent sur Moab,
        et mon cœur, sur Qir-Hérès.
${}^{12}On verra Moab s’épuiser sur le lieu sacré,
        entrer au sanctuaire pour prier,
        et ne rien obtenir.
        
           
         
${}^{13}Telle est la parole que le Seigneur prononça sur Moab, autrefois.
${}^{14}Et maintenant le Seigneur déclare :
        \\D’ici trois ans, jour pour jour,
        la gloire de Moab ne comptera plus ;
        \\malgré toute sa multitude,
        il en restera très peu, un reste insignifiant.
        
           
      
         
      \bchapter{}
${}^{1}Proclamation sur Damas.
        
           
         
        \\Voici Damas rayée d’entre les villes,
        pour devenir un champ de ruines.
${}^{2}Abandonnées, les villes d’Aroër
        seront livrées aux troupeaux :
        \\ils s’y coucheront, et personne qui les dérange.
${}^{3}Éphraïm sera privé de forteresse
        et Damas, de royauté.
        \\Il en sera du reste d’Aram
        comme il en est de la gloire d’Israël
        – oracle du Seigneur de l’univers.
${}^{4}Ce jour-là, la gloire de Jacob faiblira,
        sa chair s’amaigrira.
${}^{5}Ce sera comme à la moisson,
        quand le blé est ramassé,
        \\que les épis sont recueillis à brassée
        et rassemblés au val des Rephaïm ;
${}^{6}il n’y restera presque rien à glaner,
        comme à la cueillette des olives :
        deux ou trois olives à la cime des plus hautes branches,
        quatre ou cinq sur les meilleurs rameaux
        – oracle du Seigneur, le Dieu d’Israël.
        
           
${}^{7}Ce jour-là, l’homme regardera vers Celui qui l’a fait
        et portera les yeux vers le Saint d’Israël.
${}^{8}Il ne regardera plus vers les autels faits de ses mains ;
        il ne verra plus ce que ses doigts avaient fait,
        ni les poteaux sacrés, ni les colonnes à encens.
         
${}^{9}Ce jour-là, les villes de refuge seront abandonnées
        comme furent abandonnés les bois et les cimes
        devant les fils d’Israël :
        \\ce sera une désolation.
${}^{10}Tu as oublié ton Dieu, ton sauveur,
        tu ne t’es pas souvenu de ton Rocher, ton refuge.
        \\C’est pourquoi tu cultives des plantes de délices,
        tu sèmes des espèces étrangères.
${}^{11}Le jour même où tu les plantes, tu les vois pousser ;
        dès le matin, tu vois fleurir ce que tu as semé,
        \\mais la récolte t’échappe au jour de la maladie,
        du mal incurable.
${}^{12}Malheur ! C’est le grondement de peuples innombrables
        qui grondent comme gronde la mer,
        \\le rugissement de nations
        qui rugissent comme rugissent les eaux puissantes !
${}^{13}Ces nations rugissent
        comme rugissent les grandes eaux !
        \\Mais le Seigneur les menace, elles fuient au loin,
        chassées comme la paille par le vent des montagnes,
        comme le tourbillon, par l’ouragan.
${}^{14}Le soir venu, c’est l’épouvante ;
        avant le matin, elles ne sont plus.
        \\Telle est la part de ceux qui nous dépouillent,
        le sort de ceux qui nous mettent au pillage.
      
         
      \bchapter{}
${}^{1}Malheureux, le pays des bruissements d’ailes,
        au-delà des fleuves d’Éthiopie :
${}^{2}il envoie par mer des ambassades
        dans des barques de papyrus à la surface des eaux.
        \\Partez, rapides messagers,
        \\vers une race élancée, à la peau luisante,
        peuple redouté ici et là-bas,
        \\nation barbare et tyrannique,
        où des cours d’eau sillonnent la terre.
${}^{3}Vous tous, habitants du monde,
        vous qui peuplez la terre,
        \\quand l’étendard sera levé sur les montagnes,
        regardez !
        \\quand sonnera le cor,
        écoutez !
${}^{4}Car le Seigneur m’a déclaré :
        \\Je demeure tranquille ;
        là où je me tiens, je regarde
        \\dans la chaleur éblouissante du plein midi,
        comme un nuage de rosée dans la chaleur de la récolte.
${}^{5}Avant la récolte, quand la floraison s’achève
        et que la fleur se change en fruits mûrissants,
        \\on coupe les branches à la serpe,
        on enlève et jette les rameaux.
${}^{6}Tout sera abandonné aux rapaces des montagnes
        et aux bêtes du pays :
        \\les rapaces y passeront l’été,
        et toutes les bêtes du pays, l’hiver.
${}^{7}C’est alors qu’un présent sera porté
        au Seigneur de l’univers
        \\par le peuple élancé, à la peau luisante,
        peuple redouté ici et là-bas,
        \\nation barbare et tyrannique,
        où des cours d’eau sillonnent la terre ;
        \\ce présent sera porté
        au lieu où réside le nom du Seigneur, Dieu de l’univers :
        la montagne de Sion.
        
           
      
         
      \bchapter{}
         
${}^{1}Proclamation sur l’Égypte.
        
           
         
        \\Voici le Seigneur :
        il chevauche une nuée légère,
        il entre en Égypte ;
        \\les idoles d’Égypte vacillent devant lui,
        l’Égypte voit fondre son courage.
${}^{2}J’exciterai l’Égypte contre l’Égypte,
        on se battra frère contre frère,
        \\ami contre ami, ville contre ville,
        royaume contre royaume.
${}^{3}L’Égypte en perdra l’esprit,
        je brouillerai son projet ;
        \\ils consulteront idoles et sorciers,
        nécromanciens et devins.
${}^{4}Je livrerai l’Égypte aux mains d’un maître implacable,
        un roi tout-puissant régnera sur eux,
        – oracle du Maître et Seigneur de l’univers.
        
           
         
${}^{5}Les eaux s’épuiseront avant d’atteindre la mer,
        le fleuve tarira, s’asséchera,
${}^{6}ses canaux empesteront,
        le delta du Nil baissera, s’asséchera,
        cannes et roseaux flétriront.
${}^{7}Les joncs des bords du Nil jusqu’à son embouchure,
        les plantations près du Nil,
        \\tout sera desséché, emporté :
        il n’en restera rien.
${}^{8}Les pêcheurs se lamenteront :
        \\ils seront en deuil,
        tous ceux qui jettent l’hameçon dans le Nil ;
        \\ils dépériront,
        ceux qui tendent le filet sur les eaux.
${}^{9}Ils seront déçus, ceux qui travaillent le lin :
        cardeuses et tisserands deviendront livides.
${}^{10}Les artisans seront accablés,
        les salariés, désespérés.
        
           
         
${}^{11}Qu’ils sont stupides, les princes de Tanis !
        Les plus sages conseillers de Pharaon
        forment un conseil d’incapables.
        \\Comment pouvez-vous dire à Pharaon :
        « Je suis un fils de sage,
        un descendant des rois de jadis » ?
${}^{12}Où sont-ils, tes sages, où sont-ils ?
        Qu’ils t’instruisent donc,
        \\et l’on saura ce que le Seigneur de l’univers
        a décidé contre l’Égypte.
${}^{13}Ils déraisonnent, les princes de Tanis,
        ils délirent, les princes de Memphis ;
        \\ils font vaciller l’Égypte,
        eux qui sont la pierre d’angle de ses tribus.
${}^{14}Le Seigneur a insufflé au milieu d’elle un esprit de vertige :
        ils font vaciller l’Égypte en tout ce qu’elle fait,
        comme vacille un ivrogne en vomissant.
${}^{15}Plus rien ne se fera en Égypte,
        que ce soit par le grand ou par le petit,
        le palmier ou le roseau.
        
           
${}^{16}Ce jour-là,
        \\l’Égypte, comme les femmes,
        sera tremblante et terrifiée
        \\quand le Seigneur de l’univers lui-même
        élèvera la main contre elle.
${}^{17}Le pays de Juda sera objet d’effroi pour l’Égypte :
        chaque fois qu’on l’évoquera, elle sera terrifiée
        \\à cause du projet que le Seigneur de l’univers
        a lui-même formé contre elle.
${}^{18}Ce jour-là, il y aura au pays d’Égypte
        cinq villes pour parler la langue de Canaan
        \\et prêter serment au Seigneur de l’univers ;
        l’une d’elles se nomme « Ville-du-Soleil ».
${}^{19}Ce jour-là, il y aura un autel pour le Seigneur
        au centre du pays d’Égypte,
        et près de sa frontière une stèle pour le Seigneur.
${}^{20}Ce sera un signe, un témoin, pour le Seigneur de l’univers
        dans le pays d’Égypte :
        \\quand ils crieront vers le Seigneur
        devant ceux qui les oppriment,
        \\il leur enverra un sauveur, un défenseur
        qui les délivrera.
${}^{21}Le Seigneur se fera connaître de l’Égypte
        et l’Égypte connaîtra le Seigneur, ce jour-là ;
        \\elle le servira par des sacrifices et des offrandes,
        elle fera des vœux au Seigneur et les accomplira.
${}^{22}Le Seigneur frappera l’Égypte,
        il frappera et guérira.
        \\Elle reviendra au Seigneur
        qui l’écoutera et la guérira.
${}^{23}Ce jour-là, il y aura une route
        pour relier l’Égypte et Assour.
        \\Assour viendra en Égypte,
        et l’Égypte en Assour ;
        et l’Égypte avec Assour servira le Seigneur.
${}^{24}Ce jour-là, entre l’Égypte et Assour,
        Israël viendra en troisième,
        \\bénédiction au milieu de la terre,
${}^{25}que bénira le Seigneur Dieu de l’univers en disant :
        « Bénis soient l’Égypte, mon peuple,
        Assour, l’ouvrage de mes mains,
        et Israël, mon héritage. »
      
         
      \bchapter{}
      \begin{verse}
${}^{1}C’était l’année où le général en chef envoyé par Sargon, roi d’Assour, vint assiéger Ashdod et s’en empara. 
${}^{2}En ce temps-là, par Isaïe, fils d’Amots, le Seigneur avait parlé et dit : « Va, détache le pagne de tes reins, enlève les sandales de tes pieds. » Il avait fait ainsi, marchant dévêtu, les pieds nus. 
${}^{3}Et le Seigneur dit : « De même que mon serviteur Isaïe est allé dévêtu, les pieds nus, pendant trois ans, signe et présage pour l’Égypte et l’Éthiopie, 
${}^{4}de même, le roi d’Assour emmènera les prisonniers d’Égypte et les déportés d’Éthiopie, les jeunes et les vieux, dévêtus, les pieds nus, les fesses découvertes – telle sera la nudité de l’Égypte. 
${}^{5}Ils seront effondrés et honteux, ceux qui mettaient leur espoir dans l’Éthiopie et leur fierté dans l’Égypte. 
${}^{6}Les habitants de la côte diront ce jour-là : “Voilà ce que devient notre espoir, quand nous comptions sur cet appui pour échapper au roi d’Assour. Et nous, comment serons-nous sauvés ?” »
      
         
      
         
      \bchapter{}
         
${}^{1}Proclamation sur le Désert de la Mer.
        
           
         
        \\Comme l’ouragan qui traverse le Néguev,
        quelqu’un vient du désert, d’un pays terrible.
${}^{2}J’ai reçu une sinistre vision :
        Un ravageur qui ravage ! Un dévastateur qui dévaste !
        Montez, Élamites ! Mèdes, assiégez !
        \\Je supprime toute plainte !
        
           
         
${}^{3}Voilà pourquoi mes reins se tordent de souffrance,
        les douleurs me saisissent
        comme celles d’une femme qui accouche ;
        \\ce que j’entends me bouleverse,
        ce que je vois me terrifie.
${}^{4}Mon courage flanche,
        je tremble de peur :
        \\le crépuscule auquel j’aspire,
        Dieu le change en effroi.
${}^{5}On dressait la table, on déroulait les tapis,
        on mangeait, on buvait.
        \\Soudain : Debout, les princes ;
        préparez vos boucliers !
${}^{6}Car ainsi m’a parlé le Seigneur :
        \\« Va, place un guetteur :
        ce qu’il voit, qu’il l’annonce !
${}^{7}S’il voit un char attelé de deux chevaux
        un attelage d’âne ou de chameau,
        \\qu’il fasse attention,
        qu’il redouble d’attention ! »
${}^{8}Et le veilleur a crié :
        \\« Au poste de guet, Seigneur,
        je me tiens tout le jour.
        \\À mon poste de garde,
        je reste debout toute la nuit.
${}^{9}Voici ce qui vient :
        sur un char attelé de deux chevaux
        \\un homme qui parle et dit :
        “Elle est tombée, Babylone, elle est tombée,
        et toutes les statues de ses dieux
        gisent par terre, brisées.” »
${}^{10}À vous, fils de mon peuple, qui êtes battus
        comme les grains de mon aire,
        \\ce que j’ai entendu
        de la part du Seigneur de l’univers, Dieu d’Israël,
        je vous l’annonce.
        
           
${}^{11}Proclamation sur Douma.
         
        \\Une voix me crie de Séïr :
        « Veilleur, où en est la nuit ?
        Veilleur, où donc en est la nuit ? »
${}^{12}Le veilleur répond :
        « Le matin vient, et puis encore la nuit…
        Si vous voulez des nouvelles, interrogez,
        revenez. »
${}^{13}Proclamation « en Arabie ».
         
        \\Dans les broussailles, en Arabie,
        vous passerez la nuit,
        caravaniers de Dedane.
${}^{14}Habitants du pays de Téma,
        allez à la rencontre de l’assoiffé,
        \\portez-lui de l’eau,
        accueillez le fugitif avec du pain.
${}^{15}Car ils ont fui devant les épées,
        devant l’épée mise à nu,
        \\devant l’arc tendu,
        devant l’âpreté du combat.
         
${}^{16}Ainsi m’a parlé le Seigneur :
        \\Encore une année, jour pour jour,
        et toute la gloire de Qédar sera détruite.
${}^{17}Des archers de Qédar, nombreux et vaillants,
        il ne restera presque rien.
        Le Seigneur, le Dieu d’Israël, a parlé.
      
         
      \bchapter{}
         
${}^{1}Proclamation sur le Val de la Vision.
        
           
         
        \\Ville de Jérusalem, qu’as-tu donc
        à monter tout entière sur les terrasses,
${}^{2}ville en fête, pleine de cris,
        cité joyeuse ?
        \\Tes morts ne sont pas morts par l’épée,
        ils n’ont pas été tués dans un combat.
${}^{3}Tes chefs ont pris la fuite comme un seul homme ;
        sans même avoir tiré de l’arc, ils sont faits prisonniers.
        \\Si loin qu’ils aient fui, on les a tous retrouvés :
        ils sont prisonniers ensemble.
${}^{4}C’est pourquoi je dis :
        \\« Détournez de moi vos regards,
        je pleure amèrement !
        \\Ne cherchez pas à me consoler
        du désastre subi par la fille de mon peuple. »
        
           
         
${}^{5}Oui, c’est un jour d’affolement,
        d’effarement, d’effondrement,
        envoyé par le Seigneur, Dieu de l’univers.
        \\Dans le Val de la Vision,
        on sape la muraille,
        des cris s’élèvent vers la montagne.
${}^{6}Élam lève le carquois,
        il amène des chars et des cavaliers ;
        Qir brandit son bouclier.
${}^{7}Les plus belles de tes vallées
        sont remplies de chars,
        \\les cavaliers sont rangés devant tes portes.
${}^{8}Et Juda, lui, est privé de ses défenses !
        
           
         
        \\En regardant, ce jour-là,
        vers l’arsenal du palais royal,
${}^{9}vous avez vu comme elles sont nombreuses,
        les brèches de la cité de David.
        \\Vous avez recueilli les eaux
        dans le réservoir inférieur.
${}^{10}Vous avez recensé les maisons de Jérusalem,
        démoli des maisons pour renforcer le rempart.
${}^{11}Vous avez creusé un bassin entre les deux remparts
        pour les eaux de l’ancien réservoir.
        \\Mais vous n’avez pas regardé vers Celui qui est à l’œuvre ;
        Celui qui façonne tout depuis longtemps, vous ne l’avez pas vu.
        
           
         
${}^{12}Ce jour-là, le Seigneur, Dieu de l’univers,
        appelait à pleurer, à se lamenter,
        à se raser la tête, à se vêtir de toile à sac.
${}^{13}Et voilà qu’on se réjouit, on fait la fête ;
        on tue le bœuf, on égorge le mouton ;
        \\on mange de la viande, on boit du vin :
        « Mangeons et buvons, car demain nous mourrons ! »
${}^{14}Mais le Seigneur de l’univers
        m’a fait entendre cette révélation :
        \\« J’en fais serment : cette faute ne vous sera jamais remise,
        jusque dans la mort ».
        \\Il l’a dit, le Seigneur, Dieu de l’univers.
        
           
${}^{15}Ainsi parle le Seigneur, Dieu de l’univers :
        \\« Va trouver ce ministre,
        Shebna, le maître du palais, et dis-lui :
${}^{16}Ici, quel est ton bien ? Qui sont les tiens, ici,
        pour t’y faire creuser un tombeau,
        \\toi qui te creuses un tombeau sur une hauteur,
        et te fais tailler une demeure dans le roc ?
${}^{17}Voici que le Seigneur va te rejeter,
        il va te rejeter, grand homme,
        \\t’empaqueter comme un paquet,
${}^{18}t’enrouler, t’envoyer rouler comme une boule
        vers un pays aux vastes étendues.
        \\C’est là-bas que tu vas mourir,
        là-bas, dans tes chars prestigieux,
        toi, le déshonneur de la maison de ton maître.
        ${}^{19}Je vais te chasser de ton poste,
        t’expulser de ta place.
        ${}^{20}Et, ce jour-là, j’appellerai mon serviteur,
        Éliakim, fils d’Helcias.
        ${}^{21}Je le revêtirai de ta tunique,
        je le ceindrai de ton écharpe,
        je lui remettrai tes pouvoirs :
        \\il sera un père pour les habitants de Jérusalem
        et pour la maison de Juda.
        ${}^{22}Je mettrai sur son épaule la clef de la maison de David :
        \\s’il ouvre, personne ne fermera ;
        s’il ferme, personne n’ouvrira.
        ${}^{23}Je le planterai comme une cheville
        dans un endroit solide ;
        \\il sera un trône de gloire
        pour la maison de son père.
${}^{24}Le poids de la gloire de la maison de son père
        y sera suspendu :
        les rameaux et les pousses,
        \\et même tous les petits récipients,
        depuis les coupes jusqu’aux vases de toute sorte.
${}^{25}Ce jour-là – oracle du Seigneur de l’univers –,
        la cheville plantée dans un endroit solide lâchera,
        elle cédera, elle tombera,
        \\et la charge qui pesait sur elle sera détruite.
        Le Seigneur a parlé. »
      
         
      \bchapter{}
         
${}^{1}Proclamation sur Tyr.
        
           
         
        \\Hurlez, vaisseaux de Tarsis,
        car Tyr est dévastée : plus de maison ;
        on l’a su au retour de l’île de Chypre.
${}^{2}Restez stupéfaits, habitants de la côte,
        marchands de Sidon, ville qui envoie ses agents outre-mer.
${}^{3}Le grain du Nil, les moissons du Fleuve,
        importés par voie de mer,
        \\faisaient sa richesse :
        elle détenait le marché des nations.
${}^{4}Sois confondue, Sidon,
        car la mer, la forteresse de la mer, a dit :
        \\« Je n’ai ni accouché ni enfanté,
        je n’ai pas eu de garçon à faire grandir
        ni de fille à élever. »
${}^{5}Quand l’Égypte saura la nouvelle,
        aux nouvelles de Tyr, elle frémira.
${}^{6}Passez vers Tarsis,
        hurlez, habitants de la côte.
${}^{7}Est-ce là votre cité joyeuse ?
        Son origine remontait aux jours les plus anciens ;
        \\elle portait au loin ses pas
        pour s’y établir.
${}^{8}Qui donc a décidé ce malheur contre Tyr,
        elle qui distribuait les couronnes,
        \\dont les marchands étaient des princes
        et les commerçants, les plus influents de la terre ?
${}^{9}C’est le Seigneur de l’univers qui l’a décidé
        pour abattre l’orgueil des superbes,
        pour rabaisser tous les gens influents de la terre.
${}^{10}Cultive la terre comme les rives du Nil, fille de Tarsis,
        ton port marchand n’existe plus.
${}^{11}Le Seigneur a levé la main contre la mer,
        et fait trembler des royaumes ;
        \\il a donné l’ordre à Canaan
        de détruire ses fortifications.
${}^{12}Il a dit : « Ne bondis plus de joie :
        on t’a fait violence, vierge, fille de Sidon !
        \\Lève-toi pour passer à Chypre :
        même là-bas, tu n’auras pas de repos.
${}^{13}Regarde vers le pays des Chaldéens :
        ce peuple n’existe plus.
        \\Assour a laissé ce pays aux chats sauvages,
        il a élevé des tours d’assaut ;
        \\il a rasé ses citadelles,
        il en a fait un champ de ruines.
${}^{14}Hurlez, vaisseaux de Tarsis,
        car votre forteresse est dévastée !
        
           
         
${}^{15}Ce jour-là, il arrivera
        que Tyr tombera dans l’oubli
        \\pour soixante-dix ans :
        le temps de vie d’un roi.
        \\Au bout de ces soixante-dix ans, il en sera de Tyr
        comme de la courtisane dont parle la chanson :
${}^{16}« Prends une cithare, fais le tour de la ville,
        courtisane oubliée !
        \\Joue de ton mieux, chante encore ta chanson,
        pour qu’on se souvienne de toi ! »
${}^{17}Au bout des soixante-dix ans,
        Tyr sera visitée par le Seigneur.
        \\Elle retournera à ses profits,
        et se prostituera avec tous les royaumes du monde
        sur la face de la terre.
${}^{18}Ses gains et profits seront consacrés au Seigneur,
        ils ne seront ni amassés ni thésaurisés.
        \\Ses gains serviront à nourrir et rassasier,
        à vêtir somptueusement
        ceux qui habitent devant le Seigneur.
        
           
