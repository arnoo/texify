  
  
    
    \bbook{ESTHER}{ESTHER}
      <div class="intertitle niv10" style="margin-bottom:-1.5em;">
        • TEXTE GREC
      <sup>1a</sup> La deuxième année du règne du grand roi Assuérus, le premier jour du mois de Nissane, Mardochée, fils de Jaïre, fils de Shiméï, fils de Qish, de la tribu de Benjamin, fit un songe. <sup>1b</sup>  C’était un Juif qui habitait la ville de Suse, un personnage important, ayant une fonction à la cour. <sup>1c</sup>  Il faisait partie des captifs que Nabucodonosor, roi de Babylone, avait déportés de Jérusalem avec le roi de Juda, Jékonias.
      <sup>1d</sup> Voici le songe : cris et tumulte, le tonnerre gronde et le sol tremble, toute la terre est bouleversée. <sup>1e</sup>  Et voilà que deux énormes dragons s’avancent, prêts l’un et l’autre au combat, et ils poussent un hurlement. <sup>1f</sup>  À ce bruit, toutes les nations se préparent à la guerre contre le peuple des justes. <sup>1g</sup>  Jour de ténèbres et d’obscurité ! Souffrance, détresse, angoisse, grand bouleversement sur la terre ! <sup>1h</sup>  Bouleversé de terreur devant les maux qui l’attendent, le peuple juste tout entier se prépare à périr et crie vers Dieu ; <sup>1i</sup>  à son cri, comme d’une petite source, naît un grand fleuve, une eau abondante. <sup>1k</sup>  La lumière se lève avec le soleil ; les humbles sont exaltés et dévorent les superbes.
      <sup>1l</sup> À son réveil, Mardochée, ayant vu ce songe et pensant y découvrir le dessein de Dieu, le retint dans son cœur et, jusqu’à la nuit, s’efforça de toutes les manières d’en pénétrer le sens.
      <div class="intertitle niv10" style="margin-bottom:-1.5em;">
        • TEXTE HÉBREU
      
         
      \bchapter{}
      \begin{verse}
${}^{1}C’était au temps d’Assuérus – cet Assuérus qui régnait sur cent vingt-sept provinces, depuis l’Inde jusqu’à l’Éthiopie. 
${}^{2}En ces jours-là, comme le roi Assuérus siégeait sur son trône royal, qui est à Suse-la-Citadelle, 
${}^{3}la troisième année de son règne, il donna en sa présence un banquet pour tous ses princes et ses serviteurs, les chefs de l’armée de Perse et de Médie, les nobles et les gouverneurs des provinces. 
${}^{4}Il voulait leur montrer la richesse de sa gloire royale et la splendeur de sa grande magnificence, pendant de longs jours – cent quatre-vingts jours durant.
${}^{5}Après cette période, pour toute la population de Suse-la-Citadelle, pour les gens importants comme pour les humbles, le roi donna un banquet de sept jours dans la cour du jardin du palais royal. 
${}^{6}Des tentures blanches et violettes étaient attachées par des cordelières de lin et de pourpre à des anneaux d’argent et à des colonnes de marbre blanc. Pour le banquet, des lits d’or et d’argent étaient posés sur un pavement de porphyre, de marbre blanc, de nacre et de marbre noir. 
${}^{7}On servait à boire dans des vases d’or de différentes formes, et le vin du roi était versé avec une libéralité royale. 
${}^{8}La règle était de boire sans contrainte, car le roi avait ordonné à tous les maîtres d’hôtel de servir selon le bon plaisir de chacun.
${}^{9}La reine Vasti avait également organisé un banquet pour les femmes dans le palais du roi Assuérus.
${}^{10}Le septième jour, alors que le roi avait le cœur joyeux sous l’effet du vin, il donna l’ordre à Mehoumane, à Bizzeta, à Harbona, à Bigta, à Abagta, à Zétar, à Karkas – les sept eunuques qui étaient au service du roi Assuérus – 
${}^{11}de faire venir devant le roi la reine Vasti, portant sa couronne royale, pour montrer sa beauté aux peuples et aux princes, car elle était agréable à voir. 
${}^{12}Mais la reine Vasti refusa de venir selon l’ordre du roi transmis par les eunuques. Le roi en fut très irrité et sa colère s’enflamma.
${}^{13}Le roi s’adressa alors aux sages qui avaient la connaissance des temps, – car toute affaire royale devait aller devant les spécialistes de la loi et du droit. 
${}^{14}Les plus proches étaient Karshena, Shétar, Admata, Tarshish, Mérès, Marsena, Memoukane, les sept chefs de Perse et de Médie qui voyaient la face du roi et siégeaient au premier rang du royaume. Il leur dit : 
${}^{15}« Que faire, conformément à la loi, pour punir la reine Vasti de n’avoir pas obéi à l’ordre d’Assuérus transmis par les eunuques ? »
${}^{16}Memoukane prit la parole en présence du roi et de ses princes : « Ce n’est pas seulement contre le roi que la reine Vasti a mal agi, mais contre tous les princes et contre tous les peuples dans toutes les provinces du roi Assuérus. 
${}^{17}Car son attitude sera connue de toutes les femmes et leur fera mépriser leurs maris, quand on leur dira : “Le roi Assuérus avait ordonné de faire venir la reine Vasti en sa présence, et elle n’est pas venue !” 
${}^{18}Et dès aujourd’hui, les princesses de Perse et de Médie qui auront entendu parler de l’attitude de la reine vont se mettre à répliquer à tous les princes du roi. Ce ne seront que mépris et colère ! 
${}^{19}Si le roi le trouve bon, qu’il publie une ordonnance royale qui sera inscrite dans les lois de Perse et de Médie, et sera irrévocable : selon cette ordonnance, Vasti ne paraîtra plus en présence du roi Assuérus qui donnera son titre de reine à une autre, meilleure qu’elle. 
${}^{20}Et le décret que le roi aura publié sera connu dans tout son royaume – et il est grand ! Alors, toutes les femmes auront du respect pour leurs maris, du plus important au plus humble. »
${}^{21}La proposition plut au roi et aux princes, et le roi agit selon les conseils de Memoukane. 
${}^{22}Il envoya des lettres dans toutes ses provinces, pour chaque province selon son écriture, et pour chaque peuple selon sa langue, afin que tout homme, parlant la langue de son peuple, fût maître dans sa maison.
      
         
      \bchapter{}
      \begin{verse}
${}^{1}Après ces événements, quand la colère du roi Assuérus se fut apaisée, il se souvint de Vasti, de sa conduite et des décisions prises à son sujet. 
${}^{2}Les gens qui étaient au service du roi dirent alors : « Qu’on cherche pour le roi des jeunes filles vierges, agréables à voir ; 
${}^{3}qu’il établisse dans toutes les provinces de son royaume des commissaires, pour rassembler à Suse-la-Citadelle, dans la maison des femmes, toutes les jeunes filles vierges et agréables à voir, sous l’autorité de Hégué, eunuque du roi, gardien des femmes, qui leur donnera ce qui est nécessaire à leur toilette. 
${}^{4}La jeune fille qui plaira au roi deviendra reine à la place de Vasti. » Le conseil plut au roi, et il le suivit.
${}^{5}Or il y avait dans Suse-la-Citadelle un Juif du nom de Mardochée, fils de Jaïre, fils de Shiméï, fils de Qish, homme de Benjamin ; 
${}^{6}il avait été emmené de Jérusalem par le roi de Babylone Nabucodonosor, parmi les captifs déportés avec Jékonias, roi de Juda. 
${}^{7}Il élevait alors Hadassa – c’est Esther –, fille de son oncle, qui était orpheline de père et de mère. La jeune fille avait belle prestance et elle était agréable à voir. À la mort de son père et de sa mère, Mardochée l’avait adoptée comme fille.
${}^{8}Lorsque furent connus l’ordre du roi et son édit, de nombreuses jeunes filles furent rassemblées à Suse-la-Citadelle, sous l’autorité de Hégué. Esther fut choisie parmi elles et conduite dans la maison du roi, sous l’autorité de Hégué, gardien des femmes. 
${}^{9}La jeune fille lui plut et gagna sa faveur. Il se hâta de lui donner ce qui était nécessaire à sa toilette et à sa subsistance, il lui attribua sept suivantes, venant de la maison du roi, et l’installa avec ses suivantes dans le meilleur appartement de la maison des femmes. 
${}^{10}Esther n’avait révélé ni son peuple ni son origine, car Mardochée le lui avait interdit. 
${}^{11}Chaque jour, Mardochée passait devant la cour de la maison des femmes, pour savoir comment allait Esther et comment on la traitait.
${}^{12}Chaque jeune fille devait se présenter à son tour au roi Assuérus, au terme d’une année, où elle avait accompli son temps réglementaire de préparation : pendant six mois, elle usait d’huile de myrrhe, et, pendant six mois, de baumes et de crèmes de beauté pour les femmes. 
${}^{13}Quand elle se présentait au roi, chaque jeune fille obtenait tout ce qu’elle demandait pour l’emporter avec elle en passant de la maison des femmes au palais royal. 
${}^{14}Elle s’y rendait le soir et, le lendemain matin, elle passait dans la seconde maison des femmes, sous l’autorité de Shaashgaz, l’eunuque royal chargé des concubines. Elle ne revenait pas chez le roi, à moins que le roi ne la désire et ne la rappelle nommément.
${}^{15}Pour Esther, fille d’Abihaïl, l’oncle de ce Mardochée qui l’avait adoptée comme fille, quand vint le tour de se rendre chez le roi, elle ne demanda rien d’autre que ce qu’avait indiqué Hégué, l’eunuque du roi, gardien des femmes. Et Esther gagnait la bienveillance de tous ceux qui la voyaient. 
${}^{16}Esther fut amenée au roi Assuérus, au palais royal, au dixième mois, qui est le mois de Téveth, la septième année de son règne. 
${}^{17}Et le roi la préféra à toutes les autres femmes. Elle gagna sa bienveillance et sa faveur plus que toutes les autres jeunes filles. Il mit sur sa tête la couronne royale et la fit reine à la place de Vasti.
${}^{18}Le roi fit alors un grand banquet pour ses princes et ses serviteurs, le banquet d’Esther. Il accorda des allégements aux provinces et fit des largesses avec une libéralité royale.
${}^{19}Esther, en passant comme les jeunes filles dans la seconde maison des femmes, 
${}^{20}n’avait révélé ni son origine ni son peuple, comme le lui avait ordonné Mardochée : elle exécutait l’ordre de Mardochée, comme au temps où elle était sous sa tutelle.
${}^{21}En ces jours-là, tandis que Mardochée était assis à la porte du roi, Bigtane et Tèresh, deux eunuques du roi qui faisaient partie de la garde du seuil, furent irrités et cherchèrent à porter la main sur le roi Assuérus. 
${}^{22}L’affaire fut connue de Mardochée, qui la révéla à la reine Esther, et celle-ci en informa le roi au nom de Mardochée. 
${}^{23}Le fait ayant été vérifié et reconnu exact, les deux hommes furent pendus à une potence. L’histoire fut relatée dans le livre des Chroniques royales.
      
         
      \bchapter{}
      \begin{verse}
${}^{1}Après ces événements, le roi Assuérus distingua Amane, fils de Hamdata, du pays d’Agag ; il l’éleva en dignité et lui accorda la prééminence sur tous les princes, ses collègues. 
${}^{2}Tous les serviteurs du roi, qui étaient de service à la porte du roi, s’agenouillaient et se prosternaient devant Amane. Ainsi en avait ordonné le roi. Mais Mardochée ne s’agenouillait pas et ne se prosternait pas. 
${}^{3}Et les serviteurs du roi qui étaient de service à la porte du roi dirent à Mardochée : « Pourquoi transgresses-tu l’ordre du roi ? » 
${}^{4}Mais ils avaient beau le lui répéter tous les jours, il ne les écoutait pas. Ils le dénoncèrent à Amane, pour voir si Mardochée persisterait dans son attitude. Il leur avait fait connaître en effet qu’il était juif. 
${}^{5}Amane constata que Mardochée ne s’agenouillait pas et ne se prosternait pas devant lui, et il fut rempli de fureur. 
${}^{6}Comme on lui avait appris de quel peuple était Mardochée, il dédaigna de porter la main sur lui seul, et il résolut de faire disparaître, avec Mardochée, tous les Juifs qui étaient établis dans tout le royaume d’Assuérus.
      
         
       
${}^{7}L’an douze du règne d’Assuérus, au premier mois, qui est le mois de Nissane, on tira au sort, nommé le « Pour », en présence d’Amane, chaque jour et chaque mois. Le sort tomba sur le douzième mois, qui est le mois nommé Adar. 
${}^{8}Amane dit au roi Assuérus : « Il y a un peuple à part, dispersé au milieu des peuples, dans toutes les provinces de ton royaume. Ses lois ne ressemblent à celles d’aucun autre peuple, et ils n’observent pas les lois du roi. Le roi n’a pas intérêt à les laisser en paix. 
${}^{9}S’il plaît au roi, qu’il donne par écrit l’ordre de les faire périr, et sur leurs biens je ferai compter par les fonctionnaires dix mille talents d’argent à remettre au trésor royal. » 
${}^{10}Le roi ôta alors son anneau de sa main, et le donna à Amane, fils de Hamdata, du pays d’Agag, l’ennemi des Juifs. 
${}^{11}Le roi dit à Amane : « Je t’abandonne l’argent, le peuple aussi, pour en faire comme bon te semblera. »
${}^{12}Les scribes du roi furent alors appelés, le treizième jour du premier mois et, tout ce qu’Amane avait ordonné, on l’écrivit aux satrapes du roi, aux gouverneurs qui étaient dans chaque province, aux princes de chaque peuple, à chaque province selon son écriture et à chaque peuple selon sa langue. On écrivit au nom du roi Assuérus et on scella le document avec l’anneau royal. 
${}^{13}Et des lettres furent envoyées, par l’entremise de courriers, à toutes les provinces royales, ordonnant d’exterminer, de tuer, de faire périr tous les Juifs, depuis les jeunes jusqu’aux vieillards, les femmes comme les enfants, en un seul jour, le treizième jour du douzième mois, qui est le mois nommé Adar, et de piller leurs biens.
      <div class="intertitle niv10" style="margin-bottom:-1.5em;">
        • TEXTE GREC
      <a class="anchor bib_verset" id="bib_est_3_13_a">13a</a>Voici le texte de cette lettre :
      Le grand roi Assuérus, aux gouverneurs des cent vingt-sept provinces qui vont de l’Inde à l’Éthiopie, et aux chefs de district, leurs subordonnés, écrit ce qui suit : <a class="anchor bib_verset" id="bib_est_3_13_b">13b</a> « Placé à la tête de nations nombreuses et maître de toute la terre, j’ai voulu, sans me laisser griser par l’orgueil du pouvoir, gouverner avec bienveillance et modération, assurer à mes sujets en tout temps une vie calme, donner au royaume les bienfaits d’une libre circulation jusqu’aux frontières, restaurer la paix que tous les hommes désirent. <a class="anchor bib_verset" id="bib_est_3_13_c">13c</a> Lorsque j’ai consulté mes conseillers pour parvenir à cette fin, l’un d’entre eux à qui la sagesse, l’indéfectible dévouement, l’inébranlable fidélité, ont valu la seconde place dans le royaume, Amane, <a class="anchor bib_verset" id="bib_est_3_13_d">13d</a> nous a révélé qu’à toutes les populations répandues dans le monde se trouve mêlé un peuple hostile, opposé par ses lois à toute nation, des gens qui rejettent continuellement les ordonnances royales, au point de faire obstruction au gouvernement commun qu’à la satisfaction générale nous maintenons dans la bonne direction.
      <a class="anchor bib_verset" id="bib_est_3_13_e">13e</a>Nous avons donc reconnu que cette nation, et elle seule, s’oppose constamment à tous les hommes, se met à part en vivant selon des lois étrangères, qu’elle a des sentiments hostiles à notre gouvernement et qu’elle commet les pires méfaits, jusqu’à compromettre la stabilité du royaume. <a class="anchor bib_verset" id="bib_est_3_13_f">13f</a> Pour ces motifs, nous ordonnons que ceux qui sont désignés par les documents d’Amane, lequel est commis aux soins de nos intérêts et qui est pour nous un second père, soient exterminés par l’épée de leurs ennemis, tous, femmes et enfants inclus, sans aucune pitié ni ménagement, le quatorzième jourdu mois nommé Adar, douzième mois de l’année en cours. <a class="anchor bib_verset" id="bib_est_3_13_g">13g</a> Ainsi, ces opposants d’hier et d’aujourd’hui seront, en un seul jour, précipités violemment au séjour des morts et, pour les temps à venir, la stabilité et la tranquillité nous seront définitivement assurées. »
      <div class="intertitle niv10" style="margin-bottom:-1.5em;">
        • TEXTE HÉBREU
${}^{14}La copie de ce document, destiné à être promulgué comme loi dans chaque province, fut publiée dans toutes les populations, afin qu’elles soient prêtes pour le jour dit. 
${}^{15}Sur l’ordre du roi, les courriers partirent en hâte. La loi fut publiée à Suse-la-Citadelle.
      Le roi et Amane étaient assis et buvaient, tandis que la ville de Suse était bouleversée.
      
         
      \bchapter{}
      \begin{verse}
${}^{1}Dès qu’il apprit tout ce qui venait d’arriver, Mardochée déchira ses vêtements, se couvrit de cendre et d’une toile à sac. Il parcourut la ville en poussant un grand cri de douleur. 
${}^{2}Il alla jusqu’en face de la porte du roi, que nul ne pouvait franchir revêtu d’une toile à sac. 
${}^{3}Dans toutes les provinces, partout où étaient parvenus l’ordre du roi et son édit, ce fut parmi les Juifs un grand deuil : jeûne, larmes, lamentations ; beaucoup se couchèrent sur le sac et la cendre.
${}^{4}Les servantes d’Esther, ainsi que ses eunuques, l’avertirent, et la reine en fut toute bouleversée. Elle fit envoyer des vêtements à Mardochée, pour qu’il les mette et enlève son sac, mais il refusa. 
${}^{5}Esther appela Hatak, l’un des eunuques que le roi avait placés auprès d’elle, et lui donna l’ordre d’aller trouver Mardochée pour savoir ce qui se passait, et les raisons de sa conduite.
${}^{6}Hatak se rendit auprès de Mardochée, sur la place de la ville, en face de la porte du roi. 
${}^{7}Mardochée l’informa de tout ce qui lui était arrivé, et du montant de la somme d’argent qu’Amane avait proposé de verser au trésor royal, en échange de l’extermination des Juifs. 
${}^{8}Il lui remit une copie de l’édit promulgué à Suse pour les anéantir. Il chargea Hatak de le montrer à Esther, pour qu’elle soit informée. Il enjoignait à la reine d’aller chez le roi pour implorer sa grâce et plaider devant lui la cause de son peuple.
      <div class="intertitle niv10">
        • TEXTE GREC
      <a class="anchor bib_verset" id="bib_est_4_8_a">8a</a>« Souviens-toi des jours où tu n’étais rien, où je te nourrissais de ma main ! Car Amane, le second personnage du royaume, nous a accusés pour nous faire mourir. <a class="anchor bib_verset" id="bib_est_4_8_b">8b</a> Invoque le Seigneur, parle de nous au roi, délivre-nous de la mort. »
      <div class="intertitle niv10">
        • TEXTE HÉBREU
${}^{9}Hatak revint et rapporta à Esther les paroles de Mardochée. 
${}^{10}Elle ordonna à Hatak de lui répondre :
${}^{11}« Tous les serviteurs du roi et les habitants des provinces royales savent bien que, pour quiconque, homme ou femme, qui se rend auprès du roi dans la cour intérieure sans avoir été convoqué, il n’y a qu’une seule loi : la mort. Sauf celui auquel le roi tend son sceptre d’or : il a la vie sauve. Moi-même, cela fait trente jours que je n’ai pas été convoquée chez le roi. »
${}^{12}Les paroles d’Esther furent transmises à Mardochée 
${}^{13}qui lui fit répondre à son tour : « Ne t’imagine pas que, parce que tu es dans la maison du roi, tu en réchapperas, seule parmi les Juifs. 
${}^{14}Car si tu persistes à te taire aujourd’hui, c’est d’un autre lieu que viendront pour les Juifs soulagement et délivrance, et toi et la maison de ton père, vous périrez. Qui sait si ce n’est pas en vue d’une circonstance comme celle-ci que tu as accédé à la royauté ? » 
${}^{15}Esther fit répondre à Mardochée : 
${}^{16}« Va, rassemble tous les Juifs qui se trouvent à Suse. Jeûnez pour moi, ne mangez pas, ne buvez pas pendant trois jours, nuit et jour. Moi, je jeûnerai aussi avec mes servantes. C’est alors que j’irai chez le roi, en dépit de la loi, et s’il faut périr, je périrai. » 
${}^{17}Mardochée se retira et fit tout ce qu’Esther lui avait ordonné.
      <div class="intertitle niv10">
        • TEXTE GREC
      <a class="anchor bib_verset couleur" id="bib_est_4_17_a">17a</a>Alors, Mardochée pria le Seigneur en rappelant toutes ses œuvres\\ :
       
        <a class="anchor bib_verset couleur poem" id="bib_est_4_17_b">17b</a>« Seigneur, Seigneur, Roi souverain de l’univers,
        tout est soumis à ton pouvoir,
        \\personne ne peut s’opposer à toi
        quand tu veux sauver Israël.
         
        <a class="anchor bib_verset couleur poem" id="bib_est_4_17_c">17c</a>C’est toi qui as fait le ciel et la terre
        et toutes les merveilles qui sont sous le ciel.
        \\Tu es le Seigneur de l’univers,
        et il n’est personne qui puisse te résister, Seigneur.
         
        <a class="anchor bib_verset couleur poem" id="bib_est_4_17_d">17d</a>Tu le sais, ô Seigneur, toi qui connais tout :
        ce n’est pas insolence, orgueil ou vanité,
        \\si j’ai refusé de me prosterner devant l’orgueilleux Amane.
        – Volontiers je lui baiserais les pieds pour le salut d’Israël.
         
        <a class="anchor bib_verset couleur poem" id="bib_est_4_17_e">17e</a>Mais j’ai refusé pour ne pas mettre la gloire d’un homme
        plus haut que la gloire de Dieu :
        \\je ne me prosternerai devant personne,
        sauf devant toi, mon Seigneur,
        \\et ce n’est pas là de l’orgueil.
         
        <a class="anchor bib_verset couleur poem" id="bib_est_4_17_f">17f</a>Et maintenant, écoute-moi, Seigneur Dieu,
        ô Roi, Dieu d’Abraham,
        épargne ton peuple !
        \\Car ils ont projeté de nous perdre,
        ils veulent détruire ce peuple  , ton héritage depuis toujours.
         
        <a class="anchor bib_verset couleur poem" id="bib_est_4_17_g">17g</a>Ne méprise pas ta part,
        elle est à toi :
        \\tu nous as rachetés
        en nous faisant sortir de la terre d’Égypte.
         
        <a class="anchor bib_verset couleur poem" id="bib_est_4_17_h">17h</a>Exauce ma prière,
        sois favorable à ceux qui sont ta part d’héritage ;
        \\change notre deuil en joie,
        \\afin que nous vivions
        pour chanter ton nom, Seigneur.
        \\Ne laisse pas disparaître
        ceux dont la bouche te célèbre. »
         
        <a class="anchor bib_verset poem" id="bib_est_4_17_i">17i</a>Tous ceux d’Israël criaient de toutes leurs forces, car ils avaient la mort devant les yeux.
      <a class="anchor bib_verset couleur" id="bib_est_4_17_k">17k</a>La reine Esther, dans l’angoisse mortelle qui l’étreignait, cherchait refuge auprès du Seigneur. Elle enleva ses vêtements d’apparat et prit des vêtements de deuil et d’affliction. Au lieu de parfums précieux, elle se couvrit la tête de cendre et de poussière. Elle humilia durement son corps et le recouvrit de ses cheveux en désordre, lui qu’elle se faisait une joie de parer. Elle priait ainsi le Seigneur, Dieu d’Israël\\ :
        <a class="anchor bib_verset couleur poem" id="bib_est_4_17_l">17l</a>« Mon Seigneur, notre Roi,
        tu es l’Unique ;
        \\viens me secourir, car je suis seule,
        je n’ai pas d’autre secours que toi,
        \\et je vais risquer ma vie.
         
        <a class="anchor bib_verset couleur poem" id="bib_est_4_17_m">17m</a>Depuis ma naissance, j’entends dire,
        dans la tribu de mes pères,
        \\que toi, Seigneur, tu as choisi Israël parmi toutes les nations,
        et que parmi tous leurs ancêtres tu as choisi nos pères,
        \\pour en faire à jamais ton héritage ;
        \\tu as fait pour eux tout ce que tu avais promis.
         
        <a class="anchor bib_verset poem" id="bib_est_4_17_n">17n</a>Et maintenant, nous avons péché contre toi,
        tu nous as livrés aux mains de nos ennemis,
        \\parce que nous avons honoré leurs dieux :
        tu es juste, Seigneur.
         
        <a class="anchor bib_verset poem" id="bib_est_4_17_o">17o</a>Et maintenant, notre dur esclavage ne leur suffit plus.
        Ils ont fait un pacte avecleurs idoles,
        \\pour abolir ce que ta bouche a promis,
        faire disparaître ton héritage,
        \\fermer la bouche de ceux qui te célèbrent,
        éteindre la gloire de ta maison et les feux deton autel,
        <a class="anchor bib_verset poem" id="bib_est_4_17_p">17p</a>pour que s’ouvre la bouche des nations,
        que soient célébrés les mérites des faux dieux
        et qu’à jamais soit magnifié un roi de chair.
         
        <a class="anchor bib_verset poem" id="bib_est_4_17_q">17q</a>Ne livre pas ton sceptre, Seigneur,
        à ceux qui n’existent pas.
        \\Que nos ennemisne se moquent pas de notre chute ;
        retourne contre eux leurs projets.
        \\Du premier de nos adversaires,
        fais un exemple.
         
        <a class="anchor bib_verset couleur poem" id="bib_est_4_17_r">17r</a>Souviens-toi, Seigneur !
        Fais-toi connaître au jour de notre détresse ;
        \\donne-moi du courage, toi, le Roi des dieux,
        qui domines toute autorité.
         
        <a class="anchor bib_verset couleur poem" id="bib_est_4_17_s">17s</a>Mets sur mes lèvres un langage harmonieux
        quand je serai en présence de ce lion,
        \\et change son cœur :
        qu’il se mette à détester celui qui nous combat,
        qu’il le détruise avec tous ses partisans.
         
        <a class="anchor bib_verset couleur poem" id="bib_est_4_17_t">17t</a>Délivre-nous par ta main,
        viens me secourir car je suis seule,
        \\et je n’ai que toi, Seigneur.
         
        <a class="anchor bib_verset couleur poem" id="bib_est_4_17_u">17u</a>Tu connais tout et tu sais
        que je hais la gloire des impies,
        \\que je n’ai que dégoût pour la couche des incirconcis
        et celle de tout étranger.
         
        <a class="anchor bib_verset poem" id="bib_est_4_17_w">17w</a>Tu sais la contrainte où je suis,
        que j’ai du dégoût pour l’orgueilleux emblème
        qui est sur ma tête aux jours où je parais en public.
        \\Il m’inspire du dégoût comme un linge souillé,
        et je ne le porte pas les jours où je me repose.
         
        <a class="anchor bib_verset poem" id="bib_est_4_17_x">17x</a>Ta servante n’a pas mangé à la table d’Amane,
        ni honoré les banquets du roi,
        ni bu le vin des libations.
         
        <a class="anchor bib_verset poem" id="bib_est_4_17_y">17y</a>Ta servante n’a pas connu la joie
        depuis le jour de son élévation,
        \\si ce n’est auprès de toi, Seigneur,
        Dieu d’Abraham.
         
        <a class="anchor bib_verset poem" id="bib_est_4_17_z">17z</a>Ô Dieu, qui as pouvoir sur tous,
        écoute la voix des désespérés,
        \\délivre-nous de la main des méchants,
        et délivre-moi de ma peur ! »
      
         
      \bchapter{}
      \begin{verse}
${}^{1}Au troisième jour, lorsqu’elle eut cessé de prier, Esther quitta ses vêtements de suppliante et revêtit des habits d’apparat. <a class="anchor bib_verset" id="bib_est_5_1_a">1a</a> Ainsi dans tout son éclat, ayant invoqué le Dieu qui protège et qui sauve, elle prit avec elle ses deux servantes : sur l’une d’elles, elle s’appuyait mollement ; l’autre l’accompagnait en portant sa traîne. <a class="anchor bib_verset" id="bib_est_5_1_b">1b</a> Elle-même était rougissante, au comble de sa beauté ; son visage était radieux, comme épanoui par l’amour, mais son cœur était serré par la peur. <a class="anchor bib_verset" id="bib_est_5_1_c">1c</a> Franchissant toutes les portes, elle se trouva devant le roi. Il était assis sur son trône royal, revêtu de l’habit avec lequel il paraissait en public, rutilant d’or et de pierres précieuses : il était redoutable. <a class="anchor bib_verset" id="bib_est_5_1_d">1d</a> Il leva son visage rayonnant de gloire et, au comble de la colère, il la fixa. La reine s’effondra. Prise de faiblesse, elle changea de couleur et se pencha vers la tête de la servante qui la précédait. <a class="anchor bib_verset" id="bib_est_5_1_e">1e</a> Dieu changea le cœur du roi et l’inclina à la douceur. Saisi d’angoisse, il s’élança de son trône et la prit dans ses bras jusqu’à ce qu’elle se remît. Il la réconfortait par des paroles apaisantes : <a class="anchor bib_verset" id="bib_est_5_1_f">1f</a> « Qu’y a-t-il, Esther ? Je suis ton frère. Rassure-toi ! Tu ne mourras pas : notre décret ne vaut que pour le commun des gens. Viens avec moi ! » 
${}^{2} Levant son sceptre d’or, il le posa sur le cou d’Esther, puis il l’embrassa en disant : « Parle-moi ». <a class="anchor bib_verset" id="bib_est_5_2_a">2a</a> « Seigneur, lui dit-elle, je t’ai vu pareil à un ange de Dieu. Mon cœur a été troublé, j’ai eu peur de ta gloire. Car tu es admirable, Seigneur, et ton visage est merveilleux. » <a class="anchor bib_verset" id="bib_est_5_2_b">2b</a> Tandis qu’elle parlait, elle tomba de faiblesse. Le roi était bouleversé, et toute sa suite la réconfortait.
      
         
      <div class="intertitle niv10">
        • TEXTE HÉBREU
${}^{3}Le roi lui demanda : « Qu’y a-t-il, reine Esther ? Quelle est ta requête ? Quand ce serait la moitié du royaume, cela te serait accordé. » 
${}^{4}Esther dit : « S’il plaît au roi, que le roi vienne aujourd’hui avec Amane au banquet que je lui ai préparé. » 
${}^{5}Le roi dit alors : « Allez vite chercher Amane, pour répondre à l’invitation d’Esther ! »
      Le roi et Amane vinrent au banquet préparé par Esther. 
${}^{6}Au cours du banquet, le roi dit de nouveau à Esther : « Quelle est ta demande ? Cela te sera accordé. Quelle est ta requête ? Quand ce serait la moitié du royaume, ce sera réalisé. » 
${}^{7}Esther répondit : « Ma demande ? Ma requête ? 
${}^{8}Si j’ai trouvé grâce aux yeux du roi, s’il lui plaît d’exaucer ma demande et de réaliser ma requête, que le roi vienne encore demain avec Amane au banquet que je préparerai pour eux, et je me conformerai à la parole du roi. »
${}^{9}Ce jour-là, Amane sortit joyeux et le cœur content. Mais lorsque, à la porte du roi, il vit Mardochée qui ne se levait pas et ne se dérangeait pas à sa vue, il fut rempli de fureur contre Mardochée. 
${}^{10}Mais il se domina et rentra chez lui. Il envoya chercher ses amis et sa femme Zéresh, 
${}^{11}et, longuement, il leur parla de ses somptueuses richesses, de la multitude de ses fils, de tout ce dont le roi l’avait comblé, pour l’élever et le mettre au-dessus de ses princes et de ses serviteurs. 
${}^{12}Amane ajouta : « La reine Esther n’a fait venir que moi, avec le roi, au banquet qu’elle a préparé ; bien plus, elle vient de m’inviter encore demain avec le roi. 
${}^{13}Mais tout cela est sans intérêt pour moi, tant que je verrai Mardochée, le Juif, assis à la porte du roi. » 
${}^{14}Sa femme Zéresh et tous ses amis lui dirent alors : « Que l’on dresse une potence de cinquante coudées et, demain matin, demande au roi qu’on y pende Mardochée. Puis va te réjouir au banquet du roi ! » Le conseil plut à Amane, et il fit préparer la potence.
      
         
      \bchapter{}
      \begin{verse}
${}^{1}Or, cette nuit-là, comme le sommeil le fuyait, le roi se fit apporter le livre des Mémoires, les Chroniques, pour s’en faire donner lecture. 
${}^{2}On y trouva écrit ce que Mardochée avait révélé sur les eunuques du roi, Bigtane et Tèresh, deux des gardiens du seuil, qui avaient cherché à porter la main sur le roi Assuérus. 
${}^{3}Le roi demanda : « Quels honneurs et quelle distinction ont récompensé Mardochée pour cette révélation ? » Les jeunes serviteurs du roi lui dirent : « Rien n’a été fait pour le récompenser. » 
${}^{4}Le roi leur demanda alors : « Qui est dans la cour ? » C’était juste le moment où Amane arrivait dans la cour extérieure du palais royal pour demander au roi de faire pendre Mardochée à la potence qu’il avait fait préparer pour lui. 
${}^{5}Les jeunes serviteurs du roi lui répondirent : « C’est Amane qui se tient dans la cour. » Le roi ordonna : « Qu’il entre ! » 
${}^{6}Dès qu’il fut entré, le roi lui dit : « Comment faut-il traiter un homme que le roi désire honorer ? » Amane se dit : « Qui le roi désirerait-il honorer plus que moi ? » 
${}^{7}Il répondit donc au roi : « Le roi désire-t-il honorer quelqu’un ? 
${}^{8}Qu’on apporte un vêtement royal, parmi ceux que le roi a déjà portés, un cheval que le roi a monté et un bandeau royal qui a déjà orné sa tête. 
${}^{9}Que l’on confie vêtement et cheval à l’un des plus nobles des princes du roi. On revêtira alors l’homme que le roi désire honorer, on le conduira à cheval sur la place de la ville, et devant lui on criera : “Voilà comment on traite l’homme que le roi désire honorer !” »
${}^{10}Le roi dit à Amane : « Vite, prends le vêtement et le cheval ! Ce que tu as dit, fais-le pour Mardochée, le Juif, qui est assis à la porte du roi. Et surtout, ne néglige rien de ce que tu as dit. » 
${}^{11}Et Amane, prenant vêtement et cheval, revêtit Mardochée, le conduisit à cheval sur la place de la ville, en criant devant lui : « Voilà comment on traite l’homme que le roi désire honorer ! »
${}^{12}Puis Mardochée revint à la porte du roi tandis qu’Amane se précipitait chez lui, consterné, la tête voilée. 
${}^{13}Amane raconta à sa femme Zéresh et à tous ses amis ce qui lui était arrivé. Ses conseillers et sa femme lui dirent : « Ce Mardochée, devant qui tu commences à capituler, s’il appartient bien à la race des Juifs, tu ne pourras rien contre lui, tu capituleras totalement devant lui. »
${}^{14}Comme ils lui parlaient encore, les eunuques du roi arrivèrent et conduisirent aussitôt Amane au banquet qu’avait préparé Esther.
      
         
      \bchapter{}
      \begin{verse}
${}^{1}Le roi se rendit avec Amane au banquet de la reine Esther. 
${}^{2}Le deuxième jour, au cours du banquet, le roi dit encore à Esther : « Quelle est ta demande, ô reine Esther ? Cela te sera accordé. Quelle est ta requête ? Quand ce serait la moitié du royaume, ce sera réalisé. » 
${}^{3}La reine Esther répondit : « Si j’ai trouvé grâce à tes yeux, ô roi, et s’il plaît au roi, accorde-moi la vie – voilà ma demande. Accorde la vie à mon peuple – voilà ma requête. 
${}^{4}Car nous avons été vendus, moi et mon peuple, pour être exterminés, tués, anéantis. Si nous avions été seulement vendus comme esclaves ou servantes, je me serais tue ; car ce mauvais traitement n’aurait pas justifié que l’on dérange le roi. »
${}^{5}Le roi Assuérus prit la parole et demanda à la reine Esther : « De qui s’agit-il ? Quel est l’homme qui a osé agir ainsi ? » 
${}^{6}Esther répondit : « L’adversaire, l’ennemi, c’est Amane, c’est ce misérable. »
      Devant le roi et la reine, Amane fut terrifié. 
${}^{7}Plein de fureur, le roi se leva, quitta la salle du banquet pour gagner le jardin du palais, tandis qu’Amane se tenait près de la reine Esther et lui demandait grâce pour sa vie, voyant bien que le roi avait décidé sa perte.
${}^{8}Quand le roi revint du jardin du palais dans la salle du banquet, Amane était effondré sur le divan où se trouvait Esther. « Quoi ! dit le roi. Va-t-il encore faire violence à la reine chez moi, dans ma maison ? » Le roi prononça un ordre, et on voila le visage d’Amane. 
${}^{9}Harbona, l’un des eunuques, dit devant le roi : « Il y a justement, dans la maison d’Amane, une potence de cinquante coudées, qu’Amane avait dressée pour Mardochée, l’homme qui a parlé pour le bien du roi. » Le roi dit : « Qu’on l’y pende ! » 
${}^{10}On pendit Amane à la potence qu’il avait préparée pour Mardochée, et la fureur du roi s’apaisa.
       
      
         
      \bchapter{}
      \begin{verse}
${}^{1}Ce jour-là, le roi Assuérus donna à la reine Esther la maison d’Amane, l’adversaire des Juifs. Mardochée se présenta au roi, à qui Esther avait révélé ce qu’il était pour elle. 
${}^{2}Le roi avait repris son anneau à Amane, il l’ôta de son doigt pour le donner à Mardochée. Esther confia la gestion de la maison d’Amane à Mardochée.
      
         
${}^{3}De nouveau, Esther vint parler au roi Assuérus. Elle tomba à ses pieds, en pleurant, et elle le supplia d’écarter le malheur préparé par Amane, et tous les affreux projets qu’il avait formés contre les Juifs. 
${}^{4}Selon la coutume, le roi tendit à Esther son sceptre d’or ; alors Esther se releva et se tint debout devant lui. 
${}^{5}Puis elle dit : « S’il plaît au roi et si j’ai trouvé grâce devant lui, si ma prière ne lui paraît pas déplacée, si je suis bien vue à ses yeux, je le supplie de révoquer par de nouvelles lettres celles qu’a envoyées Amane, fils de Hamdata, du pays d’Agag, ces lettres qui ordonnaient de faire périr les Juifs dans toutes les provinces du royaume. 
${}^{6}Comment pourrais-je donc supporter le mal qu’on veut faire à mon peuple, comment pourrais-je supporter la mort de toute ma parenté ? » 
${}^{7}Le roi Assuérus répondit à la reine Esther et au Juif Mardochée : « J’ai fait cadeau à Esther de la maison d’Amane, et lui, on l’a pendu à la potence parce qu’il avait voulu porter la main sur les Juifs. 
${}^{8}Écrivez donc aux Juifs ce que vous jugerez bon au nom du roi en cachetant les lettres avec mon anneau. Car une lettre écrite au nom du roi et cachetée avec son anneau ne peut pas être révoquée. »
${}^{9}Les scribes royaux furent aussitôt convoqués. C’était le troisième mois, le mois de Sivane, le vingt-troisième jour. Sur l’ordre de Mardochée, ils écrivirent aux Juifs, aux satrapes, aux gouverneurs et aux grands officiers des provinces, depuis l’Inde jusqu’à l’Éthiopie – cent vingt-sept provinces –, à chaque province selon son écriture, et à chaque peuple selon sa langue, aux Juifs aussi selon leur écriture et selon leur langue. 
${}^{10}Ces lettres furent rédigées au nom du roi Assuérus, et cachetées avec son anneau ; elles furent portées par des courriers, montés sur des chevaux des écuries du roi. 
${}^{11}Par ces lettres, le roi accordait aux Juifs, en chaque ville, le droit de se rassembler, de mettre leur vie en sécurité, et celui d’exterminer, de tuer, de faire périr tous les gens armés des peuples ou des provinces qui voudraient les attaquer, y compris les enfants et les femmes, et celui de piller leurs biens. 
${}^{12}Cela se ferait le même jour, dans toutes les provinces du roi Assuérus, le treize du douzième mois, qui est Adar.
      <div class="intertitle niv10" style="margin-bottom:-1.5em;">
        • TEXTE GREC
      <a class="anchor bib_verset" id="bib_est_8_12_a">12a</a>Voici le texte de cette lettre. <a class="anchor bib_verset" id="bib_est_8_12_b">12b</a> « Le grand roi Assuérus, aux satrapes qui gouvernent les cent vingt-sept provinces, de l’Inde à l’Éthiopie, et à tous ceux qui s’occupent de nos affaires, salut !
      <a class="anchor bib_verset" id="bib_est_8_12_c">12c</a>Bien des gens, comblés d’honneurs par l’extrême bonté de leurs bienfaiteurs, deviennent pleins de suffisance. Non seulement ils cherchent à nuire à nos sujets, mais, incapables de supporter même ce qui devrait les contenter, ils entreprennent de comploter contre leurs propres bienfaiteurs. <a class="anchor bib_verset" id="bib_est_8_12_d">12d</a> Non seulement ils suppriment la reconnaissance du milieu des hommes, mais de plus, enivrés par les flatteries de ceux qui ignorent le bien, ils s’imaginent qu’ils vont échapper à la justice divine qui déteste le mal, alors que tout est à jamais sous le regard de Dieu. <a class="anchor bib_verset" id="bib_est_8_12_e">12e</a> Ainsi, maintes fois, ceux qui disposaient du pouvoir suprême, s’étant laissés convaincre par les amis auxquels ils avaient confié l’administration de leurs affaires, sont devenus complices de meurtres d’innocents et de malheurs irréparables : <a class="anchor bib_verset" id="bib_est_8_12_f">12f</a> ces amis-là ont trompé par leurs mensonges et leurs faux raisonnements la parfaite droiture d’intention des souverains. <a class="anchor bib_verset" id="bib_est_8_12_g">12g</a> Sans aller jusqu’aux histoires anciennes que nous venons d’évoquer, il est possible, en examinant ce qui se passe devant nous, d’observer les impiétés commises par une peste de gouvernants indignes. <a class="anchor bib_verset" id="bib_est_8_12_h">12h</a> Aussi nous veillerons à ce que, dans l’avenir, la tranquillité et la paix soient assurées à tous les hommes, dans tout le royaume, <a class="anchor bib_verset" id="bib_est_8_12_i">12i</a> en opérant les changements nécessaires, et en jugeant toujours avec un esprit favorable les affaires qui nous seront soumises.
      <a class="anchor bib_verset" id="bib_est_8_12_k">12k</a>C’est ainsi qu’Amane, fils d’Hamdata, un Macédonien, étranger en vérité au sang des Perses, et bien loin de partager notre générosité, après avoir reçu l’hospitalité chez nous, <a class="anchor bib_verset" id="bib_est_8_12_l">12l</a> a été l’objet de la bienveillance que nous portons à chaque peuple, au point qu’il a été appelé notre père, et qu’il est devenu le second personnage du royaume, devant qui tous se prosternaient ; <a class="anchor bib_verset" id="bib_est_8_12_m">12m</a> il n’a pas dominé son orgueil, il s’est employé à nous priver du pouvoir et de la vie. <a class="anchor bib_verset" id="bib_est_8_12_n">12n</a> De notre sauveur, de l’homme qui a toujours été notre bienfaiteur, Mardochée, et d’Esther, l’irréprochable compagne de notre royauté, Amane, par les manœuvres de ses raisonnements tortueux, nous avait demandé la mort, ainsi que celle de tout leur peuple. <a class="anchor bib_verset" id="bib_est_8_12_o">12o</a> Par de tels agissements, il comptait se saisir de nous quand nous serions isolés, pour remettre ensuite aux Macédoniens l’empire des Perses. <a class="anchor bib_verset" id="bib_est_8_12_p">12p</a> Nous, nous considérons que ces Juifs, voués à l’extermination par ce triple scélérat, ne sont pas des malfaiteurs, mais se gouvernent selon les lois les plus justes, <a class="anchor bib_verset" id="bib_est_8_12_q">12q</a> qu’ils sont les fils du Dieu vivant, le Très-Haut, le Très-Grand, qui a dirigé pour nous et nos ancêtres le royaume de la meilleure façon.
      <a class="anchor bib_verset" id="bib_est_8_12_r">12r</a>Vous ferez bien de ne pas tenir compte des lettres envoyées par Amane, fils d’Hamdata, leur auteur ayant été pendu aux portes de Suse, avec tous les siens : le Dieu qui a pouvoir sur touta fait justice sans délai. <a class="anchor bib_verset" id="bib_est_8_12_s">12s</a> Publiez le texte de la présente lettre en tout lieu et laissez les Juifs suivre ouvertement leurs propres coutumes. Quant à ceux qui se dresseraient pour les massacrer à la date prévue, le treizième jour du mois nommé Adar, aidez les Juifs à les repousser ce même jour. <a class="anchor bib_verset" id="bib_est_8_12_t">12t</a> Car ce jour, qui devait être un jour d’extermination pour la race élue, Dieu, le maître de tout, vient de le changer pour elle en jour d’allégresse. <a class="anchor bib_verset" id="bib_est_8_12_u">12u</a> Et vous, parmi vos fêtes officielles, célébrez ce jour solennel par toutes sortes de réjouissances, afin qu’il soit, dès maintenant et à l’avenir, jour de salut pour nous et pour les Perses de bonne volonté, mais, pour nos ennemis, un mémorial de leur ruine.
      <a class="anchor bib_verset" id="bib_est_8_12_x">12x</a>Toute ville ou contrée sans exception qui ne suivrait pas nos instructions sera impitoyablement dévastée par le fer et le feu, et elle sera rendue non seulement inhabitable aux hommes, mais hostile même aux bêtes sauvages et aux oiseaux, pour la suite des temps. »
      <div class="intertitle niv10" style="margin-bottom:-1.5em;">
        • TEXTE HÉBREU
${}^{13}La copie de ce texte, qui devait être promulgué comme loi dans chaque province, fut communiquée à tous les peuples, pour qu’au jour fixé, les Juifs puissent se venger de leurs ennemis. 
${}^{14}Sur l’ordre du roi, des courriers, montés sur des chevaux royaux, partirent immédiatement, en toute hâte. L’édit fut aussi publié à Suse-la-Citadelle. 
${}^{15}Mardochée sortit alors de chez le roi, portant un vêtement royal, violet et blanc, un grand diadème d’or, un manteau de lin et de pourpre rouge. Toute la ville de Suse criait de joie. 
${}^{16}Pour les Juifs ce n’était que lumière et joie, allégresse et gloire. 
${}^{17}Dans chaque province et chaque ville, là où parvenait l’ordre du roi, son édit, pour les Juifs ce n’était que joie, allégresse, banquets et fêtes. Parmi les peuples de la terre, beaucoup se firent juifs, car la peur des Juifs les avait saisis.
      
         
      \bchapter{}
      \begin{verse}
${}^{1}Le douzième mois, nommé Adar, le treizième jour, où entrait en vigueur l’ordre du roi, son édit, en ce jour où les ennemis des Juifs espéraient les dominer, la situation se renversa : ce furent les Juifs qui dominèrent leurs ennemis. 
${}^{2}Dans toutes les provinces du roi Assuérus, les Juifs se rassemblèrent dans les villes qu’ils habitaient, afin de frapper ceux qui avaient cherché à leur faire du mal. Personne ne leur résista, car la peur des Juifs avait saisi tous les peuples. 
${}^{3}Grands officiers des provinces, satrapes, gouverneurs, fonctionnaires du roi, tous soutenaient les Juifs, car la peur de Mardochée les avait saisis. 
${}^{4}Mardochée était en effet au palais un personnage éminent, et sa renommée se répandait dans toutes les provinces : Mardochée devenait un homme de plus en plus important.
${}^{5}Les Juifs frappèrent alors tous leurs ennemis à coups d’épée. Ce fut une tuerie, un carnage ; leurs adversaires furent livrés à leur bon plaisir. 
${}^{6}À Suse-la-Citadelle, les Juifs tuèrent et firent périr cinq cents hommes, notamment Parshandata, Dalfone, Aspata, 
${}^{8}Porata, Adalya, Aridata, 
${}^{9}Parmashta, Arissaï, Aridaï et Waïzata ; 
${}^{10}c’étaient les dix fils d’Amane, fils de Hamdata, l’adversaire des Juifs. Ils les tuèrent, mais ils ne se livrèrent pas au pillage. 
${}^{11}Le jour même, le nombre de ceux qui avaient été tués à Suse-la-Citadelle parvint à la connaissance du roi.
${}^{12}Le roi dit à la reine Esther : « À Suse-la-Citadelle, les Juifs ont tué et fait périr cinq cents hommes, et les dix fils d’Amane. Dans le reste des provinces royales, qu’ont-ils pu faire ! Quelle est ta demande ? Cela te sera accordé. Quelle est encore ta requête ? Ce sera réalisé. » 
${}^{13}Esther répondit : « S’il plaît au roi, qu’il soit accordé aux Juifs de Suse de faire encore demain ce qu’ils ont fait aujourd’hui, selon son édit, et que l’on pende à la potence les cadavres des fils d’Amane ! » 
${}^{14}Le roi dit : « Qu’il en soit fait ainsi ! » L’édit fut proclamé à Suse, et on pendit les dix fils d’Amane. 
${}^{15}Ainsi, les Juifs de Suse se rassemblèrent encore le quatorzième jour du mois qui est Adar et ils tuèrent trois cents hommes dans Suse. Mais ils ne se livrèrent pas au pillage.
${}^{16}Les autres Juifs, qui étaient dans les provinces royales, se rassemblèrent pour mettre leur vie en sécurité. Ils obtinrent le repos, délivrés de leurs ennemis, en tuant soixante-quinze mille de ceux qui les haïssaient. Mais ils ne se livrèrent pas au pillage. 
${}^{17}C’était le treizième jour du mois qui est Adar. Le quatorzième, ils se reposèrent et firent de ce jour une journée de banquets et de joie. 
${}^{18}Les Juifs de Suse, qui s’étaient rassemblés le treizième et le quatorzième jour du mois, se reposèrent le quinzième et en firent une journée de banquets et de joie. 
${}^{19}Voilà pourquoi les Juifs ruraux, habitant les bourgades rurales, font du quatorzième jour du mois nommé Adar un jour de joie, de banquets et de fête, chacun envoyant des parts de nourriture à son voisin, <a class="anchor bib_verset" id="bib_est_9_19_a">19a</a>tandis que ceux des villes passent aussi dans la joie le quinzième jour du mois nommé Adar, en envoyant des parts de nourriture à leurs voisins.
${}^{20}Mardochée consigna par écrit ces événements et envoya, dans toutes les provinces du roi Assuérus, des lettres à tous les Juifs, proches et éloignés. 
${}^{21}Il les y engageait à célébrer chaque année le quatorzième jour du mois nommé Adar, ainsi que le quinzième jour : 
${}^{22}ces jours-là, les Juifs avaient obtenu le repos, délivrés de leurs ennemis, et, ce mois-là, l’affliction s’était changée pour eux en joie, et le deuil en jour de fête. Il les conviait donc à faire de ces jours des journées de banquets et de joie, chacun envoyant des parts de nourriture à son voisin, et des dons aux pauvres.
${}^{23}Les Juifs acceptèrent comme tradition ce qu’ils avaient commencé à observer, et ce que Mardochée leur avait écrit. 
${}^{24}Amane, fils de Hamdata, du pays d’Agag, l’adversaire de tous les Juifs, avait médité de les faire périr, et avait tiré au sort, nommé en araméen le « Pour », afin de les frapper de panique et de les faire périr. 
${}^{25}Mais quand cela parvint à la connaissance du roi, il ordonna, par un écrit, de faire retomber sur la tête d’Amane le projet coupable qu’il avait formé contre les Juifs, et il le fit pendre, ainsi que ses fils, à une potence. 
${}^{26}Voilà pourquoi on appelle ces jours les Pourim, du mot araméen « Pour ». Voilà aussi pourquoi en se conformant à la lettre de Mardochée, à ce qu’ils avaient vu et à ce qui leur était arrivé, 
${}^{27}les Juifs instituèrent et établirent pour eux, pour leurs descendants et tous ceux qui s’adjoindraient à eux, le devoir de célébrer sans faute ces deux jours, selon ce qui est écrit et au temps fixé, chaque année. 
${}^{28}Ainsi, ces jours seront commémorés et célébrés de génération en génération, dans chaque clan, dans chaque province, dans chaque ville. Ces jours de Pourim devront être célébrés sans faute chez les Juifs, et leur souvenir ne disparaîtra pas de leur descendance.
${}^{29}La reine Esther, fille d’Abihayil, ainsi que Mardochée le Juif, écrivit avec toute autorité pour confirmer cette lettre de Pourim. 
${}^{30}On envoya des lettres à tous les Juifs dans les cent vingt-sept provinces du roi Assuérus, comme paroles de paix et de fidélité. 
${}^{31}On établit ces jours de Pourim aux temps fixés, comme l’avaient établi le Juif Mardochée et la reine Esther, et comme on l’avait établi pour eux et leur descendance, en y joignant des ordonnances de jeûne et de lamentations. 
${}^{32}Ainsi l’ordre d’Esther confirma l’institution des Pourim, et cela fut inscrit dans un livre.
      
         
      \bchapter{}
      \begin{verse}
${}^{1}Le roi Assuérus perçut un impôt sur le continent et les îles de la mer. 
${}^{2}Tous ses actes de puissance et de vaillance, et les détails de la haute situation que le roi avait accordée à Mardochée ne sont-ils pas écrits dans le livre des Chroniques des rois de Médie et de Perse ? 
${}^{3}Car Mardochée le Juif était le second personnage du royaume après le roi Assuérus ; il était grand aux yeux des Juifs, et aimé de la multitude de ses frères. Il recherchait le bien de son peuple et se préoccupait du bonheur de toute sa race.
      
         
      <div class="intertitle niv10" style="margin-bottom:-1.5em;">
        • TEXTE GREC
      <a class="anchor bib_verset" id="bib_est_10_3_a">3a</a>Et Mardochée dit : « C’est grâce à Dieu que tout cela est arrivé. <a class="anchor bib_verset" id="bib_est_10_3_b">3b</a> Je me souviens en effet du songe que j’ai eu à ce sujet, rien n’a été omis : <a class="anchor bib_verset" id="bib_est_10_3_c">3c</a> ni la petite source qui est devenue un fleuve, ni la lumière, ni le soleil, ni l’eau abondante. Esther est ce fleuve, elle que le roi a épousée, et qu’il a faite reine ; <a class="anchor bib_verset" id="bib_est_10_3_d">3d</a> les deux dragons, c’est moi et Amane. <a class="anchor bib_verset" id="bib_est_10_3_e">3e</a> Les nations sont celles qui se sont rassemblées pour effacer le nom des Juifs. <a class="anchor bib_verset" id="bib_est_10_3_f">3f</a> Ma nation, c’est Israël, ceux qui crièrent vers Dieu et furent sauvés. Oui, le Seigneur a sauvé son peuple, le Seigneur nous a arrachés à tous ces maux, Dieu a accompli des signes et de grands prodiges, comme il n’y en eut jamais parmi les nations. <a class="anchor bib_verset" id="bib_est_10_3_g">3g</a> C’est pourquoi il a réservé deux sorts, l’un pour le peuple de Dieu, l’autre pour toutes les nations. <a class="anchor bib_verset" id="bib_est_10_3_h">3h</a> Ces deux sorts ont trouvé accomplissement à l’heure, au temps et au jour que Dieu avait fixés pour toutes les nations : <a class="anchor bib_verset" id="bib_est_10_3_i">3i</a> Dieu s’est souvenu de son peuple, il a rendu justice à son héritage. <a class="anchor bib_verset" id="bib_est_10_3_k">3k</a> Ces quatorzième et quinzième jours du mois nommé Adar seront désormaisdes jours de rassemblement, de joie et d’allégresse devant Dieu, pour toutes les générations et à jamais, en Israël son peuple. »
      <a class="anchor bib_verset" id="bib_est_10_3_l">3l</a>La quatrième année du règne de Ptolémée et de Cléopâtre, Dosithée, qui se déclarait prêtre et lévite, ainsi que son fils Ptolémée, apportèrent la présente lettre concernant la fête de Pourim. Ils la déclaraient authentique et traduite par Lysimaque, fils de Ptolémée, un des habitants de Jérusalem.
