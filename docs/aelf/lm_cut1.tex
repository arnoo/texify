  
  
    
    \bbook{LAMENTATIONS}{LAMENTATIONS}
      <div class="intertitle niv11">
        Aleph
      
         
      \bchapter{}
${}^{1}Comment ! La voilà donc assise, solitaire,
        la ville si populeuse,
        \\semblable à une veuve,
        la reine des nations,
        \\souveraine des peuples,
        devenue esclave !
        
           
      <div class="intertitle niv11">
        Beth
${}^{2}Elle pleure, elle pleure dans la nuit,
        les larmes couvrent ses joues :
        \\personne pour la consoler
        parmi ceux qui l’aimaient ;
        \\ils l’ont trompée, tous ses amis,
        devenus ses ennemis.
      <div class="intertitle niv11">
        Guimel
${}^{3}Elle est déportée, Juda, misérable,
        durement asservie ;
        \\assise au milieu des nations,
        elle ne trouve pas de repos.
        \\Tous ses persécuteurs l’ont traquée
        jusque dans sa détresse.
      <div class="intertitle niv11">
        Daleth
${}^{4}Les routes de Sion sont en deuil,
        car personne ne vient à ses fêtes :
        \\toutes ses portes sont à l’abandon,
        ses prêtres gémissent,
        \\ses vierges s’affligent ;
        elle-même est dans l’amertume !
      <div class="intertitle niv11">
        Hé
${}^{5}Ses adversaires la dominent,
        ses ennemis sont rassurés,
        \\car le Seigneur l’afflige
        pour ses fautes sans nombre ;
        \\ses petits enfants s’en vont, captifs
        devant l’adversaire.
      <div class="intertitle niv11">
        Waw
${}^{6}De la fille de Sion toute splendeur
        s’est retirée ;
        \\ses princes, comme des cerfs
        ne trouvant plus de pâturages,
        \\sont partis à bout de forces
        devant le persécuteur.
      <div class="intertitle niv11">
        Zaïn
${}^{7}Jérusalem se rappelle tous les plaisirs
        des jours d’autrefois,
        \\aux jours de misère et d’errance
        quand son peuple tombe
        aux mains de l’adversaire ;
        \\la voyant privée de secours,
        ses adversaires rient de sa ruine.
      <div class="intertitle niv11">
        Heth
${}^{8}Elle a péché, elle a péché, Jérusalem :
        elle n’est plus que souillure ;
        \\tous ceux qui la glorifiaient la méprisent
        voyant sa nudité ;
        \\elle aussi gémit
        et se détourne.
      <div class="intertitle niv11">
        Teth
${}^{9}Son impureté a taché sa robe ;
        elle n’avait pas imaginé une telle fin :
        \\elle est descendue au plus bas ;
        personne pour la consoler.
        \\« Vois, Seigneur, ma misère :
        l’ennemi a triomphé ! »
      <div class="intertitle niv11">
        Yod
${}^{10}L’adversaire a fait main basse
        sur tous ses trésors :
        \\oui, elle a vu les païens
        entrer dans son sanctuaire,
        \\alors que tu leur avais ordonné :
        « Vous n’entrerez pas dans mon assemblée. »
      <div class="intertitle niv11">
        Kaph
${}^{11}Son peuple tout entier gémit,
        en quête de pain ;
        \\il troque ses trésors contre de la nourriture,
        pour reprendre vie :
        \\« Vois, Seigneur, et regarde
        comme je suis méprisée ! »
      <div class="intertitle niv11">
        Lamed
${}^{12}« Ô vous tous qui passez sur le chemin,
        regardez et voyez
        \\s’il est une douleur pareille
        à la douleur que j’endure,
        \\celle dont le Seigneur m’afflige,
        le jour de sa brûlante colère ! »
      <div class="intertitle niv11">
        Mem
${}^{13}D’en haut il lance un feu dans mes os
        et les piétine ;
        \\il tend un filet sous mes pas,
        il me rejette en arrière ;
        \\il me livre à l’abandon,
        malade à longueur de jour.
      <div class="intertitle niv11">
        Noun
${}^{14}Il attache de sa main
        le joug de mes péchés ;
        \\ils sont entrelacés et posés sur mon cou :
        ma force en est brisée ;
        \\le Seigneur me livre à des mains
        qui m’empêchent de me relever.
      <div class="intertitle niv11">
        Samek
${}^{15}Le Seigneur a terrassé tous les vaillants
        au milieu de mon peuple ;
        \\il a convoqué contre moi un conseil
        pour écraser mes jeunes gens ;
        \\le Seigneur a foulé au pressoir
        la vierge, fille de Juda.
      <div class="intertitle niv11">
        Aïn
${}^{16}Voilà pourquoi je pleure,
        mes yeux, mes yeux fondent en larmes :
        \\il est loin de moi, le consolateur
        qui me rendrait la vie ;
        \\mes fils sont à l’abandon :
        l’ennemi était le plus fort !
      <div class="intertitle niv11">
        Pé
${}^{17}Sion a tendu les mains :
        personne pour la consoler.
        \\Le Seigneur a donné des ordres contre Jacob
        aux adversaires qui l’entourent :
        \\Jérusalem au milieu d’eux
        n’est plus que souillure.
      <div class="intertitle niv11">
        Çadé
${}^{18}Le Seigneur, lui, est juste
        car je suis rebelle à sa parole.
        \\Écoutez donc, vous, tous les peuples,
        et voyez ma douleur :
        \\mes vierges et mes jeunes gens
        sont partis en captivité.
      <div class="intertitle niv11">
        Qoph
${}^{19}J’ai fait appel à mes amants :
        mais ils m’ont trahie.
        \\Mes prêtres et mes anciens
        expirent dans la ville,
        \\cherchant la nourriture
        pour reprendre vie.
      <div class="intertitle niv11">
        Resh
${}^{20}Vois, Seigneur, quelle est ma détresse :
        mes entrailles frémissent ;
        \\mon cœur en moi se retourne
        car j’ai persisté dans ma rébellion ;
        \\dehors, l’épée m’a privée d’enfants,
        dans la maison, c’est la mort.
      <div class="intertitle niv11">
        Shine
${}^{21}On m’entend gémir :
        personne pour me consoler.
        \\Tous mes ennemis entendent mon malheur :
        ils jubilent car c’est toi qui l’as fait.
        \\Tu amèneras le jour que tu as fixé :
        ils seront alors comme moi !
      <div class="intertitle niv11">
        Taw
${}^{22}Que leur méchanceté
        éclate à tes yeux :
        \\traite-les donc comme tu m’as traitée
        pour toutes mes fautes :
        \\nombreux sont mes gémissements,
        malade est mon cœur !
      <div class="intertitle niv11">
        Aleph
      
         
      \bchapter{}
${}^{1}Comment ! Dans sa colère, le Seigneur
        a couvert d’ombre la fille de Sion,
        \\il a précipité du ciel sur la terre
        la splendeur d’Israël,
        \\oubliant, au jour de sa colère,
        qu’elle a été le socle de ses pieds !
        
           
      <div class="intertitle niv11">
        Beth
        ${}^{2}Le Seigneur a englouti sans pitié
        tous les pâturages de Jacob ;
        \\dans son emportement, il a détruit les forteresses
        de la fille de Juda ;
        \\il a jeté à terre et profané
        le royaume et ses princes.
      <div class="intertitle niv11">
        Guimel
${}^{3}Il a brisé dans l’ardeur de sa colère
        toute la force d’Israël ;
        \\il a retiré sa main droite
        de devant l’ennemi,
        \\allumé dans Jacob un feu brûlant
        qui dévore alentour.
      <div class="intertitle niv11">
        Daleth
${}^{4}Comme un ennemi, il a tendu son arc
        et, comme un adversaire, levé sa main droite ;
        \\tout ce qui charmait les yeux,
        il l’a tué,
        \\et sur la demeure de la fille de Sion
        il a déversé sa fureur comme un feu.
      <div class="intertitle niv11">
        Hé
${}^{5}Le Seigneur est comme un ennemi :
        il a englouti Israël,
        \\englouti ses citadelles,
        rasé ses forteresses,
        \\répandu sur la fille de Juda
        plaintes et complaintes.
      <div class="intertitle niv11">
        Waw
${}^{6}Il a forcé sa clôture comme celle d’un jardin,
        rasé le lieu de la Rencontre ;
        \\le Seigneur a fait oublier dans Sion
        fêtes et sabbats ;
        \\dans l’excès de sa colère il a déshonoré
        le roi et le prêtre.
      <div class="intertitle niv11">
        Zaïn
${}^{7}Le Seigneur a délaissé son autel
        et maudit son sanctuaire ;
        \\il a livré aux mains de l’ennemi
        les remparts des citadelles ;
        \\des cris éclatent dans la maison du Seigneur
        comme aux jours de fête.
      <div class="intertitle niv11">
        Heth
${}^{8}Le Seigneur a décidé de raser la muraille
        de la fille de Sion ;
        \\il va niveler, sans retirer sa main
        avant que tout soit englouti ;
        \\il met en deuil muraille et avant-mur :
        ensemble on se désole.
      <div class="intertitle niv11">
        Teth
${}^{9}Ses portes s’enfoncent sous la terre :
        il en a détruit et brisé les barres ;
        \\son roi et ses princes sont chez les païens :
        la Loi n’existe plus,
        \\ses prophètes eux-mêmes ne reçoivent plus
        de visions du Seigneur.
      <div class="intertitle niv11">
        Yod
        ${}^{10}Les anciens de la fille de Sion,
        assis par terre, se taisent,
        \\ils ont couvert leur tête de poussière
        et revêtu des toiles à sac ;
        \\elles inclinent la tête vers la terre,
        les vierges de Jérusalem.
      <div class="intertitle niv11">
        Kaph
        ${}^{11}Mes yeux sont usés\\par les larmes,
        mes entrailles frémissent ;
        \\je vomis par terre ma bile
        face au malheur de la fille de mon peuple,
        \\alors que défaillent petits enfants et nourrissons
        sur les places de la cité.
      <div class="intertitle niv11">
        Lamed
        ${}^{12}À leur mère ils demandent :
        « Où sont le froment et le vin ? »
        \\alors qu’ils défaillent comme des blessés
        sur les places de la ville
        \\et qu’ils rendent l’âme
        sur le sein de leur mère.
      <div class="intertitle niv11">
        Mem
        ${}^{13}Que dire de toi ? À quoi te comparer,
        fille de Jérusalem ?
        \\À quoi te rendre égale pour te consoler,
        vierge, fille de Sion ?
        \\Car ton malheur est grand comme la mer !
        Qui donc te guérira ?
      <div class="intertitle niv11">
        Noun
        ${}^{14}Tes prophètes ont de toi des visions
        vides et sans valeur ;
        \\ils n’ont pas dévoilé ta faute,
        ce qui aurait ramené tes captifs\\ ;
        \\ils ont de toi des visions,
        proclamations vides et illusoires.
      <div class="intertitle niv11">
        Samek
${}^{15}Tous les passants du chemin
        battent des mains contre toi ;
        \\ils sifflent et hochent la tête
        devant la fille de Jérusalem :
        \\« Est-ce la ville que l’on disait “Toute-belle”,
        “Joie de toute la terre” ? »
      <div class="intertitle niv11">
        Pé
${}^{16}Contre toi ils ouvrent la bouche,
        tous tes ennemis,
        \\ils sifflent et grincent des dents ;
        ils disent : « Nous l’avons engloutie !
        \\Voilà bien le jour que nous espérions :
        nous y arrivons, nous le voyons ! »
      <div class="intertitle niv11">
        Aïn
${}^{17}Le Seigneur fait ce qu’il a résolu,
        il accomplit sa parole
        \\décrétée depuis les jours d’autrefois :
        il détruit sans pitié !
        \\Il réjouit à tes dépens l’ennemi,
        il accroît la force de tes adversaires.
      <div class="intertitle niv11">
        Çadé
        ${}^{18}Le cœur du peuple\\crie vers le Seigneur.
        Laisse couler le torrent de tes larmes,
        \\de jour comme de nuit,
        muraille de la fille de Sion ;
        \\ne t’accorde aucun répit,
        que tes pleurs ne tarissent pas\\ !
      <div class="intertitle niv11">
        Qoph
        ${}^{19}Lève-toi ! Pousse un cri dans la nuit
        au début de chaque veille ;
        \\déverse ton cœur comme l’eau
        devant la face du Seigneur ;
        \\élève les mains vers lui
        pour la vie de tes petits enfants
        \\qui défaillent de faim
        à tous les coins de rue\\.
      <div class="intertitle niv11">
        Resh
${}^{20}Vois, Seigneur, et regarde :
        qui as-tu traité ainsi ?
        \\Les femmes doivent-elles manger leurs enfants,
        les petits qu’elles choyaient ?
        \\Le prêtre et le prophète doivent-ils être tués
        au sanctuaire du Seigneur ?
      <div class="intertitle niv11">
        Shine
${}^{21}Ils gisent par terre dans les rues,
        l’adolescent et le vieillard ;
        \\mes vierges et mes jeunes gens
        sont tombés par l’épée :
        \\tu as tué, au jour de ta colère,
        tu as massacré sans pitié.
      <div class="intertitle niv11">
        Taw
${}^{22}Tu as convoqué, comme pour un jour de fête,
        mes terreurs de toute part ;
        \\il n’est pas de rescapé ni de survivant
        au jour de la colère du Seigneur ;
        \\ceux que j’avais nourris et choyés,
        mon ennemi les a exterminés.
      <div class="intertitle niv11">
        Aleph
      
         
      \bchapter{}
${}^{1}Je suis l’homme qui a connu la misère
        sous le bâton de Ses emportements,
${}^{2}moi qu’il a conduit et mené
        dans les ténèbres et non dans la lumière ;
${}^{3}contre moi seul, tout le jour,
        il porte et porte encore sa main.
        
           
      <div class="intertitle niv11">
        Beth
${}^{4}Il use ma chair et ma peau,
        il me brise les os ;
${}^{5}il me cerne, il m’environne
        d’amertume et de peine ;
${}^{6}il me fait habiter les ténèbres,
        comme les morts de tous les temps.
      <div class="intertitle niv11">
        Guimel
${}^{7}Il m’a emmuré, et je ne peux sortir,
        il alourdit ma chaîne :
${}^{8}j’ai beau crier et supplier,
        il étouffe ma prière ;
${}^{9}d’un bloc de pierre il barre mes routes,
        il détourne mes sentiers.
      <div class="intertitle niv11">
        Daleth
${}^{10}Pour moi, il est un ours à l’affût,
        un lion en embuscade,
${}^{11}il me fait perdre ma route, me désoriente,
        me laisse désemparé :
${}^{12}il tend son arc, il me choisit
        comme cible pour sa flèche.
      <div class="intertitle niv11">
        Hé
${}^{13}Il a planté dans mes reins
        les dards de son carquois.
${}^{14}Je suis la risée de tout mon peuple,
        leur chanson de chaque jour.
${}^{15}Il m’a gorgé d’herbes amères,
        abreuvé d’absinthe.
      <div class="intertitle niv11">
        Waw
${}^{16}Il m’a broyé les dents avec du gravier,
        il m’enfouit dans la cendre.
        ${}^{17}Tu enlèves la paix à mon âme\\,
        j’ai oublié le bonheur ;
        ${}^{18}j’ai dit : « Mon assurance a disparu,
        et l’espoir qui me venait du Seigneur. »
      <div class="intertitle niv11">
        Zaïn
        ${}^{19}Rappelle-toi ma misère et mon errance\\,
        l’absinthe et le poison.
        ${}^{20}Elle se rappelle, mon âme, elle se rappelle ;
        en moi, elle défaille.
        ${}^{21}Voici ce que je redis en mon cœur,
        et c’est pourquoi j’espère\\ :
      <div class="intertitle niv11">
        Heth
        ${}^{22}Grâce à l’amour du Seigneur\\,
        nous ne sommes pas anéantis ;
        ses tendresses ne s’épuisent pas ;
        ${}^{23}elles se renouvellent chaque matin,
        – oui, ta fidélité surabonde.
        ${}^{24}Je me dis : « Le Seigneur est mon partage,
        c’est pourquoi j’espère en lui. »
      <div class="intertitle niv11">
        Teth
        ${}^{25}Le Seigneur est bon pour qui se tourne vers lui,
        pour celui qui le cherche.
        ${}^{26}Il est bon d’espérer en silence
        le salut du Seigneur ;
${}^{27}il est bon pour l’homme de porter le joug
        dès sa jeunesse.
      <div class="intertitle niv11">
        Yod
${}^{28}Qu’il reste assis, solitaire, en silence,
        tant que le Seigneur le lui impose ;
${}^{29}qu’il tienne sa bouche contre terre :
        peut-être y a-t-il un espoir !
${}^{30}Qu’il tende la joue à qui le frappe,
        qu’il se laisse saturer d’insultes.
      <div class="intertitle niv11">
        Kaph
${}^{31}Car le Seigneur ne rejette pas
        pour toujours ;
${}^{32}s’il afflige, il fera miséricorde
        selon l’abondance de sa grâce ;
${}^{33}ce n’est pas de bon cœur qu’il humilie,
        qu’il afflige les enfants des hommes.
      <div class="intertitle niv11">
        Lamed
${}^{34}Quand on foule aux pieds
        tous les captifs d’un pays,
${}^{35}quand on fait dévier le droit d’un homme
        à la face du Très-Haut,
${}^{36}quand on lèse quelqu’un dans son procès,
        le Seigneur ne le voit-il pas ?
      <div class="intertitle niv11">
        Mem
${}^{37}Qui donc parle, et cela existe,
        sinon le Seigneur, lui qui décrète ?
${}^{38}N’est-ce pas de la bouche du Très-Haut
        que sortent malheurs et bonheur ?
${}^{39}De quoi l’homme vivant se plaindrait-il ?
        Qu’il soit plutôt vaillant contre ses péchés !
      <div class="intertitle niv11">
        Noun
${}^{40}Examinons nos chemins, scrutons-les
        et revenons au Seigneur ;
${}^{41}élevons notre cœur et nos mains
        vers Dieu qui est au ciel.
${}^{42}Nous, nous avons péché, nous avons trahi ;
        toi, tu n’as pas pardonné.
      <div class="intertitle niv11">
        Samek
${}^{43}Enveloppé de colère, tu nous as pourchassés,
        tu as tué sans pitié,
${}^{44}enveloppé dans ta nuée
        que la prière ne peut franchir.
${}^{45}Tu as fait de nous des rebuts,
        des ordures parmi les peuples.
      <div class="intertitle niv11">
        Pé
${}^{46}Ils ouvrent la bouche contre nous,
        tous nos ennemis.
${}^{47}Pour nous, l’effroi et le vertige,
        le désastre et la ruine !
${}^{48}Des torrents s’échappent de mes yeux
        sur la ruine de la fille de mon peuple.
      <div class="intertitle niv11">
        Aïn
${}^{49}Mes yeux ne cessent de couler :
        pas de répit
${}^{50}tant que le Seigneur ne se penche
        et ne voie du haut du ciel.
${}^{51}Pour toutes les filles de ma ville
        mes yeux me font mal.
      <div class="intertitle niv11">
        Çadé
${}^{52}Mes ennemis, sans raison, me chassent,
        me pourchassent comme un oiseau ;
${}^{53}ils m’ont réduit au silence de la fosse,
        m’ont recouvert d’une pierre ;
${}^{54}les eaux ont submergé ma tête ;
        j’ai dit : « Je suis perdu ! »
      <div class="intertitle niv11">
        Qoph
${}^{55}J’ai invoqué ton nom, Seigneur,
        des profondeurs de la fosse ;
${}^{56}tu m’as entendu dire : « Ne ferme pas l’oreille
        à mes soupirs, à mes clameurs ! »
${}^{57}Au jour où je t’invoquais, tu t’es fait proche
        et tu as dit : « Ne crains pas ! »
      <div class="intertitle niv11">
        Resh
${}^{58}Tu as plaidé ma cause, Seigneur :
        tu as racheté ma vie.
${}^{59}Tu vois, Seigneur, le tort qu’ils me font :
        rends-moi justice !
${}^{60}Tu vois toute leur rancune,
        tous leurs complots contre moi.
      <div class="intertitle niv11">
        Shine
${}^{61}Tu entends leurs insultes, Seigneur,
        tous leurs complots contre moi,
${}^{62}les propos de mes agresseurs et leurs murmures
        contre moi tout le jour ;
${}^{63}regarde-les, assis ou debout,
        c’est moi qu’ils chansonnent.
      <div class="intertitle niv11">
        Taw
${}^{64}Tu leur rendras la pareille, Seigneur,
        selon les œuvres de leurs mains.
${}^{65}Tu abêtiras leur cœur :
        sur eux sera ta malédiction.
${}^{66}Tu les pourchasseras de ta colère,
        tu les supprimeras de dessous ton ciel, Seigneur.
      <div class="intertitle niv11">
        Aleph
      
         
      \bchapter{}
${}^{1}Comment ! L’or s’est terni,
        \\l’or pur s’est altéré,
        \\et les pierres sacrées sont dispersées
        à tous les coins de rue !
        
           
      <div class="intertitle niv11">
        Beth
${}^{2}Les fils de Sion, si précieux,
        évalués à prix d’or fin,
        \\comment ! ils sont traités en cruches d’argile,
        ouvrages des mains d’un potier !
      <div class="intertitle niv11">
        Guimel
${}^{3}Même les chacals présentent leurs mamelles
        pour allaiter leurs petits ;
        \\la fille de mon peuple est devenue aussi cruelle
        que l’autruche du désert.
      <div class="intertitle niv11">
        Daleth
${}^{4}La langue du nourrisson assoiffé
        colle à son palais ;
        \\les petits enfants réclament du pain,
        mais nul ne leur en donne.
      <div class="intertitle niv11">
        Hé
${}^{5}Les mangeurs de mets délicats
        dépérissent dans les rues ;
        \\ceux qui vivaient dans le luxe
        se retrouvent sur le fumier.
      <div class="intertitle niv11">
        Waw
${}^{6}La faute de la fille de mon peuple
        a dépassé le péché de Sodome
        \\qui fut anéantie en un instant
        sans qu’on ait porté la main contre elle.
      <div class="intertitle niv11">
        Zaïn
${}^{7}Ses hommes voués à Dieu étaient plus brillants que neige,
        plus blancs que le lait,
        \\leurs corps, plus vermeils que le corail,
        leur apparence était de saphir.
      <div class="intertitle niv11">
        Heth
${}^{8}Leur aspect est plus noir que la suie :
        dans la rue, on ne les reconnaît plus ;
        \\ils n’ont que la peau sur les os,
        aussi sèche que du bois.
      <div class="intertitle niv11">
        Teth
${}^{9}Plus heureuses les victimes de l’épée
        que les victimes de la faim,
        \\qui s’épuisent, diaphanes,
        privées du fruit de leurs champs.
      <div class="intertitle niv11">
        Yod
${}^{10}Dans le malheur de la fille de mon peuple,
        des femmes, de leurs mains tendres,
        \\ont dû faire cuire leurs enfants
        en guise de nourriture.
      <div class="intertitle niv11">
        Kaph
${}^{11}Le Seigneur a détruit dans sa fureur,
        déversé sa brûlante colère,
        \\allumé, dans Sion, un feu
        qui dévore ses fondations.
      <div class="intertitle niv11">
        Lamed
${}^{12}Ils ne croyaient pas, les rois de la terre,
        ni aucun habitant du monde,
        \\que l’adversaire, l’ennemi, franchirait
        les portes de Jérusalem.
      <div class="intertitle niv11">
        Mem
${}^{13}Les péchés de ses prophètes en sont la cause,
        et les fautes de ses prêtres,
        \\eux qui versaient en pleine ville
        le sang des justes.
      <div class="intertitle niv11">
        Noun
${}^{14}Ils erraient en aveugles dans les rues,
        souillés de sang,
        \\et toucher leurs vêtements
        devenait interdit.
      <div class="intertitle niv11">
        Samek
${}^{15}« Arrière, impurs ! » leur criait-on,
        « Arrière, arrière ! N’y touchez pas ! »
        \\Ils fuyaient, errants, chez les païens qui disaient :
        « Ils ne s’installeront pas ! »
      <div class="intertitle niv11">
        Pé
${}^{16}La face du Seigneur les a dispersés
        et ne les regarde plus ;
        \\les prêtres ne sont pas respectés
        ni les anciens, honorés.
      <div class="intertitle niv11">
        Aïn
${}^{17}Sans répit, nos yeux se consument
        d’attendre en vain du secours ;
        \\nous guettons, nous guettons une nation
        qui ne peut pas sauver.
      <div class="intertitle niv11">
        Çadé
${}^{18}On nous a chassés et pourchassés :
        nous ne pouvons plus aller sur nos places.
        \\Notre fin approche, nos jours sont comptés ;
        notre fin est arrivée.
      <div class="intertitle niv11">
        Qoph
${}^{19}Nos poursuivants sont plus rapides
        que les aigles dans les airs ;
        \\dans les montagnes, ils nous harcèlent,
        dans le désert, ils nous traquent.
      <div class="intertitle niv11">
        Resh
${}^{20}Le Messie du Seigneur, le souffle de nos narines,
        a été pris dans leurs embûches,
        \\lui dont nous disions : « Sous sa protection,
        nous vivrons parmi les nations. »
      <div class="intertitle niv11">
        Shine
${}^{21}Jubile et réjouis-toi, fille d’Édom
        qui habites le pays de Ouç !
        \\À toi aussi, la coupe sera tendue :
        tu t’enivreras et te montreras nue.
      <div class="intertitle niv11">
        Taw
${}^{22}Ta faute est expiée, fille de Sion,
        il ne te déportera plus.
        \\Il punira ta faute, fille d’Édom,
        il dévoilera tes péchés.
      <p class="cantique" id="bib_ct-at_37"><span class="cantique_label">Cantique AT 37</span> = <span class="cantique_ref"><a class="unitex_link" href="#bib_lm_5_1">Lm 5, 1-7.15-21</a></span>
      
         
      \bchapter{}
        ${}^{1}Rappelle-toi, Seigneur, ce qui nous arrive.
        Regarde, et vois notre honte.
        
           
         
        ${}^{2}Notre héritage a passé à des inconnus,
        nos maisons, à des étrangers.
        
           
         
        ${}^{3}Nous sommes orphelins de pères,
        et nos mères sont\\veuves.
        
           
         
        ${}^{4}Notre eau, nous la buvons à prix d’argent ;
        nous achetons notre bois.
        
           
         
        ${}^{5}Nous voici pourchassés, asservis\\ ;
        exténués, nous n’avons pas de repos.
        
           
         
        ${}^{6}Nous tendons la main à l’Égypte,
        à Assour, pour notre part de pain.
        
           
         
        ${}^{7}Nos pères ont péché : ils ne sont plus,
        et c’est nous qui portons leurs fautes.
        
           
         
${}^{8}Pour maîtres nous avons des esclaves
        et nul ne nous tire de leurs mains.
        
           
         
${}^{9}Nous risquons notre vie pour du pain
        face à l’épée du désert.
        
           
         
${}^{10}Notre peau brûle comme un four
        face aux ardeurs de la faim.
        
           
         
${}^{11}Ils ont violé des femmes dans Sion,
        des vierges dans les villes de Juda.
        
           
         
${}^{12}Par leurs mains, des princes ont été pendus,
        la face des anciens n’est pas honorée.
        
           
         
${}^{13}Des jeunes gens ont porté la meule,
        des garçons, sous le poids du bois, trébuchent.
        
           
         
${}^{14}Les anciens ne tiennent plus conseil à la porte,
        et les jeunes ont cessé leurs chansons.
        
           
         
        ${}^{15}La joie de notre cœur a cessé,
        notre danse a fait place au deuil.
        
           
         
        ${}^{16}La couronne est tombée de notre tête.
        Malheur à nous, car nous avons péché !
        
           
         
        ${}^{17}Si notre cœur est malade,
        si nos yeux sont dans la nuit,
        
           
         
        ${}^{18}c’est que le mont Sion est déserté ;
        là, rôdent les renards.
        
           
         
        ${}^{19}Mais toi, Seigneur, tu sièges pour toujours ;
        ton trône est pour les âges des âges.
        
           
         
        ${}^{20}Pourquoi nous oublier sans fin,
        nous abandonner\\pour la suite des jours ?
        
           
         
        ${}^{21}Fais-nous revenir à toi, Seigneur, et nous reviendrons.
        Renouvelle pour nous les jours d’autrefois.
        
           
         
${}^{22}Nous aurais-tu voués au mépris,
        serais-tu irrité contre nous sans mesure ?
        
           
