  
  
          
            \bchapter{Psaume}
            Mon âme a soif de Dieu
${}^{1}Du maître de chœur. Poème. Des fils de Coré.
         
${}^{2}Comme un c\underline{e}rf altéré
        ch\underline{e}rche l’eau vive, *
        \\ainsi mon \underline{â}me te cherche
        t\underline{o}i, mon Dieu.
         
${}^{3}Mon âme a s\underline{o}if de Dieu,
        le Die\underline{u} vivant ; *
        \\quand pourr\underline{a}i-je m’avancer,
        par\underline{a}ître face à Dieu ?
         
${}^{4}Je n’ai d’autre p\underline{a}in que mes larmes,
        le jo\underline{u}r, la nuit, *
        \\moi qui chaque jo\underline{u}r entends dire :
        « Où est-\underline{i}l ton Dieu ? »
         
${}^{5}Je me souviens,
        et mon \underline{â}me déborde : *
        \\en ce temps-là,
        je franchiss\underline{a}is les portails !
         
        \\Je conduisais vers la mais\underline{o}n de mon Dieu
        la multit\underline{u}de en fête, *
        \\parm\underline{i} les cris de joie
        et les acti\underline{o}ns de grâce.
         
${}^{6}Pourquoi te désol\underline{e}r, ô mon âme,
        et gém\underline{i}r sur moi ? *
        \\Espère en Dieu ! De nouvea\underline{u} je rendrai grâce :
        il est mon sauveur et mon Dieu !
         
        *
         
${}^{7}Si mon \underline{â}me se désole,
        je me souvi\underline{e}ns de toi, *
        \\depuis les terres du Jourd\underline{a}in et de l’Hermon,
        depuis mon h\underline{u}mble montagne.
         
${}^{8}L’abîme appel\underline{a}nt l’abîme
        à la v\underline{o}ix de tes cataractes, *
        \\la masse de tes fl\underline{o}ts et de tes vagues
        a pass\underline{é} sur moi.
         
${}^{9}Au long du jo\underline{u}r, le Seigneur
        m’env\underline{o}ie son amour ; *
        \\et la nuit, son ch\underline{a}nt est avec moi,
        prière au Die\underline{u} de ma vie.
         
${}^{10}Je dirai à Die\underline{u}, mon rocher :
        « Pourqu\underline{o}i m’oublies-tu ? *
        \\Pourquoi v\underline{a}is-je assombri,
        press\underline{é} par l’ennemi ? »
         
${}^{11}Outrag\underline{é} par mes adversaires,
        je suis meurtr\underline{i} jusqu’aux os, *
        \\moi qui chaque jo\underline{u}r entends dire :
        « Où est-\underline{i}l ton Dieu ? »
         
${}^{12}Pourquoi te désol\underline{e}r, ô mon âme,
        et gém\underline{i}r sur moi ? *
        \\Espère en Dieu ! De nouvea\underline{u} je rendrai grâce :
        il est mon sauve\underline{u}r et mon Dieu !
      \bchapter{Psaume}
          
            \bchapter{Psaume}
            Jusqu’en ta demeure
${}^{1}Rends-moi justice, ô mon Die\underline{u}, défends ma cause
        contre un pe\underline{u}ple sans foi ; *
        \\de l’homme qui r\underline{u}se et trahit,
        l\underline{i}bère-moi.
         
${}^{2}C’est toi, Die\underline{u}, ma forteresse :
        pourqu\underline{o}i me rejeter ? *
        \\Pourquoi v\underline{a}is-je assombri,
        press\underline{é} par l’ennemi ?
         
${}^{3}Envoie ta lumi\underline{è}re et ta vérité :
        qu’elles gu\underline{i}dent mes pas *
        \\et me conduisent à ta mont\underline{a}gne sainte,
        j\underline{u}squ’en ta demeure.
         
${}^{4}J’avancerai jusqu’à l’aut\underline{e}l de Dieu,
        vers Dieu qui est to\underline{u}te ma joie ; *
        \\je te rendrai gr\underline{â}ce avec ma harpe,
        Die\underline{u}, mon Dieu !
         
${}^{5}Pourquoi te désol\underline{e}r, ô mon âme,
        et gém\underline{i}r sur moi ? *
        \\Espère en Dieu ! De nouvea\underline{u} je rendrai grâce :
        il est mon sauve\underline{u}r et mon Dieu !
      \bchapter{Psaume}
          
            \bchapter{Psaume}
            Ne nous rejette pas
${}^{1}Du maître de chœur. Des fils de Coré. Poème.
         
${}^{2}Dieu, nous av\underline{o}ns entendu dire, +
        \\et nos pères nous \underline{o}nt raconté, *
        \\quelle action tu accompl\underline{i}s de leur temps,
        aux jo\underline{u}rs d’autrefois.
         
${}^{3}Toi, par ta main, tu as déposséd\underline{é} les nations, +
        \\et ils p\underline{u}rent s’implanter ; *
        \\et tu as malmen\underline{é} des peuplades,
        et ils p\underline{u}rent s’étendre.
         
${}^{4}Ce n’était pas leur épée qui posséd\underline{a}it le pays, +
        \\ni leur bras qui les rend\underline{a}it vainqueurs, *
        \\mais ta droite et ton bras, et la lumi\underline{è}re de ta face,
        c\underline{a}r tu les aimais.
         
${}^{5}Toi, Dieu, tu \underline{e}s mon roi, *
        \\tu décides des vict\underline{o}ires de Jacob :
${}^{6}avec toi, nous batti\underline{o}ns nos ennemis ;
        \\par ton nom, nous écrasi\underline{o}ns nos adversaires.
         
${}^{7}Ce n’est pas sur mon \underline{a}rme que je compte,
        \\ni sur mon ép\underline{é}e, pour la victoire.
${}^{8}Tu nous as donné de v\underline{a}incre l’adversaire,
        \\tu as couvert notre ennem\underline{i} de honte.
         
${}^{9}Dieu était notre lou\underline{a}nge, tout le jour :
        \\sans cesse nous rendions gr\underline{â}ce à ton nom.
         
        *
         
${}^{10}Maintenant, tu nous humil\underline{i}es, tu nous rejettes,
        \\tu ne sors plus av\underline{e}c nos armées.
${}^{11}Tu nous fais pli\underline{e}r devant l’adversaire,
        \\et nos ennemis emp\underline{o}rtent le butin.
         
${}^{12}Tu nous traites en bét\underline{a}il de boucherie,
        \\tu nous disperses parm\underline{i} les nations.
${}^{13}Tu vends ton pe\underline{u}ple à vil prix,
        \\sans que tu g\underline{a}gnes à ce marché.
         
${}^{14}Tu nous exposes aux sarc\underline{a}smes des voisins,
        \\aux rires, aux moquer\underline{i}es de l’entourage.
${}^{15}Tu fais de nous la f\underline{a}ble des nations ;
        \\les étrangers ha\underline{u}ssent les épaules.
         
${}^{16}Tout le jour, ma déché\underline{a}nce est devant moi,
        \\la honte co\underline{u}vre mon visage,
${}^{17}sous les sarcasmes et les cr\underline{i}s de blasphème,
        \\sous les yeux de l’ennem\underline{i} qui se venge.
         
        *
         
${}^{18}Tout cela est venu sur nous sans que nous t’ay\underline{o}ns oublié : *
        \\nous n’avions pas trah\underline{i} ton alliance.
         
${}^{19}Notre cœur ne s’ét\underline{a}it pas détourné
        \\et nos pieds n’avaient pas quitt\underline{é} ton chemin
${}^{20}quand tu nous poussais au milie\underline{u} des chacals
        \\et nous couvrais de l’\underline{o}mbre de la mort.
         
${}^{21}Si nous avions oublié le n\underline{o}m de notre Dieu,
        \\tendu les mains vers un die\underline{u} étranger,
${}^{22}Dieu ne l’eût-il p\underline{a}s découvert,
        \\lui qui conn\underline{a}ît le fond des cœurs ?
${}^{23}C’est pour toi qu’on nous mass\underline{a}cre sans arrêt,
        \\qu’on nous traite en bét\underline{a}il d’abattoir.
         
${}^{24}Réveille-toi ! Pourquoi dors-t\underline{u}, Seigneur ?
        \\Lève-toi ! Ne nous rejette p\underline{a}s pour toujours.
${}^{25}Pourquoi détourn\underline{e}r ta face,
        \\oublier notre malhe\underline{u}r, notre misère ?
         
${}^{26}Oui, nous mord\underline{o}ns la poussière,
        \\notre ventre c\underline{o}lle à la terre.
${}^{27}Debout ! Vi\underline{e}ns à notre aide !
        \\Rachète-nous, au n\underline{o}m de ton amour.
      \bchapter{Psaume}
          
            \bchapter{Psaume}
            Ton Dieu t’a consacré
${}^{1}Du maître de chœur. Sur l’air de « Les lis ». Des fils de Coré. Poème. Chant d’amour.
         
${}^{2}D’heureuses paroles jaill\underline{i}ssent de mon cœur
        \\quand je dis mes po\underline{è}mes pour le roi
        \\d’une langue aussi vive que la pl\underline{u}me du scribe !
         
        *
         
${}^{3}Tu es beau, comme aucun des enf\underline{a}nts de l’homme,
        \\la grâce est répand\underline{u}e sur tes lèvres :
        \\oui, Dieu te bén\underline{i}t pour toujours.
         
${}^{4}Guerrier valeureux, porte l’épée de nobl\underline{e}sse et d’honneur !
${}^{5}Ton honneur, c’est de cour\underline{i}r au combat
        \\pour la justice, la clém\underline{e}nce et la vérité.
         
${}^{6}Ta main jettera la stupeur, les fl\underline{è}ches qui déchirent ;
        \\sous tes coups, les pe\underline{u}ples s’abattront,
        \\les ennemis du roi, frapp\underline{é}s en plein cœur.
         
${}^{7}Ton trône est divin, un tr\underline{ô}ne éternel ;
        \\ton sceptre royal est sc\underline{e}ptre de droiture :
${}^{8}tu aimes la justice, tu répro\underline{u}ves le mal.
         
        \\Oui, Dieu, ton Die\underline{u} t’a consacré
        \\d’une onction de joie, comme auc\underline{u}n de tes semblables ;
${}^{9}la myrrhe et l’aloès parf\underline{u}ment ton vêtement.
         
        \\Des palais d’ivoire, la mus\underline{i}que t’enchante.
${}^{10}Parmi tes bien-aimées sont des f\underline{i}lles de roi ;
        \\à ta droite, la préférée, sous les \underline{o}rs d’Ophir.
         
        *
         
${}^{11}Écoute, ma fille, reg\underline{a}rde et tends l’oreille ;
        \\oublie ton peuple et la mais\underline{o}n de ton père :
${}^{12}le roi sera sédu\underline{i}t par ta beauté.
         
        \\Il est ton Seigneur : prosterne-t\underline{o}i devant lui.
${}^{13}Alors, fille de Tyr, les plus r\underline{i}ches du peuple,
        \\chargés de présents, quêter\underline{o}nt ton sourire.
         
${}^{14}Fille de roi, elle est l\underline{à}, dans sa gloire,
        \\vêtue d’ét\underline{o}ffes d’or ;
${}^{15}on la conduit, toute par\underline{é}e, vers le roi.
         
        \\Des jeunes filles, ses compagnes, lui f\underline{o}nt cortège ;
${}^{16}on les conduit parmi les ch\underline{a}nts de fête :
        \\elles entrent au pal\underline{a}is du roi.
         
        *
         
${}^{17}À la place de tes pères se lèver\underline{o}nt tes fils ;
        \\sur toute la terre tu feras d’e\underline{u}x des princes.
         
${}^{18}Je ferai vivre ton nom pour les \underline{â}ges des âges :
        \\que les peuples te rendent grâce, toujo\underline{u}rs, à jamais !
      \bchapter{Psaume}
          
            \bchapter{Psaume}
            Il est avec nous
${}^{1}Du maître de chœur. Des fils de Coré. Sur les « ‘alamôth ». Cantique.
         
${}^{2}Dieu est pour nous ref\underline{u}ge et force,
        \\secours dans la détresse, toujo\underline{u}rs offert.
${}^{3}Nous serons sans crainte si la t\underline{e}rre est secouée,
        \\si les montagnes s’effondrent au cre\underline{u}x de la mer ;
${}^{4}ses flots peuvent mug\underline{i}r et s’enfler,
        \\les montagnes, trembl\underline{e}r dans la tempête :
         
        \[Il \underline{e}st avec nous,
        le Seigne\underline{u}r de l’univers ;
        citad\underline{e}lle pour nous,
        le Die\underline{u} de Jacob !\]
         
${}^{5}Le Fleuve, ses bras réjouissent la v\underline{i}lle de Dieu,
        \\la plus sainte des deme\underline{u}res du Très-Haut.
${}^{6}Dieu s’y tient : elle \underline{e}st inébranlable ;
        \\quand renaît le matin, Die\underline{u} la secourt.
${}^{7}Des peuples mugissent, des r\underline{è}gnes s’effondrent ;
        \\quand sa voix retentit, la t\underline{e}rre se défait.
         
${}^{8}Il \underline{e}st avec nous,
        le Seigne\underline{u}r de l’univers ;
        citad\underline{e}lle pour nous,
        le Die\underline{u} de Jacob !
         
${}^{9}Venez et voyez les \underline{a}ctes du Seigneur,
        \\comme il couvre de ru\underline{i}nes la terre.
${}^{10}Il détruit la guerre jusqu’au bo\underline{u}t du monde,
        \\il casse les arcs, brise les lances, incend\underline{i}e les chars :
${}^{11}« Arrêtez ! Sach\underline{e}z que je suis Dieu.
        \\Je domine les nations, je dom\underline{i}ne la terre. »
         
${}^{12}Il \underline{e}st avec nous,
        le Seigne\underline{u}r de l’univers ;
        citad\underline{e}lle pour nous,
        le Die\underline{u} de Jacob !
      \bchapter{Psaume}
          
            \bchapter{Psaume}
            Dieu s’élève au-dessus de tous
${}^{1}Du maître de chœur. Des fils de Coré. Psaume.
         
${}^{2}Tous les peuples, batt\underline{e}z des mains,
        \\acclamez Dieu par vos cr\underline{i}s de joie !
         
${}^{3}Car le Seigneur est le Très-Ha\underline{u}t, le redoutable,
        \\le grand roi sur to\underline{u}te la terre,
${}^{4}celui qui nous soum\underline{e}t des nations,
        \\qui tient des pe\underline{u}ples sous nos pieds ;
${}^{5}il choisit pour no\underline{u}s l’héritage,
        \\fierté de Jac\underline{o}b, son bien-aimé.
         
${}^{6}Dieu s’élève parm\underline{i} les ovations,
        \\le Seigneur, aux écl\underline{a}ts du cor.
${}^{7}Sonnez pour notre Die\underline{u}, sonnez,
        \\sonnez pour notre r\underline{o}i, sonnez !
${}^{8}Car Dieu est le r\underline{o}i de la terre :
        \\que vos mus\underline{i}ques l’annoncent !
         
${}^{9}Il règne, Die\underline{u}, sur les païens,
        \\Dieu est assis sur son tr\underline{ô}ne sacré.
${}^{10}Les chefs des peuples se s\underline{o}nt rassemblés :
        \\c’est le peuple du Die\underline{u} d’Abraham.
        \\Les princes de la t\underline{e}rre sont à Dieu
        \\qui s’élève au-dess\underline{u}s de tous.
      \bchapter{Psaume}
          
            \bchapter{Psaume}
            Dans la ville de notre Dieu
${}^{1}Cantique. Psaume. Des fils de Coré.
         
${}^{2}Il est grand, le Seigneur, hautem\underline{e}nt loué, +
        dans la v\underline{i}lle de notre Dieu, *
${}^{3}sa sainte montagne, alti\underline{è}re et belle,
        joie de to\underline{u}te la terre.
         
        \\La montagne de Sion, c’est le p\underline{ô}le du monde,
        la cit\underline{é} du grand roi ; *
${}^{4}Dieu se rév\underline{è}le, en ses palais,
        vr\underline{a}ie citadelle.
         
${}^{5}Voici que des r\underline{o}is s’étaient ligués,
        ils avanç\underline{a}ient tous ensemble ; *
${}^{6}ils ont vu, et soud\underline{a}in stupéfaits,
        pris de pan\underline{i}que, ils ont fui.
         
${}^{7}Et voilà qu’un tremblem\underline{e}nt les saisit :
        douleurs de f\underline{e}mme qui accouche ; *
${}^{8}un vent qui so\underline{u}ffle du désert
        a brisé les vaissea\underline{u}x de Tarsis.
         
${}^{9}Nous l’avions entend\underline{u}, nous l’avons vu
        dans la ville du Seigneur, Die\underline{u} de l’univers, *
        \\dans la ville de Die\underline{u}, notre Dieu,
        qui l’affermir\underline{a} pour toujours.
         
${}^{10}Dieu, nous reviv\underline{o}ns ton amour
        au milie\underline{u} de ton temple. *
${}^{11}Ta louange, c\underline{o}mme ton nom,
        couvre l’étend\underline{u}e de la terre.
         
        \\Ta main droite qui d\underline{o}nne la victoire
${}^{12}réjouit la mont\underline{a}gne de Sion ; *
        \\les villes de Jud\underline{a} exultent
        dev\underline{a}nt tes jugements.
         
${}^{13}Longez les remp\underline{a}rts de Sion,
        compt\underline{e}z ses tours ; *
${}^{14}que vos cœurs s’épr\underline{e}nnent de ses murs :
        contempl\underline{e}z ses palais.
         
        \\Et vous direz aux \underline{â}ges qui viendront :
${}^{15}« Ce Die\underline{u} est notre Dieu, *
        \\pour toujo\underline{u}rs et à jamais,
        notre gu\underline{i}de pour les siècles. »
      \bchapter{Psaume}
          
            \bchapter{Psaume}
            L’homme comblé ne dure pas
${}^{1}Du maître de chœur. Des fils de Coré. Psaume.
         
${}^{2}Écoutez cec\underline{i}, tous les peuples,
        \\entendez bien, habit\underline{a}nts de l’univers,
${}^{3}gens illustres, g\underline{e}ns obscurs,
        \\riches et pa\underline{u}vres, tous ensemble.
         
${}^{4}Ma bouche dira des par\underline{o}les de sagesse,
        \\les propos clairvoy\underline{a}nts de mon cœur ;
${}^{5}l’oreille attent\underline{i}ve aux proverbes,
        \\j’exposerai sur la cith\underline{a}re mon énigme.
         
        *
         
${}^{6}Pourquoi craindre aux jo\underline{u}rs de malheur
        \\ces fourbes qui me tal\underline{o}nnent pour m’encercler,
${}^{7}ceux qui s’appu\underline{i}ent sur leur fortune
        \\et se vantent de leurs gr\underline{a}ndes richesses ?
         
${}^{8}Nul ne peut rachet\underline{e}r son frère
        \\ni payer à Die\underline{u} sa rançon :
${}^{9}aussi cher qu’il pu\underline{i}sse payer,
        \\toute v\underline{i}e doit finir.
         
${}^{10}Peut-on vivre \underline{i}ndéfiniment
        \\sans jam\underline{a}is voir la fosse ?
${}^{11}Vous voyez les s\underline{a}ges mourir :
        \\comme le fou et l’insensé ils périssent,
        laissant à d’a\underline{u}tres leur fortune.
         
${}^{12}Ils croyaient leur mais\underline{o}n éternelle, +
        \\leur demeure établ\underline{i}e pour les siècles ;
        \\sur des terres ils avaient m\underline{i}s leur nom.
         
${}^{13}L’homme combl\underline{é} ne dure pas :
        il ressemble au bét\underline{a}il qu’on abat.
         
        *
         
${}^{14}Tel est le dest\underline{i}n des insensés
        \\et l’avenir de qui \underline{a}ime les entendre :
${}^{15}troupeau parqu\underline{é} pour les enfers
        \\et que la m\underline{o}rt mène paître.
         
        \\À l’aurore, ils feront pl\underline{a}ce au juste ;
        \\dans la mort, s’effaceront leurs visages :
        pour e\underline{u}x, plus de palais !
${}^{16}Mais Dieu rachètera ma vie aux gr\underline{i}ffes de la mort :
        \\c’est lu\underline{i} qui me prendra.
         
${}^{17}Ne crains pas l’h\underline{o}mme qui s’enrichit,
        \\qui accroît le l\underline{u}xe de sa maison :
${}^{18}aux enfers il n’emp\underline{o}rte rien ;
        \\sa gloire ne descend p\underline{a}s avec lui.
         
${}^{19}De son vivant, il s’est bén\underline{i} lui-même :
        \\« On t’applaudit car tout va bi\underline{e}n pour toi ! »
${}^{20}Mais il rejoint la lign\underline{é}e de ses ancêtres
        \\qui ne verront jamais pl\underline{u}s la lumière.
         
${}^{21}L’homme comblé qui n’est p\underline{a}s clairvoyant
        ressemble au bét\underline{a}il qu’on abat.
      \bchapter{Psaume}
          
            \bchapter{Psaume}
            Offre à Dieu le sacrifice d’action de grâce
${}^{1}Psaume. D’Asaph.
         
        \\Le Dieu des die\underline{u}x, le Seigneur,
        parle et conv\underline{o}que la terre *
        \\du sol\underline{e}il levant
        jusqu’au sol\underline{e}il couchant.
         
${}^{2}De Sion, b\underline{e}lle entre toutes,
        Die\underline{u} resplendit. *
${}^{3}Qu’il vi\underline{e}nne, notre Dieu,
        qu’il r\underline{o}mpe son silence !
         
        \\Devant lui, un fe\underline{u} qui dévore ;
        \\autour de lui, écl\underline{a}te un ouragan.
${}^{4}Il convoque les haute\underline{u}rs des cieux
        \\et la terre au jugem\underline{e}nt de son peuple :
         
${}^{5}« Assemblez devant m\underline{o}i mes fidèles,
        \\eux qui scellent d’un sacrif\underline{i}ce mon alliance. »
${}^{6}Et les cieux procl\underline{a}ment sa justice :
        \\oui, le j\underline{u}ge c’est Dieu !
         
        *
         
${}^{7}« Écoute, mon pe\underline{u}ple, je parle ; +
        \\Israël, je te pr\underline{e}nds à témoin. *
        \\Moi, Die\underline{u}, je suis ton Dieu !
         
${}^{8}« Je ne t’accuse p\underline{a}s pour tes sacrifices ;
        \\tes holocaustes sont toujo\underline{u}rs devant moi.
${}^{9}Je ne prendrai pas un seul taurea\underline{u} de ton domaine,
        \\pas un béli\underline{e}r de tes enclos.
         
${}^{10}« Tout le gibier des for\underline{ê}ts m’appartient
        \\et le bétail des ha\underline{u}ts pâturages.
${}^{11}Je connais tous les oisea\underline{u}x des montagnes ;
        \\les bêtes des ch\underline{a}mps sont à moi.
         
${}^{12}« Si j’ai faim, ir\underline{a}i-je te le dire ?
        \\Le monde et sa rich\underline{e}sse m’appartiennent.
${}^{13}Vais-je manger la ch\underline{a}ir des taureaux
        \\et boire le s\underline{a}ng des béliers ?
         
${}^{14}« Offre à Dieu le sacrif\underline{i}ce d’action de grâce,
        \\accomplis tes vœux env\underline{e}rs le Très-Haut.
${}^{15}Invoque-moi au jo\underline{u}r de détresse :
        \\je te délivrerai, et tu me r\underline{e}ndras gloire. »
         
${}^{16}Mais à l’impie, Die\underline{u} déclare : +
         
        \\« Qu’as-tu à récit\underline{e}r mes lois, *
        \\à garder mon alli\underline{a}nce à la bouche,
${}^{17}toi qui n’aimes p\underline{a}s les reproches
        \\et rejettes loin de t\underline{o}i mes paroles ?
         
${}^{18}« Si tu vois un vole\underline{u}r, tu fraternises,
        \\tu es chez toi parm\underline{i} les adultères ;
${}^{19}tu livres ta bo\underline{u}che au mal,
        \\ta langue tr\underline{a}me des mensonges.
         
${}^{20}« Tu t’assieds, tu diff\underline{a}mes ton frère,
        \\tu flétris le f\underline{i}ls de ta mère.
${}^{21}Voil\underline{à} ce que tu fais ;
        \\garder\underline{a}i-je le silence ?
         
        \\« Penses-tu que je su\underline{i}s comme toi ?
        \\Je mets cela sous tes ye\underline{u}x, et je t’accuse.
${}^{22}Comprenez donc, vo\underline{u}s qui oubliez Dieu :
        \\sinon je frappe, et p\underline{a}s de recours !
         
${}^{23}« Qui offre le sacrif\underline{i}ce d’action de grâce,
        \\celui-l\underline{à} me rend gloire :
        \\sur le chem\underline{i}n qu’il aura pris,
        \\je lui ferai voir le sal\underline{u}t de Dieu. »
      \bchapter{Psaume}
          
            \bchapter{Psaume}
            Renouvelle mon esprit
${}^{1}Du maître de chœur. Psaume. De David.
         
${}^{2}Lorsque le prophète Nathan vint à lui, après qu’il fut allé vers Bethsabée.
         
${}^{3}Pitié pour moi, mon Die\underline{u}, dans ton amour,
        \\selon ta grande miséricorde, eff\underline{a}ce mon péché.
${}^{4}Lave-moi tout enti\underline{e}r de ma faute,
        \\purifie-m\underline{o}i de mon offense.
         
${}^{5}Oui, je conn\underline{a}is mon péché,
        \\ma faute est toujo\underline{u}rs devant moi.
${}^{6}Contre toi, et toi se\underline{u}l, j’ai péché,
        \\ce qui est mal à tes ye\underline{u}x, je l’ai fait.
         
        \\Ainsi, tu peux parler et montr\underline{e}r ta justice,
        \\être juge et montr\underline{e}r ta victoire.
${}^{7}Moi, je suis n\underline{é} dans la faute,
        \\j’étais pécheur dès le s\underline{e}in de ma mère.
         
${}^{8}Mais tu veux au fond de m\underline{o}i la vérité ;
        \\dans le secret, tu m’appr\underline{e}nds la sagesse.
${}^{9}Purifie-moi avec l’hysope, \underline{e}t je serai pur ;
        \\lave-moi et je serai blanc, pl\underline{u}s que la neige.
         
${}^{10}Fais que j’entende les ch\underline{a}nts et la fête :
        \\ils danseront, les \underline{o}s que tu broyais.
${}^{11}Détourne ta f\underline{a}ce de mes fautes,
        \\enlève to\underline{u}s mes péchés.
         
${}^{12}Crée en moi un cœur p\underline{u}r, ô mon Dieu,
        \\renouvelle et raffermis au fond de m\underline{o}i mon esprit.
${}^{13}Ne me chasse p\underline{a}s loin de ta face,
        \\ne me reprends p\underline{a}s ton esprit saint.
         
${}^{14}Rends-moi la j\underline{o}ie d’être sauvé ;
        \\que l’esprit génére\underline{u}x me soutienne.
${}^{15}Aux pécheurs, j’enseigner\underline{a}i tes chemins ;
        \\vers toi, reviendr\underline{o}nt les égarés.
         
${}^{16}Libère-moi du sang versé, Die\underline{u}, mon Dieu sauveur,
        \\et ma langue acclamer\underline{a} ta justice.
${}^{17}Seigneur, o\underline{u}vre mes lèvres,
        \\et ma bouche annoncer\underline{a} ta louange.
         
${}^{18}Si j’offre un sacrif\underline{i}ce, tu n’en veux pas,
        \\tu n’acceptes p\underline{a}s d’holocauste.
${}^{19}Le sacrifice qui plaît à Dieu,
        c’est un espr\underline{i}t brisé ; *
        \\tu ne repousses pas, ô mon Dieu,
        un cœur bris\underline{é} et broyé.
         
${}^{20}Accorde à Si\underline{o}n le bonheur,
        \\relève les m\underline{u}rs de Jérusalem.
${}^{21}Alors tu accepteras de justes sacrifices,
        oblati\underline{o}ns et holocaustes ; *
        \\alors on offrira des taurea\underline{u}x sur ton autel.
      \bchapter{Psaume}
          
            \bchapter{Psaume}
            Dieu est fidèle
${}^{1}Du maître de chœur. Poème. De David.
         
${}^{2}Lorsque Doëg l’Édomite vint dire à Saül : « David est entré dans la maison d’Abimélek. »
         
${}^{3}Pourquoi te glorifi\underline{e}r du mal,
        t\underline{o}i, l’homme fort ? *
        \\Chaque jour, Die\underline{u} est fidèle.
         
${}^{4}De ta langue affil\underline{é}e comme un rasoir,
        tu prép\underline{a}res le crime, *
        \\fo\underline{u}rbe que tu es !
         
${}^{5}Tu aimes le m\underline{a}l plus que le bien,
        et plus que la vérit\underline{é}, le mensonge ; *
${}^{6}tu aimes les par\underline{o}les qui tuent,
        l\underline{a}ngue perverse.
         
${}^{7}Mais Dieu va te ruin\underline{e}r pour toujours,
        t’écraser, t’arrach\underline{e}r de ta demeure, *
        \\t’extirper de la t\underline{e}rre des vivants.
         
${}^{8}Les justes verr\underline{o}nt, ils craindront,
        ils rir\underline{o}nt de toi : +
${}^{9}« Le voilà d\underline{o}nc cet homme
        qui n’a pas mis sa f\underline{o}rce en Dieu ! *
        \\Il comptait sur ses gr\underline{a}ndes richesses,
        il se faisait f\underline{o}rt de son crime ! »
         
${}^{10}Pour moi, comme un b\underline{e}l olivier
        dans la mais\underline{o}n de Dieu, *
        \\je compte sur la fidélit\underline{é} de mon Dieu,
        sans f\underline{i}n, à jamais !
         
${}^{11}Sans fin, je veux te r\underline{e}ndre grâce,
        c\underline{a}r tu as agi. *
        \\J’espère en ton nom dev\underline{a}nt ceux qui t’aiment :
        ou\underline{i}, il est bon !
      \bchapter{Psaume}
          
            \bchapter{Psaume}
            N’ont-ils donc pas compris ?
${}^{1}Du maître de chœur. Sur « mahalath ». Poème. De David.
         
${}^{2}Dans son cœur, le fo\underline{u} déclare :
        « P\underline{a}s de Dieu ! » *
        \\Tout est corromp\underline{u}, abominable,
        pas un h\underline{o}mme de bien !
         
${}^{3}Des cieux, le Seigne\underline{u}r se penche
        v\underline{e}rs les fils d’Adam *
        \\pour voir s’il en est \underline{u}n de sensé,
        \underline{u}n qui cherche Dieu.
         
${}^{4}Tous, ils s\underline{o}nt dévoyés ;
        tous ens\underline{e}mble, pervertis : *
        \\pas un h\underline{o}mme de bien,
        pas m\underline{ê}me un seul !
         
${}^{5}N’ont-ils d\underline{o}nc pas compris,
        ces g\underline{e}ns qui font le mal ? +
        \\Quand ils mangent leur pain,
        ils m\underline{a}ngent mon peuple. *
        \\Dieu, jam\underline{a}is ils ne l’invoquent !
         
${}^{6}Et voilà qu’ils se sont m\underline{i}s à trembler,
        à trembl\underline{e}r sans raison. *
        \\Oui, Dieu a dispersé les \underline{o}s de tes assiégeants ;
        tu peux en rire : Die\underline{u} les rejette.
         
${}^{7}Qui fera ven\underline{i}r de Sion
        la délivr\underline{a}nce d’Israël ? +
        \\Quand le Seigneur ramènera les déport\underline{é}s de son peuple, *
        quelle fête en Jacob, en Isra\underline{ë}l, quelle joie !
      \bchapter{Psaume}
          
            \bchapter{Psaume}
            Mon appui entre tous
${}^{1}Du maître de chœur. Avec instruments à corde. Poème. De David.
         
${}^{2}Lorsque les gens de Ziph vinrent dire à Saül : « David n’est-il pas caché parmi nous ? »
         
${}^{3}Par ton nom, Die\underline{u}, sauve-moi,
        \\par ta puissance rends-m\underline{o}i justice ;
${}^{4}Dieu, ent\underline{e}nds ma prière,
        \\écoute les par\underline{o}les de ma bouche.
         
${}^{5}Des étrangers se sont lev\underline{é}s contre moi, +
        \\des puissants ch\underline{e}rchent ma perte :
        \\ils n’ont pas souc\underline{i} de Dieu.
         
${}^{6}Mais voici que Die\underline{u} vient à mon aide,
        \\le Seigneur est mon appu\underline{i} entre tous.
${}^{7}\[Que le mal retombe sur ce\underline{u}x qui me guettent ;
        \\par ta vérité, Seigne\underline{u}r, détruis-les.\]
         
${}^{8}De grand cœur, je t’offrir\underline{a}i le sacrifice,
        \\je rendrai grâce à ton n\underline{o}m, car il est bon !
${}^{9}Oui, il m’a délivr\underline{é} de toute angoisse :
        \\j’ai vu mes ennem\underline{i}s défaits.
      \bchapter{Psaume}
          
            \bchapter{Psaume}
            J’ai hâte d’avoir un abri
${}^{1}Du maître de chœur. Avec instruments à corde. Poème. De David.
         
${}^{2}Mon Dieu, éco\underline{u}te ma prière,
        n’écarte p\underline{a}s ma demande. *
${}^{3}Exauce-moi, je t’en pr\underline{i}e, réponds-moi ;
        inqui\underline{e}t, je me plains.
         
${}^{4}Je suis troublé par les cr\underline{i}s de l’ennemi
        et les inj\underline{u}res des méchants ; *
        \\ils me ch\underline{a}rgent de crimes,
        pleins de r\underline{a}ge, ils m’accusent.
         
${}^{5}Mon cœur se t\underline{o}rd en moi,
        la peur de la mort t\underline{o}mbe sur moi ; *
${}^{6}crainte et tremblem\underline{e}nt me pénètrent,
        un friss\underline{o}n me saisit.
         
${}^{7}Alors, j’ai dit : « Qui me donnera des \underline{a}iles de colombe ? +
        Je voler\underline{a}is en lieu sûr ; *
${}^{8}loin, très l\underline{o}in, je m’enfuirais
        pour chercher as\underline{i}le au désert. »
         
${}^{9}J’ai hâte d’av\underline{o}ir un abri
        contre ce grand v\underline{e}nt de tempête ! *
${}^{10}Divise-l\underline{e}s, Seigneur,
        mets la confusi\underline{o}n dans leur langage !
         
        \\Car je v\underline{o}is dans la ville
        disc\underline{o}rde et violence : *
${}^{11}de jour et de nu\underline{i}t, elles tournent
        en ha\underline{u}t de ses remparts.
         
        \\Au-dedans, cr\underline{i}mes et malheurs ;
${}^{12}au-ded\underline{a}ns, c’est la ruine : *
        \\fra\underline{u}de et brutalité
        ne qu\underline{i}ttent plus ses rues.
         
${}^{13}Si l’insulte me ven\underline{a}it d’un ennemi,
        je pourr\underline{a}is l’endurer ; *
        \\si mon rival s’élev\underline{a}it contre moi,
        je pourr\underline{a}is me dérober.
         
${}^{14}Mais toi, un h\underline{o}mme de mon rang,
        mon famili\underline{e}r, mon intime ! *
${}^{15}Que notre ent\underline{e}nte était bonne,
        quand nous allions d’un même pas
        dans la mais\underline{o}n de Dieu !
         
${}^{16}\[Que la m\underline{o}rt les surprenne,
        qu’ils descendent viv\underline{a}nts dans l’abîme, *
        \\car le mal hab\underline{i}te leurs demeures,
        il \underline{e}st au milieu d’eux.\]
         
        *
         
${}^{17}Pour moi, je cr\underline{i}e vers Dieu ;
        le Seigne\underline{u}r me sauvera. *
${}^{18}Le soir et le mat\underline{i}n et à midi,
        je me pl\underline{a}ins, je suis inquiet.
         
        \\Et Dieu a entend\underline{u} ma voix,
${}^{19}il m’app\underline{o}rte la paix. *
        \\Il me délivre dans le comb\underline{a}t que je menais ;
        ils étaient une fo\underline{u}le autour de moi.
         
${}^{20}Que Dieu ent\underline{e}nde et qu’il réponde,
        lui qui r\underline{è}gne dès l’origine, *
        \\à ceux-là qui ne ch\underline{a}ngent pas,
        et ne cr\underline{a}ignent pas Dieu.
         
${}^{21}Un traître a porté la m\underline{a}in sur ses amis,
        profan\underline{é} son alliance : +
${}^{22}il montre un vis\underline{a}ge séduisant,
        mais son cœur fait la guerre ; *
        \\sa parole est plus su\underline{a}ve qu’un parfum,
        mais elle \underline{e}st un poignard.
         
${}^{23}Décharge ton fardea\underline{u} sur le Seigneur :
        il prendra s\underline{o}in de toi. *
        \\Jam\underline{a}is il ne permettra
        que le j\underline{u}ste s’écroule.
         
${}^{24}Et toi, Dieu, tu les précipites au f\underline{o}nd de la tombe, +
        ces hommes qui t\underline{u}ent et qui mentent. *
        \\Ils s’en iront dans la f\underline{o}rce de l’âge ;
        moi, je m’appu\underline{i}e sur toi !
      \bchapter{Psaume}
          
            \bchapter{Psaume}
            Plus rien ne me fait peur
${}^{1}Du maître de chœur. Sur l’air de « La colombe muette dans les lointains ». De David. À mi-voix. Lorsque les Philistins s’emparèrent de lui à Gath.
         
${}^{2}Pitié, mon Dieu !
        Des hommes s’ach\underline{a}rnent contre moi ;
        \\tout le jour, ils me comb\underline{a}ttent, ils me harcèlent.
${}^{3}Ils s’acharnent, ils me gu\underline{e}ttent tout le jour ;
        \\mais là-haut, une arm\underline{é}e combat pour moi.
         
${}^{4}Le jo\underline{u}r où j’ai peur,
        \\je prends appu\underline{i} sur toi.
${}^{5}Sur Dieu dont j’ex\underline{a}lte la parole,
        \\sur Die\underline{u}, je prends appui :
        \\plus ri\underline{e}n ne me fait peur !
        \\Que peuvent sur moi des \underline{ê}tres de chair ?
         
${}^{6}Tout le jour, leurs par\underline{o}les me blessent,
        \\ils ne pensent qu’à me f\underline{a}ire du mal ;
${}^{7}à l’affût, ils épient, ils surv\underline{e}illent mes pas ;
        \\comme s’ils voul\underline{a}ient ma mort.
${}^{8}\[Vont-ils échapp\underline{e}r malgré leurs crimes ?
        \\Que ta colère, mon Dieu, ab\underline{a}tte les nations !\]
         
${}^{9}Toi qui comptes mes p\underline{a}s vagabonds,
        \\recueille en tes o\underline{u}tres mes larmes ;
        (cela n’est-il p\underline{a}s dans ton livre ?)
${}^{10}Le jour où j’appellerai, mes ennem\underline{i}s reculeront ;
        \\je le sais, Die\underline{u} est pour moi.
         
${}^{11}Sur Dieu dont j’exalte la parole,
        le Seigneur dont j’ex\underline{a}lte la parole, *
${}^{12}sur Die\underline{u}, je prends appui :
        \\plus ri\underline{e}n ne me fait peur ! *
        \\Que peuvent sur m\underline{o}i des humains ?
         
${}^{13}Mon Dieu, je tiendr\underline{a}i ma promesse,
        \\je t’offrirai des sacrif\underline{i}ces d’action de grâce ;
${}^{14}car tu m’as délivr\underline{é} de la mort
        \\et tu préserves mes pi\underline{e}ds de la chute,
        \\pour que je marche à la f\underline{a}ce de Dieu
        \\dans la lumi\underline{è}re des vivants.
      \bchapter{Psaume}
          
            \bchapter{Psaume}
            Que ta gloire domine la terre
${}^{1}Du maître de chœur. « Ne détruis pas ». De David. À mi-voix. Lorsque, dans la caverne, il fuyait Saül.
         
${}^{2}Pitié, mon Die\underline{u}, pitié pour moi !
        \\En toi je ch\underline{e}rche refuge,
        \\un refuge à l’\underline{o}mbre de tes ailes,
        \\aussi longtemps que d\underline{u}re le malheur.
         
${}^{3}Je crie vers Die\underline{u}, le Très-Haut,
        \\vers Dieu qui fera to\underline{u}t pour moi.
${}^{4}Du ciel, qu’il m’env\underline{o}ie le salut :
        \\(mon adversaire a blasphémé !).
        \\Que Dieu envoie son amo\underline{u}r et sa vérité !
         
${}^{5}Je suis au milie\underline{u} de lions
        \\et gisant parmi des b\underline{ê}tes féroces ;
        \\ils ont pour langue une \underline{a}rme tranchante,
        \\pour dents, des l\underline{a}nces et des flèches.
         
${}^{6}Dieu, lève-t\underline{o}i sur les cieux :
        \\que ta gloire dom\underline{i}ne la terre !
         
${}^{7}Ils ont tendu un fil\underline{e}t sous mes pas :
        j’all\underline{a}is succomber. *
        \\Ils ont creusé un tro\underline{u} devant moi,
        \underline{i}ls y sont tombés.
         
        *
         
${}^{8}Mon cœur est pr\underline{ê}t, mon Dieu, +
        \\mon cœ\underline{u}r est prêt ! *
        \\Je veux chant\underline{e}r, jouer des hymnes !
         
${}^{9}Éveille-t\underline{o}i, ma gloire ! +
        \\Éveillez-vous, h\underline{a}rpe, cithare, *
        \\que j’év\underline{e}ille l’aurore !
         
${}^{10}Je te rendrai grâce parmi les pe\underline{u}ples, Seigneur,
        \\et jouerai mes h\underline{y}mnes en tous pays.
${}^{11}Ton amour est plus gr\underline{a}nd que les cieux,
        \\ta vérité, plus ha\underline{u}te que les nues.
         
${}^{12}Dieu, lève-t\underline{o}i sur les cieux :
        \\que ta gloire dom\underline{i}ne la terre !
