  
  
    
    \bbook{JUGES}{JUGES}
      
         
      \bchapter{}
      \begin{verse}
${}^{1}Après la mort de Josué, les fils d’Israël consultèrent le Seigneur : « Qui de nous montera le premier attaquer les Cananéens ? » 
${}^{2}Le Seigneur répondit : « C’est Juda qui montera. Je livre le pays entre ses mains. » 
${}^{3}Juda dit alors à son frère Siméon : « Viens avec moi dans le territoire que j’ai reçu en partage, et nous combattrons les Cananéens. Puis, à mon tour, j’irai avec toi dans le territoire que tu as reçu en partage. » Siméon alla avec lui. 
${}^{4}Juda monta donc, et le Seigneur livra entre leurs mains les Cananéens et les Perizzites. À Bézeq, ils les battirent : en tout, dix mille hommes. 
${}^{5}Ayant rencontré Adoni-Bézeq à Bézeq, ils l’attaquèrent et battirent les Cananéens et les Perizzites. 
${}^{6}Adoni-Bézeq s’enfuit, mais ils le poursuivirent, le saisirent et lui coupèrent les pouces des mains et des pieds. 
${}^{7}Adoni-Bézeq dit : « Soixante-dix rois dont on avait coupé les pouces des mains et des pieds ramassaient des restes sous ma table. Ce que j’ai fait, Dieu me le rend. » On l’emmena à Jérusalem. Il y mourut.
${}^{8}Les fils de Juda attaquèrent Jérusalem, s’en emparèrent, la passèrent au fil de l’épée et mirent le feu à la ville. 
${}^{9}Ils descendirent ensuite pour combattre les Cananéens qui habitaient la Montagne, le Néguev et le Bas-Pays.
${}^{10}Puis Juda marcha contre les Cananéens qui habitaient Hébron – le nom d’Hébron était auparavant Qiryath-Arba – et il battit Shéshaï, Ahimane et Talmaï. 
${}^{11}De là, il marcha contre les habitants de Debir – le nom de Debir était auparavant Qiryath-Séfer. 
${}^{12}Caleb dit : « Celui qui vaincra Qiryath-Séfer et s’en emparera, je lui donnerai pour femme ma fille Aksa. » 
${}^{13}Otniel, fils de Qénaz, le frère cadet de Caleb, s’empara de la ville et Caleb lui donna pour femme sa fille Aksa. 
${}^{14}Dès qu’elle arriva, Otniel l’incita à demander à son père un champ. Elle descendit de son âne, et Caleb lui demanda : « Que veux-tu ? »
${}^{15}Elle lui dit : « Accorde-moi une faveur. Puisque tu m’as établie au pays du Néguev, donne-moi aussi des sources. » Caleb lui donna les sources d’en haut et les sources d’en bas.
${}^{16}Les fils d’un Qénite, parent de Moïse, montèrent de la ville des Palmiers avec les fils de Juda jusqu’au désert de Juda qui est au sud d’Arad. Ils vinrent habiter avec le peuple qui était là. 
${}^{17}Puis Juda s’en alla avec son frère Siméon. Ils battirent les Cananéens qui habitaient Sefath et vouèrent la ville à l’anathème. On lui donna le nom de Horma. 
${}^{18}Juda s’empara de Gaza et de son territoire, d’Ascalon et de son territoire, d’Éqrone et de son territoire. 
${}^{19}Le Seigneur fut avec Juda, et Juda s’empara de la Montagne, mais il ne put déposséder les habitants de la plaine, car ils avaient des chars de fer.
${}^{20}Comme l’avait prescrit Moïse, on donna Hébron à Caleb, qui en déposséda les trois fils d’Anaq. 
${}^{21}Les fils de Benjamin ne dépossédèrent pas les Jébuséens qui habitaient Jérusalem. Ceux-ci ont habité Jérusalem avec les fils de Benjamin jusqu’à ce jour.
${}^{22}La maison de Joseph, de son côté, monta à Béthel, et le Seigneur était avec eux. 
${}^{23}Ils envoyèrent d’abord reconnaître Béthel – le nom de la ville était auparavant Louz. 
${}^{24}Ceux qui étaient en reconnaissance virent un homme sortir de la ville et lui dirent : « Montre-nous donc l’accès de la ville et nous ferons preuve de loyauté envers toi. » 
${}^{25}Il leur montra l’accès de la ville. On la passa au fil de l’épée. Quant à l’informateur, ils le laissèrent aller avec tout son clan. 
${}^{26}Il s’en alla au pays des Hittites, construisit une ville et la nomma Louz. C’est encore son nom aujourd’hui.
${}^{27}Manassé ne conquit ni Beth-Shéane et ses dépendances, ni Taanak et ses dépendances, ni les habitants de Dor et ses dépendances, ni les habitants de Yibléam et ses dépendances, ni les habitants de Meguiddo et ses dépendances. Les Cananéens persistèrent à habiter ce pays. 
${}^{28}Lorsqu’Israël se fut affermi, il les soumit à la corvée, sans toutefois réussir à les déposséder. 
${}^{29}Éphraïm ne déposséda pas les Cananéens qui habitaient à Guézer, et les Cananéens habitèrent au milieu d’Éphraïm, à Guézer. 
${}^{30}Zabulon ne déposséda pas les habitants de Qitrone, ni les habitants de Nahalol ; les Cananéens habitèrent au milieu de Zabulon et furent soumis à la corvée. 
${}^{31}Asher ne déposséda pas les habitants d’Akko, ni les habitants de Sidon, ni ceux d’Ahlab, d’Akzib, de Helba, d’Afiq et de Rehob. 
${}^{32}Les Ashérites habitèrent au milieu des Cananéens habitant le pays, puisqu’ils ne les avaient pas dépossédés. 
${}^{33}Nephtali ne déposséda pas les habitants de Beth-Shèmesh, ni les habitants de Beth-Anath, et il habita au milieu des Cananéens habitant le pays. Mais les habitants de Beth-Shèmesh et de Beth-Anath furent soumis à la corvée. 
${}^{34}Les Amorites repoussèrent jusqu’à la montagne les fils de Dane, ne les laissant pas descendre dans la plaine. 
${}^{35}Les Amorites persistèrent donc à habiter Har-Hèrès, Ayyalone et Shaalbim. Mais quand la main de la maison de Joseph pesa lourdement, ils furent soumis à la corvée. 
${}^{36}Le territoire des Amorites commence à la montée des Scorpions, depuis La Roche, et va ensuite en montant.
      
         
      \bchapter{}
      \begin{verse}
${}^{1}L’ange du Seigneur monta de Guilgal à Bokim et dit : « Je vous ai fait monter d’Égypte et vous ai fait entrer dans le pays que j’avais juré de donner à vos pères. J’avais dit : “Jamais je ne romprai mon alliance avec vous, 
${}^{2}et vous, vous ne conclurez pas d’alliance avec les habitants de ce pays. Vous renverserez leurs autels.” Mais vous n’avez pas écouté ma voix ! Qu’avez-vous fait là ! 
${}^{3}Eh bien, je l’affirme : Je ne chasserai pas ces peuples devant vous, ils seront sur vos flancs, et leurs dieux deviendront pour vous un piège. » 
${}^{4}Or, dès que l’ange du Seigneur eut adressé ces paroles à tous les fils d’Israël, le peuple se mit à crier et pleura. 
${}^{5}Ils appelèrent cet endroit « Bokim » (c’est-à-dire : Pleureurs), et ils y sacrifièrent au Seigneur.
      
         
${}^{6}Josué renvoya le peuple, et les fils d’Israël se rendirent chacun dans son héritage pour prendre possession du pays. 
${}^{7}Le peuple servit le Seigneur pendant toute la vie de Josué et toute la vie des anciens qui vécurent encore après Josué et avaient vu toute la grande œuvre du Seigneur pour Israël. 
${}^{8}Josué, fils de Noun, serviteur du Seigneur, mourut à l’âge de cent dix ans. 
${}^{9}On l’ensevelit dans le territoire qu’il avait reçu en héritage à Timnath-Hèrès, dans la montagne d’Éphraïm, au nord du mont Gaash. 
${}^{10}Quand toute cette génération fut à son tour réunie à ses pères, une autre génération lui succéda, qui ne connaissait pas le Seigneur, ni l’œuvre qu’il avait faite pour Israël.
${}^{11}Les fils d’Israël firent ce qui est mal aux yeux du Seigneur, et ils servirent les Baals. 
${}^{12} Ils abandonnèrent le Seigneur, le Dieu de leurs pères, qui les avait fait sortir du pays d’Égypte, et ils suivirent d’autres dieux parmi ceux des peuples d’alentour. Ils se prosternèrent devant eux, et ils irritèrent le Seigneur. 
${}^{13} Ils abandonnèrent le Seigneur pour servir Baal et Astarté.
${}^{14}Alors la colère du Seigneur s’enflamma contre Israël. Il les livra aux mains des pillards\\, les abandonna\\aux ennemis qui les entouraient, et ils furent incapables de leur résister. 
${}^{15} Dans toutes leurs expéditions, la main du Seigneur était contre eux, pour leur malheur, comme il le leur avait dit, comme il en avait fait serment. Ils furent dans une très grande détresse.
${}^{16}Alors le Seigneur suscita\\des juges pour les sauver de la main des pillards. 
${}^{17} Mais ils n’obéissaient pas non plus à leurs juges. Ils se prostituèrent en suivant d’autres dieux, ils se prosternèrent devant eux. Ils ne tardèrent pas à se détourner du chemin où leurs pères avaient marché en obéissant aux commandements du Seigneur ; ils n’agirent pas comme eux.
${}^{18}Lorsque le Seigneur suscitait pour eux un juge, le Seigneur était avec le juge, et il les sauvait de la main de leurs ennemis aussi longtemps que le juge était en vie ; car le Seigneur se laissait émouvoir quand ils gémissaient sous la violence de leurs oppresseurs. 
${}^{19} Mais quand le juge était mort, ils recommençaient et poussaient la corruption plus loin que leurs pères : ils suivaient d’autres dieux, les servaient et se prosternaient devant eux ; ils ne renonçaient en rien à leurs pratiques ni à leur conduite obstinée.
${}^{20}La colère du Seigneur s’enflamma contre Israël. Il dit : « Puisque cette nation a transgressé mon alliance, celle que j’avais prescrite à ses pères, et qu’elle n’a pas écouté ma voix, 
${}^{21}eh bien moi, je ne déposséderai plus devant Israël aucune des nations que Josué a laissé subsister avant sa mort. » 
${}^{22}Ainsi, le Seigneur voulait mettre à l’épreuve les fils d’Israël, pour voir si, oui ou non, ils marcheraient dans ses chemins, comme l’avaient fait leurs pères. 
${}^{23}Il avait donc laissé subsister ces nations sans leur enlever trop vite leur territoire, et il ne les avait pas livrées à la main de Josué.
      
         
      \bchapter{}
      \begin{verse}
${}^{1}Voici les nations que le Seigneur laissa subsister afin de mettre par elles à l’épreuve les fils d’Israël, tous ceux qui n’avaient connu aucune des guerres de Canaan. 
${}^{2}Elles servirent à instruire les générations des fils d’Israël : ils apprirent l’art de la guerre, ceux du moins qui ne le connaissaient pas auparavant. 
${}^{3}Voici ces nations : cinq princes des Philistins et tous les Cananéens, les Sidoniens et les Hivvites qui habitaient la montagne du Liban depuis la montagne de Baal-Hermon jusqu’à l’Entrée-de-Hamath. 
${}^{4}Ces nations servirent donc à mettre à l’épreuve les fils d’Israël pour savoir s’ils écouteraient les commandements que le Seigneur avait donnés à leurs pères par l’intermédiaire de Moïse. 
${}^{5}Les fils d’Israël habitèrent au milieu des Cananéens, des Hittites, des Amorites, des Perizzites, des Hivvites, des Jébuséens ; 
${}^{6}ils prirent leurs filles pour femmes et ils donnèrent leurs filles à leurs fils ; ils servirent leurs dieux.
      
         
      <h2 class="intertitle hmbot" id="d85e56255">1. Otniel (3,7-11)</h2>
${}^{7}Les fils d’Israël firent ce qui est mal aux yeux du Seigneur. Ils oublièrent le Seigneur, leur Dieu, et ils servirent les Baals et les Ashéras. 
${}^{8}La colère du Seigneur s’enflamma contre Israël et il les abandonna aux mains de Koushane-Rishataïm, roi d’Aram-des-deux-fleuves ; les fils d’Israël servirent Koushane-Rishataïm pendant huit ans. 
${}^{9}Ils crièrent vers le Seigneur, et le Seigneur suscita pour eux un sauveur : Otniel, fils de Qenaz, frère cadet de Caleb, et il les sauva. 
${}^{10}L’Esprit du Seigneur fut sur lui et il jugea Israël. Il partit en guerre et le Seigneur lui livra Koushane-Rishataïm, roi d’Aram, et sa main fut puissante contre Koushane-Rishataïm. 
${}^{11}Le pays fut en repos pendant quarante ans. Puis Otniel, fils de Qenaz, mourut.
      <h2 class="intertitle hmbot" id="d85e56319">2. Éhoud (3,12-30)</h2>
${}^{12}Les fils d’Israël recommencèrent à faire ce qui est mal aux yeux du Seigneur, et le Seigneur donna force à Églone, roi de Moab, contre Israël, puisqu’ils faisaient ce qui est mal aux yeux du Seigneur. 
${}^{13}Églone s’adjoignit les fils d’Ammone et Amalec, puis il alla battre Israël ; ils prirent possession de la ville des Palmiers. 
${}^{14}Les fils d’Israël servirent Églone, roi de Moab, pendant dix-huit ans. 
${}^{15}Ils crièrent vers le Seigneur, et le Seigneur suscita pour eux un sauveur : Éhoud, fils de Guéra, Benjaminite, qui était gaucher. Par son intermédiaire, les fils d’Israël envoyèrent un tribut à Églone, roi de Moab.
${}^{16}Éhoud se fit un long poignard à deux tranchants, et il l’attacha sous son vêtement contre sa cuisse droite. 
${}^{17}Il offrit donc le tribut à Églone, roi de Moab, qui était un homme très gros. 
${}^{18}Dès qu’il eut fini de présenter le tribut, Éhoud raccompagna les gens qui avaient apporté ce tribut. 
${}^{19}Pour lui, arrivé aux idoles qui sont près de Guilgal, il rebroussa chemin et dit : « J’ai à te transmettre une parole confidentielle, ô roi ! » Celui-ci dit : « Silence ! » Et tous ceux qui se tenaient debout auprès de lui se retirèrent. 
${}^{20}Éhoud alla vers Églone alors qu’il se reposait dans la fraîcheur de la chambre haute qui lui était réservée. Éhoud dit : « J’ai à te transmettre une parole de Dieu. » Le roi se leva de son siège. 
${}^{21}Éhoud étendit la main gauche, prit le poignard sur sa cuisse droite et l’enfonça dans le ventre du roi. 
${}^{22}Même la poignée entra après la lame, et la graisse se referma sur la lame, car Éhoud n’avait pas retiré le poignard du ventre du roi. 
${}^{23}Alors Éhoud sortit par l’escalier extérieur, après avoir fermé derrière lui les portes de la chambre haute et mis le verrou. 
${}^{24}Quand il fut sorti, les serviteurs vinrent et constatèrent que les portes de la chambre haute étaient verrouillées, et ils dirent : « Sans doute se couvre-t-il les pieds dans la chambre bien fraîche. » 
${}^{25}Ils attendirent indéfiniment : Églone n’ouvrait toujours pas les portes de la chambre haute. Alors ils prirent la clé et ils ouvrirent : leur maître gisait à terre, mort. 
${}^{26}Quant à Éhoud, il s’était échappé pendant que les serviteurs s’attardaient ; il avait dépassé les idoles et s’échappait vers la Seïra.
${}^{27}Dès qu’il arriva, il sonna du cor dans la montagne d’Éphraïm. Les fils d’Israël descendirent de la montagne, avec Éhoud à leur tête. 
${}^{28}Il leur dit : « Suivez-moi, car le Seigneur a livré votre ennemi Moab entre vos mains. » Ils descendirent derrière lui, ils prirent à Moab le passage des gués du Jourdain et ne laissèrent plus personne traverser. 
${}^{29}En ce temps-là, ils battirent Moab, environ dix mille hommes, tous robustes et vaillants, et personne ne s’échappa. 
${}^{30}En ce jour-là, Moab fut abaissé sous la main d’Israël et le pays fut en repos pendant quatre-vingts ans.
      <h2 class="intertitle hmbot" id="d85e56427">3. Shamgar (3,31)</h2>
${}^{31}Après Éhoud, il y eut Shamgar, fils d’Anath. Brandissant un aiguillon à bœufs, il battit les Philistins au nombre de six cents hommes. Lui aussi sauva Israël.
      <h2 class="intertitle" id="d85e56452">4. Débora et Baraq (4 – 5)</h2>
      
         
      \bchapter{}
      \begin{verse}
${}^{1}Après la mort d’Éhoud, les fils d’Israël recommencèrent à faire ce qui est mal aux yeux du Seigneur. 
${}^{2}Le Seigneur les vendit à Yabine, roi de Canaan, qui régnait à Haçor. Le chef de son armée était Sissera ; celui-ci habitait Harosheth-ha-Goïm. 
${}^{3}Les fils d’Israël crièrent vers le Seigneur, car Yabine avait neuf cents chars de fer et il avait opprimé durement les fils d’Israël pendant vingt ans.
${}^{4}Or, Débora, une prophétesse, femme de Lappidoth, jugeait Israël en ce temps-là. 
${}^{5}Elle siégeait sous le Palmier de Débora, entre Rama et Béthel, dans la montagne d’Éphraïm, et les fils d’Israël venaient vers elle pour faire arbitrer leurs litiges.
${}^{6}Elle fit appeler Baraq, fils d’Abinoam, de Qèdesh en Nephtali, et elle lui dit : « Le Seigneur, Dieu d’Israël, n’a-t-il pas donné cet ordre ? “Va, fais venir au mont Tabor et prends avec toi dix mille hommes parmi les fils de Nephtali et les fils de Zabulon. 
${}^{7}Je ferai venir vers toi, au torrent de Qishone, Sissera, le chef de l’armée de Yabine, avec ses chars et ses troupes, et je le livrerai entre tes mains.” » 
${}^{8}Baraq lui dit : « Si tu marches avec moi, j’irai ; mais si tu ne marches pas avec moi, je n’irai pas. » 
${}^{9}Elle dit : « Je marcherai donc avec toi. Mais, sur la voie où tu marches, l’honneur ne sera pas pour toi : car c’est à une femme que le Seigneur abandonnera Sissera. » Débora se leva et se rendit avec Baraq à Qèdesh. 
${}^{10}Baraq convoqua Zabulon et Nephtali à Qèdesh. Dix mille hommes le suivirent, et Débora partit avec lui.
${}^{11}Hèber le Qénite s’était séparé de Qaïn et des fils de Hobab, parent de Moïse. Il avait dressé sa tente non loin du chêne de Saanaïm, près de Qèdesh.
${}^{12}On annonça à Sissera que Baraq, fils d’Abinoam, était arrivé au mont Tabor. 
${}^{13}Alors, Sissera convoqua tous ses chars, neuf cents chars de fer, ainsi que tout le peuple qui était avec lui, depuis Harosheth-ha-Goïm jusqu’au torrent de Qishone. 
${}^{14}Débora dit à Baraq : « Lève-toi ! Car c’est aujourd’hui que le Seigneur livre Sissera entre tes mains ! Le Seigneur n’est-il pas sorti devant toi ? » Baraq descendit du mont Tabor avec dix mille hommes derrière lui. 
${}^{15}Alors, le Seigneur frappa de panique Sissera, tous les chars et toute l’armée, qui fut passée au fil de l’épée devant Baraq. Sissera descendit de son char et s’enfuit à pied. 
${}^{16}Baraq poursuivit les chars et l’armée jusqu’à Harosheth-ha-Goïm, et toute l’armée de Sissera tomba au fil de l’épée ; il n’en resta pas un seul.
${}^{17}Or Sissera s’était enfui à pied vers la tente de Yaël, femme de Hèber le Qénite, car la paix régnait entre Yabine, roi de Haçor, et la maison de Hèber le Qénite. 
${}^{18}Yaël sortit au-devant de Sissera et lui dit : « Arrête-toi, mon seigneur, arrête-toi chez moi ; ne crains rien. » Il s’arrêta chez elle, dans sa tente, et elle le recouvrit d’une couverture. 
${}^{19}Il lui dit : « Peux-tu me donner à boire un peu d’eau, car j’ai soif. » Elle ouvrit l’outre de lait, le fit boire et le recouvrit. 
${}^{20}Il lui dit : « Tiens-toi à l’entrée de la tente, et si quelqu’un vient, t’interroge et demande : “Y a-t-il quelqu’un ici ?”, tu répondras : “Non.” » 
${}^{21}Mais Yaël, femme de Hèber, prit un piquet de la tente, saisit un marteau dans sa main, vint près de lui doucement, et lui enfonça dans la tempe le piquet, qui alla se planter dans la terre. Sissera qui, épuisé, était profondément endormi, mourut. 
${}^{22}Or, voici que Baraq poursuivait Sissera ! Yaël sortit à sa rencontre et lui dit : « Viens, et je te ferai voir l’homme que tu cherches. » Il entra chez elle, et voilà que Sissera gisait, mort, le piquet dans la tempe !
${}^{23}En ce jour-là, Dieu abaissa Yabine, roi de Canaan, devant les fils d’Israël. 
${}^{24}La main des fils d’Israël se fit de plus en plus lourde contre Yabine, roi de Canaan, jusqu’à ce qu’ils aient abattu Yabine, roi de Canaan.
      
         
      \bchapter{}
      \begin{verse}
${}^{1}Ce jour-là, Débora et Baraq, fils d’Abinoam, dirent et chantèrent :
      
         
       
${}^{2}« Alors qu’en Israël, on laisse flotter les chevelures,
        \\alors qu’un peuple s’offre librement,
        \\bénissez le Seigneur !
         
${}^{3}Rois, écoutez ! Prêtez l’oreille, souverains !
        \\C’est moi, c’est moi qui vais chanter pour le Seigneur,
        \\moi qui vais jouer pour le Seigneur, Dieu d’Israël !
         
${}^{4}Seigneur, quand tu sortis de Séïr,
        \\quand tu partis de la campagne d’Édom,
        \\la terre trembla, les cieux mêmes fondirent,
        \\et les nuées fondirent en eaux,
${}^{5}les montagnes furent ébranlées
        \\devant la face du Seigneur, celui du Sinaï,
        \\devant la face du Seigneur, Dieu d’Israël.
         
${}^{6}Aux jours de Shamgar, fils d’Anath,
        \\aux jours de Yaël, ne passaient plus les caravanes ;
        \\ceux qui marchaient par les sentiers
        \\prenaient des voies tortueuses.
         
${}^{7}Les guides manquaient,
        \\ils manquaient en Israël,
        \\jusqu’à ce que je me lève, moi, Débora,
        \\jusqu’à ce que je me lève, mère en Israël !
         
${}^{8}On adoptait des dieux nouveaux,
        \\alors, la guerre était aux portes.
        \\À peine voyait-on une lance, un bouclier,
        \\pour quarante mille hommes en Israël.
         
${}^{9}Le cœur va aux chefs d’Israël,
        \\à ceux du peuple qui s’offrent librement.
        \\Bénissez le Seigneur !
         
${}^{10}Vous qui montez des ânesses blanches,
        \\vous qui siégez sur des tapis,
        \\et vous qui marchez sur la route,
        \\parlez !
         
${}^{11}Dans les propos échangés auprès des abreuvoirs,
        \\là, on raconte les justes actions du Seigneur,
        \\la justice de sa force en Israël.
        \\Alors, le peuple du Seigneur est descendu aux portes.
         
${}^{12}Éveille-toi, éveille-toi, Débora !
        \\Éveille-toi, éveille-toi, lance ton chant !
        \\Lève-toi, Baraq, emmène tes captifs, ô fils d’Abinoam !
         
${}^{13}Que le reste du peuple l’emporte sur les puissants,
        \\que pour moi le Seigneur l’emporte sur les héros !
         
${}^{14}Ceux qui viennent d’Éphraïm sont en Amalec ;
        \\derrière toi, Benjamin est avec tes troupes ;
        \\de Makir sont descendus des chefs,
        \\et de Zabulon ceux qui portent le bâton de commandement.
         
${}^{15}Les princes en Issakar sont avec Débora,
        \\Issakar est fidèle à Baraq :
        \\dans la vallée, il s’est élancé sur ses pas.
         
        \\Dans les clans de Roubène, grandes intentions du cœur !
${}^{16}Pourquoi es-tu resté assis entre deux parcs,
        \\à écouter le son des flûtes auprès des troupeaux ?
        \\Dans les clans de Roubène, grandes hésitations du cœur !
         
${}^{17}Galaad est resté au-delà du Jourdain.
        \\Et Dane, pourquoi demeure-t-il sur des vaisseaux ?
        \\Asher est resté assis au bord des mers,
        \\il est resté dans ses ports.
         
${}^{18}Zabulon, le peuple qui méprise sa vie à en mourir,
        \\de même Nephtali, sur les hauteurs du pays !
         
${}^{19}Survinrent les rois, ils ont combattu,
        \\les rois de Canaan ont combattu,
        \\à Taanak, près des eaux de Meguiddo.
        \\Mais d’argent, ils n’en ont pas gagné.
         
${}^{20}Du haut des cieux, les étoiles ont combattu ;
        \\depuis leurs sentiers, elles ont combattu Sissera.
${}^{21}Le torrent de Qishone les a balayés,
        \\le torrent d’autrefois, le torrent de Qishone.
        \\Avance hardiment, ô mon âme !
         
${}^{22}Alors les sabots des chevaux ont martelé le sol.
        \\Ils galopent, ses coursiers, ils galopent !
         
${}^{23}Maudissez Méroz, dit l’ange du Seigneur !
        \\Maudissez, maudissez ses habitants :
        \\ils ne sont pas venus au secours du Seigneur,
        \\au secours du Seigneur, contre les héros.
         
${}^{24}Bénie soit parmi les femmes Yaël,
        \\la femme de Hèber, le Qénite ;
        \\parmi les femmes qui vivent sous la tente,
        \\bénie soit-elle !
         
${}^{25}Il demandait de l’eau, elle donna du lait ;
        \\dans la coupe d’honneur, elle offrit de la crème ;
${}^{26}elle étendit sa main vers un piquet,
        \\et sa droite vers un marteau de travailleurs.
        \\Elle martela Sissera et lui broya la tête,
        \\elle frappa et lui perça la tempe.
${}^{27}À ses pieds, il s’écroule, il tombe, il gît ;
        \\à ses pieds, il s’écroule, il tombe.
        \\Là, il s’écroule, il tombe, anéanti !
         
${}^{28}Par la fenêtre, elle jette un regard, la mère de Sissera,
        \\elle se lamente, à travers la claire-voie :
        \\“Pourquoi son char tarde-t-il à venir ?
        \\Pourquoi la marche de ses chars est-elle si lente ?”
         
${}^{29}Les plus sages de ses dames lui répondent,
        \\et elle se redit à elle-même :
${}^{30}“Sans doute se partagent-ils le butin qu’ils ont trouvé ?
        \\Une captive, deux captives par guerrier,
        \\des étoffes de couleur comme butin pour Sissera,
        \\comme butin, des étoffes de couleur brodées,
        \\pour son cou, une étoffe de couleur rebrodée !”
         
${}^{31}Que périssent ainsi tous tes ennemis, Seigneur,
        \\mais que tes amis soient comme le soleil
        \\quand il s’élance dans sa force ! »
       
      Et le pays fut en repos pendant quarante ans.
      <h2 class="intertitle" id="d85e57147">5. Gédéon et Abimélek (6 – 9)</h2>
      
         
      \bchapter{}
      \begin{verse}
${}^{1}Les fils d’Israël firent ce qui est mal aux yeux du Seigneur, et le Seigneur les abandonna à Madiane pendant sept ans. 
${}^{2}Madiane imposa sa puissance à Israël. À cause de Madiane, les fils d’Israël aménagèrent dans les montagnes des failles, des grottes et des lieux escarpés. 
${}^{3}Chaque fois qu’Israël avait fait les semailles, Madiane montait avec Amalec et les fils de l’Orient ; ils attaquaient Israël ; 
${}^{4}ils campaient auprès d’eux et dévastaient les produits du pays jusqu’aux abords de Gaza. Ils ne laissaient à Israël ni vivres, ni moutons, ni bœufs, ni ânes ; 
${}^{5}ils arrivaient avec leurs troupeaux et leurs tentes, comme une multitude de sauterelles. Eux et leurs chameaux étaient innombrables, et ils envahissaient le pays pour le ravager. 
${}^{6}À cause de Madiane, Israël fut réduit à une grande misère, et les fils d’Israël crièrent vers le Seigneur.
${}^{7}Comme ils criaient vers le Seigneur au sujet de Madiane, 
${}^{8}le Seigneur leur envoya un prophète, qui leur dit : « Ainsi parle le Seigneur, Dieu d’Israël : C’est moi qui vous ai fait monter d’Égypte et vous ai fait sortir de la maison d’esclavage. 
${}^{9}Je vous ai délivrés de la main des Égyptiens et de tous ceux qui vous opprimaient ; je les ai chassés devant vous et je vous ai donné leur pays. 
${}^{10}Je vous ai dit : “Je suis le Seigneur, votre Dieu. Vous ne craindrez pas les dieux des Amorites dont vous habitez le pays.” Mais vous n’avez pas écouté ma voix ! »
${}^{11}L’ange du Seigneur vint s’asseoir sous le térébinthe d’Ofra, qui appartenait à Joas, de la famille d’Abièzer. Gédéon, son fils, battait le blé dans le pressoir, pour le soustraire au pillage des Madianites. 
${}^{12} L’ange du Seigneur lui apparut et lui dit : « Le Seigneur est avec toi, vaillant guerrier ! » 
${}^{13} Gédéon lui répondit : « Pardon, mon Seigneur ! Si le Seigneur est avec nous, pourquoi tout ceci nous est-il arrivé ? Que sont devenus tous ces prodiges que nous ont racontés nos pères ? Ils nous disaient : “Est-ce que le Seigneur ne nous a pas fait monter d’Égypte ?” Mais aujourd’hui le Seigneur nous a abandonnés, en nous livrant au pouvoir de Madiane… »
${}^{14}Alors le Seigneur regarda Gédéon et lui dit : « Avec la force qui est en toi, va sauver Israël du pouvoir de Madiane. N’est-ce pas moi qui t’envoie ? » 
${}^{15} Gédéon reprit : « Pardon, mon Seigneur ! Comment sauverais-je Israël ? Mon clan est le plus faible dans la tribu de Manassé, et moi je suis le plus petit dans la maison de mon père ! » 
${}^{16} Le Seigneur lui répondit : « Je serai avec toi\\, et tu battras les Madianites comme s’ils n’étaient qu’un seul homme. » 
${}^{17} Gédéon lui dit : « Si j’ai trouvé grâce à tes yeux, donne-moi un signe que c’est bien toi qui me parles. 
${}^{18} Ne t’éloigne pas d’ici avant que je revienne vers toi. Je vais chercher mon offrande et je la placerai devant toi. » Le Seigneur répondit : « Je resterai jusqu’à ton retour. »
${}^{19}Gédéon s’en alla, il prépara un chevreau, et avec une mesure de farine il fit des pains sans levain. Il mit la viande dans une corbeille, et le jus dans un pot, puis il\\apporta tout cela sous le térébinthe et le lui présenta. 
${}^{20}L’ange de Dieu lui dit : « Prends la viande et les pains sans levain, pose-les sur ce rocher et répands le jus. » Gédéon obéit. 
${}^{21}Alors l’ange du Seigneur étendit le bâton qu’il tenait à la main, et il toucha la viande et les pains sans levain. Le feu jaillit de la roche, consuma la viande et les pains sans levain, et l’ange du Seigneur disparut. 
${}^{22}Alors Gédéon comprit que c’était l’ange du Seigneur, et il dit : « Malheur à moi\\, Seigneur mon Dieu ! Pourquoi donc ai-je vu l’ange du Seigneur face à face ? » 
${}^{23}Le Seigneur lui répondit : « Que la paix soit avec toi ! Sois sans crainte ; tu ne mourras pas. » 
${}^{24}À cet endroit, Gédéon bâtit un autel au Seigneur sous le vocable de Seigneur-de-la-paix. Jusqu’à ce jour, cet autel est encore à Ofra d’Abièzer.
${}^{25}Cette nuit-là, le Seigneur dit à Gédéon : « Prends le taureau de ton père et un deuxième taureau âgé de sept ans. Puis tu démoliras l’autel de Baal qui est à ton père, et tu couperas le Poteau sacré qui se trouve à côté de lui. 
${}^{26}Tu bâtiras au Seigneur ton Dieu un autel selon les règles, au sommet de cette citadelle. Tu prendras alors le deuxième taureau et tu l’offriras en holocauste sur le bois du Poteau sacré que tu auras coupé. » 
${}^{27}Gédéon prit alors avec lui dix de ses serviteurs et fit ce que lui avait ordonné le Seigneur. Mais, comme il craignait sa famille et les gens de la ville, plutôt que de le faire de jour, il préféra agir de nuit. 
${}^{28}Le lendemain, les gens de la ville, levés de bon matin, virent que l’autel de Baal était renversé, que le Poteau sacré qui se trouvait à côté de lui avait été coupé, et qu’un taureau avait été offert en holocauste sur l’autel qui venait d’être bâti. 
${}^{29}Ils se dirent alors les uns aux autres : « Qui a fait cela ? » Ils cherchèrent et s’informèrent, et ils dirent : « C’est Gédéon, fils de Joas, qui a fait cela. » 
${}^{30}Les gens de la ville dirent à Joas : « Fais sortir ton fils ! Il doit mourir, car il a détruit l’autel de Baal et coupé le Poteau sacré qui se trouvait à côté de lui. » 
${}^{31}Mais Joas répondit à ceux qui se tenaient près de lui : « Est-ce à vous de défendre Baal ? Est-ce à vous de le sauver ? Celui qui défendra Baal sera mis à mort avant le matin. Si Baal est dieu, qu’il se défende lui-même, puisqu’on a renversé son autel ! » 
${}^{32}Ce jour-là, on donna à Gédéon le nom de Yeroubbaal, c’est-à-dire : « Que Baal s’en prenne à lui », puisqu’il a renversé son autel.
${}^{33}Tout Madiane, Amalec et les fils de l’Orient se coalisèrent et, ayant passé le Jourdain, vinrent camper dans la plaine de Yizréel. 
${}^{34}l’Esprit du Seigneur revêtit Gédéon, celui-ci sonna du cor, et Abièzer se regroupa derrière lui. 
${}^{35}Il envoya des messagers dans tout Manassé qui, lui aussi, se regroupa derrière lui ; il envoya des messagers dans Asher, dans Zabulon et dans Nephtali, qui montèrent à leur rencontre.
${}^{36}Gédéon dit alors à Dieu : « Si vraiment, comme tu l’as dit, tu veux te servir de moi pour sauver Israël, 
${}^{37}je vais étendre une toison de laine sur l’aire de battage et, s’il n’y a de rosée que sur la toison et si tout le sol est sec, je saurai que c’est par moi que tu veux sauver Israël, comme tu l’as dit. » 
${}^{38} Il en fut ainsi. Le lendemain, Gédéon se leva ; il pressa la toison, il en exprima la rosée, une pleine coupe. 
${}^{39}Gédéon dit encore à Dieu : « Que ta colère ne s’enflamme pas contre moi ! Laisse-moi te parler encore une fois ! Permets-moi de faire une fois encore l’épreuve de la toison : que seule la toison soit sèche, et qu’il y ait de la rosée sur tout le sol ! » 
${}^{40}Dieu fit ainsi cette nuit-là : seule la toison fut sèche, et il y eut de la rosée sur tout le sol.
      
         
      \bchapter{}
      \begin{verse}
${}^{1}Yeroubbaal – c’est-à-dire Gédéon – et tout le peuple qui était avec lui se levèrent de bon matin et vinrent camper près de Ein-Harod. Le camp de Madiane se trouvait plus au nord, au pied de la colline de Moré, dans la vallée. 
${}^{2}Le Seigneur dit à Gédéon : « Le peuple qui est avec toi est trop nombreux pour que je livre Madiane entre ses mains. Israël pourrait s’en glorifier et dire : “C’est ma main qui m’a sauvé.” 
${}^{3}Et maintenant, crie ceci au peuple : “Ceux qui ont peur et tremblent, qu’ils s’en retournent et partent par le mont Galaad !” » Vingt-deux mille hommes s’en retournèrent, et il en resta dix mille.
${}^{4}Le Seigneur dit à Gédéon : « Ce peuple est encore trop nombreux ! Fais-le descendre au bord de l’eau, et là, pour toi, je le mettrai à l’épreuve. Celui dont je te dirai : “Il doit aller avec toi”, il ira avec toi, et celui dont je te dirai : “Il ne doit pas aller avec toi”, il n’ira pas ! » 
${}^{5}Alors Gédéon fit descendre le peuple au bord de l’eau, et le Seigneur dit à Gédéon : « Tous ceux qui laperont l’eau comme des chiens avec leur langue, tu les mettras à part. Tu feras de même pour ceux qui se mettront à genoux pour boire. » 
${}^{6}Ceux qui lapèrent en portant la main à la bouche furent au nombre de trois cents ; tout le reste du peuple s’était mis à genoux pour boire de l’eau. 
${}^{7}Le Seigneur dit à Gédéon : « C’est avec les trois cents hommes qui ont lapé que je vous sauverai et que je livrerai Madiane entre tes mains. Que le reste du peuple s’en aille chacun chez soi. » 
${}^{8}Les trois cents hommes prirent les provisions du peuple ainsi que leurs cors. Puis Gédéon renvoya tous les hommes d’Israël, chacun à sa tente, ne retenant que les trois cents. Le camp de Madiane était au-dessous du sien, dans la vallée.
${}^{9}Cette nuit-là, le Seigneur dit à Gédéon : « Lève-toi, descends au camp, car je le livre entre tes mains. 
${}^{10}Mais si tu as peur de descendre, va d’abord au camp avec Poura, ton serviteur. 
${}^{11}Tu entendras ce qu’on y dit. Ton courage en sera fortifié, et tu pourras alors descendre dans le camp. » Gédéon alla donc avec Poura, son serviteur, jusqu’aux avant-postes du camp. 
${}^{12}Madiane, Amalec et tous les fils de l’Orient étaient étendus dans la vallée, aussi nombreux que des sauterelles. Leurs chameaux étaient innombrables, comme est innombrable le sable au bord de la mer. 
${}^{13}Au moment où Gédéon arrivait, un homme racontait un songe à son camarade : « Je viens d’avoir un songe : une galette de pain d’orge tournoyait dans le camp de Madiane ; elle arriva jusqu’à la tente, la heurta, la fit tomber et la bouleversa. Et voilà la tente effondrée ! » 
${}^{14}Son camarade lui répondit : « Ce ne peut être que l’épée de Gédéon, fils de Joas, l’Israélite. Dieu livre entre ses mains Madiane et tout son camp. » 
${}^{15}Quand Gédéon eut entendu le récit du songe et son interprétation, il se prosterna, puis revint au camp d’Israël et dit : « Levez-vous, car le Seigneur livre entre vos mains le camp de Madiane. »
${}^{16}Gédéon sépara les trois cents hommes en trois groupes. Il leur remit à tous des cors et des cruches vides dans lesquelles ils mirent des torches. 
${}^{17}Il leur dit : « Vous me regarderez et vous ferez comme moi ! Quand je serai arrivé à la limite du camp, vous ferez comme je ferai. 
${}^{18}Je sonnerai du cor, ainsi que tous ceux qui seront avec moi. Alors, vous aussi, vous sonnerez du cor tout autour du camp, et vous direz : “Pour le Seigneur et pour Gédéon !” »
${}^{19}Gédéon et les cent hommes qui étaient avec lui arrivèrent à la limite du camp vers minuit ; on venait de relever les sentinelles. Ils sonnèrent du cor et brisèrent les cruches qu’ils tenaient à la main. 
${}^{20}Alors les trois groupes sonnèrent du cor et brisèrent les cruches. De la main gauche, ils saisirent les torches et ils crièrent : « Guerre pour le Seigneur et pour Gédéon ! » 
${}^{21}Pendant qu’ils étaient debout autour du camp, chacun à sa place, tous les hommes du camp se mirent à courir, à pousser des cris, et ils s’enfuirent. 
${}^{22}Et, tandis que retentissaient les trois cents cors, le Seigneur fit que, dans tout le camp, chacun tourna l’épée contre son compagnon. Tous s’enfuirent jusqu’à Beth-ha-Shitta, du côté de Céréra, et jusqu’au bord de la rivière d’Abel-Mehola, près de Tabbath.
${}^{23}Alors les hommes d’Israël venus de Nephtali, d’Asher et de tout Manassé se regroupèrent et poursuivirent Madiane. 
${}^{24}Gédéon envoya des messagers dans toute la montagne d’Éphraïm pour dire : « Descendez à la rencontre de Madiane et occupez avant eux les points d’eau jusqu’à Beth-Bara, ainsi que le Jourdain. » Tous les hommes d’Éphraïm se regroupèrent et ils occupèrent les points d’eau jusqu’à Beth-Bara, ainsi que le Jourdain. 
${}^{25}Ils s’emparèrent des deux chefs de Madiane, Oreb et Zéèb. Ils tuèrent Oreb au Rocher d’Oreb, et Zéèb au Pressoir de Zéèb. Puis ils poursuivirent Madiane et rapportèrent à Gédéon les têtes d’Oreb et de Zéèb, par-delà le Jourdain.
      
         
      \bchapter{}
      \begin{verse}
${}^{1}Les gens d’Éphraïm dirent à Gédéon : « Qu’est-ce que tu nous as fait là ? Tu ne nous as pas appelés quand tu es parti combattre Madiane. » Et ils se querellèrent violemment avec lui. 
${}^{2}Gédéon leur répondit : « Je n’ai rien fait, en comparaison de ce que vous avez fait. Ce qu’on ramasse après la vendange dans la vigne d’Éphraïm ne vaut-il pas mieux que la récolte d’Abièzer ? 
${}^{3}C’est entre vos mains que Dieu a livré les chefs de Madiane, Oreb et Zéèb. Que pouvais-je faire en comparaison de ce que vous avez fait ? » Cette parole calma leurs esprits.
      
         
${}^{4}Gédéon arriva au Jourdain et le passa avec ses trois cents hommes. Ils étaient épuisés par la poursuite. 
${}^{5}Gédéon dit aux gens de Souccoth : « Donnez donc des couronnes de pain à la troupe qui marche avec moi, car ils sont épuisés. Je poursuis Zèba et Salmounna, les rois de Madiane. » 
${}^{6}Mais les chefs de Souccoth répondirent : « Tiens-tu déjà en ton pouvoir Zèba et Salmounna, pour que nous donnions du pain à ton armée ? » 
${}^{7}« Eh bien, répliqua Gédéon, quand le Seigneur m’aura livré Zèba et Salmounna, je vous écraserai avec les épines et les chardons du désert ! » 
${}^{8}Il monta de là à Penouël, il parla de la même manière, et les gens de Penouël répondirent comme l’avaient fait les gens de Souccoth. 
${}^{9}Gédéon répliqua aux gens de Penouël : « Quand je reviendrai sain et sauf, je détruirai cette tour ! »
${}^{10}Zèba et Salmounna se trouvaient dans le Qarqor avec leurs armées, qui comptaient environ quinze mille hommes, tout ce qui restait de l’armée des fils de l’Orient : cent vingt mille guerriers étaient tombés. 
${}^{11}Gédéon monta par la route des nomades, à l’est de Nobah et de Yogboha, et il battit l’armée, alors qu’elle se croyait en sécurité. 
${}^{12}Zèba et Salmounna s’enfuirent, mais Gédéon les poursuivit, s’empara des deux rois de Madiane et sema la panique dans toute l’armée.
${}^{13}Gédéon, fils de Joas, revint du combat par la montée de Hèrès. 
${}^{14}Il arrêta un jeune homme de Souccoth, et il l’interrogea. Celui-ci écrivit pour lui les noms des chefs et des anciens de Souccoth : soixante-dix-sept hommes. 
${}^{15}Gédéon se rendit alors auprès des gens de Souccoth et il leur dit : « Voici Zèba et Salmounna, à propos desquels vous m’aviez mis au défi, en me disant : “Tiens-tu déjà en ton pouvoir Zèba et Salmounna, pour que nous donnions du pain à tes hommes épuisés ?” » 
${}^{16}Il prit les anciens de la ville et, avec des épines et des chardons du désert, il donna une leçon aux hommes de Souccoth. 
${}^{17}Il renversa aussi la tour de Penouël et tua les habitants de la ville.
${}^{18}Puis il dit à Zèba et à Salmounna : « Comment étaient les hommes que vous avez tués au Tabor ? » Ils répondirent : « Ils étaient comme toi. Ils avaient chacun l’air d’un fils de roi. » 
${}^{19}Il leur dit : « C’étaient mes frères, les fils de ma mère ! Par la vie du Seigneur, si vous les aviez laissé vivre, je ne vous tuerais pas. » 
${}^{20}Puis il dit à Yéter, son fils aîné : « Lève-toi et tue-les ! » Mais le garçon ne tira pas son épée : il n’osait pas, car il était encore jeune. 
${}^{21}Zèba et Salmounna dirent alors : « Lève-toi toi-même et frappe-nous, car la bravoure, c’est l’homme. » Alors Gédéon se leva et tua Zèba et Salmounna, et il prit les amulettes en forme de croissants de lune qui étaient au cou de leurs chameaux.
${}^{22}Les gens d’Israël dirent à Gédéon : « Sois notre maître, toi, puis ton fils, puis ton petit-fils, car tu nous as sauvés de la main de Madiane. » 
${}^{23}Gédéon répondit : « Moi, je ne serai pas votre maître, pas plus que mon fils. C’est le Seigneur qui sera votre maître. » 
${}^{24}Gédéon ajouta : « Je veux vous faire une requête : Que chacun de vous me donne un anneau de son butin. » Les ennemis avaient en effet des anneaux d’or, car c’étaient des Ismaélites. 
${}^{25}« Nous les donnerons volontiers », répondirent-ils. Ils étendirent un manteau, et chacun y jeta un anneau de son butin. 
${}^{26}Le poids des anneaux d’or qu’il avait demandés s’éleva à celui de mille sept cents pièces d’or, sans compter les amulettes, les pendants d’oreille, les vêtements de pourpre que portaient les rois de Madiane et les colliers qui étaient au cou de leurs chameaux. 
${}^{27}Gédéon en fit un objet de culte, un éphod, qu’il installa dans sa ville, à Ofra. Tout Israël vint s’y prostituer, et cet éphod devint un piège pour Gédéon et pour sa maison.
${}^{28}Madiane fut soumis par les fils d’Israël et ne releva plus la tête. Ainsi, au temps de Gédéon, le pays fut en repos pendant quarante ans. 
${}^{29}Yeroubbaal – c’est-à-dire Gédéon –, fils de Joas, alla résider dans sa maison. 
${}^{30}Il eut soixante-dix fils, issus de lui, car il avait beaucoup de femmes. 
${}^{31}Sa concubine, qui habitait Sichem, lui enfanta, elle aussi, un fils, qu’il nomma Abimélek. 
${}^{32}Gédéon, fils de Joas, mourut après une heureuse vieillesse et il fut enseveli dans le tombeau de Joas, son père, à Ofra d’Abièzer.
${}^{33}Après la mort de Gédéon, les fils d’Israël recommencèrent à se prostituer aux Baals et ils adoptèrent pour Dieu Baal-Berith. 
${}^{34}Ils ne se souvinrent plus du Seigneur, leur Dieu, qui les avait délivrés de la main de tous leurs ennemis d’alentour, 
${}^{35}et ils ne firent preuve d’aucune fidélité envers la maison de Yeroubbaal-Gédéon, après tout le bien qu’il avait fait à Israël.
       
      
         
      \bchapter{}
      \begin{verse}
${}^{1}Abimélek, fils de Yeroubbaal, alla trouver à Sichem les frères de sa mère et leur parla, ainsi qu’à tout le clan de la maison paternelle de sa mère. Il leur dit : 
${}^{2}« Faites entendre ceci à tous les notables de Sichem : Que vaut-il mieux pour vous ? Avoir pour maîtres les soixante-dix fils de Yeroubbaal, ou avoir pour maître un seul homme ? Souvenez-vous que moi, je suis de vos os et de votre chair. » 
${}^{3}Les frères de sa mère firent entendre ces paroles à tous les notables de Sichem. Leur cœur pencha pour Abimélek, car ils se disaient : « C’est notre frère ! » 
${}^{4}Ils lui donnèrent soixante-dix pièces d’argent du temple de Baal-Berith, avec lesquelles il recruta des vauriens et des aventuriers qui marchèrent à sa suite. 
${}^{5}Puis il entra dans la maison de son père, à Ofra, et il tua ses soixante-dix frères, les fils de Yeroubbaal, sur une même pierre. Seul survécut Yotam, le plus jeune, qui s’était caché.
      
         
${}^{6}Tous les notables de Sichem et ceux de la maison du Terre-Plein\\se réunirent et vinrent proclamer roi Abimélek, près du chêne de la Pierre-Dressée qui est à Sichem. 
${}^{7} On l’annonça à Yotam. Celui-ci vint se poster sur le sommet du mont Garizim et il cria de toutes ses forces :
        \\« Écoutez-moi, notables de Sichem,
        \\et Dieu vous écoutera !
        ${}^{8}Un jour, les arbres se mirent en campagne
        \\pour se donner un roi et le consacrer par l’onction.
        \\Ils dirent à l’olivier :
        \\“Sois notre roi !”
        ${}^{9}L’olivier leur répondit :
        \\“Faudra-t-il que je renonce à mon huile,
        \\qui sert à honorer Dieu et les hommes,
        \\pour aller me balancer au-dessus des autres arbres ?”
        ${}^{10}Alors les arbres dirent au figuier :
        \\“Viens, toi, sois notre roi !”
        ${}^{11}Le figuier leur répondit :
        \\“Faudra-t-il que je renonce
        \\à la douceur et à la saveur de mes fruits,
        \\pour aller me balancer au-dessus des autres arbres ?”
        ${}^{12}Les arbres dirent alors à la vigne :
        \\“Viens, toi, sois notre roi !”
        ${}^{13}La vigne leur répondit :
        \\“Faudra-t-il que je renonce à mon vin,
        \\qui réjouit Dieu et les hommes,
        \\pour aller me balancer au-dessus des autres arbres ?”
        ${}^{14}Alors tous les arbres dirent au buisson d’épines :
        \\“Viens, toi, sois notre roi !”
        ${}^{15}Et le buisson d’épines répondit aux arbres :
        \\“Si c’est de bonne foi
        \\que vous me consacrez par l’onction
        \\pour être votre roi,
        \\venez vous abriter sous mon ombre ;
        \\sinon, qu’un feu sorte du buisson d’épines
        \\et dévore jusqu’aux cèdres du Liban !” »
${}^{16}« Maintenant, avez-vous agi dans la fidélité et l’intégrité en faisant roi Abimélek ? Avez-vous bien agi à l’égard de Yeroubbaal et de sa famille ? Avez-vous agi selon le mérite de ses actes ? 
${}^{17}Mon père a combattu pour vous, il a risqué pour vous sa vie, il vous a arrachés aux mains de Madiane. 
${}^{18}Mais vous vous êtes levés aujourd’hui contre sa maison, vous avez tué ses soixante-dix fils sur une même pierre, vous avez proclamé roi, sur les notables de Sichem, Abimélek, le fils de sa servante, parce qu’il est votre frère. 
${}^{19}Si, en ce jour, vous avez agi dans la fidélité et l’intégrité envers Yeroubbaal et sa maison, qu’Abimélek fasse votre joie, et vous la sienne ! 
${}^{20}Mais s’il n’en est pas ainsi, qu’un feu sorte d’Abimélek, et qu’il dévore les notables de Sichem, ainsi que la maison du Terre-Plein ! Et que des notables de Sichem et de la maison du Terre-Plein, un feu sorte pour dévorer Abimélek ! »
${}^{21}Yotam prit la fuite et disparut. Il alla s’établir à Beér, par crainte d’Abimélek, son frère.
${}^{22}Abimélek gouverna Israël pendant trois ans, 
${}^{23}puis Dieu envoya un esprit de discorde entre Abimélek et les notables de Sichem, qui le trahirent. 
${}^{24}Il fallait que soit vengée la violence faite aux soixante-dix fils de Yeroubbaal, et que leur sang retombe sur Abimélek, leur frère, qui les avait tués, et sur les notables de Sichem qui l’avaient soutenu. 
${}^{25}Pour lui faire du tort, les notables de Sichem mirent en embuscade, au sommet des montagnes, des hommes qui dépouillaient tous ceux qui passaient près d’eux sur le chemin. On en informa Abimélek.
${}^{26}Gaal, fils d’Ébed, accompagné de ses frères, passa par Sichem et obtint la confiance de ses notables. 
${}^{27}Ils sortirent dans la campagne pour vendanger leurs vignes, ils foulèrent le raisin et organisèrent des réjouissances. Au temple de leurs dieux, ils mangèrent et burent. Puis ils maudirent Abimélek. 
${}^{28}Gaal, fils d’Ébed, dit alors : « Qui est Abimélek ? Qu’est-ce que Sichem pour que nous servions Abimélek ? N’est-il pas le fils de Yeroubbaal, et Zeboul, son lieutenant ? Servez les hommes de Hamor, père de Sichem. Pourquoi, nous autres, servirions-nous celui-là ? 
${}^{29}Si seulement j’avais en main ce peuple, j’écarterais Abimélek ! Je lui dirais : “Renforce ton armée, et viens combattre !” »
${}^{30}Zeboul, gouverneur de la ville, apprit ce que disait Gaal, fils d’Ébed. Il se mit en colère. 
${}^{31}Il envoya secrètement des messagers à Abimélek pour lui dire : « Gaal, fils d’Ébed, est arrivé à Sichem avec ses frères, et ils excitent la ville contre toi. 
${}^{32}Maintenant donc, lève-toi de nuit, avec ta troupe, et mets-toi en embuscade dans la campagne. 
${}^{33}Puis, le matin, au lever du soleil, tu partiras et tu lanceras un assaut contre la ville. Quand Gaal et la troupe qui le suit sortiront à ta rencontre, tu verras ce que tu pourras leur faire. »
${}^{34}Abimélek partit de nuit avec sa troupe ; ils s’embusquèrent près de Sichem, en se divisant en quatre groupes. 
${}^{35}Comme Gaal, fils d’Ébed, sortait et se trouvait à l’entrée de la porte de la ville, Abimélek et la troupe qui l’accompagnait surgirent de l’embuscade. 
${}^{36}À leur vue, Gaal dit à Zeboul : « Voici une troupe qui descend du sommet des montagnes ! » Zeboul lui dit : « C’est l’ombre des montagnes que tu prends pour des hommes ! » 
${}^{37}Gaal insista et dit : « Voici une troupe qui descend du côté du lieu-dit « Nombril de la terre », et un autre groupe qui vient par le chemin du Chêne des Devins. » 
${}^{38}Zeboul lui dit : « Qu’as-tu fait de ta langue, toi qui disais : “Qui est Abimélek pour que nous le servions ?” N’est-ce pas la troupe que tu méprisais ? Sors donc maintenant, et attaque-le ! »
${}^{39}Gaal sortit à la tête des notables de Sichem, et il combattit contre Abimélek. 
${}^{40}Abimélek poursuivit Gaal, qui s’était enfui. Il y eut de nombreuses victimes à l’entrée de la porte. 
${}^{41}Abimélek résida alors à Arouma, et Zeboul expulsa Gaal et ses frères, pour les empêcher de résider à Sichem.
${}^{42}Le lendemain, les gens sortirent dans la campagne ; Abimélek en fut informé. 
${}^{43}Il prit sa troupe, la partagea en trois groupes et se mit en embuscade dans la campagne. Quand il vit les gens sortir de la ville, il surgit contre eux et les tailla en pièces. 
${}^{44}Tandis qu’Abimélek et le groupe qui était avec lui se déployaient et prenaient position à l’entrée de la porte de la ville, les deux autres groupes se déployaient contre tous ceux qui étaient dans la campagne, et les taillaient en pièces. 
${}^{45}Tout ce jour-là, Abimélek donna l’assaut à la ville, il s’en empara et tua tous ses habitants. Puis il la détruisit et y sema du sel. 
${}^{46}Quand les notables de la Tour-de-Sichem l’apprirent, ils se rendirent dans la crypte du temple d’El-Berith. 
${}^{47}On avertit Abimélek de leur rassemblement. 
${}^{48}Il monta avec sa troupe sur le mont Salmone. Il prit une hache à double tranchant, coupa une branche d’arbre, la prit et la mit sur son épaule. Il dit à la troupe qui était avec lui : « Ce que vous m’avez vu faire, hâtez-vous de le faire comme moi. » 
${}^{49}Tous les hommes de la troupe coupèrent, eux aussi, chacun une branche ; ils suivirent Abimélek, jetèrent les branches dans la crypte et la brûlèrent sur ceux qui s’y trouvaient. Ainsi périrent tous les habitants de la Tour-de-Sichem – environ mille hommes et femmes.
${}^{50}Abimélek se rendit alors à Tébès ; il assiégea la ville et s’en empara. 
${}^{51}Il y avait au milieu de la ville une tour fortifiée. Tous les hommes, les femmes et tous les notables de la ville s’y étaient réfugiés. Après avoir fermé la porte derrière eux, ils étaient montés sur la terrasse de la tour. 
${}^{52}Abimélek arriva jusqu’à la tour, l’attaqua et s’approcha de l’entrée de la tour pour y mettre le feu. 
${}^{53}Mais une femme lança une meule sur sa tête et lui fracassa le crâne. 
${}^{54}Abimélek appela aussitôt son écuyer et lui dit : « Tire ton épée, donne-moi la mort, de peur que l’on ne dise de moi : “C’est une femme qui l’a tué.” » Alors son écuyer le transperça de son épée, et il mourut. 
${}^{55}Quand les hommes d’Israël virent qu’Abimélek était mort, ils s’en allèrent chacun chez soi.
${}^{56}Ainsi, Dieu fit retomber sur Abimélek le mal qu’il avait fait à son père en tuant ses soixante-dix frères. 
${}^{57}Et tout le mal commis par les hommes de Sichem, Dieu le fit retomber sur leur tête. C’est ainsi que s’accomplit la malédiction de Yotam, fils de Yeroubbaal.
      <h2 class="intertitle hmbot" id="d85e58431">6. Tola (10,1-2)</h2>
      
         
      \bchapter{}
      \begin{verse}
${}^{1}Après Abimélek, ce fut Tola, fils de Poua, fils de Dodo, qui se leva pour sauver Israël. Il était d’Issakar, et il habitait Shamir, dans la montagne d’Éphraïm. 
${}^{2}Il fut juge en Israël pendant vingt-trois ans, puis il mourut et fut enseveli à Shamir.
      
         
      <h2 class="intertitle hmbot" id="d85e58445">7. Yaïr (10,3-5)</h2>
${}^{3}Après lui, ce fut Yaïr de Galaad qui jugea Israël pendant vingt-deux ans. 
${}^{4}Il avait trente fils qui montaient trente ânons et possédaient trente villes, qu’on appelle encore aujourd’hui les Douars de Yaïr, au pays de Galaad. 
${}^{5}Puis Yaïr mourut et il fut enseveli à Qamone.
      <h2 class="intertitle hmbot" id="d85e58460">8. Jephté (10,6 – 12,7)</h2>
${}^{6}Les fils d’Israël recommencèrent à faire ce qui est mal aux yeux du Seigneur. Ils rendirent un culte aux Baals et aux Astartés, aux dieux d’Aram, de Sidon, de Moab, aux dieux des fils d’Ammone et aux dieux des Philistins. Ils abandonnèrent le Seigneur et ne le servirent plus. 
${}^{7}La colère du Seigneur s’enflamma contre Israël, et il les vendit aux Philistins et aux fils d’Ammone. 
${}^{8}Ceux-ci écrasèrent et tourmentèrent les fils d’Israël cette année-là et, pendant dix-huit ans, tous ceux qui étaient au-delà du Jourdain, dans le pays des Amorites de Galaad. 
${}^{9}Les fils d’Ammone passèrent le Jourdain pour combattre aussi Juda, Benjamin et la maison d’Éphraïm, et la détresse d’Israël fut très grande. 
${}^{10}Alors les fils d’Israël crièrent vers le Seigneur en disant : « Nous avons péché contre toi, car nous avons abandonné notre Dieu pour servir les Baals. » 
${}^{11}Le Seigneur dit aux fils d’Israël : « Lorsque les Égyptiens, les Amorites, les fils d’Ammone, les Philistins, 
${}^{12}les Sidoniens, les Amalécites et les Madianites vous ont opprimés, et que vous avez crié vers moi, ne vous ai-je pas sauvés de leurs mains ? 
${}^{13}Mais vous, vous m’avez abandonné et vous avez servi d’autres dieux. C’est pourquoi je ne vous sauverai pas à nouveau. 
${}^{14}Allez, criez vers les dieux que vous avez choisis ! Qu’ils vous sauvent, eux, au temps de votre détresse ! » 
${}^{15}Les fils d’Israël dirent au Seigneur : « Nous avons péché ! Toi, fais de nous tout ce qui est bon à tes yeux ; mais, aujourd’hui, délivre-nous ! » 
${}^{16}Ils enlevèrent les dieux étrangers et servirent le Seigneur, qui ne put supporter la souffrance d’Israël.
${}^{17}Les fils d’Ammone se rassemblèrent et campèrent à Galaad. Les fils d’Israël se réunirent et campèrent à Mispa. 
${}^{18}Le peuple, les chefs de Galaad, se dirent l’un à l’autre : « Quel est l’homme qui entreprendra le combat contre les fils d’Ammone ? Il sera le chef de tous les habitants de Galaad. »
      
         
      \bchapter{}
      \begin{verse}
${}^{1}Jephté de Galaad était un vaillant guerrier. Il était le fils d’une prostituée. C’était Galaad qui l’avait engendré. 
${}^{2}Puis la femme de Galaad lui enfanta des fils qui, lorsqu’ils eurent grandi, chassèrent Jephté, en lui disant : « Tu n’auras pas de part d’héritage dans la maison de notre père, car tu es le fils d’une autre femme, toi ! » 
${}^{3}Jephté s’enfuit loin de ses frères et s’établit, dans le pays de Tob. Des vauriens s’associèrent à lui et ils faisaient avec lui des expéditions.
${}^{4}Quelque temps après, les fils d’Ammone vinrent combattre Israël. 
${}^{5}Comme les fils d’Ammone combattaient Israël, les anciens de Galaad allèrent chercher Jephté au pays de Tob. 
${}^{6}Ils lui dirent : « Viens, sois notre commandant, et nous combattrons les fils d’Ammone. » 
${}^{7}Jephté répondit aux anciens de Galaad : « N’est-ce pas votre haine qui m’a chassé de la maison de mon père ? Pourquoi venez-vous vers moi, maintenant que vous êtes dans la détresse ? » 
${}^{8}Les anciens de Galaad dirent à Jephté : « C’est bien pour cela que maintenant nous sommes revenus à toi, pour que tu viennes avec nous combattre les fils d’Ammone, et pour que tu sois notre chef et celui de tous les habitants de Galaad. » 
${}^{9}Jephté répondit aux anciens de Galaad : « Si vous me faites revenir pour combattre les fils d’Ammone et que le Seigneur les livre à ma merci, moi, je serai votre chef. » 
${}^{10}Les anciens de Galaad lui dirent alors : « Le Seigneur sera témoin entre nous, si nous n’agissons pas selon ta parole. » 
${}^{11}Jephté partit avec les anciens de Galaad ; le peuple en fit son chef et son commandant. Jephté redit toutes ses paroles devant le Seigneur, à Mispa.
${}^{12}Jephté envoya des messagers au roi des Ammonites pour lui dire : « Que me veux-tu, pour être venu faire la guerre à mon pays ? » 
${}^{13}Le roi des Ammonites répondit aux messagers de Jephté : « C’est parce qu’Israël, quand il est monté d’Égypte, s’est emparé de mon pays depuis l’Arnon jusqu’au Yabboq et au Jourdain. Maintenant rends ce territoire pacifiquement. »
${}^{14}Jephté envoya de nouveau des messagers au roi des Ammonites 
${}^{15}et lui dit : « Ainsi parle Jephté : Israël n’a pas pris le pays de Moab, ni celui des fils Ammone. 
${}^{16}En effet, quand il est monté d’Égypte, Israël a marché dans le désert, jusqu’à la mer des Roseaux, et il est arrivé à Cadès. 
${}^{17}Israël envoya des messagers au roi d’Édom, pour lui dire : “Permets-moi, je t’en prie, de traverser ton pays.” Le roi d’Édom ne voulut rien entendre. Israël envoya aussi des messagers au roi de Moab, qui refusa. Israël demeura donc à Cadès. 
${}^{18}Puis il marcha dans le désert, contourna le pays d’Édom et le pays de Moab, et arriva à l’est de Moab. Ils campèrent au-delà de l’Arnon et n’entrèrent pas dans le territoire de Moab, dont l’Arnon marque la frontière. 
${}^{19}Israël envoya des messagers à Séhone, roi des Amorites, roi de Heshbone pour lui dire : “Permets, nous t’en prions, que nous traversions ton pays, pour nous rendre là où nous voulons aller.” 
${}^{20}Mais Séhone n’accepta pas qu’Israël traverse son territoire. Il rassembla tout son peuple, qui campa à Yahça, et il livra bataille à Israël. 
${}^{21}Le Seigneur, Dieu d’Israël, livra Séhone et tout son peuple entre les mains d’Israël, qui les battit. Israël prit possession de tout le pays des Amorites, qui habitaient alors ce pays. 
${}^{22}Ils occupèrent entièrement le territoire des Amorites, depuis l’Arnon jusqu’au Yabboq, depuis le désert jusqu’au Jourdain. 
${}^{23}Et maintenant que le Seigneur, Dieu d’Israël, a dépossédé les Amorites en faveur de son peuple Israël, toi, tu voudrais le déposséder ! 
${}^{24}Ne possèdes-tu pas ce que Camosh, ton dieu, t’a permis de posséder ? Et ne posséderions-nous pas tout ce que le Seigneur nous a donné ? 
${}^{25}Vaudrais-tu donc mieux que Balaq, fils de Cippor, roi de Moab ? A-t-il osé chercher querelle à Israël ? A-t-il osé combattre contre lui ? 
${}^{26}Lorsqu’il y a trois cents ans, Israël s’est établi à Heshbone et dans ses dépendances, à Aroër et dans ses dépendances, dans toutes les villes qui sont sur les bords de l’Arnon, pourquoi ne pas les avoir reprises à ce moment-là ? 
${}^{27}Ce n’est pas moi qui ai fait une faute contre toi, mais c’est toi qui agis mal envers moi en me combattant ! Que le Seigneur, le Juge, juge aujourd’hui entre les fils d’Israël et les fils d’Ammone. » 
${}^{28}Mais le roi des Ammonites n’écouta pas les paroles que Jephté lui avait fait adresser.
${}^{29}L’Esprit du Seigneur s’empara de Jephté, et il traversa les pays de Galaad et Manassé, et Mispa de Galaad. De là il passa la frontière des fils d’Ammone. 
${}^{30} Jephté fit alors ce vœu au Seigneur : « Si tu livres les fils d’Ammone entre mes mains, 
${}^{31} la première personne qui sortira de ma maison pour venir à ma rencontre quand je reviendrai victorieux\\appartiendra au Seigneur, et\\je l’offrirai en sacrifice d’holocauste. » 
${}^{32} Jephté passa chez les fils d’Ammone pour les attaquer, et le Seigneur les livra entre ses mains. 
${}^{33} Il les battit depuis Aroër jusqu’à proximité de Minnith et jusqu’à Abel-Keramim, soit le territoire de vingt villes. Ce fut une très grande défaite, et les fils d’Ammone durent se soumettre aux fils d’Israël.
${}^{34}Lorsque Jephté revint à Mispa, comme il arrivait à sa maison, voici que sa fille sortit à sa rencontre en dansant au son des tambourins. C’était son unique enfant ; en dehors d’elle, il n’avait ni fils ni fille. 
${}^{35}Dès qu’il l’aperçut, il déchira ses vêtements et s’écria : « Hélas, ma fille, tu m’accables ! C’est toi qui fais mon malheur ! J’ai parlé trop vite devant le Seigneur, et je ne peux pas reprendre ma parole. » 
${}^{36}Elle lui répondit : « Mon père, tu as parlé trop vite devant le Seigneur, traite-moi donc selon ta parole, puisque maintenant le Seigneur t’a vengé de tes ennemis, les fils d’Ammone. » 
${}^{37}Et elle ajouta : « Je ne te demande qu’une chose : laisse-moi un répit de deux mois. J’irai dans les montagnes pour pleurer ma virginité avec mes amies\\. » 
${}^{38}Il lui dit : « Va ! » Et il la laissa partir pour deux mois. Elle s’en alla donc, avec ses amies, dans la montagne, et pleura sa virginité. 
${}^{39}Les deux mois écoulés, elle revint vers son père, et il accomplit à son égard le vœu qu’il avait prononcé. Elle ne s’était pas unie à un homme. Et c’est une coutume en Israël que, 
${}^{40}d’année en année, les filles d’Israël aillent honorer la fille de Jephté de Galaad quatre jours par an.
      
         
      \bchapter{}
      \begin{verse}
${}^{1}Les hommes d’Éphraïm se regroupèrent et passèrent vers le nord. Ils dirent à Jephté : « Pourquoi es-tu passé pour combattre les fils d’Ammone sans nous avoir appelés à marcher avec toi ? Nous brûlerons ta maison sur toi. » 
${}^{2}Jephté leur répliqua : « J’étais en grand conflit, moi et mon peuple, avec les fils d’Ammone. Lorsque j’ai fait appel à vous, vous ne m’avez pas sauvé de leurs mains. 
${}^{3}Quand j’ai vu que vous ne me sauveriez pas, j’ai risqué ma vie, je suis passé chez les fils d’Ammone. Le Seigneur les a livrés entre mes mains. Pourquoi êtes-vous montés contre moi aujourd’hui pour me combattre ? » 
${}^{4}Jephté regroupa alors tous les hommes de Galaad et attaqua Éphraïm. Les hommes de Galaad battirent ceux d’Éphraïm, qui disaient : « Vous êtes des rescapés d’Éphraïm, gens de Galaad, au milieu d’Éphraïm, au milieu de Manassé. » 
${}^{5}Galaad s’empara des gués du Jourdain, près d’Éphraïm. Et lorsqu’un des rescapés d’Éphraïm disait : « Je voudrais traverser », les hommes de Galaad lui demandaient : « Es-tu d’Éphraïm ? » S’il répondait : « Non », 
${}^{6}ils lui disaient : « Eh bien, dis : Shibboleth ! » Lui prononçait : « Sibboleth », car il n’arrivait pas à dire le mot correctement. Alors, on le saisissait et on l’égorgeait près des gués du Jourdain. À cette époque, tombèrent quarante-deux mille hommes d’Éphraïm.
${}^{7}Jephté jugea Israël pendant six ans. Puis Jephté le Galaadite mourut et fut enseveli dans sa ville, en Galaad.
      <h2 class="intertitle hmbot" id="d85e58838">9. Ibsane (12,8-10)</h2>
${}^{8}Après lui, ce fut Ibsane de Bethléem qui jugea Israël. 
${}^{9}Il avait trente fils et trente filles. Il maria ses filles au-dehors et fit venir du dehors des femmes pour ses fils. Il jugea Israël pendant sept ans. 
${}^{10}Puis Ibsane mourut et fut enseveli à Bethléem.
      <h2 class="intertitle hmbot" id="d85e58864">10. Élone (12,11-12)</h2>
${}^{11}Après lui, ce fut Élone de Zabulon qui jugea Israël, pendant dix ans. 
${}^{12}Puis Élone mourut et fut enseveli à Ayyalone, en terre de Zabulon.
      <h2 class="intertitle hmbot" id="d85e58881">11. Abdone (12,13-15)</h2>
${}^{13}Après lui, ce fut Abdone, fils de Hillel de Piréatone, qui jugea Israël. 
${}^{14}Il avait quarante fils et trente petits-fils qui montaient soixante-dix ânons. Il jugea Israël pendant huit ans. 
${}^{15}Puis Abdone, fils de Hillel, mourut et fut enseveli à Piréatone, au pays d’Éphraïm, dans la montagne des Amalécites.
      <h2 class="intertitle" id="d85e58910">12. Samson (13 – 16)</h2>
      
         
      \bchapter{}
      \begin{verse}
${}^{1}Les fils d’Israël recommencèrent à faire ce qui est mal aux yeux du Seigneur, et le Seigneur les livra entre les mains des Philistins pendant quarante ans.
${}^{2}Il y avait un homme de Soréa, du clan de Dane, nommé Manoah. Sa femme était stérile et n’avait pas eu d’enfant. 
${}^{3} L’ange du Seigneur apparut à cette femme et lui dit : « Tu es stérile et tu n’as pas eu d’enfant. 
${}^{4} Mais tu vas concevoir et enfanter\\un fils. Désormais, fais bien attention : ne bois ni vin ni boisson forte, et ne mange aucun aliment impur, 
${}^{5} car tu vas concevoir et enfanter un fils. Le rasoir ne passera pas sur sa tête, car il sera voué à Dieu\\dès le sein de sa mère. C’est lui qui entreprendra de sauver Israël de la main des Philistins. » 
${}^{6} La femme s’en alla dire à son mari : « Un homme de Dieu est venu me trouver ; il avait l’apparence d’un ange de Dieu tant il était imposant. Je ne lui ai pas demandé d’où il venait, et il ne m’a pas fait connaître son nom. 
${}^{7} Mais il m’a dit : “Tu vas devenir enceinte et enfanter un fils. Désormais ne bois ni vin ni boisson forte, et ne mange aucun aliment impur, car l’enfant sera voué à Dieu dès le sein de sa mère et jusqu’au jour de sa mort !” »
${}^{8}Alors, Manoah implora le Seigneur et dit : « Je t’en prie, Seigneur, que l’homme de Dieu que tu as envoyé revienne vers nous, et qu’il nous enseigne ce que nous devrons faire pour l’enfant qui va naître. » 
${}^{9}Dieu écouta la voix de Manoah, et l’ange de Dieu revint trouver la femme, qui était assise dans le champ, en l’absence de son mari. 
${}^{10}Aussitôt, elle courut annoncer à son mari : « Voici que m’est apparu l’homme qui est venu me trouver l’autre jour. » 
${}^{11}Manoah se leva et suivit sa femme ; il vint vers l’homme et lui dit : « Est-ce toi, l’homme qui a parlé à cette femme ? » Il répondit : « C’est moi. » 
${}^{12}Manoah dit : « Maintenant que ta parole va se réaliser, quelle sera la règle de conduite à l’égard de l’enfant, et que devra-t-il faire ? » 
${}^{13}L’ange du Seigneur dit à Manoah : « Que ta femme s’abstienne de tout ce que je lui ai interdit : 
${}^{14}elle ne doit rien manger qui provienne du fruit de la vigne ; qu’elle ne boive ni vin ni boisson forte ; qu’elle ne mange aucun aliment impur ; tout ce que je lui ai ordonné, qu’elle l’observe. » 
${}^{15}Manoah dit à l’ange du Seigneur : « Permets, je t’en prie, que nous te retenions et que nous te préparions un chevreau. » 
${}^{16}L’ange du Seigneur répondit à Manoah : « Même si tu me retenais, je ne mangerais pas de ton pain. Offre plutôt un holocauste au Seigneur. » Manoah ne savait pas que l’homme était l’ange du Seigneur. 
${}^{17}Il lui dit : « Quel est ton nom, pour que nous puissions t’honorer lorsque tes paroles se réaliseront ? » 
${}^{18}L’ange du Seigneur lui répondit : « Pourquoi demandes-tu mon nom, alors qu’il est merveilleux ? » 
${}^{19}Manoah prit le chevreau et l’offrande de céréales et, sur le rocher, il en fit l’holocauste au Seigneur, à celui qui fait des merveilles. Manoah et sa femme regardaient. 
${}^{20}Or, quand la flamme monta de l’autel vers le ciel, l’ange du Seigneur monta dans la flamme de l’autel. Voyant cela, Manoah et sa femme tombèrent face contre terre. 
${}^{21}Désormais l’ange du Seigneur ne leur apparut plus. Manoah comprit que c’était l’ange du Seigneur. 
${}^{22}Il dit à sa femme : « Nous allons sûrement mourir, car nous avons vu Dieu. » 
${}^{23}Mais sa femme lui dit : « Si le Seigneur voulait nous faire mourir, il n’aurait accepté de notre main ni holocauste ni offrande ; il ne nous aurait pas donné pareilles choses à voir, et à entendre maintenant. » 
${}^{24}La femme enfanta un fils, et elle lui donna le nom de Samson. L’enfant grandit, le Seigneur le bénit, 
${}^{25}et l’Esprit du Seigneur commença à s’emparer de lui à Mahané-Dane, entre Soréa et Eshtaol.
      
         
      \bchapter{}
      \begin{verse}
${}^{1}Samson descendit à Timna et y remarqua une femme parmi les filles des Philistins. 
${}^{2}Il remonta l’annoncer à son père et à sa mère, et il leur dit : « À Timna, j’ai remarqué une femme parmi les filles des Philistins. Prenez-la-moi donc pour femme. » 
${}^{3}Son père lui dit, ainsi que sa mère : « N’y a-t-il pas assez de femmes parmi les filles de tes frères et dans tout mon peuple, pour que tu ailles prendre femme chez les Philistins, ces incirconcis ? » Mais Samson répondit à son père : « Prends-la-moi. Elle me plaît. » 
${}^{4}Son père et sa mère ne savaient pas que le Seigneur inspirait ce choix et qu’il cherchait une occasion de conflit avec les Philistins qui, en ce temps-là, dominaient Israël.
${}^{5}Samson descendit donc à Timna avec son père et sa mère. Alors qu’ils arrivaient aux vignes de Timna, voici qu’un jeune lion se précipita sur lui en rugissant. 
${}^{6}L’Esprit du Seigneur s’empara de lui, et, sans rien en main, Samson déchira le lion, comme on déchire un chevreau. Il ne raconta pas à son père et à sa mère ce qu’il avait fait. 
${}^{7}Puis il continua sa route. Il parla à la femme, elle lui plut. 
${}^{8}Il revint quelques jours après pour la prendre, mais il fit un détour pour revoir le cadavre du lion. Il y avait dans sa carcasse un essaim d’abeilles et du miel. 
${}^{9}Il en recueillit dans ses mains et, chemin faisant, il en mangea. Arrivé chez son père et sa mère, il leur en donna, et ils en mangèrent. Mais il ne leur raconta pas qu’il avait recueilli le miel dans la carcasse du lion.
${}^{10}Son père descendit alors chez cette femme, et Samson y donna un festin, comme le font habituellement les jeunes gens. 
${}^{11}Dès qu’on le vit, on lui donna trente compagnons pour rester avec lui. 
${}^{12}Samson leur dit : « Je vais vous proposer une énigme. Si vous me l’expliquez au cours des sept jours du festin, et que vous en trouviez le sens, je vous donnerai trente tuniques et trente habits de rechange. 
${}^{13}Si vous ne pouvez me l’expliquer, c’est vous qui me donnerez trente tuniques et trente habits de rechange. » Ils lui dirent alors : « Propose ton énigme, nous écoutons. » 
${}^{14}Samson leur dit :
        \\« De celui qui mange est sorti ce qui se mange,
        \\et du fort est sorti le doux. »
      Au bout de trois jours, les jeunes gens n’avaient pas encore pu expliquer l’énigme. 
${}^{15}Ils dirent à la femme de Samson : « Séduis ton mari, pour qu’il nous explique l’énigme. Sinon, nous te brûlerons, toi et la maison de ton père. Est-ce pour nous déposséder que vous nous avez invités, oui ou non ? » 
${}^{16}La femme de Samson se mit à pleurer tout contre lui. Elle lui disait : « Tu ne fais que me haïr, tu ne m’aimes pas ! Cette énigme que tu as proposée aux fils de mon peuple, tu ne me l’as pas expliquée ! » Il lui répondit : « Je ne l’ai même pas expliquée à mon père et à ma mère ! Et à toi, je l’expliquerais ! » 
${}^{17}Elle pleura tout contre lui pendant les sept jours que dura le festin, et, le septième jour, à force d’être harcelé, il finit par lui expliquer l’énigme. Et elle l’expliqua à son tour aux fils de son peuple. 
${}^{18}Au septième jour, avant le coucher du soleil, les gens de la ville dirent à Samson :
        \\« Qu’y a-t-il de plus doux que le miel,
        \\de plus fort que le lion ? »
        \\Samson leur dit :
        \\« Si vous n’aviez pas labouré avec ma génisse,
        \\vous n’auriez pas trouvé mon énigme. »
${}^{19}L’Esprit du Seigneur s’empara de lui. Il descendit à Ascalon, tua trente de ses habitants, prit leurs vêtements et donna les habits de rechange à ceux qui avaient expliqué l’énigme. Puis, plein de colère, il remonta vers la maison de son père. 
${}^{20}Quant à la femme de Samson, elle fut donnée à l’un de ses compagnons, dont il avait fait son ami.
      
         
      \bchapter{}
      \begin{verse}
${}^{1}Quelque temps après, à l’époque de la moisson des blés, Samson rendit visite à sa femme en apportant un chevreau, et déclara : « Je veux entrer dans la chambre à coucher de ma femme. » Mais le père de sa femme ne lui permit pas d’entrer, 
${}^{2}et il dit à Samson : « Je me suis dit que, vraiment, tu la haïssais, et je l’ai donnée à ton compagnon. Sa sœur cadette ne vaut-elle pas mieux qu’elle ? Qu’elle soit ta femme à la place de l’aînée ! » 
${}^{3}Samson leur dit : « Cette fois, je serai quitte envers les Philistins si je leur fais du mal. »
${}^{4}Il partit, attrapa trois cents renards, prit des torches et, mettant les renards queue contre queue, il plaça une torche entre les deux queues, au milieu. 
${}^{5}Puis il mit le feu aux torches et lâcha les renards dans les blés mûrs des Philistins ; il incendia aussi bien les meules que les épis, et même les vignes et les oliviers. 
${}^{6}Les Philistins demandèrent : « Qui a fait cela ? » On leur répondit : « C’est Samson, gendre du Timnite, car celui-ci a repris sa femme et l’a donnée à son compagnon. » Alors les Philistins montèrent et ils brûlèrent la femme et son père. 
${}^{7}Samson dit alors : « Puisque vous agissez de la sorte, je ne m’arrêterai qu’après m’être vengé de vous ! » 
${}^{8}Il les battit à plate couture, et ce fut une terrible défaite. Puis il descendit dans une faille du rocher d’Étam, où il demeura.
${}^{9}Les Philistins montèrent camper en Juda, et firent une incursion à Lèhi. 
${}^{10}Les hommes de Juda leur dirent : « Pourquoi êtes-vous montés nous combattre ? » Ils répondirent : « C’est pour ligoter Samson, pour le traiter comme il nous a traités. » 
${}^{11}Trois mille hommes de Juda descendirent vers la faille du rocher d’Étam, et ils dirent à Samson : « Ne sais-tu pas que les Philistins sont nos maîtres ? Que nous as-tu fait là ? » Samson répondit : « Je les ai traités comme ils m’ont traité. » 
${}^{12}Ils lui dirent : « C’est pour te ligoter que nous sommes descendus, pour te livrer aux Philistins. » Samson leur dit : « Jurez-moi que vous ne me tuerez pas vous-mêmes. » 
${}^{13}Ils répondirent : « Non, nous voulons seulement te ligoter et te livrer aux Philistins. Nous ne voulons pas te mettre à mort. » Ils le lièrent avec deux cordes neuves et le firent remonter du rocher.
${}^{14}Comme il approchait de Lèhi, les Philistins vinrent à sa rencontre avec des cris de joie, mais alors l’Esprit du Seigneur s’empara de lui : les cordes qui lui liaient les bras devinrent comme des fils de lin consumés par le feu, et les liens qui retenaient ses mains se dénouèrent. 
${}^{15}Puis, trouvant une mâchoire d’âne encore fraîche, il étendit la main pour s’en saisir, et avec elle abattit mille hommes. 
${}^{16}Samson dit :
        \\« Avec une mâchoire d’âne, valant deux ânesses,
        \\avec une mâchoire d’âne, j’ai abattu mille hommes. »
${}^{17}Dès qu’il eut achevé de parler, il jeta la mâchoire loin de lui et il appela ce lieu Ramath-Lèhi (c’est-à-dire : Hauteur de la Mâchoire). 
${}^{18}Comme Samson avait très soif, il invoqua le Seigneur en disant : « C’est toi qui as accordé à ton serviteur cette grande victoire. Vas-tu maintenant me laisser mourir de soif et tomber aux mains des incirconcis ? » 
${}^{19}Alors, Dieu fendit la cavité qui se trouve à Lèhi, et il en sortit de l’eau. Samson en but, il reprit ses esprits et se ranima. Voilà pourquoi on a donné à cette source le nom d’Enn-ha-Qoré (c’est-à-dire : la Source de celui qui invoque). Elle se trouve à Lèhi jusqu’à ce jour. 
${}^{20}Samson jugea Israël, à l’époque des Philistins, pendant vingt ans.
      
         
      \bchapter{}
      \begin{verse}
${}^{1}Samson se rendit à Gaza. Il y vit une prostituée et entra chez elle. 
${}^{2}On dit aux gens de Gaza : « Samson est venu ici. » Ils firent des rondes et guettèrent toute la nuit à la porte de la ville. Ils se tinrent tranquilles toute la nuit, en se disant : « Attendons la lumière du matin, et alors nous le tuerons. » 
${}^{3}Mais Samson resta couché jusqu’au milieu de la nuit. Il se leva alors, saisit les battants de la porte de la ville et les deux montants, les arracha avec leur verrou, les mit sur ses épaules et les emporta au sommet de la montagne qui est en face d’Hébron.
      
         
${}^{4}Après ces événements, il s’éprit d’une femme de la vallée de Soreq, nommée Dalila. 
${}^{5}Les princes des Philistins vinrent la trouver et lui dirent : « Séduis Samson : vois en quoi réside sa grande force et comment on peut triompher de lui. Alors nous le ligoterons pour le maîtriser, et nous te donnerons chacun onze cents pièces d’argent. »
${}^{6}Dalila dit à Samson : « Explique-moi, je t’en prie, d’où vient ta grande force, et comment tu devrais être ligoté pour qu’on te maîtrise. » 
${}^{7}Samson lui dit : « Si on me liait avec sept cordes d’arc neuves, qui n’ont pas été séchées, je perdrais ma vigueur, et je serais comme n’importe quel homme. » 
${}^{8}Les princes des Philistins firent apporter à Dalila sept cordes neuves qui n’avaient pas été séchées, et Dalila le lia avec ces cordes. 
${}^{9}Des hommes étaient embusqués dans sa chambre. Elle lui cria : « Les Philistins sont sur toi, Samson ! » Celui-ci rompit les cordes qui enserraient ses bras, comme se rompt un cordon d’étoupe à l’approche du feu. On ne découvrit donc pas le secret de sa force.
${}^{10}Dalila dit alors à Samson : « Tu t’es moqué de moi ; tu as menti. Révèle-moi maintenant comment tu devrais être ligoté. » 
${}^{11}Il lui répondit : « Si on me liait avec des cordes neuves et non travaillées, je perdrais ma vigueur, et je serais comme n’importe quel homme. » 
${}^{12}Dalila le lia avec des cordes neuves, puis elle lui cria : « Les Philistins sont sur toi, Samson ! » Des hommes étaient embusqués dans sa chambre ; mais il rompit les cordes qui lui enserraient les bras comme si c’était du fil.
${}^{13}Dalila dit encore à Samson : « Jusqu’ici, tu t’es moqué de moi, et tu m’as menti. Révèle-moi comment tu devrais être ligoté ! » Samson lui dit : « Si tu tissais les sept tresses de ma chevelure avec la chaîne d’un tissu, et si tu les resserrais avec un peigne de tisserand, alors je perdrais ma vigueur, et je serais comme n’importe quel homme. » 
${}^{14}Elle le laissa s’endormir, tissa les tresses de sa chevelure avec la chaîne, les resserra avec le peigne, puis elle lui cria : « Les Philistins sont sur toi, Samson ! » Samson s’éveilla, et il arracha le peigne, la navette et la chaîne.
${}^{15}Dalila lui dit alors : « Comment peux-tu me dire : “Je t’aime”, alors que tu ne m’ouvres pas ton cœur ! Voici trois fois que tu te joues de moi. Tu ne m’as pas révélé d’où vient ta grande force ! » 
${}^{16}Tous les jours, elle le harcelait, répétant les mêmes paroles. Samson, excédé à en mourir, 
${}^{17}lui ouvrit tout son cœur. Il lui dit : « Le rasoir n’a jamais passé sur ma tête, car je suis voué à Dieu depuis le sein de ma mère. Si j’étais rasé, je perdrais toute ma vigueur, et je serais comme n’importe quel homme. » 
${}^{18}Dalila vit qu’il lui avait ouvert tout son cœur, et elle fit appeler les princes des Philistins en leur disant : « Venez, car cette fois, il m’a ouvert tout son cœur. » Les princes des Philistins se rendirent chez elle, avec l’argent en main. 
${}^{19}Elle le laissa s’endormir sur ses genoux, et elle fit appel à un homme qui rasa les sept tresses de sa chevelure. Alors, il commença à faiblir, et sa vigueur l’abandonna. 
${}^{20}Dalila lui cria : « Les Philistins sont sur toi, Samson ! » Il s’éveilla et dit : « J’en sortirai comme les autres fois et je me dégagerai. » Mais il ne savait pas que le Seigneur s’était éloigné de lui. 
${}^{21}Les Philistins le saisirent et lui crevèrent les yeux ; ils l’emmenèrent à Gaza et le lièrent avec une double chaîne de bronze. Samson tournait une meule dans sa prison. 
${}^{22}Mais, après qu’il eût été rasé, ses cheveux recommencèrent à pousser.
${}^{23}Les princes des Philistins se réunirent pour offrir un grand sacrifice à Dagone, leur dieu, et se livrer à des réjouissances. Ils disaient : « Notre dieu a livré entre nos mains Samson, notre ennemi. » 
${}^{24}Dès que le peuple le vit, il loua son dieu et l’acclama en disant :
        \\« Notre dieu a livré entre nos mains notre ennemi,
        \\celui qui dévastait notre pays
        \\et qui multipliait nos morts. »
${}^{25}Et comme leur cœur était joyeux, ils dirent : « Appelez Samson, et qu’il nous divertisse ! » On envoya chercher Samson dans sa prison, et il se livra à des bouffonneries devant eux, puis on le plaça entre les colonnes. 
${}^{26}Samson dit au garçon qui le tenait par la main : « Guide-moi et fais-moi toucher les colonnes sur lesquelles repose le temple, pour que je m’y appuie. » 
${}^{27}Le temple était rempli d’hommes et de femmes. Il y avait là tous les princes des Philistins et, sur la terrasse, environ trois mille hommes et femmes qui s’étaient divertis en regardant Samson. 
${}^{28}Il invoqua le Seigneur en disant : « Je t’en prie, Seigneur Dieu, souviens-toi de moi, rends-moi ma force encore une fois et que, d’un seul coup, je me venge des Philistins pour mes deux yeux. » 
${}^{29}Il tâta alors les deux colonnes du milieu, sur lesquelles reposait le temple, prit appui contre l’une avec son bras droit, et contre l’autre avec son bras gauche. 
${}^{30}Il s’écria : « Que je meure avec les Philistins ! » Puis il pesa de toutes ses forces, et l’édifice s’effondra sur les princes et sur tout le peuple qui se trouvait là. Ceux qu’il fit mourir en mourant furent plus nombreux que ceux qu’il avait fait mourir pendant sa vie. 
${}^{31}Ses frères et toute la maison de son père descendirent et l’emportèrent. Ils remontèrent et l’ensevelirent entre Coréa et Eshtaol, dans le tombeau de Manoah, son père. Samson avait jugé Israël pendant vingt ans.
      
         
      \bchapter{}
      \begin{verse}
${}^{1}Il y avait un homme de la montagne d’Éphraïm nommé Mikayehou. 
${}^{2}Il dit à sa mère : « Les onze cents pièces d’argent que l’on t’a dérobées, et pour lesquelles tu as proféré une malédiction que tu m’as répétée, les voici : C’est moi qui les avais prises ! » Sa mère dit : « Sois béni du Seigneur, mon fils ! » 
${}^{3}Il rendit les onze cents pièces d’argent à sa mère, qui lui dit : « En fait, j’avais consacré de moi-même cet argent au Seigneur pour mon fils, afin d’en faire une statue, une idole en métal fondu. Aussi vais-je maintenant te le rendre. » 
${}^{4}Lorsqu’il eut rendu l’argent à sa mère, elle prit deux cents pièces d’argent qu’elle donna au fondeur. Celui-ci en fit une statue, une idole en métal fondu, qui fut placée dans la maison de Mikayehou. 
${}^{5}Cet homme, Mika, avait un sanctuaire personnel. Il fit donc faire un éphod et des idoles domestiques, et il donna l’investiture à l’un de ses fils, qui devint son prêtre. 
${}^{6}À cette époque, il n’y avait pas de roi en Israël, et chacun faisait ce qui lui plaisait.
${}^{7}Il y avait un jeune homme de Bethléem, en Juda, appartenant au clan de Juda, qui était lévite et résidait là comme étranger. 
${}^{8}Il avait quitté la ville de Bethléem de Juda pour aller là où il trouverait à s’établir. Chemin faisant, il parvint, dans la montagne d’Éphraïm, à la maison de Mika. 
${}^{9}Mika lui demanda : « D’où viens-tu ? » Il lui répondit : « Je suis un lévite de Bethléem de Juda. Je me suis mis en route afin de trouver là où m’établir. » 
${}^{10}Alors Mika lui dit : « Reste avec moi ; tu seras pour moi un père et un prêtre. Je te donnerai dix pièces d’argent par an, tes vêtements et ta nourriture. » 
${}^{11}Le lévite consentit à se fixer chez lui, et il devint comme l’un de ses fils. 
${}^{12}Mika donna l’investiture au lévite, qui devint son prêtre et demeura dans sa maison. 
${}^{13}Mika se dit : « Maintenant, je sais que le Seigneur me fera du bien, puisque ce lévite est devenu mon prêtre. »
      
         
      \bchapter{}
      \begin{verse}
${}^{1}En ces jours-là, il n’y avait pas de roi en Israël. Or, en ces jours-là, la tribu de Dane se cherchait un territoire pour y habiter, car jamais jusque-là, elle n’en avait obtenu au milieu des tribus d’Israël. 
${}^{2}Les hommes de Dane envoyèrent donc de chez eux cinq hommes, appartenant à leur clan, hommes vaillants de Soréa et d’Eshtaol, pour espionner le pays et pour l’explorer. On leur dit : « Allez explorer le pays ! » Ils atteignirent, dans la montagne d’Éphraïm, la maison de Mika et y passèrent la nuit.
${}^{3}Comme ils s’approchaient de la maison de Mika, ils reconnurent la voix du jeune lévite, se dirigèrent de son côté et lui demandèrent : « Qui t’a fait venir ici ? Que fais-tu là ? Qu’est-ce qui t’y retient ? » 
${}^{4}Il répondit : « Mika m’a donné beaucoup d’avantages, il m’a engagé et je suis devenu son prêtre. » 
${}^{5}Ils lui demandèrent : « Consulte donc Dieu, pour que nous sachions si le voyage que nous entreprenons réussira. » 
${}^{6}Le prêtre leur répondit : « Allez en paix ! Le voyage que vous entreprenez est sous le regard du Seigneur. »
${}^{7}Les cinq hommes s’en allèrent et arrivèrent à Laïsh. Ils virent que les gens qui y habitaient vivaient en sécurité, tranquilles et confiants, à la manière des Sidoniens, personne ne blâmant dans le pays le détenteur du pouvoir ; ces gens étaient loin des Sidoniens et ils n’avaient affaire avec personne. 
${}^{8}Ils revinrent alors vers leurs frères, à Soréa et à Eshtaol, et leurs frères leur demandèrent : « Que nous rapportez-vous ? » 
${}^{9}Ils répondirent : « Levons-nous ! Montons contre eux : nous avons vu leur pays. C’est un excellent pays. Et vous demeurez muets ! Mettez-vous en marche sans hésiter, pour aller conquérir ce territoire ! 
${}^{10}En arrivant, vous trouverez un peuple confiant. Le pays est étendu. Dieu l’a mis entre vos mains, ce lieu où rien ne manque de ce que l’on peut avoir sur la terre ! »
${}^{11}Six cents hommes équipés d’armes de guerre partirent donc de là, du clan de Dane, de Soréa et d’Eshtaol. 
${}^{12}Ils montèrent camper à Qiryath-Yearim, en Juda. Voilà pourquoi, jusqu’à ce jour, on appelle ce lieu Mahané-Dane (c’est-à-dire : Camp de Dane). Il est situé à l’ouest deQiryath-Yearim. 
${}^{13}De là, ils passèrent dans la montagne d’Éphraïm et atteignirent la maison de Mika. 
${}^{14}Les cinq hommes qui étaient allés espionner le pays de Laïsh prirent la parole et dirent à leurs frères : « Savez-vous qu’il y a ici, dans ces maisons, un éphod, des idoles domestiques, une statue, une idole en métal fondu ? Et maintenant, vous savez ce que vous avez à faire ! » 
${}^{15}Ils firent un détour pour se rendre à la maison du jeune lévite, la maison de Mika, et ils le saluèrent. 
${}^{16}Les six cents hommes de Dane, équipés de leurs armes de guerre, prirent position à l’entrée de la porte. 
${}^{17}Les cinq hommes qui étaient allés espionner le pays montèrent, entrèrent là et prirent la statue, l’éphod, les idoles domestiques, l’idole en métal fondu. Le prêtre se tenait debout à l’entrée de la porte avec les six cents hommes équipés d’armes de guerre. 
${}^{18}Comme ils étaient entrés dans la maison de Mika, et avaient pris la statue, l’éphod, les idoles domestiques, l’idole en métal fondu, le prêtre leur demanda : « Que faites-vous ? 
${}^{19}– Tais-toi, lui dirent-ils, mets la main sur ta bouche et viens avec nous ! Sois pour nous un père et un prêtre ! Vaut-il mieux pour toi être prêtre dans la maison d’un seul homme, ou être prêtre d’une tribu ou d’un clan en Israël ? » 
${}^{20}Le prêtre se réjouit en son cœur, il prit l’éphod, les idoles domestiques, la statue, et se plaça au milieu de la troupe.
${}^{21}Ils reprirent leur chemin, et partirent en plaçant devant eux les enfants, le bétail et les bagages. 
${}^{22}Ils étaient déjà loin de la maison de Mika, quand les habitants des maisons voisines de celle de Mika s’ameutèrent et se mirent à poursuivre les fils de Dane. 
${}^{23}Comme ils interpellaient les fils de Dane, ceux-ci se retournèrent et dirent à Mika : « Qu’as-tu à ameuter ces gens ? » 
${}^{24}Il leur répondit : « Les dieux que je m’étais faits, vous les avez pris, ainsi que le prêtre, et vous partez. Que me reste-t-il ? Comment pouvez-vous me dire : “Qu’est-ce que tu as ?” » 
${}^{25} Les fils de Dane lui dirent : « Qu’on ne t’entende plus ! Sinon nos hommes exaspérés pourraient bien vous attaquer, et tu risquerais ta vie et celle de ta maison ! » 
${}^{26}Les fils de Dane poursuivirent leur chemin, et Mika, voyant qu’ils étaient les plus forts, fit demi-tour et revint chez lui.
${}^{27}Ainsi, après avoir pris les dieux que Mika avait faits, et le prêtre qui le servait, les fils de Dane marchèrent contre Laïsh, contre une population tranquille et confiante. Ils la passèrent au fil de l’épée et incendièrent la ville. 
${}^{28}Personne ne vint à son aide, car elle était loin de Sidon et n’avait affaire avec personne. La ville se trouve en effet dans la plaine qui s’étend vers Beth-Rehob. Ils rebâtirent la ville et s’y établirent. 
${}^{29}Ils nommèrent la ville « Dane », du nom de Dane leur père, qui était né de Jacob, mais à l’origine, le nom de cette ville était Laïsh. 
${}^{30}Les fils de Dane érigèrent pour eux la statue. Jonathan, fils de Guershom, fils de Moïse, puis ses fils furent prêtres de la tribu de Dane, jusqu’au jour de la déportation. 
${}^{31}Ils installèrent pour leur usage la statue que Mika avait faite. Elle demeura à Silo, aussi longtemps que la maison de Dieu y subsista.
      
         
      \bchapter{}
      \begin{verse}
${}^{1}En ces jours-là – il n’y avait pas de roi en Israël –, un lévite qui résidait aux confins de la montagne d’Éphraïm prit comme concubine une femme de Bethléem de Juda. 
${}^{2}Sa concubine se fâcha contre lui, puis elle le quitta et revint à la maison de son père, à Bethléem de Juda, où elle séjourna quatre mois entiers. 
${}^{3}Son mari partit la retrouver, pour parler à son cœur et la ramener. Il avait avec lui son serviteur et deux ânes. Sa concubine le fit entrer dans la maison de son père. Le père de la jeune femme le vit et, tout joyeux, vint à sa rencontre. 
${}^{4}Son beau-père, le père de la jeune femme, le retint, et le lévite resta chez lui pendant trois jours. Ils mangèrent, burent et passèrent la nuit en cet endroit. 
${}^{5}Le quatrième jour, ils se levèrent de bon matin et le lévite se disposait à partir quand le père de la jeune femme dit à son gendre : « Restaure-toi en mangeant un morceau de pain, vous partirez après. » 
${}^{6}Ils s’assirent tous deux, mangèrent et burent ensemble. Alors le père de la jeune femme dit à son gendre : « Je t’en prie, passe encore la nuit. Que ton cœur soit content ! » 
${}^{7}Comme l’homme se levait pour partir, il céda finalement à l’insistance de son beau-père et passa une autre nuit en cet endroit. 
${}^{8}Le cinquième jour, il se leva de bon matin pour partir, mais le père de la jeune femme lui dit : « Restaure-toi, je t’en prie, et attardez-vous jusqu’au déclin du jour. » Ils mangèrent tous deux. 
${}^{9}L’homme se préparait à partir avec sa concubine et son serviteur ; mais son beau-père lui dit : « Voici que le jour faiblit vers le soir. Passez donc la nuit ! Voici la tombée du jour. Passe la nuit ici et que ton cœur soit content ! Demain, de bon matin, vous prendrez la route, et tu regagneras ta tente. » 
${}^{10}Mais l’homme ne voulut pas passer la nuit. Il se leva et partit. Il arriva en vue de Jébus – c’est-à-dire Jérusalem. Il avait avec lui deux ânes bâtés, ainsi que sa concubine et son serviteur.
      
         
${}^{11}Quand ils furent près de Jébus, le jour avait beaucoup baissé. Le serviteur dit à son maître : « Allons ! Faisons un détour vers cette ville des Jébuséens ! Nous y passerons la nuit. » 
${}^{12}Son maître lui répondit : « Non, nous ne ferons pas un détour vers cette ville étrangère, où il n’y a aucun fils d’Israël ; nous pousserons jusqu’à Guibéa. 
${}^{13}Allons ! dit-il à son serviteur, rapprochons-nous d’une de ces localités, Guibéa ou Rama, pour y passer la nuit. » 
${}^{14}Poussant plus loin, ils s’en allèrent. Le soleil se couchait quand ils approchèrent de Guibéa de Benjamin. 
${}^{15}Ils firent un détour pour passer la nuit à Guibéa. Le lévite entra, s’assit sur la place, mais personne ne lui offrit l’hospitalité pour la nuit.
${}^{16}Voici qu’un vieillard, le soir venu, rentrait de son travail des champs. Il était originaire de la montagne d’Éphraïm, mais il résidait à Guibéa, dont les habitants étaient Benjaminites. 
${}^{17}Levant les yeux, il remarqua le voyageur sur la place de la ville. Il lui demanda : « Où vas-tu, et d’où viens-tu ? » 
${}^{18}L’homme lui répondit : « Partis de Bethléem de Juda, nous faisons route vers l’arrière-pays de la montagne d’Éphraïm. C’est de là que je suis originaire. Je me suis rendu à Bethléem de Juda et je retourne dans ma maison. Personne ne m’a offert l’hospitalité. 
${}^{19}Pourtant, nous avons de la paille et du fourrage pour nos ânes ; j’ai aussi du pain et du vin pour moi, pour ta servante et pour le jeune homme qui accompagne tes serviteurs. Nous ne manquons de rien ! » 
${}^{20}Le vieillard dit alors : « Sois en paix ; laisse-moi pourvoir à tous tes besoins, mais ne passe pas la nuit sur la place. » 
${}^{21}Il le fit entrer dans sa maison et donna du fourrage aux ânes. Les voyageurs se lavèrent les pieds, ils mangèrent et ils burent.
${}^{22}Pendant qu’ils se restauraient, des hommes de la ville, de vrais vauriens, cernèrent la maison. Ils frappèrent à coups redoublés contre la porte et dirent au vieillard, propriétaire de la maison : « Fais sortir l’homme qui est entré chez toi pour que nous le connaissions ! »
${}^{23}Le propriétaire de la maison alla au-devant d’eux et leur dit : « Non, mes frères, non, ne faites pas le mal ! Après que cet homme a été reçu dans ma maison, ne commettez pas cette infamie ! 
${}^{24}Voici ma fille, qui est vierge ; je vais la faire sortir. Abusez d’elle ! Faites avec elle ce qui vous semblera bon, mais ne commettez pas contre cet homme un acte infâme. » 
${}^{25}Les hommes de la ville ne voulurent pas l’écouter. Alors, le lévite saisit sa concubine et la leur amena dehors. Ils s’unirent à elle et s’en amusèrent toute la nuit, jusqu’au matin. Quand vint l’aurore, ils la relâchèrent.
${}^{26}Comme le matin approchait, la femme s’en vint tomber à l’entrée de la maison de l’homme chez qui était son mari, et elle resta là jusqu’à ce qu’il fît jour. 
${}^{27}Au petit matin, son mari se leva, ouvrit la porte de la maison, sortit pour reprendre sa route. Voici que sa concubine gisait à l’entrée de la maison, les mains sur le seuil ! 
${}^{28}« Lève-toi, lui dit-il, et partons ! » Il n’obtint pas de réponse. Il la mit sur son âne, partit et rentra chez lui. 
${}^{29}Une fois arrivé dans sa maison, il prit un couteau, saisit sa concubine, la dépeça, membre après membre, en douze morceaux, qu’il envoya dans tout le territoire d’Israël. 
${}^{30}Quiconque voyait cela disait : « Jamais ne s’est fait, jamais ne s’est vu un crime aussi affreux, depuis le jour où les fils d’Israël sont montés du pays d’Égypte jusqu’à ce jour ! » Le lévite avait donné cet ordre à ses messagers : « Vous parlerez ainsi à tous les hommes d’Israël : “Un crime aussi affreux a-t-il jamais été commis depuis le jour où les fils d’Israël sont montés du pays d’Égypte jusqu’à ce jour ? Réfléchissez, tenez conseil, prononcez-vous !” »
      
         
      \bchapter{}
      \begin{verse}
${}^{1}Tous les fils d’Israël sortirent donc et, comme un seul homme, la communauté s’assembla, depuis Dane jusqu’à Bershéba, y compris le pays de Galaad, auprès du Seigneur, à Mispa. 
${}^{2}Les chefs de tout le peuple et toutes les tribus d’Israël se rendirent à l’assemblée du peuple de Dieu : quatre cent mille fantassins, sachant tirer l’épée. 
${}^{3}Et les fils de Benjamin apprirent que les fils d’Israël étaient montés à Mispa.
      Les fils d’Israël demandèrent : « Racontez comment ce crime a été commis ! » 
${}^{4}Le lévite, le mari de la femme qui avait été assassinée, prit la parole et dit : « J’étais venu avec ma concubine à Guibéa de Benjamin pour y passer la nuit. 
${}^{5}Les notables de Guibéa se sont ligués contre moi et ont cerné pendant la nuit la maison où je me trouvais. Ils avaient résolu de me tuer. Ils ont fait violence à ma concubine, elle en est morte. 
${}^{6}Alors, j’ai pris ma concubine, je l’ai coupée en morceaux que j’ai envoyés dans toute l’étendue de l’héritage d’Israël, car on avait commis en Israël un acte scandaleux et infâme. 
${}^{7}Vous voici tous, fils d’Israël ! Délibérez et prenez une décision ici même. » 
${}^{8}Tout le peuple se leva comme un seul homme en disant : « Aucun d’entre nous ne reviendra à sa tente, aucun d’entre nous ne regagnera sa maison. 
${}^{9}Et maintenant, voilà ce que nous allons faire contre Guibéa : nous tirerons au sort ; 
${}^{10}nous prendrons, dans toutes les tribus d’Israël, dix hommes sur cent, cent sur mille et mille sur dix mille. Ils procureront des vivres aux gens, à ceux qui iront punir Guibéa de Benjamin pour toute l’infamie qu’elle a commise en Israël. » 
${}^{11}Tous les hommes d’Israël, associés comme un seul homme, s’unirent contre la ville.
${}^{12}Les tribus d’Israël envoyèrent des messagers dans toute la tribu de Benjamin pour dire : « Quel est ce crime qui a été commis chez vous ? 
${}^{13}Maintenant, livrez ces hommes, ces vauriens qui habitent Guibéa. Nous les mettrons à mort, et nous ôterons ainsi le mal d’Israël. » Les fils de Benjamin ne voulurent pas écouter la voix de leurs frères, les fils d’Israël.
${}^{14}Les fils de Benjamin quittèrent leurs villes et se réunirent à Guibéa, pour faire la guerre aux fils d’Israël. 
${}^{15}Ce jour-là, on recensa les fils de Benjamin qui étaient venus des villes : vingt-six mille hommes sachant tirer l’épée, sans compter les habitants de Guibéa, sept cents hommes d’élite, dénombrés à part. 
${}^{16}Dans cette troupe, il y avait sept cents hommes d’élite gauchers. Chacun d’eux pouvait, avec la pierre de sa fronde, lancer une pierre sur un cheveu sans le manquer. 
${}^{17}Les fils d’Israël furent également dénombrés, sans compter Benjamin : ils étaient quatre cent mille hommes sachant tirer l’épée, tous hommes de guerre. 
${}^{18}Ils se mirent en route et montèrent à Béthel pour consulter Dieu. Les fils d’Israël demandèrent : « Qui de nous montera le premier à l’attaque contre les fils de Benjamin ? » Le Seigneur répondit : « C’est Juda qui montera le premier. »
${}^{19}Au matin, les fils d’Israël partirent et ils installèrent leur camp près de Guibéa. 
${}^{20}Les hommes d’Israël sortirent combattre Benjamin et se rangèrent en bataille devant Guibéa. 
${}^{21}Les fils de Benjamin sortirent de Guibéa, et ce jour-là, ils firent mordre la poussière à vingt-deux mille hommes d’Israël. 
${}^{22}Mais le peuple, les hommes d’Israël, reprirent courage, et de nouveau, ils se rangèrent en bataille au même endroit que le premier jour. 
${}^{23}Les fils d’Israël vinrent pleurer devant le Seigneur, jusqu’au soir. Ils consultèrent le Seigneur en disant : « Devons-nous recommencer à combattre les fils de Benjamin, notre frère ? » Le Seigneur répondit : « Montez contre lui. »
${}^{24}Le deuxième jour, les fils d’Israël s’approchèrent des fils de Benjamin. 
${}^{25}Les fils de Benjamin sortirent de Guibéa à leur rencontre ce deuxième jour, et ils firent encore mordre la poussière à dix-huit mille hommes parmi les fils d’Israël, tous sachant tirer l’épée. 
${}^{26}Tous les fils d’Israël et tout le peuple montèrent à Béthel. Là, ils pleurèrent, assis devant le Seigneur ; ils jeûnèrent ce jour-là jusqu’au soir ; ils offrirent des holocaustes et des sacrifices de paix devant le Seigneur. 
${}^{27}Ils le consultèrent, car l’arche de l’Alliance de Dieu s’y trouvait en ces jours-là. 
${}^{28}En ces jours-là, en effet, Pinhas, fils d’Éléazar, fils d’Aaron, se tenait devant elle. Ils demandèrent : « Devons-nous recommencer à combattre les fils de Benjamin, notre frère, ou devons-nous y renoncer ? » Le Seigneur répondit : « Montez, car demain, je les livrerai entre vos mains. »
${}^{29}Israël plaça des hommes en embuscade autour de Guibéa 
${}^{30}et, le troisième jour, les fils d’Israël montèrent contre les fils de Benjamin et se rangèrent devant Guibéa, comme les autres fois. 
${}^{31}Les fils de Benjamin sortirent à la rencontre du peuple et se laissèrent attirer loin de la ville. Ils commencèrent, comme les autres fois, à faire des victimes parmi le peuple, environ trente hommes d’Israël, sur les routes qui mènent, l’une à Béthel, l’autre à Guibéa, en rase campagne. 
${}^{32}Les fils de Benjamin se dirent : « Les voilà battus devant nous comme précédemment ! » Mais les fils d’Israël s’étaient dit : « Nous allons fuir et les attirer loin de la ville sur les routes. » 
${}^{33}Ils quittèrent tous leurs positions, se mirent en ordre de bataille à Baal-Tamar, tandis que l’embuscade d’Israël surgissait de la position qu’elle occupait, à l’ouest de Guéba. 
${}^{34}Dix mille hommes d’élite, pris dans tout Israël, arrivèrent en face de Guibéa. La bataille fut acharnée. Les fils de Benjamin ne savaient pas que le malheur allait s’abattre sur eux. 
${}^{35}Le Seigneur frappa Benjamin devant Israël. Ce jour-là, les fils d’Israël firent périr vingt-cinq mille cent hommes de Benjamin, tous sachant tirer l’épée. 
${}^{36}Les fils de Benjamin virent qu’ils étaient battus.
      Les hommes d’Israël cédèrent du terrain à Benjamin, car ils comptaient sur l’embuscade qu’ils avaient tendue contre Guibéa. 
${}^{37}Ceux de l’embuscade s’élancèrent rapidement sur Guibéa, se déployèrent et passèrent toute la ville au fil de l’épée. 
${}^{38}Or, il y avait une convention entre les hommes d’Israël et ceux de l’embuscade : ces derniers devaient faire monter de la ville, en guise de signal, un panache de fumée. 
${}^{39}Alors, les hommes d’Israël engagés dans le combat feraient volte-face. Benjamin avait déjà abattu une trentaine d’hommes parmi eux, et se disait : « Les voilà encore battus comme lors du combat précédent ! » 
${}^{40}Mais le signal, une colonne de fumée, avait commencé à s’élever de la ville, et Benjamin se retourna : voici que la ville tout entière montait en feu vers le ciel. 
${}^{41}Alors, les hommes d’Israël firent volte-face, et les hommes de Benjamin furent épouvantés. Ils virent que le malheur les avait frappés. 
${}^{42}Ils s’enfuirent devant les hommes d’Israël en direction du désert, mais les combattants les talonnaient, et ceux qui venaient de la ville, les prenant à revers, les massacraient. 
${}^{43}Benjamin fut cerné, poursuivi sans répit, écrasé en face de Guibéa, du côté du soleil levant. 
${}^{44}Dix-huit mille hommes de Benjamin tombèrent, tous hommes de valeur. 
${}^{45}Les autres tournèrent le dos et s’enfuirent au désert, vers le roc de Rimmone. Cinq mille hommes furent exécutés sur les routes, on en poursuivit jusqu’à Guidéom et, de ceux-ci, on en tua deux mille.
${}^{46}Le total des Benjaminites qui tombèrent ce jour-là fut de vingt-cinq mille hommes sachant tirer l’épée, tous hommes de valeur. 
${}^{47}Les six cents hommes qui s’étaient enfuis au désert, vers le roc de Rimmone, y restèrent quatre mois. 
${}^{48}Les hommes d’Israël revinrent vers les fils de Benjamin, les passèrent au fil de l’épée : la population des villes, avec le bétail et tout ce qui s’y trouvait. Ils incendièrent aussi toutes les villes qui se trouvaient sur leur chemin.
      
         
      \bchapter{}
      \begin{verse}
${}^{1}Les hommes d’Israël avaient fait ce serment à Mispa : « Aucun d’entre nous ne donnera sa fille pour femme à un Benjaminite. » 
${}^{2}Le peuple se rendit à Béthel et là, il se tint assis devant Dieu, jusqu’au soir. Ils poussèrent des cris, sanglotèrent, 
${}^{3}en disant : « Pourquoi, Seigneur, Dieu d’Israël, se fait-il qu’il manque aujourd’hui à Israël une de ses tribus ? » 
${}^{4}Le lendemain, de bon matin, le peuple se leva, bâtit un autel, offrit des holocaustes et des sacrifices de paix. 
${}^{5}Les fils d’Israël se demandèrent : « Quelle est celle, parmi toutes les tribus d’Israël, qui n’est pas montée à l’assemblée, auprès du Seigneur ? » Car ils avaient fait ce serment solennel : ceux qui ne seraient pas montés à Mispa auprès du Seigneur seraient mis à mort. 
${}^{6}Les fils d’Israël furent pris de pitié pour Benjamin, leur frère, et ils dirent : « Aujourd’hui, une des tribus a été retranchée d’Israël. 
${}^{7}Que ferons-nous pour donner des femmes à ceux qui restent, alors que nous avons fait serment, par le Seigneur, de ne pas leur donner nos filles pour femmes ? »
      
         
${}^{8}Ils demandèrent alors : « Y a-t-il quelqu’un, parmi les tribus d’Israël, qui n’est pas monté auprès du Seigneur à Mispa ? » Voici que, de Yabesh-de-Galaad, personne n’était venu au camp, à l’assemblée. 
${}^{9}La population avait été recensée, et en effet, il n’y avait là aucun des habitants de Yabesh-de-Galaad. 
${}^{10}La communauté y envoya douze mille hommes de valeur, et leur donna cet ordre : « Allez, et passez au fil de l’épée les habitants de Yabesh-de-Galaad, y compris les femmes et les enfants. 
${}^{11}Voici ce que vous ferez : tout mâle et toute femme ayant partagé la couche d’un homme, vous les vouerez à l’anathème. » 
${}^{12}Ils trouvèrent, parmi les habitants de Yabesh-de-Galaad, quatre cents jeunes filles vierges, qui n’avaient partagé la couche d’aucun homme, et ils les amenèrent au camp, à Silo, qui est au pays de Canaan. 
${}^{13}Toute la communauté envoya des messagers aux fils de Benjamin qui s’étaient réfugiés au roc de Rimmone, et ils leur proposèrent la paix. 
${}^{14}Les Benjaminites revinrent alors, et on leur donna les femmes laissées en vie parmi celles de Yabesh-de-Galaad, mais ils n’en trouvèrent pas assez pour eux.
${}^{15}Le peuple fut pris de pitié pour Benjamin, car le Seigneur avait fait une brèche parmi les tribus d’Israël. 
${}^{16}Les anciens de la communauté dirent alors : « Que ferons-nous pour que ceux qui restent aient des femmes, puisque les femmes de la tribu de Benjamin ont été exterminées ? » 
${}^{17}Ils se demandaient : « Comment Benjamin peut-il avoir une postérité, pour qu’une tribu ne soit pas effacée d’Israël ? 
${}^{18}Car nous ne pouvons leur donner des femmes parmi nos filles. » En effet, les fils d’Israël avaient prêté ce serment : « Maudit soit celui qui donnera une femme à Benjamin. » 
${}^{19}Mais ils se dirent : « Il y a chaque année une fête du Seigneur à Silo. » – La ville se trouve au nord de Béthel, à l’orient de la route qui va de Béthel à Sichem, et au sud de Lébona.
${}^{20}Ils donnèrent ces instructions aux fils de Benjamin :
      « Allez vous embusquer dans les vignes. 
${}^{21}Vous guetterez et, quand les filles de Silo sortiront pour danser en chœur, vous sortirez des vignes, vous vous emparerez chacun d’une femme parmi les filles de Silo, et vous vous en irez au pays de Benjamin. 
${}^{22}Et si leurs pères ou leurs frères viennent protester auprès de nous, nous leur dirons : “Pour elles, faites-nous grâce, car nous n’avons pu prendre des femmes pendant le combat, et vous, vous auriez été coupables si vous les leur aviez données.” » 
${}^{23}Les fils de Benjamin agirent ainsi. Ils prirent un nombre de femmes égal au leur parmi les danseuses qu’ils avaient enlevées, puis ils partirent et revinrent dans leur héritage. Ils rebâtirent leurs villes et y habitèrent.
${}^{24}À ce moment-là, les fils d’Israël s’en allèrent, chacun vers sa tribu et vers son clan ; ils s’en retournèrent de là chacun dans son héritage. 
${}^{25}En ces jours-là, il n’y avait pas de roi en Israël. Chacun faisait ce qui lui plaisait.
