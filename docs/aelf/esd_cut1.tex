  
  
    
    \bbook{ESDRAS}{ESDRAS}
      
         
      \bchapter{}
      \begin{verse}
${}^{1}La première année du règne\\de Cyrus, roi de Perse, pour que soit accomplie la parole du Seigneur proclamée par Jérémie, le Seigneur inspira Cyrus, roi de Perse. Et celui-ci fit publier dans tout son royaume – et même consigner par écrit : 
${}^{2}« Ainsi parle Cyrus, roi de Perse : Le Seigneur, le Dieu du ciel, m’a donné tous les royaumes de la terre ; et il m’a chargé de lui bâtir une maison à Jérusalem, en Juda. 
${}^{3}Quiconque parmi vous fait partie de\\son peuple, que son Dieu soit avec lui, qu’il monte à Jérusalem, en Juda, et qu’il bâtisse la maison du Seigneur, le Dieu d’Israël, le Dieu qui est à Jérusalem. 
${}^{4}En tout lieu où résident ceux qui restent d’Israël\\, que la population leur vienne en aide : qu’on leur fournisse argent, or, dons en nature, bétail, qu’on y joigne des offrandes volontaires pour la maison de Dieu qui est à Jérusalem. »
${}^{5}Alors les chefs de famille de Juda et de Benjamin, les prêtres et les Lévites, bref, tous ceux à qui Dieu avait inspiré cette décision\\, se mirent en route et montèrent à Jérusalem pour bâtir la maison du Seigneur ; 
${}^{6}tous leurs voisins leur apportèrent de l’aide : argent, or, dons en nature, bétail, objets précieux en quantité, sans compter toutes sortes d’offrandes volontaires.
${}^{7}Le roi Cyrus fit retirer les objets de la maison du Seigneur que Nabucodonosor avait enlevés de Jérusalem pour les mettre dans la maison de son dieu. 
${}^{8}Cyrus, roi de Perse, les fit retirer par le soin du trésorier Mithridate qui en fit le compte pour Sheshbassar, le prince de Juda. 
${}^{9}En voici le compte : 30 plats d’or, 1 000 plats d’argent, 29 couteaux, 
${}^{10}30 coupes en or, 410 coupes en argent de deuxième ordre, ainsi que 1 000 autres objets. 
${}^{11}En tout, il y avait 5 400 objets d’or et d’argent. Sheshbassar emporta le tout au moment où les exilés montèrent de Babylone à Jérusalem.
      
         
      \bchapter{}
      \begin{verse}
${}^{1}Voici les gens de la province de Juda qui sont remontés de la captivité et de l’exil – Nabucodonosor, roi de Babylone, les avait déportés à Babylone. Ils retournèrent à Jérusalem et à Juda, chacun dans sa ville ; 
${}^{2}ils vinrent avec Zorobabel, Josué, Néhémie, Seraya, Reélaya, Mardochée, Bilshane, Mispar, Bigwaï, Rehoum, Baana.
      Voici le nombre des hommes du peuple d’Israël : 
${}^{3}les fils de Paréosh : 2 172, 
${}^{4}les fils de Shefatya : 372, 
${}^{5}les fils d’Arah : 775, 
${}^{6}les fils de Pahath-Moab, c’est-à-dire les fils de Josué Joab : 2 812 ; 
${}^{7}les fils d’Élam : 1 254 ; 
${}^{8}les fils de Zattou : 945 ; 
${}^{9}les fils de Zakkaï : 760 ; 
${}^{10}les fils de Bani : 642 ; 
${}^{11}les fils de Bébaï : 623 ; 
${}^{12}les fils d’Azgad : 1 222 ; 
${}^{13}les fils d’Adoniqam : 666 ; 
${}^{14}les fils de Bigwaï : 2 056 ; 
${}^{15}les fils d’Adine : 454 ; 
${}^{16}les fils d’Ater de Yehizkiya : 98 ; 
${}^{17}les fils de Bésaï : 323 ; 
${}^{18}les fils de Yora : 112 ; 
${}^{19}les fils de Hashoum : 223 ; 
${}^{20}les fils de Guibbar : 95 ; 
${}^{21}les fils de Bethléem : 123 ; 
${}^{22}hommes de Netofa : 56 ; 
${}^{23}hommes d’Anatoth : 128 ; 
${}^{24}les fils d’Azmaveth : 42 ; 
${}^{25}les fils de Qiryath-Yearim, Kefira et Beéroth : 743 ; 
${}^{26}les fils de Harama et Guéba : 621 ; 
${}^{27}hommes de Mikmas : 122 ; 
${}^{28}hommes de Béthel et d’Aï : 223 ; 
${}^{29}les fils de Nébo : 52 ; 
${}^{30}les fils de Magbish : 156 ; 
${}^{31}les fils d’un autre Élam : 1 254 ; 
${}^{32}les fils de Harim : 320 ; 
${}^{33}les fils de Lod, Hadid et Ono : 725 ; 
${}^{34}les fils de Jéricho : 345 ; 
${}^{35}les fils de Senaa : 3 630.
${}^{36}Les prêtres : les fils de Yedaya, de la maison de Josué : 973 ; 
${}^{37}les fils d’Immer : 1 052 ; 
${}^{38}les fils de Pashehour : 1 247 ; 
${}^{39}les fils de Harim : 1 017.
${}^{40}Les Lévites : les fils de Josué et ceux de Qadmiel, c’est-à-dire les fils de Hodavya : 74.
${}^{41}Les chantres : les fils d’Asaph : 128.
${}^{42}Les fils des portiers : les fils de Shalloum, les fils d’Ater, les fils de Talmone, les fils d’Akkoub, les fils de Hatita, les fils de Shobaï, en tout : 139.
${}^{43}Les servants : les fils de Siha, les fils de Hasoufa, les fils de Tabbaoth, 
${}^{44}les fils de Quéros, les fils de Siaha, les fils de Padone, 
${}^{45}les fils de Lebana, les fils de Hagaba, les fils d’Akkoub, 
${}^{46}les fils de Hagab, les fils de Shalmaï, les fils de Hanane, 
${}^{47}les fils de Guiddel, les fils de Gahar, les fils de Reaya, 
${}^{48}les fils de Recine, les fils de Neqoda, les fils de Gazzam, 
${}^{49}les fils de Ouzza, les fils de Passéah, les fils de Béçaï, 
${}^{50}les fils d’Asna, les fils de Meounim, les fils de Nefoussim, 
${}^{51}les fils de Bakbouk, les fils de Hakoufa, les fils de Harhour, 
${}^{52}les fils de Baslouth, les fils de Mehida, les fils de Harsha, 
${}^{53}les fils de Barkos, les fils de Sissera, les fils de Tamah, 
${}^{54}les fils de Neciah, les fils de Hatifa.
${}^{55}Les fils des serviteurs de Salomon : les fils de Sotaï, les fils de Hassofèreth, les fils de Perouda, 
${}^{56}les fils de Yaala, les fils de Darqone, les fils de Guiddel, 
${}^{57}les fils de Shefatya, les fils de Hattil, les fils de Pokèreth-Hassebaïm, les fils d’Ami. 
${}^{58}Tous les servants et les fils des serviteurs de Salomon étaient donc au nombre de 392.
${}^{59}Et voici ceux qui sont montés de Tel-Mélah, Tel-Harsha, Keroub-Addane, Immer, mais qui n’ont pu établir que leur famille et leur race étaient bien d’Israël : 
${}^{60}les fils de Delaya, les fils de Tobie, les fils de Neqoda : 652 ; 
${}^{61}de même que certains d’entre les prêtres : les fils de Hobaya, les fils de Haqqos, les fils de Barzillaï, celui qui avait pris pour femme l’une des filles de Barzillaï le Galaadite et avait été appelé de ce nom. 
${}^{62}Ceux-là recherchèrent leur registre généalogique, mais on ne le trouva pas. Ils furent alors déclarés impurs et exclus du sacerdoce. 
${}^{63}Et le gouverneur leur interdit de manger des aliments très saints, jusqu’à ce qu’un prêtre se tienne devant Dieu pour le consulter par les Ourim et les Toummim.
${}^{64}L’assemblée tout entière était de 42 360 personnes, 
${}^{65}sans compter leurs serviteurs et leurs servantes, au nombre de 7 337. Il y avait aussi 200 chanteurs et chanteuses. 
${}^{66}Leurs chevaux étaient au nombre de 736, leurs mulets de 245, 
${}^{67}leurs chameaux de 435, les ânes de 6 720.
${}^{68}À leur arrivée à la maison du Seigneur qui est à Jérusalem, certains chefs de famille firent des offrandes volontaires pour la maison de Dieu, afin qu’elle soit rétablie à son emplacement. 
${}^{69}Selon leurs possibilités, ils versèrent au trésor de l’œuvre 61 000 pièces d’or et 5 000 pièces d’argent, et aussi 100 tuniques sacerdotales. 
${}^{70}Alors s’installèrent dans leurs villes prêtres, Lévites, une partie du peuple, chantres et portiers ainsi que les servants. Voilà donc tout Israël dans ses villes.
      
         
      \bchapter{}
      \begin{verse}
${}^{1}Quand arriva le septième mois, les fils d’Israël étant dans leurs villes, le peuple se réunit comme un seul homme à Jérusalem. 
${}^{2}Alors Josué, fils de Yosadaq, se leva avec ses frères les prêtres, ainsi que Zorobabel, fils de Salathiel, avec ses frères, et ils bâtirent l’autel du Dieu d’Israël, pour y offrir des holocaustes, selon ce qui est écrit dans la loi de Moïse, l’homme de Dieu. 
${}^{3}Ils érigèrent l’autel sur ses fondations, terrifiés qu’ils étaient par les peuples des autres pays, et ils y offrirent des holocaustes pour le Seigneur, ceux du matin et ceux du soir. 
${}^{4}Puis ils célébrèrent la fête des Tentes, comme cela est écrit : ils offraient l’holocauste jour après jour, selon le nombre fixé par la coutume pour chaque jour, 
${}^{5}et après cela, l’holocauste perpétuel, les holocaustes pour les nouvelles lunes et pour toutes les solennités saintes du Seigneur, et les holocaustes pour tous ceux qui faisaient des offrandes volontaires au Seigneur.
${}^{6}Dès le premier jour du septième mois, ils commencèrent à offrir des holocaustes au Seigneur, et pourtant les fondations du temple du Seigneur n’étaient pas encore posées. 
${}^{7}Ils donnèrent de l’argent aux tailleurs de pierre et aux charpentiers ; ils donnèrent également des vivres, de la boisson et de l’huile aux Sidoniens et aux Tyriens pour qu’ils fassent venir du bois de cèdre depuis le Liban, par la mer jusqu’à Yaffo, selon l’autorisation reçue de Cyrus, roi de Perse. 
${}^{8}La deuxième année de leur arrivée à la maison de Dieu à Jérusalem, au deuxième mois, Zorobabel, fils de Salathiel, et Josué, fils de Yosadaq, avec le reste de leurs frères, les prêtres, les Lévites et tous ceux qui étaient revenus de la captivité à Jérusalem, se mirent à l’ouvrage ; et ils établirent les Lévites, ceux qui avaient vingt ans et plus, pour qu’ils surveillent les travaux de la maison du Seigneur. 
${}^{9}Alors Josué, ses fils et ses frères, Qadmiel et ses fils, les fils de Hodavya, ainsi que les fils de Hénadad, leurs fils et leurs frères, Lévites, se chargèrent, tous ensemble, de surveiller ceux qui travaillaient pour la maison de Dieu.
${}^{10}Alors les maçons posèrent les fondations du temple du Seigneur, tandis que prenaient place les prêtres revêtus de leurs ornements, avec les trompettes, puis les Lévites, fils d’Asaph, avec les cymbales, afin de louer le Seigneur selon les indications de David, roi d’Israël. 
${}^{11}Ils louaient le Seigneur et lui rendaient grâce, en chantant : « Il est bon, éternel est son amour pour Israël. » Tout le peuple faisait de grandes ovations en louant le Seigneur, à cause de la fondation de la maison du Seigneur. 
${}^{12}Beaucoup de prêtres, de Lévites et de chefs de famille parmi les plus âgés, qui avaient vu la première maison, pleuraient à chaudes larmes, tandis que sous leurs yeux on posait les fondations de cette maison-ci. Mais beaucoup élevaient la voix en joyeuses ovations. 
${}^{13}Le peuple ne pouvait distinguer entre le bruit des ovations joyeuses et le bruit des pleurs. Le peuple faisait de grandes ovations dont le bruit s’entendait de très loin.
      
         
      \bchapter{}
      \begin{verse}
${}^{1}Quand les adversaires de Juda et de Benjamin apprirent que les rapatriés construisaient un temple au Seigneur, le Dieu d’Israël, 
${}^{2}ils vinrent trouver Zorobabel et les chefs de famille, et leur dirent : « Nous voulons bâtir avec vous ! Comme vous, en effet, nous cherchons votre Dieu et lui offrons des sacrifices, depuis le temps d’Asarhaddone, roi d’Assour, celui qui nous a fait monter ici. » 
${}^{3}Zorobabel, Josué et le reste des chefs de famille d’Israël leur répondirent : « Ce n’est pas à vous et à nous de bâtir une maison pour notre Dieu ; c’est à nous seuls de la bâtir pour le Seigneur, le Dieu d’Israël, comme nous l’a ordonné le roi Cyrus, roi de Perse. »
${}^{4}Alors les gens du pays découragèrent le peuple de Juda et l’intimidèrent pour l’empêcher de construire. 
${}^{5}Ils soudoyèrent des conseillers contre le peuple pour faire échouer son projet, durant tout le temps de Cyrus, roi de Perse, et jusqu’au règne de Darius, roi de Perse.
${}^{6}Sous le règne de Xerxès, au commencement de son règne, ils rédigèrent une accusation contre les habitants de Juda et de Jérusalem. 
${}^{7}Au temps d’Artaxerxès, Bishlam, Mithridate, Tabéel, ainsi que leurs autres collègues écrivirent à Artaxerxès, roi de Perse. Le texte de la lettre était écrit en caractères araméens et traduit en cette langue.
      <div class="intertitle niv10 arameen">
        • TEXTE ARAMÉEN (4,8 – 6,18)
${}^{8}Le chancelier Rehoum et le secrétaire Shimshaï écrivirent la lettre suivante au roi Artaxerxès au sujet de Jérusalem : 
${}^{9}« Le chancelier Rehoum, le secrétaire Shimshaï, et le reste de leurs collègues, les gens de Dine, d’Afarsatak, de Tarpel, d’Afaras, d’Ourouk, de Babylone, de Suse, c’est-à-dire les Élamites, 
${}^{10}et les autres peuples que le grand et illustre Asnappar a déportés et a fait résider dans les villes de Samarie et dans le reste de la Transeuphratène, etc. »
${}^{11}Voici la copie de la lettre qu’ils envoyèrent : « Au roi Artaxerxès, tes serviteurs, gens de Transeuphratène, etc. 
${}^{12}Qu’il soit porté à la connaissance du roi que les Juifs, qui sont montés de chez toi pour venir vers nous à Jérusalem, reconstruisent la ville rebelle et mauvaise ; ils relèvent les murs et posent les fondations. 
${}^{13}Maintenant, qu’il soit porté à la connaissance du roi que si cette ville est reconstruite et les murs relevés, alors ils ne verseront plus de tribut, ni d’impôt, ni de droit de passage, ce qui finalement portera préjudice aux rois. 
${}^{14}Or, parce que nous mangeons le sel du palais et qu’il nous paraît inacceptable de voir le roi tourné en dérision, nous envoyons au roi des informations à ce sujet, 
${}^{15}pour qu’on fasse des recherches dans le livre des Mémoires de tes pères. Dans ce livre des Mémoires, tu trouveras et tu apprendras que cette ville est une ville rebelle, qu’elle porte préjudice aux rois et aux provinces, et que depuis les temps anciens on y a fomenté des révoltes. C’est à cause de cela que cette ville a été détruite. 
${}^{16}Nous faisons connaître au roi que si cette ville est reconstruite, si les murs en sont relevés, dès lors tu n’auras plus de possessions en Transeuphratène. »
${}^{17}Le roi envoya cette réponse : « Au chancelier Rehoum, au secrétaire Shimshaï et à leurs autres collègues qui habitent à Samarie et dans le reste de la Transeuphratène : Paix ! Maintenant donc, 
${}^{18}l’acte officiel que vous nous avez envoyé a été lu mot à mot devant moi. 
${}^{19}Sur mon ordre, on a fait des recherches et on a découvert que, depuis les temps anciens, cette ville se soulève contre les rois et qu’elle est travaillée par la révolte et la sédition. 
${}^{20}Il y eut à Jérusalem des rois puissants qui dominèrent toute la Transeuphratène ; on leur versait tribut, impôt et droit de passage. 
${}^{21}C’est pourquoi donnez l’ordre d’arrêter le travail de ces gens ; que cette ville ne soit pas rebâtie jusqu’à ce que j’en aie donné l’ordre. 
${}^{22}Gardez-vous d’agir avec négligence en cette affaire, de peur que le mal ne s’accroisse au préjudice des rois. »
${}^{23}Dès que la copie de l’acte officiel du roi Artaxerxès fut lue devant Rehoum et le secrétaire Shimshaï ainsi que leurs collègues, ils allèrent en toute hâte à Jérusalem auprès des Juifs et ils leur firent arrêter le travail par la force et la violence. 
${}^{24}Alors on arrêta le travail de la maison de Dieu à Jérusalem. Il fut interrompu jusqu’à la deuxième année du règne de Darius, roi de Perse.
      
         
      \bchapter{}
      \begin{verse}
${}^{1}Les prophètes Aggée et Zacharie, le fils d’Iddo, prophétisèrent pour les Juifs qui vivaient en Juda et à Jérusalem, au nom du Dieu d’Israël qui était sur eux. 
${}^{2}Alors Zorobabel, fils de Salathiel, et Josué, fils de Yosadaq, se levèrent et se mirent à bâtir la maison de Dieu à Jérusalem ; avec eux, il y avait les prophètes de Dieu qui les aidaient. 
${}^{3}Mais à ce moment-là, Tatnaï, gouverneur de Transeuphratène, Shetar-Boznaï et leurs collègues vinrent vers eux et leur dirent : « Qui vous a donné l’ordre de rebâtir cette Maison, de relever cette construction ? » 
${}^{4}Et ils ajoutèrent : « Quels sont les noms des hommes qui bâtissent cet édifice ? » 
${}^{5}Mais les anciens des Juifs étaient sous le regard de leur Dieu : on ne leur fit pas arrêter le travail jusqu’à ce que le rapport parvienne chez Darius et que revienne ensuite l’acte officiel à ce sujet.
${}^{6}Copie de la lettre envoyée au roi Darius par Tatnaï, gouverneur de Transeuphratène, Shetar-Boznaï et ses collègues, les fonctionnaires de Transeuphratène. 
${}^{7}Ils lui envoyèrent un message où il était écrit : « Au roi Darius, toute paix ! 
${}^{8}Qu’il soit porté à la connaissance du roi que nous sommes allés dans la province de Juda, à la maison du grand Dieu. Elle se construit en pierres de taille, et des madriers sont placés dans les murs. Ce travail s’accomplit avec soin et il progresse par leurs mains. 
${}^{9}Alors, nous avons questionné ces anciens et nous leur avons dit : “Qui vous a donné l’ordre de rebâtir cette Maison, de relever cette construction ?” 
${}^{10}Nous leur avons aussi demandé leurs noms pour te les faire connaître ; nous avons ainsi écrit les noms des hommes qui sont à leur tête. »
${}^{11}Voici la réponse qui nous parvint : « Nous sommes les serviteurs du Dieu du ciel et de la terre, et nous rebâtissons la Maison construite jadis, il y a de longues années. Un grand roi d’Israël l’avait construite et achevée. 
${}^{12}Mais après que nos pères irritèrent le Dieu du ciel, il les livra aux mains du Chaldéen Nabucodonosor, roi de Babylone ; celui-ci détruisit alors cette Maison et déporta le peuple à Babylone. 
${}^{13}Toutefois, la première année de Cyrus, roi de Babylone, le roi Cyrus donna l’ordre de bâtir cette maison de Dieu. 
${}^{14}De plus, les objets de la maison de Dieu, en or et en argent, ceux que Nabucodonosor avait fait enlever du sanctuaire de Jérusalem pour les emporter au sanctuaire de Babylone, le roi Cyrus les fit enlever du sanctuaire de Babylone et remettre au nommé Sheshbassar, celui qu’il avait établi gouverneur. 
${}^{15}Le roi lui dit : “Prends ces objets et va les déposer dans le sanctuaire de Jérusalem, et que la maison de Dieu soit rebâtie sur son emplacement.” 
${}^{16}Alors ce Sheshbassar est venu et a posé les fondations de la maison de Dieu à Jérusalem. Depuis ce moment et jusqu’à maintenant, elle est en construction mais elle n’est pas achevée. 
${}^{17}Maintenant donc, s’il plaît au roi, que l’on recherche, là-bas, dans la maison du trésor royal à Babylone, s’il y a bien eu un ordre donné par le roi Cyrus en vue de bâtir cette maison de Dieu à Jérusalem. Alors, qu’on nous envoie la décision du roi à ce sujet. »
      
         
      \bchapter{}
      \begin{verse}
${}^{1}Alors le roi Darius donna l’ordre de faire des recherches dans la maison du trésor où étaient déposées les archives à Babylone. 
${}^{2}Dans la forteresse d’Ecbatane qui est dans la province de Médie, on trouva un rouleau où il était écrit : « Mémoire. 
${}^{3}La première année du roi Cyrus, le roi Cyrus a donné cet ordre au sujet de la maison de Dieu à Jérusalem : La Maison sera rebâtie là où l’on offre des sacrifices et où se trouvent ses fondations ; sa hauteur sera de soixante coudées, et sa largeur de soixante coudées. 
${}^{4}Il y aura trois rangées de pierres de taille et une rangée de madriers. La dépense sera couverte par la maison du roi. 
${}^{5}De plus, on restituera les objets de la maison de Dieu, en or et en argent, ceux que Nabucodonosor avait fait enlever du sanctuaire de Jérusalem et qu’il avait emportés à Babylone. Chaque objet retrouvera sa place dans le sanctuaire de Jérusalem. Tu les déposeras dans la maison de Dieu. »
${}^{6}Alors Darius ordonna : « Maintenant, Tatnaï, gouverneur de Transeuphratène, Shetar-Boznaï et vous leurs collègues, fonctionnaires de Transeuphratène, éloignez-vous de là ! 
${}^{7}Laissez le gouverneur de Juda et les anciens des Juifs travailler à cette maison de Dieu : ils doivent la\\rebâtir sur son emplacement. 
${}^{8}Voici mes ordres concernant votre ligne de conduite envers les\\anciens des Juifs pour la reconstruction de cette maison de Dieu : les dépenses de ces gens leur seront remboursées, exactement et sans interruption, sur les fonds royaux, c’est-à-dire sur l’impôt de la province\\. 
${}^{9}Ce qui est nécessaire pour les holocaustes du Dieu du ciel : les jeunes taureaux, les béliers, les agneaux, le blé, le sel, le vin et l’huile, que cela leur soit donné sans aucune négligence, jour après jour, selon les indications des prêtres de Jérusalem. 
${}^{10}Ils pourront ainsi apporter des offrandes d’agréable odeur au Dieu du ciel et intercéder pour la vie du roi et de ses fils. 
${}^{11}Voici mes ordres concernant quiconque transgressera cet édit : qu’on arrache une poutre de sa maison, elle sera dressée pour qu’il y soit empalé ; et à cause de cette transgression, on réduira sa maison en tas d’ordures. 
${}^{12}Que le Dieu qui fait résider son nom à Jérusalem renverse tout roi et tout peuple qui, transgressant cet édit, étendra la main pour détruire cette maison de Dieu qui est à Jérusalem. Moi, Darius, j’ai donné cet ordre. Qu’il soit strictement exécuté ! »
${}^{13}Alors Tatnaï, gouverneur de Transeuphratène, Shetar-Boznaï et leurs collègues exécutèrent strictement l’ordre envoyé par le roi Darius.
${}^{14}Les anciens des Juifs continuèrent avec succès les travaux de construction, encouragés par la parole des prophètes Aggée et Zacharie\\le fils d’Iddo. Ils achevèrent la construction conformément à l’ordre du Dieu d’Israël, selon les décrets de Cyrus et de Darius\\. 
${}^{15} La maison de Dieu\\fut achevée le troisième jour du mois nommé Adar, dans la sixième année du règne de Darius.
${}^{16}Les fils d’Israël, les prêtres, les Lévites et le reste des rapatriés\\célébrèrent dans la joie la dédicace de cette maison de Dieu. 
${}^{17} Ils immolèrent, pour cette dédicace\\, cent taureaux, deux cents béliers, quatre cents agneaux et, en sacrifice pour le péché de tout Israël, douze boucs, d’après le nombre des tribus d’Israël. 
${}^{18} Puis ils installèrent les prêtres selon leurs classes, et les Lévites selon leurs groupes, pour le service de Dieu à Jérusalem, suivant les prescriptions du livre de Moïse.
${}^{19}Les rapatriés célébrèrent la Pâque\\le quatorzième jour du premier mois\\. 
${}^{20}Tous les prêtres et tous les Lévites, sans exception, s’étaient purifiés : tous étaient purs. Ils immolèrent donc la Pâque pour tous les rapatriés, pour leurs frères les prêtres, et pour eux-mêmes. 
${}^{21}Les fils d’Israël, revenus d’exil, mangèrent la Pâque avec tous ceux qui, pour se joindre à eux, s’étaient séparés de l’impureté des païens de ce pays, afin de chercher le Seigneur, le Dieu d’Israël. 
${}^{22}Sept jours durant, ils célébrèrent avec joie la fête des pains sans levain, car le Seigneur les avait remplis de joie et avait tourné vers eux le cœur du roi d’Assour, pour que celui-ci les soutienne dans les travaux de la maison de Dieu, le Dieu d’Israël.
      
         
      \bchapter{}
      \begin{verse}
${}^{1}Après ces événements, sous le règne d’Artaxerxès, roi de Perse, Esdras, fils de Seraya, fils d’Azarya, fils de Hilqiya, 
${}^{2}fils de Shalloum, fils de Sadoc, fils d’Ahitoub, 
${}^{3}fils d’Amarya, fils d’Azarya, fils de Merayoth, 
${}^{4}fils de Zerahya, fils de Ouzzi, fils de Bouqqi, 
${}^{5}fils d’Abishoua, fils de Pinhas, fils d’Éléazar, fils d’Aaron, le chef des prêtres, 
${}^{6}lui, Esdras, monta de Babylone. Il était scribe, versé dans la loi de Moïse, celle qu’avait donnée le Seigneur, le Dieu d’Israël. Le roi lui avait donné tout ce qu’il avait demandé, car la main du Seigneur son Dieu était sur lui. 
${}^{7}Parmi les fils d’Israël et parmi les prêtres, les Lévites, les chantres, les portiers et les servants, quelques-uns montèrent à Jérusalem, la septième année du roi Artaxerxès. 
${}^{8}Esdras arriva à Jérusalem le cinquième mois, pendant la septième année du roi. 
${}^{9}En effet, le premier jour du premier mois, Esdras avait organisé lui-même le voyage depuis Babylone. Le premier jour du cinquième mois, il arriva à Jérusalem, car la main bienfaisante de son Dieu était sur lui. 
${}^{10}Esdras, en effet, avait appliqué son cœur à étudier la loi du Seigneur, à la mettre en pratique et à enseigner les décrets et les ordonnances en Israël.
      
         
${}^{11}Voici la copie de l’acte officiel que le roi Artaxerxès avait donné au prêtre-scribe Esdras, qui était instruit des commandements du Seigneur et de ses décrets au sujet d’Israël :
       
      <div class="intertitle niv10 arameen">
        • TEXTE ARAMÉEN (7,12-26)
${}^{12}« Artaxerxès, le roi des rois, au prêtre Esdras, scribe de la loi du Dieu du ciel, paix parfaite ! Maintenant donc, 
${}^{13}j’ai donné cet ordre : Dans mon royaume, tous ceux qui font partie du peuple d’Israël, de ses prêtres ou de ses Lévites, et sont volontaires pour se rendre à Jérusalem, qu’ils s’y rendent avec toi ! 
${}^{14}En effet, tu es envoyé de la part du roi et de ses sept conseillers pour faire une enquête au sujet de Juda et de Jérusalem, conformément à la loi de ton Dieu, dont le livre est dans ta main. 
${}^{15}Tu devras aussi porter l’argent et l’or que le roi et ses conseillers donnent volontairement au Dieu d’Israël dont la demeure est à Jérusalem, 
${}^{16}ainsi que tout l’argent et l’or que tu trouveras dans toute la province de Babylone, avec les dons volontaires que le peuple et les prêtres apporteront pour la maison de leur Dieu à Jérusalem. 
${}^{17}En conséquence, tu auras soin d’acheter avec cet argent des taureaux, des béliers, des agneaux, et ce qui convient pour les offrandes de céréales et les libations qui les accompagnent ; alors tu les présenteras sur l’autel de la maison de votre Dieu à Jérusalem. 
${}^{18}Ce que toi et tes frères jugerez bon de faire avec le reste de l’argent et de l’or, vous le ferez selon la volonté de votre Dieu. 
${}^{19}Les objets qui te seront donnés pour le service de la maison de ton Dieu, dépose-les devant le Dieu de Jérusalem. 
${}^{20}Le reste de ce qui est nécessaire à la maison de ton Dieu, et qu’il te revient de fournir, tu le fourniras sur le compte du trésor royal. 
${}^{21}Moi, le roi Artaxerxès, je donne l’ordre à tous les trésoriers de Transeuphratène de faire strictement tout ce que vous demandera le prêtre Esdras, qui est scribe de la loi du Dieu du ciel, 
${}^{22}ceci jusqu’à concurrence de cent talents d’argent, cinquante tonnes de froment, quatre mille cinq cents litres de vin, quatre mille cinq cents litres d’huile, et du sel à volonté. 
${}^{23}Tout ce qu’ordonne le Dieu du ciel, qu’on l’exécute avec zèle pour la maison du Dieu du ciel, de peur que sa colère ne pèse sur le royaume du roi et de ses fils. 
${}^{24}On vous fait savoir qu’en ce qui concerne tous les prêtres et les Lévites, les chantres, les portiers, les servants et les serviteurs de cette maison de Dieu, il n’est permis de lever sur eux ni tribut, ni impôt, ni droit de passage. 
${}^{25}Quant à toi, Esdras, selon la sagesse de ton Dieu, dont le livre est dans ta main, établis des juges et des magistrats qui exercent la justice pour tout le peuple de Transeuphratène et pour tous ceux qui connaissent les lois de ton Dieu ; et à ceux qui ne les connaissent pas, vous les ferez connaître.
${}^{26}Celui qui n’accomplira pas strictement la loi de ton Dieu et la loi du roi, que la sentence lui soit appliquée : soit la mort, soit le bannissement, soit une amende, soit la prison. »
${}^{27}Béni soit le Seigneur, le Dieu de nos pères, qui a donné ainsi au cœur du roi d’honorer la maison du Seigneur à Jérusalem. 
${}^{28}Il m’a obtenu la faveur du roi, de ses conseillers et de tous les officiers du roi les plus valeureux. Quant à moi, j’ai été affermi, car la main du Seigneur mon Dieu était sur moi ; et j’ai rassemblé des chefs d’Israël pour monter avec moi à Jérusalem.
      
         
      \bchapter{}
      \begin{verse}
${}^{1}Voici, avec leurs généalogies, les chefs de famille qui montèrent avec moi de Babylone, sous le règne du roi Artaxerxès : 
${}^{2}Des fils de Pinhas, il y eut Guershom ; des fils d’Itamar : Daniel ; des fils de David : Hattoush ; 
${}^{3}des fils de Shekanya… ; des fils de Paréosh : Zacharie, avec qui furent enregistrés 150 hommes ; 
${}^{4}des fils de Pahath-Moab : Élyehoénaï, fils de Zerahya, et avec lui 200 hommes ; 
${}^{5}des fils de Zattou : Shekanya, fils de Yahaziel, et avec lui 300 hommes ; 
${}^{6}des fils d’Adine : Ébed, fils de Jonathan, et avec lui 50 hommes ; 
${}^{7}des fils d’Élam : Isaïe, fils d’Atalya, et avec lui 70 hommes ; 
${}^{8}des fils de Shefatya : Zebadya, fils de Mikaël, et avec lui 80 hommes ; 
${}^{9}des fils de Joab : Abdias, fils de Yeïel, et avec lui 218 hommes ; 
${}^{10}des fils de Bani : Shelomith, fils de Yosifya, et avec lui 160 hommes ; 
${}^{11}des fils de Bébaï : Zacharie, fils de Bébaï, et avec lui 28 hommes ; 
${}^{12}des fils d’Azgad : Yohanane, fils de Haqqatane, et avec lui 110 hommes ; 
${}^{13}des fils d’Adoniqam, les derniers dont voici les noms : Élifèleth, Yeiël et Shemaya, et avec eux 60 hommes ; 
${}^{14}des fils de Bigwaï : Outaï et Zakkour, et avec lui 70 hommes.
      
         
${}^{15}Je les ai rassemblés près du fleuve qui coule en direction d’Ahawa, et là nous avons campé durant trois jours. J’ai considéré avec attention le peuple et les prêtres, mais je n’ai trouvé là aucun des fils de Lévi. 
${}^{16}Alors j’ai envoyé les chefs Élièzer, Ariel, Shemaya, Elnatane et Yarib, Elnatane, Nathan, Zacharie, Meshoullam, ainsi que Yoyarib et Elnatane, chargés d’expliquer la Loi, 
${}^{17}et leur ai donné des ordres concernant Iddo, chef de la localité de Kasifya. J’ai mis dans leur bouche les paroles à transmettre à Iddo et à ses frères, les servants de la localité de Kasifya, afin de nous amener des serviteurs pour la maison de notre Dieu. 
${}^{18}Comme la main bienfaisante de notre Dieu était sur nous, ils nous ont amené un homme sensé, l’un des fils de Mahli, fils de Lévi, fils d’Israël, à savoir Shérébya avec ses fils et ses frères, au nombre de dix-huit, 
${}^{19}ainsi que Hashabya et Isaïe d’entre les fils de Merari, ses frères et leurs fils, au nombre de vingt. 
${}^{20}Parmi les servants que David et les princes avaient donnés pour le service des Lévites, il y en eut 220, tous désignés par leurs noms. 
${}^{21}Là, au bord du fleuve Ahawa, j’ai proclamé un jeûne afin de nous humilier devant notre Dieu, et de lui demander un heureux voyage, pour nous, pour nos enfants et pour tous nos biens. 
${}^{22}En effet, j’avais honte de solliciter du roi une troupe et des cavaliers pour nous protéger de l’ennemi pendant la route. Nous avions dit au roi : « La main de notre Dieu est sur tous ceux qui le cherchent, pour leur bonheur, mais sa force et sa colère sont sur tous ceux qui l’abandonnent. »
${}^{23}Alors nous avons jeûné et demandé cela à Dieu, et il nous a exaucés. 
${}^{24}Puis j’ai pris à part douze chefs des prêtres avec Shérébya, Hashabia et avec eux dix de leurs frères. 
${}^{25}Et je leur ai remis une somme d’argent et d’or, ainsi que les objets du culte. Telle était la contribution apportée à la maison de notre Dieu, par le roi, ses conseillers, ses princes et tous ceux d’Israël qui se trouvaient là. 
${}^{26}J’ai remis entre leurs mains 650 talents d’argent, des objets d’argent d’une valeur de 100 talents, 100 talents d’or, 
${}^{27}20 coupes d’or d’une valeur de 1 000 pièces, deux beaux objets d’un bronze brillant, précieux comme l’or. 
${}^{28}Je leur ai déclaré : « Vous êtes consacrés au Seigneur, les objets sont consacrés, et l’argent et l’or sont une offrande volontaire au Seigneur, le Dieu de vos pères. 
${}^{29}Soyez vigilants et gardez-les, jusqu’à ce que vous les remettiez aux chefs des prêtres et aux Lévites, ainsi qu’aux chefs de famille d’Israël, à Jérusalem, dans les salles de la maison du Seigneur. »
${}^{30}Alors les prêtres et les Lévites reçurent la somme d’argent et d’or et les objets du culte afin de les apporter à Jérusalem, à la maison de notre Dieu. 
${}^{31}Nous sommes partis du fleuve Ahawa le douze du premier mois, pour aller à Jérusalem. La main de notre Dieu était sur nous : il nous délivra des mains de l’ennemi et du pillard en embuscade. 
${}^{32}Nous sommes arrivés à Jérusalem et nous y sommes restés trois jours au repos. 
${}^{33}Le quatrième jour, nous avons remis la somme d’argent et d’or et les objets du culte dans la maison de notre Dieu, entre les mains du prêtre Merémoth, fils d’Ourias, avec qui était Éléazar, fils de Pinhas ; il y avait aussi avec eux les Lévites Yozabad, fils de Josué, et Noadya, fils de Binnouï. 
${}^{34}Nombre et poids, tout y était. Alors le poids total fut consigné par écrit à ce moment-là.
${}^{35}Ceux qui revinrent de la captivité, les rapatriés, offrirent en holocauste pour le Dieu d’Israël 12 taureaux pour tout Israël, 96 béliers, 77 agneaux, 12 boucs en sacrifice pour le péché : le tout comme holocauste pour le Seigneur. 
${}^{36}Puis ils transmirent les ordonnances du roi aux satrapes du roi et aux gouverneurs de Transeuphratène, et ils apportèrent leur soutien au peuple et à la maison de Dieu.
      
         
      \bchapter{}
      \begin{verse}
${}^{1}Quand cela fut achevé, les chefs s’approchèrent de moi, Esdras, et dirent : « Le peuple d’Israël, les prêtres et les Lévites ne se sont pas séparés des peuples des autres pays en ce qui concerne les abominations des Cananéens, des Hittites, des Perizzites, des Jébuséens, des Ammonites, des Moabites, des Égyptiens et des Amorites. 
${}^{2}Car eux et leurs fils ont pris femme parmi leurs filles ; ainsi la race sainte a été mélangée aux peuples des autres pays. Les princes et les notables ont été les premiers à mettre la main à cette infidélité. »
${}^{3}Lorsque j’entendis cela, je déchirai mon vêtement et mon manteau, je m’arrachai les cheveux de la tête et les poils de la barbe, et je m’assis accablé. 
${}^{4}Tous ceux qui tremblaient aux paroles du Dieu d’Israël se réunirent auprès de moi à cause de cette infidélité des rapatriés, et moi, je restai assis accablé, jusqu’à l’offrande du soir.
${}^{5}À l’heure de\\l’offrande du soir, je me relevai de ma prostration ; le vêtement et le manteau déchirés, je tombai à genoux ; les mains tendues vers le Seigneur mon Dieu, 
${}^{6}je dis : « Mon Dieu, j’ai trop de honte et de confusion pour lever mon visage vers toi, mon Dieu. Nos fautes sans nombre nous submergent\\, nos offenses se sont amoncelées jusqu’au ciel. 
${}^{7}Depuis les jours de nos pères et aujourd’hui encore, grande est notre offense : c’est à cause de nos fautes que nous avons été livrés, nous, nos rois et nos prêtres, aux mains des rois étrangers, à l’épée, à la captivité, au pillage et à la honte\\, qui nous accablent encore\\aujourd’hui. 
${}^{8}Or, voici que depuis peu de temps\\la pitié du Seigneur notre Dieu a laissé subsister pour nous des rescapés et nous a permis de nous fixer en son lieu saint ; ainsi, notre Dieu a fait briller nos yeux, il nous a rendu un peu de vie dans notre servitude. 
${}^{9}Car nous sommes asservis ; mais, dans cette servitude, notre Dieu ne nous a pas abandonnés : il nous a concilié la faveur des rois de Perse, il nous a rendu la vie, pour que nous puissions restaurer la maison de notre Dieu et relever ses ruines, afin d’avoir un abri solide\\en Juda et à Jérusalem.
${}^{10}Et maintenant, notre Dieu, que dirons-nous après cela ? Car nous avons abandonné tes commandements, 
${}^{11}ceux que tu as prescrits par tes serviteurs les prophètes, en ces termes : “Le pays dans lequel vous entrez pour en prendre possession est un pays souillé par la souillure des peuples des autres pays et par les abominations dont ils l’ont rempli d’un bout à l’autre par leur impureté. 
${}^{12}Et maintenant, ne donnez pas vos filles à leurs fils, ne prenez pas leurs filles pour vos fils, ne vous inquiétez jamais de leur prospérité ni de leur bonheur, afin que vous deveniez forts, que vous mangiez les biens du pays et que vous les laissiez en héritage à vos fils pour toujours.” 
${}^{13}Or, après tout ce qui nous est arrivé, à cause de nos actions mauvaises et de la gravité de notre offense – et bien que toi, notre Dieu, tu aies laissé de côté quelques-unes de nos fautes, et qu’ainsi tu aies maintenu des rescapés –, 
${}^{14}pourrions-nous recommencer à violer tes commandements, et à nous lier par le mariage à ces peuples abominables ? Ne vas-tu pas t’irriter contre nous au point de nous exterminer sans laisser ni reste ni rescapés ? 
${}^{15}Seigneur, Dieu d’Israël, tu es juste, car aujourd’hui encore, nous subsistons comme des rescapés. Nous voici devant toi avec nos offenses, bien qu’il soit impossible, à cause d’elles, de se tenir devant ta face. »
      
         
      \bchapter{}
      \begin{verse}
${}^{1}Comme Esdras priait et faisait cette confession, pleurant et prosterné devant la maison de Dieu, une assemblée très nombreuse du peuple d’Israël, hommes, femmes et enfants, se réunit auprès de lui ; alors le peuple versa d’abondantes larmes. 
${}^{2}Shekanya, fils de Yeïel, l’un des fils d’Élam, répondit à Esdras et déclara : « Nous avons été infidèles envers notre Dieu en épousant des femmes étrangères prises parmi les gens du pays. Mais maintenant il y a encore une espérance pour Israël ! 
${}^{3}Nous allons maintenant prendre l’engagement devant notre Dieu de faire sortir du milieu de nous toutes les femmes et leur progéniture, selon la décision de mon seigneur Esdras et de ceux qui tremblent au commandement de notre Dieu. On agira ainsi conformément à la Loi. 
${}^{4}Lève-toi, car l’affaire dépend de toi ; nous sommes avec toi. Sois fort et agis ! » 
${}^{5}Alors Esdras se leva et fit jurer aux chefs des prêtres lévites et à tout Israël d’agir conformément à cette parole ; et ils jurèrent. 
${}^{6}Esdras se leva du lieu où il était, en face de la maison de Dieu. Il se dirigea vers le logement de Yohanane, fils d’Élyashib. Mais arrivé là, il ne mangea pas de pain et ne but pas d’eau, car il était dans le deuil à cause de l’infidélité des exilés.
${}^{7}Alors en Juda et à Jérusalem, on fit parvenir une annonce à tous les rapatriés pour qu’ils se rassemblent à Jérusalem. 
${}^{8}Quiconque ne viendrait pas dans les trois jours, selon la décision des chefs et des anciens, aurait tous ses biens frappés d’anathème et lui-même serait séparé de l’assemblée des exilés.
${}^{9}Alors tous les hommes de Juda et de Benjamin se rassemblèrent à Jérusalem, dans les trois jours ; c’était le vingtième jour du neuvième mois. Tout le peuple s’installa sur la place de la maison de Dieu, tout tremblant en raison de la circonstance et à cause de la pluie ! 
${}^{10}Le prêtre Esdras se leva et leur dit : « Vous avez été infidèles en épousant des femmes étrangères, et vous avez aggravé l’offense d’Israël. 
${}^{11}Mais maintenant, rendez grâce au Seigneur, le Dieu de vos pères, et faites sa volonté : séparez-vous des gens du pays et des femmes étrangères. » 
${}^{12}Toute l’assemblée répondit d’une voix forte : « C’est vrai ! À nous d’agir comme tu as dit ! 
${}^{13}Mais le peuple est nombreux, et c’est la saison des pluies ; on n’a pas la force de rester dehors. Et ce n’est pas l’affaire d’un jour ou deux, car nous sommes nombreux à avoir péché en ce domaine. 
${}^{14}Que nos chefs se tiennent donc là pour toute l’assemblée ; tous ceux qui dans nos villes ont épousé des femmes étrangères viendront aux dates prévues, et avec eux les anciens de chaque ville et ses juges, jusqu’à ce que l’ardeur de la colère de notre Dieu se détourne de nous, au sujet de cette affaire. »
${}^{15}Cependant, Jonathan, fils d’Asahel, et Yahzeya, fils de Tiqva, s’opposèrent à cela ; Meshoullam et le lévite Shabbetaï les soutinrent. 
${}^{16}Les rapatriés agirent comme on l’avait dit. Le prêtre Esdras mit à part des hommes, chefs de famille, pour chaque famille, tous désignés par leurs noms. Alors ils siégèrent le premier jour du dixième mois pour examiner l’affaire. 
${}^{17}Et le premier jour du premier mois, ils en eurent fini avec tous les hommes qui avaient pris des femmes étrangères.
${}^{18}On trouva des fils de prêtres qui avaient épousé des femmes étrangères.
      Parmi les fils de Josué, fils de Yosadaq, et ses frères, il y eut Maaséya, Élièzer, Yarib et Guedalya. 
${}^{19}Ils s’engagèrent par serment à faire sortir leurs femmes d’au milieu d’eux, et aussi à offrir un bélier pour la réparation de leur offense.
${}^{20}Parmi les fils d’Immer : Hanani et Zebadya.
${}^{21}Parmi les fils de Harim : Maaséya, Élie, Shemaya, Yeïel et Ouziya.
${}^{22}Parmi les fils de Pashehour : Élyoénaï, Maaséya, Ismaël, Netanel, Yozabad et Élasa.
${}^{23}Parmi les Lévites : Yozabad, Shiméï, Qélaya – c’est-à-dire Qelita –, Petahya, Juda et Élièzer.
${}^{24}Parmi les chantres : Élyashib.
      Parmi les portiers : Shalloum, Télem et Ouri.
${}^{25}Quant à ceux d’Israël, on trouva parmi les fils de Paréosh : Ramya, Yizziya, Malkiya, Miyamine, Éléazar, Malkiya et Benaya ;
${}^{26}parmi les fils d’Élam : Mattanya, Zacharie, Yeïel, Abdi, Yerémoth et Élie ;
${}^{27}parmi les fils de Zattou : Élyoénaï, Élyashib, Mattanya, Yerémoth, Zabad et Aziza ;
${}^{28}parmi les fils de Bébaï : Yohanane, Hananya, Zabbaï et Atlaï ;
${}^{29}parmi les fils de Bani : Meshoullam, Mallouk, Adaya, Yashoub, Sheal, Yeramoth ;
${}^{30}parmi les fils de Pahath-Moab : Adna, Kelal, Benaya, Maaséya, Mattanya, Beçalel, Binnouï, Manassé ;
${}^{31}les fils de Harim : Élièzer, Yishshiya, Malkiya, Shemaya, Siméon, 
${}^{32}Benjamin, Mallouk, Shemarya ;
${}^{33}parmi les fils de Hashoum : Mattenaï, Mattata, Zabad, Élifèleth, Yerémaï, Manassé, Shiméï ;
${}^{34}parmi les fils de Bani : Maadaï, Amram, Ouël, 
${}^{35}Benaya, Bédya, Kelouhou, 
${}^{36}Wanya, Merémoth, Élyashib, 
${}^{37}Mattanya, Mattenaï et Yaasaï, 
${}^{38}Bani et Binnouï, Shiméï, 
${}^{39}Shèlèmya, Nathan, Adaya, 
${}^{40}Maknadebaï, Shashaï, Sharaï, 
${}^{41}Azaréel, Shèlèmyahou, Shemarya, 
${}^{42}Shalloum, Amarya, Joseph ;
${}^{43}parmi les fils de Nébo : Yeiël, Mattitya, Zabad, Zebina, Yaddaï, Joël, Benaya.
${}^{44}Tous ceux-là avaient pris des femmes étrangères, et de plusieurs d’entre elles ils avaient eu des fils.
