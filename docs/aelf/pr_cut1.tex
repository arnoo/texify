  
  
    
    \bbook{PROVERBES}{PROVERBES}
      
         
      \bchapter{}
${}^{1}Proverbes de Salomon, fils de David, roi d’Israël.
${}^{2}Veux-tu connaître la sagesse et l’instruction,
        avoir l’intelligence des propos intelligents,
${}^{3}veux-tu acquérir une instruction éclairée,
        – la justice, le jugement, la droiture –,
${}^{4}veux-tu rendre astucieux les naïfs,
        donner aux jeunes gens savoir et perspicacité ?
${}^{5}Que le sage écoute, il progressera encore,
        et l’homme intelligent apprendra à diriger :
${}^{6}il saisira les proverbes et les traits d’esprit,
        les propos des sages et leurs énigmes.
        
           
         
${}^{7}Le savoir commence avec la crainte du Seigneur !
        Sagesse et instruction, l’insensé les méprise.
        
           
         
${}^{8}Écoute, mon fils, les leçons de ton père,
        ne néglige pas l’enseignement de ta mère :
${}^{9}c’est comme une couronne de grâce sur ta tête,
        un collier à ton cou.
        
           
${}^{10}Mon fils, si des mauvais garçons veulent t’entraîner,
        ne les suis pas !
${}^{11}Ils vont te dire : « Marche avec nous,
        nous allons faire un coup sanglant,
        traquer un innocent, pour voir !
${}^{12}Nous allons, comme la Mort, le dévorer vif,
        tout entier, pareil à ceux qui descendent à la fosse ;
${}^{13}nous trouverons le magot,
        un vrai butin à remplir nos maisons !
${}^{14}Tente ta chance avec nous,
        nous ferons tous bourse commune ! »
${}^{15}Eh bien, mon fils, ne marche pas avec eux,
        ne mets pas les pieds sur leurs sentiers !
${}^{16}Car ils vont au mal d’un pied rapide,
        ils ont hâte de verser le sang.
${}^{17}– Rien ne sert de tendre un filet, dit-on,
        si l’oiseau le voit.
${}^{18}Eux, c’est contre eux-mêmes qu’ils montent ce coup sanglant :
        ils traquent leur propre vie.
${}^{19}Telle est la voie que briguent les brigands :
        elle leur coûtera la vie.
${}^{20}La Sagesse, au-dehors, lance un appel ;
        sur les places, elle élève sa voix ;
${}^{21}au-dessus du tumulte, elle crie ;
        à l’entrée des portes de la ville, elle tient ce discours :
         
${}^{22}« Combien de temps encore, étourdis,
        allez-vous aimer l’étourderie ?
        \\– Les insolents n’aspirent qu’à l’insolence,
        et les insensés refusent la connaissance !
${}^{23}Tournerez-vous longtemps le dos quand je critique ?
        Contre vous, je laisserai libre cours à mon humeur,
        je vous ferai savoir ce que j’ai à vous dire.
         
${}^{24}Quand j’ai appelé, vous avez rechigné,
        quand j’ai tendu la main, nul ne s’en est soucié !
${}^{25}Vous avez récusé tous mes conseils,
        vous n’avez pris à cœur aucune de mes critiques.
${}^{26}Eh bien, moi aussi, lors de votre malheur, je rirai,
        je serai sarcastique quand viendra l’épouvante,
${}^{27}quand elle viendra comme une tourmente,
        et que le malheur sera là, comme une tornade,
        quand viendront sur vous la détresse et l’angoisse.
         
${}^{28}Alors on m’appellera et je ne répondrai pas,
        on me cherchera et ne me trouvera pas,
${}^{29}car ils ont refusé la connaissance
        et n’ont pas choisi la crainte du Seigneur.
${}^{30}Ils n’ont pas pris à cœur mes conseils,
        ils ont dénigré chacune de mes critiques :
${}^{31}alors, ils dégusteront les fruits de leur conduite,
        ils pourront se gaver de leurs intrigues.
         
${}^{32}Oui, l’indocilité des étourdis leur sera fatale,
        et l’insouciance des insensés les perdra.
${}^{33}Celui qui m’écoute demeure en sécurité,
        à l’abri, sans malheur à redouter. »
      
         
      \bchapter{}
        ${}^{1}Mon fils, accueille mes paroles,
        conserve précieusement mes préceptes,
        ${}^{2}l’oreille attentive à la sagesse,
        le cœur incliné vers la raison.
        ${}^{3}Oui, si tu fais appel à l’intelligence,
        si tu invoques la raison,
        ${}^{4}si tu la recherches comme l’argent,
        si tu creuses comme un chercheur de trésor,
        ${}^{5}alors tu comprendras la crainte du Seigneur,
        tu découvriras la connaissance de Dieu.
        ${}^{6}Car c’est le Seigneur qui donne la sagesse ;
        connaissance et raison sortent de sa bouche.
        ${}^{7}Il réserve aux hommes droits la réussite :
        pour qui marche dans l’intégrité, il est un bouclier,
        ${}^{8}gardien des sentiers du droit,
        veillant sur le chemin de ses fidèles.
        ${}^{9}Alors tu comprendras la justice, le jugement\\, la droiture,
        seuls sentiers qui mènent au bonheur.
${}^{10}Car la sagesse viendra dans ton cœur,
        la connaissance fera tes délices,
${}^{11}la perspicacité te gardera,
        la raison veillera sur toi.
        
           
${}^{12}Tu seras préservé des chemins du mal,
        de l’homme aux propos pervers,
${}^{13}de ceux qui délaissent la route droite
        pour aller sur les chemins de ténèbre :
${}^{14}ils prennent plaisir à faire le mal,
        ils se complaisent dans la pire des perversités ;
${}^{15}leurs routes sont tortueuses,
        ils ne font que dévier sur leurs pistes.
         
${}^{16}Tu seras préservé de la femme d’un autre,
        l’étrangère aux paroles enjôleuses,
${}^{17}celle qui a délaissé l’ami de sa jeunesse,
        oublié l’alliance de son Dieu.
${}^{18}Sa maison incline vers la mort,
        ses détours mènent aux Ombres ;
${}^{19}quiconque va chez elle n’en reviendra jamais,
        il n’atteindra jamais la route de la vie.
${}^{20}C’est pourquoi il te faut prendre le bon chemin,
        garder la route des justes :
${}^{21}les hommes droits habiteront le pays,
        les gens intègres y resteront,
${}^{22}mais les méchants seront extirpés du pays,
        les fourbes en seront arrachés.
      
         
      \bchapter{}
${}^{1}Mon fils, n’oublie pas mon enseignement ;
        que ton cœur observe mes préceptes :
${}^{2}la longueur de tes jours, les années de ta vie,
        et ta paix en seront augmentées.
        
           
         
${}^{3}Que fidélité et loyauté ne te quittent pas,
        attache-les à ton cou,
        écris-les sur les tablettes de ton cœur !
${}^{4}Tu trouveras grâce et seras rayonnant
        aux yeux de Dieu et des hommes.
        
           
         
${}^{5}De tout ton cœur, fais confiance au Seigneur,
        ne t’appuie pas sur ton intelligence.
${}^{6}Reconnais-le, où que tu ailles,
        c’est lui qui aplanit ta route.
        
           
         
${}^{7}Ne te complais pas dans ta sagesse,
        crains le Seigneur, écarte-toi du mal !
${}^{8}Voilà le traitement pour ton corps,
        l’élixir pour tes os.
        
           
         
${}^{9}Rends gloire au Seigneur avec tes biens,
        donne-lui les prémices de ton revenu :
${}^{10}tes greniers se rempliront de blé,
        le vin nouveau débordera de tes cuves.
        
           
         
${}^{11}Mon fils, ne rejette pas les leçons du Seigneur,
        ne dédaigne pas ses critiques,
${}^{12}car le Seigneur reprend celui qu’il aime,
        comme fait un père pour le fils qu’il chérit.
        
           
${}^{13}Heureux qui trouve la sagesse,
        qui accède à la raison !
${}^{14}C’est une bonne affaire, meilleure qu’une affaire d’argent,
        plus rentable que l’or.
${}^{15}La sagesse est plus précieuse que les perles,
        rien ne l’égale :
${}^{16}dans sa main droite, longueur de jours,
        dans sa main gauche, richesse et gloire !
${}^{17}Ses chemins sont chemins de délices,
        tous ses sentiers, des lieux de paix.
${}^{18}Pour qui la tient, elle est arbre de vie ;
        qui la saisit est un homme heureux.
         
${}^{19}Le Seigneur a fondé la terre avec sagesse ;
        il a établi les cieux avec intelligence.
${}^{20}C’est par sa science que les abîmes se sont ouverts
        et que, des nuages, perle la rosée.
         
${}^{21}Mon fils, ne perds jamais de vue
        le savoir-faire et la perspicacité :
${}^{22}ils te seront force de vie,
        une parure à ton cou.
${}^{23}Alors tu iras ton chemin avec assurance,
        ton pied n’achoppera pas.
${}^{24}Au moment de dormir, nulle anxiété ;
        une fois endormi, ton sommeil sera doux.
${}^{25}Tu n’as rien à craindre, ni l’angoisse soudaine,
        ni la tourmente qui surprend les méchants :
${}^{26}c’est le Seigneur qui sera ton assurance,
        il gardera ton pied des embûches.
        ${}^{27}Ne refuse pas un bienfait à qui tu le dois,
        quand ce geste est à ta portée.
        ${}^{28}Ne dis pas à ton prochain : « Va-t’en, tu reviendras,
        je donnerai demain ! », alors que tu as de quoi.
        ${}^{29}Ne travaille pas au malheur de ton prochain,
        alors qu’il vit sans méfiance auprès de toi.
        ${}^{30}Ne cherche pas de vaine querelle
        à qui ne t’a pas fait de mal.
        ${}^{31}N’envie pas l’homme violent,
        n’adopte pas ses procédés.
        ${}^{32}Car le Seigneur a horreur des gens tortueux ;
        il ne s’attache qu’aux hommes droits.
        ${}^{33}Malédiction du Seigneur sur la maison du méchant,
        bénédiction sur la demeure des justes.
        ${}^{34}Il se moque des moqueurs,
        aux humbles il accorde sa grâce.
${}^{35}Aux sages, la gloire en partage,
        aux insensés, toute la honte !
      
         
      \bchapter{}
${}^{1}Fils, écoutez les leçons d’un père,
        soyez attentifs et vous connaîtrez l’intelligence.
${}^{2}Oui, c’est une valeur sûre que je vous transmets,
        ne négligez pas mon enseignement !
${}^{3}Moi aussi, j’ai été un fils pour mon père,
        enfant chéri, unique aux yeux de ma mère.
${}^{4}Voilà comment il m’instruisait :
        \\« Que ton cœur reçoive mes paroles,
        garde mes préceptes et tu vivras ;
${}^{5}acquiers la sagesse, acquiers l’intelligence,
        n’oublie pas, ne te détourne pas de ce que dit ma bouche ;
${}^{6}la sagesse, ne l’abandonne pas, elle te gardera,
        aime-la, elle veillera sur toi. »
        ${}^{7}Ainsi commence la sagesse : elle s’acquiert !
        Cède tout ce que tu as pour acquérir l’intelligence.
        ${}^{8}Ouvre-lui la voie\\, elle t’élèvera ;
        si tu l’embrasses, elle fera ta gloire.
        ${}^{9}Elle posera sur ta tête un diadème de grâce,
        elle te couronnera de splendeur.
        
           
        ${}^{10}Écoute, mon fils, accueille mes paroles,
        les années de ta vie en seront augmentées.
        ${}^{11}Je te conduis par un chemin de sagesse,
        je te fais cheminer par des sentiers de droiture.
        ${}^{12}Nulle entrave à ta marche :
        si tu cours, tu ne trébucheras pas.
        ${}^{13}Tiens-toi à la discipline, ne te relâche pas,
        veille sur elle : elle est ta vie.
${}^{14}Sur la route des méchants, ne t’engage pas ;
        ne t’avance pas sur le chemin des malfaiteurs :
${}^{15}évite-le, n’y passe pas,
        détourne-toi de lui, passe au-delà !
${}^{16}Car ils ne dorment pas qu’ils n’aient commis le mal ;
        le sommeil les fuit tant qu’ils n’ont fait chuter personne.
${}^{17}C’est de méchanceté qu’ils se nourrissent ;
        d’un vin de violence ils s’abreuvent.
${}^{18}La route des justes est lumière d’aurore,
        sa clarté s’accroît jusqu’au grand jour.
${}^{19}Le chemin des méchants, c’est la ténèbre :
        ils trébuchent sans savoir sur quoi.
         
${}^{20}Mon fils, sois attentif à mes paroles,
        prête l’oreille à mes propos ;
${}^{21}ne les perds pas de vue,
        garde-les au profond de ton cœur :
${}^{22}pour qui les trouve, ils sont la vie,
        la guérison de son être de chair.
${}^{23}Par-dessus tout, veille sur ton cœur,
        c’est de lui que jaillit la vie.
${}^{24}Éloigne de ta bouche les propos retors,
        écarte la perfidie de tes lèvres.
${}^{25}Sache regarder en face,
        dirige tes yeux droit devant toi !
${}^{26}Aplanis la piste sous tes pieds :
        tous tes chemins seront sûrs.
${}^{27}Ne dévie ni à droite ni à gauche ;
        du mal, écarte ton pied !
      
         
      \bchapter{}
${}^{1}Mon fils, sois attentif à ma sagesse,
        prête l’oreille à mes raisons ;
${}^{2}pour garder un esprit avisé,
        que tes lèvres s’en tiennent au vrai savoir !
${}^{3}Oui, le miel coule des lèvres de la femme d’un autre ;
        plus que l’huile, onctueuse est sa bouche,
${}^{4}mais elle laisse à la fin amertume d’absinthe,
        blessure d’une épée à deux tranchants.
${}^{5}Vers la mort descendent ses pas,
        son pied touche au séjour des morts ;
${}^{6}jamais elle n’ouvrira un chemin de vie ;
        ses pistes se perdent sans qu’elle en sache rien.
        
           
         
${}^{7}Maintenant, mon fils, écoute-moi,
        ne t’écarte pas de ce que dit ma bouche :
${}^{8}éloigne de cette femme ton chemin,
        n’approche pas du seuil de sa maison.
${}^{9}Sinon, tu laisseras chez d’autres ta vigueur
        et tes années au mari sans pitié !
${}^{10}Oui, des étrangers dévoreront ton énergie,
        tu travailleras dur pour la maison d’un autre,
${}^{11}si bien qu’à la fin, tu hurleras,
        ton corps et ta chair épuisés.
${}^{12}« Ah, diras-tu, comment ai-je pu haïr la discipline,
        compter pour rien les avertissements ?
${}^{13}Je n’ai pas écouté les maîtres ;
        je n’ai pas prêté l’oreille à ceux qui me formaient.
${}^{14}Pour un peu, le pire me serait arrivé
        devant la communauté rassemblée. »
        
           
${}^{15}Bois de l’eau à ta citerne,
        des eaux vives de ton puits !
${}^{16}Tes sources iraient-elles se répandre au-dehors,
        couler en ruisseaux sur les places ?
${}^{17}Qu’elles soient pour toi,
        pour toi seul, sans partage !
${}^{18}Que ta fontaine soit bénie,
        qu’elle soit ta joie, la femme de ta jeunesse,
${}^{19}biche de tes amours, gracieuse gazelle !
        \\Laisse-toi toujours enivrer de ses charmes,
        reste éperdu d’amour pour elle !
${}^{20}Pourquoi, mon fils, t’éprendre d’une autre,
        enlacer une étrangère ?
         
${}^{21}Le Seigneur a les yeux sur les chemins de l’homme,
        il observe toutes ses pistes.
${}^{22}Les crimes du méchant se retournent contre lui,
        il est captif des liens de son péché.
${}^{23}Il mourra, faute de discipline ;
        par trop de folie, il se perdra.
      
         
      \bchapter{}
${}^{1}Mon fils, si tu t’es porté caution pour un proche,
        si tu as dit : « Marché conclu ! » pour un étranger,
${}^{2}si tu es piégé par tes propres paroles,
        prisonnier de tes propres paroles,
${}^{3}alors, fais ceci, mon fils, pour t’en sortir,
        puisque te voilà entre les mains d’un autre :
        \\va, humilie-toi, insiste auprès de lui,
${}^{4}interdis tout sommeil à tes yeux,
        et tout répit à tes paupières,
${}^{5}échappe-toi, comme la gazelle loin du chasseur,
        comme l’oiseau de la main de l’oiseleur.
        
           
${}^{6}Va vers la fourmi, paresseux !
        Regarde-la marcher et deviens sage :
${}^{7}elle n’a pas de supérieur,
        ni surveillant ni gouverneur,
${}^{8}et tout l’été, elle fait ses provisions,
        elle amasse, à la moisson, de quoi manger.
${}^{9}Combien de temps vas-tu rester couché, paresseux ?
        Quand vas-tu émerger de ton sommeil ?
${}^{10}Un somme par-ci, une sieste par-là,
        s’allonger un moment, se croiser les bras,
${}^{11}et voilà que survient la pauvreté, comme un rôdeur,
        la misère, comme un garde bien armé.
${}^{12}C’est un vaurien, un faux-jeton :
        il se promène, tordant sa bouche,
${}^{13}lançant des clins d’œil, des appels du pied,
        donnant ses consignes avec les doigts ;
${}^{14}le cœur pervers, il prépare des mauvais coups,
        à tout moment, il déclenche des querelles.
${}^{15}Voilà pourquoi, soudain, vient sa ruine,
        brusquement il est brisé,
        et c’est sans remède.
         
${}^{16}Il y a six choses que le Seigneur déteste,
        sept qu’il a en horreur :
${}^{17}le regard hautain, la langue menteuse,
        les mains qui versent le sang innocent,
${}^{18}le cœur occupé de projets coupables,
        les pieds qui s’empressent de courir au mal,
${}^{19}le faux témoin qui ment comme il respire,
        et l’homme qui déclenche des querelles entre frères.
${}^{20}Garde les préceptes de ton père, mon fils,
        ne rejette pas l’enseignement de ta mère ;
${}^{21}tiens-les toujours fixés à ton cœur,
        attache-les à ton cou :
${}^{22}dans tes démarches, ils te guideront,
        dans ton sommeil, ils te garderont,
        à ton réveil, ils te tiendront compagnie.
${}^{23}Car ces préceptes sont une lampe,
        l’enseignement, une lumière :
        instruction et discipline sont un chemin de vie.
${}^{24}Ainsi, tu seras gardé de la femme mauvaise,
        des propos enjôleurs de l’étrangère ;
${}^{25}ne convoite pas sa beauté dans ton cœur,
        ne succombe pas à ses œillades !
${}^{26}À la prostituée, il suffit de gagner son pain ;
        la femme mariée, elle, pourchasse celui qui l’enrichira.
${}^{27}Peut-on mettre du feu dans sa poche
        sans brûler ses vêtements ?
${}^{28}Peut-on marcher sur des charbons ardents
        sans se griller les pieds ?
${}^{29}De même, courir après la femme de son prochain :
        nul n’y touchera sans en être puni.
${}^{30}Point de mépris pour un voleur
        si c’est la faim qui l’a poussé ;
${}^{31}mais s’il est pris, il doit payer sept fois plus :
        tous les biens de sa maison !
${}^{32}L’adultère, lui, est un écervelé,
        il fait d’une femme le bourreau de sa vie.
${}^{33}Il recevra coups et affronts,
        son déshonneur ne s’effacera pas
${}^{34}car un mari jaloux devient fou de rage,
        il est sans pitié au jour de la vengeance :
${}^{35}nul dédommagement ne l’apaisera ;
        de multiples cadeaux ne le fléchiront pas.
       
      
         
      \bchapter{}
${}^{1}Mon fils, garde mes paroles,
        conserve précieusement mes préceptes ;
${}^{2}garde mes préceptes et tu vivras,
        garde mon enseignement comme la prunelle de tes yeux ;
${}^{3}attache-les à tes doigts,
        inscris-les sur la tablette de ton cœur.
${}^{4}Dis à la sagesse : « Tu es ma sœur ! » ;
        à l’intelligence, donne le nom de « familière ».
${}^{5}Alors tu seras gardé de la femme d’un autre,
        de l’étrangère aux propos enjôleurs.
        
           
${}^{6}Comme j’étais à la fenêtre, chez moi,
        je regardais à travers la claire-voie ;
${}^{7}je vis quelques étourdis
        et, parmi ces garçons, je remarquai un jeune écervelé :
${}^{8}il passait par la venelle, près d’un recoin,
        il filait vers une maison,
${}^{9}le soir venu, à la tombée du jour,
        s’enfonçant dans les ténèbres de la nuit.
${}^{10}Et voici qu’une femme s’en va vers lui
        avec des allures de prostituée aux aguets ;
${}^{11}elle trépigne, provocante,
        incapable de rester à la maison ;
${}^{12}un pas dans la ruelle, deux pas sur la place,
        elle guette à tous les coins de rue.
${}^{13}Elle l’attrape et l’embrasse
        et, d’un air effronté, lui déclare :
${}^{14}« Je dois offrir un sacrifice de paix,
        et aujourd’hui, je viens accomplir mon vœu ;
${}^{15}aussi, je suis sortie pour te rencontrer,
        je cherchais à te voir et je t’ai trouvé !
${}^{16}J’ai paré mon lit de couvertures,
        d’un tissu d’Égypte multicolore ;
${}^{17}et sur ma couche, j’ai répandu la myrrhe,
        l’aloès et le cinnamome.
${}^{18}Viens ! Que l’amour nous enivre jusqu’au matin,
        prenons du plaisir à nous aimer !
${}^{19}Tu sais, mon mari n’est pas chez lui,
        il est en voyage au loin ;
${}^{20}il a emporté l’argent de la bourse,
        il ne rentrera qu’à la pleine lune. »
${}^{21}Avec tout son talent, elle le séduit
        et, de ses lèvres enjôleuses, elle l’entraîne.
${}^{22}Soudain le voilà qui la suit,
        comme le bœuf qu’on mène à l’abattoir,
        comme l’insensé qui gambade avant d’être châtié,
${}^{23}jusqu’à ce qu’une flèche lui ait crevé le foie.
        \\Ainsi l’oiseau se jette dans le filet,
        sans savoir qu’il y va de sa vie.
         
${}^{24}Et maintenant, fils, écoute-moi,
        fais attention à mes propos :
${}^{25}ne laisse pas ton cœur se détourner vers ses chemins,
        ne t’égare pas sur ses sentiers,
${}^{26}car nombreux sont ceux qu’elle a blessés à mort,
        innombrables, tous ceux qu’elle a tués.
${}^{27}Sa maison, c’est le chemin du séjour des morts,
        la descente aux enfers.
      
         
      \bchapter{}
${}^{1}N’est-ce pas la Sagesse qui appelle,
        la raison qui élève sa voix ?
${}^{2}En haut de la montée, sur la route,
        postée à la jonction des chemins,
${}^{3}près des portes, aux abords de la cité,
        à l’entrée des passages, elle clame :
${}^{4}« C’est vous, les humains, que j’appelle,
        ma voix s’adresse aux fils d’Adam :
${}^{5}vous, les naïfs, devenez habiles,
        vous, les insensés, devenez raisonnables.
${}^{6}Écoutez-bien, mon discours est capital,
        j’ouvre mes lèvres pour dire la droiture.
${}^{7}Oui, c’est la vérité que je ne cesse d’annoncer,
        mes lèvres ont la malice en horreur.
${}^{8}Les paroles de ma bouche ne sont que justice ;
        en elles, rien d’oblique ni de retors :
${}^{9}toutes sont claires pour qui a l’intelligence,
        et droites pour qui a trouvé la connaissance.
${}^{10}Choisissez mes leçons et non pas l’argent,
        la connaissance plutôt que l’or fin.
${}^{11}– La sagesse vaut mieux que les perles :
        rien ne l’égale.
        
           
         
${}^{12}Moi, la Sagesse, j’habite avec l’habileté,
        j’ai appris à connaître bien des finesses.
${}^{13}– La crainte du Seigneur, c’est la haine du mal.
        \\Je hais l’orgueil, l’arrogance,
        le chemin du mal et la bouche perverse.
${}^{14}À moi le conseil et l’efficacité ;
        c’est moi l’intelligence, à moi la vigueur !
${}^{15}Par moi, les rois agissent en rois
        et les souverains édictent ce qui est juste,
${}^{16}par moi, les princes agissent en princes :
        tous les chefs ont autorité dans le pays.
${}^{17}Moi, j’aime ceux qui m’aiment ;
        ceux qui me recherchent me trouvent.
${}^{18}Avec moi, la richesse et la gloire,
        fortune durable et juste prospérité.
${}^{19}Mon fruit est meilleur que l’or, que l’or fin,
        ce qui vient de moi, meilleur qu’un argent purifié.
${}^{20}Sur le chemin de la justice je m’avance,
        sur le sentier du droit.
${}^{21}Je donne un bel héritage à ceux qui m’aiment,
        je remplis leurs trésors.
        
           
        ${}^{22}Le Seigneur m’a faite pour lui\\,
        principe\\de son action,
        première de ses œuvres\\, depuis toujours.
        ${}^{23}Avant les siècles j’ai été formée\\,
        dès le commencement, avant l’apparition de la terre.
        ${}^{24}Quand les abîmes n’existaient pas encore, je fus enfantée,
        quand n’étaient pas les sources jaillissantes.
        ${}^{25}Avant que les montagnes ne soient fixées,
        avant les collines, je fus enfantée,
        ${}^{26}avant que le Seigneur\\n’ait fait la terre et l’espace\\,
        les éléments primitifs du monde.
         
        ${}^{27}Quand il établissait les cieux, j’étais là,
        quand il traçait l’horizon à la surface de l’abîme,
        ${}^{28}qu’il amassait les nuages dans les hauteurs
        et maîtrisait\\les sources de l’abîme,
        ${}^{29}quand il imposait à la mer ses limites,
        si bien que les eaux ne peuvent enfreindre son ordre,
        quand il établissait\\les fondements de la terre.
        ${}^{30}Et moi, je grandissais\\à ses côtés.
         
        \\Je faisais ses délices jour après jour,
        jouant devant lui à tout moment,
        ${}^{31}jouant dans l’univers, sur sa terre,
        et trouvant mes délices avec les fils des hommes.
         
${}^{32}Et maintenant, fils, écoutez-moi.
        Heureux ceux qui gardent mes chemins !
${}^{33}Écoutez l’instruction et devenez sages,
        ne la négligez pas.
${}^{34}Heureux l’homme qui m’écoute,
        qui veille à ma porte jour après jour,
        qui monte la garde devant chez moi.
${}^{35}Qui me trouve a trouvé la vie,
        c’est une bienveillance du Seigneur.
${}^{36}Qui m’offense se fait tort à lui-même :
        me haïr, c’est aimer la mort ! »
      <p class="cantique" id="bib_ct-at_8"><span class="cantique_label">Cantique AT 8</span> = <span class="cantique_ref"><a class="unitex_link" href="#bib_pr_9_1">Pr 9, 1-6.10-12</a></span>
      
         
      \bchapter{}
        ${}^{1}La Sagesse a bâti sa maison,
        elle a taillé sept colonnes.
        ${}^{2}Elle a tué ses bêtes, et préparé son vin,
        puis a dressé la table.
        
           
         
        ${}^{3}Elle a envoyé ses servantes, elle appelle
        sur les hauteurs de la cité :
        ${}^{4}« Vous, étourdis, passez par ici ! »
        
           
         
        \\À qui manque de bon sens, elle dit\\ :
        ${}^{5}« Venez, mangez de mon pain,
        buvez le vin que j’ai préparé.
        ${}^{6}Quittez l’étourderie et vous vivrez,
        prenez le chemin de l’intelligence. »
        
           
       
${}^{7}\[Qui corrige l’insolent ne reçoit que mépris,
        qui reprend le méchant s’en trouve sali.
${}^{8}Ne reprends pas l’insolent, il va te haïr ;
        reprends le sage, il t’aimera.
${}^{9}Si tu donnes au sage, il devient plus sage ;
        si tu instruis le juste, il progresse encore.\]
       
        ${}^{10}La sagesse commence avec la crainte du Seigneur,
        connaître le Dieu\\saint, voilà l’intelligence !
        ${}^{11}– « Par moi, se multiplient tes jours
        et s’augmentent les années de ta vie. »
        ${}^{12}Si tu es sage, c’est pour toi que tu es sage ;
        si tu fais l’insolent, toi seul en pâtiras.
${}^{13}Dame Folie fait du tapage ;
        c’est une étourdie qui ne connaît rien.
${}^{14}Assise à la porte de sa maison,
        elle trône sur les hauteurs de la cité
${}^{15}pour appeler les passants
        qui vont droit leur chemin :
${}^{16}« Vous, étourdis, passez par ici ! »
        À qui manque de bon sens, elle dit :
${}^{17}« Bien douce est l’eau qu’on a volée,
        savoureux, le pain pris en secret ! »
${}^{18}Et lui ne sait pas que des Ombres sont là,
        que ses invités descendent au séjour des morts.
      
         
      \bchapter{}
${}^{1}Proverbes de Salomon.
        
           
         
        \\Le fils sage fait la joie de son père,
        le fils insensé désole sa mère.
        
           
         
${}^{2}Bien mal acquis ne profite jamais :
        c’est la justice qui délivre de la mort.
        
           
         
${}^{3}Le Seigneur ne laisse pas le juste mourir de faim,
        il rejette l’avidité des méchants.
        
           
         
${}^{4}Main nonchalante appauvrit,
        main diligente enrichit.
        
           
         
${}^{5}Qui récolte en été est quelqu’un d’avisé,
        qui dort à la moisson est digne de mépris.
        
           
         
${}^{6}Bénédictions sur la tête du juste !
        La bouche du méchant dissimule sa violence.
        
           
         
${}^{7}On se souvient du juste pour le bénir,
        mais le renom des méchants se flétrit.
        
           
         
${}^{8}Un cœur sage accepte des règles ;
        un sot bavard court à sa perte.
        
           
         
${}^{9}Qui marche droit marche en sécurité,
        qui louvoie sur son chemin sera démasqué.
        
           
         
${}^{10}Qui fait des clins d’œil provoque des troubles,
        qui reproche avec franchise fait œuvre de paix.
        
           
         
${}^{11}La bouche du juste est source de vie,
        la bouche du méchant dissimule sa violence.
        
           
         
${}^{12}La haine suscite des querelles,
        l’amour couvre toutes les offenses.
        
           
         
${}^{13}Sur les lèvres intelligentes se trouve la sagesse,
        et le bâton, sur le dos de l’écervelé !
        
           
         
${}^{14}Les sages gardent leur savoir comme un trésor,
        mais la bouche du sot, c’est le désastre imminent.
        
           
         
${}^{15}La fortune du riche est sa citadelle ;
        la misère, la terreur des faibles.
        
           
         
${}^{16}Le salaire du juste lui sert à vivre ;
        les gains du méchant ne servent qu’à pécher !
        
           
         
${}^{17}Qui retient une leçon devient chemin de vie ;
        qui néglige les avertissements fourvoie.
        
           
         
${}^{18}Qui a le mensonge aux lèvres dissimule sa haine ;
        qui propage la calomnie est un insensé.
        
           
         
${}^{19}À trop parler on n’évite pas le péché :
        qui tient sa langue est bien avisé.
        
           
         
${}^{20}Argent de bon aloi, la langue du juste !
        Le cœur des méchants n’a guère de valeur.
        
           
         
${}^{21}Les propos des justes nourrissent la multitude,
        mais les sots meurent d’avoir l’esprit borné.
        
           
         
${}^{22}La bénédiction du Seigneur enrichit,
        et l’effort de l’homme n’y ajoute rien.
        
           
         
${}^{23}Le plaisir de l’insensé : commettre des horreurs ;
        celui de l’homme réfléchi : la sagesse.
        
           
         
${}^{24}Ce que redoute le méchant lui échoit,
        ce que désirent les justes leur est accordé.
        
           
         
${}^{25}Que passe une tempête, et le méchant n’est plus ;
        le juste reste inébranlable.
        
           
         
${}^{26}Vinaigre sur les dents, fumée dans les yeux,
        tel est le paresseux pour ceux qui l’emploient !
        
           
         
${}^{27}La crainte du Seigneur accroît les jours,
        les années des méchants sont comptées.
        
           
         
${}^{28}L’espérance des justes est joie ;
        pour les méchants, tout espoir est perdu.
        
           
         
${}^{29}Le chemin du Seigneur est un lieu sûr pour l’homme intègre
        et un désastre pour ceux qui font le mal !
        
           
         
${}^{30}Jamais le juste ne sera ébranlé ;
        les méchants n’habiteront pas le pays.
        
           
         
${}^{31}La bouche du juste a pour fruit la sagesse,
        la langue perverse sera coupée.
        
           
         
${}^{32}Les lèvres du juste savent être bienveillantes,
        les méchants n’ont que perversité à la bouche.
        
           
       
      
         
      \bchapter{}
${}^{1}Le Seigneur a horreur des balances truquées,
        le poids exact lui plaît.
        
           
         
${}^{2}Que vienne l’arrogance, viendra le mépris !
        La sagesse est avec les humbles.
        
           
         
${}^{3}L’intégrité des honnêtes gens les guidera,
        la perfidie des fourbes les mènera à la ruine.
        
           
         
${}^{4}La fortune n’est d’aucune utilité au jour de la colère,
        c’est la justice qui délivre de la mort.
        
           
         
${}^{5}La justice des gens intègres leur trace un droit chemin,
        et le méchant succombe à sa propre méchanceté.
        
           
         
${}^{6}La justice des honnêtes gens les rendra libres,
        les fourbes restent prisonniers de leur convoitise.
        
           
         
${}^{7}À la mort du méchant, son espoir périt :
        c’est peine perdue de compter sur les richesses !
        
           
         
${}^{8}Un juste est tiré de l’angoisse,
        un méchant y tombe à sa place.
        
           
         
${}^{9}Par sa langue, l’impie détruit son voisin :
        les justes le savent et se tirent d’affaire.
        
           
         
${}^{10}Le succès des justes réjouit la cité,
        la perte des méchants fait exploser sa joie !
        
           
         
${}^{11}La bénédiction due aux hommes droits grandit la cité,
        la bouche des méchants l’anéantit.
        
           
         
${}^{12}Qui se moque du voisin est un écervelé ;
        l’homme intelligent se tait.
        
           
         
${}^{13}Qui colporte des cancans trahira le secret,
        l’homme de confiance n’en soufflera mot.
        
           
         
${}^{14}Sans l’art de gouverner, un peuple tombe :
        un conseil élargi le sauvera.
        
           
         
${}^{15}Qui cautionne un inconnu s’attire des ennuis ;
        refuser toute caution, c’est plus sûr !
        
           
         
${}^{16}Une femme charmante reçoit des hommages,
        l’homme énergique capte la richesse.
        
           
         
${}^{17}Un homme bon se fait du bien à lui-même,
        un homme dur se rend malheureux.
        
           
         
${}^{18}Le salaire du méchant n’est qu’illusion ;
        à qui sème la justice, la récompense est assurée.
        
           
         
${}^{19}Oui, la justice mène à la vie ;
        qui poursuit le mal va vers la mort !
        
           
         
${}^{20}Le Seigneur a horreur des esprits retors ;
        les gens à la conduite intègre lui plaisent.
        
           
         
${}^{21}Promis, juré, le malfaiteur ne restera pas impuni,
        mais la race des justes sera sauve !
        
           
         
${}^{22}Comme l’anneau d’or au groin d’un pourceau,
        une jolie femme dépourvue de tact !
        
           
         
${}^{23}Les justes n’ont qu’un désir : le bien !
        Pour les méchants, rien d’autre à espérer que la colère !
        
           
         
${}^{24}Tel est prodigue et s’enrichit encore ;
        tel autre, économe à l’excès, n’y gagne que misère !
        
           
         
${}^{25}Une personne généreuse deviendra prospère ;
        qui donne à boire sera lui-même désaltéré.
        
           
         
${}^{26}Le peuple maudit celui qui accapare le blé,
        il bénit celui qui le met sur le marché.
        
           
         
${}^{27}Qui poursuit le bien cherche aussi à plaire ;
        tendre au mal, c’est en faire les frais.
        
           
         
${}^{28}Qui se fie à sa richesse tombera ;
        comme un arbre verdoyant, le juste fleurira.
        
           
         
${}^{29}Qui perturbe sa maison héritera le vent,
        et le sot sera l’esclave de l’homme au cœur sage !
        
           
         
${}^{30}Le fruit du juste devient arbre de vie :
        le sage entraîne les autres à sa suite.
        
           
         
${}^{31}Si le juste reçoit rétribution sur cette terre,
        combien plus le méchant et le pécheur !
        
           
       
      
         
      \bchapter{}
${}^{1}Qui aime la discipline est ami du savoir,
        qui déteste la critique est un abruti.
        
           
         
${}^{2}L’homme de bien gagne la faveur divine ;
        le Seigneur condamne le rusé.
        
           
         
${}^{3}Nul ne trouve appui dans la méchanceté,
        la racine des justes tient ferme.
        
           
         
${}^{4}Une femme parfaite est la couronne de son mari,
        une femme sans pudeur, le cancer de ses os.
        
           
         
${}^{5}Les projets du juste sont droits,
        les manœuvres du méchant, des tromperies.
        
           
         
${}^{6}Les paroles des méchants sont embûches meurtrières ;
        la bouche des honnêtes gens les en garde.
        
           
         
${}^{7}Bouscule les méchants : ils disparaissent !
        La maison des justes tient bon.
        
           
         
${}^{8}On loue quelqu’un pour son bon sens ;
        l’esprit tordu mérite le mépris.
        
           
         
${}^{9}Mieux vaut être méprisé, mais avoir un serviteur,
        que faire l’important et manquer de pain !
        
           
         
${}^{10}Le juste connaît les besoins de ses bêtes ;
        le méchant n’a que cruauté à la place du cœur !
        
           
         
${}^{11}Qui travaille sa terre aura du pain à satiété,
        qui poursuit des chimères est un écervelé.
        
           
         
${}^{12}Le méchant convoite des proies misérables,
        la racine du juste donnera du fruit.
        
           
         
${}^{13}Des lèvres criminelles sont un piège dangereux,
        mais le juste échappe à son étreinte.
        
           
         
${}^{14}Chacun, du fruit de sa bouche, peut tirer grand bien,
        comme chacun recueille le salaire de ses mains.
        
           
         
${}^{15}Le chemin de l’insensé paraît droit à ses yeux,
        mais un sage accepte le conseil.
        
           
         
${}^{16}L’insensé manifeste, sur l’heure, son dépit ;
        bien avisé qui reste sourd à l’injure !
        
           
         
${}^{17}Le témoin véridique manifeste ce qui est juste ;
        le faux témoin ne cesse de tromper.
        
           
         
${}^{18}Les bavards, c’est comme des coups d’épée,
        mais la langue des sages guérit.
        
           
         
${}^{19}Parole vraie subsiste à jamais,
        propos menteur, le temps d’un clin d’œil !
        
           
         
${}^{20}Déception pour le cœur de ceux qui trament le mal,
        joie pour les conseillers de paix !
        
           
         
${}^{21}Aucun malheur ne frappe le juste,
        mais les méchants sont comblés de maux.
        
           
         
${}^{22}Le Seigneur a horreur des lèvres menteuses ;
        qui agit loyalement lui plaît.
        
           
         
${}^{23}L’homme avisé ne montre pas sa science
        mais le cœur de l’insensé crie sa folie.
        
           
         
${}^{24}La main active aura le pouvoir,
        le paresseux aura les corvées !
        
           
         
${}^{25}Un souci dans le cœur déprime,
        une bonne parole ramène la joie.
        
           
         
${}^{26}Le juste montre la voie à son ami,
        le chemin des méchants les égare.
        
           
         
${}^{27}Point de gibier à rôtir chez les fainéants ;
        le bien le plus précieux de l’homme est son activité.
        
           
         
${}^{28}Sur le sentier de la justice, la vie !
        Qui prend ce chemin ne mourra pas.
        
           
       
      
         
      \bchapter{}
${}^{1}Le fils sage écoute les leçons de son père,
        l’insolent reste sourd aux menaces.
${}^{2}On peut tirer de la parole un fruit savoureux,
        mais les fourbes n’ont faim que de violence.
        
           
         
${}^{3}Qui surveille ses lèvres garde son âme,
        qui ouvre trop le bec court au désastre.
        
           
         
${}^{4}Le paresseux soupire, et rien ne vient ;
        les gens actifs ne restent pas sur leur faim.
        
           
         
${}^{5}Le juste déteste le mensonge,
        le méchant empeste et fait scandale.
        
           
         
${}^{6}Leur justice protège ceux qui marchent droit ;
        le péché fait la ruine des méchants.
        
           
         
${}^{7}Tel joue au riche, qui n’a rien du tout,
        et tel joue au pauvre, qui a de grands biens.
        
           
         
${}^{8}La fortune d’un homme lui sert de rançon ;
        pour le pauvre aucun bruit de menace !
        
           
         
${}^{9}La lumière des justes est joyeuse ;
        la lampe des méchants s’éteint.
        
           
         
${}^{10}La vaine prétention cause des polémiques ;
        la sagesse est avec celui qui prend conseil.
        
           
         
${}^{11}Fortune trop soudaine s’évanouira ;
        qui amasse peu à peu la verra grossir.
        
           
         
${}^{12}Attente prolongée : cœur dolent ;
        désir satisfait : arbre de vie.
        
           
         
${}^{13}Qui fait fi des conseils le paiera ;
        qui respecte un ordre aura sa récompense.
        
           
         
${}^{14}L’enseignement du sage est source de vie :
        il détourne des pièges mortels.
        
           
         
${}^{15}Un bon discernement permet de trouver grâce ;
        le chemin des traîtres est interminable.
        
           
         
${}^{16}Tout homme avisé agit à bon escient,
        mais l’insensé déploie sa folie.
        
           
         
${}^{17}Un messager malveillant tombera dans le malheur ;
        le remède, c’est un ambassadeur fidèle.
        
           
         
${}^{18}Misère et mépris à qui refuse l’éducation,
        gloire à celui qui tient compte des avertissements !
        
           
         
${}^{19}Un désir comblé est bien doux pour l’âme.
        Les insensés ont horreur de s’écarter du mal.
        
           
         
${}^{20}Qui fait route avec les sages deviendra sage ;
        qui fréquente les insensés tournera mal.
        
           
         
${}^{21}Le malheur poursuit les pécheurs,
        le bonheur récompense les justes.
        
           
         
${}^{22}À ses petits-enfants l’homme de bien transmet son héritage ;
        au juste est réservée la fortune du pécheur.
        
           
         
${}^{23}Il y a beaucoup à manger quand les pauvres défrichent ;
        que l’un vienne à périr, quelle injustice !
        
           
         
${}^{24}Qui ménage sa trique n’aime pas son fils,
        qui l’aime vraiment veille à le corriger.
        
           
         
${}^{25}Le juste mange à satiété,
        l’estomac des méchants reste vide.
        
           
       
      
         
      \bchapter{}
${}^{1}Sagesse de femme bâtit sa maison ;
        Folie la détruit de sa propre main.
        
           
         
${}^{2}Qui craint le Seigneur marche avec droiture,
        qui dévie en ses chemins le méprise.
        
           
         
${}^{3}Sur la bouche de l’insensé pointe l’orgueil,
        les lèvres des sages s’en garderont.
        
           
         
${}^{4}Nulle bête de trait : la mangeoire est vide ;
        les récoltes sont belles quand le taureau est fort.
        
           
         
${}^{5}Un témoin véridique ne ment pas ;
        le faux témoin ment comme il respire.
        
           
         
${}^{6}L’insolent cherche la sagesse, mais en vain !
        Le savoir est à la portée de l’homme intelligent.
        
           
         
${}^{7}Détourne-toi de l’insensé :
        tu n’apprendras de ses lèvres rien qui vaille !
        
           
         
${}^{8}La sagesse de l’homme avisé éclaire son chemin ;
        la folie des insensés ne fait que les tromper.
        
           
         
${}^{9}Les fous se moquent des sacrifices d’expiation ;
        les gens honnêtes y trouvent grâce.
        
           
         
${}^{10}Seul le cœur connaît sa peine,
        et à sa joie, nul ne prend part.
        
           
         
${}^{11}La maison des méchants sera rasée,
        la demeure des honnêtes gens sera florissante.
        
           
         
${}^{12}Il y a un chemin qui semble droit,
        mais au terme, ce sont des chemins de mort.
        
           
         
${}^{13}Même dans le rire, un cœur peut s’attrister,
        et au terme, la joie se changer en affliction !
        
           
         
${}^{14}Un cœur pervers se satisfait de sa conduite,
        et plus encore, un homme de bien !
        
           
         
${}^{15}Le naïf croit tout ce qu’on lui dit,
        l’homme avisé regarde où il met les pieds.
        
           
         
${}^{16}Le sage craint le mal et s’en détourne ;
        l’insensé fonce, plein d’assurance.
        
           
         
${}^{17}L’homme impatient fait des sottises,
        et l’intrigant se rend odieux.
        
           
         
${}^{18}Les naïfs ont en partage la bêtise ;
        la science est la couronne des gens avisés.
        
           
         
${}^{19}Des mauvais s’inclineront devant les bons,
        des méchants attendront à la porte du juste.
        
           
         
${}^{20}Un pauvre est rejeté, même par son ami,
        mais il y a foule pour aduler un riche.
        
           
         
${}^{21}Qui méprise son prochain est un pécheur.
        Heureux qui a pitié des petites gens !
        
           
         
${}^{22}Ne s’égarent-ils pas, les artisans du mal ?
        Ils sont fidèles et loyaux, les artisans du bien.
        
           
         
${}^{23}En tout labeur on trouve du profit,
        mais le bavardage ne mène qu’à l’indigence.
        
           
         
${}^{24}La couronne des sages, c’est leur richesse,
        mais la folie des insensés reste folie !
        
           
         
${}^{25}Il sauve des vies, le témoin qui dit vrai ;
        le faux témoin ne cesse de tromper.
        
           
         
${}^{26}La crainte du Seigneur assure puissamment :
        il est un refuge pour ses enfants.
        
           
         
${}^{27}La crainte du Seigneur est source de vie :
        elle détourne des pièges mortels.
        
           
         
${}^{28}Un peuple nombreux donne prestige au roi ;
        le déclin de la population consterne le prince !
        
           
         
${}^{29}L’homme patient a du discernement ;
        l’impulsif arbore sa folie.
        
           
         
${}^{30}Un cœur paisible est vie pour le corps ;
        la passion est un cancer pour les os.
        
           
         
${}^{31}Qui exploite le faible insulte Dieu qui l’a fait ;
        qui a pitié du misérable l’honore.
        
           
         
${}^{32}Le méchant est terrassé par sa malice ;
        dans la mort même, le juste garde confiance.
        
           
         
${}^{33}Dans un cœur intelligent repose la Sagesse ;
        au milieu des insensés, elle se fera connaître !
        
           
         
${}^{34}La justice grandit une nation ;
        le péché est la honte des peuples.
        
           
         
${}^{35}Le serviteur avisé a la faveur du roi,
        l’effronté n’a droit qu’à sa fureur.
        
           
       
      
         
      \bchapter{}
${}^{1}Une réponse paisible calme la fureur,
        un mot blessant déclenche la colère.
        
           
         
${}^{2}La langue des sages rend la science aimable,
        la bouche des insensés ne débite que des sottises.
        
           
         
${}^{3}En tout lieu, les yeux du Seigneur :
        ils observent les méchants et les bons.
        
           
         
${}^{4}La langue qui réconforte est un arbre de vie ;
        perfide, elle vous coupe le souffle.
        
           
         
${}^{5}Un sot se moque des leçons de son père ;
        bien avisé qui tient compte des avertissements !
        
           
         
${}^{6}La maison du juste regorge de richesse,
        les gains du méchant sont précaires !
        
           
         
${}^{7}Les lèvres des sages sèment le savoir ;
        pour le cœur des insensés, il n’en va pas ainsi !
        
           
         
${}^{8}Le sacrifice des méchants, le Seigneur l’a en horreur,
        mais il accueille la prière des honnêtes gens.
        
           
         
${}^{9}Le chemin du méchant fait horreur au Seigneur ;
        il aime celui qui recherche la justice.
        
           
         
${}^{10}Rude leçon pour qui s’écarte du sentier :
        refuser l’avertissement, c’est la mort !
        
           
         
${}^{11}Le séjour des morts et l’abîme sont devant le Seigneur :
        à plus forte raison le cœur des hommes !
        
           
         
${}^{12}Un insolent n’aime pas qu’on le reprenne ;
        aussi ne va-t-il guère avec les sages.
        
           
         
${}^{13}À cœur joyeux, visage épanoui ;
        à cœur chagrin, soupirs désolés.
        
           
         
${}^{14}Un cœur intelligent recherche le savoir,
        la bouche des insensés se repaît de sottises.
        
           
         
${}^{15}Pour le malheureux, tous les jours sont mauvais ;
        pour le cœur content, c’est un perpétuel festin.
        
           
         
${}^{16}Mieux vaut peu, avec la crainte du Seigneur,
        qu’un grand trésor et ses embarras.
        
           
         
${}^{17}Mieux vaut un plat de légumes servi avec amour
        que du veau gras et de la haine.
        
           
         
${}^{18}Un homme irascible provoque la querelle,
        un homme patient apaise la dispute.
        
           
         
${}^{19}Le chemin du paresseux est un buisson de ronces,
        la route des honnêtes gens est dégagée.
        
           
         
${}^{20}Le fils sage fait la joie de son père ;
        insensé, l’homme qui méprise sa mère !
        
           
         
${}^{21}L’écervelé se complaît dans sa bêtise,
        l’homme intelligent va droit son chemin.
        
           
         
${}^{22}Sans concertation, le projet avorte ;
        un conseil élargi lui donne consistance.
        
           
         
${}^{23}Quel plaisir de trouver la répartie !
        Qu’elle est bonne, la parole qui tombe à pic !
        
           
         
${}^{24}Pour l’homme avisé, le sentier de vie est une montée ;
        il évite ainsi la descente au séjour des morts.
        
           
         
${}^{25}La maison des orgueilleux, le Seigneur la renverse ;
        il fixe les bornes du terrain de la veuve.
        
           
         
${}^{26}Le Seigneur a horreur des calculs pervers ;
        les paroles bienveillantes sont pures.
        
           
         
${}^{27}Péril en la demeure pour l’homme âpre au gain !
        Qui refuse tout pot-de-vin vivra.
        
           
         
${}^{28}Le juste réfléchit en son cœur avant de répondre ;
        la bouche du méchant vomit le mal.
        
           
         
${}^{29}Le Seigneur est loin des méchants,
        mais il écoute la prière des justes.
        
           
         
${}^{30}Un regard lumineux réjouit le cœur,
        une bonne nouvelle donne des forces.
        
           
         
${}^{31}Qui sait écouter les leçons de la vie
        aura sa place entre les sages.
        
           
         
${}^{32}Qui refuse l’éducation se méprise lui-même,
        qui écoute les leçons gagne en intelligence.
        
           
       
${}^{33}La crainte du Seigneur est école de sagesse :
        avant les honneurs, l’humilité !
       
      
         
      \bchapter{}
${}^{1}Dans son cœur, l’homme propose ;
        par sa parole, Dieu dispose.
        
           
         
${}^{2}Chacun trouve sa conduite pure,
        mais le Seigneur pèse les esprits.
        
           
         
${}^{3}Remets ton action au Seigneur,
        et tes projets réussiront.
        
           
         
${}^{4}Le Seigneur a tout fait selon son dessein,
        même le méchant, pour les jours de malheur.
        
           
         
${}^{5}Le Seigneur a horreur des prétentieux :
        promis, juré, ils ne resteront pas impunis.
        
           
         
${}^{6}Fidélité et loyauté effacent une faute,
        la crainte du Seigneur détourne du mal.
        
           
         
${}^{7}Quand le Seigneur apprécie la conduite de l’homme,
        il lui concilie même ses ennemis.
        
           
         
${}^{8}Mieux vaut peu dans la justice
        que de grands profits hors du droit !
        
           
         
${}^{9}L’homme, en son cœur, fait des projets de route,
        et le Seigneur dirige ses pas.
        
           
       
${}^{10}Sentence sur les lèvres du roi :
        quand il juge, sa bouche est infaillible.
         
${}^{11}Une balance juste plaît au Seigneur,
        chacun des poids est son œuvre.
         
${}^{12}Les rois ont le mal en horreur,
        car la justice affermit le trône.
         
${}^{13}Des lèvres justes ont la faveur du roi,
        il aime celui qui parle avec droiture.
         
${}^{14}La fureur du roi est messagère de mort,
        mais un sage saura l’apaiser.
         
${}^{15}La lumière sur le visage du roi donne la vie,
        sa faveur est comme une pluie de printemps.
         
${}^{16}Mieux vaut acquérir la sagesse que l’or,
        et l’intelligence plutôt que l’argent.
         
${}^{17}La route des hommes droits se détourne du mal ;
        qui surveille son chemin garde sa vie.
         
${}^{18}L’orgueil précède l’effondrement,
        et la prétention, la chute.
         
${}^{19}Mieux vaut être humble parmi des gens modestes,
        que partager un butin avec des orgueilleux.
         
${}^{20}Un homme avisé trouvera le bonheur ;
        qui se fie au Seigneur est plus heureux encore !
         
${}^{21}La sagesse du cœur, on l’appelle intelligence ;
        la douceur des paroles aide à apprendre.
         
${}^{22}Le bon sens, pour qui le possède, est source de vie,
        mais les fous n’enseignent que folie.
         
${}^{23}Le sage, en son cœur, affine ses propos,
        et sa parole aide à apprendre.
         
${}^{24}Les paroles aimables sont un rayon de miel :
        douces au palais, elles redonnent des forces.
         
${}^{25}Il y a un chemin qui semble droit,
        mais au terme, ce sont des chemins de mort.
         
${}^{26}L’estomac du travailleur l’oblige à travailler,
        c’est la faim qui le presse !
         
${}^{27}Le vaurien prépare un mauvais coup :
        sur ses lèvres, c’est comme un feu dévorant.
         
${}^{28}Un pervers sème la discorde,
        le calomniateur divise les amis.
         
${}^{29}Un violent entraîne son compagnon
        et lui fait prendre le mauvais chemin.
         
${}^{30}S’il ferme les yeux, c’est qu’il médite un coup pervers ;
        s’il serre les lèvres, le mal est fait !
         
${}^{31}Les cheveux blancs sont une couronne splendide :
        on la trouve sur les chemins de la justice.
         
${}^{32}L’homme patient vaut mieux que le héros :
        mieux vaut maîtriser son humeur que prendre une ville.
         
${}^{33}On tire au sort avec un dé,
        mais le Seigneur décide de tout.
       
      
         
      \bchapter{}
${}^{1}Mieux vaut du pain sec, et la paix,
        qu’une salle de banquet pleine de discorde.
        
           
         
${}^{2}Un serviteur avisé supplantera le fils indigne ;
        avec les frères il aura part à l’héritage.
        
           
         
${}^{3}Comme le creuset pour l’argent, et le fourneau pour l’or,
        le Seigneur éprouve les cœurs.
        
           
         
${}^{4}Le malfaiteur s’intéresse à qui parle de fraude,
        le menteur tend l’oreille aux propos pernicieux.
        
           
         
${}^{5}Qui se moque d’un pauvre insulte Dieu qui l’a fait,
        qui se réjouit du malheur ne restera pas impuni.
        
           
         
${}^{6}Couronne des vieillards, leurs petits-enfants !
        Fierté des fils, leur père !
        
           
         
${}^{7}L’abruti n’a que faire d’un langage distingué,
        et moins encore, un prince, du langage menteur.
        
           
         
${}^{8}Les pots-de-vin portent chance, aux yeux du donateur :
        où qu’il aille, tout lui sourit !
        
           
         
${}^{9}Qui cherche l’amitié ignore l’offense ;
        qui revient sur l’affaire divise les amis.
        
           
         
${}^{10}Un reproche fait plus d’effet à l’homme intelligent
        que cent coups de bâton à l’insensé !
        
           
         
${}^{11}Au révolté qui n’aspire qu’au mal,
        on dépêche un émissaire sans pitié.
        
           
         
${}^{12}Plutôt tomber sur une ourse privée de ses petits
        que sur un insensé en plein délire !
        
           
         
${}^{13}Qui rend le mal pour le bien
        voit le mal ne plus quitter sa maison.
        
           
         
${}^{14}Début de querelle, torrent qui déferle :
        avant d’en venir au procès, renonce !
        
           
         
${}^{15}Justifier le coupable, donner tort à l’innocent :
        deux choses dont le Seigneur a horreur !
        
           
         
${}^{16}À quoi bon de l’argent dans la main d’un insensé ?
        Pour acheter la sagesse ? Il n’a rien dans la tête !
        
           
         
${}^{17}On a des amis en tout temps,
        mais un frère est là pour le temps de la détresse.
        
           
         
${}^{18}C’est un écervelé, l’homme qui dit : « Marché conclu ! »,
        et se porte caution pour un proche.
        
           
         
${}^{19}Qui aime la querelle aime le péché !
        La folie des grandeurs entraîne la chute !
        
           
         
${}^{20}Un esprit retors ne trouve pas le bonheur ;
        qui a la langue perfide tombe dans le malheur.
        
           
         
${}^{21}Engendrer un insensé, quelle tristesse !
        Le père d’un imbécile n’est pas à la fête.
        
           
         
${}^{22}À cœur joyeux, santé florissante !
        L’esprit chagrin dessèche jusqu’à l’os.
        
           
         
${}^{23}Le méchant accepte un pot-de-vin sous le manteau
        pour entraver le cours de la justice.
        
           
         
${}^{24}L’homme intelligent a devant lui la sagesse ;
        l’insensé la cherche des yeux au bout du monde.
        
           
         
${}^{25}Un fils insensé ne fait qu’irriter son père,
        et désole sa mère.
        
           
         
${}^{26}Punir un innocent est inadmissible ;
        frapper des gens de bien n’est pas correct !
        
           
         
${}^{27}Qui sait tenir sa langue a du discernement ;
        qui garde son sang-froid est homme de réflexion.
        
           
         
${}^{28}S’il se tait, même un sot passe pour sage ;
        bien malin, celui qui ne dit mot !
        
           
       
      
         
      \bchapter{}
${}^{1}Qui reste en marge n’en fait qu’à sa tête ;
        tout bon conseil l’exaspère.
        
           
         
${}^{2}L’insensé n’a pas envie de réfléchir,
        mais seulement d’étaler ses idées.
        
           
         
${}^{3}Quand arrive le méchant, arrive aussi le mépris ;
        avec les affronts, le déshonneur.
        
           
         
${}^{4}Une eau profonde, les paroles de l’homme !
        Torrent jaillissant, la source de la sagesse !
        
           
         
${}^{5}Inadmissible de réhabiliter un coupable
        en privant le juste de son droit !
        
           
         
${}^{6}L’insensé a sur les lèvres la contestation ;
        sa bouche incite à en venir aux coups.
        
           
         
${}^{7}Les propos de l’insensé font sa ruine,
        ses lèvres lui sont un piège.
        
           
         
${}^{8}Les mots du calomniateur, quel régal !
        On s’en délecte jusqu’au plus profond.
        
           
         
${}^{9}Celui qui se relâche dans son travail
        et celui qui le sabote sont frères !
        
           
         
${}^{10}Le nom du Seigneur est une tour fortifiée ;
        le juste y court : hors d’atteinte !
        
           
         
${}^{11}La fortune du riche est sa citadelle :
        il y voit son rempart.
        
           
         
${}^{12}Avant la chute, le cœur s’exalte ;
        avant la gloire, il s’humilie.
        
           
         
${}^{13}Rétorquer avant d’écouter :
        sottise et confusion !
        
           
         
${}^{14}Un bon moral rétablit le malade ;
        un moral au plus bas, qui le relèvera ?
        
           
         
${}^{15}Un cœur intelligent veut acquérir la connaissance,
        l’oreille des sages la recherche.
        
           
         
${}^{16}Des cadeaux ouvrent les portes,
        ils introduisent chez les grands.
        
           
         
${}^{17}Qui plaide en premier semble avoir raison ;
        vienne la partie adverse, elle contestera.
        
           
         
${}^{18}Le tirage au sort met fin à la querelle,
        il départage même les puissants.
        
           
         
${}^{19}Un frère offensé se ferme plus qu’une forteresse :
        la querelle bloque comme le verrou d’une citadelle.
        
           
         
${}^{20}Chacun, du fruit de sa bouche, tire profit,
        du produit de ses lèvres il se rassasie.
        
           
         
${}^{21}La mort et la vie sont au pouvoir de la langue ;
        qui aime la parole mangera de son fruit.
        
           
         
${}^{22}Il trouve le bonheur celui qui trouve femme :
        c’est une bienveillance du Seigneur.
        
           
         
${}^{23}Aux supplications du pauvre
        le riche répond durement.
        
           
         
${}^{24}Il y a des relations qui tournent mal,
        mais il y a l’ami plus attaché qu’un frère.
        
           
       
      
         
      \bchapter{}
${}^{1}Mieux vaut un pauvre à la conduite intègre
        qu’un homme aux propos retors : c’est lui l’insensé !
        
           
         
${}^{2}Sans la réflexion, le zèle ne vaut rien ;
        qui se précipite manque son but.
        
           
         
${}^{3}La sottise d’un homme entrave sa route,
        et c’est contre Dieu qu’il tourne sa colère !
        
           
         
${}^{4}La fortune multiplie les relations,
        mais le faible voit partir son seul ami.
        
           
         
${}^{5}Un faux témoin ne reste pas impuni :
        il ment comme il respire, il n’en réchappera pas.
        
           
         
${}^{6}Il y a foule pour flatter un prince,
        et chacun veut être l’ami d’un mécène.
        
           
         
${}^{7}Tous les frères du pauvre le repoussent,
        et, bien sûr, ses amis ont pris leurs distances ;
        il en parle encore, mais ils ne sont plus là !
        
           
         
${}^{8}Qui acquiert du jugement s’aime soi-même,
        qui reste clairvoyant trouve le bonheur.
        
           
         
${}^{9}Un faux témoin ne reste pas impuni :
        il ment comme il respire, il se perdra.
        
           
         
${}^{10}Qu’un insensé ait la vie douce, c’est choquant ;
        qu’un esclave commande aux princes, c’est pire encore !
        
           
         
${}^{11}Un homme de bon sens retient sa colère,
        il met son point d’honneur à passer sur l’offense.
        
           
         
${}^{12}Rugissement d’un lion, la colère du roi ;
        rosée sur l’herbe, sa faveur !
        
           
         
${}^{13}Un fils stupide, c’est une calamité pour son père ;
        les querelles d’une épouse,
        c’est un goutte-à-goutte qui n’en finit pas.
        
           
         
${}^{14}Maison et fortune sont héritage paternel,
        mais du Seigneur vient l’épouse avisée.
        
           
         
${}^{15}La paresse fait tomber dans l’apathie :
        l’homme indolent aura faim.
        
           
         
${}^{16}Qui garde une règle se garde lui-même,
        qui ne veille pas à sa conduite mourra.
        
           
         
${}^{17}Qui prend pitié du faible prête au Seigneur :
        il saura lui rendre son bienfait.
        
           
         
${}^{18}Corrige ton fils, tant qu’il y a de l’espoir ;
        mais ne t’emporte pas jusqu’à causer sa mort !
        
           
         
${}^{19}Une colère violente doit être punie ;
        si tu la tolères, tu incites à recommencer !
        
           
         
${}^{20}Écoute les conseils, accepte la correction :
        tu finiras par t’assagir !
        
           
         
${}^{21}Il y a bien des projets dans le cœur d’un homme ;
        le dessein du Seigneur, lui, se réalisera.
        
           
         
${}^{22}Ce qu’on attend d’un homme, c’est la bonne foi :
        mieux vaut un pauvre qu’un menteur !
        
           
         
${}^{23}La crainte du Seigneur mène à la vie :
        vie comblée, lieu de repos inaccessible au malheur !
        
           
         
${}^{24}Le paresseux plonge sa main dans le plat,
        il ne la ramènera même pas jusqu’à sa bouche !
        
           
         
${}^{25}Tape sur l’insolent, et l’étourdi en tirera la leçon ;
        critique l’homme intelligent, lui saura comprendre !
        
           
         
${}^{26}Qui agresse son père et met en fuite sa mère
        est un fils indigne par qui vient le déshonneur !
        
           
         
${}^{27}Cesse, mon fils, d’écouter les leçons,
        et tu iras te perdre loin du savoir !
        
           
         
${}^{28}Un vaurien témoigne en se moquant du droit,
        le méchant laisse sa bouche vomir des insanités !
        
           
         
${}^{29}Le châtiment est fait pour les insolents,
        et les coups pour l’échine des insensés !
        
           
       
      
         
      \bchapter{}
${}^{1}Le vin provoque, une boisson forte perturbe :
        quiconque s’y aventure perd toute raison.
        
           
         
${}^{2}Rugissement d’un lion, la menace du roi :
        qui se fâche avec lui risque sa vie.
        
           
         
${}^{3}Gloire à l’homme qui évite un procès !
        Tous les insensés s’y précipitent !
        
           
         
${}^{4}En automne, le paresseux ne laboure pas ;
        au temps de la moisson, il cherche… et il n’y a rien !
        
           
         
${}^{5}Les projets du cœur sont une eau profonde,
        l’homme raisonnable y puisera.
        
           
         
${}^{6}Tout un chacun proteste de sa bonne foi,
        mais un homme de confiance, qui le trouvera ?
        
           
         
${}^{7}Le juste marche droit :
        heureux ses enfants après lui !
        
           
         
${}^{8}Au tribunal, quand le roi préside,
        d’un regard, il dissipe le mal !
        
           
         
${}^{9}Qui peut dire : « J’ai purifié mon cœur,
        maintenant, je suis sans péché » ?
        
           
         
${}^{10}Deux poids, deux mesures :
        le Seigneur en a horreur !
        
           
         
${}^{11}On jugera d’un jeune à sa façon de faire :
        travaille-t-il proprement, selon les règles ?
        
           
         
${}^{12}L’oreille qui entend, l’œil qui voit :
        le Seigneur les a faits l’un et l’autre.
        
           
         
${}^{13}Ne prends pas le goût du sommeil, tu deviendrais pauvre ;
        tiens les yeux ouverts, tu seras rassasié.
        
           
         
${}^{14}« Mauvais ! Mauvais ! » dit l’acheteur,
        mais, tournant les talons, il se frotte les mains !
        
           
         
${}^{15}De l’or, il y en a ; des pierres précieuses, il n’en manque pas ;
        mais la parole intelligente, c’est perle rare !
        
           
         
${}^{16}Quelqu’un s’est-il porté garant pour un tiers, saisis son manteau ;
        s’il l’a fait pour des étrangers, prends-lui un gage !
        
           
         
${}^{17}Il est délicieux, le pain de la fraude,
        mais après, la bouche en reste pleine de gravier !
        
           
         
${}^{18}Avec des projets bien concertés, te voilà sûr ;
        mène la guerre en bon stratège !
        
           
         
${}^{19}Qui colporte des cancans trahira le secret :
        ne fréquente pas les bavards !
        
           
         
${}^{20}Qui maudit père et mère,
        sa lampe s’éteindra au cœur de la nuit !
        
           
         
${}^{21}Une fortune acquise en toute hâte
        finira par te porter malheur !
        
           
         
${}^{22}Ne dis pas : « Je rendrai le mal qu’on m’a fait ! » ;
        compte sur le Seigneur, il te sauvera !
        
           
         
${}^{23}Tricher sur le poids, le Seigneur en a horreur :
        rien de bon dans une balance truquée !
        
           
         
${}^{24}C’est le Seigneur qui dirige les pas de l’homme :
        un mortel, comment saurait-il où il va ?
        
           
         
${}^{25}Voici un piège : dire à la légère : « C’est sacré »,
        et n’y réfléchir qu’après s’être engagé !
        
           
         
${}^{26}Le roi sage disperse au vent les méchants :
        il a traîné sur eux la roue du battage.
        
           
         
${}^{27}C’est une lampe divine, l’esprit de l’homme ;
        il explore le tréfonds de son être.
        
           
       
${}^{28}Fidélité, loyauté forment la garde du roi,
        la fidélité affermira son trône !
         
${}^{29}La beauté de la jeunesse, c’est sa vigueur ;
        la parure des anciens, leurs cheveux blancs !
         
${}^{30}Faire saigner la plaie prévient l’infection ;
        ainsi la correction guérit en profondeur !
       
      
         
      \bchapter{}
        ${}^{1}Le Seigneur dispose du cœur du roi
        comme d’un canal d’irrigation,
        \\il le dirige où il veut.
        
           
         
        ${}^{2}La conduite d’un homme est toujours droite à ses yeux,
        mais c’est le Seigneur qui pèse les cœurs.
        
           
         
        ${}^{3}Accomplir la justice et le droit
        plaît au Seigneur plus que le sacrifice.
        
           
         
        ${}^{4}Regarder de haut, se rengorger :
        ainsi brillent les méchants, mais ce n’est que péché.
        
           
         
        ${}^{5}Les plans de l’homme actif lui assurent du profit ;
        mais\\la précipitation conduit à l’indigence.
        
           
         
        ${}^{6}Une fortune acquise par le mensonge :
        illusion fugitive de qui cherche la mort.
        
           
         
${}^{7}Les méchants seront victimes de leur violence,
        car ils refusent d’agir selon le droit.
        
           
         
${}^{8}Les méthodes de l’imposteur sont tortueuses ;
        un homme honnête agit droitement.
        
           
         
${}^{9}Mieux vaut loger dans un coin de terrasse
        que partager sa maison avec une mégère.
        
           
         
        ${}^{10}Le méchant ne désire que le mal ;
        il n’a pas un regard de pitié pour son prochain.
        
           
         
        ${}^{11}Quand on punit l’insolent, l’étourdi devient sage ;
        le sage, il suffit de le raisonner pour qu’il comprenne\\.
        
           
         
        ${}^{12}Le juste considère le clan du méchant :
        le méchant pervertit les autres pour leur malheur\\.
        
           
         
        ${}^{13}Qui fait la sourde oreille à la clameur des faibles
        criera lui-même sans obtenir de réponse.
        
           
         
${}^{14}On apaise une colère par un présent discret,
        la fureur violente par un cadeau sous le manteau.
        
           
       
${}^{15}L’exercice du droit est une joie pour le juste ;
        c’est la terreur des malfaisants.
         
${}^{16}L’homme qui erre loin des voies de la raison
        ira reposer parmi les Ombres.
         
${}^{17}Qui aime faire la fête sera sans le sou,
        qui aime le vin et les parfums ne fera pas fortune.
         
${}^{18}Le méchant paiera pour le juste,
        et le fourbe, à la place des honnêtes gens.
         
${}^{19}Mieux vaut aller vivre au désert
        qu’avec une mégère acariâtre !
         
${}^{20}Dans la demeure du sage, abondance et trésor désirable ;
        l’insensé n’en fait qu’une bouchée !
         
${}^{21}Qui poursuit justice et fidélité
        trouvera la vie et la gloire !
         
${}^{22}Le sage a pris d’assaut une ville de héros,
        il en a rasé la citadelle, son lieu sûr.
         
${}^{23}Qui garde sa bouche et sa langue
        se garde lui-même de bien des angoisses.
         
${}^{24}L’orgueilleux, le prétentieux, a pour nom : « l’Insolent » !
        Il n’agit que dans l’orgueil et l’excès.
         
${}^{25}Les désirs du paresseux le tuent,
        car ses mains refusent de passer à l’acte.
${}^{26}Au long du jour, il va de désir en désir,
        tandis que le juste donne sans compter.
         
${}^{27}Le sacrifice des méchants est une horreur,
        d’autant plus qu’ils le font dans un esprit perfide.
         
${}^{28}Le faux témoin périra ;
        seul peut parler l’homme qui a entendu.
         
${}^{29}Le méchant prend un air assuré ;
        l’homme droit, lui, avance de pied ferme.
         
${}^{30}Point de sagesse, point de raison,
        point de projet qui tienne en face du Seigneur !
         
${}^{31}On équipe son cheval pour le jour du combat,
        mais c’est le Seigneur qui détient la victoire.
       
      
         
      \bchapter{}
${}^{1}Plus désirable un nom qu’une grande richesse,
        plus que l’or et l’argent, la faveur !
        
           
         
${}^{2}Un riche et un pauvre se rencontrent :
        l’un comme l’autre, le Seigneur les a faits.
        
           
         
${}^{3}Bien avisé qui voit le malheur et se met à l’abri ;
        l’étourdi passe outre, il le paiera.
        
           
         
${}^{4}Fruit de l’humilité : la crainte du Seigneur,
        et la richesse, et la gloire, et la vie !
        
           
         
${}^{5}Épines et embûches sur le chemin du pervers :
        qui tient à la vie s’en écarte.
        
           
         
${}^{6}Éduque un jeune à mesure de son développement :
        jamais il ne déviera, même l’âge venu.
        
           
         
${}^{7}Le riche a pouvoir sur les pauvres :
        le débiteur sera l’esclave de son créancier.
        
           
         
${}^{8}Qui sème la fraude récolte le malheur :
        le fléau de sa frénésie lui sera fatal.
        
           
         
${}^{9}Qui a bon cœur sera béni :
        il partage son pain avec le faible.
        
           
         
${}^{10}Chasse l’insolent, la querelle s’en va,
        affronts et procès prennent fin.
        
           
         
${}^{11}Qui s’attache à purifier son cœur a des paroles aimables,
        le roi est son ami.
        
           
         
${}^{12}Les yeux du Seigneur veillent sur le savoir,
        mais il réduit à rien les propos trompeurs.
        
           
         
${}^{13}Le paresseux dit : « Un lion dehors !
        Si je sors, je suis mort ! »
        
           
         
${}^{14}Gouffre profond, la bouche de la femme d’un autre,
        y tombent ceux que réprouve le Seigneur !
        
           
         
${}^{15}La folie s’agrippe au cœur d’un jeune :
        le bâton de la correction lui fait lâcher prise.
        
           
         
${}^{16}Qui opprime le faible le rend fort,
        qui donne au riche se rend pauvre !
        
           
${}^{17}Tends l’oreille, écoute les paroles des sages,
        que ton cœur s’attache à mon savoir :
${}^{18}tu prendras plaisir à garder en toi ces paroles,
        toutes prêtes à venir sur tes lèvres.
${}^{19}Pour que ta confiance soit dans le Seigneur,
        je vais t’instruire aujourd’hui, toi aussi.
         
${}^{20}N’ai-je pas écrit pour toi trente chapitres
        pleins de conseils et de science,
${}^{21}afin que tu reconnaisses la parole vraie,
        que tu fasses un rapport sûr à celui qui t’envoie ?
         
${}^{22}Ne profite pas de la faiblesse du faible,
        n’écrase pas le pauvre au tribunal,
${}^{23}car le Seigneur plaidera leur cause,
        il ravira la vie de leurs ravisseurs.
         
${}^{24}Ne fréquente pas le coléreux,
        n’approche pas l’homme irascible ;
${}^{25}sinon, tu prendras leurs manières,
        tu seras pris au piège.
         
${}^{26}Ne sois pas de ceux qui disent : « Marché conclu ! »
        et se portent caution pour un emprunt :
${}^{27}si tu n’as pas de quoi rembourser,
        pourquoi risquer la saisie du lit où tu dors ?
         
${}^{28}Ne déplace pas une borne ancienne :
        ce sont tes ancêtres qui l’ont posée.
         
${}^{29}As-tu vu un homme habile à l’ouvrage ?
        C’est au service des rois qu’il a sa place,
        il ne l’a pas auprès des gens obscurs.
      
         
      \bchapter{}
${}^{1}Si tu es à la table d’un grand,
        fais bien attention à ce qui est devant toi ;
${}^{2}mets un couteau sur ta gorge
        si tu es gourmand ;
${}^{3}ne lorgne pas les bons plats :
        en manger te décevra !
        
           
         
${}^{4}Ne cours pas après la richesse,
        qu’elle cesse de t’obséder !
${}^{5}L’as-tu suivie des yeux ? Elle a disparu !
        Car elle se donne des ailes ;
        comme un aigle, elle s’envole vers le ciel !
        
           
         
${}^{6}Ne partage pas le pain de l’envieux,
        ne lorgne pas ses bons plats !
${}^{7}Car il calcule tout, il est ainsi fait ;
        il te dit : « Mange et bois ! »,
        mais il n’est pas de cœur avec toi.
${}^{8}La bouchée sitôt avalée, tu vas la vomir,
        et tu en seras pour tes compliments !
        
           
         
${}^{9}À l’oreille d’un sot ne dis mot :
        il n’a que mépris pour tes paroles sensées !
        
           
         
${}^{10}Ne déplace pas une borne ancienne,
        n’empiète pas sur la terre des orphelins
${}^{11}car leur Défenseur est puissant,
        il plaiderait leur cause contre toi.
        
           
         
${}^{12}Dispose ton cœur à l’instruction
        et tes oreilles aux paroles du savoir.
        
           
         
${}^{13}N’hésite pas à corriger ton garçon,
        il ne va pas mourir pour des coups de baguette !
${}^{14}Toi, par des coups de baguette,
        c’est de la tombe que tu le sauveras !
        
           
${}^{15}Mon fils, si tu as le cœur sage,
        mon cœur à moi se réjouira,
${}^{16}et j’exulterai de tout mon être
        quand tes lèvres parleront avec droiture.
         
${}^{17}Que ton cœur n’envie pas les pécheurs,
        mais qu’il reste tout le jour dans la crainte du Seigneur :
${}^{18}il y a, certes, un avenir,
        tu n’auras pas espéré en vain.
         
${}^{19}Et toi, mon fils, écoute et sois un sage ;
        garde ton cœur dans le droit chemin !
${}^{20}Ne sois pas de ceux qui s’enivrent
        et qui font bonne chère,
${}^{21}car l’ivrogne et le glouton courent à la ruine ;
        ils se réveillent un jour vêtus de haillons.
         
${}^{22}Écoute ton père, c’est lui qui t’a engendré ;
        ne méprise pas ta mère en ses vieux jours.
${}^{23}– Achète la vérité, ne la vends jamais !
        De même, la sagesse, l’instruction, l’intelligence ! – 
${}^{24}Il exulte, le père d’un homme juste ;
        celui qui engendre un sage est comblé de joie.
${}^{25}Que se réjouissent ton père et ta mère,
        qu’elle exulte, celle qui t’a donné le jour !
         
${}^{26}Donne-moi ton cœur, mon fils ;
        que tes yeux suivent mes pas !
${}^{27}La prostituée est un gouffre profond,
        l’étrangère est un puits dont on ne peut sortir.
${}^{28}Elle aussi, comme un voleur, est aux aguets,
        multipliant les perfidies entre les hommes.
${}^{29}Pour qui les « aïe ! » ? Pour qui les « hou-là-là ! » ?
        Pour qui les querelles ? Pour qui les soupirs ?
        \\Pour qui les coups à tort et à travers ?
        Pour qui le regard trouble ?
${}^{30}Pour ceux qui perdent leur temps à s’enivrer,
        à courir après les boissons fortes !
${}^{31}Ne lorgne pas le vin qui rougeoie,
        si beaux que soient ses reflets dans la coupe,
        car il va droit au but :
${}^{32}il finit par mordre comme un serpent,
        il pique comme une vipère ;
${}^{33}tes yeux verront d’étranges choses,
        tu diras des absurdités,
${}^{34}tu seras comme pris du mal de mer
        comme à la dérive tout en haut d’un mât :
${}^{35}« On m’a frappé, mais je n’ai pas mal,
        on m’a battu, mais je ne sais plus…
        \\Quand vais-je me réveiller ?
        J’en redemanderai encore ! »
       
      
         
      \bchapter{}
${}^{1}N’envie pas les méchants,
        ne rêve pas d’être avec eux,
${}^{2}car ils n’ont dans le cœur que violence
        et sur les lèvres, des paroles menaçantes.
        
           
         
${}^{3}Avec la sagesse on se bâtit une maison,
        avec l’intelligence on la rend solide,
${}^{4}avec du savoir-faire on remplit les pièces
        de mille biens précieux et beaux.
        
           
         
${}^{5}Au sage appartient la force,
        celui qui a l’expérience augmente son pouvoir.
${}^{6}C’est en bon stratège que tu dois mener la guerre :
        élargis ton conseil et tu réussiras !
        
           
         
${}^{7}La sagesse vole trop haut pour le fou :
        au conseil de la cité, il ne dit mot.
${}^{8}Qui fait le mal avec préméditation
        mérite le nom de roi des rusés.
${}^{9}Le fou n’a de ruse que pour le péché ;
        l’insolent, les gens l’ont en horreur.
        
           
         
${}^{10}Tu faiblis aux jours difficiles ?
        Difficile de croire à ta force !
${}^{11}Sauve les condamnés à mort,
        fais grâce à ceux qu’on traîne au supplice.
${}^{12}Tu oses dire : « Eh oui, nous ne savions pas. »
        Mais Celui qui pèse les cœurs, ne comprend-il pas ?
        \\Celui qui observe ton âme, il sait, lui ;
        il rendra à chacun selon ses actes.
        
           
         
${}^{13}Mange du miel, mon fils, c’est très bon,
        un vrai nectar, douceur pour ton palais !
${}^{14}Ainsi en sera-t-il pour ton âme, si tu goûtes à la sagesse :
        si tu la trouves, tu auras un avenir,
        tu n’auras pas espéré en vain.
        
           
         
${}^{15}Ne guette pas le domaine du juste,
        ne ravage pas sa propriété.
${}^{16}Car le juste tombe sept fois mais se relève,
        alors que les méchants s’effondrent dans le malheur.
        
           
         
${}^{17}Si ton ennemi tombe, ne te réjouis pas ;
        s’il s’effondre, ne jubile pas :
${}^{18}le Seigneur verrait cela d’un mauvais œil
        et détournerait de lui sa colère !
        
           
         
${}^{19}Ne t’indigne pas à la vue des malfaiteurs,
        ne jalouse pas les méchants,
${}^{20}car le mauvais n’a pas d’avenir :
        la lampe des méchants s’éteindra.
        
           
         
${}^{21}Crains le Seigneur, mon fils, et aussi le roi,
        ne fréquente pas le contestataire,
${}^{22}car soudain surgira son désastre ;
        l’échec que les deux lui réservent, qui le sait ?
        
           
${}^{23}Voici encore ce que disent les Sages :
         
        \\Être partial dans un jugement, ce n’est jamais bon !
${}^{24}Quiconque dit au coupable : « Tu es innocent »
        sera vilipendé par la foule, honni par les nations.
${}^{25}Mais à ceux qui sanctionnent, on saura gré ;
        sur eux viendra une heureuse bénédiction.
         
${}^{26}Il donne un vrai baiser,
        celui qui répond franchement.
         
${}^{27}Assure au-dehors ton travail,
        mène-le à bien dans ton champ ;
        ensuite, tu construiras ta maison.
         
${}^{28}Ne charge pas sans preuve ton prochain :
        tes lèvres seraient-elles trompeuses ?
${}^{29}Ne dis pas : « Il me l’a fait, je le lui ferai :
        je rendrai à chacun selon ses actes ! »
${}^{30}Je suis passé près du champ d’un paresseux,
        près de la vigne d’un homme écervelé.
${}^{31}Voici que des broussailles avaient poussé partout,
        le sol était couvert de ronces,
        la murette en pierres était par terre.
${}^{32}Voyant cela, moi, je réfléchis ;
        je regarde et j’en tire une leçon :
${}^{33}un somme par-ci, une sieste par-là,
        s’allonger un moment, se croiser les bras,
${}^{34}et voilà que survient la pauvreté qui rôdait,
        la misère, comme un garde bien armé.
      
         
      \bchapter{}
${}^{1}Voici encore des proverbes de Salomon, que transcrivirent les gens d’Ézékias, roi de Juda.
        
           
         
${}^{2}La gloire de Dieu, c’est de voiler ;
        la gloire des rois, c’est de scruter.
${}^{3}Et le ciel en sa hauteur, et la terre en sa profondeur,
        et le cœur des rois, nul ne peut les scruter.
        
           
         
${}^{4}Ôte les scories de l’argent,
        il en sortira un vase d’orfèvre ;
${}^{5}ôte la corruption de l’entourage du roi,
        son trône s’affermira dans la justice.
        
           
         
${}^{6}Ne cherche pas à briller devant le roi,
        ne te mets pas à la place des grands ;
${}^{7}mieux vaut que l’on te dise : « Monte ici »,
        plutôt que d’être rabaissé devant un prince.
        
           
         
        \\Ce que tu as vu de tes yeux,
${}^{8}ne te presse pas d’en faire état dans un procès :
        \\sinon, comment mettre un point final
        aux accusations de ton adversaire ?
        
           
         
${}^{9}Vide ta querelle avec ton adversaire,
        sans divulguer les confidences d’un tiers :
${}^{10}sinon, qui le saura te fera honte,
        et cet affront sera sans appel.
        
           
         
${}^{11}Pommes d’or incrustées d’argent,
        la parole dite à point nommé.
${}^{12}Anneau d’or, collier d’or pur,
        la critique du sage pour l’oreille attentive.
        
           
         
${}^{13}Fraîcheur de neige un jour de moisson,
        tel est le messager fidèle, pour qui l’envoie,
        vrai réconfort pour son maître !
        
           
         
${}^{14}Des nuages et du vent, mais point de pluie :
        tel est l’homme qui promet monts et merveilles !
        
           
         
${}^{15}À force de patience on peut fléchir un juge :
        une langue délicate peut broyer un os.
        
           
         
${}^{16}Tu as trouvé du miel ? Manges-en à ta faim !
        Si tu t’en gaves, tu le vomiras.
        
           
         
${}^{17}Ne va pas trop souvent chez ton voisin !
        Si tu exagères, il te haïra.
        
           
         
${}^{18}Coup de massue, poignard, flèche acérée,
        le faux témoignage qui vient d’un ami !
        
           
         
${}^{19}Dent qui branle, pied qui flanche,
        le secours d’un traître au jour mauvais.
        
           
         
${}^{20}C’est retirer le manteau un jour de gel,
        verser du vinaigre sur une plaie,
        que de chanter des chansons à qui va mal !
        
           
         
${}^{21}Si ton ennemi a faim, donne-lui du pain à manger ;
        s’il a soif, donne-lui de l’eau à boire ;
${}^{22}ce sont des braises que tu places sur sa tête,
        et le Seigneur te le rendra.
        
           
         
${}^{23}Le vent du nord est gros de la pluie ;
        parler à mots couverts fait monter la colère.
        
           
         
${}^{24}Mieux vaut loger dans un coin de terrasse
        que partager sa maison avec une mégère.
        
           
         
${}^{25}De l’eau fraîche pour une gorge sèche,
        les bonnes nouvelles d’un pays lointain.
        
           
         
${}^{26}Une source trouble, une fontaine polluée,
        le juste qui perd pied devant un méchant.
        
           
         
${}^{27}Manger du miel tant et plus, ce n’est pas bon ;
        courir après la gloire, non plus.
        
           
         
${}^{28}Ville éventrée, sans rempart,
        l’homme qui ne maîtrise pas ses humeurs !
        
           
      
         
      \bchapter{}
${}^{1}Pas plus que neige en été ou pluie à la moisson,
        la gloire ne convient à un sot.
        
           
         
${}^{2}Comme un battement d’ailes, comme un envol d’hirondelles,
        maudire sans raison ne mène à rien.
        
           
         
${}^{3}Un fouet pour le cheval, une bride pour l’âne,
        tel le bâton pour l’échine des sots !
        
           
         
${}^{4}Ne réponds pas à l’insensé selon sa folie,
        sinon tu vas lui ressembler, toi aussi !
${}^{5}Réponds à l’insensé selon sa folie,
        sinon il va se prendre pour un sage !
${}^{6}Il en aura les bras coupés, l’injure à la bouche,
        celui qui confie une mission à l’insensé.
        
           
         
${}^{7}Elles vont de travers, les jambes du boiteux,
        comme un proverbe dans la bouche du sot.
        
           
         
${}^{8}Autant bloquer le caillou dans la fronde
        que de faire honneur à un sot.
        
           
         
${}^{9}Le harpon que brandit un ivrogne,
        ainsi le proverbe dans la bouche d’un sot.
        
           
         
${}^{10}C’est beaucoup de blessures pour tous
        que d’embaucher l’insensé ou le premier venu.
        
           
         
${}^{11}Comme le chien retourne à son vomi,
        l’insensé revient à ses folies.
        
           
         
${}^{12}As-tu vu quelqu’un qui se prend pour un sage ?
        D’un sot, tu peux attendre davantage.
        
           
${}^{13}Le paresseux dit : « Il y a un fauve sur le chemin,
        un lion qui rôde par les rues ! »
         
${}^{14}La porte tourne sur ses gonds,
        le paresseux se retourne sur son lit.
         
${}^{15}Le paresseux plonge sa main dans le plat :
        quel effort pour la ramener à sa bouche !
         
${}^{16}Le paresseux se croit plus sage
        que sept qui répondraient à bon escient.
         
${}^{17}Il attrape un chien par les oreilles,
        celui qui se mêle des querelles d’autrui.
         
${}^{18}Comme un homme qui fait le fou
        et lance des brandons et des flèches de mort,
${}^{19}ainsi l’homme qui trompe son prochain,
        et puis déclare : « Mais je plaisantais ! »
${}^{20}Quand le bois vient à manquer, le feu s’éteint ;
        faute de calomniateur, cesse la querelle.
${}^{21}Du charbon sur les braises, du bois sur le feu :
        ainsi l’homme querelleur attise le procès.
         
${}^{22}Les mots du calomniateur, quel régal !
        On s’en délecte jusqu’au plus profond.
         
${}^{23}Scories d’argent plaquées sur un tesson,
        des lèvres de feu avec un cœur mauvais !
         
${}^{24}L’ennemi se cache derrière ses lèvres,
        il dissimule, au fond de lui, sa tromperie.
${}^{25}Au charme de sa voix ne te fie pas,
        il a, dans le cœur, sept projets abominables.
${}^{26}Sa ruse a beau cacher sa haine,
        sa malice apparaîtra au grand jour.
         
${}^{27}Qui creuse une fosse y tombera,
        qui fait rouler une pierre la verra revenir sur lui.
         
${}^{28}Langue menteuse a de la haine pour ses victimes,
        bouche mielleuse mène à la ruine.
       
      
         
      \bchapter{}
${}^{1}Ne te félicite pas du lendemain,
        tu ne sais pas ce qu’aujourd’hui va enfanter.
        
           
         
${}^{2}Qu’un autre te loue, mais pas ta bouche,
        n’importe qui, mais pas tes lèvres !
        
           
         
${}^{3}Ça pèse, la pierre, c’est lourd, le sable,
        plus lourd encore, les humeurs d’un sot !
        
           
         
${}^{4}La rage est sans pitié, la colère, impétueuse ;
        mais qui tiendra devant la jalousie ?
        
           
         
${}^{5}Mieux vaut franche critique
        qu’amitié hypocrite !
        
           
         
${}^{6}Un ami donne des coups loyalement ;
        les baisers de l’ennemi sont trompeurs.
        
           
         
${}^{7}Ventre repu fait fi du miel ;
        à ventre affamé tout est doux, même le fiel !
        
           
         
${}^{8}Comme l’oiseau migrant loin de son nid,
        ainsi va l’émigré loin de chez lui.
        
           
         
${}^{9}Baumes et parfums mettent le cœur en joie :
        la douceur de l’ami l’emporte sur les rêves.
${}^{10}Ne laisse pas tomber ton ami, ni l’ami de ton père ;
        ne va pas chez ton frère le jour où tu es ruiné :
        mieux vaut proche voisin que frère lointain !
        
           
         
${}^{11}Sois un sage, mon fils, pour ma plus grande joie,
        et j’aurai de quoi répondre à qui m’insulte.
        
           
         
${}^{12}Bien avisé qui voit le malheur et se met à l’abri ;
        l’étourdi passe outre, il le paiera.
        
           
         
${}^{13}Quelqu’un s’est-il porté garant pour un tiers, saisis son manteau ;
        s’il l’a fait pour des étrangers, prends-lui un gage !
        
           
         
${}^{14}Bénir un ami, tôt matin, à grands cris ?
        Autant le maudire !
        
           
         
${}^{15}La gouttière qui ne cesse de couler un jour de pluie
        et l’épouse querelleuse, c’est tout un :
${}^{16}autant vouloir retenir le vent
        ou prendre de l’huile avec ses doigts !
        
           
         
${}^{17}Le fer s’aiguise avec le fer,
        et l’homme s’aiguise à rencontrer son prochain !
        
           
         
${}^{18}Qui soigne son figuier en mangera les fruits,
        qui veille sur son maître en recevra l’honneur.
        
           
         
${}^{19}Comme un visage voit son reflet dans l’eau,
        ainsi l’homme se voit-il en son cœur.
        
           
         
${}^{20}Le séjour des morts et l’abîme sont insatiables,
        les yeux de l’homme aussi sont insatiables !
        
           
         
${}^{21}On éprouve l’argent au creuset, l’or au fourneau,
        et l’homme par la bouche de qui fait son éloge.
        
           
         
${}^{22}Quand tu pilerais le sot dans un mortier,
        au milieu du grain, avec le pilon,
        sa sottise ne se détacherait pas de lui !
        
           
${}^{23}Juge bien de la mine de tes moutons,
        et prends soin de tes troupeaux,
${}^{24}car la richesse n’est pas éternelle
        ni les trésors assurés d’âge en âge !
${}^{25}On fauche le foin, le regain apparaît,
        on ramasse l’herbe des montagnes :
${}^{26}les brebis te paieront le vêtement,
        les boucs, le prix d’un champ ;
${}^{27}le lait de chèvre suffira à te nourrir,
        à nourrir ta maison, faire vivre tes servantes.
      
         
      \bchapter{}
${}^{1}Le coupable s’enfuit quand nul ne le poursuit,
        les innocents ont l’assurance des lions.
        
           
         
${}^{2}Quand le pays se révolte, les chefs sont légion ;
        vienne un homme de bon sens et d’expérience, l’ordre règne.
        
           
         
${}^{3}Un homme pauvre qui exploite les faibles,
        c’est le déluge : plus rien à manger !
        
           
         
${}^{4}Ceux qui rejettent la loi font l’éloge du méchant,
        ceux qui observent la loi leur font la guerre.
        
           
         
${}^{5}La racaille ne comprend rien au droit ;
        qui cherche le Seigneur comprendra tout !
        
           
         
${}^{6}Mieux vaut un pauvre à la conduite intègre
        qu’un homme tortueux et louvoyant, même riche !
        
           
         
${}^{7}Qui garde la loi est un fils intelligent ;
        qui s’encanaille fait la honte de son père.
        
           
         
${}^{8}Qui accroît son bien par usure et intérêts
        amasse pour qui aura pitié des faibles.
        
           
         
${}^{9}Qui fait la sourde oreille à la loi,
        sa prière n’inspirera que dégoût.
        
           
         
${}^{10}Qui égare les gens honnêtes sur la voie du mal
        tombera lui-même dans son piège ;
        les gens intègres hériteront le bonheur.
        
           
         
${}^{11}Il se prend pour un sage, l’homme riche,
        mais un faible, perspicace, le démasque !
        
           
         
${}^{12}Le triomphe des justes, on le fête avec éclat ;
        que surgissent des méchants : plus personne !
        
           
         
${}^{13}Qui cache ses fautes ne réussira pas ;
        qui les avoue et s’en détourne obtiendra miséricorde.
        
           
         
${}^{14}Heureux l’homme qui reste vulnérable !
        Qui endurcit son cœur tombera dans le malheur !
        
           
         
${}^{15}Un lion qui rugit, un ours qui charge,
        ainsi le criminel qui domine un peuple faible !
        
           
         
${}^{16}Moins le prince est malin, plus il est rapace !
        Qui renonce à ses intérêts verra de longs jours.
        
           
         
${}^{17}Un homme traqué par le sang qu’il a versé
        fuira jusqu’à la tombe : n’allez pas le retenir !
        
           
         
${}^{18}Qui marche droit sera sauvé ;
        qui louvoie entre deux routes sur l’une des deux tombera !
        
           
         
${}^{19}Qui travaille sa terre aura du pain à satiété ;
        qui poursuit des chimères aura de la misère à satiété !
        
           
         
${}^{20}Un homme de confiance est comblé de bénédictions.
        Courir après la fortune n’a rien d’innocent.
        
           
         
${}^{21}Être partial dans un jugement n’est jamais bon ;
        pour une bouchée de pain, un homme s’en rend coupable.
        
           
         
${}^{22}L’envieux louche sur la fortune
        sans voir que l’indigence s’abat sur lui !
        
           
         
${}^{23}On finit par aimer l’homme qui ose critiquer
        bien plus que l’homme à la langue mielleuse !
        
           
         
${}^{24}Qui met son père et sa mère sur la paille
        en disant : « Ce n’est pas ma faute ! »
        n’est rien d’autre qu’un brigand.
        
           
         
${}^{25}Qui donne libre cours à ses envies provoque des querelles,
        qui se fie au Seigneur sera comblé !
        
           
         
${}^{26}Ne se fier qu’à soi-même, c’est folie !
        Seul le chemin de la sagesse permet d’en réchapper.
        
           
         
${}^{27}Qui donne au pauvre ne manquera de rien ;
        qui détourne les yeux sera chargé de malédictions.
        
           
         
${}^{28}Quand surgissent les méchants, on se cache ;
        quand ils périssent, nombreux sont les justes !
        
           
       
      
         
      \bchapter{}
${}^{1}Qui raidit la nuque sous la critique
        sera brisé, soudain, sans appel !
        
           
         
${}^{2}Quand se multiplient les justes, le peuple est en joie ;
        sous la domination des méchants, il gémit.
        
           
         
${}^{3}Qui a la sagesse pour amie réjouit son père ;
        qui fréquente les prostituées y perdra son bien.
        
           
         
${}^{4}Un roi qui fait justice affermit le pays ;
        rapace, il le ruinera.
        
           
         
${}^{5}Flatter son prochain,
        c’est tendre un filet sous ses pas !
        
           
         
${}^{6}Piège pour le méchant, sa révolte !
        Bonheur et joie pour le juste !
        
           
         
${}^{7}Le juste connaît la cause des faibles,
        le méchant l’ignore.
        
           
         
${}^{8}Les provocateurs embrasent la cité,
        les sages font retomber la colère.
        
           
         
${}^{9}Le sage est-il en procès avec un fou,
        qu’on se fâche ou plaisante,
        plus moyen d’avoir la paix !
        
           
         
${}^{10}Les meurtriers haïssent l’homme intègre,
        les honnêtes gens ne lui veulent que du bien.
        
           
         
${}^{11}L’insensé à toute heure exprime ses humeurs,
        le sage a du recul et les tempère.
        
           
         
${}^{12}Quand le maître prête l’oreille aux mensonges,
        tous les serviteurs tournent mal.
        
           
         
${}^{13}Le pauvre et l’exploiteur se rencontrent :
        à tous deux, le Seigneur donne la lumière !
        
           
         
${}^{14}Le roi qui rend justice aux faibles selon la vérité
        ne cesse d’affermir son trône.
        
           
         
${}^{15}Coups de bâton et remontrances procurent la sagesse ;
        un jeune, renvoyé, fait la honte de sa mère.
        
           
         
${}^{16}Plus les méchants se multiplient, plus le crime prolifère,
        mais les justes seront témoins de leur chute.
        
           
         
${}^{17}Corrige ton fils, il te donnera du repos,
        il fera tes délices !
        
           
         
${}^{18}Faute de prophétie, le peuple se relâche ;
        heureux celui qui observe la loi !
        
           
         
${}^{19}Pour corriger un serviteur, la parole ne suffit pas,
        car il comprend mais n’obtempère pas.
        
           
         
${}^{20}As-tu vu quelqu’un qui parle à tout propos ?
        D’un sot, tu peux attendre davantage.
        
           
         
${}^{21}Qui gâte un serviteur en son jeune âge
        n’en fera qu’un insolent.
        
           
         
${}^{22}Un coléreux provoque des querelles,
        un fou furieux multiplie les crimes.
        
           
         
${}^{23}L’orgueil d’un homme l’humiliera,
        l’esprit humble obtiendra la gloire.
        
           
         
${}^{24}Le complice d’un voleur met sa vie en danger,
        si, appelé à témoigner, il ne le dénonce pas.
        
           
         
${}^{25}Les peurs d’un homme lui sont autant de pièges :
        qui se fie au Seigneur reste hors d’atteinte.
        
           
         
${}^{26}Beaucoup recherchent la faveur d’un chef,
        mais c’est du Seigneur que chacun tient son droit.
        
           
         
${}^{27}Les justes ont horreur des gens pervers,
        le méchant a horreur de qui marche droit.
        
           
      
         
      \bchapter{}
${}^{1}Paroles d’Agour, fils de Yaqé, de Massa.
        Oracle de cet homme pour Ytiel, pour Ytiel et Oukal.
        
           
         
${}^{2}Oui, je suis le plus stupide des hommes,
        aucune intelligence humaine en moi :
${}^{3}je n’ai pas étudié la sagesse,
        mais la science du Dieu saint, je la connais.
        
           
         
${}^{4}Qui est monté au ciel et en est descendu ?
        Qui a retenu le vent au creux de sa main ?
        \\Qui a serré les eaux dans son manteau ?
        Qui a fixé toutes les limites de la terre ?
        \\Quel est son nom ?
        Quel est le nom de son fils ? Sans doute, tu le sais !
        
           
         
        ${}^{5}Toute parole de Dieu est éprouvée au feu ;
        il est un bouclier pour qui s’abrite en lui\\.
        ${}^{6}N’ajoute rien à ce qu’il dit :
        il te le reprocherait comme un mensonge.
        
           
         
        ${}^{7}Seigneur\\,\\je n’ai que deux choses à te demander,
        ne me les refuse pas avant que je meure !
        ${}^{8}Éloigne de moi mensonge et fausseté,
        ne me donne ni pauvreté ni richesse,
        accorde-moi seulement ma part de pain.
        ${}^{9}Car, dans l’abondance, je pourrais te renier
        en disant : « Le Seigneur, qui est-ce ? »
        \\Ou alors, la misère ferait de moi un voleur,
        et je profanerais le nom de mon Dieu !
        
           
         
${}^{10}Ne calomnie pas devant son maître un serviteur,
        il te maudirait, et tu en porterais la faute.
        
           
         
${}^{11}Quelle génération ! Ils maudissent leur père
        et ne bénissent pas leur mère.
${}^{12}Quelle génération ! Ils se croient purs
        et ne balaient pas devant leur porte.
${}^{13}Quelle génération ! Ils prennent un air supérieur
        et vous regardent de haut.
${}^{14}Quelle génération ! Ils ont des dents comme des épées,
        des mâchoires comme des scies
        \\pour dévorer les pauvres du pays,
        retrancher les malheureux d’entre les humains.
        
           
${}^{15}La sangsue a deux filles, « Apporte » et « Apporte »,
        cela fait trois insatiables
        \\et quatre encore qui ne diront jamais « Assez ! » :
${}^{16}le séjour des morts et le ventre stérile,
        \\la terre jamais gorgée d’eau,
        et le feu qui ne dit jamais « Assez ! »
         
${}^{17}L’œil qui toise un père,
        qui dédaigne d’obéir à une mère,
        \\les corbeaux du torrent le crèveront,
        les vautours le dévoreront.
         
${}^{18}Il y a trois merveilles qui me dépassent,
        quatre dont je ne sais rien :
${}^{19}le chemin de l’aigle dans le ciel,
        le chemin du serpent sur le rocher,
        \\le chemin du navire en haute mer
        et le chemin de l’homme chez la jeune fille.
${}^{20}Et puis il y a le chemin de la femme adultère :
        comme si, après y avoir goûté,
        \\elle pouvait s’essuyer la bouche et déclarer :
        « Je n’ai rien fait de mal ! »
         
${}^{21}Il y a trois causes aux tremblements de terre,
        quatre situations que la terre ne peut supporter :
${}^{22}un esclave qui se prend pour le roi,
        un abruti, la panse pleine,
${}^{23}une peste qui trouve un mari
        et une servante qui supplante sa maîtresse.
         
${}^{24}Il y en a quatre, tout petits sur la terre,
        mais sages entre les sages :
${}^{25}les fourmis, race bien faible,
        qui font en été leurs provisions ;
${}^{26}les damans, race chétive,
        qui, dans le rocher, se font un gîte ;
${}^{27}point de roi chez les sauterelles,
        mais elles avancent toutes en bon ordre ;
${}^{28}le lézard, on l’attrape à la main,
        mais il est chez lui au palais du roi.
         
${}^{29}Il y en a trois qui ont fière allure,
        quatre dont la démarche est superbe :
${}^{30}le lion, le plus vaillant des animaux,
        qui ne recule devant rien,
${}^{31}le coq sur ses ergots, et le bouc,
        enfin, le roi, à la tête de son armée.
         
${}^{32}Si tu t’es emporté sottement,
        et si tu t’en rends compte,
        mets la main sur ta bouche !
${}^{33}Car, du lait battu, sort du beurre ;
        d’un coup sur le nez, sort du sang ;
        d’un coup de colère, sort un procès.
      
         
      \bchapter{}
${}^{1}Paroles de Lemouël, roi de Massa,
        instruit par sa mère.
        
           
         
${}^{2}Non, mon fils,
        non, fils de mes entrailles,
        non, fils de mes vœux !
${}^{3}Ne livre pas aux femmes ta vigueur,
        ta conduite à celles qui perdent les rois !
${}^{4}Jamais les rois, Lemouël,
        jamais les rois ne doivent boire de vin,
        ni les princes de boissons fortes,
${}^{5}de peur qu’en buvant ils n’oublient les lois
        et ne lèsent tous les miséreux de leurs droits.
${}^{6}Donne une boisson forte à qui va mourir,
        du vin à qui trouve la vie trop amère :
${}^{7}qu’il boive et qu’il oublie sa misère,
        qu’il cesse de remâcher ses tourments !
${}^{8}Prends la parole en faveur du muet,
        pour la cause de tous les affligés ;
${}^{9}prends la parole et dis le droit
        pour la cause du pauvre et du malheureux !
        
           
      <div class="intertitle niv11">
        Aleph
        ${}^{10}Une femme parfaite\\, qui la trouvera ?
        Elle est précieuse plus que les perles !
      <div class="intertitle niv11">
        Beth
        ${}^{11}Son mari peut lui faire confiance :
        il ne manquera pas de ressources\\.
      <div class="intertitle niv11">
        Guimel
        ${}^{12}Elle fait son bonheur, et non pas sa ruine,
        tous les jours de sa vie.
      <div class="intertitle niv11">
        Daleth
        ${}^{13}Elle sait choisir la laine et le lin,
        et ses mains travaillent volontiers.
      <div class="intertitle niv11">
        Hé
${}^{14}Elle est comme les navires marchands,
        faisant venir ses vivres de très loin.
      <div class="intertitle niv11">
        Waw
${}^{15}Elle est debout quand il fait encore nuit
        pour préparer les repas de sa maison
        et donner ses ordres aux servantes.
      <div class="intertitle niv11">
        Zaïn
${}^{16}A-t-elle des visées sur un champ ? Elle l’acquiert.
        Avec le produit de son travail, elle plante une vigne.
      <div class="intertitle niv11">
        Heth
${}^{17}Elle rayonne de force
        et retrousse ses manches !
      <div class="intertitle niv11">
        Teth
${}^{18}Elle s’assure de la bonne marche des affaires,
        sa lampe ne s’éteint pas de la nuit.
      <div class="intertitle niv11">
        Yod
        ${}^{19}Elle tend la main vers la quenouille,
        ses doigts dirigent le fuseau.
      <div class="intertitle niv11">
        Kaph
        ${}^{20}Ses doigts s’ouvrent en faveur du pauvre,
        elle tend la main au malheureux.
      <div class="intertitle niv11">
        Lamed
${}^{21}Elle ne craint pas la neige pour sa maisonnée,
        car tous les siens ont des vêtements doublés.
      <div class="intertitle niv11">
        Mem
${}^{22}Elle s’est fait des couvertures,
        des vêtements de pourpre et de lin fin.
      <div class="intertitle niv11">
        Noun
${}^{23}Aux portes de la ville, on reconnaît son mari
        siégeant parmi les anciens du pays.
      <div class="intertitle niv11">
        Samek
${}^{24}Elle fabrique de l’étoffe pour la vendre,
        elle propose des ceintures au marchand.
      <div class="intertitle niv11">
        Aïn
${}^{25}Revêtue de force et de splendeur,
        elle sourit à l’avenir.
      <div class="intertitle niv11">
        Pé
${}^{26}Sa bouche s’exprime avec sagesse
        et sa langue enseigne la bonté.
      <div class="intertitle niv11">
        Çadé
${}^{27}Attentive à la marche de sa maison,
        elle ne mange pas le pain de l’oisiveté.
      <div class="intertitle niv11">
        Qoph
${}^{28}Ses fils, debout, la disent bienheureuse
        et son mari fait sa louange :
      <div class="intertitle niv11">
        Resh
${}^{29}« Bien des femmes ont fait leurs preuves,
        mais toi, tu les surpasses toutes ! »
      <div class="intertitle niv11">
        Shine
        ${}^{30}Le charme est trompeur et la beauté s’évanouit ;
        seule, la femme qui craint le Seigneur mérite la louange.
      <div class="intertitle niv11">
        Taw
        ${}^{31}Célébrez-la\\pour les fruits de son travail :
        et qu’aux portes de la ville, ses œuvres disent sa louange !
