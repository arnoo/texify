  
  
    
    \bbook{BARUC}{BARUC}
      
         
      \bchapter{}
      \begin{verse}
${}^{1}Voici les paroles du document que Baruc, fils de Nérias, fils de Maasaias, fils de Sédécias, fils d’Asadias, fils de Helkias, écrivit à Babylone, 
${}^{2}la cinquième année, le septième jour du mois, à l’époque même où les Chaldéens avaient pris Jérusalem et l’avaient livrée au feu. 
${}^{3}Baruc donna lecture de ce document en présence de Jékonias, fils de Joakim, roi de Juda, et de tout le peuple venu pour entendre le Livre, 
${}^{4}en présence des notables et des fils des rois, en présence des anciens et de tout le peuple, du plus petit jusqu’au plus grand, tous ceux qui résidaient à Babylone, au bord de la rivière Soud. 
${}^{5}On pleurait, on jeûnait et on priait devant le Seigneur. 
${}^{6}On récolta de l’argent selon les possibilités de chacun 
${}^{7}et on l’envoya à Jérusalem, au prêtre Joakim, fils de Helkias, fils de Salom, ainsi qu’aux autres prêtres et à tout le peuple, ceux qui se trouvaient avec lui à Jérusalem. 
${}^{8}Joakim, en effet, avait récupéré les objets de la maison du Seigneur, enlevés du Temple, pour les rapporter au pays de Juda, le dixième jour du mois nommé siwân. C’étaient les objets d’argent fabriqués sur l’ordre de Sédécias, fils de Josias, roi de Juda, 
${}^{9}après que Nabucodonosor, roi de Babylone, eut déporté Jékonias, les princes, les forgerons, les notables et le peuple du pays, de Jérusalem à Babylone.
${}^{10}Ils dirent : Voici que nous vous envoyons de l’argent. Avec cette somme, achetez de quoi faire des holocaustes et des sacrifices pour le péché, achetez de l’encens, préparez des offrandes que vous déposerez sur l’autel du Seigneur notre Dieu. 
${}^{11}Priez pour la vie de Nabucodonosor, roi de Babylone, et pour celle de son fils Baltasar, afin que leurs jours se prolongent aussi longtemps que le ciel demeurera au-dessus de la terre. 
${}^{12}Alors, le Seigneur nous donnera la force et illuminera nos yeux ; nous vivrons sous la protection de Nabucodonosor, roi de Babylone, et de son fils Baltasar ; nous les servirons durant de nombreux jours et nous trouverons grâce devant eux. 
${}^{13}Priez aussi pour nous le Seigneur notre Dieu, car nous avons péché contre le Seigneur notre Dieu, et jusqu’à ce jour la fureur et la colère du Seigneur ne se sont pas détournées de nous. 
${}^{14}Vous lirez ce document, que nous vous envoyons pour le proclamer dans la maison du Seigneur au jour de la Fête et aux jours appropriés. 
${}^{15}Vous direz :
      Au Seigneur notre Dieu appartient la justice, mais à nous la honte sur le visage comme on le voit aujourd’hui : honte pour l’homme de Juda et les habitants de Jérusalem, 
${}^{16} pour nos rois et nos chefs, pour nos prêtres, nos prophètes et nos pères ; 
${}^{17} oui, nous avons péché contre le Seigneur, 
${}^{18} nous lui avons désobéi, nous n’avons pas écouté la voix du Seigneur notre Dieu, qui nous disait de suivre les préceptes que le Seigneur nous avait mis sous les yeux. 
${}^{19} Depuis le jour où le Seigneur a fait sortir nos pères du pays d’Égypte jusqu’à ce jour, nous n’avons pas cessé de désobéir au Seigneur notre Dieu ; dans notre légèreté, nous n’avons pas écouté sa voix.
${}^{20}Aussi, comme on le voit aujourd’hui, le malheur s’est attaché à nous, avec la malédiction que le Seigneur avait fait prononcer par son serviteur Moïse, au jour où il a fait sortir nos pères du pays d’Égypte pour nous donner une terre ruisselant de lait et de miel. 
${}^{21} Nous n’avons pas écouté la voix du Seigneur notre Dieu, à travers toutes les paroles des prophètes qu’il nous envoyait. 
${}^{22} Chacun de nous, selon la pensée de son cœur mauvais, est allé servir d’autres dieux et faire ce qui est mal aux yeux du Seigneur notre Dieu.
      
         
      \bchapter{}
      \begin{verse}
${}^{1}Le Seigneur a donc accompli la parole qu’il avait prononcée contre nous, contre nos juges, qui gouvernèrent Israël, contre nos rois et nos chefs, contre les gens d’Israël et de Juda. 
${}^{2}Sous l’immensité du ciel, il ne se produisit jamais rien de semblable à ce que le Seigneur fit advenir à Jérusalem, conformément à ce qui est écrit dans la loi de Moïse, 
${}^{3}à savoir que nous mangerions l’un la chair de son fils, l’autre la chair de sa fille ! 
${}^{4}Et le Seigneur a livré nos pères aux mains de tous les royaumes d’alentour, à l’outrage et à la désolation parmi tous les peuples d’alentour, où il les a dispersés. 
${}^{5}Ils ont été soumis, au lieu de dominer, car nous avons péché contre le Seigneur notre Dieu en n’écoutant pas sa voix.
${}^{6}Au Seigneur notre Dieu appartient la justice, mais à nous et à nos pères la honte sur le visage, comme on le voit aujourd’hui. 
${}^{7}Ces malheurs que le Seigneur avait prononcés contre nous se sont tous abattus sur nous. 
${}^{8}Nous n’avons pas apaisé la face du Seigneur, nous ne nous sommes pas détournés chacun des pensées de son cœur mauvais. 
${}^{9}Aussi, dans sa vigilance, le Seigneur a déclenché contre nous ces malheurs. Oui, il est juste, le Seigneur, en tout ce qu’il nous a commandé de faire. 
${}^{10}Mais nous n’avons pas écouté sa voix qui nous disait de suivre les préceptes que le Seigneur nous avait mis sous les yeux.
${}^{11}Et maintenant, Seigneur Dieu d’Israël, écoute, toi qui as fait sortir ton peuple du pays d’Égypte, par la force de ta main, avec signes et prodiges, par ta grande puissance et la vigueur de ton bras, toi qui t’es fait un nom, comme on le voit aujourd’hui. 
${}^{12}Nous avons péché, nous avons été impies, nous avons été injustes, Seigneur notre Dieu. En raison de tous tes jugements, 
${}^{13}que ta fureur se détourne de nous, car nous ne sommes plus qu’un petit nombre parmi les nations où tu nous as dispersés. 
${}^{14}Entends, Seigneur, notre prière et notre supplication ; délivre-nous à cause de toi-même ; fais-nous trouver grâce devant ceux qui nous ont déportés, 
${}^{15}afin que la terre entière sache que tu es le Seigneur notre Dieu, puisque ton nom a été invoqué sur Israël et sa descendance. 
${}^{16}Regarde, Seigneur, du haut de ta demeure sainte, et pense à nous. Incline ton oreille, Seigneur, écoute. 
${}^{17}Ouvre les yeux, Seigneur, et vois. Car ce ne sont pas les défunts dans le séjour des morts, eux dont le souffle est enlevé des entrailles, qui rendront gloire et justice au Seigneur, 
${}^{18}mais l’âme comblée de tristesse, l’homme qui marche courbé et sans force, les yeux défaillants et l’âme affamée, ceux-là te rendront gloire et justice, Seigneur.
${}^{19}Non, ce n’est pas à cause des actions justes de nos pères et de nos rois, que nous déposons notre supplique devant ta face, Seigneur notre Dieu. 
${}^{20}Car tu as déchaîné sur nous ta fureur et ta colère, comme tu l’avais déclaré par tes serviteurs les prophètes, en disant :
${}^{21}« Ainsi parle le Seigneur : Courbez les épaules et servez le roi de Babylone. Alors, vous resterez dans le pays que j’ai donné à vos pères. 
${}^{22}Mais si vous n’écoutez pas l’appel du Seigneur à servir le roi de Babylone, 
${}^{23}je ferai cesser dans les villes de Juda et dans les rues de Jérusalem le chant d’allégresse et le chant de joie, le chant de l’époux et le chant de l’épouse. Tout le pays sera désolé, vidé de ses habitants. »
${}^{24}Or, nous n’avons pas écouté ton appel à servir le roi de Babylone, et tu as accompli les paroles que tu avais prononcées par tes serviteurs les prophètes, à savoir que les ossements de nos rois et de nos pères seraient arrachés à leur tombeau. 
${}^{25}Et voici qu’ils furent jetés dehors, à la brûlure du jour et au gel de la nuit. Nos rois et nos pères sont morts dans de cruelles souffrances, par la famine, l’épée et l’exil.
${}^{26}Tu as établi la maison sur laquelle a été invoqué ton nom, comme elle est en ce jour, à cause de la méchanceté de la maison d’Israël et de la maison de Juda. 
${}^{27}Tu as agi envers nous selon ton entière bienveillance et ton immense tendresse, Seigneur notre Dieu, 
${}^{28}comme tu l’avais déclaré par ton serviteur Moïse, le jour où tu lui ordonnas de mettre par écrit ta Loi en présence des fils d’Israël, en disant :
${}^{29}« Vraiment, si vous n’écoutez pas ma voix, cette grande, cette immense multitude elle-même sera réduite à un petit nombre parmi les nations où je les disperserai ! 
${}^{30}Oui, je sais bien qu’ils ne m’écouteront pas, car c’est un peuple à la nuque raide. Mais dans le pays de leur exil, ils rentreront en eux-mêmes, 
${}^{31}et ils sauront que moi, je suis le Seigneur leur Dieu. Je leur donnerai un cœur et des oreilles qui entendent. 
${}^{32}Et là, dans le pays de leur exil, ils me loueront, ils se souviendront de mon nom, 
${}^{33}ils se repentiront de leur obstination et de leurs actions mauvaises, au souvenir du destin de leurs pères, qui avaient péché devant le Seigneur. 
${}^{34}Alors, je les ferai revenir dans le pays que j’ai promis par serment à leurs pères Abraham, Isaac et Jacob, et ils y seront maîtres. Je les multiplierai, ils ne diminueront plus ! 
${}^{35}J’établirai pour eux une alliance éternelle : je serai leur Dieu et eux, ils seront mon peuple. Jamais plus, je ne déplacerai mon peuple Israël loin du pays que je leur ai donné. »
      
         
      \bchapter{}
      \begin{verse}
${}^{1}Seigneur, Souverain de l’univers, Dieu d’Israël, une âme angoissée, un esprit découragé crie vers toi. 
${}^{2}Écoute, Seigneur, et prends pitié, car nous avons péché contre toi. 
${}^{3}Toi, en effet, tu demeures à jamais ; nous, nous sommes à jamais perdus. 
${}^{4}Seigneur, Souverain de l’univers, Dieu d’Israël, écoute donc la prière des morts d’Israël, des fils de ceux qui ont péché contre toi, qui n’ont pas écouté la voix du Seigneur leur Dieu, de sorte que les malheurs se sont attachés à nous. 
${}^{5}Ne te souviens pas des injustices de nos pères, mais souviens-toi, en cette heure, de ta main et de ton nom. 
${}^{6}Car tu es le Seigneur notre Dieu, et nous voulons te louer, Seigneur. 
${}^{7}Oui, c’est pour cela que tu as mis ta crainte en notre cœur, pour que nous invoquions ton nom. Nous voulons te louer en notre exil, puisque nous avons détourné de notre cœur toute l’injustice de nos pères qui ont péché contre toi. 
${}^{8}Nous voici aujourd’hui dans cet exil, où tu nous as dispersés, pour y être objet d’outrage et de malédiction, et pour notre amendement, après toutes les fautes de nos pères, qui s’étaient éloignés du Seigneur notre Dieu.
      
         
        ${}^{9}Écoute, Israël, les commandements de vie,
        prête l’oreille pour acquérir la connaissance.
        ${}^{10}Pourquoi donc, Israël,
        \\pourquoi es-tu exilé\\chez tes ennemis\\,
        vieillissant sur une terre étrangère,
        ${}^{11}souillé par le contact des cadavres,
        inscrit parmi les habitants du séjour des morts ?
        ${}^{12}– Parce que tu as abandonné la Source de la Sagesse !
        ${}^{13}Si tu avais suivi les chemins de Dieu,
        tu vivrais dans la paix pour toujours.
        ${}^{14}Apprends où se trouvent
        et la connaissance, et la force, et l’intelligence ;
        \\pour savoir en même temps où se trouvent
        de longues années de vie,
        la lumière des yeux et la paix.
         
        ${}^{15}Mais qui donc a découvert la demeure de la Sagesse\\,
        qui a pénétré jusqu’à ses trésors ?
${}^{16}Où sont-ils, les chefs des nations,
        ceux qui domptent les bêtes de la terre,
${}^{17}qui se jouent des oiseaux du ciel,
        et qui entassent l’argent et l’or,
        \\– ces biens auxquels les hommes accordent leur confiance –
        mais dont les possessions n’ont pas de fin ?
${}^{18}Où sont-ils, ceux qui travaillent l’argent avec soin,
        mais dont les œuvres ne laissent pas de traces ?
${}^{19}Ils ont disparu, ils sont descendus dans le séjour des morts,
        et d’autres se sont levés à leur place.
${}^{20}De plus jeunes ont vu le jour,
        se sont installés sur la terre,
        \\mais le chemin du savoir, ils ne l’ont pas connu,
${}^{21}ils n’ont pas compris ses sentiers,
        ils ne l’ont pas saisi.
        \\Même leurs fils sont restés loin de son chemin.
${}^{22}De la Sagesse, on n’a rien entendu en Canaan ;
        en Témane, nul ne l’a vue.
${}^{23}Ni les fils d’Agar, recherchant l’intelligence sur la terre,
        ni les marchands de Merrane et de Témane,
        \\ni les conteurs de fables, ni les chercheurs d’intelligence
        n’ont connu le chemin de la Sagesse.
        \\Ils n’ont pas gardé mémoire de ses sentiers.
${}^{24}Ô Israël, comme elle est grande, la maison de Dieu,
        comme il est vaste, le domaine qui lui appartient !
${}^{25}Grand et sans borne,
        élevé, sans mesure !
${}^{26}C’est là que naquirent les célèbres géants des origines,
        de haute stature, experts à la guerre.
${}^{27}Ce n’est pas eux que Dieu a choisis,
        il ne leur a pas montré le chemin du savoir.
${}^{28}Ils ont péri, par manque de connaissance,
        ils ont péri à cause de leur imprudence.
${}^{29}Qui est monté au ciel pour saisir la Sagesse
        et la faire descendre des nuées ?
${}^{30}Qui a traversé la mer pour la trouver
        et la rapporter au prix d’un or très fin ?
${}^{31}Nul ne connaît son chemin,
        nul n’examine son sentier.
        ${}^{32}Mais celui qui sait tout en connaît le chemin,
        il l’a découvert par son intelligence.
        \\Il a pour toujours aménagé la terre,
        et l’a peuplée de troupeaux.
        ${}^{33}Il lance la lumière, et elle prend sa course ;
        il la rappelle, et elle obéit en tremblant.
        ${}^{34}Les étoiles brillent, joyeuses, à leur poste de veille ;
        ${}^{35}il les appelle, et elles répondent : « Nous voici ! »
        \\Elles brillent avec joie pour celui qui les a faites.
        ${}^{36}C’est lui qui est notre Dieu :
        aucun autre ne lui est comparable.
        ${}^{37}Il a découvert les chemins du savoir,
        et il les a confiés à Jacob, son serviteur,
        à Israël, son bien-aimé.
         
        ${}^{38}Ainsi, la Sagesse\\est apparue sur la terre,
        elle a vécu parmi les hommes.
      
         
      \bchapter{}
        ${}^{1}Elle est le livre des préceptes de Dieu,
        la Loi qui demeure éternellement :
        \\tous ceux qui l’observent vivront,
        ceux qui l’abandonnent mourront.
        ${}^{2} Reviens, Jacob, saisis-la de nouveau ;
        à sa lumière, marche vers la splendeur :
        ${}^{3}ne laisse pas ta gloire à un autre,
        tes privilèges à un peuple étranger.
        ${}^{4}Heureux sommes-nous, Israël !
        Car ce qui plaît à Dieu, nous le connaissons.
        
           
        ${}^{5}Courage, mon peuple,
        toi qui es la part d’Israël réservée à Dieu\\ !
        ${}^{6}Vous avez été vendus aux nations païennes,
        mais ce n’était pas pour votre anéantissement ;
        \\vous avez excité la colère de Dieu :
        c’est pour cela que vous avez été livrés à vos adversaires.
        ${}^{7}Car vous avez irrité votre Créateur
        en offrant des sacrifices aux démons et non à Dieu.
        ${}^{8}Vous avez oublié le Dieu éternel,
        lui qui vous a nourris.
        \\Vous avez aussi attristé Jérusalem,
        elle qui vous a élevés,
        ${}^{9}car elle a vu fondre sur vous la colère qui vient de Dieu,
        \\et elle a dit :
        \\« Écoutez, voisines de Sion,
        Dieu m’a infligé un deuil cruel.
        ${}^{10}J’ai vu la captivité
        que l’Éternel a infligée à mes fils et à mes filles.
        ${}^{11}Je les avais élevés dans la joie,
        je les ai laissés partir dans les larmes et le deuil.
        ${}^{12}Que nul ne se réjouisse de mon sort,
        à moi qui suis veuve et délaissée par tout le monde\\.
        \\J’ai été abandonnée
        à cause des péchés de mes enfants,
        \\parce qu’ils se sont détournés de la loi de Dieu.
${}^{13}Ils n’ont pas reconnu ses jugements,
        \\ni marché sur les chemins des commandements de Dieu,
        ni suivi les sentiers de l’éducation
        conforme à sa justice.
${}^{14}Qu’elles viennent, les voisines de Sion !
        \\Souvenez-vous de la captivité que l’Éternel a infligée
        à mes fils et à mes filles.
${}^{15}Car il a fait venir contre eux une nation lointaine,
        une nation insolente et de langue étrangère,
        \\sans respect pour le vieillard,
        sans pitié pour le petit enfant.
${}^{16}Ils ont emmené les fils bien-aimés de la veuve ;
        ils l’ont rendue solitaire en la privant de ses filles.
${}^{17}Et moi, comment pourrais-je vous aider ?
${}^{18}Celui qui vous a infligé ces malheurs,
        lui seul vous arrachera à la main de vos ennemis.
${}^{19}Allez, mes enfants, allez votre chemin !
        Moi, délaissée, je reste solitaire.
${}^{20}J’ai quitté la robe de paix,
        j’ai revêtu le sac du suppliant ;
        \\vers l’Éternel je lancerai mon cri,
        au long de mes jours.
${}^{21}Courage, mes enfants, criez vers Dieu !
        \\Il vous arrachera au pouvoir,
        à la main des ennemis.
${}^{22}Car moi, j’ai mis dans l’Éternel mon espérance,
        pour qu’il vous accorde le salut.
        \\Et il m’est venu une joie,
        de la part du Dieu Saint,
        \\en raison de la miséricorde qui bientôt vous sera envoyée
        par l’Éternel, votre Sauveur.
${}^{23}Dans le deuil et les larmes, je vous ai laissés partir ;
        \\mais Dieu vous ramènera vers moi, pour toujours,
        dans la joie et l’allégresse.
${}^{24}Comme les voisines de Sion voient maintenant votre captivité,
        ainsi verront-elles bientôt le salut que Dieu vous accordera,
        \\qui viendra vers vous avec grande gloire,
        dans la splendeur de l’Éternel.
${}^{25}Mes enfants, supportez avec patience
        la colère qui vous est venue de Dieu.
        \\L’ennemi t’a poursuivi,
        mais bientôt tu verras sa ruine,
        tu mettras ton pied sur sa nuque.
${}^{26}Mes tendres enfants ont marché par de rudes chemins ;
        ils ont été enlevés comme un troupeau
        emporté par l’ennemi.
        ${}^{27}Courage, mes enfants, criez vers Dieu !
        Celui qui vous a infligé l’épreuve\\se souviendra de vous.
        ${}^{28}Votre pensée vous a égarés loin de Dieu ;
        \\une fois convertis,
        mettez dix fois plus d’ardeur à le chercher.
        ${}^{29}Car celui qui a fait venir sur vous ces calamités
        fera venir sur vous la joie éternelle,
        en assurant votre salut. »
         
${}^{30}Courage, Jérusalem !
        Il te consolera, celui qui t’a donné un nom.
${}^{31}Malheur à ceux qui t’ont maltraitée
        et se sont réjouis de ta chute,
${}^{32}malheur aux villes dont tes enfants furent esclaves,
        malheur à celle qui reçut tes fils !
${}^{33}Comme elle s’était réjouie de ta chute
        et avait pris plaisir à ton effondrement,
        \\ainsi sera-t-elle attristée par sa propre dévastation.
${}^{34}Je lui ôterai sa fierté de ville populeuse,
        je changerai son insolence en deuil.
${}^{35}Un feu lui surviendra de la part de l’Éternel
        pour de longs jours ;
        \\elle sera la demeure des démons
        pour plus longtemps encore.
${}^{36}Regarde vers l’orient, Jérusalem,
        et vois l’allégresse qui te vient de Dieu.
${}^{37}Voici : ils reviennent, les fils que tu laissas partir ;
        ils reviennent, rassemblés du levant au couchant
        par la parole du Dieu Saint,
        \\ils se réjouissent de la gloire de Dieu.
      
         
      \bchapter{}
        ${}^{1}Jérusalem, quitte ta robe de tristesse et de misère,
        et revêts la parure de la gloire de Dieu pour toujours,
        ${}^{2}enveloppe-toi dans le manteau de la justice de Dieu,
        mets sur ta tête le diadème de la gloire de l’Éternel.
        ${}^{3}Dieu va déployer ta splendeur partout sous le ciel,
        ${}^{4}car Dieu, pour toujours, te donnera ces noms :
        \\« Paix-de-la-justice »
        et « Gloire-de-la-piété-envers-Dieu ».
        ${}^{5}Debout, Jérusalem ! tiens-toi sur la hauteur,
        et regarde vers l’orient :
        \\vois tes enfants rassemblés du couchant au levant
        par la parole du Dieu\\Saint ;
        \\ils se réjouissent parce que Dieu se souvient.
        ${}^{6}Tu les avais vus partir\\à pied,
        emmenés par les ennemis,
        \\et Dieu te les ramène, portés en triomphe,
        comme sur un trône royal.
        ${}^{7}Car Dieu a décidé
        que les hautes montagnes et les collines éternelles seraient abaissées,
        et que les vallées seraient comblées :
        \\ainsi la terre sera aplanie,
        afin qu’Israël chemine en sécurité
        dans la gloire de Dieu.
        ${}^{8}Sur l’ordre de Dieu,
        les forêts et les arbres odoriférants
        donneront à Israël leur ombrage ;
        ${}^{9}car Dieu conduira Israël dans la joie,
        à la lumière de sa gloire,
        avec sa miséricorde et sa justice.
        
           
