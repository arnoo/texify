  
  
      
         
      \bchapter{}
      \begin{verse}
${}^{1}Sédécias avait vingt et un ans lorsqu’il devint roi, et il régna onze ans à Jérusalem. Sa mère s’appelait Hamoutal, fille de Jérémie ; elle était de Libna. 
${}^{2}Il fit ce qui est mal aux yeux du Seigneur, tout comme avait fait Joakim. 
${}^{3}C’est à cause de la colère du Seigneur qu’il en fut ainsi à Jérusalem et en Juda, jusqu’à ce qu’il les rejette loin de sa face. Mais Sédécias se révolta contre le roi de Babylone.
${}^{4}La neuvième année du règne de Sédécias, le dixième jour du dixième mois, Nabucodonosor, roi de Babylone, vint attaquer Jérusalem avec toute son armée. Les Chaldéens établirent leur camp devant la ville qu’ils entourèrent d’un retranchement. 
${}^{5}La ville fut assiégée jusqu’à la onzième année du règne de Sédécias.
${}^{6}Le neuvième jour du quatrième mois, comme la famine était devenue terrible dans la ville et que les gens du pays n’avaient plus de pain, 
${}^{7}une brèche fut ouverte dans le rempart de la ville. Mais tous les hommes de guerre prirent la fuite ; ils sortirent de la ville, la nuit, par la porte du double rempart, près du jardin du roi, dans la direction de la plaine du Jourdain, pendant que les Chaldéens cernaient la ville.
${}^{8}Les troupes chaldéennes poursuivirent le roi et le rattrapèrent dans la plaine de Jéricho ; toute son armée en débandade l’avait abandonné. 
${}^{9}Les Chaldéens s’emparèrent du roi, ils le menèrent à Ribla, au pays de Hamath, auprès du roi de Babylone qui prononça la sentence contre lui. 
${}^{10}Le roi de Babylone égorgea les fils de Sédécias, sous les yeux de leur père. De même, tous les princes de Juda, il les égorgea à Ribla. 
${}^{11}Puis il creva les yeux de Sédécias et le fit attacher avec une double chaîne de bronze. Alors, le roi de Babylone l’emmena à Babylone où il le fit mettre en prison jusqu’au jour de sa mort.
${}^{12}Le dixième jour du cinquième mois, la dix-neuvième année du règne de Nabucodonosor, roi de Babylone, Nabouzardane, commandant de la garde, celui qui se tient en présence du roi de Babylone, fit son entrée à Jérusalem. 
${}^{13}Il incendia la maison du Seigneur et la maison du roi ; il incendia toutes les maisons de Jérusalem – toutes les maisons des notables. 
${}^{14}Toutes les troupes chaldéennes qui étaient avec lui abattirent tous les remparts de Jérusalem. 
${}^{15}Nabouzardane déporta une partie du petit peuple, ceux qui étaient restés dans la ville, les déserteurs qui s’étaient ralliés au roi de Babylone, ainsi que le reste des artisans. 
${}^{16}Il laissa seulement une partie du petit peuple de la campagne, pour avoir des vignerons et des laboureurs.
${}^{17}Les colonnes de bronze qui se trouvaient dans la maison du Seigneur, les bases et la Mer de bronze qui se trouvaient dans la maison du Seigneur, les Chaldéens les brisèrent et ils en emportèrent tout le bronze à Babylone. 
${}^{18}Ils prirent également les vases, les pelles, les ciseaux, les bols pour l’aspersion, les gobelets et tous les objets de bronze qui servaient au culte. 
${}^{19}Le commandant de la garde prit les récipients, les brûle-parfums, les bols pour l’aspersion, les chaudrons, les chandeliers, les gobelets et les timbales, tout ce qui était en or et tout ce qui était en argent. 
${}^{20}Les deux colonnes, la Mer – qui était unique –, les douze bœufs en bronze qui étaient au-dessous, les bases que le roi Salomon avait faites pour la maison du Seigneur, tous ces objets étaient d’un poids de bronze qu’on ne pouvait évaluer. 
${}^{21}Les colonnes avaient chacune dix-huit coudées de hauteur ; un cordon de douze coudées en faisait le tour. Chacune était creuse, d’une épaisseur de quatre doigts ; 
${}^{22}elle était surmontée d’un chapiteau de bronze, et la hauteur du chapiteau était de cinq coudées. Il y avait un filet et des grenades tout autour du chapiteau. Le tout était en bronze. La deuxième colonne, avec ses grenades, était semblable à la première. 
${}^{23}Il y avait quatre-vingt-seize grenades, en relief. Sur le filet tout autour, le total des grenades était de cent.
${}^{24}Le commandant de la garde prit Seraya, chef des prêtres, Sophonie, prêtre en second, et les trois gardiens du seuil. 
${}^{25}Dans la ville, il prit un dignitaire, celui qui était préposé aux hommes de guerre, sept hommes parmi les familiers du roi qui furent trouvés dans la ville, puis le secrétaire du chef de l’armée, chargé d’enrôler les gens du pays, et soixante hommes parmi les gens du pays, qui se trouvaient au milieu de la ville. 
${}^{26}Nabouzardane, commandant de la garde, les ayant pris, les amena au roi de Babylone, à Ribla. 
${}^{27}Le roi de Babylone les frappa et les mit à mort, à Ribla, au pays de Hamath. Et Juda fut déporté loin de sa terre.
       
${}^{28}Voici le nombre des gens que Nabucodonosor déporta la septième année : 3 023 Judéens ; 
${}^{29}la dix-huitième année de Nabucodonosor : 832 personnes de Jérusalem ; 
${}^{30}la vingt-troisième année de Nabucodonosor, Nabouzardane, commandant de la garde, déporta 745 Judéens. Au total : 4 600 personnes.
${}^{31}La trente-septième année de la déportation de Jékonias, roi de Juda, le douzième mois, le vingt-cinq du mois, Évil-Mérodak, roi de Babylone, l’année même où il devint roi, fit grâce à Jékonias, roi de Juda, et le fit sortir de prison. 
${}^{32}Il lui parla avec bonté et lui accorda un rang plus élevé que celui des rois qui étaient avec lui à Babylone. 
${}^{33}Il lui fit quitter ses vêtements de prisonnier, et Jékonias prit désormais ses repas en présence du roi, tous les jours de sa vie. 
${}^{34}Sa subsistance fut assurée en permanence par le roi de Babylone, jour après jour, jusqu’au jour de sa mort, tous les jours de sa vie.
