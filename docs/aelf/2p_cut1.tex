  
  
    
    \bbook{DEUXIÈME LETTRE DE SAINT PIERRE}{DEUXIÈME LETTRE DE SAINT PIERRE}
      
         
      \bchapter{}
        ${}^{1}Syméon Pierre,
        serviteur et apôtre de Jésus Christ,
        \\à ceux qui ont reçu en partage
        une foi d’aussi grand prix que la nôtre,
        par la justice de notre Dieu et Sauveur Jésus Christ.
        ${}^{2}Que la grâce et la paix
        vous soient accordées en abondance
        \\par la vraie connaissance de Dieu
        et de Jésus notre Seigneur.
        
           
${}^{3}Sa puissance divine nous a fait don de tout ce qui permet de vivre avec piété, grâce à la vraie connaissance de celui qui nous a appelés par la gloire et la force qui lui appartiennent. 
${}^{4}De la sorte nous sont accordés les dons promis, si précieux et si grands, pour que, par eux, vous deveniez participants de la nature divine, et que vous échappiez à la dégradation produite dans le monde par la convoitise. 
${}^{5}Et pour ces motifs, faites tous vos efforts pour joindre à votre foi la vertu, à la vertu la connaissance de Dieu, 
${}^{6}à la connaissance de Dieu la maîtrise de soi, à la maîtrise de soi la persévérance, à la persévérance la piété, 
${}^{7}à la piété la fraternité, à la fraternité l’amour. 
${}^{8}Si vous avez tout cela en abondance, vous n’êtes pas inactifs ni stériles pour la vraie connaissance de notre Seigneur Jésus Christ. 
${}^{9}Mais celui qui en est dépourvu est myope au point d’être aveugle : il oublie qu’il a été purifié de ses péchés d’autrefois. 
${}^{10}C’est pourquoi, frères, redoublez d’efforts pour confirmer l’appel et le choix dont vous avez bénéficié ; en agissant de la sorte, vous ne risquez pas de tomber. 
${}^{11}C’est ainsi que vous sera généreusement accordée l’entrée dans le royaume éternel de notre Seigneur et Sauveur Jésus Christ.
${}^{12}Voilà pourquoi je tiendrai toujours à vous remettre cela en mémoire, bien que vous le sachiez et que vous soyez affermis dans la vérité qui est déjà là. 
${}^{13}Et il me paraît juste, tant que je suis ici-bas, de vous tenir éveillés par ces rappels, 
${}^{14}car je sais que bientôt je partirai d’ici-bas, comme notre Seigneur Jésus Christ me l’a fait savoir. 
${}^{15}Mais je redoublerai d’efforts pour qu’après mon départ vous puissiez en toute occasion faire mémoire de cela.
${}^{16}En effet, ce n’est pas en ayant recours à des récits imaginaires sophistiqués que nous vous avons fait connaître la puissance et la venue de notre Seigneur Jésus Christ, mais c’est pour avoir été les témoins oculaires de sa grandeur. 
${}^{17}Car il a reçu de Dieu le Père l’honneur et la gloire quand, depuis la Gloire magnifique, lui parvint une voix qui disait : Celui-ci est mon Fils, mon bien-aimé ; en lui j’ai toute ma joie. 
${}^{18}Cette voix venant du ciel, nous l’avons nous-mêmes entendue quand nous étions avec lui sur la montagne sainte. 
${}^{19}Et ainsi se confirme pour nous la parole prophétique ; vous faites bien de fixer votre attention sur elle, comme sur une lampe brillant dans un lieu obscur jusqu’à ce que paraisse le jour et que l’étoile du matin se lève dans vos cœurs. 
${}^{20}Car vous savez cette chose primordiale : pour aucune prophétie de l’Écriture il ne peut y avoir d’interprétation individuelle, 
${}^{21}puisque ce n’est jamais par la volonté d’un homme qu’un message prophétique a été porté : c’est portés par l’Esprit Saint que des hommes ont parlé de la part de Dieu.
      
         
      \bchapter{}
      \begin{verse}
${}^{1}Mais il y eut aussi des prophètes de mensonge dans le peuple, comme il y aura parmi vous des maîtres de mensonge, qui introduiront des hérésies menant à la perdition et renieront le Maître souverain qui les a rachetés. Ils préparent pour bientôt leur perdition. 
${}^{2}Beaucoup les suivront dans leurs débauches ; à cause d’eux, suivre le chemin de la vérité fera l’objet d’outrages, 
${}^{3}et dans leur cupidité, ils vous exploiteront par des discours factices ; leur condamnation est en cours depuis longtemps, et leur perdition n’est pas en sommeil.
${}^{4}Car Dieu n’a pas épargné les anges qui avaient péché, mais il les a livrés, enchaînés, aux ténèbres infernales, où ils sont gardés pour le jugement. 
${}^{5}Il n’a pas non plus épargné le monde des origines, mais, quand il a fait venir le déluge sur le monde des impies, il a protégé huit personnes, dont Noé qui proclamait la justice. 
${}^{6}Il a condamné aussi les villes de Sodome et Gomorrhe à la catastrophe en les réduisant en cendres ; il en a fait un exemple pour montrer aux impies ce qui les attend. 
${}^{7}Mais il a délivré Loth, le juste, accablé par la conduite débauchée de ces gens dévoyés : 
${}^{8}en effet, avec ce qu’il voyait et entendait, ce juste, en habitant au milieu d’eux, mettait, jour après jour, son âme de juste à la torture, à cause de leurs actions contraires à la loi. 
${}^{9}Le Seigneur peut donc délivrer de l’épreuve ceux qui pratiquent la piété, mais les injustes, il les garde pour le jour du jugement afin de les punir, 
${}^{10}ceux-là surtout qui, par convoitise impure, suivent les inclinations de la chair et dédaignent la seigneurie de Dieu. Présomptueux, arrogants, ils outragent sans trembler les anges appelés « Gloires », 
${}^{11}alors que d’autres anges, supérieurs en force et en puissance, ne portent pas contre ceux-ci un jugement outrageant de la part du Seigneur.
${}^{12}Ces gens-là sont comme des bêtes privées de raison, engendrées par la nature pour être capturées et détruites ; outrageant ce qu’ils ignorent, ils seront détruits comme ces bêtes seront détruites ; 
${}^{13}ils subiront l’injustice comme salaire de leur injustice. Ils pensent trouver leur plaisir à vivre dans les délices en plein jour, ils ne sont que taches et défauts, en se délectant de leurs tromperies quand ils font bombance avec vous. 
${}^{14}Ils ont les yeux remplis du désir d’adultère et sont insatiables de péchés. Ils séduisent les âmes mal affermies, ils ont le cœur exercé à la cupidité : ce sont des enfants de malédiction. 
${}^{15}Abandonnant le droit chemin, ils se sont égarés en s’engageant sur le chemin de Balaam fils de Bosor ; celui-ci fut heureux de recevoir un salaire d’injustice, 
${}^{16}mais il reçut une leçon pour sa transgression : une bête de somme sans voix s’est mise à parler avec une voix humaine et s’est opposée à la folie du prophète. 
${}^{17}Ces gens-là sont des sources sans eau, des brumes chassées par la tempête ; l’obscurité des ténèbres leur est réservée. 
${}^{18}En proférant des énormités vides de sens, ils séduisent, par des convoitises nées de la chair, par les débauches, ceux qui viennent à peine d’échapper aux gens qui vivent dans l’égarement. 
${}^{19}Ils leur promettent la liberté, alors qu’eux-mêmes sont esclaves de la corruption : on est, en effet, esclave de ce qui vous domine. 
${}^{20}Car si des hommes, par la vraie connaissance de notre Seigneur et Sauveur Jésus Christ, ont échappé aux souillures du monde, et qu’ils se trouvent à nouveau empêtrés et dominés par elles, leur état est pire à la fin qu’au début. 
${}^{21}Il aurait mieux valu pour eux ne pas avoir connu le chemin de la justice que de l’avoir connu et de s’être détournés du saint commandement qui leur avait été transmis. 
${}^{22}Il leur arrive ce que dit en vérité le proverbe : Le chien retourne à son vomissement, et : La truie, sitôt lavée, se vautre dans la boue.
      
         
      \bchapter{}
      \begin{verse}
${}^{1}Bien-aimés, c’est déjà la deuxième lettre que je vous écris. Dans l’une et l’autre, je fais appel à votre mémoire, afin de réveiller en vous une intelligence claire, 
${}^{2}pour que vous vous souveniez des paroles dites à l’avance par les saints prophètes, et du commandement de vos apôtres, qui est celui du Seigneur et Sauveur. 
${}^{3}Sachez d’abord que, dans les derniers jours, des moqueurs viendront avec leurs moqueries, allant au gré de leurs convoitises, 
${}^{4}et disant : « Où en est la promesse de son avènement ? En effet, depuis que les pères se sont endormis dans la mort, tout reste pareil depuis le début de la création. »
${}^{5}En prétendant cela, ils oublient que, jadis, il y avait des cieux, ainsi qu’une terre sortie de l’eau et constituée au milieu de l’eau grâce à la parole de Dieu. 
${}^{6}Par ces mêmes éléments, le monde d’alors périt dans les eaux du déluge. 
${}^{7}Mais les cieux et la terre de maintenant, la même parole les réserve et les garde pour le feu, en vue du jour où les hommes impies seront jugés et périront.
${}^{8}Bien-aimés, il est une chose qui ne doit pas vous échapper : pour le Seigneur, un seul jour est comme mille ans, et mille ans sont comme un seul jour. 
${}^{9}Le Seigneur ne tarde pas à tenir sa promesse, alors que certains prétendent qu’il a du retard. Au contraire, il prend patience envers vous, car il ne veut pas en laisser quelques-uns se perdre, mais il veut que tous parviennent à la conversion.
${}^{10}Cependant le jour du Seigneur viendra, comme un voleur. Alors les cieux disparaîtront avec fracas, les éléments embrasés seront dissous, la terre, avec tout ce qu’on a fait ici-bas, ne pourra y échapper. 
${}^{11}Ainsi, puisque tout cela est en voie de dissolution, vous voyez quels hommes vous devez être, en vivant dans la sainteté et la piété, 
${}^{12}vous qui attendez, vous qui hâtez l’avènement du jour de Dieu, ce jour où les cieux enflammés seront dissous, où les éléments embrasés seront en fusion. 
${}^{13}Car ce que nous attendons, selon la promesse du Seigneur, c’est un ciel nouveau et une terre nouvelle où résidera la justice.
${}^{14}C’est pourquoi, bien-aimés, en attendant cela, faites tout pour qu’on vous trouve sans tache ni défaut, dans la paix. 
${}^{15}Et dites-vous bien que la longue patience de notre Seigneur, c’est votre salut, comme vous l’a écrit également Paul, notre frère bien-aimé, avec la sagesse qui lui a été donnée. 
${}^{16}C’est ce qu’il dit encore dans toutes les lettres où il traite de ces sujets ; on y trouve des textes difficiles à comprendre, que torturent des gens sans instruction et sans solidité, comme ils le font pour le reste des Écritures : cela les mène à leur propre perdition.
${}^{17}Quant à vous, bien-aimés, vous voilà prévenus ; prenez garde : ne vous laissez pas entraîner dans l’égarement des gens dévoyés, et n’abandonnez pas l’attitude de fermeté qui est la vôtre. 
${}^{18}Mais continuez à grandir dans la grâce et la connaissance de notre Seigneur et Sauveur, Jésus Christ. À lui la gloire, dès maintenant et jusqu’au jour de l’éternité. Amen.
