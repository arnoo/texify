  
  
    
    \bbook{SAGESSE}{SAGESSE}
      
         
      \bchapter{}
        ${}^{1}Aimez la justice, vous qui gouvernez la terre,
        ayez sur le Seigneur des pensées droites,
        cherchez-le avec un cœur simple,
        ${}^{2}car il se laisse trouver\\par ceux qui ne le mettent pas à l’épreuve,
        il se manifeste à ceux qui ne refusent pas de croire en lui.
        ${}^{3}Les pensées tortueuses éloignent de Dieu,
        et sa puissance confond les insensés qui la provoquent.
        ${}^{4}Car la Sagesse ne peut entrer dans une âme qui veut le mal,
        ni habiter dans un corps asservi au péché.
        
           
         
        ${}^{5}L’Esprit saint, éducateur des hommes, fuit l’hypocrisie,
        il se détourne des projets sans intelligence,
        quand survient l’injustice, il la confond\\.
        ${}^{6}La Sagesse est un esprit ami des hommes,
        mais elle ne laissera pas le blasphémateur
        impuni pour ses paroles ;
        \\car Dieu scrute ses reins,
        avec clairvoyance il observe son cœur,
        il écoute les propos de sa bouche.
        ${}^{7}L’esprit du Seigneur remplit l’univers :
        lui qui tient ensemble tous les êtres,
        il entend\\toutes les voix.
        ${}^{8}C’est pourquoi nul n’est à l’abri
        lorsqu’il tient des propos injustes :
        \\la Justice qui confond les coupables
        ne l’épargnera pas.
        ${}^{9}Sur les intentions de l’impie, il y aura une enquête,
        le bruit de ses paroles parviendra jusqu’au Seigneur
        qui le confondra pour ses forfaits.
        ${}^{10}Une oreille attentive écoute tout ;
        même le murmure des récriminations ne reste pas caché.
        ${}^{11}Gardez-vous donc d’une récrimination inutile,
        et plutôt que de dire du mal, retenez votre langue,
        \\car un propos tenu en cachette ne restera pas sans effet :
        la bouche qui calomnie détruit l’âme.
        
           
         
        ${}^{12}Ne courez pas après la mort en dévoyant votre vie,
        n’attirez pas la catastrophe par les œuvres de vos mains.
        ${}^{13}Dieu n’a pas fait la mort,
        il ne se réjouit pas de voir mourir les êtres vivants.
        ${}^{14}Il les a tous créés pour qu’ils subsistent ;
        ce qui naît dans le monde est porteur de vie\\ :
        on n’y trouve pas de poison qui fasse mourir.
        \\La puissance de la Mort ne règne pas sur la terre,
        ${}^{15}car la justice est immortelle.
        
           
        ${}^{16}Pourtant, les impies ont invité la Mort, du geste et de la voix ;
        la tenant pour amie, pour elle ils se consument ;
        \\ils ont fait un pacte avec elle :
        ils méritent bien de lui appartenir.
      
         
      \bchapter{}
        ${}^{1}Ils ne sont pas dans la vérité
        lorsqu’ils raisonnent ainsi en eux-mêmes :
        \\« Notre existence est brève et triste,
        rien ne peut guérir l’homme au terme de sa vie,
        on n’a jamais vu personne revenir du séjour des morts.
${}^{2}Nous sommes nés par hasard,
        et après, nous serons comme si nous n’avions pas existé ;
        \\le souffle de nos narines, c’est de la fumée,
        \\et la pensée, une étincelle
        qui jaillit au battement de notre cœur :
${}^{3}si elle s’éteint, le corps s’en ira en cendres,
        et l’esprit se dissipera comme l’air léger.
${}^{4}Avec le temps, notre nom tombera dans l’oubli,
        et nul ne saura plus ce que nous avons fait.
        \\Notre vie passera comme un nuage, sans laisser de traces ;
        elle se dissipera comme la brume
        chassée par les rayons du soleil, écrasée par sa chaleur.
${}^{5}Nos jours passent comme une ombre,
        l’heure de notre fin ne peut être reculée :
        elle est scellée, et nul ne revient.
        
           
         
${}^{6}Alors allons-y ! Jouissons des biens qui sont là ;
        vite, profitons des créatures, tant que nous sommes jeunes :
${}^{7}enivrons-nous de bons vins et de parfums,
        ne laissons pas échapper la fleur du printemps,
${}^{8}couronnons-nous de roses en boutons,
        avant qu’elles ne soient fanées !
${}^{9}Qu’aucun de nous ne manque à nos orgies,
        laissons partout des signes de réjouissance,
        car c’est là notre part et c’est là notre lot !
${}^{10}Écrasons le pauvre et sa justice,
        soyons sans ménagement pour la veuve,
        et sans égard pour le vieillard aux cheveux blancs.
${}^{11}Que notre force soit la norme de la justice,
        car ce qui est faible s’avère inutile.
        
           
        ${}^{12}Attirons le juste dans un piège, car il nous contrarie,
        il s’oppose à nos entreprises,
        \\il nous reproche de désobéir à la loi de Dieu\\,
        et nous accuse d’infidélités à notre éducation.
        ${}^{13}Il prétend posséder la connaissance de Dieu,
        et se nomme lui-même enfant du Seigneur.
        ${}^{14}Il est un démenti pour nos idées,
        sa seule présence nous pèse ;
        ${}^{15}car il mène une vie en dehors du commun,
        sa conduite est étrange.
        ${}^{16}Il nous tient pour des gens douteux,
        se détourne de nos chemins comme de la boue\\.
        \\Il proclame heureux le sort final des justes
        et se vante d’avoir Dieu pour père\\.
        ${}^{17}Voyons si ses paroles sont vraies,
        regardons comment il en sortira.
         
        ${}^{18}Si le juste est fils de Dieu\\,
        Dieu l’assistera, et l’arrachera aux mains de ses adversaires.
        ${}^{19}Soumettons-le à des outrages et à des tourments ;
        nous saurons ce que vaut sa douceur,
        nous éprouverons sa patience.
        ${}^{20}Condamnons-le à une mort infâme,
        puisque, dit-il, quelqu’un interviendra pour lui\\. »
         
        ${}^{21}C’est ainsi que raisonnent ces gens-là, mais ils s’égarent ;
        leur méchanceté les a rendus aveugles.
        ${}^{22}Ils ne connaissent pas les secrets de Dieu,
        ils n’espèrent pas que la sainteté puisse être récompensée,
        ils n’estiment pas qu’une âme irréprochable puisse être glorifiée.
        ${}^{23}Or, Dieu a créé l’homme pour l’incorruptibilité,
        il a fait de lui une image de sa propre identité\\.
         
        ${}^{24}C’est par la jalousie du diable
        que la mort est entrée dans le monde ;
        \\ils en font l’expérience,
        ceux qui prennent parti pour lui.
      <p class="cantique" id="bib_ct-at_9a"><span class="cantique_label">Cantique AT 9a</span> = <span class="cantique_ref"><a class="unitex_link" href="#bib_sg_3_1">Sg 3, 1-7a</a></span>
      <p class="cantique" id="bib_ct-at_9b"><span class="cantique_label">Cantique AT 9b</span> = <span class="cantique_ref"><a class="unitex_link" href="#bib_sg_3_7">Sg 3, 7-9</a></span>
      
         
      \bchapter{}
      <div class="box_noborder cantique_chap">
            ${}^{1}Mais\\les âmes des justes sont dans la main de Dieu ;
            aucun tourment n’a de prise sur eux.
            ${}^{2}Aux yeux de l’insensé, ils ont paru mourir ; +
            \\leur départ\\est compris comme un malheur,
            ${}^{3}et leur éloignement, comme une fin\\ :
            mais ils sont dans la paix.
            ${}^{4}Au regard des hommes, ils ont subi un châtiment,
            mais l’espérance de l’immortalité les comblait.
            ${}^{5}Après de faibles peines,
            de grands bienfaits les attendent,
            \\car Dieu les a mis à l’épreuve
            et trouvés dignes de lui.
            ${}^{6}Comme l’or au creuset, il les a éprouvés ;
            comme une offrande parfaite, il les accueille\\.
      <div class="box_noborder cantique_chap">
            ${}^{7}Au temps de sa visite, ils resplendiront\\ :
            comme l’étincelle qui court sur la paille, ils avancent.
            ${}^{8}Ils jugeront les nations, ils auront pouvoir sur les peuples,
            et le Seigneur régnera sur eux pour les siècles.
            ${}^{9}Qui met en lui sa foi comprendra la vérité ;
            ceux qui sont fidèles resteront, dans l’amour, près de lui.
            \\Pour ses amis\\, grâce et miséricorde :
            il visitera ses élus\\.
       
${}^{10}Mais les impies subiront une peine à la mesure de leurs pensées,
        car ils ont méprisé le juste et abandonné le Seigneur.
${}^{11}Misérables, ceux qui tiennent pour rien
        la sagesse et sa discipline de vie :
        \\vide est leur espérance, vaines leurs fatigues,
        inutiles leurs œuvres,
${}^{12}folles leurs femmes, méchants leurs enfants,
        et leur descendance, maudite !
${}^{13}Heureuse la femme stérile, si elle est sans souillure
        et n’a pas connu d’union coupable ;
        elle aura son fruit lorsque les âmes seront visitées.
${}^{14}Heureux aussi l’eunuque dont la main n’a pas fait de mal,
        et qui n’a pas nourri de ressentiment contre le Seigneur :
        \\une faveur spéciale lui sera accordée pour sa fidélité,
        et, dans le temple du Seigneur,
        il aura une part très douce à son cœur.
${}^{15}Les efforts vertueux produisent un fruit de renommée ;
        ce qui s’enracine dans la sagesse ne peut manquer son but.
         
${}^{16}Mais les fruits de l’infidélité ne parviendront pas à maturité,
        la descendance d’une union illégitime disparaîtra.
${}^{17}Même s’ils vivent longtemps, ils seront comptés pour rien,
        et pour finir, on méprisera leur vieillesse ;
${}^{18}s’ils meurent tôt,
        ils n’auront ni espérance ni consolation au jour du verdict :
${}^{19}pénible est la destinée d’une génération injuste !
       
      
         
      \bchapter{}
${}^{1}Mieux vaut ne pas avoir d’enfant
        et posséder la vertu :
        \\elle laisse un souvenir immortel,
        puisqu’elle est connue de Dieu comme des hommes.
${}^{2}Présente, on l’imite,
        absente, on la regrette ;
        \\et dans l’éternité, elle triomphe, portant sa couronne :
        elle a vaincu dans des luttes aux prix incorruptibles.
${}^{3}Mais la nombreuse progéniture des impies n’aboutira à rien ;
        issue de boutures bâtardes,
        \\elle ne prendra pas racine en profondeur
        et n’aura aucun fondement solide.
${}^{4}Même si pour un temps elle déploie des branches,
        elle est prête à tomber, et le vent l’ébranlera,
        une bourrasque la déracinera.
${}^{5}Les rameaux, encore tendres, seront brisés,
        et leur fruit sera perdu :
        \\trop vert pour être mangé,
        il ne sera bon à rien.
${}^{6}Les enfants nés de sommeils coupables
        sont des témoins à charge :
        \\lors de l’enquête, chacun dénonce
        la perversité des parents.
        
           
        ${}^{7}Au contraire, même s’il meurt avant l’âge,
        le juste trouvera le repos.
        ${}^{8}La dignité du vieillard ne tient pas au grand âge,
        elle ne se mesure pas au nombre des années.
        ${}^{9}Pour l’homme, la sagesse tient lieu de cheveux blancs,
        une vie sans tache vaut une longue vieillesse.
        ${}^{10}Il a su plaire à Dieu, et Dieu l’a aimé ;
        il vivait au milieu des pécheurs : il en fut retiré.
        ${}^{11}Il a été enlevé,
        de peur que le mal ne corrompe sa conscience,
        pour que le mensonge n’égare pas son âme.
        ${}^{12}Car la fascination du mal fait perdre de vue le bien,
        le tourbillon de la convoitise trouble un esprit sans malice.
        ${}^{13}Arrivé au but en peu de temps,
        il a parcouru tous les âges de la vie.
        ${}^{14}Parce qu’il plaisait au Seigneur,
        celui-ci, sans attendre, l’a retiré d’un monde mauvais.
         
        \\Les gens voient cela sans comprendre ;
        \\il ne leur vient pas à l’esprit
        ${}^{15}que Dieu accorde à ses élus grâce et miséricorde,
        et qu’il intervient pour ceux qui lui sont fidèles\\.
${}^{16}Le juste qui meurt est condamnation des impies vivants ;
        et la jeunesse rapidement parvenue au but
        condamne la longue vieillesse de l’homme injuste.
${}^{17}Ils verront donc la mort du sage
        sans comprendre ce que le Seigneur a décidé à son égard,
        ni dans quel but il l’a mis en sûreté.
${}^{18}Ils verront et n’en tiendront aucun compte ;
        ces gens-là, le Seigneur s’en moquera !
${}^{19}Après cette vie, ils ne seront plus qu’un infâme cadavre,
        un objet de dérision parmi les morts, pour toujours ;
        \\il les fera tomber, muets, la tête la première,
        il les ébranlera jusqu’aux profondeurs ;
        \\jusqu’à la fin, ils resteront en friche,
        ils seront dans la douleur,
        et leur souvenir périra.
${}^{20}Les impies viendront, tout tremblants,
        quand on fera le compte de leurs péchés,
        et leurs crimes se dresseront contre eux pour les accuser.
       
      
         
      \bchapter{}
${}^{1}Alors le juste se tiendra debout, plein d’assurance,
        en présence de ceux qui l’ont opprimé,
        de ceux qui méprisaient sa peine.
${}^{2}À sa vue, ils seront pris d’une peur épouvantable,
        sidérés de le voir sauvé contre toute attente ;
${}^{3}saisis par le remords, ils se diront entre eux,
        la gorge serrée, incapables de reprendre souffle :
${}^{4}« Le voilà, celui que nous tournions jadis en ridicule !
        \\Nous en faisions la cible de nos sarcasmes,
        fous que nous étions !
        \\Nous trouvions absurde sa manière de vivre
        et infâme sa mort !
${}^{5}Pourquoi est-il compté parmi les fils de Dieu ?
        Pourquoi partage-t-il le sort des saints ?
        
           
         
${}^{6}En fait, nous nous sommes égarés loin du chemin de la vérité,
        la lumière de la justice ne nous a pas éclairés
        et le soleil ne s’est pas levé sur nous.
${}^{7}Nous nous sommes soûlés d’injustice, sur les sentiers de perdition,
        nous avons traversé des déserts impraticables,
        mais le chemin du Seigneur, nous ne l’avons pas connu.
${}^{8}À quoi nous a servi l’orgueil,
        et que nous ont rapporté la richesse et la prétention ?
        
           
         
${}^{9}Tout cela a passé comme une ombre,
        comme une rumeur fugace.
${}^{10}Comme le navire traverse une mer agitée
        sans qu’on puisse retrouver la trace de son passage,
        ni le sillage de sa coque sur les vagues…
${}^{11}Comme l’oiseau vole à travers l’espace
        sans qu’on trouve aucune empreinte de son parcours :
        \\du battement de ses ailes, il fouette l’air léger,
        le fend avec violence dans le sifflement de son vol
        et le traverse sans qu’on trouve signe de ce passage…
${}^{12}Comme la flèche lancée vers la cible
        déchire l’air aussitôt refermé,
        si bien qu’on ignore quelle fut sa trajectoire…
${}^{13}Ainsi de nous :
        à peine nés, nous avons disparu,
        \\nous n’avons pu montrer aucun signe de vertu
        et, dans notre malice, nous nous sommes consumés. »
        
           
         
${}^{14}L’espoir de l’impie est comme un brin de paille emporté par le vent,
        comme l’écume légère chassée par l’ouragan ;
        \\il se dissipe comme fumée au vent,
        il passe son chemin
        comme le souvenir de l’hôte d’un jour.
        
           
${}^{15}Les justes, eux, vivent pour toujours,
        le Seigneur détient leur récompense,
        le Très-Haut prend soin d’eux.
${}^{16}Aussi recevront-ils de la main du Seigneur
        le royaume de splendeur et le diadème de beauté,
        \\car de sa droite il les protégera,
        et son bras les couvrira.
${}^{17}Il prendra pour armure son ardeur jalouse,
        il armera la création pour réprimer ses ennemis.
${}^{18}Il revêtira la justice comme cuirasse
        et mettra comme casque le jugement sans appel ;
${}^{19}il prendra comme bouclier l’invincible sainteté ;
${}^{20}en guise d’épée, il affûtera sa colère implacable,
        et l’univers, à ses côtés, combattra les insensés.
${}^{21}Flèches bien ajustées, les éclairs partiront,
        comme tirés depuis l’arc bien tendu des nuages,
        et frapperont leur cible ;
${}^{22}d’une catapulte jailliront des grêlons pleins de fureur,
        l’eau de la mer se déchaînera contre ces gens-là,
        et les fleuves les submergeront implacablement.
${}^{23}Un souffle puissant se lèvera contre eux
        et, tel un ouragan, les dispersera ;
        \\l’injustice transformera la terre entière en désert,
        et la dépravation renversera les trônes des puissants.
      
         
      \bchapter{}
        ${}^{1}Écoutez donc, ô rois, et comprenez ;
        instruisez-vous, juges de toute la terre\\.
        ${}^{2}Soyez attentifs, vous qui dominez les foules,
        qui vous vantez de la multitude de vos peuples.
        ${}^{3}Car la domination vous a été donnée par le Seigneur,
        et le pouvoir, par le Très-Haut\\,
        \\lui qui examinera votre conduite
        et scrutera vos intentions.
        ${}^{4}En effet, vous êtes les ministres de sa royauté ;
        \\si donc vous n’avez pas rendu la justice avec droiture,
        ni observé la Loi,
        ni vécu selon les intentions de Dieu,
        ${}^{5}il fondra sur vous, terrifiant et rapide,
        car un jugement implacable s’exerce sur les grands ;
        ${}^{6}au petit, par pitié, on pardonne,
        mais les puissants seront jugés avec puissance.
        
           
         
        ${}^{7}Le Maître de l’univers ne reculera devant personne\\,
        la grandeur ne lui en impose pas ;
        \\car les petits comme les grands, c’est lui qui les a faits :
        il prend soin de tous pareillement.
        ${}^{8}Les puissants seront soumis à une enquête rigoureuse.
        ${}^{9}C’est donc pour vous, souverains, que je parle,
        afin que vous appreniez la sagesse
        et que vous évitiez la chute,
        ${}^{10}car ceux qui observent saintement les lois saintes
        seront reconnus saints,
        \\et ceux qui s’en instruisent
        y trouveront leur défense.
        ${}^{11}Recherchez mes paroles, désirez-les ;
        elles feront votre éducation.
        
           
        ${}^{12}La Sagesse est resplendissante,
        elle ne se flétrit pas.
        \\Elle se laisse aisément contempler
        par ceux qui l’aiment,
        \\elle se laisse trouver
        par ceux qui la cherchent.
        ${}^{13}Elle devance leurs désirs
        en se faisant connaître la première.
        ${}^{14}Celui qui la cherche dès l’aurore ne se fatiguera pas :
        il la trouvera assise à sa porte.
        ${}^{15}Penser à elle est la perfection du discernement,
        et celui qui veille à cause d’elle
        sera bientôt délivré du souci.
        ${}^{16}Elle va et vient à la recherche de ceux qui sont dignes d’elle ;
        \\au détour des sentiers,
        elle leur apparaît avec un visage souriant ;
        \\dans chacune de leurs pensées,
        elle vient à leur rencontre.
${}^{17}Le commencement de la Sagesse,
        c’est le désir vrai d’être instruit ;
        \\le souci de l’instruction, c’est l’amour ;
${}^{18}l’amour, c’est de garder ses lois ;
        \\observer les lois, c’est l’assurance de l’incorruptibilité,
${}^{19}et l’incorruptibilité rend proche de Dieu.
${}^{20}Ainsi le désir de la Sagesse élève à la royauté.
${}^{21}Si donc vous, souverains des peuples,
        vous prenez plaisir aux trônes et aux sceptres,
        \\rendez hommage à la Sagesse
        afin de régner pour toujours.
${}^{22}Mais qu’est-ce que la Sagesse et comment est-elle née ?
        Je vais l’exposer.
        \\Loin de vous en cacher les mystères,
        je suivrai ses traces depuis le principe de son origine
        \\et je mettrai en lumière ce que l’on connaît d’elle ;
        je n’écarterai pas mon chemin de la vérité,
${}^{23}et ne marcherai jamais avec l’esprit de jalousie,
        car cela n’a rien de commun avec la sagesse.
${}^{24}Au contraire, une multitude de sages
        est salut pour le monde,
        \\et un roi qui gouverne avec discernement,
        c’est le bien-être du peuple.
${}^{25}Aussi laissez-vous instruire par mes paroles :
        vous en tirerez profit.
      
         
      \bchapter{}
${}^{1}Moi aussi, je suis un mortel, pareil à tous,
        descendant du premier homme façonné à partir de la terre ;
        \\au ventre d’une mère, j’ai été sculpté dans la chair,
${}^{2}jusqu’au dixième mois, j’ai pris consistance dans le sang,
        \\à partir de la semence virile
        et du plaisir, compagnon du sommeil.
${}^{3}Moi aussi, en naissant, j’ai aspiré l’air commun,
        je suis tombé sur la même terre où tous ont à souffrir ;
        \\et mon premier cri, comme pour tous, ce fut des pleurs.
${}^{4}J’ai été élevé dans les langes, avec sollicitude.
${}^{5}En fait, aucun roi n’a connu d’autre début dans l’existence :
${}^{6}pour tout être humain, il n’y a qu’une façon d’entrer dans la vie,
        et une seule d’en sortir.
        
           
         
        ${}^{7}Aussi j’ai prié\\,
        et le discernement m’a été donné.
        \\J’ai supplié,
        et l’esprit de la Sagesse est venu en moi.
        ${}^{8}Je l’ai préférée aux trônes et aux sceptres ;
        à côté d’elle, j’ai tenu pour rien la richesse ;
        ${}^{9}je ne l’ai pas comparée à la pierre la plus précieuse ;
        tout l’or du monde auprès d’elle n’est qu’un peu de sable,
        et, en face d’elle, l’argent sera regardé comme de la boue.
        ${}^{10}Plus que la santé et la beauté, je l’ai aimée ;
        je l’ai choisie de préférence à la lumière,
        parce que sa clarté ne s’éteint pas.
        ${}^{11}Tous les biens me sont venus avec elle
        et, par ses mains, une richesse incalculable.
${}^{12}Je me suis réjoui de tous ces biens,
        les sachant gouvernés par la Sagesse ;
        j’ignorais pourtant qu’elle en était aussi la mère.
        
           
${}^{13}Ce que j’ai appris sans calcul, je le partage sans réserve,
        je ne veux rien dissimuler de ses richesses :
${}^{14}la Sagesse est pour les hommes un trésor inépuisable,
        ceux qui l’acquièrent gagnent l’amitié de Dieu,
        car les bienfaits de l’éducation les recommandent auprès de lui.
         
        ${}^{15}Que Dieu m’accorde de parler comme je comprends,
        et de concevoir une pensée à la mesure de ses dons,
        \\puisque lui-même guide la Sagesse
        et dirige les sages ;
        ${}^{16}car nous sommes dans sa main :
        nous-mêmes, nos paroles,
        toute notre intelligence et notre savoir-faire.
         
${}^{17}C’est lui qui m’a donné une connaissance exacte du réel,
        pour que je comprenne la structure de l’univers
        et l’activité des éléments,
${}^{18}le commencement, la fin et le milieu des temps,
        l’alternance des solstices et le changement des saisons,
${}^{19}le cycle des années et la position des astres,
${}^{20}la nature des animaux et l’instinct des bêtes sauvages,
        \\l’impulsion des esprits et les raisonnements de l’homme,
        la variété des plantes et les vertus des racines.
${}^{21}Toute la réalité, cachée ou apparente, je l’ai connue,
        car la Sagesse, artisan de l’univers, m’a instruit.
        ${}^{22}Il y a dans la Sagesse un esprit
        intelligent et saint,
        unique et multiple,
        subtil et rapide ;
        perçant, net, clair et intact ;
        <p class="verset_anchor" id="para_bib_sg_7_23">ami du bien, vif, 
${}^{23}irrésistible,
        bienfaisant, ami des hommes ;
        ferme, sûr et paisible,
        tout-puissant et observant tout,
        pénétrant tous les esprits,
        même les plus intelligents, les plus purs, les plus subtils.
        ${}^{24}La Sagesse, en effet, se meut d’un mouvement
        qui surpasse tous les autres ;
        elle traverse et pénètre toute chose à cause de sa pureté.
         
        ${}^{25}Car elle est la respiration de la puissance de Dieu,
        l’émanation toute pure de la gloire du Souverain de l’univers ;
        aussi rien de souillé ne peut l’atteindre.
        ${}^{26}Elle est le rayonnement de la lumière éternelle,
        le miroir sans tache de l’activité de Dieu,
        l’image de sa bonté\\.
         
        ${}^{27}Comme elle est unique, elle peut tout ;
        et sans sortir d’elle-même, elle renouvelle l’univers.
        \\D’âge en âge, elle se transmet à des âmes saintes,
        pour en faire des prophètes et des amis de Dieu.
        ${}^{28}Car Dieu n’aime que celui qui vit avec la Sagesse.
        ${}^{29}Elle est plus belle que le soleil,
        elle surpasse toutes les constellations ;
        \\si on la compare à la lumière du jour,
        on la trouve bien supérieure,
        ${}^{30}car le jour s’efface devant la nuit,
        mais contre la Sagesse le mal ne peut rien.
      
         
      \bchapter{}
        ${}^{1}Elle déploie sa vigueur d’un bout du monde à l’autre,
        elle gouverne l’univers avec bonté.
        
           
${}^{2}C’est elle que j’ai aimée et recherchée depuis ma jeunesse,
        j’ai cherché à la prendre pour épouse,
        je suis devenu l’amant de sa beauté.
${}^{3}Elle manifeste la gloire de sa propre naissance
        puisqu’elle partage la vie de Dieu,
        et que le maître de l’univers lui a donné son amour.
${}^{4}Elle est initiée aux mystères de la science de Dieu,
        c’est elle qui décide de ses œuvres.
${}^{5}Si la richesse est un bien désirable en cette vie,
        qu’y a-t-il de plus riche que la Sagesse,
        elle qui met en œuvre toutes choses ?
${}^{6}Si l’intelligence humaine\\peut accomplir une œuvre,
        qui, plus que la Sagesse, est l’artisan de l’univers ?
${}^{7}Veut-on devenir juste ?
        Les labeurs de la Sagesse produisent les vertus :
        \\elle enseigne la tempérance et la prudence,
        la justice et la force d’âme,
        \\et rien n’est plus utile aux hommes dans l’existence.
${}^{8}Désire-t-on encore profiter de sa grande expérience ?
        \\Elle connaît le passé et conçoit l’avenir,
        elle sait le sens caché des paroles et la solution des énigmes ;
        \\les signes et les prodiges, elle les prévoit,
        ainsi que les temps et les moments favorables.
${}^{9}J’ai donc résolu d’amener la Sagesse à partager ma vie,
        car je savais qu’elle serait ma conseillère pour bien agir,
        mon réconfort dans les soucis et la tristesse.
${}^{10}Grâce à elle, j’aurai la gloire auprès des foules,
        et l’honneur auprès des anciens, malgré ma jeunesse.
${}^{11}Au tribunal, on reconnaîtra ma perspicacité ;
        devant moi les puissants seront dans l’admiration.
${}^{12}Si je me tais, ils attendront ;
        si je parle, ils prêteront l’oreille ;
        \\si je prolonge mon discours,
        ils se garderont de m’interrompre.
${}^{13}Grâce à la Sagesse, j’aurai l’immortalité,
        je laisserai à la postérité un souvenir éternel.
${}^{14}Je dirigerai des peuples,
        et des nations me seront soumises.
${}^{15}S’ils entendent parler de moi,
        des souverains redoutables prendront peur.
        \\Je montrerai ma valeur dans l’assemblée du peuple,
        et ma bravoure à la guerre.
${}^{16}Quand j’entrerai chez moi, je me reposerai près d’elle,
        car sa compagnie est sans amertume ;
        \\partager sa vie ne cause pas de peine,
        seulement plaisir et joie.
${}^{17}J’ai raisonné ainsi en moi-même,
        j’ai pesé dans mon cœur les réflexions que voici :
        \\l’immortalité se trouve dans l’union avec la Sagesse ;
${}^{18}il y a dans sa tendresse une jouissance supérieure,
        dans les travaux de ses mains, une richesse inépuisable,
        dans sa fréquentation assidue, le discernement ;
        \\et l’on trouve la célébrité en partageant ce qu’elle enseigne ;
        aussi, je la courtisais et cherchais comment la prendre pour épouse.
${}^{19}Certes, j’étais un enfant d’une heureuse nature,
        et j’avais reçu une âme bonne,
${}^{20}ou plutôt, étant bon,
        j’étais venu dans un corps sans souillure ;
${}^{21}mais je savais que je ne pourrais jamais obtenir la sagesse
        si Dieu ne me la donnait,
        \\et il me fallait déjà du discernement
        pour savoir de qui viendrait ce bienfait.
        \\Je me tournai donc vers le Seigneur et lui fis cette prière,
        en disant de tout mon cœur :
      <p class="cantique" id="bib_ct-at_10"><span class="cantique_label">Cantique AT 10</span> = <span class="cantique_ref"><a class="unitex_link" href="#bib_sg_9_1">Sg 9, 1-6.9-11</a></span>
      
         
      \bchapter{}
      <div class="box_noborder cantique_chap">
            ${}^{1}« Dieu de mes\\pères et Seigneur de miséricorde\\,
            par ta parole tu fis l’univers,
            ${}^{2}Tu formas l’homme par ta Sagesse
            pour qu’il soit maître de tes créatures,
            ${}^{3}qu’il gouverne le monde avec justice et sainteté,
            qu’il rende, avec droiture, ses jugements.
            ${}^{4}Donne-moi la Sagesse, assise auprès de toi\\ ;
            ne me retranche pas du nombre de tes enfants :
            ${}^{5}je suis ton serviteur, le fils de ta servante, +
            \\un homme frêle et qui dure peu,
            trop faible pour comprendre les préceptes et les lois.
            ${}^{6}Le plus accompli des enfants des hommes,
            s’il lui manque la Sagesse que tu donnes,
            sera compté pour rien.
             
${}^{7}\[Tu m’as choisi pour régner sur ton peuple,
            pour gouverner tes fils et tes filles ;
${}^{8}tu m’as ordonné de bâtir un temple sur ta montagne sainte,
            un autel dans la ville où tu demeures,
            imitation de la demeure sainte que tu fondas dès l’origine.\]
             
            ${}^{9}Or la Sagesse est avec toi, elle qui sait tes œuvres ;
            elle était là quand tu fis l’univers ;
            \\elle connaît ce qui plaît à tes yeux,
            ce qui est conforme à tes décrets.
            ${}^{10}Des cieux très saints, daigne l’envoyer,
            fais-la descendre du trône de ta gloire.
            \\Qu’elle travaille à mes côtés
            et m’apprenne ce qui te plaît.
            ${}^{11}Car elle sait tout, comprend tout,
            guidera mes actes avec prudence,
            me gardera par sa gloire.
       
${}^{12}Alors mes œuvres te seront agréables,
        je jugerai ton peuple avec justice,
        et serai digne du trône de mon père.
        ${}^{13}Quel homme peut découvrir les intentions de Dieu ?
        Qui peut comprendre les volontés du Seigneur\\ ?
        ${}^{14}Les réflexions des mortels sont incertaines,
        et nos pensées, instables ;
        ${}^{15}car un corps périssable appesantit notre âme,
        et cette enveloppe d’argile
        alourdit notre esprit aux mille pensées.
        ${}^{16}Nous avons peine à nous représenter ce qui est sur terre,
        et nous trouvons avec effort ce qui est à notre portée ;
        \\ce qui est dans les cieux, qui donc l’a découvert ?
        ${}^{17}Et qui aurait connu ta volonté,
        si tu n’avais pas donné la Sagesse
        et envoyé d’en haut ton Esprit Saint ?
        ${}^{18}C’est ainsi que les sentiers des habitants de la terre
        sont devenus droits ;
        \\c’est ainsi que les hommes ont appris ce qui te plaît
        et, par la Sagesse, ont été sauvés. »
      <p class="cantique" id="bib_ct-at_11"><span class="cantique_label">Cantique AT 11</span> = <span class="cantique_ref"><a class="unitex_link" href="#bib_sg_10_17">Sg 10, 17-21</a></span>
      
         
      \bchapter{}
${}^{1}La Sagesse elle-même a veillé
        \\sur celui qui fut façonné le premier, créé seul,
        le père du monde ;
        \\puis elle l’arracha à sa propre faute,
${}^{2}et lui donna la force de dominer toute chose.
${}^{3}Or un homme injuste, pris de colère, se détourna d’elle
        et périt de cette rage fratricide ;
${}^{4}à cause de lui, la terre fut submergée par le déluge,
        mais la Sagesse, de nouveau, la sauva
        en pilotant le juste sur un simple morceau de bois.
        
           
         
${}^{5}Lorsque les nations, unanimes dans le mal, furent confondues,
        c’est elle qui reconnut le juste, le garda irréprochable devant Dieu,
        et assez fort pour surmonter sa tendresse envers son enfant.
${}^{6}Alors que les impies périssaient,
        la Sagesse délivra un homme juste,
        qui fuyait le feu s’abattant sur les Cinq-Villes.
${}^{7}En témoignage de leur perversité reste une terre aride et fumante,
        des plantes dont les fruits ne mûrissent plus,
        \\et, en mémoire d’une âme qui ne voulut pas croire,
        se dresse une colonne de sel.
        
           
         
${}^{8}Car ceux qui dédaignent la Sagesse
        non seulement se privent de la connaissance du vrai bien,
        \\mais laissent encore aux vivants un souvenir de leur folie,
        pour que leur égarement ne passe pas inaperçu.
${}^{9}Quant à ceux qui servent la Sagesse,
        elle les délivre de leurs épreuves.
        
           
         
${}^{10}Elle guida elle-même sur des sentiers droits
        un homme juste qui fuyait la colère de son frère :
        \\elle lui fit voir en songe le royaume de Dieu
        et lui donna la connaissance de réalités saintes ;
        \\elle récompensa ses efforts et multiplia le fruit de ses labeurs ;
${}^{11}elle l’assista contre la cupidité de ceux qui l’exploitaient,
        et le rendit riche.
${}^{12}La Sagesse l’a protégé de ses ennemis
        et l’a préservé de tous les pièges,
        elle lui a donné la victoire dans un combat redoutable ;
        \\elle lui a fait comprendre ainsi
        que la piété est plus puissante que tout.
        
           
         
${}^{13}Quand le juste eut été vendu,
        la Sagesse ne l’a pas abandonné :
        mais elle l’a arraché au péché,
${}^{14}elle est descendue avec lui au fond de sa prison,
        dans les chaînes elle ne l’a pas quitté,
        \\jusqu’au jour où elle lui donna le sceptre royal
        et le pouvoir sur ceux qui l’avaient opprimé.
        \\Elle a convaincu de mensonge ses accusateurs
        et lui a donné une gloire éternelle.
        
           
${}^{15}La Sagesse a délivré d’une nation d’oppresseurs
        le peuple saint, la race irréprochable.
${}^{16}Elle entra dans l’âme d’un serviteur du Seigneur
        et, par des prodiges et des signes, s’est opposée à des rois redoutables.
       
      <div class="box_noborder cantique_chap">
            ${}^{17}La Sagesse a récompensé les saints de leurs peines,
            les a conduits sur un chemin de merveilles.
            \\Le jour, elle fut pour eux un abri,
            et la nuit, une clarté\\d’étoiles\\.
            ${}^{18}Elle leur fit traverser la mer Rouge
            et franchir les eaux profondes.
            ${}^{19}Elle noya leurs ennemis,
            et les vomit du fond de l’abîme.
            ${}^{20}Alors les justes ont dépouillé les impies :
            ils ont clamé, Seigneur, ton saint nom,
            et chanté, d’un seul cœur, ta main qui les défend\\.
            ${}^{21}La Sagesse a ouvert la bouche des muets
            et délié la langue des tout-petits.
      
         
      \bchapter{}
${}^{1}La Sagesse a mené à bien leur entreprise
        \\par la main d’un saint prophète :
${}^{2}ils ont traversé un désert inhabitable
        et planté leurs tentes en des lieux jamais foulés ;
${}^{3}ils ont résisté aux assaillants,
        repoussé les ennemis.
        
           
${}^{4}Quand ils souffrirent de la soif, ils t’invoquèrent,
        et l’eau leur fut donnée d’un roc escarpé :
        un remède à leur soif jaillit du rocher.
${}^{5}Ce qui avait servi à châtier leurs ennemis,
        cela même fut pour eux un bienfait dans leur détresse :
${}^{6}tandis qu’un fleuve au cours immuable
        était troublé d’un sang boueux
${}^{7}en punition du décret infanticide,
        \\tu donnas aux tiens, contre tout espoir, une eau abondante ;
${}^{8}tu leur montrais par cette expérience de la soif
        comment tu avais châtié leurs adversaires.
${}^{9}Au cours de cette épreuve en effet,
        bien que corrigés avec miséricorde,
        \\ils comprirent quelles tortures étaient infligées aux impies,
        jugés avec colère.
         
${}^{10}Les tiens, tu les as mis à l’épreuve
        comme un père donne un avertissement,
        \\mais les impies, tu les as mis au supplice,
        comme un roi inflexible lorsqu’il condamne.
${}^{11}Après le départ des tiens,
        ils n’étaient pas moins affligés qu’avant.
${}^{12}Une tristesse redoublée les saisit,
        un gémissement au souvenir des événements passés.
${}^{13}Lorsqu’ils apprirent que l’instrument même de leur châtiment
        était source de bienfait pour les autres,
        ils reconnurent le Seigneur.
${}^{14}Au terme de ces événements,
        après avoir souffert une soif bien différente de celle des justes,
        \\ils admirèrent celui qu’ils avaient, tout-petit, exposé à la mort,
        rejeté, et, plus tard, congédié en se moquant.
${}^{15}Pour prix des projets insensés que formait leur injustice
        – car ils s’égaraient jusqu’à vénérer de stupides reptiles
        et de misérables bêtes –,
        \\tu leur envoyas en punition une multitude d’animaux stupides,
${}^{16}afin qu’ils comprennent
        que l’on est châtié par où l’on pèche.
${}^{17}Certes, elle n’était pas embarrassée, ta main toute-puissante,
        elle qui a créé le monde à partir d’une matière informe :
        \\elle aurait pu envoyer contre eux
        une bande d’ours ou de lions féroces,
${}^{18}ou des monstres inconnus et pleins de rage, créés tout exprès,
        exhalant un souffle de feu,
        \\répandant une fumée pestilentielle,
        ou lançant de leurs yeux de terribles éclairs.
${}^{19}Non seulement leurs ravages auraient pu les anéantir,
        mais leur vision déjà les aurait fait périr d’effroi.
${}^{20}D’ailleurs, il aurait suffi d’un souffle pour qu’ils soient renversés,
        chassés par la Justice,
        \\dispersés en tous sens par le souffle de ta puissance.
         
        \\Mais toi, Seigneur\\, tu as tout réglé
        avec mesure, nombre et poids.
${}^{21}Car ta grande puissance est toujours à ton service,
        et qui peut résister à la force de ton bras ?
${}^{22}Le monde entier est devant toi
        comme un rien sur la balance,
        \\comme la goutte de rosée matinale
        qui descend sur la terre.
        ${}^{23}Pourtant, tu as pitié de tous les hommes\\,
        parce que tu peux tout.
        \\Tu fermes les yeux sur leurs péchés,
        pour qu’ils se convertissent.
        ${}^{24}Tu aimes en effet tout ce qui existe,
        tu n’as de répulsion envers aucune de tes œuvres ;
        \\si tu avais haï quoi que ce soit,
        tu ne l’aurais pas créé.
        ${}^{25}Comment aurait-il subsisté,
        si tu ne l’avais pas voulu ?
        \\Comment serait-il resté vivant,
        si tu ne l’avais pas appelé ?
        ${}^{26}En fait, tu épargnes tous les êtres, parce qu’ils sont à toi,
        Maître qui aimes les vivants,
      
         
      \bchapter{}
        ${}^{1}toi dont le souffle impérissable les anime tous.
        ${}^{2}Ceux qui tombent, tu les reprends peu à peu,
        tu les avertis, tu leur rappelles en quoi ils pèchent,
        \\pour qu’ils se détournent du mal
        et croient en toi, Seigneur.
        
           
${}^{3}Ainsi, ceux qui habitaient autrefois ta Terre sainte ;
${}^{4}tu les avais pris en haine pour leurs abominables pratiques,
        œuvres de sorcellerie et rites sacrilèges :
${}^{5}ils tuaient leurs enfants sans aucune pitié,
        et faisaient des festins de chair humaine, de viscères et de sang.
        \\Ces adeptes des mystères,
${}^{6}ces parents meurtriers d’êtres sans défense,
        tu avais décidé de les faire périr par la main de nos pères,
${}^{7}afin que, sur la Terre qui t’est chère entre toutes,
        viennent s’établir des enfants de Dieu dignes d’elle.
${}^{8}Et pourtant, même ceux-là, tu les as traités avec ménagement
        parce qu’ils étaient des êtres humains.
        \\Tu n’as envoyé contre eux, en avant-coureurs de ton armée,
        que des frelons, ces insectes dangereux,
        pour ne pas hâter leur extermination.
${}^{9}Tu aurais bien pu livrer ces impies aux mains des justes,
        dans une bataille rangée,
        \\ou les anéantir d’un coup
        par des fauves redoutables ou une parole tranchante,
${}^{10}mais en exerçant ta justice sans hâte,
        tu leur offrais l’occasion du repentir.
        \\Tu n’ignorais pourtant pas que leur nature était viciée,
        et leur malice, innée,
        \\que leur mentalité ne changerait jamais,
${}^{11}car c’était une descendance maudite dès l’origine.
        \\Ce n’est par crainte de personne
        que tu leur offrais l’impunité de leurs péchés.
${}^{12}Qui donc osera dire : « Qu’as-tu fait ? »
        Qui contestera ta sentence ?
        \\Qui te citera en justice
        pour avoir détruit des peuples que tu as toi-même créés ?
        \\Qui encore viendra s’opposer à toi
        et prendre la défense d’hommes injustes ?
        ${}^{13}Il n’y a pas d’autre dieu que toi,
        qui prenne soin de toute chose\\ :
        tu montres ainsi que tes jugements ne sont pas injustes.
${}^{14}Il n’est pas de roi ou de souverain qui puisse te braver
        pour défendre ceux que tu as châtiés.
         
${}^{15}Parce que tu es juste, tu gouvernes l’univers avec justice ;
        tu estimes incompatible avec ta puissance
        de condamner celui qui ne mérite pas d’être puni.
        ${}^{16}Ta force est à l’origine\\de ta justice,
        et ta domination sur toute chose
        te permet d’épargner toute chose.
        ${}^{17}Tu montres ta force
        si l’on ne croit pas à la plénitude de ta puissance,
        et ceux qui la bravent sciemment, tu les réprimes.
        ${}^{18}Mais toi qui disposes de la force,
        tu juges avec indulgence,
        \\tu nous gouvernes avec beaucoup de ménagement,
        car tu n’as qu’à vouloir pour exercer ta puissance.
        ${}^{19}Par ton exemple tu as enseigné à ton peuple
        que le juste doit être humain ;
        \\à tes fils tu as donné une belle espérance :
        après la faute tu accordes la conversion.
${}^{20}Les ennemis de tes enfants,
        \\tu les as punis avec un grand souci d’indulgence
        alors qu’ils méritaient la mort,
        tu leur as donné le temps et l’occasion de renoncer au mal.
${}^{21}Mais tes fils, avec combien plus de scrupules les as-tu jugés,
        toi qui avais fait en faveur de leurs pères des serments et des alliances :
        magnifiques promesses !
${}^{22}Ainsi, tu modères le châtiment de nos ennemis,
        pour nous apprendre à méditer ta bonté lorsque nous jugeons,
        et à compter sur ta miséricorde lorsque nous sommes jugés.
${}^{23}C’est pourquoi ceux qui menaient une vie absurde et injuste,
        tu les as tourmentés par leurs propres abominations ;
${}^{24}ils avaient vraiment dépassé les bornes de l’erreur et de l’égarement,
        prenant pour divinités les plus vils des animaux immondes,
        et se laissant abuser comme des tout-petits sans intelligence.
${}^{25}Aussi, comme des gamins déraisonnables,
        tu les as tournés en ridicule pour les punir.
${}^{26}Mais ceux qui ne comprennent pas les réprimandes pour enfants
        font l’expérience d’un jugement vraiment divin.
${}^{27}Exaspérés par ces bestioles qui les faisaient souffrir,
        et se voyant châtiés par celles-là mêmes qu’ils prenaient pour des dieux,
        \\ils durent reconnaître pour un Dieu véritable
        celui dont autrefois ils ne voulaient rien savoir :
        c’est pourquoi la condamnation suprême s’abattit sur eux.
      
         
      \bchapter{}
        ${}^{1}De nature, ils sont inconsistants,
        \\tous ces gens qui restent dans l’ignorance de Dieu :
        \\à partir de ce qu’ils voient de bon,
        ils n’ont pas été capables de connaître Celui qui est ;
        \\en examinant ses œuvres,
        ils n’ont pas reconnu l’Artisan\\.
        ${}^{2}Mais c’est le feu, le vent, la brise légère,
        la ronde des étoiles, la violence des flots,
        \\les luminaires du ciel gouvernant le cours du monde,
        qu’ils ont regardés comme des dieux.
        ${}^{3}S’ils les ont pris pour des dieux,
        sous le charme de leur beauté,
        \\ils doivent savoir
        combien le Maître de ces choses leur est supérieur,
        car l’Auteur même de la beauté est leur créateur.
        ${}^{4}Et si c’est leur puissance et leur efficacité qui les ont frappés,
        ils doivent comprendre, à partir de ces choses,
        combien est plus puissant Celui qui les a faites.
        ${}^{5}Car à travers la grandeur et la beauté des créatures,
        on peut contempler, par analogie, leur Auteur.
        
           
         
        ${}^{6}Et pourtant, ces hommes ne méritent qu’un blâme léger ;
        car c’est peut-être en cherchant Dieu et voulant le trouver,
        qu’ils se sont égarés :
        ${}^{7}plongés au milieu de ses œuvres,
        ils poursuivent leur recherche
        et se laissent prendre aux apparences :
        \\ce qui s’offre à leurs yeux est si beau !
        ${}^{8}Encore une fois, ils n’ont pas d’excuse.
        ${}^{9}S’ils ont poussé la science à un degré tel
        qu’ils sont capables d’avoir une idée
        sur le cours éternel des choses,
        \\comment n’ont-ils pas découvert plus vite
        Celui qui en est le Maître ?
        
           
${}^{10}Mais malheureux, car ils espèrent en des choses mortes,
        ceux qui ont appelé « divinités »
        des ouvrages de mains humaines,
        \\de l’or et de l’argent travaillés avec art,
        figurant des êtres vivants,
        \\ou une pierre quelconque, ouvrage d’une main d’autrefois !
${}^{11}Ainsi un bûcheron, qui a scié un arbre facile à transporter,
        il en a raclé toute l’écorce selon les règles,
        \\et, avec tout l’art qui convient,
        il a fabriqué un objet, pour les besoins de la vie courante.
${}^{12}Les chutes de son ouvrage,
        il les a fait brûler pour préparer sa nourriture,
        puis il s’est rassasié.
${}^{13}Quant à la chute qui ne pouvait servir à rien,
        ce bout de bois tordu et plein de nœuds,
        \\il s’est mis à le tailler pour occuper ses loisirs,
        et, en amateur, il l’a sculpté,
        \\il lui a donné une figure humaine
${}^{14}ou la ressemblance d’un quelconque animal.
        \\Il l’a recouvert de vermillon,
        en passant la surface au rouge ;
        tous les défauts du bois, il les a recouverts.
${}^{15}Il lui a fait une digne résidence
        et l’a installé dans le mur, bien fixé avec du fer.
${}^{16}Il a pris grand soin qu’il ne tombe pas,
        le sachant incapable de se soutenir lui-même :
        ce n’est en effet qu’une image qui a besoin de soutien.
${}^{17}Et pourtant, quand il prie pour acquérir des biens,
        pour se marier et avoir des enfants,
        il n’a pas honte de s’adresser à cet objet inanimé ;
        \\pour obtenir la santé, il invoque ce qui est faible ;
${}^{18}pour la vie, il implore ce qui est mort ;
        \\pour sa sécurité, il supplie la plus totale incompétence ;
        pour voyager, il recourt à ce qui ne peut faire un pas ;
${}^{19}et pour son gagne-pain, son ouvrage, l’heureux travail de ses mains,
        il demande l’efficacité aux mains les plus inefficaces.
      
         
      \bchapter{}
${}^{1}Cet autre, sur le point d’appareiller
        \\pour traverser les flots sauvages,
        \\invoque un morceau de bois,
        plus pourri que le bateau qui l’emporte !
${}^{2}Or ce bateau a été conçu par désir de profit,
        et la sagesse de l’artisan l’a construit.
${}^{3}Mais c’est ta providence, ô Père, qui tient la barre,
        car tu as ouvert un chemin dans la mer,
        un sentier sûr au milieu des flots.
${}^{4}Tu as montré par là que tu peux sauver de tout danger,
        même si l’on embarque sans être du métier.
${}^{5}Tu veux que les œuvres de ta Sagesse ne restent pas stériles ;
        c’est pourquoi des hommes osent confier leur vie à un peu de bois,
        \\et, traversant la mer houleuse sur un radeau,
        ils restent sains et saufs.
${}^{6}Ainsi, aux origines, quand périssaient des géants orgueilleux,
        \\l’espoir du monde, se réfugiant sur un radeau,
        préserva une semence pour de nouvelles générations,
        car c’est ta main qui tenait la barre.
        
           
         
${}^{7}Béni est le bois par lequel advient la justice !
${}^{8}Mais l’idole, elle, est maudite, avec celui qui l’a fabriquée,
        \\celui-ci pour en avoir été l’artisan,
        et cet objet corruptible, pour avoir reçu le nom de dieu.
${}^{9}Car Dieu déteste autant l’impie que son impiété,
${}^{10}et l’œuvre sera châtiée avec son auteur.
${}^{11}C’est pourquoi, contre les idoles des nations,
        il y aura aussi une intervention divine,
        \\car elles sont devenues, parmi les créatures de Dieu, une abomination,
        un scandale pour les âmes des hommes,
        un piège sous les pas des insensés.
        
           
${}^{12}L’idée de faire des idoles, c’est le commencement de la prostitution ;
        leur invention a corrompu la vie.
${}^{13}Il n’y avait pas d’idoles au commencement,
        et il n’y en aura pas toujours.
${}^{14}C’est par la vaine imagination des hommes
        qu’elles sont entrées dans le monde ;
        aussi une fin rapide leur est-elle réservée.
         
${}^{15}Un père, affligé par un deuil prématuré,
        a fait le portrait de son enfant trop vite enlevé ;
        \\cet être humain qui n’est plus qu’un cadavre,
        il l’honore maintenant comme un dieu,
        et il a transmis aux siens les rites d’un culte secret ;
${}^{16}puis, avec le temps, cet usage impie s’est fortifié :
        il a pris force de loi.
         
        \\De même, sur l’ordre des souverains,
        on rendait un culte à leurs statues :
${}^{17}les gens qui habitaient trop loin
        pour venir les voir en personne et les honorer
        firent réaliser chez eux leur effigie :
        \\on fabriqua une image bien visible du roi vénéré,
        pour faire une cour empressée à l’absent comme s’il était présent.
${}^{18}Le culte s’étendit même
        à ceux qui ne connaissaient pas le souverain,
        grâce à l’ambition de l’artiste ;
${}^{19}celui-ci, désireux sans doute de plaire à son maître,
        mit tout son art à le représenter plus beau que nature.
${}^{20}Alors la foule fut séduite par le charme de l’œuvre,
        et celui qui était jusque-là honoré comme un homme
        fut bientôt objet d’adoration.
${}^{21}Ce fut un piège mortel,
        lorsque des hommes, devenus esclaves du malheur ou du pouvoir,
        attribuèrent à la pierre et au bois le Nom incommunicable.
${}^{22}De plus, il ne leur suffit pas d’errer
        au sujet de la connaissance de Dieu,
        \\mais, alors que leur vie est pleine de conflits dus à l’ignorance,
        ils donnent le nom de « paix » à ces fléaux si grands :
${}^{23}meurtres rituels d’enfants, célébrations de mystères occultes,
        délires et cortèges au cérémonial extravagant ;
${}^{24}ainsi, ils ne respectent plus la pureté ni de la vie ni du mariage,
        mais ils conspirent pour s’entretuer
        et s’infligent les tourments de l’adultère.
${}^{25}Tout est mêlé : sang et meurtre, vol et fraude,
        corruption, déloyauté, sédition, parjure,
${}^{26}confusion des valeurs, oubli des bienfaits,
        souillure morale, perversion sexuelle,
        désordre dans le mariage, adultère et débauche.
         
${}^{27}Oui, le culte des idoles sans nom
        est le commencement, la cause et le comble de tout mal.
         
${}^{28}Ceux qui participent à ces réjouissances
        délirent ou profèrent de faux oracles,
        vivent dans l’injustice ou se parjurent aussitôt :
${}^{29}quand on se fie à des idoles inanimées,
        on peut jurer à tort et à travers, sans redouter aucun préjudice.
${}^{30}Mais c’est à double titre que la justice les poursuivra :
        parce qu’ils se sont mépris sur Dieu,
        en s’attachant aux idoles,
        \\et que, dans leur perfidie, ils ont fait de faux serments
        par mépris de la sainteté.
${}^{31}Ce qui sanctionne la faute des coupables,
        ce n’est pas la puissance de ce que l’on prend à témoin
        mais la justice réservée aux pécheurs.
      
         
      \bchapter{}
${}^{1}Toi, notre Dieu, tu es bon et véridique,
        \\tu es patient et tu gouvernes l’univers avec miséricorde.
${}^{2}Même si nous en venions à pécher, nous resterions à toi,
        conscients de ta souveraineté ;
        \\mais nous ne pécherons pas,
        conscients que nous sommes comptés pour tiens.
${}^{3}Car te connaître est la parfaite justice,
        et la conscience de ta souveraineté est racine d’immortalité.
${}^{4}Nous ne nous sommes pas laissé égarer
        par les inventions humaines d’un art maléfique,
        ni par le labeur stérile de ceux qui peignent en trompe-l’œil :
        \\ce sont des formes aux touches de couleurs disparates,
${}^{5}dont la vue excite la passion des insensés,
        pleins de désir pour la forme inerte d’une figure morte.
${}^{6}Ils sont amants du mal
        et ne valent pas mieux que l’objet de leurs espérances,
        \\ceux qui les fabriquent,
        ceux qui les désirent et ceux qui les adorent.
        
           
${}^{7}Voici un potier qui pétrit de la terre glaise à grand-peine,
        et façonne un par un des objets à notre usage.
        \\Avec la même glaise, il façonne
        les vases les plus nobles et les plus grossiers,
        procédant, pour tous, de la même manière.
        \\Quelle sera la fonction de chaque objet ?
        C’est le potier qui en décide !
${}^{8}Et puis, avec la même glaise,
        il perd sa peine à façonner un dieu de vanité,
        \\lui qui, né de la terre il y a peu de temps,
        retournera bientôt à la terre dont il fut tiré,
        lorsqu’on lui réclamera son âme.
         
${}^{9}Eh bien, il n’a pas souci de la mort qui l’attend,
        ni de la brièveté de sa vie,
        \\mais il rivalise avec les orfèvres et les fondeurs d’argent,
        il imite ceux qui coulent le bronze,
        et se fait gloire de sa contrefaçon.
${}^{10}Son cœur n’est que cendre,
        son espoir, plus vil que de la terre,
        et son existence, plus méprisable que de la glaise :
${}^{11}il a ignoré celui qui l’a façonné,
        qui lui a insufflé une âme agissante,
        et, de son souffle, l’a doté d’un esprit de vie.
${}^{12}Il a pris notre existence pour un jeu d’enfant,
        la vie pour une fête et un marché :
        \\« Il faut tirer profit de tout, dit-il,
        même du mal. »
${}^{13}Mieux que quiconque, celui-là sait qu’il pèche,
        puisque, avec de la terre comme matériau,
        il se fait créateur d’objets fragiles aussi bien que d’idoles.
${}^{14}Mais ils sont tous vraiment fous,
        et plus misérables qu’une âme infantile,
        les ennemis et les oppresseurs de ton peuple.
${}^{15}Car ils ont même pris pour des dieux toutes les idoles des nations,
        qui n’ont pas l’usage des yeux pour voir,
        \\pas de narines pour respirer,
        ni d’oreilles pour entendre,
        \\ni de doigts aux mains pour toucher,
        et dont les pieds sont incapables de marcher.
${}^{16}Celui qui les a faites n’est qu’un homme ;
        celui qui les a façonnées n’a reçu le souffle de vie qu’à titre de prêt.
        \\Nul homme n’a le pouvoir de façonner un dieu qui lui soit semblable ;
${}^{17}mortel, il fait une œuvre morte de ses mains de faussaire ;
        \\il a plus de valeur que les objets qu’il adore,
        car lui, il a reçu la vie ; eux, jamais !
${}^{18}Ces gens-là adorent même les bêtes les plus odieuses,
        des bêtes pires que les autres en fait de stupidité,
${}^{19}sans rien de la beauté qui pourrait séduire chez certains animaux ;
        non, elles ont échappé à l’approbation de Dieu, à sa bénédiction.
      <p class="cantique" id="bib_ct-at_12"><span class="cantique_label">Cantique AT 12</span> = <span class="cantique_ref"><a class="unitex_link" href="#bib_sg_16_20">Sg 16,20-21.26</a> ; <a class="unitex_link" href="#bib_sg_17_1">17,1</a></span>
      
         
      \bchapter{}
${}^{1}Voilà pourquoi ils ont mérité d’être châtiés
        \\par des animaux de ce genre,
        \\et tourmentés par toutes sortes de bêtes.
${}^{2}À la place de ce châtiment,
        c’est un bienfait que tu as accordé à ton peuple ;
        \\en réponse à sa faim dévorante,
        tu as préparé une nourriture au goût extraordinaire : des cailles.
${}^{3}Ainsi les premiers, malgré la nécessité de se nourrir,
        perdaient jusqu’à l’envie de manger,
        à la vue des bêtes répugnantes qui leur étaient envoyées ;
        \\les seconds, au contraire, après une courte disette,
        recevaient un aliment au goût extraordinaire.
${}^{4}Il fallait qu’une famine implacable frappe les oppresseurs,
        mais, pour les autres, il suffisait que leur soit montré
        comment leurs ennemis étaient tourmentés.
        
           
${}^{5}Et même, quand s’abattit sur les tiens
        la fureur terrible de bêtes venimeuses,
        \\lorsqu’ils périssaient sous la morsure de serpents tortueux,
        ta colère ne persista pas jusqu’à la fin.
${}^{6}C’est en guise d’avertissement qu’ils avaient été alarmés
        pour un peu de temps,
        \\mais ils possédaient un signe de salut,
        qui leur rappelait le commandement de ta Loi.
${}^{7}Celui qui se tournait vers ce signe était sauvé,
        non pas à cause de ce qu’il regardait,
        mais par toi, le Sauveur de tous.
${}^{8}Ainsi tu as prouvé à nos ennemis
        que tu es Celui qui délivre de tout mal.
${}^{9}Eux périrent sous la morsure des sauterelles et des mouches,
        sans qu’on ait trouvé de remède,
        \\parce qu’ils méritaient d’être châtiés par de telles bêtes.
${}^{10}Tes fils, en revanche,
        \\même la dent des serpents venimeux n’a pu les vaincre,
        car ta miséricorde est intervenue et les a guéris.
${}^{11}Afin de se rappeler tes paroles,
        ils étaient piqués par l’aiguillon, puis aussitôt délivrés,
        \\pour qu’en évitant de tomber dans un oubli profond,
        ils restent attentifs à ton action bienfaisante.
${}^{12}Ni plante ni onguent ne fut leur remède,
        mais ta Parole, Seigneur, elle qui guérit tout.
${}^{13}Toi, tu as pouvoir sur la vie et sur la mort,
        tu fais descendre aux portes des enfers, et tu en ramènes.
${}^{14}L’homme, dans sa malice, peut tuer,
        mais il ne peut pas faire revenir le souffle qui est parti
        et ne peut délivrer l’âme emmenée captive.
         
${}^{15}Il est impossible d’échapper à ta main.
${}^{16}Les impies, qui refusaient de te connaître,
        furent soumis au fouet par la force de ton bras,
        \\pourchassés par des pluies étranges, des grêles,
        d’impitoyables trombes d’eau,
        \\et dévorés par le feu.
${}^{17}Le plus extraordinaire, c’est que ce feu redoublait d’énergie
        au milieu de l’eau qui aurait dû l’éteindre.
        \\L’univers se pose en défenseur des justes.
${}^{18}Tantôt la flamme se faisait douce,
        pour ne pas consumer les bêtes envoyées contre les impies
        \\– et à ce spectacle, ils pouvaient se reconnaître
        poursuivis par un jugement divin –,
${}^{19}tantôt la flamme flambait jusqu’au milieu de l’eau,
        excédant le pouvoir habituel du feu,
        pour ravager les produits d’une terre injuste.
         
        ${}^{20}À l’inverse, tu donnais à ton peuple une nourriture d’ange ;
        tu envoyais du ciel un pain tout préparé,
        obtenu sans effort,
        \\un pain aux multiples saveurs
        qui comblait tous les goûts,
        ${}^{21}substance qui révélait ta douceur
        envers tes enfants,
        \\qui servait le désir de chacun
        et s’accordait à ses vœux\\.
${}^{22}\[Elle était neige et glace,
        mais résistait au feu et ne fondait pas,
        \\pour leur faire savoir que le feu avait flamboyé dans la grêle
        et lancé des éclairs dans la pluie
        afin de ravager les récoltes des ennemis ;
${}^{23}ce même feu, en revanche,
        allait jusqu’à oublier son pouvoir,
        pour que les justes soient nourris.
         
${}^{24}La création, docile à te servir, toi, son Auteur,
        se tend comme un arc pour châtier les injustes,
        et se détend pour combler de biens ceux qui se fient à toi.
${}^{25}C’est pourquoi, se prêtant, une fois encore, à toute transformation,
        elle était au service de ce que tu donnais :
        \\une nourriture universelle
        accordée au désir de ceux qui t’imploraient.\]
        ${}^{26}Ainsi les fils que tu aimais, Seigneur, devaient l’apprendre :
        l’homme n’est pas nourri par le fruit des semences ;
        ta parole maintient celui qui croit en toi\\.
${}^{27}\[Car ce que le feu ne pouvait détruire
        fondait à la chaleur d’un simple rayon de soleil.
${}^{28}On saurait ainsi qu’il faut devancer le soleil pour te rendre grâce
        et venir à ta rencontre au lever du jour.
${}^{29}L’espoir de l’ingrat fondra comme le givre hivernal,
        il s’écoulera comme une eau qui se perd.\]
       
      
         
      \bchapter{}
        ${}^{1}Qu’ils sont grands, tes jugements,
        \\et difficiles à faire comprendre !
        Ceux qui n’en sont pas instruits s’égarent.
        
           
${}^{2}Des gens sans loi, qui prétendaient asservir une nation sainte,
        se retrouvaient enchaînés par les ténèbres ;
        \\prisonniers d’une longue nuit,
        comme enfermés sous un toit,
        \\bannis de la Providence éternelle,
        ils gisaient.
${}^{3}Ils avaient cru pouvoir passer inaperçus,
        dissimulant leurs fautes sous le voile opaque de l’oubli !
        \\Ils furent dispersés, en proie à d’horribles frayeurs,
        terrorisés par des hallucinations.
${}^{4}Car même le réduit où ils étaient enfermés
        ne les protégeait pas de la peur :
        \\des bruits fracassants retentissaient tout autour,
        et des spectres sinistres à la face lugubre apparaissaient ;
${}^{5}aucun feu, si puissant soit-il,
        ne parvenait à produire de la lumière,
        \\et la lueur flamboyante des étoiles
        n’osait pas éclairer cette nuit de cauchemar.
${}^{6}Quelque chose paraissait seulement
        qui brûlait de soi-même en répandant la terreur :
        \\quand cessait la vision, ils demeuraient dans l’angoisse
        et redoutaient plus encore ce qu’ils venaient de voir.
         
${}^{7}Les artifices de la magie restaient sans effet :
        démenti humiliant pour cette prétendue sagesse !
${}^{8}Ceux qui se targuaient de savoir repousser
        les angoisses et les troubles d’une âme malade,
        étaient à présent malades d’une anxiété risible.
${}^{9}Même si aucun phénomène troublant ne les menaçait,
        le passage des bêtes et le sifflement des serpents
        suffisaient à les terroriser :
${}^{10}ils mouraient de peur,
        redoutant d’ouvrir les yeux
        sur les ténèbres qu’ils ne pouvaient fuir.
         
${}^{11}La méchanceté est lâche
        lorsque son propre témoignage la condamne,
        \\toujours elle voit grandir les obstacles
        quand sa conscience l’oppresse ;
${}^{12}car la peur n’est rien d’autre
        que la défaillance des secours de la raison :
${}^{13}moins on compte intérieurement sur cette aide,
        plus grandit l’ignorance de ce qui cause le tourment.
         
${}^{14}Cette nuit sans pouvoir avait surgi
        des tréfonds du séjour des morts, lui aussi sans pouvoir,
        et ils étaient plongés dans ce même sommeil,
${}^{15}à la fois saisis d’hallucinations monstrueuses
        et paralysés par la défaillance de leur âme :
        une peur soudaine, inattendue, avait fondu sur eux.
${}^{16}Ainsi celui qui tombait là, quel qu’il fût,
        était retenu captif d’une prison sans grilles.
${}^{17}Laboureur ou berger,
        ou manœuvre à la peine en un lieu désert,
        \\chacun, à l’improviste, était frappé d’un sort inéluctable ;
${}^{18}car une même chaîne de ténèbres les tenait tous liés.
         
        \\Et le vent qui sifflait,
        et le chant mélodieux des oiseaux dans l’épaisse ramure,
        \\le bruit rythmé des eaux puissantes,
${}^{19}le grand fracas d’un éboulis de pierre,
        \\la course invisible d’animaux bondissants,
        les cris, le rugissement des bêtes féroces,
        \\et l’écho répercuté au creux des montagnes,
        tout les paralysait d’épouvante.
${}^{20}Car le monde entier resplendissait de lumière
        et s’adonnait librement à ses activités ;
${}^{21}mais sur eux seuls s’étendait une nuit pesante,
        image des ténèbres qui les recevraient bientôt ;
        \\et ils se sentaient eux-mêmes plus pesants que les ténèbres.
       
      
         
      \bchapter{}
${}^{1}Pour tes fidèles, au contraire, c’était une lumière magnifique.
        \\Les autres, qui entendaient des voix sans voir de formes,
        les disaient bienheureux d’avoir échappé à ces souffrances,
${}^{2}ils les remerciaient de ne pas se venger
        après tant de mauvais traitements,
        et les suppliaient de partir.
${}^{3}À l’inverse, tu as donné aux tiens une colonne de feu,
        guide pour un voyage inconnu,
        soleil bienveillant pour une migration splendide.
${}^{4}Les autres méritaient bien d’être privés de lumière
        et prisonniers des ténèbres,
        \\puisqu’ils avaient tenu en captivité tes fils,
        par qui devait être donnée au monde
        la lumière incorruptible de la Loi.
        
           
${}^{5}Ils avaient décidé de tuer les nouveau-nés du peuple saint,
        dont un seul enfant, exposé à la mort, avait été sauvé ;
        \\alors, pour les confondre,
        tu leur as enlevé un grand nombre d’enfants,
        et tu les as fait périr, tous à la fois, dans les eaux immenses.
        ${}^{6}Cette nuit avait été connue d’avance par nos Pères ;
        assurés des promesses auxquelles ils avaient cru,
        ils étaient dans la joie.
        ${}^{7}Et ton peuple accueillit à la fois le salut des justes
        et la ruine de leurs ennemis.
        ${}^{8}En même temps que tu frappais nos adversaires,
        tu nous appelais à la gloire.
        ${}^{9}Dans le secret de leurs maisons\\,
        les fidèles descendants des justes offraient un sacrifice\\,
        \\et ils consacrèrent d’un commun accord cette loi divine :
        que les saints partageraient aussi bien le meilleur que le pire ;
        et déjà ils entonnaient les chants de louange des Pères.
         
${}^{10}À ces chants répondait la clameur dissonante des ennemis
        une voix plaintive résonnait partout,
        une lamentation sur leurs enfants morts.
${}^{11}Un même châtiment frappait l’esclave et son maître :
        l’homme du peuple et le roi souffraient pareillement.
${}^{12}Tous à la fois, frappés de cet unique fléau,
        avaient des morts sans nombre,
        \\et, pour les ensevelir, les vivants n’y suffisaient pas,
        puisqu’en un rien de temps
        leur descendance la plus précieuse avait péri.
${}^{13}Devant ce qu’ils prenaient pour des sortilèges,
        ils étaient restés tout à fait incrédules,
        \\mais devant l’extermination des premiers-nés,
        ils confessèrent que ce peuple était fils de Dieu.
         
        ${}^{14}Un silence paisible enveloppait toute chose,
        et la nuit de la Pâque\\était au milieu de son cours rapide ;
        ${}^{15}alors, du haut du ciel, venant de ton trône royal, Seigneur\\,
        ta Parole toute-puissante
        fondit en plein milieu de ce pays de détresse,
        \\comme un guerrier impitoyable,
        portant l’épée tranchante de ton décret inflexible\\.
        ${}^{16}Elle s’arrêta, et sema partout la mort ;
        elle touchait au ciel et marchait aussi sur la terre.
${}^{17}Alors, brusquement,
        de fantastiques visions de cauchemar les secouèrent,
        de soudaines frayeurs les assaillirent,
${}^{18}et chacun s’abattait à demi-mort, ici ou là,
        manifestant la cause de sa mort,
${}^{19}car les songes qui les avaient troublés les avaient prévenus,
        afin que nul ne meure sans savoir la raison d’un tel malheur.
${}^{20}L’épreuve de la mort atteignit aussi les justes :
        au désert, il y eut un grand massacre.
        \\Mais la colère divine ne dura pas longtemps ;
${}^{21}sans attendre, un homme irréprochable prit leur défense,
        muni des armes de son sacerdoce :
        la prière, et l’encens offert pour le pardon.
        \\Il affronta la fureur et mit un terme au fléau,
        montrant qu’il était bien ton serviteur.
${}^{22}Il triompha du courroux, non par sa vigueur
        ni la puissance des armes ;
        \\mais, par la parole, il fit céder le Vengeur
        en faisant mémoire des serments et des alliances de nos pères.
${}^{23}Comme déjà les cadavres s’amoncelaient,
        il s’interposa, repoussa la colère,
        et lui barra le chemin des vivants.
${}^{24}Sur sa longue robe sacerdotale figurait l’univers entier ;
        \\le nom glorieux des patriarches
        était gravé sur quatre rangs de pierres,
        et ta majesté reposait sur le diadème de sa tête.
${}^{25}Voilà ce qui fit reculer l’Exterminateur,
        voilà ce qui le remplit de crainte.
        La seule épreuve de la colère avait suffi.
      
         
      \bchapter{}
${}^{1}Mais sur les impies une fureur implacable s’abattit jusqu’au bout,
        \\car ce qu’ils allaient faire était connu d’avance :
${}^{2}après avoir permis aux tiens de s’en aller
        et même pressé leur départ,
        ils changeraient d’avis et les poursuivraient.
${}^{3}Comme ils étaient encore en plein deuil
        et gémissaient sur les tombes de leurs morts,
        \\il leur vint une autre idée absurde :
        poursuivre comme des fugitifs
        ceux qu’ils avaient suppliés de partir.
${}^{4}Une juste contrainte les poussait à cette extrémité,
        effaçant de leur mémoire les événements passés :
        \\ils porteraient ainsi à son comble le châtiment
        par un supplice qui leur manquait encore.
${}^{5}Tandis que ton peuple ferait l’expérience d’un voyage extraordinaire,
        ceux-là trouveraient une mort étrange.
        
           
         
        ${}^{6}La création entière, dans sa propre nature,
        était remodelée au service de tes décrets,
        pour que tes enfants soient gardés sains et saufs.
        ${}^{7}On vit la nuée recouvrir le camp de son ombre,
        on vit la terre sèche émerger là où il n’y avait eu que de l’eau ;
        \\de la mer Rouge surgit un chemin sans obstacles
        et, des flots impétueux, une plaine verdoyante.
        ${}^{8}C’est là que le peuple entier, protégé par ta main, traversa
        en contemplant des prodiges merveilleux.
        ${}^{9}Ils étaient comme des chevaux dans un pré,
        ils bondissaient comme des agneaux
        \\et chantaient ta louange, Seigneur :
        tu les avais délivrés.
        
           
${}^{10}Ils se rappelaient encore les événements de leur exil,
        \\comment la terre, et non les animaux,
        avait engendré des moustiques,
        \\comment le Fleuve, et non des bêtes aquatiques,
        avait vomi quantité de grenouilles ;
${}^{11}et par la suite, ils virent encore
        un nouveau mode de génération pour les oiseaux,
        \\lorsque, poussés par le désir,
        ils réclamèrent des nourritures délicates :
${}^{12}pour leur réconfort, des cailles surgirent de la mer !
${}^{13}Quant aux sanctions qui frappèrent les pécheurs,
        elles furent annoncées par de violents éclairs.
        \\Il était juste qu’ils souffrent en raison de leurs crimes,
        car ils avaient vraiment fait preuve d’une haine cruelle
        envers les étrangers.
${}^{14}D’autres, jadis, n’avaient pas accueilli des inconnus de passage,
        mais eux réduisirent en esclavage
        des étrangers qui étaient leurs bienfaiteurs.
${}^{15}Bien plus : les premiers n’échapperont pas à l’intervention divine
        pour avoir reçu les étrangers de façon odieuse,
${}^{16}mais eux, après avoir accueilli par des fêtes
        ceux qui partageaient déjà les mêmes droits,
        se mirent à les maltraiter, les soumettant à de terribles corvées !
${}^{17}Eux aussi furent donc frappés de cécité,
        comme les premiers devant la porte du juste :
        \\ils étaient enveloppés de ténèbres sans fond,
        et chacun cherchait un passage vers sa propre porte.
${}^{18}Alors les éléments naturels s’accordèrent différemment entre eux,
        comme, sur la harpe, les notes varient selon le rythme,
        tandis que toujours demeure la musique.
        \\On peut le voir avec précision
        en observant ces événements :
${}^{19}des animaux terrestres devenaient aquatiques,
        et ceux qui nagent se mettaient à marcher sur la terre.
${}^{20}Le feu redoublait de puissance dans l’eau,
        et l’eau oubliait qu’il est de sa nature d’éteindre le feu ;
${}^{21}en revanche, les flammes ne consumaient pas
        la chair des êtres fragiles qui s’y déplaçaient ;
        \\elles ne faisaient pas fondre non plus
        ce qui semblait du givre facile à fondre :
        la nourriture d’immortalité.
         
${}^{22}Ainsi, Seigneur, en toutes choses,
        tu as donné à ton peuple grandeur et gloire,
        \\et tu n’as pas manqué, en tout temps et en tout lieu,
        d’être présent à ses côtés.
