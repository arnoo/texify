  
  
      <h2 class="intertitle hmbot" id="d85e276315">1. Arrestation et jugement de Jérémie (26)</h2>
      
         
      \bchapter{}
      \begin{verse}
${}^{1}Au début du règne de Joakim, fils de Josias, roi de Juda, il y eut cette parole venant du Seigneur : 
${}^{2} Ainsi parle le Seigneur : Tiens-toi dans la cour de la maison du Seigneur. Aux gens de toutes les villes de Juda qui viennent se prosterner dans la maison du Seigneur, tu diras toutes les paroles que je t’ai ordonné de leur dire ; n’en retranche pas un mot. 
${}^{3} Peut-être écouteront-ils, et reviendront-ils chacun de son mauvais chemin ? Alors je renoncerai au mal que je projette de leur faire à cause de la malice de leurs actes. 
${}^{4} Tu leur diras donc : Ainsi parle le Seigneur : Si vous ne m’écoutez pas, si vous ne marchez pas selon ma Loi, celle que j’ai mise sous vos yeux, 
${}^{5} si vous n’écoutez pas les paroles de mes serviteurs les prophètes, que je vous envoie inlassablement, et que vous n’avez pas écoutés, 
${}^{6} je traiterai cette Maison comme celle de Silo, et ferai de cette ville un exemple de malédiction pour toutes les nations de la terre.
      
         
       
${}^{7}Les prêtres, les prophètes et tout le peuple entendirent Jérémie proclamer ces paroles dans la maison du Seigneur. 
${}^{8}Et quand Jérémie eut fini de dire à tout le peuple tout ce que le Seigneur lui avait ordonné de dire, les prêtres, les prophètes et tout le peuple se saisirent de lui en disant : « Tu vas mourir ! 
${}^{9}Pourquoi prophétises-tu, au nom du Seigneur, que cette Maison deviendra comme celle de Silo, que cette ville sera dévastée et vidée de ses habitants ? » Et tout le peuple se rassembla autour de Jérémie dans la maison du Seigneur. 
${}^{10}Lorsque les princes de Juda apprirent ces événements, ils montèrent de la maison du roi à la maison du Seigneur et siégèrent à l’entrée de la Porte-Neuve-du-Seigneur.
       
${}^{11}Alors les prêtres et les prophètes dirent aux princes et à tout le peuple : « Cet homme mérite la mort, car il a prophétisé contre cette ville ; vous l’avez entendu de vos oreilles. » 
${}^{12} À son tour Jérémie s’adressa à tous les princes et à tout le peuple : « C’est le Seigneur qui m’a envoyé prophétiser contre cette Maison et contre cette ville, et dire\\toutes les paroles que vous avez entendues. 
${}^{13} Et maintenant, rendez meilleurs vos chemins et vos actes, écoutez la voix du Seigneur votre Dieu ; alors il renoncera au malheur qu’il a proféré contre vous. 
${}^{14} Quant à moi, me voici entre vos mains, faites de moi ce qui vous semblera bon et juste. 
${}^{15} Mais sachez-le bien : si vous me faites mourir, vous allez vous charger d’un sang innocent, vous-mêmes et cette ville et tous\\ses habitants. Car c’est vraiment le Seigneur qui m’a envoyé vers vous proclamer toutes ces paroles pour que vous les entendiez\\. »
${}^{16}Alors les princes et tout le peuple dirent aux prêtres et aux prophètes : « Cet homme ne mérite pas la mort, car c’est au nom du Seigneur notre Dieu qu’il nous a parlé. » 
${}^{17}Quelques-uns des anciens du pays se levèrent pour dire à toute l’assemblée du peuple : 
${}^{18}« Michée de Morèsheth, qui était prophète au temps d’Ézékias, roi de Juda, a déclaré à tout le peuple de Juda : Ainsi parle le Seigneur de l’univers :
        \\Sion sera un champ qu’on laboure,
        Jérusalem, un monceau de décombres,
        \\et la montagne du Temple,
        des lieux sacrés envahis par la forêt.
${}^{19}Est-ce que le roi de Juda, Ézékias, et tous les gens de Juda l’ont fait mourir ? Le roi n’a-t-il pas craint le Seigneur, n’a-t-il pas apaisé le visage du Seigneur, si bien que le Seigneur a renoncé au malheur qu’il avait proféré contre eux ? Et nous, nous irions faire ce grand mal à nous-mêmes ? »
       
${}^{20}Il y eut aussi un homme qui prophétisait au nom du Seigneur. C’était Ourias, fils de Shemayahou, originaire de Qirath-Yearim ; il prophétisait contre cette ville et contre ce pays dans les mêmes termes que Jérémie. 
${}^{21}Le roi Joakim, avec tous ses guerriers et ses princes, en fut informé et chercha à le faire mourir. Ourias, en l’apprenant, prit peur, s’enfuit et parvint en Égypte. 
${}^{22}Alors le roi Joakim envoya des hommes en Égypte : Elnatane, fils d’Akbor, et quelques autres avec lui. 
${}^{23}Ils firent sortir Ourias d’Égypte et l’amenèrent au roi Joakim qui le frappa de l’épée et fit jeter son cadavre dans la fosse commune. 
${}^{24}Mais comme la protection\\d’Ahiqam, fils de Shafane, était acquise à Jérémie, il échappa aux mains de ceux qui voulaient le faire mourir.
      <h2 class="intertitle" id="d85e276542">2. Conflits avec les prophètes au temps de Sédécias (27 – 29)</h2>
      
         
      \bchapter{}
      \begin{verse}
${}^{1}Au début du règne de Sédécias, fils de Josias, roi de Juda, cette parole fut adressée à Jérémie de la part du Seigneur. 
${}^{2}Ainsi m’a parlé le Seigneur : Fais-toi des liens et des jougs, place-les sur ta nuque. 
${}^{3}Puis, envoie-les au roi d’Édom, au roi de Moab et au roi des Ammonites, au roi de Tyr et au roi de Sidon, par les messagers venus à Jérusalem auprès de Sédécias, roi de Juda. 
${}^{4}Et tu leur ordonneras de dire à leurs maîtres : Ainsi parle le Seigneur de l’univers, le Dieu d’Israël :
        \\Vous parlerez ainsi à vos maîtres :
${}^{5}C’est moi qui ai fait la terre,
        l’homme et le bétail sur la face de la terre,
        \\par ma grande force et mon bras étendu,
        et je les donne à qui bon me semble.
${}^{6}Et moi, maintenant, je livre tous ces pays aux mains de Nabucodonosor, roi de Babylone, mon serviteur ; je lui donne même les bêtes sauvages, pour qu’elles le servent ; 
${}^{7}toutes les nations le serviront, lui, son fils et son petit-fils, jusqu’à ce que vienne pour son pays le temps où il sera lui-même asservi à des nations nombreuses et à de grands rois. 
${}^{8}Quant à la nation ou au royaume qui ne servira pas Nabucodonosor, roi de Babylone, qui ne livrera pas sa nuque au joug du roi de Babylone, cette nation, je la châtierai par l’épée, la famine et la peste – oracle du Seigneur – avant de la faire achever par sa main. 
${}^{9}Et vous, n’écoutez pas vos prophètes ni vos devins, ni vos songes, ni vos astrologues, ni vos mages, eux qui vous disent : « Non, vous ne servirez pas le roi de Babylone ! » 
${}^{10}C’est le mensonge qu’ils vous prophétisent ! Et à cause de cela, vous serez bannis de votre terre : je vous chasserai, et vous périrez. 
${}^{11}Mais la nation qui mettra sa nuque sous le joug du roi de Babylone et qui le servira, je lui donnerai le repos sur sa terre – oracle du Seigneur ; elle pourra la cultiver et l’habiter.
${}^{12}À Sédécias, roi de Juda, je parlai dans les mêmes termes ; je lui dis : Mettez votre nuque sous le joug du roi de Babylone, servez-le, lui et son peuple, et vous vivrez. 
${}^{13}Pourquoi devriez-vous mourir, toi et ton peuple, par l’épée, la famine et la peste, comme le Seigneur l’a dit à la nation qui ne servirait pas le roi de Babylone ? 
${}^{14}N’écoutez pas les paroles des prophètes qui vous disent : « Non, vous ne servirez pas le roi de Babylone ! » C’est le mensonge qu’ils vous prophétisent ! 
${}^{15}Je ne les ai pas envoyés – oracle du Seigneur –, mais ils prophétisent en mon nom le mensonge ! À cause de cela, je vais vous chasser, et vous périrez, vous et vos prophètes.
${}^{16}Quant aux prêtres et à tout ce peuple, je leur déclarai : Ainsi parle le Seigneur. N’écoutez pas les paroles de vos prophètes qui prophétisent pour vous dire : « Maintenant et sans tarder, les objets de la maison du Seigneur vont être ramenés de Babylone ! », car c’est le mensonge qu’ils vous prophétisent. 
${}^{17}Ne les écoutez pas ! Servez le roi de Babylone, et vous vivrez. Pourquoi cette ville serait-elle une ruine ? 
${}^{18}S’ils sont prophètes, si la parole du Seigneur est avec eux, qu’ils supplient donc le Seigneur de l’univers pour que les objets qui sont encore dans la maison du Seigneur, dans la maison du roi de Juda et dans Jérusalem, ne s’en aillent pas à Babylone ! 
${}^{19}Car ainsi parle le Seigneur de l’univers au sujet des colonnes, de la Mer et des bases, ainsi que des autres objets qui sont encore dans cette ville, 
${}^{20}ce que n’a pas enlevé Nabucodonosor, roi de Babylone, quand il déporta de Jérusalem à Babylone Jékonias, fils de Joakim, roi de Juda, avec tous les notables de Juda et de Jérusalem. 
${}^{21}Oui, ainsi parle le Seigneur de l’univers, le Dieu d’Israël, à propos des objets qui sont encore dans la maison du Seigneur, dans la maison du roi de Juda et dans Jérusalem : 
${}^{22}ils seront emportés à Babylone et ils y resteront jusqu’au jour où je m’en occuperai – oracle du Seigneur. Alors, je les ferai remonter et revenir en ce lieu.
      
         
      \bchapter{}
      \begin{verse}
${}^{1}Cette année-là, au début du règne de Sédécias, roi de Juda, la quatrième année, au cinquième mois, le prophète Ananie, fils d’Azzour, originaire de Gabaon, me dit dans la maison du Seigneur, en présence des prêtres et de tout le peuple : 
${}^{2} « Ainsi parle le Seigneur de l’univers, le Dieu d’Israël : J’ai brisé le joug du roi de Babylone ! 
${}^{3} Dans deux ans, jour pour jour, je ferai revenir en ce lieu tous les objets de la maison du Seigneur que Nabucodonosor, roi de Babylone, a enlevés\\pour les emporter à Babylone. 
${}^{4} Je ramènerai ici Jékonias, fils de Joakim, roi de Juda, avec tous les déportés de Juda qui sont partis à Babylone – oracle du Seigneur –, car je vais briser le joug du roi de Babylone ! »
${}^{5}Le prophète Jérémie répondit au prophète Ananie en présence des prêtres et de tout le peuple, qui se tenaient dans la maison du Seigneur. 
${}^{6} Il lui dit : « Amen ! Que le Seigneur agisse ainsi, que le Seigneur accomplisse ta prophétie : qu’il fasse revenir\\de Babylone les objets de la maison du Seigneur et tous les déportés. 
${}^{7} Cependant, écoute bien cette parole que je vais te faire entendre, à toi et à tout le peuple : 
${}^{8} Les prophètes qui nous ont précédés, toi et moi, depuis bien longtemps, ont prophétisé contre de nombreux pays et de grands royaumes la guerre, le malheur et la peste. 
${}^{9} Le prophète qui annonce la paix n’est reconnu comme prophète vraiment envoyé par le Seigneur, que si sa parole s’accomplit. »
${}^{10}Alors le prophète Ananie enleva le joug que le prophète Jérémie s’étais mis sur la nuque, et il le brisa. 
${}^{11} Et Ananie déclara en présence de tout le peuple : « Ainsi parle le Seigneur : De la même manière, dans deux ans, jour pour jour, je briserai le joug de Nabucodonosor, roi de Babylone, pour en délivrer\\toutes les nations. » Alors le prophète Jérémie alla son chemin.
       
${}^{12}La parole du Seigneur fut adressée à Jérémie après que le prophète Ananie eut brisé le joug qui était sur sa nuque\\. 
${}^{13} « Va dire à Ananie : Ainsi parle le Seigneur : Tu as brisé un joug\\de bois, mais à sa place tu feras\\un joug de fer. 
${}^{14} Car ainsi parle le Seigneur de l’univers, le Dieu d’Israël : C’est un joug de fer que je mets sur la nuque de toutes ces nations, pour qu’elles servent Nabucodonosor, roi de Babylone. Et elles le serviront. Je lui ai donné même les bêtes sauvages. » 
${}^{15} Le prophète Jérémie dit alors au prophète Ananie : « Écoute bien, Ananie : le Seigneur ne t’a pas envoyé, et toi, tu rassures ce peuple par un mensonge. 
${}^{16} C’est pourquoi, ainsi parle le Seigneur : Je te renvoie de la surface de la terre ; tu mourras cette année, car c’est la révolte que tu as prêchée contre le Seigneur. » 
${}^{17} Le prophète Ananie mourut cette même année, au septième mois.
      
         
      \bchapter{}
      \begin{verse}
${}^{1}Voici les termes de la lettre que le prophète Jérémie envoya de Jérusalem à ceux des anciens qui survivaient en exil, aux prêtres, aux prophètes et à tout le peuple, que Nabucodonosor avait déportés de Jérusalem à Babylone. 
${}^{2}C’était après que le roi Jékonias eut quitté Jérusalem avec la reine mère, les dignitaires, les princes de Juda et de Jérusalem, les artisans et forgerons. 
${}^{3}Cette lettre fut confiée à Élasa, fils de Shafane, et à Guemarya, fils de Hilqiya, que Sédécias, roi de Juda, avait envoyés à Babylone auprès de Nabucodonosor, roi de Babylone. Elle disait :
${}^{4}« Ainsi parle le Seigneur de l’univers, le Dieu d’Israël, à tous les exilés que j’ai déportés de Jérusalem à Babylone :
${}^{5}Bâtissez des maisons et habitez-les,
        plantez des jardins et mangez de leurs fruits.
${}^{6}Prenez des femmes et engendrez des fils et des filles,
        prenez des femmes pour vos fils ;
        donnez vos filles en mariage,
        et qu’elles enfantent des fils et des filles ;
        \\multipliez-vous là-bas,
        et ne diminuez pas !
${}^{7}Recherchez la paix
        en faveur de la ville où je vous ai déportés,
        \\et intercédez pour elle auprès du Seigneur,
        car de sa paix dépend votre paix.
         
${}^{8}Oui, ainsi parle le Seigneur de l’univers, le Dieu d’Israël :
        \\Ne vous laissez abuser
        ni par les prophètes qui sont au milieu de vous,
        ni par vos devins !
        \\N’écoutez pas vos songes,
        les songes que vous provoquez !
${}^{9}Car c’est le mensonge qu’ils prophétisent en mon nom.
        \\Je ne les ai pas envoyés
        – oracle du Seigneur.
         
${}^{10}Oui, ainsi parle le Seigneur :
        \\Dès que les soixante-dix ans seront révolus pour Babylone,
        je vous visiterai,
        \\j’accomplirai pour vous ma parole de bonheur,
        en vous ramenant en ce lieu.
${}^{11}Car moi, je connais les pensées que je forme à votre sujet
        – oracle du Seigneur –,
        \\pensées de paix et non de malheur,
        pour vous donner un avenir et une espérance.
${}^{12}Vous m’invoquerez, vous approcherez, vous me prierez,
        et je vous écouterai.
${}^{13}Vous me chercherez et vous me trouverez ;
        oui, recherchez-moi de tout votre cœur.
${}^{14}Je me laisserai trouver par vous
        – oracle du Seigneur –,
        \\et je ramènerai vos captifs.
        \\Je vous rassemblerai de toutes les nations et de tous les lieux
        où je vous avais chassés
        – oracle du Seigneur –,
        \\et je vous ramènerai au lieu dont je vous avais exilés.
       
${}^{15}Puisque vous dites : “Le Seigneur a suscité pour nous des prophètes à Babylone”, 
${}^{16}ainsi parle le Seigneur au roi qui siège sur le trône de David et à tout le peuple qui habite cette ville, à vos frères qui ne sont pas partis en exil avec vous, 
${}^{17}ainsi parle le Seigneur de l’univers :
        \\Je vais envoyer chez eux l’épée, la famine et la peste ;
        je les traiterai comme des figues pourries,
        si mauvaises qu’elles sont immangeables.
${}^{18}Je les persécuterai avec l’épée, la famine et la peste ;
        \\je ferai d’eux un objet de stupeur
        pour tous les royaumes de la terre,
        \\un objet d’imprécation et de désolation,
        de dérision et d’insulte,
        parmi toutes les nations où je les aurai chassés,
${}^{19}car ils n’ont pas écouté mes paroles
        – oracle du Seigneur –,
        \\ils n’ont pas écouté mes serviteurs les prophètes
        que je leur ai envoyés inlassablement
        – oracle du Seigneur.
       
${}^{20}Quant à vous, écoutez la parole du Seigneur, vous, tous les déportés que j’ai envoyés de Jérusalem à Babylone : 
${}^{21}Ainsi parle le Seigneur de l’univers, le Dieu d’Israël, à Acab, fils de Kolaya, et à Sédécias, fils de Maaséya, qui, en mon nom, vous prophétisent le mensonge.
        \\Je vais les livrer aux mains de Nabucodonosor, roi de Babylone,
        qui les frappera sous vos yeux.
${}^{22}Et désormais on formulera cette malédiction
        chez tous les déportés de Juda qui se trouvent à Babylone :
        \\“Que le Seigneur te traite comme Sédécias et Acab,
        grillés au feu par le roi de Babylone !”
${}^{23}C’est qu’ils ont commis une infamie en Israël
        et se sont livrés à l’adultère
        avec les femmes de leur prochain.
        \\Et ils ont dit, en mon nom, une parole de mensonge,
        alors que je ne leur avais rien ordonné.
        \\Moi, je le sais, j’en suis témoin
        – oracle du Seigneur. »
${}^{24}À Shemayahou, le Néhélamite, tu diras : 
${}^{25}Ainsi parle le Seigneur de l’univers, le Dieu d’Israël : Tu as envoyé des lettres en ton nom à tout le peuple qui se trouve à Jérusalem, au prêtre Sophonie, fils de Maaséya, et à tous les prêtres, pour dire ceci : 
${}^{26}« Le Seigneur a fait de toi un prêtre à la place du prêtre Joad, pour surveiller dans la maison du Seigneur tout homme qui délire ou fait le prophète, afin de l’attacher au pilori ou au carcan. 
${}^{27}Alors, pourquoi ne corriges-tu pas Jérémie d’Anatoth qui fait le prophète parmi vous ? 
${}^{28}Car il nous a envoyé à Babylone le message suivant : “Ce sera long ! Bâtissez des maisons et habitez-les, plantez des jardins et mangez de leurs fruits.” »
       
${}^{29}Or le prêtre Sophonie avait lu cette lettre au prophète Jérémie. 
${}^{30}Alors la parole du Seigneur fut adressée à Jérémie : 
${}^{31}Envoie ce message à tous les exilés : « Ainsi parle le Seigneur au sujet de Shemayahou, le Néhélamite :
        \\Parce que Shemayahou a prophétisé parmi vous,
        sans que je l’envoie,
        \\afin de vous rassurer par un mensonge,
${}^{32}à cause de cela – ainsi parle le Seigneur –
        \\je vais sévir contre Shemayahou, le Néhélamite,
        et contre sa descendance.
        \\Aucun d’entre eux n’aura sa place au milieu de ce peuple,
        aucun ne verra le bien que je vais faire à mon peuple
        – oracle du Seigneur –,
        \\car c’est la révolte qu’il a prêchée contre le Seigneur. »
      <h2 class="intertitle" id="d85e277275">3. Livret de consolation. Oracles d’espérance (30 – 33)</h2>
      
         
      \bchapter{}
      \begin{verse}
${}^{1}Parole du Seigneur adressée à Jérémie\\ : 
${}^{2} « Ainsi parle le Seigneur, le Dieu d’Israël : Écris dans un livre toutes les paroles que je t’ai dites.
${}^{3}Car voici venir des jours – oracle du Seigneur – où je ramènerai les captifs de mon peuple Israël et Juda, dit le Seigneur ; je les ramènerai dans le pays que j’ai donné à leurs pères, et ils en prendront possession. »
${}^{4}Voici donc les paroles que le Seigneur a dites à Israël et à Juda :
${}^{5}Ainsi parle le Seigneur :
        \\Nous avons entendu un cri d’effroi,
        c’est la terreur et non la paix.
${}^{6}Interrogez donc et voyez :
        un mâle peut-il enfanter ?
        \\Pourquoi tout homme que je vois a-t-il les mains sur les hanches
        comme une femme qui enfante ?
        \\Pourquoi tous les visages sont-ils soudain livides ?
${}^{7}Malheur ! Il est grand, ce jour-là,
        à nul autre semblable !
        \\C’est le temps de l’angoisse pour Jacob,
        mais il en sera sauvé.
${}^{8}Il arrivera en ce jour-là – oracle du Seigneur de l’univers –
        que je briserai le joug qui est sur ta nuque
        et je romprai tes liens.
        \\Alors, ils ne seront plus asservis à des étrangers.
${}^{9}Ils serviront le Seigneur leur Dieu
        et David, leur roi que je susciterai pour eux.
${}^{10}Mais toi, Jacob mon serviteur, ne crains pas
        – oracle du Seigneur –,
        \\ne tremble pas, Israël,
        \\car je vais te sauver des terres lointaines,
        sauver ta descendance de la terre où elle est captive.
        \\Jacob reviendra, il sera en sécurité,
        tranquille, sans personne qui l’inquiète,
${}^{11}car je suis avec toi pour te sauver
        – oracle du Seigneur.
        \\Oui, j’exterminerai toutes les nations
        parmi lesquelles je t’ai dispersé ;
        \\et toi, je ne t’exterminerai pas,
        \\mais je te corrigerai selon le droit
        et ne te laisserai pas impuni.
         
        ${}^{12}Ainsi parle le Seigneur :
        \\Sion\\, incurable est ta blessure,
        et profonde, ta plaie.
        ${}^{13}Nul ne défend ta cause pour qu’on soigne ton ulcère ;
        pas de remède pour le cicatriser.
        ${}^{14}Tous tes amants t’ont oubliée,
        aucun ne te recherche.
        \\Oui, comme un ennemi je t’ai blessée
        – sévère correction !
        \\Sur la masse de tes fautes,
        tes péchés n’ont cessé de s’accroître.
        ${}^{15}Qu’as-tu à crier à cause de ta blessure ?
        Ta peine est incurable.
        \\Sur la masse de tes fautes,
        tes péchés n’ont cessé de s’accroître :
        c’est pourquoi\\je t’ai infligé cela.
${}^{16}Mais tous ceux qui te dévorent seront dévorés ;
        tous tes adversaires, oui tous, iront en captivité.
        \\Ceux qui te dépouillent seront dépouillés,
        et tous ceux qui te pillent, je les livrerai au pillage.
${}^{17}Oui, je vais cicatriser et guérir ta plaie
        – oracle du Seigneur –,
        \\Sion, toi qu’on appelle « Expulsée »,
        « Celle que nul ne recherche ».
         
        ${}^{18}Ainsi parle le Seigneur :
        \\Voici que je vais restaurer les tentes de Jacob,
        pour ses demeures j’aurai de la compassion ;
        \\la ville sera rebâtie sur ses ruines,
        la citadelle sera rétablie en sa juste place.
        ${}^{19}Les actions de grâce en jailliront
        avec des cris de joie.
        \\Bien loin de diminuer ses fils\\, je les multiplierai ;
        bien loin de les abaisser, je les glorifierai.
        ${}^{20}Ils seront comme autrefois,
        leur communauté se maintiendra devant moi,
        car je punirai tous ses oppresseurs.
        ${}^{21}Jacob aura pour maître l’un des siens,
        un chef qui sera issu de lui.
        \\Je lui permettrai d’approcher
        et il aura accès auprès de moi.
        \\Qui donc, en effet, a jamais osé
        de lui-même s’approcher de moi ?
        – oracle du Seigneur.
        ${}^{22}Vous serez mon peuple,
        et moi, je serai votre Dieu.
${}^{23}Voici la tempête du Seigneur ;
        sa fureur éclate, la tempête s’installe,
        elle tournoie sur la tête des méchants.
${}^{24}L’ardente colère du Seigneur ne se détournera pas
        avant d’avoir agi et réalisé les desseins de son cœur.
        \\Dans les derniers jours, vous le comprendrez.
      <p class="cantique" id="bib_ct-at_36"><span class="cantique_label">Cantique AT 36</span> = <span class="cantique_ref"><a class="unitex_link" href="#bib_jr_31_10">Jr 31, 10-14</a></span>
      
         
      \bchapter{}
        ${}^{1}En ce temps-là – oracle du Seigneur –,
        je serai le Dieu de toutes les familles\\d’Israël,
        et elles seront mon peuple.
        ${}^{2}Ainsi parle le Seigneur :
        \\Il a trouvé grâce dans le désert,
        le peuple qui a échappé au massacre\\ ;
        \\Israël est en route vers Celui qui le fait reposer.
        
           
         
        ${}^{3}Depuis les lointains, le Seigneur m’est apparu :
        \\Je t’aime d’un amour éternel,
        aussi je te garde ma fidélité.
        ${}^{4}De nouveau je te bâtirai,
        et tu seras rebâtie, vierge d’Israël.
        \\De nouveau tu prendras tes tambourins de fête
        pour te mêler aux danses joyeuses.
        ${}^{5}De nouveau tu planteras des vignes
        dans les montagnes de Samarie,
        \\et ceux qui les planteront
        en goûteront le premier fruit.
        ${}^{6}Un jour viendra où les veilleurs crieront
        dans la montagne d’Éphraïm :
        \\« Debout, montons à Sion,
        vers le Seigneur notre Dieu ! »
        
           
         
        ${}^{7}Car ainsi parle le Seigneur :
        \\Poussez des cris de joie pour Jacob,
        acclamez la première des nations !
        \\Faites résonner vos louanges et criez tous :
        \\« Seigneur, sauve ton peuple,
        le reste d’Israël ! »
        
           
         
        ${}^{8}Voici que je les fais revenir du pays du nord,
        que je les rassemble des confins de la terre ;
        \\parmi eux, tous ensemble, l’aveugle et le boiteux,
        la femme enceinte et la jeune accouchée :
        c’est une grande assemblée qui revient.
        ${}^{9}Ils avancent dans les pleurs et les supplications,
        \\je les mène, je les conduis vers les cours d’eau
        par un droit chemin où ils ne trébucheront pas.
        \\Car je suis un père pour Israël,
        Éphraïm est mon fils aîné.
        
           
        ${}^{10}Écoutez, nations, la parole du Seigneur !
        \\Annoncez dans les îles lointaines :
        \\« Celui qui dispersa Israël le rassemble,
        il le garde, comme un berger son troupeau.
        ${}^{11}Le Seigneur a libéré Jacob,
        l’a racheté des mains d’un plus fort.
         
        ${}^{12}Ils viennent, criant de joie, sur les hauteurs de Sion :
        ils affluent vers les biens du Seigneur,
        \\le froment, le vin nouveau et l’huile fraîche,
        les génisses et les brebis du troupeau\\.
        \\Ils auront l’âme comme un jardin tout irrigué ;
        ils verront la fin de leur détresse.
         
        ${}^{13}La\\jeune fille se réjouit, elle danse ;
        jeunes gens, vieilles gens, tous ensemble !
        \\Je change leur deuil en joie,
        les réjouis, les console après la peine.
        ${}^{14}Je nourris mes prêtres de festins ;
        \\mon peuple se rassasie de mes biens »
        – oracle du Seigneur.
${}^{15}Ainsi parle le Seigneur :
        \\Un cri s’élève dans Rama,
        une plainte et des pleurs d’amertume.
        \\C’est Rachel qui pleure ses fils ;
        elle refuse d’être consolée,
        car ses fils ne sont plus.
${}^{16}Ainsi parle le Seigneur :
        \\Retiens le cri de tes pleurs
        et les larmes de tes yeux.
        \\Car il y a un salaire pour ta peine,
        – oracle du Seigneur :
        ils reviendront du pays de l’ennemi.
${}^{17}Il y a un espoir pour ton avenir,
        – oracle du Seigneur :
        tes fils reviendront sur leur territoire.
${}^{18}J’entends bien Éphraïm se plaindre :
        \\« Tu m’as corrigé, et je suis corrigé,
        comme un jeune taureau non dressé.
        \\Fais-moi revenir, et je reviendrai,
        car c’est toi qui es le Seigneur mon Dieu.
${}^{19}Oui, je me repens après être revenu ;
        \\après avoir reconnu qui je suis,
        je me frappe la poitrine.
        \\Je rougis et je suis confus,
        car je porte la honte de ma jeunesse. »
         
${}^{20}Éphraïm n’est-il pas pour moi un fils précieux,
        n’est-il pas un enfant de délices,
        \\puisque son souvenir ne me quitte plus
        chaque fois que j’ai parlé de lui ?
        \\Voilà pourquoi, à cause de lui, mes entrailles frémissent ;
        oui, je lui ferai miséricorde
        – oracle du Seigneur.
         
${}^{21}Dresse pour toi des signaux,
        pose pour toi des jalons ;
        \\sois attentive à la route,
        au chemin sur lequel tu as marché !
        \\Reviens, vierge d’Israël,
        reviens ici, vers tes villes !
${}^{22}Combien de temps, fille rebelle,
        vas-tu encore vagabonder ?
        \\Le Seigneur crée du nouveau dans le pays :
        la femme entourera l’homme !
${}^{23}Ainsi parle le Seigneur de l’univers, le Dieu d’Israël : On dira encore cette parole au pays de Juda et dans ses villes, quand je ramènerai leurs captifs :
        \\« Que le Seigneur te bénisse,
        demeure de justice,
        montagne de sainteté ! »
${}^{24}Là, habiteront ensemble
        les gens de Juda et de toutes ses villes,
        les laboureurs et les nomades.
${}^{25}Car je vais désaltérer l’âme qui défaille ;
        toute âme en détresse, je la comblerai.
       
${}^{26}Là-dessus, je me suis réveillé et j’ai compris. Que mon sommeil avait été agréable !
       
${}^{27}Voici venir des jours – oracle du Seigneur –,
        \\où j’ensemencerai la maison d’Israël et la maison de Juda
        d’une semence d’homme et d’une semence de bétail.
${}^{28}Et, de même que j’ai veillé sur eux
        pour arracher et renverser,
        \\pour démolir et détruire,
        pour provoquer le malheur,
        \\de même je veillerai sur eux
        pour bâtir et planter
        – oracle du Seigneur.
${}^{29}En ces jours-là, on ne dira plus :
        \\« Les pères ont mangé du raisin vert,
        et les dents des fils en sont irritées. »
${}^{30}Mais chacun mourra pour sa propre faute ;
        \\tout homme qui mangera du raisin vert,
        ses propres dents en seront irritées.
${}^{31}Voici venir des jours – oracle du Seigneur –, où je conclurai avec la maison d’Israël et avec la maison de Juda une alliance nouvelle. 
${}^{32} Ce ne sera pas comme l’alliance que j’ai conclue avec leurs pères, le jour où je les ai pris par la main pour les faire sortir du pays d’Égypte : mon alliance, c’est eux qui l’ont rompue, alors que moi, j’étais leur maître – oracle du Seigneur.
${}^{33}Mais voici quelle sera l’alliance que je conclurai avec la maison d’Israël quand ces jours-là seront passés – oracle du Seigneur. Je mettrai ma Loi au plus profond d’eux-mêmes ; je l’inscrirai sur leur cœur. Je serai leur Dieu, et ils seront mon peuple. 
${}^{34} Ils n’auront plus à instruire chacun son compagnon, ni chacun son frère en disant : « Apprends à connaître\\le Seigneur ! » Car tous me connaîtront, des plus petits jusqu’aux plus grands – oracle du Seigneur. Je pardonnerai leurs fautes, je ne me rappellerai plus leurs péchés.
        ${}^{35}Ainsi parle le Seigneur,
        \\lui qui a fait le soleil
        pour éclairer pendant le jour,
        \\qui a établi les lois de la lune et des étoiles
        pour éclairer pendant la nuit,
        \\qui soulève la mer et fait mugir ses flots\\.
        Son nom est « Le Seigneur de l’univers »\\.
        ${}^{36}Si jamais ces lois disparaissent devant ma face
        – oracle du Seigneur –,
        \\alors la descendance d’Israël
        cessera, elle aussi, pour toujours
        d’être une nation devant ma face.
        ${}^{37}Ainsi parle le Seigneur :
        \\Si jamais on peut mesurer le ciel, là-haut,
        et sonder les fondations de la terre, en bas,
        \\alors moi aussi, je rejetterai toute la descendance d’Israël
        pour tout ce qu’elle aura fait
        – oracle du Seigneur.
${}^{38}Voici venir des jours – oracle du Seigneur – où la ville sera reconstruite pour le Seigneur, depuis la tour de Hananéel jusqu’à la porte de l’Angle. 
${}^{39}On sortira de nouveau le cordeau à mesurer, pour le tendre sur la colline de Gareb ; ensuite, on le tournera vers Goa. 
${}^{40}Et toute la vallée des cadavres et des cendres, tous les cimetières, jusqu’au torrent du Cédron, jusqu’à l’angle de la porte des Chevaux, à l’est, tout sera consacré au Seigneur. On n’arrachera plus, on ne démolira plus jamais.
      
         
      \bchapter{}
      \begin{verse}
${}^{1}Parole du Seigneur adressée à Jérémie, la dixième année du règne de Sédécias, roi de Juda ; c’était la dix-huitième année du règne de Nabucodonosor.
${}^{2}L’armée du roi de Babylone assiégeait alors Jérusalem, et le prophète Jérémie était retenu prisonnier dans la cour de garde, celle de la maison du roi de Juda. 
${}^{3}C’est là que Sédécias, roi de Juda, l’avait enfermé en lui disant : « Pourquoi fais-tu cette prophétie ? Tu as dit : “Ainsi parle le Seigneur : Je vais livrer cette ville aux mains du roi de Babylone qui la prendra. 
${}^{4}Et Sédécias, roi de Juda, n’échappera pas aux mains des Chaldéens, mais il sera bel et bien livré aux mains du roi de Babylone ; il lui parlera face à face et ses yeux verront ses yeux. 
${}^{5}À Babylone, il emmènera Sédécias, qui y restera jusqu’à ce que je le visite – oracle du Seigneur. Si vous faites la guerre aux Chaldéens, vous ne réussirez pas !” ».
       
${}^{6}Or, voici ce que dit Jérémie : Cette parole du Seigneur m’a été adressée : 
${}^{7}« Hanaméel, le fils de ton oncle Shalloum, va venir te trouver pour te dire : “Achète-toi mon champ d’Anatoth, c’est toi qui as droit de rachat pour l’acquérir !” » 
${}^{8}Hanaméel, le fils de mon oncle, vint me trouver dans la cour de garde, selon la parole du Seigneur, et il me dit : « Achète donc mon champ d’Anatoth, au pays de Benjamin, car tu as droit de propriété et droit de rachat. Achète-le ! » Je compris que c’était là une parole du Seigneur 
${}^{9}et j’achetai le champ d’Anatoth à Hanaméel, le fils de mon oncle, et je lui pesai l’argent : dix-sept pièces d’argent. 
${}^{10}Je rédigeai un acte, le scellai devant ceux que j’avais pris comme témoins, et je pesai l’argent dans une balance. 
${}^{11}Puis, je pris l’acte d’acquisition, la partie scellée – avec l’ordre et les clauses – et la partie ouverte. 
${}^{12}Et je remis l’acte d’acquisition à Baruc, fils de Nériya, fils de Mahséya, sous les yeux de Hanaméel, fils de mon oncle, sous les yeux des témoins signataires de l’acte et sous les yeux de tous les Judéens qui se trouvaient dans la cour de garde. 
${}^{13}Sous leurs yeux, j’ordonnai ceci à Baruc : 
${}^{14}« Ainsi parle le Seigneur de l’univers, le Dieu d’Israël : Prends ces documents, cet acte d’acquisition, la partie scellée et la partie ouverte, et dépose-les dans un vase en terre cuite, pour qu’ils se conservent longtemps ; 
${}^{15}car, ainsi parle le Seigneur de l’univers, le Dieu d’Israël : Dans ce pays, on achètera encore des maisons, des champs et des vignes. »
       
${}^{16}Après avoir remis l’acte d’acquisition à Baruc, fils de Nériya, j’adressai au Seigneur cette prière : 
${}^{17}« Ah ! Seigneur mon Dieu, c’est toi qui as fait le ciel et la terre par ta grande force et ton bras étendu, et rien n’est impossible pour toi. 
${}^{18}Tu montres ta fidélité à des milliers, mais la faute des pères, tu la fais payer à leurs fils après eux. Le Dieu grand, le valeureux, son nom est “Le Seigneur de l’univers”. 
${}^{19}Grand par tes desseins et riche en exploits, toi dont les yeux sont ouverts sur tous les chemins des hommes afin de rendre à chacun selon sa conduite et selon le fruit de ses actes, 
${}^{20}toi qui as opéré des signes et des prodiges jusqu’à ce jour, au pays d’Égypte, en Israël et dans l’humanité, tu t’es fait le nom que tu portes en ce jour. 
${}^{21}Tu fis sortir ton peuple Israël du pays d’Égypte par des signes et des prodiges, à main forte et à bras étendu, dans une grande crainte. 
${}^{22}Tu leur donnas ce pays que tu avais promis par serment à leurs pères, pays ruisselant de lait et de miel. 
${}^{23}Ils y sont entrés, ils en ont pris possession, mais ils n’ont pas écouté ta voix, ils n’ont pas marché selon tes lois, ils n’ont rien fait de tout ce que tu leur avais ordonné ; et tu as provoqué contre eux tout ce malheur. 
${}^{24}Voici que les remblais s’élèvent pour prendre la ville ; par l’épée, la famine et la peste, la ville est livrée aux mains des Chaldéens qui lui font la guerre. Ce que tu as dit se réalise : toi-même, tu le vois. 
${}^{25}Et toi, Seigneur mon Dieu, tu me dis, alors que la ville est livrée aux mains des Chaldéens : “Achète-toi ce champ à prix d’argent et prends des témoins !” »
       
${}^{26}Alors la parole du Seigneur fut adressée à Jérémie : 
${}^{27}« Moi, je suis le Seigneur, le Dieu de tout être de chair. Y a-t-il quelque chose qui me soit impossible ? 
${}^{28}C’est pourquoi – ainsi parle le Seigneur – je vais livrer cette ville aux mains des Chaldéens, aux mains de Nabucodonosor, roi de Babylone, et il la prendra. 
${}^{29}Les Chaldéens qui font la guerre à cette ville entreront dans cette ville et y mettront le feu, incendiant les maisons où l’on brûle sur les terrasses de l’encens au dieu Baal, où l’on verse des libations à d’autres dieux ; ce qui m’offense. 
${}^{30}Oui, depuis leur jeunesse, les fils d’Israël et les fils de Juda ne font que ce qui est mal à mes yeux ; les fils d’Israël ne font que m’offenser par les œuvres de leurs mains – oracle du Seigneur. 
${}^{31}Oui, cette ville excite en moi colère et fureur depuis le jour de sa fondation jusqu’à aujourd’hui, au point que je l’écarte de ma présence, 
${}^{32}à cause de tout le mal que les fils d’Israël et les fils de Juda ont fait pour m’offenser, eux, leurs rois et leurs princes, leurs prêtres et leurs prophètes, les gens de Juda et les habitants de Jérusalem ! 
${}^{33}Ils ont tourné vers moi leur dos, et non leur visage. Inlassablement, je les instruisais, mais aucun d’eux n’écoutait pour accepter la leçon. 
${}^{34}Ils ont installé leurs horreurs pour souiller la Maison sur laquelle mon nom est invoqué. 
${}^{35}Ils ont édifié, au Val-de-la-Géhenne, les lieux sacrés de Baal pour y faire passer par le feu leurs fils et leurs filles en l’honneur de Moloch. Cela, je ne l’ai pas ordonné, ce n’est pas venu à mon esprit. Commettre une telle abomination, c’est faire pécher Juda ! »
       
${}^{36}Mais maintenant, ainsi parle le Seigneur, le Dieu d’Israël, au sujet de cette ville dont vous dites qu’elle est livrée aux mains du roi de Babylone par l’épée, la famine et la peste : 
${}^{37}« Je vais les rassembler de tous les pays où je les ai chassés dans ma colère, ma fureur, ma grande irritation ; je les ramènerai en ce lieu et les ferai habiter en sécurité. 
${}^{38}Ils seront mon peuple, et moi, je serai leur Dieu. 
${}^{39}Je leur donnerai un seul cœur, un seul chemin, afin qu’ils me craignent chaque jour, pour leur bonheur et celui de leurs fils après eux. 
${}^{40}Je conclurai avec eux une alliance éternelle : je ne cesserai pas de les suivre pour les rendre heureux et je mettrai ma crainte en leur cœur pour qu’ils ne s’écartent pas de moi. 
${}^{41}J’aurai de la joie à les rendre heureux ; en vérité, je les planterai dans ce pays, de tout mon cœur et de toute mon âme. »
       
${}^{42}Oui, ainsi parle le Seigneur : « De même que j’ai fait venir sur ce peuple tout ce grand malheur, de même, je fais venir sur eux tout le bonheur dont je parle. 
${}^{43}On achètera des champs dans ce pays dont vous dites : « C’est une terre désolée, sans hommes ni bétail, livrée aux mains des Chaldéens ! » 
${}^{44}On achètera des champs à prix d’argent, on rédigera des actes, on les scellera devant ceux qu’on aura pris comme témoins, au pays de Benjamin, aux alentours de Jérusalem, dans les villes de Juda, les villes de la Montagne, les villes du Bas-Pays et les villes du Néguev, quand je ramènerai leurs captifs – oracle du Seigneur. »
      
         
      \bchapter{}
      \begin{verse}
${}^{1}La parole du Seigneur fut adressée de nouveau à Jérémie, alors qu’il était encore détenu dans la cour de garde. 
${}^{2}Ainsi parle le Seigneur, qui a fait et façonné ce qui est stable, lui dont le nom est « Le Seigneur » : 
${}^{3}Invoque-moi, et je te répondrai, je te révélerai des choses grandes et inaccessibles que tu ne connais pas. 
${}^{4}Oui, ainsi parle le Seigneur, le Dieu d’Israël, à propos des maisons de cette ville et des maisons des rois de Juda, qui seront renversées face aux remblais et à l’épée. 
${}^{5}On viendra combattre les Chaldéens et remplir les maisons avec les cadavres des gens que j’ai frappés dans ma colère et ma fureur, car j’avais caché mon visage à cette ville à cause de toute leur méchanceté. 
${}^{6}Je vais cicatriser sa plaie et la guérir, je les guérirai. Je leur ferai voir à profusion la paix et la stabilité. 
${}^{7}Je ramènerai les captifs de Juda et les captifs d’Israël, je les rétablirai comme autrefois. 
${}^{8}Je les purifierai de toute la faute qu’ils ont commise envers moi ; je pardonnerai toutes les fautes qu’ils ont commises envers moi, en se révoltant contre moi. 
${}^{9}Cette ville fera ma joyeuse renommée, elle sera ma louange et ma parure, devant toutes les nations de la terre, quand elles apprendront tout le bonheur que je lui donne. Elles frémiront de crainte en voyant tout le bonheur et toute la paix que je lui donne.
      
         
       
${}^{10}Ainsi parle le Seigneur : Dans ce lieu dont vous dites : « C’est un désert sans hommes ni bétail ! », dans les villes de Juda et les rues de Jérusalem dévastées, sans habitants, ni hommes ni bétail, on entendra de nouveau 
${}^{11}les chants d’allégresse et les chants de joie, le chant de l’époux et le chant de l’épousée, le chant de ceux qui présentent le sacrifice d’action de grâce dans la maison du Seigneur, en disant :
        \\« Rendez grâce au Seigneur de l’univers !
        \\Oui, le Seigneur est bon :
        éternel est son amour ! »
      Car je ramènerai les captifs du pays ; ce sera comme autrefois, dit le Seigneur.
       
${}^{12}Ainsi parle le Seigneur de l’univers : Dans ce lieu déserté tant par les hommes que par le bétail, et dans toutes les villes de la contrée, il y aura encore un enclos où les bergers feront reposer leurs brebis. 
${}^{13}Dans les villes de la Montagne, du Bas-Pays et du Néguev, dans le pays de Benjamin, aux alentours de Jérusalem et dans les villes de Juda, les brebis passeront encore sous les mains de celui qui les compte, dit le Seigneur.
        ${}^{14}Voici venir des jours – oracle du Seigneur – 
        où j’accomplirai la parole de bonheur
        \\que j’ai adressée à la maison d’Israël
        et à la maison de Juda :
        ${}^{15}En ces jours-là, en ce temps-là,
        \\je ferai germer pour David un Germe de justice,
        et il exercera dans le pays le droit et la justice.
        ${}^{16}En ces jours-là, Juda sera sauvé,
        Jérusalem habitera en sécurité,
        \\et voici comment on la nommera :
        « Le-Seigneur-est-notre-justice. »
${}^{17}Oui, ainsi parle le Seigneur :
        \\Il ne manquera jamais à David
        un homme qui siège sur le trône de la maison d’Israël ;
${}^{18}et aux prêtres lévites, ne manquera jamais en ma présence
        un homme qui fasse monter l’holocauste,
        qui brûle l’offrande et présente chaque jour le sacrifice.
         
${}^{19}La parole du Seigneur fut adressée à Jérémie :
${}^{20}Ainsi parle le Seigneur :
        \\Si vous pouviez rompre mon alliance avec le jour
        et mon alliance avec la nuit,
        \\de sorte que jour et nuit
        ne viennent plus en leur temps,
${}^{21}alors serait aussi rompue mon alliance avec mon serviteur David :
        il n’aurait plus de fils régnant sur son trône.
        \\Et il en serait de même pour les prêtres lévites qui me servent.
${}^{22}Autant que l’armée du ciel impossible à dénombrer,
        autant que le sable de la mer impossible à compter,
        \\ainsi multiplierai-je la descendance de mon serviteur David
        et les Lévites qui me servent.
       
${}^{23}La parole du Seigneur fut adressée à Jérémie : 
${}^{24}N’as-tu pas remarqué ce que disent les gens de ce peuple : « Le Seigneur a rejeté les deux familles qu’il avait choisies » ? En parlant ainsi, ils méprisent mon peuple qui n’est plus pour eux une nation. 
${}^{25}Ainsi parle le Seigneur :
        \\Si je n’avais pas établi mon alliance avec le jour et la nuit,
        ni les lois du ciel et de la terre,
${}^{26}alors je pourrais aussi rejeter
        la descendance de Jacob et de mon serviteur David,
        \\et ne pas prendre en elle ceux qui gouvernent
        la descendance d’Abraham, d’Isaac et de Jacob.
        \\Mais non ! Je ramènerai leurs captifs,
        pour eux j’aurai de la compassion.
      <h2 class="intertitle" id="d85e279364">4. Siège de Jérusalem au temps de Sédécias (34 – 35)</h2>
      
         
      \bchapter{}
      \begin{verse}
${}^{1}Parole du Seigneur adressée à Jérémie, alors que Nabucodonosor, roi de Babylone, toute son armée, tous les royaumes de la terre soumis à son pouvoir et tous les peuples, étaient en guerre contre Jérusalem et contre toutes les villes qui en dépendent :
      
         
       
${}^{2}Ainsi parle le Seigneur, le Dieu d’Israël :
        \\Va dire à Sédécias, roi de Juda, va lui dire :
        \\« Ainsi parle le Seigneur :
        \\Je vais livrer cette ville
        aux mains du roi de Babylone qui l’incendiera.
${}^{3}Et toi, tu n’échapperas pas à ses mains,
        mais tu seras bel et bien capturé,
        livré entre ses mains.
        \\Tes yeux verront les yeux du roi de Babylone,
        il te parlera face à face,
        et tu iras à Babylone.
${}^{4}Cependant, écoute la parole du Seigneur, Sédécias, roi de Juda.
        \\Ainsi parle le Seigneur à ton sujet :
        \\Ce n’est pas par l’épée que tu mourras ;
${}^{5}tu mourras en paix.
        \\Comme on a brûlé des parfums pour tes pères,
        les rois d’autrefois qui t’ont précédé,
        \\ainsi, on en brûlera pour toi,
        et l’on fera cette lamentation :
        \\“Quel malheur, monseigneur !”
        Telle est la parole que j’ai dite, moi
        – oracle du Seigneur. »
       
${}^{6}Le prophète Jérémie adressa toutes ces paroles à Sédécias, roi de Juda, dans Jérusalem. 
${}^{7}L’armée du roi de Babylone était alors en guerre contre Jérusalem et contre les villes de Juda qui tenaient encore : Lakish et Azéqa ; c’étaient, en effet, les seules villes fortifiées qui restaient parmi les villes de Juda.
${}^{8}Parole du Seigneur adressée à Jérémie, après que le roi Sédécias eut conclu avec tout le peuple de Jérusalem une alliance qui proclamait l’affranchissement des esclaves. 
${}^{9}Chacun devait renvoyer libre son esclave hébreu, homme ou femme, de sorte que personne ne soit asservi à son frère judéen. 
${}^{10}Tous ceux qui étaient entrés dans l’alliance, les princes et tous les gens du peuple, acceptèrent de renvoyer libres leurs esclaves, hommes et femmes, de sorte qu’ils ne leur soient plus asservis. Ils obéirent donc et les renvoyèrent. 
${}^{11}Mais après cela, s’étant ravisés, ils firent revenir leurs esclaves, hommes et femmes, qu’ils avaient renvoyés libres, et les réduisirent à nouveau en servitude.
       
${}^{12}Alors, la parole du Seigneur fut adressée à Jérémie : 
${}^{13}Ainsi parle le Seigneur, le Dieu d’Israël : Moi, j’ai conclu une alliance avec vos pères, le jour où je les ai fait sortir du pays d’Égypte, de la maison de servitude, en déclarant : 
${}^{14}« Au bout de sept ans, chacun renverra son frère hébreu qui se sera vendu à lui. Il te servira durant six ans ; puis tu le renverras libre, de chez toi. » Mais vos pères ne m’ont pas écouté, ils n’ont pas prêté l’oreille. 
${}^{15}Vous, en ces jours-ci, vous vous étiez convertis, vous aviez fait ce qui est juste à mes yeux en proclamant chacun l’affranchissement de son prochain ; vous aviez conclu devant moi une alliance dans la Maison sur laquelle mon nom est invoqué. 
${}^{16}Mais vous vous êtes ravisés et vous avez profané mon nom : vous avez fait revenir les esclaves, hommes et femmes, que vous aviez renvoyés pleinement libres, et vous les avez réduits de nouveau à être vos esclaves.
${}^{17}C’est pourquoi, ainsi parle le Seigneur : Vous, en fait, vous ne m’avez pas obéi quand chacun de vous proclamait l’affranchissement de son frère, de son prochain ; alors moi, je vais proclamer contre vous – oracle du Seigneur – l’affranchissement de l’épée, de la peste et de la famine, et je ferai de vous un objet de stupeur pour tous les royaumes de la terre. 
${}^{18}Je livrerai les hommes qui ont transgressé mon alliance, qui n’ont pas accompli les paroles de l’alliance conclue devant moi, quand ils avaient coupé en deux un veau, et qu’ils étaient passés entre ses morceaux ; 
${}^{19}je livrerai les princes de Juda et les princes de Jérusalem, les dignitaires et les prêtres et tous les gens du pays, ceux qui sont passés entre les morceaux du veau, 
${}^{20}je les livrerai aux mains de leurs ennemis et aux mains de ceux qui en veulent à leur vie. Leurs cadavres serviront de pâture aux oiseaux du ciel et aux bêtes de la terre. 
${}^{21}Quant à Sédécias, roi de Juda, et à ses princes, je les livrerai aux mains de leurs ennemis, aux mains de ceux qui en veulent à leur vie, aux mains de l’armée du roi de Babylone, qui a levé le siège de Jérusalem. 
${}^{22}Voici que je donne un ordre – oracle du Seigneur : Je les fais revenir vers cette ville pour qu’ils l’attaquent, la prennent et l’incendient. Et je ferai des villes de Juda une désolation ; plus personne n’y habitera.
      
         
      \bchapter{}
      \begin{verse}
${}^{1}Parole du Seigneur adressée à Jérémie, au temps de Joakim, fils de Josias, roi de Juda : 
${}^{2}« Va trouver la famille des Récabites ; tu leur parleras et tu les amèneras à la maison du Seigneur, dans l’une des salles, et tu leur feras boire du vin. » 
${}^{3}Je pris donc Yaazanya, fils de Yirmeyahou, fils de Habacinya, ainsi que ses frères et tous ses fils, toute la famille des Récabites. 
${}^{4}Je les amenai à la maison du Seigneur, dans la salle des fils de Hanane, fils de Yigdaliahou, l’homme de Dieu – salle qui était à côté de celle des princes, au-dessus de la salle de Maaséyahou, fils de Shalloum, le gardien du seuil. 
${}^{5}Puis, devant les fils de la famille des Récabites, je plaçai des amphores pleines de vin ainsi que des coupes, et je leur dis : « Buvez du vin ! » 
${}^{6}Mais ils répondirent : « Nous ne boirons pas de vin, car Yonadab, fils de Récab, notre père, nous a donné cet ordre :
        \\“Vous ne boirez pas de vin,
        ni vous, ni vos fils, à jamais.
${}^{7}Vous ne bâtirez pas de maisons,
        vous ne ferez pas de semailles,
        \\vous ne planterez pas de vignes,
        vous ne posséderez rien de tout cela.
        \\Car vous habiterez sous la tente,
        tout au long de vos jours,
        \\afin de vivre de nombreux jours
        sur le sol où vous êtes comme des étrangers.”
${}^{8}Nous avons obéi à Yonadab, fils de Récab, notre père, en tout ce qu’il nous a ordonné : ne pas boire de vin tout au long de nos jours, nous, nos femmes, nos fils et nos filles ; 
${}^{9}ne pas bâtir de maisons pour y habiter. Nous ne possédons ni vignes, ni champs, ni semailles, 
${}^{10}nous habitons sous la tente. Nous obéissons et nous faisons en tous points ce que Yonadab notre père nous a ordonné. 
${}^{11}Mais lorsque Nabucodonosor, roi de Babylone, a envahi le pays, nous nous sommes dit : “Venez ! Entrons à Jérusalem pour échapper à l’armée des Chaldéens et à l’armée d’Aram !” Et depuis, nous habitons dans Jérusalem. »
${}^{12}Alors la parole du Seigneur fut adressée à Jérémie. 
${}^{13}« Ainsi parle le Seigneur de l’univers, le Dieu d’Israël : Va dire aux gens de Juda et aux habitants de Jérusalem : N’accepterez-vous pas la leçon en écoutant mes paroles ? – oracle du Seigneur. 
${}^{14}On a observé les paroles de Yonadab, fils de Récab : il avait défendu à ses fils de boire du vin, et ils n’en ont pas bu jusqu’à ce jour, car ils ont obéi à l’ordre de leur père. Et moi, qui vous ai parlé inlassablement, vous ne m’avez pas obéi ! 
${}^{15}Inlassablement je vous ai envoyé tous mes serviteurs les prophètes, pour dire : “Revenez chacun de votre mauvais chemin, rendez meilleurs vos actes et n’allez pas suivre d’autres dieux pour les servir ; vous habiterez sur le sol que je vous ai donné, à vous et à vos pères.” Mais vous n’avez pas prêté l’oreille, vous ne m’avez pas écouté. 
${}^{16}Ainsi, les fils de Yonadab, fils de Récab, ont observé l’ordre reçu de leur père, alors que ce peuple ne m’a pas obéi ! 
${}^{17}C’est pourquoi, ainsi parle le Seigneur, Dieu de l’univers, Dieu d’Israël : Je vais faire venir sur Juda et sur tous les habitants de Jérusalem tout le malheur que j’ai proféré contre eux ; parce que je leur ai parlé et ils n’ont pas écouté, je les ai appelés et ils n’ont pas répondu. »
${}^{18}Puis Jérémie dit à la famille des Récabites : « Ainsi parle le Seigneur de l’univers, le Dieu d’Israël : Vous avez obéi à l’ordre de Yonadab votre père, vous avez gardé tous ses commandements, vous avez fait en tous points ce qu’il vous a ordonné ; 
${}^{19}c’est pourquoi – ainsi parle le Seigneur de l’univers, le Dieu d’Israël – il ne manquera jamais à Yonadab, fils de Récab, un homme de sa descendance qui se tienne tous les jours devant moi. »
      <h2 class="intertitle" id="d85e279741">5. Les épreuves de Jérémie (36 – 45)</h2>
      
         
      \bchapter{}
      \begin{verse}
${}^{1}La quatrième année du règne de Joakim, fils de Josias, roi de Juda, cette parole du Seigneur fut adressée à Jérémie : 
${}^{2}« Procure-toi un rouleau ; tu écriras dessus toutes les paroles que je t’ai dites concernant Israël, Juda et toutes les nations, depuis le jour où j’ai commencé à te parler, au temps de Josias, jusqu’aujourd’hui. 
${}^{3}Peut-être qu’en apprenant tout le mal que je projette de leur faire, les gens de la maison de Juda reviendront chacun de son mauvais chemin, et je pardonnerai leurs fautes et leurs péchés. » 
${}^{4}Jérémie appela Baruc, fils de Nériya. Baruc écrivit dans le rouleau, sous la dictée de Jérémie, toutes les paroles que le Seigneur lui avait dites.
${}^{5}Alors Jérémie donna cet ordre à Baruc : « Je suis empêché, je ne peux entrer à la maison du Seigneur. 
${}^{6}Mais toi, tu y entreras, et les paroles du Seigneur que tu as écrites dans le rouleau sous ma dictée, tu les proclameras aux oreilles du peuple, dans la maison du Seigneur, le jour du jeûne. Et tu les proclameras aussi aux oreilles de tous les gens de Juda qui viennent de leurs villes. 
${}^{7}Peut-être leur supplication atteindra-t-elle le Seigneur. Peut-être chacun d’eux reviendra-t-il de son mauvais chemin. Car elle est grande, la colère du Seigneur, la fureur qu’il a exprimée envers ce peuple. » 
${}^{8}Baruc, fils de Nériya, se conforma en tout à l’ordre du prophète Jérémie : à la maison du Seigneur, il lut dans le livre les paroles du Seigneur.
       
${}^{9}La cinquième année du règne de Joakim, fils de Josias, roi de Juda, au neuvième mois, on proclama un jeûne devant le Seigneur pour tout le peuple de Jérusalem et pour tout le peuple qui venait des villes de Juda, à Jérusalem. 
${}^{10}Alors, à la maison du Seigneur, Baruc lut dans le livre les paroles de Jérémie ; c’était dans la salle de Guemaryahou, fils du scribe Shafane, dans la cour d’en haut, à l’entrée de la Porte Neuve de la maison du Seigneur ; il les proclama aux oreilles de tout le peuple.
${}^{11}Or, Michée, fils de Guemaryahou, fils de Shafane, entendit toutes les paroles du Seigneur écrites dans le livre. 
${}^{12}Il descendit à la maison du roi, dans la salle du scribe, et là, tous les princes tenaient séance. Il y avait le scribe Élishama, Delayahou, fils de Shemayahou, Elnathane, fils d’Akbor, Guemaryahou, fils de Shafane, Sédécias, fils de Hananyahou, et tous les autres princes. 
${}^{13}Michée leur rapporta toutes les paroles qu’il avait entendues quand Baruc les avait lues dans le livre aux oreilles du peuple.
       
${}^{14}Alors, tous les princes dépêchèrent vers Baruc Yehoudi, fils de Netanyahou, fils de Shèlèmyahou, fils de Koushi, pour lui dire : « Ce rouleau dont tu as fait la lecture aux oreilles du peuple, prends-le et viens ! » Baruc, fils de Nériya, prit donc le rouleau et s’en vint vers eux. 
${}^{15}Ils lui dirent : « Assieds-toi et fais-nous la lecture ! » Et Baruc leur en fit la lecture. 
${}^{16}Or, tandis qu’ils écoutaient toutes les paroles, ils se tournèrent l’un vers l’autre, pris de frayeur, et ils dirent à Baruc : « Il nous faut absolument rapporter au roi toute cette affaire ! » 
${}^{17}Puis ils interrogèrent Baruc : « Raconte-nous comment tu as écrit toutes ces paroles venant de sa bouche ! » 
${}^{18}Baruc leur répondit : « De sa bouche, il me dictait toutes ces paroles et moi, je les écrivais dans le livre avec de l’encre. » 
${}^{19}Les princes dirent alors à Baruc : « Va-t’en, cache-toi, ainsi que Jérémie ! Que personne ne sache où vous êtes ! » 
${}^{20}Puis ils se rendirent chez le roi, dans la cour du palais, après avoir déposé le rouleau dans la salle du scribe Élishama. Et ils rapportèrent aux oreilles du roi toute l’affaire.
${}^{21}Alors le roi envoya Yehoudi chercher le rouleau. Celui-ci alla le chercher dans la salle du scribe Élishama et en fit la lecture au roi ainsi qu’à tous les princes qui se tenaient debout auprès de lui. 
${}^{22}Le roi était assis dans la maison d’hiver – on était au neuvième mois – et un brasero était allumé devant lui. 
${}^{23}Chaque fois que Yehoudi avait lu trois ou quatre colonnes, le roi, avec le canif du scribe, les déchirait et les jetait dans le feu du brasero, jusqu’à ce que le rouleau eût entièrement disparu dans le feu du brasero. 
${}^{24}Mais ni le roi ni aucun de ses serviteurs, entendant toutes ces paroles, ne furent effrayés et ne déchirèrent leurs vêtements. 
${}^{25}Pourtant Elnathane, Delayahou et Guemaryahou avaient supplié le roi de ne pas brûler le rouleau, mais lui ne les avait pas écoutés. 
${}^{26}Ensuite, le roi ordonna à Yerahméel, fils du roi, à Serayahou, fils d’Azriel, et à Shèlèmyahou, fils d’Abdéel, de se saisir du scribe Baruc et du prophète Jérémie. Mais le Seigneur les avait cachés.
${}^{27}Alors la parole du Seigneur fut adressée à Jérémie, après que le roi eut brûlé le rouleau et les paroles que Baruc avait écrites sous la dictée de Jérémie : 
${}^{28}« Retourne te procurer un autre rouleau et écris dessus toutes les paroles qui étaient déjà dans le premier rouleau, celui qu’a brûlé Joakim, roi de Juda. 
${}^{29}Et contre Joakim, roi de Juda, tu diras : Ainsi parle le Seigneur : Toi, tu as brûlé ce rouleau en disant : “Pourquoi y avoir écrit : Le roi de Babylone viendra certainement, il détruira ce pays et en fera disparaître hommes et bétail ?” 
${}^{30}À cause de cela, ainsi parle le Seigneur contre Joakim, roi de Juda : Il n’aura personne pour lui succéder sur le trône de David. Son cadavre sera exposé à la chaleur du jour et au froid de la nuit. 
${}^{31}Je les châtierai de leurs fautes, lui, sa descendance et ceux qui le servent. Je ferai venir sur eux, sur les habitants de Jérusalem et sur les gens de Juda, tout le malheur que j’ai proféré contre eux sans qu’ils n’écoutent. »
${}^{32}Jérémie se procura un autre rouleau et le donna au scribe Baruc, fils de Nériya, lequel y écrivit sous la dictée de Jérémie toutes les paroles du livre que Joakim, roi de Juda, avait brûlé dans le feu. Beaucoup d’autres paroles semblables y furent ajoutées.
      
         
      \bchapter{}
      \begin{verse}
${}^{1}Le roi Sédécias, fils de Josias, régna à la place de Konias, fils de Joakim ; c’est lui, en effet, que Nabucodonosor, roi de Babylone, avait établi roi sur le pays de Juda. 
${}^{2}Cependant, ni lui, ni ses serviteurs, ni les gens du pays n’écoutaient les paroles que le Seigneur leur adressait par l’intermédiaire du prophète Jérémie.
${}^{3}Le roi Sédécias envoya Yehoucal, fils de Shèlèmya, et le prêtre Sophonie, fils de Maaséya, dire au prophète Jérémie : « De grâce, intercède pour nous auprès du Seigneur notre Dieu ! » 
${}^{4}Jérémie allait et venait au milieu du peuple : on ne l’avait pas encore mis en prison. 
${}^{5}L’armée de Pharaon était sortie d’Égypte et les Chaldéens qui assiégeaient Jérusalem, en apprenant cette nouvelle, avaient levé le siège de Jérusalem.
${}^{6}Alors la parole du Seigneur fut adressée au prophète Jérémie : 
${}^{7}« Ainsi parle le Seigneur, le Dieu d’Israël : Voici ce que vous direz au roi de Juda qui vous envoie vers moi pour me consulter : L’armée de Pharaon, qui est sortie pour vous porter secours, va retourner dans son pays d’Égypte. 
${}^{8}Les Chaldéens reviendront attaquer cette ville, ils la prendront et l’incendieront. 
${}^{9}Ainsi parle le Seigneur : Ne vous abusez pas vous-mêmes en disant : “Les Chaldéens vont s’en aller de chez nous, pour de bon !” Car ils ne s’en iront pas. 
${}^{10}Et même si vous battiez toute l’armée des Chaldéens en guerre contre vous, au point qu’il n’en reste plus que des blessés graves, chacun d’eux se redresserait dans sa tente et viendrait incendier cette ville. »
       
${}^{11}Or, tandis que l’armée des Chaldéens levait le siège de Jérusalem à cause de l’armée de Pharaon, 
${}^{12}Jérémie sortait de Jérusalem pour aller au pays de Benjamin toucher sa part d’héritage au milieu des siens. 
${}^{13}Comme il était à la porte de Benjamin, il y rencontra un certain Yiriya, fils de Shèlèmya, fils de Hananya, chef du poste de garde, qui se saisit du prophète Jérémie en lui disant : « Voilà que tu te rends chez les Chaldéens ! » 
${}^{14}Jérémie lui répondit : « C’est faux ! Je ne me rends pas aux Chaldéens ! » Mais Yiriya ne l’écouta pas, il se saisit de Jérémie et le conduisit auprès des princes. 
${}^{15}Les princes s’emportèrent contre Jérémie, le frappèrent et le mirent en détention dans la maison du scribe Jonathan, transformée en prison. 
${}^{16}Ainsi Jérémie se retrouva-t-il dans un cachot, sous les voûtes. Et Jérémie demeura là, de nombreux jours.
${}^{17}C’est alors que le roi Sédécias l’envoya chercher ; il le questionna en cachette dans sa maison et lui demanda : « Y a-t-il une parole venant du Seigneur ? » Jérémie lui répondit : « Il y en a une ! » Et il lui dit : « Tu seras livré aux mains du roi de Babylone. » 
${}^{18}Puis Jérémie dit au roi Sédécias : « En quoi ai-je péché contre toi, contre tes serviteurs et contre ce peuple, pour que vous m’ayez mis en prison ? 
${}^{19}Où sont-ils vos prophètes qui prophétisaient : “Le roi de Babylone ne viendra pas contre vous ni contre ce pays” ? 
${}^{20}Mais maintenant, écoute, je t’en prie, monseigneur le roi ! Laisse-toi toucher par ma supplication : ne me fais pas retourner à la maison du scribe Jonathan ; que je n’y meure pas ! » 
${}^{21}Alors le roi Sédécias ordonna que Jérémie soit consigné dans la cour de garde et qu’on lui donne chaque jour une couronne de pain, de la rue des Boulangers, jusqu’à ce que tout le pain de la ville soit épuisé. Et Jérémie demeura dans la cour de garde.
      
         
      \bchapter{}
      \begin{verse}
${}^{1}Shefatya, fils de Mattane, Guedalyahou, fils de Pashehour, Youkal, fils de Shèlèmyahou, et Pashehour, fils de Malkiya, entendirent les paroles que Jérémie adressait à tout le peuple : 
${}^{2}« Ainsi parle le Seigneur : Qui restera dans cette ville mourra par l’épée, la famine ou la peste. Mais qui en sortira pour se rendre aux Chaldéens, celui-là vivra : il aura la vie sauve, comme part de butin ; il vivra. 
${}^{3}Ainsi parle le Seigneur : Cette ville sera bel et bien livrée aux mains de l’armée du roi de Babylone, qui la prendra. »
${}^{4}Alors les princes dirent au roi Sédécias\\ : « Que cet homme soit mis à mort : en parlant comme il le fait, il démoralise\\tout ce qui reste de combattant dans la ville, et toute la population. Ce n’est pas le bonheur du peuple qu’il cherche, mais son malheur. » 
${}^{5}Le roi Sédécias répondit : « Il est entre vos mains, et le roi ne peut rien contre vous ! » 
${}^{6}Alors ils se saisirent de Jérémie et le jetèrent dans la citerne de Melkias, fils du roi, dans la cour de garde. On le descendit avec des cordes. Dans cette citerne il n’y avait pas d’eau, mais de la boue, et Jérémie enfonça dans la boue.
       
${}^{7}Ébed-Mélek l’Éthiopien, dignitaire de la maison du roi, apprit qu’on avait mis Jérémie dans la citerne. Comme le roi siégeait à la porte de Benjamin, 
${}^{8}Ébed-Mélek sortit de la maison du roi et vint lui dire : 
${}^{9}« Monseigneur le roi, ce que ces gens-là ont fait au prophète Jérémie, c’est mal ! Ils l’ont jeté dans la citerne, il va y mourir de faim car on n’a plus de pain dans la ville ! » 
${}^{10}Alors le roi donna cet ordre à Ébed-Mélek l’Éthiopien : « Prends trente hommes avec toi, et fais remonter de la citerne le prophète Jérémie avant qu’il ne meure. » 
${}^{11}Ébed-Mélek prit les hommes avec lui et entra dans la maison du roi, au sous-sol de la réserve. Il s’y procura de vieux chiffons usés et déchirés qu’il fit passer à Jérémie, dans la citerne, au moyen de cordes. 
${}^{12}Ébed-Mélek l’Éthiopien dit à Jérémie : « Mets donc ces vieux chiffons sous tes aisselles, par-dessous les cordes ! » C’est ce que fit Jérémie. 
${}^{13}Alors, ils tirèrent Jérémie avec les cordes et le firent remonter de la citerne. Et Jérémie demeura dans la cour de garde.
${}^{14}Le roi Sédécias envoya chercher le prophète Jérémie et le fit venir auprès de lui, à la troisième entrée de la maison du Seigneur. Le roi dit à Jérémie : « J’ai à te demander une parole ; ne me la cache pas ! » 
${}^{15}Jérémie répondit à Sédécias : « Si je te révèle quelque chose, ne me feras-tu pas mourir ? Et si je te donne un conseil, tu ne m’écouteras pas ! » 
${}^{16}Alors le roi Sédécias fit en secret ce serment à Jérémie : « Par le Seigneur vivant qui nous a donné cette vie, je ne te ferai pas mourir, et je ne te livrerai pas aux mains de ces gens qui en veulent à ta vie. » 
${}^{17}Jérémie dit à Sédécias : « Ainsi parle le Seigneur, Dieu de l’univers, Dieu d’Israël : Si tu sors pour te rendre aux officiers du roi de Babylone, tu auras la vie sauve, et cette ville ne sera pas incendiée. Vous vivrez, toi et ceux de ta maison. 
${}^{18}Mais si tu ne sors pas pour te rendre aux officiers du roi de Babylone, cette ville sera livrée aux mains des Chaldéens qui l’incendieront ; et toi, tu n’échapperas pas à leurs mains. » 
${}^{19}Le roi Sédécias dit à Jérémie : « Moi, je redoute les Judéens qui sont passés aux Chaldéens. Ils pourraient me livrer à eux et se jouer de moi. » 
${}^{20}Jérémie répondit : « Ils ne te livreront pas. Écoute donc la voix du Seigneur qui s’adresse à toi quand je te parle, et tout ira bien pour toi, tu auras la vie sauve. 
${}^{21}Mais si tu refuses de te rendre, voici ce que le Seigneur m’a montré : 
${}^{22}Toutes les femmes qui sont restées dans la maison du roi de Juda, on les conduit aux officiers du roi de Babylone. Et elles te disent :
        \\“Ils t’ont séduit, ils ont gagné,
        tes bons amis !
        \\Ton pied enfonce dans le bourbier,
        eux sont partis !”
${}^{23}Toutes tes femmes et tes fils, on les conduit aux Chaldéens, et toi, tu n’échappes pas à leurs mains : tu es saisi par les mains du roi de Babylone, et cette ville est incendiée. » 
${}^{24}Sédécias dit à Jérémie : « Que personne ne sache rien de ces paroles, sinon tu en mourrais. 
${}^{25}Mais si les princes apprennent que j’ai parlé avec toi, s’ils viennent te trouver en disant : “Rapporte-nous les paroles échangées entre toi et le roi, ne nous cache rien, sinon nous te ferons mourir !”, 
${}^{26}tu leur diras : “J’ai adressé au roi cette supplication : Ne me fais pas retourner chez Jonathan pour y mourir !” » 
${}^{27}De fait, tous les princes vinrent trouver Jérémie pour le questionner. Et il leur rapporta l’entretien dans les termes mêmes que le roi lui avait indiqués. Alors, sans insister ils le laissèrent, et la parole de l’oracle ne fut pas divulguée. 
${}^{28}Et Jérémie demeura dans la cour de garde jusqu’au jour où Jérusalem fut prise ; il y était encore quand Jérusalem fut prise.
      
         
      \bchapter{}
      \begin{verse}
${}^{1}La neuvième année du règne de Sédécias, roi de Juda, au dixième mois, Nabucodonosor, roi de Babylone, arriva devant Jérusalem avec toute son armée, et il en fit le siège. 
${}^{2}La onzième année de Sédécias, le neuvième jour du quatrième mois, une brèche fut ouverte dans le rempart de la ville. 
${}^{3}Alors entrèrent tous les officiers du roi de Babylone et ils siégèrent à la porte du Milieu ; il y avait Nergal-Sarécer, Samgar-Nebou, Sarsekim, grand dignitaire, Nergal-Sarécer, grand mage, et tout le reste des officiers du roi de Babylone.
${}^{4}En les voyant, Sédécias, roi de Juda, et tous les hommes de guerre prirent la fuite ; ils sortirent de la ville, la nuit, vers le jardin du roi, par la porte du double rempart ; et ils s’éloignèrent en direction de la Araba. 
${}^{5}Mais l’armée des Chaldéens les poursuivit et rattrapa Sédécias dans la plaine de Jéricho. Il fut capturé et mené à Ribla, au pays de Hamath, auprès de Nabucodonosor, roi de Babylone, qui prononça la sentence contre lui. 
${}^{6}Le roi de Babylone égorgea les fils de Sédécias, sous les yeux de leur père, à Ribla. Le roi de Babylone égorgea tous les notables de Juda. 
${}^{7}Puis il creva les yeux de Sédécias et le fit attacher avec une double chaîne de bronze pour l’emmener à Babylone. 
${}^{8}Les Chaldéens incendièrent la maison du roi et les maisons du peuple, et ils abattirent les remparts de Jérusalem. 
${}^{9}Ensuite, Nabouzardane, commandant de la garde, déporta à Babylone ceux du peuple qui restaient encore dans la ville, ainsi que les déserteurs qui s’étaient rendus à lui, bref, le reste du peuple. 
${}^{10}Nabouzardane, commandant de la garde, laissa au pays de Juda ceux qui étaient pauvres, qui ne possédaient rien, et il leur donna, ce jour-là, des vignes et des cultures.
${}^{11}Au sujet de Jérémie, Nabucodonosor, roi de Babylone, avait donné ses ordres à Nabouzardane, commandant de la garde, en lui disant : 
${}^{12}« Prends-le et veille sur lui, ne lui fais aucun mal, mais traite-le comme il te le demandera. » 
${}^{13}Nabouzardane, commandant de la garde, Naboushazbane, grand dignitaire, et Nergal-Sarécer, grand mage, ainsi que tous les grands officiers du roi de Babylone étaient chargés de cette mission. 
${}^{14}Ils envoyèrent prendre Jérémie dans la cour de garde et le confièrent à Godolias, fils d’Ahiqam, fils de Shafane, pour le conduire chez lui. Et il demeura au milieu du peuple.
${}^{15}La parole du Seigneur fut adressée à Jérémie, alors qu’il était détenu dans la cour de garde : 
${}^{16}« Va dire à Ébed-Mélek l’Éthiopien : “Ainsi parle le Seigneur de l’univers, le Dieu d’Israël : Je vais faire venir sur cette ville mes paroles de malheur et non de bonheur. Ce jour-là, elles se réaliseront devant toi. 
${}^{17}Mais, ce jour-là, je te délivrerai – oracle du Seigneur –, et tu ne seras pas livré aux mains des hommes qui t’épouvantent. 
${}^{18}Oui, je te ferai échapper, à coup sûr, et tu ne tomberas pas sous l’épée ; tu auras la vie sauve, comme part de butin, puisque tu as mis ta confiance en moi – oracle du Seigneur.” »
      
         
      \bchapter{}
      \begin{verse}
${}^{1}Parole du Seigneur adressée à Jérémie, après que Nabouzardane, commandant de la garde, l’eut renvoyé de Rama. C’est là qu’il l’avait pris, quand il était enchaîné au milieu de tous les déportés de Jérusalem et de Juda sur le point d’être exilés à Babylone.
${}^{2}Le commandant de la garde prit donc Jérémie et lui dit : « C’est le Seigneur ton Dieu qui avait annoncé un tel malheur contre ce lieu 
${}^{3}et qui l’a fait venir. Le Seigneur a réalisé ce qu’il avait annoncé. Cela vous est arrivé parce que vous avez péché contre le Seigneur et n’avez pas écouté sa voix. 
${}^{4}Mais maintenant, je délie en ce jour les chaînes de tes mains. S’il est bon à tes yeux de venir avec moi à Babylone, viens ! Je veillerai sur toi. Mais s’il est mauvais à tes yeux de venir avec moi, ne viens pas ! Regarde : toute la terre est devant toi ; là où tu jugeras bon d’aller, va ! » 
${}^{5}Et comme Jérémie tardait à s’en retourner, il ajouta : « Retourne auprès de Godolias, fils d’Ahiqam, fils de Shafane, que le roi de Babylone a établi gouverneur des villes de Juda, et demeure avec lui au milieu du peuple, ou bien va partout où tu jugeras bon d’aller. » Puis le commandant de la garde lui remit des vivres, avec un cadeau, et il le renvoya. 
${}^{6}Jérémie se rendit auprès de Godolias, fils d’Ahiqam, à Mispa, et demeura avec lui, au milieu du peuple, parmi ceux qui étaient restés dans le pays.
${}^{7}Tous les officiers de l’armée, qui étaient dans la campagne, apprirent, eux et leurs hommes, que le roi de Babylone avait établi gouverneur du pays Godolias, fils d’Ahiqam, et qu’il lui avait confié hommes, femmes et enfants, le petit peuple qui n’était pas déporté à Babylone. 
${}^{8}Alors, ils se rendirent, eux et leurs hommes, auprès de Godolias à Mispa. C’étaient Ismaël, fils de Netanyahou, Yohanane et Jonathan, fils de Qaréah, Seraya, fils de Tanhoumeth, les fils d’Éfaï, de Netofa, et Jézanyahou, fils du Maakatite. 
${}^{9}Godolias, fils d’Ahiqam, fils de Shafane, leur fit un serment, à eux et à leurs hommes. Il leur dit : « N’ayez pas peur de servir les Chaldéens, demeurez dans le pays, servez le roi de Babylone, et tout ira bien pour vous. 
${}^{10}Quant à moi, je vais demeurer à Mispa, pour me tenir à la disposition des Chaldéens qui viennent chez nous. Mais vous, faites la récolte du vin, des fruits et de l’huile, emplissez vos jarres et demeurez dans les villes que vous occupez. »
${}^{11}Tous les Judéens qui se trouvaient en Moab, chez les fils d’Ammone, en Édom et dans tout autre pays, apprirent aussi que le roi de Babylone avait laissé subsister un reste en Juda et qu’il avait établi gouverneur Godolias, fils d’Ahiqam, fils de Shafane. 
${}^{12}Tous les Judéens revinrent donc de tous les lieux où ils avaient été chassés. Ils se rendirent au pays de Juda, auprès de Godolias à Mispa. Et ils firent une très abondante récolte de vin et de fruits.
${}^{13}Yohanane, fils de Qaréah, et tous les officiers de l’armée qui étaient dans la campagne se rendirent auprès de Godolias à Mispa. 
${}^{14}Ils lui dirent : « Ne sais-tu pas que Baalis, roi des fils d’Ammone, a envoyé Ismaël, fils de Netanya, pour attenter à ta vie ? » Mais Godolias, fils d’Ahiqam, refusa de les croire. 
${}^{15}Alors Yohanane, fils de Qaréah, dit en secret à Godolias à Mispa : « J’irai abattre Ismaël, fils de Netanya, sans que personne le sache ! Pourquoi le laisser attenter à ta vie ? Pourquoi tous les Judéens rassemblés autour de toi devraient-ils se disperser, et le reste de Juda périr ? » 
${}^{16}Mais Godolias, fils d’Ahiqam, répondit à Yohanane, fils de Qaréah : « Ne fais pas cela ! Car c’est faux, ce que tu dis au sujet d’Ismaël. »
       
      
         
      \bchapter{}
      \begin{verse}
${}^{1}Mais le septième mois, Ismaël, fils de Netanya, fils d’Élishama, qui était de sang royal, se rendit, avec les grands officiers du roi et dix hommes, auprès de Godolias, fils d’Ahiqam, à Mispa. Et là, à Mispa, ils prirent un repas ensemble. 
${}^{2}Alors Ismaël, fils de Netanya, se leva avec les dix hommes qui l’accompagnaient, et ils frappèrent de l’épée Godolias, fils d’Ahiqam, fils de Shafane. Ils firent mourir celui que le roi de Babylone avait établi gouverneur du pays. 
${}^{3}Et Ismaël frappa à mort tous les Judéens qui étaient avec Godolias à Mispa, et les Chaldéens qui se trouvaient là – des hommes de guerre.
${}^{4}Le deuxième jour après le meurtre de Godolias, alors que personne encore ne savait rien, 
${}^{5}arrivèrent des hommes venant de Sichem, de Silo et de Samarie. Au nombre de quatre-vingts, la barbe rasée, les vêtements déchirés, le corps marqué d’incisions, ils portaient des offrandes et de l’encens pour la maison du Seigneur. 
${}^{6}Ismaël, fils de Netanya, sortit de Mispa à leur rencontre. Il avançait en pleurant et leur dit au moment où il les rejoignait : « Venez chez Godolias, fils d’Ahiqam ! » 
${}^{7}Mais quand ils arrivèrent au milieu de la ville, Ismaël, fils de Netanya, les égorgea, aidé de ses hommes, et les jeta au fond de la citerne.
${}^{8}Il se trouva cependant, parmi eux, dix hommes qui dirent à Ismaël : « Ne nous fais pas mourir ! Nous avons dans les champs des provisions cachées : du blé, de l’orge, de l’huile et du miel. » Alors, Ismaël renonça à les faire mourir avec leurs frères. 
${}^{9}La citerne, où Ismaël avait jeté tous les cadavres des hommes qu’il avait frappés à mort, était la grande citerne faite par le roi Asa quand il luttait contre Basha, roi d’Israël. Celle-là, Ismaël, fils de Netanyahou, la remplit de ses victimes.
${}^{10}Puis Ismaël emmena captif tout le reste du peuple qui était à Mispa : les filles du roi avec tous ceux du peuple qui restaient à Mispa et que Nabouzardane, commandant de la garde, avait confiés à Godolias fils d’Ahiqam. Ismaël fils de Netanya, les emmenant captifs, se mit en route pour passer chez les fils d’Ammone. 
${}^{11}Lorsque Yohanane, fils de Qaréah, et tous les officiers de l’armée qui se trouvaient avec lui apprirent tout le mal qu’avait fait Ismaël fils de Netanya, 
${}^{12}ils rassemblèrent tous leurs hommes et partirent en guerre contre Ismaël fils de Netanya. Ils le trouvèrent aux Grandes-Eaux de Gabaon. 
${}^{13}À la vue de Yohanane, fils de Qaréah, et de tous les officiers qui l’accompagnaient, tous ceux du peuple qui étaient avec Ismaël furent saisis de joie. 
${}^{14}Faisant volte-face, tous ceux qu’Ismaël avait emmenés captifs depuis Mispa se retournèrent et s’en allèrent vers Yohanane fils de Qaréah. 
${}^{15}Quant à Ismaël fils de Netanya, il échappa à Yohanane avec huit de ses hommes et s’en alla chez les fils d’Ammone.
${}^{16}Alors Yohanane, fils de Qaréah, et tous les officiers de l’armée qui l’accompagnaient, rassemblèrent tout le reste du peuple ramené de Mispa d’auprès d’Ismaël fils de Netanya, après que celui-ci eut frappé à mort Godolias fils d’Ahiqam ; ils rassemblèrent hommes de guerre, femmes, enfants et dignitaires, ramenés de Gabaon. 
${}^{17}Ils se mirent en route et s’arrêtèrent au campement de Kimham, à côté de Bethléem, pour gagner ensuite l’Égypte. 
${}^{18}En effet, ils avaient peur des Chaldéens, depuis qu’Ismaël, fils de Netanya, avait frappé à mort Godolias fils d’Ahiqam, établi gouverneur du pays par le roi de Babylone.
      
         
      \bchapter{}
      \begin{verse}
${}^{1}Alors tous les officiers de l’armée, notamment Yohanane, fils de Qaréah, et Azarya, fils de Hoshaya, avec tous ceux du peuple, des plus petits jusqu’aux plus grands, s’avancèrent 
${}^{2}et dirent au prophète Jérémie : « Laisse-toi toucher par notre supplication ! Intercède pour nous auprès du Seigneur ton Dieu, pour tout ce reste, car nous restons peu du grand nombre que nous étions, comme tu le vois de tes yeux ! 
${}^{3}Que le Seigneur ton Dieu nous indique la route que nous devons suivre et ce que nous devons faire ! » 
${}^{4}Le prophète Jérémie leur dit : « C’est entendu ! Selon votre demande, je vais intercéder auprès du Seigneur votre Dieu. Toute parole que le Seigneur vous donnera en réponse, je vous l’annoncerai, sans rien vous cacher. » 
${}^{5}De leur côté, ils dirent à Jérémie : « Que le Seigneur soit contre nous un témoin fidèle et sûr, si nous n’agissons pas conformément à toute parole que le Seigneur ton Dieu t’adressera pour nous ! 
${}^{6}Que cela semble bon ou mauvais, nous écouterons la voix du Seigneur notre Dieu auprès de qui nous t’envoyons. Et ainsi, tout ira bien pour nous, puisque nous écouterons la voix du Seigneur notre Dieu. »
      
         
       
${}^{7}Au bout de dix jours, la parole du Seigneur fut adressée à Jérémie. 
${}^{8}Alors il convoqua Yohanane, fils de Qaréah, tous les officiers de l’armée qui étaient avec lui, et tout le peuple, des plus petits jusqu’aux plus grands. 
${}^{9}Il leur dit : « Ainsi parle le Seigneur, le Dieu d’Israël, auprès de qui vous m’avez envoyé pour qu’il se laisse toucher par votre supplication : 
${}^{10}Si vous revenez habiter dans ce pays, je vous bâtirai, je ne démolirai pas, je vous planterai, je n’arracherai pas, car je me repens du mal que je vous ai fait. 
${}^{11}N’ayez pas peur du roi de Babylone, devant qui vous avez peur ! N’ayez pas peur de lui – oracle du Seigneur – car je suis avec vous pour vous sauver et vous délivrer de sa main. 
${}^{12}Je vous ferai prendre en compassion, et il aura de la compassion pour vous : il vous laissera revenir sur votre sol. 
${}^{13}Mais si vous dites : “Nous n’habiterons plus dans ce pays”, refusant ainsi d’écouter la voix du Seigneur votre Dieu, 
${}^{14}si vous dites : “Non ! C’est au pays d’Égypte que nous irons pour ne plus voir la guerre, ni entendre l’appel du cor, ni souffrir la faim ; c’est là-bas que nous habiterons”, 
${}^{15}alors, reste de Juda, écoutez maintenant la parole du Seigneur ! Ainsi parle le Seigneur de l’univers, le Dieu d’Israël : Si vous persistez à tourner votre visage vers l’Égypte pour y aller, si vous allez séjourner là-bas, 
${}^{16}l’épée dont vous avez peur, c’est là-bas qu’elle vous atteindra, au pays d’Égypte ; la famine qui fait votre inquiétude, c’est là-bas qu’elle vous poursuivra, en Égypte, et c’est là-bas que vous mourrez. 
${}^{17}Et tous les hommes qui auront tourné leur visage vers l’Égypte pour aller séjourner là-bas, ils mourront par l’épée, la famine et la peste. Il n’y aura ni survivant ni rescapé face au malheur que je ferai venir contre eux. 
${}^{18}Oui, ainsi parle le Seigneur de l’univers, le Dieu d’Israël : De même que mon ardente colère s’est déversée sur les habitants de Jérusalem, de même se déversera sur vous ma colère lorsque vous irez en Égypte. Alors vous deviendrez un objet d’imprécation, de désolation, de malédiction et d’insulte, et vous ne reverrez plus ce lieu.
${}^{19}Reste de Juda, le Seigneur vous a parlé. N’allez pas en Égypte ! Sachez-le bien : aujourd’hui, je suis témoin contre vous. 
${}^{20}Vous avez perdu la tête quand vous m’avez envoyé auprès du Seigneur votre Dieu, en me disant : “Intercède pour nous auprès du Seigneur notre Dieu ! Tout ce que dira le Seigneur notre Dieu, tu nous l’annonceras, et nous le ferons.” 
${}^{21}Or, aujourd’hui, je vous l’annonce, mais vous n’écoutez pas la voix du Seigneur votre Dieu, ni rien de ce qu’il m’envoie vous dire ! 
${}^{22}Maintenant, sachez-le bien : vous mourrez par l’épée, la famine et la peste dans le lieu où vous désirez aller pour y séjourner. »
      
         
      \bchapter{}
      \begin{verse}
${}^{1}Quand Jérémie eut achevé de prononcer devant tout le peuple toutes les paroles du Seigneur leur Dieu, ces paroles pour lesquelles le Seigneur leur Dieu l’avait envoyé, 
${}^{2}alors Azarya fils de Hoshaya, Yohanane fils de Qaréah, et tous ces gens pleins d’arrogance répondirent à Jérémie : « C’est un mensonge que tu dis ! Le Seigneur notre Dieu ne t’a pas envoyé nous dire : “N’allez pas en Égypte pour y séjourner !”, 
${}^{3}mais c’est Baruc, fils de Nériya, qui t’excite contre nous, afin de nous livrer aux mains des Chaldéens qui vont nous faire mourir ou nous déporter à Babylone. »
${}^{4}Ainsi, ni Yohanane fils de Qaréah, ni aucun des officiers de l’armée, ni personne dans le peuple n’écouta la voix du Seigneur, en demeurant au pays de Juda. 
${}^{5}Yohanane fils de Qaréah et tous les officiers de l’armée emmenèrent tout le reste de Juda, ceux qui étaient revenus séjourner au pays de Juda, depuis toutes les nations où ils avaient été chassés : 
${}^{6}hommes, femmes et enfants, ainsi que les filles du roi et tous ceux que Nabouzardane, commandant de la garde, avait confiés à Godolias, fils d’Ahiqam, fils de Shafane, notamment le prophète Jérémie et Baruc, fils de Nériya. 
${}^{7}Refusant d’écouter la voix du Seigneur, ils allèrent donc au pays d’Égypte et arrivèrent à Tapanès.
${}^{8}Or, à Tapanès, la parole du Seigneur fut adressée à Jérémie : 
${}^{9}« De tes mains, prends de grosses pierres et, aux yeux des Judéens, enfouis-les dans l’argile, près du four à briques, à l’entrée de la maison de Pharaon, à Tapanès. 
${}^{10}Tu leur diras : Ainsi parle le Seigneur de l’univers, le Dieu d’Israël : Voici que j’envoie chercher mon serviteur Nabucodonosor, roi de Babylone ; je placerai son trône sur les pierres que j’ai enfouies, et il étendra son baldaquin au-dessus d’elles. 
${}^{11}Il viendra et frappera le pays d’Égypte.
        \\Qui est pour la mort, qu’il aille à la mort !
        \\Qui est pour la captivité, à la captivité !
        \\Qui est pour l’épée, qu’il aille à l’épée !
${}^{12}Je mettrai le feu aux maisons des dieux de l’Égypte. Nabucodonosor les brûlera, il emmènera les dieux en captivité ; comme un berger s’enveloppe de son manteau, il s’enveloppera du pays d’Égypte, et de là repartira en paix. 
${}^{13}Il brisera les stèles de Beth-Shèmesh au pays d’Égypte et il incendiera les maisons des dieux de l’Égypte. »
      
         
      \bchapter{}
      \begin{verse}
${}^{1}Parole qui fut adressée à Jérémie pour tous les Judéens habitant au pays d’Égypte, ceux qui habitent à Migdol, Tapanès, Noph et au pays de Patros. 
${}^{2}« Ainsi parle le Seigneur de l’univers, le Dieu d’Israël : Vous avez vu tout le malheur que j’ai fait venir sur Jérusalem et sur toutes les villes de Juda ; aujourd’hui, les voici en ruine, et sans habitants. 
${}^{3}C’est à cause du mal qu’ils ont fait pour m’offenser, en allant encenser et servir d’autres dieux, qui leur étaient inconnus, comme à vous et à vos pères. 
${}^{4}Je vous ai envoyé inlassablement tous mes serviteurs les prophètes, pour dire : “Ne faites donc pas cette chose abominable que je déteste !” 
${}^{5}Mais ils n’ont pas écouté ni prêté l’oreille, pour se détourner de leur mal et cesser de brûler de l’encens à d’autres dieux. 
${}^{6}Alors mon ardente colère s’est déversée, elle a embrasé les villes de Juda et les rues de Jérusalem, les réduisant en ruine et en désolation, comme on le voit aujourd’hui. 
${}^{7}Et maintenant, ainsi parle le Seigneur, Dieu de l’univers, le Dieu d’Israël : Pourquoi vous faire à vous-mêmes ce grand mal, retrancher du milieu de Juda homme et femme, petit enfant et nourrisson, et ne pas laisser subsister un reste ? 
${}^{8}Pourquoi m’offenser par les œuvres de vos mains, en brûlant de l’encens à d’autres dieux, dans le pays d’Égypte, là où vous êtes venus séjourner ? Pourquoi être retranchés de toutes les nations de la terre et devenir pour elles un objet de malédiction et d’insulte ? 
${}^{9}Avez-vous oublié les méfaits de vos pères, les méfaits des rois de Juda et ceux de leurs femmes, vos propres méfaits et ceux de vos femmes, commis au pays de Juda et dans les rues de Jérusalem ? 
${}^{10}Jusqu’à ce jour, ils n’ont éprouvé ni contrition ni crainte ; ils n’ont pas marché selon ma loi et mes décrets que j’avais placés devant vous et devant vos pères. 
${}^{11}C’est pourquoi, ainsi parle le Seigneur de l’univers, le Dieu d’Israël : Je vais tourner mon visage contre vous, pour votre malheur, et retrancher tout Juda. 
${}^{12}Je prendrai le reste de Juda, ceux qui ont tourné leur visage vers l’Égypte pour aller y séjourner, et ils périront tous, ils tomberont au pays d’Égypte, ils périront par l’épée et la famine, du plus petit jusqu’au plus grand. Par l’épée et la famine ils mourront et ils deviendront un objet d’imprécation et de désolation, de malédiction et d’insulte. 
${}^{13}Je châtierai ceux qui habitent au pays d’Égypte, comme j’ai châtié Jérusalem, par l’épée, la famine et la peste. 
${}^{14}Il n’y aura ni rescapé ni survivant parmi le reste de Juda, ceux qui sont venus séjourner au pays d’Égypte. Quant au pays de Juda où ils désirent ardemment revenir pour y habiter, ils n’y reviendront pas, sauf quelques rescapés. »
      
         
       
${}^{15}Tous les hommes qui savaient que leurs femmes brûlaient de l’encens à d’autres dieux et toutes les femmes présentes – donc une grande assemblée –, c’est-à-dire le peuple habitant au pays d’Égypte à Patros, tous firent à Jérémie cette réponse : 
${}^{16}« Pour ce qui est de la parole que tu nous as dite au nom du Seigneur, nous ne t’écouterons pas. 
${}^{17}Mais nous voulons agir selon toute la parole qui est sortie de notre bouche : brûler de l’encens à la Reine du ciel et lui verser des libations comme nous le faisions dans les villes de Juda et les rues de Jérusalem, nous et nos pères, nos rois et nos princes ; car alors, rassasiés de pain, nous étions heureux et ne connaissions pas le malheur. 
${}^{18}Depuis que nous avons cessé de brûler de l’encens à la Reine du ciel et de lui verser des libations, nous manquons de tout et nous périssons par l’épée et la famine. » 
${}^{19}Les femmes ajoutèrent : « Lorsque nous brûlons de l’encens à la Reine du ciel et lui versons des libations, est-ce à l’insu de nos maris que nous lui faisons des gâteaux qui la représentent et lui versons des libations ? »
${}^{20}Alors Jérémie s’adressa à tout le peuple, aux hommes et aux femmes, à tous ceux qui lui avaient ainsi répondu. Il leur dit : 
${}^{21}« Cet encens que vous avez brûlé dans les villes de Juda et dans les rues de Jérusalem, vous et vos pères, vos rois et vos princes, avec les gens du pays, est-ce que le Seigneur ne s’en est pas souvenu ? N’en a-t-il pas gardé mémoire ? 
${}^{22}Le Seigneur ne pouvait plus supporter la malice de vos actes et les abominations que vous avez commises. Alors votre pays est devenu une ruine, un objet de désolation et de malédiction, sans aucun habitant, comme on le voit aujourd’hui. 
${}^{23}Vous avez brûlé de l’encens, vous avez péché contre le Seigneur, vous n’avez pas écouté la voix du Seigneur, vous n’avez pas marché selon sa loi, ses décrets et ses exigences : voilà pourquoi ce malheur vous est arrivé, comme on le voit aujourd’hui. »
${}^{24}Puis Jérémie dit à tout le peuple, hommes et femmes : « Écoutez la parole du Seigneur, vous tous de Juda, vous qui êtes au pays d’Égypte ! 
${}^{25}Ainsi parle le Seigneur de l’univers, le Dieu d’Israël : Vous et vos femmes, vous avez dit de votre bouche et vous avez accompli de vos mains, vous qui déclariez : “Nous réaliserons les vœux que nous avons faits de brûler de l’encens à la Reine du ciel et de lui verser des libations.” Eh bien, acquittez-vous donc de vos vœux ! Réalisez-les complètement ! 
${}^{26}Cependant, écoutez la parole du Seigneur, vous tous de Juda, vous qui habitez au pays d’Égypte : Voici que j’en ai fait le serment par mon grand nom, dit le Seigneur. Plus personne en Juda n’invoquera mon nom, aucun de ceux qui disent, par tout le pays d’Égypte : “Vive le Seigneur mon Dieu !” 
${}^{27}Voici que je veille sur eux pour le malheur et non pour le bonheur ; ils périront, tous les hommes de Juda qui sont au pays d’Égypte, par l’épée et la famine, jusqu’à leur anéantissement. 
${}^{28}Mais quelques-uns, peu nombreux, échappant à l’épée, s’en retourneront du pays d’Égypte au pays de Juda. Alors, tout le reste de Juda, ceux qui sont venus séjourner au pays d’Égypte, ceux-là sauront quelle parole se réalise : la mienne ou la leur ! 
${}^{29}Et voici pour vous – oracle du Seigneur – le signe que je vais vous châtier en ce lieu, afin que vous sachiez que mes paroles de malheur contre vous se réalisent vraiment : 
${}^{30}– ainsi parle le Seigneur – je vais livrer le pharaon Hophra, roi d’Égypte, aux mains de ses ennemis, aux mains de ceux qui en veulent à sa vie, comme j’ai livré Sédécias, roi de Juda, aux mains de Nabucodonosor, roi de Babylone, son ennemi, celui qui en voulait à sa vie. »
      
         
      \bchapter{}
      \begin{verse}
${}^{1}Parole que le prophète Jérémie adressa à Baruc fils de Nériya, quand celui-ci écrivait sous la dictée de Jérémie ces paroles dans un livre, la quatrième année du règne de Joakim, fils de Josias, roi de Juda. 
${}^{2}« Ainsi parle le Seigneur, le Dieu d’Israël, à ton sujet, Baruc, 
${}^{3}toi qui as dit : “Malheur à moi ! Le Seigneur ajoute le tourment à ma douleur ; je m’épuise à force de gémir, sans trouver de repos.” 
${}^{4}Voici ce que tu diras à Baruc : Ainsi parle le Seigneur : Ce que j’ai bâti, je vais le démolir ; ce que j’ai planté, je vais l’arracher, c’est-à-dire tout ce pays. 
${}^{5}Et toi, tu demandes pour toi de grandes choses ! Ne demande rien, car je vais faire venir le malheur sur tout être de chair – oracle du Seigneur –, mais en tous lieux où tu iras, comme part de butin, je t’accorderai la vie sauve. »
      
         
