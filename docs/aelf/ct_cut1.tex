  
  
    
    \bbook{CANTIQUE DES CANTIQUES}{CANTIQUE DES CANTIQUES}
      
         
      \bchapter{}
${}^{1}Le Cantique des Cantiques.
        \\De Salomon.
        
           
      <div class="intertitle niv7">
        ELLE
${}^{2}Qu’il me donne
        les baisers de sa bouche :
        \\meilleures que le vin
        sont tes amours !
         
${}^{3}Délice,
        l’odeur de tes parfums ;
        \\ton nom,
        un parfum qui s’épanche :
        \\ainsi t’aiment les jeunes filles !
         
${}^{4}Entraîne-moi : à ta suite,
        courons !
        \\Le roi m’a fait entrer
        en ses demeures.
       
      <div class="intertitle niv7">
        CHŒUR
        \\En toi, notre fête et notre joie !
        \\Nous redirons tes amours,
        meilleures que le vin :
        \\il est juste de t’aimer !
      <div class="intertitle niv7">
        ELLE
${}^{5}Noire, je le suis, mais belle,
        filles de Jérusalem,
        \\pareille aux tentes de Qédar,
        aux tissus de Salma.
         
${}^{6}Ne regardez pas à ma peau noire :
        c’est le soleil qui m’a brunie.
        \\Les fils de ma mère se sont fâchés contre moi :
        ils m’ont mise à garder les vignes.
         
        \\Ma vigne, la mienne,
        je ne l’ai pas gardée…
         
${}^{7}Raconte-moi,
        bien-aimé de mon âme,
        \\où tu mènes paître tes brebis,
        où tu les couches aux heures de midi,
        \\que je n’aille plus m’égarer
        vers les troupeaux de tes compagnons.
       
      <div class="intertitle niv7">
        CHŒUR
${}^{8}Si tu ne le sais pas,
        ô belle entre les femmes,
        \\va dehors sur les traces du troupeau
        et mène paître tes jeunes chèvres
        vers les tentes des bergers.
       
      <div class="intertitle niv7">
        LUI
${}^{9}Cavale attelée aux chars de Pharaon,
        ainsi tu m’apparais, ô mon amie !
${}^{10}Quel charme, tes joues entre tes boucles,
        ton cou entre les perles !
${}^{11}Nous te ferons des boucles d’or,
        incrustées d’argent.
       
      <div class="intertitle niv7">
        ELLE
${}^{12}Quand le roi est dans ses enclos,
        mon parfum répand sa bonne odeur.
         
${}^{13}Mon bien-aimé, pour moi,
        est un sachet de myrrhe :
        \\entre mes seins,
        il passera la nuit.
         
${}^{14}Mon bien-aimé, pour moi,
        est un rameau de cypre
        \\parmi les vignes
        d’Enn-Guèdi.
       
      <div class="intertitle niv7">
        LUI
${}^{15}Ah ! Que tu es belle,
        mon amie !
        \\Ah ! Que tu es belle :
        tes yeux sont des colombes !
       
      <div class="intertitle niv7">
        ELLE
${}^{16}Ah ! Que tu es beau,
        mon bien-aimé :
        \\tu es la grâce même !
         
        \\La verdure
        est notre lit ;
${}^{17}les cèdres forment
        les poutres de notre maison
        \\et les cyprès,
        nos lambris.
       
      <div class="intertitle niv7">
        ELLE
      <p class="cantique"><span class="cantique_label"><a href="#bib_ct-at_8bis">Cantique AT 8bis</a></span> = <span class="cantique_ref"><a class="unitex_link" href="#bib_ct_2_10">Ct 2,10cd.14ef</a> ; <a class="unitex_link" href="#bib_ct_4_8">4,8ab</a>.9a,11ef.12acd.15</span>
      
         
      \bchapter{}
${}^{1}Je suis la rose du Sarone,
        le lis des vallées.
        
           
       
      <div class="intertitle niv7">
        LUI
${}^{2}Comme le lis entre les ronces,
        ainsi mon amie entre les jeunes filles.
       
      <div class="intertitle niv7">
        ELLE
${}^{3}Comme un pommier entre les arbres de la forêt,
        ainsi mon bien-aimé entre les jeunes hommes.
         
        \\J’ai désiré son ombre
        et je m’y suis assise :
        \\son fruit est doux
        à mon palais.
         
${}^{4}Il m’a menée
        vers la maison du vin :
        \\l’enseigne au-dessus de moi est
        « Amour ».
         
${}^{5}Soutenez-moi par des gâteaux de raisins,
        \\fortifiez-moi avec des pommes,
        \\car je suis malade d’amour !
         
${}^{6}Son bras gauche est sous ma tête,
        et sa droite m’étreint.
       
      <div class="intertitle niv7">
        LUI
${}^{7}Je vous en conjure, filles de Jérusalem,
        par les gazelles, par les biches des champs,
        \\n’éveillez pas, ne réveillez pas l’Amour,
        avant qu’il le veuille.
      <div class="intertitle niv7">
        ELLE
        ${}^{8}La voix de mon bien-aimé !
        C’est lui, il vient…
        \\Il bondit sur les montagnes,
        il court sur les collines,
        ${}^{9}mon bien-aimé, pareil à la gazelle,
        au faon de la biche.
         
        \\Le voici, c’est lui qui se tient
        derrière notre mur :
        \\il regarde aux fenêtres,
        guette par le treillage.
         
        ${}^{10}Il parle, mon bien-aimé,
        il me dit :
        \\Lève-toi, mon amie, ma toute belle,
        et viens…
         
        ${}^{11}Vois,
        l’hiver s’en est allé,
        \\les pluies ont cessé,
        elles se sont enfuies.
         
        ${}^{12}Sur la terre apparaissent les fleurs,
        le temps des chansons est venu
        \\et la voix de la tourterelle s’entend
        sur notre terre.
         
        ${}^{13}Le figuier a formé ses premiers fruits,
        la vigne fleurie exhale sa bonne odeur.
        \\Lève-toi, mon amie, ma toute belle,
        et viens…
         
        ${}^{14}Ma colombe, dans les fentes du rocher,
        dans les retraites escarpées,
        \\que je voie ton visage,
        que j’entende ta voix !
        \\Ta voix est douce,
        et ton visage, charmant.
         
      <div class="intertitle niv7">
        CHŒUR
${}^{15}Attrapez-nous ces renards,
        ces petits renards
        \\qui ravagent les vignes,
        car nos vignes sont en fleurs !
       
      <div class="intertitle niv7">
        ELLE
        ${}^{16}Mon bien-aimé est à moi,
        et moi, je suis à lui
        \\qui mène paître ses brebis
        parmi les lis.
         
${}^{17}Avant le souffle du jour
        et la fuite des ombres,
        \\toi, retourne…
         
        \\Sois pareil à la gazelle, mon bien-aimé,
        au faon de la biche,
        \\sur les montagnes escarpées.
      <div class="box_other filet_bleu">
          <h3 class="intertitle cantique_chap" id="bib_ct-at_8bis">Cantique AT 8bis</h3><a class="cantique_chap" href="#bib_ct_2">2</a>
            <a class="cantique_verset" href="#bib_ct_2_10"><span class="cantique_verset_in">10</span></a>Lève-toi, mon amie, ma toute belle,
            et viens…
            <a class="cantique_verset" href="#bib_ct_2_14"><span class="cantique_verset_in">14</span></a>Ta voix est douce,
            et ton visage, charmant.
           <a class="cantique_chap" href="#bib_ct_4">4</a>
            <a class="cantique_verset" href="#bib_ct_4_8"><span class="cantique_verset_in">8</span></a>Avec moi, du Liban, ô fiancée,
            avec moi, du Liban, tu viendras.
            <a class="cantique_verset" href="#bib_ct_4_9"><span class="cantique_verset_in">9</span></a>Tu as blessé mon cœur.
             
            <a class="cantique_verset" href="#bib_ct_4_11"><span class="cantique_verset_in">11</span></a>L’odeur de tes vêtements
            est comme l’odeur du Liban.
            <a class="cantique_verset" href="#bib_ct_4_12"><span class="cantique_verset_in">12</span></a>Jardin fermé, source scellée.
             
            <a class="cantique_verset" href="#bib_ct_4_15"><span class="cantique_verset_in">15</span></a>Ô source des jardins,
            puits d’eaux vives
            qui ruissellent du Liban !
      <div class="intertitle niv7">
        ELLE
      
         
      \bchapter{}
${}^{1}Sur mon lit, la nuit, j’ai cherché
        celui que mon âme désire ;
        \\je l’ai cherché ;
        je ne l’ai pas trouvé.
        
           
         
${}^{2}Oui, je me lèverai, je tournerai dans la ville,
        par les rues et les places :
        \\je chercherai
        celui que mon âme désire ;
        \\je l’ai cherché ;
        je ne l’ai pas trouvé.
        
           
         
${}^{3}Ils m’ont trouvée, les gardes,
        eux qui tournent dans la ville :
        \\« Celui que mon âme désire,
        l’auriez-vous vu ? »
        
           
         
${}^{4}À peine les avais-je dépassés,
        j’ai trouvé celui que mon âme désire :
        \\je l’ai saisi
        et ne le lâcherai pas
        \\que je l’aie fait entrer dans la maison de ma mère,
        dans la chambre de celle qui m’a conçue.
        
           
       
      <div class="intertitle niv7">
        LUI
${}^{5}Je vous en conjure, filles de Jérusalem,
        par les gazelles, par les biches des champs,
        \\n’éveillez pas, ne réveillez pas l’Amour,
        avant qu’il le veuille.
      <div class="intertitle niv7">
        CHŒUR
${}^{6}Qu’est-ce là
        qui monte du désert
        \\comme une colonne de fumée
        odorante d’encens et de myrrhe,
        \\de tous les aromates des marchands ?
         
${}^{7}– C’est la litière de Salomon !
        \\Tout autour, soixante braves
        d’entre les braves d’Israël,
${}^{8}tous armés de glaives,
        entraînés à la guerre,
        \\chacun son épée à la hanche
        contre les terreurs de la nuit.
         
${}^{9}Le palanquin fait pour le roi Salomon
        est en bois du Liban ;
${}^{10}il a fait les colonnes d’argent,
        le toit en or, le siège de pourpre ;
        \\l’intérieur fut tapissé avec amour
        par les filles de Jérusalem.
         
${}^{11}– Sortez et regardez, filles de Sion,
        le roi Salomon
        \\avec la couronne
        dont sa mère le couronna
        \\au jour de ses noces,
        au jour de la joie de son cœur.
      <div class="intertitle niv7">
        LUI
      
         
      \bchapter{}
${}^{1}Ah ! Que tu es belle, mon amie !
        Ah ! Que tu es belle !
        \\Tes yeux sont des colombes
        au travers de ton voile.
        
           
         
        \\Ta chevelure : un troupeau de chèvres
        qui dévalent du mont Galaad.
${}^{2}Tes dents : un troupeau de brebis tondues
        qui remontent du bain ;
        \\chacune a sa jumelle,
        nulle n’en est privée.
        
           
         
${}^{3}Comme un ruban d’écarlate, tes lèvres ;
        tes paroles : une harmonie.
        \\Comme une moitié de grenade, ta joue
        au travers de ton voile.
        
           
         
${}^{4}Ton cou : la tour de David,
        harmonieusement élevée ;
        \\mille boucliers sont suspendus,
        toutes les armes des braves.
        
           
         
${}^{5}Tes deux seins : deux faons,
        jumeaux d’une gazelle ;
        \\ils pâturent parmi les lis.
        
           
         
${}^{6}Avant le souffle du jour
        et la fuite des ombres,
        \\j’irai à la montagne de la myrrhe,
        à la colline de l’encens.
        
           
         
${}^{7}Tu es toute belle, ô mon amie !
        Nulle tache en toi !
        
           
      <div class="intertitle niv7">
        LUI
        ${}^{8}Avec moi, du Liban, ô fiancée,
        avec moi, du Liban, tu viendras.
         
        \\Tu regarderas du haut de l’Amana,
        des hauteurs du Sanir et de l’Hermon,
        \\depuis les repaires des lions,
        depuis les montagnes des léopards.
         
        ${}^{9}Tu as blessé mon cœur,
        ma sœur fiancée.
        \\tu as blessé mon cœur,
        d’un seul de tes regards,
        d’un seul anneau de ton collier.
         
${}^{10}Qu’elles sont belles, tes amours,
        ma sœur fiancée !
        \\Qu’elles sont bonnes, tes amours :
        meilleures que le vin !
        \\L’odeur de tes parfums,
        une exquise senteur !
         
        ${}^{11}Un miel pur coule de tes lèvres,
        ô fiancée,
        \\le miel et le lait,
        sous ta langue ;
        \\l’odeur de tes vêtements
        est comme l’odeur du Liban.
         
        ${}^{12}Jardin fermé,
        ma sœur fiancée,
        \\fontaine close,
        source scellée.
         
${}^{13}Tes formes élancées :
        un paradis de grenades
        aux fruits délicieux,
         
        \\le nard et le cypre,
${}^{14}le nard et le safran,
        cannelle, cinnamome,
        \\et tous les arbres à encens,
        la myrrhe et l’aloès,
        tous les plus fins arômes.
         
        ${}^{15}Ô source des jardins,
        puits d’eaux vives
        qui ruissellent du Liban !
         
      <div class="intertitle niv7">
        ELLE
${}^{16}Éveille-toi, Vent du nord !
        Viens, Vent du sud !
        \\Souffle sur mon jardin
        et ses arômes s’exhaleront !
        \\Qu’il entre dans son jardin, mon bien-aimé,
        qu’il en mange les fruits délicieux.
       
      <div class="intertitle niv7">
        LUI
      
         
      \bchapter{}
${}^{1}Je suis entré dans mon jardin,
        ma sœur fiancée :
        \\j’ai recueilli ma myrrhe, avec mes aromates,
        j’ai mangé mon pain et mon miel,
        \\j’ai bu mon vin et mon lait.
        
           
       
      <div class="intertitle niv7">
        CHŒUR
        \\Mangez, amis !
        Buvez, bien-aimés, enivrez-vous !
      <div class="intertitle niv7">
        ELLE
${}^{2}Je dors,
        mais mon cœur veille…
        \\C’est la voix de mon bien-aimé !
        Il frappe !
       
      <div class="intertitle niv7">
        LUI
        \\– Ouvre-moi, ma sœur, mon amie,
        ma colombe, ma toute pure,
        \\car ma tête est humide de rosée
        et mes boucles, des gouttes de la nuit.
       
      <div class="intertitle niv7">
        ELLE
${}^{3}– J’ai ôté ma tunique :
        devrais-je la remettre ?
        \\J’ai lavé mes pieds :
        devrais-je les salir ?
         
${}^{4}Mon bien-aimé a passé la main
        par la fente de la porte ;
        \\mes entrailles ont frémi :
        c’était lui !
         
${}^{5}Je me suis levée
        pour ouvrir à mon bien-aimé,
        les mains ruisselantes de myrrhe.
        \\Mes doigts répandaient cette myrrhe
        sur la barre du verrou.
         
${}^{6}J’ai ouvert à mon bien-aimé :
        mon bien-aimé s’était détourné,
        il avait disparu.
         
        \\Quand il parlait,
        je rendais l’âme…
        \\Je l’ai cherché :
        je ne l’ai pas trouvé.
        \\Je l’appelai :
        il n’a pas répondu.
         
${}^{7}Ils m’ont trouvée, les gardes,
        eux qui tournent dans la ville :
        \\ils m’ont frappée,
        ils m’ont blessée,
        \\ils ont arraché mon voile,
        les gardes des remparts !
         
${}^{8}Je vous en conjure, filles de Jérusalem,
        si vous trouvez mon bien-aimé,
        \\que lui direz-vous ?
        Que je suis malade d’amour.
       
      <div class="intertitle niv7">
        CHŒUR
${}^{9}Qu’a-t-il, ton bien-aimé, de plus qu’un autre,
        ô belle entre les femmes ?
        \\Qu’a-t-il, ton bien-aimé, de plus qu’un autre
        que tu nous adjures ainsi ?
      <div class="intertitle niv7">
        ELLE
${}^{10}Mon bien-aimé est clair et vermeil :
        on le distingue entre dix mille !
         
${}^{11}Sa tête est d’or,
        d’un or pur.
        \\Ses boucles, d’un noir de corbeau,
        ondulent.
         
${}^{12}Ses yeux sont comme des colombes
        au bord d’un ruisseau
        \\qui baignent dans le lait
        et reposent, tranquilles.
         
${}^{13}Ses joues : un parterre d’arômes,
        des corbeilles de senteurs.
        \\Ses lèvres, des lis,
        un ruissellement de myrrhe.
         
${}^{14}Ses bras, des torsades d’or
        serties de topazes.
        \\Son ventre : un bloc d’ivoire,
        couvert de saphirs.
         
${}^{15}Ses jambes : des colonnes de marbre
        posées sur des socles d’or pur.
        \\Son aspect est celui du Liban :
        comme le cèdre, sans rival !
         
${}^{16}Sa bouche est pur délice,
        tout, en lui, est désirable.
        \\Tel est mon bien-aimé ;
        tel est mon aimé, filles de Jérusalem.
      <div class="intertitle niv7">
        CHŒUR
      
         
      \bchapter{}
${}^{1}Où est parti ton bien-aimé,
        ô belle entre les femmes ?
        \\Où s’en est allé ton bien-aimé
        que nous le cherchions avec toi ?
        
           
       
      <div class="intertitle niv7">
        ELLE
${}^{2}Mon bien-aimé est descendu dans son jardin,
        dans les parterres d’aromates,
        \\pour mener ses brebis paître aux jardins,
        et pour cueillir des lis.
         
${}^{3}Je suis à mon bien-aimé,
        mon bien-aimé est à moi,
        \\lui qui mène paître ses brebis
        parmi les lis.
      <div class="intertitle niv7">
        LUI
${}^{4}Tu es belle, ô mon amie,
        comme Tirsa,
        \\splendide comme Jérusalem,
        terrible comme des bataillons !
         
${}^{5}Détourne de moi tes yeux,
        car ils me troublent.
         
        \\Ta chevelure : un troupeau de chèvres
        qui dévalent du Galaad.
         
${}^{6}Tes dents : un troupeau de brebis
        qui remontent du bain ;
        \\chacune a sa jumelle,
        nulle n’en est privée.
         
${}^{7}Comme une moitié de grenade, ta joue
        au travers de ton voile.
         
${}^{8}Soixante sont les reines,
        quatre-vingts, les compagnes,
        sans nombre, les jeunes filles.
         
${}^{9}Unique est ma colombe,
        ma parfaite,
        \\unique pour sa mère,
        merveille pour qui l’a mise au monde.
         
        \\Les jeunes filles l’ont vue,
        l’ont dite bienheureuse ;
        \\reines et compagnes
        ont chanté ses louanges :
         
${}^{10}« Qui donc est celle qui surgit,
        semblable à l’aurore,
        \\belle autant que la lune, brillante comme le soleil,
        terrible comme des bataillons ? »
         
${}^{11}Je suis descendu au jardin du noyer
        voir le vallon qui verdoie,
        \\voir si la vigne bourgeonne,
        si les grenadiers sont en fleurs…
       
      <div class="intertitle niv7">
        ELLE
${}^{12}Je ne sais plus, mon âme m’a transportée
        sur les chars de mon peuple-prince.
      <div class="intertitle niv7">
        CHŒUR
      
         
      \bchapter{}
${}^{1}Reviens, reviens,
        ô Sulamite !
        \\Reviens, reviens :
        que nous t’admirions !
        
           
         
        \\– Qu’admirez-vous de la Sulamite
        tandis qu’elle danse au milieu des deux chœurs ?
        
           
         
${}^{2}Comme ils sont beaux, tes pieds,
        dans tes sandales,
        fille de prince !
        \\Les courbes de tes hanches
        dessinent des colliers,
        œuvre de mains artistes.
        
           
         
${}^{3}Ton nombril : une coupe ronde
        où le vin ne tarit pas.
        \\Ton ventre : un monceau de blé
        dans un enclos de lis.
        
           
         
${}^{4}Tes deux seins : deux faons,
        jumeaux d’une gazelle.
${}^{5}Ton cou :
        une tour d’ivoire.
        
           
         
        \\Tes yeux : les vasques de Heshbone
        à la porte de Bath-Rabbim,
        \\et ton nez, comme la Tour du Liban,
        sentinelle tournée vers Damas.
        
           
         
${}^{6}Ta tête se dresse
        comme le Carmel.
        \\Sa parure est de pourpre ;
        un roi s’est pris dans ces tresses.
        
           
       
      <div class="intertitle niv7">
        LUI
${}^{7}Ah ! Que tu es belle !
        Que tu es douce,
        \\amour, en tes caresses !
         
${}^{8}Tu es élancée comme le palmier,
        tes seins en sont les grappes.
${}^{9}J’ai dit : je monterai au palmier,
        j’en saisirai les fruits.
         
        \\Tes seins, qu’ils soient comme des grappes de raisins,
        ton haleine, comme une odeur de pomme,
${}^{10}ta bouche, un vin exquis…
       
      <div class="intertitle niv7">
        ELLE
        \\Il s’écoule vers mon bien-aimé,
        abreuvant des lèvres endormies.
         
${}^{11}Je suis à mon bien-aimé :
        vers moi, monte son désir.
      <div class="intertitle niv7">
        ELLE
${}^{12}Viens, mon bien-aimé…
         
        \\Nous sortirons dans les champs,
        nous passerons la nuit dans la campagne.
${}^{13}Au matin, nous irons dans les vignes,
        nous verrons si les pampres fleurissent,
        \\si le bourgeon s’est ouvert,
        si les grenadiers sont en fleurs.
         
        \\Là, je t’offrirai mes amours…
${}^{14}Les mandragores ont exhalé leur parfum…
         
        \\Et à nos portes, toutes les délices,
        les nouvelles comme les anciennes.
        \\Mon bien-aimé,
        je les ai gardées pour toi.
         
      
         
      \bchapter{}
${}^{1}Que n’es-tu pour moi un frère,
        nourri aux seins de ma mère ?
        \\Je te rencontrerais au dehors
        et je t’embrasserais
        sans paraître méprisable !
        
           
         
${}^{2}Je t’emmènerais, je t’introduirais
        dans la maison de ma mère :
        \\tu m’initierais…
        \\Je t’abreuverais de vin parfumé,
        de ma liqueur de grenade.
        
           
         
${}^{3}Sa main gauche est sous ma tête,
        et sa droite m’étreint.
        
           
       
      <div class="intertitle niv7">
        LUI
${}^{4}Je vous en conjure, filles de Jérusalem,
        \\n’éveillez pas, ne réveillez pas l’Amour,
        avant qu’il le veuille.
      <div class="intertitle niv7">
        CHŒUR
${}^{5}Qui donc est celle-ci
        qui monte du désert
        appuyée sur son bien-aimé ?
       
      <div class="intertitle niv7">
        LUI
        \\Sous le pommier, je t’éveille,
        là où ta mère t’a enfantée ;
        là, elle t’a enfantée et mise au monde.
       
      <div class="intertitle niv7">
        ELLE
        ${}^{6}Pose-moi comme un sceau sur ton cœur,
        comme un sceau sur ton bras.
       
      <div class="intertitle niv7">
        CHŒUR
        \\Car l’amour est fort comme la Mort,
        la passion, implacable comme l’Abîme :
        \\ses flammes sont des flammes de feu,
        fournaise divine.
         
        ${}^{7}Les grandes eaux ne pourront éteindre l’amour,
        ni les fleuves l’emporter.
      <div class="intertitle niv7">
        CHŒUR
        \\Un homme donnerait-il toutes les richesses de sa maison
        pour acheter l’amour,
        \\il ne recueillerait que mépris.
         
${}^{8}Nous avons une petite sœur
        qui n’a pas encore de seins.
        \\Que ferons-nous pour notre sœur
        le jour où l’on parlera d’elle ?
         
${}^{9}Sera-t-elle un rempart ?
        Nous lui bâtirons un créneau d’argent.
        \\Sera-t-elle une porte ?
        Nous la munirons d’une barre de cèdre.
       
      <div class="intertitle niv7">
        ELLE
${}^{10}– « Je suis un rempart,
        mes seins sont des tours !
        \\Et je suis devenue à ses yeux
        celle qui a trouvé la paix. »
       
      <div class="intertitle niv7">
        LUI
${}^{11}Salomon avait une vigne à Baal-Hamone :
        il remit la vigne à des gardiens.
        \\Chacun devait payer pour son fruit
        mille pièces d’argent.
         
${}^{12}Ma vigne, à moi, je l’ai sous mes yeux.
        \\À toi, Salomon, les mille pièces,
        \\et deux cents aux gardiens de son fruit.
         
${}^{13}Toi, l’habitante des jardins,
        \\des compagnons guettent ta voix.
        \\Donne-moi de l’entendre…
       
      <div class="intertitle niv7">
        ELLE
${}^{14}Fuis, mon bien-aimé,
        \\pareil à la gazelle, au faon de la biche,
        \\sur des montagnes embaumées…
