  
  
          
            \bchapter{Psaume}
            Je dois vivre en exil
${}^{1}Cantique des montées.
         
        \\Dans ma détresse, j’ai cri\underline{é} vers le Seigneur,
        et lu\underline{i} m’a répondu. *
${}^{2}Seigneur, délivre-moi de la l\underline{a}ngue perfide,
        de la bo\underline{u}che qui ment.
         
${}^{3}Que t’infliger, ô l\underline{a}ngue perfide,
        et qu’ajout\underline{e}r encore ? *
${}^{4}La flèche meurtri\underline{è}re du guerrier,
        et la br\underline{a}ise des genêts.
         
${}^{5}Malheur à moi : je dois v\underline{i}vre en exil *
        et camp\underline{e}r dans un désert !
         
${}^{6}Trop longtemps, j’ai véc\underline{u} parmi ces gens
        qui ha\underline{ï}ssent la paix.*
${}^{7}Je ne veux que la p\underline{a}ix, mais quand je parle
        ils ch\underline{e}rchent la guerre.
      \bchapter{Psaume}
          
            \bchapter{Psaume}
            Le Seigneur, ton gardien
${}^{1}Cantique pour les montées.
         
        \\Je lève les ye\underline{u}x vers les montagnes :
        \\d’où le seco\underline{u}rs me viendra-t-il ?
${}^{2}Le secours me viendr\underline{a} du Seigneur
        \\qui a fait le ci\underline{e}l et la terre.
         
${}^{3}Qu’il empêche ton pi\underline{e}d de glisser,
        \\qu’il ne dorme p\underline{a}s, ton gardien.
${}^{4}Non, il ne dort p\underline{a}s, ne sommeille pas,
        \\le gardi\underline{e}n d’Israël.
         
${}^{5}Le Seigneur, ton gardien, le Seigne\underline{u}r, ton ombrage,
        \\se ti\underline{e}nt près de toi.
${}^{6}Le soleil, pendant le jour, ne pourr\underline{a} te frapper,
        \\ni la l\underline{u}ne, durant la nuit.
         
${}^{7}Le Seigneur te garder\underline{a} de tout mal,
        \\il garder\underline{a} ta vie.
${}^{8}Le Seigneur te gardera, au dép\underline{a}rt et au retour,
        \\mainten\underline{a}nt, à jamais.
      \bchapter{Psaume}
          
            \bchapter{Psaume}
            Paix sur Jérusalem !
${}^{1}Cantique des montées. De David.
         
        \\Quelle j\underline{o}ie quand on m’a dit :
        \\« Nous irons à la mais\underline{o}n du Seigneur ! »
         
${}^{2}Maintenant notre m\underline{a}rche prend fin
        \\devant tes p\underline{o}rtes, Jérusalem !
${}^{3}Jérusalem, te voic\underline{i} dans tes murs :
        \\ville où tout ens\underline{e}mble ne fait qu’un !
         
${}^{4}C’est là que montent les tribus,
        les trib\underline{u}s du Seigneur, *
        \\là qu’Israël doit rendre grâce
        au n\underline{o}m du Seigneur.
${}^{5}C’est là le si\underline{è}ge du droit, *
        \\le siège de la mais\underline{o}n de David.
         
${}^{6}Appelez le bonhe\underline{u}r sur Jérusalem :
        \\« P\underline{a}ix à ceux qui t’aiment !
${}^{7}Que la paix r\underline{è}gne dans tes murs,
        \\le bonhe\underline{u}r dans tes palais ! »
         
${}^{8}À cause de mes fr\underline{è}res et de mes proches,
        \\je dirai : « P\underline{a}ix sur toi ! »
${}^{9}À cause de la maison du Seigne\underline{u}r notre Dieu,
        \\je dés\underline{i}re ton bien.
      \bchapter{Psaume}
          
            \bchapter{Psaume}
            Les yeux levés vers toi
${}^{1}Cantique des montées.
         
        \\Vers toi j’ai les ye\underline{u}x levés,
        \\vers toi qui \underline{e}s au ciel.
         
${}^{2}Comme les yeux de l’esclave
        vers la m\underline{a}in de son maître, +
        \\comme les yeux de la servante
        vers la m\underline{a}in de sa maîtresse, *
        \\nos yeux, levés vers le Seigneur notre Dieu,
        att\underline{e}ndent sa pitié.
         
${}^{3}Pitié pour nous, Seigne\underline{u}r, pitié pour nous :
        \\notre âme est rassasi\underline{é}e de mépris.
         
${}^{4}C’en est trop,
        nous s\underline{o}mmes rassasiés *
        \\du rire des satisfaits,
        du mépr\underline{i}s des orgueilleux !
      \bchapter{Psaume}
          
            \bchapter{Psaume}
            Le filet s’est rompu
${}^{1}Cantique des montées. De David.
         
        \\Sans le Seigneur qui ét\underline{a}it pour nous,
        – qu’Isra\underline{ë}l le redise – +
${}^{2}sans le Seigneur qui ét\underline{a}it pour nous
        quand des h\underline{o}mmes nous assaillirent, *
${}^{3}alors ils nous aval\underline{a}ient tout vivants,
        dans le fe\underline{u} de leur colère.
         
${}^{4}Alors le fl\underline{o}t passait sur nous,
        le torr\underline{e}nt nous submergeait ; *
${}^{5}alors nous éti\underline{o}ns submergés
        par les fl\underline{o}ts en furie.
         
${}^{6}Bén\underline{i} soit le Seigneur *
        \\qui n’a pas fait de nous
        la pr\underline{o}ie de leurs dents !
         
${}^{7}Comme un oiseau, nous av\underline{o}ns échappé
        au fil\underline{e}t du chasseur ; *
        \\le fil\underline{e}t s’est rompu :
        nous av\underline{o}ns échappé.
         
${}^{8}Notre secours est le n\underline{o}m du Seigneur *
        \\qui a fait le ci\underline{e}l et la terre.
      \bchapter{Psaume}
          
            \bchapter{Psaume}
            Il entoure son peuple
${}^{1}Cantique des montées.
         
        \\Qui s’appu\underline{i}e sur le Seigneur
        ress\underline{e}mble au mont Sion : *
        \\il \underline{e}st inébranlable,
        il deme\underline{u}re à jamais.
         
${}^{2}Jérusalem, des mont\underline{a}gnes l’entourent ; *
        \\ainsi le Seigneur : il ento\underline{u}re son peuple
        mainten\underline{a}nt et toujours.
         
${}^{3}Jamais le sc\underline{e}ptre de l’impie
        ne pèsera sur la p\underline{a}rt des justes, *
        \\de peur que la m\underline{a}in des justes
        ne se t\underline{e}nde vers l’idole.
         
${}^{4}Sois bon pour qui est b\underline{o}n, Seigneur,
        pour l’h\underline{o}mme au cœur droit. *
${}^{5}Mais ceux qui r\underline{u}sent et qui trahissent,
        que le Seigneur les rej\underline{e}tte avec les méchants !
         
        Paix sur Israël !
      \bchapter{Psaume}
          
            \bchapter{Psaume}
            Ramène nos captifs
${}^{1}Cantique des montées.
         
        \\Quand le Seigneur ramena les capt\underline{i}fs à Sion, *
        \\nous éti\underline{o}ns comme en rêve !
         
${}^{2}Alors notre bouche était pl\underline{e}ine de rires,
        nous poussi\underline{o}ns des cris de joie ;
        \\alors on disait parm\underline{i} les nations :
        « Quelles merveilles fait pour e\underline{u}x le Seigneur ! »
${}^{3}Quelles merveilles le Seigne\underline{u}r fit pour nous :
        nous éti\underline{o}ns en grande fête !
         
        *
         
${}^{4}Ramène, Seigne\underline{u}r, nos captifs,
        \\comme les torr\underline{e}nts au désert.
         
${}^{5}Qui s\underline{è}me dans les larmes
        moiss\underline{o}nne dans la joie : +
${}^{6}il s’en va, il s’en v\underline{a} en pleurant, 
        il j\underline{e}tte la semence ;
        \\il s’en vient, il s’en vi\underline{e}nt dans la joie,
        il rapp\underline{o}rte les gerbes.
      \bchapter{Psaume}
          
            \bchapter{Psaume}
            Si le Seigneur ne bâtit
${}^{1}Cantique des montées. De Salomon.
         
        \\Si le Seigneur ne bât\underline{i}t la maison,
        les bâtisseurs trav\underline{a}illent en vain ; *
        \\si le Seigneur ne g\underline{a}rde la ville,
        c’est en vain que v\underline{e}illent les gardes.
         
${}^{2}En vain tu dev\underline{a}nces le jour,
        tu retardes le mom\underline{e}nt de ton repos, +
        \\tu manges un p\underline{a}in de douleur : *
        Dieu comble son bien-aim\underline{é} quand il dort.
         
${}^{3}Des fils, voilà ce que d\underline{o}nne le Seigneur,
        des enfants, la récomp\underline{e}nse qu’il accorde ; *
${}^{4}comme des flèches aux m\underline{a}ins d’un guerrier,
        ainsi les f\underline{i}ls de la jeunesse.
         
${}^{5}Heureux l’h\underline{o}mme vaillant
        qui a garni son carqu\underline{o}is de telles armes ! *
        \\S’ils affrontent leurs ennem\underline{i}s sur la place,
        ils ne seront p\underline{a}s humiliés.
      \bchapter{Psaume}
          
            \bchapter{Psaume}
            Tu verras les fils de tes fils
${}^{1}Cantique des montées.
         
        \\Heureux qui cr\underline{a}int le Seigneur
        \\et marche sel\underline{o}n ses voies !
${}^{2}Tu te nourriras du trav\underline{a}il de tes mains :
        \\Heureux es-tu ! À t\underline{o}i, le bonheur !
         
${}^{3}Ta femme sera dans ta maison
        comme une v\underline{i}gne généreuse, *
        \\et tes fils, autour de la table,
        comme des pl\underline{a}nts d’olivier.
         
${}^{4}Voilà comment sera béni
        l’homme qui cr\underline{a}int le Seigneur. *
${}^{5}De Sion, que le Seigne\underline{u}r te bénisse  !
         
        \\Tu verras le bonheur de Jérusalem
        tous les jo\underline{u}rs de ta vie, *
${}^{6}et tu verras les f\underline{i}ls de tes fils.
         
        Paix sur Israël !
      \bchapter{Psaume}
          
            \bchapter{Psaume}
            Ils ne m’ont pas soumis
${}^{1}Cantique des montées.
         
        \\Que de mal ils m’ont f\underline{a}it dès ma jeunesse,
        – à Isra\underline{ë}l de le dire – *
${}^{2}que de mal ils m’ont f\underline{a}it dès ma jeunesse :
        ils ne m’ont p\underline{a}s soumis !
         
${}^{3}Sur mon dos, des laboure\underline{u}rs ont labouré
        et creus\underline{é} leurs sillons ; *
${}^{4}mais le Seigne\underline{u}r, le juste,
        a brisé l’attel\underline{a}ge des impies.
         
${}^{5}Qu’ils soient tous humili\underline{é}s, rejetés,
        les ennem\underline{i}s de Sion ! *
${}^{6}Qu’ils deviennent comme l’h\underline{e}rbe des toits,
        aussit\underline{ô}t desséchée !
         
${}^{7}Les moissonneurs n’en font p\underline{a}s une poignée,
        ni les lie\underline{u}rs une gerbe, *
${}^{8}et les passants ne pe\underline{u}vent leur dire :
        « La bénédiction du Seigne\underline{u}r soit sur vous ! »
         
        \\Au nom du Seigneur, no\underline{u}s vous bénissons.
      \bchapter{Psaume}
          
            \bchapter{Psaume}
            Près du Seigneur, abonde le rachat
${}^{1}Cantique des montées.
         
        \\Des profondeurs je crie vers t\underline{o}i, Seigneur,
${}^{2}Seigneur, éco\underline{u}te mon appel ! *
        \\Que ton oreille se f\underline{a}sse attentive
        au cr\underline{i} de ma prière !
         
${}^{3}Si tu retiens les fa\underline{u}tes, Seigneur,
        Seigneur, qu\underline{i} subsistera ? *
${}^{4}Mais près de toi se tro\underline{u}ve le pardon
        pour que l’h\underline{o}mme te craigne.
         
${}^{5}J’espère le Seigneur de to\underline{u}te mon âme ; *
        \\je l’espère, et j’att\underline{e}nds sa parole.
         
${}^{6}Mon âme att\underline{e}nd le Seigneur
        plus qu’un veilleur ne gu\underline{e}tte l’aurore. *
        \\Plus qu’un veilleur ne gu\underline{e}tte l’aurore,
${}^{7}attends le Seigne\underline{u}r, Israël.
         
        \\Oui, près du Seigne\underline{u}r, est l’amour ;
        près de lui, ab\underline{o}nde le rachat. *
${}^{8}C’est lui qui rachèter\underline{a} Israël
        de to\underline{u}tes ses fautes.
      \bchapter{Psaume}
          
            \bchapter{Psaume}
            Comme un petit enfant
${}^{1}Cantique des montées. De David.
         
        \\Seigneur, je n’ai p\underline{a}s le cœur fier
        ni le reg\underline{a}rd ambitieux ; *
        \\je ne poursu\underline{i}s ni grands desseins,
        ni merv\underline{e}illes qui me dépassent.
         
${}^{2}Non, mais je ti\underline{e}ns mon âme
        ég\underline{a}le et silencieuse ; *
        \\mon âme est en m\underline{o}i comme un enfant,
        comme un petit enf\underline{a}nt contre sa mère.
         
${}^{3}Attends le Seigne\underline{u}r, Israël, *
        mainten\underline{a}nt et à jamais.
      \bchapter{Psaume}
          
            \bchapter{Psaume}
            Pour l’amour de David
${}^{1}Cantique des montées.
         
        \\Souviens-toi, Seigne\underline{u}r, de David
        \\et de sa gr\underline{a}nde soumission
${}^{2}quand il fit au Seigne\underline{u}r un serment,
        \\une promesse au Puiss\underline{a}nt de Jacob :
         
${}^{3}« Jamais je n’entrer\underline{a}i sous ma tente,
        \\et jamais ne m’étendr\underline{a}i sur mon lit,
${}^{4}j’interdirai tout somm\underline{e}il à mes yeux
        \\et tout rép\underline{i}t à mes paupières,
${}^{5}avant d’avoir trouvé un lie\underline{u} pour le Seigneur,
        \\une demeure pour le Puiss\underline{a}nt de Jacob. »
         
${}^{6}Voici qu’on nous l’ann\underline{o}nce à Éphrata,
        \\nous l’avons trouv\underline{é}e près de Yagar.
${}^{7}Entrons dans la deme\underline{u}re de Dieu,
        \\prosternons-nous aux pi\underline{e}ds de son trône.
         
${}^{8}Monte, Seigneur, vers le lie\underline{u} de ton repos,
        \\toi, et l’\underline{a}rche de ta force !
${}^{9}Que tes prêtres soient vêt\underline{u}s de justice,
        \\que tes fid\underline{è}les crient de joie !
         
${}^{10}Pour l’amour de Dav\underline{i}d, ton serviteur,
        \\ne repousse pas la f\underline{a}ce de ton messie.
         
        *
         
${}^{11}Le Seigneur l’a jur\underline{é} à David,
        \\et jamais il ne reprendr\underline{a} sa parole :
        \\« C’est un homme iss\underline{u} de toi
        \\que je placer\underline{a}i sur ton trône.
         
${}^{12}« Si tes fils g\underline{a}rdent mon alliance,
        \\les volontés que je leur f\underline{a}is connaître,
        \\leurs fils, eux auss\underline{i}, à tout jamais,
        \\siègeront sur le trône dress\underline{é} pour toi. »
         
${}^{13}Car le Seigneur a fait ch\underline{o}ix de Sion ;
        \\elle est le séjo\underline{u}r qu’il désire :
${}^{14}« Voilà mon rep\underline{o}s à tout jamais,
        \\c’est le séjour que j’av\underline{a}is désiré.
         
${}^{15}« Je bénirai, je bénir\underline{a}i ses récoltes
        \\pour rassasier de p\underline{a}in ses pauvres.
${}^{16}Je vêtirai de gl\underline{o}ire ses prêtres,
        \\et ses fidèles crieront, crier\underline{o}nt de joie.
         
${}^{17}« Là, je ferai germer la f\underline{o}rce de David ;
        \\pour mon messie, j’ai allum\underline{é} une lampe.
${}^{18}Je vêtirai ses ennem\underline{i}s de honte,
        \\mais, sur lui, la cour\underline{o}nne fleurira. »
      \bchapter{Psaume}
          
            \bchapter{Psaume}
            Vivre ensemble, être unis
${}^{1}Cantique des montées. De David.
         
        \\Oui, il est bon, il est do\underline{u}x pour des frères *
        de vivre ens\underline{e}mble et d’être unis !
         
${}^{2}On dirait un ba\underline{u}me précieux,
        un parf\underline{u}m sur la tête, +
        \\qui descend sur la barbe, la b\underline{a}rbe d’Aaron, *
        qui descend sur le b\underline{o}rd de son vêtement.
         
${}^{3}On dirait la ros\underline{é}e de l’Hermon *
        qui descend sur les coll\underline{i}nes de Sion.
        \\C’est là que le Seigneur env\underline{o}ie la bénédiction, *
        la v\underline{i}e pour toujours.
      \bchapter{Psaume}
          
            \bchapter{Psaume}
            Au long des nuits
${}^{1}Cantique des montées.
         
        \\Vous tous, béniss\underline{e}z le Seigneur,
        \\vous qui serv\underline{e}z le Seigneur,
        \\qui veillez dans la mais\underline{o}n du Seigneur
        \\au l\underline{o}ng des nuits.
         
${}^{2}Levez les mains v\underline{e}rs le sanctuaire,
        \\et béniss\underline{e}z le Seigneur.
${}^{3}Que le Seigneur te bén\underline{i}sse de Sion,
        \\lui qui a fait le ci\underline{e}l et la terre !
      \bchapter{Psaume}
          
            \bchapter{Psaume}
            Tout ce que veut le Seigneur, il le fait
${}^{1}Alléluia !
         
        \\Louez le n\underline{o}m du Seigneur,
        \\louez-le, servite\underline{u}rs du Seigneur
${}^{2}qui veillez dans la mais\underline{o}n du Seigneur,
        \\dans les parvis de la mais\underline{o}n de notre Dieu.
         
${}^{3}Louez la bont\underline{é} du Seigneur,
        \\célébrez la douce\underline{u}r de son nom.
${}^{4}C’est Jacob que le Seigne\underline{u}r a choisi,
        \\Israël dont il a f\underline{a}it son bien.
         
${}^{5}Je le sais, le Seigne\underline{u}r est grand :
        \\notre Maître est plus gr\underline{a}nd que tous les dieux.
${}^{6}Tout ce que veut le Seigne\underline{u}r, il le fait *
        \\au ciel et sur la terre,
        dans les mers et jusqu’au f\underline{o}nd des abîmes.
         
${}^{7}De l’horizon, il fait mont\underline{e}r les nuages ; +
        \\il lance des éclairs, et la plu\underline{i}e ruisselle ; *
        \\il libère le vent qu’il ten\underline{a}it en réserve.
         
${}^{8}Il a frappé les aîn\underline{é}s de l’Égypte,
        \\les premiers-nés de l’h\underline{o}mme et du bétail.
${}^{9}Il envoya des signes et des prodiges,
        chez toi, t\underline{e}rre d’Égypte, *
        \\sur Pharaon et to\underline{u}s ses serviteurs.
         
${}^{10}Il a frappé des nati\underline{o}ns en grand nombre
        \\et fait périr des r\underline{o}is valeureux :
${}^{11}(Séhone, le roi des Amorites, Og, le r\underline{o}i de Bashane,
        \\et tous les roya\underline{u}mes de Canaan ;)
${}^{12}il a donné leur pa\underline{y}s en héritage,
        \\en héritage à Isra\underline{ë}l, son peuple.
         
        *
         
${}^{13}Pour toujours, Seigne\underline{u}r, ton nom !
        \\D’âge en âge, Seigne\underline{u}r, ton mémorial !
${}^{14}Car le Seigneur rend just\underline{i}ce à son peuple :
        \\par égard pour ses servite\underline{u}rs, il se reprend.
         
${}^{15}Les idoles des nations : \underline{o}r et argent,
        \\ouvrages de m\underline{a}ins humaines.
${}^{16}Elles ont une bo\underline{u}che et ne parlent pas,
        \\des ye\underline{u}x et ne voient pas.
         
${}^{17}Leurs oreilles n’ent\underline{e}ndent pas,
        \\et dans leur bouche, p\underline{a}s le moindre souffle.
${}^{18}Qu’ils deviennent comme elles, tous ce\underline{u}x qui les font,
        \\ceux qui mettent leur f\underline{o}i en elles.
         
${}^{19}Maison d’Israël, bén\underline{i}s le Seigneur,
        \\maison d’Aaron, bén\underline{i}s le Seigneur,
${}^{20}maison de Lévi, bén\underline{i}s le Seigneur,
        \\et vous qui le craignez, béniss\underline{e}z le Seigneur !
         
${}^{21}Béni soit le Seigne\underline{u}r depuis Sion,
        \\lui qui hab\underline{i}te Jérusalem !
      \bchapter{Psaume}
          
            \bchapter{Psaume}
            Éternel est son amour
${}^{1}Rendez grâce au Seigne\underline{u}r : il est bon,
        étern\underline{e}l est son amour !
${}^{2}Rendez gr\underline{â}ce au Dieu des dieux,
        étern\underline{e}l est son amour !
${}^{3}Rendez grâce au Seigne\underline{u}r des seigneurs,
        étern\underline{e}l est son amour !
         
${}^{4}Lui seul a fait de gr\underline{a}ndes merveilles,
        étern\underline{e}l est son amour !
${}^{5}lui qui fit les cie\underline{u}x avec sagesse,
        étern\underline{e}l est son amour !
${}^{6}qui affermit la t\underline{e}rre sur les eaux,
        étern\underline{e}l est son amour !
         
${}^{7}Lui qui a fait les gr\underline{a}nds luminaires,
        étern\underline{e}l est son amour !
${}^{8}le soleil qui r\underline{è}gne sur le jour,
        étern\underline{e}l est son amour !
${}^{9}la lune et les ét\underline{o}iles, sur la nuit,
        étern\underline{e}l est son amour !
         
${}^{10}Lui qui frappa les Égypti\underline{e}ns dans leurs aînés,
        étern\underline{e}l est son amour !
${}^{11}et fit sortir Isra\underline{ë}l de leur pays,
        étern\underline{e}l est son amour !
${}^{12}d’une main forte et d’un br\underline{a}s vigoureux,
        étern\underline{e}l est son amour !
         
${}^{13}Lui qui fendit la mer Ro\underline{u}ge en deux parts,
        étern\underline{e}l est son amour !
${}^{14}et fit passer Isra\underline{ë}l en son milieu,
        étern\underline{e}l est son amour !
${}^{15}y rejetant Phara\underline{o}n et ses armées,
        étern\underline{e}l est son amour !
         
${}^{16}Lui qui mena son pe\underline{u}ple au désert,
        étern\underline{e}l est son amour !
${}^{17}qui frappa des pr\underline{i}nces fameux,
        étern\underline{e}l est son amour !
${}^{18}et fit périr des r\underline{o}is redoutables,
        étern\underline{e}l est son amour !
         
${}^{19}Séhone, le r\underline{o}i des Amorites,
        étern\underline{e}l est son amour !
${}^{20}et Og, le r\underline{o}i de Bashane,
        étern\underline{e}l est son amour !
         
${}^{21}pour donner leur pa\underline{y}s en héritage,
        étern\underline{e}l est son amour !
${}^{22}en héritage à Isra\underline{ë}l, son serviteur,
        étern\underline{e}l est son amour !
         
${}^{23}Il se souvient de no\underline{u}s, les humiliés,
        étern\underline{e}l est son amour !
${}^{24}il nous tira de la m\underline{a}in des oppresseurs,
        étern\underline{e}l est son amour !
         
${}^{25}À toute chair, il d\underline{o}nne le pain,
        étern\underline{e}l est son amour !
${}^{26}Rendez gr\underline{â}ce au Dieu du ciel,
        étern\underline{e}l est son amour !
      \bchapter{Psaume}
          
            \bchapter{Psaume}
            Si je t’oublie, Jérusalem…
${}^{1}Au bord des fle\underline{u}ves de Babylone
        nous étions ass\underline{i}s et nous pleurions, +
        \\nous souven\underline{a}nt de Sion ; *
${}^{2}aux sa\underline{u}les des alentours
        nous avi\underline{o}ns pendu nos harpes.
         
${}^{3}C’est l\underline{à} que nos vainqueurs
        nous demand\underline{è}rent des chansons, +
        \\et nos bourrea\underline{u}x, des airs joyeux : *
        \\« Chantez-no\underline{u}s, disaient-ils,
        quelque ch\underline{a}nt de Sion. »
         
${}^{4}Comm\underline{e}nt chanterions-nous
        un ch\underline{a}nt du Seigneur +
        \\sur une t\underline{e}rre étrangère ? *
${}^{5}Si je t’oubl\underline{i}e, Jérusalem,
        que ma main dr\underline{o}ite m’oublie !
         
${}^{6}Je ve\underline{u}x que ma langue
        s’att\underline{a}che à mon palais +
        \\si je p\underline{e}rds ton souvenir, *
        \\si je n’él\underline{è}ve Jérusalem,
        au somm\underline{e}t de ma joie.
         
${}^{7}\[Souviens-t\underline{o}i, Seigneur,
        des f\underline{i}ls du pays d’Édom, +
        \\et de ce jo\underline{u}r à Jérusalem *
        \\où ils cri\underline{a}ient : « Détruisez-la,
        détruisez-l\underline{a} de fond en comble ! »
         
${}^{8}Ô Babyl\underline{o}ne misérable, +
        \\heure\underline{u}x qui te revaudra
        les ma\underline{u}x que tu nous valus ; *
${}^{9}heureux qui saisir\underline{a} tes enfants,
        pour les bris\underline{e}r contre le roc !\]
      \bchapter{Psaume}
          
            \bchapter{Psaume}
            Le Seigneur fait tout pour moi
${}^{1}De David.
         
        \\De tout mon cœur, Seigne\underline{u}r, je te rends grâce :
        \\tu as entendu les par\underline{o}les de ma bouche.
        \\Je te chante en prés\underline{e}nce des anges,
${}^{2}vers ton temple sacr\underline{é}, je me prosterne.
         
        \\Je rends grâce à ton nom pour ton amo\underline{u}r et ta vérité,
        \\car tu élèves, au-dessus de tout, ton n\underline{o}m et ta parole.
${}^{3}Le jour où tu répond\underline{i}s à mon appel,
        \\tu fis grandir en mon \underline{â}me la force.
         
${}^{4}Tous les rois de la t\underline{e}rre te rendent grâce
        \\quand ils entendent les par\underline{o}les de ta bouche.
${}^{5}Ils chantent les chem\underline{i}ns du Seigneur :
        \\« Qu’elle est grande, la gl\underline{o}ire du Seigneur ! »
         
${}^{6}Si haut que soit le Seigneur, il v\underline{o}it le plus humble ;
        \\de loin, il reconn\underline{a}ît l’orgueilleux.
${}^{7}Si je marche au milieu des ang\underline{o}isses, tu me fais vivre,
        \\ta main s’abat sur mes ennem\underline{i}s en colère.
         
        \\Ta droite me r\underline{e}nd vainqueur.
${}^{8}Le Seigneur fait to\underline{u}t pour moi !
        \\Seigneur, étern\underline{e}l est ton amour :
        \\n’arrête pas l’œ\underline{u}vre de tes mains.
      \bchapter{Psaume}
          
            \bchapter{Psaume}
            Seigneur, tu sais
${}^{1}Du maître de chœur. De David. Psaume.
         
        \\Tu me scrutes, Seigne\underline{u}r, et tu sais ! +
${}^{2}Tu sais quand je m’ass\underline{o}is, quand je me lève ;
        \\de très loin, tu pén\underline{è}tres mes pensées.
         
${}^{3}Que je marche ou me rep\underline{o}se, tu le vois,
        \\tous mes chemins te s\underline{o}nt familiers.
${}^{4}Avant qu’un mot ne parvi\underline{e}nne à mes lèvres,
        \\déjà, Seigne\underline{u}r, tu le sais.
         
${}^{5}Tu me devances et me poursu\underline{i}s, tu m’enserres,
        \\tu as mis la m\underline{a}in sur moi.
${}^{6}Savoir prodigie\underline{u}x qui me dépasse,
        \\hauteur que je ne pu\underline{i}s atteindre !
         
${}^{7}Où donc aller, l\underline{o}in de ton souffle ?
        \\où m’enfuir, l\underline{o}in de ta face ?
${}^{8}Je gravis les cie\underline{u}x : tu es là ;
        \\je descends chez les m\underline{o}rts : te voici.
         
${}^{9}Je prends les \underline{a}iles de l’aurore
        \\et me pose au-del\underline{à} des mers :
${}^{10}même là, ta m\underline{a}in me conduit,
        \\ta main dr\underline{o}ite me saisit.
         
        *
         
${}^{11}J’avais dit : « Les tén\underline{è}bres m’écrasent ! »
        \\mais la nuit devient lumi\underline{è}re autour de moi.
${}^{12}Même la ténèbre pour t\underline{o}i n’est pas ténèbre,
        \\et la nuit comme le jo\underline{u}r est lumière !
         
${}^{13}C’est toi qui as cré\underline{é} mes reins,
        \\qui m’as tissé dans le s\underline{e}in de ma mère.
${}^{14}Je reconnais devant toi le prodige,
        l’être étonn\underline{a}nt que je suis : *
        \\étonnantes sont tes œuvres
        toute mon \underline{â}me le sait.
         
${}^{15}Mes os n’étaient pas cach\underline{é}s pour toi *
        \\quand j’étais façonné dans le secret,
        modelé aux entr\underline{a}illes de la terre.
         
${}^{16}J’étais encore inachev\underline{é}, tu me voyais ; *
        \\sur ton livre, tous mes jours étaient inscrits,
        recensés avant qu’un se\underline{u}l ne soit !
         
        *
         
${}^{17}Que tes pensées sont pour m\underline{o}i difficiles,
        \\Dieu, que leur s\underline{o}mme est imposante !
${}^{18}Je les compte : plus nombre\underline{u}ses que le sable !
        \\Je m’éveille : je suis enc\underline{o}re avec toi.
         
${}^{19}\[Dieu, si tu extermin\underline{a}is l’impie !
        \\Hommes de sang, éloignez-vo\underline{u}s de moi !
${}^{20}Tes adversaires prof\underline{a}nent ton nom :
        \\ils le pron\underline{o}ncent pour détruire.
         
${}^{21}Comment ne pas haïr tes ennem\underline{i}s, Seigneur,
        \\ne pas avoir en dégo\underline{û}t tes assaillants ?
${}^{22}Je les hais d’une h\underline{a}ine parfaite,
        \\je les tiens pour mes pr\underline{o}pres ennemis.\]
         
${}^{23}Scrute-moi, mon Dieu, tu saur\underline{a}s ma pensée ;
        \\éprouve-moi, tu connaîtr\underline{a}s mon cœur.
${}^{24}Vois si je prends le chem\underline{i}n des idoles,
        \\et conduis-moi sur le chem\underline{i}n d’éternité.
      \bchapter{Psaume}
          
            \bchapter{Psaume}
            Contre l’homme violent, défends-moi
${}^{1}Du maître de chœur. Psaume. De David.
         
${}^{2}Délivre-moi, Seigneur, de l’h\underline{o}mme mauvais,
        \\contre l’homme viol\underline{e}nt, défends-moi,
${}^{3}contre ceux qui préméd\underline{i}tent le mal
        \\et tout le jour entreti\underline{e}nnent la guerre,
${}^{4}qui dardent leur l\underline{a}ngue de vipère,
        \\leur langue charg\underline{é}e de venin.
         
${}^{5}Garde-moi, Seigneur, de la m\underline{a}in des impies,
        \\contre l’homme viol\underline{e}nt, défends-moi,
        \\contre ceux qui méd\underline{i}tent ma chute,
${}^{6}les arrogants qui m’ont tend\underline{u} des pièges ;
        \\sur mon passage ils ont m\underline{i}s un filet,
        \\ils ont dressé contre m\underline{o}i des embûches.
         
${}^{7}Je dis au Seigneur : « Mon Die\underline{u}, c’est toi ! »
        \\Seigneur, entends le cr\underline{i} de ma prière.
${}^{8}Tu es la force qui me sauve, M\underline{a}ître, Seigneur ;
        \\au jour du combat, tu prot\underline{è}ges ma tête.
${}^{9}Ne cède pas, Seigneur, au dés\underline{i}r des impies,
        \\ne permets pas que leurs intr\underline{i}gues réussissent !
         
${}^{10}\[Sur la tête de ce\underline{u}x qui m’encerclent,
        \\que retombe le p\underline{o}ids de leurs injures !
${}^{11}Que des braises ple\underline{u}vent sur eux !
        \\Qu’ils soient jetés à la fosse et jam\underline{a}is ne se relèvent !
${}^{12}L’insulteur ne tiendra p\underline{a}s sur la terre :
        \\le violent, le mauvais, sera traqu\underline{é} à mort.\]
         
${}^{13}Je le sais, le Seigneur rendra just\underline{i}ce au malheureux,
        \\il fera dr\underline{o}it au pauvre.
${}^{14}Oui, les justes rendront gr\underline{â}ce à ton nom,
        \\les hommes droits siéger\underline{o}nt en ta présence.
      \bchapter{Psaume}
          
            \bchapter{Psaume}
            L’offrande du soir
${}^{1}Psaume. De David.
         
        \\Seigneur, je t’appelle : acco\underline{u}rs vers moi !
        \\Écoute mon appel quand je cr\underline{i}e vers toi !
${}^{2}Que ma prière devant toi s’él\underline{è}ve comme un encens,
        \\et mes mains, comme l’offr\underline{a}nde du soir.
         
${}^{3}Mets une garde à mes l\underline{è}vres, Seigneur,
        \\veille au se\underline{u}il de ma bouche.
${}^{4}Ne laisse pas mon cœur pench\underline{e}r vers le mal
        \\ni devenir complice des h\underline{o}mmes malfaisants.
         
        \\Jamais je ne goûter\underline{a}i leurs plaisirs :
${}^{5}que le juste me reprenne et me corr\underline{i}ge avec bonté.
        \\Que leurs parfums, ni leurs poisons, ne to\underline{u}chent ma tête !
        \\Ils font du mal : je me ti\underline{e}ns en prière.
         
${}^{6}Voici leurs juges précipit\underline{é}s contre le roc,
        \\eux qui prenaient plais\underline{i}r à m’entendre dire :
${}^{7}« Comme un sol qu’on reto\underline{u}rne et défonce,
        \\nos os sont dispersés à la gue\underline{u}le des enfers ! »
         
${}^{8}Je regarde vers toi, Seigne\underline{u}r, mon Maître ;
        \\tu es mon refuge : ép\underline{a}rgne ma vie !
${}^{9}Garde-moi du fil\underline{e}t qui m’est tendu,
        \\des embûches qu’ont dress\underline{é}es les malfaisants.
         
${}^{10}\[Les impies tomber\underline{o}nt dans leur piège ;
        \\seul, m\underline{o}i, je passerai.\]
      \bchapter{Psaume}
          
            \bchapter{Psaume}
            Personne qui pense à moi
${}^{1}Poème. De David. Lorsqu’il était dans la caverne. Prière.
         
${}^{2}À pleine voix, je cr\underline{i}e vers le Seigneur !
        \\À pleine voix, je suppl\underline{i}e le Seigneur !
${}^{3}Je répands devant lu\underline{i} ma plainte,
        \\devant lui, je d\underline{i}s ma détresse.
         
${}^{4}Lorsque le so\underline{u}ffle me manque,
        toi, tu s\underline{a}is mon chemin. *
        \\Sur le senti\underline{e}r où j’avance,
        un pi\underline{è}ge m’est tendu.
         
${}^{5}Regarde à mes côt\underline{é}s, et vois :
        pers\underline{o}nne qui me connaisse ! *
        \\Pour moi, il n’est pl\underline{u}s de refuge :
        personne qui p\underline{e}nse à moi !
         
${}^{6}J’ai crié vers t\underline{o}i, Seigneur ! *
        \\J’ai dit : « Tu \underline{e}s mon abri,
        ma part, sur la t\underline{e}rre des vivants. »
         
${}^{7}Sois attent\underline{i}f à mes appels :
        je suis rédu\underline{i}t à rien ; *
        \\délivre-moi de ce\underline{u}x qui me poursuivent :
        ils sont plus f\underline{o}rts que moi.
         
${}^{8}Tire-moi de la pris\underline{o}n où je suis,
        que je rende gr\underline{â}ce à ton nom. *
        \\Autour de moi, les j\underline{u}stes feront cercle
        pour le bi\underline{e}n que tu m’as fait.
      \bchapter{Psaume}
          
            \bchapter{Psaume}
            Que ton souffle me guide
${}^{1}Psaume. De David.
         
        \\Seigneur, ent\underline{e}nds ma prière ; +
        \\dans ta justice éco\underline{u}te mes appels, *
        \\dans ta fidélit\underline{é} réponds-moi.
${}^{2}N’entre pas en jugement av\underline{e}c ton serviteur :
        \\aucun vivant n’est j\underline{u}ste devant toi.
         
${}^{3}L’ennemi ch\underline{e}rche ma perte,
        \\il foule au s\underline{o}l ma vie ;
        \\il me fait habit\underline{e}r les ténèbres
        \\avec les m\underline{o}rts de jadis.
${}^{4}Le souffle en m\underline{o}i s’épuise,
        \\mon cœur au fond de m\underline{o}i s’épouvante.
         
${}^{5}Je me souviens des jours d’autrefois,
        je me redis to\underline{u}tes tes actions, *
        \\sur l’œuvre de tes m\underline{a}ins je médite.
${}^{6}Je tends les m\underline{a}ins vers toi,
        \\me voici devant toi comme une t\underline{e}rre assoiffée.
         
${}^{7}Vite, réponds-m\underline{o}i, Seigneur :
        \\je suis à bo\underline{u}t de souffle !
        \\Ne me cache p\underline{a}s ton visage :
        \\je serais de ceux qui t\underline{o}mbent dans la fosse.
         
${}^{8}Fais que j’entende au mat\underline{i}n ton amour,
        \\car je c\underline{o}mpte sur toi.
        \\Montre-moi le chem\underline{i}n que je dois prendre :
        \\vers toi, j’él\underline{è}ve mon âme !
         
${}^{9}Délivre-moi de mes ennem\underline{i}s, Seigneur :
        \\j’ai un abr\underline{i} auprès de toi.
${}^{10}Apprends-moi à f\underline{a}ire ta volonté,
        \\car tu \underline{e}s mon Dieu.
        \\Ton so\underline{u}ffle est bienfaisant :
        \\qu’il me guide en un pa\underline{y}s de plaines.
         
${}^{11}Pour l’honneur de ton nom, Seigne\underline{u}r, fais-moi vivre ;
        \\à cause de ta justice, tire-m\underline{o}i de la détresse.
${}^{12}\[À cause de ton amour, tu détruir\underline{a}s mes ennemis ;
        \\tu feras périr mes adversaires, car je su\underline{i}s ton serviteur.\]
      \bchapter{Psaume}
          
            \bchapter{Psaume}
            Heureux qui a pour Dieu le Seigneur
${}^{1}De David.
         
        \\Béni soit le Seigne\underline{u}r, mon rocher ! +
        \\Il exerce mes m\underline{a}ins pour le combat, *
        \\il m’entr\underline{a}îne à la bataille.
         
${}^{2}Il est mon alli\underline{é}, ma forteresse,
        \\ma citadelle, celu\underline{i} qui me libère ;
        \\il est le boucli\underline{e}r qui m’abrite,
        \\il me donne pouv\underline{o}ir sur mon peuple.
         
${}^{3}Qu’est-ce que l’homme,
        pour que tu le conn\underline{a}isses, Seigneur, *
        \\le fils d’un homme, pour que tu c\underline{o}mptes avec lui ?
${}^{4}L’homme est sembl\underline{a}ble à un souffle,
        \\ses jours sont une \underline{o}mbre qui passe.
         
${}^{5}Seigneur, incline les cie\underline{u}x et descends ;
        \\touche les mont\underline{a}gnes : qu’elles brûlent !
${}^{6}Décoche des écl\underline{a}irs de tous côtés,
        \\tire des flèches et rép\underline{a}nds la terreur.
         
${}^{7}Des hauteurs, tends-moi la m\underline{a}in, délivre-moi, *
        \\sauve-moi du gouffre des eaux,
        de l’emprise d’un pe\underline{u}ple étranger :
${}^{8}il dit des par\underline{o}les mensongères,
        \\sa main est une m\underline{a}in parjure.
         
${}^{9}Pour toi, je chanter\underline{a}i un chant nouveau,
        \\pour toi, je jouerai sur la h\underline{a}rpe à dix cordes,
${}^{10}pour toi qui donnes aux r\underline{o}is la victoire
        \\et sauves de l’épée meurtrière Dav\underline{i}d, ton serviteur.
         
${}^{11}Délivre-m\underline{o}i, sauve-moi
        \\de l’emprise d’un pe\underline{u}ple étranger :
        \\il dit des par\underline{o}les mensongères,
        \\sa main est une m\underline{a}in parjure.
         
        *
         
${}^{12}Que nos fils soient par\underline{e}ils à des plants
        bien ven\underline{u}s dès leur jeune âge ; *
        \\nos filles, par\underline{e}illes à des colonnes
        sculpt\underline{é}es pour un palais !
         
${}^{13}Nos greniers, rempl\underline{i}s, débordants,
        regorger\underline{o}nt de biens ; *
        \\les troupeaux, par milli\underline{e}rs, par myriades,
        emplir\underline{o}nt nos campagnes !
         
${}^{14}Nos vassaux nous rester\underline{o}nt soumis,
        pl\underline{u}s de défaites ; *
        \\plus de br\underline{è}ches dans nos murs,
        plus d’al\underline{e}rtes sur nos places !
         
${}^{15}Heureux le pe\underline{u}ple ainsi comblé !
        \\Heureux le peuple qui a pour Die\underline{u} « Le Seigneur » !
      \bchapter{Psaume}
          
            \bchapter{Psaume}
            Tu rassasies avec bonté tout ce qui vit
${}^{1}Louange. De David.
         
        \\Je t’exalterai, mon Die\underline{u}, mon Roi,
        \\je bénirai ton nom toujo\underline{u}rs et à jamais !
         
${}^{2}Chaque jour je te b\underline{é}nirai,
        \\je louerai ton nom toujo\underline{u}rs et à jamais.
${}^{3}Il est grand, le Seigneur, hautem\underline{e}nt loué ;
        \\à sa grandeur, il n’est p\underline{a}s de limite.
         
${}^{4}D’âge en âge, on vanter\underline{a} tes œuvres,
        \\on proclamer\underline{a} tes exploits.
${}^{5}Je redirai le réc\underline{i}t de tes merveilles,
        \\ton éclat, ta gl\underline{o}ire et ta splendeur.
         
${}^{6}On dira ta f\underline{o}rce redoutable ;
        \\je raconter\underline{a}i ta grandeur.
${}^{7}On rappellera tes imm\underline{e}nses bontés ;
        \\tous acclamer\underline{o}nt ta justice.
         
${}^{8}Le Seigneur est tendr\underline{e}sse et pitié,
        \\lent à la col\underline{è}re et plein d’amour ;
${}^{9}la bonté du Seigne\underline{u}r est pour tous,
        \\sa tendresse, pour to\underline{u}tes ses œuvres.
         
${}^{10}Que tes œuvres, Seigne\underline{u}r, te rendent grâce
        \\et que tes fid\underline{è}les te bénissent !
${}^{11}Ils diront la gl\underline{o}ire de ton règne,
        \\ils parler\underline{o}nt de tes exploits,
         
${}^{12}annonçant aux h\underline{o}mmes tes exploits,
        \\la gloire et l’écl\underline{a}t de ton règne :
${}^{13}ton règne, un r\underline{è}gne éternel,
        \\ton empire, pour les \underline{â}ges des âges.
         
        \\Le Seigneur est vrai en to\underline{u}t ce qu’il dit,
        \\fidèle en to\underline{u}t ce qu’il fait.
${}^{14}Le Seigneur souti\underline{e}nt tous ceux qui tombent,
        \\il redresse to\underline{u}s les accablés.
         
${}^{15}Les yeux sur toi, to\underline{u}s, ils espèrent :
        \\tu leur donnes la nourrit\underline{u}re au temps voulu ;
${}^{16}tu o\underline{u}vres ta main :
        \\tu rassasies avec bont\underline{é} tout ce qui vit.
         
${}^{17}Le Seigneur est juste en to\underline{u}tes ses voies,
        \\fidèle en to\underline{u}t ce qu’il fait.
${}^{18}Il est proche de ce\underline{u}x qui l’invoquent,
        \\de tous ceux qui l’inv\underline{o}quent en vérité.
         
${}^{19}Il répond au désir de ce\underline{u}x qui le craignent ;
        \\il écoute leur cr\underline{i} : il les sauve.
${}^{20}Le Seigneur garder\underline{a} tous ceux qui l’aiment,
        \\mais il détruir\underline{a} tous les impies.
         
${}^{21}Que ma bouche proclame les lou\underline{a}nges du Seigneur ! *
        \\Son nom très saint, que toute chair le bénisse
        toujo\underline{u}rs et à jamais !
      \bchapter{Psaume}
          
            \bchapter{Psaume}
            Heureux qui s’appuie sur Dieu
${}^{1}Alléluia !
         
        \\Chante, ô mon âme, la lou\underline{a}nge du Seigneur ! +
${}^{2}Je veux louer le Seigne\underline{u}r tant que je vis, *
        \\chanter mes hymnes pour mon Die\underline{u} tant que je dure.
         
${}^{3}Ne comptez p\underline{a}s sur les puissants,
        \\des fils d’homme qui ne pe\underline{u}vent sauver !
${}^{4}Leur souffle s’en va : ils reto\underline{u}rnent à la terre ;
        \\et ce jour-là, pér\underline{i}ssent leurs projets.
         
${}^{5}Heureux qui s’appuie sur le Die\underline{u} de Jacob,
        \\qui met son espoir dans le Seigne\underline{u}r son Dieu,
${}^{6}lui qui a fait le ci\underline{e}l et la terre
        \\et la mer et to\underline{u}t ce qu’ils renferment !
         
        \\Il garde à jam\underline{a}is sa fidélité,
${}^{7}il fait just\underline{i}ce aux opprimés ;
        \\aux affamés, il d\underline{o}nne le pain ;
        \\le Seigneur dél\underline{i}e les enchaînés.
         
${}^{8}Le Seigneur ouvre les ye\underline{u}x des aveugles,
        \\le Seigneur redr\underline{e}sse les accablés,
        \\le Seigneur \underline{a}ime les justes,
${}^{9}le Seigneur prot\underline{è}ge l’étranger.
         
        \\Il soutient la ve\underline{u}ve et l’orphelin,
        \\il égare les p\underline{a}s du méchant.
${}^{10}D’âge en âge, le Seigne\underline{u}r régnera :
        \\ton Dieu, ô Si\underline{o}n, pour toujours !
      \bchapter{Psaume}
          
            \bchapter{Psaume}
            Le Seigneur élève les humbles
${}^{1}Alléluia !
         
        Il est bon de fêt\underline{e}r notre Dieu,
        il est beau de chant\underline{e}r sa louange !
         
${}^{2}Le Seigneur rebât\underline{i}t Jérusalem,
        \\il rassemble les déport\underline{é}s d’Israël ;
${}^{3}il guér\underline{i}t les cœurs brisés
        \\et s\underline{o}igne leurs blessures.
         
${}^{4}Il compte le n\underline{o}mbre des étoiles,
        \\il donne à chac\underline{u}ne un nom ;
${}^{5}il est grand, il est f\underline{o}rt, notre Maître :
        \\nul n’a mesur\underline{é} son intelligence.
${}^{6}Le Seigneur él\underline{è}ve les humbles
        \\et rabaisse jusqu’à t\underline{e}rre les impies.
         
${}^{7}Entonnez pour le Seigne\underline{u}r l’action de grâce,
        jouez pour notre Die\underline{u} sur la cithare !
         
${}^{8}Il couvre le ci\underline{e}l de nuages,
        \\il prépare la plu\underline{i}e pour la terre ;
        \\il fait germer l’h\underline{e}rbe sur les montagnes
        \\et les plantes pour l’us\underline{a}ge des hommes ;
${}^{9}il donne leur pât\underline{u}re aux troupeaux,
        \\aux petits du corbea\underline{u} qui la réclament.
         
${}^{10}La force des chevaux n’est p\underline{a}s ce qu’il aime,
        \\ni la vigueur des guerri\underline{e}rs, ce qui lui plaît ;
${}^{11}mais le Seigneur se plaît avec ce\underline{u}x qui le craignent,
        \\avec ceux qui esp\underline{è}rent son amour.
      \bchapter{Psaume}
          
            \bchapter{Psaume}
            Pas un peuple qu’il ait ainsi traité
${}^{12}Glorifie le Seigne\underline{u}r, Jérusalem !
        Célèbre ton Die\underline{u}, ô Sion !
         
${}^{13}Il a consolidé les b\underline{a}rres de tes portes,
        \\dans tes murs il a bén\underline{i} tes enfants ;
${}^{14}il fait régner la p\underline{a}ix à tes frontières,
        \\et d’un pain de from\underline{e}nt te rassasie.
         
${}^{15}Il envoie sa par\underline{o}le sur la terre :
        \\rapide, son v\underline{e}rbe la parcourt.
${}^{16}Il étale une tois\underline{o}n de neige,
        \\il sème une poussi\underline{è}re de givre.
         
${}^{17}Il jette à poign\underline{é}es des glaçons ;
        \\devant ce froid, qu\underline{i} pourrait tenir ?
${}^{18}Il envoie sa parole : survi\underline{e}nt le dégel ;
        \\il répand son so\underline{u}ffle : les eaux coulent.
         
${}^{19}Il révèle sa par\underline{o}le à Jacob,
        \\ses volontés et ses l\underline{o}is à Israël.
${}^{20}Pas un peuple qu’il ait ains\underline{i} traité ;
        \\nul autre n’a conn\underline{u} ses volontés.
         
        Alléluia !
      \bchapter{Psaume}
          
            Du haut des cieux, depuis la terre, louez-le
${}^{1}Alléluia !
         
        \\Louez le Seigneur du haut des cieux,
        \\louez-l\underline{e} dans les hauteurs.
${}^{2}Vous, tous ses \underline{a}nges, louez-le,
        \\louez-le, to\underline{u}s les univers.
         
${}^{3}Louez-le, sol\underline{e}il et lune,
        \\louez-le, tous les \underline{a}stres de lumière ;
${}^{4}vous, cieux des cie\underline{u}x, louez-le,
        \\et les eaux des haute\underline{u}rs des cieux.
         
${}^{5}Qu’ils louent le n\underline{o}m du Seigneur :
        \\sur son ordre ils f\underline{u}rent créés ;
${}^{6}c’est lui qui les pos\underline{a} pour toujours
        \\sous une loi qui ne p\underline{a}ssera pas.
         
        *
         
${}^{7}Louez le Seigne\underline{u}r depuis la terre,
        \\monstres mar\underline{i}ns, tous les abîmes ;
${}^{8}feu et grêle, n\underline{e}ige et brouillard,
        \\vent d’ouragan qui accompl\underline{i}s sa parole ;
         
${}^{9}les montagnes et to\underline{u}tes les collines,
        \\les arbres des verg\underline{e}rs, tous les cèdres ;
${}^{10}les bêtes sauvages et to\underline{u}s les troupeaux,
        \\le reptile et l’oisea\underline{u} qui vole ;
         
${}^{11}les rois de la t\underline{e}rre et tous les peuples,
        \\les princes et tous les j\underline{u}ges de la terre ;
${}^{12}tous les jeunes g\underline{e}ns et jeunes filles,
        \\les vieillards c\underline{o}mme les enfants.
         
        *
         
${}^{13}Qu’ils louent le n\underline{o}m du Seigneur,
        \\le seul au-dess\underline{u}s de tout nom ;
        \\sur le ciel et sur la t\underline{e}rre, sa splendeur :
${}^{14}il accroît la vigue\underline{u}r de son peuple.
         
        \\Louange de to\underline{u}s ses fidèles,
        \\des fils d’Israël, le pe\underline{u}ple de ses proches !
         
        Alléluia !
      \bchapter{Psaume}
          
            Louez-le dans l’assemblée de ses fidèles
${}^{1}Alléluia !
         
        \\Chantez au Seigne\underline{u}r un chant nouveau,
        \\louez-le dans l’assembl\underline{é}e de ses fidèles !
${}^{2}En Israël, j\underline{o}ie pour son créateur ;
        \\dans Sion, allégr\underline{e}sse pour son Roi !
${}^{3}Dansez à la lou\underline{a}nge de son nom,
        \\jouez pour lui, tambour\underline{i}ns et cithares !
         
${}^{4}Car le Seigneur \underline{a}ime son peuple,
        \\il donne aux humbles l’écl\underline{a}t de la victoire.
${}^{5}Que les fidèles ex\underline{u}ltent, glorieux,
        \\criant leur joie à l’he\underline{u}re du triomphe.
${}^{6}Qu’ils proclament les él\underline{o}ges de Dieu,
        \\tenant en main l’ép\underline{é}e à deux tranchants.
         
${}^{7}Tirer venge\underline{a}nce des nations,
        \\infliger aux pe\underline{u}ples un châtiment,
${}^{8}charger de ch\underline{a}înes les rois,
        \\jeter les pr\underline{i}nces dans les fers,
${}^{9}leurs appliquer la sent\underline{e}nce écrite,
        \\c’est la fiert\underline{é} de ses fidèles.
         
        Alléluia !
      \bchapter{Psaume}
          
            Que tout être vivant chante louange
${}^{1}Alléluia !
         
        \\Louez Dieu dans son t\underline{e}mple saint,
        \\louez-le au ci\underline{e}l de sa puissance ;
${}^{2}louez-le pour ses acti\underline{o}ns éclatantes,
        \\louez-le sel\underline{o}n sa grandeur !
         
${}^{3}Louez-le en sonn\underline{a}nt du cor,
        \\louez-le sur la h\underline{a}rpe et la cithare ;
${}^{4}louez-le par les c\underline{o}rdes et les flûtes,
        \\louez-le par la d\underline{a}nse et le tambour !
         
${}^{5}Louez-le par les cymb\underline{a}les sonores,
        \\louez-le par les cymb\underline{a}les triomphantes !
${}^{6}Et que tout \underline{ê}tre vivant
        \\chante lou\underline{a}nge au Seigneur !
         
        Alléluia !
