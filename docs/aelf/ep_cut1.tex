  
  
    
    \bbook{LETTRE AUX ÉPHÉSIENS}{LETTRE AUX ÉPHÉSIENS}
      <p class="cantique" id="bib_ct-nt_4"><span class="cantique_label">Cantique NT 4</span> = <span class="cantique_ref"><a class="unitex_link" href="#bib_ep_1_3">Ep 1, 3-10</a></span>
      
         
      \bchapter{}
        ${}^{1}Paul, apôtre du Christ Jésus
        par la volonté de Dieu,
        \\à ceux qui sont sanctifiés et habitent Éphèse,
        eux qui croient au Christ Jésus.
        ${}^{2}À vous, la grâce et la paix
        \\de la part de Dieu notre Père
        et du Seigneur Jésus Christ.
        
           
        ${}^{3}Béni soit Dieu, le Père
        \\de notre Seigneur Jésus Christ\\ !
         
        \\Il nous a bénis et comblés
        des bénédictions de l’Esprit,
        \\au ciel\\, dans le Christ.
         
        ${}^{4}Il nous a choisis, dans le Christ,
        avant la fondation du monde,
        \\pour que nous soyons saints, immaculés
        devant lui, dans l’amour.
         
        ${}^{5}Il nous a prédestinés
        à être, pour lui, des fils adoptifs
        \\par Jésus, le Christ.
         
        \\Ainsi l’a voulu sa bonté\\,
        ${}^{6}à la louange de gloire de sa grâce,
        \\la grâce qu’il nous donne
        dans le Fils\\bien-aimé.
         
        ${}^{7}En lui, par son sang,
        \\nous avons la rédemption,
        le pardon de nos fautes.
         
        \\C’est la richesse de la grâce
        ${}^{8}que Dieu\\a fait déborder jusqu’à nous
        \\en toute sagesse et intelligence.
         
        ${}^{9}Il nous dévoile\\ainsi le mystère de sa volonté,
        \\selon que sa bonté l’avait prévu dans le Christ :
         
        ${}^{10}pour mener les temps à leur plénitude\\,
        \\récapituler toutes choses dans le Christ,
        \\celles du ciel et celles de la terre.
         
        ${}^{11}En lui, nous sommes devenus
        le domaine particulier de Dieu,
        \\nous y avons été prédestinés
        selon le projet de celui qui réalise tout ce qu’il a décidé :
        <p class="verset_anchor" id="para_bib_ep_1_12">il a voulu 
${}^{12}que nous vivions
        à la louange de sa gloire,
        \\nous qui avons d’avance espéré dans le Christ.
         
        ${}^{13}En lui, vous aussi,
        \\après avoir écouté la parole de vérité,
        l’Évangile de votre salut,
        \\et après y avoir cru,
        \\vous avez reçu la marque de l’Esprit Saint.
         
        \\Et l’Esprit promis par Dieu
        ${}^{14}est une première avance sur notre héritage,
        \\en vue de la rédemption que nous obtiendrons,
        à la louange de sa gloire.
${}^{15}C’est pourquoi moi aussi, ayant entendu parler de la foi que vous avez dans le Seigneur Jésus, et de votre amour pour tous les fidèles, 
${}^{16}je ne cesse pas de rendre grâce, quand je fais mémoire de vous dans mes prières : 
${}^{17}que le Dieu de notre Seigneur Jésus Christ, le Père dans sa gloire, vous donne un esprit de sagesse qui vous le révèle et vous le fasse vraiment connaître. 
${}^{18}Qu’il ouvre à sa lumière les yeux de votre cœur, pour que vous sachiez quelle espérance vous ouvre son appel, la gloire sans prix de l’héritage que vous partagez avec les fidèles, 
${}^{19}et quelle puissance incomparable il déploie pour nous, les croyants : c’est l’énergie, la force, la vigueur 
${}^{20}qu’il a mise en œuvre dans le Christ quand il l’a ressuscité d’entre les morts et qu’il l’a fait asseoir à sa droite dans les cieux. 
${}^{21}Il l’a établi au-dessus de tout être céleste : Principauté, Souveraineté, Puissance et Domination, au-dessus de tout nom que l’on puisse nommer, non seulement dans le monde présent mais aussi dans le monde à venir. 
${}^{22}Il a tout mis sous ses pieds et, le plaçant plus haut que tout, il a fait de lui la tête de l’Église 
${}^{23}qui est son corps, et l’Église, c’est l’accomplissement total du Christ, lui que Dieu comble totalement de sa plénitude.
      
         
      \bchapter{}
      \begin{verse}
${}^{1}Et vous, vous étiez des morts, par suite des fautes et des péchés 
${}^{2}qui marquaient autrefois votre conduite, soumise aux forces mauvaises de ce monde, au prince du mal qui s’interpose entre le ciel et nous, et dont le souffle est maintenant à l’œuvre en ceux qui désobéissent à Dieu. 
${}^{3}Et nous aussi, nous étions tous de ceux-là, quand nous vivions suivant les convoitises de notre chair, cédant aux caprices de la chair et des pensées, nous qui étions, de par nous-mêmes, voués à la colère comme tous les autres.
${}^{4}Mais Dieu est riche en miséricorde ; à cause du grand amour dont il nous a aimés, 
${}^{5}nous qui étions des morts par suite de nos fautes, il nous a donné la vie avec le Christ : c’est bien par grâce que vous êtes sauvés. 
${}^{6}Avec lui, il nous a ressuscités et il nous a fait siéger aux cieux, dans le Christ Jésus. 
${}^{7}Il a voulu ainsi montrer, au long des âges futurs, la richesse surabondante de sa grâce, par sa bonté pour nous dans le Christ Jésus. 
${}^{8}C’est bien par la grâce que vous êtes sauvés, et par le moyen de la foi. Cela ne vient pas de vous, c’est le don de Dieu. 
${}^{9}Cela ne vient pas des actes : personne ne peut en tirer orgueil. 
${}^{10}C’est Dieu qui nous a faits, il nous a créés dans le Christ Jésus, en vue de la réalisation d’œuvres bonnes qu’il a préparées d’avance pour que nous les pratiquions.
${}^{11}Vous qui autrefois étiez païens, traités de « non-circoncis » par ceux qui se disent circoncis à cause d’une opération pratiquée dans la chair, souvenez-vous donc 
${}^{12}qu’en ce temps-là vous n’aviez pas le Christ, vous n’aviez pas droit de cité avec Israël, vous étiez étrangers aux alliances et à la promesse, vous n’aviez pas d’espérance et, dans le monde, vous étiez sans Dieu. 
${}^{13}Mais maintenant, dans le Christ Jésus, vous qui autrefois étiez loin, vous êtes devenus proches par le sang du Christ. 
${}^{14}C’est lui, le Christ, qui est notre paix : des deux, le Juif et le païen, il a fait une seule réalité ; par sa chair crucifiée, il a détruit ce qui les séparait, le mur de la haine ; 
${}^{15}il a supprimé les prescriptions juridiques de la loi de Moïse. Ainsi, à partir des deux, le Juif et le païen, il a voulu créer en lui un seul Homme nouveau en faisant la paix, 
${}^{16}et réconcilier avec Dieu les uns et les autres en un seul corps par le moyen de la croix ; en sa personne, il a tué la haine. 
${}^{17}Il est venu annoncer la bonne nouvelle de la paix, la paix pour vous qui étiez loin, la paix pour ceux qui étaient proches. 
${}^{18}Par lui, en effet, les uns et les autres, nous avons, dans un seul Esprit, accès auprès du Père.
${}^{19}Ainsi donc, vous n’êtes plus des étrangers ni des gens de passage, vous êtes concitoyens des saints, vous êtes membres de la famille de Dieu, 
${}^{20}car vous avez été intégrés dans la construction qui a pour fondations les Apôtres et les prophètes ; et la pierre angulaire, c’est le Christ Jésus lui-même. 
${}^{21}En lui, toute la construction s’élève harmonieusement pour devenir un temple saint dans le Seigneur. 
${}^{22}En lui, vous êtes, vous aussi, les éléments d’une même construction pour devenir une demeure de Dieu par l’Esprit Saint.
      
         
      \bchapter{}
      \begin{verse}
${}^{1}Moi, Paul, qui suis en prison à cause du Christ Jésus, je le suis pour vous qui venez des nations païennes. 
${}^{2}Vous avez appris, je pense, en quoi consiste la grâce que Dieu m’a donnée pour vous : 
${}^{3}par révélation, il m’a fait connaître le mystère, comme je vous l’ai déjà écrit brièvement. 
${}^{4}En me lisant, vous pouvez vous rendre compte de l’intelligence que j’ai du mystère du Christ. 
${}^{5}Ce mystère n’avait pas été porté à la connaissance des hommes des générations passées, comme il a été révélé maintenant à ses saints Apôtres et aux prophètes, dans l’Esprit. 
${}^{6}Ce mystère, c’est que toutes les nations sont associées au même héritage, au même corps, au partage de la même promesse, dans le Christ Jésus, par l’annonce de l’Évangile. 
${}^{7}De cet Évangile je suis devenu ministre par le don de la grâce que Dieu m’a accordée par l’énergie de sa puissance.
${}^{8}À moi qui suis vraiment le plus petit de tous les fidèles, la grâce a été donnée d’annoncer aux nations l’insondable richesse du Christ, 
${}^{9}et de mettre en lumière pour tous le contenu du mystère qui était caché depuis toujours en Dieu, le créateur de toutes choses ; 
${}^{10}ainsi, désormais, les Puissances célestes elles-mêmes connaissent, grâce à l’Église, les multiples aspects de la Sagesse de Dieu. 
${}^{11}C’est le projet éternel que Dieu a réalisé dans le Christ Jésus notre Seigneur. 
${}^{12}Et notre foi au Christ nous donne l’assurance nécessaire pour accéder auprès de Dieu en toute confiance. 
${}^{13}Aussi, je vous demande de ne pas vous décourager devant les épreuves que j’endure pour vous : elles sont votre gloire.
${}^{14}C’est pourquoi je tombe à genoux devant le Père, 
${}^{15}de qui toute paternité au ciel et sur la terre tient son nom. 
${}^{16}Lui qui est si riche en gloire, qu’il vous donne la puissance de son Esprit, pour que se fortifie en vous l’homme intérieur. 
${}^{17}Que le Christ habite en vos cœurs par la foi ; restez enracinés dans l'amour, établis dans l'amour. 
${}^{18}Ainsi vous serez capables de comprendre avec tous les fidèles quelle est la largeur, la longueur, la hauteur, la profondeur… 
${}^{19}Vous connaîtrez ce qui dépasse toute connaissance : l’amour du Christ. Alors vous serez comblés jusqu’à entrer dans toute la plénitude de Dieu.
${}^{20}À Celui qui peut réaliser, par la puissance qu’il met à l’œuvre en nous, infiniment plus que nous ne pouvons demander ou même concevoir, 
${}^{21}gloire à lui dans l’Église et dans le Christ Jésus pour toutes les générations dans les siècles des siècles. Amen.
      
         
      \bchapter{}
      \begin{verse}
${}^{1}Moi qui suis en prison à cause du Seigneur, je vous exhorte donc à vous conduire d’une manière digne de votre vocation : 
${}^{2}ayez beaucoup d’humilité, de douceur et de patience, supportez-vous les uns les autres avec amour ; 
${}^{3}ayez soin de garder l’unité dans l’Esprit par le lien de la paix. 
${}^{4}Comme votre vocation vous a tous appelés à une seule espérance, de même il y a un seul Corps et un seul Esprit. 
${}^{5}Il y a un seul Seigneur, une seule foi, un seul baptême, 
${}^{6}un seul Dieu et Père de tous, au-dessus de tous, par tous, et en tous.
${}^{7}À chacun d’entre nous, la grâce a été donnée selon la mesure du don fait par le Christ. 
${}^{8}C’est pourquoi l’Écriture dit :
        \\Il est monté sur la hauteur, il a capturé des captifs,
        \\il a fait des dons aux hommes.
${}^{9}Que veut dire : Il est monté ? – Cela veut dire qu’il était d’abord descendu dans les régions inférieures de la terre. 
${}^{10}Et celui qui était descendu est le même qui est monté au-dessus de tous les cieux pour remplir l’univers. 
${}^{11}Et les dons qu’il a faits, ce sont les Apôtres, et aussi les prophètes, les évangélisateurs, les pasteurs et ceux qui enseignent. 
${}^{12}De cette manière, les fidèles sont organisés pour que les tâches du ministère soient accomplies et que se construise le corps du Christ, 
${}^{13}jusqu’à ce que nous parvenions tous ensemble à l’unité dans la foi et la pleine connaissance du Fils de Dieu, à l’état de l’Homme parfait, à la stature du Christ dans sa plénitude. 
${}^{14}Alors, nous ne serons plus comme des petits enfants, nous laissant secouer et mener à la dérive par tous les courants d’idées, au gré des hommes qui emploient la ruse pour nous entraîner dans l’erreur. 
${}^{15}Au contraire, en vivant dans la vérité de l’amour, nous grandirons pour nous élever en tout jusqu’à celui qui est la Tête, le Christ. 
${}^{16}Et par lui, dans l’harmonie et la cohésion, tout le corps poursuit sa croissance, grâce aux articulations qui le maintiennent, selon l’énergie qui est à la mesure de chaque membre. Ainsi le corps se construit dans l’amour.
${}^{17}Je vous le dis, j’en témoigne dans le Seigneur : vous ne devez plus vous conduire comme les païens qui se laissent guider par le néant de leur pensée. 
${}^{18}Ils ont l’intelligence remplie de ténèbres, ils sont étrangers à la vie de Dieu, à cause de l’ignorance qui est en eux, à cause de l’endurcissement de leur cœur ; 
${}^{19}ayant perdu le sens moral, ils se sont livrés à la débauche au point de s’adonner sans retenue à toute sorte d’impureté.
${}^{20}Mais vous, ce n’est pas ainsi que l’on vous a appris à connaître le Christ, 
${}^{21}si du moins l’annonce et l’enseignement que vous avez reçus à son sujet s’accordent à la vérité qui est en Jésus. 
${}^{22}Il s’agit de vous défaire de votre conduite d’autrefois, c’est-à-dire de l’homme ancien corrompu par les convoitises qui l’entraînent dans l’erreur. 
${}^{23}Laissez-vous renouveler par la transformation spirituelle de votre pensée. 
${}^{24}Revêtez-vous de l’homme nouveau, créé, selon Dieu, dans la justice et la sainteté conformes à la vérité.
${}^{25}Débarrassez-vous donc du mensonge, et dites la vérité, chacun à son prochain, parce que nous sommes membres les uns des autres. 
${}^{26}Si vous êtes en colère, ne tombez pas dans le péché ; que le soleil ne se couche pas sur votre colère. 
${}^{27}Ne donnez pas prise au diable. 
${}^{28}Que le voleur cesse de voler ; qu’il prenne plutôt la peine de travailler honnêtement de ses mains, afin d’avoir de quoi partager avec celui qui est dans le besoin. 
${}^{29}Aucune parole mauvaise ne doit sortir de votre bouche ; mais, s’il en est besoin, que ce soit une parole bonne et constructive, profitable à ceux qui vous écoutent. 
${}^{30}N’attristez pas le Saint Esprit de Dieu, qui vous a marqués de son sceau en vue du jour de votre délivrance. 
${}^{31}Amertume, irritation, colère, éclats de voix ou insultes, tout cela doit être éliminé de votre vie, ainsi que toute espèce de méchanceté. 
${}^{32}Soyez entre vous pleins de générosité et de tendresse. Pardonnez-vous les uns aux autres, comme Dieu vous a pardonné dans le Christ.
      
         
      \bchapter{}
      \begin{verse}
${}^{1}Oui, cherchez à imiter Dieu, puisque vous êtes ses enfants bien-aimés. 
${}^{2}Vivez dans l’amour, comme le Christ nous a aimés et s’est livré lui-même pour nous, s’offrant en sacrifice à Dieu, comme un parfum d’agréable odeur.
${}^{3}Comme il convient aux fidèles la débauche, l’impureté sous toutes ses formes et la soif de posséder sont des choses qu’on ne doit même plus évoquer chez vous ; 
${}^{4}pas davantage de propos grossiers, stupides ou scabreux – tout cela est déplacé – mais qu’il y ait plutôt des actions de grâce. 
${}^{5}Sachez-le bien : ni les débauchés, ni les dépravés, ni les profiteurs – qui sont de vrais idolâtres – ne reçoivent d’héritage dans le royaume du Christ et de Dieu ; 
${}^{6}ne laissez personne vous égarer par de vaines paroles. Tout cela attire la colère de Dieu sur ceux qui désobéissent. 
${}^{7}N’ayez donc rien de commun avec ces gens-là. 
${}^{8}Autrefois, vous étiez ténèbres ; maintenant, dans le Seigneur, vous êtes lumière ; conduisez-vous comme des enfants de lumière – 
${}^{9}or la lumière a pour fruit tout ce qui est bonté, justice et vérité – 
${}^{10}et sachez reconnaître ce qui est capable de plaire au Seigneur. 
${}^{11}Ne prenez aucune part aux activités des ténèbres, elles ne produisent rien de bon ; démasquez-les plutôt. 
${}^{12}Ce que ces gens-là font en cachette, on a honte même d’en parler. 
${}^{13}Mais tout ce qui est démasqué est rendu manifeste par la lumière, 
${}^{14}et tout ce qui devient manifeste est lumière. C’est pourquoi l’on dit :
        \\Réveille-toi, ô toi qui dors,
        \\relève-toi d’entre les morts,
        \\et le Christ t’illuminera.
${}^{15}Prenez bien garde à votre conduite : ne vivez pas comme des fous, mais comme des sages. 
${}^{16}Tirez parti du temps présent, car nous traversons des jours mauvais. 
${}^{17}Ne soyez donc pas insensés, mais comprenez bien quelle est la volonté du Seigneur. 
${}^{18}Ne vous enivrez pas de vin, car il porte à l’inconduite ; soyez plutôt remplis de l’Esprit Saint. 
${}^{19}Dites entre vous des psaumes, des hymnes et des chants inspirés, chantez le Seigneur et célébrez-le de tout votre cœur. 
${}^{20}À tout moment et pour toutes choses, au nom de notre Seigneur Jésus Christ, rendez grâce à Dieu le Père.
${}^{21}Par respect pour le Christ, soyez soumis les uns aux autres ; 
${}^{22}les femmes, à leur mari, comme au Seigneur Jésus ; 
${}^{23}car, pour la femme, le mari est la tête, tout comme, pour l’Église, le Christ est la tête, lui qui est le Sauveur de son corps. 
${}^{24}Eh bien ! puisque l’Église se soumet au Christ, qu’il en soit toujours de même pour les femmes à l’égard de leur mari.
${}^{25}Vous, les hommes, aimez votre femme à l’exemple du Christ : il a aimé l’Église, il s’est livré lui-même pour elle, 
${}^{26}afin de la rendre sainte en la purifiant par le bain de l’eau baptismale, accompagné d’une parole ; 
${}^{27}il voulait se la présenter à lui-même, cette Église, resplendissante, sans tache, ni ride, ni rien de tel ; il la voulait sainte et immaculée. 
${}^{28}C’est de la même façon que les maris doivent aimer leur femme : comme leur propre corps. Celui qui aime sa femme s’aime soi-même. 
${}^{29}Jamais personne n’a méprisé son propre corps : au contraire, on le nourrit, on en prend soin. C’est ce que fait le Christ pour l’Église, 
${}^{30}parce que nous sommes les membres de son corps. Comme dit l’Écriture : 
${}^{31}À cause de cela, l’homme quittera son père et sa mère, il s’attachera à sa femme, et tous deux ne feront plus qu’un. 
${}^{32}Ce mystère est grand : je le dis en référence au Christ et à l’Église. 
${}^{33}Pour en revenir à vous, chacun doit aimer sa propre femme comme lui-même, et la femme doit avoir du respect pour son mari.
      
         
      \bchapter{}
      \begin{verse}
${}^{1}Vous, les enfants, obéissez à vos parents dans le Seigneur, car c’est cela qui est juste : 
${}^{2}Honore ton père et ta mère, c’est le premier commandement qui soit assorti d’une promesse : 
${}^{3}ainsi tu seras heureux et tu auras longue vie sur la terre.
${}^{4}Et vous, les parents, ne poussez pas vos enfants à la colère, mais élevez-les en leur donnant une éducation et des avertissements inspirés par le Seigneur.
${}^{5}Vous, les esclaves, obéissez à vos maîtres d’ici-bas comme au Christ, avec crainte et profond respect, dans la simplicité de votre cœur. 
${}^{6}Ne le faites pas seulement sous leurs yeux, par souci de plaire à des hommes, mais comme des esclaves du Christ qui accomplissent la volonté de Dieu de tout leur cœur, 
${}^{7}et qui font leur travail d’esclaves volontiers, comme pour le Seigneur et non pas pour des hommes. 
${}^{8}Car vous savez bien que chacun, qu’il soit esclave ou libre, sera rétribué par le Seigneur selon le bien qu’il aura fait. 
${}^{9}Et vous, les maîtres, agissez de même avec vos esclaves, laissez de côté les menaces. Car vous savez bien que, pour eux comme pour vous, le Maître est dans le ciel, et il est impartial envers les personnes.
${}^{10}Enfin, puisez votre énergie dans le Seigneur et dans la vigueur de sa force. 
${}^{11}Revêtez l’équipement de combat donné par Dieu, afin de pouvoir tenir contre les manœuvres du diable. 
${}^{12}Car nous ne luttons pas contre des êtres de sang et de chair, mais contre les Dominateurs de ce monde de ténèbres, les Principautés, les Souverainetés, les esprits du mal qui sont dans les régions célestes. 
${}^{13}Pour cela, prenez l’équipement de combat donné par Dieu ; ainsi, vous pourrez résister quand viendra le jour du malheur, et tout mettre en œuvre pour tenir bon. 
${}^{14}Oui, tenez bon, ayant autour des reins le ceinturon de la vérité, portant la cuirasse de la justice, 
${}^{15}les pieds chaussés de l’ardeur à annoncer l’Évangile de la paix, 
${}^{16}et ne quittant jamais le bouclier de la foi, qui vous permettra d’éteindre toutes les flèches enflammées du Mauvais. 
${}^{17}Prenez le casque du salut et le glaive de l’Esprit, c’est-à-dire la parole de Dieu.
${}^{18}En toute circonstance, que l’Esprit vous donne de prier et de supplier : restez éveillés, soyez assidus à la supplication pour tous les fidèles. 
${}^{19}Priez aussi pour moi : qu’une parole juste me soit donnée quand j’ouvre la bouche pour faire connaître avec assurance le mystère de l’Évangile 
${}^{20}dont je suis l’ambassadeur, dans mes chaînes. Priez donc afin que je trouve dans l’Évangile pleine assurance pour parler comme je le dois.
${}^{21}Et vous, vous saurez ce que je deviens et ce que je fais, car Tychique, le frère bien-aimé, le fidèle ministre dans le Seigneur, vous informera de tout. 
${}^{22}Je l’envoie spécialement auprès de vous, afin que vous ayez de nos nouvelles et qu’il réconforte vos cœurs.
${}^{23}Que la paix soit avec les frères, ainsi que l’amour et la foi, de la part de Dieu le Père et du Seigneur Jésus Christ.
${}^{24}Que la grâce soit avec tous ceux qui aiment notre Seigneur Jésus Christ d’un amour impérissable.
