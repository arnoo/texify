  
  
    
    \bbook{ACTES DES APÔTRES}{ACTES DES APÔTRES}
      
         
      \bchapter{}
      \begin{verse}
${}^{1}Cher Théophile, dans mon premier livre, j’ai parlé de tout ce que Jésus a fait et enseigné, depuis le moment où il commença, 
${}^{2}jusqu’au jour où il fut enlevé au ciel, après avoir, par l’Esprit Saint, donné ses instructions aux Apôtres qu’il avait choisis. 
${}^{3}C’est à eux qu’il s’est présenté vivant après sa Passion ; il leur en a donné bien des preuves, puisque, pendant quarante jours, il leur est apparu et leur a parlé du royaume de Dieu.
      
         
${}^{4}Au cours d’un repas qu’il prenait avec eux, il leur donna l’ordre de ne pas quitter Jérusalem, mais d’y attendre que s’accomplisse la promesse du Père. Il déclara : « Cette promesse, vous l’avez entendue de ma bouche : 
${}^{5}alors que Jean a baptisé avec l’eau, vous, c’est dans l’Esprit Saint que vous serez baptisés d’ici peu de jours. » 
${}^{6}Ainsi réunis, les Apôtres l’interrogeaient : « Seigneur, est-ce maintenant le temps où tu vas rétablir le royaume pour Israël ? » 
${}^{7}Jésus leur répondit : « Il ne vous appartient pas de connaître les temps et les moments que le Père a fixés de sa propre autorité. 
${}^{8}Mais vous allez recevoir une force quand le Saint-Esprit viendra sur vous ; vous serez alors mes témoins à Jérusalem, dans toute la Judée et la Samarie, et jusqu’aux extrémités de la terre. »
${}^{9}Après ces paroles, tandis que les Apôtres le regardaient, il s’éleva, et une nuée vint le soustraire à leurs yeux. 
${}^{10}Et comme ils fixaient encore le ciel où Jésus s’en allait, voici que, devant eux, se tenaient deux hommes en vêtements blancs, 
${}^{11}qui leur dirent : « Galiléens, pourquoi restez-vous là à regarder vers le ciel ? Ce Jésus qui a été enlevé au ciel d’auprès de vous, viendra de la même manière que vous l’avez vu s’en aller vers le ciel. »
${}^{12}Alors, ils retournèrent à Jérusalem depuis le lieu-dit « mont des Oliviers » qui en est proche, – la distance de marche ne dépasse pas ce qui est permis le jour du sabbat. 
${}^{13}À leur arrivée, ils montèrent dans la chambre haute où ils se tenaient habituellement ; c’était Pierre, Jean, Jacques et André, Philippe et Thomas, Barthélemy et Matthieu, Jacques fils d’Alphée, Simon le Zélote, et Jude fils de Jacques. 
${}^{14}Tous, d’un même cœur, étaient assidus à la prière, avec des femmes, avec Marie la mère de Jésus, et avec ses frères.
${}^{15}En ces jours-là, Pierre se leva au milieu des frères qui étaient réunis au nombre d’environ cent vingt personnes, et il déclara : 
${}^{16}« Frères, il fallait que l’Écriture s’accomplisse. En effet, par la bouche de David, l’Esprit Saint avait d’avance parlé de Judas, qui en est venu à servir de guide aux gens qui ont arrêté Jésus : 
${}^{17}ce Judas était l’un de nous et avait reçu sa part de notre ministère ; 
${}^{18}puis, avec le salaire de l’injustice, il acheta un domaine ; il tomba la tête la première, son ventre éclata, et toutes ses entrailles se répandirent. 
${}^{19}Tous les habitants de Jérusalem en furent informés, si bien que ce domaine fut appelé dans leur propre dialecte Hakeldama, c’est-à-dire Domaine-du-Sang. 
${}^{20}Car il est écrit au livre des Psaumes :
      Que son domainedevienne un désert,
      et que personne n’y habite,
      <p class="retrait0">et encore :
      Qu’un autre prenne sa charge.
${}^{21}Or, il y a des hommes qui nous ont accompagnés durant tout le temps où le Seigneur Jésus a vécu parmi nous, 
${}^{22}depuis le commencement, lors du baptême donné par Jean, jusqu’au jour où il fut enlevé d’auprès de nous. Il faut donc que l’un d’entre eux devienne, avec nous, témoin de sa résurrection. » 
${}^{23}On en présenta deux : Joseph appelé Barsabbas, puis surnommé Justus, et Matthias. 
${}^{24}Ensuite, on fit cette prière : « Toi, Seigneur, qui connais tous les cœurs, désigne lequel des deux tu as choisi 
${}^{25}pour qu’il prenne, dans le ministère apostolique, la place que Judas a désertée en allant à la place qui est désormais la sienne. » 
${}^{26}On tira au sort entre eux, et le sort tomba sur Matthias, qui fut donc associé par suffrage aux onze Apôtres.
      
         
      \bchapter{}
      \begin{verse}
${}^{1}Quand arriva le jour de la Pentecôte, au terme des cinquante jours, ils se trouvaient réunis tous ensemble. 
${}^{2}Soudain un bruit survint du ciel comme un violent coup de vent : la maison où ils étaient assis en fut remplie tout entière. 
${}^{3}Alors leur apparurent des langues qu’on aurait dites de feu, qui se partageaient, et il s’en posa une sur chacun d’eux. 
${}^{4}Tous furent remplis d’Esprit Saint : ils se mirent à parler en d’autres langues, et chacun s’exprimait selon le don de l’Esprit.
${}^{5}Or, il y avait, résidant à Jérusalem, des Juifs religieux, venant de toutes les nations sous le ciel. 
${}^{6}Lorsque ceux-ci entendirent la voix qui retentissait, ils se rassemblèrent en foule. Ils étaient en pleine confusion parce que chacun d’eux entendait dans son propre dialecte ceux qui parlaient. 
${}^{7}Dans la stupéfaction et l’émerveillement, ils disaient : « Ces gens qui parlent ne sont-ils pas tous Galiléens ? 
${}^{8}Comment se fait-il que chacun de nous les entende dans son propre dialecte, sa langue maternelle ? 
${}^{9}Parthes, Mèdes et Élamites, habitants de la Mésopotamie, de la Judée et de la Cappadoce, de la province du Pont et de celle d’Asie, 
${}^{10}de la Phrygie et de la Pamphylie, de l’Égypte et des contrées de Libye proches de Cyrène, Romains de passage, 
${}^{11}Juifs de naissance et convertis, Crétois et Arabes, tous nous les entendons parler dans nos langues des merveilles de Dieu. » 
${}^{12}Ils étaient tous dans la stupéfaction et la perplexité, se disant l’un à l’autre : « Qu’est-ce que cela signifie ? » 
${}^{13}D’autres se moquaient et disaient : « Ils sont pleins de vin doux ! »
${}^{14}Alors Pierre, debout avec les onze autres Apôtres, éleva la voix et leur fit cette déclaration : « Vous, Juifs, et vous tous qui résidez à Jérusalem, sachez bien ceci, prêtez l’oreille à mes paroles. 
${}^{15}Non, ces gens-là ne sont pas ivres comme vous le supposez, car c’est seulement la troisième heure du jour. 
${}^{16}Mais ce qui arrive a été annoncé par le prophète Joël :
${}^{17}Il arrivera dans les derniers jours, dit Dieu,
        \\que je répandrai mon Esprit sur toute créature :
        \\vos fils et vos filles prophétiseront,
        \\vos jeunes gens auront des visions,
        \\et vos anciens auront des songes.
${}^{18}Même sur mes serviteurs et sur mes servantes,
        \\je répandrai mon Esprit en ces jours-là,
        \\et ils prophétiseront.
${}^{19}Je ferai des prodiges en haut dans le ciel,
        \\et des signes en bas sur la terre :
        \\du sang, du feu, un nuagede fumée.
${}^{20}Le soleil sera changé en ténèbres,
        \\et la lune sera changée en sang,
        \\avant que vienne le jour du Seigneur,
        \\jour grand et manifeste.
${}^{21}Alors, quiconque invoquera le nom du Seigneur
        \\sera sauvé.
${}^{22}Hommes d’Israël, écoutez les paroles que voici. Il s’agit de Jésus le Nazaréen, homme que Dieu a accrédité auprès de vous en accomplissant par lui des miracles, des prodiges et des signes au milieu de vous, comme vous le savez vous-mêmes. 
${}^{23}Cet homme, livré selon le dessein bien arrêté et la prescience de Dieu, vous l’avez supprimé en le clouant sur le bois par la main des impies. 
${}^{24}Mais Dieu l’a ressuscité en le délivrant des douleurs de la mort, car il n’était pas possible qu’elle le retienne en son pouvoir. 
${}^{25}En effet, c’est de lui que parle David dans le psaume :
        \\Je voyais le Seigneur devant moi sans relâche :
        \\il est à ma droite, je suis inébranlable.
        ${}^{26}C’est pourquoi mon cœur est en fête,
        \\et ma langue exulte de joie ;
        \\ma chair elle-même reposera dans l’espérance :
        ${}^{27}tu ne peux m’abandonner au séjour des morts
        \\ni laisser ton fidèle voir la corruption.
        ${}^{28}Tu m’as appris des chemins de vie,
        \\tu me rempliras d’allégresse par ta présence.
${}^{29}Frères, il est permis de vous dire avec assurance, au sujet du patriarche David, qu’il est mort, qu’il a été enseveli, et que son tombeau est encore aujourd’hui chez nous. 
${}^{30}Comme il était prophète, il savait que Dieu lui avait juré de faire asseoir sur son trône un homme issu de lui. 
${}^{31}Il a vu d’avance la résurrection du Christ, dont il a parlé ainsi : Il n’a pas été abandonné à la mort, et sa chair n’a pas vu la corruption. 
${}^{32}Ce Jésus, Dieu l’a ressuscité ; nous tous, nous en sommes témoins. 
${}^{33}Élevé par la droite de Dieu, il a reçu du Père l’Esprit Saint qui était promis, et il l’a répandu sur nous, ainsi que vous le voyez et l’entendez. 
${}^{34}David, en effet, n’est pas monté au ciel, bien qu’il dise lui-même :
        \\Le Seigneur a dit à mon Seigneur :
        “Siège à ma droite,
${}^{35}jusqu’à ce que j’aie placé tes ennemis
        comme un escabeau sous tes pieds.”
${}^{36}Que toute la maison d’Israël le sache donc avec certitude : Dieu l’a fait Seigneur et Christ, ce Jésus que vous aviez crucifié. »
${}^{37}Les auditeurs furent touchés au cœur ; ils dirent à Pierre et aux autres Apôtres : « Frères, que devons-nous faire ? » 
${}^{38}Pierre leur répondit : « Convertissez-vous, et que chacun de vous soit baptisé au nom de Jésus Christ pour le pardon de ses péchés ; vous recevrez alors le don du Saint-Esprit. 
${}^{39}Car la promesse est pour vous, pour vos enfants et pour tous ceux qui sont loin, aussi nombreux que le Seigneur notre Dieu les appellera. » 
${}^{40}Par bien d’autres paroles encore, Pierre les adjurait et les exhortait en disant : « Détournez-vous de cette génération tortueuse, et vous serez sauvés. » 
${}^{41}Alors, ceux qui avaient accueilli la parole de Pierre furent baptisés. Ce jour-là, environ trois mille personnes se joignirent à eux.
${}^{42}Ils étaient assidus à l’enseignement des Apôtres et à la communion fraternelle, à la fraction du pain et aux prières. 
${}^{43}La crainte de Dieu était dans tous les cœurs à la vue des nombreux prodiges et signes accomplis par les Apôtres.
${}^{44}Tous les croyants vivaient ensemble, et ils avaient tout en commun ; 
${}^{45}ils vendaient leurs biens et leurs possessions, et ils en partageaient le produit entre tous en fonction des besoins de chacun.
${}^{46}Chaque jour, d’un même cœur, ils fréquentaient assidûment le Temple, ils rompaient le pain dans les maisons, ils prenaient leurs repas avec allégresse et simplicité de cœur ; 
${}^{47}ils louaient Dieu et avaient la faveur du peuple tout entier. Chaque jour, le Seigneur leur adjoignait ceux qui allaient être sauvés.
      
         
      \bchapter{}
      \begin{verse}
${}^{1}Pierre et Jean montaient au Temple pour la prière de l’après-midi, à la neuvième heure. 
${}^{2}On y amenait alors un homme, infirme de naissance, que l’on installait chaque jour à la porte du Temple, appelée la « Belle-Porte », pour qu’il demande l’aumône à ceux qui entraient. 
${}^{3}Voyant Pierre et Jean qui allaient entrer dans le Temple, il leur demanda l’aumône. 
${}^{4}Alors Pierre, ainsi que Jean, fixa les yeux sur lui, et il dit : « Regarde-nous ! » 
${}^{5}L’homme les observait, s’attendant à recevoir quelque chose de leur part. 
${}^{6}Pierre déclara : « De l’argent et de l’or, je n’en ai pas ; mais ce que j’ai, je te le donne : au nom de Jésus Christ le Nazaréen, lève-toi et marche. » 
${}^{7}Alors, le prenant par la main droite, il le releva et, à l’instant même, ses pieds et ses chevilles s’affermirent. 
${}^{8}D’un bond, il fut debout et il marchait. Entrant avec eux dans le Temple, il marchait, bondissait, et louait Dieu. 
${}^{9}Et tout le peuple le vit marcher et louer Dieu. 
${}^{10}On le reconnaissait : c’est bien lui qui était assis à la « Belle-Porte » du Temple pour demander l’aumône. Et les gens étaient frappés de stupeur et désorientés devant ce qui lui était arrivé.
      
         
${}^{11}L’homme ne lâchait plus Pierre et Jean. Tout le peuple accourut vers eux au Portique dit de Salomon. Les gens étaient stupéfaits. 
${}^{12}Voyant cela, Pierre interpella le peuple : « Hommes d’Israël, pourquoi vous étonner ? Pourquoi fixer les yeux sur nous, comme si c’était en vertu de notre puissance personnelle ou de notre piété que nous lui avons donné de marcher ? 
${}^{13}Le Dieu d’Abraham, d’Isaac et de Jacob, le Dieu de nos pères, a glorifié son serviteur Jésus, alors que vous, vous l’aviez livré, vous l’aviez renié en présence de Pilate qui était décidé à le relâcher. 
${}^{14}Vous avez renié le Saint et le Juste, et vous avez demandé qu’on vous accorde la grâce d’un meurtrier. 
${}^{15}Vous avez tué le Prince de la vie, lui que Dieu a ressuscité d’entre les morts, nous en sommes témoins. 
${}^{16}Tout repose sur la foi dans le nom de Jésus Christ : c’est ce nom lui-même qui vient d’affermir cet homme que vous regardez et connaissez ; oui, la foi qui vient par Jésus l’a rétabli dans son intégrité physique, en votre présence à tous.
${}^{17}D’ailleurs, frères, je sais bien que vous avez agi dans l’ignorance, vous et vos chefs. 
${}^{18}Mais Dieu a ainsi accompli ce qu’il avait d’avance annoncé par la bouche de tous les prophètes : que le Christ, son Messie, souffrirait. 
${}^{19}Convertissez-vous donc et tournez-vous vers Dieu pour que vos péchés soient effacés. 
${}^{20}Ainsi viendront les temps de la fraîcheur de la part du Seigneur, et il enverra le Christ Jésus qui vous est destiné. 
${}^{21}Il faut en effet que le ciel l’accueille jusqu’à l’époque où tout sera rétabli, comme Dieu l’avait dit par la bouche des saints, ceux d’autrefois, ses prophètes. 
${}^{22}Moïse a déclaré : Le Seigneur votre Dieu suscitera pour vous, du milieu de vos frères, un prophète comme moi : vous l’écouterez en tout ce qu’il vous dira. 
${}^{23}Quiconque n’écoutera pas ce prophète sera retranché du peuple. 
${}^{24}Ensuite, tous les prophètes qui ont parlé depuis Samuel et ses successeurs, aussi nombreux furent-ils, ont annoncé les jours où nous sommes. 
${}^{25}C’est vous qui êtes les fils des prophètes et de l’Alliance que Dieu a conclue avec vos pères, quand il disait à Abraham : En ta descendance seront bénies toutes les familles de la terre. 
${}^{26}C’est pour vous d’abord que Dieu a suscité son Serviteur, et il l’a envoyé vous bénir, pourvu que chacun de vous se détourne de sa méchanceté. »
      
         
      \bchapter{}
      \begin{verse}
${}^{1}Comme Pierre et Jean parlaient encore au peuple, les prêtres survinrent, avec le commandant du Temple et les sadducéens ; 
${}^{2}ils étaient excédés de les voir enseigner le peuple et annoncer, en la personne de Jésus, la résurrection d’entre les morts. 
${}^{3}Ils les firent arrêter et placer sous bonne garde jusqu’au lendemain, puisque c’était déjà le soir. 
${}^{4}Or, beaucoup de ceux qui avaient entendu la Parole devinrent croyants ; à ne compter que les hommes, il y en avait environ cinq mille.
${}^{5}Le lendemain se réunirent à Jérusalem les chefs du peuple, les anciens et les scribes. 
${}^{6}Il y avait là Hanne le grand prêtre, Caïphe, Jean, Alexandre, et tous ceux qui appartenaient aux familles de grands prêtres. 
${}^{7}Ils firent amener Pierre et Jean au milieu d’eux et les questionnèrent : « Par quelle puissance, par le nom de qui, avez-vous fait cette guérison ? »
${}^{8}Alors Pierre, rempli de l’Esprit Saint, leur déclara : « Chefs du peuple et anciens, 
${}^{9}nous sommes interrogés aujourd’hui pour avoir fait du bien à un infirme, et l’on nous demande comment cet homme a été sauvé. 
${}^{10}Sachez-le donc, vous tous, ainsi que tout le peuple d’Israël : c’est par le nom de Jésus le Nazaréen, lui que vous avez crucifié mais que Dieu a ressuscité d’entre les morts, c’est par lui que cet homme se trouve là, devant vous, bien portant. 
${}^{11}Ce Jésus est la pierre méprisée de vous, les bâtisseurs, mais devenue la pierre d’angle. 
${}^{12}En nul autre que lui, il n’y a de salut, car, sous le ciel, aucun autre nom n’est donné aux hommes, qui puisse nous sauver. »
${}^{13}Constatant l’assurance de Pierre et de Jean, et se rendant compte que c’était des hommes sans culture et de simples particuliers, ils étaient surpris ; d’autre part, ils reconnaissaient en eux ceux qui étaient avec Jésus. 
${}^{14}Mais comme ils voyaient, debout avec eux, l’homme qui avait été guéri, ils ne trouvaient rien à redire. 
${}^{15}Après leur avoir ordonné de quitter la salle du Conseil suprême, ils se mirent à discuter entre eux. 
${}^{16}Ils disaient : « Qu’allons-nous faire de ces gens-là ? Il est notoire, en effet, qu’ils ont opéré un miracle ; cela fut manifeste pour tous les habitants de Jérusalem, et nous ne pouvons pas le nier. 
${}^{17}Mais pour en limiter la diffusion dans le peuple, nous allons les menacer afin qu’ils ne parlent plus à personne en ce nom-là. »
${}^{18}Ayant rappelé Pierre et Jean, ils leur interdirent formellement de parler ou d’enseigner au nom de Jésus. 
${}^{19}Ceux-ci leur répliquèrent : « Est-il juste devant Dieu de vous écouter, plutôt que d’écouter Dieu ? À vous de juger. 
${}^{20}Quant à nous, il nous est impossible de nous taire sur ce que nous avons vu et entendu. » 
${}^{21}Après de nouvelles menaces, ils les relâchèrent, faute d’avoir trouvé le moyen de les punir : c’était à cause du peuple, car tout le monde rendait gloire à Dieu pour ce qui était arrivé. 
${}^{22}En effet, l’homme qui avait bénéficié de ce miracle de guérison avait plus de quarante ans.
${}^{23}Lorsque Pierre et Jean eurent été relâchés, ils se rendirent auprès des leurs et rapportèrent tout ce que les grands prêtres et les anciens leur avaient dit. 
${}^{24}Après avoir écouté, tous, d’un même cœur, élevèrent leur voix vers Dieu en disant : « Maître, toi, tu as fait le ciel et la terre et la mer et tout ce qu’ils renferment. 
${}^{25}Par l’Esprit Saint, tu as mis dans la bouche de notre père David, ton serviteur, les paroles que voici :
        \\Pourquoi ce tumulte des nations,
        \\ce vain murmure des peuples ?
        ${}^{26}Les rois de la terre se sont dressés,
        \\les chefs se sont ligués entre eux
        \\contre le Seigneur et contre son Christ ?
${}^{27}Et c’est vrai : dans cette ville, Hérode et Ponce Pilate, avec les nations et le peuple d’Israël, se sont ligués contre Jésus, ton Saint, ton Serviteur, le Christ à qui tu as donné l’onction ; 
${}^{28}ils ont fait tout ce que tu avais décidé d’avance dans ta puissance et selon ton dessein. 
${}^{29}Et maintenant, Seigneur, sois attentif à leurs menaces : donne à ceux qui te servent de dire ta parole avec une totale assurance. 
${}^{30}Étends donc ta main pour que se produisent guérisons, signes et prodiges, par le nom de Jésus, ton Saint, ton Serviteur. »
${}^{31}Quand ils eurent fini de prier, le lieu où ils étaient réunis se mit à trembler, ils furent tous remplis du Saint-Esprit et ils disaient la parole de Dieu avec assurance.
${}^{32}La multitude de ceux qui étaient devenus croyants avait un seul cœur et une seule âme ; et personne ne disait que ses biens lui appartenaient en propre, mais ils avaient tout en commun. 
${}^{33}C’est avec une grande puissance que les Apôtres rendaient témoignage de la résurrection du Seigneur Jésus, et une grâce abondante reposait sur eux tous. 
${}^{34}Aucun d’entre eux n’était dans l’indigence, car tous ceux qui étaient propriétaires de domaines ou de maisons les vendaient, 
${}^{35}et ils apportaient le montant de la vente pour le déposer aux pieds des Apôtres ; puis on le distribuait en fonction des besoins de chacun.
${}^{36}Il y avait un lévite originaire de Chypre, Joseph, surnommé Barnabé par les Apôtres, ce qui se traduit : « homme du réconfort ». 
${}^{37}Il vendit un champ qu’il possédait et en apporta l’argent qu’il déposa aux pieds des Apôtres.
      
         
      \bchapter{}
      \begin{verse}
${}^{1}Un homme du nom d’Ananie, avec son épouse Saphira, vendit une propriété ; 
${}^{2}il détourna pour lui une partie du montant de la vente, de connivence avec sa femme, et il apporta le reste pour le déposer aux pieds des Apôtres. 
${}^{3}Pierre lui dit : « Ananie, comment se fait-il que Satan a envahi ton cœur, pour que tu mentes à l’Esprit, l’Esprit Saint, et que tu détournes pour toi une partie du montant du domaine ? 
${}^{4}Tant que tu le possédais, il était bien à toi, et après la vente, tu pouvais disposer de la somme, n’est-ce pas ? Alors, pourquoi ce projet a-t-il germé dans ton cœur ? Tu n’as pas menti aux hommes, mais à Dieu. » 
${}^{5}En entendant ces paroles, Ananie tomba, et il expira. Une grande crainte saisit tous ceux qui apprenaient la nouvelle. 
${}^{6}Les jeunes gens se levèrent, enveloppèrent le corps, et ils l’emportèrent pour l’enterrer.
${}^{7}Il se passa environ trois heures, puis sa femme entra sans savoir ce qui était arrivé. 
${}^{8}Pierre l’interpella : « Dis-moi : le domaine, c’est bien à ce prix-là que vous l’avez cédé ? » Elle dit : « Oui, c’est à ce prix-là. » 
${}^{9}Pierre reprit : « Pourquoi cet accord entre vous pour mettre à l’épreuve l’Esprit du Seigneur ? Voici que sont à la porte les pas de ceux qui ont enterré ton mari ; ils vont t’emporter ! » 
${}^{10}Aussitôt, elle tomba à ses pieds, et elle expira. Les jeunes gens, qui rentraient, la trouvèrent morte, et ils l’emportèrent pour l’enterrer auprès de son mari. 
${}^{11}Une grande crainte saisit toute l’Église et tous ceux qui apprenaient cette nouvelle.
${}^{12}Par les mains des Apôtres, beaucoup de signes et de prodiges s’accomplissaient dans le peuple. Tous les croyants, d’un même cœur, se tenaient sous le portique de Salomon. 
${}^{13}Personne d’autre n’osait se joindre à eux ; cependant tout le peuple faisait leur éloge ; 
${}^{14}de plus en plus, des foules d’hommes et de femmes, en devenant croyants, s’attachaient au Seigneur. 
${}^{15}On allait jusqu’à sortir les malades sur les places, en les mettant sur des civières et des brancards : ainsi, au passage de Pierre, son ombre couvrirait l’un ou l’autre. 
${}^{16}La foule accourait aussi des villes voisines de Jérusalem, en amenant des gens malades ou tourmentés par des esprits impurs. Et tous étaient guéris.
${}^{17}Alors intervint le grand prêtre, ainsi que tout son entourage, c’est-à-dire le groupe des sadducéens, qui étaient remplis d’une ardeur jalouse pour la Loi. 
${}^{18}Ils mirent la main sur les Apôtres et les placèrent publiquement sous bonne garde. 
${}^{19}Mais, pendant la nuit, l’ange du Seigneur ouvrit les portes de la prison et les fit sortir. Il leur dit : 
${}^{20}« Partez, tenez-vous dans le Temple et là, dites au peuple toutes ces paroles de vie. » 
${}^{21}Ils l’écoutèrent ; dès l’aurore, ils entrèrent dans le Temple, et là, ils enseignaient.
      Alors arriva le grand prêtre, ainsi que son entourage. Ils convoquèrent le Conseil suprême, toute l’assemblée des anciens d’Israël, et ils envoyèrent chercher les Apôtres dans leur cachot. 
${}^{22}En arrivant, les gardes ne les trouvèrent pas à la prison. Ils revinrent donc annoncer : 
${}^{23}« Nous avons trouvé le cachot parfaitement verrouillé, et les gardes en faction devant les portes ; mais, quand nous avons ouvert, nous n’avons trouvé personne à l’intérieur. » 
${}^{24}Ayant entendu ce rapport, le commandant du Temple et les grands prêtres, tout perplexes, se demandaient ce qu’il adviendrait de cette affaire. 
${}^{25}Là-dessus, quelqu’un vient leur annoncer : « Les hommes que vous aviez mis en prison, voilà qu’ils se tiennent dans le Temple et enseignent le peuple ! » 
${}^{26}Alors, le commandant partit avec son escorte pour les ramener, mais sans violence, parce qu’ils avaient peur d’être lapidés par le peuple.
${}^{27}Ayant amené les Apôtres, ils les présentèrent au Conseil suprême, et le grand prêtre les interrogea : 
${}^{28}« Nous vous avions formellement interdit d’enseigner au nom de celui-là, et voilà que vous remplissez Jérusalem de votre enseignement. Vous voulez donc faire retomber sur nous le sang de cet homme ! »
${}^{29}En réponse, Pierre et les Apôtres déclarèrent : « Il faut obéir à Dieu plutôt qu’aux hommes. 
${}^{30}Le Dieu de nos pères a ressuscité Jésus, que vous aviez exécuté en le suspendant au bois du supplice. 
${}^{31}C’est lui que Dieu, par sa main droite, a élevé, en faisant de lui le Prince et le Sauveur, pour accorder à Israël la conversion et le pardon des péchés. 
${}^{32}Quant à nous, nous sommes les témoins de tout cela, avec l’Esprit Saint, que Dieu a donné à ceux qui lui obéissent. »
${}^{33}Ceux qui les avaient entendus étaient exaspérés et projetaient de les supprimer. 
${}^{34}Alors, dans le Conseil suprême, intervint un pharisien nommé Gamaliel, docteur de la Loi, qui était honoré par tout le peuple. Il ordonna de les faire sortir un instant, 
${}^{35}puis il dit : « Vous, Israélites, prenez garde à ce que vous allez faire à ces gens-là. 
${}^{36}Il y a un certain temps, se leva Theudas qui prétendait être quelqu’un, et à qui se rallièrent quatre cents hommes environ ; il a été supprimé, et tous ses partisans ont été mis en déroute et réduits à rien. 
${}^{37}Après lui, à l’époque du recensement, se leva Judas le Galiléen qui a entraîné beaucoup de monde derrière lui. Il a péri lui aussi, et tous ses partisans ont été dispersés. 
${}^{38}Eh bien, dans la circonstance présente, je vous le dis : ne vous occupez plus de ces gens-là, laissez-les. En effet, si leur résolution ou leur entreprise vient des hommes, elle tombera. 
${}^{39}Mais si elle vient de Dieu, vous ne pourrez pas les faire tomber. Ne risquez donc pas de vous trouver en guerre contre Dieu. »
      Les membres du Conseil se laissèrent convaincre ; 
${}^{40}ils rappelèrent alors les Apôtres et, après les avoir fait fouetter, ils leur interdirent de parler au nom de Jésus, puis ils les relâchèrent. 
${}^{41}Quant à eux, quittant le Conseil suprême, ils repartaient tout joyeux d’avoir été jugés dignes de subir des humiliations pour le nom de Jésus. 
${}^{42}Tous les jours, au Temple et dans leurs maisons, sans cesse, ils enseignaient et annonçaient la Bonne Nouvelle : le Christ, c’est Jésus.
      
         
      \bchapter{}
      \begin{verse}
${}^{1}En ces jours-là, comme le nombre des disciples augmentait, les frères de langue grecque récriminèrent contre ceux de langue hébraïque, parce que les veuves de leur groupe étaient désavantagées dans le service quotidien.
${}^{2}Les Douze convoquèrent alors l’ensemble des disciples et leur dirent : « Il n’est pas bon que nous délaissions la parole de Dieu pour servir aux tables. 
${}^{3}Cherchez plutôt, frères, sept d’entre vous, des hommes qui soient estimés de tous, remplis d’Esprit Saint et de sagesse, et nous les établirons dans cette charge. 
${}^{4}En ce qui nous concerne, nous resterons assidus à la prière et au service de la Parole. » 
${}^{5}Ces propos plurent à tout le monde, et l’on choisit : Étienne, homme rempli de foi et d’Esprit Saint, Philippe, Procore, Nicanor, Timon, Parménas et Nicolas, un converti au judaïsme, originaire d’Antioche. 
${}^{6}On les présenta aux Apôtres, et après avoir prié, ils leur imposèrent les mains.
${}^{7}La parole de Dieu était féconde, le nombre des disciples se multipliait fortement à Jérusalem, et une grande foule de prêtres juifs parvenaient à l’obéissance de la foi.
${}^{8}Étienne, rempli de la grâce et de la puissance de Dieu, accomplissait parmi le peuple des prodiges et des signes éclatants. 
${}^{9}Intervinrent alors certaines gens de la synagogue dite des Affranchis, ainsi que des Cyrénéens et des Alexandrins, et aussi des gens originaires de Cilicie et de la province d’Asie. Ils se mirent à discuter avec Étienne, 
${}^{10}mais sans pouvoir résister à la sagesse et à l’Esprit qui le faisaient parler. 
${}^{11}Alors ils soudoyèrent des hommes pour qu’ils disent : « Nous l’avons entendu prononcer des paroles blasphématoires contre Moïse et contre Dieu. » 
${}^{12}Ils ameutèrent le peuple, les anciens et les scribes, et, s’étant saisis d’Étienne à l’improviste, ils l’amenèrent devant le Conseil suprême. 
${}^{13}Ils produisirent de faux témoins, qui disaient : « Cet individu ne cesse de proférer des paroles contre le Lieu saint et contre la Loi. 
${}^{14}Nous l’avons entendu affirmer que ce Jésus, le Nazaréen, détruirait le Lieu saint et changerait les coutumes que Moïse nous a transmises. » 
${}^{15}Tous ceux qui siégeaient au Conseil suprême avaient les yeux fixés sur Étienne, et ils virent que son visage était comme celui d’un ange.
      
         
      \bchapter{}
      \begin{verse}
${}^{1}Le grand prêtre demanda : « Cela est-il exact ? » 
${}^{2}Étienne dit alors : « Frères et pères, écoutez ! Le Dieu de gloire est apparu à notre père Abraham, quand il était en Mésopotamie avant de venir habiter Harrane, 
${}^{3}et il lui a dit : Sors de ton pays et de ta parenté, et va dans le pays que je te montrerai. 
${}^{4}Alors, étant sorti du pays des Chaldéens, il vint habiter Harrane ; après la mort de son père, Dieu le fit émigrer de là-bas vers le pays où vous-mêmes habitez maintenant. 
${}^{5}Et là, il ne lui donna rien en héritage, pas même de quoi poser le pied. Mais il promit de lui donner ce pays en possession ainsi qu’à sa descendance après lui, alors qu’il n’avait pas encore d’enfant.
${}^{6}Dieu lui déclara que ses descendants seraient des immigrés en terre étrangère, que l’on en ferait des esclaves et qu’on les maltraiterait pendant quatre cents ans. 
${}^{7}“Mais, dit Dieu, la nation dont ils seront esclaves, moi, je la jugerai, et après cela ils sortiront et ils me rendront un culte en ce lieu.” 
${}^{8}Et Dieu lui donna l’alliance de la circoncision. Ainsi, Abraham engendra Isaac et le circoncit le huitième jour. Isaac fit de même pour Jacob, et Jacob pour les douze patriarches. 
${}^{9}Les patriarches, jaloux de Joseph, le vendirent pour être conduit en Égypte. Mais Dieu était avec lui, 
${}^{10}et il le tira de toutes ses épreuves. Il lui donna grâce et sagesse devant Pharaon, roi d’Égypte, et celui-ci le mit à la tête de l’Égypte et de toute la maison royale. 
${}^{11}Puis une famine arriva sur toute l’Égypte et Canaan, ainsi qu’une grande détresse, et nos pères ne trouvaient plus de nourriture. 
${}^{12}Quand Jacob apprit qu’il y avait du blé en Égypte, il y envoya nos pères une première fois. 
${}^{13}À la deuxième fois, Joseph se fit reconnaître par ses frères, et ainsi, son origine fut dévoilée à Pharaon. 
${}^{14}Joseph envoya chercher son père Jacob et toute sa parenté, à savoir soixante-quinze personnes. 
${}^{15}Jacob, alors, descendit en Égypte ; il y mourut, ainsi que nos pères. 
${}^{16}Ils furent transportés à Sichem et déposés dans le tombeau qu’Abraham avait acheté à prix d’argent aux fils de Hemmor, à Sichem.
${}^{17}Comme approchait le temps où devait s’accomplir la promesse par laquelle Dieu s’était engagé envers Abraham, le peuple devint fécond et se multiplia en Égypte, 
${}^{18}jusqu’à ce qu’un autre roi qui n’avait pas connu Joseph arrive au pouvoir en Égypte. 
${}^{19}Ayant pris des dispositions perverses contre notre peuple, il maltraita nos pères, au point de leur faire abandonner leurs nouveaux-nés pour qu’ils ne puissent pas vivre. 
${}^{20}C’est à ce moment que Moïse vint au monde ; il était beau sous le regard de Dieu. Il fut élevé pendant trois mois dans la maison de son père, 
${}^{21}puis abandonné. La fille de Pharaon le recueillit et l’éleva comme son propre fils. 
${}^{22}Moïse fut éduqué dans toute la sagesse des Égyptiens ; il était puissant par ses paroles et par ses actes. 
${}^{23}Comme il avait atteint l’âge de quarante ans, la pensée lui vint d’aller visiter ses frères, les fils d’Israël. 
${}^{24}Voyant que l’un d’entre eux était maltraité, il prit sa défense et frappa l’Égyptien pour venger celui qui était agressé. 
${}^{25}Il pensait que ses frères comprendraient que Dieu leur donnait, par lui, le salut ; mais eux ne comprirent pas. 
${}^{26}Le lendemain, il se fit voir à eux pendant qu’ils se battaient, et il essayait de rétablir la paix entre eux en leur disant : « Vous êtes frères : pourquoi vous faire du mal les uns aux autres ? » 
${}^{27}Mais celui qui maltraitait son compagnon repoussa Moïse en disant : Qui t’a établi chef et juge sur nous ? 
${}^{28}Veux-tu me tuer comme tu as tué hier l’Égyptien ? 
${}^{29}À ces mots, Moïse s’enfuit, et il séjourna en immigré dans le pays de Madiane, où il engendra deux fils.
${}^{30}Quarante années s’écoulèrent ; un ange lui apparut au désert du mont Sinaï dans la flamme d’un buisson en feu. 
${}^{31}Ayant vu, Moïse s’étonna de la vision, et lorsqu’il s’approcha pour regarder, la voix du Seigneur se fit entendre : 
${}^{32}Je suis le Dieu de tes pères, le Dieu d’Abraham, d’Isaac et de Jacob. Moïse se mit à trembler, et il n’osait plus regarder. 
${}^{33}Le Seigneur lui dit : Retire les sandales de tes pieds, car le lieu où tu te tiens est une terre sainte. 
${}^{34}J’ai vu, oui, j’ai vu la misère de mon peuple qui est en Égypte ; j’ai entendu leurs gémissements et je suis descendu pour les délivrer. Et maintenant, va ! Je t’envoie en Égypte.
${}^{35}Ce Moïse que l’on avait rejeté en disant : Qui t’a établi chef et juge ?, Dieu l’a envoyé comme chef et libérateur, avec l’aide de l’ange qui lui était apparu dans le buisson. 
${}^{36}C’est lui qui les a fait sortir en faisant des prodiges et des signes au pays d’Égypte, à la mer Rouge, et au désert pendant quarante ans. 
${}^{37}C’est ce Moïse qui a dit aux fils d’Israël : Dieu suscitera pour vous, du milieu de vos frères, un prophète comme moi. 
${}^{38}C’est lui qui était présent lors de l’assemblée au désert, avec l’ange qui lui parlait sur le mont Sinaï et avec nos pères : il reçut des paroles vivantes pour nous les donner, 
${}^{39}mais nos pères n’ont pas voulu lui obéir ; bien plus, ils le repoussèrent. De cœur ils retournaient en Égypte, 
${}^{40}quand ils dirent à Aaron : Fabrique-nous des dieux qui marcheront devant nous. Car ce Moïse qui nous a fait sortir du pays d’Égypte, nous ne savons pas ce qui lui est arrivé. 
${}^{41}Et en ces jours-là, ils fabriquèrent un veau et offrirent un sacrifice à cette idole : ils se réjouissaient de l’œuvre de leurs mains ! 
${}^{42}Alors Dieu se détourna et les laissa rendre un culte à l’armée du ciel, comme il est écrit dans le livre des prophètes :
        \\Des victimes et des sacrifices, m’en avez-vous présentés
        \\pendant quarante ans au désert, maison d’Israël ?
${}^{43}Mais vous avez porté la tente de Molok
        \\et l’étoile de votre dieu Réphane,
        \\ces images que vous avez fabriquées
        \\pour vous prosterner devant elles.
        \\Je vous déporterai au-delà de Babylone !
${}^{44}Nos pères, dans le désert, avaient la tente du Témoignage. Elle avait été faite d’après les ordres de Celui qui parlait à Moïse et qui lui en avait montré le modèle. 
${}^{45}Après avoir reçu cette tente, nos pères, avec Josué, la firent entrer dans le pays que les nations possédaient avant que Dieu les chasse loin du visage de nos pères. Cela dura jusqu’au temps de David. 
${}^{46}Celui-ci trouva grâce devant Dieu et il pria afin de trouver une demeure au Dieu de Jacob. 
${}^{47}Mais ce fut Salomon qui lui construisit une maison. 
${}^{48}Pourtant, le Très-Haut n’habite pas dans ce qui est fait de main d’homme, comme le dit le prophète :
${}^{49}Le ciel est mon trône,
        \\et la terre, l’escabeau de mes pieds.
        \\Quelle maison me bâtirez-vous, dit le Seigneur,
        \\quel sera le lieu de mon repos ?
${}^{50}N’est-ce pas ma main qui a fait tout cela ?
${}^{51}Vous qui avez la nuque raide, vous dont le cœur et les oreilles sont fermés à l’Alliance, depuis toujours vous résistez à l’Esprit Saint ; vous êtes bien comme vos pères ! 
${}^{52}Y a-t-il un prophète que vos pères n’aient pas persécuté ? Ils ont même tué ceux qui annonçaient d’avance la venue du Juste, celui-là que maintenant vous venez de livrer et d’assassiner. 
${}^{53}Vous qui aviez reçu la loi sur ordre des anges, vous ne l’avez pas observée. »
${}^{54}Ceux qui écoutaient ce discours avaient le cœur exaspéré et grinçaient des dents contre Étienne. 
${}^{55}Mais lui, rempli de l’Esprit Saint, fixait le ciel du regard : il vit la gloire de Dieu, et Jésus debout à la droite de Dieu. 
${}^{56}Il déclara : « Voici que je contemple les cieux ouverts et le Fils de l’homme debout à la droite de Dieu. » 
${}^{57}Alors ils poussèrent de grands cris et se bouchèrent les oreilles. Tous ensemble, ils se précipitèrent sur lui, 
${}^{58}l’entraînèrent hors de la ville et se mirent à le lapider. Les témoins avaient déposé leurs vêtements aux pieds d’un jeune homme appelé Saul. 
${}^{59}Étienne, pendant qu’on le lapidait, priait ainsi : « Seigneur Jésus, reçois mon esprit. » 
${}^{60}Puis, se mettant à genoux, il s’écria d’une voix forte : « Seigneur, ne leur compte pas ce péché. » Et, après cette parole, il s’endormit dans la mort.
      
         
      \bchapter{}
       
      \begin{verse}
${}^{1}Quant à Saul, il approuvait ce meurtre.
      <a class="anchor verset_lettre" id="bib_ac_8_1_b"/>Ce jour-là, éclata une violente persécution contre l’Église de Jérusalem. Tous se dispersèrent dans les campagnes de Judée et de Samarie, à l’exception des Apôtres. 
${}^{2}Des hommes religieux ensevelirent Étienne et célébrèrent pour lui un grand deuil. 
${}^{3}Quant à Saul, il ravageait l’Église, il pénétrait dans les maisons, pour en arracher hommes et femmes, et les jeter en prison.
${}^{4}Ceux qui s’étaient dispersés annonçaient la Bonne Nouvelle de la Parole là où ils passaient. 
${}^{5}C’est ainsi que Philippe, l’un des Sept, arriva dans une ville de Samarie, et là il proclamait le Christ. 
${}^{6}Les foules, d’un même cœur, s’attachaient à ce que disait Philippe, car elles entendaient parler des signes qu’il accomplissait, ou même les voyaient. 
${}^{7}Beaucoup de possédés étaient délivrés des esprits impurs, qui sortaient en poussant de grands cris. Beaucoup de paralysés et de boiteux furent guéris. 
${}^{8}Et il y eut dans cette ville une grande joie.
${}^{9}Or il y avait déjà dans la ville un homme du nom de Simon ; il pratiquait la magie et frappait de stupéfaction la population de Samarie, prétendant être un grand personnage. 
${}^{10}Et tous, du plus petit jusqu’au plus grand, s’attachaient à lui en disant : « Cet homme est la Puissance de Dieu, celle qu’on appelle la Grande. » 
${}^{11}Ils s’attachaient à lui du fait que depuis un certain temps il les stupéfiait par ses pratiques magiques. 
${}^{12}Mais quand ils crurent Philippe qui annonçait la Bonne Nouvelle concernant le règne de Dieu et le nom de Jésus Christ, hommes et femmes se firent baptiser. 
${}^{13}Simon lui-même devint croyant et, après avoir reçu le baptême, il ne quittait plus Philippe ; voyant les signes et les actes de grande puissance qui se produisaient, il était stupéfait.
${}^{14}Les Apôtres, restés à Jérusalem, apprirent que la Samarie avait accueilli la parole de Dieu. Alors ils y envoyèrent Pierre et Jean. 
${}^{15}À leur arrivée, ceux-ci prièrent pour ces Samaritains afin qu’ils reçoivent l’Esprit Saint ; 
${}^{16}en effet, l’Esprit n’était encore descendu sur aucun d’entre eux : ils étaient seulement baptisés au nom du Seigneur Jésus. 
${}^{17}Alors Pierre et Jean leur imposèrent les mains, et ils reçurent l’Esprit Saint.
${}^{18}Simon, voyant que l’Esprit était donné par l’imposition des mains des Apôtres, leur offrit de l’argent 
${}^{19}en disant : « Donnez-moi ce pouvoir, à moi aussi, pour que tous ceux à qui j’imposerai les mains reçoivent l’Esprit Saint. » 
${}^{20}Pierre lui dit : « Périsse ton argent, et toi avec, puisque tu as estimé pouvoir acheter le don de Dieu à prix d’argent ! 
${}^{21}Tu n’as aucune part, aucun droit, en ce domaine, car devant Dieu ton cœur manque de droiture. 
${}^{22}Détourne-toi donc de ce mal que tu veux faire, et prie le Seigneur : il te pardonnera peut-être cette pensée que tu as dans le cœur. 
${}^{23}Car je le vois bien : tu es plein d’aigreur amère, tu es enchaîné dans l’injustice. » 
${}^{24}Simon répondit : « Priez vous-mêmes pour moi le Seigneur, afin que rien ne m’arrive de ce que vous avez dit. »
${}^{25}Quant à Pierre et Jean, ayant rendu témoignage et proclamé la parole du Seigneur, ils retournèrent à Jérusalem en annonçant l’Évangile à un grand nombre de villages samaritains.
${}^{26}L’ange du Seigneur adressa la parole à Philippe en disant : « Mets-toi en marche en direction du sud, prends la route qui descend de Jérusalem à Gaza ; elle est déserte. »
${}^{27}Et Philippe se mit en marche. Or, un Éthiopien, un eunuque, haut fonctionnaire de Candace, la reine d’Éthiopie, et administrateur de tous ses trésors, était venu à Jérusalem pour adorer. 
${}^{28}Il en revenait, assis sur son char, et lisait le prophète Isaïe. 
${}^{29}L’Esprit dit à Philippe : « Approche, et rejoins ce char. » 
${}^{30}Philippe se mit à courir, et il entendit l’homme qui lisait le prophète Isaïe ; alors il lui demanda : « Comprends-tu ce que tu lis ? » 
${}^{31}L’autre lui répondit : « Et comment le pourrais-je s’il n’y a personne pour me guider ? » Il invita donc Philippe à monter et à s’asseoir à côté de lui. 
${}^{32}Le passage de l’Écriture qu’il lisait était celui-ci :
        \\Comme une brebis, il fut conduit à l’abattoir ;
        \\comme un agneau muet devant le tondeur,
        \\il n’ouvre pas la bouche.
        ${}^{33}Dans son humiliation,
        \\il n’a pas obtenu justice.
        \\Sa descendance, qui en parlera ?
        \\Car sa vie est retranchée de la terre.
${}^{34}Prenant la parole, l’eunuque dit à Philippe : « Dis-moi, je te prie : de qui le prophète parle-t-il ? De lui-même, ou bien d’un autre ? » 
${}^{35}Alors Philippe prit la parole et, à partir de ce passage de l’Écriture, il lui annonça la Bonne Nouvelle de Jésus. 
${}^{36}Comme ils poursuivaient leur route, ils arrivèrent à un point d’eau, et l’eunuque dit : « Voici de l’eau : qu’est-ce qui empêche que je sois baptisé ? » 
${}^{38}Il fit arrêter le char, ils descendirent dans l’eau tous les deux, et Philippe baptisa l’eunuque. 
${}^{39}Quand ils furent remontés de l’eau, l’Esprit du Seigneur emporta Philippe ; l’eunuque ne le voyait plus, mais il poursuivait sa route, tout joyeux. 
${}^{40}Philippe se retrouva dans la ville d’Ashdod, il annonçait la Bonne Nouvelle dans toutes les villes où il passait jusqu’à son arrivée à Césarée.
      
         
      \bchapter{}
      \begin{verse}
${}^{1}Saul était toujours animé d’une rage meurtrière contre les disciples du Seigneur. Il alla trouver le grand prêtre 
${}^{2}et lui demanda des lettres pour les synagogues de Damas, afin que, s’il trouvait des hommes et des femmes qui suivaient le Chemin du Seigneur, il les amène enchaînés à Jérusalem.
${}^{3}Comme il était en route et approchait de Damas, soudain une lumière venant du ciel l’enveloppa de sa clarté. 
${}^{4}Il fut précipité à terre ; il entendit une voix qui lui disait : « Saul, Saul, pourquoi me persécuter ? » 
${}^{5}Il demanda : « Qui es-tu, Seigneur ? » La voix répondit : « Je suis Jésus, celui que tu persécutes. 
${}^{6}Relève-toi et entre dans la ville : on te dira ce que tu dois faire. » 
${}^{7}Ses compagnons de route s’étaient arrêtés, muets de stupeur : ils entendaient la voix, mais ils ne voyaient personne. 
${}^{8}Saul se releva de terre et, bien qu’il eût les yeux ouverts, il ne voyait rien. Ils le prirent par la main pour le faire entrer à Damas. 
${}^{9}Pendant trois jours, il fut privé de la vue et il resta sans manger ni boire.
${}^{10}Or, il y avait à Damas un disciple nommé Ananie. Dans une vision, le Seigneur lui dit : « Ananie ! » Il répondit : « Me voici, Seigneur. » 
${}^{11}Le Seigneur reprit : « Lève-toi, va dans la rue appelée rue Droite, chez Jude : tu demanderas un homme de Tarse nommé Saul. Il est en prière, 
${}^{12}et il a eu cette vision : un homme, du nom d’Ananie, entrait et lui imposait les mains pour lui rendre la vue. » 
${}^{13}Ananie répondit : « Seigneur, j’ai beaucoup entendu parler de cet homme, et de tout le mal qu’il a fait subir à tes fidèles à Jérusalem. 
${}^{14}Il est ici, après avoir reçu de la part des grands prêtres le pouvoir d’enchaîner tous ceux qui invoquent ton nom. » 
${}^{15}Mais le Seigneur lui dit : « Va ! car cet homme est l’instrument que j’ai choisi pour faire parvenir mon nom auprès des nations, des rois et des fils d’Israël. 
${}^{16}Et moi, je lui montrerai tout ce qu’il lui faudra souffrir pour mon nom. »
${}^{17}Ananie partit donc et entra dans la maison. Il imposa les mains à Saul, en disant : « Saul, mon frère, celui qui m’a envoyé, c’est le Seigneur, c’est Jésus qui t’est apparu sur le chemin par lequel tu venais. Ainsi, tu vas retrouver la vue, et tu seras rempli d’Esprit Saint. » 
${}^{18}Aussitôt tombèrent de ses yeux comme des écailles, et il retrouva la vue. Il se leva, puis il fut baptisé. 
${}^{19}Alors il prit de la nourriture et les forces lui revinrent.
      Il passa quelques jours à Damas avec les disciples 
${}^{20}et, sans plus attendre, il proclamait Jésus dans les synagogues, affirmant que celui-ci est le Fils de Dieu. 
${}^{21}Tous ceux qui écoutaient étaient stupéfaits et disaient : « N’est-ce pas lui qui, à Jérusalem, s’acharnait contre ceux qui invoquent ce nom-là, et n’est-il pas venu ici afin de les ramener enchaînés chez les grands prêtres ? » 
${}^{22}Mais Saul, avec une force de plus en plus grande, réfutait les Juifs qui habitaient Damas, en démontrant que Jésus est le Christ. 
${}^{23}Assez longtemps après, les Juifs tinrent conseil en vue de le supprimer. 
${}^{24}Saul fut informé de leur machination. On faisait même garder les portes de la ville jour et nuit afin de pouvoir le supprimer. 
${}^{25}Alors ses disciples le prirent de nuit ; ils le firent descendre dans une corbeille, jusqu’en bas, de l’autre côté du rempart.
${}^{26}Arrivé à Jérusalem, Saul cherchait à se joindre aux disciples, mais tous avaient peur de lui, car ils ne croyaient pas que lui aussi était un disciple. 
${}^{27}Alors Barnabé le prit avec lui et le présenta aux Apôtres ; il leur raconta comment, sur le chemin, Saul avait vu le Seigneur, qui lui avait parlé, et comment, à Damas, il s’était exprimé avec assurance au nom de Jésus. 
${}^{28}Dès lors, Saul allait et venait dans Jérusalem avec eux, s’exprimant avec assurance au nom du Seigneur. 
${}^{29}Il parlait aux Juifs de langue grecque, et discutait avec eux. Mais ceux-ci cherchaient à le supprimer. 
${}^{30}Mis au courant, les frères l’accompagnèrent jusqu’à Césarée et le firent partir pour Tarse.
${}^{31}L’Église était en paix dans toute la Judée, la Galilée et la Samarie ; elle se construisait et elle marchait dans la crainte du Seigneur ; réconfortée par l’Esprit Saint, elle se multipliait.
${}^{32}Or, il arriva que Pierre, parcourant tout le pays, se rendit aussi chez les fidèles qui habitaient Lod. 
${}^{33}Il y trouva un homme du nom d’Énéas, alité depuis huit ans parce qu’il était paralysé. 
${}^{34}Pierre lui dit : « Énéas, Jésus Christ te guérit, lève-toi et fais ton lit toi-même. » Et aussitôt il se leva. 
${}^{35}Alors tous les habitants de Lod et de la plaine de Sarone purent le voir, et ils se convertirent en se tournant vers le Seigneur.
${}^{36}Il y avait aussi à Jaffa une femme disciple du Seigneur, nommée Tabitha, ce qui se traduit : Dorcas (c’est-à-dire : Gazelle). Elle était riche des bonnes œuvres et des aumônes qu’elle faisait. 
${}^{37}Or, il arriva en ces jours-là qu’elle tomba malade et qu’elle mourut. Après la toilette funèbre, on la déposa dans la chambre haute. 
${}^{38}Comme Lod est près de Jaffa, les disciples, apprenant que Pierre s’y trouvait, lui envoyèrent deux hommes avec cet appel : « Viens chez nous sans tarder. » 
${}^{39}Pierre se mit en route avec eux. À son arrivée on le fit monter à la chambre haute. Toutes les veuves en larmes s’approchèrent de lui ; elles lui montraient les tuniques et les manteaux confectionnés par Dorcas quand celle-ci était avec elles. 
${}^{40}Pierre mit tout le monde dehors ; il se mit à genoux et pria ; puis il se tourna vers le corps, et il dit : « Tabitha, lève-toi ! » Elle ouvrit les yeux et, voyant Pierre, elle se redressa et s’assit. 
${}^{41}Pierre, lui donnant la main, la fit lever. Puis il appela les fidèles et les veuves et la leur présenta vivante. 
${}^{42}La chose fut connue dans toute la ville de Jaffa, et beaucoup crurent au Seigneur. 
${}^{43}Pierre resta assez longtemps à Jaffa, chez un certain Simon, qui travaillait le cuir.
      
         
      \bchapter{}
      \begin{verse}
${}^{1}Il y avait à Césarée un homme du nom de Corneille, centurion de la cohorte appelée Italique. 
${}^{2}C’était quelqu’un de grande piété qui craignait Dieu, lui et tous les gens de sa maison ; il faisait de larges aumônes au peuple juif et priait Dieu sans cesse. 
${}^{3}Vers la neuvième heure du jour, il eut la vision très claire d’un ange de Dieu qui entrait chez lui et lui disait : « Corneille ! » 
${}^{4}Celui-ci le fixa du regard et, saisi de crainte, demanda : « Qu’y a-t-il, Seigneur ? » L’ange lui répondit : « Tes prières et tes aumônes sont montées devant Dieu pour qu’il se souvienne de toi. 
${}^{5}Et maintenant, envoie des hommes à Jaffa et fais venir un certain Simon surnommé Pierre : 
${}^{6}il est logé chez un autre Simon qui travaille le cuir et dont la maison est au bord de la mer. » 
${}^{7}Après le départ de l’ange qui lui avait parlé, il appela deux de ses domestiques et l’un des soldats attachés à son service, un homme de grande piété. 
${}^{8}Leur ayant tout expliqué, il les envoya à Jaffa.
      
         
${}^{9}Le lendemain, tandis qu’ils étaient en route et s’approchaient de la ville, Pierre monta sur la terrasse de la maison, vers midi, pour prier. 
${}^{10}Saisi par la faim, il voulut prendre quelque chose. Pendant qu’on lui préparait à manger, il tomba en extase. 
${}^{11}Il contemplait le ciel ouvert et un objet qui descendait : on aurait dit une grande toile tenue aux quatre coins, et qui se posait sur la terre. 
${}^{12}Il y avait dedans tous les quadrupèdes, tous les reptiles de la terre et tous les oiseaux du ciel.
${}^{13}Et une voix s’adressa à lui : « Debout, Pierre, offre-les en sacrifice, et mange ! » 
${}^{14}Pierre dit : « Certainement pas, Seigneur ! Je n’ai jamais pris d’aliment interdit et impur ! » 
${}^{15}À nouveau, pour la deuxième fois, la voix s’adressa à lui : « Ce que Dieu a déclaré pur, toi, ne le déclare pas interdit. » 
${}^{16}Cela se produisit par trois fois et, aussitôt après, l’objet fut emporté au ciel.
${}^{17}Comme Pierre était tout perplexe sur ce que pouvait signifier cette vision, voici que les envoyés de Corneille, s’étant renseignés sur la maison de Simon, survinrent à la porte. 
${}^{18}Ils appelèrent pour demander : « Est-ce que Simon surnommé Pierre est logé ici ? » 
${}^{19}Comme Pierre réfléchissait encore à sa vision, l’Esprit lui dit : « Voilà trois hommes qui te cherchent. 
${}^{20}Eh bien, debout, descends, et pars avec eux sans hésiter, car c’est moi qui les ai envoyés. » 
${}^{21}Pierre descendit trouver les hommes et leur dit : « Me voici, je suis celui que vous cherchez. Pour quelle raison êtes-vous là ? » 
${}^{22}Ils répondirent : « Le centurion Corneille, un homme juste, qui craint Dieu, et à qui toute la nation juive rend un bon témoignage, a été averti par un ange saint de te faire venir chez lui et d’écouter tes paroles. » 
${}^{23}Il les fit entrer et leur donna l’hospitalité.
      Le lendemain, il se mit en route avec eux ; quelques frères de Jaffa l’accompagnèrent. 
${}^{24}Le jour suivant, il entra à Césarée. Corneille les attendait, et avait rassemblé sa famille et ses amis les plus proches. 
${}^{25}Comme Pierre arrivait, Corneille vint à sa rencontre et, tombant à ses pieds, il se prosterna.
${}^{26}Mais Pierre le releva en disant : « Lève-toi. Je ne suis qu’un homme, moi aussi. » 
${}^{27}Tout en conversant avec lui, il entra et il trouva beaucoup de gens réunis. 
${}^{28}Il leur dit : « Vous savez qu’un Juif n’est pas autorisé à fréquenter un étranger ni à entrer en contact avec lui. Mais à moi, Dieu a montré qu’il ne fallait déclarer interdit ou impur aucun être humain. 
${}^{29}C’est pourquoi, quand vous m’avez envoyé chercher, je suis venu sans réticence. J’aimerais donc savoir pour quelle raison vous m’avez envoyé chercher. » 
${}^{30}Corneille dit alors : « Il y a maintenant quatre jours, j’étais en train de prier chez moi à la neuvième heure, au milieu de l’après-midi, quand un homme au vêtement éclatant se tint devant moi, 
${}^{31}et me dit : “Corneille, ta prière a été exaucée, et Dieu s’est souvenu de tes aumônes. 
${}^{32}Envoie donc quelqu’un à Jaffa pour convoquer Simon surnommé Pierre ; il est logé chez un autre Simon qui travaille le cuir et dont la maison est au bord de la mer.” 
${}^{33}Je t’ai donc aussitôt envoyé chercher, et toi, en venant, tu as bien agi. Maintenant donc, nous sommes tous là devant Dieu pour écouter tout ce que le Seigneur t’a chargé de nous dire. »
${}^{34}Alors Pierre prit la parole et dit : « En vérité, je le comprends, Dieu est impartial : 
${}^{35}il accueille, quelle que soit la nation, celui qui le craint et dont les œuvres sont justes. 
${}^{36}Telle est la parole qu’il a envoyée aux fils d’Israël, en leur annonçant la bonne nouvelle de la paix par Jésus Christ, lui qui est le Seigneur de tous.
${}^{37}Vous savez ce qui s’est passé à travers tout le pays des Juifs, depuis les commencements en Galilée, après le baptême proclamé par Jean : 
${}^{38}Jésus de Nazareth, Dieu lui a donné l’onction d’Esprit Saint et de puissance. Là où il passait, il faisait le bien et guérissait tous ceux qui étaient sous le pouvoir du diable, car Dieu était avec lui. 
${}^{39}Et nous, nous sommes témoins de tout ce qu’il a fait dans le pays des Juifs et à Jérusalem. Celui qu’ils ont supprimé en le suspendant au bois du supplice, 
${}^{40}Dieu l’a ressuscité le troisième jour. Il lui a donné de se manifester, 
${}^{41}non pas à tout le peuple, mais à des témoins que Dieu avait choisis d’avance, à nous qui avons mangé et bu avec lui après sa résurrection d’entre les morts. 
${}^{42}Dieu nous a chargés d’annoncer au peuple et de témoigner que lui-même l’a établi Juge des vivants et des morts. 
${}^{43}C’est à Jésus que tous les prophètes rendent ce témoignage : Quiconque croit en lui reçoit par son nom le pardon de ses péchés. »
${}^{44}Pierre parlait encore quand l’Esprit Saint descendit sur tous ceux qui écoutaient la Parole. 
${}^{45}Les croyants qui accompagnaient Pierre, et qui étaient juifs d’origine, furent stupéfaits de voir que, même sur les nations, le don de l’Esprit Saint avait été répandu. 
${}^{46}En effet, on les entendait parler en langues et chanter la grandeur de Dieu. Pierre dit alors : 
${}^{47}« Quelqu’un peut-il refuser l’eau du baptême à ces gens qui ont reçu l’Esprit Saint tout comme nous ? » 
${}^{48}Et il donna l’ordre de les baptiser au nom de Jésus Christ. Alors ils lui demandèrent de rester quelques jours avec eux.
      
         
      \bchapter{}
      \begin{verse}
${}^{1}Les Apôtres et les frères qui étaient en Judée avaient appris que les nations, elles aussi, avaient reçu la parole de Dieu. 
${}^{2}Lorsque Pierre fut de retour à Jérusalem, ceux qui étaient juifs d’origine le prirent à partie, 
${}^{3}en disant : « Tu es entré chez des hommes qui ne sont pas circoncis, et tu as mangé avec eux ! »
${}^{4}Alors Pierre reprit l’affaire depuis le commencement et leur exposa tout dans l’ordre, en disant : 
${}^{5}« J’étais dans la ville de Jaffa, en train de prier, et voici la vision que j’ai eue dans une extase : c’était un objet qui descendait. On aurait dit une grande toile tenue aux quatre coins ; venant du ciel, elle se posa près de moi. 
${}^{6}Fixant les yeux sur elle, je l’examinai et je vis les quadrupèdes de la terre, les bêtes sauvages, les reptiles et les oiseaux du ciel. 
${}^{7}J’entendis une voix qui me disait : “Debout, Pierre, offre-les en sacrifice, et mange !” 
${}^{8}Je répondis : “Certainement pas, Seigneur ! Jamais aucun aliment interdit ou impur n’est entré dans ma bouche.” 
${}^{9}Une deuxième fois, du haut du ciel la voix répondit : “Ce que Dieu a déclaré pur, toi, ne le déclare pas interdit.” 
${}^{10}Cela se produisit par trois fois, puis tout fut remonté au ciel.
${}^{11}Et voici qu’à l’instant même, devant la maison où j’étais, survinrent trois hommes qui m’étaient envoyés de Césarée. 
${}^{12}L’Esprit me dit d’aller avec eux sans hésiter. Les six frères qui sont ici m’ont accompagné, et nous sommes entrés chez le centurion Corneille. 
${}^{13}Il nous raconta comment il avait vu l’ange se tenir dans sa maison et dire : “Envoie quelqu’un à Jaffa pour chercher Simon surnommé Pierre. 
${}^{14}Celui-ci t’adressera des paroles par lesquelles tu seras sauvé, toi et toute ta maison.”
${}^{15}Au moment où je prenais la parole, l’Esprit Saint descendit sur ceux qui étaient là, comme il était descendu sur nous au commencement. 
${}^{16}Alors je me suis rappelé la parole que le Seigneur avait dite : “Jean a baptisé avec l’eau, mais vous, c’est dans l’Esprit Saint que vous serez baptisés.” 
${}^{17}Et si Dieu leur a fait le même don qu’à nous, parce qu’ils ont cru au Seigneur Jésus Christ, qui étais-je, moi, pour empêcher l’action de Dieu ? »
${}^{18}En entendant ces paroles, ils se calmèrent et ils rendirent gloire à Dieu, en disant : « Ainsi donc, même aux nations, Dieu a donné la conversion qui fait entrer dans la vie ! »
${}^{19}Les frères dispersés par la tourmente qui se produisit lors de l’affaire d’Étienne allèrent jusqu’en Phénicie, puis à Chypre et Antioche, sans annoncer la Parole à personne d’autre qu’aux Juifs. 
${}^{20}Parmi eux, il y en avait qui étaient originaires de Chypre et de Cyrène, et qui, en arrivant à Antioche, s’adressaient aussi aux gens de langue grecque pour leur annoncer la Bonne Nouvelle : Jésus est le Seigneur. 
${}^{21}La main du Seigneur était avec eux : un grand nombre de gens devinrent croyants et se tournèrent vers le Seigneur.
${}^{22}La nouvelle parvint aux oreilles de l’Église de Jérusalem, et l’on envoya Barnabé jusqu’à Antioche. 
${}^{23}À son arrivée, voyant la grâce de Dieu à l’œuvre, il fut dans la joie. Il les exhortait tous à rester d’un cœur ferme attachés au Seigneur. 
${}^{24}C’était en effet un homme de bien, rempli d’Esprit Saint et de foi. Une foule considérable s’attacha au Seigneur.
${}^{25}Barnabé partit alors à Tarse chercher Saul. 
${}^{26}L’ayant trouvé, il l’amena à Antioche. Pendant toute une année, ils participèrent aux assemblées de l’Église, ils instruisirent une foule considérable. Et c’est à Antioche que, pour la première fois, les disciples reçurent le nom de « chrétiens ».
${}^{27}En ces jours-là, des prophètes descendirent de Jérusalem à Antioche. 
${}^{28}L’un d’eux, nommé Agabus, se leva pour signifier sous l’action de l’Esprit qu’il y aurait une grande famine sur toute la terre ; celle-ci se produisit sous l’empereur Claude. 
${}^{29}Alors les disciples décidèrent d’envoyer de l’aide, chacun selon ses moyens, aux frères qui habitaient en Judée ; 
${}^{30}ce qu’ils firent en l’adressant aux Anciens, par l’intermédiaire de Barnabé et de Saul.
      
         
      \bchapter{}
      \begin{verse}
${}^{1}À cette époque, le roi Hérode Agrippa se saisit de certains membres de l’Église pour les mettre à mal. 
${}^{2}Il supprima Jacques, frère de Jean, en le faisant décapiter. 
${}^{3}Voyant que cette mesure plaisait aux Juifs, il décida aussi d’arrêter Pierre. C’était les jours des Pains sans levain. 
${}^{4}Il le fit appréhender, emprisonner, et placer sous la garde de quatre escouades de quatre soldats ; il voulait le faire comparaître devant le peuple après la Pâque. 
${}^{5}Tandis que Pierre était ainsi détenu dans la prison, l’Église priait Dieu pour lui avec insistance.
      
         
${}^{6}Hérode allait le faire comparaître. Or, Pierre dormait, cette nuit-là, entre deux soldats ; il était attaché avec deux chaînes et des gardes étaient en faction devant la porte de la prison.
${}^{7}Et voici que survint l’ange du Seigneur, et une lumière brilla dans la cellule. Il réveilla Pierre en le frappant au côté et dit : « Lève-toi vite. » Les chaînes lui tombèrent des mains. 
${}^{8}Alors l’ange lui dit : « Mets ta ceinture et chausse tes sandales. » Ce que fit Pierre. L’ange ajouta : « Enveloppe-toi de ton manteau et suis-moi. » 
${}^{9}Pierre sortit derrière lui, mais il ne savait pas que tout ce qui arrivait grâce à l’ange était bien réel ; il pensait qu’il avait une vision. 
${}^{10}Passant devant un premier poste de garde, puis devant un second, ils arrivèrent au portail de fer donnant sur la ville. Celui-ci s’ouvrit tout seul devant eux. Une fois dehors, ils s’engagèrent dans une rue, et aussitôt l’ange le quitta.
${}^{11}Alors, se reprenant, Pierre dit : « Vraiment, je me rends compte maintenant que le Seigneur a envoyé son ange, et qu’il m’a arraché aux mains d’Hérode et à tout ce qu’attendait le peuple juif. » 
${}^{12}S’étant repéré, il se rendit à la maison de Marie, la mère de Jean surnommé Marc, où se trouvaient rassemblées un certain nombre de personnes qui priaient. 
${}^{13}Il frappa au battant du portail : une jeune servante nommée Rhodè s’approcha pour écouter. 
${}^{14}Elle reconnut la voix de Pierre et, dans sa joie, au lieu d’ouvrir la porte, elle rentra en courant annoncer que Pierre était là, devant le portail. 
${}^{15}On lui dit : « Tu délires ! » Mais elle soutenait qu’il en était bien ainsi. Et eux disaient : « C’est son ange. » 
${}^{16}Cependant Pierre continuait à frapper ; ayant ouvert, ils le virent et furent dans la stupéfaction. 
${}^{17}D’un geste de la main, il leur demanda le silence et leur raconta comment le Seigneur l’avait fait sortir de la prison. Il leur dit alors : « Annoncez-le à Jacques et aux frères. » Puis il sortit et s’en alla vers un autre lieu.
${}^{18}Au lever du jour, il y eut une belle agitation chez les soldats : qu’était donc devenu Pierre ? 
${}^{19}Hérode le fit rechercher, sans réussir à le trouver. Ayant fait comparaître les gardes, il donna l’ordre de les emmener au supplice. Puis, de Judée, il descendit à Césarée, où il séjourna.
${}^{20}Hérode était en conflit aigu avec les habitants de Tyr et de Sidon. S’étant mis d’accord, ceux-ci vinrent se présenter devant lui. Après avoir gagné à leur cause Blastos, le chambellan du roi, ils sollicitaient une solution pacifique, car leur contrée dépendait du domaine royal pour son approvisionnement. 
${}^{21}Au jour fixé, Hérode, ayant revêtu les habits royaux et siégeant à la tribune, se mit à les haranguer.
${}^{22}Le peuple l’acclamait à grands cris : « C’est la voix d’un dieu, et non d’un homme ! » 
${}^{23}Mais soudain, l’ange du Seigneur le frappa, parce qu’il n’avait pas rendu gloire à Dieu. Rongé par les vers, il expira.
${}^{24}La parole de Dieu était féconde et se multipliait. 
${}^{25}Barnabé et Saul, une fois leur service accompli en faveur de Jérusalem, s’en retournèrent à Antioche, en prenant avec eux Jean surnommé Marc.
      
         
      \bchapter{}
      \begin{verse}
${}^{1}Or il y avait dans l’Église qui était à Antioche des prophètes et des hommes chargés d’enseigner : Barnabé, Syméon appelé Le Noir, Lucius de Cyrène, Manahène, compagnon d’enfance d’Hérode le Tétrarque, et Saul. 
${}^{2}Un jour qu’ils célébraient le culte du Seigneur et qu’ils jeûnaient, l’Esprit Saint leur dit : « Mettez à part pour moi Barnabé et Saul en vue de l’œuvre à laquelle je les ai appelés. » 
${}^{3}Alors, après avoir jeûné et prié, et leur avoir imposé les mains, ils les laissèrent partir.
      
         
${}^{4}Eux donc, envoyés par le Saint-Esprit, descendirent à Séleucie et de là s’embarquèrent pour Chypre ; 
${}^{5}arrivés à Salamine, ils annonçaient la parole de Dieu dans les synagogues des Juifs. Ils avaient Jean-Marc comme auxiliaire.
${}^{6}Ayant traversé toute l’île jusqu’à Paphos, ils rencontrèrent un mage, un faux prophète ; c’était un juif du nom de Barjésus, 
${}^{7}qui vivait auprès du proconsul Sergius Paulus, un homme avisé. Celui-ci fit venir Barnabé et Saul car il avait le désir d’entendre la parole de Dieu. 
${}^{8}Alors, en face d’eux se dressa Élymas « le mage » – car ainsi se traduit son nom –, qui cherchait à détourner le proconsul de la foi. 
${}^{9}Mais Saul, appelé aussi Paul, rempli d’Esprit Saint, le fixa du regard et dit : 
${}^{10}« Toi qui es plein de toute sorte de fausseté et de méchanceté, fils du diable, ennemi de tout ce qui est juste, n’en finiras-tu pas de faire dévier les chemins du Seigneur, qui sont droits ? 
${}^{11}Maintenant, voici que la main du Seigneur est sur toi : tu vas être aveugle, tu ne verras plus le soleil jusqu’au moment fixé. » Et aussitôt tombèrent sur lui brouillard et ténèbres ; il tournait en rond, cherchant une main pour le guider. 
${}^{12}Alors le proconsul, ayant vu ce qui s’était passé, devint croyant, car il était frappé par l’enseignement du Seigneur.
${}^{13}Paul et ceux qui l’accompagnaient s’embarquèrent à Paphos et arrivèrent à Pergé en Pamphylie. Mais Jean-Marc les abandonna pour s’en retourner à Jérusalem. 
${}^{14}Quant à eux, ils poursuivirent leur voyage au-delà de Pergé et arrivèrent à Antioche de Pisidie. Le jour du sabbat, ils entrèrent à la synagogue et prirent place. 
${}^{15}Après la lecture de la Loi et des Prophètes, les chefs de la synagogue leur envoyèrent dire : « Frères, si vous avez une parole d’exhortation pour le peuple, parlez. »
${}^{16}Paul se leva, fit un signe de la main et dit : « Israélites, et vous aussi qui craignez Dieu, écoutez : 
${}^{17}Le Dieu de ce peuple, le Dieu d’Israël a choisi nos pères ; il a fait grandir son peuple pendant le séjour en Égypte et il l’en a fait sortir à bras étendu. 
${}^{18}Pendant une quarantaine d’années, il les a supportés au désert 
${}^{19}et, après avoir exterminé tour à tour sept nations au pays de Canaan, il a partagé pour eux ce pays en héritage. 
${}^{20}Tout cela dura environ quatre cent cinquante ans. Ensuite, il leur a donné des juges, jusqu’au prophète Samuel. 
${}^{21}Puis ils demandèrent un roi, et Dieu leur donna Saül, fils de Kish, homme de la tribu de Benjamin, pour quarante années. 
${}^{22}Après l’avoir rejeté, Dieu a, pour eux, suscité David comme roi, et il lui a rendu ce témoignage : J’ai trouvé David, fils de Jessé ; c’est un homme selon mon cœur qui réalisera toutes mes volontés. 
${}^{23}De la descendance de David, Dieu, selon la promesse, a fait sortir un sauveur pour Israël : c’est Jésus, 
${}^{24}dont Jean le Baptiste a préparé l’avènement, en proclamant avant lui un baptême de conversion pour tout le peuple d’Israël. 
${}^{25}Au moment d’achever sa course, Jean disait : “Ce que vous pensez que je suis, je ne le suis pas. Mais le voici qui vient après moi, et je ne suis pas digne de retirer les sandales de ses pieds.”
${}^{26}Vous, frères, les fils de la lignée d’Abraham et ceux parmi vous qui craignent Dieu, c’est à nous que la parole du salut a été envoyée. 
${}^{27}En effet, les habitants de Jérusalem et leurs chefs ont méconnu Jésus, ainsi que les paroles des prophètes qu’on lit chaque sabbat ; or, en le jugeant, ils les ont accomplies. 
${}^{28}Sans avoir trouvé en lui aucun motif de condamnation à mort, ils ont demandé à Pilate qu’il soit supprimé. 
${}^{29}Et, après avoir accompli tout ce qui était écrit de lui, ils l’ont descendu du bois de la croix et mis au tombeau. 
${}^{30}Mais Dieu l’a ressuscité d’entre les morts. 
${}^{31}Il est apparu pendant bien des jours à ceux qui étaient montés avec lui de Galilée à Jérusalem, et qui sont maintenant ses témoins devant le peuple.
${}^{32}Et nous, nous vous annonçons cette Bonne Nouvelle : la promesse faite à nos pères, 
${}^{33}Dieu l’a pleinement accomplie pour nous, leurs enfants, en ressuscitant Jésus, comme il est écrit au psaume deux :
        \\Tu es mon fils ;
        \\moi, aujourd’hui, je t’ai engendré.
${}^{34}De fait, Dieu l’a ressuscité des morts sans plus de retour à la condition périssable, comme il l’avait déclaré en disant : Je vous donnerai les réalités saintes promises à David, celles qui sont dignes de foi. 
${}^{35}C’est pourquoi celui-ci dit dans un autre psaume : Tu donneras à ton fidèle de ne pas voir la corruption. 
${}^{36}En effet, David, après avoir, pour sa génération, servi le dessein de Dieu, s’endormit dans la mort, fut déposé auprès de ses pères et il a vu la corruption. 
${}^{37}Mais celui que Dieu a ressuscité n’a pas vu la corruption. 
${}^{38}Sachez-le donc, frères, grâce à Jésus, le pardon des péchés vous est annoncé ; alors que, par la loi de Moïse, vous ne pouvez pas être délivrés de vos péchés ni devenir justes, 
${}^{39}par Jésus, tout homme qui croit devient juste. 
${}^{40}Prenez donc garde de ne pas être atteints par ce qui a été dit dans les Prophètes :
${}^{41}Vous, les arrogants, regardez,
        \\soyez dans la stupeur, disparaissez,
        \\car je fais une œuvre en votre temps,
        \\une œuvre à laquelle vous ne croiriez pas
        \\si on vous la racontait. »
${}^{42}À leur sortie de la synagogue, les gens les invitaient à leur parler encore de tout cela le prochain sabbat. 
${}^{43}Une fois l’assemblée dispersée, beaucoup de Juifs et de convertis qui adorent le Dieu unique les suivirent. Paul et Barnabé, parlant avec eux, les encourageaient à rester attachés à la grâce de Dieu.
${}^{44}Le sabbat suivant, presque toute la ville se rassembla pour entendre la parole du Seigneur. 
${}^{45}Quand les Juifs virent les foules, ils s’enflammèrent de jalousie ; ils contredisaient les paroles de Paul et l’injuriaient. 
${}^{46}Paul et Barnabé leur déclarèrent avec assurance : « C’est à vous d’abord qu’il était nécessaire d’adresser la parole de Dieu. Puisque vous la rejetez et que vous-mêmes ne vous jugez pas dignes de la vie éternelle, eh bien ! nous nous tournons vers les nations païennes. 
${}^{47}C’est le commandement que le Seigneur nous a donné :
        \\J’ai fait de toi la lumière des nations
        \\pour que, grâce à toi,
        \\le salut parvienne jusqu’aux extrémités de la terre. »
${}^{48}En entendant cela, les païens étaient dans la joie et rendaient gloire à la parole du Seigneur ; tous ceux qui étaient destinés à la vie éternelle devinrent croyants. 
${}^{49}Ainsi la parole du Seigneur se répandait dans toute la région.
${}^{50}Mais les Juifs provoquèrent l’agitation parmi les femmes de qualité adorant Dieu, et parmi les notables de la cité ; ils se mirent à poursuivre Paul et Barnabé, et les expulsèrent de leur territoire. 
${}^{51}Ceux-ci secouèrent contre eux la poussière de leurs pieds et se rendirent à Iconium, 
${}^{52}tandis que les disciples étaient remplis de joie et d’Esprit Saint.
      
         
      \bchapter{}
      \begin{verse}
${}^{1}À Iconium, la même chose se produisit : Paul et Barnabé entrèrent dans la synagogue des Juifs, et parlèrent de telle façon qu’un grand nombre de Juifs et de Grecs devinrent croyants. 
${}^{2}Mais ceux des Juifs qui avaient refusé de croire se mirent à exciter les païens et à les monter contre les frères. 
${}^{3}Paul et Barnabé séjournèrent là un certain temps. Ils mettaient leur assurance dans le Seigneur : celui-ci rendait témoignage à l’annonce de la parole de sa grâce, et il leur donnait d’accomplir par leurs mains des signes et des prodiges. 
${}^{4}La population de la ville se trouva divisée : les uns étaient pour les Juifs, les autres pour les Apôtres. 
${}^{5}Il y eut un mouvement chez les non-Juifs et chez les Juifs, avec leurs chefs, pour recourir à la violence et lapider Paul et Barnabé. 
${}^{6}Lorsque ceux-ci s’en aperçurent, ils se réfugièrent en Lycaonie dans les cités de Lystres et de Derbé et dans leurs territoires environnants. 
${}^{7}Là encore, ils annonçaient la Bonne Nouvelle.
      
         
${}^{8}Or, à Lystres, il y avait un homme qui était assis, incapable de se tenir sur ses pieds. Infirme de naissance, il n’avait jamais pu marcher. 
${}^{9}Cet homme écoutait les paroles de Paul. Celui-ci le fixa du regard et vit qu’il avait la foi pour être sauvé. 
${}^{10}Alors il lui dit d’une voix forte : « Lève-toi, tiens-toi droit sur tes pieds. » L’homme se dressa d’un bond : il marchait. 
${}^{11}En voyant ce que Paul venait de faire, les foules s’écrièrent en lycaonien : « Les dieux se sont faits pareils aux hommes, et ils sont descendus chez nous ! » 
${}^{12}Ils donnaient à Barnabé le nom de Zeus, et à Paul celui d’Hermès, puisque c’était lui le porte-parole. 
${}^{13}Le prêtre du temple de Zeus, situé hors de la ville, fit amener aux portes de celle-ci des taureaux et des guirlandes. Il voulait offrir un sacrifice avec les foules.
${}^{14}Informés de cela, les Apôtres Barnabé et Paul déchirèrent leurs vêtements et se précipitèrent dans la foule en criant : 
${}^{15}« Pourquoi faites-vous cela ? Nous aussi, nous sommes des hommes pareils à vous, et nous annonçons la Bonne Nouvelle : détournez-vous de ces vaines pratiques, et tournez-vous vers le Dieu vivant, lui qui a fait le ciel, la terre, la mer, et tout ce qu’ils contiennent. 
${}^{16}Dans les générations passées, il a laissé toutes les nations suivre leurs chemins. 
${}^{17}Pourtant, il n’a pas manqué de donner le témoignage de ses bienfaits, puisqu’il vous a envoyé du ciel la pluie et des saisons fertiles pour vous combler de nourriture et de bien-être. » 
${}^{18}En parlant ainsi, ils empêchèrent, mais non sans peine, la foule de leur offrir un sacrifice.
${}^{19}Alors des Juifs arrivèrent d’Antioche de Pisidie et d’Iconium ; ils se rallièrent les foules, ils lapidèrent Paul et le traînèrent hors de la ville, pensant qu’il était mort. 
${}^{20}Mais, quand les disciples firent cercle autour de lui, il se releva et rentra dans la ville. Le lendemain, avec Barnabé, il partit pour Derbé.
${}^{21}Ils annoncèrent la Bonne Nouvelle à cette cité et firent bon nombre de disciples. Puis ils retournèrent à Lystres, à Iconium et à Antioche de Pisidie ; 
${}^{22}ils affermissaient le courage des disciples ; ils les exhortaient à persévérer dans la foi, en disant : « Il nous faut passer par bien des épreuves pour entrer dans le royaume de Dieu. » 
${}^{23}Ils désignèrent des Anciens pour chacune de leurs Églises et, après avoir prié et jeûné, ils confièrent au Seigneur ces hommes qui avaient mis leur foi en lui.
${}^{24}Ils traversèrent la Pisidie et se rendirent en Pamphylie. 
${}^{25}Après avoir annoncé la Parole aux gens de Pergé, ils descendirent au port d’Attalia, 
${}^{26}et s’embarquèrent pour Antioche de Syrie, d’où ils étaient partis ; c’est là qu’ils avaient été remis à la grâce de Dieu pour l’œuvre qu’ils avaient accomplie. 
${}^{27}Une fois arrivés, ayant réuni l’Église, ils rapportèrent tout ce que Dieu avait fait avec eux, et comment il avait ouvert aux nations la porte de la foi.
${}^{28}Ils passèrent alors un certain temps avec les disciples.
      
         
      \bchapter{}
      \begin{verse}
${}^{1}Des gens, venus de Judée à Antioche, enseignaient les frères en disant : « Si vous n’acceptez pas la circoncision selon la coutume qui vient de Moïse, vous ne pouvez pas être sauvés. » 
${}^{2}Cela provoqua un affrontement ainsi qu’une vive discussion engagée par Paul et Barnabé contre ces gens-là. Alors on décida que Paul et Barnabé, avec quelques autres frères, monteraient à Jérusalem auprès des Apôtres et des Anciens pour discuter de cette question. 
${}^{3}L’Église d’Antioche facilita leur voyage. Ils traversèrent la Phénicie et la Samarie en racontant la conversion des nations, ce qui remplissait de joie tous les frères.
${}^{4}À leur arrivée à Jérusalem, ils furent accueillis par l’Église, les Apôtres et les Anciens, et ils rapportèrent tout ce que Dieu avait fait avec eux.
${}^{5}Alors quelques membres du groupe des pharisiens qui étaient devenus croyants intervinrent pour dire qu’il fallait circoncire les païens et leur ordonner d’observer la loi de Moïse.
${}^{6}Les Apôtres et les Anciens se réunirent pour examiner cette affaire. 
${}^{7}Comme cela provoquait une intense discussion, Pierre se leva et leur dit : « Frères, vous savez bien comment Dieu, dans les premiers temps, a manifesté son choix parmi vous : c’est par ma bouche que les païens ont entendu la parole de l’Évangile et sont venus à la foi. 
${}^{8}Dieu, qui connaît les cœurs, leur a rendu témoignage en leur donnant l’Esprit Saint tout comme à nous ; 
${}^{9}sans faire aucune distinction entre eux et nous, il a purifié leurs cœurs par la foi. 
${}^{10}Maintenant, pourquoi donc mettez-vous Dieu à l’épreuve en plaçant sur la nuque des disciples un joug que nos pères et nous-mêmes n’avons pas eu la force de porter ? 
${}^{11}Oui, nous le croyons, c’est par la grâce du Seigneur Jésus que nous sommes sauvés, de la même manière qu’eux. »
${}^{12}Toute la multitude garda le silence, puis on écouta Barnabé et Paul exposer tous les signes et les prodiges que Dieu avait accomplis grâce à eux parmi les nations. 
${}^{13}Quand ils eurent terminé, Jacques prit la parole et dit : « Frères, écoutez-moi. 
${}^{14}Simon-Pierre vous a exposé comment, dès le début, Dieu est intervenu pour prendre parmi les nations un peuple qui soit à son nom. 
${}^{15}Les paroles des prophètes s’accordent avec cela, puisqu’il est écrit :
        ${}^{16}Après cela, je reviendrai
        \\pour reconstruire la demeure de David,
        \\qui s’est écroulée ;
        \\j’en reconstruirai les parties effondrées,
        \\je la redresserai ;
        ${}^{17}alors le reste des hommes cherchera le Seigneur,
        \\oui, toutes les nations
        \\sur lesquelles mon nom a été invoqué,
        \\– déclare le Seigneur, qui fait ces choses
        ${}^{18}connues depuis toujours.
${}^{19}Dès lors, moi, j’estime qu’il ne faut pas tracasser ceux qui, venant des nations, se tournent vers Dieu, 
${}^{20}mais écrivons-leur de s’abstenir des souillures des idoles, des unions illégitimes, de la viande non saignée et du sang. 
${}^{21}Car, depuis les temps les plus anciens, Moïse a, dans chaque ville, des gens qui proclament sa Loi, puisque, dans les synagogues, on en fait la lecture chaque sabbat. »
${}^{22}Alors les Apôtres et les Anciens décidèrent avec toute l’Église de choisir parmi eux des hommes qu’ils enverraient à Antioche avec Paul et Barnabé. C’étaient des hommes qui avaient de l’autorité parmi les frères : Jude, appelé aussi Barsabbas, et Silas.
${}^{23}Voici ce qu’ils écrivirent de leur main :
        \\« Les Apôtres et les Anciens, vos frères,
        \\aux frères issus des nations,
        \\qui résident à Antioche, en Syrie et en Cilicie,
        \\salut !
${}^{24}Attendu que certains des nôtres, comme nous l’avons appris, sont allés, sans aucun mandat de notre part, tenir des propos qui ont jeté chez vous le trouble et le désarroi, 
${}^{25}nous avons pris la décision, à l’unanimité, de choisir des hommes que nous envoyons chez vous, avec nos frères bien-aimés Barnabé et Paul, 
${}^{26}eux qui ont fait don de leur vie pour le nom de notre Seigneur Jésus Christ.
${}^{27}Nous vous envoyons donc Jude et Silas, qui vous confirmeront de vive voix ce qui suit : 
${}^{28}L’Esprit Saint et nous-mêmes avons décidé de ne pas faire peser sur vous d’autres obligations que celles-ci, qui s’imposent : 
${}^{29}vous abstenir des viandes offertes en sacrifice aux idoles, du sang, des viandes non saignées et des unions illégitimes. Vous agirez bien, si vous vous gardez de tout cela. Bon courage ! »
${}^{30}On laissa donc partir les délégués, et ceux-ci descendirent alors à Antioche. Ayant réuni la multitude des disciples, ils remirent la lettre. 
${}^{31}À sa lecture, tous se réjouirent du réconfort qu’elle apportait. 
${}^{32}Jude et Silas, qui étaient aussi prophètes, parlèrent longuement aux frères pour les réconforter et les affermir. 
${}^{33}Après quelque temps, les frères les laissèrent repartir en paix vers ceux qui les avaient envoyés. 
${}^{35}Quant à Paul et Barnabé, ils séjournaient à Antioche, où ils enseignaient et, avec beaucoup d’autres, annonçaient la Bonne Nouvelle de la parole du Seigneur.
${}^{36}Quelque temps après, Paul dit à Barnabé : « Retournons donc visiter les frères en chacune des villes où nous avons annoncé la parole du Seigneur, pour voir où ils en sont. » 
${}^{37}Barnabé voulait emmener aussi Jean appelé Marc. 
${}^{38}Mais Paul n’était pas d’avis d’emmener cet homme, qui les avait quittés à partir de la Pamphylie et ne les avait plus accompagnés dans leur tâche. 
${}^{39}L’exaspération devint telle qu’ils se séparèrent l’un de l’autre. Barnabé emmena Marc et s’embarqua pour Chypre. 
${}^{40}Paul, lui, choisit pour compagnon Silas et s’en alla, remis par les frères à la grâce du Seigneur. 
${}^{41}Il traversait la Syrie et la Cilicie, en affermissant les Églises.
      
         
      \bchapter{}
      \begin{verse}
${}^{1}Il arriva ensuite à Derbé, puis à Lystres. Il y avait là un disciple nommé Timothée ; sa mère était une Juive devenue croyante, mais son père était Grec. 
${}^{2}À Lystres et à Iconium, les frères lui rendaient un bon témoignage. 
${}^{3}Paul désirait l’emmener ; il le prit avec lui et le fit circoncire à cause des Juifs de la région, car ils savaient tous que son père était Grec.
${}^{4}Dans les villes où Paul et ses compagnons passaient, ils transmettaient les décisions prises par les Apôtres et les Anciens de Jérusalem, pour qu’elles entrent en vigueur. 
${}^{5}Les Églises s’affermissaient dans la foi et le nombre de leurs membres augmentait chaque jour.
${}^{6}Paul et ses compagnons traversèrent la Phrygie et le pays des Galates, car le Saint-Esprit les avait empêchés de dire la Parole dans la province d’Asie. 
${}^{7}Arrivés en Mysie, ils essayèrent d’atteindre la Bithynie, mais l’Esprit de Jésus s’y opposa. 
${}^{8}Ils longèrent alors la Mysie et descendirent jusqu’à Troas.
${}^{9}Pendant la nuit, Paul eut une vision : un Macédonien lui apparut, debout, qui lui faisait cette demande : « Passe en Macédoine et viens à notre secours. » 
${}^{10}À la suite de cette vision de Paul, nous avons aussitôt cherché à partir pour la Macédoine, car nous en avons déduit que Dieu nous appelait à y porter la Bonne Nouvelle.
${}^{11}De Troas nous avons gagné le large et filé tout droit sur l’île de Samothrace, puis, le lendemain, sur Néapolis, 
${}^{12}et ensuite sur Philippes, qui est une cité du premier district de Macédoine et une colonie romaine. Nous avons passé un certain temps dans cette ville 
${}^{13}et, le jour du sabbat, nous en avons franchi la porte pour rejoindre le bord de la rivière, où nous pensions trouver un lieu de prière. Nous nous sommes assis, et nous avons parlé aux femmes qui s’étaient réunies. 
${}^{14}L’une d’elles nommée Lydie, une négociante en étoffes de pourpre, originaire de la ville de Thyatire, et qui adorait le Dieu unique, écoutait. Le Seigneur lui ouvrit l’esprit pour la rendre attentive à ce que disait Paul. 
${}^{15}Quand elle fut baptisée, elle et tous les gens de sa maison, elle nous adressa cette invitation : « Si vous avez reconnu ma foi au Seigneur, venez donc dans ma maison pour y demeurer. » C’est ainsi qu’elle nous a forcé la main.
${}^{16}Comme nous allions au lieu de prière, voilà que vint à notre rencontre une jeune servante qui était possédée par un esprit de divination ; elle rapportait de gros bénéfices à ses maîtres par ses oracles. 
${}^{17}Elle se mit à nous suivre, Paul et nous, et elle criait : « Ces hommes sont des serviteurs du Dieu Très-Haut ; ils vous annoncent le chemin du salut. »
${}^{18}Elle faisait cela depuis plusieurs jours quand Paul, excédé, se retourna et dit à l’esprit : « Au nom de Jésus Christ, je te l’ordonne : Sors ! » Et à l’instant même il sortit.
${}^{19}Les maîtres, voyant s’en aller l’espoir de leurs bénéfices, se saisirent de Paul et de Silas et les traînèrent sur la place publique auprès des autorités. 
${}^{20}Puis, ils les firent comparaître devant les magistrats en disant : « Ces gens troublent notre cité : ils sont Juifs, 
${}^{21}et ils prônent des coutumes que nous n’avons pas le droit d’accepter ni de pratiquer, nous qui sommes citoyens romains. » 
${}^{22}Alors, la foule se déchaîna contre Paul et Silas. Les magistrats ordonnèrent de leur arracher les vêtements pour leur donner la bastonnade. 
${}^{23}Après les avoir roués de coups, on les jeta en prison, en donnant au geôlier la consigne de les surveiller de près. 
${}^{24}Pour appliquer cette consigne, il les mit tout au fond de la prison, avec les pieds coincés dans des blocs de bois.
${}^{25}Vers le milieu de la nuit, Paul et Silas priaient et chantaient les louanges de Dieu, et les autres détenus les écoutaient. 
${}^{26}Tout à coup, il y eut un violent tremblement de terre, qui secoua les fondations de la prison : à l’instant même, toutes les portes s’ouvrirent, et les liens de tous les détenus se détachèrent. 
${}^{27}Le geôlier, tiré de son sommeil, vit que les portes de la prison étaient ouvertes ; croyant que les détenus s’étaient évadés, il dégaina son épée et il était sur le point de se donner la mort.
${}^{28}Mais Paul se mit à crier d’une voix forte : « Ne va pas te faire de mal, nous sommes tous là. » 
${}^{29}Ayant réclamé de la lumière, le geôlier se précipita et, tout tremblant, se jeta aux pieds de Paul et de Silas. 
${}^{30}Puis il les emmena dehors et leur demanda : « Que dois-je faire pour être sauvé, mes seigneurs ? » 
${}^{31}Ils lui répondirent : « Crois au Seigneur Jésus, et tu seras sauvé, toi et toute ta maison. » 
${}^{32}Ils lui annoncèrent la parole du Seigneur, ainsi qu’à tous ceux qui vivaient dans sa maison. 
${}^{33}À l’heure même, en pleine nuit, le geôlier les emmena pour laver leurs plaies. Aussitôt, il reçut le baptême avec tous les siens. 
${}^{34}Puis il fit monter chez lui Paul et Silas, il fit préparer la table et, avec toute sa maison, il laissa déborder sa joie de croire en Dieu.
${}^{35}Quand il fit jour, les magistrats envoyèrent leurs gardes dire au geôlier : « Relâche ces gens ! » 
${}^{36}Le geôlier rapporta ces paroles à Paul : « Les magistrats ont envoyé dire de vous relâcher. Sortez donc maintenant et partez en paix. » 
${}^{37}Mais Paul dit aux gardes : « Ils nous ont fait flageller en public sans jugement, alors que nous sommes citoyens romains, ils nous ont jetés en prison ; et maintenant, c’est à la dérobée qu’ils nous expulsent ! Il n’en est pas question : qu’ils viennent eux-mêmes nous faire sortir ! » 
${}^{38}Les gardes rapportèrent ces paroles aux magistrats. Ceux-ci furent pris de peur en apprenant que c’étaient des Romains. 
${}^{39}Ils vinrent donc les apaiser ; ils les firent sortir en leur demandant de quitter la ville. 
${}^{40}Une fois sortis de la prison, Paul et Silas entrèrent chez Lydie ; ils virent les frères et les réconfortèrent, puis ils partirent.
      
         
      \bchapter{}
      \begin{verse}
${}^{1}Ayant traversé Amphipolis et Apollonia, ils arrivèrent à Thessalonique, où les Juifs avaient une synagogue. 
${}^{2}Suivant son habitude, Paul entra chez eux. Pendant trois sabbats, il discuta avec eux à partir des Écritures, 
${}^{3}dont il ouvrait le sens pour établir que le Christ devait souffrir et ressusciter d’entre les morts ; il ajoutait : « Le Christ, c’est ce Jésus que moi, je vous annonce. » 
${}^{4}Quelques-uns d’entre eux se laissèrent convaincre et s’attachèrent à Paul et à Silas, avec une grande multitude de Grecs qui adoraient Dieu et avec un bon nombre de femmes de notables. 
${}^{5}Mais les Juifs, pris de jalousie, ramassèrent sur la place publique quelques vauriens ; ayant provoqué des attroupements, ils semaient le trouble dans la ville. Ils marchèrent jusqu’à la maison de Jason, à la recherche de Paul et de Silas, pour les faire comparaître devant le peuple.
${}^{6}Ne les trouvant pas, ils traînèrent Jason et quelques frères devant les magistrats, en criant : « Ceux qui ont semé le désordre dans le monde entier, voilà qu’ils sont ici, 
${}^{7}et Jason les accueille ! Ils contreviennent tous aux édits de l’empereur en disant qu’il y a un autre roi : Jésus. » 
${}^{8}Ces Juifs jetèrent ainsi le trouble parmi la foule et les magistrats, qui entendaient cela. 
${}^{9}On fit payer une caution à Jason et aux autres avant de les relâcher.
${}^{10}Aussitôt, les frères firent partir de nuit vers Bérée Paul et Silas qui, dès leur arrivée, se rendirent à la synagogue des Juifs. 
${}^{11}Ceux-ci avaient des sentiments plus nobles que ceux de Thessalonique, et ils accueillirent la Parole de tout leur cœur, interrogeant chaque jour les Écritures pour voir si ce que l’on disait était exact. 
${}^{12}Beaucoup d’entre eux devinrent donc croyants, ainsi que des femmes grecques de qualité et un bon nombre d’hommes.
${}^{13}Mais quand les Juifs de Thessalonique apprirent qu’à Bérée aussi la parole de Dieu était annoncée par Paul, ils vinrent là encore bouleverser les foules et jeter le trouble. 
${}^{14}Alors, aussitôt, les frères firent partir Paul pour qu’il poursuive sa route jusqu’à la mer, tandis que Silas et Timothée restaient là. 
${}^{15}Ceux qui escortaient Paul le conduisirent jusqu’à Athènes. Puis ils s’en retournèrent, porteurs d’un message, avec l’ordre, pour Silas et Timothée, de rejoindre Paul le plus tôt possible.
${}^{16}Pendant que Paul les attendait à Athènes, il avait l’esprit exaspéré en observant la ville livrée aux idoles. 
${}^{17}Il discutait donc à la synagogue avec les Juifs et ceux qui adorent Dieu, ainsi qu’avec ceux qu’il rencontrait chaque jour sur l’Agora. 
${}^{18}Il y avait même des philosophes épicuriens et stoïciens qui venaient s’entretenir avec lui. Certains disaient : « Que peut-il bien vouloir dire, ce radoteur ? » Et d’autres : « On dirait un prêcheur de divinités étrangères. » Ils disaient cela parce que Paul se faisait le messager de « Jésus » et de « Résurrection ».
${}^{19}Ils vinrent le prendre pour le conduire à l’Aréopage. Ils lui disaient : « Pouvons-nous savoir quel est cet enseignement nouveau que tu proposes ? 
${}^{20}Tu nous rebats les oreilles de choses étranges. Nous voulons donc savoir ce que cela signifie. » 
${}^{21}Tous les Athéniens, en effet, ainsi que les étrangers de passage, ne consacraient leur temps à rien d’autre que dire ou écouter la dernière nouveauté.
${}^{22}Alors Paul, debout au milieu de l’Aréopage, fit ce discours : « Athéniens, je peux observer que vous êtes, en toutes choses, des hommes particulièrement religieux. 
${}^{23}En effet, en me promenant et en observant vos monuments sacrés, j’ai même trouvé un autel avec cette inscription : “Au dieu inconnu.” Or, ce que vous vénérez sans le connaître, voilà ce que, moi, je viens vous annoncer. 
${}^{24}Le Dieu qui a fait le monde et tout ce qu’il contient, lui qui est Seigneur du ciel et de la terre, n’habite pas des sanctuaires faits de main d’homme ; 
${}^{25}il n’est pas non plus servi par des mains humaines, comme s’il avait besoin de quoi que ce soit, lui qui donne à tous la vie, le souffle et tout le nécessaire.
${}^{26}À partir d’un seul homme, il a fait tous les peuples pour qu’ils habitent sur toute la surface de la terre, fixant les moments de leur histoire et les limites de leur habitat ; 
${}^{27}Dieu les a faits pour qu’ils le cherchent et, si possible, l’atteignent et le trouvent, lui qui, en fait, n’est pas loin de chacun de nous. 
${}^{28}Car c’est en lui que nous avons la vie, le mouvement et l’être. Ainsi l’ont également dit certains de vos poètes :
        \\Nous sommes de sa descendance.
${}^{29}Si donc nous sommes de la descendance de Dieu, nous ne devons pas penser que la divinité est pareille à une statue d’or, d’argent ou de pierre sculptée par l’art et l’imagination de l’homme. 
${}^{30}Et voici que Dieu, sans tenir compte des temps où les hommes l’ont ignoré, leur enjoint maintenant de se convertir, tous et partout. 
${}^{31}En effet, il a fixé le jour où il va juger la terre avec justice, par un homme qu’il a établi pour cela, quand il l’a accrédité auprès de tous en le ressuscitant d’entre les morts. »
${}^{32}Quand ils entendirent parler de résurrection des morts, les uns se moquaient, et les autres déclarèrent : « Là-dessus nous t’écouterons une autre fois. »
${}^{33}C’est ainsi que Paul, se retirant du milieu d’eux, s’en alla. 
${}^{34}Cependant quelques hommes s’attachèrent à lui et devinrent croyants. Parmi eux, il y avait Denys, membre de l’Aréopage, et une femme nommée Damaris, ainsi que d’autres avec eux.
      
         
      \bchapter{}
      \begin{verse}
${}^{1}Après cela, Paul s’éloigna d’Athènes et se rendit à Corinthe. 
${}^{2}Il y trouva un Juif nommé Aquila, originaire de la province du Pont, récemment arrivé d’Italie, ainsi que sa femme Priscille ; l’empereur Claude, en effet, avait pris la décision d’éloigner de Rome tous les Juifs. Paul entra en relation avec eux ; 
${}^{3}comme ils avaient le même métier, il demeurait chez eux et y travaillait, car ils étaient, de leur métier, fabricants de tentes. 
${}^{4}Chaque sabbat, Paul discutait à la synagogue et s’efforçait de convaincre aussi bien les Juifs que les Grecs.
${}^{5}Quand Silas et Timothée furent arrivés de Macédoine, Paul se consacra entièrement à la Parole, attestant aux Juifs que le Christ, c’est Jésus. 
${}^{6}Devant leur opposition et leurs injures, Paul secoua ses vêtements et leur dit : « Que votre sang soit sur votre tête ! Moi, je n’ai rien à me reprocher. Désormais, j’irai vers les païens. »
${}^{7}Quittant la synagogue, il alla chez un certain Titius Justus, qui adorait le Dieu unique ; sa maison était tout à côté de la synagogue. 
${}^{8}Or Crispus, chef de synagogue, crut au Seigneur, avec toute sa maison. Beaucoup de Corinthiens, apprenant cela, devenaient croyants et se faisaient baptiser.
${}^{9}Une nuit, le Seigneur dit à Paul dans une vision : « Sois sans crainte : parle, ne garde pas le silence. 
${}^{10}Je suis avec toi, et personne ne s’en prendra à toi pour te maltraiter, car dans cette ville j’ai pour moi un peuple nombreux. » 
${}^{11}Paul y séjourna un an et demi et il leur enseignait la parole de Dieu.
${}^{12}Sous le proconsulat de Gallion en Grèce, les Juifs, unanimes, se dressèrent contre Paul et l’amenèrent devant le tribunal, 
${}^{13}en disant : « La manière dont cet individu incite les gens à adorer le Dieu unique est contraire à la loi. » 
${}^{14}Au moment où Paul allait ouvrir la bouche, Gallion déclara aux Juifs : « S’il s’agissait d’un délit ou d’un méfait grave, je recevrais votre plainte à vous, Juifs, comme il se doit. 
${}^{15}Mais s’il s’agit de débats sur des mots, sur des noms et sur la Loi qui vous est propre, cela vous regarde. Être juge en ces affaires, moi je m’y refuse. » 
${}^{16}Et il les chassa du tribunal. 
${}^{17}Tous alors se saisirent de Sosthène, chef de synagogue, et se mirent à le frapper devant le tribunal, tandis que Gallion restait complètement indifférent.
${}^{18}Paul demeura encore assez longtemps à Corinthe. Puis il fit ses adieux aux frères et s’embarqua pour la Syrie, accompagné de Priscille et d’Aquila. À Cencrées, il s’était fait raser la tête, car le vœu qui le liait avait pris fin. 
${}^{19}Ils arrivèrent à Éphèse ; il laissa là ses compagnons, mais lui, entrant à la synagogue, se mit à discuter avec les Juifs. 
${}^{20}Comme ceux-ci lui demandaient de rester plus longtemps, il n’accepta pas. 
${}^{21}En faisant ses adieux, il dit : « Je reviendrai encore chez vous, si Dieu le veut. » Et, quittant Éphèse, il reprit la mer.
${}^{22}Ayant débarqué à Césarée, il monta saluer l’Église de Jérusalem, puis descendit à Antioche. 
${}^{23}Après y avoir passé quelque temps, Paul partit. Il parcourut successivement le pays galate et la Phrygie, en affermissant tous les disciples.
${}^{24}Or, un Juif nommé Apollos, originaire d’Alexandrie, venait d’arriver à Éphèse. C’était un homme éloquent, versé dans les Écritures. 
${}^{25}Il avait été instruit du Chemin du Seigneur ; dans la ferveur de l’Esprit, il parlait et enseignait avec précision ce qui concerne Jésus, mais, comme baptême, il ne connaissait que celui de Jean. 
${}^{26}Il se mit donc à parler avec assurance à la synagogue. Quand Priscille et Aquilas l’entendirent, ils le prirent à part et lui exposèrent avec plus de précision le Chemin de Dieu.
${}^{27}Comme Apollos voulait se rendre en Grèce, les frères l’y encouragèrent, et écrivirent aux disciples de lui faire bon accueil. Quand il fut arrivé, il rendit de grands services à ceux qui étaient devenus croyants par la grâce de Dieu. 
${}^{28}En effet, avec vigueur il réfutait publiquement les Juifs, en démontrant par les Écritures que le Christ, c’est Jésus.
      
         
      \bchapter{}
      \begin{verse}
${}^{1}Pendant qu’Apollos était à Corinthe, Paul traversait le haut pays ; il arriva à Éphèse, où il trouva quelques disciples. 
${}^{2}Il leur demanda : « Lorsque vous êtes devenus croyants, avez-vous reçu l’Esprit Saint ? » Ils lui répondirent : « Nous n’avons même pas entendu dire qu’il y a un Esprit Saint. » 
${}^{3}Paul reprit : « Quel baptême avez-vous donc reçu ? » Ils répondirent : « Celui de Jean le Baptiste. » 
${}^{4}Paul dit alors : « Jean donnait un baptême de conversion : il disait au peuple de croire en celui qui devait venir après lui, c’est-à-dire en Jésus. » 
${}^{5}Après l’avoir entendu, ils se firent baptiser au nom du Seigneur Jésus. 
${}^{6}Et quand Paul leur eut imposé les mains, l’Esprit Saint vint sur eux, et ils se mirent à parler en langues mystérieuses et à prophétiser. 
${}^{7}Ils étaient une douzaine d’hommes au total.
${}^{8}Paul se rendit à la synagogue où, pendant trois mois, il prit la parole avec assurance ; il discutait et usait d’arguments persuasifs à propos du royaume de Dieu. 
${}^{9}Certains s’endurcissaient et refusaient de croire ; devant la multitude, ils dénigraient le Chemin du Seigneur Jésus. C’est pourquoi Paul se sépara d’eux. Il prit les disciples à part et s’entretenait chaque jour avec eux dans l’école de Tyrannos. 
${}^{10}Cela dura deux ans, si bien que tous les habitants de la province d’Asie, Juifs et Grecs, entendirent la parole du Seigneur.
${}^{11}Par les mains de Paul, Dieu faisait des miracles peu ordinaires, 
${}^{12}à tel point que l’on prenait des linges ou des mouchoirs qui avaient touché sa peau, pour les appliquer sur les malades ; alors les maladies les quittaient et les esprits mauvais sortaient.
${}^{13}Certains exorcistes juifs itinérants entreprirent de prononcer le nom du Seigneur Jésus sur ceux qui étaient possédés par les esprits mauvais, en disant : « Je vous exorcise par ce Jésus que Paul proclame. » 
${}^{14}Les sept fils d’un certain Scéva, un grand prêtre juif, agissaient ainsi. 
${}^{15}Mais l’esprit mauvais leur répondit : « Jésus, je le connais ; Paul, je sais qui c’est ; mais vous, qui êtes-vous ? » 
${}^{16}Et, bondissant sur eux, l’homme en qui était l’esprit mauvais les maîtrisa tous avec une telle violence, qu’ils s’enfuirent de la maison, tout nus et couverts de blessures. 
${}^{17}Cela fut connu par tous les Juifs et les Grecs habitant Éphèse ; la crainte s’empara de tous, et l’on exaltait le nom du Seigneur Jésus.
${}^{18}Beaucoup de ceux qui étaient devenus croyants venaient confesser publiquement les pratiques auxquelles ils s’étaient livrés. 
${}^{19}Bon nombre de ceux qui avaient pratiqué les sciences occultes rassemblaient leurs livres et les brûlaient devant tout le monde ; on en évalua le prix : cela faisait cinquante mille pièces d’argent. 
${}^{20}Ainsi, par la force du Seigneur, la Parole était féconde et gagnait en vigueur.
${}^{21}Après ces événements, Paul forma le projet de passer par la Macédoine et la Grèce pour aller à Jérusalem ; il disait : « Après être allé là-bas, il faudra que je voie également Rome. » 
${}^{22}Ayant alors envoyé en Macédoine deux de ses auxiliaires, Timothée et Éraste, lui-même resta un certain temps dans la province d’Asie.
${}^{23}C’est à cette époque qu’il y eut des troubles non négligeables à propos du Chemin du Seigneur Jésus. 
${}^{24}Un orfèvre nommé Démétrios, qui fabriquait des sanctuaires d’Artémis en argent, procurait aux artisans des bénéfices non négligeables. 
${}^{25}Il les réunit, avec les ouvriers qui exerçaient des métiers du même genre, et il leur dit : « Mes amis, vous savez que ces bénéfices sont la source de notre prospérité. 
${}^{26}Or vous voyez bien et vous entendez ce que l’on dit : non seulement à Éphèse mais dans presque toute la province d’Asie, ce Paul, par sa persuasion, a dévoyé toute une foule de gens, en disant que les dieux faits de main d’homme ne sont pas des dieux. 
${}^{27}Cela risque non seulement de jeter le discrédit sur notre profession, mais encore de faire compter pour rien le temple d’Artémis, la grande déesse, et bientôt de la priver de son prestige, elle qui est adorée par toute l’Asie et le monde entier. » 
${}^{28}Remplis de fureur, les auditeurs criaient : « Grande est l’Artémis des Éphésiens ! »
${}^{29}La confusion gagna la ville entière, et les gens se précipitèrent tous ensemble au théâtre, en y entraînant avec eux les Macédoniens Gaïos et Aristarque, compagnons de voyage de Paul. 
${}^{30}Or Paul voulait rejoindre l’assemblée du peuple, mais les disciples ne le laissaient pas faire, 
${}^{31}et quelques-uns des dirigeants de la province, qui étaient ses amis, lui envoyèrent un message pour l’exhorter à ne pas s’exposer en allant au théâtre. 
${}^{32}Les uns criaient une chose, les autres une autre : en effet, l’assemblée était en pleine confusion, et la plupart ne savaient même pas pourquoi ils étaient réunis. 
${}^{33}Des gens dans la foule expliquèrent l’affaire à un certain Alexandre, que les Juifs poussaient en avant. Celui-ci, faisant un geste de la main, voulait plaider devant l’assemblée. 
${}^{34}Mais quand on découvrit qu’il était Juif, tous se mirent à crier d’une seule voix pendant près de deux heures : « Grande est l’Artémis des Éphésiens ! »
${}^{35}Le secrétaire de la cité, ayant calmé la foule, prit la parole : « Éphésiens, quel homme en ce monde ignore que la cité d’Éphèse est la gardienne du temple de la grande Artémis et de sa statue tombée du ciel ? 
${}^{36}Cela est incontestable. Il vous faut donc garder votre calme et éviter toute action précipitée. 
${}^{37}Vous avez amené ici ces hommes, qui n’ont commis ni vol sacrilège, ni blasphème contre notre déesse. 
${}^{38}Si donc Démétrios et les artisans qui l’accompagnent ont un grief contre quelqu’un, il existe des jours d’audience, et il y a des proconsuls : qu’ils portent plainte. 
${}^{39}Mais si vous avez d’autres requêtes, cela se réglera à l’assemblée prévue par la loi. 
${}^{40}En effet, avec l’affaire d’aujourd’hui, nous risquons d’être accusés d’émeute, car nous ne pourrons alléguer aucun motif pour rendre compte de ce rassemblement. » Ayant ainsi parlé, il renvoya l’assemblée.
      
         
      \bchapter{}
      \begin{verse}
${}^{1}Quand le tumulte se fut calmé, Paul fit venir les disciples et les encouragea ; puis, les ayant salués, il se mit en route pour la Macédoine.
      
         
${}^{2}Après avoir traversé la région en adressant aux disciples de nombreuses paroles d’encouragement, il arriva en Grèce 
${}^{3}et y passa trois mois. Il allait prendre la mer pour la Syrie, lorsqu’à la suite d’un complot des Juifs contre lui, il décida de repasser par la Macédoine. 
${}^{4}Il était accompagné par Sopatros, fils de Pyrrhos de Bérée, par Aristarque et Secundus de Thessalonique, par Gaïos de Derbé, par Timothée, ainsi que par Tychique et Trophime de la province d’Asie. 
${}^{5}Ces derniers étaient partis en avant et nous attendaient à Troas. 
${}^{6}Quant à nous, après la Pâque, nous avons embarqué à Philippes ; et, au bout de cinq jours, nous les avons rejoints à Troas, où nous avons passé sept jours.
${}^{7}Le premier jour de la semaine, nous étions rassemblés pour rompre le pain, et Paul, qui devait partir le lendemain, s’entretenait avec ceux qui étaient là. Il continua de parler jusqu’au milieu de la nuit, 
${}^{8}car, dans la salle du haut où nous étions rassemblés, il y avait suffisamment de lampes. 
${}^{9}Un jeune garçon nommé Eutyque, assis sur le rebord de la fenêtre, fut gagné par un profond sommeil tandis que Paul prolongeait l’entretien ; pris par le sommeil, il tomba du troisième étage et, quand on le souleva, il était mort. 
${}^{10}Paul descendit, se précipita sur lui et le prit dans ses bras en disant : « Ne vous agitez pas ainsi : le souffle de vie est en lui ! » 
${}^{11}Il remonta, rompit le pain et mangea ; puis il conversa avec eux assez longtemps, jusqu’à l’aube ; ensuite il s’en alla. 
${}^{12}Quant au garçon, on l’emmena bien vivant, et ce fut un immense réconfort.
${}^{13}Pour nous, ayant pris les devants par bateau, nous avons gagné le large pour Assos, où nous devions reprendre Paul ; celui-ci, en effet, devait y aller par la route : ainsi en avait-il disposé. 
${}^{14}Lorsqu’il nous a rejoints à Assos, nous l’avons repris pour aller à Mitylène. 
${}^{15}Nous avons embarqué le lendemain et, de là, nous sommes parvenus en face de Khios ; le jour suivant, nous avons fait la traversée jusqu’à Samos, et le jour d’après nous sommes allés jusqu’à Milet. 
${}^{16}En effet, Paul avait pris la décision de passer au large d’Éphèse pour ne pas avoir à rester trop longtemps dans la province d’Asie, car il se hâtait pour être, si possible, à Jérusalem le jour de la Pentecôte.
${}^{17}Depuis Milet, il envoya un message à Éphèse pour convoquer les Anciens de cette Église. 
${}^{18}Quand ils furent arrivés auprès de lui, il leur adressa la parole : « Vous savez comment je me suis toujours comporté avec vous, depuis le premier jour où j’ai mis le pied en Asie : 
${}^{19}j’ai servi le Seigneur en toute humilité, dans les larmes et les épreuves que m’ont values les complots des Juifs ; 
${}^{20}je n’ai rien négligé de ce qui était utile, pour vous annoncer l’Évangile et vous donner un enseignement en public ou de maison en maison. 
${}^{21}Je rendais témoignage devant Juifs et Grecs pour qu’ils se convertissent à Dieu et croient en notre Seigneur Jésus.
${}^{22}Et maintenant, voici que je suis contraint par l’Esprit de me rendre à Jérusalem, sans savoir ce qui va m’arriver là-bas. 
${}^{23}Je sais seulement que l’Esprit Saint témoigne, de ville en ville, que les chaînes et les épreuves m’attendent. 
${}^{24}Mais en aucun cas, je n’accorde du prix à ma vie, pourvu que j’achève ma course et le ministère que j’ai reçu du Seigneur Jésus : rendre témoignage à l’évangile de la grâce de Dieu.
${}^{25}Et maintenant, je sais que vous ne reverrez plus mon visage, vous tous chez qui je suis passé en proclamant le Royaume. 
${}^{26}C’est pourquoi j’atteste aujourd’hui devant vous que je suis pur du sang de tous, 
${}^{27}car je n’ai rien négligé pour vous annoncer tout le dessein de Dieu.
${}^{28}Veillez sur vous-mêmes, et sur tout le troupeau dont l’Esprit Saint vous a établis responsables, pour être les pasteurs de l’Église de Dieu, qu’il s’est acquise par son propre sang. 
${}^{29}Moi, je sais qu’après mon départ, des loups redoutables s’introduiront chez vous et n’épargneront pas le troupeau. 
${}^{30}Même du milieu de vous surgiront des hommes qui tiendront des discours pervers pour entraîner les disciples à leur suite. 
${}^{31}Soyez donc vigilants, et souvenez-vous que, durant trois ans, nuit et jour, je n’ai cessé, dans les larmes, de reprendre chacun d’entre vous.
${}^{32}Et maintenant, je vous confie à Dieu et à la parole de sa grâce, lui qui a le pouvoir de construire l’édifice et de donner à chacun l’héritage en compagnie de tous ceux qui ont été sanctifiés. 
${}^{33}Je n’ai convoité ni l’argent ni l’or ni le vêtement de personne. 
${}^{34}Vous le savez bien vous-mêmes : les mains que voici ont pourvu à mes besoins et à ceux de mes compagnons. 
${}^{35}En toutes choses, je vous ai montré qu’en se donnant ainsi de la peine, il faut secourir les faibles et se souvenir des paroles du Seigneur Jésus, car lui-même a dit : Il y a plus de bonheur à donner qu’à recevoir. »
${}^{36}Quand Paul eut ainsi parlé, il s’agenouilla et pria avec eux tous. 
${}^{37}Tous se mirent à pleurer abondamment ; ils se jetaient au cou de Paul et l’embrassaient ; 
${}^{38}ce qui les affligeait le plus, c’est la parole qu’il avait dite : « Vous ne verrez plus mon visage ». Puis on l’accompagna jusqu’au bateau.
      
         
      \bchapter{}
      \begin{verse}
${}^{1}Alors, après nous être séparés d’eux, nous avons gagné le large et filé droit sur Cos, le lendemain sur Rhodes, et de là sur Patara. 
${}^{2}Puis, ayant trouvé un bateau qui faisait la traversée vers la Phénicie, nous sommes montés à bord et nous avons gagné le large. 
${}^{3}Arrivés en vue de Chypre, nous avons laissé l’île sur notre gauche ; nous avons navigué vers la Syrie et nous avons débarqué à Tyr : c’est là, en effet, que le bateau déchargeait sa cargaison. 
${}^{4}Ayant trouvé les disciples, nous sommes restés sept jours avec eux ; ceux-ci, poussés par l’Esprit, disaient à Paul de ne pas monter à Jérusalem. 
${}^{5}Mais quand notre séjour a été achevé, nous sommes partis et nous avons repris la route, accompagnés jusqu’en dehors de la ville par tous, y compris les femmes et les enfants. À genoux sur le rivage, nous avons prié ; 
${}^{6}après nous être dit adieu les uns aux autres, nous avons embarqué à nouveau sur le bateau, tandis qu’ils retournaient chez eux.
${}^{7}Quant à nous, achevant notre traversée, de Tyr nous sommes arrivés à Ptolémaïs ; ayant salué les frères, nous avons passé une journée chez eux.
${}^{8}Partis le lendemain, nous sommes allés à Césarée, nous sommes entrés dans la maison de Philippe, l’évangélisateur, qui était l’un des Sept, et nous sommes restés chez lui. 
${}^{9}Il avait quatre filles non mariées, qui prophétisaient.
${}^{10}Comme nous restions là plusieurs jours, un prophète nommé Agabos descendit de Judée. 
${}^{11}Il vint vers nous, enleva la ceinture de Paul, se ligota les pieds et les mains, et déclara : « Voici ce que dit l’Esprit Saint : L’homme à qui appartient cette ceinture, les Juifs le ligoteront de la sorte à Jérusalem et le livreront aux mains des nations. »
${}^{12}Quand nous avons entendu cela, nous et les frères qui habitaient là, nous l’exhortions à ne pas monter à Jérusalem. 
${}^{13}Alors Paul répondit : « Que faites-vous là à pleurer et à me briser le cœur ? Moi je suis prêt, non seulement à me laisser ligoter, mais encore à mourir à Jérusalem pour le nom du Seigneur Jésus. » 
${}^{14}N’ayant pu le persuader, nous n’avons pas insisté, et nous avons dit : « Que la volonté du Seigneur soit faite. »
${}^{15}À la fin du séjour, nos préparatifs étant achevés, nous sommes montés à Jérusalem. 
${}^{16}Quelques disciples, venus avec nous de Césarée, nous conduisirent chez un certain Mnason de Chypre, un disciple des premiers jours, pour que nous y recevions l’hospitalité.
${}^{17}Les frères, à notre arrivée à Jérusalem, nous ont fait bon accueil. 
${}^{18}Le lendemain, Paul s’est rendu avec nous chez Jacques, où sont arrivés tous les Anciens. 
${}^{19}Après les avoir salués, il leur expliqua en détail ce que Dieu avait fait parmi les nations à travers son ministère.
${}^{20}L’ayant écouté, ils glorifiaient Dieu. Ils dirent à Paul : « Tu vois, frère, combien de dizaines de milliers de Juifs sont devenus croyants, et ils ont tous une ardeur jalouse pour la Loi. 
${}^{21}Or ils ont entendu ce que l’on colporte à ton sujet : par ton enseignement, tu détournes de Moïse tous les Juifs des nations, en leur disant de ne pas circoncire leurs enfants et de ne pas suivre les coutumes. 
${}^{22}Que faut-il donc faire ? De toute façon, ils apprendront ton arrivée. 
${}^{23}Fais donc ce que nous allons te dire. Nous avons ici quatre hommes qui sont tenus par un vœu. 
${}^{24}Prends-les avec toi, accomplis la purification en même temps qu’eux, et paie ce qu’il faut pour qu’ils se fassent raser la tête. Alors tout le monde saura qu’il n’y a rien de vrai dans ce que l’on colporte sur toi, mais que tu marches, toi aussi, en gardant la Loi. 
${}^{25}Quant aux croyants venus des nations, nous leur avons écrit nos décisions : ils doivent se garder des viandes offertes en sacrifice aux idoles, du sang, de la viande non saignée, et des unions illégitimes. »
${}^{26}Alors, le lendemain, Paul prit ces hommes avec lui, accomplit la purification en même temps qu’eux, et il entra dans le Temple pour indiquer à quelle date, le temps de la purification étant achevé, l’offrande serait présentée pour chacun d’eux.
${}^{27}Les sept jours de la purification allaient s’achever, quand les Juifs venus de la province d’Asie, voyant Paul dans le Temple, semèrent la confusion dans toute la foule et mirent la main sur lui, 
${}^{28}en s’écriant : « Israélites, au secours ! Voilà l’homme qui, auprès de tous et partout, répand son enseignement contre le peuple, contre la Loi et contre ce Lieu ! Bien plus, il a aussi fait entrer des Grecs dans le Temple, il a souillé ce Lieu saint ! »
${}^{29}En effet, ils avaient vu auparavant Trophime d’Éphèse avec Paul dans la ville, et ils pensaient que celui-ci l’avait introduit dans le Temple. 
${}^{30}La ville tout entière s’agita, le peuple accourut de toutes parts, on se saisit de Paul et on l’entraîna hors du Temple, dont on ferma aussitôt les portes. 
${}^{31}Tandis qu’on cherchait à le tuer, l’officier romain commandant la cohorte fut informé que tout Jérusalem était en pleine confusion. 
${}^{32}Il prit immédiatement avec lui des soldats et des centurions, et descendit en courant vers la foule. À la vue du commandant et des soldats, on cessa de frapper Paul. 
${}^{33}Alors le commandant s’approcha, se saisit de lui et ordonna de l’attacher avec deux chaînes ; puis il demanda qui il était et ce qu’il avait fait. 
${}^{34}Dans la foule, les uns hurlaient une chose, les autres une autre. Ne réussissant pas à savoir quelque chose de précis à cause du tumulte, il donna l’ordre de conduire Paul à la forteresse. 
${}^{35}En arrivant aux marches de l’escalier, on dut le faire porter par les soldats à cause de la violence de la foule, 
${}^{36}car la multitude du peuple suivait en criant : « Mort à cet homme ! »
${}^{37}Comme on allait le faire entrer dans la forteresse, Paul dit au commandant : « M’est-il permis de te dire quelque chose ? » Il répondit : « Tu sais le grec ? 
${}^{38}Tu n’es donc pas l’Égyptien qui, voici quelques jours, a soulevé et entraîné au désert les quatre mille bandits ? » 
${}^{39}Paul dit : « Moi, je suis un Juif, de Tarse en Cilicie, citoyen d’une ville qui n’est pas insignifiante ! Je t’en prie, permets-moi de parler au peuple. » 
${}^{40}Avec sa permission, Paul, debout sur les marches, fit signe de la main au peuple. Un grand silence s’établit, et il prit la parole en araméen :
      
         
      \bchapter{}
      \begin{verse}
${}^{1}« Frères et pères, écoutez ce que j’ai à vous dire maintenant pour ma défense. » 
${}^{2}Quand ils l’entendirent s’adresser à eux en araméen, le calme se fit plus grand encore. Il leur dit : 
${}^{3}« Je suis Juif, né à Tarse en Cilicie, mais élevé ici dans cette ville, où, à l’école de Gamaliel, j’ai reçu une éducation strictement conforme à la Loi de nos pères ; j’avais pour Dieu une ardeur jalouse, comme vous tous aujourd’hui. 
${}^{4}J’ai persécuté à mort ceux qui suivent le Chemin du Seigneur Jésus ; j’arrêtais hommes et femmes, et les jetais en prison ; 
${}^{5}le grand prêtre et tout le collège des anciens peuvent en témoigner. Ces derniers m’avaient donné des lettres pour nos frères de Damas où je me rendais : je devais ramener à Jérusalem, ceux de là-bas, enchaînés, pour qu’ils subissent leur châtiment.
${}^{6}Donc, comme j’étais en route et que j’approchais de Damas, soudain vers midi, une grande lumière venant du ciel m’enveloppa de sa clarté. 
${}^{7}Je tombai sur le sol, et j’entendis une voix me dire : “Saul, Saul, pourquoi me persécuter ?” 
${}^{8}Et moi je répondis : “Qui es-tu, Seigneur ? – Je suis Jésus le Nazaréen, celui que tu persécutes.” 
${}^{9}Ceux qui étaient avec moi virent la lumière, mais n’entendirent pas la voix de celui qui me parlait. 
${}^{10}Alors je dis : “Que dois-je faire, Seigneur ?” Le Seigneur me répondit : “Relève-toi, va jusqu’à Damas ; et là on te dira tout ce qu’il t’est prescrit de faire.”
${}^{11}Comme je n’y voyais plus rien, à cause de l’éclat de cette lumière, je me rendis à Damas, conduit par la main de mes compagnons. 
${}^{12}Or, Ananie, un homme religieux selon la Loi, à qui tous les Juifs résidant là rendaient un bon témoignage, 
${}^{13}vint se placer près de moi et me dit : “Saul, mon frère, retrouve la vue.” Et moi, au même instant, je retrouvai la vue, et je le vis. 
${}^{14}Il me dit encore : “Le Dieu de nos pères t’a destiné à connaître sa volonté, à voir celui qui est le Juste et à entendre la voix qui sort de sa bouche. 
${}^{15}Car tu seras pour lui, devant tous les hommes, le témoin de ce que tu as vu et entendu. 
${}^{16}Et maintenant, pourquoi tarder ? Lève-toi et reçois le baptême, sois lavé de tes péchés en invoquant son nom.”
${}^{17}Revenu à Jérusalem, j’étais en prière dans le Temple quand je tombai en extase. 
${}^{18}Je vis le Seigneur qui me disait : “Hâte-toi, sors vite de Jérusalem, car ils n’accueilleront pas ton témoignage à mon sujet.” 
${}^{19}Et moi je répondis : “Seigneur, ces gens le savent bien : c’est moi qui allais d’une synagogue à l’autre pour mettre en prison et faire flageller ceux qui croyaient en toi ; 
${}^{20}et quand on versait le sang d’Étienne ton témoin, je me tenais là, moi aussi ; j’étais d’accord, et je gardais les vêtements de ses meurtriers.” 
${}^{21}Il me dit alors : “Va, car moi je vais t’envoyer au loin, vers les nations.” »
${}^{22}Jusqu’à cette parole, les gens l’écoutaient. Mais alors, ils se mirent à élever la voix : « Débarrassez la terre d’un tel individu ! Il ne faut pas qu’il vive ! » 
${}^{23}Ils poussaient des cris, arrachaient leurs vêtements, jetaient de la poussière en l’air. 
${}^{24}Alors le commandant ordonna de le faire entrer dans la forteresse. Il dit de procéder à un interrogatoire par le fouet, afin de savoir pour quel motif on criait contre lui de cette manière.
${}^{25}Comme on l’étendait en l’attachant avec des courroies, Paul dit au centurion de service : « Un citoyen romain, qui n’a même pas été jugé, avez-vous le droit de lui donner le fouet ? » 
${}^{26}Quand le centurion entendit cela, il alla trouver le commandant pour le mettre au courant : « Qu’allais-tu faire ? Cet homme est un Romain ! » 
${}^{27}Le commandant alla trouver Paul et lui demanda : « Dis-moi : tu es romain, toi ? – Oui, répondit-il. » 
${}^{28}Le commandant reprit : « Moi, j’ai payé une grosse somme pour acquérir cette citoyenneté. » Paul répliqua : « Moi, je l’ai de naissance. » 
${}^{29}Aussitôt, ceux qui allaient procéder à l’interrogatoire se retirèrent ; et le commandant prit peur en se rendant compte que c’était un citoyen romain et qu’il l’avait fait ligoter.
${}^{30}Le lendemain, le commandant voulut savoir avec certitude de quoi les Juifs l’accusaient. Il lui fit enlever ses liens ; puis il convoqua les grands prêtres et tout le Conseil suprême, et il fit descendre Paul pour l’amener devant eux.
      
         
      \bchapter{}
      \begin{verse}
${}^{1}Fixant du regard le Conseil suprême, Paul déclara :
      « Frères, c’est en toute bonne conscience que je me suis comporté devant Dieu jusqu’à ce jour. » 
${}^{2}Le grand prêtre Ananias ordonna à ceux qui étaient auprès de lui de le frapper sur la bouche. 
${}^{3}Alors Paul lui dit : « C’est Dieu qui va te frapper, espèce de mur blanchi ! Tu sièges ici pour me juger conformément à la Loi, et contrairement à la Loi tu donnes l’ordre de me frapper ! » 
${}^{4}Ceux qui étaient là dirent : « Tu insultes le grand prêtre de Dieu ? » 
${}^{5}Paul reprit : « Je ne savais pas, frères, que c’était le grand prêtre. Il est écrit en effet : Tu ne diras pas de mal d’un chef de ton peuple. »
${}^{6}Sachant que le Conseil suprême se répartissait entre sadducéens et pharisiens, Paul s’écria devant eux : « Frères, moi, je suis pharisien, fils de pharisiens. C’est à cause de notre espérance, la résurrection des morts, que je passe en jugement. » 
${}^{7}À peine avait-il dit cela, qu’il y eut un affrontement entre pharisiens et sadducéens, et l’assemblée se divisa. 
${}^{8}En effet, les sadducéens disent qu’il n’y a pas de résurrection, pas plus que d’ange ni d’esprit, tandis que les pharisiens professent tout cela. 
${}^{9}Il se fit alors un grand vacarme. Quelques scribes du côté des pharisiens se levèrent et protestèrent vigoureusement : « Nous ne trouvons rien de mal chez cet homme. Et si c’était un esprit qui lui avait parlé, ou un ange ? »
${}^{10}L’affrontement devint très violent, et le commandant craignit que Paul ne se fasse écharper. Il ordonna à la troupe de descendre pour l’arracher à la mêlée et le ramener dans la forteresse.
${}^{11}La nuit suivante, le Seigneur vint auprès de Paul et lui dit : « Courage ! Le témoignage que tu m’as rendu à Jérusalem, il faut que tu le rendes aussi à Rome. »
${}^{12}Lorsqu’il fit jour, les Juifs organisèrent un rassemblement où ils se jurèrent, sous peine d’anathème, de ne plus manger ni boire tant qu’ils n’auraient pas tué Paul. 
${}^{13}Les auteurs de cette conjuration étaient plus de quarante. 
${}^{14}Ils vinrent trouver les grands prêtres et les anciens pour leur dire : « Nous nous sommes juré, sous peine d’anathème, de ne prendre aucune nourriture tant que nous n’aurons pas tué Paul. 
${}^{15}Alors vous, d’accord avec le Conseil suprême, faites un rapport au commandant pour qu’il le fasse comparaître devant vous sous prétexte de mener une enquête plus approfondie sur son cas. Nous nous tenons prêts pour le supprimer avant qu’il n’arrive. »
${}^{16}Mais le fils de la sœur de Paul eut connaissance du guet-apens ; il se présenta à la forteresse et, une fois entré, avertit Paul. 
${}^{17}Paul alors appela l’un des centurions et lui dit : « Emmène ce garçon chez le commandant : il doit l’avertir de quelque chose. » 
${}^{18}Le centurion le prit avec lui et le mena chez le commandant, auquel il dit : « Le prisonnier Paul m’a appelé pour me demander de t’amener ce jeune garçon qui a quelque chose à te dire. » 
${}^{19}Le commandant prit celui-ci par la main, l’emmena à l’écart et l’interrogea en particulier : « De quoi dois-tu m’avertir ? » 
${}^{20}Il répondit : « Les Juifs ont convenu de te demander de faire comparaître Paul demain devant le Conseil suprême sous prétexte d’une information plus approfondie sur son cas. 
${}^{21}Mais toi, ne leur fais pas confiance ; en effet, parmi eux plus de quarante hommes préparent un guet-apens contre lui : ils se sont juré, sous peine d’anathème, de ne plus manger ni boire tant qu’ils ne l’auront pas supprimé. Et maintenant, ils se tiennent prêts en attendant ton accord. » 
${}^{22}Le commandant renvoya le jeune garçon en lui donnant cette consigne : « Ne raconte à personne que tu m’as rapporté tout cela. »
${}^{23}Il appela alors deux centurions et leur dit : « Que deux cents soldats, soixante-dix cavaliers et deux cents auxiliaires se tiennent prêts à prendre la route de Césarée à partir de la troisième heure de la nuit ; 
${}^{24}qu’on prépare aussi des montures pour transférer Paul en toute sécurité auprès du gouverneur Félix. » 
${}^{25}Il écrivit une lettre dont voici le contenu :
${}^{26}« Claudius Lysias, au Très excellent Félix, gouverneur, salut. 
${}^{27}L’homme que voici, dont les Juifs se sont emparés, allait être supprimé par eux. Je suis alors intervenu avec la troupe pour le soustraire au danger, ayant appris qu’il est citoyen romain. 
${}^{28}Voulant connaître le motif pour lequel les Juifs l’accusaient, je l’ai fait comparaître devant leur Conseil suprême. 
${}^{29}J’ai constaté qu’il était accusé pour des questions relatives à leur Loi, sans aucun chef d’accusation méritant la mort ou la prison. 
${}^{30}Après dénonciation devant moi d’un complot contre cet homme, je te l’ai envoyé immédiatement, en donnant également aux accusateurs la consigne d’exposer devant toi ce qu’ils ont contre lui. »
${}^{31}Les soldats prirent donc Paul conformément aux ordres reçus, et ils le conduisirent de nuit jusqu’à Antipatris. 
${}^{32}Le lendemain, ils laissèrent partir avec lui les cavaliers et regagnèrent la forteresse. 
${}^{33}À leur arrivée à Césarée, après avoir remis la lettre au gouverneur, ils lui présentèrent Paul. 
${}^{34}Le gouverneur lut la lettre et demanda de quelle province il était ; apprenant qu’il était de Cilicie, 
${}^{35}il dit : « Je t’entendrai quand tes accusateurs se présenteront, eux aussi. » Et il ordonna de l’incarcérer au prétoire d’Hérode.
      
         
      \bchapter{}
      \begin{verse}
${}^{1}Cinq jours plus tard, le grand prêtre Ananias descendit à Césarée avec quelques anciens et un avocat, un certain Tertullus. Ils exposèrent devant le gouverneur leurs griefs contre Paul. 
${}^{2}On fit appeler celui-ci, et Tertullus commença son discours d’accusation : « Nous qui jouissons d’une grande paix grâce à toi et aux réformes dont ta prévoyance a fait bénéficier cette nation, 
${}^{3}nous accueillons, de toute manière et en tout lieu, ce qui nous vient de toi, Très excellent Félix, avec une immense reconnaissance. 
${}^{4}Mais pour ne pas t’importuner davantage, je te prie de nous écouter un instant avec toute ta bienveillance. 
${}^{5}Nous avons constaté que cet homme est un fléau ; il suscite l’émeute chez tous les Juifs du monde entier, étant le chef du groupe des Nazaréens. 
${}^{6}Il a même tenté de profaner le Temple ; alors nous l’avons arrêté. 
${}^{8}En l’interrogeant lui-même, tu pourras mieux connaître tout ce dont nous l’accusons. » 
${}^{9}Les Juifs appuyèrent ce discours en affirmant qu’il en était bien ainsi.
${}^{10}Le gouverneur lui ayant fait signe de parler, Paul répliqua : « Sachant que, depuis des années, tu as cette nation sous ta juridiction, c’est avec confiance que je présente la défense de ma cause. 
${}^{11}Tu peux vérifier qu’il n’y a pas plus de douze jours que je suis monté à Jérusalem pour adorer. 
${}^{12}On ne m’a pas trouvé dans le Temple en train de discuter avec qui que ce soit, ni dans les synagogues ou en ville en train d’ameuter la foule, 
${}^{13}et ils ne peuvent alléguer aucun fait à l’appui de ce dont ils m’accusent maintenant. 
${}^{14}Mais je le déclare devant toi : c’est selon le Chemin du Seigneur – ce qu’ils désignent comme un groupe – que je rends un culte au Dieu de nos pères ; je crois à tout ce qu’il y a dans la Loi et à tout ce qui est écrit dans les prophètes ; 
${}^{15}mon espérance en Dieu, et ce qu’ils attendent eux-mêmes, c’est qu’il va y avoir une résurrection des justes et des injustes. 
${}^{16}C’est pourquoi, moi aussi, je m’efforce de garder une conscience irréprochable en toute chose devant Dieu et devant les hommes. 
${}^{17}Au bout de plusieurs années, je suis venu apporter le produit des aumônes destinées à ma nation, et présenter des offrandes rituelles. 
${}^{18}C’est à cette occasion qu’on m’a trouvé dans le Temple après une cérémonie de purification, sans qu’il y ait eu ni attroupement ni tumulte. 
${}^{19}Il y avait, ce jour-là, des Juifs venus de la province d’Asie ; ils devraient se présenter devant toi et m’accuser s’ils avaient quelque chose contre moi. 
${}^{20}Ou bien alors, que ceux qui sont là disent quel délit ils ont constaté quand j’ai comparu devant le Conseil suprême. 
${}^{21}À moins qu’il ne s’agisse de cette seule parole que j’ai criée, debout au milieu d’eux : “C’est à cause de la résurrection des morts que je passe aujourd’hui en jugement devant vous.” »
${}^{22}Félix, qui avait une connaissance approfondie de ce qui concerne le Chemin du Seigneur, ajourna l’audience en disant : « Quand le commandant Lysias descendra de Jérusalem, je rendrai une sentence sur votre affaire. » 
${}^{23}Il donna l’ordre au centurion de garder Paul en détention avec un régime adouci, et sans empêcher les siens de lui rendre des services.
${}^{24}Quelques jours plus tard, Félix vint avec sa femme Drusille, qui était juive. Il envoya chercher Paul et l’écouta parler de la foi au Christ Jésus. 
${}^{25}Mais quand l’entretien porta sur la justice, la maîtrise de soi et le jugement à venir, Félix fut pris de peur et déclara : « Pour le moment, retire-toi ; je te rappellerai à une prochaine occasion. » 
${}^{26}Il n’en espérait pas moins que Paul lui donnerait de l’argent ; c’est pourquoi il l’envoyait souvent chercher pour parler avec lui.
${}^{27}Deux années s’écoulèrent ; Félix reçut comme successeur Porcius Festus. Voulant accorder une faveur aux Juifs, Félix avait laissé Paul en prison.
      
         
      \bchapter{}
      \begin{verse}
${}^{1}Trois jours après avoir rejoint sa province, Festus monta de Césarée à Jérusalem. 
${}^{2}Les grands prêtres et les notables juifs exposèrent devant lui leurs griefs contre Paul ; avec insistance, 
${}^{3}ils demandaient comme une faveur le transfert de Paul à Jérusalem ; en fait, ils préparaient un guet-apens pour le supprimer en chemin. 
${}^{4}Festus répondit que Paul était détenu à Césarée, et que lui-même allait repartir incessamment. 
${}^{5}Il déclara : « Que ceux d’entre vous qui sont experts en la matière descendent avec moi, et présentent leur accusation s’il y a quelque chose à reprocher à cet homme. »
${}^{6}Ayant passé chez eux huit à dix jours au plus, il redescendit à Césarée. Le lendemain, il siégea au tribunal, et ordonna d’amener Paul. 
${}^{7}Quand celui-ci fut arrivé, les Juifs descendus de Jérusalem l’entourèrent et multiplièrent contre lui de graves motifs d’accusation qu’ils ne pouvaient pas démontrer, 
${}^{8}tandis que Paul se défendait : « Je n’ai commis de faute ni contre la loi des Juifs, ni contre le Temple, ni contre l’empereur. » 
${}^{9}Festus, voulant accorder une faveur aux Juifs, s’adressa à Paul : « Veux-tu monter à Jérusalem pour y être jugé sur cette affaire en ma présence ? » 
${}^{10}Paul répondit : « Je suis ici devant le tribunal impérial : c’est là qu’il me faut être jugé. Je ne suis coupable de rien contre les Juifs, comme toi-même tu t’en rends fort bien compte. 
${}^{11}Si donc je suis coupable et si j’ai fait quelque chose qui mérite la mort, je ne refuse pas de mourir. Mais s’il ne reste rien des accusations que ces gens-là portent contre moi, personne ne peut leur faire la faveur de me livrer à eux. J’en appelle à l’empereur. » 
${}^{12}Alors, après avoir conféré avec son conseil, Festus déclara : « Tu en as appelé à l’empereur, tu iras devant l’empereur. »
${}^{13}Quelques jours plus tard, le roi Agrippa et Bérénice vinrent à Césarée saluer le gouverneur Festus. 
${}^{14}Comme ils passaient là plusieurs jours, Festus exposa au roi la situation de Paul en disant : « Il y a ici un homme que mon prédécesseur Félix a laissé en prison. 
${}^{15}Quand je me suis trouvé à Jérusalem, les grands prêtres et les anciens des Juifs ont exposé leurs griefs contre lui en réclamant sa condamnation. 
${}^{16}J’ai répondu que les Romains n’ont pas coutume de faire la faveur de livrer qui que ce soit lorsqu’il est accusé, avant qu’il soit confronté avec ses accusateurs et puisse se défendre du chef d’accusation.
${}^{17}Ils se sont donc retrouvés ici, et sans aucun délai, le lendemain même, j’ai siégé au tribunal et j’ai donné l’ordre d’amener cet homme. 
${}^{18}Quand ils se levèrent, les accusateurs n’ont mis à sa charge aucun des méfaits que, pour ma part, j’aurais supposés. 
${}^{19}Ils avaient seulement avec lui certains débats au sujet de leur propre religion, et au sujet d’un certain Jésus qui est mort, mais que Paul affirmait être en vie. 
${}^{20}Quant à moi, embarrassé devant la suite à donner à l’instruction, j’ai demandé à Paul s’il voulait aller à Jérusalem pour y être jugé sur cette affaire. 
${}^{21}Mais Paul a fait appel pour être gardé en prison jusqu’à la décision impériale. J’ai donc ordonné de le garder en prison jusqu’au renvoi de sa cause devant l’empereur. » 
${}^{22}Agrippa dit à Festus : « Je voudrais bien, moi aussi, entendre cet homme. » Il répondit : « Demain, tu l’entendras. »
${}^{23}Le lendemain, Agrippa et Bérénice arrivèrent donc en grand apparat et firent leur entrée dans la salle d’audience, escortés par les autorités militaires et les principaux personnages de la cité. Sur l’ordre de Festus, Paul fut amené. 
${}^{24}Festus prit la parole : « Roi Agrippa, et vous tous qui êtes là avec nous, vous voyez devant vous l’homme au sujet duquel toute la multitude des Juifs m’a sollicité, tant à Jérusalem qu’ici même, en criant qu’il ne devait plus rester en vie. 
${}^{25}Quant à moi, j’ai compris qu’il n’avait rien fait qui mérite la mort ; mais comme lui-même en a appelé à l’empereur, j’ai pris la décision de l’envoyer à Rome. 
${}^{26}Je n’ai rien de précis à écrire sur son compte au seigneur l’empereur ; c’est pourquoi je l’ai fait comparaître devant vous, et surtout devant toi, roi Agrippa, afin qu’après cette audience j’aie quelque chose à écrire. 
${}^{27}En effet, il ne me semble pas raisonnable d’envoyer un prisonnier sans signifier les charges retenues contre lui. »
      
         
      \bchapter{}
      \begin{verse}
${}^{1}Alors Agrippa s’adressa à Paul : « Tu es autorisé à plaider ta cause. » Après avoir levé la main, Paul présenta sa défense : 
${}^{2}« Sur tous les points dont je suis accusé par les Juifs, je m’estime heureux, roi Agrippa, d’avoir à présenter ma défense aujourd’hui devant toi, 
${}^{3}d’autant plus que tu es un connaisseur de toutes les coutumes des Juifs et de tous leurs débats. Voilà pourquoi je te prie de m’écouter avec patience.
${}^{4}Ce qu’a été ma vie depuis ma jeunesse, comment dès le début j’ai vécu parmi mon peuple et à Jérusalem, cela, tous les Juifs le savent. 
${}^{5}Ils me connaissent depuis longtemps, et ils témoigneront, s’ils le veulent bien, que j’ai vécu en pharisien, c’est-à-dire dans le groupe le plus strict quant à notre pratique religieuse. 
${}^{6}Et maintenant, si je suis là en jugement, c’est parce que je mets mon espérance en la promesse faite par Dieu à nos pères, 
${}^{7}promesse dont nos douze tribus espèrent l’accomplissement, elles qui rendent un culte à Dieu jour et nuit avec persévérance. C’est pour cette espérance, ô roi, que je suis accusé par les Juifs. 
${}^{8}Pourquoi, chez vous, juge-t-on incroyable que Dieu ressuscite les morts ?
${}^{9}Pour moi, j’ai pensé qu’il fallait combattre très activement le nom de Jésus le Nazaréen. 
${}^{10}C’est ce que j’ai fait à Jérusalem : j’ai moi-même emprisonné beaucoup de fidèles, en vertu des pouvoirs reçus des grands prêtres ; et quand on les mettait à mort, j’avais apporté mon suffrage. 
${}^{11}Souvent, je passais de synagogue en synagogue et je les forçais à blasphémer en leur faisant subir des sévices ; au comble de la fureur, je les persécutais jusque dans les villes hors de Judée.
${}^{12}C’est ainsi que j’allais à Damas muni d’un pouvoir et d’une procuration des grands prêtres ; 
${}^{13}en plein midi, sur la route, ô roi, j’ai vu, venant du ciel, une lumière plus éclatante que le soleil, qui m’enveloppa, moi et ceux qui m’accompagnaient. 
${}^{14}Tous, nous sommes tombés à terre, et j’ai entendu une voix qui me disait en araméen : “Saul, Saul, pourquoi me persécuter ? Il est dur pour toi de résister à l’aiguillon.” 
${}^{15}Et moi je dis : “Qui es-tu, Seigneur ?” Le Seigneur répondit : “Je suis Jésus, celui que tu persécutes. 
${}^{16}Mais relève-toi, et tiens-toi debout ; voici pourquoi je te suis apparu : c’est pour te destiner à être serviteur et témoin de ce moment où tu m’as vu, et des moments où je t’apparaîtrai encore, 
${}^{17}pour te délivrer de ton peuple et des non-Juifs. Moi, je t’envoie vers eux, 
${}^{18}pour leur ouvrir les yeux, pour les ramener des ténèbres vers la lumière et du pouvoir de Satan vers Dieu, afin qu’ils reçoivent, par la foi en moi, le pardon des péchés et une part d’héritage avec ceux qui ont été sanctifiés.”
${}^{19}Dès lors, roi Agrippa, je n’ai pas désobéi à cette vision céleste, 
${}^{20}mais j’ai parlé d’abord aux gens de Damas et à ceux de Jérusalem, puis à tout le pays de Judée et aux nations païennes ; je les exhortais à se convertir et à se tourner vers Dieu, en adoptant un comportement accordé à leur conversion. 
${}^{21}Voilà pourquoi les Juifs se sont emparés de moi dans le Temple, pour essayer d’en finir avec moi. 
${}^{22}Fort du secours que j’ai reçu de Dieu, j’ai tenu bon jusqu’à ce jour pour rendre témoignage devant petits et grands. Je n’ai rien dit en dehors de ce que les prophètes et Moïse avaient prédit, 
${}^{23}à savoir que le Christ, exposé à la souffrance et premier ressuscité d’entre les morts, devait annoncer la lumière à notre peuple et aux nations. »
${}^{24}Il en était là de sa défense, quand Festus s’écria : « Tu délires, Paul ! Ta grande érudition te fait délirer ! » 
${}^{25}Mais Paul répliqua : « Je ne délire pas, Très excellent Festus ! Mais je parle un langage de vérité et de bon sens. 
${}^{26}Le roi, à qui je m’adresse avec assurance, est au courant de ces événements ; je suis convaincu qu’aucun d’eux ne lui a échappé, car ce n’est pas dans un coin perdu que cela s’est fait. 
${}^{27}Roi Agrippa, crois-tu aux prophètes ? Je sais bien que tu y crois. » 
${}^{28}Agrippa dit alors à Paul : « Encore un peu, et tu me persuades de me faire chrétien ! » 
${}^{29}Paul répliqua : « Plaise à Dieu que, tôt ou tard, non seulement toi, mais encore tous ceux qui m’écoutent aujourd’hui, vous deveniez tel que je suis – sauf les chaînes que voici ! »
${}^{30}Le roi se leva, puis le gouverneur, ainsi que Bérénice et ceux qui étaient assis avec eux. 
${}^{31}S’étant retirés, ils se disaient entre eux : « Cet homme ne fait rien qui mérite la mort ou la prison. » 
${}^{32}Et Agrippa dit à Festus : « Cet homme aurait pu être relâché, s’il n’en avait pas appelé à l’empereur. »
      
         
      \bchapter{}
      \begin{verse}
${}^{1}Quand notre embarquement pour l’Italie a été décidé, on a confié Paul et quelques autres prisonniers à un centurion nommé Julius, de la cohorte Augusta. 
${}^{2}Montés à bord d’un bateau d’Adramyttium sur le point d’appareiller pour les côtes de la province d’Asie, nous avons gagné le large, ayant avec nous Aristarque, un Macédonien de Thessalonique. 
${}^{3}Le lendemain, nous avons abordé à Sidon ; et Julius, qui traitait Paul avec humanité, lui a permis d’aller voir ses amis et de bénéficier de leur sollicitude. 
${}^{4}De là, nous avons repris la mer et longé Chypre pour nous abriter des vents contraires. 
${}^{5}Nous avons traversé la mer qui borde la Cilicie et la Pamphylie, et débarqué à Myre en Lycie. 
${}^{6}Là, le centurion a trouvé un bateau d’Alexandrie en partance pour l’Italie, et nous a fait monter à bord. 
${}^{7}Pendant plusieurs jours, nous avons navigué lentement, et nous sommes arrivés avec peine à la hauteur de Cnide, mais le vent ne nous a pas permis d’en approcher. Nous avons alors longé la Crète à l’abri du vent, au large du cap Salmoné 
${}^{8}que nous avons doublé avec peine, et nous sommes arrivés à un endroit appelé « Bons Ports », près de la ville de Lasaïa.
${}^{9}Il s’était écoulé pas mal de temps, puisque même le jeûne du Grand Pardon était déjà passé, et déjà la navigation était devenue dangereuse, si bien que Paul ne cessait de les avertir : 
${}^{10}« Mes amis, je vois que la navigation ne se fera pas sans dommages ni beaucoup de pertes, non seulement pour la cargaison et le bateau, mais encore pour nos vies. » 
${}^{11}Mais le centurion faisait davantage confiance au pilote et à l’armateur qu’aux paroles de Paul. 
${}^{12}Et comme le port n’était pas adapté pour y passer l’hiver, la plupart ont été d’avis de reprendre la mer, afin d’atteindre, si possible, Phénix, un port de Crète ouvert à la fois vers le sud-ouest et le nord-ouest, et d’y passer l’hiver.
${}^{13}Comme un léger vent du sud s’était mis à souffler, ils s’imaginaient pouvoir réaliser leur projet ; ayant donc levé l’ancre, ils essayaient de longer de près la Crète. 
${}^{14}Mais presque aussitôt, venant des hauteurs de l’île, s’est déchaîné le vent d’ouragan qu’on appelle euraquilon. 
${}^{15}Le bateau a été emporté, sans pouvoir tenir contre le vent : nous sommes donc partis à la dérive. 
${}^{16}En passant à l’abri d’un îlot appelé Cauda, nous avons réussi, non sans peine, à garder la maîtrise de la chaloupe. 
${}^{17}On l’a hissée à bord, puis on a utilisé des câbles de secours pour ceinturer le bateau : craignant d’aller s’échouer sur les hauts-fonds de la Syrte, on a fait descendre l’ancre flottante, et ainsi on continuait à dériver. 
${}^{18}Le lendemain, comme la tempête nous secouait avec violence, on a jeté le superflu par-dessus bord. 
${}^{19}Le troisième jour, les matelots ont lancé, de leurs propres mains, le gréement du bateau à la mer. 
${}^{20}Depuis bien des jours, ni le soleil ni les étoiles ne se montraient et une tempête d’une violence peu commune continuait à sévir : désormais, tout espoir d’être sauvés nous était enlevé.
${}^{21}Les gens n’avaient plus rien mangé depuis longtemps. Alors Paul, debout au milieu d’eux, a pris la parole : « Mes amis, il fallait m’obéir et ne pas quitter la Crète pour gagner le large : on aurait évité ces dommages et ces pertes ! 
${}^{22}Mais maintenant, je vous exhorte à garder confiance, car aucun de vous n’y laissera la vie, seul le bateau sera perdu. 
${}^{23}Cette nuit, en effet, s’est présenté à moi un ange du Dieu à qui j’appartiens et à qui je rends un culte. 
${}^{24}Il m’a dit : “Sois sans crainte, Paul, il faut que tu te présentes devant l’empereur, et voici que, pour toi, Dieu fait grâce à tous ceux qui sont sur le bateau avec toi.” 
${}^{25}Alors, gardez confiance, mes amis ! J’ai foi en Dieu : il en sera comme il m’a été dit. 
${}^{26}Nous devons échouer sur une île. »
${}^{27}Or, la quatorzième nuit que nous dérivions sur la mer Adria, vers minuit, les matelots ont pressenti l’approche d’une terre. 
${}^{28}Ils ont lancé la sonde et trouvé vingt brasses ; un peu plus loin, ils l’ont lancée de nouveau et trouvé quinze brasses. 
${}^{29}Craignant que nous n’allions échouer sur des rochers, ils ont jeté quatre ancres à l’arrière, et ils appelaient de leurs vœux la venue du jour. 
${}^{30}C’est alors qu’ils ont cherché à s’enfuir du bateau, et qu’ils ont descendu la chaloupe à la mer sous prétexte d’aller tirer les ancres de la proue. 
${}^{31}Paul a dit alors au centurion et aux soldats : « Si ces gens-là ne restent pas sur le bateau, vous ne pouvez pas être sauvés. » 
${}^{32}À ce moment, les soldats ont coupé les filins de la chaloupe et l’ont laissé partir.
${}^{33}En attendant que le jour se lève, Paul exhortait tout le monde à prendre de la nourriture : « Voilà aujourd’hui le quatorzième jour que vous restez dans l’expectative, sans manger ni rien prendre. 
${}^{34}Je vous exhorte donc à prendre de la nourriture, car il y va de votre salut : aucun de vous ne perdra un cheveu de sa tête. » 
${}^{35}Ayant dit cela, il a pris du pain, il a rendu grâce à Dieu devant tous, il l’a rompu, et il s’est mis à manger. 
${}^{36}Alors tous, retrouvant confiance, ont eux aussi pris de la nourriture. 
${}^{37}Nous étions en tout deux cent soixante-seize personnes sur le bateau. 
${}^{38}Une fois rassasiés, on cherchait à alléger le bateau en jetant les vivres à la mer.
${}^{39}Quand il fit jour, on ne reconnaissait pas la terre, mais on apercevait une baie avec une plage, vers laquelle on voulait, si possible, faire avancer le bateau. 
${}^{40}Les matelots ont alors décroché les ancres pour les abandonner à la mer, ils ont détaché les câbles des gouvernails et hissé une voile au vent pour gagner la plage. 
${}^{41}Mais ayant touché un banc de sable, ils ont fait échouer le navire. La proue, qui s’était enfoncée, restait immobile, tandis que la poupe se disloquait sous la violence des vagues. 
${}^{42}Les soldats ont eu alors l’intention de tuer les prisonniers pour éviter que l’un d’eux s’enfuie à la nage. 
${}^{43}Mais le centurion, voulant sauver Paul, les a empêchés de réaliser leur projet ; il a ordonné de gagner la terre : à ceux qui savaient nager, en se jetant à l’eau les premiers, 
${}^{44}aux autres soit sur des planches, soit sur des débris du bateau. C’est ainsi que tous sont parvenus à terre sains et saufs.
      
         
      \bchapter{}
      \begin{verse}
${}^{1}Une fois sauvés, nous avons découvert que l’île s’appelait Malte. 
${}^{2}Les indigènes nous ont traités avec une humanité peu ordinaire. Ils avaient allumé un grand feu, et ils nous ont tous pris avec eux, car la pluie s’était mise à tomber et il faisait froid. 
${}^{3}Or comme Paul avait ramassé une brassée de bois mort et l’avait jetée dans le feu, la chaleur a fait sortir une vipère qui s’est accrochée à sa main. 
${}^{4}À la vue de la bête suspendue à sa main, les indigènes se disaient entre eux : « Cet homme est sûrement un meurtrier : il est sorti sain et sauf de la mer, mais la justice divine ne permet pas qu’il reste en vie. » 
${}^{5}Or Paul a secoué la bête pour la faire tomber dans le feu, et il n’en a éprouvé aucun mal, 
${}^{6}alors que les gens s’attendaient à le voir enfler ou tomber raide mort. Après avoir attendu un bon moment, et vu qu’il ne lui arrivait rien d’anormal, ils ont changé complètement d’avis : ils disaient que Paul était un dieu.
${}^{7}Il y avait là une propriété appartenant à Publius, le premier magistrat de l’île ; il nous a accueillis et, pendant trois jours, nous a donné une hospitalité cordiale. 
${}^{8}Or son père était au lit, atteint de fièvre et de dysenterie. Paul est allé le voir, il a prié, lui a imposé les mains et lui a rendu la santé. 
${}^{9}À la suite de cet événement, tous les autres malades de l’île venaient à lui et ils étaient guéris. 
${}^{10}On nous a comblés d’honneurs et, lorsque nous avons pris la mer, on nous a fourni tout ce dont nous avions besoin.
${}^{11}C’est au bout de trois mois que nous avons repris la mer à bord d’un navire d’Alexandrie, portant comme emblème les Dioscures, et qui avait passé l’hiver dans l’île. 
${}^{12}Nous avons abordé à Syracuse et nous y sommes restés trois jours. 
${}^{13}Après avoir levé l’ancre, nous avons atteint Reggio. Le lendemain, est survenu un vent du sud, et en deux jours nous sommes arrivés à Pouzzoles. 
${}^{14}Nous y avons trouvé des frères qui nous ont invités à passer sept jours chez eux.
      Voici comment nous sommes arrivés à Rome. 
${}^{15}De la ville, les frères, qui avaient entendu parler de nous, sont venus à notre rencontre jusqu’au lieu-dit Forum-d’Appius et à celui des Trois-Tavernes. En les voyant, Paul a rendu grâce à Dieu et repris courage. 
${}^{16}À notre arrivée à Rome, il a reçu l’autorisation d’habiter en ville avec le soldat qui le gardait.
${}^{17}Trois jours après, il fit appeler les notables des Juifs. Quand ils arrivèrent, il leur dit : « Frères, moi qui n’ai rien fait contre notre peuple et les coutumes reçues de nos pères, je suis prisonnier depuis Jérusalem où j’ai été livré aux mains des Romains. 
${}^{18}Après m’avoir interrogé, ceux-ci voulaient me relâcher, puisque, dans mon cas, il n’y avait aucun motif de condamnation à mort. 
${}^{19}Mais, devant l’opposition des Juifs, j’ai été obligé de faire appel à l’empereur, sans vouloir pour autant accuser ma nation. 
${}^{20}C’est donc pour ce motif que j’ai demandé à vous voir et à vous parler, car c’est à cause de l’espérance d’Israël que je porte ces chaînes. » 
${}^{21}Ils lui répondirent : « Pour notre part, nous n’avons pas reçu à ton sujet de lettre en provenance de Judée, et aucun frère venu ici n’a rapporté ou dit du mal de toi. 
${}^{22}Nous souhaitons pourtant apprendre de toi ce que tu penses, car nous avons été informés que votre groupe est contesté partout. »
${}^{23}Après lui avoir fixé une date, ils vinrent le trouver en plus grand nombre dans son logement. Paul rendait témoignage au royaume de Dieu, dans ce qu’il leur exposait, et il s’efforçait de les convaincre à propos de Jésus, en partant de la loi de Moïse ainsi que des Prophètes. Cela dura depuis le matin jusqu’au soir. 
${}^{24}Les uns se laissaient convaincre par de telles paroles, les autres refusaient de croire. 
${}^{25}N’étant pas d’accord les uns avec les autres, ils s’en allaient, quand Paul leur adressa cette seule parole : « L’Esprit Saint a bien parlé, quand il a dit à vos pères par le prophète Isaïe :
${}^{26}Va dire à ce peuple :
        \\Vous aurez beau écouter, vous ne comprendrez pas.
        \\Vous aurez beau regarder, vous ne verrez pas.
${}^{27}Le cœur de ce peuple s’est alourdi :
        \\ils sont devenus durs d’oreille,
        \\ils se sont bouché les yeux,
        \\de peur que leurs yeux ne voient,
        \\que leurs oreilles n’entendent,
        \\que leur cœur ne comprenne,
        \\qu’ils ne se convertissent,
        \\– et moi, je les guérirai.
${}^{28}Donc, sachez-le : c’est aux nations que ce salut de Dieu a été envoyé. Les nations, elles, écouteront. »
${}^{30}Paul demeura deux années entières dans le logement qu’il avait loué ; il accueillait tous ceux qui venaient chez lui ; 
${}^{31}il annonçait le règne de Dieu et il enseignait ce qui concerne le Seigneur Jésus Christ avec une entière assurance et sans obstacle.
