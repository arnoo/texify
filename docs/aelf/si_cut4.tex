  
  
      
         
      \bchapter{}
${}^{1}Certains reproches sont intempestifs ;
        certains silences dénotent quelqu’un d’intelligent.
${}^{2}Mieux vaut s’expliquer que rester en colère.
${}^{3}Qui avoue ses faiblesses évite d’être humilié.
${}^{4}Qui veut établir la justice par la violence
        est comme l’eunuque qui voudrait déflorer une jeune fille.
${}^{5}Tel se montre sage en gardant le silence,
        tel autre se rend odieux par son bavardage.
${}^{6}Tel garde le silence, car il n’a rien à répondre,
        tel autre, par sens de l’opportunité.
${}^{7}L’homme sage se tait jusqu’au moment opportun,
        que le sot, imbu de lui-même, laissera passer.
${}^{8}Qui parle trop se rend insupportable,
        qui se croit tout permis devient odieux.
        \\C’est bien de montrer du repentir après un blâme,
        tu éviteras ainsi de t’obstiner dans le mal.
        
           
${}^{9}D’un malheur on peut tirer profit,
        et une aubaine peut tourner au désastre.
${}^{10}Certains cadeaux sont offerts en pure perte,
        et d’autres te rapportent le double.
${}^{11}Parfois la gloire conduit à l’humiliation,
        et parfois d’humbles gens ont pu relever la tête.
${}^{12}Voici quelqu’un qui achète beaucoup pour pas cher,
        mais paie encore sept fois le prix.
${}^{13}Le sage se fait aimer pour ses seules paroles,
        mais les largesses des sots sont prodiguées en vain.
${}^{14}Le cadeau de l’insensé ne te profitera pas
        – pas plus que celui de l’envieux qui donne à contrecœur –
        car ses yeux attendent bien plus en retour.
${}^{15}Il donne peu et reproche beaucoup,
        il ouvre grand la bouche comme un crieur public,
        \\il prête aujourd’hui et demain redemande :
        quel homme détestable !
${}^{16}Le sot déclare : « Je n’ai pas d’ami,
        nul n’a de gratitude pour mes bienfaits ;
        ceux qui mangent mon pain sont mauvaises langues. »
${}^{17}Que de gens, si souvent, le tournent en ridicule !
        Ce qu’il possède, il ne l’apprécie pas à sa juste valeur,
        comme de ne pas posséder le laisse indifférent.
${}^{18}Mieux vaut choir sur le pavé que déchoir par la langue ;
        ainsi la chute des méchants est-elle soudaine.
${}^{19}Un homme mal élevé est comme l’histoire malvenue
        que l’imbécile a toujours à la bouche.
${}^{20}Venant d’un sot le proverbe reste sans effet,
        car il n’est jamais dit à propos.
${}^{21}Tel est préservé du péché par son indigence,
        son repos n’est pas troublé par le remords.
${}^{22}Tel perd son âme par respect humain,
        et c’est par égard pour un insensé qu’il la perd.
${}^{23}Tel autre, par timidité, promet l’impossible à son ami
        et tout ce qu’il y gagne est de s’en faire un ennemi.
         
${}^{24}Le mensonge est une faute grave,
        l’imbécile l’a toujours à la bouche.
${}^{25}Mieux vaut être voleur que menteur invétéré,
        mais tous deux courent à leur perte.
${}^{26}Mentir par habitude est déshonorant,
        la honte du menteur lui colle à la peau.
${}^{27}Le sage, par ses discours, se fait apprécier,
        l’homme de bon sens est estimé des grands.
${}^{28}Celui qui travaille la terre voit son blé s’entasser,
        celui qui gagne l’estime des grands voit ses torts pardonnés.
${}^{29}Présents et cadeaux aveuglent les sages ;
        comme un bâillon ils étouffent les reproches.
${}^{30}Une sagesse cachée, un trésor enfoui :
        à quoi peuvent-ils servir l’un et l’autre ?
${}^{31}Mieux vaut cacher sa sottise
        que dissimuler sa sagesse.
${}^{32}Mieux vaut rechercher le Seigneur avec une persévérance obstinée
        que diriger sa vie en refusant tout maître.
      
         
      \bchapter{}
${}^{1}Mon fils, tu as péché ?
        \\Ne recommence plus,
        mais demande pardon pour le passé.
${}^{2}Comme devant le serpent, fuis devant le péché,
        car si tu approches, il te mordra.
        \\Ses dents sont les dents d’un lion :
        elles arrachent la vie des hommes.
${}^{3}Toute transgression de la loi est une épée à deux tranchants,
        dont la blessure est inguérissable.
${}^{4}La folie des grandeurs et l’insolence dilapident une fortune ;
        c’est pourquoi la maison du prétentieux s’écroulera.
${}^{5}La prière du pauvre va droit de ses lèvres aux oreilles de Dieu :
        il ne tardera pas à faire justice.
${}^{6}Qui refuse les remarques suit les traces des pécheurs ;
        qui craint le Seigneur se convertit dans son cœur.
${}^{7}Le beau parleur se fait connaître partout,
        mais ses points faibles sont perçus par quelqu’un d’avisé.
${}^{8}Construire sa maison avec l’argent d’autrui,
        c’est amasser des pierres pour sa tombe.
${}^{9}Un groupe de gens sans loi est un paquet de filasse
        qui finira dans la flamme et le feu.
${}^{10}La route des pécheurs est plane, sans aucune pierre,
        mais l’abîme de la Mort est au bout.
        
           
${}^{11}Celui qui observe la Loi reste maître de ses pensées ;
        la crainte du Seigneur trouve sa perfection dans la sagesse.
${}^{12}Qui n’est pas habile n’apprend rien,
        mais il est une habileté qui déborde d’amertume.
${}^{13}La connaissance du sage est débordante : elle coule à flots,
        et ses conseils sont source de vie.
${}^{14}Le cerveau du sot est une cruche fêlée :
        il ne retient rien de ce qu’il apprend.
${}^{15}Une parole sage vient-elle aux oreilles de qui est instruit,
        il l’approuve et renchérit.
        \\Un débauché l’entend, elle lui déplaît
        et il la rejette loin derrière lui.
${}^{16}Les commentaires du sot sont un bagage pesant ;
        mais à la conversation de qui est sensé on trouve plaisir.
${}^{17}On cherche à entendre quelqu’un d’intelligent dans l’assemblée,
        et chacun médite en son cœur sur ce qu’il dit.
         
${}^{18}La sagesse que possède le sot est une maison en ruines,
        et le savoir d’un idiot n’est que propos incohérents.
${}^{19}L’instruction est une entrave aux pieds de l’imbécile ;
        elle est une chaîne à sa main droite.
${}^{20}Le sot ne peut rire sans élever la voix,
        l’homme avisé sourit discrètement.
${}^{21}L’instruction est une parure d’or sur qui a du bon sens ;
        elle est un bracelet à son bras droit.
${}^{22}Le sot entre dans les maisons d’un pas rapide,
        l’homme d’expérience se présente à l’entrée avec modestie.
${}^{23}Dès la porte, le rustre plonge le regard à l’intérieur,
        tandis que l’homme bien élevé reste dehors.
${}^{24}Écouter aux portes dénote un manque d’éducation,
        quelqu’un de sensé aurait honte à le faire.
${}^{25}Les lèvres des gens irréfléchis débitent des sottises,
        mais les paroles des personnes sensées sont pesées à la balance.
${}^{26}Les sots n’ont de cœur que leur bouche,
        mais les sages n’ont de bouche que leur cœur.
${}^{27}Quand un impie maudit son adversaire,
        il se maudit lui-même.
${}^{28}Le médisant se salit lui-même,
        et, de son entourage, il se fait détester.
      
         
      \bchapter{}
${}^{1}Le paresseux ressemble à une pierre crottée,
        tous le persiflent à cause de son déshonneur.
${}^{2}Le paresseux ressemble à un tas de fumier :
        tous ceux qui en ramassent secouent la main.
${}^{3}Un fils mal élevé est la honte de son père ;
        la naissance d’une fille lui est un préjudice.
${}^{4}Une fille intelligente trouvera un mari,
        mais la dévergondée fait le désespoir de son père.
${}^{5}Une femme effrontée est une honte pour son père et son mari,
        ils la mépriseront, l’un et l’autre.
${}^{6}Une remontrance mal à propos
        est comme de la musique un jour de deuil,
        mais le fouet et la correction sont sagesse en tout temps.
${}^{7}Des enfants qui ne manquent de rien et se conduisent honnêtement
        font oublier la modeste origine de leurs parents.
${}^{8}Des enfants qui se targuent d’être méprisants et mal élevés
        ternissent la noble origine de leur famille.
        
           
${}^{9}Instruire un sot, c’est recoller des débris,
        c’est vouloir réveiller quelqu’un abruti de sommeil.
${}^{10}S’entretenir avec un sot, c’est parler à quelqu’un d’assoupi ;
        à la fin il demandera : « De quoi s’agit-il ? »
${}^{11}Pleure sur un mort : il est privé de lumière ;
        pleure sur un sot : il est privé d’intelligence.
        \\Mais pleure un peu moins sur le mort : il a trouvé le repos,
        alors que la vie du sot est pire que la mort.
${}^{12}Pour un mort, le deuil dure sept jours ;
        pour le sot et l’impie, tous les jours de leur vie.
${}^{13}Ne multiplie pas les entretiens avec un insensé,
        et ne va pas chez un idiot,
        car il est grossier et ne respectera rien de ce qui est à toi.
        \\Garde-toi de lui si tu ne veux pas avoir d’ennuis
        ni être éclaboussé par la fange où il se vautre.
        \\Détourne-toi de lui, tu auras la paix
        au lieu d’être exaspéré par sa bêtise.
${}^{14}Qu’y a-t-il de plus lourd que le plomb ?
        Et quel est son nom : n’est-ce pas « le fou » ?
${}^{15}Sac de sable, bloc de sel ou masse de fer
        sont un fardeau moins lourd qu’un insensé.
${}^{16}Une charpente de bois bien ajustée à une construction
        ne se laisse pas disloquer par un tremblement de terre ;
        \\de même, un cœur fixé dans le dessein qu’il a mûri
        ne se laisse pas ébranler au moment critique.
${}^{17}Un cœur qui s’appuie sur une réflexion intelligente tient bon,
        comme le crépi de mortier d’un mur sans défaut.
${}^{18}Une palissade posée sur une hauteur
        ne résiste pas au vent ;
        \\de même, un cœur lâche, nourri de sottes pensées,
        ne résiste pas à la peur.
${}^{19}Qui blesse l’œil fait couler des larmes ;
        qui blesse le cœur révèle les sentiments.
${}^{20}Qui jette une pierre à des oiseaux les fait fuir ;
        qui offense un ami brise l’amitié.
${}^{21}Si tu as brandi l’épée contre un ami,
        ne désespère pas : un retour est possible.
${}^{22}Si tu as eu des mots avec un ami,
        ne t’inquiète pas : la réconciliation est possible.
        \\Mais s’il s’agit d’injures, de mépris,
        d’un secret trahi ou d’un coup perfide,
        cela fera fuir n’importe quel ami.
${}^{23}Gagne la confiance de ton prochain tandis qu’il est pauvre
        pour qu’aux jours de son bonheur, avec lui, tu sois comblé ;
        \\au moment de la détresse, reste-lui fidèle :
        tu auras une part quand il héritera.
        \\Car il ne faut pas toujours mépriser l’apparence
        ni admirer un riche qui manque d’esprit.
${}^{24}Vapeurs et fumées précèdent l’incendie ;
        de même, les injures annoncent l’effusion de sang.
${}^{25}Je n’aurai pas honte de prendre la défense d’un ami ;
        je ne me détournerai pas de lui.
${}^{26}Mais s’il m’arrive du mal par sa faute,
        quiconque l’apprendra se gardera de lui.
${}^{27}Qui placera une garde à ma bouche
        et, sur mes lèvres, le sceau du discernement,
        \\pour que je ne tombe pas,
        que ma langue ne cause pas ma perte ?
      
         
      \bchapter{}
${}^{1}Seigneur, Père et Maître de ma vie,
        ne permets pas qu’elles me fassent tomber.
        
           
         
${}^{2}Qui dressera mes pensées par le fouet
        et mon cœur par une discipline de sagesse,
        \\sans laisser passer mes erreurs
        ni fermer les yeux sur mes péchés ?
${}^{3}Sans quoi mes erreurs vont se multiplier,
        mes péchés abonder,
        \\je tomberai devant mes adversaires
        et mes ennemis s’en réjouiront,
        eux qui sont loin d’espérer en ta miséricorde.
${}^{4}Seigneur, Père et Dieu de ma vie,
        ne m’abandonne pas à leur caprice,
        \\ne laisse pas mes yeux être effrontés
${}^{5}et détourne de moi la convoitise.
${}^{6}Les désirs charnels et la dépravation,
        qu’ils n’aient pas prise sur moi ;
        ne me laisse pas céder à la débauche.
        
           
${}^{7}Mes enfants, apprenez comment on discipline sa langue :
        qui la surveille ne se laisse pas prendre en défaut.
${}^{8}Le pécheur sera pris au piège de ses propres lèvres,
        l’insolent et l’orgueilleux s’y empêtrent.
${}^{9}Ne prends pas l’habitude de faire des serments,
        ni de prononcer le nom du Dieu Saint.
${}^{10}De même que l’esclave harcelé et puni sans cesse
        garde la trace des coups,
        \\ainsi celui qui jure et invoque à tout propos le nom divin
        ne sera certainement pas purifié de son péché.
${}^{11}Qui multiplie les serments accumule les transgressions,
        et le fouet ne sortira pas de sa maison.
        \\S’il ne tient pas son serment, son péché sera sur lui ;
        s’il s’en moque, il pèche doublement ;
        \\s’il fait un faux serment, il n’aura pas de pardon,
        et sa maison sera accablée de malheurs.
         
${}^{12}Il y a un langage qui conduit à la mort :
        qu’il soit banni de l’héritage de Jacob !
        \\Car ceux qui sont religieux doivent s’en abstenir
        et ne pas se vautrer dans le péché.
${}^{13}Ne prends pas l’habitude de dire des grossièretés,
        car elles font pécher en paroles.
${}^{14}Souviens-toi de ton père et de ta mère
        quand tu sièges parmi des grands,
        \\de peur de te laisser aller devant eux
        et de dire une insanité, par habitude.
        \\Tu en viendrais à regretter d’être né
        et à maudire le jour de ta naissance.
${}^{15}Un homme habitué à parler grossièrement
        ne se corrigera jamais.
${}^{16}Deux sortes d’individus multiplient les péchés,
        et une troisième attire la colère :
        \\celui dont la passion brûle comme un brasier
        – elle ne s’éteindra pas avant d’être assouvie ;
        \\puis, l’homme qui livre son corps à la débauche
        – il n’aura pas de repos avant d’être consumé par le feu :
${}^{17}pour ce débauché, toute pâture est délectable,
        il ne se calmera qu’une fois mort ;
${}^{18}et enfin, l’homme infidèle au lit conjugal ;
        il se dit en lui-même : « Qui me voit ?
        \\L’obscurité m’entoure, les murs me cachent.
        Personne ne m’observe : qu’ai-je à craindre ?
        De mes péchés, le Très-Haut ne gardera pas mémoire. »
${}^{19}Le regard des hommes, voilà sa terreur.
        \\Il ignore que les yeux du Seigneur
        sont mille fois plus lumineux que le soleil :
        \\ils observent tous les chemins des hommes
        et pénètrent les plus secrets recoins.
${}^{20}Toutes choses lui étaient connues, avant d’être créées ;
        elles le sont encore après leur achèvement.
${}^{21}L’adultère recevra son châtiment sur la place publique,
        pris sur le fait quand il ne s’y attendait pas.
         
${}^{22}Il en va de même pour la femme qui trompe son mari
        et lui donne un héritier conçu d’un autre homme.
${}^{23}Premièrement, elle a désobéi à la Loi du Très-Haut ;
        deuxièmement, elle a commis une faute contre son mari ;
        \\troisièmement, elle s’est débauchée dans l’adultère
        et elle a des enfants conçus d’un autre homme.
${}^{24}Elle sera traînée devant l’assemblée
        et l’on fera une enquête sur ses enfants.
${}^{25}Ils ne pourront prendre racine :
        ses branches ne porteront pas de fruit ;
${}^{26}son souvenir restera maudit
        et sa honte ne s’effacera pas.
${}^{27}Ceux qui restent le sauront :
        rien ne vaut la crainte du Seigneur
        et rien n’est plus doux que d’observer ses commandements.
${}^{28}Suivre le Seigneur, quelle gloire immense !
        Être accueilli par lui, c’est pour toi longue vie.
      
         
      \bchapter{}
        ${}^{1}La Sagesse divine\\proclame son propre éloge,
        au milieu de son peuple elle célèbre sa gloire.
        ${}^{2}Dans l’assemblée du Très-Haut elle prend la parole,
        devant le Dieu Puissant\\elle se glorifie :
        
           
         
        ${}^{3}« Je suis sortie de la bouche du Très-Haut
        et, comme la brume, j’ai couvert la terre.
        ${}^{4}J’ai dressé ma tente dans les hauteurs du ciel,
        et la colonne de nuée était mon trône.
${}^{5}J’ai parcouru seule la voûte des cieux
        et me suis promenée dans le fond des abîmes.
${}^{6}Des flots de la mer, de la terre entière,
        de tout peuple et de toute nation j’ai fait mon domaine.
${}^{7}Parmi eux tous, j’ai cherché le lieu de mon repos,
        une part d’héritage où m’établir.
        ${}^{8}Le Créateur de toutes choses m’a donné un ordre,
        celui qui m’a créée a fixé ma demeure.
        \\Il m’a dit : “Viens demeurer parmi les fils de Jacob,
        reçois ta part d’héritage en Israël,
        enracine-toi dans le peuple élu\\.”
        
           
         
        ${}^{9}Dès le commencement, avant les siècles\\, il m’a créée,
        et pour les siècles je subsisterai ;
        ${}^{10}dans la demeure sainte,
        j’ai assuré mon service en sa présence.
        \\Ainsi, je me suis fixée dans Sion,
        ${}^{11}il m’a fait demeurer dans la cité bien-aimée,
        et dans Jérusalem j’exerce ma puissance.
        ${}^{12}Je me suis enracinée dans un peuple glorieux,
        dans le domaine du Seigneur, dans son héritage :
        j’habite au milieu de l’assemblée des saints  .
${}^{13}Je me suis dressée comme un cèdre sur le Liban,
        un cyprès dans la montagne de l’Hermon.
${}^{14}Je me suis dressée comme un palmier à Enn-Guèdi,
        comme les plants de laurier-rose à Jéricho,
        comme un bel olivier dans la plaine ;
        comme un platane je me suis dressée.
${}^{15}Comme le cinnamome et l’acanthe aromatique j’ai donné mon parfum,
        comme une myrrhe précieuse j’ai exhalé mes senteurs,
        comme le galbanum, l’ambre et le storax,
        comme un nuage d’encens dans la tente de la Rencontre.
${}^{16}Comme un térébinthe j’ai déployé mes rameaux,
        rameaux de grâce et de gloire.
${}^{17}Comme une vigne, j’ai donné des sarments pleins de grâce
        et mes fleurs sont des fruits de gloire et de richesse.
${}^{18}Je suis la mère du bel amour,
        de la crainte de Dieu  et de la connaissance
        et aussi de la sainte espérance.
        \\J’ai reçu toute grâce
        pour montrer le chemin et la vérité.
        \\En moi est toute espérance de vie et de force\\.
        
           
         
        ${}^{19}Venez à moi, vous qui me désirez,
        rassasiez-vous de mes fruits.
        ${}^{20}Mon souvenir est plus doux que le miel,
        mon héritage, plus doux qu’un rayon de miel.
        \\Mon souvenir demeure dans la suite des âges  .
        ${}^{21}Ceux qui me mangent auront encore faim,
        ceux qui me boivent auront encore soif.
${}^{22}Celui qui m’obéit ne sera pas déçu.
        Ceux qui travaillent avec moi\\ne seront pas pécheurs.
        Ceux qui me mettent en lumière auront la vie éternelle. »
        
           
${}^{23}Tout cela, c’est le livre de l’alliance du Dieu Très-Haut,
        la Loi que Moïse nous a prescrite,
        héritage laissé aux assemblées de Jacob.
${}^{24}Continuez à vous affermir dans le Seigneur ;
        attachez-vous à lui afin qu’il vous fortifie.
        \\Le Seigneur souverain de l’univers est le Dieu unique ;
        il n’y a pas d’autre Sauveur que lui.
${}^{25}La Loi fait abonder la sagesse comme les eaux du Pishone,
        comme le fleuve Tigre au temps des fruits nouveaux ;
${}^{26}elle fait déborder l’intelligence comme l’Euphrate,
        comme le Jourdain au temps de la moisson.
${}^{27}Elle fait couler l’instruction comme le Nil,
        comme le Guihone au temps de la vendange.
${}^{28}De même que le premier n’a jamais fini de connaître la sagesse,
        de même le dernier n’a pas encore repéré sa trace.
${}^{29}Car sa pensée est vaste, plus que la mer,
        et son projet, plus profond que l’océan.
${}^{30}Quant à moi, j’étais comme un canal venu du fleuve,
        comme un aqueduc menant vers un paradis.
${}^{31}Je me suis dit : « J’arroserai mon jardin,
        je vais irriguer mon parterre. »
        \\Et voici que mon canal est devenu un fleuve,
        et mon fleuve, une mer !
${}^{32}Je vais encore faire luire l’instruction comme l’aurore
        et porter au loin sa lumière.
${}^{33}Je vais encore répandre la doctrine comme une prophétie
        et je la léguerai aux générations à venir.
${}^{34}Voyez : ce n’est pas pour moi seul que j’ai peiné,
        mais pour tous ceux qui cherchent la sagesse.
      
         
      \bchapter{}
${}^{1}Il est trois choses dont mon âme est éprise
        et qui sont belles devant le Seigneur et devant les hommes :
        \\la concorde entre frères, l’amitié entre proches,
        une femme et un homme en parfaite harmonie.
${}^{2}Mais il est trois sortes de personnes que mon âme déteste
        et dont la manière de vivre m’irrite terriblement :
        \\le pauvre plein d’orgueil, le riche qui ment
        et le vieillard vicieux, dépourvu de bon sens.
        
           
         
${}^{3}Ce que tu n’as pas amassé dans ta jeunesse,
        comment le trouverais-tu dans ta vieillesse ?
${}^{4}Qu’il est beau, à l’âge des cheveux blancs, d’avoir du jugement
        et, dans la vieillesse, de savoir conseiller !
${}^{5}Qu’elle est belle, la sagesse des anciens,
        de même que la réflexion et le conseil, chez les gens vénérables !
${}^{6}La couronne des vieillards, c’est leur grande expérience ;
        leur fierté, c’est la crainte du Seigneur.
        
           
         
${}^{7}Il y a neuf choses qui me viennent à l’esprit,
        et qu’au fond de moi j’estime heureuses,
        et ma langue peut encore en nommer une dixième :
        \\un homme qui trouve sa joie dans ses enfants,
        celui qui voit, de son vivant, la ruine de ses ennemis ;
${}^{8}heureux aussi celui qui vit avec une femme intelligente,
        celui qui ne laboure pas avec un bœuf et un âne,
        \\celui qui n’a pas failli par sa langue,
        celui qui n’a pas servi un maître indigne de lui ;
${}^{9}heureux encore celui qui parvient au discernement,
        et celui qui sait intéresser son auditoire.
${}^{10}Qu’il est grand, enfin, celui qui a trouvé la sagesse !
        Mais nul ne surpasse celui qui craint le Seigneur.
${}^{11}Car la crainte du Seigneur est au-dessus de tout ;
        celui qui la possède, à qui peut-on le comparer ?
${}^{12}L’amour du Seigneur commence avec la crainte ;
        l’attachement, avec la confiance.
        
           
${}^{13}Toutes les blessures, mais pas la blessure du cœur !
        Toutes les méchancetés, mais pas la méchanceté d’une femme.
${}^{14}Tous les mauvais coups, mais pas les coups de ceux qui me haïssent !
        Toutes les revanches, mais pas la revanche des ennemis.
${}^{15}Il n’est pire venin que le venin du serpent,
        il n’est pire fureur que la fureur d’une femme.
${}^{16}J’habiterais plus volontiers avec un lion ou un dragon
        qu’avec une femme mauvaise.
${}^{17}La méchanceté d’une femme altère ses traits,
        sa mine sombre lui donne l’air d’un ours.
${}^{18}Son mari va chez les voisins prendre ses repas
        et ne peut réprimer de profonds soupirs.
${}^{19}Toute malice est peu de chose à côté de la malice d’une femme :
        que tombe sur elle le sort des pécheurs !
${}^{20}Une montée sablonneuse sous les pas d’un vieillard :
        telle est la femme bavarde pour l’homme qui cherche le calme.
${}^{21}Par la beauté d’une femme, ne te laisse pas subjuguer ;
        ne convoite pas une femme.
${}^{22}Colère, impudence et grande honte
        quand la femme entretient son mari !
${}^{23}Cœur abattu, visage sombre,
        blessure du cœur, voilà la femme mauvaise ;
        \\les mains négligentes, le pas traînant,
        voilà celle qui ne rend pas heureux son mari.
${}^{24}C’est par une femme qu’a commencé le péché,
        et c’est à cause d’elle que nous mourons tous.
${}^{25}Ne laisse pas l’eau se répandre,
        ni une femme méchante parler sans frein.
${}^{26}Si elle ne marche pas au doigt et à l’œil,
        sépare-toi d’elle.
      
         
      \bchapter{}
        ${}^{1}Heureux l’homme qui a une bonne épouse :
        le nombre de ses jours sera doublé.
        ${}^{2}La femme courageuse fait la joie de son mari :
        il possédera le bonheur tout au long de sa vie.
        ${}^{3}Une femme de valeur, voilà le bon parti,
        la part que le Seigneur donne à ceux qui le craignent ;
        ${}^{4}riches ou pauvres, ils ont le cœur joyeux,
        en toute circonstance leur visage est souriant.
        
           
${}^{5}Il y a trois choses que redoute mon cœur,
        et une quatrième m’épouvante :
        \\une calomnie qui court la ville, un attroupement de foule,
        une fausse accusation – tout cela est pire que la mort.
${}^{6}Mais c’est crève-cœur et affliction qu’une femme jalouse d’une rivale,
        sa langue est un fouet qui n’épargne personne.
${}^{7}Une femme mauvaise est comme un joug qui blesse les bœufs ;
        vouloir s’en rendre maître, c’est saisir un scorpion !
${}^{8}Une femme qui boit, c’est révoltant :
        elle ne pourra cacher son déshonneur.
${}^{9}L’impudeur d’une femme se lit dans ses yeux effrontés
        et se reconnaît au jeu de ses paupières.
${}^{10}Si ta fille est aventureuse, monte bonne garde,
        de peur qu’elle ne profite de la moindre occasion.
${}^{11}Sur ses regards aguicheurs exerce une surveillance,
        et ne t’étonne pas si elle te déshonore.
${}^{12}Comme le voyageur assoiffé ouvre la bouche
        et boit la première eau qui est à sa portée,
        \\elle s’assied devant chaque seuil,
        et à toute flèche elle ouvre son carquois.
        ${}^{13}La grâce de la femme enchante son mari,
        et ses talents lui donnent le bien-être\\.
        ${}^{14}Une femme qui sait se taire est un don du Seigneur.
        Rien ne vaut une femme\\préparée à sa tâche.
        ${}^{15}C’est la grâce des grâces\\qu’une femme discrète.
        Une âme qui se maîtrise est un trésor sans prix.
        ${}^{16}Un lever de soleil sur les montagnes du Seigneur :
        ainsi, la beauté d’une épouse parfaite est la lumière de sa maison.
         
${}^{17}Une lampe qui brille sur le chandelier saint,
        tel est un beau visage sur un corps bien formé.
${}^{18}Des colonnes d’or sur un socle d’argent,
        ainsi de belles jambes sur des talons solides.
         
${}^{19}Mon fils, préserve la fleur de ta jeunesse
        et ne gaspille pas ta force au-dehors.
${}^{20}Cherche dans toute la plaine un lot de bonne terre,
        sèmes-y ton grain, fais confiance à tes origines.
${}^{21}Ainsi, tes enfants après toi
        pourront avec fierté vanter leurs origines.
${}^{22}Une femme qui se laisse acheter ne vaut pas un crachat,
        alors que la femme fidèle à son mari est comme une tour :
        ceux qui s’y attaquent y trouvent la mort.
${}^{23}Une femme impie sera le partage du pécheur,
        la femme pieuse est donnée à celui qui craint le Seigneur.
${}^{24}Une femme impudique consume sa vie dans le déshonneur,
        la jeune femme pudique est réservée, même avec son mari.
${}^{25}Une femme provocante ne sera pas plus respectée qu’un chien,
        celle qui a de la modestie craindra le Seigneur.
${}^{26}Une femme qui honore son mari paraîtra sage aux yeux de tous,
        celle qui le méprise sera jugée impie pour son orgueil.
        \\Heureux l’homme qui a une bonne épouse :
        le nombre de ses années sera doublé.
${}^{27}Une femme criarde et bavarde
        est comme la trompette au milieu des batailles.
        \\Quiconque vit dans ces conditions
        passera sa vie dans les fracas de la guerre.
${}^{28}Il y a deux choses qui affligent mon cœur,
        et une troisième qui me met en colère :
        \\un soldat réduit à la misère,
        des hommes intelligents rejetés avec mépris,
        \\un homme qui délaisse la justice pour le péché ;
        celui-là, le Seigneur le destine à périr par l’épée.
${}^{29}Un négociant évitera difficilement les fautes,
        un commerçant ne saurait être sans péché.
      
         
      \bchapter{}
${}^{1}Beaucoup pèchent par amour du profit ;
        qui cherche à s’enrichir ferme les yeux sur le mal.
${}^{2}Un clou s’enfonce dans la jointure des pierres,
        entre la vente et l’achat s’insinue le péché.
${}^{3}Qui ne s’attache vivement à la crainte du Seigneur,
        sa maison ne tardera pas à tomber en ruine.
        
           
        ${}^{4}Quand on secoue le tamis, il reste les déchets ;
        de même, les petits côtés d’un homme
        apparaissent dans ses propos.
        ${}^{5}Le four éprouve les vases du potier ;
        on juge l’homme en le faisant parler.
        ${}^{6}C’est le fruit qui manifeste la qualité de l’arbre ;
        ainsi la parole fait connaître les sentiments\\.
        ${}^{7}Ne fais pas l’éloge de quelqu’un avant qu’il ait parlé,
        c’est alors qu’on pourra le juger.
${}^{8}Si tu poursuis la justice, tu l’atteindras
        et t’en revêtiras comme d’un manteau de gloire.
${}^{9}Les oiseaux vont nicher auprès de leurs semblables,
        la vérité revient toujours vers ceux qui la pratiquent.
${}^{10}Le lion traque sa proie ;
        ainsi le péché traque ceux qui pratiquent l’injustice.
${}^{11}La conversation de qui est religieux, toujours, est sagesse,
        mais l’insensé change comme la lune.
${}^{12}Sois sur tes gardes avec des gens stupides,
        mais attarde-toi en compagnie de gens réfléchis.
${}^{13}La conversation des sots est agaçante :
        seuls débauche et péché les font rire.
${}^{14}Celui qui multiplie les jurons fait dresser les cheveux sur la tête ;
        quand il cherche querelle, on se bouche les oreilles.
${}^{15}La querelle des orgueilleux fera couler le sang :
        leurs invectives sont pénibles à entendre.
${}^{16}Qui trahit les secrets détruit la confiance,
        il ne trouvera plus d’ami selon son cœur.
${}^{17}Montre de l’affection à ton ami et sois-lui fidèle,
        mais, si tu trahis ses secrets, ne cours pas après lui :
${}^{18}aussi vrai que l’on perd un des siens quand il meurt,
        tu as perdu l’amitié de ton prochain.
${}^{19}Comme on laisse échapper un oiseau de sa main,
        tu as laissé partir ton prochain :
        tu ne le rattraperas jamais.
${}^{20}Ne le poursuis pas : il est déjà loin,
        il s’est enfui comme une gazelle s’échappe du filet.
${}^{21}Car une blessure se soigne,
        on se réconcilie après l’injure,
        \\mais pour qui trahit un secret, plus d’espoir !
${}^{22}Qui fait des clins d’œil prépare un mauvais coup,
        nul ne pourra l’en détourner.
${}^{23}En ta présence, il est tout miel,
        il s’extasie devant tes propos,
        \\mais par-derrière il change de langage
        et utilise tes paroles pour te faire chuter.
${}^{24}Je déteste bien des choses, mais rien autant que lui,
        et le Seigneur aussi le déteste.
${}^{25}Qui jette une pierre en l’air la jette sur sa tête,
        qui frappe en traître recevra le contrecoup.
${}^{26}Qui creuse une fosse y tombera,
        qui tend un piège y sera pris.
${}^{27}Qui fait le mal le verra se retourner contre lui,
        sans savoir d’où cela lui vient.
${}^{28}L’orgueilleux insulte et se moque,
        mais la vengeance, comme un lion, le traque.
${}^{29}Ils seront pris au piège, ceux que réjouit la chute des gens religieux,
        et la douleur les consumera dès avant leur mort.
        ${}^{30}Rancune et colère, voilà des choses abominables
        où le pécheur est passé maître.
      
         
      \bchapter{}
        ${}^{1}Celui qui se venge
        \\éprouvera la vengeance du Seigneur ;
        celui-ci tiendra un compte rigoureux de ses péchés.
        ${}^{2}Pardonne à ton prochain le tort qu’il t’a fait ;
        alors, à ta prière, tes péchés seront remis.
        ${}^{3}Si un homme nourrit de la colère contre un autre homme,
        comment peut-il demander à Dieu la guérison ?
        ${}^{4}S’il n’a pas de pitié pour un homme, son semblable,
        comment peut-il supplier pour ses péchés à lui ?
        ${}^{5}Lui qui est un pauvre mortel\\, il garde rancune ;
        qui donc lui pardonnera ses péchés ?
        ${}^{6}Pense à ton sort final et renonce à toute haine,
        pense à ton déclin et à ta mort,
        et demeure fidèle aux commandements.
        ${}^{7}Pense aux commandements
        et ne garde pas de rancune envers le prochain,
        \\pense à l’Alliance du Très-Haut
        et sois indulgent pour qui ne sait pas.
        
           
${}^{8}Reste à l’écart des querelles, et tu pécheras moins,
        car un homme emporté attise les querelles.
${}^{9}Le pécheur sème le trouble entre les amis
        et jette la division parmi ceux qui vivent en paix.
${}^{10}Le feu brûle pour autant qu’on l’alimente,
        une querelle s’échauffe pour autant qu’on s’entête.
        \\La fureur d’un homme est à la mesure de sa force,
        et il gonfle sa colère en proportion de sa richesse.
${}^{11}Une dispute soudaine allume un feu,
        une querelle subite fait couler le sang.
${}^{12}Souffle sur une braise, elle s’enflamme,
        crache dessus, elle s’éteint ;
        l’un comme l’autre vient de ta bouche.
${}^{13}Maudits soient le diffamateur et la langue fourbe :
        ils ont perdu bien des gens qui vivaient en paix.
${}^{14}La langue calomniatrice en a fait tomber beaucoup
        et les a chassés de nation en nation ;
        \\elle a détruit des villes fortes
        et renversé la maison de gens puissants.
${}^{15}La langue calomniatrice a fait répudier des femmes courageuses,
        les privant du fruit de leurs travaux.
${}^{16}Celui qui l’écoute ne trouvera jamais de repos
        et ne pourra pas habiter en paix.
${}^{17}Un coup de fouet laisse une meurtrissure,
        un coup donné par la langue brise les os.
${}^{18}Beaucoup sont tombés sous le tranchant de l’épée,
        combien plus sont tombés victimes de la langue !
${}^{19}Heureux qui est à l’abri de ses atteintes,
        qui n’a pas été exposé à sa fureur,
        \\ni soumis à son joug,
        ni lié par ses chaînes.
${}^{20}Car son joug est un joug de fer
        et ses chaînes sont des chaînes de bronze.
${}^{21}La mort qu’elle inflige est une mort terrible,
        mieux vaut encore le séjour d’en-bas.
${}^{22}Mais elle n’a pas d’emprise sur les gens religieux,
        et sa flamme ne les brûlera pas.
${}^{23}Ceux qui abandonnent le Seigneur tomberont en son pouvoir :
        elle les consumera sans s’éteindre ;
        \\elle s’élancera sur eux comme un lion,
        comme une panthère, elle les déchirera.
${}^{24}Vois : tu entoures ton domaine d’une haie d’épines,
        tu mets sous clé ton argent et ton or,
${}^{25}eh bien, pèse aussi tes mots sur une balance
        et mets à ta bouche porte et verrou.
${}^{26}Prends garde que la langue ne te fasse trébucher,
        tu tomberais devant celui qui te guette.
      
         
      \bchapter{}
${}^{1}C’est faire œuvre de miséricorde que prêter à son prochain ;
        lui venir en aide, c’est observer les commandements.
${}^{2}Prête à ton prochain quand il est dans le besoin,
        et restitue en temps voulu ce que tu as emprunté.
${}^{3}Tiens ta parole, sois loyal envers ton prochain,
        et tu trouveras toujours le nécessaire.
${}^{4}Beaucoup considèrent ce qu’on leur prête comme s’ils l’avaient trouvé
        et mettent en difficulté ceux qui les secourent.
${}^{5}Tant qu’on n’a pas reçu, on couvre de baisers les mains du prochain,
        et, pour ses richesses, on lui parle humblement.
        \\Mais, au moment de l’échéance, on traîne en longueur,
        on rembourse par des mots embarrassés,
        et l’on incrimine les circonstances.
${}^{6}Si l’on est solvable, on rend tout au plus la moitié,
        et le prêteur peut considérer cela comme une aubaine.
        \\Sinon, le voilà frustré de son bien ;
        il s’offre même le luxe d’un ennemi
        \\qui le rembourse en insultes et malédictions,
        le payant de mépris au lieu d’estime.
${}^{7}Beaucoup se refusent à prêter, non par méchanceté,
        mais par crainte d’être dépouillés pour rien.
        
           
${}^{8}Pourtant, laisse-toi toucher par le malheureux,
        ne lui fais pas attendre ton aumône.
${}^{9}Viens en aide au pauvre, comme il est prescrit :
        il est dans la misère, ne le renvoie pas les mains vides.
${}^{10}Pour un frère et un ami, sacrifie ton argent,
        plutôt que de le laisser rouiller dans ton coffre.
${}^{11}Investis ta fortune selon les commandements du Très-Haut :
        cela te rapportera plus que l’or.
${}^{12}Fais l’aumône, elle s’entassera dans tes greniers,
        elle te préservera de tout malheur.
${}^{13}Mieux qu’un solide bouclier, mieux qu’une lourde lance,
        elle te défendra contre l’ennemi.
${}^{14}L’homme de bien se porte caution pour son prochain,
        mais celui qui est sans scrupule l’abandonnera.
${}^{15}N’oublie pas les bienfaits de qui s’est porté garant :
        il s’est engagé en personne pour toi.
${}^{16}Le pécheur dilapide les biens de son garant ;
${}^{17}dans son ingratitude, il abandonne celui qui l’a sauvé.
${}^{18}Une caution a ruiné bien des gens prospères
        et les a secoués comme une mer démontée ;
        \\elle a contraint à l’exil des hommes puissants,
        qui ont erré parmi des nations étrangères.
${}^{19}Un pécheur qui, pour le profit, s’expose comme garant
        s’expose en fait aux poursuites judiciaires.
${}^{20}Viens en aide à ton prochain selon tes moyens,
        mais prends garde à ne pas t’exposer à la ruine.
${}^{21}L’essentiel dans la vie, c’est l’eau, le pain, le vêtement
        et une maison pour protéger son intimité.
${}^{22}Mieux vaut une existence de pauvre dans sa cabane
        que faire bonne chère chez les autres.
${}^{23}Que tu aies peu ou beaucoup, montre-toi content,
        et tu n’entendras pas le reproche d’être un étranger.
${}^{24}C’est une triste vie que d’aller de maison en maison,
        où tu restes un étranger et n’oses pas ouvrir la bouche.
${}^{25}Tu fais le service et présentes à boire sans qu’on te dise merci,
        et, en plus, tu entends des paroles dures :
${}^{26}« Viens ici, l’étranger, prépare la table ;
        et aussitôt prêt, donne-moi à manger !
${}^{27}Va-t’en, l’étranger, laisse ta place à quelqu’un d’honorable !
        Mon frère vient me rendre visite, j’ai besoin de la maison. »
${}^{28}Il est pénible pour un homme qui se respecte
        de s’entendre rabrouer par son hôte et insulter par un créancier.
