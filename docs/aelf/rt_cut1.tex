  
  
    
    \bbook{RUTH}{RUTH}
      
         
      \bchapter{}
      \begin{verse}
${}^{1}À l’époque où gouvernaient les Juges, il y eut une famine dans le pays. Un homme de Bethléem de Juda émigra avec sa femme et ses deux fils pour s’établir dans la région appelée Champs-de-Moab. 
${}^{2}L’homme se nommait Élimélek (c’est-à-dire : Mon-Dieu-est-roi), sa femme : Noémi (c’est-à-dire : Ma-gracieuse) et ses deux fils : Mahlone (c’est-à-dire : Maladie) et Kilyone (c’est-à-dire : Épuisement). C’était des Éphratéens de Bethléem de Juda. Ils arrivèrent aux Champs-de-Moab et y restèrent. 
${}^{3}Élimélek, le mari de Noémi, mourut, et Noémi resta seule avec ses deux fils. 
${}^{4}Ceux-ci épousèrent deux Moabites ; l’une s’appelait Orpa (c’est-à-dire : Volte-face)\\et l’autre, Ruth (c’est-à-dire : Compagne)\\. Ils demeurèrent là une dizaine d’années. 
${}^{5}Mahlone et Kilyone moururent à leur tour, et Noémi resta privée de ses deux fils et de son mari. 
${}^{6}Alors, avec ses belles-filles, elle se prépara à quitter les Champs-de-Moab et à retourner chez elle, car elle avait appris\\que le Seigneur avait visité son peuple et lui donnait du pain.
${}^{7}Elle partit donc de l’endroit où elle habitait, accompagnée de ses deux belles-filles. Et elles prirent le chemin du retour vers le pays de Juda. 
${}^{8}Alors Noémi dit à ses deux belles-filles : « Allez, retournez chacune à la maison de votre mère. Que le Seigneur vous montre le même attachement que vous avez eu envers nos morts et envers moi ! 
${}^{9}Que le Seigneur vous donne de trouver chacune un foyer stable, avec un mari. » Et Noémi les embrassa, mais elles élevèrent la voix et se mirent à pleurer. 
${}^{10}Elles lui dirent : « Nous voulons retourner avec toi vers ton peuple. » 
${}^{11}Mais Noémi reprit : « Retournez chez vous, mes filles ! Pourquoi venir avec moi ? Pourrais-je encore avoir des fils à vous donner comme maris ? 
${}^{12}Retournez, mes filles, allez ! Oui, je suis bien trop vieille pour avoir un mari. Quand bien même je dirais : “Il y a encore de l’espoir ; je vais appartenir à un homme cette nuit et j’aurai des fils”, 
${}^{13}même dans ce cas, auriez-vous la patience d’attendre qu’ils grandissent ? Pourriez-vous vous passer d’homme aussi longtemps ? Non, mes filles ! Mon sort est trop amer pour que vous le partagiez. Car c’est contre moi que la main du Seigneur s’est levée. »
${}^{14}Alors les deux belles-filles, de nouveau, élevèrent la voix et se mirent à pleurer. Orpa embrassa sa belle-mère, mais Ruth restait attachée à ses pas\\. 
${}^{15}Noémi lui dit : « Tu vois, ta belle-sœur est retournée vers son peuple et vers ses dieux. Retourne, toi aussi\\, comme ta belle-sœur. »
${}^{16}Ruth lui répondit : « Ne me force pas à t’abandonner et à m’éloigner de toi, car
        \\où tu iras, j’irai ;
        \\où tu t’arrêteras\\, je m’arrêterai ;
        \\ton peuple sera mon peuple,
        \\et ton Dieu sera mon Dieu.
${}^{17}Où tu mourras, je mourrai ;
        \\et là je serai enterrée.
        \\Que le Seigneur me traite ainsi,
        \\qu’il fasse pire encore,
        \\si ce n’est pas la mort seule
        \\qui nous sépare ! »
${}^{18}Voyant qu’elle était résolue à l’accompagner, Noémi cessa de lui parler de cela. 
${}^{19}Ainsi, elles allaient leur chemin, toutes les deux, jusqu’à ce qu’elles arrivent à Bethléem. À leur arrivée à Bethléem, toute la ville fut en émoi. Les femmes disaient : « Est-ce bien là Noémi ? » 
${}^{20}Mais elle leur dit : « Ne m’appelez plus Noémi (Ma-gracieuse), appelez-moi Mara (Amertume). Car le Puissant m’a remplie d’amertume. 
${}^{21}J’étais partie comblée, mais le Seigneur me ramène les mains vides. Pourquoi m’appeler encore Noémi ? Le Seigneur m’a humiliée, le Puissant m’a fait du mal ! »
${}^{22}Noémi revint\\donc des Champs-de-Moab avec sa belle-fille, Ruth la Moabite. Elles arrivèrent à Bethléem au début de la moisson de l’orge.
      
         
      \bchapter{}
      \begin{verse}
${}^{1}Noémi avait un parent du côté de son mari Élimélek ; c’était un riche propriétaire du même clan ; il s’appelait Booz (c'est-à-dire : En-lui-la force). 
${}^{2}Ruth la Moabite dit à Noémi : « Laisse-moi aller glaner dans les champs, derrière celui aux yeux de qui je trouverai grâce. » Elle lui répondit : « Va, ma fille. » 
${}^{3}Ruth partit donc glaner dans les champs derrière les moissonneurs. Elle se trouva par bonheur dans la parcelle d’un champ appartenant à Booz, du clan d’Élimélek. 
${}^{4}Et voici que Booz arriva de Bethléem. Il dit aux moissonneurs : « Le Seigneur soit avec vous ! » Et ceux-ci lui répondirent : « Que le Seigneur te bénisse ! » 
${}^{5}Booz demanda à son serviteur, le chef des moissonneurs : « À qui appartient cette jeune femme ? » 
${}^{6}Celui-ci lui répondit : « Cette jeune femme est une Moabite. Elle est revenue avec Noémi des Champs-de-Moab. 
${}^{7}Elle a dit : “Laisse-moi glaner et ramasser ce qui tombe des gerbes, derrière les moissonneurs.” Depuis qu’elle est arrivée, elle est restée debout, depuis ce matin jusqu’à maintenant. C’est à peine si elle s’est reposée. »
${}^{8}Booz dit à Ruth : « Tu m’entends bien, n’est-ce pas, ma fille ? Ne va pas glaner dans un autre champ. Ne t’éloigne pas de celui-ci, mais attache-toi aux pas de mes servantes. 
${}^{9}Regarde dans quel champ on moissonne, et suis-les. N’ai-je pas interdit aux serviteurs de te molester ? Si tu as soif, va boire aux cruches ce que les serviteurs auront puisé. » 
${}^{10}Alors Ruth se prosterna face contre terre et lui dit : « Pourquoi ai-je trouvé grâce à tes yeux, pourquoi t’intéresser à moi, moi qui suis une étrangère ? » 
${}^{11}Booz lui répondit : « On m’a dit et répété tout ce que tu as fait pour ta belle-mère après la mort de ton mari, comment tu as quitté ton père, ta mère et le pays de ta parenté, pour te rendre chez un peuple que tu n’avais jamais connu de ta vie\\. 
${}^{12}Que le Seigneur te rende en bien ce que tu as fait ! Qu’elle soit complète, la récompense dont te comblera le Seigneur, le Dieu d’Israël, sous les ailes de qui tu es venue t’abriter ! » 
${}^{13}Et Ruth lui dit : « Que je trouve toujours grâce à tes yeux, mon seigneur ! Oui, tu m’as consolée ; oui, tu as parlé au cœur de ta servante, à moi qui ne suis même pas comme l’une de tes servantes. »
${}^{14}Au moment du repas, Booz lui dit : « Approche-toi ; mange de ce pain, trempe ton morceau dans la vinaigrette. » Elle s’assit à côté des moissonneurs, et Booz lui passa des épis grillés. Elle mangea, fut rassasiée et garda le reste. 
${}^{15}Alors elle se leva pour aller glaner, et Booz donna cet ordre à ses serviteurs : « Qu’elle glane aussi entre les gerbes. Ne la rabrouez pas ! 
${}^{16}Et laissez même tomber des épis des brassées. Abandonnez-les, elle glanera. Ne la tracassez pas ! »
${}^{17}Elle glana dans le champ jusqu’au soir ; puis elle égrena ce qu’elle avait glané : elle avait recueilli une quarantaine de mesures d’orge. 
${}^{18}Elle l’emporta et revint en ville. Elle montra à sa belle-mère ce qu’elle avait glané ; ce qu’elle avait gardé après s’être rassasiée, elle le sortit aussi pour le lui donner. 
${}^{19}Sa belle-mère lui dit : « Où donc as-tu glané aujourd’hui ? Où as-tu travaillé ? Béni soit celui qui s’est intéressé à toi ! » Elle raconta alors à sa belle-mère chez qui elle avait travaillé et lui dit : « L’homme chez qui j’ai travaillé aujourd’hui s’appelle Booz. » 
${}^{20}Noémi dit à sa belle-fille : « Il est béni du Seigneur, celui qui n’a pas oublié ses liens avec les vivants et les morts. » Et elle ajouta : « Cet homme est l’un de nos proches parents, l’un de ceux qui ont sur nous droit de rachat. » 
${}^{21}Ruth la Moabite dit : « Il m’a même déclaré : “Tu t’attacheras aux pas de mes serviteurs jusqu’à ce qu’ils aient terminé toute ma moisson.” » 
${}^{22}Noémi dit alors à Ruth, sa belle-fille : « C’est bien, ma fille, que tu ailles avec ses servantes ; ainsi tu ne seras pas maltraitée dans un autre champ. » 
${}^{23}Elle s’attacha donc aux pas des servantes de Booz pour glaner jusqu’à la fin de la moisson de l’orge et de la moisson du blé. Et elle habitait avec sa belle-mère.
      
         
      \bchapter{}
      \begin{verse}
${}^{1}Noémi, sa belle-mère, dit à Ruth : « Ma fille, ne devrais-je pas chercher à t’établir pour que tu sois heureuse ? 
${}^{2}Et maintenant, Booz n’est-il pas notre parent, lui dont tu as suivi les servantes ? Voici que, cette nuit, il vanne lui-même l’orge sur l’aire. 
${}^{3}Va te baigner, te parfumer et mettre ton manteau. Tu descendras sur l’aire. Ne te fais pas reconnaître de l’homme avant qu’il ait fini de manger et de boire. 
${}^{4}Quand il sera couché, tu sauras où il se couche. Alors, va, découvre-lui les pieds, et là, tu te coucheras. Lui t’indiquera ce que tu devras faire. » 
${}^{5}Et Ruth lui répondit : « Tout ce que tu me dis, je le ferai. »
${}^{6}Elle descendit sur l’aire et fit tout ce que sa belle-mère lui avait ordonné. 
${}^{7}Booz mangea et but. Puis, le cœur content, il alla se coucher contre la meule. Alors, Ruth s’approcha discrètement, découvrit les pieds de Booz et se coucha. 
${}^{8}Or, au milieu de la nuit, l’homme frissonna, il se tourna pour voir : et voici qu’une femme était couchée à ses pieds ! 
${}^{9}Il demanda : « Qui es-tu ? » Elle répondit : « C’est moi, Ruth ta servante. Étends sur ta servante le pan de ton manteau, car c’est toi qui as droit de rachat. » 
${}^{10}Alors, il dit : « Sois bénie du Seigneur, ma fille ! Ce geste d’attachement est encore plus beau que le premier : tu n’as pas recherché les jeunes gens, pauvres ou riches. 
${}^{11}Et maintenant, ma fille, n’aie pas peur ; tout ce que tu diras, je le ferai pour toi, car tout le monde ici sait que tu es une femme parfaite. 
${}^{12}C’est vrai que j’ai droit de rachat, mais il existe un plus proche parent que moi qui a droit de rachat. 
${}^{13}Passe donc la nuit ici, et demain matin, s’il veut te racheter, eh bien ! qu’il te rachète ! Mais s’il ne le veut pas, c’est moi qui te rachèterai, aussi vrai que le Seigneur est vivant ! Reste couchée jusqu’au matin ! » 
${}^{14}Elle resta donc couchée à ses pieds jusqu’au matin, mais elle se leva avant qu’on puisse reconnaître qui que ce soit. Car Booz se disait : « Il ne faut pas qu’on apprenne que cette femme est venue sur l’aire. » 
${}^{15}Il lui dit alors : « Présente le châle que tu portes et tiens-le bien. » Elle le tint donc ; il mesura six mesures d’orge et l’aida à s’en charger. Puis il rentra en ville. 
${}^{16}Ruth revint chez sa belle-mère qui lui demanda : « Que t’est-il arrivé, ma fille ? » Alors Ruth lui raconta tout ce que l’homme avait fait pour elle 
${}^{17}et elle ajouta : « Il m’a donné ces six mesures d’orge en me disant : “Ne rentre pas chez ta belle-mère les mains vides.” » 
${}^{18}Noémi lui dit : « Reste ici, ma fille, jusqu’à ce que tu saches comment l’affaire aboutira. Car cet homme n’aura de cesse qu’il n’ait conclu cette affaire, aujourd’hui même. »
      
         
      \bchapter{}
      \begin{verse}
${}^{1}Booz était monté à la porte de la ville, et il s’y était assis. Et voici que vint à passer celui dont Booz avait parlé, celui qui avait droit de rachat. Booz l’appela : « Hé, toi ! Arrête-toi un peu, viens t’asseoir ici ! » Il s’arrêta et il s’assit. 
${}^{2}Booz prit alors dix hommes parmi les anciens d’Israël et leur dit : « Venez vous asseoir ici pour siéger. » Et ils s’assirent. 
${}^{3}Puis il s’adressa à celui qui avait droit de rachat : « La parcelle du champ qui appartenait à notre frère Élimélek, Noémi, qui vient de revenir des Champs-de-Moab, la met en vente. 
${}^{4}Et moi, je me suis dit que j’allais t’en informer en disant : “Veux-tu, devant ceux qui siègent ici, devant les anciens du peuple, veux-tu acquérir ce champ ?” Si tu veux exercer ton droit de rachat, fais-le, mais si tu ne veux pas l’exercer, déclare-le moi, pour que je le sache. En effet, personne, sauf toi, ne peut exercer ce droit, sinon moi après toi. » Alors l’autre dit : « Moi, je veux l’exercer. » 
${}^{5}Booz reprit : « Le jour où, de la main de Noémi, tu prends possession du champ, tu prends également possession de Ruth la Moabite, la femme de celui qui est mort, afin que le nom du mort reste attaché à son héritage. » 
${}^{6}Alors, celui qui avait droit de rachat dit : « Je ne pourrais pas exercer mon droit de rachat sans détruire mon propre héritage. Toi, exerce donc le droit de rachat, puisque je ne le peux pas. »
${}^{7}Or, jadis en Israël, pour le rachat ou pour l’échange, afin de conclure toute affaire, l’un enlevait sa sandale et la donnait à l’autre. En Israël, cela servait de témoignage. 
${}^{8}Celui qui avait droit de rachat dit alors à Booz : « À toi de te porter acquéreur ! » Et il enleva sa sandale. 
${}^{9}Booz dit aux anciens et à tout le peuple : « Aujourd’hui, vous en êtes témoins : de la main de Noémi, j’ai pris possession de tout ce qui appartenait à Élimélek ainsi qu’à Kilyone et Mahlone. 
${}^{10}J’ai également pris pour femme Ruth, la Moabite, la femme de Mahlone, afin que le nom du mort reste attaché à son héritage et ne soit pas effacé parmi ses frères ni à la porte de sa ville. Vous en êtes témoins, aujourd’hui. » 
${}^{11}Tout le peuple qui se trouvait à la porte de la ville, ainsi que les anciens, répondirent : « Nous en sommes témoins. Que le Seigneur rende la femme qui entre dans ta maison comme Rachel et comme Léa qui, à elles deux, ont bâti la maison d’Israël !
      Fais fortune en Éphrata !
      Fais-toi un nom à Bethléem !
${}^{12}Puisse la descendance que le Seigneur te donnera par cette jeune femme rendre ta maison comme la maison de Pérès que Tamar enfanta à Juda ! »
${}^{13}Booz prit donc Ruth comme épouse, elle devint sa femme et il s’unit à elle. Le Seigneur lui accorda de concevoir, et elle enfanta un fils. 
${}^{14}Les femmes de Bethléem\\dirent à Noémi : « Béni soit le Seigneur qui aujourd’hui ne t’a pas laissée sans quelqu’un pour te racheter ! Que son nom soit célébré en Israël ! 
${}^{15}Cet enfant te fera revivre, il sera l’appui de ta vieillesse : il est né de ta belle-fille qui t’aime, et qui vaut mieux pour toi que sept fils. » 
${}^{16}Noémi prit l’enfant, le mit sur son sein, et se chargea de l’élever.
${}^{17}Les voisines lui donnèrent son nom. Elles disaient : « Il est né un fils à Noémi. » Et elles le nommèrent Obed (c'est-à-dire : serviteur). Ce fut le père de Jessé, qui fut le père de David.
${}^{18}Voici la descendance de Pérès :
      <p class="retrait1">Pérès engendra Esrone.
${}^{19}Esrone engendra Ram,
      <p class="retrait1">Ram engendra Aminadab.
${}^{20}Aminadab engendra Naassone,
      <p class="retrait1">Naassone engendra Salmone ;
${}^{21}Salmone engendra Booz,
      <p class="retrait1">Booz engendra Obed ;
${}^{22}Obed engendra Jessé,
      <p class="retrait1">et Jessé engendra David.
