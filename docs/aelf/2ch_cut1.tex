  
  
    
    \bbook{DEUXIÈME LIVRE DES CHRONIQUES}{DEUXIÈME LIVRE DES CHRONIQUES}
      
         
      \bchapter{}
      \begin{verse}
${}^{1}Salomon, fils de David, s’affermit dans sa royauté. Le Seigneur son Dieu était avec lui ; il l’éleva au sommet de la grandeur.
      
         
${}^{2}Salomon s’adressa à tout Israël : aux officiers de millier et de centaine, aux juges et à tous les responsables de tout Israël, chefs des familles. 
${}^{3}Salomon et toute l’assemblée avec lui se rendirent au lieu sacré de Gabaon. C’est là que se trouvait la tente de la Rencontre de Dieu, celle que Moïse, le serviteur du Seigneur, avait faite dans le désert. 
${}^{4}Mais l’arche de Dieu, David l’avait fait monter de Qiryath-Yearim vers l’endroit qu’il avait fait préparer pour elle ; pour elle, en effet, il avait dressé une tente à Jérusalem. 
${}^{5}À Gabaon, devant la Demeure du Seigneur, se trouvait l’autel de bronze, qu’avait fait Beçalel, fils d’Ouri, fils de Hour. Alors Salomon et l’assemblée consultèrent le Seigneur. 
${}^{6}En ce lieu, Salomon monta à l’autel de bronze qui était devant le Seigneur, près de la tente de la Rencontre, et il y offrit en sacrifice mille holocaustes.
${}^{7}Cette nuit-là, Dieu apparut à Salomon et lui dit : « Demande ce que je dois te donner. » 
${}^{8}Salomon répondit à Dieu : « Tu as traité David, mon père, avec une grande fidélité, et tu m’as fait roi à sa place. 
${}^{9}À présent, Seigneur Dieu, la parole que tu as adressée à David mon père se vérifie, car c’est toi qui m’as fait roi sur un peuple aussi nombreux que la poussière de la terre. 
${}^{10}Maintenant, Seigneur, donne-moi sagesse et connaissance, pour que je sache comment me comporter à la tête de ce peuple. Qui, en effet, peut gouverner ce grand peuple qui est le tien ? » 
${}^{11}Dieu répondit à Salomon : « Puisque c’est cela que tu as pris à cœur et que tu ne m’as demandé ni richesse, ni biens, ni gloire, ni la vie de tes ennemis, puisque tu ne m’as pas demandé non plus de longs jours, mais que tu as demandé pour toi sagesse et connaissance, afin de gouverner mon peuple sur lequel je te fais roi, 
${}^{12}la sagesse et la connaissance te sont données. Je te donnerai aussi la richesse, les biens et la gloire, comme aucun roi n’en a eu avant toi et comme aucun n’en aura après toi. » 
${}^{13}Salomon quitta le lieu sacré de Gabaon pour Jérusalem, loin de la tente de la Rencontre, et il régna sur Israël.
${}^{14}Salomon rassembla des chars et des cavaliers ; il eut ainsi mille quatre cents chars et douze mille cavaliers ; il les cantonna dans les villes de garnison et près du roi, à Jérusalem. 
${}^{15}À Jérusalem, le roi fit abonder l’argent et l’or autant que les pierres, et les cèdres autant que les sycomores dans le Bas-Pays. 
${}^{16}Les chevaux de Salomon provenaient d’Égypte, et le groupe des courtiers du roi, à Qewé, les achetaient pour un prix convenu. 
${}^{17}Ces derniers montaient et ramenaient d’Égypte un char pour six cents pièces d’argent et un cheval pour cent cinquante. Il en était de même pour tous les rois hittites et pour les rois d’Aram que ces courtiers approvisionnaient.
${}^{18}Salomon donna l’ordre de bâtir une maison pour le nom du Seigneur et, pour lui-même, une maison royale.
      
         
      \bchapter{}
      \begin{verse}
${}^{1}Salomon engagea soixante-dix mille porteurs, quatre-vingt mille ouvriers pour les carrières dans la montagne et, avec eux, trois mille six cents contremaîtres. 
${}^{2}Salomon envoya ce message à Houram, roi de Tyr : « Tu as agi en faveur de mon père David, à qui tu as fait parvenir des cèdres pour se bâtir une maison où habiter. 
${}^{3}Or voici que je vais bâtir une maison pour le nom du Seigneur mon Dieu ; je la lui consacrerai ; on y brûlera de l’encens aromatique devant lui ; on y disposera la rangée de pains pour l’offrande perpétuelle et on y fera des holocaustes matin et soir, pour les sabbats, les nouvelles lunes et les solennités du Seigneur notre Dieu. Cela incombe à Israël pour toujours. 
${}^{4}La maison que je vais bâtir sera grande, car notre Dieu est plus grand que tous les dieux. 
${}^{5}Mais qui serait capable de lui bâtir une maison, quand les cieux et les hauteurs des cieux ne peuvent le contenir ? Et qui suis-je, moi, pour lui bâtir une maison, si ce n’était en vue de brûler de l’encens devant lui ? 
${}^{6}Maintenant, envoie-moi un homme habile à travailler l’or, l’argent, le bronze, le fer, la pourpre rouge, le carmin et la pourpre violette, et connaissant l’art de la gravure ; il se joindra aux artisans qui sont près de moi en Juda et à Jérusalem, ceux que David mon père a préparés. 
${}^{7}Fais-moi parvenir du Liban des bois de cèdre, de cyprès et de santal, car je sais que tes serviteurs peuvent abattre les arbres du Liban. Et voici que mes serviteurs seront avec tes serviteurs, 
${}^{8}pour me préparer des bois en quantité, car la Maison que je vais bâtir sera grande et merveilleuse. 
${}^{9}À tes serviteurs, les bûcherons qui abattront ces arbres, je donne, pour les nourrir, vingt mille quintaux de blé, vingt mille quintaux d’orge, ainsi que vingt mille mesures de vin et vingt mille mesures d’huile. »
${}^{10}Houram, roi de Tyr, répondit par une lettre qu’il envoya à Salomon : « C’est parce que le Seigneur aime son peuple qu’il t’a établi roi sur eux. » 
${}^{11}Houram dit encore : « Béni soit le Seigneur, Dieu d’Israël, lui qui a fait le ciel et la terre ! Il a donné au roi David un fils sage, prudent et intelligent, qui bâtira une maison pour le Seigneur et, pour lui-même, une maison royale. 
${}^{12}Maintenant j’envoie un homme habile et intelligent, appelé Houram-Abi ; 
${}^{13}il est le fils d’une femme originaire de Dane, mais son père est de Tyr. Il sait travailler l’or, l’argent, le bronze, le fer, la pierre, le bois, la pourpre rouge, la pourpre violette, le lin et le carmin ; il connaît l’art de la gravure et saura réaliser toute œuvre qui lui sera confiée, en collaborant avec tes artisans et ceux de mon seigneur David, ton père. 
${}^{14}Et maintenant, le blé, l’orge, l’huile et le vin dont mon seigneur a parlé, qu’il les envoie à ses serviteurs. 
${}^{15}De notre côté, nous abattrons des arbres du Liban, selon tous tes besoins ; nous te les apporterons à Jaffa par radeaux sur la mer, et toi, tu les feras monter à Jérusalem. »
${}^{16}Salomon recensa tous les étrangers qui résidaient dans le pays d’Israël, après le recensement qu’avait fait David son père : on en trouva cent cinquante-trois mille six cents. 
${}^{17}Il en employa soixante-dix mille comme porteurs, quatre-vingt mille comme ouvriers pour les carrières dans la montagne et trois mille six cents comme contremaîtres pour faire travailler les gens.
      
         
      \bchapter{}
      \begin{verse}
${}^{1}Alors Salomon commença à bâtir la maison du Seigneur à Jérusalem, sur le mont Moriya, là où le Seigneur était apparu à David, son père, à l’emplacement que David avait préparé sur l’aire d’Ornane le Jébuséen. 
${}^{2}C’est au deuxième mois de la quatrième année de son règne qu’il commença à bâtir.
${}^{3}Les fondations établies par Salomon pour bâtir la maison de Dieu étaient, en longueur, de soixante coudées d’ancienne mesure et, en largeur, de vingt coudées. 
${}^{4}Devant la Grande Salle de la Maison, le vestibule était long de vingt coudées dans le sens de la largeur de la Maison ; il avait cent vingt coudées de haut. Salomon le plaqua d’or pur à l’intérieur. 
${}^{5}La Grande Maison, il la recouvrit de bois de cyprès, qu’il recouvrit ensuite d’or pur, rehaussé de palmiers et de chaînettes. 
${}^{6}Il revêtit la Maison d’une parure de pierres précieuses. L’or était de l’or de Parwaïm ! 
${}^{7}Il recouvrit d’or la Maison : les poutres, les seuils, les murs et les portes. Il fit graver des kéroubim sur les murs. 
${}^{8}Puis, il fit la salle du Saint des Saints. Elle était longue de vingt coudées dans le sens de la largeur de la Maison, et large de vingt coudées. Il la recouvrit d’or pur, d’un poids de six cents lingots. 
${}^{9}Le poids des clous était celui de cinquante pièces d’or. Il recouvrit d’or les chambres hautes.
${}^{10}Dans la salle du Saint des Saints, Salomon fit deux kéroubim, ouvrage de métal fondu, et on les plaqua d’or. 
${}^{11}Les ailes des kéroubim avaient une longueur de vingt coudées ; l’une des ailes du premier kéroub, d’une longueur de cinq coudées, touchait le mur de la Maison ; l’autre, également de cinq coudées, touchait celle du second kéroub. 
${}^{12}Une aile du second kéroub, de cinq coudées, touchait le mur de la Maison ; l’autre, également de cinq coudées, rejoignait l’aile de l’autre kéroub. 
${}^{13}Déployées, les ailes de ces kéroubim mesuraient vingt coudées. Ils se tenaient debout sur leurs pieds, la face tournée vers la Maison. 
${}^{14}Salomon fit le rideau de pourpre violette et de pourpre rouge, de carmin et de lin ; et il le rehaussa de kéroubim.
${}^{15}Il fit devant la Maison deux colonnes de trente-cinq coudées de haut ; à leur sommet, le chapiteau avait cinq coudées. 
${}^{16}Il fit aussi des chaînettes en collier et les plaça au sommet des colonnes. Il fit encore cent grenades et les plaça sur les chaînettes. 
${}^{17}Il dressa les colonnes sur le devant de la Grande Salle, l’une à droite et l’autre à gauche. Il donna à celle de droite le nom de Yakine (ce qui signifie : « Il rend stable »), à celle de gauche le nom de Boaz (ce qui signifie : « En lui la force »).
      
         
      \bchapter{}
      \begin{verse}
${}^{1}Il fit un autel de bronze, dont la longueur était de vingt coudées, la largeur de vingt coudées et la hauteur de dix coudées.
${}^{2}Il fit la Mer, bassin en métal fondu, de dix coudées de diamètre, car son pourtour était circulaire. Elle avait cinq coudées de haut. Un cordeau de trente coudées en aurait fait le tour. 
${}^{3}Au-dessous, des figures de bœufs encerclaient la Mer, tout autour, dix par coudées. Les bœufs étaient disposés sur deux rangées, fondus ensemble avec la Mer. 
${}^{4}La Mer était dressée sur douze bœufs : trois faisaient face au nord, trois faisaient face à l’ouest, trois faisaient face au sud, trois faisaient face à l’est. La Mer reposait directement dessus, leurs arrière-trains étaient tournés vers l’intérieur. 
${}^{5}L’épaisseur de la Mer était d’une largeur de paume, son rebord était comme le bord d’une coupe, en forme de fleur de lis. Sa contenance était de trois mille mesures.
${}^{6}Il fit dix cuves, qu’il disposa ainsi : cinq à droite et cinq à gauche. Elles servaient aux ablutions, mais on y nettoyait aussi ce qui était utile pour l’holocauste. Les prêtres faisaient leurs ablutions dans la Mer. 
${}^{7}Il fit dix chandeliers d’or selon le modèle prescrit, et il les plaça dans la Grande Salle : cinq à droite et cinq à gauche. 
${}^{8}Il fit dix tables et les plaça dans la Grande Salle : cinq à droite et cinq à gauche. Il fit cent bols en or pour l’aspersion. 
${}^{9}Il fit le parvis des prêtres et la grande cour, ainsi que les portes pour la grande cour. Et ces portes, il les plaqua de bronze. 
${}^{10}Quant à la Mer, il la posa du côté droit, au sud-est.
${}^{11}Houram fit les vases, les pelles, et les bols pour l’aspersion. Houram acheva le travail qu’il avait à faire pour le roi Salomon dans la maison de Dieu : 
${}^{12}les deux colonnes, les arrondis des chapiteaux sur le sommet des deux colonnes ; les deux filets pour couvrir les deux arrondis des chapiteaux sur le sommet des colonnes ; 
${}^{13}les quatre cents grenades destinées aux deux filets – deux rangées de grenades par filet –, de façon à couvrir les deux arrondis des chapiteaux, jusque sur le flanc des colonnes ; 
${}^{14}il fit les bases et fit les cuves sur les bases ; 
${}^{15}la Mer – une seule –, avec les douze bœufs en dessous d’elle ; 
${}^{16}les vases, les pelles, les fourchettes : tous les objets. Houram-Abiv les avait fabriqués en bronze luisant, pour le roi Salomon dans la maison du Seigneur. 
${}^{17}C’est dans la région du Jourdain, entre Souccoth et Seréda, que le roi les avait fait mouler dans des couches d’argile. 
${}^{18}Salomon fit tous ces objets. Il y en avait tant que l’on ne pouvait estimer le poids du bronze employé.
${}^{19}Salomon fit tous les objets de la maison de Dieu : l’autel d’or, et les tables pour le pain de l’offrande ; 
${}^{20}les chandeliers et leurs lampes, en or fin, que l’on allume selon la règle devant le Saint des Saints ; 
${}^{21}les fleurs, les lampes et les pincettes, en or, d’un or incomparable ; 
${}^{22}les ciseaux, les bols pour l’aspersion, les gobelets, les brûle-parfums, en or fin ; l’entrée de la Maison, ses portes intérieures pour le Saint des Saints, et les portes de la Maison pour la Grande Salle : en or !
      
         
      \bchapter{}
      \begin{verse}
${}^{1}Ainsi fut parachevé tout le travail entrepris par Salomon pour la maison du Seigneur. Salomon fit apporter les objets sacrés de David son père : l’argent, l’or et tous les ustensiles ; il les déposa dans les trésors de la maison de Dieu.
      
         
${}^{2}Salomon rassembla à Jérusalem les anciens d’Israël et tous les chefs des tribus, les chefs de famille des fils d’Israël, pour aller chercher l’arche de l’Alliance du Seigneur dans la Cité de David, c’est-à-dire à Sion. 
${}^{3}Tous les hommes d’Israël se rassemblèrent auprès du roi au septième mois, durant la fête des Tentes. 
${}^{4}Quand tous les anciens d’Israël furent arrivés, les Lévites se chargèrent de l’Arche. 
${}^{5}Ils emportèrent l’Arche et la tente de la Rencontre avec tous les objets sacrés qui s’y trouvaient ; ce sont les prêtres lévites qui les transportèrent. 
${}^{6}Le roi Salomon, et toute la communauté d’Israël qu’il avait convoquée\\devant l’Arche, offrirent en sacrifice des moutons et des bœufs : il y en avait tant qu’on ne pouvait les compter. 
${}^{7}Puis les prêtres transportèrent l’arche de l’Alliance du Seigneur à sa place, dans la chambre sacrée que l’on appelle le Saint des Saints, sous les ailes des kéroubim. 
${}^{8}Ceux-ci étendaient leurs ailes au-dessus de l’emplacement de l’Arche : ils abritaient l’Arche et ses barres. 
${}^{9}Les barres étaient si longues que l’on pouvait voir leurs extrémités depuis l’Arche, devant le Saint des Saints ; mais on ne les voyait pas de l’extérieur. Elles y sont encore à ce jour. 
${}^{10}Dans l’Arche, il n’y avait rien, sinon les deux tables de la Loi\\que Moïse y avait placées, au mont Horeb, quand le Seigneur avait conclu alliance avec les fils d’Israël à leur sortie d’Égypte.
${}^{11}Alors les prêtres sortirent du sanctuaire. En effet, tous les prêtres qui se trouvaient là s’étaient sanctifiés sans observer l’ordre des classes ; 
${}^{12}les Lévites qui étaient chantres étaient au complet : Asaf, Hémane, Yedoutoune, leurs fils et leurs frères, vêtus de lin, se tenaient debout avec des cymbales, des harpes et des cithares, à l’orient de l’autel ; il y avait auprès d’eux cent vingt prêtres sonnant de la trompette. 
${}^{13}Tous ensemble, ceux qui jouaient de la trompette et ceux qui chantaient louaient et célébraient le Seigneur d’une seule voix. Élevant la voix au son des trompettes, des cymbales et des instruments de musique, ils louaient le Seigneur : « Car il est bon, éternel est son amour. »
      Alors la Maison, la maison du Seigneur, fut remplie par une nuée, 
${}^{14} et, à cause d’elle, les prêtres durent interrompre le service divin : la gloire du Seigneur remplissait la maison de Dieu.
      
         
      \bchapter{}
      \begin{verse}
${}^{1}Alors Salomon s’écria :
        \\« Le Seigneur déclare demeurer dans la nuée obscure.
        ${}^{2}Et maintenant, je t’ai construit, Seigneur\\,
        \\une maison somptueuse,
        \\un lieu où tu habiteras éternellement. »
${}^{3}Puis le roi se retourna et bénit toute l’assemblée d’Israël ; toute l’assemblée d’Israël se tenait debout. 
${}^{4}Il dit :
      « Béni soit le Seigneur, Dieu d’Israël ! De ses mains, il a accompli ce qu’il avait dit de sa bouche à David mon père : 
${}^{5}“Depuis le jour où j’ai fait sortir mon peuple du pays d’Égypte, je n’ai choisi aucune ville entre toutes les tribus d’Israël pour y construire une maison où serait mon nom, et je n’ai pas choisi d’homme pour être chef de mon peuple Israël. 
${}^{6}Mais j’ai choisi Jérusalem pour que mon nom y demeure, et j’ai choisi David pour qu’il soit le chef de mon peuple Israël.” 
${}^{7}Or David, mon père, avait pris à cœur de construire une maison pour le nom du Seigneur, Dieu d’Israël. 
${}^{8}Mais le Seigneur a dit à David, mon père : “Tu as pris à cœur de construire une maison pour mon nom, et tu as bien fait de prendre cela à cœur. 
${}^{9}Cependant, ce n’est pas toi qui construiras la maison, mais ton fils, issu de toi : c’est lui qui construira la maison pour mon nom.” 
${}^{10}Le Seigneur a réalisé la parole qu’il avait dite, et j’ai succédé à David, mon père, je me suis assis sur le trône d’Israël, comme l’avait dit le Seigneur, et j’ai construit la maison pour le nom du Seigneur, Dieu d’Israël. 
${}^{11}Là j’ai placé l’Arche où se trouve l’Alliance du Seigneur, qu’il a conclue avec les fils d’Israël. »
${}^{12}Salomon se plaça devant l’autel du Seigneur, en face de toute l’assemblée d’Israël, et il étendit les mains. 
${}^{13}En effet, Salomon avait dressé une estrade de bronze et l’avait installée au milieu de la cour ; elle avait cinq coudées de long, cinq coudées de large et trois coudées de haut. Il y prit place, fléchit les genoux en face de toute l’assemblée d’Israël, puis il étendit les mains vers le ciel, 
${}^{14}et fit cette prière :
      « Seigneur, Dieu d’Israël, il n’y a pas de Dieu comme toi, ni dans les cieux ni sur la terre : tu gardes ton Alliance et ta fidélité envers tes serviteurs, quand ils marchent devant toi de tout leur cœur. 
${}^{15}Tu as gardé pour ton serviteur David, mon père, ce que tu lui avais dit ; et ce que tu lui avais dit de ta bouche, de ta main aujourd’hui tu l’as accompli. 
${}^{16}Maintenant, Seigneur, Dieu d’Israël, par égard pour ton serviteur David, mon père, garde la parole que tu lui avais dite : “Tes descendants qui siégeront sur le trône d’Israël ne seront pas écartés de ma présence, pourvu que tes fils veillent à suivre leur chemin en marchant selon ma Loi, comme tu as marché devant moi.” 
${}^{17}Maintenant, Seigneur, Dieu d’Israël, que se vérifie la parole que tu as dite à ton serviteur David !
${}^{18}Est-ce que, vraiment, Dieu habiterait avec l’homme sur la terre ? Les cieux et les hauteurs des cieux ne peuvent te contenir : encore moins cette Maison que j’ai bâtie ! 
${}^{19}Sois attentif à la prière et à la supplication de ton serviteur. Écoute, Seigneur mon Dieu, la prière et le cri qu’il lance vers toi. 
${}^{20}Que tes yeux soient ouverts jour et nuit sur cette Maison, sur ce lieu dont tu as dit que là tu mettrais ton nom. Écoute donc la prière que ton serviteur fera en ce lieu. 
${}^{21}Écoute les supplications de ton serviteur et de ton peuple Israël, lorsqu’ils prieront en ce lieu. Toi, depuis les cieux où tu habites, écoute et pardonne.
${}^{22}Quand un homme aura péché contre son prochain et qu’on lui imposera un serment qui peut se retourner contre lui, s’il vient à prêter ce serment devant ton autel dans cette Maison, 
${}^{23}toi, depuis les cieux, écoute, agis et juge tes serviteurs. Le coupable, rends-lui son dû : que sa conduite retombe sur sa tête ; déclare juste le juste : traite-le selon sa justice !
${}^{24}Quand ton peuple Israël aura été battu devant l’ennemi, parce qu’il aura péché contre toi, s’il revient et célèbre ton nom, s’il prie et te supplie dans cette Maison, 
${}^{25}toi, depuis les cieux, écoute, pardonne le péché de ton peuple Israël et fais-les revenir sur le sol que tu leur as donné ainsi qu’à leurs pères.
${}^{26}Lorsque les cieux seront fermés et qu’il n’y aura pas de pluie, parce que les fils d’Israël auront péché contre toi, s’ils prient vers ce lieu et célèbrent ton nom, s’ils se détournent de leur péché, parce que tu les auras humiliés, 
${}^{27}toi, dans les cieux, écoute, pardonne le péché de tes serviteurs et de ton peuple Israël. Tu leur enseigneras le bon chemin par où ils doivent marcher, et tu accorderas la pluie à ta terre, celle que tu as donnée à ton peuple en héritage.
${}^{28}Lorsqu’il y aura la famine dans le pays, lorsqu’il y aura la peste, la rouille et la nigelle du blé, les sauterelles et les criquets, lorsque son ennemi assiégera une ville dans le pays, en tout fléau, en toute maladie, 
${}^{29}quel que soit le motif de la prière ou de la supplication émanant de tout homme ou de tout ton peuple Israël, dès l’instant où chacun reconnaît sa plaie et sa douleur, et qu’il tend les mains vers cette maison, 
${}^{30}toi, depuis les cieux où tu habites, écoute, pardonne. Traite chacun selon toute sa conduite, puisque tu connais son cœur – toi seul, en effet, connais le cœur de l’homme –, 
${}^{31}afin qu’ils te craignent, en marchant dans tes voies, tous les jours qu’ils vivront sur le sol que tu as donné à nos pères.
${}^{32}Si, à cause de ton grand nom, à cause de ta main forte et de ton bras étendu, un étranger, qui n’est pas de ton peuple Israël, vient d’un pays lointain prier dans cette Maison, 
${}^{33}toi, depuis les cieux où tu habites, écoute-le. Exauce toutes les demandes de l’étranger. Ainsi, tous les peuples de la terre, comme ton peuple Israël, vont reconnaître ton nom et te craindre. Et ils sauront que ton nom est invoqué sur cette Maison que j’ai bâtie.
${}^{34}Lorsque ton peuple partira en guerre contre ses ennemis, dans la direction où tu l’auras envoyé, et qu’il te priera, tourné vers cette ville que tu as choisie et vers la Maison que j’ai bâtie pour ton nom, 
${}^{35}toi, depuis les cieux, écoute leur prière et leur supplication, et rends-leur justice.
${}^{36}Lorsqu’ils pécheront contre toi – car il n’est pas d’être humain qui ne commette quelque péché – et que tu seras irrité contre eux, alors tu les livreras à la merci de leurs ennemis, et leurs vainqueurs les emmèneront captifs dans un pays lointain ou proche. 
${}^{37}Si, au pays où ils auront été emmenés captifs, ils rentrent en eux-mêmes, s’ils se repentent, s’ils élèvent vers toi leur supplication dans le pays de leur captivité, en disant : “Nous avons péché, nous avons commis une faute, nous avons fait ce qui est mal” ; 
${}^{38}s’ils reviennent à toi de tout leur cœur et de toute leur âme, au pays de leur captivité, là où ils ont été emmenés captifs, et s’ils prient, tournés vers le pays que tu as donné à leurs pères, vers la ville que tu as choisie et vers la Maison que j’ai bâtie pour ton nom, 
${}^{39}toi, depuis les cieux où tu habites, écoute leur prière et leur supplication, rends-leur justice et pardonne à ton peuple qui a péché contre toi.
${}^{40}À présent, mon Dieu, que tes yeux restent ouverts,
        et tes oreilles attentives à la prière faite en ce lieu !
${}^{41}Et maintenant, monte, Seigneur Dieu,
        vers le lieu de ton repos,
        toi, et l’arche de ta force !
        \\Que tes prêtres, Seigneur Dieu, soient vêtus de salut,
        que tes fidèles exultent dans le bonheur !
${}^{42}Seigneur Dieu, ne repousse pas la face de ton messie,
        souviens-toi de ce que David, ton serviteur,
        a fait avec fidélité ! »
      
         
      \bchapter{}
      \begin{verse}
${}^{1}Quand Salomon eut achevé de prier, le feu descendit du ciel, il dévora l’holocauste et les sacrifices ; et la gloire du Seigneur remplit la Maison. 
${}^{2}Les prêtres ne purent entrer dans la maison du Seigneur, car la gloire du Seigneur remplissait la maison du Seigneur. 
${}^{3}Tous les fils d’Israël, voyant le feu descendre et la gloire du Seigneur sur la Maison, s’inclinèrent face contre terre sur le dallage ; ils se prosternèrent pour rendre grâce au Seigneur « car il est bon, éternel est son amour ! »
${}^{4}Le roi et tout le peuple offraient des sacrifices devant le Seigneur. 
${}^{5}Le roi Salomon offrit en sacrifice vingt-deux mille bœufs et cent vingt mille moutons. C’est ainsi que le roi et tout le peuple firent la dédicace de la maison de Dieu. 
${}^{6}Les prêtres se tenaient à leur poste, ainsi que les Lévites avec les instruments de musique du Seigneur, ceux qu’avait faits le roi David, afin de rendre grâce au Seigneur « car éternel est son amour ! », comme au temps où, par eux, David psalmodiait. Les prêtres, en face d’eux, sonnaient de la trompette, et tout Israël se tenait debout.
${}^{7}Salomon consacra le milieu de la cour qui était devant la maison du Seigneur. C’est là, en effet, qu’il offrit les holocaustes et les graisses des sacrifices de paix, car l’autel de bronze que Salomon avait fait ne pouvait pas contenir l’holocauste, l’offrande de céréales et les graisses. 
${}^{8}Salomon – et tout Israël avec lui – célébra, en ce temps-là, la fête pendant sept jours. Ce fut un très grand rassemblement depuis l’Entrée-de-Hamath jusqu’au Torrent d’Égypte. 
${}^{9}On avait célébré la dédicace de l’autel pendant sept jours ; le huitième jour, on fit une réunion solennelle, et la fête dura encore sept jours. 
${}^{10}Le vingt-troisième jour du septième mois, Salomon renvoya le peuple à ses tentes. Joyeux, ils avaient le cœur content pour le bien que le Seigneur avait accordé à David, à Salomon, et à Israël, son peuple.
${}^{11}Salomon acheva la maison du Seigneur et la maison du roi, et tout ce que Salomon avait désiré faire dans la maison du Seigneur et dans sa propre maison, il le mena à bien. 
${}^{12}Alors le Seigneur apparut à Salomon durant la nuit, et il lui dit : « J’ai entendu ta prière et j’ai choisi pour moi ce lieu comme maison de sacrifices. 
${}^{13}Si je ferme le ciel et qu’il n’y ait pas de pluie, si je commande à la sauterelle de dévorer le pays, si j’envoie la peste dans mon peuple, 
${}^{14}et que mon peuple, sur qui est prononcé mon nom, s’incline et prie, s’il recherche ma face et revient de sa conduite mauvaise, moi, j’écouterai depuis les cieux, je pardonnerai son péché et je guérirai son pays. 
${}^{15}Maintenant mes yeux sont ouverts, et mes oreilles attentives à la prière faite en ce lieu. 
${}^{16}À présent, j’ai choisi et consacré cette Maison, afin que mon nom y soit à jamais ; mes yeux et mon cœur y seront pour toujours. 
${}^{17}Pour toi, si tu marches devant moi comme l’a fait David, ton père, afin d’agir en tout selon mes commandements, et si tu gardes mes décrets et mes ordonnances, 
${}^{18}alors je maintiendrai le trône de ta royauté selon ce que j’ai conclu avec David, ton père, en lui disant : “Aucun des tiens régnant sur Israël ne sera écarté.”
${}^{19}Mais si vous vous détournez, et si vous abandonnez les commandements et les décrets que j’ai placés devant vous, si vous suivez et servez d’autres dieux et vous prosternez devant eux, 
${}^{20}alors j’arracherai les fils d’Israël de ma terre, celle que je leur ai donnée ; et cette Maison que j’ai consacrée à mon nom, je la rejetterai loin de ma face ; et j’en ferai la fable et la risée de tous les peuples. 
${}^{21}Cette Maison, qui avait été sublime, sera objet de stupéfaction pour quiconque passera près d’elle ; il dira : “Pourquoi donc le Seigneur a-t-il agi de cette manière envers ce pays et envers cette Maison ?” 
${}^{22}On lui répondra : “C’est qu’ils ont abandonné le Seigneur, Dieu de leurs pères, lui qui les avait fait sortir du pays d’Égypte. Ils se sont attachés à d’autres dieux, devant lesquels ils se sont prosternés et qu’ils ont servis. Voilà pourquoi il a fait venir sur eux tout ce malheur.” »
      
         
      \bchapter{}
      \begin{verse}
${}^{1}Au terme des vingt années pendant lesquelles Salomon avait bâti la maison du Seigneur et sa propre maison, 
${}^{2}Salomon rebâtit les villes que lui avait données Houram ; il y établit les fils d’Israël. 
${}^{3}Puis Salomon se rendit à Hamath-Soba et il s’en empara. 
${}^{4}Il rebâtit Tadmor dans le désert, ainsi que toutes les villes d’entrepôts qu’il avait bâties dans la région de Hamath. 
${}^{5}Il rebâtit Beth-Horone-le-Haut et Beth-Horone-le-Bas – c’étaient des villes fortifiées avec des remparts, des portes et des verrous – ; 
${}^{6}puis il rebâtit Baalath et toutes les villes d’entrepôts appartenant à Salomon, toutes les villes de garnison pour les chars et celles des cavaliers. Salomon construisit tout ce qu’il désirait, dans Jérusalem, au Liban et dans tout le pays soumis à son autorité.
${}^{7}Il restait toute une population de Hittites, d’Amorites, de Perizzites, de Hivvites et de Jébuséens, qui n’étaient pas d’Israël. 
${}^{8}Ceux d’entre leurs fils qui, après eux, étaient restés dans le pays, et que les fils d’Israël n’avaient pas exterminés, Salomon les réquisitionna pour la corvée, jusqu’à ce jour. 
${}^{9}Mais, d’entre les fils d’Israël, Salomon ne soumit personne au servage pour ses travaux, car ils étaient des hommes de guerre, chefs de ses écuyers, chefs de ses chars et de ses cavaliers. 
${}^{10}Voici le nombre des chefs des préposés aux travaux du roi Salomon : deux cent cinquante qui commandaient le peuple.
${}^{11}Salomon fit monter la fille de Pharaon de la Cité de David à la maison qu’il avait bâtie pour elle. Il disait en effet : « Une femme ne peut demeurer pour moi dans la maison de David, roi d’Israël, car les lieux où l’arche du Seigneur est entrée sont saints. »
${}^{12}En ce temps-là, Salomon offrait des holocaustes au Seigneur sur l’autel du Seigneur qu’il avait bâti devant le Vestibule. 
${}^{13}Il offrait des sacrifices suivant le rituel de chaque jour, selon le commandement de Moïse, pour les sabbats, les nouvelles lunes et les solennités ; de même, trois fois par an, pour la fête des Pains sans levain, la fête des Semaines et la fête des Tentes. 
${}^{14}Puis, selon ce que David son père avait ordonné, il établit les classes des prêtres dans leur service, les Lévites dans leurs fonctions pour louer et officier en présence des prêtres selon le rituel de chaque jour, ainsi que les portiers suivant leurs classes aux diverses portes ; car tel était le commandement de David, l’homme de Dieu. 
${}^{15}On ne s’écarta en rien du commandement donné par le roi au sujet des prêtres et des Lévites, même en ce qui concerne les trésors. 
${}^{16}Tout le travail de Salomon fut bien mené, dès avant le jour de la fondation de la maison du Seigneur jusqu’à son achèvement. Parfaite était la maison du Seigneur !
${}^{17}Alors Salomon se rendit à Écione-Guéber et vers Eilath, sur le rivage de la mer, au pays d’Édom. 
${}^{18}Houram, par l’intermédiaire de ses serviteurs, lui dépêcha des navires ainsi que des serviteurs connaissant bien la mer. Ils arrivèrent avec les serviteurs de Salomon à Ophir et s’y procurèrent quatre cent cinquante lingots d’or, qu’ils rapportèrent au roi Salomon.
      
         
      \bchapter{}
      \begin{verse}
${}^{1}La reine de Saba avait entendu parler de la renommée de Salomon. Pour le mettre à l’épreuve en lui proposant des énigmes, elle vint à Jérusalem avec une escorte imposante : des chameaux chargés d’aromates, d’une énorme quantité d’or et de pierres précieuses. Quand elle fut parvenue auprès de Salomon, elle lui exposa les questions qu’elle avait préparées, 
${}^{2}mais Salomon trouva réponse à tout et ne fut arrêté par aucune difficulté. 
${}^{3}Lorsque la reine de Saba vit la sagesse de Salomon, la maison qu’il avait bâtie, 
${}^{4}les plats servis à sa table, le logement de ses officiers, la tenue du service et l’habillement des serviteurs, ses sommeliers et leur habillement, les holocaustes qu’il offrait à la maison du Seigneur, elle en eut le souffle coupé, 
${}^{5}et elle dit au roi : « Ce que j’ai entendu dire dans mon pays sur toi et sur ta sagesse, c’est donc vrai ! 
${}^{6}Je ne voulais pas croire ce qu’on disait, avant de venir et de voir de mes yeux ; mais vraiment, on ne m’en avait pas appris la moitié ! Tu surpasses en sagesse la renommée qui était venue jusqu’à moi. 
${}^{7}Heureux tes gens, heureux tes serviteurs que voici, eux qui se tiennent continuellement devant toi et qui entendent ta sagesse ! 
${}^{8}Béni soit le Seigneur ton Dieu, qui t’a montré sa bienveillance en te plaçant sur son trône comme roi pour le Seigneur ton Dieu. Parce que ton Dieu aime Israël et veut le faire subsister à jamais, il t’a établi roi sur eux pour exercer le droit et la justice. » 
${}^{9}Elle fit présent au roi de cent vingt lingots d’or, d’une grande quantité d’aromates et de pierres précieuses ; il n’y avait jamais eu d’aromates pareils à ceux que la reine de Saba avait donnés au roi Salomon.
      
         
${}^{10}De même, les serviteurs de Houram et les serviteurs de Salomon avaient apporté de l’or d’Ophir. Ils avaient également apporté du bois de santal et des pierres précieuses. 
${}^{11}Avec ce bois de santal, le roi fit des parquets pour la maison du Seigneur et la maison du roi ; on en fit aussi des cithares et des harpes pour les chantres. On n’avait rien vu de tel auparavant au pays de Juda ! 
${}^{12}Le roi Salomon offrit à la reine de Saba tout ce qui répondait à ses désirs, en plus de la contrepartie de ce qu’elle avait apporté au roi. Puis elle s’en retourna dans son pays avec ses serviteurs.
${}^{13}En une seule année, le poids de l’or qui parvenait à Salomon était de six cent soixante-six lingots d’or, 
${}^{14}sans compter les péages des voyageurs et des commerçants. Tous les rois d’Arabie et les gouverneurs du pays apportaient de l’or et de l’argent à Salomon. 
${}^{15}Le roi Salomon fit deux cents grands boucliers d’or battu – il utilisait six cents pièces d’or battu pour un grand bouclier – 
${}^{16}et trois cents petits boucliers d’or battu – il utilisait trois cents pièces d’or pour un petit bouclier. Le roi les plaça dans la maison de la Forêt du Liban. 
${}^{17}Le roi fit aussi un grand trône d’ivoire, qu’il plaqua d’or pur. 
${}^{18}Ce trône avait six degrés ; il y avait un marchepied en or derrière le trône, et des bras de chaque côté du siège ; deux lions étaient debout près des bras 
${}^{19}et douze lions se tenaient là, sur les six degrés, de chaque côté. Dans aucun royaume, on ne fit chose pareille.
${}^{20}Toutes les coupes du roi Salomon étaient en or, et tous les objets de la maison de la Forêt du Liban, en or fin : on ne faisait aucun cas de l’argent au temps de Salomon. 
${}^{21}Car le roi avait des navires qui allaient à Tarsis avec les serviteurs de Houram et, une fois tous les trois ans, les navires de Tarsis arrivaient, apportant or et argent, ivoires, singes et paons.
${}^{22}Le roi Salomon devint le plus grand de tous les rois de la terre en richesse et en sagesse. 
${}^{23}Tous les rois de la terre cherchaient à rencontrer Salomon face à face, pour entendre la sagesse que Dieu avait mise en son cœur. 
${}^{24}Chacun apportait son offrande : objets d’argent et objets d’or, vêtements, armes et aromates, chevaux et mulets ; et ainsi d’année en année.
${}^{25}Salomon avait quatre mille stalles pour chevaux, des chars et douze mille cavaliers, qu’il cantonna dans les villes de garnison et auprès de lui, à Jérusalem.
${}^{26}Il dominait sur tous les rois, depuis l’Euphrate – le fleuve –, jusqu’au pays des Philistins et jusqu’à la frontière de l’Égypte. 
${}^{27}À Jérusalem, le roi fit abonder l’argent autant que les pierres, et les cèdres autant que les sycomores dans le Bas-Pays. 
${}^{28}Pour Salomon, on faisait venir des chevaux d’Égypte et de tous les pays.
${}^{29}Le reste des actions de Salomon,
        \\des premières aux dernières,
        \\cela n’est-il pas écrit dans les Actes du prophète Nathan,
        \\dans la Prophétie d’Ahias de Silo,
        \\et dans la Vision de Yédo le voyant
        \\au sujet de Jéroboam, fils de Nebath ?
${}^{30}Salomon régna à Jérusalem sur tout Israël pendant quarante ans.
${}^{31}Puis Salomon reposa avec ses pères,
        \\et on l’ensevelit dans la Cité de David son père.
        \\Son fils Roboam régna à sa place.
      
         
      \bchapter{}
      \begin{verse}
${}^{1}Roboam se rendit à Sichem ; c’est à Sichem, en effet, que tout Israël était venu pour le faire roi. 
${}^{2}Jéroboam, fils de Nebath, apprit la nouvelle. Il était alors en Égypte, où il s’était enfui loin du roi Salomon. Il revint d’Égypte. 
${}^{3}On envoya chercher Jéroboam, et il vint, ainsi que tout Israël. Ils s’adressèrent à Roboam en ces termes : 
${}^{4}« Ton père a rendu pénible notre joug ; maintenant, allège la pénible servitude imposée par ton père, et le joug pesant qu’il nous a infligé ; alors nous te servirons. » 
${}^{5}Il leur répondit : « Attendez trois jours, puis revenez vers moi. » Et le peuple se retira.
${}^{6}Le roi Roboam prit conseil des anciens, ceux qui s’étaient tenus en présence de son père Salomon, de son vivant. Il leur dit : « Quelle réponse conseillez-vous de faire à ce peuple ? » 
${}^{7}Ils lui dirent alors : « Si tu te montres bon envers ce peuple, si tu les accueilles, si tu leur adresses des paroles bienveillantes, ils seront tes serviteurs pour toujours. » 
${}^{8}Mais il négligea le conseil que lui donnaient les anciens ; il prit conseil des jeunes gens qui avaient grandi avec lui et se tenaient en sa présence. 
${}^{9}Il leur demanda : « Que conseillez-vous ? Quelle réponse allons-nous faire à ce peuple qui m’a parlé en disant : “Allège donc le joug que nous a infligé ton père” ? » 
${}^{10}Les jeunes gens qui avaient grandi avec lui répondirent : « Voici ce que tu répondras à ce peuple qui t’a parlé en disant : “Ton père a rendu pesant notre joug mais toi, pour nous, allège-le !” Voici ce que tu leur diras : “Mon petit doigt est plus fort que les reins de mon père. 
${}^{11}S’il est vrai que mon père vous accablait sous un joug pesant, je vais, moi, ajouter encore à votre joug. Mon père vous a corrigés avec des lanières ? Eh bien, moi, ce sera avec des fouets à pointes de fer !” »
${}^{12}Le troisième jour, Jéroboam revint, ainsi que tout le peuple, auprès de Roboam, conformément à la parole du roi : « Revenez vers moi le troisième jour. » 
${}^{13}Le roi leur répondit durement, Roboam négligeant ainsi le conseil des anciens. 
${}^{14}Il parla au peuple en suivant le conseil des jeunes gens. Il leur dit : « Mon père a rendu pesant votre joug, je vais, moi, y ajouter encore. Mon père vous a corrigés avec des lanières ? Et bien moi, ce sera avec des fouets à pointes de fer ! » 
${}^{15}Ainsi, le roi n’écouta pas le peuple ; en effet, la tournure que prenaient les choses venait de Dieu, pour que s’accomplisse la parole que le Seigneur avait dite par l’intermédiaire d’Ahias de Silo à Jéroboam, fils de Nebath. 
${}^{16}Tout Israël vit que le roi ne les avait pas écoutés. Le peuple rétorqua au roi :
        \\« Quelle part avons-nous chez David ?
        \\Pas d’héritage chez le fils de Jessé !
        \\Chacun de vous à ses tentes, Israël !
        \\Maintenant, David, pourvois donc à ta maison ! »
      Et tout Israël s’en alla dans ses tentes.
${}^{17}Quant aux fils d’Israël qui demeuraient dans les villes de Juda, Roboam régna sur eux. 
${}^{18}Le roi Roboam envoya Adoram, le chef de la corvée ; mais les fils d’Israël le lapidèrent, et il mourut. Et le roi Roboam se vit contraint de monter sur un char pour s’enfuir à Jérusalem. 
${}^{19}Les dix tribus d’Israël rejetèrent la maison de David, jusqu’à ce jour.
      
         
      \bchapter{}
      \begin{verse}
${}^{1}Roboam arriva à Jérusalem. Il rassembla la maison de Juda et de Benjamin : cent quatre-vingt mille guerriers d’élite, pour combattre Israël, afin de rendre la royauté à Roboam. 
${}^{2}Alors la parole du Seigneur fut adressée à Shemaya, homme de Dieu : 
${}^{3}« Parle à Roboam, fils de Salomon, roi de Juda, ainsi qu’à tout Israël, dans le territoire de Juda et de Benjamin : 
${}^{4}“Ainsi parle le Seigneur : Ne montez pas, ne faites pas la guerre à vos frères. Retournez chacun chez soi, car je suis moi-même à l’origine de cette affaire.” » Alors ils écoutèrent les paroles du Seigneur et ils s’en retournèrent, au lieu de marcher contre Jéroboam.
${}^{5}Roboam résida à Jérusalem et construisit en Juda des villes fortes. 
${}^{6}Il construisit Bethléem, Étam, Teqoa, 
${}^{7}Beth-Sour, Soko, Adoullam, 
${}^{8}ainsi que Gath, Marésha, Zif, 
${}^{9}Adoraïm, Lakish, Azéqa, 
${}^{10}Soréa, Ayyalone et Hébron, qui sont des villes fortes en Juda et en Benjamin. 
${}^{11}Il les fortifia puissamment et y plaça des gouverneurs, ainsi que des dépôts de vivres, d’huile et de vin. 
${}^{12}Dans chacune de ces villes, il y avait des boucliers et des lances. Il rendit ces villes extrêmement fortes, et il eut pour lui Juda et Benjamin.
${}^{13}Les prêtres et les Lévites qui se trouvaient dans tout Israël vinrent de tout leur territoire se présenter devant lui. 
${}^{14}Les Lévites, en effet, abandonnèrent leurs pâturages et leurs propriétés, et vinrent en Juda et à Jérusalem, parce que Jéroboam et ses fils les avaient écartés du sacerdoce du Seigneur. 
${}^{15}Jéroboam institua des prêtres à lui pour les lieux sacrés, pour les boucs et les veaux, idoles qu’il avait fabriquées. 
${}^{16}À la suite des Lévites, les hommes de toutes les tribus d’Israël qui prenaient à cœur de chercher le Seigneur, Dieu d’Israël, vinrent à Jérusalem pour sacrifier au Seigneur, Dieu de leurs pères. 
${}^{17}Pendant trois ans, ils rendirent plus fort le royaume de Juda et ils soutinrent Roboam, fils de Salomon, car pendant trois ans ils marchèrent dans le chemin de David et de Salomon.
${}^{18}Roboam prit pour femme Mahalath, fille de Yerimoth, fils de David, et d’Abihaïl, fille d’Éliab, fils de Jessé. 
${}^{19}Elle lui enfanta des fils : Yéoush, Shemarya et Zaham. 
${}^{20}Après elle, Roboam prit Maaka, fille d’Absalom, qui lui enfanta Abiya, Attaï, Ziza et Shelomith. 
${}^{21}Roboam aima Maaka, fille d’Absalom, plus que toutes ses autres femmes et concubines. Il eut en effet dix-huit femmes et soixante concubines, et il engendra vingt-huit fils et soixante filles. 
${}^{22}Aussi Roboam donna-t-il le premier rang à Abiya, fils de Maaka, comme chef parmi ses frères, afin de le faire roi. 
${}^{23}Il eut l’intelligence de disperser tous ses fils dans toutes les régions de Juda et de Benjamin, et dans toutes les villes fortes ; il leur procura des vivres en abondance et demanda pour eux des femmes en grand nombre.
      
         
      \bchapter{}
      \begin{verse}
${}^{1}Quand Roboam eut affermi sa royauté et fut devenu fort, il abandonna la Loi du Seigneur, et tout Israël le suivit. 
${}^{2}La cinquième année du règne de Roboam, Shishaq, roi d’Égypte, monta contre Jérusalem, parce que le peuple avait été infidèle au Seigneur. 
${}^{3}Avec mille deux cents chars, soixante mille cavaliers, et une foule innombrable venue d’Égypte avec lui : Libyens, Soukkiens, Éthiopiens, 
${}^{4}il s’empara des villes fortes de Juda et atteignit Jérusalem. 
${}^{5}Le prophète Shemaya vint trouver Roboam et les princes de Juda, qui s’étaient regroupés à Jérusalem pour fuir Shishaq. Il leur dit : « Ainsi parle le Seigneur : Vous, vous m’avez abandonné, eh bien, moi aussi, je vous ai abandonnés à la main de Shishaq. » 
${}^{6}Les princes d’Israël et le roi s’inclinèrent et dirent : « Le Seigneur est juste. » 
${}^{7}Le Seigneur vit qu’ils s’inclinaient. La parole du Seigneur fut adressée à Shemaya : « Ils se sont inclinés ; moi, je ne les détruirai pas. Je leur permettrai sous peu d’en réchapper, et ma colère ne se déversera pas sur Jérusalem par la main de Shishaq. 
${}^{8}Mais ils deviendront ses esclaves et ils sauront la différence entre me servir et servir les royaumes des autres pays. »
${}^{9}Shishaq, roi d’Égypte, monta contre Jérusalem. Il s’empara des trésors de la maison du Seigneur et des trésors de la maison du roi. Il s’empara de tout. Il s’empara même des boucliers d’or qu’avait faits Salomon. 
${}^{10}Le roi Roboam fit, pour les remplacer, des boucliers de bronze, et les confia aux chefs des gardes, à la porte de la maison du roi. 
${}^{11}Chaque fois que le roi se rendait à la maison du Seigneur, les gardes venaient prendre ces boucliers, puis ils les rapportaient dans la salle des gardes.
${}^{12}Parce que le roi s’était incliné, la colère du Seigneur se détourna de lui et ne détruisit pas le pays jusqu’à l’extermination : il y avait encore en Juda quelque chose de bon.
${}^{13}Le roi Roboam s’affermit à Jérusalem et y régna. Roboam avait quarante et un ans lorsqu’il devint roi, et il régna dix-sept ans à Jérusalem, la ville que le Seigneur avait choisie parmi toutes les tribus d’Israël pour y mettre son nom. Sa mère s’appelait Naama, l’Ammonite. 
${}^{14}Il fit ce qui est mal, car il n’appliqua pas son cœur à chercher le Seigneur.
${}^{15}Les actions de Roboam, des premières aux dernières,
        \\cela n’est-il pas écrit dans les Actes du prophète Shemaya,
        \\et de Iddo le voyant, selon l’ordre généalogique ?
        \\Il y eut continuellement la guerre entre Roboam et Jéroboam.
${}^{16}Roboam reposa avec ses pères.
        \\Il fut enseveli dans la Cité de David.
        \\Son fils Abiya régna à sa place.
      
         
      \bchapter{}
      \begin{verse}
${}^{1}La dix-huitième année du règne de Jéroboam, Abiya devint roi sur Juda. 
${}^{2}Il régna trois ans à Jérusalem. Sa mère s’appelait Maaka, fille d’Ouriel ; elle était de Guibéa. Il y eut la guerre entre Abiya et Jéroboam. 
${}^{3}Abiya engagea le combat avec une armée de vaillants soldats, quatre cent mille hommes d’élite. Et Jéroboam se rangea en bataille en face de lui, avec huit cent mille hommes d’élite, de vaillants guerriers.
${}^{4}Abiya se dressa au sommet du mont Semaraïm, qui est dans la montagne d’Éphraïm. Il cria : « Écoutez-moi, Jéroboam et tout Israël ! 
${}^{5}Ne savez-vous pas que le Seigneur, Dieu d’Israël, a donné à David la royauté sur Israël, pour toujours, à lui et à ses fils, par une alliance indestructible ? 
${}^{6}Mais Jéroboam, fils de Nebath, serviteur de Salomon, fils de David, s’est dressé et s’est révolté contre son maître. 
${}^{7}Autour de lui se sont groupés des gens de rien, des vauriens, et ils ont vaincu Roboam, fils de Salomon. Roboam, jeune et de cœur timoré, n’a pu leur tenir tête. 
${}^{8}Et maintenant, vous parlez de tenir tête à la royauté du Seigneur qui est entre les mains des fils de David ! Vous êtes une foule immense, et vous avez apporté les veaux d’or que Jéroboam vous a faits pour qu’ils soient vos dieux. 
${}^{9}N’avez-vous pas expulsé les prêtres du Seigneur, les fils d’Aaron et les Lévites, pour vous faire des prêtres comme les peuples des autres pays ? Quiconque vient recevoir l’investiture en apportant un jeune taureau et sept béliers devient prêtre de ce qui n’est pas Dieu ! 
${}^{10}Mais nous, notre Dieu, c’est le Seigneur, et nous ne l’avons pas abandonné ! Les prêtres qui servent le Seigneur sont fils d’Aaron, et il y a des Lévites en fonction. 
${}^{11}Ils font brûler pour le Seigneur des holocaustes matin après matin et soir après soir, avec de l’encens aromatique ; ils disposent la rangée de pains sur la table pure, ils allument le chandelier d’or et ses lampes soir après soir. Car nous gardons l’observance du Seigneur notre Dieu, mais vous, vous avez abandonné le Seigneur ! 
${}^{12}Voici que Dieu est avec nous, en tête, ainsi que ses prêtres et les trompettes de l’acclamation pour les faire sonner contre vous. Fils d’Israël, ne combattez pas le Seigneur, Dieu de vos pères, car vous ne gagnerez pas ! »
${}^{13}Jéroboam les fit contourner, sur leurs arrières, par une troupe placée en embuscade. Ainsi son armée se trouva devant Juda, et l’embuscade derrière. 
${}^{14}Ceux de Juda firent volte-face, et voici qu’il leur fallait combattre devant et derrière. Alors ils crièrent vers le Seigneur, et les prêtres sonnèrent de la trompette. 
${}^{15}Les hommes de Juda poussèrent le cri de guerre, et, au cri de guerre des hommes de Juda, Dieu battit Jéroboam et tout Israël, en face d’Abiya et des gens de Juda. 
${}^{16}Les fils d’Israël s’enfuirent devant Juda, et Dieu les livra entre leurs mains. 
${}^{17}Abiya et son peuple leur infligèrent une grande défaite : du côté d’Israël tombèrent, frappés à mort, cinq cent mille hommes d’élite. 
${}^{18}En ce temps-là, les fils d’Israël furent abaissés, et, au contraire, les fils de Juda rendus plus forts, parce qu’ils s’étaient appuyés sur le Seigneur, Dieu de leurs pères.
${}^{19}Abiya poursuivit Jéroboam et lui prit des villes : Béthel et ses dépendances, Yeshana et ses dépendances, Éphrone et ses dépendances. 
${}^{20}Jéroboam ne retrouva plus sa puissance durant le règne d’Abiya. Le Seigneur le frappa, et il mourut. 
${}^{21}Quant à Abiya, il s’affermit. Il prit quatorze femmes et engendra vingt-deux fils et seize filles.
${}^{22}Le reste des actions d’Abiya,
        \\sa conduite et ses actions,
        \\sont écrites dans le Commentaire du prophète Iddo.
${}^{23}Abiya reposa avec ses pères,
        \\et on l’ensevelit dans la Cité de David.
        \\Son fils Asa régna à sa place.
        \\De son temps, le pays fut tranquille pendant dix ans.
      
         
      \bchapter{}
      \begin{verse}
${}^{1}Asa fit ce qui est bon et droit aux yeux du Seigneur son Dieu. 
${}^{2}Il supprima les autels d’origine étrangère et les lieux sacrés ; il brisa les stèles et abattit les poteaux sacrés. 
${}^{3}Il dit à Juda de chercher le Seigneur, Dieu de leurs pères, et de pratiquer la Loi et le Commandement. 
${}^{4}Il supprima de toutes les villes de Juda les lieux sacrés et les colonnes à encens. Et le royaume fut tranquille sous son règne. 
${}^{5}Il construisit en Juda des villes fortes, car le pays était tranquille. Et il n’y eut pas de guerre contre lui en ces années-là, car le Seigneur lui avait procuré le repos.
${}^{6}Asa dit aux gens de Juda : « Construisons ces villes, entourons-les d’un rempart, avec des tours, des portes et des verrous. Tout le pays est encore devant nous. Comme nous avons cherché le Seigneur notre Dieu, il nous a cherchés et nous a donné le repos de tous côtés. »
      Ils construisirent donc et le firent avec succès. 
${}^{7}Asa avait comme armée trois cent mille hommes de Juda portant le grand bouclier et la lance, et deux cent quatre-vingt mille hommes de Benjamin portant le petit bouclier et tirant à l’arc ; tous étaient de vaillants guerriers.
${}^{8}Zèrah l’Éthiopien marcha contre eux avec une armée de mille milliers d’hommes et trois cents chars, et il parvint jusqu’à Marésha. 
${}^{9}Asa marcha au-devant de lui, et ils se rangèrent en bataille dans la vallée de Sefata, près de Marésha. 
${}^{10}Asa invoqua le Seigneur son Dieu, en disant : « Seigneur, il n’y a pas de différence pour toi entre aider le puissant et aider celui qui est sans force. Alors, aide-nous, Seigneur notre Dieu, car nous nous appuyons sur toi, et nous sommes venus, en ton nom, contre cette multitude. Tu es le Seigneur notre Dieu : qu’un mortel ne l’emporte pas sur toi ! »
${}^{11}Le Seigneur battit les Éthiopiens devant Asa et devant les gens de Juda, et les Éthiopiens s’enfuirent. 
${}^{12}Asa et le peuple qui était avec lui les poursuivirent jusqu’à Guérar. Il tomba tant d’Éthiopiens qu’il n’y eut aucun survivant, parce qu’ils avaient été taillés en pièces devant le Seigneur et devant son armée. Asa et le peuple emportèrent un très grand butin. 
${}^{13}Ils frappèrent toutes les villes aux alentours de Guérar, car la terreur du Seigneur s’était emparée d’elles. Ils pillèrent toutes les villes, car il s’y trouvait un grand butin. 
${}^{14}Ils s’en prirent même aux enclos des troupeaux, ils emmenèrent du petit bétail en quantité et des chameaux. Puis ils revinrent à Jérusalem.
      
         
      \bchapter{}
      \begin{verse}
${}^{1}L’Esprit de Dieu fut sur Azarias, fils d’Oded, 
${}^{2}qui sortit au-devant d’Asa et lui dit : « Écoutez-moi, Asa, et tout Juda et Benjamin ! Le Seigneur est avec vous quand vous êtes avec lui. Si vous le cherchez, il se laisse trouver par vous ; mais si vous l’abandonnez, il vous abandonne. 
${}^{3}Pendant longtemps, Israël a été sans vrai Dieu, sans prêtre pour enseigner et sans Loi. 
${}^{4}Mais dans sa détresse il est revenu vers le Seigneur, Dieu d’Israël ; ils l’ont cherché, et le Seigneur s’est laissé trouver par eux. 
${}^{5}En ce temps-là, il n’y avait pas de sécurité pour ceux qui allaient et venaient, car de nombreux désordres pesaient sur tous les habitants du pays. 
${}^{6}On se heurtait, nation contre nation, ville contre ville ; car Dieu les frappait de panique par toutes sortes de détresses. 
${}^{7}Mais vous, soyez forts, et que vos mains ne faiblissent pas, car vos actions auront leur récompense. »
${}^{8}Lorsqu’il entendit ces paroles et la prophétie d’Azarias, fils d’Oded, Asa prit courage. Il fit disparaître les horreurs de tout le pays de Juda et de Benjamin, ainsi que des villes dont il s’était emparé dans la montagne d’Éphraïm. Et il remit en état l’autel du Seigneur qui se trouvait devant le Vestibule du Seigneur. 
${}^{9}Asa rassembla tout Juda et Benjamin, et ceux d’Éphraïm, de Manassé et de Siméon qui résidaient parmi eux. Car des gens d’Israël étaient passés en grand nombre de son côté, en voyant que le Seigneur son Dieu était avec lui. 
${}^{10}Ils se rassemblèrent à Jérusalem, le troisième mois de la quinzième année du règne d’Asa. 
${}^{11}Ce jour-là, ils offrirent au Seigneur, sur le butin qu’ils avaient ramené, sept cents bœufs et sept mille moutons. 
${}^{12}Ils s’engagèrent par une alliance à chercher le Seigneur, Dieu de leurs pères, de tout leur cœur et de toute leur âme. 
${}^{13}Et quiconque ne chercherait pas le Seigneur, Dieu d’Israël, serait mis à mort, le petit comme le grand, homme ou femme. 
${}^{14}Ils prêtèrent serment au Seigneur, d’une voix forte, avec des ovations, des trompettes et des cors. 
${}^{15}Tous ceux de Juda se réjouirent de ce serment, car ils l’avaient prêté de tout leur cœur, et c’est de leur plein gré qu’ils avaient cherché le Seigneur qui se laissa trouver par eux. Et le Seigneur leur procura le repos de tous côtés.
${}^{16}Asa destitua Maaka, grand-mère du roi, du titre de reine mère, parce qu’elle avait fabriqué une Infamie pour Ashéra. Asa abattit cette Infamie, la réduisit en poussière et la brûla dans le ravin du Cédron. 
${}^{17}Mais les lieux sacrés ne disparurent pas d’Israël, bien que le cœur d’Asa fût intègre tous les jours de sa vie. 
${}^{18}Il fit apporter à la maison de Dieu les objets sacrés de son père et ceux qu’il avait lui-même consacrés : l’argent, l’or et les ustensiles. 
${}^{19}Il n’y eut plus de guerre jusqu’à la trente-cinquième année du règne d’Asa.
      
         
      \bchapter{}
      \begin{verse}
${}^{1}La trente-sixième année du règne d’Asa, Baasa, roi d’Israël, monta contre Juda. Il fortifia la ville de Rama pour empêcher quiconque de communiquer avec Asa, roi de Juda. 
${}^{2}Asa prit de l’argent et de l’or dans les trésors de la maison du Seigneur et de la maison du roi pour en envoyer à Ben-Hadad, roi d’Aram, qui résidait à Damas. Il lui fit dire : 
${}^{3}« Il y a une alliance entre moi et toi, comme entre mon père et ton père ! Voici que je t’envoie de l’argent et de l’or. Va, romps ton alliance avec Baasa, roi d’Israël : qu’il s’éloigne de moi ! » 
${}^{4}Ben-Hadad écouta le roi Asa. Il envoya les chefs de ses armées contre les villes d’Israël. Il frappa les villes de Yone, Dane, Abel-Maïm et tous les entrepôts des villes de Nephtali. 
${}^{5}Lorsque Baasa apprit cela, il cessa aussitôt de fortifier Rama et arrêta ses travaux. 
${}^{6}Alors Asa, le roi, prit avec lui tous les gens de Juda. Ils enlevèrent de Rama les pierres et le bois dont Baasa s’était servi pour la construire, et Asa les réemploya pour fortifier Guéba et Mispa.
${}^{7}En ce temps-là, Hanani le Voyant vint trouver Asa, roi de Juda, et lui dit : « Parce que tu t’es appuyé sur le roi d’Aram et que tu ne t’es pas appuyé sur le Seigneur ton Dieu, à cause de cela l’armée du roi d’Aram s’est échappée de tes mains. 
${}^{8}Les Éthiopiens et les Libyens ne formaient-ils pas une armée nombreuse, avec des chars et des chevaux en grande quantité ? Et cependant, parce que tu t’es appuyé sur le Seigneur, il les a livrés entre tes mains. 
${}^{9}Car les yeux du Seigneur scrutent toute la terre, pour affermir ceux dont le cœur est tout entier à lui. Tu as agi en insensé dans cette affaire : désormais, il y aura des guerres contre toi. » 
${}^{10}Asa se mit en colère contre le voyant et le fit mettre en prison, aux fers, car il était furieux contre lui à cause de ces paroles. Dans le même temps, Asa maltraita une partie du peuple.
${}^{11}Les actions d’Asa, des premières aux dernières,
        \\voici qu’elles sont écrites dans le Livre des rois de Juda et d’Israël.
${}^{12}La trente-troisième année de son règne, Asa eut les pieds malades, d’une très grave maladie. Pourtant, même pendant sa maladie, il n’eut pas recours au Seigneur, mais aux médecins.
${}^{13}Asa reposa avec ses pères :
        \\il mourut la quarante et unième année de son règne.
${}^{14}On l’ensevelit dans le tombeau
        \\qu’il s’était fait creuser dans la Cité de David.
        \\On l’étendit sur un lit couvert d’aromates et de toutes sortes de baumes préparés selon l’art des embaumeurs. Et on alluma pour lui un brasier grandiose.
      
         
      \bchapter{}
      \begin{verse}
${}^{1}Son fils Josaphat régna à sa place. Il s’affermit contre Israël. 
${}^{2}Il installa des troupes dans toutes les villes fortifiées de Juda, il plaça des préfets dans le pays de Juda et dans les villes d’Éphraïm dont s’était emparé Asa, son père.
${}^{3}Le Seigneur fut avec Josaphat car celui-ci marcha dans les chemins où David, son père, avait marché en ses débuts. Il ne chercha pas les Baals, 
${}^{4}mais c’est le Dieu de son père qu’il chercha ; il marcha selon ses commandements, sans agir comme Israël. 
${}^{5}Le Seigneur confirma la royauté entre ses mains. Tout Juda offrait des présents à Josaphat, qui eut en abondance richesse et gloire. 
${}^{6}Son cœur s’exalta dans les chemins du Seigneur, et il supprima de nouveau en Juda les lieux sacrés et les poteaux sacrés.
${}^{7}La troisième année de son règne, il envoya ses officiers Ben-Haïl, Abdias, Zacharie, Nathanaël et Michée, pour donner un enseignement dans les villes de Juda. 
${}^{8}Des Lévites les accompagnaient : Shemayahou, Netanyahou, Zebadyahou, Asaël, Shemiramoth, Jonathan, Adonias, Tobie, Tob-Adonias, et avec eux les prêtres Élishama et Joram. 
${}^{9}Ils enseignaient dans Juda, emportant avec eux le livre de la Loi du Seigneur ; ils firent le tour de toutes les villes de Juda, donnant un enseignement parmi le peuple. 
${}^{10}La terreur du Seigneur s’empara de tous les royaumes des pays qui entouraient Juda, et ils ne firent pas la guerre à Josaphat. 
${}^{11}De chez les Philistins, on apporta à Josaphat des présents et de l’argent comme tribut ; même les Arabes lui amenèrent du petit bétail : sept mille sept cents béliers et sept mille sept cents boucs. 
${}^{12}Josaphat devenait de plus en plus important. Il construisit en Juda des citadelles et des villes d’entrepôts.
${}^{13}Il avait une nombreuse main-d’œuvre dans les villes de Juda, et des hommes de guerre, vaillants guerriers, à Jérusalem. 
${}^{14}En voici le dénombrement, selon la maison de leurs pères : pour Juda, comme officiers de millier, il y avait l’officier Adna, et avec lui trois cent mille vaillants guerriers ; 
${}^{15}à ses côtés, l’officier Yohanane, et avec lui deux cent quatre-vingt mille hommes ; 
${}^{16}à ses côtés, Amasya, fils de Zikri, engagé volontaire pour le Seigneur, et avec lui deux cent mille vaillants guerriers.
${}^{17}De Benjamin, il y avait Élyada, vaillant guerrier, et avec lui deux cent mille hommes armés de l’arc et du bouclier ; 
${}^{18}à ses côtés, Yehozabad, et avec lui cent quatre-vingt mille hommes équipés pour la guerre.
${}^{19}Tels étaient ceux qui servaient le roi, sans compter ceux que le roi avait placés dans les villes fortes, sur tout le territoire de Juda.
      
         
      \bchapter{}
      \begin{verse}
${}^{1}Josaphat possédait en abondance richesses et gloire, et il s’allia par un mariage avec Acab. 
${}^{2}Après quelques années, il descendit chez Acab, à Samarie. Acab sacrifia pour lui et le peuple qui l’accompagnait quantité de moutons et de bœufs, et il l’incita à monter contre Ramoth-de-Galaad. 
${}^{3}Acab, roi d’Israël, dit à Josaphat, roi de Juda : « Viendrais-tu avec moi à Ramoth-de-Galaad ? » Il lui répondit : « Il en est de moi comme de toi, de mon peuple comme de ton peuple ; nous serons avec toi au combat. »
      
         
${}^{4}Josaphat dit au roi d’Israël : « Mais consulte d’abord la parole du Seigneur. » 
${}^{5}Le roi d’Israël réunit les prophètes au nombre de quatre cents. Il leur demanda : « Irons-nous à Ramoth-de-Galaad pour combattre, ou dois-je y renoncer ? » Ils dirent : « Monte ! Dieu livrera la ville aux mains du roi. » 
${}^{6}Mais Josaphat reprit : « N’y a-t-il ici aucun autre prophète du Seigneur, par qui nous pourrions le consulter ? » 
${}^{7}Le roi d’Israël répondit à Josaphat : « Il y a encore un homme par qui nous pourrions consulter le Seigneur, mais moi, je le hais, car il ne prophétise aucun bonheur à mon sujet, mais toujours du malheur. Il s’agit de Michée, fils de Yimla. » Josaphat répliqua : « Que le roi ne parle pas ainsi ! » 
${}^{8}Le roi d’Israël appela un dignitaire et lui dit : « Vite, fais venir Michée, fils de Yimla ! »
${}^{9}Le roi d’Israël et Josaphat, roi de Juda, siégeaient, en tenue d’apparat, chacun sur son trône. Ils siégeaient sur l’esplanade à l’entrée de la porte de Samarie. Et, devant eux, tous les prophètes se mettaient à prophétiser. 
${}^{10}Sédécias, fils de Kenahana, s’était fabriqué des cornes de fer. Il disait : « Ainsi parle le Seigneur : Avec cela tu pourfendras Aram jusqu’à l’exterminer. » 
${}^{11}Tous les prophètes prophétisaient de la même manière ; ils disaient : « Monte à Ramoth-de-Galaad ! Tu réussiras ! Le Seigneur livrera la ville aux mains du roi. »
${}^{12}Le messager qui était allé appeler Michée lui dit : « Voici les paroles des prophètes ; d’une seule voix ils annoncent du bien pour le roi. Que ta parole soit donc conforme à celle de chacun : annonce du bien ! » 
${}^{13}Michée répondit : « Par le Seigneur qui est vivant ! Ce que mon Dieu dira, c’est cela que j’annoncerai ! » 
${}^{14}Il entra chez le roi qui lui dit : « Michée, irons-nous à Ramoth-de-Galaad pour combattre, ou dois-je y renoncer ? » Il répondit : « Montez ! Vous réussirez. Ses habitants seront livrés entre vos mains. » 
${}^{15}Le roi lui rétorqua : « Combien de fois devrai-je t’adjurer de me dire seulement la vérité au nom du Seigneur ? » 
${}^{16}Michée dit alors :
        \\« J’ai vu tout Israël dispersé sur les montagnes
        \\comme des brebis sans berger.
        \\Le Seigneur a dit :
        \\“Ces gens n’ont plus de maître ;
        \\qu’ils retournent en paix, chacun dans sa maison.” »
${}^{17}Le roi d’Israël dit à Josaphat : « Ne te l’avais-je pas dit ? Il ne prophétise à mon sujet rien de bon, mais seulement du mal ! » 
${}^{18}Michée reprit : « Eh bien ! Écoutez la parole du Seigneur ! J’ai vu le Seigneur qui siégeait sur son trône ; toute l’armée des cieux se tenait à sa droite et à sa gauche. 
${}^{19}Le Seigneur demanda : “Qui séduira Acab, roi d’Israël, pour qu’il monte et qu’il tombe à Ramoth-de-Galaad ?” Ils répondirent, l’un disant une chose, l’autre disant autre chose. 
${}^{20}Alors un esprit s’avança et se tint en présence du Seigneur. Il dit : “Moi, je le séduirai.” Le Seigneur reprit : “De quelle manière ?” 
${}^{21}Il répondit : “J’avancerai, je deviendrai esprit de mensonge dans la bouche de tous ses prophètes.” Le Seigneur déclara : “Tu le séduiras, tu l’auras même en ton pouvoir. Avance, et fais comme tu as dit.” »
${}^{22}Michée continua : « Maintenant donc, voici que le Seigneur a mis un esprit de mensonge dans la bouche des prophètes qui sont là, voici que le Seigneur annonce contre toi le malheur. »
${}^{23}Sédécias, fils de Kenahana, s’approcha et frappa Michée sur la joue, en disant : « Par quel chemin l’esprit du Seigneur s’est-il échappé de moi pour te parler ? » 
${}^{24}Michée répondit : « Eh bien ! Le jour où tu fuiras dans une chambre retirée pour te cacher, tu le verras. » 
${}^{25}Le roi d’Israël donna cet ordre : « Saisissez-vous de Michée et remettez-le aux mains d’Amone, gouverneur de la ville, et à Joas, le fils du roi. 
${}^{26}Vous direz : “Ainsi parle le roi : Mettez cet homme en prison, nourrissez-le de rations réduites de pain et d’eau jusqu’à ce que je revienne sain et sauf”. » 
${}^{27}Michée reprit : « Si vraiment tu reviens sain et sauf, c’est que le Seigneur n’a pas parlé par ma bouche. » Il ajouta : « Vous, tous les peuples, écoutez ! »
${}^{28}Le roi d’Israël et Josaphat, roi de Juda, montèrent à Ramoth-de-Galaad. 
${}^{29}Le roi d’Israël dit à Josaphat : « Je vais me déguiser pour marcher au combat, mais toi, revêts ta tenue. » Le roi d’Israël se déguisa pour marcher au combat. 
${}^{30}Le roi d’Aram avait donné cet ordre à ses commandants de chars : « Vous n’attaquerez ni petit ni grand, mais uniquement le roi d’Israël ! » 
${}^{31}Lorsque les commandants de chars virent Josaphat, ils dirent : « C’est le roi d’Israël. » Et ils l’encerclèrent pour l’attaquer. Mais Josaphat poussa un cri. Le Seigneur le secourut, et Dieu les entraîna loin de lui. 
${}^{32}Alors les commandants de chars virent que ce n’était pas le roi d’Israël et ils se détournèrent de lui. 
${}^{33}Un homme tira de l’arc au hasard et atteignit le roi d’Israël entre les attaches et la cuirasse. Le roi dit au conducteur du char : « Tourne bride et fais-moi sortir du champ de bataille, car je me sens mal ! » 
${}^{34}Le combat, ce jour-là, devint très violent. Le roi d’Israël se maintint debout sur son char, face aux Araméens, jusqu’au soir. Et il mourut au coucher du soleil.
       
      
         
      \bchapter{}
      \begin{verse}
${}^{1}Josaphat, roi de Juda, revint en paix dans sa maison, à Jérusalem. 
${}^{2}Jéhu, un voyant, fils de Hanani, sortit à sa rencontre. Il dit au roi Josaphat : « Fallait-il porter secours au méchant ? Est-ce que tu aimes ceux qui haïssent le Seigneur ? C’est pourquoi la colère du Seigneur est sur toi. 
${}^{3}Toutefois, il y a aussi en toi quelque chose de bon, puisque tu as fait disparaître du pays les poteaux sacrés ; et que tu as appliqué ton cœur à chercher Dieu. »
      
         
${}^{4}Josaphat habitait à Jérusalem. De nouveau, il visita le peuple depuis Bershéba jusqu’à la montagne d’Éphraïm et il les fit revenir vers le Seigneur, le Dieu de leurs pères. 
${}^{5}Il établit des juges dans le pays, dans toutes les villes fortifiées de Juda, un pour chaque ville. 
${}^{6}Il dit aux juges : « Soyez attentifs à ce que vous ferez, car ce n’est pas selon les hommes que vous jugez, mais selon le Seigneur ; il est avec vous quand vous prononcez un jugement. 
${}^{7}Et maintenant, que pèse sur vous la crainte du Seigneur ! Prenez garde à ce que vous faites, car dans le Seigneur notre Dieu il n’y a ni perfidie, ni partialité, ni vénalité. »
${}^{8}À Jérusalem également Josaphat établit des Lévites, des prêtres et des chefs de famille d’Israël, pour juger selon le Seigneur et régler les litiges ; ils habitaient à Jérusalem. 
${}^{9}Il leur donna des ordres, en disant : « Voici comment vous agirez, avec la crainte du Seigneur, dans la fidélité et d’un cœur sans partage. 
${}^{10}Chaque fois que vos frères, établis dans leurs villes, porteront devant vous un litige concernant une affaire de sang ou concernant la Loi – commandement, décrets ou ordonnances –, vous les avertirez, pour qu’ils ne se rendent pas coupables envers le Seigneur, et que sa colère ne soit pas sur vous ni sur vos frères. En agissant ainsi, vous ne vous rendrez pas coupables.
${}^{11}Et voici qu’Amarias, le chef des prêtres, veillera sur vous pour tout ce qui concerne le Seigneur, et Zebadias, fils de Yishmaël, chef de la maison de Juda, pour tout ce qui concerne le roi ; les Lévites vous serviront de scribes. Soyez forts et agissez ! Et que le Seigneur soit avec celui qui fait le bien ! »
      
         
      \bchapter{}
      \begin{verse}
${}^{1}À quelque temps de là, les fils de Moab et les fils d’Ammone, et avec eux des Méounites, vinrent faire la guerre à Josaphat. 
${}^{2}On vint en informer Josaphat : « Une grande multitude s’avance contre toi, venant d’au-delà de la Mer, du pays d’Édom ; elle est déjà à Haceçone-Tamar, c’est-à-dire Enn-Guèdi. »
${}^{3}Josaphat prit peur et décida de consulter le Seigneur ; puis il proclama un jeûne pour tout Juda. 
${}^{4}Les gens de Juda se rassemblèrent pour chercher secours auprès du Seigneur ; c’est de toutes les villes de Juda que l’on vint chercher le Seigneur. 
${}^{5}Alors Josaphat se tint debout au milieu de l’assemblée de Juda, à Jérusalem, dans la maison du Seigneur, devant le nouveau parvis. 
${}^{6}Il dit : « Seigneur, Dieu de nos pères, n’es-tu pas le Dieu qui est au ciel ? N’est-ce pas toi qui domines sur tous les royaumes des nations ? Dans ta main, force et puissance ; nul ne peut tenir devant toi. 
${}^{7}N’est-ce pas toi, ô notre Dieu, qui, face à ton peuple Israël, as dépossédé les habitants de ce pays, et l’as donné pour toujours à la descendance d’Abraham ton ami ? 
${}^{8}Là ils se sont établis et ils ont construit un sanctuaire pour ton nom. Ils disaient : 
${}^{9}“Si un malheur tombe sur nous, épée, châtiment, peste ou famine, nous nous tiendrons devant cette Maison et devant toi, puisque ton nom est dans cette Maison. Nous crierons vers toi du fond de notre détresse ; toi, tu entendras et tu sauveras.”
${}^{10}Maintenant, voici les fils d’Ammone, de Moab et de la montagne de Séïr, chez qui tu n’as pas laissé entrer les fils d’Israël lorsqu’ils venaient du pays d’Égypte : ils se sont écartés d’eux sans les anéantir ! 
${}^{11}Les voici qui nous récompensent en venant nous chasser de ton héritage, dont tu nous as fait hériter. 
${}^{12}Ô notre Dieu, ne rendras-tu pas contre eux un jugement ? Car nous sommes sans force en face de cette foule immense qui marche contre nous. Nous ne savons que faire, mais nos yeux se tournent vers toi. »
${}^{13}Tous les gens de Juda se tenaient debout devant le Seigneur, et même leurs petits enfants, leurs femmes et leurs fils. 
${}^{14}Il y avait là Yahaziel, fils de Zacharie, fils de Benaya, fils de Yehiel, fils de Mattanya ; il était lévite, du groupe des fils d’Asaf. L’Esprit du Seigneur fut sur lui, au milieu de l’assemblée. 
${}^{15}Yahaziel s’écria : « Soyez attentifs, vous tous de Juda et habitants de Jérusalem, et toi, roi Josaphat ! Ainsi vous parle le Seigneur : Ne craignez pas, ne vous effrayez pas devant cette foule immense ; car ce combat n’est pas le vôtre, mais celui de Dieu. 
${}^{16}Demain, descendez vers eux ; voici qu’ils arrivent par la montée de Ciç ; vous les trouverez à l’extrémité du ravin près du désert de Yerouël. 
${}^{17}Mais là, vous n’aurez pas à combattre ; restez sur place et prenez position ; vous verrez comment le Seigneur va vous sauver. Juda et Jérusalem, ne craignez pas, ne vous effrayez pas : demain, sortez à leur rencontre, le Seigneur sera avec vous. »
${}^{18}Josaphat se mit à genoux, face contre terre. Tous les gens de Juda et les habitants de Jérusalem tombèrent devant la face du Seigneur et se prosternèrent devant lui. 
${}^{19}Les Lévites appartenant aux fils des Qehatites et aux fils des Coréites se levèrent pour louer à pleine voix le Seigneur, Dieu d’Israël.
${}^{20}De grand matin, ils se levèrent et partirent pour le désert de Teqoa. À leur départ, Josaphat, debout, s’écria : « Écoutez-moi, gens de Juda et habitants de Jérusalem. Ayez confiance dans le Seigneur votre Dieu, et vous tiendrez ; ayez confiance en ses prophètes, et vous réussirez. » 
${}^{21}Après avoir pris conseil du peuple, il mit en place des hommes qui chantaient le Seigneur et louaient la splendeur de sa sainteté. Précédant la troupe, ils disaient : « Rendez grâce au Seigneur, éternel est son amour ! » 
${}^{22}Au moment où ils entonnaient l’acclamation et la louange, le Seigneur tendit une embuscade aux fils d’Ammone, de Moab et de la montagne de Séïr, qui marchaient contre Juda. Ceux-ci furent battus. 
${}^{23}Les fils d’Ammone et de Moab se dressèrent contre les habitants de la montagne de Séïr pour les vouer à l’anathème et les anéantir. Puis, lorsqu’ils en eurent fini avec les habitants de Séïr, ils s’entraînèrent les uns les autres à se détruire.
${}^{24}Les gens de Juda parvinrent au promontoire d’où l’on a vue sur le désert, et ils se tournèrent vers la foule : elle n’était plus que cadavres gisant à terre, et pas un survivant ! 
${}^{25}Josaphat vint avec son peuple piller leur butin ; ils trouvèrent du bétail en quantité, des biens, des vêtements, et des objets précieux. Ils en ramassèrent plus qu’ils ne pouvaient porter. Il leur fallut trois jours pour piller le butin, car il était considérable. 
${}^{26}Le quatrième jour, ils se rassemblèrent dans la vallée de la Bénédiction. Là, en effet, ils bénirent le Seigneur. Voilà pourquoi on a donné à ce lieu le nom de « vallée de la Bénédiction », jusqu’à ce jour. 
${}^{27}Tous les hommes de Juda et de Jérusalem, Josaphat à leur tête, se mirent en marche pour retourner à Jérusalem dans la joie, car le Seigneur les avait réjouis au détriment de leurs ennemis. 
${}^{28}Ils entrèrent à Jérusalem au son des harpes, des cithares et des trompettes, jusqu’à la maison du Seigneur. 
${}^{29}La terreur de Dieu s’empara de tous les royaumes des pays environnants, lorsqu’ils apprirent que le Seigneur avait combattu contre les ennemis d’Israël. 
${}^{30}Alors le règne de Josaphat fut calme. Son Dieu lui donna le repos de tous côtés.
${}^{31}Josaphat régna sur Juda. Il avait trente-cinq ans lorsqu’il devint roi, et il régna vingt-cinq ans à Jérusalem. Le nom de sa mère était Azouba ; elle était fille de Shilki.
${}^{32}Il marcha dans le chemin de son père Asa ; il ne s’en détourna pas, faisant ce qui est droit aux yeux du Seigneur. 
${}^{33}Toutefois les lieux sacrés ne disparurent pas, et le peuple n’attachait toujours pas son cœur au Dieu de ses pères.
${}^{34}Le reste des actions de Josaphat,
        \\des premières aux dernières,
        \\cela est écrit dans les Annales de Jéhu, fils de Hanani,
        \\qui ont été reportées dans le Livre des rois d’Israël.
${}^{35}Après cela, Josaphat, roi de Juda, s’associa avec le roi d’Israël, Ocozias, dont la conduite était mauvaise. 
${}^{36}Il s’associa avec lui afin de construire des navires pour aller à Tarsis. Ils construisirent les navires à Écione-Guéber. 
${}^{37}Mais Élièzer, fils de Dodavahou, de Marésha, prophétisa contre Josaphat en disant : « Parce que tu t’es associé à Ocozias, le Seigneur a fait une brèche dans tes œuvres. » Les navires se brisèrent et ne purent aller à Tarsis.
      
         
      \bchapter{}
${}^{1}Josaphat reposa avec ses pères.
        \\Il fut enseveli avec eux dans la Cité de David.
        \\Son fils Joram régna à sa place.
        
           
${}^{2}Celui-ci avait des frères, fils de Josaphat : Azaria, Yehiel, Zacharie, Azarias, Mikaël et Shefatias. Tous ceux-là étaient fils de Josaphat, roi d’Israël. 
${}^{3}Leur père leur avait donné de nombreux cadeaux en argent, en or et en objets précieux, ainsi que des villes fortes en Juda. Mais la royauté, il l’avait donnée à Joram, parce qu’il était l’aîné. 
${}^{4}Joram s’établit à la tête du royaume de son père, et s’y affermit ; alors il tua par l’épée tous ses frères et quelques princes d’Israël.
${}^{5}Joram avait trente-deux ans lorsqu’il devint roi, et il régna huit ans à Jérusalem. 
${}^{6}Il marcha dans le chemin des rois d’Israël, comme avait fait la maison d’Acab, car il avait pour femme une fille d’Acab. Et il fit ce qui est mal aux yeux du Seigneur. 
${}^{7}Mais le Seigneur ne voulut pas détruire la maison de David, à cause de l’alliance qu’il avait conclue avec David et parce qu’il avait promis de lui donner, à lui et à ses fils, une lampe pour toujours.
${}^{8}Du temps de Joram, le pays d’Édom se révolta contre la domination de Juda et se donna un roi. 
${}^{9}Joram passa la frontière avec ses officiers et tous ses chars. S’étant levé de nuit, il battit les Édomites qui l’encerclaient, lui et les commandants de chars. 
${}^{10}Édom fut en révolte contre la domination de Juda jusqu’à ce jour. Alors, en ce temps-là, la ville de Libna se révolta contre sa domination.
      Joram, en effet, avait abandonné le Seigneur, Dieu de ses pères. 
${}^{11}Il fit même des lieux sacrés dans les montagnes de Juda. Il poussa les habitants de Jérusalem à se prostituer, et dévoya les gens de Juda. 
${}^{12}Une lettre du prophète Élie lui parvint, qui disait : « Ainsi parle le Seigneur, le Dieu de ton père David : Tu n’as pas marché dans les chemins de Josaphat, ton père, ni dans les chemins d’Asa, roi de Juda, 
${}^{13}mais tu as marché dans le chemin des rois d’Israël, tu as poussé à la prostitution les gens de Juda et les habitants de Jérusalem comme l’avait fait la maison d’Acab, tu as même tué tes frères – la maison de ton père –, et ils étaient meilleurs que toi. 
${}^{14}C’est pourquoi le Seigneur va frapper d’un grand fléau ton peuple, tes fils, tes femmes et tout ce qui t’appartient. 
${}^{15}Toi, tu seras frappé de nombreuses maladies, d’un mal d’entrailles tel que, par l’effet de ce mal, tu te videras jour après jour de tes entrailles. »
       
${}^{16}Le Seigneur excita contre Joram l’esprit des Philistins et des Arabes voisins des Éthiopiens. 
${}^{17}Ils montèrent contre le pays de Juda et l’envahirent. Ils s’emparèrent de toutes les richesses qui se trouvaient dans la maison du roi, ainsi que de ses fils et de ses femmes. Il ne lui resta plus qu’un fils, Ocozias, le plus jeune. 
${}^{18}Après tout cela, le Seigneur le frappa aux entrailles d’un mal incurable. 
${}^{19}De jour en jour, par l’effet de son mal il se vidait de ses entrailles jusqu’à la fin de la deuxième année, et il mourut dans de cruelles souffrances. Le peuple ne fit pas pour lui de brasier comme on l’avait fait pour ses pères.
${}^{20}Joram avait trente-deux ans lorsqu’il devint roi,
        \\et il régna huit ans à Jérusalem.
        \\Il partit sans laisser de regrets.
        \\On l’ensevelit dans la Cité de David,
        \\mais non pas dans les tombeaux des rois.
      
         
      \bchapter{}
      \begin{verse}
${}^{1}Les habitants de Jérusalem firent roi à la place de Joram Ocozias, son plus jeune fils, car la troupe qui avait investi le camp avec les Arabes avait mis à mort tous les aînés. Ainsi devint roi Ocozias, fils de Joram, roi de Juda. 
${}^{2}Il avait quarante-deux ans lorsqu’il devint roi, et il régna un an à Jérusalem. Sa mère s’appelait Athalie, fille d’Omri. 
${}^{3}Lui aussi marcha dans les chemins de la maison d’Acab, car sa mère lui donnait de mauvais conseils. 
${}^{4}Il fit ce qui est mal aux yeux du Seigneur, comme les gens de la maison d’Acab, car après la mort de son père, ils furent ses conseillers, pour son malheur. 
${}^{5}C’est même sur leur conseil qu’il partit avec Joram, fils d’Acab, roi d’Israël, pour combattre à Ramoth-de-Galaad Hazaël, roi d’Aram. Mais les Araméens blessèrent Joram. 
${}^{6}Il revint à Yizréel se faire soigner des blessures qu’il avait reçues dans le combat contre Hazaël, roi d’Aram.
      Ocozias, fils de Joram, roi de Juda, descendit à Yizréel pour voir Joram, fils d’Acab, parce qu’il était blessé. 
${}^{7}De cette visite à Joram Dieu fit la perte d’Ocozias. À son arrivée, celui-ci sortit avec Joram à la rencontre de Jéhu, fils de Nimshi, à qui le Seigneur avait donné l’onction pour supprimer la maison d’Acab. 
${}^{8}Alors que Jéhu faisait justice de la maison d’Acab, il trouva les princes de Juda et les neveux d’Ocozias qui étaient à son service, et les tua. 
${}^{9}Puis il chercha Ocozias. On se saisit d’Ocozias dans Samarie où il se cachait, on l’amena à Jéhu et on le mit à mort. On l’ensevelit, car on disait : « Il est le fils de Josaphat, et celui-ci a cherché le Seigneur de tout son cœur. » De la maison d’Ocozias, il n’y eut plus personne en mesure de régner.
${}^{10}Lorsqu’Athalie, mère d’Ocozias, apprit que son fils était mort, elle entreprit de faire périr toute la descendance royale, dans la maison de Juda. 
${}^{11}Mais Josabeth, fille du roi Joram, prit Joas, fils d’Ocozias, pour le soustraire au massacre. Elle le cacha, lui et sa nourrice, dans une chambre de la maison du Seigneur. Josabeth, fille du roi Joram et femme du prêtre Joad, – elle était en effet la sœur d’Ocozias – le dissimula ainsi aux regards d’Athalie, qui ne put le mettre à mort. 
${}^{12}Joas demeura avec Josabeth pendant six ans, caché dans la maison de Dieu, tandis qu’Athalie régnait sur le pays.
      
         
      \bchapter{}
      \begin{verse}
${}^{1}Au bout de sept ans, le prêtre Joad prit une décision courageuse. Il appela auprès de lui les officiers de centaine : Azaria, fils de Jéroham, Ismaël, fils de Yohanane, Azarias, fils d’Obed, Maaséya, fils d’Adaya, et Élishafath, fils de Zikri, qui avaient fait alliance avec lui. 
${}^{2}Ils parcoururent le pays de Juda et rassemblèrent, de toutes les villes de Juda, les Lévites et les chefs de famille d’Israël ; puis ils vinrent à Jérusalem. 
${}^{3}Toute l’assemblée conclut une alliance avec le roi dans la maison de Dieu. Et Joad leur dit : « Voici le fils du roi : il doit régner, selon ce que le Seigneur a déclaré au sujet des fils de David. 
${}^{4}Voilà ce que vous allez faire : un tiers d’entre vous, ceux qui entrent en service le jour du sabbat, prêtres et Lévites, garderont les portes ; 
${}^{5}un tiers se tiendra dans la maison du roi, un tiers à la porte de la Fondation, et tout le peuple dans les parvis de la maison du Seigneur. 
${}^{6}Que personne n’entre dans la maison du Seigneur, sauf les prêtres et les Lévites de service ; eux pourront entrer, car ils sont consacrés, mais tout le peuple observera l’ordre du Seigneur. 
${}^{7}Les Lévites feront cercle autour du roi, chacun les armes à la main. Celui qui entrera dans la Maison sera mis à mort. Et vous accompagnerez le roi quand il part en campagne et en revient. »
${}^{8}Les Lévites et tout Juda exécutèrent tous les ordres du prêtre Joad. Chacun prit ses hommes, ceux qui entraient en service le jour du sabbat, et ceux qui en sortaient. Car le prêtre Joad n’avait exempté aucune des classes. 
${}^{9}Le prêtre Joad remit aux officiers de centaine les lances, les grands et les petits boucliers du roi David, qui étaient conservés dans la maison de Dieu. 
${}^{10}Il fit placer tout le peuple, les javelots à la main, devant l’autel, du côté sud et du côté nord de la Maison, afin d’entourer le futur roi. 
${}^{11}Alors on fit avancer le fils du roi, on lui remit le diadème et le Témoignage, et on le fit roi. Puis Joad et ses fils lui donnèrent l’onction, et on cria : « Vive le roi ! »
${}^{12}Athalie entendit la clameur du peuple, qui courait et acclamait le roi. Elle accourut vers le peuple à la maison du Seigneur. 
${}^{13}Et voilà ce qu’elle vit : le roi debout devant sa colonne, à l’entrée ; auprès de lui, les officiers et les trompettes, et tout le peuple du pays criant sa joie tandis que les trompettes sonnaient, et que les chantres, accompagnés des instruments de musique, dirigeaient les acclamations. Alors, Athalie déchira ses vêtements et s’écria : « Trahison ! Trahison ! » 
${}^{14}Le prêtre Joad fit sortir les officiers de centaine qui commandaient la troupe et leur donna cet ordre : « Faites-la sortir de la Maison, entre les rangs. Si quelqu’un veut la suivre, qu’il soit mis à mort par l’épée. » En effet, le prêtre Joad avait interdit de la mettre à mort dans la maison du Seigneur. 
${}^{15}On mit la main sur elle, et elle arriva à la maison du roi par l’entrée de la porte des Chevaux. C’est là qu’elle fut mise à mort.
${}^{16}Joad conclut une alliance entre lui-même, tout le peuple et le roi, pour que le peuple soit le peuple du Seigneur. 
${}^{17}Alors, tout le peuple entra dans le temple de Baal et le démolit. Ils brisèrent ses autels et ses statues et, devant les autels, ils tuèrent Matane, prêtre de Baal. 
${}^{18}Joad posta ensuite des gardes devant la maison du Seigneur sous l’autorité des prêtres lévites. David avait réparti ceux-ci dans la maison du Seigneur pour qu’ils offrent, comme cela est écrit dans la loi de Moïse, les holocaustes du Seigneur, dans la joie et les chants, selon les indications de David. 
${}^{19}Il posta des gardes aux portes de la maison du Seigneur, pour que n’y entre, en aucun cas, une personne en état d’impureté. 
${}^{20}Il prit ensuite les officiers, les notables, ceux qui avaient autorité sur le peuple, et tous les gens du pays. Il fit descendre le roi de la maison du Seigneur. Ils entrèrent, par la Porte du Haut, dans la maison du roi et ils firent asseoir le roi sur le trône de la royauté. 
${}^{21}Tous les gens du pays étaient dans la joie, et la ville retrouva le calme. Quant à Athalie, on l’avait mise à mort par l’épée.
      
         
      \bchapter{}
      \begin{verse}
${}^{1}Joas avait sept ans lorsqu’il devint roi, et il régna quarante ans à Jérusalem. Le nom de sa mère était Cibya ; elle était de Bershéba. 
${}^{2}Joas fit ce qui est droit aux yeux du Seigneur pendant toute la vie du prêtre Joad. 
${}^{3}Joad lui fit épouser deux femmes, dont il eut des fils et des filles.
${}^{4}Après quoi, Joas eut à cœur de restaurer la maison du Seigneur. 
${}^{5}Il réunit les prêtres et les Lévites et leur dit : « Allez dans les villes de Juda, recueillez de l’argent auprès de tous les fils d’Israël, pour réparer d’année en année la Maison de votre Dieu, et hâtez-vous de le faire ! » Mais les Lévites ne se hâtèrent pas. 
${}^{6}Le roi convoqua Joad, le chef des prêtres, et lui dit : « Pourquoi n’as-tu pas exigé des Lévites qu’ils apportent, de Juda et de Jérusalem, l’impôt que Moïse, serviteur du Seigneur, et l’assemblée d’Israël ont fixé pour la tente du Témoignage ? 
${}^{7}Athalie, l’impiété en personne, ainsi que ses fils ont en effet laissé se dégrader la maison de Dieu. Ils ont même utilisé pour le culte de Baal les objets sacrés de la maison du Seigneur. »
${}^{8}Le roi ordonna alors de fabriquer un coffre et de le placer à la porte de la maison du Seigneur, à l’extérieur. 
${}^{9}On proclama dans Juda et dans Jérusalem qu’il fallait apporter au Seigneur l’impôt que Moïse, serviteur de Dieu, avait fixé à Israël dans le désert. 
${}^{10}Tous les princes et tout le peuple se réjouirent. Ils apportèrent l’impôt et le versèrent dans le coffre jusqu’à paiement complet. 
${}^{11}Chaque fois que les Lévites apportaient le coffre aux inspecteurs du roi, si on voyait qu’il y avait beaucoup d’argent, le secrétaire du roi et l’intendant du grand prêtre venaient vider le coffre et le remportaient pour le remettre à sa place. Ainsi faisaient-ils chaque jour, et ils récoltaient beaucoup d’argent. 
${}^{12}Le roi et Joad remettaient l’argent au maître d’œuvre, pour le service de la maison du Seigneur. Celui-ci engageait des tailleurs de pierre et des charpentiers pour restaurer la maison du Seigneur, et aussi des artisans du fer et du bronze pour réparer la maison du Seigneur. 
${}^{13}Les maîtres d’œuvre se mirent au travail, et, entre leurs mains, les travaux de réfection progressèrent. Ils remirent en état la maison de Dieu et la consolidèrent. 
${}^{14}Lorsqu’ils eurent achevé, ils apportèrent devant le roi et Joad le reste de l’argent, avec lequel on fabriqua des objets pour la maison du Seigneur, objets pour le service et pour les holocaustes, coupes et autres ustensiles d’or et d’argent. On offrit continuellement des holocaustes dans la maison du Seigneur, pendant toute la vie de Joad.
${}^{15}Joad devint très âgé, il fut rassasié de jours et il mourut. Il avait cent trente ans quand il mourut. 
${}^{16}On l’ensevelit dans la Cité de David avec les rois, parce qu’il avait fait du bien en Israël, à l’égard de Dieu et de sa Maison.
${}^{17}Après\\la mort de Joad, les princes de Juda vinrent se prosterner devant le roi, et alors le roi les écouta. 
${}^{18} Les gens abandonnèrent la maison du Seigneur, Dieu de leurs pères, pour servir les poteaux sacrés et les idoles. À cause de cette infidélité, la colère de Dieu\\s’abattit sur Juda et sur Jérusalem. 
${}^{19} Pour les ramener à lui, Dieu envoya chez eux des prophètes. Ceux-ci transmirent le message, mais personne ne les écouta. 
${}^{20} Dieu revêtit de son esprit Zacharie, le fils du prêtre Joad. Zacharie se présenta devant le peuple et lui dit : « Ainsi parle Dieu : Pourquoi transgressez-vous les commandements du Seigneur ? Cela fera votre malheur : puisque vous avez abandonné le Seigneur, le Seigneur vous abandonne. » 
${}^{21} Ils s’ameutèrent alors contre lui et, par commandement du roi, le lapidèrent sur le parvis de la maison du Seigneur\\.  
${}^{22} Le roi Joas, en faisant mourir Zacharie, fils de Joad, oubliait la fidélité que Joad lui avait témoignée. Zacharie s’était écrié en mourant : « Que le Seigneur le voie, et qu’il fasse justice ! »
${}^{23}Or, à la fin de l’année, l’armée d’Aram monta contre le roi Joas et arriva en Juda et à Jérusalem. Ses hommes massacrèrent tous les princes du peuple\\et envoyèrent tout le butin au roi de Damas. 
${}^{24} L’armée d’Aram ne comptait qu’un petit nombre d’hommes, et pourtant le Seigneur leur livra une armée très importante, parce que les gens de Juda\\avaient abandonné le Seigneur, Dieu de leurs pères ; et Joas reçut le châtiment qu’il méritait\\. 
${}^{25} Lorsque les Araméens\\partirent, le laissant dans de grandes souffrances, ses serviteurs complotèrent contre lui parce qu’il avait répandu le sang du fils\\du prêtre Joad, et ils le tuèrent sur son lit. Il mourut, et on l’ensevelit dans la Cité de David, mais non pas dans les tombeaux des rois.
${}^{26}Voici ceux qui complotèrent contre lui : Zabad, fils d’une Ammonite nommée Shiméath, et Yehozabad, fils d’une Moabite nommée Shimrith.
${}^{27}Ce qui concerne ses fils,
        \\les nombreuses prophéties lancées contre lui,
        \\les aménagements de la maison de Dieu,
        \\cela est écrit dans le Commentaire du livre des Rois.
        \\Son fils Amasias régna à sa place.
      
         
      \bchapter{}
      \begin{verse}
${}^{1}Amasias avait vingt-cinq ans lorsqu’il devint roi, et il régna vingt-neuf ans à Jérusalem. Sa mère s’appelait Yoadane : elle était de Jérusalem. 
${}^{2}Il fit ce qui est droit aux yeux du Seigneur, mais non pas d’un cœur sans partage. 
${}^{3}Lorsque sa royauté fut affermie, il tua ceux de ses serviteurs qui avaient frappé le roi son père. 
${}^{4}Mais leurs fils, il ne les mit pas à mort, selon ce qui est écrit dans la Loi, le Livre de Moïse, où le Seigneur a donné cet ordre : « Les pères ne mourront pas à la place des fils, les fils ne mourront pas à la place des pères, mais chacun mourra pour son propre péché. »
${}^{5}Amasias réunit les gens de Juda et les organisa en familles, avec officiers de millier et officiers de centaine, pour tout Juda et Benjamin. Il recensa ceux qui avaient vingt ans et plus : il trouva trois cent mille hommes d’élite, aptes à rejoindre l’armée, maniant la lance et le bouclier. 
${}^{6}De plus, il recruta en Israël, pour cent talents d’argent, cent mille vaillants guerriers. 
${}^{7}Mais un homme de Dieu vint le trouver pour lui dire : « Ô roi, il ne faut pas qu’une armée d’Israël se joigne à toi, car le Seigneur n’est pas avec Israël, avec tous ces fils d’Éphraïm. 
${}^{8}Si elle vient, tu auras beau combattre avec courage, Dieu te fera trébucher devant l’ennemi, car Dieu a le pouvoir de secourir et de faire trébucher. » 
${}^{9}Amasias dit à l’homme de Dieu : « Que faire alors des cent talents que j’ai donnés à la troupe d’Israël ? » L’homme de Dieu répondit : « Le Seigneur peut te donner beaucoup plus que cela. » 
${}^{10}Amasias mit à part la troupe qui lui était venue d’Éphraïm, et la renvoya chez elle. Mais la colère de ces gens s’enflamma fort contre Juda, et ils retournèrent chez eux enflammés de colère.
${}^{11}Amasias, ayant pris de l’assurance, emmena son peuple ; il se rendit dans la vallée du Sel et frappa dix mille des fils de Séïr. 
${}^{12}Les fils de Juda en prirent vivants dix mille autres, qu’ils emmenèrent au sommet de la Roche. Ils les précipitèrent du sommet de la Roche, et tous s’écrasèrent au sol. 
${}^{13}Quant aux hommes de troupe qu’Amasias avait renvoyés pour qu’ils n’aillent pas à la guerre avec lui, ils se jetèrent sur les villes de Juda, depuis Samarie jusqu’à Beth-Horone. Ils frappèrent trois mille habitants et s’emparèrent d’un grand butin.
${}^{14}À son retour, après avoir battu les Édomites, Amasias ramena les dieux des fils de Séïr, les établit comme ses dieux, se prosterna devant eux et pour eux brûla de l’encens. 
${}^{15}Le Seigneur s’enflamma de colère contre Amasias et lui envoya un prophète pour lui dire : « Pourquoi consultes-tu les dieux de ce peuple, alors qu’ils n’ont pas su sauver leur peuple de ta main ? » 
${}^{16}Pendant qu’il parlait, Amasias lui dit : « Ne t’avons-nous pas institué conseiller du roi ? Arrête ! Faut-il qu’on te frappe ? » Le prophète s’arrêta, mais il dit : « Je sais que Dieu a résolu de te détruire, parce que tu as fait cela et que tu n’as pas écouté mon conseil. »
${}^{17}Après avoir tenu conseil, Amasias, roi de Juda, envoya des messagers à Joas, fils de Joakaz, fils de Jéhu, roi d’Israël, pour lui dire : « Viens, et affrontons-nous ! » 
${}^{18}Joas, roi d’Israël, envoya à son tour des messagers à Amasias, roi de Juda, pour lui dire : « Le chardon du Liban envoya dire au cèdre du Liban : “Donne ta fille pour femme à mon fils”. Mais une bête sauvage du Liban est passée, elle a piétiné le chardon. 
${}^{19}Tu te dis que tu as vaincu Édom, alors ton cœur s’enorgueillit et se glorifie. Reste donc chez toi ! Pourquoi provoquer le malheur et tomber, toi et Juda avec toi ? » 
${}^{20}Mais Amasias n’écouta pas. Cela était voulu par Dieu afin qu’ils soient livrés, lui et son peuple, aux mains de l’adversaire, pour avoir consulté les dieux du pays d’Édom. 
${}^{21}Alors Joas, roi d’Israël, se mit en marche, et ils s’affrontèrent, lui et Amasias, roi de Juda, à Beth-Shèmesh, qui appartient à Juda. 
${}^{22}Ceux de Juda furent battus devant Israël et ils s’enfuirent chacun à sa tente.
${}^{23}Quant à Amasias, roi de Juda, fils de Joas, fils d’Ocozias, Joas, roi d’Israël, le fit prisonnier à Beth-Shèmesh et le ramena à Jérusalem. Il fit au rempart de Jérusalem une brèche de quatre cents coudées, depuis la porte d’Éphraïm jusqu’à la porte de l’Angle. 
${}^{24}Joas prit tout l’or et tout l’argent, tous les objets qui se trouvaient dans la maison de Dieu, sous la garde d’Obed-Édom, et les trésors de la maison du roi, ainsi que des otages. Puis il retourna à Samarie.
${}^{25}Amasias, fils de Joas, roi de Juda, vécut encore quinze ans après la mort de Joas, fils de Joakaz, roi d’Israël.
${}^{26}Le reste des actions d’Amasias,
        \\des premières aux dernières,
        \\cela n’est-il pas écrit dans le Livre des rois de Juda et d’Israël ?
${}^{27}À partir du moment où Amasias se détourna du Seigneur, certains, à Jérusalem, tramèrent un complot contre lui. Il s’enfuit dans la ville de Lakish ; mais on envoya des gens à sa poursuite à Lakish, et c’est là qu’il fut mis à mort. 
${}^{28}Puis, on le transporta sur des chevaux, et on l’ensevelit avec ses pères dans la Cité de David.
      
         
      \bchapter{}
      \begin{verse}
${}^{1}Tout le peuple de Juda prit Ozias, âgé de seize ans, et le fit roi, à la place de son père Amasias. 
${}^{2}C’est lui qui rebâtit Eilath et la réintégra en Juda, après que le roi Amasias eut reposé avec ses pères. 
${}^{3}Ozias avait seize ans lorsqu’il devint roi, et il régna cinquante-deux ans à Jérusalem. Sa mère s’appelait Jékolie ; elle était de Jérusalem. 
${}^{4}Il fit ce qui est droit aux yeux du Seigneur, tout comme avait fait Amasias, son père. 
${}^{5}Il s’appliqua à rechercher Dieu tant que vécut Zacharie, qui avait l’intelligence des visions de Dieu ; et tout le temps qu’il rechercha le Seigneur, Dieu le fit réussir.
${}^{6}Il partit en guerre contre les Philistins, il démantela le rempart de Gath, le rempart de Yabné, le rempart d’Ashdod, et construisit des villes dans la région d’Ashdod et chez les Philistins. 
${}^{7}Dieu lui vint en aide contre les Philistins, contre les Arabes qui habitaient à Gour-Baal, et contre les Méounites. 
${}^{8}Les Ammonites payèrent tribut à Ozias, dont la renommée parvint jusqu’à l’entrée de l’Égypte, car il était devenu extrêmement puissant.
${}^{9}Ozias construisit des tours à Jérusalem, sur la porte de l’Angle, sur la porte de la Vallée, et sur le Contrefort ; et il les fortifia. 
${}^{10}Il construisit des tours dans le désert et creusa de nombreuses citernes, car il avait de nombreux troupeaux dans le Bas-Pays et sur le Plateau. Il avait aussi des cultivateurs et des vignerons, dans les montagnes et au Carmel, car il aimait la terre.
${}^{11}Ozias avait une armée entraînée, capable d’aller au combat, répartie par troupes selon le nombre des hommes recensés par Yéiël, le secrétaire, et Maasias, le scribe ; elle était sous l’autorité de Hananias, l’un des officiers du roi. 
${}^{12}Le nombre total des chefs de famille des vaillants guerriers était de deux mille six cents. 
${}^{13}Ils avaient sous leur autorité une armée de trois cent sept mille cinq cents hommes, d’une grande valeur au combat, pour aider le roi en face de l’ennemi. 
${}^{14}Ozias procura à toute cette armée des boucliers, des lances, des casques, des cuirasses, des arcs, et des frondes avec leurs pierres. 
${}^{15}Il fit également fabriquer à Jérusalem des engins conçus par un ingénieur, destinés à être placés sur les tours et aux angles, pour lancer des traits et de grosses pierres. Sa renommée se répandit au loin car il fut merveilleusement aidé, au point qu’il devint puissant.
${}^{16}Mais lorsqu’il fut devenu puissant, son cœur s’enorgueillit jusqu’à le perdre, et il fut infidèle au Seigneur son Dieu. Il entra dans le temple du Seigneur pour brûler de l’encens sur l’autel de l’encens. 
${}^{17}Le prêtre Azarias entra après lui, avec quatre-vingts prêtres du Seigneur, des hommes courageux. 
${}^{18}Ils s’opposèrent au roi Ozias et lui dirent : « Ce n’est pas à toi, Ozias, de brûler de l’encens pour le Seigneur, mais aux prêtres, descendants d’Aaron, qui ont été consacrés pour brûler de l’encens. Sors du sanctuaire, car tu as été infidèle, et cela ne te vaudra pas la gloire qui vient du Seigneur Dieu. »
${}^{19}Ozias, qui tenait à la main un encensoir pour brûler de l’encens, se mit en rage. Tandis qu’il était en rage contre les prêtres, la lèpre apparut sur son front, en leur présence, dans la maison du Seigneur, devant l’autel de l’encens. 
${}^{20}Le grand prêtre Azarias et tous les prêtres se tournèrent vers lui, et voici que son front était couvert de lèpre ! En toute hâte ils l’expulsèrent, et lui-même se pressa de sortir, parce que le Seigneur l’avait frappé. 
${}^{21}Le roi Ozias fut lépreux jusqu’au jour de sa mort, et il habita, lépreux, dans une maison à l’écart ; en effet, il était exclu de la maison du Seigneur. Son fils Yotam, maître du palais du roi, gouvernait les gens du pays.
${}^{22}Le reste des actions d’Ozias, des premières aux dernières,
        \\le prophète Isaïe, fils d’Amots, les a écrites.
${}^{23}Ozias reposa avec ses pères,
        \\et on l’ensevelit avec eux
        \\dans le champ de la sépulture des rois,
        \\car on disait : « Il est lépreux ! »
        \\Son fils Yotam régna à sa place.
      
         
      \bchapter{}
      \begin{verse}
${}^{1}Yotam avait vingt-cinq ans lorsqu’il devint roi, et il régna seize ans à Jérusalem. Sa mère s’appelait Yerousha, fille de Sadoc. 
${}^{2}Il fit ce qui est droit aux yeux du Seigneur, tout comme avait fait Ozias, son père, sans toutefois entrer dans le temple du Seigneur. Mais le peuple continuait à se corrompre.
${}^{3}C’est Yotam qui bâtit la porte Haute de la maison du Seigneur ; il fit beaucoup de travaux sur le rempart d’Ophel. 
${}^{4}Il construisit des villes dans la montagne de Juda, et, dans les régions boisées, des citadelles et des tours. 
${}^{5}C’est lui qui combattit le roi des fils d’Ammone et l’emporta sur eux. Les fils d’Ammone lui donnèrent, cette année-là, cent talents d’argent, dix mille quintaux de blé et dix mille quintaux d’orge. Et ils lui en apportèrent autant, la deuxième et la troisième année. 
${}^{6}Yotam s’affermit, car il marchait constamment en présence du Seigneur son Dieu.
${}^{7}Le reste des actions de Yotam,
        \\toutes ses guerres et ce qu’il a entrepris,
        \\voici que cela est écrit dans le Livre des rois d’Israël et de Juda.
${}^{8}Il avait vingt-cinq ans lorsqu’il devint roi,
        \\et il régna seize ans à Jérusalem.
${}^{9}Yotam reposa avec ses pères,
        \\et on l’ensevelit dans la Cité de David.
        \\Son fils Acaz régna à sa place.
      
         
      \bchapter{}
      \begin{verse}
${}^{1}Acaz avait vingt ans lorsqu’il devint roi, et il régna seize ans à Jérusalem. Il ne fit pas ce qui est droit aux yeux du Seigneur, comme avait fait David, son ancêtre. 
${}^{2}Il marcha dans le chemin des rois d’Israël, et il fit même des idoles de métal fondu en l’honneur des Baals. 
${}^{3}C’est lui qui brûla de l’encens dans la vallée de Ben-Hinnom, et fit passer ses fils par le feu, selon les coutumes abominables des nations que le Seigneur avait dépossédées devant les fils d’Israël. 
${}^{4}Il offrit des sacrifices et brûla de l’encens dans les lieux sacrés, sur les collines et sous tout arbre verdoyant.
${}^{5}Le Seigneur son Dieu le livra aux mains du roi des Araméens. Ceux-ci le vainquirent, et firent un grand nombre de prisonniers qu’ils emmenèrent à Damas. Il fut également livré aux mains du roi d’Israël qui lui infligea une grande défaite. 
${}^{6}En un seul jour, Pèqah, fils de Remalyahou, tua en Juda cent vingt mille hommes, tous de vaillants guerriers, car ils avaient abandonné le Seigneur, Dieu de leurs pères. 
${}^{7}Zikri, guerrier d’Éphraïm, tua Maasias, fils du roi, Azriqam, chef de la maison du roi, et Elqana, le second du roi. 
${}^{8}Les fils d’Israël firent prisonniers parmi leurs frères deux cent mille femmes, fils et filles. Ils prirent aussi un grand butin qu’ils emportèrent à Samarie.
${}^{9}Il y avait là un prophète du Seigneur, nommé Oded. Il sortit au-devant de l’armée qui arrivait à Samarie, et dit : « Voici que, dans sa fureur contre Juda, le Seigneur, Dieu de vos pères, les a livrés entre vos mains, et vous les avez tués avec une rage qui est parvenue jusqu’au ciel. 
${}^{10}Et maintenant, vous parlez de soumettre les fils de Juda et de Jérusalem pour qu’ils soient vos serviteurs et vos servantes. Mais n’avez-vous pas à vous reprocher quelque offense envers le Seigneur votre Dieu ? 
${}^{11}Alors, maintenant écoutez-moi, et renvoyez les prisonniers que vous avez faits parmi vos frères, car l’ardeur de la colère du Seigneur est sur vous. »
${}^{12}Quelques-uns d’entre les chefs des fils d’Éphraïm, Azarias, fils de Yohanane, Bérékia, fils de Meshillémoth, Yehizqia, fils de Shalloum, et Amasa, fils de Hadlaï, se dressèrent contre ceux qui revenaient de l’expédition. 
${}^{13}Ils leur dirent : « Ne faites pas entrer ici les prisonniers : ce serait nous charger d’une offense envers le Seigneur, et ajouter à nos péchés et à nos offenses. Notre offense est déjà grande, et l’ardeur de la colère du Seigneur est sur Israël. » 
${}^{14}La troupe abandonna les prisonniers et le butin devant les princes et toute l’assemblée. 
${}^{15}Alors se levèrent les hommes désignés par leurs noms, qui prirent en charge les prisonniers. Avec le butin, ils habillèrent tous ceux d’entre eux qui étaient nus, leur donnèrent de quoi se vêtir et se chausser, les firent manger et boire, et les soignèrent. Ils conduisirent sur des ânes tous les éclopés et les emmenèrent à Jéricho, la ville des Palmiers, auprès de leurs frères. Puis ils retournèrent à Samarie.
${}^{16}À cette époque, le roi Acaz envoya demander de l’aide aux rois d’Assour : 
${}^{17}les Édomites étaient encore venus, avaient battu Juda et emmené des prisonniers ; 
${}^{18}les Philistins s’étaient répandus dans les villes du Bas-Pays et du Néguev de Juda ; ils avaient pris Beth-Shèmesh, Ayyalone, Guedéroth, Soko et ses dépendances, Timna et ses dépendances, Guimzo et ses dépendances, et s’y étaient installés. 
${}^{19}En effet, le Seigneur humiliait Juda à cause d’Acaz, roi d’Israël, qui laissait Juda se relâcher et qui était gravement infidèle au Seigneur.
${}^{20}Téglath-Phalasar, roi d’Assour, vint l’attaquer et l’assiégea, au lieu de le soutenir. 
${}^{21}En effet, Acaz avait dépouillé la maison du Seigneur, la maison du roi et des princes, pour faire des présents au roi d’Assour. Mais cela ne lui fut d’aucune aide. 
${}^{22}Au temps où il était assiégé, le roi Acaz, lui, continuait à être infidèle au Seigneur. 
${}^{23}Il offrit des sacrifices aux dieux de Damas qui l’avaient frappé, car il se disait : « Puisque les dieux des rois d’Aram leur viennent en aide, je leur offrirai des sacrifices, et ils me viendront en aide ! » Mais ces dieux furent une cause de chute pour lui et pour tout Israël.
${}^{24}Acaz rassembla les objets de la maison de Dieu et les brisa ; il ferma les portes de la maison du Seigneur, et se fit des autels dans tous les coins de Jérusalem. 
${}^{25}Dans chacune des villes de Juda, il établit des lieux sacrés pour brûler de l’encens à d’autres dieux. Et il provoqua l’indignation du Seigneur, Dieu de ses pères.
${}^{26}Le reste de ses actions et toutes ses entreprises,
        \\des premières aux dernières,
        \\voici que cela est écrit dans le Livre des rois de Juda et d’Israël.
${}^{27}Acaz reposa avec ses pères,
        \\et on l’ensevelit dans la ville, à Jérusalem,
        \\car on ne le mit pas dans les tombeaux des rois d’Israël.
        \\Son fils Ézékias régna à sa place.
      
         
      \bchapter{}
      \begin{verse}
${}^{1}Ézékias devint roi à l’âge de vingt-cinq ans et il régna vingt-neuf ans à Jérusalem. Le nom de sa mère était Abiya, fille de Zacharie. 
${}^{2}Il fit ce qui est droit aux yeux du Seigneur, tout comme avait fait David, son ancêtre. 
${}^{3}La première année de son règne, le premier mois, il rouvrit les portes de la maison du Seigneur et les répara. 
${}^{4}Il fit venir les prêtres et les Lévites et, les ayant réunis sur la place de l’Orient, 
${}^{5}il leur dit : « Écoutez-moi, Lévites ! Maintenant, sanctifiez-vous, sanctifiez la maison du Seigneur, Dieu de vos pères, et du sanctuaire enlevez la souillure. 
${}^{6}Car nos pères ont été infidèles, ils ont fait ce qui est mal aux yeux du Seigneur notre Dieu. Ils l’ont abandonné, ils ont détourné leur visage de la demeure du Seigneur, et lui ont tourné le dos. 
${}^{7}Ils ont même fermé les portes du Vestibule, ils ont éteint les lampes, ils ont cessé de brûler de l’encens et d’offrir l’holocauste dans le sanctuaire du Dieu d’Israël. 
${}^{8}La colère du Seigneur s’est abattue sur Juda et sur Jérusalem. Il les a livrés à la terreur, à la désolation et à la moquerie, comme vous le voyez de vos yeux. 
${}^{9}Voilà pourquoi nos pères sont tombés sous l’épée, et nos fils, nos filles et nos femmes sont en captivité. 
${}^{10}Maintenant j’ai l’intention de conclure une alliance avec le Seigneur, Dieu d’Israël, pour que se détourne de nous l’ardeur de sa colère. 
${}^{11}Maintenant, mes fils, ne soyez plus négligents, car c’est vous que le Seigneur a choisis, pour vous tenir devant lui et le servir, pour être ses serviteurs et brûler de l’encens. »
${}^{12}Alors les Lévites se mirent à l’œuvre : parmi les fils de Qehath : Mahath fils d’Amasaï, Joël fils d’Azarias ; parmi les fils de Merari : Qish fils d’Abdi, Azarias fils de Yehallélel ; parmi les Guershonites : Yoah fils de Zimma, Éden fils de Yoah ; 
${}^{13}parmi les fils d’Éliçafane : Shimri et Yéiël ; parmi les fils d’Asaf : Zacharie et Mattanyahou ; 
${}^{14}parmi les fils de Hémane : Yeïel et Shiméï ; parmi les fils de Yedoutoune : Shemaya et Ouzziël. 
${}^{15}Ils réunirent leurs frères, se sanctifièrent et, selon l’ordre du roi conforme aux paroles du Seigneur, ils vinrent purifier la maison du Seigneur.
${}^{16}Les prêtres entrèrent à l’intérieur de la maison du Seigneur pour la purifier. Ils emportèrent sur le parvis de la maison du Seigneur tous les objets impurs qu’ils trouvèrent dans le temple du Seigneur. Les Lévites les ramassèrent pour les emporter dehors, dans le ravin du Cédron. 
${}^{17}Ils commencèrent cette sanctification le premier jour du premier mois. Le huitième jour du mois, ils entrèrent dans le Vestibule du Seigneur. Ils mirent huit jours à sanctifier la maison du Seigneur, et le seizième jour du premier mois ils avaient achevé.
${}^{18}Ils se rendirent ensuite dans les appartements du roi Ézékias et lui dirent : « Nous avons purifié toute la maison du Seigneur, l’autel des holocaustes et tous ses ustensiles, la table où l’on dispose les pains de l’offrande et tous ses ustensiles. 
${}^{19}Et tous les objets que, pendant son règne, le roi Acaz avait profanés par son infidélité, nous les avons remis en état et sanctifiés : les voici devant l’autel du Seigneur. »
${}^{20}Le roi Ézékias, s’étant levé de bon matin, réunit les princes de la ville et monta à la maison du Seigneur. 
${}^{21}On amena sept taureaux, sept béliers, sept agneaux et sept boucs, en sacrifice pour la faute, à l’intention du royaume, du sanctuaire et de Juda. Le roi dit aux prêtres, descendants d’Aaron, d’offrir les holocaustes sur l’autel du Seigneur. 
${}^{22}On immola les bœufs, et les prêtres recueillirent le sang dont on aspergea l’autel ; on immola les béliers, et de leur sang on aspergea l’autel ; on immola les agneaux, et de leur sang on aspergea l’autel. 
${}^{23}Ensuite, devant le roi et l’assemblée, on fit avancer les boucs destinés au sacrifice pour la faute, et on leur imposa les mains. 
${}^{24}Les prêtres les immolèrent et, de leur sang répandu sur l’autel, ils firent un sacrifice pour la faute, en expiation pour tout Israël. C’est en effet pour tout Israël que le roi avait ordonné l’holocauste et le sacrifice pour la faute.
${}^{25}Il plaça les Lévites dans la maison du Seigneur, au son des cymbales, des harpes et des cithares, selon le commandement de David, de Gad, le voyant du roi, et du prophète Nathan. Car ce commandement venait du Seigneur par l’intermédiaire de ses prophètes. 
${}^{26}Quand les Lévites, avec les instruments de David, et les prêtres, avec les trompettes, eurent pris place, 
${}^{27}Ézékias ordonna d’offrir l’holocauste sur l’autel. Au moment où commença l’holocauste commencèrent aussi le cantique au Seigneur et la sonnerie des trompettes, avec accompagnement des instruments de David, roi d’Israël. 
${}^{28}Toute l’assemblée se prosterna ; on chanta le cantique, et les trompettes sonnèrent jusqu’à l’achèvement de l’holocauste.
${}^{29}Lorsqu’on eut achevé d’offrir l’holocauste, le roi et tous ceux qui étaient avec lui s’inclinèrent et se prosternèrent. 
${}^{30}Le roi Ézékias et les princes dirent aux Lévites de célébrer le Seigneur en reprenant les paroles de David et d’Asaf le Voyant. Ils célébrèrent donc dans la joie, puis se mirent à genoux et se prosternèrent. 
${}^{31}Ézékias prit alors la parole et dit : « Maintenant, vous avez reçu l’investiture du Seigneur ; approchez-vous donc, offrez des sacrifices d’action de grâce dans la maison du Seigneur. » Alors l’assemblée offrit des sacrifices d’action de grâce, et tous ceux que leur cœur y incitait offrirent des holocaustes. 
${}^{32}Le nombre des holocaustes qu’offrit l’assemblée fut de soixante-dix bœufs, cent béliers et deux cents agneaux, le tout en holocauste pour le Seigneur. 
${}^{33}On consacra encore six cents bœufs et trois mille brebis. 
${}^{34}Toutefois les prêtres n’étaient pas assez nombreux pour dépecer toutes les victimes de l’holocauste. Alors leurs frères, les Lévites, les aidèrent jusqu’à ce que le travail soit achevé, et que les prêtres se soient sanctifiés. Les Lévites, en effet, avaient mis plus d’empressement que les prêtres à se sanctifier. 
${}^{35}Il y eut aussi de nombreux holocaustes, outre les graisses des sacrifices de paix et les libations pour les holocaustes. Ainsi fut rétabli le service de la maison du Seigneur. 
${}^{36}Ézékias et le peuple se réjouirent de ce que Dieu avait réalisé pour le peuple, car cela s’était fait rapidement.
      
         
      \bchapter{}
      \begin{verse}
${}^{1}Ézékias envoya des messagers à tout Israël et Juda – il écrivit même des lettres à Éphraïm et Manassé –, les invitant à venir à la maison du Seigneur à Jérusalem, pour célébrer la Pâque en l’honneur du Seigneur, Dieu d’Israël. 
${}^{2}Le roi, ses princes et toute l’assemblée délibérèrent à Jérusalem pour célébrer la Pâque au deuxième mois. 
${}^{3}En effet, on ne pouvait pas la célébrer en son temps, puisque les prêtres ne s’étaient pas sanctifiés en assez grand nombre, et que le peuple ne s’était pas réuni à Jérusalem. 
${}^{4}Ce projet parut juste aux yeux du roi et de toute l’assemblée. 
${}^{5}On décida de faire passer une proclamation dans tout Israël, depuis Bershéba jusqu’à Dane, pour que l’on vienne célébrer la Pâque à Jérusalem en l’honneur du Seigneur, Dieu d’Israël. Car seul un petit nombre l’avait célébrée selon ce qui est écrit ! 
${}^{6}Les coursiers partirent, avec des lettres écrites de la main du roi et de ses princes, dans tout Israël et Juda, pour dire, selon le commandement du roi : « Fils d’Israël, revenez au Seigneur, Dieu d’Abraham, d’Isaac et d’Israël, afin qu’il revienne, lui, à ceux d’entre vous qui restent, après avoir échappé à la main des rois d’Assour. 
${}^{7}Ne soyez pas comme vos pères, ni comme vos frères, qui ont été infidèles au Seigneur, Dieu de leurs pères : il les a livrés à la désolation, comme vous le voyez. 
${}^{8}Maintenant, ne raidissez pas votre nuque comme vos pères. Tendez la main vers le Seigneur en geste de soumission. Venez à son sanctuaire, qu’il a consacré pour toujours. Servez le Seigneur votre Dieu, pour que se détourne de vous l’ardeur de sa colère. 
${}^{9}Si vous revenez au Seigneur, vos frères et vos fils trouveront miséricorde auprès de ceux qui les ont emmenés captifs, et ils pourront revenir en ce pays ; car le Seigneur votre Dieu est tendre et miséricordieux, et il n’écartera pas de vous sa face, si vous revenez à lui. »
${}^{10}Les coursiers passèrent ainsi de ville en ville, dans le pays d’Éphraïm et de Manassé, jusqu’à Zabulon. Mais on se riait d’eux, on se moquait d’eux. 
${}^{11}Toutefois, quelques hommes d’Asher, de Manassé et de Zabulon s’humilièrent et vinrent à Jérusalem. 
${}^{12}La main de Dieu fut aussi sur Juda, pour leur donner un cœur loyal, afin qu’ils exécutent le commandement du roi et des princes selon la parole du Seigneur. 
${}^{13}Un peuple nombreux se réunit à Jérusalem pour célébrer au deuxième mois la fête des Pains sans levain : ce fut une très nombreuse assemblée. 
${}^{14}Ils se levèrent pour supprimer les autels qui étaient dans Jérusalem ; ils supprimèrent toutes les tables où l’on brûlait l’encens et les jetèrent dans le ravin du Cédron.
${}^{15}Ils immolèrent la Pâque, le quatorzième jour du deuxième mois. Les prêtres et les Lévites, emplis de confusion, s’étaient sanctifiés et avaient fait venir les victimes pour les holocaustes dans la maison du Seigneur. 
${}^{16}Ils se tenaient à leur poste, selon leur coutume, conformément à la loi de Moïse, homme de Dieu. Les prêtres faisaient l’aspersion avec le sang qu’apportaient les Lévites. 
${}^{17}Bon nombre dans l’assemblée ne s’étaient pas sanctifiés. Les Lévites étaient donc chargés, à la place de tous ceux qui n’étaient pas purs, d’immoler les victimes pascales pour accomplir un acte saint envers le Seigneur. 
${}^{18}Or, une grande partie du peuple, beaucoup de gens d’Éphraïm, de Manassé, d’Issakar et de Zabulon, ne s’étaient pas purifiés, en sorte qu’ils mangèrent la Pâque sans se conformer à ce qui est prescrit. Alors Ézékias intercéda pour eux en disant : « Que le Seigneur, lui qui est bon, pardonne à 
${}^{19}tous ceux qui ont appliqué leur cœur à chercher Dieu, le Seigneur, Dieu de leurs pères, bien qu’ils n’aient pas la pureté qui convient aux choses saintes. » 
${}^{20}Le Seigneur exauça Ézékias, il épargna le peuple.
${}^{21}Les fils d’Israël qui se trouvaient à Jérusalem célébrèrent la fête des Pains sans levain pendant sept jours, en grande joie. Jour après jour, les Lévites et les prêtres louaient le Seigneur avec les puissants instruments de musique du Seigneur. 
${}^{22}Ézékias s’adressa au cœur de tous les Lévites qui montraient une grande intelligence dans le service du Seigneur. Ils partagèrent pendant sept jours le repas de la solennité ; ils offraient des sacrifices de paix et célébraient le Seigneur, Dieu de leurs pères. 
${}^{23}Puis toute l’assemblée se mit d’accord pour prolonger de sept jours la célébration, et on fêta les sept jours dans la joie. 
${}^{24}Car Ézékias, roi de Juda, avait prélevé pour l’assemblée mille jeunes taureaux et sept mille moutons, et les princes avaient prélevé pour l’assemblée mille jeunes taureaux et dix mille moutons. Les prêtres, en grand nombre, s’étaient sanctifiés. 
${}^{25}Toute l’assemblée de Juda, les prêtres et les Lévites, et toute l’assemblée venue d’Israël, furent dans la joie, ainsi que les immigrés venus du pays d’Israël et résidant en Juda.
${}^{26}Il y eut grande joie dans Jérusalem, car, depuis le temps de Salomon, fils de David, roi d’Israël, rien de pareil ne s’était produit à Jérusalem. 
${}^{27}Les prêtres lévites se levèrent pour bénir le peuple. Leur voix fut entendue, et leur prière parvint aux cieux, au séjour de la Sainteté.
      
         
      \bchapter{}
      \begin{verse}
${}^{1}Lorsque tout cela fut achevé, tous les fils d’Israël qui étaient présents partirent pour les villes de Juda. Ils brisèrent les stèles, coupèrent les poteaux sacrés, détruisirent complètement les lieux sacrés et les autels dans tout le territoire de Juda et de Benjamin, d’Éphraïm et de Manassé. Ensuite, tous les fils d’Israël retournèrent dans leurs villes, chacun dans sa propriété.
${}^{2}Ézékias établit les classes des prêtres et des Lévites, classe par classe, chacun selon son service, qu’il soit prêtre ou lévite, pour s’occuper des holocaustes et des sacrifices de paix, pour officier, rendre grâce et louer, aux portes des camps du Seigneur. 
${}^{3}Le roi fit don d’une part de ses biens pour les holocaustes, holocaustes du matin et du soir, holocaustes des sabbats, des nouvelles lunes et des solennités, selon ce qui est écrit dans la Loi du Seigneur. 
${}^{4}Il dit au peuple qui habitait Jérusalem de donner la part qui revient aux prêtres et aux Lévites, pour qu’ils s’attachent fermement à la Loi du Seigneur. 
${}^{5}Dès que cette parole se fut répandue, les fils d’Israël offrirent en abondance les prémices du froment, du vin nouveau, de l’huile fraîche, du miel et de tous les produits des champs. Ils apportèrent la dîme de tout, en abondance. 
${}^{6}Les fils d’Israël et de Juda qui habitaient dans les villes de Juda, apportèrent à leur tour la dîme du gros et du petit bétail, ainsi que la dîme des offrandes saintes consacrées au Seigneur leur Dieu. Et ils en firent des tas et des tas. 
${}^{7}Le troisième mois, on commença à former les tas, et le septième mois, tout était terminé. 
${}^{8}Ézékias et les princes vinrent voir tout ce qui avait été entassé, et ils bénirent le Seigneur et son peuple Israël. 
${}^{9}Ézékias interrogea à ce sujetles prêtres et les Lévites. 
${}^{10}Azarias, le chef des prêtres, de la maison de Sadoc, lui dit : « Depuis que l’on a commencé d’apporter les contributions dans la maison du Seigneur, nous avons mangé et nous avons été rassasiés. Il en est resté en abondance, car le Seigneur a béni son peuple, et le reste forme cette grande quantité. »
${}^{11}Ézékias ordonna de préparer des salles dans la maison du Seigneur, et on les prépara. 
${}^{12}On apporta fidèlement les contributions, la dîme et les offrandes saintes. Le lévite Konanias en fut chargé, et son frère Shiméï le secondait. 
${}^{13}Yeïel, Azazyahou, Nahath, Asahel, Yerimoth, Yozabad, Éliël, Yismakya, Mahath et Benaya étaient surveillants, sous l’autorité de Konanias et de son frère Shiméï, par disposition du roi Ézékias et d’Azarias, chef de la maison de Dieu. 
${}^{14}Le lévite Coré, fils de Yimna, gardien de la porte de l’orient, fut préposé aux offrandes volontaires présentées à Dieu, afin de répartir les contributions revenant au Seigneur, et les offrandes très saintes. 
${}^{15}Éden, Minyamine, Josué, Shemayahou, Amaryahou et Shekanyahou l’assistaient fidèlement dans les villes sacerdotales, pour faire la distribution à leurs frères prêtres, grands et petits, selon leurs classes. 
${}^{16}En plus des hommes déjà enregistrés, à partir de trois ans et au-dessus, tous ceux qui entraient dans la maison du Seigneur recevaient chaque jour quelque chose pour leur service, selon leurs fonctions et d’après leurs classes. 
${}^{17}Les prêtres furent enregistrés selon la maison de leurs pères, et les Lévites, à partir de vingt ans et au-dessus, selon leurs fonctions et leurs classes. 
${}^{18}Furent enregistrés aussi tous leurs jeunes enfants, leurs femmes, leurs fils et leurs filles, toute l’assemblée, car en toute fidélité ils s’occupaient saintement des offrandes saintes. 
${}^{19}En ce qui concerne les fils d’Aaron, les prêtres, qui se trouvaient dans les terres à pâturage autour de leurs villes, il y avait dans chaque ville des hommes désignés par leurs noms pour distribuer des parts à chacun des prêtres et à quiconque était enregistré comme lévite.
${}^{20}Voilà ce que fit Ézékias dans tout le pays de Juda. Il fit ce qui est bon, droit et vrai devant le Seigneur son Dieu. 
${}^{21}Dans toute œuvre qu’il entreprit pour le service de la maison de Dieu, pour la Loi et les commandements, cherchant son Dieu, il agit de tout son cœur et il réussit.
      
         
      \bchapter{}
      \begin{verse}
${}^{1}Après ces événements et ces actes de loyauté, Sennakérib, roi d’Assour, se mit en marche et entra en Juda. Il campa contre les villes fortifiées et donna l’ordre d’en forcer les remparts. 
${}^{2}Quand Ézékias vit que Sennakérib arrivait avec l’intention d’attaquer Jérusalem, 
${}^{3}il tint conseil avec ses officiers et ses vaillants guerriers pour obstruer les eaux des sources qui étaient en dehors de la ville, et ils lui apportèrent leur aide. 
${}^{4}Une foule nombreuse se rassembla, et l’on obstrua toutes les sources, ainsi que le torrent qui coulait au milieu du pays. On disait : « Pourquoi les rois d’Assour trouveraient-ils, en arrivant, de l’eau en abondance ? » 
${}^{5}Ézékias se mit courageusement à rebâtir tout le rempart qui était en ruine et à restaurer les tours. Il édifia un second mur à l’extérieur, fortifia le Terre-Plein de la Cité de David, et fabriqua en quantité des javelots et des boucliers. 
${}^{6}Il plaça les officiers de l’armée à la tête du peuple, les réunit près de lui sur la place à la porte de la ville, et, s’adressant à leur cœur, il leur dit : 
${}^{7}« Soyez forts et courageux ! Ne craignez pas, ne vous effrayez pas devant le roi d’Assour, ni devant la multitude qui est avec lui. Car il y a plus grand avec nous qu’avec lui : 
${}^{8}avec lui, une force humaine ; avec nous, le Seigneur notre Dieu, pour nous secourir et mener nos combats. » Le peuple trouva soutien dans les paroles d’Ézékias, roi de Juda.
      
         
${}^{9}Après cela, Sennakérib, roi d’Assour, qui se trouvait lui-même, avec toute son armée, devant la ville de Lakish, envoya ses serviteurs à Jérusalem vers Ézékias, roi de Juda, et vers tous les habitants de Juda qui étaient à Jérusalem, pour leur dire : 
${}^{10}« Ainsi parle Sennakérib, roi d’Assour : En quoi mettez-vous votre confiance, vous qui êtes assiégés dans Jérusalem ? 
${}^{11}Ézékias n’est-il pas en train de vous duper, pour vous faire mourir de faim et de soif, lorsqu’il dit : “Le Seigneur notre Dieu nous délivrera de la main du roi d’Assour” ? 
${}^{12}N’est-ce pas lui, Ézékias, qui a supprimé ses lieux sacrés et ses autels, en disant aux gens de Juda et de Jérusalem ces mots : “C’est devant un seul autel que vous vous prosternerez et que vous brûlerez de l’encens” ? 
${}^{13}Ne savez-vous pas ce que nous avons fait, moi et mes pères, à tous les peuples des autres pays ? Les dieux des nations ont-ils été capables de délivrer leur pays de ma main ? 
${}^{14}De tous les dieux de ces nations que mes pères ont voués à l’anathème, quel est celui qui a pu délivrer son peuple de ma main, pour que votre dieu puisse vous délivrer de ma main ? 
${}^{15}Et maintenant, qu’Ézékias ne vous trompe pas et ne vous dupe pas ! Ne vous fiez pas à lui, car aucun dieu d’aucune nation ni d’aucun royaume ne peut délivrer son peuple de ma main, ni de la main de mes pères ! À plus forte raison, vos dieux ne vous délivreront pas de ma main ! »
${}^{16}Les serviteurs de Sennakérib parlèrent encore contre le Seigneur Dieu et contre Ézékias, son serviteur. 
${}^{17}Sennakérib écrivit également des lettres pour insulter le Seigneur, Dieu d’Israël. Voilà ce qu’il disait contre lui : « De même que les dieux des nations des pays n’ont pu délivrer leur peuple de ma main, de même le dieu d’Ézékias ne délivrera pas son peuple de ma main. » 
${}^{18}Les serviteurs de Sennakérib crièrent cela d’une voix forte, en judéen, au peuple de Jérusalem qui était sur le rempart, pour lui faire peur et l’épouvanter, afin de s’emparer de la ville. 
${}^{19}Ils parlaient du Dieu de Jérusalem comme des dieux des autres peuples de la terre, qui sont ouvrage de mains humaines.
${}^{20}Dans cette situation, le roi Ézékias et le prophète Isaïe, fils d’Amots, se mirent en prière et supplièrent le ciel. 
${}^{21}Alors le Seigneur envoya un ange qui anéantit tous les vaillants guerriers, les chefs et les officiers, dans le camp du roi d’Assour. Celui-ci retourna dans son pays, la honte au visage. Il entra dans la maison de son dieu, et quelques-uns de ses propres enfants le mirent à mort par l’épée. 
${}^{22}Ainsi, le Seigneur sauva Ézékias et les habitants de Jérusalem de la main de Sennakérib, roi d’Assour, et de la main de tous les ennemis : il leur assura la tranquillité de tous côtés. 
${}^{23}Beaucoup de gens apportèrent à Jérusalem des offrandes pour le Seigneur, et de riches présents à Ézékias, roi de Juda. Désormais, son prestige grandit aux yeux de toutes les nations.
${}^{24}En ces jours-là, Ézékias fut atteint d’une maladie mortelle. Il pria le Seigneur qui lui parla et lui accorda un prodige. 
${}^{25}Mais Ézékias ne répondit pas au bienfait reçu, car son cœur s’était enflé d’orgueil. Aussi la colère du Seigneur fut-elle sur lui, sur Juda et Jérusalem. 
${}^{26}Alors Ézékias, dont le cœur était orgueilleux, s’humilia, et les habitants de Jérusalem avec lui. Et la colère du Seigneur ne vint pas sur eux pendant la vie d’Ézékias. 
${}^{27}Ézékias eut richesse et gloire en très grande abondance. Il amassa des trésors : argent, or, pierres précieuses, aromates, boucliers, et toute sorte d’objets précieux. 
${}^{28}Il eut des entrepôts pour ses provisions de froment, de vin nouveau et d’huile fraîche, des étables pour toute espèce de bétail, et des troupeaux pour ses parcs. 
${}^{29}Il se construisit des villes, il eut de nombreux troupeaux de petit et de gros bétail, car Dieu lui avait donné de très grands biens.
${}^{30}C’est lui, Ézékias, qui obstrua la sortie supérieure des eaux du Guihone, et les détourna en bas vers l’ouest de la Cité de David. Ézékias réussit dans toutes ses entreprises. 
${}^{31}Et même, face aux ambassadeurs que les princes de Babylone lui envoyèrent afin de s’informer du prodige qui avait eu lieu dans le pays, c’est pour le mettre à l’épreuve que Dieu l’abandonna, pour connaître le fond de son cœur.
${}^{32}Le reste des actions d’Ézékias, ce qu’il a fait avec fidélité,
        \\voici que cela est écrit dans la Vision du prophète Isaïe, fils d’Amots,
        \\et dans le livre des Annales des rois de Juda et d’Israël.
${}^{33}Ézékias reposa avec ses pères,
        \\et on l’ensevelit dans la partie haute
        \\des tombeaux des fils de David.
        \\À sa mort, tous les gens de Juda
        \\et les habitants de Jérusalem lui rendirent hommage.
        \\Son fils Manassé régna à sa place.
      
         
      \bchapter{}
      \begin{verse}
${}^{1}Manassé avait douze ans lorsqu’il devint roi, et il régna cinquante-cinq ans à Jérusalem. 
${}^{2}Il fit ce qui est mal aux yeux du Seigneur, selon les coutumes abominables des nations que le Seigneur avait dépossédées devant les fils d’Israël. 
${}^{3}Il rebâtit les lieux sacrés qu’avait détruits Ézékias, son père, et il fit élever des autels aux Baals. Il fabriqua des poteaux sacrés ; il se prosterna devant toute l’armée des cieux et s’en fit le serviteur. 
${}^{4}Il bâtit des autels dans la maison du Seigneur, alors que le Seigneur avait dit : « Dans Jérusalem, mon nom sera à jamais. »
${}^{5}Manassé bâtit aussi des autels à toute l’armée des cieux, dans les deux cours de la maison du Seigneur. 
${}^{6}C’est lui qui fit passer ses fils par le feu, dans la vallée de Ben-Hinnome. Il pratiqua divination, incantation et enchantement, il interrogea les spectres et les esprits. Il fit de maintes façons ce qui est mal aux yeux du Seigneur, pour provoquer son indignation. 
${}^{7}Il plaça dans la maison de Dieu la statue de l’idole qu’il avait faite. Or, Dieu avait dit à David et à son fils Salomon : « Dans cette Maison, et dans Jérusalem que j’ai choisie parmi toutes les tribus d’Israël, je mettrai mon nom à jamais. 
${}^{8}Je ne ferai plus s’écarter les pas d’Israël loin de la terre que j’ai destinée à vos pères, pourvu qu’ils veillent à faire tout ce que je leur ai ordonné, selon toute la Loi, les décrets et les ordonnances, transmis par Moïse. » 
${}^{9}Manassé égara les gens de Juda et les habitants de Jérusalem, de sorte qu’ils firent le mal, plus encore que les nations que le Seigneur avait anéanties devant les fils d’Israël. 
${}^{10}Le Seigneur s’adressa à Manassé et à son peuple, mais ils n’y prêtèrent pas attention.
${}^{11}Alors le Seigneur fit venir contre eux les officiers de l’armée du roi d’Assour. Ils prirent Manassé avec des crochets, l’attachèrent avec une double chaîne de bronze et l’emmenèrent à Babylone. 
${}^{12}Dans son angoisse, Manassé apaisa le visage du Seigneur son Dieu en s’humiliant profondément devant le Dieu de ses pères. 
${}^{13}Il le pria ; le Seigneur l’exauça, entendit sa supplication : il le fit revenir à Jérusalem dans son royaume. Et Manassé reconnut que le Seigneur est Dieu. 
${}^{14}Après cela, il bâtit un rempart à l’extérieur de la Cité de David, à l’ouest, vers Guihone, dans la vallée, jusqu’à l’entrée de la porte des Poissons, de manière à entourer l’Ophel ; ce rempart, il le fit très élevé. Il plaça des officiers de l’armée dans toutes les villes fortifiées de Juda.
${}^{15}Il supprima de la maison du Seigneur les dieux étrangers et l’idole, ainsi que tous les autels qu’il avait bâtis sur la montagne de la maison du Seigneur et dans Jérusalem, et les jeta hors de la ville. 
${}^{16}Il restaura l’autel du Seigneur, il y offrit des sacrifices de paix et d’action de grâce ; puis il ordonna aux gens de Juda de servir le Seigneur, Dieu d’Israël. 
${}^{17}Toutefois, le peuple sacrifiait encore dans les lieux sacrés, mais seulement au Seigneur son Dieu.
${}^{18}Le reste des actions de Manassé,
        \\la prière qu’il adressa à son Dieu,
        \\et les paroles des voyants qui lui parlèrent
        \\au nom du Seigneur, Dieu d’Israël,
        \\cela se trouve dans les Actes des rois d’Israël.
${}^{19}Sa prière, et comment il fut exaucé,
        \\tous ses péchés et ses infidélités,
        \\les endroits où il bâtit des lieux sacrés,
        \\et dressa des poteaux sacrés et des idoles
        \\avant de s’être humilié,
        \\voici que cela est écrit dans les Actes de Hozaï.
${}^{20}Manassé reposa avec ses pères,
        \\et on l’ensevelit dans sa maison.
        \\Son fils Amone régna à sa place.
${}^{21}Amone avait vingt-deux ans lorsqu’il devint roi, et il régna deux ans à Jérusalem. 
${}^{22}Il fit ce qui est mal aux yeux du Seigneur, comme avait fait Manassé, son père. Amone offrit des sacrifices à toutes les idoles qu’avait fabriquées Manassé, son père, et il les servit. 
${}^{23}Il ne s’humilia pas devant le Seigneur, comme l’avait fait Manassé, son père. En effet, lui, Amone, multiplia les offenses. 
${}^{24}Ses serviteurs complotèrent contre lui ; ils le mirent à mort dans sa maison. 
${}^{25}Mais les gens du pays frappèrent tous ceux qui avaient comploté contre le roi Amone, et ce sont les gens du pays qui firent roi son fils Josias à sa place.
      
         
      \bchapter{}
      \begin{verse}
${}^{1}Josias avait huit ans lorsqu’il devint roi, et il régna trente et un ans à Jérusalem. 
${}^{2}Il fit ce qui est droit aux yeux du Seigneur, et il marcha sur les chemins de David, son ancêtre ; il ne s’en écarta ni à droite ni à gauche.
${}^{3}La huitième année de son règne, alors qu’il était encore un jeune homme, il commença à chercher le Dieu de David, son ancêtre. La douzième année, il commença à purifier le pays de Juda et Jérusalem des lieux sacrés, des poteaux sacrés, des idoles sculptées ou fondues. 
${}^{4}On détruisit en sa présence les autels des Baals. Les colonnes à encens qui étaient en haut, au-dessus d’eux, il les abattit ; les poteaux sacrés et les idoles sculptées ou fondues, il les brisa, les réduisit en poussière qu’il répandit ensuite sur les tombeaux de ceux qui leur avaient offert des sacrifices. 
${}^{5}Il brûla les ossements des prêtres sur leurs autels, et purifia le pays de Juda et Jérusalem. 
${}^{6}Dans les villes de Manassé, d’Éphraïm, de Siméon, et jusqu’en Nephtali, sur toutes les places, 
${}^{7}il détruisit les autels et les poteaux sacrés ; les idoles, il les brisa et les réduisit en poussière ; toutes les colonnes à encens, il les abattit dans tout le pays d’Israël. Et il revint à Jérusalem.
${}^{8}La dix-huitième année de son règne, après avoir purifié le pays et la maison du Seigneur, Josias envoya Shafane, fils d’Açalyahou, Maaséyahou, gouverneur de la ville, et Yoah, fils de Yoahaz, archiviste, pour réparer la maison du Seigneur son Dieu. 
${}^{9}Ils se rendirent chez Helcias, le grand-prêtre, et lui donnèrent l’argent apporté à la maison de Dieu, celui que les Lévites, gardiens du seuil, avaient recueilli de Manassé, d’Éphraïm, et de tout le reste d’Israël, de tout Juda et Benjamin, et des habitants de Jérusalem. 
${}^{10}On le remit entre les mains des maîtres d’œuvre, préposés à la maison du Seigneur, on le remit à ces derniers, qui travaillaient à la maison du Seigneur, pour consolider et réparer la Maison. 
${}^{11}Ceux-ci le remirent aux charpentiers et aux ouvriers du bâtiment, pour acheter des pierres de taille et du bois pour les assemblages, et refaire la charpente des bâtiments qu’avaient endommagés les rois de Juda.
${}^{12}Ces hommes accomplissaient leur travail avec honnêteté. Ils étaient sous la surveillance de Yahath et d’Obadyahou, Lévites d’entre les fils de Merari. Zacharie et Meshoullam, d’entre les fils de Qehath, étaient chargés de les diriger, ainsi que tous les Lévites habiles à jouer des instruments de musique. 
${}^{13}Ils surveillaient les porteurs et dirigeaient tous les artisans dans les divers services. D’autres Lévites étaient secrétaires, scribes et portiers.
${}^{14}Comme on retirait l’argent apporté à la maison du Seigneur, le prêtre Helcias trouva le livre de la Loi du Seigneur transmise par Moïse. 
${}^{15}Helcias prit la parole et dit au secrétaire Shafane : « J’ai trouvé le livre de la Loi dans la maison du Seigneur. » Et Helcias donna le livre à Shafane. 
${}^{16}Shafane porta le livre au roi et, de plus, lui rendit compte de ce qui s’était passé : « Tout ce qui a été confié à tes serviteurs, ils l’ont fait. 
${}^{17}L’argent qui se trouvait dans la maison du Seigneur, ils l’ont versé et remis entre les mains des préposés et des maîtres d’œuvre. » 
${}^{18}Shafane, le secrétaire, annonça au roi : « Le prêtre Helcias m’a donné un livre. » Et Shafane fit au roi la lecture de ce livre.
${}^{19}Après avoir entendu les paroles de la Loi, le roi déchira ses vêtements. 
${}^{20}Il donna cet ordre à Helcias, à Ahiqam, fils de Shafane, à Abdone, fils de Mika, au secrétaire Shafane, ainsi qu’à Asaya, serviteur du roi : 
${}^{21}« Allez consulter le Seigneur, pour moi et pour ce qui reste d’Israël et de Juda, au sujet des paroles du livre qu’on vient de retrouver. La fureur du Seigneur est grande : elle s’est déversée sur nous, parce que nos pères n’ont pas observé la parole du Seigneur et n’ont pas pratiqué tout ce qui est écrit dans ce livre. »
${}^{22}Alors Helcias et ceux que le roi avait désignés allèrent chez la prophétesse Houlda, femme du gardien des vêtements Shalloum, fils de Toqhath, fils de Hasra. Elle habitait à Jérusalem, dans la ville nouvelle. Quand ils lui eurent parlé de cette affaire, 
${}^{23}elle leur dit : « Ainsi parle le Seigneur, Dieu d’Israël : Dites à l’homme qui vous a envoyés vers moi : 
${}^{24}“Ainsi parle le Seigneur : Moi, je vais faire venir un malheur sur ce lieu et sur ses habitants, accomplissant ainsi toutes les malédictions écrites dans le livre qu’on a lu en présence du roi de Juda. 
${}^{25}Parce qu’ils m’ont abandonné, et qu’ils ont brûlé de l’encens pour d’autres dieux, afin de provoquer mon indignation par toutes les œuvres de leurs mains, ma fureur s’est déversée sur ce lieu et ne s’éteindra plus !” 
${}^{26}Mais au roi de Juda qui vous a envoyés consulter le Seigneur, vous direz : “Ainsi parle le Seigneur, Dieu d’Israël : Ces paroles, tu les as entendues : 
${}^{27}puisque ton cœur s’est attendri et que tu t’es humilié devant Dieu, quand tu as entendu ses paroles contre ce lieu et ses habitants, puisque tu t’es humilié devant moi, que tu as déchiré tes vêtements et pleuré devant moi, eh bien ! moi aussi, j’ai entendu – oracle du Seigneur. 
${}^{28}Moi, je te réunirai à tes pères, tu seras ramené en paix dans leurs tombeaux ; tes yeux ne verront rien de tout le malheur que je fais venir sur ce lieu et sur ses habitants.” » Helcias et ses compagnons rapportèrent au roi la réponse.
${}^{29}Le roi fit convoquer tous les anciens de Juda et de Jérusalem, 
${}^{30}et il monta à la maison du Seigneur avec tous les gens de Juda, les habitants de Jérusalem, les prêtres et les Lévites, et tout le peuple, du plus grand au plus petit. Il lut devant eux toutes les paroles du livre de l’Alliance retrouvé dans la maison du Seigneur. 
${}^{31}Le roi était debout, à sa place, et il conclut l’Alliance en présence du Seigneur, afin de suivre le Seigneur, d’observer ses commandements, ses édits et ses décrets, de tout son cœur et de toute son âme, et d’accomplir les paroles de l’Alliance inscrites dans ce Livre. 
${}^{32}Il fit s’engager tous ceux qui se trouvaient à Jérusalem et en Benjamin ; et les habitants de Jérusalem agirent conformément à l’alliance de Dieu, le Dieu de leurs pères. 
${}^{33}Josias supprima toutes les abominations de tous les territoires appartenant aux fils d’Israël, et il obligea tous ceux qui se trouvaient en Israël à servir le Seigneur, leur Dieu. Tant qu’il vécut, ils ne s’écartèrent pas du Seigneur, le Dieu de leurs pères.
      
         
      \bchapter{}
      \begin{verse}
${}^{1}Josias célébra à Jérusalem une Pâque en l’honneur du Seigneur ; on immola la Pâque le quatorzième jour du premier mois. 
${}^{2}Il établit les prêtres dans leurs fonctions et les encouragea pour le service de la maison du Seigneur. 
${}^{3}Il dit aux Lévites qui enseignaient tout Israël, eux qui étaient consacrés au Seigneur : « Placez l’Arche sainte dans la Maison que Salomon, fils de David, roi d’Israël, a construite : vous n’avez plus à la porter sur vos épaules. Maintenant, servez le Seigneur votre Dieu et son peuple Israël. 
${}^{4}Disposez-vous par familles, selon vos classes, comme l’a écrit David, roi d’Israël, et comme l’a prescrit Salomon, son fils. 
${}^{5}Tenez-vous dans le sanctuaire, selon les branches des familles, pour vos frères, les gens du peuple, et selon la répartition des familles des Lévites. 
${}^{6}Immolez la Pâque, sanctifiez-vous et préparez-la pour vos frères, afin qu’ils puissent la célébrer en se conformant à la parole du Seigneur transmise par Moïse. »
${}^{7}Josias préleva pour les gens du peuple, pour tous ceux qui étaient là, du petit bétail, agneaux et chevreaux, tout cela en victimes pour la Pâque, soit trente mille bêtes, ainsi que trois mille bœufs ; l’ensemble provenait du cheptel du roi. 
${}^{8}Ses princes prélevèrent des offrandes volontaires au profit du peuple, des prêtres et des Lévites. Helcias, Zacharie et Yeïel, chefs de la maison de Dieu, donnèrent aux prêtres, en victimes pour la Pâque, deux mille six cents agneaux et trois cents bœufs. 
${}^{9}Konanias, et ses frères Shemayahou et Nathanaël, Hashabyahou, Yeïel et Yozabad, chefs des Lévites, prélevèrent au profit des Lévites, en victimes pour la Pâque, cinq mille agneaux et cinq cents bœufs. 
${}^{10}Le service fut ainsi organisé : les prêtres se tenaient à leur place, de même les Lévites, en fonction de leurs classes, selon le commandement du roi. 
${}^{11}On immola la Pâque, les prêtres firent l’aspersion avec le sang reçu de la main des Lévites, et ceux-ci écorchaient les victimes. 
${}^{12}Les Lévites mirent à part les victimes destinées à l’holocauste, afin de les remettre aux gens du peuple, selon les branches des familles, pour qu’ils les présentent au Seigneur, comme cela est écrit dans le livre de Moïse. On fit de même pour les bœufs. 
${}^{13}Ils firent rôtir la Pâque, comme cela est ordonné. Ils firent cuire les autres viandes sacrées dans des chaudrons, des pots et des plats, et s’empressèrent de les porter à tous les gens du peuple. 
${}^{14}Ils firent ensuite les préparatifs pour eux-mêmes et pour les prêtres ; car les prêtres, descendants d’Aaron, furent occupés jusqu’à la nuit à offrir l’holocauste et les graisses. Les Lévites firent donc les préparatifs pour eux-mêmes et pour les prêtres, descendants d’Aaron. 
${}^{15}Les chantres, fils d’Asaf, étaient à leur place, selon le commandement de David, d’Asaf, de Hémane et de Yedoutoune, le voyant du roi, et les portiers se tenaient à chaque porte. Aucun d’entre eux n’avait à interrompre son service, puisque leurs frères Lévites faisaient pour eux les préparatifs.
${}^{16}C’est ainsi que, ce jour-là, fut organisé tout le service du Seigneur, en vue de célébrer la Pâque et d’offrir des holocaustes sur l’autel du Seigneur, selon le commandement du roi Josias. 
${}^{17}Les fils d’Israël qui étaient présents célébrèrent la Pâque à ce moment-là, ainsi que la fête des Pains sans levain, pendant sept jours. 
${}^{18}On n’avait pas célébré de Pâque comme celle-là en Israël depuis le temps du prophète Samuel, et aucun des rois d’Israël n’avait célébré une Pâque semblable à celle que célébrèrent Josias, les prêtres et les Lévites, tous les gens de Juda et d’Israël qui étaient présents, et les habitants de Jérusalem. 
${}^{19}C’est la dix-huitième année du règne de Josias que l’on célébra cette Pâque.
${}^{20}Après tous ces événements, lorsque Josias eut réparé la maison du Seigneur, Nékao, roi d’Égypte, monta pour combattre à Karkémish, sur l’Euphrate, et Josias marcha à sa rencontre. 
${}^{21}Nékao lui envoya des messagers pour lui dire : « Que me veux-tu, roi de Juda ? Ce n’est pas contre toi que je marche aujourd’hui, mais contre une autre maison royale avec laquelle je suis en guerre, et Dieu m’a dit de me hâter. Laisse donc agir Dieu qui est avec moi, de peur qu’il ne te détruise ! » 
${}^{22}Mais Josias ne changea pas d’avis ; bien plus, il se déguisa pour l’attaquer. Il n’écouta pas les paroles de Nékao, qui venaient de la bouche de Dieu. Il vint livrer bataille dans la plaine de Meguiddo. 
${}^{23}Les archers tirèrent sur le roi Josias, qui dit à ses serviteurs : « Emportez-moi, car je suis gravement blessé. » 
${}^{24}Ses serviteurs le sortirent de son char, le placèrent sur un autre de ses chars et l’emmenèrent à Jérusalem, où il mourut. On l’ensevelit dans les tombeaux de ses pères ; tout Juda et Jérusalem prirent le deuil de Josias. 
${}^{25}Jérémie composa une lamentation sur Josias. Tous les chanteurs et les chanteuses ont mentionné Josias dans leurs lamentations, jusqu’à ce jour. On en fit une règle en Israël, et ces chants sont consignés dans les lamentations.
${}^{26}Le reste des actions de Josias,
        \\ce qu’il a fait avec fidélité,
        \\conformément à ce qui est écrit dans la Loi du Seigneur,
${}^{27}ses actions donc, des premières aux dernières,
        \\voilà qu’elles sont écrites dans le Livre des rois d’Israël et de Juda.
      
         
      \bchapter{}
      \begin{verse}
${}^{1}Les gens du pays prirent alors Joakaz, fils de Josias, et le firent roi à la place de son père, à Jérusalem. 
${}^{2}Joakaz avait vingt-trois ans lorsqu’il devint roi, et il régna trois mois à Jérusalem. 
${}^{3}Le roi d’Égypte le destitua à Jérusalem et imposa au pays un tribut de cent talents d’argent et d’un talent d’or. 
${}^{4}Le roi d’Égypte fit roi Élyakim, frère de Joakaz, sur Juda et Jérusalem, et il changea son nom en celui de Joakim. Quant à son frère Joakaz, Nékao le prit et l’emmena en Égypte.
      
         
${}^{5}Joakim avait vingt-cinq ans lorsqu’il devint roi, et il régna onze ans à Jérusalem. Il fit ce qui est mal aux yeux du Seigneur son Dieu. 
${}^{6}Nabucodonosor, roi de Babylone, monta contre lui et l’enchaîna d’une double chaîne de bronze pour l’emmener à Babylone. 
${}^{7}Nabucodonosor emporta à Babylone une partie des objets de la maison du Seigneur et les déposa dans son palais, à Babylone.
${}^{8}Le reste des actions de Joakim,
        \\les abominations qu’il commit et ce qui lui est arrivé,
        \\voici que cela est écrit dans le Livre des rois d’Israël et de Juda.
        \\Son fils Jékonias régna à sa place.
${}^{9}Jékonias avait huit ans lorsqu’il devint roi, et il régna trois mois et dix jours à Jérusalem. Il fit ce qui est mal aux yeux du Seigneur. 
${}^{10}Au retour du printemps, le roi Nabucodonosor l’envoya chercher et le fit emmener à Babylone, avec les objets précieux de la maison du Seigneur. Il fit roi, sur Juda et sur Jérusalem, Sédécias, frère de Jékonias.
${}^{11}Sédécias avait vingt et un ans lorsqu’il devint roi, et il régna onze ans à Jérusalem. 
${}^{12}Il fit ce qui est mal aux yeux du Seigneur son Dieu, et ne s’humilia pas devant le prophète Jérémie, qui parlait au nom du Seigneur. 
${}^{13}Il se révolta même contre le roi Nabucodonosor qui lui avait fait prêter serment par Dieu. Il raidit sa nuque et endurcit son cœur, plutôt que de revenir au Seigneur, Dieu d’Israël.
${}^{14}Tous les chefs des prêtres et du peuple multipliaient les infidélités, en imitant toutes les abominations des nations païennes, et ils profanaient la Maison que le Seigneur avait consacrée à Jérusalem. 
${}^{15} Le Seigneur, le Dieu de leurs pères, sans attendre et sans se lasser, leur envoyait des messagers, car il avait pitié de son peuple et de sa Demeure. 
${}^{16} Mais eux tournaient en dérision les envoyés de Dieu, méprisaient ses paroles, et se moquaient de ses prophètes ; finalement, il n’y eut plus de remède à la fureur grandissante du Seigneur contre son peuple\\.
${}^{17}Alors le Seigneur fit monter contre eux le roi des Chaldéens, qui tua par l’épée les jeunes gens à l’intérieur du sanctuaire, n’épargna ni le jeune homme ni la jeune fille, ni le vieillard ni l’infirme : le Seigneur les livra tous entre ses mains. 
${}^{18}Tous les objets, grands ou petits, de la maison de Dieu, les trésors de la maison du Seigneur et les trésors du roi et de ses princes, Nabucodonosor emporta tout cela à Babylone. 
${}^{19}Les Babyloniens brûlèrent la maison de Dieu, détruisirent le rempart de Jérusalem, incendièrent tous ses palais, et réduisirent à rien tous leurs objets précieux. 
${}^{20}Nabucodonosor déporta à Babylone ceux qui avaient échappé au massacre\\ ; ils devinrent les esclaves du roi et de ses fils jusqu’au temps de la domination des Perses. 
${}^{21}Ainsi s’accomplit la parole du Seigneur proclamée par Jérémie :\\« La terre sera dévastée et elle se reposera durant soixante-dix ans, jusqu’à ce qu’elle ait compensé par ce repos tous les sabbats profanés\\. »
${}^{22}Or, la première année du règne\\de Cyrus, roi de Perse, pour que soit accomplie la parole du Seigneur proclamée par Jérémie, le Seigneur inspira Cyrus, roi de Perse. Et celui-ci fit publier dans tout son royaume – et même consigner par écrit – : 
${}^{23} « Ainsi parle Cyrus, roi de Perse : Le Seigneur, le Dieu du ciel, m’a donné tous les royaumes de la terre ; et il m’a chargé de lui bâtir une maison à Jérusalem, en Juda. Quiconque parmi vous fait partie de\\son peuple, que le Seigneur son Dieu soit avec lui, et qu’il monte à Jérusalem\\ ! »
