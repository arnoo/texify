  
  
    
    <h2 class="title">LETTRE<br/>
    AUX GALATES</h2>
      
         
      \bchapter{}
        ${}^{1}Paul, apôtre,
        \\– envoyé non par des hommes,
        ni par l’intermédiaire d’un homme,
        mais par Jésus Christ
        et par Dieu le Père qui l’a ressuscité d’entre les morts, –
        ${}^{2}ainsi que tous les frères qui sont avec moi :
        \\aux Églises du pays galate.
${}^{3}À vous, la grâce et la paix
        \\de la part de Dieu notre Père
        et du Seigneur Jésus Christ,
${}^{4}qui s’est donné pour nos péchés,
        \\afin de nous arracher à ce monde mauvais,
        selon la volonté de Dieu notre Père,
${}^{5}à qui soit la gloire pour les siècles des siècles. Amen.
        
           
${}^{6}Je m’étonne que vous abandonniez si vite celui qui vous a appelés par la grâce du Christ, et que vous passiez à un Évangile différent. 
${}^{7}Ce n'en est pas un autre : il y a seulement des gens qui jettent le trouble parmi vous et qui veulent changer l’Évangile du Christ. 
${}^{8}Pourtant, si nous-mêmes, ou si un ange du ciel vous annonçait un Évangile différent de celui que nous vous avons annoncé, qu’il soit anathème ! 
${}^{9}Nous l’avons déjà dit, et je le répète encore : si quelqu’un vous annonce un Évangile différent de celui que vous avez reçu, qu’il soit anathème ! 
${}^{10}Maintenant, est-ce par des hommes ou par Dieu que je veux me faire approuver ? Est-ce donc à des hommes que je cherche à plaire ? Si j’en étais encore à plaire à des hommes, je ne serais pas serviteur du Christ.
${}^{11}Frères, je tiens à ce que vous le sachiez, l’Évangile que j’ai proclamé n’est pas une invention humaine. 
${}^{12}Ce n’est pas non plus d’un homme que je l’ai reçu ou appris, mais par révélation de Jésus Christ.
${}^{13}Vous avez entendu parler du comportement que j’avais autrefois dans le judaïsme : je menais une persécution effrénée contre l’Église de Dieu, et je cherchais à la détruire. 
${}^{14}J’allais plus loin dans le judaïsme que la plupart de mes frères de race qui avaient mon âge, et, plus que les autres, je défendais avec une ardeur jalouse les traditions de mes pères. 
${}^{15}Mais Dieu m’avait mis à part dès le sein de ma mère ; dans sa grâce, il m’a appelé ; et il a trouvé bon 
${}^{16}de révéler en moi son Fils, pour que je l’annonce parmi les nations païennes. Aussitôt, sans prendre l'avis de personne, 
${}^{17}sans même monter à Jérusalem pour y rencontrer ceux qui étaient Apôtres avant moi, je suis parti pour l’Arabie et, de là, je suis retourné à Damas. 
${}^{18}Puis, trois ans après, je suis monté à Jérusalem pour faire la connaissance de Pierre, et je suis resté quinze jours auprès de lui. 
${}^{19}Je n’ai vu aucun des autres Apôtres sauf Jacques, le frère du Seigneur. 
${}^{20}En vous écrivant cela, – je le déclare devant Dieu – je ne mens pas. 
${}^{21}Ensuite, je me suis rendu dans les régions de Syrie et de Cilicie. 
${}^{22}Mais pour les Églises de Judée qui sont dans le Christ, mon visage restait inconnu ; 
${}^{23}elles avaient simplement entendu dire : « Celui qui nous persécutait naguère annonce aujourd’hui la foi qu’il cherchait alors à détruire. » 
${}^{24}Et l’on rendait gloire à Dieu à mon sujet.
      
         
      \bchapter{}
      \begin{verse}
${}^{1}Puis, au bout de quatorze ans, je suis de nouveau monté à Jérusalem ; j’étais avec Barnabé, et j’avais aussi emmené Tite. 
${}^{2}J’y montais à la suite d’une révélation, et j’y ai exposé l’Évangile que je proclame parmi les nations ; je l’ai exposé en privé, aux personnages les plus importants, car je ne voulais pas risquer de courir ou d’avoir couru pour rien. 
${}^{3}Eh bien ! Tite, mon compagnon, qui est grec, n’a même pas été obligé de se faire circoncire. 
${}^{4}Il y avait pourtant les faux frères, ces intrus, qui s’étaient infiltrés comme des espions pour voir quelle liberté nous avons dans le Christ Jésus, leur but étant de nous réduire en esclavage ; 
${}^{5}mais, pas un seul instant, nous n’avons accepté de nous soumettre à eux, afin de maintenir pour vous la vérité de l’Évangile. 
${}^{6}Quant à ceux qui étaient tenus pour importants – mais ce qu’ils étaient alors ne compte guère pour moi, car Dieu est impartial envers les personnes –, ces gens importants ne m’ont imposé aucune obligation supplémentaire, 
${}^{7}mais au contraire, ils ont constaté que l’annonce de l’Évangile m’a été confiée pour les incirconcis (c’est-à-dire les païens), comme elle l’a été à Pierre pour les circoncis (c’est-à-dire les Juifs). 
${}^{8}En effet, si l’action de Dieu a fait de Pierre l’Apôtre des circoncis, elle a fait de moi l’Apôtre des nations païennes. 
${}^{9}Ayant reconnu la grâce qui m’a été donnée, Jacques, Pierre et Jean, qui sont considérés comme les colonnes de l’Église, nous ont tendu la main, à moi et à Barnabé, en signe de communion, montrant par là que nous sommes, nous, envoyés aux nations, et eux, aux circoncis. 
${}^{10}Ils nous ont seulement demandé de nous souvenir des pauvres, ce que j’ai pris grand soin de faire.
      
         
${}^{11}Mais quand Pierre est venu à Antioche, je me suis opposé à lui ouvertement, parce qu’il était dans son tort. 
${}^{12}En effet, avant l’arrivée de quelques personnes de l’entourage de Jacques, Pierre prenait ses repas avec les fidèles d’origine païenne. Mais après leur arrivée, il prit l’habitude de se retirer et de se tenir à l’écart, par crainte de ceux qui étaient d’origine juive. 
${}^{13}Tous les autres fidèles d’origine juive jouèrent la même comédie que lui, si bien que Barnabé lui-même se laissa entraîner dans ce jeu. 
${}^{14}Mais quand je vis que ceux-ci ne marchaient pas droit selon la vérité de l’Évangile, je dis à Pierre devant tout le monde : « Si toi qui es juif, tu vis à la manière des païens et non des Juifs, pourquoi obliges-tu les païens à suivre les coutumes juives ? »
${}^{15}Nous, nous sommes des Juifs de naissance, et non pas de ces pécheurs d’origine païenne. 
${}^{16}Cependant, nous avons reconnu que ce n’est pas en pratiquant la loi de Moïse que l’homme devient juste devant Dieu, mais seulement par la foi en Jésus Christ ; c’est pourquoi nous avons cru, nous aussi, au Christ Jésus pour devenir des justes par la foi au Christ, et non par la pratique de la Loi, puisque, par la pratique de la Loi, personne ne deviendra juste. 
${}^{17}S’il était vrai qu’en cherchant à devenir des justes grâce au Christ, nous avons été trouvés pécheurs, nous aussi, cela ne voudrait-il pas dire que le Christ est au service du péché ? Il n’en est rien, bien sûr ! 
${}^{18}Si maintenant je revenais à la Loi que j’ai rejetée, reconstruisant ainsi ce que j’ai démoli, j’attesterais que j’ai eu tort de la rejeter. 
${}^{19}Par la Loi, je suis mort à la Loi afin de vivre pour Dieu ; avec le Christ, je suis crucifié. 
${}^{20}Je vis, mais ce n’est plus moi, c’est le Christ qui vit en moi. Ce que je vis aujourd’hui dans la chair, je le vis dans la foi au Fils de Dieu qui m’a aimé et s’est livré lui-même pour moi. 
${}^{21}Il n’est pas question pour moi de rejeter la grâce de Dieu. En effet, si c’était par la Loi qu’on devient juste, alors le Christ serait mort pour rien.
      
         
      \bchapter{}
      \begin{verse}
${}^{1}Galates stupides, qui donc vous a ensorcelés ? À vos yeux, pourtant, Jésus Christ a été présenté crucifié. 
${}^{2}Je n’ai qu’une question à vous poser : l’Esprit Saint, l’avez-vous reçu pour avoir pratiqué la Loi, ou pour avoir écouté le message de la foi ? 
${}^{3}Comment pouvez-vous être aussi fous ? Après avoir commencé par l’Esprit, allez-vous, maintenant, finir par la chair ? 
${}^{4}Auriez-vous vécu de si grandes choses en vain ? Si encore ce n’était qu’en vain ! 
${}^{5}Celui qui vous fait don de l’Esprit et qui réalise des miracles parmi vous, le fait-il parce que vous pratiquez la Loi, ou parce que vous écoutez le message de la foi ?
      
         
${}^{6}C’est ainsi qu’Abraham eut foi en Dieu, et il lui fut accordé d’être juste. 
${}^{7}Comprenez-le donc : ceux qui se réclament de la foi, ce sont eux, les fils d’Abraham. 
${}^{8}D’ailleurs, l’Écriture avait prévu, au sujet des nations, que Dieu les rendrait justes par la foi, et elle avait annoncé d’avance à Abraham cette bonne nouvelle : En toi seront bénies toutes les nations. 
${}^{9}Ainsi, ceux qui se réclament de la foi sont bénis avec Abraham, le croyant.
${}^{10}Quant à ceux qui se réclament de la pratique de la Loi, ils sont tous sous la menace d’une malédiction, car il est écrit : Maudit soit celui qui ne s’attache pas à mettre en pratique tout ce qui est écrit dans le livre de la Loi. 
${}^{11}Il est d’ailleurs clair que par la Loi personne ne devient juste devant Dieu, car, comme le dit l’Écriture, celui qui est juste par la foi, vivra, 
${}^{12}et la Loi ne procède pas de la foi, mais elle dit : Celui qui met en pratique les commandements vivra à cause d’eux. 
${}^{13}Quant à cette malédiction de la Loi, le Christ nous en a rachetés en devenant, pour nous, objet de malédiction, car il est écrit : Il est maudit, celui qui est pendu au bois du supplice. 
${}^{14}Tout cela pour que la bénédiction d’Abraham s’étende aux nations païennes dans le Christ Jésus, et que nous recevions, par la foi, l’Esprit qui a été promis.
${}^{15}Frères, j’emploie ici un langage humain. Quand un homme a fait un testament en bonne et due forme, personne ne peut l’annuler ou lui ajouter des clauses. 
${}^{16}Or, les promesses ont été faites à Abraham ainsi qu’à sa descendance ; l’Écriture ne dit pas « et à tes descendants », comme si c’était pour plusieurs, mais et à ta descendance, comme pour un seul, qui est le Christ. 
${}^{17}Alors je dis ceci : le testament fait par Dieu en bonne et due forme n’est pas révoqué par la Loi intervenue quatre cent trente ans après, ce qui abolirait la promesse. 
${}^{18}Car si l’héritage s’obtient par la Loi, ce n’est plus par une promesse. Or c’est par une promesse que Dieu accorda sa faveur à Abraham.
${}^{19}Alors pourquoi la Loi ? Elle a été ajoutée, pour que les transgressions soient rendues manifestes, jusqu’à la venue de la descendance à qui ont été faites les promesses, et elle a été établie par des anges par l’entremise d’un médiateur. 
${}^{20}Ce médiateur en représente plus d’un, mais Dieu, lui, est un. 
${}^{21}La Loi est-elle donc contre les promesses de Dieu ? Absolument pas. S’il nous avait été donné une loi capable de nous faire vivre, alors vraiment la Loi rendrait juste. 
${}^{22}Mais l’Écriture a tout enfermé sous la domination du péché, afin que ce soit par la foi en Jésus Christ que la promesse s’accomplisse pour les croyants.
${}^{23}Avant que vienne la foi en Jésus Christ, nous étions des prisonniers, enfermés sous la domination de la Loi, jusqu’au temps où cette foi devait être révélée. 
${}^{24}Ainsi, la Loi, comme un guide, nous a menés jusqu’au Christ pour que nous obtenions de la foi la justification. 
${}^{25}Et maintenant que la foi est venue, nous ne sommes plus soumis à ce guide. 
${}^{26}Car tous, dans le Christ Jésus, vous êtes fils de Dieu par la foi. 
${}^{27}En effet, vous tous que le baptême a unis au Christ, vous avez revêtu le Christ ; 
${}^{28}il n’y a plus ni juif ni grec, il n’y a plus ni esclave ni homme libre, il n’y a plus l’homme et la femme, car tous, vous ne faites plus qu’un dans le Christ Jésus. 
${}^{29}Et si vous appartenez au Christ, vous êtes de la descendance d’Abraham : vous êtes héritiers selon la promesse.
      
         
      \bchapter{}
      \begin{verse}
${}^{1}Je m’explique. Tant que l’héritier est un petit enfant, il ne diffère en rien d’un esclave, alors qu’il est le maître de toute la maison ; 
${}^{2}mais il est soumis aux gérants et aux intendants jusqu’à la date fixée par le père. 
${}^{3}De même nous aussi, quand nous étions des petits enfants, nous étions en situation d’esclaves, soumis aux forces qui régissent le monde. 
${}^{4}Mais lorsqu’est venue la plénitude des temps, Dieu a envoyé son Fils, né d’une femme et soumis à la loi de Moïse, 
${}^{5}afin de racheter ceux qui étaient soumis à la Loi et pour que nous soyons adoptés comme fils. 
${}^{6}Et voici la preuve que vous êtes des fils : Dieu a envoyé l’Esprit de son Fils dans nos cœurs, et cet Esprit crie « Abba ! », c’est-à-dire : Père ! 
${}^{7}Ainsi tu n’es plus esclave, mais fils, et puisque tu es fils, tu es aussi héritier : c’est l’œuvre de Dieu.
      
         
${}^{8}Jadis, quand vous ne connaissiez pas Dieu, vous étiez esclaves de ces dieux qui, en réalité, n’en sont pas. 
${}^{9}Mais maintenant que vous avez connu Dieu – ou plutôt que vous avez été connus par lui – comment pouvez-vous de nouveau vous tourner vers ces forces inconsistantes et misérables, dont vous voulez de nouveau être esclaves comme autrefois ? 
${}^{10}Vous vous pliez à des règles concernant les jours, les mois, les temps, les années ! 
${}^{11}J’ai bien peur de m’être donné, en vain, de la peine pour vous.
${}^{12}Frères, je vous en prie, devenez comme moi, car moi je suis devenu comme vous. Assurément, vous ne m’avez fait aucun tort. 
${}^{13}Vous le savez : c’est par suite d’une maladie que je vous ai annoncé l’Évangile pour la première fois ; 
${}^{14}et l’épreuve qu’était pour vous ce corps malade, vous ne l’avez pas repoussée avec dégoût, mais vous m’avez accueilli comme un ange de Dieu, comme le Christ Jésus lui-même. 
${}^{15}Où donc est votre bonheur d’alors ? Je vous en rends témoignage : si vous aviez pu, vous vous seriez arraché les yeux pour me les donner. 
${}^{16}Suis-je donc devenu votre ennemi pour vous avoir dit la vérité ? 
${}^{17}Certains ont pour vous un attachement qui n’est pas bon ; en fait, ils voudraient vous isoler pour que vous vous attachiez à eux. 
${}^{18}Mieux vaut un attachement de bonne qualité en tout temps, et pas seulement quand je suis chez vous. 
${}^{19}Mes enfants, vous que j’enfante à nouveau dans la douleur jusqu’à ce que le Christ soit formé en vous, 
${}^{20}je voudrais être maintenant près de vous et pouvoir changer le ton de ma voix, car je ne sais comment m’y prendre avec vous.
${}^{21}Dites-moi, vous qui voulez être soumis à la Loi, n’entendez-vous pas ce que dit la Loi ? 
${}^{22}Il y est écrit en effet qu’Abraham a eu deux fils, l’un né de la servante, et l’autre de la femme libre. 
${}^{23}Le fils de la servante a été engendré selon la chair ; celui de la femme libre l’a été en raison d’une promesse de Dieu. 
${}^{24}Ces événements ont un sens symbolique : les deux femmes sont les deux Alliances. La première Alliance, celle du mont Sinaï, qui met au monde des enfants esclaves, c’est Agar, la servante. 
${}^{25}Agar est le mont Sinaï en Arabie, elle correspond à la Jérusalem actuelle, elle qui est esclave ainsi que ses enfants, 
${}^{26}tandis que la Jérusalem d’en haut est libre, et c’est elle, notre mère. 
${}^{27}L’Écriture dit en effet :
        \\Réjouis-toi, femme stérile, toi qui n’enfantes pas ;
        \\éclate en cris de joie,
        \\toi qui ne connais pas les douleurs de l’enfantement,
        \\car les enfants de la femme délaissée sont plus nombreux
        \\que ceux de la femme qui a son mari.
${}^{28}Et vous, frères, vous êtes, comme Isaac, des enfants de la promesse. 
${}^{29}Mais de même qu’autrefois le fils engendré selon la chair persécutait le fils engendré selon l’Esprit, de même en est-il aujourd’hui. 
${}^{30}Or, que dit l’Écriture ? Renvoie la servante et son fils, car le fils de la servante ne peut être héritier avec le fils de la femme libre.
${}^{31}Dès lors, frères, nous ne sommes pas les enfants d’une servante, nous sommes ceux de la femme libre.
      
         
      \bchapter{}
      \begin{verse}
${}^{1}C’est pour que nous soyons libres que le Christ nous a libérés. Alors tenez bon, ne vous mettez pas de nouveau sous le joug de l’esclavage. 
${}^{2}Moi, Paul, je vous le déclare : si vous vous faites circoncire, le Christ ne vous sera plus d’aucun secours. 
${}^{3}Je l’atteste encore une fois : tout homme qui se fait circoncire est dans l’obligation de pratiquer la loi de Moïse tout entière. 
${}^{4}Vous qui cherchez la justification par la Loi, vous vous êtes séparés du Christ, vous êtes déchus de la grâce. 
${}^{5}Nous, c’est par l’Esprit, en effet, que de la foi nous attendons la justice espérée. 
${}^{6}Car, dans le Christ Jésus, ce qui a de la valeur, ce n’est pas que l’on soit circoncis ou non, mais c’est la foi, qui agit par la charité.
${}^{7}Votre course partait bien. Qui vous a empêchés d’obéir à la vérité ? 
${}^{8}Cette influence-là ne vient pas de Celui qui vous appelle. 
${}^{9}Un peu de levain suffit pour que toute la pâte fermente. 
${}^{10}Moi, j’ai dans le Seigneur la conviction que vous, vous n’adopterez pas une autre façon de penser. Quant à celui qui met le trouble chez vous, il en subira la sanction, quel qu’il soit. 
${}^{11}Et moi, frères, si, comme certains le prétendent, je prêche encore la circoncision, pourquoi suis-je encore persécuté ? Car alors cette prédication abolirait le scandale de la Croix. 
${}^{12}Qu’ils aillent donc jusqu’à se mutiler, ceux qui sèment le désordre chez vous.
${}^{13}Vous, frères, vous avez été appelés à la liberté. Mais que cette liberté ne soit pas un prétexte pour votre égoïsme ; au contraire, mettez-vous, par amour, au service les uns des autres. 
${}^{14}Car toute la Loi est accomplie dans l’unique parole que voici : Tu aimeras ton prochain comme toi-même. 
${}^{15}Mais si vous vous mordez et vous dévorez les uns les autres, prenez garde : vous allez vous détruire les uns les autres. 
${}^{16}Je vous le dis : marchez sous la conduite de l’Esprit Saint, et vous ne risquerez pas de satisfaire les convoitises de la chair. 
${}^{17}Car les tendances de la chair s’opposent à l’Esprit, et les tendances de l’Esprit s’opposent à la chair. En effet, il y a là un affrontement qui vous empêche de faire tout ce que vous voudriez. 
${}^{18}Mais si vous vous laissez conduire par l’Esprit, vous n’êtes pas soumis à la Loi.
${}^{19}On sait bien à quelles actions mène la chair : inconduite, impureté, débauche, 
${}^{20}idolâtrie, sorcellerie, haines, rivalité, jalousie, emportements, intrigues, divisions, sectarisme, 
${}^{21}envie, beuveries, orgies et autres choses du même genre. Je vous préviens, comme je l’ai déjà fait : ceux qui commettent de telles actions ne recevront pas en héritage le royaume de Dieu. 
${}^{22}Mais voici le fruit de l’Esprit : amour, joie, paix, patience, bonté, bienveillance, fidélité, 
${}^{23}douceur et maîtrise de soi. En ces domaines, la Loi n’intervient pas. 
${}^{24}Ceux qui sont au Christ Jésus ont crucifié en eux la chair, avec ses passions et ses convoitises. 
${}^{25}Puisque l’Esprit nous fait vivre, marchons sous la conduite de l’Esprit. 
${}^{26}Ne cherchons pas la vaine gloire ; entre nous, pas de provocation, pas d’envie les uns à l’égard des autres.
      
         
      \bchapter{}
      \begin{verse}
${}^{1}Frères, si quelqu’un est pris en faute, vous, les spirituels, remettez-le dans le droit chemin en esprit de douceur ; mais prenez garde à vous-mêmes : vous pourriez être tentés, vous aussi. 
${}^{2}Portez les fardeaux les uns des autres : ainsi vous accomplirez la loi du Christ. 
${}^{3}Si quelqu’un pense être quelque chose alors qu’il n’est rien, il se fait illusion sur lui-même. 
${}^{4}Que chacun examine sa propre action ; ainsi, c’est seulement par rapport à lui-même qu’il trouvera ses motifs de fierté et non par rapport aux autres. 
${}^{5}Chacun, en effet, portera sa propre charge.
${}^{6}Celui qui reçoit l’enseignement de la Parole doit donner, à celui qui la lui transmet, une part de tous ses biens. 
${}^{7}Ne vous égarez pas : Dieu ne se laisse pas narguer. Ce que l’on a semé, on le récoltera. 
${}^{8}Celui qui a semé en vue de sa propre chair récoltera ce que produit la chair : la corruption ; mais celui qui a semé en vue de l’Esprit récoltera ce que produit l’Esprit : la vie éternelle. 
${}^{9}Ne nous lassons pas de faire le bien, car, le moment venu, nous récolterons, si nous ne perdons pas courage. 
${}^{10}Ainsi donc, lorsque nous en avons l’occasion, travaillons au bien de tous, et surtout à celui de nos proches dans la foi.
${}^{11}Regardez ce que j’écris en grandes lettres pour vous de ma propre main. 
${}^{12}Tous ceux qui veulent faire humainement bonne figure, ce sont ceux-là qui vous obligent à la circoncision ; ils le font seulement afin de ne pas être persécutés pour la croix du Christ. 
${}^{13}Car ceux qui se font circoncire n’observent pas eux-mêmes la Loi ; ils veulent seulement vous imposer la circoncision afin que votre chair soit pour eux un motif de fierté. 
${}^{14}Mais pour moi, que la croix de notre Seigneur Jésus Christ reste ma seule fierté. Par elle, le monde est crucifié pour moi, et moi pour le monde. 
${}^{15}Ce qui compte, ce n’est pas d’être circoncis ou incirconcis, c’est d’être une création nouvelle. 
${}^{16}Pour tous ceux qui marchent selon cette règle de vie et pour l’Israël de Dieu, paix et miséricorde.
${}^{17}Dès lors, que personne ne vienne me tourmenter, car je porte dans mon corps les marques des souffrances de Jésus.
${}^{18}Frères, que la grâce de notre Seigneur Jésus Christ soit avec votre esprit. Amen.
