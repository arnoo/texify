  
  
    
    \bbook{PREMIER LIVRE DES MARTYRS D’ISRAËL}{PREMIER LIVRE DES MARTYRS D’ISRAËL}
      
         
      \bchapter{}
      \begin{verse}
${}^{1}Alexandre, fils de Philippe, de Macédoine, quitta le pays des Grecs pour affronter Darius, le roi des Perses et des Mèdes. Après l’avoir vaincu, il régna à sa place ; auparavant il régnait déjà sur le monde grec. 
${}^{2}Il livra de multiples batailles, s’empara de nombreuses forteresses et fit périr les rois de la région. 
${}^{3}Il poussa jusqu’aux extrémités de la terre et ramassa le butin d’une multitude de nations. Devant lui, la terre resta muette. Son cœur s’exalta à l’extrême. 
${}^{4}Il rassembla une armée très puissante, soumit des provinces, des nations et des souverains, qui durent lui payer l’impôt. 
${}^{5}Après quoi, il fut contraint de s’aliter et comprit qu’il allait mourir. 
${}^{6}Il convoqua ses auxiliaires les plus illustres, élevés avec lui depuis le jeune âge et, de son vivant, il partagea entre eux son royaume. 
${}^{7}Alexandre avait régné douze ans quand il mourut. 
${}^{8}Alors, ceux qu’il avait mis en fonction exercèrent le pouvoir, chacun dans sa région. 
${}^{9}Après sa mort, ils portèrent tous le diadème, et leurs fils après eux, durant de longues années. Et ils multiplièrent les malheurs sur la terre.
      
         
${}^{10}De leur descendance surgit un homme de péché\\, Antiocos Épiphane, fils du roi Antiocos le Grand\\. Il avait séjourné à Rome comme otage, et il devint roi en l’année 137 de l’empire grec\\. 
${}^{11} À cette époque\\, surgirent en Israël des hommes infidèles à la Loi, et ils séduisirent beaucoup de gens, car ils disaient : « Allons, faisons alliance avec les nations qui nous entourent. En effet, depuis que nous avons rompu avec elles, il nous est arrivé beaucoup de malheurs. »
${}^{12}Ce langage parut judicieux, 
${}^{13} et quelques-uns, dans le peuple, s’empressèrent d’aller trouver le roi. Celui-ci leur permit d’adopter les usages des nations. 
${}^{14} Ils construisirent un gymnase à Jérusalem, selon la coutume des nations ; 
${}^{15} ils effacèrent les traces de leur circoncision, renièrent l’Alliance sainte, s’associèrent aux gens des nations, et se vendirent pour faire le mal.
${}^{16}Quand Antiocos vit son pouvoir bien établi, il projeta de devenir aussi roi d’Égypte et de régner sur les deux royaumes. 
${}^{17}Il entra en Égypte avec une armée imposante, des chars, des éléphants, des cavaliers ainsi qu’une grande flotte. 
${}^{18}Il livra bataille à Ptolémée, le roi d’Égypte, qui battit en retraite et s’enfuit ; il y eut beaucoup de victimes. 
${}^{19}Antiocos s’empara des villes fortes d’Égypte et ramassa le butin du pays.
${}^{20}Après sa victoire sur l’Égypte, en l’an 143 de l’empire grec, Antiocos s’en retourna et monta contre Israël et Jérusalem avec une armée imposante. 
${}^{21}Il entra dans le sanctuaire avec arrogance et fit main basse sur l’autel d’or, le chandelier de lumière avec tous ses accessoires, 
${}^{22}la table des offrandes, les coupes à libation, les vases, les encensoirs d’or, le voile et les couronnes. Même le revêtement doré de la façade du Temple, il l’arracha en entier. 
${}^{23}Il prit aussi l’argent, l’or et les objets précieux, ainsi que les trésors cachés qu’il découvrait. 
${}^{24}Avec tout cela, il retourna dans son pays. Comme il avait fait un carnage et proféré des paroles d’une extrême arrogance, 
${}^{25}il y eut un grand deuil dans tout Israël.
${}^{26}Chefs et anciens poussèrent des lamentations,
        \\jeunes filles et jeunes gens dépérirent,
        \\et la beauté des femmes s’altéra.
${}^{27}Tout jeune marié entonna un chant de deuil ;
        \\assise dans la chambre nuptiale, l’épouse était en larmes.
${}^{28}La terre trembla à cause de ses habitants,
        \\et toute la maison de Jacob fut revêtue de honte.
${}^{29}Deux ans plus tard, le roi envoya dans les villes de Juda un commissaire pour percevoir les impôts. Celui-ci vint à Jérusalem avec une armée importante. 
${}^{30}Il adressa aux habitants des paroles de paix : c’était pour les tromper, mais ils le crurent. Puis, il attaqua la ville à l’improviste, la frappa durement et fit périr beaucoup de gens en Israël. 
${}^{31}Il prit le butin de la ville, incendia celle-ci, détruisit les maisons et le mur d’enceinte. 
${}^{32}Ses hommes réduisirent en captivité les femmes et les enfants et confisquèrent le bétail. 
${}^{33}Puis, ils rebâtirent la Cité de David, avec un haut rempart fortifié et de puissantes tours. Elle devint leur citadelle. 
${}^{34}Là, ils installèrent une race perverse, des gens infidèles à la Loi. Pour accroître leur force, 
${}^{35}ils y amassèrent des armes et de la nourriture, ils y déposèrent le butin de Jérusalem qu’ils avaient rassemblé. C’était un véritable piège, 
${}^{36}un lieu d’embuscade contre le sanctuaire, un cruel adversaire pour Israël en tout temps.
${}^{37}Autour du sanctuaire, ils répandirent le sang des innocents,
        \\ils souillèrent le Lieu saint.
${}^{38}À cause d’eux s’enfuirent les habitants de Jérusalem,
        \\et la ville devint une colonie d’étrangers,
        \\une étrangère pour sa descendance,
        \\car ses propres enfants l’avaient abandonnée.
${}^{39}Son sanctuaire fut dévasté comme un désert,
        \\ses fêtes se changèrent en deuil,
        \\ses sabbats en dérision,
        \\son honneur en mépris.
${}^{40}Elle fut remplie de honte,
        \\à la mesure même de sa gloire passée,
        \\et sa fierté se changea en larmes.
${}^{41}Le roi Antiocos\\prescrivit à tous les habitants de son royaume de ne faire désormais qu’un seul peuple, 
${}^{42}et d’abandonner leurs coutumes particulières. Toutes les nations païennes se conformèrent à cet ordre. 
${}^{43}En Israël, beaucoup suivirent volontiers la religion du roi, offrirent des sacrifices aux idoles, et profanèrent le sabbat. 
${}^{44}Le roi envoya également à Jérusalem et aux villes de Juda des émissaires, avec des lettres qui leur prescrivaient de suivre des coutumes étrangères à leur pays. 
${}^{45}Ils devaient bannir du sanctuaire holocaustes, sacrifice et libation, profaner les sabbats et les fêtes, 
${}^{46}souiller le sanctuaire et les fidèles, 
${}^{47}élever des autels, des lieux de culte et des idoles, offrir en sacrifice des porcs et d’autres animaux impurs. 
${}^{48}Ils devaient laisser leurs fils incirconcis et se rendre eux-mêmes abominables par toutes sortes de pratiques impures et de profanations, 
${}^{49}de manière à oublier la Loi et à changer toutes les observances. 
${}^{50}Quiconque n’agirait pas selon l’ordre du roi serait mis à mort. 
${}^{51}C’est en ces termes que le roi écrivit à tous ses sujets. Il établit des inspecteurs pour tout le peuple et ordonna aux villes de Juda d’offrir des sacrifices dans chaque ville. 
${}^{52}Beaucoup de gens du peuple, tous ceux qui délaissaient la Loi, se rallièrent à eux et firent du mal dans le pays. 
${}^{53}Ils obligèrent Israël à se cacher dans tous ses lieux de refuge.
${}^{54}Le quinzième jour du neuvième mois\\, en l’année 145, Antiocos éleva sur l’autel des sacrifices l’Abomination de la désolation\\, et, dans les villes de Juda autour de Jérusalem, ses partisans\\élevèrent des autels païens. 
${}^{55}Ils brûlèrent de l’encens aux portes des maisons et sur les places. 
${}^{56}Tous les livres de la Loi qu’ils découvraient, ils les jetaient au feu après les avoir lacérés. 
${}^{57}Si l’on découvrait chez quelqu’un un livre de l’Alliance, si quelqu’un se conformait à la Loi, le décret du roi le faisait mettre à mort. 
${}^{58}Mois après mois, dans les villes, on sévissait durement contre ceux d’Israël qui étaient pris en infraction. 
${}^{59}Le vingt-cinq de chaque mois, on sacrifiait sur l’autel dressé sur l’autel des sacrifices. 
${}^{60}Selon l’ordre du roi, les femmes qui avaient fait circoncire leurs enfants, on les mettait à mort, 
${}^{61}avec leurs nourrissons suspendus à leur cou. On exécutait aussi les membres de leurs familles, et ceux qui avaient opéré la circoncision. 
${}^{62}Cependant, beaucoup en Israël résistèrent et eurent le courage de ne manger aucun aliment impur. 
${}^{63}Ils acceptèrent de mourir pour ne pas être souillés par ce qu’ils mangeaient, et pour ne pas profaner l’Alliance sainte ; et de fait, ils moururent. 
${}^{64}C’est ainsi que s’abattit sur Israël une grande colère.
      
         
      \bchapter{}
      \begin{verse}
${}^{1}En ces jours-là se leva Mattathias, fils de Jean, fils de Syméon, prêtre de la descendance de Joarib. Il quitta Jérusalem et s’établit à Modine. 
${}^{2}Il avait cinq fils : Jean surnommé Gaddi, 
${}^{3}Simon appelé Thassi, 
${}^{4}Judas appelé Maccabée, 
${}^{5}Éléazar appelé Awarane, Jonathan appelé Apphous.
      
         
${}^{6}À la vue des sacrilèges qui se commettaient en Juda et à Jérusalem, Mattathias dit :
${}^{7}« Malheur à moi !
        \\Suis-je né pour voir la ruine de mon peuple,
        \\la ruine de la Ville sainte,
        \\et rester assis là,
        \\tandis qu’elle est livrée aux mains des ennemis,
        \\et le sanctuaire, aux mains des étrangers ?
${}^{8}Son Temple est devenu semblable à un homme déshonoré,
${}^{9}les objets qui faisaient sa gloire
        \\ont été emportés comme prises de guerre.
        \\Sur ses places, on a massacré ses petits enfants,
        \\ses jeunes gens sont tombés sous l’épée de l’ennemi.
${}^{10}Est-il une nation qui n’a pas confisqué
        \\une part de sa puissance royale,
        \\qui ne s’est emparée de son butin ?
${}^{11}Toute sa parure lui a été arrachée.
        \\Elle était libre : la voilà réduite en esclavage !
${}^{12}Oui, le Lieu saint, notre beauté et notre gloire,
        \\ils l’ont dévasté,
        \\les nations l’ont profané.
${}^{13}À quoi bon vivre encore ? »
${}^{14}Mattathias et ses fils déchirèrent leurs tuniques, s’enveloppèrent de toile à sac et menèrent un grand deuil.
${}^{15}Les hommes envoyés par le roi pour contraindre les gens à l’apostasie arrivèrent dans la ville de Modine pour y organiser des sacrifices. 
${}^{16} Beaucoup en Israël allèrent à eux ; Mattathias et ses fils vinrent à la réunion. 
${}^{17} Les envoyés du roi prirent la parole pour dire à Mattathias : « Tu es un chef honoré et puissant dans cette ville, soutenu par des fils et des frères. 
${}^{18} Avance donc le premier, et exécute l’ordre du roi, comme l’ont fait toutes les nations, les hommes de Juda et ceux qui sont restés à Jérusalem. Alors, toi et tes fils, vous serez les amis du roi. Toi et tes fils, vous serez comblés d’argent, d’or et de cadeaux nombreux. » 
${}^{19} Mattathias répondit d’une voix forte : « Toutes les nations qui appartiennent aux États du roi peuvent bien lui obéir en rejetant chacune la religion de ses pères, et se conformer à ses commandements ; 
${}^{20} mais moi, mes fils et mes frères, nous suivrons l’Alliance\\de nos pères. 
${}^{21} Que le Ciel\\nous préserve d’abandonner la Loi et ses préceptes ! 
${}^{22} Nous n’obéirons pas aux ordres du roi, nous ne dévierons pas de notre religion, ni à droite ni à gauche. »
${}^{23}Dès qu’il eut fini de prononcer ces paroles, un Juif s’avança en présence de tout le monde pour offrir le sacrifice, selon l’ordre du roi, sur cet autel de Modine. 
${}^{24} À cette vue, Mattathias s’enflamma d’indignation et frémit jusqu’au fond de lui-même ; il laissa monter en lui une légitime colère, courut à l’homme et l’égorgea sur l’autel. 
${}^{25} Quant à l’envoyé du roi, qui voulait contraindre à offrir le sacrifice, Mattathias le tua à l’instant même, et il renversa l’autel. 
${}^{26} Il s’enflamma d’ardeur pour la Loi comme jadis Pinhas contre Zimri\\. 
${}^{27} Alors Mattathias se mit à crier d’une voix forte à travers la ville : « Ceux qui sont enflammés d’une ardeur jalouse pour la Loi, et qui soutiennent l’Alliance, qu’ils sortent tous de la ville\\à ma suite. » 
${}^{28} Il s’enfuit dans la montagne avec ses fils, en abandonnant tout ce qu’ils avaient dans la ville.
${}^{29}Alors, beaucoup de ceux qui recherchaient la justice et la Loi s’en allèrent vivre au désert, 
${}^{30}avec leurs fils, leurs femmes et leur bétail. Le malheur s’était appesanti sur eux. 
${}^{31}On annonça aux hommes du roi et aux troupes stationnées à Jérusalem, dans la Cité de David, que des hommes qui avaient rejeté la prescription du roi étaient descendus dans les lieux cachés du désert. 
${}^{32}En grand nombre, ils coururent à leur poursuite. Ils les rattrapèrent et dressèrent leur camp en face d’eux. Ils se préparèrent à les attaquer le jour du sabbat. 
${}^{33}Ils leur dirent : « En voilà assez ! Sortez pour obéir à l’ordre du roi. Alors, vous aurez la vie sauve. » 
${}^{34}Mais ils répondirent : « Nous ne sortirons pas. Nous n’obéirons pas à l’ordre du roi de profaner le jour du sabbat. » 
${}^{35}Aussitôt assaillis, 
${}^{36}ils ne ripostèrent pas. Ils ne lancèrent pas une seule pierre. Ils ne barricadèrent pas même l’entrée de leurs refuges. 
${}^{37}Ils disaient : « Mourons tous en préservant notre droiture. Le ciel et la terre nous sont témoins que vous nous faites périr injustement. » 
${}^{38}On leur donna donc l’assaut en plein sabbat, et ils moururent, eux, leurs femmes, leurs enfants, leur bétail ; en tout, un millier de personnes.
${}^{39}En apprenant ces événements, Mattathias et ses amis furent profondément attristés. 
${}^{40}Ils se dirent l’un à l’autre : « Si nous agissons tous comme nos frères, si nous ne luttons pas pour nos vies et nos lois, les nations païennes auront tôt fait de nous exterminer de la terre. » 
${}^{41}Ce jour-là, ils prirent la décision suivante : « Si un homme, quel qu’il soit, vient nous attaquer le jour du sabbat, nous lui ferons face. Ainsi, nous ne mourrons pas tous, comme nos frères qui sont morts dans leurs refuges. »
${}^{42}Alors, la communauté des Assidéens se joignit à eux. C’étaient de vaillants guerriers d’Israël, tous profondément attachés à la Loi. 
${}^{43}Tous ceux qui fuyaient le malheur vinrent également grossir leurs rangs et les soutenir. 
${}^{44}Ils constituèrent une armée. Dans leur colère, ils frappèrent les pécheurs et, dans leur fureur, les hommes infidèles à la Loi. Ceux qui restaient s’enfuirent auprès des païens pour trouver le salut. 
${}^{45}En circulant dans le pays, Mattathias et ses amis renversèrent les autels païens. 
${}^{46}À tous les jeunes enfants incirconcis qu’ils trouvaient sur le territoire d’Israël, ils imposèrent de force la circoncision. 
${}^{47}Ils chassèrent les fils d’arrogance, et tout ce qu’ils entreprenaient réussissait. 
${}^{48}Ils arrachèrent la Loi de la main des païens et des rois. Ils ne laissèrent pas triompher le pécheur.
${}^{49}Mattathias, sentant que la mort approchait, dit à ses fils :
        \\« Voici maintenant le règne de l’arrogance et du mépris,
        \\le temps du bouleversement, l’explosion de la colère.
        ${}^{50}Maintenant, mes enfants, défendez la Loi avec ardeur,
        \\et donnez votre vie pour l’Alliance de nos pères\\.
        ${}^{51}Souvenez-vous de tout ce qu’ils ont fait autrefois,
        \\et vous obtiendrez une grande gloire et un nom immortel.
        ${}^{52}Abraham a montré sa foi dans l’épreuve,
        \\et c’est pourquoi il fut reconnu comme juste\\.
${}^{53}Au temps de sa détresse, Joseph a observé les préceptes
        \\et il est devenu maître de l’Égypte.
${}^{54}Pinhas, notre ancêtre, a reçu, pour son zèle ardent,
        \\l’alliance d’un sacerdoce éternel.
${}^{55}Josué, pour avoir accompli sa mission,
        \\est devenu juge en Israël.
${}^{56}Caleb, pour son témoignage devant l’assemblée du peuple,
        \\a reçu une terre en héritage.
        ${}^{57}David, pour sa fidélité\\,
        \\a reçu un trône royal en héritage éternel.
        ${}^{58}Élie, pour son ardeur à défendre la Loi,
        \\a été enlevé au ciel.
        ${}^{59}Ananias, Azarias et Misaël, pour avoir gardé la foi\\,
        \\ont été sauvés de la fournaise.
        ${}^{60}Daniel, pour sa droiture,
        \\a été arraché à la gueule des lions.
        ${}^{61}Retenez donc bien ceci de génération en génération :
        \\tous ceux qui gardent l’espérance\\auront la force de résister.
        ${}^{62}Ne redoutez pas les menaces de l’homme pécheur,
        \\car sa gloire s’en ira en pourriture.
        ${}^{63}Aujourd’hui, il s’élève, et demain on ne le trouvera plus ;
        \\car il sera redevenu poussière, et ses projets auront échoué.
        ${}^{64}Mes enfants, soyez courageux et fermes dans la Loi,
        \\car la Loi sera votre gloire.
${}^{65}« Voici votre frère Syméon, je sais qu’il est de bon conseil. Écoutez-le toujours, et il sera pour vous comme un père. 
${}^{66}Judas Maccabée est un homme vaillant depuis sa jeunesse. C’est lui qui commandera votre armée et mènera le combat contre les peuples. 
${}^{67}Quant à vous, rassemblez tous ceux qui observent la Loi, et vengez votre peuple. 
${}^{68}Rendez aux nations païennes le mal qu’elles vous ont fait et attachez-vous aux préceptes de la Loi. » 
${}^{69}Puis, Mattathias leur donna sa bénédiction, et il fut réuni à ses pères. 
${}^{70}Il mourut en l’an 146 de l’empire grec. Il fut enseveli dans le tombeau de famille à Modine, et tout Israël se lamenta en se frappant la poitrine.
      
         
      \bchapter{}
      \begin{verse}
${}^{1}Après la mort de Mattathias, son fils Judas, appelé Maccabée, se leva à sa place. 
${}^{2}Tous ses frères et tous les partisans de son père lui vinrent en aide et menèrent avec allégresse le combat pour Israël.
      
         
${}^{3}Judas étendit le renom glorieux de son peuple.
        \\Revêtu de son armure, comme un héros,
        \\et fort de ses armes de guerre,
        \\il livra des batailles,
        \\protégeant le camp de son épée.
${}^{4}Tel un lion en pleine action,
        \\pareil au lionceau rugissant vers sa proie,
${}^{5}il traquait les hommes infidèles à la Loi, il les pourchassait
        \\et livrait au feu ceux qui troublaient son peuple.
${}^{6}Devant la crainte qu’il inspirait,
        \\les hommes infidèles à la Loi furent abaissés
        \\et tous les malfaisants, pris de panique.
        \\Par sa main advint le salut.
${}^{7}À bien des rois il rendit la vie amère,
        \\il réjouit Jacob par ses exploits.
        \\Son souvenir est à jamais béni.
${}^{8}Il parcourut les villes de Juda,
        \\il en extermina les impies.
        \\Il détourna la colère qui pesait sur Israël.
${}^{9}Son nom retentit jusqu’aux extrémités de la terre
        \\et il rassembla ceux qui allaient périr.
${}^{10}Alors, Apollonios rassembla des païens et une troupe importante venue de Samarie, pour combattre Israël. 
${}^{11}Judas en fut informé. Il sortit à sa rencontre, le battit et le tua. Il y eut de nombreuses victimes, et les survivants s’enfuirent. 
${}^{12}On ramassa le butin, et Judas s’empara du glaive d’Apollonios, pour s’en servir tous les jours au combat. 
${}^{13}Sérone, commandant de l’armée de Syrie, apprit que Judas avait réuni autour de lui un groupe, une assemblée de fidèles, pour marcher au combat. 
${}^{14}Il se dit : « Je me ferai un nom et je me couvrirai de gloire dans le royaume. Je combattrai Judas et ses compagnons, qui méprisent l’ordre du roi. » 
${}^{15}Il partit donc à son tour, accompagné d’une forte troupe d’hommes impies, pour tirer vengeance des fils d’Israël. 
${}^{16}Tandis qu’il s’approchait de la montée de Bethorone, Judas sortit à sa rencontre avec une poignée d’hommes. 
${}^{17}À la vue de la troupe qui montait vers eux, ses hommes dirent à Judas : « Comment pourrons-nous, en si petit nombre, lutter contre une foule si nombreuse et si forte ? De plus, nous sommes exténués, car nous n’avons rien mangé aujourd’hui ! » 
${}^{18}Judas leur répondit : « Il arrive facilement qu’une multitude tombe aux mains d’un petit nombre. Pour le Ciel, peu importe d’opérer le salut au moyen de beaucoup d’hommes ou seulement de quelques-uns. 
${}^{19}En effet, la victoire au combat ne dépend pas de l’importance de l’armée : c’est du Ciel que vient la force. 
${}^{20}Eux, ils viennent vers nous, débordants d’orgueil et de mépris pour la Loi, afin de nous détruire, ainsi que nos femmes et nos enfants, et de nous dépouiller de nos biens. 
${}^{21}Nous, nous combattons pour nos vies et nos lois. 
${}^{22}Le Ciel lui-même les écrasera devant nous. Ne les craignez donc pas ! »
${}^{23}Après ces paroles, il se rua sur eux à l’improviste. Sérone et sa troupe furent écrasés devant lui. 
${}^{24}On les poursuivit dans la descente de Bethorone, jusqu’à la plaine. Huit cents hommes environ tombèrent ; les autres s’enfuirent au pays des Philistins. 
${}^{25}Judas et ses frères commencèrent à inspirer de la crainte, et la panique s’empara des nations païennes d’alentour. 
${}^{26}Le nom de Judas parvint jusqu’aux oreilles du roi, et tous les païens commentaient ses actes de guerre.
${}^{27}Lorsqu’il entendit ces récits, le roi Antiocos fut transporté de colère. Il fit rassembler toutes les troupes de son royaume, une armée très puissante. 
${}^{28}Il ouvrit son trésor, donna aux troupes leur solde pour un an, et leur commanda de se tenir prêtes à toute éventualité. 
${}^{29}Il s’aperçut alors que l’argent manquait dans ses coffres et que le produit des impôts de la province était maigre, à cause de la révolte et de la misère provoquées dans le pays par la suppression des coutumes ancestrales. 
${}^{30}Il eut peur de n’être plus en mesure, comme c’était arrivé une fois ou l’autre, de faire les mêmes dépenses et largesses qu’auparavant. En effet, il avait l’habitude de distribuer des présents d’une main plus généreuse encore que ses prédécesseurs. 
${}^{31}Se trouvant dans une grande perplexité, il décida de se rendre en Perse, pour y lever les impôts des provinces et récolter beaucoup d’argent. 
${}^{32}Lysias, personnage illustre et membre de la famille royale, fut laissé à la tête des affaires du royaume, depuis les rives de l’Euphrate jusqu’aux frontières de l’Égypte. 
${}^{33}Le roi lui confia aussi l’éducation de son fils Antiocos, jusqu’à son retour. 
${}^{34}Il lui laissa la moitié des troupes et les éléphants, et l’instruisit de toutes ses volontés. Pour ce qui est des habitants de la Judée et de Jérusalem, 
${}^{35}Lysias devait leur envoyer une armée qui les anéantirait, et qui exterminerait la force d’Israël et le reste de ceux qui étaient à Jérusalem, en effaçant de ce lieu jusqu’à leur souvenir. 
${}^{36}Il devait installer des étrangers sur tout leur territoire et lotir leur pays. 
${}^{37}Le roi prit avec lui l’autre moitié des troupes et partit d’Antioche, la capitale du royaume, en l’an 147 de l’empire grec. Il franchit l’Euphrate et traversa le haut pays.
${}^{38}Lysias choisit Ptolémée, fils de Dorymène, Nicanor et Gorgias, personnages puissants parmi les amis du roi. 
${}^{39}Il envoya avec eux quarante mille fantassins et sept mille cavaliers, pour aller dévaster le pays de Juda, selon l’ordre du roi. 
${}^{40}Partis avec toute leur armée, ils arrivèrent près d’Emmaüs et ils établirent leur camp dans la plaine. 
${}^{41}Les marchands de la région entendirent parler d’eux. Munis d’une grande quantité d’argent et d’or, ainsi que d’entraves, ils vinrent au camp avec l’intention d’emmener comme esclaves les fils d’Israël. Un contingent de Syrie et du pays des Philistins rejoignit cette armée.
${}^{42}Judas et ses frères virent que la menace s’aggravait et que des armées campaient sur leur territoire. Ils apprirent aussi que le roi avait ordonné de faire périr leur peuple jusqu’au dernier. 
${}^{43}Ils se dirent entre eux : « Relevons notre peuple de sa ruine. Combattons pour notre peuple et pour le Lieu saint ! » 
${}^{44}Afin d’être prête au combat, la communauté se réunit pour prier, pour implorer miséricorde et pitié.
${}^{45}Jérusalem est inhabitée, comme un désert.
        \\De ses enfants, aucun n’entre, aucun ne sort.
        \\On a foulé aux pieds le sanctuaire.
        \\Des étrangers occupent la citadelle,
        \\des païens s’y installent.
        \\En Jacob, les cris de joie se sont tus,
        \\le son des flûtes et des harpes s’est éteint.
       
${}^{46}Ils se rendirent tous ensemble à Maspha, en face de Jérusalem, car autrefois il y avait eu à Maspha un lieu de prière pour Israël. 
${}^{47}Ce jour-là, ils jeûnèrent. Ils s’enveloppèrent de toile à sac et, la tête couverte de cendres, ils déchirèrent leurs tuniques. 
${}^{48}Ils déroulèrent le livre de la Loi pour le consulter au sujet de ce que les païens, eux, cherchaient à savoir en interrogeant les images de leurs idoles. 
${}^{49}Ils apportèrent les habits sacerdotaux, les prémices et les dîmes. Ils réunirent les gens qui arrivaient au terme fixé pour leur vœu de naziréat, 
${}^{50}et ils crièrent à pleine voix vers le Ciel, en disant : « Que faire de ces gens-là ? Où les emmener, 
${}^{51}puisque ton Lieu saint a été piétiné, profané, et que tes prêtres sont dans le deuil et l’humiliation ? 
${}^{52}Voici les païens ligués contre nous. Ils veulent nous exterminer : toi, tu connais leurs intentions à notre égard. 
${}^{53}Comment pourrons-nous leur résister, si tu ne nous aides pas ? » 
${}^{54}Ils sonnèrent de la trompette et poussèrent de grands cris.
${}^{55}Après cela, Judas établit des chefs du peuple : chefs de millier, de centaine, de cinquantaine et de dizaine. 
${}^{56}À ceux qui bâtissaient leurs maisons, ou qui venaient de se fiancer, ou de planter une vigne, et à ceux qui avaient peur, il dit de s’en retourner chez eux, conformément à la Loi. 
${}^{57}L’armée se mit alors en marche, et vint camper au sud d’Emmaüs. 
${}^{58}« Équipez-vous, leur dit Judas, et soyez courageux. Tenez-vous prêts à combattre demain matin ces païens qui se sont ligués pour nous détruire, nous et notre Lieu saint. 
${}^{59}Car il vaut mieux mourir au combat que de voir les malheurs de notre nation et de notre Lieu saint. 
${}^{60}Ce que le Ciel aura voulu, il l’accomplira. »
      
         
      \bchapter{}
      \begin{verse}
${}^{1}Gorgias prit avec lui cinq mille fantassins et mille cavaliers d’élite, et ce détachement partit de nuit, 
${}^{2}afin de faire irruption dans le camp des Juifs et de fondre sur eux à l’improviste. Les gens de la citadelle lui servaient de guide. 
${}^{3}Judas l’apprit, et il partit avec ses guerriers pour battre l’armée royale à Emmaüs, 
${}^{4}profitant de ce que les troupes ennemies se trouvaient encore dispersées en dehors du camp. 
${}^{5}Durant la nuit, Gorgias pénétra dans le camp de Judas et n’y trouva personne. Il se mit à chercher dans la montagne, car il disait : « Ces gens-là fuient devant nous. »
${}^{6}Au moment où le jour se levait, on put voir Judas dans la plaine avec trois mille hommes, mais ils n’avaient pas les armures ni les glaives qu’ils auraient voulu. 
${}^{7}Ils apercevaient le camp des païens, puissant et fortifié. Les cavaliers qui l’entouraient étaient tous des gens experts au combat. 
${}^{8}Judas dit aux hommes qui l’accompagnaient : « Ne craignez pas leur grand nombre et ne redoutez pas leur assaut. 
${}^{9}Rappelez-vous que nos pères furent sauvés à la mer Rouge, quand Pharaon les poursuivait avec son armée. 
${}^{10}Maintenant, crions vers le Ciel : s’il veut bien de nous, il se souviendra de l’Alliance avec nos pères et il écrasera aujourd’hui cette armée, sous nos yeux. 
${}^{11}Alors, toutes les nations sauront qu’il y a un rédempteur et un sauveur pour Israël. » 
${}^{12}Les étrangers levèrent les yeux et, voyant que l’on marchait contre eux, 
${}^{13}ils sortirent du camp pour livrer bataille. Les hommes de Judas sonnèrent de la trompette 
${}^{14}et engagèrent le combat. Les païens furent battus et s’enfuirent vers la plaine, 
${}^{15}mais tous ceux qui étaient restés en arrière tombèrent sous l’épée. On les poursuivit jusqu’à Gazara, jusqu’aux plaines de l’Idumée, d’Azôt et de Jamnia : environ trois mille d’entre eux tombèrent. 
${}^{16}Revenu de la poursuite avec sa troupe, 
${}^{17}Judas dit au peuple : « Ne soyez pas avides de butin, car un autre combat nous attend. 
${}^{18}Gorgias et sa troupe sont dans la montagne, tout près de nous. Maintenant donc, tenez tête à nos ennemis et combattez-les. Après cela, vous ramasserez le butin en toute liberté. » 
${}^{19}À peine Judas achevait-il sa phrase que l’on put voir au sommet de la montagne une section ennemie en train de guetter. 
${}^{20}Ces hommes virent que leur armée avait été mise en déroute et que le camp était en flammes : la fumée, encore visible, révélait ce qui s’était passé. 
${}^{21}Ce spectacle les remplit d’effroi. Voyant que, dans la plaine, l’armée de Judas se tenait prête pour la bataille, 
${}^{22}ils s’enfuirent tous au pays des Philistins. 
${}^{23}Alors, Judas revint pour le pillage du camp. Ils ramassèrent beaucoup d’or et d’argent, des étoffes de pourpre violette et de pourpre marine, ainsi que de grandes richesses. 
${}^{24}À leur retour, ils louaient et bénissaient le Ciel en chantant : « Il est bon, éternelle est sa miséricorde. » 
${}^{25}Ce jour-là, il y eut une grande délivrance en Israël.
${}^{26}Ceux des étrangers qui s’étaient sauvés allèrent chez Lysias pour lui rapporter tous ces événements. 
${}^{27}En les entendant, celui-ci fut bouleversé et très irrité : les choses ne s’étaient pas passées en Israël comme il aurait voulu, et le résultat n’était pas ce que lui avait ordonné le roi.
${}^{28}L’année suivante, Lysias rassembla soixante mille hommes d’élite et cinq mille cavaliers, pour repartir en campagne. 
${}^{29}Ils vinrent en Idumée et campèrent à Bethsour. Judas se porta à leur rencontre avec dix mille hommes. 
${}^{30}Quand il vit la puissance de l’armée ennemie, Judas fit cette prière : « Tu es béni, Sauveur d’Israël, toi qui as brisé l’élan du puissant guerrier par la main de ton serviteur David, toi qui as livré le camp des Philistins aux mains de Jonathan, fils de Saül, et aux mains de son écuyer. 
${}^{31}De même, enferme cette armée entre les mains de ton peuple Israël. Qu’ils aient honte de leurs troupes et de leurs cavaliers ! 
${}^{32}Mets en eux la peur, fais fondre l’assurance qu’ils placent dans leur force. Qu’ils soient ébranlés par une défaite écrasante ! 
${}^{33}Renverse-les par l’épée de ceux qui t’aiment. Alors, tous ceux qui connaissent ton nom te célébreront par des hymnes. » 
${}^{34}Le combat s’engagea et, dans le corps à corps, l’armée de Lysias perdit près de cinq mille hommes. 
${}^{35}Lorsqu’il vit la déroute de son armée, et combien l’armée de Judas était devenue intrépide, prête à vivre ou à mourir avec le même courage, Lysias repartit pour Antioche. Là, il recruta une armée de mercenaires, afin de revenir en force en Judée.
${}^{36}Alors Judas et ses frères déclarèrent : « Voilà nos ennemis écrasés, montons purifier le Lieu saint et en faire la dédicace. » 
${}^{37}Toute l’armée se rassembla, et ils montèrent à la montagne de Sion. 
${}^{38}Là, ils virent le sanctuaire dévasté, l’autel profané, les portes complètement brûlées. Dans les parvis, la végétation avait poussé comme dans un bois ou sur une montagne, et les salles des prêtres étaient détruites. 
${}^{39}Ils déchirèrent leurs tuniques, se frappèrent la poitrine, répandirent de la cendre sur leur tête 
${}^{40}et tombèrent, la face contre terre. Au signal donné par les trompettes, ils poussèrent des cris vers le Ciel.
${}^{41}Alors, Judas donna l’ordre à quelques hommes de combattre les occupants de la citadelle, pendant la purification du Lieu saint. 
${}^{42}Il choisit des prêtres irréprochables et très attachés à la Loi. 
${}^{43}Ceux-ci purifièrent le Lieu saint et emportèrent les pierres souillées dans un endroit impur. 
${}^{44}Ils se demandèrent ce qu’il fallait faire de l’autel des holocaustes, qui avait été profané. 
${}^{45}Ils eurent la bonne idée de le démolir, pour écarter tout reproche, du fait que les païens l’avaient souillé. Ils démolirent donc l’autel, 
${}^{46}et transportèrent les pierres sur la montagne de la Demeure, dans un endroit approprié, en attendant la venue d’un prophète qui se prononcerait à leur sujet. 
${}^{47}Conformément à la Loi, ils prirent des pierres non taillées et bâtirent un autel nouveau, sur le modèle du précédent. 
${}^{48}Ils restaurèrent aussi le Lieu saint et l’intérieur de la Demeure ; ils sanctifièrent les parvis. 
${}^{49}Ils introduisirent au cœur du sanctuaire les nouveaux ustensiles sacrés qu’ils avaient fabriqués, le chandelier, l’autel des parfums et la table des offrandes. 
${}^{50}Ils firent brûler de l’encens sur l’autel et allumèrent les lampes du chandelier, qui illuminèrent le sanctuaire. 
${}^{51}Ils placèrent les pains de l’offrande sur la table et tendirent les rideaux. Ils achevèrent ainsi tous les travaux qu’ils avaient entrepris.
${}^{52}Le vingt-cinquième jour du neuvième mois, c’est-à-dire le mois de Kisléou, en l’année 148, de grand matin, 
${}^{53}les prêtres offrirent le sacrifice prescrit par la Loi sur le nouvel autel qu’ils avaient construit. 
${}^{54}On fit la dédicace de l’autel au chant des hymnes, au son des cithares, des harpes et des cymbales. C’était juste l’anniversaire du jour où les païens l’avaient profané. 
${}^{55}Le peuple entier se prosterna la face contre terre pour adorer, puis ils bénirent le Ciel qui avait fait aboutir leur effort. 
${}^{56}Pendant huit jours, ils célébrèrent la dédicace de l’autel, en offrant, dans l’allégresse, des holocaustes, des sacrifices de communion et d’action de grâce. 
${}^{57}Ils ornèrent la façade du Temple de couronnes d’or et de boucliers, ils en restaurèrent les entrées et les salles et y replacèrent des portes. 
${}^{58}Il y eut une grande allégresse dans le peuple, et l’humiliation infligée par les païens fut effacée. 
${}^{59}Judas Maccabée\\décida, avec ses frères et toute l’assemblée d’Israël, que l’anniversaire de la dédicace de l’autel serait célébré pendant huit jours chaque année à cette date\\, dans la joie et l’allégresse.
${}^{60}En ce temps-là, on édifia tout autour de la montagne de Sion un rempart élevé, avec de puissantes tours, de peur que les païens ne viennent piétiner ces lieux comme auparavant. 
${}^{61}Judas y établit une garnison. Il fortifia aussi Bethsour, pour que le peuple possède une forteresse en face de l’Idumée.
      
         
      \bchapter{}
      \begin{verse}
${}^{1}Lorsque les nations d’alentour apprirent que l’autel des sacrifices avait été reconstruit et le sanctuaire restauré dans son état antérieur, elles en furent très irritées. 
${}^{2}Elles prirent la décision de supprimer les descendants de Jacob qui vivaient au milieu d’elles, et commencèrent à en tuer parmi le peuple pour les exterminer.
${}^{3}Judas faisait la guerre aux fils d’Ésaü en Idumée, au pays d’Akrabattène, parce qu’ils encerclaient Israël. Il les frappa durement, les refoula et ramassa le butin. 
${}^{4}Puis il se souvint de la méchanceté des fils de Baïane. Ils étaient un piège et un obstacle pour le peuple, par les embuscades qu’ils dressaient sur les chemins. 
${}^{5}Il les enferma dans leurs tours, les assiégea et les voua à l’anathème. Il incendia leurs tours avec tous ceux qui s’y trouvaient. 
${}^{6}Ensuite, il se rendit chez les fils d’Ammone. Il y trouva une forte troupe et un peuple nombreux, conduit par Timothée. 
${}^{7}Il leur livra un grand nombre de combats, si bien que ceux-ci furent écrasés devant lui et vaincus. 
${}^{8}Il emporta d’assaut Jazer, ainsi que les villages qui en dépendent, puis il revint en Judée.
${}^{9}Les païens de Galaad se liguèrent contre les gens d’Israël établis sur leur territoire, afin de les exterminer. Ceux-ci se réfugièrent dans la forteresse de Dathéma. 
${}^{10}De là, ils envoyèrent des lettres à Judas et à ses frères, pour leur dire : « Les nations païennes d’alentour sont liguées contre nous, afin de nous exterminer. 
${}^{11}Elles se préparent à venir prendre d’assaut la forteresse où nous nous sommes réfugiés, et c’est Timothée qui commande leur armée. 
${}^{12}Viens donc maintenant, arrache-nous à leur main, car beaucoup d’entre nous sont déjà tombés. 
${}^{13}Tous nos frères du pays de Toubias ont été mis à mort. On a emmené en captivité leurs femmes et leurs enfants, confisqué leurs biens, et fait périr en ces lieux près d’un millier d’hommes. » 
${}^{14}La lecture de ces lettres n’était pas encore achevée, que d’autres messagers survinrent. Ils venaient de Galilée. Leurs vêtements étaient déchirés, et ils apportaient les mêmes nouvelles. 
${}^{15}« Ceux de Ptolémaïs, disaient-ils, de Tyr et de Sidon se sont ligués contre nous, de même que toute la Galilée des Étrangers, pour nous faire disparaître. »
${}^{16}En apprenant ces nouvelles, Judas et le peuple réunirent une grande assemblée, pour délibérer sur la manière d’aider leurs frères opprimés et assaillis. 
${}^{17}Judas dit à son frère Simon : « Choisis-toi des hommes et va délivrer tes frères qui sont en Galilée. Moi, j’irai avec mon frère Jonathan au pays de Galaad. » 
${}^{18}Il laissa en Judée Joseph, fils de Zacharie, et Azarias, chef du peuple, avec le reste de l’armée, pour assurer la garde. 
${}^{19}Il leur donna ces instructions : « Gouvernez ce peuple, mais n’engagez pas de combat avec les païens, jusqu’à notre retour. » 
${}^{20}Trois mille hommes furent détachés pour accompagner Simon en Galilée, tandis que Judas emmenait huit mille hommes au pays de Galaad.
${}^{21}Simon se rendit en Galilée et livra de nombreux combats aux païens, qui furent écrasés devant lui. 
${}^{22}Il les poursuivit jusqu’à la porte de Ptolémaïs. Trois mille hommes environ tombèrent parmi les païens, et Simon ramassa le butin. 
${}^{23}Il recueillit ses frères de Galilée et d’Arbatta, avec leurs femmes, leurs enfants et toutes leurs possessions. Il les conduisit en Judée dans l’allégresse.
${}^{24}De leur côté, Judas Maccabée et son frère Jonathan franchirent le Jourdain et marchèrent pendant trois jours dans le désert. 
${}^{25}Ils rencontrèrent les Nabatéens, qui les accueillirent de manière pacifique et leur racontèrent tout ce qui était arrivé à leurs frères, au pays de Galaad : 
${}^{26}un grand nombre d’entre eux se trouvaient enfermés à Bossorra et à Bossor près d’Aléma, à Kaspho, à Maked et à Carnaïn, qui sont toutes de grandes et fortes villes. 
${}^{27}Il y en avait aussi dans les autres villes du pays de Galaad. L’ennemi avait pris ses dispositions pour donner l’assaut le lendemain à ces forteresses, s’en emparer et exterminer en un jour tous ceux qui s’y trouvaient. 
${}^{28}Aussitôt, Judas et son armée prirent à travers le désert la direction de Bossorra. Ils s’en emparèrent, passèrent toute la population masculine au fil de l’épée, ramassèrent tout le butin et incendièrent la ville. 
${}^{29}De là, ils repartirent de nuit et marchèrent jusqu’aux abords de la forteresse de Dathéma. 
${}^{30}Au point du jour, en levant les yeux, ils aperçurent une foule innombrable, qui dressait des échelles et des machines de guerre pour s’emparer de la forteresse ; déjà on attaquait. 
${}^{31}Judas vit que le combat était engagé : le cri de la ville s’élevait jusqu’au ciel, au son des trompettes et des hurlements. 
${}^{32}Il dit aux hommes de son armée : « Combattez aujourd’hui pour nos frères. »
${}^{33}Il les fit marcher en trois bataillons sur les arrières de l’ennemi. Ils sonnèrent de la trompette et prièrent à grands cris. 
${}^{34}Alors, l’armée de Timothée reconnut que c’était Maccabée, et elle s’enfuit à son approche. Judas les frappa durement, et huit mille hommes environ tombèrent ce jour-là. 
${}^{35}Ensuite, il se tourna vers Aléma, lui donna l’assaut et s’en empara. Il en tua la population masculine, ramassa le butin et incendia la ville. 
${}^{36}De là, il partit s’emparer de Kaspho, de Maked, de Bossor et des autres villes du pays de Galaad.
${}^{37}Quant à Timothée, après ces événements, il rassembla une autre armée et prit position en face de Raphone, sur l’autre rive du torrent. 
${}^{38}Judas envoya des hommes observer le camp ennemi, et ils lui firent ce rapport : « Tous les païens des alentours sont rassemblés auprès de Timothée. C’est une armée très nombreuse : 
${}^{39}même des Arabes ont été recrutés comme auxiliaires. Ils campent sur l’autre rive du torrent, prêts à venir te combattre. » Judas se porta à leur rencontre 
${}^{40}et, avec son armée, il s’approcha du torrent. Alors, Timothée dit aux chefs de son armée : « S’il traverse le premier, nous ne pourrons pas lui résister, car il aura un grand avantage sur nous. 
${}^{41}Mais s’il a peur et s’arrête de l’autre côté de la rivière, nous traverserons et nous l’emporterons sur lui. » 
${}^{42}Lorsqu’il arriva au bord du torrent, Judas y plaça les scribes du peuple et leur donna cet ordre : « Ne laissez personne s’installer ici, mais que tous les hommes aillent au combat. » 
${}^{43}Il traversa le premier à la rencontre de l’ennemi, et tout le peuple le suivit. Tous les païens furent écrasés devant lui. Ils jetèrent leurs armes et s’enfuirent vers le lieu de culte de Carnaïn. 
${}^{44}Cette ville, les hommes de Judas la prirent d’assaut et ils mirent le feu à son lieu de culte, avec tous ceux qui s’y trouvaient. Carnaïn fut renversée. Dès ce moment, il ne fut plus possible de résister à Judas Maccabée.
${}^{45}Alors, Judas rassembla tous ceux d’Israël qui vivaient au pays de Galaad, du plus petit jusqu’au plus grand, avec leurs femmes, leurs enfants et tout ce qu’ils possédaient. Cette troupe immense se mit en route vers la Judée. 
${}^{46}Ils arrivèrent à Éphrone, ville importante et très puissante qui se trouvait sur leur chemin. On ne pouvait la contourner ni à droite ni à gauche ; il fallait la traverser. 
${}^{47}Les gens de la ville leur refusèrent le passage en barricadant les portes avec des blocs de pierre. 
${}^{48}Judas leur envoya des messagers de paix, pour leur dire : « Nous allons traverser votre pays pour aller dans le nôtre. Personne ne vous fera de mal. Nous ne ferons que passer à pied. » Mais ils ne voulaient pas lui ouvrir. 
${}^{49}Alors, Judas fit proclamer dans le camp l’ordre de prendre position, chacun à l’endroit où il se trouvait. 
${}^{50}Les soldats prirent position, et Judas attaqua la ville tout ce jour-là et toute la nuit. La ville tomba entre ses mains. 
${}^{51}Il fit passer toute la population masculine au fil de l’épée, il détruisit la ville de fond en comble, en prit le butin et la traversa en marchant sur les corps des tués. 
${}^{52}On franchit le Jourdain, en direction de la Grande Plaine qui se trouve en face de Bethsane. 
${}^{53}Tout au long du chemin, Judas allait et venait pour regrouper les retardataires et encourager le peuple, jusqu’à son arrivée en Judée. 
${}^{54}Ils gravirent la montagne de Sion, tout remplis de joie et d’allégresse. Là, ils offrirent des holocaustes, car ils étaient revenus en paix, sans avoir perdu aucun des leurs.
${}^{55}À l’époque où Judas et Jonathan étaient au pays de Galaad et leur frère Simon en Galilée devant Ptolémaïs, 
${}^{56}les deux chefs de l’armée restée en Judée, Joseph, fils de Zacharie, et Azarias, entendirent parler de leurs actes de bravoure et des combats qu’ils avaient livrés. 
${}^{57}Ils se dirent : « Nous aussi, faisons-nous un nom, et allons combattre les païens des alentours. » 
${}^{58}Ils donnèrent des ordres aux hommes de l’armée qui étaient avec eux et marchèrent sur Jamnia. 
${}^{59}Gorgias sortit de la ville avec ses hommes pour engager le combat contre eux. 
${}^{60}Joseph et Azarias furent mis en déroute. On les poursuivit jusqu’aux frontières de la Judée. Environ deux mille hommes d’Israël tombèrent ce jour-là. 
${}^{61}Ce fut une grande déroute pour le peuple, car ils avaient désobéi à Judas et à ses frères, dans l’idée d’accomplir, eux aussi, des actes de bravoure. 
${}^{62}Mais ils n’étaient pas de la même race que ces hommes sur qui reposait le salut d’Israël.
${}^{63}La renommée du vaillant Judas et de ses frères devint très grande dans tout Israël et dans toutes les nations où l’on entendait citer leur nom. 
${}^{64}On se pressait autour d’eux pour les acclamer. 
${}^{65}Judas repartit avec ses frères pour combattre les fils d’Ésaü dans la région du Sud. Il frappa Hébron et les villages qui en dépendent, il démolit ses fortifications et incendia les tours de ses remparts. 
${}^{66}Puis il se mit en marche vers le pays des Philistins et traversa la ville de Marisa. 
${}^{67}Ce jour-là, il y eut des prêtres qui tombèrent au combat : dans un geste inconsidéré, pour faire acte de bravoure, ils étaient allés combattre eux aussi. 
${}^{68}Judas se tourna ensuite vers Azôt, dans la région des Philistins. Il renversa leurs autels, fit brûler les images sculptées de leurs dieux et ramassa le butin de leurs villes. Après cela, il revint en Judée.
      
         
      \bchapter{}
      \begin{verse}
${}^{1}Le roi Antiocos parcourait le haut pays. Il apprit alors qu’il y avait en Perse une ville, Élymaïs, fameuse par ses richesses, son argent et son or ; 
${}^{2} son temple, extrêmement riche, contenait des casques en or, des cuirasses et des armes, laissés là par Alexandre, fils de Philippe et roi de Macédoine, qui régna le premier sur les Grecs. 
${}^{3} Antiocos arriva, et il tenta de prendre la ville et de la piller, mais il n’y réussit pas, parce que les habitants avaient été informés de son projet. 
${}^{4} Ils lui résistèrent et livrèrent bataille, si bien qu’il prit la fuite et battit en retraite, accablé de chagrin, pour retourner à Babylone. 
${}^{5} Il était encore en Perse quand on vint lui annoncer la déroute des troupes qui avaient pénétré en Judée ; 
${}^{6} Lysias, en particulier, qui avait été envoyé avec un important matériel, avait fait demi-tour devant les Juifs ; ceux-ci s’étaient renforcés grâce aux armes, au matériel et au butin saisis sur les troupes qu’ils avaient battues ; 
${}^{7} ils avaient renversé l’Abomination\\qu’Antiocos avait élevée à Jérusalem sur l’autel ; enfin, ils avaient reconstruit comme auparavant de hautes murailles autour du sanctuaire et autour de la ville royale de Bethsour. 
${}^{8} Quand le roi apprit ces nouvelles, il fut saisi de frayeur et profondément ébranlé. Il s’écroula sur son lit et tomba malade sous le coup du chagrin, parce que les événements n’avaient pas répondu à son attente. 
${}^{9} Il resta ainsi pendant plusieurs jours, car son profond chagrin se renouvelait sans cesse. Lorsqu’il se rendit compte qu’il allait mourir, 
${}^{10} il appela tous ses amis et leur dit : « Le sommeil s’est éloigné de mes yeux ; l’inquiétude accable mon cœur, 
${}^{11} et je me dis : À quelle profonde détresse en suis-je arrivé ? Dans quel abîme suis-je plongé maintenant ? J’étais bon et aimé au temps de ma puissance. 
${}^{12} Mais maintenant je me rappelle le mal que j’ai fait à Jérusalem : tous les objets d’argent et d’or qui s’y trouvaient, je les ai pris ; j’ai fait exterminer les habitants de la Judée sans aucun motif. 
${}^{13} Je reconnais que tous mes malheurs viennent de là, et voici que je meurs dans un profond chagrin sur une terre étrangère. »
${}^{14}Il appela Philippe, l’un de ses amis, et l’établit à la tête de tout son royaume. 
${}^{15}Il lui donna son diadème, son vêtement royal et son anneau, pour le charger de l’éducation de son fils Antiocos, en vue de la royauté. 
${}^{16}Le roi Antiocos mourut en ce lieu, en l’an 149 de l’empire grec. 
${}^{17}Quant à Lysias, à la nouvelle de la mort du roi, il fit monter sur le trône le jeune Antiocos, dont il assurait l’éducation depuis l’enfance. Il le surnomma Eupator (c’est-à-dire : né d’un père noble).
${}^{18}À Jérusalem, les occupants de la citadelle bloquaient Israël autour du Lieu saint. Ils cherchaient à leur faire du mal en toute occasion et constituaient un appui pour les païens. 
${}^{19}Décidé à les exterminer, Judas convoqua tout le peuple pour les assiéger. 
${}^{20}On se rassembla et on mit le siège devant la citadelle en l’an 150. On construisit des catapultes et d’autres machines de guerre. 
${}^{21}Mais certains des assiégés parvinrent à rompre le blocus. Quelques impies, des hommes d’Israël, se joignirent à eux. 
${}^{22}Ils se rendirent chez le nouveau roi et lui dirent : « Combien de temps vas-tu attendre pour faire justice et venger nos frères ? 
${}^{23}Nous, nous étions heureux de servir ton père, de nous conduire selon ses ordres et d’observer ses décrets. 
${}^{24}Pour cette raison, nos compatriotes nous ont traités comme des étrangers. Bien plus, ils ont tué ceux d’entre nous qu’ils trouvaient et ils ont pillé nos biens. 
${}^{25}D’ailleurs, ce n’est pas seulement sur nous qu’ils ont porté la main, mais aussi sur tous les pays voisins. 
${}^{26}Aujourd’hui, ils ont pris position autour de la citadelle de Jérusalem pour s’en emparer. Ils ont fortifié le sanctuaire, ainsi que la ville de Bethsour. 
${}^{27}Si tu ne te hâtes pas de les devancer, ils en feront encore davantage et tu ne pourras plus les arrêter. »
${}^{28}À ces paroles, le roi se mit en colère. Il réunit tous ses amis, les commandants de son armée et ceux de la cavalerie. 
${}^{29}Des troupes mercenaires, venues des royaumes étrangers et des îles de la mer, se joignirent à eux. 
${}^{30}Ses forces s’élevaient ainsi à cent mille fantassins, vingt mille cavaliers et trente-deux éléphants de combat. 
${}^{31}Ils traversèrent l’Idumée et mirent le siège devant Bethsour. Pendant de nombreux jours, ils combattirent cette ville. Ils fabriquèrent aussi des machines de guerre, mais les assiégés firent une sortie, les incendièrent et se défendirent avec courage. 
${}^{32}Alors, Judas laissa la citadelle et prit position à Bethzakaria, en face du camp du roi. 
${}^{33}Le lendemain, à l’aube, le roi lança son armée pleine d’ardeur sur le chemin de Bethzakaria. Les troupes se préparèrent à l’attaque et on sonna de la trompette. 
${}^{34}On présenta aux éléphants du jus de raisin et de mûres, pour les exciter au combat. 
${}^{35}Les bêtes furent réparties entre les bataillons. Près de chacune se tenaient mille hommes cuirassés de mailles et coiffés d’un casque de bronze, ainsi que cinq cents cavaliers d’élite. 
${}^{36}Ceux-ci prévenaient tous les mouvements de la bête et l’accompagnaient partout, sans jamais s’en éloigner. 
${}^{37}Pour protéger chaque bête, une solide tour de bois avait été fixée sur elle par des sangles. À l’intérieur se tenaient les trois guerriers qui combattaient sur la bête, en plus de son cornac. 
${}^{38}Le roi disposa le reste de la cavalerie sur les deux flancs de l’armée, pour harceler l’ennemi et protéger les bataillons. 
${}^{39}Quand le soleil frappa de sa lumière les boucliers d’or et de bronze, les montagnes furent illuminées par leur reflet et elles resplendirent comme des torches de feu. 
${}^{40}Une partie de l’armée royale se déploya sur les crêtes des montagnes et une autre en contrebas. Ils avançaient avec assurance et en bon ordre. 
${}^{41}L’angoisse s’emparait de tous ceux qui entendaient la rumeur de cette multitude, le bruit de sa marche et le cliquetis des armes entrechoquées, car cette armée était vraiment immense et très puissante. 
${}^{42}Judas et son armée s’avancèrent pour engager le combat : six cents hommes de l’armée du roi tombèrent. 
${}^{43}Éléazar, surnommé Awarane, le frère de Judas, vit que l’une des bêtes était équipée d’un harnais royal cuirassé et surpassait toutes les autres : il supposa que le roi était dessus. 
${}^{44}Alors il se sacrifia pour sauver son peuple et acquérir un nom immortel. 
${}^{45}Il se précipita avec intrépidité vers la bête, au milieu du bataillon, tuant à droite et à gauche, si bien qu’on s’écarta devant lui de part et d’autre. 
${}^{46}Il se glissa sous l’éléphant et, par en dessous, lui porta un coup mortel. La bête s’écroula sur lui et il mourut sur place. 
${}^{47}Voyant alors la puissance du roi et l’ardeur de ses forces, ceux d’Israël se replièrent.
${}^{48}L’armée royale monta vers Jérusalem à leur rencontre. Toute la Judée et la montagne de Sion furent mises en état de siège. 
${}^{49}Le roi fit la paix avec les gens de Bethsour, qui sortirent de la ville car ils n’avaient pas suffisamment de vivres pour y soutenir un siège. C’était en effet l’année du repos sabbatique. 
${}^{50}Le roi s’empara de Bethsour et y établit une garnison. 
${}^{51}Ensuite, il assiégea le sanctuaire de Jérusalem pendant de nombreux jours. Il installa des catapultes et des machines de guerre, lance-flammes, lance-pierres, lance-flèches et frondes. 
${}^{52}Pour riposter à ces machines, les assiégés en fabriquèrent aussi et le combat se prolongea pendant de nombreux jours. 
${}^{53}Mais il n’y avait pas de provisions dans les dépôts, car c’était la septième année. En outre, les réfugiés qui étaient arrivés en Judée, après avoir été sauvés de la main des païens, avaient consommé les dernières réserves. 
${}^{54}On ne laissa donc qu’une poignée d’hommes dans le Lieu saint, parce qu’on était en proie à la famine. Les autres se dispersèrent, chacun de son côté.
${}^{55}Or, Philippe, que le roi Antiocos avait désigné avant de mourir pour élever son fils Antiocos en vue de la royauté, 
${}^{56}était revenu de Perse et de Médie avec les troupes qui avaient accompagné le roi. Il cherchait à se mettre à la tête des affaires. 
${}^{57}À cette nouvelle, Lysias se hâta de donner le signal du départ. Il dit au roi, aux généraux et aux hommes : « Nous nous épuisons de jour en jour, nous n’avons que peu de vivres, la place que nous assiégeons est bien fortifiée et les affaires du royaume reposent sur nous. 
${}^{58}Tendons maintenant la main droite à ces hommes, faisons la paix avec eux et avec toute leur nation. 
${}^{59}Accordons-leur de vivre selon leurs coutumes, comme auparavant. En effet, c’est parce que nous avions aboli leurs coutumes, qu’ils se sont mis en colère et qu’ils ont fait tout cela. » 
${}^{60}Ce discours fut apprécié par le roi et par les chefs. Lysias envoya à ceux d’Israël des propositions de paix et ceux-ci les acceptèrent. 
${}^{61}Le roi et les chefs s’engagèrent envers eux par serment. Alors, les assiégés sortirent de la forteresse, 
${}^{62}et le roi fit son entrée sur la montagne de Sion. Mais lorsqu’il vit les fortifications du lieu, il viola son serment et ordonna de démolir tout le mur d’enceinte. 
${}^{63}Puis il partit en toute hâte et retourna à Antioche, où il trouva Philippe qui s’était rendu maître de la ville. Il lui livra bataille et s’empara de la ville par la force.
      
         
      \bchapter{}
      \begin{verse}
${}^{1}En l’an 151 de l’empire grec, Démétrios, fils de Séleucos, quitta Rome et se rendit avec une poignée d’hommes dans une ville du littoral, où il inaugura son règne. 
${}^{2}Or, tandis qu’il s’avançait vers Antioche, résidence royale de ses pères, l’armée captura Antiocos et Lysias, pour les lui amener. 
${}^{3}Il en fut informé et dit : « Ne me faites pas voir leur visage. » 
${}^{4}L’armée les tua, et Démétrios s’assit sur le trône royal. 
${}^{5}Alors, tous les hommes sans foi ni loi que l’on pouvait trouver en Israël se rendirent auprès de lui, sous la conduite d’Alkime, qui convoitait la charge de grand prêtre. 
${}^{6}Ils se mirent à accuser leur propre peuple devant le roi, en disant : « Judas et ses frères ont fait périr tous tes amis ; ils nous ont dispersés hors de notre pays. 
${}^{7}Envoie donc maintenant un homme de confiance : qu’il vienne voir tous les ravages dont Judas s’est rendu coupable envers nous et envers le domaine du roi, et qu’il les punisse, lui, ses frères et tous leurs auxiliaires. » 
${}^{8}Le roi choisit Bacchidès, un des amis du roi, gouverneur de la province de Transeuphratène, un grand du royaume, fidèle au roi.
      
         
${}^{9}Il l’envoya avec Alkime, l’impie, auquel il conféra la charge de grand prêtre et prescrivit de se venger des fils d’Israël. 
${}^{10}Ils arrivèrent en Judée avec une troupe nombreuse. Ils envoyèrent des messagers auprès de Judas et de ses frères, pour leur adresser de fausses paroles de paix. 
${}^{11}Mais en voyant qu’ils étaient accompagnés d’une troupe nombreuse, les fils d’Israël n’accordèrent aucun crédit à leur discours. 
${}^{12}Un groupe de scribes se réunit toutefois chez Alkime et Bacchidès, pour rechercher une solution équitable. 
${}^{13}Parmi les fils d’Israël, les Assidéens furent les premiers à solliciter la paix, 
${}^{14}car ils se disaient : « C’est un prêtre de la descendance d’Aaron, qui est venu avec les troupes ; il ne commettra pas d’injustice envers nous. » 
${}^{15}Bacchidès leur adressa des paroles de paix et leur fit ce serment : « Nous ne chercherons à vous faire aucun mal, ni à vous, ni à vos amis. » 
${}^{16}Ils le crurent, mais lui, il fit saisir soixante hommes de leur groupe et les fit périr en un seul jour, selon la parole de l’Écriture :
${}^{17}« Ils ont dispersé les corpsde tes fidèles,
        \\ils ont versé leur sang aux alentours de Jérusalem.
        \\Et personne pour les ensevelir ! »
${}^{18}Alors, tout le peuple fut saisi de crainte et de terreur, et l’on disait : « Il n’y a chez eux ni vérité ni justice, car ils ont violé leur engagement et le serment qu’ils avaient fait. »
${}^{19}Bacchidès quitta Jérusalem et prit position à Bethzaïth. Il envoya capturer beaucoup d’hommes qui pourtant s’étaient ralliés à lui, ainsi que quelques-uns du peuple ; il les fit immoler et jeter dans le grand puits. 
${}^{20}Il remit le gouvernement de la province à Alkime et lui laissa une troupe armée pour le soutenir. Bacchidès revint auprès du roi, 
${}^{21}et Alkime lutta pour se faire admettre comme grand prêtre. 
${}^{22}Tous ceux qui troublaient le peuple se groupèrent autour de lui. Ils se rendirent maîtres de la Judée et causèrent beaucoup de tort en Israël. 
${}^{23}Judas se rendit compte que tout le mal causé par Alkime et ses compagnons surpassait celui que les païens avaient infligé aux fils d’Israël. 
${}^{24}Il parcourut à la ronde tous les territoires de la Judée, pour tirer vengeance des déserteurs et les empêcher de circuler dans le pays. 
${}^{25}Alkime vit que Judas et ses compagnons étaient devenus plus forts. Alors, sachant qu’il ne pourrait leur résister, il retourna chez le roi et les accabla des pires accusations.
${}^{26}Le roi envoya Nicanor, l’un de ses généraux parmi les plus illustres, un homme qui haïssait profondément Israël. Il lui donna l’ordre d’exterminer ce peuple. 
${}^{27}Nicanor vint à Jérusalem avec une troupe nombreuse. Il adressa à Judas et à ses frères de fausses paroles de paix. 
${}^{28}Il leur dit : « Qu’il n’y ait pas de combat entre nous. Je viendrai avec une petite escorte pour une entrevue pacifique. » 
${}^{29}En arrivant chez Judas, il le salua amicalement, mais ses soldats se tenaient prêts à capturer Judas. 
${}^{30}Celui-ci se rendit compte que Nicanor était venu chez lui avec des intentions perfides. Il eut très peur de lui et refusa l’entrevue. 
${}^{31}Nicanor comprit que son plan était dévoilé. Alors, il marcha contre Judas pour le combattre près de Kapharsalama. 
${}^{32}Du côté de Nicanor, cinq cents hommes environ tombèrent, tandis que les autres s’enfuirent vers la Cité de David.
${}^{33}Après ces événements, Nicanor se rendit à la montagne de Sion. Quelques prêtres sortirent du Lieu saint, avec des anciens du peuple, pour le saluer de manière pacifique et lui montrer l’holocauste qu’on offrait pour le roi. 
${}^{34}Mais lui les tourna en dérision, les outragea et proféra des paroles arrogantes. 
${}^{35}Il jura avec colère : « Si Judas et son armée ne sont pas livrés maintenant entre mes mains, et que je reviens sans encombre, je mettrai le feu à cette Demeure. » Et il sortit furieux. 
${}^{36}Les prêtres rentrèrent et, debout devant l’autel et le Temple, ils fondirent en larmes. Ils dirent : 
${}^{37}« Toi qui as choisi cette Demeure pour que ton nom y soit invoqué, tu as voulu qu’elle soit pour ton peuple une maison de prière et de supplication. 
${}^{38}Exerce ta vengeance sur cet homme et sur son armée. Que l’épée les fasse périr ! Souviens-toi de leurs blasphèmes et ne leur accorde aucun repos ! »
${}^{39}Nicanor sortit de Jérusalem et prit position à Bethorone, où un contingent de l’armée syrienne vint le rejoindre. 
${}^{40}Judas prit position à Adassa, avec trois mille hommes. Il fit cette prière : 
${}^{41}« Le jour où les messagers du roi ont proféré des blasphèmes, ton ange est sorti pour frapper cent quatre-vingt-cinq mille d’entre eux. 
${}^{42}De même, aujourd’hui, écrase devant nous cette armée et fais comprendre aux survivants que leur chef a mal parlé contre ton Lieu saint. Oui, juge-le selon sa méchanceté. » 
${}^{43}Les armées engagèrent le combat le treize du mois nommé Adar. L’armée de Nicanor fut écrasée et lui-même fut tué le tout premier au combat. 
${}^{44}Lorsqu’ils virent que leur chef était tombé, les soldats de Nicanor jetèrent leurs armes et s’enfuirent. 
${}^{45}On les poursuivit sur un parcours d’une journée, depuis Adassa jusqu’aux abords de Gazara, les trompettes sonnant derrière eux pour donner le signal. 
${}^{46}De tous les villages judéens alentour, on sortit pour les prendre de flanc et les rabattre. Tous tombèrent par l’épée ; pas un seul n’en réchappa. 
${}^{47}On ramassa le butin et les dépouilles. On coupa la tête de Nicanor et sa main droite, qu’il avait levée dans un geste d’arrogance ; on les emporta pour les exposer aux abords de Jérusalem. 
${}^{48}Le peuple fut transporté de joie et fêta ce jour-là comme un grand jour d’allégresse. 
${}^{49}On décida de célébrer ce jour-là chaque année, le treize du mois nommé Adar. 
${}^{50}La Judée connut alors quelques jours de tranquillité.
      
         
      \bchapter{}
      \begin{verse}
${}^{1}Judas entendit parler des Romains : c’étaient de vaillants guerriers, mais bien disposés envers tous ceux qui se rangeaient à leurs côtés et accordant leur amitié à tous ceux qui venaient à eux. 
${}^{2}On lui raconta leurs guerres et les exploits accomplis par eux chez les Gaulois, qu’ils avaient vaincus et soumis à l’impôt. 
${}^{3}On lui raconta aussi tout ce qu’ils avaient fait dans la province d’Espagne pour s’emparer des mines d’argent et d’or qui s’y trouvaient : 
${}^{4}ils s’étaient rendus maîtres de tout ce pays grâce à leur habileté et à leur persévérance, car ce pays était fort éloigné de chez eux. Les rois venus des extrémités de la terre pour les attaquer, ils les avaient finalement écrasés en les frappant durement, tandis que d’autres devaient leur payer un tribut chaque année. 
${}^{5}Philippe et Persée, rois des Macédoniens, et tous ceux qui s’étaient dressés contre eux, ils leur avaient fait la guerre et les avaient vaincus. 
${}^{6}Antiocos le Grand, roi de l’Asie, avait entrepris de les combattre avec cent vingt éléphants, de la cavalerie, des chars et une armée très nombreuse. Lui aussi avait été écrasé devant eux. 
${}^{7}Les Romains l’avaient capturé vivant et l’avaient obligé, lui et ses successeurs, à verser un lourd tribut. Il avait dû livrer des otages et céder 
${}^{8}le pays indien, la Médie et la Lydie, parmi ses plus belles provinces, au profit du roi Eumène. 
${}^{9}Les habitants de la Grèce, eux aussi, avaient décidé d’aller exterminer les Romains. 
${}^{10}Quand ceux-ci l’apprirent, ils envoyèrent un seul général pour leur faire la guerre. Il y eut beaucoup de victimes parmi les Grecs ; leurs femmes et leurs enfants furent emmenés en captivité. Les Romains pillèrent leurs biens et se rendirent maîtres de leur pays. Ils démantelèrent leurs forteresses et les réduisirent à une servitude qui dure jusqu’à ce jour. 
${}^{11}Les autres royaumes et les îles, tous ceux qui leur avaient résisté, ils les avaient dévastés et asservis. 
${}^{12}Mais envers leurs amis et ceux qui s’appuyaient sur eux, ils gardaient leur amitié. Ils soumettaient les rois proches ou lointains, si bien qu’en entendant prononcer leur nom, tout le monde les redoutait. 
${}^{13}Ceux dont ils voulaient soutenir la royauté régnaient, mais ils déposaient aussi ceux qu’ils voulaient. Ils étaient à l’apogée de leur puissance. 
${}^{14}Pourtant, pas un seul d’entre eux n’avait porté le diadème, ni revêtu le manteau de pourpre en signe de gloire personnelle. 
${}^{15}Ils s’étaient donné un sénat dont les trois cent vingt membres se réunissaient chaque jour pour délibérer en permanence des affaires du peuple et en assurer le bon ordre. 
${}^{16}Ils confiaient chaque année à un seul homme la charge de les gouverner et d’exercer la domination sur tout leur territoire. Tous lui obéissaient, à lui seul, sans qu’il y ait chez eux ni envie ni jalousie.
      
         
${}^{17}Judas choisit Eupolème, fils de Jean, fils d’Akkôs, et Jason, fils d’Éléazar ; il les envoya à Rome pour conclure amitié et alliance. 
${}^{18}Il espérait ainsi que les Romains, voyant le joug de servitude imposé à Israël par le royaume des Grecs, l’en délivreraient. 
${}^{19}Ces hommes se rendirent donc à Rome. Au bout d’un très long voyage, ils entrèrent au Sénat et prirent la parole. Ils dirent : 
${}^{20}« Judas, celui que l’on surnomme Maccabée, ainsi que ses frères et tout le peuple des Juifs nous ont envoyés pour conclure avec vous une alliance de paix, afin d’être inscrits au nombre de vos alliés et amis. » 
${}^{21}Cette affaire parut bonne aux yeux des Romains. 
${}^{22}Voici la copie de la lettre qu’ils gravèrent sur des tables de bronze et qu’ils envoyèrent à Jérusalem pour y être un mémorial de paix et d’alliance : 
${}^{23}« Prospérité aux Romains et à la nation des Juifs, sur mer et sur terre, à jamais ! Loin d’eux l’épée et l’ennemi ! 
${}^{24}Mais si une guerre menace Rome la première, ou l’un de ses alliés en tout lieu où s’exerce sa domination, 
${}^{25}la nation des Juifs combattra avec elle de tout cœur, selon les exigences du moment. 
${}^{26}Aux combattants, on ne donnera rien, on ne fournira ni blé, ni armes, ni argent, ni vaisseaux. Ainsi en a décidé Rome. Ils tiendront leurs engagements sans rien recevoir en échange. 
${}^{27}De même, si une guerre touche d’abord la nation des Juifs, les Romains combattront avec elle de toute leur âme, selon les exigences du moment. 
${}^{28}Aux alliés, on ne donnera ni blé, ni armes, ni argent, ni vaisseaux. Ainsi en a décidé Rome. Ils tiendront leurs engagements loyalement. 
${}^{29}C’est en ces termes que les Romains ont statué avec le peuple des Juifs. 
${}^{30}Si les uns ou les autres décident d’ajouter ou de retrancher quelque chose à ces termes, ils devront le faire d’un commun accord : toutes ces additions ou suppressions seront alors valables de plein droit. 
${}^{31}Quant au roi Démétrios, qui accable de malheurs le peuple d’Israël, nous lui avons écrit en ces termes : “Pourquoi fais-tu peser ton joug sur les Juifs, nos amis et alliés ? 
${}^{32}S’ils se plaignent encore de toi, nous soutiendrons leur cause et nous te combattrons, sur mer et sur terre.” »
      
         
      \bchapter{}
      \begin{verse}
${}^{1}Le roi Démétrios apprit que Nicanor et ses troupes étaient tombés au combat. Il renvoya donc en Judée Bacchidès et Alkime, accompagnés de l’aile droite de l’armée. 
${}^{2}Ils se mirent en route pour la Galilée ; ils assiégèrent Maisalôth, dans le territoire d’Arbèles, ils s’en emparèrent et firent périr un grand nombre de vies humaines. 
${}^{3}Le premier mois de l’an 152 de l’empire grec, ils dressèrent leur camp près de Jérusalem, 
${}^{4}puis ils partirent et se dirigèrent vers Béerzaïth avec vingt mille fantassins et deux mille cavaliers. 
${}^{5}Quant à Judas, il avait dressé son camp à Élassa. Trois mille guerriers d’élite étaient avec lui. 
${}^{6}Mais ces hommes, en voyant la multitude des ennemis, furent pris de panique et beaucoup se glissèrent hors du camp ; il n’en resta plus que huit cents. 
${}^{7}Judas vit que son armée avait fondu, alors que le combat était imminent. Il en eut le cœur brisé, parce qu’il n’avait plus le temps de rassembler les siens. 
${}^{8}Désemparé, il dit à ceux qui étaient restés : « Debout ! Montons à la rencontre de nos adversaires. Peut-être pourrons-nous leur tenir tête ! » 
${}^{9}Mais eux l’en dissuadaient par ces mots : « Pour l’instant, nous ne pouvons rien faire, sinon sauver nos vies. Nous reviendrons plus tard, avec nos frères, pour reprendre la lutte. Nous sommes trop peu nombreux ! » 
${}^{10}Judas leur répliqua : « Il ne sera pas dit que j’ai choisi la fuite. Si notre heure est arrivée, mourons courageusement pour nos frères et ne laissons pas ternir notre gloire. »
${}^{11}L’armée ennemie quitta le camp et se posta en vue de l’affrontement. Sa cavalerie avait été partagée en deux groupes, ses frondeurs et ses archers marchaient en tête et ses meilleurs guerriers se trouvaient tous au premier rang. 
${}^{12}Bacchidès, lui, se tenait à l’aile droite. Les bataillons se mirent en marche sur deux fronts, au son des trompettes. Les hommes de Judas sonnèrent, eux aussi, des trompettes, 
${}^{13}et la terre fut ébranlée par le vacarme des armées. Le combat fit rage du matin jusqu’au soir. 
${}^{14}Judas vit que Bacchidès se tenait sur la droite avec la partie la plus forte de son armée. Entouré de tous les guerriers les plus ardents, 
${}^{15}il réussit à enfoncer l’aile droite et la poursuivit jusqu’à la montagne d’Azôt. 
${}^{16}En voyant la déroute de l’aile droite, ceux de l’aile gauche se rabattirent sur les pas de Judas et de ses compagnons, les prenant à revers. 
${}^{17}Le combat devint acharné et il y eut beaucoup de victimes de part et d’autre. 
${}^{18}C’est alors que Judas succomba, lui aussi. Tous les autres s’enfuirent. 
${}^{19}Jonathan et Simon emportèrent leur frère Judas. Ils l’ensevelirent dans le tombeau de ses pères, à Modine. 
${}^{20}Tout Israël le pleura et se lamenta sur lui pendant de nombreux jours, dans le deuil et l’affliction. On disait : 
${}^{21}« Comment est-il tombé, le héros qui sauvait Israël ? » 
${}^{22}Tout ce que l’on pourrait dire encore au sujet de Judas, de ses guerres, de ses exploits et de son prestige, n’a pas été mis par écrit : il y en avait trop !
${}^{23}Après la mort de Judas, on vit réapparaître sur tout le territoire d’Israël les hommes infidèles à la Loi. Tous ceux qui commettaient l’injustice se redressèrent. 
${}^{24}Comme il y avait alors une très grande famine, le pays se rallia à eux. 
${}^{25}Bacchidès choisit les impies pour régenter la province. 
${}^{26}Ces hommes se mirent à rechercher et traquer les amis de Judas, afin de les faire comparaître devant Bacchidès, qui les punissait et les tournait en dérision. 
${}^{27}Ce fut une grande épreuve pour Israël, une oppression telle qu’on n’en avait plus vue de semblable depuis la disparition des prophètes. 
${}^{28}Tous les amis de Judas se rassemblèrent et dirent à Jonathan : 
${}^{29}« Depuis la mort de ton frère Judas, il n’y a plus d’homme pareil à lui pour marcher contre l’ennemi et contre Bacchidès, contre tous ceux qui sont hostiles à notre nation. 
${}^{30}C’est pourquoi nous te choisissons aujourd’hui à sa place comme chef et comme guide, pour prendre la tête de notre combat. » 
${}^{31}Jonathan reçut donc le commandement ce jour-là. Il succéda ainsi à son frère Judas.
${}^{32}Lorsqu’il en fut informé, Bacchidès chercha à le faire périr. 
${}^{33}Sachant cela, Jonathan et son frère Simon s’enfuirent avec tous leurs compagnons au désert de Thékoé. Là, ils établirent leur campement près de l’eau de la citerne Asfar. 
${}^{34}Bacchidès en fut informé le jour du sabbat. Il se rendit lui-même avec toute son armée au-delà du Jourdain.
${}^{35}Jonathan envoya son frère Jean, responsable des équipements, demander à ses amis les Nabatéens l’autorisation de déposer chez eux ses bagages, qui étaient considérables. 
${}^{36}Mais les fils de Jambri, ceux de Mèdaba, firent une sortie. Ils s’emparèrent de Jean et de tout ce qu’il avait, et ils repartirent avec ce butin. 
${}^{37}Par la suite, on annonça à Jonathan et à son frère Simon que les fils de Jambri célébraient un grand mariage. La fiancée, qui était la fille d’un notable très important de Canaan, était escortée en grande pompe depuis Nabatha. 
${}^{38}Se souvenant de la mort sanglante de leur frère Jean, Jonathan et les siens montèrent se cacher dans un repli de la montagne. 
${}^{39}Ils levèrent les yeux et virent, au milieu d’un cortège bruyant et de tous les bagages, le fiancé, ses amis et ses frères, qui marchaient dans leur direction, avec des tambourins, des instruments de musique et des armes en grand nombre. 
${}^{40}De leur embuscade, ils se jetèrent sur eux et les massacrèrent. Il y eut beaucoup de victimes, et les survivants s’enfuirent dans la montagne. On emporta tout le butin. 
${}^{41}La noce se changea en deuil, et le son de la musique en lamentations. 
${}^{42}C’est ainsi qu’ils tirèrent vengeance du sang de leur frère. Ensuite, ils regagnèrent les marais du Jourdain.
${}^{43}Bacchidès en fut informé. Le jour du sabbat, il se rendit jusqu’aux rives du Jourdain avec une armée nombreuse. 
${}^{44}Jonathan dit à ses hommes : « Debout ! Combattons pour sauver nos vies, car aujourd’hui, ce n’est pas comme hier et les jours précédents. 
${}^{45}Voici que le combat est à la fois devant et derrière nous. De part et d’autre, c’est l’eau du Jourdain, le marais et le fourré : il n’y a pas d’endroit où s’esquiver. 
${}^{46}Criez donc maintenant vers le Ciel, afin d’échapper aux mains de nos ennemis. » 
${}^{47}Le combat s’engagea. Jonathan étendit la main pour frapper Bacchidès, mais celui-ci esquiva le coup en se rejetant en arrière. 
${}^{48}Jonathan et ses compagnons sautèrent dans le Jourdain et atteignirent l’autre rive à la nage, mais leurs ennemis ne traversèrent pas le Jourdain à leur suite. 
${}^{49}Ce jour-là, environ mille hommes succombèrent du côté de Bacchidès.
${}^{50}Celui-ci retourna à Jérusalem. Par la suite, Bacchidès fit construire des villes fortes en Judée : la forteresse de Jéricho, celles d’Emmaüs, de Bethorone et de Béthel, de Tamnatha, de Pharathone et de Tephone, avec de hauts remparts, des portes et des verrous. 
${}^{51}Il y plaça des garnisons pour harceler Israël. 
${}^{52}Il fortifia aussi la ville de Bethsour, Gazara et la citadelle ; il y plaça des troupes et des réserves de nourriture. 
${}^{53}Les fils des chefs de la province furent pris en otages et retenus sous bonne garde dans la citadelle de Jérusalem.
${}^{54}Le deuxième mois de l’an 153 de l’empire grec, Alkime, le grand prêtre, ordonna d’abattre le mur de la cour intérieure du Lieu saint, détruisant ainsi l’œuvre des prophètes. Il fit commencer la démolition. 
${}^{55}Mais, à ce moment-là, Alkime eut une attaque, si bien qu’il fallut interrompre ses travaux. Sa bouche se ferma ; atteint de paralysie, il ne fut plus capable de dire un mot, ni de donner des ordres concernant sa maison. 
${}^{56}Alkime mourut à cette époque, en proie à de vives souffrances. 
${}^{57}Voyant qu’Alkime était mort, Bacchidès retourna chez le roi, et la Judée connut deux années de tranquillité.
${}^{58}En ce temps-là, tous les hommes infidèles à la Loi se rassemblèrent pour délibérer. Ils se disaient : « Voici que Jonathan et ses hommes vivent tranquilles et sans méfiance. Faisons donc venir Bacchidès maintenant : il les arrêtera tous, en une nuit. » 
${}^{59}Ils allèrent en délibérer avec lui, 
${}^{60}et Bacchidès se mit en route avec une troupe nombreuse. En cachette, il envoya des lettres à tous ses alliés de Judée pour qu’ils arrêtent Jonathan et ses compagnons. Mais ils ne purent le faire car leur projet fut éventé. 
${}^{61}En revanche, une cinquantaine d’hommes du pays, parmi ceux qui avaient été à l’origine de ce forfait, furent pris et massacrés. 
${}^{62}Jonathan, Simon et leurs compagnons se retirèrent à Bethbassi, dans le désert. Ils rebâtirent les ruines de cette place et la fortifièrent. 
${}^{63}Bacchidès en fut informé. Il rassembla toute sa troupe et fit également appel à ses partisans de Judée. 
${}^{64}Il vint prendre position en face de Bethbassi, lui donna l’assaut pendant de nombreux jours et construisit des machines de guerre. 
${}^{65}Laissant son frère Simon dans la ville, Jonathan sortit dans la campagne avec un détachement. 
${}^{66}Il battit Odomera et ses frères, ainsi que les fils de Phasirone, dans leur campement. Ils se mirent à attaquer et à monter avec les troupes. 
${}^{67}Alors, Simon et ses hommes sortirent de la ville et incendièrent les machines de guerre. 
${}^{68}Ils combattirent Bacchidès, qui dut plier devant eux, profondément accablé parce que son plan et son expédition avaient échoué. 
${}^{69}Rempli d’un violent ressentiment contre ces hommes infidèles à la Loi qui lui avaient suggéré de venir dans la région, il en tua un grand nombre. Puis, il décida de rentrer chez lui. 
${}^{70}À cette nouvelle, Jonathan lui envoya des messagers pour conclure avec lui la paix et régler la restitution des prisonniers. 
${}^{71}Bacchidès accepta et se conforma aux propositions de Jonathan. Il lui jura de ne plus chercher à lui faire du mal, pour le reste de sa vie. 
${}^{72}Après avoir restitué les prisonniers qu’il avait auparavant capturés en Judée, il retourna dans son pays et ne revint plus sur leur territoire. 
${}^{73}L’épée se reposa en Israël, et Jonathan s’installa à Macmas, où il devint juge pour le peuple. Il fit disparaître les impies du milieu d’Israël.
      
         
      \bchapter{}
      \begin{verse}
${}^{1}En l’an 160 de l’empire grec, Alexandre Épiphane, qui se prétendait fils d’Antiocos, vint s’emparer de Ptolémaïs. Il y fut accueilli, et c’est là qu’il inaugura son règne. 
${}^{2}Apprenant cela, le roi Démétrios rassembla des troupes en très grand nombre et marcha contre lui pour le combattre. 
${}^{3}Il écrivit à Jonathan une lettre dont les paroles pacifiques étaient pleines de flatteries à son égard. 
${}^{4}Il se disait en effet : « Hâtons-nous de consolider la paix avec ces gens-là, avant que Jonathan ne la fasse avec Alexandre contre nous. 
${}^{5}Car il se souviendra de tous les malheurs dont nous l’avons accablé, lui, ses frères et sa nation. » 
${}^{6}Démétrios lui donna donc l’autorisation de rassembler des troupes, de fabriquer des armes et de se dire son allié. Il ordonna aussi qu’on remette à Jonathan les otages retenus dans la citadelle. 
${}^{7}Ce dernier vint à Jérusalem. Il lut ce message en présence de tout le peuple et des occupants de la citadelle. 
${}^{8}Ceux-ci éprouvèrent une grande crainte en apprenant que le roi avait donné à Jonathan l’autorisation de rassembler des troupes. 
${}^{9}Les gens de la citadelle lui remirent donc les otages, et Jonathan les rendit à leurs familles. 
${}^{10}Jonathan s’installa à Jérusalem. Il se mit à rebâtir et rénover la ville. 
${}^{11}Aux entrepreneurs des travaux, il ordonna de construire les remparts et l’enceinte de la montagne de Sion en pierres de taille, pour la fortifier. Et ils le firent. 
${}^{12}Les étrangers qui occupaient les places fortes construites par Bacchidès prirent la fuite. 
${}^{13}Chacun abandonna son poste pour retourner dans son pays. 
${}^{14}Seule Bethsour, devenue un lieu de refuge, ne fut pas désertée : quelques-uns de ceux qui avaient abandonné la Loi et les préceptes y demeurèrent.
${}^{15}Le roi Alexandre fut mis au courant des promesses que Démétrios avait faites à Jonathan. On lui raconta les combats et les exploits de cet homme et de ses frères, ainsi que les peines qu’ils avaient endurées. 
${}^{16}Alors, le roi s’exclama : « Trouverons-nous jamais un seul homme pareil à celui-là ? Nous ferons de lui, sans tarder, notre ami et notre allié ! » 
${}^{17}Il lui écrivit donc une lettre rédigée en ces termes :
${}^{18}« Le roi Alexandre à son frère Jonathan, salut ! 
${}^{19}Nous avons entendu dire de toi que tu es un vaillant guerrier et que tu mérites d’être notre ami. 
${}^{20}C’est pourquoi, à dater de ce jour, nous t’établissons grand prêtre de ta nation : tu auras le titre d’ami du roi et nous t’envoyons un manteau de pourpre ainsi qu’une couronne d’or ; tu seras de notre parti et tu nous garderas ton amitié. » 
${}^{21}Jonathan revêtit donc les ornements sacrés le septième mois de l’an 160 de l’empire grec, à l’occasion de la fête des Tentes. Il rassembla des troupes et fabriqua des armes en quantité.
${}^{22}Quand Démétrios apprit tout cela, il en fut contrarié. Il dit : 
${}^{23}« Qu’avons-nous donc fait pour qu’Alexandre s’empare avant nous de l’amitié des Juifs en vue d’affermir sa position ? 
${}^{24}Je vais leur adresser, moi aussi, un appel pressant. Je leur promettrai une situation élevée et des présents, afin qu’ils m’accordent leur soutien. » 
${}^{25}Il leur écrivit donc en ces termes :
      « Le roi Démétrios à la nation des Juifs, salut ! 
${}^{26}Vous avez respecté les conventions faites avec nous, et vous êtes restés nos amis. Vous n’êtes pas passés du côté de nos ennemis. Nous avons appris tout cela avec joie. 
${}^{27}Continuez à nous garder votre confiance, et nous récompenserons votre attitude par des bienfaits : 
${}^{28}nous vous accorderons de nombreuses remises d’impôts et nous vous ferons des présents. 
${}^{29}Dès maintenant, je vous libère, je décharge tous les Juifs des contributions, de la taxe sur le sel et des couronnes. 
${}^{30}D’autre part, à compter de ce jour, je fais remise à perpétuité du tiers des produits du sol et de la moitié des fruits des arbres qui me reviennent, au bénéfice du pays de Juda et des trois districts de Samarie-Galilée, qui lui sont annexés. 
${}^{31}Jérusalem sera sainte et exonérée de dîmes et de droits, ainsi que son territoire. 
${}^{32}Je renonce aussi à toute autorité sur la citadelle de Jérusalem et je la cède au grand prêtre, qui pourra y établir une garnison choisie par ses soins. 
${}^{33}Je rends gratuitement la liberté à toute personne juive qui aurait été emmenée en captivité hors du pays de Juda, n’importe où dans mon royaume. Tous, ils seront exonérés d’impôts, même pour leur bétail. 
${}^{34}Toutes les fêtes, les sabbats, les nouvelles lunes et les jours fixés pour les solennités, ainsi que les trois jours qui précèdent celles-ci et les trois jours qui les suivent, seront tous des jours d’exemption de péage et de remise des dettes pour tous les Juifs de mon royaume. 
${}^{35}Absolument personne n’aura le droit d’exiger d’eux la moindre chose ou de les inquiéter au sujet de quelque affaire que ce soit. 
${}^{36}On enrôlera des Juifs dans les armées royales, jusqu’au nombre de trente mille hommes. Ils toucheront la même solde que toutes les troupes du roi. 
${}^{37}Certains d’entre eux seront affectés aux grandes forteresses royales. D’autres seront nommés aux postes de confiance du royaume. Leurs officiers et leurs chefs seront choisis dans leurs rangs ; ils se conformeront à leurs lois, comme le roi l’a autorisé pour le pays de Juda. 
${}^{38}Quant aux trois districts de la province de Samarie qui ont été annexés à la Judée, ils seront rattachés à la Judée, de telle sorte qu’ils ne dépendent que d’un seul homme et n’obéissent qu’à la seule autorité du grand prêtre. 
${}^{39}Je fais don de Ptolémaïs et du territoire qui en dépend au Lieu saint de Jérusalem, pour couvrir les dépenses du culte. 
${}^{40}Je donne en outre, chaque année, quinze mille pièces d’argent prélevées sur les revenus du roi dans les lieux appropriés. 
${}^{41}Tout le surplus, que les fonctionnaires n’ont pas versé comme ils le faisaient les années précédentes, ils l’affecteront désormais aux travaux de la Demeure. 
${}^{42}En outre, les cinq mille pièces d’argent que l’on prélevait chaque année sur les revenus du Lieu saint, on ne les prendra plus, car ils reviennent aux prêtres qui assurent le service liturgique. 
${}^{43}Si quelqu’un se réfugie dans le temple de Jérusalem ou dans son enceinte, à cause d’une dette envers le trésor royal ou de toute autre affaire, il sera libre, avec tous les biens qu’il possède dans mon royaume. 
${}^{44}Les frais des travaux de construction et de restauration du Lieu saint seront mis au compte du roi. 
${}^{45}Les frais occasionnés par la construction des remparts de Jérusalem et la fortification de son enceinte, ainsi que par la construction d’autres remparts en Judée, seront également mis au compte du roi. »
${}^{46}Quand Jonathan et le peuple entendirent ces paroles, ils refusèrent d’y croire et de les prendre en considération : ils avaient encore en mémoire tout le mal que Démétrios avait commis en Israël, toute l’oppression qu’il avait fait peser sur eux. 
${}^{47}Ils accordèrent donc leur préférence à Alexandre, qui avait été le premier à leur adresser des paroles de paix. Ils étaient pour lui des alliés permanents. 
${}^{48}Alors, le roi Alexandre rassembla des troupes nombreuses et prit position contre Démétrios. 
${}^{49}Les deux rois engagèrent le combat, et l’armée d’Alexandre prit la fuite. Démétrios le poursuivit et prit le dessus. 
${}^{50}Il mena le combat avec acharnement jusqu’au coucher du soleil, mais, ce jour-là, Démétrios succomba.
${}^{51}Alexandre envoya des messagers à Ptolémée, roi d’Égypte, pour lui dire :
${}^{52}« Rentré dans mon royaume, je me suis assis sur le trône de mes pères, j’ai pris le pouvoir, j’ai écrasé Démétrios et je me suis rendu maître de mon territoire. 
${}^{53}Dans le combat que nous nous sommes livré, lui et son armée ont été écrasés par nous. Alors, je me suis assis sur son trône royal. 
${}^{54}Maintenant, concluons donc une amitié l’un avec l’autre : donne-moi ta fille en mariage ; je deviendrai ton gendre et je te donnerai, ainsi qu’à elle, des présents dignes de toi. »
${}^{55}Le roi Ptolémée répondit en ces termes : « Heureux le jour où, rentré dans le pays de tes pères, tu t’es assis sur leur trône royal ! 
${}^{56}Maintenant, je ferai donc pour toi ce que tu as écrit, mais viens me rejoindre à Ptolémaïs, afin que nous nous voyions l’un l’autre. Alors, je ferai de toi mon gendre, comme tu l’as dit. »
${}^{57}Ptolémée partit d’Égypte avec Cléopâtre, sa fille, et vint à Ptolémaïs en l’an 162 de l’empire grec. 
${}^{58}Le roi Alexandre vint le rejoindre et Ptolémée lui donna Cléopâtre, sa fille. On célébra son mariage à Ptolémaïs, avec beaucoup d’éclat, comme il convient aux rois. 
${}^{59}Le roi Alexandre écrivit à Jonathan de venir le rejoindre. 
${}^{60}Celui-ci se rendit donc à Ptolémaïs, avec éclat. Il y rencontra les deux rois et leur donna, ainsi qu’à leurs amis, de l’argent et de l’or. Il fit de nombreux présents et trouva grâce à leurs yeux. 
${}^{61}Alors se liguèrent contre lui des hommes infidèles à la Loi, la peste d’Israël. Ils l’accusèrent devant le roi, mais celui-ci ne leur prêta aucune attention. 
${}^{62}Il ordonna même que l’on enlève à Jonathan sa tunique pour le revêtir de pourpre, ce que l’on fit. 
${}^{63}Le roi le fit asseoir auprès de lui et dit à ses officiers : « Allez avec lui au milieu de la ville et faites proclamer que personne n’a le droit de porter plainte contre lui pour quelque motif que ce soit, ni de le tourmenter sous aucun prétexte. » 
${}^{64}Alors, en voyant les honneurs qu’on lui rendait par cette proclamation et le manteau de pourpre dont il était revêtu, ses accusateurs s’enfuirent tous. 
${}^{65}Le roi le combla d’honneurs. Il l’inscrivit au nombre de ses premiers amis et le nomma général et gouverneur. 
${}^{66}Aussi, Jonathan retourna à Jérusalem dans la paix et l’allégresse.
${}^{67}En l’an 165 de l’empire grec, Démétrios Nicator, fils de Démétrios, vint de Crète dans le pays de ses pères. 
${}^{68}À cette nouvelle, le roi Alexandre, fort contrarié, retourna à Antioche.
${}^{69}Démétrios confia le commandement à Apollonios, qui était gouverneur de la Cœlé-Syrie. Celui-ci rassembla une grande armée, vint camper à Jamnia et envoya dire au grand prêtre Jonathan : 
${}^{70}« Tu es vraiment le seul à te dresser contre nous. À cause de toi, me voici devenu un objet de dérision et de honte. Pourquoi exercer ton pouvoir contre nous dans les montagnes de Judée ? 
${}^{71}Si tu as confiance dans tes troupes, descends donc maintenant vers nous, dans la plaine : nous y confronterons nos forces, car l’armée des villes est à mes côtés. 
${}^{72}Informe-toi et tu sauras qui nous sommes, moi-même et les autres qui me prêtent main-forte. On prétend qu’il ne vous est pas possible de tenir devant nous puisque, par deux fois déjà, tes pères ont été mis en déroute dans leur propre pays. 
${}^{73}Maintenant, tu ne pourras pas résister à la cavalerie et à une aussi grande armée dans cette plaine où ne se trouve ni pierre ni caillou, ni aucun refuge. »
${}^{74}Quand il entendit ces paroles d’Apollonios, Jonathan, piqué au vif, choisit dix mille hommes et sortit de Jérusalem. Son frère Simon le rejoignit pour lui prêter main-forte. 
${}^{75}On dressa le camp devant Joppé, dont les habitants avaient fermé les portes, car il y avait là une garnison d’Apollonios. Le combat commença. 
${}^{76}Pris de peur, les habitants ouvrirent les portes, et Jonathan se rendit maître de Joppé. 
${}^{77}Quand il apprit cela, Apollonios déploya trois mille cavaliers et une troupe nombreuse ; il se dirigea vers Azôt comme pour traverser le pays, tandis qu’en même temps, il s’avançait dans la plaine, confiant dans le grand nombre de ses cavaliers. 
${}^{78}Jonathan le poursuivit en direction d’Azôt, et les armées engagèrent le combat. 
${}^{79}Or, Apollonios avait laissé à l’arrière mille cavaliers en embuscade. 
${}^{80}Jonathan fut informé de ce piège dressé derrière lui. Les cavaliers encerclèrent son armée et lancèrent des flèches sur ses hommes, du matin jusqu’au soir. 
${}^{81}Mais ils tinrent bon, comme Jonathan l’avait ordonné, tandis que les chevaux se fatiguaient. 
${}^{82}C’est alors que Simon entraîna son armée et attaqua les bataillons adverses, dont la cavalerie était épuisée. Les ennemis furent écrasés par Simon et s’enfuirent. 
${}^{83}La cavalerie se dispersa dans la plaine. Les fuyards parvinrent à Azôt et pénétrèrent dans le sanctuaire de Dagone, leur idole, pour y trouver le salut. 
${}^{84}Mais Jonathan incendia Azôt et les villes d’alentour. Il ramassa le butin et livra au feu le temple de Dagone avec ceux qui s’y étaient réfugiés. 
${}^{85}Près de huit mille hommes périrent par le glaive et par le feu. 
${}^{86}Jonathan partit de là et dressa son camp près d’Ascalon : les habitants de cette ville sortirent à sa rencontre avec de grands honneurs. 
${}^{87}Puis, Jonathan revint à Jérusalem avec ses compagnons. Ils ramenaient un grand butin. 
${}^{88}Lorsque le roi Alexandre fut informé de ces événements, il décerna de nouveaux honneurs à Jonathan. 
${}^{89}Il lui envoya une agrafe d’or, comme il est d’usage de l’accorder aux parents du roi. Il lui donna également en propriété la ville d’Akkarone et tout son territoire.
      
         
      \bchapter{}
      \begin{verse}
${}^{1}Ptolémée, le roi d’Égypte, rassembla une armée aussi nombreuse que le sable au bord de la mer, ainsi qu’une grande flotte, car il désirait se rendre maître, par la ruse, du royaume d’Alexandre et l’annexer à son propre royaume. 
${}^{2}Il se rendit donc en Syrie avec des paroles de paix. Les habitants des villes lui ouvraient leurs portes et sortaient au-devant de lui, car le roi Alexandre avait donné l’ordre de l’accueillir, lui qui était son beau-père. 
${}^{3}Mais, dans chacune des villes où il pénétrait, Ptolémée laissait des troupes en garnison. 
${}^{4}Comme il s’approchait d’Azôt, on lui montra le temple de Dagone incendié, Azôt et ses environs détruits, les cadavres dispersés çà et là et les restes calcinés de ceux qui avaient été brûlés dans le combat. On les avait entassés sur le parcours du roi. 
${}^{5}On rapporta au roi les actions de Jonathan, pour susciter sa réprobation, mais il garda le silence. 
${}^{6}Quant à Jonathan, il se rendit, en grand apparat, au-devant du roi à Joppé. Ils se saluèrent mutuellement et passèrent la nuit en ce lieu. 
${}^{7}Ensuite, Jonathan accompagna le roi jusqu’au fleuve nommé Éleuthère. Puis, il retourna à Jérusalem.
${}^{8}Le roi Ptolémée se rendit maître des villes du littoral jusqu’à Séleucie-sur-Mer. Il nourrissait de mauvaises intentions à l’égard d’Alexandre. 
${}^{9}Il envoya des messagers au roi Démétrios, pour lui dire : « Viens, faisons alliance l’un avec l’autre. Je te donnerai ma fille, celle qui est la femme d’Alexandre, et tu régneras sur le royaume de ton père. 
${}^{10}Je regrette, en effet, d’avoir donné ma fille à Alexandre, car il a cherché à me tuer. » 
${}^{11}Or, s’il lui reprochait cela, c’est parce qu’il convoitait son royaume. 
${}^{12}Il lui reprit donc sa fille et la donna à Démétrios. Il changea d’attitude envers Alexandre, et leur hostilité devint manifeste. 
${}^{13}Alors, Ptolémée fit son entrée à Antioche et ceignit le diadème de l’Asie, de sorte qu’il portait sur son front deux diadèmes : celui de l’Égypte et celui de l’Asie. 
${}^{14}À cette époque, le roi Alexandre se trouvait en Cilicie, parce que les habitants de cette région s’étaient révoltés. 
${}^{15}Lorsqu’il apprit les manœuvres de Ptolémée, il marcha contre lui pour le combattre. Mais Ptolémée sortit à sa rencontre avec une forte troupe et le mit en déroute. 
${}^{16}Alexandre s’enfuit en Arabie pour s’y mettre à l’abri, et le roi Ptolémée triompha. 
${}^{17}Zabdiel l’Arabe trancha la tête d’Alexandre et l’envoya à Ptolémée. 
${}^{18}Mais le roi Ptolémée lui-même mourut le surlendemain, et les garnisons qu’il avait laissées dans les villes fortifiées furent massacrées par les habitants. 
${}^{19}Démétrios devint roi en l’an 167 de l’empire grec.
${}^{20}En ces jours-là, Jonathan rassembla les habitants de la Judée pour attaquer la citadelle de Jérusalem. Il dressa contre elle un grand nombre de machines de guerre. 
${}^{21}Or, quelques hommes infidèles à la Loi, des gens qui haïssaient leur propre nation, se rendirent chez le roi Démétrios pour lui annoncer que Jonathan assiégeait la citadelle. 
${}^{22}Apprenant cela, Démétrios fut très irrité. Dès qu’il l’apprit, il donna aussitôt le signal du départ pour se rendre à Ptolémaïs. Il écrivit à Jonathan de lever le siège et de venir au plus vite s’entretenir avec lui à Ptolémaïs. 
${}^{23}Quand Jonathan l’apprit, il ordonna de continuer le siège. Puis, il choisit pour l’accompagner quelques-uns des anciens d’Israël et quelques prêtres, et il affronta lui-même le danger. 
${}^{24}Chargé d’argent, d’or, de vêtements précieux et d’autres présents en quantité, il se rendit à Ptolémaïs chez le roi et trouva grâce à ses yeux. 
${}^{25}Quelques hommes de sa nation, infidèles à la Loi, essayèrent bien de l’accuser devant le roi, 
${}^{26}mais celui-ci agit à son égard de la même manière que ses prédécesseurs et l’éleva en présence de tous ses amis. 
${}^{27}Il le confirma dans sa charge de grand prêtre et dans toutes les autres fonctions honorifiques qu’il avait auparavant. Il le fit compter au nombre de ses premiers amis. 
${}^{28}Jonathan demanda alors au roi d’exempter d’impôts la Judée et les trois districts de la Samarie, en lui promettant une somme de trois cents talents. 
${}^{29}Le roi accepta. Au sujet de toutes ces choses, il écrivit à Jonathan une lettre ainsi tournée :
${}^{30}« Le roi Démétrios à Jonathan, son frère, et à la nation des Juifs, salut ! 
${}^{31}Voici la copie de la lettre que nous avons écrite à votre sujet à notre parent Lasthénès. Nous vous l’adressons aussi, pour que vous en preniez connaissance. 
${}^{32}“Le roi Démétrios à son père Lasthénès, salut ! 
${}^{33}À la nation des Juifs, qui sont nos amis et se conduisent envers nous avec droiture, nous sommes décidé à faire du bien en raison de leurs bons sentiments à notre égard. 
${}^{34}Nous leur confirmons la possession du territoire de la Judée et celle des trois districts d’Aphéréma, de Lydda et de Ramathaïm, qui ont été détachés de la Samarie et annexés à la Judée, avec tous les lieux qui en dépendent. Leurs revenus appartiennent à tous ceux qui offrent des sacrifices à Jérusalem, en échange des redevances royales qu’auparavant le roi prélevait chaque année sur les produits de leur terre et de leurs arbres. 
${}^{35}Quant aux autres droits que nous avons sur les dîmes et les impôts qui nous reviennent, sur les marais salants et les couronnes qui nous sont dues, nous leur en faisons remise totale à dater de ce jour. 
${}^{36}Pas une seule de ces dispositions ne sera abrogée, à dater de ce jour et pour la suite des temps. 
${}^{37}Prenez donc soin, maintenant, d’en faire une copie. Qu’on la remette à Jonathan et qu’elle soit placée bien en vue sur la montagne sainte.” »
${}^{38}Le roi Démétrios vit qu’en sa présence le pays connaissait la tranquillité et que rien ne lui résistait. Il démobilisa toute son armée et renvoya chacun dans sa propre région, à l’exception des troupes étrangères qu’il avait recrutées dans les îles des nations. Pour cette raison, il s’attira l’hostilité de toutes les troupes qu’il tenait de ses pères. 
${}^{39}Or, Tryphon, un ancien partisan d’Alexandre, constata que toutes les troupes murmuraient contre Démétrios. Il se rendit alors chez Imalkoué l’Arabe, qui élevait Antiocos, le jeune fils d’Alexandre. 
${}^{40}Il lui demandait avec insistance de lui remettre l’enfant, pour qu’il règne à la place de son père. Il l’informa de toutes les dispositions prises par Démétrios et de l’hostilité que lui vouaient ses armées. Il demeura là pendant de nombreux jours. 
${}^{41}Jonathan, de son côté, fit demander au roi Démétrios de retirer ses garnisons de la citadelle de Jérusalem ainsi que des autres forteresses de Judée, car elles étaient toujours en état de guerre avec Israël. 
${}^{42}Démétrios fit porter cette réponse à Jonathan : « Non seulement je ferai cela pour toi et pour ta nation, mais je te couvrirai d’honneurs, toi et ta nation, si l’occasion se présente. 
${}^{43}Mais maintenant, tu ferais bien d’envoyer des hommes pour combattre à mes côtés, car toutes mes troupes ont fait défection. »
${}^{44}Jonathan lui envoya donc à Antioche trois mille hommes, de vaillants guerriers, dont l’arrivée réjouit le roi. 
${}^{45}Les gens se rassemblèrent au milieu de la ville, au nombre d’environ cent vingt mille, dans l’intention de faire périr le roi. 
${}^{46}Celui-ci se réfugia dans son palais, tandis que les gens de la ville envahissaient les rues et commençaient le combat. 
${}^{47}Le roi appela les Juifs à son secours. Ceux-ci, tous ensemble, se groupèrent autour de lui, puis se dispersèrent dans la ville, où ils tuèrent, ce jour-là, environ cent mille personnes. 
${}^{48}Ils incendièrent la ville, ramassèrent une grande quantité de butin ce jour-là et ils sauvèrent le roi. 
${}^{49}Voyant que les Juifs s’étaient rendus maîtres de la ville comme ils le voulaient, les habitants perdirent courage et se mirent à crier vers le roi, en le suppliant ainsi : 
${}^{50}« Tends-nous la main droite, et que les Juifs cessent de nous combattre, nous et notre ville ! » 
${}^{51}Ils jetèrent leurs armes et firent la paix. Les Juifs furent couverts de gloire en présence du roi et aux yeux de tout son royaume. Leur renom s’étendit dans le royaume, et ils regagnèrent Jérusalem chargés d’un riche butin. 
${}^{52}Alors, le roi Démétrios s’assit sur le trône royal et, en sa présence, le pays connut la tranquillité. 
${}^{53}Mais il renia toutes ses promesses et traita Jonathan comme un étranger. Il l’opprima durement, sans plus reconnaître les services rendus.
${}^{54}Après ces événements, Tryphon revint, accompagné d’Antiocos. Cet enfant, tout jeune encore, devint roi et porta le diadème. 
${}^{55}Autour de lui se rassemblèrent toutes les troupes dont Démétrios s’était débarrassé. Elles combattirent Démétrios, qui prit la fuite et fut mis en déroute. 
${}^{56}Tryphon s’empara des éléphants et se rendit maître de la ville d’Antioche.
${}^{57}Le jeune Antiocos écrivit à Jonathan en ces termes : « Je te confirme dans la charge de grand prêtre. Je t’établis à la tête des quatre districts et je te compte parmi les amis du roi. » 
${}^{58}Il lui envoya en même temps des objets d’or et un service de table, avec l’autorisation de boire dans des coupes d’or, de porter un vêtement de pourpre et une agrafe d’or. 
${}^{59}Quant à Simon, le frère de Jonathan, Antiocos le nomma gouverneur militaire de la région qui s’étend de l’Échelle de Tyr, débarcadère de cette ville, jusqu’aux frontières de l’Égypte.
${}^{60}Alors, Jonathan partit et se mit à parcourir la Transeuphratène et ses villes. Toutes les troupes de Syrie se rangèrent à ses côtés pour combattre avec lui. Il se rendit à Ascalon dont les habitants l’accueillirent avec beaucoup d’honneurs. 
${}^{61}De là, il vint à Gaza, mais les gens de cette ville lui fermèrent les portes. Il en fit le siège et livra ses faubourgs au feu et au pillage. 
${}^{62}Alors, les habitants de Gaza supplièrent Jonathan. Il leur tendit la main droite en signe de paix. Cependant il prit en otages les fils de leurs chefs et les envoya à Jérusalem. Puis, il parcourut le pays jusqu’à Damas.
${}^{63}Jonathan apprit que les généraux de Démétrios se trouvaient près de Kédès de Galilée avec une armée nombreuse et qu’ils voulaient l’écarter de sa charge. 
${}^{64}Il se porta à leur rencontre, mais laissa son frère Simon dans le pays. 
${}^{65}Simon prit position devant Bethsour et l’attaqua pendant de nombreux jours en bloquant ses issues. 
${}^{66}Les habitants lui demandèrent la paix, ce qu’il leur accorda. Cependant, il les expulsa, prit la ville et y installa une garnison. 
${}^{67}De leur côté, Jonathan et son armée, qui campaient près du lac de Gennésar, arrivèrent de grand matin dans la plaine d’Assor. 
${}^{68}L’armée des étrangers s’avançait à sa rencontre dans la plaine, tandis qu’un détachement se trouvait embusqué derrière lui, dans les montagnes. Les étrangers avancèrent à sa rencontre, 
${}^{69}et les hommes de l’embuscade, surgissant de leur position, engagèrent le combat. 
${}^{70}Tous les hommes de Jonathan prirent la fuite. Personne ne resta, à l’exception de deux chefs de ses troupes : Mattathias, fils d’Absalom, et Judas, fils de Kalphi. 
${}^{71}Jonathan déchira ses vêtements, répandit de la poussière sur sa tête et pria. 
${}^{72}Puis, il retourna au combat et provoqua la déroute de ses ennemis, qui s’enfuirent. 
${}^{73}À cette vue, ceux des siens qui avaient fui revinrent auprès de lui. Ensemble, ils poursuivirent les ennemis jusqu’à leur camp, qui se trouvait à Kédès. Eux-mêmes campèrent en ce lieu. 
${}^{74}En ce jour-là, parmi les étrangers, trois mille hommes environ tombèrent. Ensuite, Jonathan retourna à Jérusalem.
      
         
      \bchapter{}
      \begin{verse}
${}^{1}Voyant que les circonstances lui étaient favorables, Jonathan choisit des hommes qu’il envoya à Rome pour confirmer et renouveler l’amitié avec les Romains. 
${}^{2}À Sparte également, ainsi qu’en d’autres régions, il envoya des lettres rédigées dans le même sens. 
${}^{3}Ces hommes se rendirent donc à Rome. Ils entrèrent au Sénat et s’y exprimèrent en ces termes : « Le grand prêtre Jonathan et la nation des Juifs nous ont envoyés renouveler l’amitié et l’alliance avec eux, comme par le passé. » 
${}^{4}On leur donna des lettres pour les autorités locales, recommandant de les escorter en paix vers le pays de Juda.
${}^{5}Voici la copie de la lettre que Jonathan écrivit aux gens de Sparte :
${}^{6}« Le grand prêtre Jonathan, le Conseil des anciens de la nation, les prêtres et le reste du peuple des Juifs, aux Spartiates leurs frères, salut ! 
${}^{7}Autrefois déjà, une lettre fut envoyée au grand prêtre Onias de la part d’Aréios qui régnait parmi vous. Cette lettre attestait que vous êtes nos frères, comme le montre la copie ci-dessous. 
${}^{8}Onias reçut avec honneur l’homme qui était envoyé et prit la lettre où il était clairement question d’alliance et d’amitié. 
${}^{9}Pour notre part, sans en éprouver le besoin, puisque nous avons le réconfort des livres saints qui sont entre nos mains, 
${}^{10}nous nous sommes cependant permis d’envoyer quelqu’un pour renouveler la fraternité et l’amitié qui nous lient à vous, afin de ne pas nous comporter comme des étrangers à votre égard. En effet, il s’est écoulé beaucoup de temps depuis que vous nous avez envoyé cette lettre. 
${}^{11}Sans cesse, en toute occasion, aux fêtes comme aux autres jours appropriés, nous nous souvenons de vous dans les sacrifices que nous offrons et dans nos prières, car il est juste et convenable de penser à des frères. 
${}^{12}Oui, nous nous réjouissons de votre gloire. 
${}^{13}Quant à nous, nous avons été assaillis par bien des épreuves et bien des guerres. Les rois d’alentour nous ont combattus. 
${}^{14}Au moment de ces guerres, nous n’avons pas voulu vous importuner, pas plus que nos autres alliés et amis, 
${}^{15}car nous avons, pour nous aider, l’aide du Ciel. Ainsi avons-nous été délivrés de nos ennemis. Eux, ils ont été humiliés. 
${}^{16}Nous avons donc choisi Nouménios, fils d’Antiocos, et Antipater, fils de Jason, et nous les avons envoyés auprès des Romains pour renouveler l’amitié et l’alliance établies antérieurement avec eux. 
${}^{17}Nous leur avons prescrit d’aller aussi chez vous, de vous saluer et de vous remettre de notre part la lettre concernant le renouvellement de nos liens de fraternité. 
${}^{18}Vous voudrez donc bien nous répondre à ce sujet. »
${}^{19}Voici la copie de la lettre qui avait été envoyée à Onias :
${}^{20}« Aréios, roi des Spartiates, à Onias le grand prêtre, salut ! 
${}^{21}Dans un document sur les Spartiates et les Juifs, on a découvert qu’ils sont frères et qu’ils sont de la race d’Abraham. 
${}^{22}Maintenant que nous savons cela, vous voudrez bien nous adresser vos salutations. 
${}^{23}Pour notre part, nous vous répondons ceci : votre bétail et vos biens sont à nous, et les nôtres sont à vous. C’est pourquoi nous ordonnons qu’on vous apporte un message en ce sens. »
${}^{24}Jonathan apprit que les généraux de Démétrios étaient revenus avec une armée plus nombreuse qu’auparavant, pour lui faire la guerre. 
${}^{25}Il quitta donc Jérusalem et se porta à leur rencontre vers le pays de Hamath, sans leur laisser le temps de pénétrer dans son propre pays. 
${}^{26}Il envoya des espions dans leur camp. Ceux-ci revinrent lui annoncer que les ennemis avaient pris leurs dispositions pour l’attaquer durant la nuit. 
${}^{27}Au coucher du soleil, Jonathan ordonna à ses hommes de veiller et de garder les armes sous la main, pour être prêts au combat pendant toute la nuit. Il plaça aussi des sentinelles tout autour du camp. 
${}^{28}Lorsque les adversaires apprirent que Jonathan et ses hommes étaient prêts au combat, ils prirent peur. Le cœur plein d’effroi, ils se retirèrent, tout en allumant des feux dans leur camp. 
${}^{29}Jonathan et ses hommes ne s’aperçurent de rien jusqu’au matin, car ils voyaient briller les feux. 
${}^{30}Jonathan se lança à leur poursuite, mais sans les rejoindre, car ils avaient déjà franchi le fleuve Éleuthère. 
${}^{31}Alors, il se tourna contre la tribu arabe des Zabadéens. Il les battit et s’empara de leur butin. 
${}^{32}Après avoir donné le signal du départ, il se rendit à Damas et parcourut toute la région. 
${}^{33}De son côté, Simon sortit et s’avança jusqu’à Ascalon et jusqu’aux forteresses voisines. De là, il se tourna vers Joppé et s’en empara, 
${}^{34}car il avait appris que ses habitants voulaient livrer la forteresse aux hommes de Démétrios. Il y installa donc une garnison pour la défendre.
${}^{35}De retour à Jérusalem, Jonathan réunit les anciens du peuple et prit avec eux la décision de bâtir des forteresses en Judée, 
${}^{36}de surélever les remparts de Jérusalem et d’élever un grand mur de séparation entre la ville et la citadelle, pour isoler complètement celle-ci et empêcher d’y vendre ou d’y acheter. 
${}^{37}On se rassembla donc pour rebâtir la ville ; une partie du mur du torrent, à l’est de la ville, s’était écroulée ; on restaura aussi le quartier appelé Kaphénata. 
${}^{38}Quant à Simon, il rebâtit Adida dans le Bas-Pays. Il la fortifia et la munit de portes et de verrous.
${}^{39}Or, Tryphon avait le projet de devenir roi sur l’Asie, de ceindre le diadème et de porter la main sur le jeune roi Antiocos. 
${}^{40}Cependant, comme il craignait que Jonathan ne le laisse pas faire et le combatte, il cherchait le moyen de s’emparer de lui pour le faire périr. Il partit donc pour Bethsane. 
${}^{41}Jonathan sortit à sa rencontre et marcha sur Bethsane avec quarante mille hommes d’élite. 
${}^{42}Voyant qu’il était accompagné d’une troupe nombreuse, Tryphon n’osa pas porter la main sur lui. 
${}^{43}Au contraire, il l’accueillit avec honneur, le recommanda à tous ses amis, lui fit des cadeaux et ordonna à ses amis et à ses troupes d’obéir à Jonathan comme à lui-même. 
${}^{44}Il dit à Jonathan : « Pourquoi as-tu imposé cette fatigue à tout ce peuple, alors qu’aucune guerre ne nous menace ? 
${}^{45}Renvoie-les donc dans leurs foyers, choisis quelques hommes pour t’escorter et viens avec moi à Ptolémaïs. Je te livrerai cette ville et les autres forteresses, ainsi que le reste des troupes et tous les fonctionnaires. Ensuite, je m’en retournerai, car c’est dans ce but que je suis venu. » 
${}^{46}Jonathan lui fit confiance et agit selon ce qu’il avait dit : il renvoya ses troupes, qui retournèrent au pays de Juda. 
${}^{47}Il ne garda avec lui que trois mille hommes, dont il laissa deux mille en Galilée ; les mille autres l’accompagnèrent. 
${}^{48}Mais dès que Jonathan fut entré dans Ptolémaïs, les habitants fermèrent les portes, se saisirent de lui et tuèrent par l’épée tous ceux qui étaient entrés avec lui. 
${}^{49}Tryphon envoya des troupes et des cavaliers en Galilée et dans la Grande Plaine pour faire périr tous les hommes de Jonathan. 
${}^{50}Ceux-ci comprirent qu’il avait été arrêté et qu’il était perdu, lui et ses compagnons. Ils s’encouragèrent mutuellement et marchèrent en rangs serrés, prêts au combat. 
${}^{51}Voyant qu’ils étaient résolus à défendre leur vie, leurs poursuivants firent demi-tour. 
${}^{52}Et tous les hommes de Jonathan rentrèrent sans encombre au pays de Juda. Ils pleurèrent Jonathan et ses compagnons, accablés eux-mêmes par une forte crainte. Tout Israël mena un grand deuil. 
${}^{53}Alors, toutes les nations païennes d’alentour cherchèrent à les anéantir, car elles se disaient : « Ils n’ont personne pour les gouverner ni les secourir. C’est le moment de les attaquer et d’effacer leur souvenir du milieu des hommes. »
      
         
      \bchapter{}
      \begin{verse}
${}^{1}Simon apprit que Tryphon avait réuni une grande armée pour se rendre au pays de Juda et le ravager. 
${}^{2}Voyant que le peuple tremblait d’épouvante, il monta à Jérusalem, rassembla le peuple 
${}^{3}et l’exhorta en ces termes : « Vous savez bien vous-mêmes tout ce que nous avons fait, moi, mes frères et la maison de mon père, en faveur des lois et du Lieu saint. Vous connaissez les guerres et les angoisses que nous avons traversées. 
${}^{4}C’est pour cela, pour Israël, que tous mes frères ont péri. Je suis resté, moi seul. 
${}^{5}Maintenant, quelle que soit la détresse de ce temps, il ne sera pas dit que j’ai épargné ma vie ! Car je ne suis pas meilleur que mes frères. 
${}^{6}Au contraire, je vengerai ma nation et le Lieu saint, vos femmes et vos enfants, puisque toutes les nations païennes, poussées par la haine, se sont liguées pour nous anéantir. » 
${}^{7}L’esprit du peuple se ranima dès qu’il entendit ces paroles 
${}^{8}et tous répondirent d’une voix forte : « C’est toi qui es notre guide, à la place de Judas et de Jonathan, ton frère. 
${}^{9}Prends la tête de notre combat, et tout ce que tu nous diras, nous le ferons. » 
${}^{10}Simon rassembla donc tous les hommes de guerre ; il se hâta d’achever les remparts de Jérusalem et fortifia la ville tout autour. 
${}^{11}Il envoya à Joppé Jonathan, fils d’Absalom, accompagné d’une troupe importante. Celui-ci en chassa les habitants et s’y établit.
      
         
${}^{12}Tryphon quitta Ptolémaïs avec une armée nombreuse, pour pénétrer dans le pays de Juda ; il emmenait avec lui Jonathan, sous bonne garde. 
${}^{13}Simon, lui, établit son camp à Adida, en face de la plaine. 
${}^{14}Apprenant que Simon avait pris la relève de son frère Jonathan et qu’il s’apprêtait à lui livrer bataille, Tryphon lui envoya des messagers pour lui dire : 
${}^{15}« Ton frère Jonathan doit de l’argent au trésor royal, en raison des fonctions qu’il exerçait ; c’est pour cela que nous le retenons captif. 
${}^{16}Envoie donc maintenant une somme de cent talents d’argent et deux de ses fils en otages, de peur qu’une fois relâché, il ne se dresse contre nous. À cette condition, nous le relâcherons. » 
${}^{17}Simon se rendit bien compte que ces paroles étaient trompeuses, mais il envoya chercher l’argent et les jeunes enfants, de peur de s’attirer une grande hostilité de la part du peuple. 
${}^{18}On aurait pu dire : « C’est parce que Simon ne lui a pas envoyé l’argent et les jeunes enfants, que Jonathan a péri. » 
${}^{19}Simon envoya donc les deux jeunes enfants et la somme de cent talents, mais Tryphon renia sa parole et ne relâcha pas Jonathan. 
${}^{20}Après quoi, il se mit en marche pour envahir le pays et le dévaster. Il fit un détour par le chemin d’Adora, mais Simon et son armée lui faisaient obstacle partout où il passait. 
${}^{21}De leur côté, les occupants de la citadelle de Jérusalem envoyèrent des messagers auprès de Tryphon, le pressant de les rejoindre par le désert et de leur faire parvenir des vivres. 
${}^{22}Tryphon prépara donc toute sa cavalerie, mais la neige tomba cette nuit-là en telle abondance qu’il ne put s’y rendre. Il quitta ce lieu pour le pays de Galaad. 
${}^{23}Aux approches de Baskama, il tua Jonathan, qui fut enseveli en ce lieu. 
${}^{24}Puis, il rebroussa chemin et retourna dans son pays.
${}^{25}Simon envoya recueillir les ossements de son frère Jonathan et l’ensevelit à Modine, la ville de ses pères. 
${}^{26}Tout Israël mena un grand deuil et se lamenta sur lui pendant de nombreux jours. 
${}^{27}Sur le tombeau de son père et de ses frères, Simon bâtit un monument assez élevé pour être vu de loin, en pierres polies à l’arrière comme en façade. 
${}^{28}Il y dressa sept pyramides, l’une en face de l’autre, en l’honneur de son père, de sa mère et de ses quatre frères. 
${}^{29}Il leur fit des socles et les entoura de hautes colonnes. Sur ces colonnes, en souvenir éternel, il plaça un décor d’armes et des vaisseaux sculptés en relief pour être vus de tous ceux qui navigueraient sur la mer. 
${}^{30}Tel fut le monument funéraire qu’il éleva à Modine et qui existe encore aujourd’hui.
${}^{31}Cependant Tryphon, dans sa duplicité à l’égard du jeune roi Antiocos, finit par le tuer. 
${}^{32}Il devint roi à sa place, s’empara du diadème de l’Asie et causa beaucoup de tort au pays. 
${}^{33}Quant à Simon, il rebâtit les forteresses de Judée, les entoura de hautes tours et de remparts élevés, munis de portes et de verrous. Dans ces forteresses, il entreposa des vivres. 
${}^{34}Simon choisit quelques hommes qu’il envoya au roi Démétrios pour que celui-ci accorde au pays une remise d’impôts, car tous les actes de Tryphon n’avaient été que pillage.
${}^{35}Le roi Démétrios lui envoya une réponse conforme à sa demande et lui écrivit la lettre suivante :
${}^{36}« Le roi Démétrios, à Simon, grand prêtre et ami des rois, aux anciens et à la nation des Juifs, salut ! 
${}^{37}Nous avons reçu volontiers la couronne d’or et la palme que vous avez envoyées ; nous sommes disposés à conclure avec vous une paix complète et à prescrire aux fonctionnaires de vous accorder les remises d’impôts. 
${}^{38}Tout ce que nous avons statué en votre faveur reste confirmé. Les forteresses que vous avez bâties, qu’elles soient vôtres ! 
${}^{39}Nous vous faisons grâce des négligences et des fautes commises jusqu’à ce jour. Nous renonçons à la couronne que vous nous devez. S’il est encore pour Jérusalem d’autres redevances à payer, qu’il ne soit plus rien exigé à l’avenir ! 
${}^{40}Enfin, s’il en est parmi vous qui sont aptes à être enrôlés dans notre garde du corps, qu’ils se fassent inscrire, et qu’il y ait la paix entre nous ! »
${}^{41}C’est en l’année 170 de l’empire grec que le joug des nations païennes fut retiré d’Israël, 
${}^{42}et le peuple se mit à mentionner sur les actes et les contrats : « La première année de Simon, grand prêtre éminent, chef de l’armée et guide des Juifs ».
${}^{43}En ces jours-là, Simon mit le siège devant Gazara et l’encercla. Il construisit une tour roulante, l’amena tout près de la ville, attaqua un de ses bastions et s’en empara. 
${}^{44}De la tour, les combattants se précipitèrent dans la ville, où il se produisit une véritable panique. 
${}^{45}Les habitants de la ville montèrent sur les remparts avec leurs femmes et leurs enfants, déchirèrent leurs vêtements et crièrent d’une voix forte pour supplier Simon de leur tendre la main droite en signe de paix. 
${}^{46}Ils disaient : « Ne nous traite pas selon nos méchancetés mais selon ta miséricorde. » 
${}^{47}Simon le leur accorda et cessa le combat, mais les expulsa de la ville. Puis il purifia les maisons où se trouvaient les idoles. C’est ainsi qu’il fit son entrée dans la ville, au chant des hymnes et des bénédictions. 
${}^{48}Il en expulsa toute impureté et y installa des hommes fidèles à la Loi. Il la fortifia et s’y fit construire une résidence.
${}^{49}De leur côté, les occupants de la citadelle de Jérusalem étaient empêchés de sortir et de se rendre à la campagne, d’acheter et de vendre. Ils eurent terriblement faim et bon nombre d’entre eux furent emportés par la famine. 
${}^{50}Ils crièrent à Simon de conclure avec eux la paix, ce qu’il leur accorda. Cependant il les expulsa de ce lieu et purifia la citadelle de toute souillure. 
${}^{51}Alors, il y fit son entrée, le vingt-troisième jour du deuxième mois de l’année 171, avec des acclamations et des branches de palmier, au son des cithares, des cymbales et des harpes, avec des hymnes et des cantiques, parce qu’un grand ennemi avait été brisé et jeté hors d’Israël. 
${}^{52}Simon décida de commémorer ce jour-là chaque année dans l’allégresse. Il fortifia aussi la montagne du Temple du côté de la citadelle. C’est là qu’il habita, lui et les siens. 
${}^{53}Voyant que son fils Jean était devenu un homme, Simon le plaça à la tête de toutes les troupes. Jean résidait à Gazara.
      
         
      \bchapter{}
      \begin{verse}
${}^{1}En l’an 172 de l’empire grec, le roi Démétrios rassembla son armée et se rendit en Médie pour y chercher de l’aide afin de combattre Tryphon. 
${}^{2}Arsace, roi de Perse et de Médie, apprit que Démétrios était entré sur son territoire ; il envoya un de ses généraux pour le capturer vivant. 
${}^{3}Celui-ci partit donc et battit l’armée de Démétrios. Il le captura et l’amena chez Arsace, qui le fit mettre en prison.
      
         
${}^{4}Tout au long des jours de Simon,
        \\le pays connut la tranquillité.
        \\Simon rechercha le bien de sa nation ;
        \\son autorité fut appréciée de tous,
        \\autant que sa gloire,
        \\tout au long de ses jours.
${}^{5}En pleine gloire, il prit Joppé, en fit un port,
        \\s’ouvrant ainsi l’accès aux îles de la mer.
${}^{6}Il élargit les frontières de sa nation,
        \\affermit son pouvoir sur le pays
${}^{7}et rassembla de nombreux captifs.
        \\Il se rendit maître de Gazara,
        \\de Bethsour et de la citadelle
        \\d’où il extirpa toute impureté.
        \\Il n’y eut personne pour lui résister.
${}^{8}Chacun cultivait sa terre dans la paix ;
        \\la terre donnait ses produits,
        \\et les arbres des champs leur fruit.
${}^{9}Les vieillards s’installaient sur les places publiques,
        \\tous, ils s’entretenaient de leur bonheur,
        \\et les jeunes gens revêtaient de splendides tenues guerrières.
${}^{10}Il approvisionna les villes,
        \\en fit des instruments de sa force,
        \\et le renom de sa gloire atteignit le bout du monde.
${}^{11}Il instaura la paix dans le pays :
        \\Israël connut une immense allégresse.
${}^{12}Chacun pouvait s’asseoir sous sa vigne ou son figuier
        \\sans avoir à redouter personne.
${}^{13}Tout adversaire disparut de leur pays,
        \\et les rois, en ce temps-là, furent écrasés.
${}^{14}Simon releva tous les humbles de son peuple.
        \\Il observa fidèlement la Loi,
        \\supprimant tout homme infidèle et malfaisant.
${}^{15}Il favorisa la splendeur du Lieu saint,
        \\multipliant la richesse de son mobilier.
${}^{16}La nouvelle de la mort de Jonathan parvint à Rome et jusqu’à Sparte où l’on en fut très contrarié. 
${}^{17}Mais lorsqu’on apprit que son frère Simon, devenu grand prêtre à sa place, gouvernait le pays et les villes qui s’y trouvaient, 
${}^{18}on inscrivit sur des tables de bronze le renouvellement de l’amitié et de l’alliance conclues autrefois avec Judas et Jonathan, ses frères. 
${}^{19}Lecture en fut donnée devant l’assemblée, à Jérusalem.
${}^{20}Voici la copie de la lettre qu’envoyèrent les Spartiates :
      « Les magistrats et la ville des Spartiates à Simon, grand prêtre, ainsi qu’aux anciens, aux prêtres et au reste du peuple des Juifs, leurs frères, salut ! 
${}^{21}Les ambassadeurs que vous avez envoyés auprès de notre peuple nous ont informés de votre gloire et de votre prestige. Nous nous sommes réjouis de leur visite. 
${}^{22}Nous avons inscrit leurs déclarations parmi les décisions du peuple, en ces termes : “Nouménios, fils d’Antiocos, et Antipater, fils de Jason, ambassadeurs des Juifs, sont venus chez nous pour renouveler leur amitié à notre égard.” 
${}^{23}Le peuple a jugé bon de recevoir ces hommes avec honneur et de déposer la copie de leur discours aux archives publiques, afin que le peuple de Sparte en garde le souvenir. Une copie de cette lettre fut établie à l’intention de Simon, le grand prêtre. »
${}^{24}Après cela, Simon envoya Nouménios à Rome avec un grand bouclier d’or pesant mille livres, pour confirmer l’alliance avec eux.
${}^{25}Apprenant ces événements, le peuple dit : « Quel témoignage de reconnaissance donnerons-nous à Simon et à ses fils ? 
${}^{26}Car il s’est montré ferme, lui-même, ainsi que ses frères et la maison de son père, pour combattre et repousser les ennemis d’Israël. Ils ont assuré au peuple la liberté. » On fit donc graver une inscription sur des tables de bronze, que l’on plaça sur des stèles à la montagne de Sion.
${}^{27}Voici le texte de cette inscription :
      « Le dix-huitième jour du mois d’Élul, en l’an 172, la troisième année du règne de Simon, grand prêtre éminent, en Asaramel, 
${}^{28}en présence de la grande assemblée des prêtres, du peuple, des chefs de la nation et des anciens du pays, on nous a fait connaître ce qui suit.
${}^{29}Lors des nombreux combats qui ont eu lieu dans le pays, Simon fils de Mattathias, descendant des fils de Joarib, ainsi que ses frères affrontèrent eux-mêmes le danger. Ils ont tenu tête aux adversaires de leur nation, afin que soient maintenus leur Lieu saint et la Loi. Ils ont ainsi couvert leur nation d’une grande gloire. 
${}^{30}Jonathan rassembla sa nation et devint pour elle grand prêtre, puis il fut réuni aux pères de son peuple. 
${}^{31}Leurs ennemis voulurent envahir le pays pour le ravager et porter la main sur le Lieu saint. 
${}^{32}C’est alors que se leva Simon. Il combattit pour sa nation ; il dépensa une grande partie de ses propres biens pour équiper les hommes de l’armée de sa nation et leur payer une solde. 
${}^{33}Il fortifia les villes de la Judée, ainsi que Bethsour, ville frontière où était auparavant l’arsenal de l’ennemi. Il y plaça une garnison juive. 
${}^{34}Il fortifia Joppé, sur la côte, et Gazara, aux confins de la région d’Azôt, ville autrefois peuplée d’ennemis. Il y installa des Juifs. Toutes ces villes, il les munit de ce qui était nécessaire à leur prospérité. 
${}^{35}Le peuple vit la fidélité de Simon et la gloire qu’il avait décidé de procurer à sa nation. Dès lors, il le choisit comme guide et grand prêtre, en raison de tout ce qu’il avait accompli, de la justice et de la fidélité qu’il avait gardées envers sa nation, et parce qu’il avait travaillé de toutes les manières à la grandeur de son peuple. 
${}^{36}Durant ses jours, grâce à lui, on réussit à extirper les païens du pays, même ceux qui occupaient la Cité de David à Jérusalem. Les païens en avaient fait leur citadelle et, chaque fois qu’ils en sortaient, ils profanaient les abords du Lieu saint, portant gravement atteinte à sa pureté. 
${}^{37}Simon installa en ce lieu des soldats juifs et le fortifia pour la sécurité du pays et de la ville. Il suréleva aussi les remparts de Jérusalem.
${}^{38}Pour tout cela, le roi Démétrios le confirma dans sa charge de grand prêtre, 
${}^{39}le mit au nombre de ses amis et le couvrit de grands honneurs. 
${}^{40}Il avait appris, en effet, que les Romains traitaient les Juifs d’amis, d’alliés et de frères, et qu’ils avaient accueilli avec honneur les ambassadeurs de Simon. 
${}^{41}Il avait appris aussi que les Juifs et les prêtres avaient jugé bon de nommer Simon guide et grand prêtre à perpétuité, jusqu’à ce que se lève un prophète digne de foi. 
${}^{42}En outre, ils lui avaient confié le commandement de l’armée, la responsabilité du Lieu saint, la charge de nommer les chefs de travaux, les administrateurs du pays, de l’armement et des fortifications. 
${}^{43}Tous devaient lui obéir. Tous les actes du pays devaient être rédigés en son nom, et lui-même devait être revêtu de pourpre et porter des ornements d’or. 
${}^{44}Il ne sera permis à personne, ni dans le peuple ni parmi les prêtres, de rejeter une seule de ces dispositions ou de contredire les ordres de Simon, de convoquer sans lui une réunion dans le pays, de revêtir la pourpre ou de porter une agrafe d’or. 
${}^{45}Quiconque agira à l’encontre de ces décisions ou en rejettera une, sera passible d’une sanction. 
${}^{46}Tout le peuple a jugé bon d’accorder à Simon le droit d’agir selon ces dispositions. 
${}^{47}Simon a accepté. Il a jugé bon d’exercer la charge de grand prêtre, d’être chef de l’armée, gouverneur des Juifs et des prêtres, et d’être à la tête de tous. 
${}^{48}On a décidé de mettre cela par écrit sur des tables de bronze, placées en évidence dans l’enceinte du Lieu saint, 
${}^{49}tandis que des copies en seront déposées dans la chambre du trésor, à la disposition de Simon et de ses fils. »
      
         
      \bchapter{}
      \begin{verse}
${}^{1}Antiocos, fils du roi Démétrios, envoya, des îles de la mer, une lettre à Simon, prêtre et gouverneur des Juifs, ainsi qu’à toute la nation. 
${}^{2}Elle était ainsi conçue :
      « Le roi Antiocos à Simon, grand prêtre et gouverneur, et à la nation des Juifs, salut ! 
${}^{3}Des scélérats se sont emparés du royaume de nos pères, mais je veux revendiquer la possession du royaume, afin de le rétablir tel qu’il était auparavant. J’ai donc recruté un grand nombre de troupes mercenaires, j’ai équipé des vaisseaux de guerre. 
${}^{4}Et je veux débarquer dans le pays, afin de poursuivre ceux qui l’ont ravagé et qui ont dévasté de nombreuses villes dans mon royaume. 
${}^{5}Je te confirme donc maintenant toutes les remises d’impôts que t’ont accordées les rois qui m’ont précédé, et la dispense qu’ils t’ont consentie de tous les autres présents. 
${}^{6}Je t’autorise à frapper une monnaie propre à ton pays. 
${}^{7}Jérusalem et le Lieu saint sont libres. Tout l’armement que tu as fabriqué, les forteresses que tu as bâties et qui sont en ton pouvoir, tu peux les conserver. 
${}^{8}Tout ce que tu dois au trésor royal, avec tout ce que tu lui devrais à l’avenir, que cela te soit remis dès maintenant et pour toujours. 
${}^{9}Lorsque nous aurons rétabli notre royauté, nous te couvrirons de grands honneurs, toi-même, ta nation et le Temple, au point de rendre votre gloire éclatante aux yeux du monde entier. »
${}^{10}L’an 174 de l’empire grec, Antiocos partit pour le pays de ses pères. Toutes les troupes se rallièrent à lui, si bien qu’il ne resta qu’une poignée d’hommes avec Tryphon. 
${}^{11}Celui-ci, poursuivi par Antiocos, s’enfuit et parvint à Dora, au bord de la mer : 
${}^{12}il se rendait compte en effet que les malheurs s’étaient accumulés sur lui et que ses troupes l’avaient abandonné. 
${}^{13}Antiocos prit position devant Dora avec cent vingt mille combattants et huit mille cavaliers. 
${}^{14}Il encercla la ville, tandis que ses vaisseaux attaquaient par la mer. Il bloqua les issues de la ville par terre et par mer, ne laissant personne sortir ni entrer.
${}^{15}Entre-temps, Nouménios et ses compagnons revinrent de Rome avec des lettres adressées aux rois et aux pays. En voici le texte :
${}^{16}« Lucius, consul des Romains, au roi Ptolémée, salut ! 
${}^{17}Les ambassadeurs des Juifs, envoyés par le grand prêtre Simon et par le peuple juif, sont venus chez nous en amis et en alliés, pour renouveler l’amitié et l’alliance de jadis. 
${}^{18}Ils ont apporté un bouclier d’or de mille livres. 
${}^{19}Nous avons donc jugé bon d’écrire aux rois et aux pays de ne pas leur causer de tort, de ne pas leur faire la guerre, à eux, ni à leurs villes, ni à leur pays, et de ne pas s’allier à ceux qui leur feraient la guerre. 
${}^{20}Nous avons aussi décidé d’accepter de leur part le bouclier. 
${}^{21}Si donc des scélérats ont fui leur pays pour se rendre auprès de vous, livrez-les au grand prêtre Simon, afin qu’il les punisse selon la loi juive. »
${}^{22}La même lettre fut adressée au roi Démétrios, à Attale, à Ariarathe, à Arsace 
${}^{23}et dans tous les pays : à Sampsamé, aux Spartiates, à Délos, à Myndos, à Sicyone, en Carie, à Samos, en Pamphylie, en Lycie, à Halicarnasse, à Rhodes, à Phasélis, à Cos, à Sidé, à Arados, à Gortyne, à Cnide, à Chypre et à Cyrène. 
${}^{24}Une copie de cette lettre fut établie à l’intention de Simon, le grand prêtre.
${}^{25}Le roi Antiocos avait donc pris position devant Dora, dans son faubourg. Sans répit, il lançait contre elle ses détachements et fabriquait des machines de guerre. Il bloquait Tryphon pour que nul ne puisse ni sortir ni entrer. 
${}^{26}Simon, pour sa part, lui envoya deux mille hommes d’élite pour combattre à ses côtés, ainsi que de l’argent, de l’or et un matériel considérable. 
${}^{27}Non seulement le roi refusa de les recevoir, mais il révoqua tout ce qui avait été convenu auparavant avec Simon et traita celui-ci comme un étranger. 
${}^{28}Il lui envoya Athénobios, l’un de ses amis, pour lui communiquer ceci : « Vous occupez Joppé et Gazara, ainsi que la citadelle de Jérusalem ; or, ce sont des villes de mon royaume. 
${}^{29}Vous avez dévasté leur territoire et causé beaucoup de tort au pays, en vous rendant maîtres de nombreuses localités de mon royaume. 
${}^{30}Maintenant, rendez-moi donc les villes dont vous vous êtes emparés et payez les impôts des régions que vous avez soumises hors des frontières de la Judée. 
${}^{31}Sinon, donnez en échange la somme de cinq cents talents d’argent et cinq cents autres talents pour les dommages causés et les impôts des villes. Faute de quoi, nous viendrons vous faire la guerre. »
${}^{32}Arrivé à Jérusalem, Athénobios, l’ami du roi, vit la magnificence de Simon, un meuble avec des coupes d’or et d’argent et le faste extraordinaire dont il s’entourait : il en fut stupéfait. Il lui fit alors connaître les paroles du roi. 
${}^{33}Simon lui répondit : « Ce n’est pas une terre étrangère que nous avons prise, ni la propriété d’autrui que nous avons conquise, mais bien l’héritage de nos pères, injustement occupé pendant quelque temps par nos ennemis. 
${}^{34}Nous avons simplement profité d’une occasion favorable pour récupérer l’héritage de nos pères. 
${}^{35}Quant à Joppé et Gazara, que tu réclames, ces villes causaient beaucoup de tort à notre peuple et dévastaient notre pays. Pour elles, nous donnerons la somme de cent talents. » 
${}^{36}L’envoyé ne lui répondit mot. Il s’en retourna furieux chez le roi, l’informa de ces paroles, de la magnificence de Simon et de tout ce qu’il avait vu. Et le roi fut pris d’une violente colère.
       
${}^{37}Or, Tryphon monta sur un bateau et s’enfuit vers Orthosia. 
${}^{38}Le roi institua Kendébée général en chef de la zone du littoral. Il lui donna des troupes d’infanterie et de cavalerie, 
${}^{39}avec pour mission de prendre position en face de la Judée, de rebâtir Kédrone, de consolider ses portes et de faire la guerre au peuple. Puis il se lança à la poursuite de Tryphon. 
${}^{40}Kendébée, lui, parvenu à Jamnia, se mit à provoquer le peuple juif, à faire des incursions en Judée, à ramener des prisonniers et à massacrer. 
${}^{41}Il rebâtit Kédrone et y plaça des cavaliers et des fantassins pour opérer des sorties et parcourir les routes de Judée, comme le roi le lui avait ordonné.
      
         
      \bchapter{}
      \begin{verse}
${}^{1}Jean monta de Gazara à Jérusalem et informa Simon, son père, des manœuvres de Kendébée. 
${}^{2}Alors, Simon fit venir ses deux fils aînés, Judas et Jean, et leur dit : « Moi, mes frères et la maison de mon père, nous avons fait la guerre aux ennemis d’Israël depuis notre jeunesse jusqu’à ce jour, et, bien souvent, il nous a été donné de délivrer Israël. 
${}^{3}Maintenant, je suis vieux, tandis que vous, par la miséricorde du Ciel, vous avez l’âge qu’il faut. Prenez donc ma place et celle de mon frère, partez combattre pour notre nation, et que le secours du Ciel soit avec vous ! » 
${}^{4}Il choisit dans le pays vingt mille hommes de guerre et des cavaliers. Ceux-ci marchèrent contre Kendébée et passèrent la nuit à Modine. 
${}^{5}Levés dès l’aube, ils avançaient dans la plaine, lorsqu’une armée nombreuse de fantassins et de cavaliers vint à leur rencontre. Un torrent les séparait encore. 
${}^{6}Jean prit position en face d’eux, lui et son groupe. Quand il vit que celui-ci avait peur de franchir le torrent, il traversa le premier. À cette vue, tous traversèrent derrière lui. 
${}^{7}Il répartit son groupe et plaça les cavaliers au milieu des fantassins, car la cavalerie adverse était très nombreuse. 
${}^{8}On sonna de la trompette. Kendébée fut mis en déroute avec son armée ; il y eut beaucoup de victimes parmi eux, et les survivants s’enfuirent vers la forteresse. 
${}^{9}C’est alors que fut blessé Judas, le frère de Jean ; mais Jean poursuivit l’ennemi jusqu’à ce que Kendébée arrive à Kédrone, la forteresse qu’il avait rebâtie. 
${}^{10}Les ennemis s’enfuirent jusqu’aux tours qui sont dans la campagne d’Azôt, et Jean les incendia. Environ deux mille hommes parmi eux succombèrent. Après cela, Jean retourna en paix en Judée.
      
         
${}^{11}Ptolémée, fils d’Aboubos, avait été nommé gouverneur militaire de la plaine de Jéricho et possédait beaucoup d’argent et d’or. 
${}^{12}Il était en effet le gendre du grand prêtre. 
${}^{13}Son cœur s’exalta : il voulut se rendre maître du pays. Il forma donc le projet perfide de supprimer Simon et ses fils. 
${}^{14}Or, Simon faisait une tournée d’inspection dans les villes du pays, pour veiller à leur administration. Le onzième mois – c’est-à-dire le mois de Shebath – de l’an 177 de l’empire grec, il descendit à Jéricho avec ses fils Mattathias et Judas. 
${}^{15}Avec ruse, le fils d’Aboubos les reçut dans une petite forteresse appelée Dôk, qu’il avait fait construire, et il leur offrit un grand festin. Or, il avait caché des hommes en ce lieu. 
${}^{16}Lorsque Simon et ses fils furent ivres, Ptolémée bondit avec ses hommes ; ils prirent leurs armes et se jetèrent sur Simon dans la salle du festin. Ils le tuèrent, ainsi que ses deux fils et quelques-uns de ses serviteurs. 
${}^{17}Ptolémée commit ainsi une grave trahison et il rendit le mal pour le bien.
${}^{18}Il écrivit au roi pour l’informer de ce qu’il avait fait. Il lui demanda d’envoyer des troupes à son secours et de lui donner le gouvernement du pays et des villes de Judée. 
${}^{19}Il envoya des hommes à Gazara, pour supprimer Jean. Il convoqua par lettres les commandants de l’armée, afin de leur donner de l’argent, de l’or et des présents. 
${}^{20}Il envoya d’autres hommes encore pour prendre possession de Jérusalem et de la montagne du Temple. 
${}^{21}Mais quelqu’un prit les devants et se rendit à Gazara pour annoncer à Jean la mort de son père et de ses frères ; il ajouta : « On a envoyé quelqu’un pour te tuer, toi aussi. » 
${}^{22}À cette nouvelle, Jean fut complètement hors de lui. Il se saisit des hommes qui étaient venus le supprimer et les tua, car il avait appris que ces hommes cherchaient à le faire périr. 
${}^{23}Tout ce que l’on pourrait dire encore au sujet de Jean, de ses guerres et de ses actes de bravoure, des remparts qu’il fit construire et de ses faits et gestes, 
${}^{24}tout cela est raconté dans le livre des Annales de son pontificat, depuis le jour où il devint grand prêtre à la place de son père.
