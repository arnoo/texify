  
  
      <h2 class="intertitle" id="d85e35570">1. Du Sinaï au désert de Parane (11 – 12)</h2>
      
         
      \bchapter{}
      \begin{verse}
${}^{1}Or le peuple se répandit en plaintes qui arrivèrent aux oreilles du Seigneur ; cela lui déplut. Quand il entendit, sa colère s’enflamma ; le feu du Seigneur s’alluma contre eux et dévora une extrémité du camp. 
${}^{2}Alors le peuple cria vers Moïse. Moïse intercéda auprès du Seigneur, et le feu s’apaisa. 
${}^{3}On appela ce lieu Tabeéra (c’est-à-dire : Incendie) car le feu du Seigneur s’y était allumé contre eux.
      
         
      <h3 class="intertitle">Protestation du peuple pour de la viande.<br/>
      Découragement de Moïse</h3>
${}^{4}Il y avait un ramassis de gens qui était mêlé au peuple ; ceux-ci furent saisis de convoitise. <a class="anchor verset_lettre" id="bib_nb_11_4_b"/>Même les fils d’Israël se remirent à pleurer : « Ah ! qui donc nous donnera de la viande à manger ? 
${}^{5}Nous nous rappelons encore le poisson que nous mangions pour rien en Égypte, et les concombres, les melons, les poireaux, les oignons et l’ail ! 
${}^{6}Maintenant notre gorge est desséchée ; nous ne voyons jamais rien que de la manne ! »
${}^{7}La manne était comme des grains de coriandre\\, elle ressemblait à de l’ambre jaune\\. 
${}^{8} Le peuple se dispersait pour la recueillir ; puis on la broyait sous la meule, ou on l’écrasait au pilon ; enfin on la cuisait dans la marmite et on en faisait des galettes. Elle avait le goût d’une friandise à l’huile. 
${}^{9} Lorsque, pendant la nuit, la rosée descendait sur le camp, la manne descendait sur elle.
${}^{10}Moïse entendit pleurer le peuple, groupé par clans, chacun à l’entrée de sa tente. Le Seigneur s’enflamma d’une grande colère. Cela déplut à Moïse\\, 
${}^{11}et il dit au Seigneur : « Pourquoi traiter si mal ton serviteur ? Pourquoi n’ai-je pas trouvé grâce à tes yeux que tu m’aies imposé le fardeau de tout ce peuple ? 
${}^{12}Est-ce moi qui ai conçu tout ce peuple, est-ce moi qui l’ai enfanté, pour que tu me dises : “Comme on\\porte un nourrisson, porte ce peuple dans tes bras jusqu’au pays que j’ai juré de donner\\à tes pères” ? 
${}^{13}Où puis-je trouver de la viande pour en donner à tout ce peuple, quand ils viennent pleurer près de moi en disant : “Donne-nous de la viande à manger” ? 
${}^{14}Je ne puis, à moi seul, porter tout ce peuple : c’est trop lourd pour moi. 
${}^{15}Si c’est ainsi que tu me traites, tue-moi donc ; oui, tue-moi, si j’ai trouvé grâce à tes yeux. Que je ne voie pas mon malheur ! »
${}^{16}Le Seigneur dit alors à Moïse : « Rassemble-moi soixante-dix hommes parmi les anciens d’Israël, connus par toi comme des anciens et des scribes du peuple. Tu les amèneras à la tente de la Rencontre, où ils se présenteront avec toi. 
${}^{17}Là, je descendrai pour te parler, et je prendrai une part de l’esprit qui est sur toi pour le mettre sur eux. Ainsi ils porteront avec toi le fardeau de ce peuple, et tu ne seras plus seul à le porter. 
${}^{18}Au peuple, tu diras : Sanctifiez-vous pour demain ! Et vous mangerez de la viande, car les oreilles du Seigneur ont entendu vos pleurs quand vous disiez : “Qui nous donnera de la viande à manger ? Comme nous étions bien en Égypte !” Eh bien ! Le Seigneur vous donnera de la viande, et vous en mangerez ! 
${}^{19}Vous n’en mangerez pas seulement un jour, deux jours, cinq jours, dix jours, vingt jours, 
${}^{20}mais tout un mois, jusqu’à ce qu’elle vous sorte par le nez, et que vous en ayez la nausée. Tout cela parce que vous avez rejeté le Seigneur qui est au milieu de vous et que vous avez pleuré devant lui en disant : “Pourquoi donc sommes-nous sortis d’Égypte ?” »
${}^{21}Moïse répliqua : « Le peuple au milieu duquel je suis compte 600 000 hommes à pied, et toi, tu dis : “Je leur donnerai de la viande, et ils mangeront pendant tout un mois !” 
${}^{22}Égorgera-t-on pour eux du petit et du gros bétail ? Et cela leur suffirait-il ? Tous les poissons de la mer, si on pouvait les ramasser pour eux, cela leur suffirait-il ? » 
${}^{23}Et le Seigneur dit à Moïse : « La main du Seigneur serait-elle trop courte ? Maintenant tu vas voir si ma parole se réalise pour toi, oui ou non ! »
${}^{24}Moïse sortit pour transmettre au peuple les paroles du Seigneur. Puis il réunit soixante-dix hommes parmi les anciens du peuple et les plaça autour de la Tente.
${}^{25}Le Seigneur descendit dans la nuée pour parler avec Moïse. Il prit une part de l’esprit qui reposait sur celui-ci, et le mit sur les soixante-dix anciens. Dès que l’esprit reposa sur eux, ils se mirent à prophétiser, mais cela ne dura pas\\.
${}^{26}Or, deux hommes étaient restés dans le camp ; l’un s’appelait Eldad, et l’autre Médad. L’esprit reposa sur eux ; eux aussi avaient été choisis\\, mais ils ne s’étaient pas rendus à la Tente, et c’est dans le camp qu’ils se mirent à prophétiser. 
${}^{27}Un jeune homme courut annoncer à Moïse : « Eldad et Médad prophétisent dans le camp ! » 
${}^{28}Josué, fils de Noun, auxiliaire\\de Moïse depuis sa jeunesse, prit la parole : « Moïse, mon maître, arrête-les ! » 
${}^{29}Mais Moïse lui dit : « Serais-tu jaloux pour moi ? Ah ! Si le Seigneur pouvait faire de tout son peuple un peuple de prophètes ! Si le Seigneur pouvait mettre son esprit sur eux ! » 
${}^{30}Puis Moïse se retira dans le camp et, avec lui, les anciens d’Israël.
${}^{31}Envoyé par le Seigneur, le vent se leva ; depuis la mer, il amena des cailles, il les rabattit sur le camp et tout autour du camp sur une largeur d’une journée de marche à peu près ; elles couvraient la surface du sol sur deux coudées d’épaisseur environ. 
${}^{32}Le peuple resta debout tout ce jour-là, toute la nuit et toute la journée du lendemain ; ils ramassèrent les cailles. Celui qui en eut le moins en ramassa dix grandes mesures. Ils prirent beaucoup de temps pour les étaler tout autour du camp. 
${}^{33}La viande était encore entre leurs dents, ils n’avaient pas fini de la mâcher que déjà la colère du Seigneur s’enflammait contre le peuple et qu’il frappait le peuple ; il le frappa d’un très grand coup. 
${}^{34}On appela donc ce lieu Qibroth-ha-Taawa (c’est-à-dire : Tombeaux-de-la-convoitise) car c’est là qu’on enterra la foule de ceux qui avaient été pris de convoitise.
${}^{35}De Qibroth-ha-Taawa, le peuple partit pour Hacéroth, et il resta à Hacéroth.
      
         
      \bchapter{}
      \begin{verse}
${}^{1}Parce que Moïse avait épousé une femme éthiopienne\\, sa sœur\\Miryam et son frère\\Aaron se mirent à le critiquer. 
${}^{2} Ils disaient : « Le Seigneur parle-t-il uniquement par Moïse ? Ne parle-t-il pas aussi par nous ? » Le Seigneur entendit. 
${}^{3} – Or, Moïse était très humble, l’homme le plus humble que la terre ait porté. 
${}^{4} Soudain, le Seigneur dit à Moïse, à Aaron et à Miryam : « Sortez tous les trois pour aller à la tente de la Rencontre. » Ils sortirent tous les trois. 
${}^{5} Le Seigneur descendit dans la colonne de nuée et s’arrêta à l’entrée de la Tente. Il appela Aaron et Miryam ; tous deux s’avancèrent, et il leur dit : 
${}^{6} « Écoutez bien mes paroles : Quand il y a parmi vous un prophète du Seigneur\\, je me fais connaître à lui dans une vision, je lui parle dans un songe. 
${}^{7} Il n’en est pas ainsi pour mon serviteur Moïse, lui qui, dans toute ma maison, est digne de confiance\\ : 
${}^{8} c’est de vive voix que je lui parle, dans une vision claire et non pas en énigmes ; ce qu’il regarde, c’est la forme même du Seigneur. Pourquoi avez-vous osé\\critiquer mon serviteur Moïse ? »
${}^{9}La colère du Seigneur s’enflamma contre eux, puis il s’en alla. 
${}^{10}La nuée s’éloigna de la tente, et voici : Miryam était couverte d’une lèpre blanche\\comme de la neige. Aaron se tourna vers elle, et voici qu’elle était lépreuse. 
${}^{11}Il dit alors à Moïse : « Je t’en supplie, mon seigneur, ne fais pas retomber sur nous ce péché que nous avons eu la folie de commettre. 
${}^{12}Que Miryam ne soit pas comme l’enfant mort-né\\dont la chair est à demi rongée lorsqu’il sort du sein de sa mère ! » 
${}^{13}Moïse cria vers le Seigneur : « Dieu, je t’en prie, guéris-la ! » 
${}^{14}Mais le Seigneur dit à Moïse : « Si son père lui crachait au visage, n’aurait-elle pas honte pendant sept jours ? Qu’elle soit donc exclue du camp pendant sept jours ; après quoi, elle sera réintégrée. » 
${}^{15}Miryam fut donc exclue du camp pendant sept jours, et le peuple ne leva pas le camp avant qu’elle ne soit réintégrée.
${}^{16}Ensuite, le peuple partit d’Hacéroth et ils campèrent dans le désert de Parane.
      <h2 class="intertitle" id="d85e36123">2. De Cadès et autour de Cadès (13 – 20,13)</h2>
      
         
      \bchapter{}
      \begin{verse}
${}^{1}Le Seigneur parla à Moïse. Il dit : 
${}^{2}« Envoie des hommes pour explorer le pays de Canaan que je donne aux fils d’Israël. Vous enverrez un homme par tribu patriarcale, chacun d’eux sera un responsable. »
${}^{3}Moïse les envoya donc du désert de Parane, sur l’ordre du Seigneur : tous ces hommes étaient les chefs des fils d’Israël. 
${}^{4}Et voici leurs noms :
      pour la tribu de Roubène, Shammoua, fils de Zakkour ;
${}^{5}pour la tribu de Siméon, Shafath, fils de Hori ;
${}^{6}pour la tribu de Juda, Caleb, fils de Yefounnè ;
${}^{7}pour la tribu d’Issakar, Yiguéal, fils de Joseph ;
${}^{8}pour la tribu d’Éphraïm, Osée, fils de Noun ;
${}^{9}pour la tribu de Benjamin, Palti, fils de Rafou ;
${}^{10}pour la tribu de Zabulon, Gaddiël, fils de Sôdi ;
${}^{11}pour la tribu de Joseph – tribu de Manassé – Gaddi, fils de Soussi ;
${}^{12}pour la tribu de Dane, Ammiël, fils de Guemalli ;
${}^{13}pour la tribu d’Asher, Setour, fils de Mikaël ;
${}^{14}pour la tribu de Nephtali, Nahbi, fils de Wofsi ;
${}^{15}pour la tribu de Gad, Guéouël, fils de Maki.
${}^{16}Voilà les noms des hommes que Moïse envoya explorer le pays. Moïse donna à Osée, fils de Noun, le nom de Josué.
${}^{17}Moïse les envoya explorer le pays de Canaan. Il leur dit : « Montez par le Néguev, montez dans la montagne. 
${}^{18}Regardez le pays : comment est-il ? Regardez la population qui l’habite : est-elle forte ou faible, nombreuse ou pas ? 
${}^{19}Comment est le pays où cette population habite : est-il bon ou mauvais ? Comment sont les villes où cette population habite : sont-elles des campements ou des forteresses ? 
${}^{20}Comment est ce pays : sa terre est-elle grasse ou maigre ? Y pousse-t-il ou non des arbres ? Rassemblez vos forces et prenez les fruits du pays. »
      Or c’était le moment des premiers raisins. 
${}^{21}Ils montèrent donc explorer le pays depuis le désert de Cine jusqu’à Rehob, l’Entrée-de-Hamath. 
${}^{22}Ils montèrent par le Néguev et arrivèrent à Hébron où habitaient Ahimane, Sheshaï et Talmaï, descendants d’Anaq. Hébron avait été bâtie sept ans avant Tanis en Égypte. 
${}^{23}Ils allèrent jusqu’à la vallée d’Eshkol où ils coupèrent un sarment et une grappe de raisin. Ils la portaient à deux au moyen d’une perche. Ils avaient aussi cueilli des grenades et des figues. 
${}^{24}On appela cet endroit la vallée d’Eshkol (c’est-à-dire : la Grappe) à cause de la grappe que les fils d’Israël avaient coupée là-bas.
${}^{25}Au bout de quarante jours, ces envoyés\\revinrent, après avoir exploré le pays. 
${}^{26} Ils allèrent trouver Moïse, Aaron et toute la communauté des fils d’Israël, à Cadès, dans le désert de Parane. Ils firent leur rapport devant eux et devant toute la communauté, et ils leur montrèrent les fruits du pays.
${}^{27}Ils firent ce récit à Moïse : « Nous sommes allés dans le pays où tu nous as envoyés. Vraiment, il ruisselle de lait et de miel, et voici ses fruits. 
${}^{28} Cependant le peuple qui l’habite est puissant, ses villes sont fortifiées et très grandes. Nous y avons même vu des descendants d’Anaq. 
${}^{29} Les Amalécites\\habitent le pays du Néguev ; les Hittites, les Jébuséens et les Amorites habitent la montagne ; les Cananéens habitent le bord de la mer et les rives du Jourdain. »
${}^{30}Caleb imposa silence au peuple qui faisait face à\\Moïse et il dit : « Allons-y ! Montons prendre possession de ce pays. Oui, nous nous en rendrons maîtres. » 
${}^{31} Mais les autres hommes de l’expédition\\répliquèrent : « Nous ne pouvons pas marcher contre ce peuple, car il est plus fort que nous. » 
${}^{32} Et, s’adressant aux fils d’Israël, ils se mirent à dénigrer le pays qu’ils avaient exploré : « Le pays que nous avons parcouru pour l’explorer est un pays qui dévore ceux qui veulent s’y installer\\. Tous les hommes que nous y avons vus sont de très haute taille. 
${}^{33} Nous y avons même vu des géants, des fils d’Anaq, des descendants\\de géants. À côté d’eux, nous avions l’air de sauterelles, et c’est bien ainsi qu’ils nous voyaient. »
      
         
      \bchapter{}
      \begin{verse}
${}^{1}Alors toute la communauté éleva la voix, se mit à crier ; et le peuple pleura cette nuit-là. 
${}^{2}Tous les fils d’Israël récriminèrent contre Moïse et Aaron. La communauté tout entière leur dit : « Ah ! Si nous étions morts au pays d’Égypte ou si nous étions morts dans ce désert ! 
${}^{3}Pourquoi le Seigneur nous conduit-il vers ce pays ? Pour que nous tombions par l’épée ? Nos femmes et nos enfants deviendraient un butin ! Ne serait-il pas mieux pour nous de retourner en Égypte ? » 
${}^{4}Et ils se dirent l’un à l’autre : « Donnons-nous un chef et retournons en Égypte. »
${}^{5}Moïse et Aaron tombèrent face contre terre devant toute l’assemblée de la communauté des fils d’Israël. 
${}^{6}Josué, fils de Noun, et Caleb, fils de Yefounnè, qui, parmi d’autres, avaient exploré le pays déchirèrent leurs vêtements. 
${}^{7}Puis il parlèrent à toute la communauté des fils d’Israël. Ils dirent : « Le pays que nous avons parcouru pour l’explorer, ce pays est très, très bon. 
${}^{8}Si le Seigneur nous est favorable, il nous fera entrer dans ce pays et il nous le donnera. C’est un pays ruisselant de lait et de miel. 
${}^{9}Mais ne vous révoltez pas contre le Seigneur. Ne craignez pas les gens du pays : nous n’en ferons qu’une bouchée ! Leur ombre protectrice s’est éloignée d’eux, et le Seigneur est avec nous. Ne les craignez pas ! »
${}^{10}Toute la communauté parlait de les lapider, mais la gloire du Seigneur apparut dans la tente de la Rencontre à tous les fils d’Israël. 
${}^{11}Et le Seigneur dit à Moïse : « Combien de temps encore ce peuple me méprisera-t-il ? Combien de temps refuseront-ils de croire en moi, de croire tous les signes que j’ai accomplis au milieu d’eux ? 
${}^{12}Je le frapperai de la peste et je le déposséderai. Et de toi je ferai une nation plus grande et plus puissante que lui ! » 
${}^{13}Moïse répondit au Seigneur : « Mais cela, les Égyptiens l’apprendront, car ce peuple, tu l’as fait sortir de chez eux par ta puissance, 
${}^{14}et ils le diront à l’habitant de Canaan ! Ils avaient appris que toi, Seigneur, tu es au milieu de ce peuple, que toi, Seigneur, tu te laisses voir les yeux dans les yeux, que ta nuée se tient au-dessus d’eux, que toi, tu marches devant eux, dans une colonne de nuée le jour, dans une colonne de feu la nuit. 
${}^{15}Et tu ferais mourir ce peuple comme un seul homme ! Alors les nations qui ont entendu parler de toi diraient : 
${}^{16}“Le Seigneur n’était pas capable de faire entrer ce peuple dans le pays qu’il avait juré de leur donner. Aussi les a-t-il massacrés dans le désert !” 
${}^{17}Que maintenant donc se déploie la puissance de mon Seigneur ! En effet, tu avais déclaré : 
${}^{18}“Le Seigneur est lent à la colère et plein d’amour, il supporte faute et transgression, mais il ne laisse rien passer, car il punit la faute des pères sur les fils, jusqu’à la troisième et la quatrième génération.” 
${}^{19}Selon la grandeur de ton amour, pardonne donc la faute de ce peuple, comme tu as porté ce peuple d’Égypte jusqu’ici ! »
${}^{20}Alors le Seigneur dit : « Je pardonne selon ta parole. 
${}^{21}Mais, aussi vrai que je suis vivant et que la gloire du Seigneur emplit toute la terre, 
${}^{22}oui, tous ces hommes qui ont vu ma gloire et mes signes accomplis en Égypte et dans le désert, qui m’ont mis à l’épreuve dix fois déjà et n’ont pas écouté ma voix, 
${}^{23}eh bien, jamais ils ne verront le pays que j’ai juré de donner à leurs pères. Aucun de ceux qui me méprisent ne le verra. 
${}^{24}Mais mon serviteur Caleb, puisqu’il est animé d’un autre esprit et qu’il adhère à moi sans réserve, je le ferai entrer dans le pays où il s’est rendu, et sa descendance en prendra possession. 
${}^{25}Les Amalécites et les Cananéens habitent dans la vallée : demain, faites demi-tour et partez au désert en direction de la mer des Roseaux. »
${}^{26}Le Seigneur parla à Moïse et à son frère\\Aaron. Il dit : 
${}^{27}« Combien de temps aurai-je affaire à cette communauté mauvaise qui récrimine contre moi ? Les récriminations des fils d’Israël\\contre moi, je les ai entendues. 
${}^{28}Tu leur diras : “Aussi vrai que je suis vivant – oracle du Seigneur –, je vous traiterai selon vos paroles mêmes qui sont arrivées jusqu’à mes oreilles\\. 
${}^{29}Vous tous qu’on a recensés\\, les hommes de vingt ans et plus, vous qui avez récriminé contre moi, vos cadavres resteront dans ce désert. 
${}^{30}Jamais vous n’entrerez dans le pays où j’ai juré, la main levée, de vous faire demeurer, sauf Caleb, fils de Yefounnè, et Josué, fils de Noun. 
${}^{31}Mais vos enfants, ceux dont vous aviez dit qu’ils deviendraient un butin, je les y ferai entrer et ils connaîtront le pays que vous avez dédaigné. 
${}^{32}Vos cadavres, à vous, resteront dans ce désert, 
${}^{33}et vos fils seront bergers dans le désert pendant quarante ans, ils porteront le poids de vos prostitutions jusqu’à totale disparition de vos cadavres dans le désert. 
${}^{34}Vous avez exploré le pays pendant quarante jours, chaque jour vaudra une année : vous porterez donc le poids de vos fautes pendant quarante ans, et vous saurez ce qu’il en coûte d’encourir ma réprobation.” 
${}^{35}Moi, le Seigneur, j’ai parlé. Oui, c’est ainsi que je traiterai cette communauté mauvaise liguée\\contre moi. Dans ce désert, tous finiront leur vie : là, ils mourront. »
${}^{36}Quant aux hommes que Moïse avait envoyés explorer le pays et qui, à leur retour, avaient poussé toute la communauté à récriminer contre Moïse en dénigrant le pays, 
${}^{37}ces hommes qui, par méchanceté, avaient dénigré le pays moururent en présence du Seigneur, frappés par un fléau. 
${}^{38}Josué, fils de Noun, et Caleb, fils de Yefounnè, furent les seuls survivants des hommes qui étaient allés explorer le pays.
${}^{39}Moïse rapporta les paroles du Seigneur à tous les fils d’Israël, et le peuple mena un grand deuil. 
${}^{40}Le lendemain, ils se levèrent de bon matin et montèrent au sommet de la montagne occupée par les gens du pays. Ils s’étaient dit : « Nous, nous monterons à l’endroit dont a parlé le Seigneur. Car nous avons péché. » 
${}^{41}Mais Moïse dit : « Pourquoi transgresser l’ordre du Seigneur ? Cela ne réussira pas ! 
${}^{42}Ne montez pas, car le Seigneur n’est pas au milieu de vous ! Et vous ne serez pas battus par vos ennemis. 
${}^{43}En effet, les Amalécites et les Cananéens sont là devant vous, et vous tomberez par l’épée. Puisque vous vous êtes détournés du Seigneur, le Seigneur ne sera pas avec vous ! »
${}^{44}Mais ils s’obstinèrent à monter au sommet de la montagne, alors que ni l’arche de l’Alliance du Seigneur ni Moïse ne quittaient le milieu du camp. 
${}^{45}Les Amalécites et les Cananéens qui habitaient cette montagne descendirent, les frappèrent et les mirent en pièces jusqu’à Horma.
      
         
      \bchapter{}
      \begin{verse}
${}^{1}Le Seigneur parla à Moïse. Il dit : 
${}^{2}« Parle aux fils d’Israël. Tu leur diras : Quand vous entrerez dans le pays où vous habiterez, le pays que je vous donne, 
${}^{3}et que vous présenterez au Seigneur la nourriture offerte, holocauste ou sacrifice, pour accomplir un vœu ou pour rendre grâce, ou bien à l’occasion de vos fêtes, donc lorsque vous présenterez du gros ou du petit bétail en agréable odeur pour le Seigneur, 
${}^{4}alors celui qui apporte au Seigneur son présent réservé apportera une offrande d’un dixième de fleur de farine pétrie avec un quart de mesure d’huile 
${}^{5}et aussi un quart de mesure de vin pour la libation : tu l’ajouteras à l’holocauste ou au sacrifice, pour chaque agneau. 
${}^{6}Si c’est un bélier, tu feras une offrande de deux dixièmes de fleur de farine pétrie avec un tiers de mesure d’huile, 
${}^{7}et un tiers de mesure de vin pour la libation : tu l’apporteras en agréable odeur pour le Seigneur. 
${}^{8}Et quand tu offriras un jeune taureau, soit en holocauste soit en sacrifice, pour accomplir un vœu ou comme sacrifice de paix pour le Seigneur, 
${}^{9}tu apporteras, en plus du jeune taureau, une offrande de trois dixièmes de fleur de farine pétrie avec une demi-mesure d’huile, 
${}^{10}et tu apporteras une demi-mesure de vin pour la libation : c’est une nourriture offerte, en agréable odeur pour le Seigneur. 
${}^{11}On fera de même pour chaque taureau, chaque bélier, ou pour les têtes de petit bétail, moutons ou chèvres. 
${}^{12}Quel que soit le nombre de ceux que vous offrirez, vous ferez la même chose pour chacun, oui, quel que soit leur nombre. 
${}^{13}Tout Israélite de souche fera les choses ainsi, lorsqu’il apportera de la nourriture offerte, en agréable odeur pour le Seigneur. 
${}^{14}L’immigré qui réside chez vous ou quiconque réside parmi vous pour plusieurs générations, lorsqu’il présentera une nourriture offerte, en agréable odeur pour le Seigneur, il fera comme vous ferez. 
${}^{15}Pour l’assemblée, il n’y aura qu’un seul rituel pour vous et pour l’immigré résidant ; c’est un décret perpétuel, d’âge en âge. Il en ira de même pour vous et pour l’immigré devant le Seigneur. 
${}^{16}Il y aura une seule loi, un seul droit, pour vous et pour l’immigré résidant chez vous. »
${}^{17}Puis le Seigneur parla à Moïse. Il dit : 
${}^{18}« Parle aux fils d’Israël. Tu leur diras : Quand vous entrerez dans le pays où je vais vous faire entrer, 
${}^{19}et que vous mangerez de la nourriture du pays, vous en prélèverez une part pour le Seigneur. 
${}^{20}Comme prémices de vos fournées, vous prélèverez un gâteau ; vous le prélèverez de la même manière que vous prélevez du grain sur l’aire à battre. 
${}^{21}Des prémices de vos fournées, vous donnerez une part au Seigneur, et ceci d’âge en âge.
${}^{22}Lorsque vous commettez une faute par inadvertance, et ne pratiquez pas tous ces commandements dont le Seigneur a parlé à Moïse, 
${}^{23}– tout ce que le Seigneur vous a ordonné par Moïse, depuis le jour où le Seigneur a donné l’ordre et par la suite, d’âge en âge –, 
${}^{24}si la faute par inadvertance a été commise à l’insu de la communauté, toute la communauté présentera un taureau en holocauste, en agréable odeur pour le Seigneur, avec son offrande de céréales et sa libation, suivant le rite, et elle présentera un bouc en sacrifice pour la faute. 
${}^{25}Alors le prêtre fera le rite d’expiation sur toute la communauté des fils d’Israël, et il leur sera pardonné car il s’agit d’une faute par inadvertance ; quant à eux, ils auront apporté devant le Seigneur leur présent réservé : une nourriture offerte pour le Seigneur, ainsi que le bouc de leur sacrifice pour la faute, tout cela pour leur faute par inadvertance. 
${}^{26}Il sera donc pardonné à toute la communauté des fils d’Israël ainsi qu’à l’immigré résidant parmi eux, car c’est tout le peuple qui a commis une faute par inadvertance.
${}^{27}Si une personne commet une faute, une faute par inadvertance, elle apportera, en sacrifice pour la faute, une chèvre de l’année. 
${}^{28}Devant le Seigneur, le prêtre fera le rite d’expiation sur la personne qui a commis une faute par inadvertance ; il fera donc le rite d’expiation sur elle, et elle sera pardonnée. 
${}^{29}Pour l’Israélite de souche – parmi les fils d’Israël – et pour l’immigré résidant parmi eux, une seule loi sera appliquée à qui commet une faute par inadvertance.
${}^{30}Mais la personne qui, israélite de souche ou immigrée, agit délibérément, c’est le Seigneur qu’elle outrage : cette personne sera retranchée du milieu de son peuple. 
${}^{31}Puisqu’elle a méprisé la parole du Seigneur et violé son ordre, elle devra être retranchée, oui, retranchée de son peuple : sa faute est en elle ! »
${}^{32}Alors que les fils d’Israël étaient dans le désert, on trouva un homme ramassant du bois le jour du sabbat. 
${}^{33}Ceux qui l’avaient trouvé ramassant du bois l’amenèrent à Moïse, à Aaron et à toute la communauté. 
${}^{34}On le mit sous bonne garde, car il n’y avait pas de décision sur ce qu’on devait lui faire.
${}^{35}Alors le Seigneur dit à Moïse : « L’homme sera mis à mort : toute la communauté le lapidera en dehors du camp. » 
${}^{36}Comme le Seigneur l’avait ordonné à Moïse, toute la communauté le fit donc sortir hors du camp, on le lapida et il mourut.
${}^{37}Le Seigneur parla à Moïse. Il dit : 
${}^{38}« Parle aux fils d’Israël. Tu leur diras qu’ils se fassent une frange aux pans de leurs vêtements, et ceci d’âge en âge, et qu’ils placent sur la frange du pan de leur vêtement un cordon de pourpre violette. 
${}^{39}Vous aurez donc une frange ; chaque fois que vous la regarderez, vous vous rappellerez tous les commandements du Seigneur et vous les mettrez en pratique ; vous ne vous laisserez pas entraîner, comme les explorateurs, par vos cœurs et vos yeux qui vous mèneraient à la prostitution. 
${}^{40}Ainsi vous vous rappellerez et vous mettrez en pratique tous mes commandements, et vous serez saints pour votre Dieu. 
${}^{41}Je suis le Seigneur votre Dieu, moi qui vous ai fait sortir du pays d’Égypte pour être votre Dieu. Je suis le Seigneur votre Dieu. »
      
         
      \bchapter{}
      \begin{verse}
${}^{1}Coré, fils de Yicehar, fils de Qehath, fils de Lévi, prit avec lui Datane et Abiram, fils d’Éliab, ainsi que One, fils de Pèleth, des descendants de Roubène. 
${}^{2}Ils se dressèrent face à Moïse, ainsi que deux cent cinquante hommes ; c’étaient des fils d’Israël, responsables de communauté, délégués de l’assemblée, hommes de renom. 
${}^{3}Ils s’assemblèrent contre Moïse et contre Aaron, et leur dirent : « On en a assez de vous ! Oui, toute la communauté des fils d’Israël, eux tous, ils sont saints, et le Seigneur se tient au milieu d’eux. Pourquoi donc vous élevez-vous au-dessus de l’assemblée du Seigneur ? »
${}^{4}En entendant ces mots, Moïse tomba face contre terre, 
${}^{5}puis il parla à Coré et à toute sa communauté. Il dit : « Demain matin, le Seigneur fera connaître qui lui appartient, qui est saint, et celui-là, il le fera approcher de lui ; celui donc qu’il aura choisi, il le fera approcher de lui. 
${}^{6}Faites ceci : procurez-vous des brûle-parfums, vous, Coré et toute ta communauté. 
${}^{7}Demain, mettez-y du feu et, au-dessus, placez de l’encens devant le Seigneur. Alors, l’homme que le Seigneur choisira, c’est celui-là qui est saint. On en a assez de vous, fils de Lévi ! »
${}^{8}Moïse dit encore à Coré : « Écoutez donc, fils de Lévi ! 
${}^{9}Ne vous suffit-il pas que le Dieu d’Israël vous ait séparés de la communauté d’Israël en vous faisant approcher de lui, pour accomplir le service de la demeure du Seigneur et vous tenir face à la communauté quand vous officiez pour eux ? 
${}^{10}Il t’a fait approcher, toi et, avec toi, tous tes frères, les fils de Lévi, et vous réclamez encore le sacerdoce ! 
${}^{11}Serait-ce pour cela que toi et toute ta communauté, vous vous liguez contre le Seigneur ? Et Aaron, que vous a-t-il fait pour que vous récriminiez contre lui ? »
${}^{12}Moïse envoya appeler Datane et Abiram, fils d’Éliab, mais ils répondirent : « Nous ne monterons pas ! 
${}^{13}Ne te suffit-il pas de nous avoir fait venir d’un pays ruisselant de lait et de miel pour nous faire mourir dans le désert ? Et tu veux encore te poser en prince au-dessus de nous ! 
${}^{14}Oh non, ce n’est pas dans un pays ruisselant de lait et de miel que tu nous as fait entrer ! Tu ne nous as pas donné des champs et des vignobles en héritage ! Même si tu crevais les yeux de ces hommes, nous ne monterons pas ! »
${}^{15}Alors Moïse s’enflamma d’une grande colère et il dit au Seigneur : « Ne tourne pas ta face vers leur offrande : je n’ai pas enlevé un seul de leurs ânes, je n’ai maltraité aucun d’entre eux ! » 
${}^{16}Puis Moïse dit à Coré : « Toi et toute ta communauté, soyez-là demain devant le Seigneur, toi, eux tous et Aaron ! 
${}^{17}Prenez chacun votre brûle-parfum, vous y mettrez de l’encens ; puis vous présenterez chacun votre brûle-parfum devant le Seigneur : il y aura deux cent cinquante brûle-parfums. Toi et Aaron, vous aurez aussi chacun votre brûle-parfum. » 
${}^{18}Chacun prit donc son brûle-parfum, y mit du feu et plaça de l’encens par-dessus ; ils se tinrent à l’entrée de la tente de la Rencontre, ainsi que Moïse et Aaron. 
${}^{19}Coré avait rassemblé contre ceux-ci toute la communauté à l’entrée de la tente de la Rencontre. La gloire du Seigneur apparut alors à toute la communauté. 
${}^{20}Le Seigneur parla à Moïse et Aaron. Il dit : 
${}^{21}« Séparez-vous de cette communauté : je vais les exterminer à l’instant ! » 
${}^{22}Mais eux tombèrent face contre terre et parlèrent ainsi : « Ô Dieu, Dieu des esprits qui animent toute chair, un seul homme est en faute, et c’est contre toute la communauté que tu t’irriterais ! »
${}^{23}Alors le Seigneur parla à Moïse, il dit : 
${}^{24}« Parle à la communauté, tu diras : “Éloignez-vous des abords de la demeure de Coré, Datane et Abiram !” » 
${}^{25}Moïse se leva et alla trouver Datane et Abiram. Les anciens d’Israël le suivaient. 
${}^{26}Il parla à la communauté. Il dit : « Écartez-vous donc des tentes de ces hommes, de ces coupables ! Ne touchez à rien de ce qui leur appartient, de peur que vous ne périssiez pour toutes leurs fautes ! » 
${}^{27}Ils s’éloignèrent donc des abords de la demeure de Coré, Datane et Abiram, tandis que Datane et Abiram sortaient et se tenaient debout à l’entrée de leurs tentes avec leurs femmes, leurs fils et leurs jeunes enfants. 
${}^{28}Alors, Moïse dit : « À ceci vous saurez que c’est le Seigneur qui m’a envoyé pour accomplir toutes ces œuvres, je ne les accomplis pas de moi-même : 
${}^{29}si ces gens meurent de la mort de tous les hommes, s’ils subissent le sort de tous les hommes, c’est que le Seigneur ne m’a pas envoyé ! 
${}^{30}Mais si le Seigneur crée quelque chose, si le sol, ouvrant sa gueule, les engloutit, avec tout ce qui leur appartient, s’ils descendent vivants au séjour des morts, alors vous saurez que ces hommes ont méprisé le Seigneur. »
${}^{31}Comme il achevait de dire toutes ces paroles, le sol, en dessous d’eux, se fendit, 
${}^{32}la terre ouvrit sa gueule et les engloutit avec leurs familles, ainsi que tous les gens de Coré et tous leurs biens. 
${}^{33}Ils descendirent, vivants, au séjour des morts, eux et tout ce qui leur appartenait ; la terre les recouvrit et ils disparurent du milieu de l’assemblée. 
${}^{34}Et tout Israël qui était autour d’eux s’enfuit à leurs cris, en disant : « Pourvu que la terre ne nous engloutisse pas ! »
${}^{35}Un feu jaillit d’auprès du Seigneur et dévora les deux cent cinquante hommes qui présentaient l’encens.
      
         
      \bchapter{}
      \begin{verse}
${}^{1}Le Seigneur parla à Moïse. Il dit : 
${}^{2}« Ordonne à Éléazar, fils du prêtre Aaron, de retirer du brasier les brûle-parfums et de répandre au loin leur feu, car ils sont saints. 
${}^{3}Les brûle-parfums de ces hommes qui ont payé de leur vie leur péché, qu’on en fasse des plaques martelées pour le revêtement de l’autel, car ils ont été apportés devant le Seigneur et ils sont saints. Qu’ils deviennent un signe pour les fils d’Israël ! »
${}^{4}Le prêtre Éléazar prit donc les brûle-parfums de bronze que les hommes brûlés vifs avaient apportés, et on en fit des plaques pour le revêtement de l’autel. 
${}^{5}C’est un mémorial pour les fils d’Israël, afin qu’aucun homme profane, c’est-à-dire étranger à la lignée d’Aaron, ne s’approche pour brûler de l’encens devant le Seigneur et afin que personne ne soit traité comme Coré et sa communauté, selon ce que le Seigneur lui avait dit par l’intermédiaire de Moïse !
${}^{6}Le lendemain, toute la communauté des fils d’Israël se mit à récriminer contre Moïse et contre Aaron : « C’est vous, disaient-ils, qui avez fait mourir le peuple du Seigneur ! » 
${}^{7}Or, tandis que la communauté se rassemblait contre eux, Moïse et Aaron se tournèrent vers la tente de la Rencontre : voici que la nuée l’avait couverte, et la gloire du Seigneur apparut. 
${}^{8}Alors Moïse et Aaron allèrent devant la tente de la Rencontre. 
${}^{9}Le Seigneur parla à Moïse. Il dit : 
${}^{10}« Éloignez-vous de cette communauté ; je vais les exterminer à l’instant ! » Moïse et Aaron tombèrent face contre terre. 
${}^{11}Puis Moïse dit à Aaron : « Prends le brûle-parfum, mets-y du feu de l’autel et place dessus de l’encens ; emporte-le vite vers la communauté et fais sur eux le rite d’expiation, car la colère du Seigneur a éclaté et le fléau a commencé ! » 
${}^{12}Aaron prit le brûle-parfum comme Moïse le lui avait dit et il courut vers la communauté où, en effet, le fléau avait commencé parmi le peuple. Il mit l’encens et fit le rite d’expiation sur le peuple. 
${}^{13}Il se tint entre les morts et les vivants, et le fléau s’arrêta. 
${}^{14}14 700 moururent du fléau, sans compter les morts dans l’affaire de Coré. 
${}^{15}Aaron retourna vers Moïse à l’entrée de la tente de la Rencontre. Le fléau s’était arrêté.
${}^{16}Le Seigneur parla à Moïse. Il dit : 
${}^{17}« Parle aux fils d’Israël et fais-toi remettre par eux une branche pour chaque tribu, remise par tous les responsables de tribu : cela fera douze branches. Tu écriras le nom de chacun sur sa branche. 
${}^{18}Mais sur la branche de Lévi tu écriras le nom d’Aaron car il n’y aura qu’une seule branche par chef de tribu. 
${}^{19}Tu déposeras les branches dans la tente de la Rencontre, devant le Témoignage, là où je vous rencontre. 
${}^{20}L’homme que j’aurai choisi sera celui dont la branche fleurira : ainsi, j’apaiserai, en les écartant de moi, les récriminations que les fils d’Israël élèvent contre vous. »
${}^{21}Moïse parla aux fils d’Israël et chacun de leurs responsables lui donna une branche, une branche par responsable selon leur tribu, soit douze branches ; et la branche d’Aaron se trouvait au milieu de leurs branches. 
${}^{22}Puis Moïse déposa les branches devant le Seigneur dans la tente du Témoignage. 
${}^{23}Le lendemain, Moïse entra dans la tente du Témoignage, et voici : la branche d’Aaron avait fleuri pour la maison de Lévi ; la branche avait fait éclore une floraison, fleurir des fleurs et mûrir des amandes ! 
${}^{24}Moïse enleva toutes les branches qui étaient devant le Seigneur pour les montrer aux fils d’Israël : ils les virent et chacun reprit sa branche. 
${}^{25}Alors le Seigneur dit à Moïse : « Remets la branche d’Aaron devant le Témoignage, garde-la comme un signe pour les fils rebelles : ainsi tu mettras fin à leurs récriminations contre moi et ils ne mourront pas. » 
${}^{26}Moïse fit comme le Seigneur le lui avait ordonné. Il fit ainsi.
${}^{27}Les fils d’Israël dirent à Moïse : « Vois ! Nous expirons, nous sommes perdus ; tous, nous sommes perdus ! 
${}^{28}Car : “Quiconque s’approche de la demeure du Seigneur, celui-là mourra”. Allons-nous expirer jusqu’au dernier ? »
      
         
      \bchapter{}
      \begin{verse}
${}^{1}Le Seigneur dit à Aaron : « Toi, tes fils et toute ta tribu, vous porterez le poids des fautes commises envers le sanctuaire ; mais seuls, toi et tes fils avec toi, vous porterez le poids des fautes commises dans l’exercice de votre sacerdoce. 
${}^{2}Ainsi donc, tes frères, la branche de Lévi, tribu de ton père, fais-les approcher avec toi ; ils seront tes adjoints et t’assisteront, tandis que toi et tes fils, vous vous tiendrez devant la tente du Témoignage. 
${}^{3}Ils assureront ta charge, et la charge de la Tente tout entière, mais ils n’approcheront pas des objets du sanctuaire ni de l’autel, de peur qu’ils ne meurent, eux et vous avec eux. 
${}^{4}Ils seront tes adjoints dans la charge de la tente de la Rencontre, pour tout le service de la Tente ; aucun profane ne s’approchera de vous. 
${}^{5}Mais vous, vous assurerez la charge du sanctuaire et celle de l’autel ; ainsi il n’y aura plus de colère contre les fils d’Israël. 
${}^{6}Voici que moi, j’ai pris vos frères, les Lévites, parmi les fils d’Israël, je vous en fais don, eux qui sont donnés au Seigneur pour accomplir le service de la tente de la Rencontre. 
${}^{7}Toi et tes fils avec toi, vous exercerez le sacerdoce pour tout ce qui concerne l’autel et ce qui est derrière le rideau ; ce sera votre service. Je vous donne le sacerdoce comme on donne un service. Le profane qui s’approchera sera mis à mort. »
${}^{8}Le Seigneur dit à Aaron : « Et voici que moi, je te confie la charge de prélever pour moi les parts sur toutes les choses saintes des fils d’Israël : je te les ai données, du fait de l’onction, à toi et à tes fils ; c’est un décret perpétuel. 
${}^{9}Parmi les choses très saintes qui ne sont pas destinées au feu, voici ce qui sera pour toi : tous les présents réservés que me rapportent les fils d’Israël, à savoir l’offrande de céréales, le sacrifice pour la faute et le sacrifice de réparation ; ce sont là, pour toi et tes fils, les plus saintes des choses saintes. 
${}^{10}Les plus saintes des choses saintes, tu les mangeras ; toute personne de sexe masculin en mangera ; ce sera pour toi une chose sainte.
${}^{11}Tu recevras aussi les parts prélevées sur leurs dons, sur tout ce que présentent les fils d’Israël avec le geste d’élévation. Je te les donne, à toi ainsi qu’à tes fils et tes filles ; c’est un décret perpétuel ; quiconque, dans ta maison, sera en état de pureté en mangera. 
${}^{12}Le meilleur de l’huile fraîche, le meilleur du vin nouveau et du froment, ces prémices qu’ils donneront au Seigneur, je te les donne. 
${}^{13}Les premiers fruits de la terre qu’ils apporteront au Seigneur seront pour toi : tous ceux qui, dans ta maison, seront en état de pureté en mangeront. 
${}^{14}Tout ce qui, en Israël, est voué à l’anathème sera pour toi.
${}^{15}Le premier-né de tout être de chair, homme ou bétail, qu’on apportera au Seigneur, sera pour toi ; mais tu devras racheter les premiers-nés des hommes et tu rachèteras les premiers-nés des animaux impurs. 
${}^{16}Tu les rachèteras lorsqu’ils auront un mois, au prix qu’ils valent, c’est-à-dire cinq sicles d’argent, en sicle du sanctuaire – soit vingt guéras par sicle. 
${}^{17}Mais les premiers-nés de la vache, de la brebis ou de la chèvre, tu ne les rachèteras pas : ils sont une chose sainte. De leur sang, tu aspergeras l’autel, et tu feras fumer leur graisse : c’est une nourriture offerte, en agréable odeur pour le Seigneur. 
${}^{18}Leur viande sera pour toi, ainsi que la poitrine présentée avec le geste d’élévation et la cuisse droite.
${}^{19}Tout ce que les fils d’Israël prélèveront pour le Seigneur sur les choses saintes, je te le donne, à toi ainsi qu’à tes fils et à tes filles ; c’est un décret perpétuel, une alliance perpétuelle conclue avec le rite du sel devant le Seigneur, pour toi et ta descendance. »
${}^{20}Le Seigneur dit à Aaron : « Tu n’auras pas d’héritage sur la terre des fils d’Israël, tu n’auras aucun lot au milieu d’eux : c’est moi qui serai ton lot et ton héritage au milieu d’eux !
${}^{21}Voici qu’aux fils de Lévi, je donne en héritage toute la dîme perçue en Israël en échange du service qu’ils rendront, le service de la tente de la Rencontre. 
${}^{22}Ainsi les fils d’Israël n’auront plus à s’approcher de la tente de la Rencontre, ce qui les chargerait d’une faute passible de mort. 
${}^{23}Ce sont les Lévites qui accompliront le service de la tente de la Rencontre et porteront le poids de leur péché : ce sera pour vos descendants un décret perpétuel. Les Lévites ne recevront pas d’héritage au milieu des fils d’Israël. 
${}^{24}En effet, c’est la dîme prélevée par les fils d’Israël pour le Seigneur que je donne en héritage aux Lévites. C’est pourquoi je leur dis qu’ils n’auront pas d’héritage au milieu des fils d’Israël. »
${}^{25}Le Seigneur parla à Moïse. Il dit : 
${}^{26}« Tu parleras aux Lévites. Tu diras : Lorsque vous recevrez des fils d’Israël la dîme que je vous donne comme part d’héritage, vous en prélèverez une partie pour le Seigneur : la dîme de la dîme. 
${}^{27}La part que vous prélèverez sera considérée comme l’équivalent du froment ramassé sur l’aire ou de la coulée du pressoir. 
${}^{28}Ainsi, vous aussi, vous prélèverez la part du Seigneur sur toutes les dîmes que vous recevrez des fils d’Israël. Là-dessus, vous donnerez la part du Seigneur au prêtre Aaron. 
${}^{29}Sur tout ce qui vous sera donné vous prélèverez toute la part du Seigneur ; c’est sur tout ce qu’il y a de meilleur que vous retiendrez la part sanctifiée.
${}^{30}Tu leur diras encore ceci : Ce que vous aurez prélevé de meilleur aura la même valeur pour vous, les Lévites, que le produit de l’aire et du pressoir. 
${}^{31}Vous en mangerez n’importe où, vous et vos familles, car c’est votre salaire en échange de votre service dans la tente de la Rencontre. 
${}^{32}En ce cas, vous ne porterez le poids d’aucune faute, quand vous prélèverez le meilleur ; vous ne profanerez pas les choses saintes des fils d’Israël, et ainsi vous ne mourrez pas. »
      
         
      \bchapter{}
      \begin{verse}
${}^{1}Le Seigneur parla à Moïse et Aaron. Il dit : 
${}^{2}« Voici la disposition de la loi que le Seigneur a prescrite. Parle aux fils d’Israël : Qu’ils te procurent une vache rousse sans défaut, sans tare, qui n’a jamais porté le joug. 
${}^{3}Vous la donnerez au prêtre Éléazar ; il la fera sortir hors du camp et on l’égorgera devant lui. 
${}^{4}Le prêtre Éléazar prendra du sang de la vache avec son doigt et, de ce sang, il fera sept fois l’aspersion vers l’entrée de la tente de la Rencontre. 
${}^{5}Puis on brûlera la vache sous ses yeux ; la peau, la chair, le sang et même la bouse, on les brûlera. 
${}^{6}Ensuite le prêtre prendra du bois de cèdre, de l’hysope et du cramoisi éclatant et les jettera au milieu du brasier où se consume la vache. 
${}^{7}Puis le prêtre lavera ses vêtements et baignera son corps dans l’eau ; après quoi il rentrera au camp. Le prêtre restera impur jusqu’au soir. 
${}^{8}Celui qui a brûlé la vache lavera également ses vêtements dans l’eau et baignera son corps dans l’eau. Il restera impur jusqu’au soir. 
${}^{9}Un homme en état de pureté recueillera les cendres de la vache et les déposera hors du camp dans un lieu pur. La communauté des fils d’Israël les gardera pour la préparation de l’eau lustrale. C’est un sacrifice pour la faute. 
${}^{10}Puis, celui qui a recueilli les cendres de la vache lavera ses vêtements ; il restera impur jusqu’au soir. Ce sera un décret perpétuel pour les fils d’Israël et pour l’immigré résidant parmi eux.
      
         
${}^{11}« Celui qui touche un mort – n’importe quel corps humain – restera impur pendant sept jours. 
${}^{12}Il se purifiera avec l’eau lustrale le troisième jour et sera en état de pureté le septième jour. Mais s’il ne se purifie pas le troisième jour, il ne sera pas en état de pureté le septième jour. 
${}^{13}Quiconque touche un mort – un être humain qui est mort – et ne se purifie pas rend impure la demeure du Seigneur. Celui-là sera donc retranché d’Israël. Puisque l’eau lustrale n’a pas été répandue sur lui, il est impur, il reste en état d’impureté.
${}^{14}Voici la loi : quand un homme meurt dans une tente, quiconque entre dans la tente restera impur pendant sept jours, ainsi que tout ce qui se trouve dans la tente. 
${}^{15}Tout récipient ouvert, non fermé par un couvercle, sera impur.
${}^{16}Celui qui, dans un champ, touche une victime de l’épée, ou un mort de mort naturelle ou des ossements humains ou encore un tombeau, celui-là restera impur pendant sept jours. 
${}^{17}Pour cet homme impur, on prendra de la cendre provenant du brasier du sacrifice pour la faute et on la mettra dans un récipient en y ajoutant de l’eau vive. 
${}^{18}Puis, un homme en état de pureté prendra un rameau d’hysope, le plongera dans cette eau et aspergera la tente, ainsi que tous les objets et les personnes qui s’y trouvent. On fera de même pour celui qui aura touché des ossements, ou une victime de l’épée, ou un mort de mort naturelle, ou encore un tombeau. 
${}^{19}Le troisième et le septième jour, celui qui est pur aspergera l’impur ; et lorsqu’il l’aura purifié le septième jour, il lavera ses vêtements, se baignera dans l’eau et, le soir, il sera pur. 
${}^{20}Mais l’homme qui est impur et ne se sera pas purifié de sa faute, celui-là sera retranché de l’assemblée, car il aura rendu impur le sanctuaire du Seigneur ; l’eau lustrale n’ayant pas été répandue sur lui, il reste en état d’impureté.
${}^{21}Ce sera pour eux un décret perpétuel. Celui qui aura aspergé avec l’eau lustrale lavera ses vêtements ; celui qui aura touché l’eau lustrale restera impur jusqu’au soir. 
${}^{22}Tout ce que touchera l’impur sera impur, et la personne qui ensuite y touchera restera impure jusqu’au soir. »
      
         
      \bchapter{}
      \begin{verse}
${}^{1}Le premier mois de l’année\\, toute la communauté des fils d’Israël arriva dans le désert de Cine. Le peuple s’établit à Cadès. C’est là que Miryam mourut et qu’elle fut enterrée. 
${}^{2} Comme il n’y avait pas d’eau pour la communauté, ils se rassemblèrent\\contre Moïse et Aaron. 
${}^{3} Le peuple chercha querelle à Moïse\\, en disant : « Ah ! si seulement nous avions expiré, quand nos frères ont expiré devant le Seigneur ! 
${}^{4} Pourquoi avoir amené l’assemblée du Seigneur dans ce désert où nous allons mourir, nous et nos bêtes ? 
${}^{5} Pourquoi nous avoir fait monter d’Égypte, et nous avoir amenés dans ce lieu de malheur où l’on ne peut rien semer, où il n’y a ni figuiers, ni vignes, ni grenadiers, et même pas d’eau à boire ! »
${}^{6}Moïse et Aaron quittèrent l’assemblée et se rendirent à l’entrée de la tente de la Rencontre. Ils tombèrent face contre terre, et la gloire du Seigneur leur apparut. 
${}^{7} Le Seigneur parla à Moïse. Il dit : 
${}^{8} « Prends ton bâton de chef\\et, avec ton frère Aaron, rassemble la communauté. Puis, sous leurs yeux, vous parlerez au rocher, et il donnera son eau. Pour eux tu feras jaillir l’eau du rocher, et tu feras boire la communauté et ses bêtes. »
${}^{9}Comme il en avait reçu l’ordre, Moïse prit le bâton qui était placé devant le Seigneur. 
${}^{10} Moïse et Aaron réunirent l’assemblée en face du rocher, et Moïse leur dit : « Écoutez donc, rebelles. Est-ce que nous pouvons faire jaillir de l’eau pour vous de ce rocher ? » 
${}^{11} Moïse leva la main et, de son bâton, il frappa le rocher par deux fois : l’eau jaillit en abondance, et la communauté put boire et abreuver ses bêtes.
${}^{12}Le Seigneur dit alors à Moïse et à son frère\\Aaron : « Puisque vous n’avez pas eu assez de foi pour manifester ma sainteté devant les fils d’Israël, vous ne ferez pas entrer cette assemblée dans le pays que je lui donne. »
${}^{13}Ce sont les eaux de Mériba (c’est-à-dire : les eaux du Défi)\\ où les fils d’Israël ont défié le Seigneur, et où le Seigneur a manifesté parmi eux sa sainteté.
      <h2 class="intertitle" id="d85e38072">3. De Cadès aux steppes de Moab (20,14 – 21)</h2>
${}^{14}De Cadès, Moïse envoya des messagers au roi d’Édom pour lui dire : « Ainsi parle ton frère Israël. Tu sais toutes les difficultés que nous avons rencontrées. 
${}^{15}Nos pères sont descendus en Égypte et nous y avons habité longtemps. Puis les Égyptiens nous ont maltraités, nous et nos pères. 
${}^{16}Alors nous avons crié vers le Seigneur, et il a entendu notre voix : il a envoyé un ange et nous a fait sortir d’Égypte. Nous voici donc à Cadès, ville à la frontière de ton territoire. 
${}^{17}Laisse-nous, je te prie, traverser ton pays ! Nous ne passerons ni dans les champs ni dans les vignes ; nous ne boirons pas l’eau des puits ; nous irons par la voie Royale sans dévier ni à droite ni à gauche jusqu’à ce que nous ayons traversé ton territoire. » 
${}^{18}Mais Édom répondit : « Tu ne passeras pas chez moi, sinon je sortirai à ta rencontre avec l’épée ! » 
${}^{19}Les fils d’Israël lui dirent : « Nous monterons par la route et si, moi et mes troupeaux, nous buvons de ton eau, je te la paierai. Après tout, ce n’est pas une affaire de me laisser passer à pied. » 
${}^{20}Mais Édom dit : « Non ! Tu ne passeras pas ! » Puis il sortit à la rencontre d’Israël avec une masse de gens et en grande force. 
${}^{21}Ainsi Édom refusa-t-il à Israël la permission de traverser son territoire, et Israël prit une autre direction.
${}^{22}Les fils d’Israël – toute la communauté – partirent de Cadès et arrivèrent à Hor-la-Montagne. 
${}^{23}Le Seigneur s’adressa à Moïse et Aaron à Hor-la-Montagne, sur la frontière du pays d’Édom, et il dit : 
${}^{24}« Aaron va être réuni aux siens ; en effet, il n’entrera pas dans le pays que je donne aux fils d’Israël, puisque vous avez été rebelles à ma parole aux eaux de Mériba. 
${}^{25}Toi, Moïse, prends avec toi Aaron et son fils Éléazar, et fais-les monter à Hor-la-Montagne. 
${}^{26}Enlève à Aaron ses vêtements, tu en vêtiras son fils Éléazar. C’est là qu’Aaron sera réuni aux siens, qu’il mourra. »
${}^{27}Moïse fit comme le Seigneur l’avait ordonné : ils montèrent à Hor-la-Montagne aux yeux de toute la communauté. 
${}^{28}Moïse enleva ses vêtements à Aaron et en vêtit son fils Éléazar. Aaron mourut là, au sommet de la montagne. Puis Moïse, accompagné d’Éléazar, descendit de la montagne. 
${}^{29}Toute la communauté constata qu’Aaron avait expiré, et toute la maison d’Israël le pleura pendant trente jours.
      
         
      \bchapter{}
      \begin{verse}
${}^{1}Le roi d’Arad, un Cananéen habitant le Néguev, apprit qu’Israël arrivait par le chemin des Atarim ; il combattit contre Israël et il lui fit des prisonniers. 
${}^{2}Alors Israël fit ce vœu au Seigneur : « Si tu consens à livrer ce peuple entre mes mains, je vouerai ses villes à l’anathème. » 
${}^{3}Le Seigneur écouta la voix d’Israël et lui livra les Cananéens. On les voua à l’anathème, eux et leurs villes, et l’on donna à cet endroit le nom de Horma (c’est-à-dire : Voué-à-l’anathème).
      
         
${}^{4}Ils quittèrent Hor-la-Montagne par la route de la mer des Roseaux en contournant le pays d’Édom. <a class="anchor verset_lettre" id="bib_nb_21_4_b"/>Mais en chemin, le peuple perdit courage. 
${}^{5}Il récrimina contre Dieu et contre Moïse : « Pourquoi nous avoir fait monter d’Égypte ? Était-ce pour nous faire mourir dans le désert, où il n’y a ni pain ni eau ? Nous sommes dégoûtés de cette nourriture misérable ! »
${}^{6}Alors le Seigneur envoya contre le peuple des serpents à la morsure brûlante\\, et beaucoup en moururent dans le peuple d’Israël. 
${}^{7} Le peuple vint vers Moïse et dit : « Nous avons péché, en récriminant contre le Seigneur et contre toi. Intercède auprès du Seigneur pour qu’il éloigne de nous les serpents. » Moïse intercéda pour le peuple, 
${}^{8} et le Seigneur dit à Moïse : « Fais-toi un serpent brûlant\\, et dresse-le au sommet d’un mât : tous ceux qui auront été mordus, qu’ils le regardent, alors ils vivront ! » 
${}^{9} Moïse fit un serpent de bronze et le dressa au sommet du mât. Quand un homme était mordu par un serpent, et qu’il regardait vers le serpent de bronze, il restait en vie !
${}^{10}Les fils d’Israël partirent et campèrent à Oboth. 
${}^{11}Puis ils partirent d’Oboth et campèrent aux Ruines-des-Abarim dans le désert qui est en face de Moab, du côté du soleil levant. 
${}^{12}Ils partirent de là et campèrent au torrent de Zèred. 
${}^{13}Ils partirent de là et campèrent au-delà de l’Arnon qui, dans le désert, sort du territoire des Amorites. L’Arnon, en effet, marque la frontière de Moab, entre Moab et les Amorites.
${}^{14}C’est pourquoi on mentionne dans le livre des Guerres du Seigneur « Wahéb en Soufa, les torrents de l’Arnon, 
${}^{15}et la pente de ces torrents qui est inclinée vers le site d’Ar et longe la frontière de Moab. »
${}^{16}De là ils allèrent à Beér (c’est-à-dire : le Puits). À propos de ce puits, le Seigneur avait dit à Moïse : « Rassemble le peuple et je leur donnerai de l’eau ! »
${}^{17}Alors Israël chanta cette chanson :
        « Sur le puits.
        Acclamez-le,
${}^{18}ce puits que les princes ont creusé,
        que les nobles du peuple ont foré
        avec leur sceptre, avec leurs cannes ! »
      Du désert, ils allèrent à Mattanah ; 
${}^{19}de Mattanah à Nahaliël et de Nahaliël à Bamoth. 
${}^{20}Et de Bamoth au Vallon qui se trouve aux Champs-de-Moab, vers le sommet du Pisga qui fait face à la steppe et la domine.
${}^{21}Alors Israël envoya des messagers à Séhone, le roi des Amorites, pour lui dire : 
${}^{22}« Laisse-moi traverser ton pays ! Nous ne ferons pas de détour par les champs ni par les vignes, nous ne boirons pas l’eau des puits, nous irons par la voie Royale jusqu’à ce que nous ayons traversé ton territoire. » 
${}^{23}Mais Séhone ne permit pas à Israël de traverser son territoire. Il rassembla tout son peuple et sortit à la rencontre d’Israël en direction du désert ; il arriva à Yahaç et combattit contre Israël. 
${}^{24}Israël le passa au fil de l’épée et prit possession de son pays, de l’Arnon jusqu’au Yabboq, jusqu’à la frontière des fils d’Ammone, car la frontière des fils d’Ammone était fortifiée. 
${}^{25}Israël prit toutes ces villes, il s’installa dans toutes les villes des Amorites, à Heshbone et dans tout ce qui en dépendait. 
${}^{26}En effet, Heshbone était la ville de Séhone, le roi des Amorites qui avait fait la guerre au précédent roi de Moab et avait conquis de sa main tout le pays jusqu’à l’Arnon.
${}^{27}C’est pourquoi les poètes disent :
        « Venez à Heshbone, qu’elle soit rebâtie !
        Qu’elle soit restaurée, la ville de Séhone !
${}^{28}Car un feu est sorti de Heshbone,
        une flamme de la cité de Séhone :
        elle a dévoré Ar-Moab,
        les Baals des lieux sacrés de l’Arnon !
${}^{29}Malheur à toi, Moab !
        Tu es perdu, peuple de Camosh !
        On fait de tes fils, des fugitifs,
        on fait de tes filles, des captives
        pour Séhone, roi des Amorites !
${}^{30}Nous les avons percés de flèches,
        tout a péri de Heshbone jusqu’à Dibone ;
        nous avons tout dévasté jusqu’à Nofah,
        et jusqu’à Madaba. »
${}^{31}Israël s’installa donc dans le pays des Amorites. 
${}^{32}Moïse envoya des espions à Yazèr. Les fils d’Israël s’emparèrent des villes qui en dépendaient et chassèrent les Amorites qui s’y trouvaient.
${}^{33}Puis ils firent demi-tour et montèrent par la route du Bashane. Og, le roi de Bashane, sortit à leur rencontre avec tout son peuple pour livrer bataille, à Édréï. 
${}^{34}Mais le Seigneur dit à Moïse : « Ne le crains pas ! Je l’ai livré entre tes mains avec tout son peuple et tout son pays : tu le traiteras comme tu as traité Séhone, le roi des Amorites qui siégeait à Heshbone. » 
${}^{35}Ils le battirent donc, ainsi que ses fils et tout son peuple, au point de ne laisser aucun survivant, et ils conquirent le pays.
