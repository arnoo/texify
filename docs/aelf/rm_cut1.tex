  
  
    
    \bbook{LETTRE AUX ROMAINS}{LETTRE AUX ROMAINS}
      
         
      \bchapter{}
        ${}^{1}Paul, serviteur du Christ Jésus,
        appelé à être Apôtre,
        mis à part pour l’Évangile de Dieu,
        \\à tous les bien-aimés de Dieu qui sont à Rome.
        ${}^{2}Cet Évangile, que Dieu avait promis d’avance
        par ses prophètes dans les saintes Écritures,
        ${}^{3}concerne son Fils qui, selon la chair,
        est né de la descendance de David
        ${}^{4}et, selon l’Esprit de sainteté,
        a été établi dans sa puissance de Fils de Dieu
        \\par sa résurrection d’entre les morts,
        lui, Jésus Christ, notre Seigneur.
        ${}^{5}Pour que son nom soit reconnu,
        nous avons reçu par lui grâce et mission d’Apôtre,
        \\afin d’amener à l’obéissance de la foi
        toutes les nations païennes,
        ${}^{6}dont vous faites partie,
        vous aussi que Jésus Christ a appelés.
        ${}^{7}À vous qui êtes appelés à être saints,
        la grâce et la paix
        \\de la part de Dieu notre Père
        et du Seigneur Jésus Christ.
        
           
${}^{8}Tout d’abord, je rends grâce à mon Dieu par Jésus Christ pour vous tous, puisque la nouvelle de votre foi se répand dans le monde entier. 
${}^{9}Car Dieu m’en est témoin, lui à qui je rends un culte spirituel en annonçant l’Évangile de son Fils : je fais sans cesse mémoire de vous, 
${}^{10}lorsqu’à tout moment, dans mes prières, je demande que, par la volonté de Dieu, l’occasion me soit bientôt donnée de venir enfin chez vous. 
${}^{11}J’ai en effet un très vif désir de vous voir, pour vous communiquer l’un ou l’autre don de l’Esprit, afin que vous en soyez fortifiés, – 
${}^{12}je veux dire, afin que nous soyons réconfortés ensemble chez vous, par la foi que nous avons en commun, vous et moi. 
${}^{13}Je ne veux pas vous le laisser ignorer, frères : j’ai bien souvent eu l’intention de venir chez vous, et j’en ai été empêché jusqu’à maintenant ; je pensais obtenir chez vous quelque fruit comme chez les autres nations païennes. 
${}^{14}J’ai des devoirs envers tous : Grecs et non-Grecs, savants et ignorants ; 
${}^{15}de là cet élan qui me pousse à vous annoncer l’Évangile à vous aussi qui êtes à Rome.
${}^{16}En effet, je n’ai pas honte de l’Évangile, car il est puissance de Dieu pour le salut de quiconque est devenu croyant, le Juif d’abord, et le païen. 
${}^{17}Dans cet Évangile se révèle la justice donnée par Dieu, celle qui vient de la foi et conduit à la foi, comme il est écrit : Celui qui est juste par la foi, vivra.
      <h2 class="intertitle" id="d85e375183">1. Les hommes, tous pécheurs, rendus justes par la foi (1,18 – 4)</h2>
${}^{18}Or la colère de Dieu se révèle du haut du ciel contre toute impiété et contre toute injustice des hommes qui, par leur injustice, font obstacle à la vérité. 
${}^{19}En effet, ce que l’on peut connaître de Dieu est clair pour eux, car Dieu le leur a montré clairement. 
${}^{20}Depuis la création du monde, on peut voir avec l’intelligence, à travers les œuvres de Dieu, ce qui de lui est invisible : sa puissance éternelle et sa divinité. Ils n’ont donc pas d’excuse, 
${}^{21}puisque, malgré leur connaissance de Dieu, ils ne lui ont pas rendu la gloire et l’action de grâce que l’on doit à Dieu. Ils se sont laissé aller à des raisonnements sans valeur, et les ténèbres ont rempli leurs cœurs privés d’intelligence. 
${}^{22}Ces soi-disant sages sont devenus fous ; 
${}^{23}ils ont échangé la gloire du Dieu impérissable contre des idoles représentant l’être humain périssable ou bien des volatiles, des quadrupèdes et des reptiles.
${}^{24}Voilà pourquoi, à cause des convoitises de leurs cœurs, Dieu les a livrés à l’impureté, de sorte qu’ils déshonorent eux-mêmes leur corps. 
${}^{25}Ils ont échangé la vérité de Dieu contre le mensonge ; ils ont vénéré la création et lui ont rendu un culte plutôt qu’à son Créateur, lui qui est béni éternellement. Amen.
${}^{26}C’est pourquoi Dieu les a livrés à des passions déshonorantes. Chez eux, les femmes ont échangé les rapports naturels pour des rapports contre nature. 
${}^{27}De même, les hommes ont abandonné les rapports naturels avec les femmes pour brûler de désir les uns pour les autres ; les hommes font avec les hommes des choses infâmes, et ils reçoivent en retour dans leur propre personne le salaire dû à leur égarement. 
${}^{28}Et comme ils n’ont pas jugé bon de garder la vraie connaissance de Dieu, Dieu les a livrés à une façon de penser dépourvue de jugement. Ils font ce qui est inconvenant ; 
${}^{29}ils sont remplis de toutes sortes d’injustice, de perversité, de soif de posséder, de méchanceté, ne respirant que jalousie, meurtre, rivalité, ruse, dépravation ; ils sont détracteurs, 
${}^{30}médisants, ennemis de Dieu, insolents, orgueilleux, fanfarons, ingénieux à faire le mal, révoltés contre leurs parents ; 
${}^{31}ils sont sans intelligence, sans loyauté, sans affection, sans pitié. 
${}^{32}Ils savent bien que, d’après le juste décret de Dieu, ceux qui font de telles choses méritent la mort ; et eux, non seulement ils les font, mais encore ils approuvent ceux qui les font.
      
         
      \bchapter{}
      \begin{verse}
${}^{1}De même, toi, l’homme qui juge, tu n’as aucune excuse, qui que tu sois : quand tu juges les autres, tu te condamnes toi-même car tu fais comme eux, toi qui juges. 
${}^{2}Or, nous savons que Dieu juge selon la vérité ceux qui font de telles choses. 
${}^{3}Et toi, l’homme qui juge ceux qui font de telles choses et les fais toi-même, penses-tu échapper au jugement de Dieu ? 
${}^{4}Ou bien méprises-tu ses trésors de bonté, de longanimité et de patience, en refusant de reconnaître que cette bonté de Dieu te pousse à la conversion ? 
${}^{5}Avec ton cœur endurci, qui ne veut pas se convertir, tu accumules la colère contre toi pour ce jour de colère, où sera révélé le juste jugement de Dieu, 
${}^{6}lui qui rendra à chacun selon ses œuvres. 
${}^{7}Ceux qui font le bien avec persévérance et recherchent ainsi la gloire, l’honneur et une existence impérissable, recevront la vie éternelle ; 
${}^{8}mais les intrigants, qui se refusent à la vérité pour se donner à l’injustice, subiront la colère et la fureur. 
${}^{9}Oui, détresse et angoisse pour tout homme qui commet le mal, le Juif d’abord, et le païen. 
${}^{10}Mais gloire, honneur et paix pour quiconque fait le bien, le Juif d’abord, et le païen. 
${}^{11}Car Dieu est impartial.
${}^{12}En effet, tous ceux qui ont péché sans la loi de Moïse périront aussi sans la Loi ; et tous ceux qui ont péché en ayant la Loi seront jugés au moyen de la Loi. 
${}^{13}Car ce n’est pas ceux qui écoutent la Loi qui sont justes devant Dieu, mais ceux qui pratiquent la Loi, ceux-là seront justifiés. 
${}^{14}Quand des païens qui n’ont pas la Loi pratiquent spontanément ce que prescrit la Loi, eux qui n’ont pas la Loi sont à eux-mêmes leur propre loi. 
${}^{15}Ils montrent ainsi que la façon d’agir prescrite par la Loi est inscrite dans leur cœur, et leur conscience en témoigne, ainsi que les arguments par lesquels ils se condamnent ou s’approuvent les uns les autres. 
${}^{16}Cela apparaîtra le jour où ce qui est caché dans les hommes sera jugé par Dieu conformément à l’Évangile que j’annonce par le Christ Jésus.
${}^{17}Mais toi qui portes le nom de Juif, qui te reposes sur la Loi, qui mets ta fierté en Dieu, 
${}^{18}toi qui connais sa volonté et qui discernes l’essentiel parce que tu es à l’école de la Loi, 
${}^{19}toi qui es convaincu d’être toi-même guide des aveugles, lumière de ceux qui sont dans les ténèbres, 
${}^{20}éducateur des insensés, maître des tout-petits, toi qui es convaincu de posséder dans la Loi l’expression même de la connaissance et de la vérité, 
${}^{21}bref, toi qui instruis les autres, tu ne t’instruis pas toi-même ! toi qui proclames qu’il ne faut pas voler, tu voles ! 
${}^{22}toi qui dis de ne pas commettre l’adultère, tu le commets ! toi qui as horreur des idoles, tu pilles leurs temples ! 
${}^{23}toi qui mets ta fierté dans la Loi, tu déshonores Dieu en transgressant la Loi, 
${}^{24}car, comme le dit l’Écriture, à cause de vous, le nom de Dieu est bafoué parmi les nations.
${}^{25}Sans doute, la circoncision est utile si tu pratiques la Loi ; mais si tu transgresses la Loi, malgré ta circoncision tu es devenu non-circoncis. 
${}^{26}À l’inverse si le non-circoncis garde les préceptes de la Loi, ne sera-t-il pas considéré comme s’il était circoncis ? 
${}^{27}Celui qui n’est pas circoncis dans son corps mais qui accomplit la Loi te jugera, toi qui transgresses la Loi tout en ayant la lettre de la Loi et la circoncision. 
${}^{28}Ce n’est pas ce qui est visible qui fait le Juif, ce n’est pas la marque visible dans la chair qui fait la circoncision ; 
${}^{29}mais c’est ce qui est caché qui fait le Juif : sa circoncision est celle du cœur, selon l’Esprit et non selon la lettre, et sa louange ne vient pas des hommes, mais de Dieu.
      
         
      \bchapter{}
      \begin{verse}
${}^{1}Mais alors, quelle est la supériorité du Juif ? Quelle est l’utilité de la circoncision ? 
${}^{2}Grande, à tous égards ! D’abord, parce que c’est aux Juifs que les paroles de Dieu ont été confiées. 
${}^{3}Mais alors, qu’en est-il ? Si certains ont refusé de croire, leur manque de foi va-t-il donc empêcher Dieu d’être digne de foi ? 
${}^{4}Pas du tout ! Qu’il soit bien entendu que Dieu est véridique et tout homme menteur, comme il est écrit :
        \\Ainsi, on te reconnaît juste lorsque tu parles
        \\et tu triomphes lorsqu’on te met en jugement.
${}^{5}Et si notre injustice manifeste la justice de Dieu, que dirons-nous ? Que Dieu est injuste quand il donne libre cours à sa colère ? – Et là, je parle de manière humaine. 
${}^{6}Mais pas du tout ! Sinon, comment Dieu jugera-t-il le monde ? 
${}^{7}Et si la vérité de Dieu éclate pour sa gloire grâce à mon mensonge, pourquoi suis-je encore condamné comme pécheur ? 
${}^{8}Faut-il dire : « Faisons le mal pour qu’il en sorte du bien », comme certains nous accusent injurieusement de le dire ? Ceux-là méritent leur condamnation.
${}^{9}Alors ? Avons-nous une supériorité ? Pas entièrement ! Nous avons déjà montré que tous, Juifs et païens, sont sous la domination du péché. 
${}^{10}Voici en effet ce qui est écrit :
        \\Il n’y a pas un juste, pas même un seul,
${}^{11}il n’y a personne de sensé,
        \\personne qui cherche Dieu ;
${}^{12}Tous, ils sont dévoyés ; tous ensemble, pervertis :
        \\pas un homme de bien, pas même un seul.
${}^{13}Leur gosier est un sépulcre béant,
        \\et leur langue sert à tromper.
        \\Leurs lèvres sont chargées d’un venin de vipère.
${}^{14}Leur bouche déborde d’imprécations et d’amertume.
${}^{15}Leurs pieds sont rapides quand ils vont répandre le sang.
${}^{16}Sur leurs chemins, ruine et misère ;
${}^{17}ils ne connaissent pas le chemin de la paix.
${}^{18}Leurs yeux ne voient pas
        \\qu’il faut craindre Dieu.
${}^{19}Or nous le savons : tout ce que dit la Loi, elle le déclare pour ceux qui sont sujets de la Loi, afin que toute bouche soit fermée, et que le monde entier soit soumis au jugement de Dieu. 
${}^{20}Ainsi, par la pratique de la Loi, personne ne deviendra juste devant Dieu. En effet, la Loi fait seulement connaître le péché.
${}^{21}Mais aujourd’hui, indépendamment de la Loi, Dieu a manifesté en quoi consiste sa justice : la Loi et les prophètes en sont témoins. 
${}^{22}Et cette justice de Dieu, donnée par la foi en Jésus Christ, elle est offerte à tous ceux qui croient. En effet, il n’y a pas de différence : 
${}^{23}tous les hommes ont péché, ils sont privés de la gloire de Dieu, 
${}^{24}et lui, gratuitement, les fait devenir justes par sa grâce, en vertu de la rédemption accomplie dans le Christ Jésus. 
${}^{25}Car le projet de Dieu était que le Christ soit instrument de pardon, en son sang, par le moyen de la foi. C’est ainsi que Dieu voulait manifester sa justice, lui qui, dans sa longanimité, avait fermé les yeux sur les péchés commis autrefois. 
${}^{26}Il voulait manifester, au temps présent, en quoi consiste sa justice, montrer qu’il est juste et rend juste celui qui a foi en Jésus.
${}^{27}Alors, y a-t-il de quoi s'enorgueillir ? Absolument pas. Par quelle loi ? Par celle des œuvres que l’on pratique ? Pas du tout. Mais par la loi de la foi. 
${}^{28}En effet, nous estimons que l’homme devient juste par la foi, indépendamment de la pratique de la loi de Moïse. 
${}^{29}Ou bien, Dieu serait-il seulement le Dieu des Juifs ? N’est-il pas aussi le Dieu des nations ? Bien sûr, il est aussi le Dieu des nations, 
${}^{30}puisqu’il n’y a qu’un seul Dieu : il rendra justes en vertu de la foi ceux qui ont reçu la circoncision, et aussi, au moyen de la foi, ceux qui ne l’ont pas reçue. 
${}^{31}Sommes-nous en train d’abolir la Loi au moyen de la foi ? Pas du tout ! Au contraire, nous confirmons la Loi.
      
         
      \bchapter{}
      \begin{verse}
${}^{1}Que dirons-nous alors d’Abraham, notre ancêtre selon la chair ? Qu’a-t-il obtenu ? 
${}^{2}Si Abraham était devenu un homme juste par la pratique des œuvres, il aurait pu en tirer fierté, mais pas devant Dieu. 
${}^{3}Or, que dit l’Écriture ? Abraham eut foi en Dieu, et il lui fut accordé d’être juste. 
${}^{4}Si quelqu’un accomplit un travail, son salaire ne lui est pas accordé comme un don gratuit, mais comme un dû. 
${}^{5}Au contraire, si quelqu’un, sans rien accomplir, a foi en Celui qui rend juste l’homme impie, il lui est accordé d’être juste par sa foi. 
${}^{6}C’est ainsi que le psaume de David proclame heureux l’homme à qui Dieu accorde d’être juste, indépendamment de la pratique des œuvres :
        ${}^{7}Heureux ceux dont les offenses ont été remises,
        \\et les péchés, effacés.
        ${}^{8}Heureux l’homme dont le péché
        \\n’est pas compté par le Seigneur.
${}^{9}Cette béatitude-là concerne-t-elle seulement ceux qui ont la circoncision, ou bien aussi ceux qui ne l’ont pas ? Nous disons, en effet : « C’est pour sa foi qu’il a été accordé à Abraham d’être juste. » 
${}^{10}Et quand cela lui fut-il accordé ? Après la circoncision ? ou avant ? Non pas après, mais avant. 
${}^{11}Et il reçut le signe de la circoncision comme la marque de la justice obtenue par la foi avant d’être circoncis. De cette façon, il est le père de tous ceux qui croient sans avoir la circoncision, pour qu’à eux aussi, il soit accordé d’être justes ; 
${}^{12}et il est également le père des circoncis, ceux qui non seulement ont la circoncision, mais qui marchent aussi sur les traces de la foi de notre père Abraham avant sa circoncision.
${}^{13}Car ce n’est pas en vertu de la Loi que la promesse de recevoir le monde en héritage a été faite à Abraham et à sa descendance, mais en vertu de la justice obtenue par la foi. 
${}^{14}En effet, si l’on devient héritier par la Loi, alors la foi est sans contenu, et la promesse, abolie. 
${}^{15}Car la Loi aboutit à la colère de Dieu, mais là où il n’y a pas de Loi, il n’y a pas non plus de transgression. 
${}^{16}Voilà pourquoi on devient héritier par la foi : c’est une grâce, et la promesse demeure ferme pour tous les descendants d’Abraham, non pour ceux qui se rattachent à la Loi seulement, mais pour ceux qui se rattachent aussi à la foi d’Abraham, lui qui est notre père à tous. 
${}^{17}C’est bien ce qui est écrit : J’ai fait de toi le père d’un grand nombre de nations. Il est notre père devant Dieu en qui il a cru, Dieu qui donne la vie aux morts et qui appelle à l’existence ce qui n’existe pas. 
${}^{18}Espérant contre toute espérance, il a cru ; ainsi est-il devenu le père d’un grand nombre de nations, selon cette parole : Telle sera la descendance que tu auras ! 
${}^{19}Il n’a pas faibli dans la foi quand, presque centenaire, il considéra que son corps était déjà marqué par la mort et que Sara ne pouvait plus enfanter. 
${}^{20}Devant la promesse de Dieu, il n’hésita pas, il ne manqua pas de foi, mais il trouva sa force dans la foi et rendit gloire à Dieu, 
${}^{21}car il était pleinement convaincu que Dieu a la puissance d’accomplir ce qu’il a promis. 
${}^{22}Et voilà pourquoi il lui fut accordé d’être juste. 
${}^{23}En disant que cela lui fut accordé, l’Écriture ne s’intéresse pas seulement à lui, 
${}^{24}mais aussi à nous, car cela nous sera accordé puisque nous croyons en Celui qui a ressuscité d’entre les morts Jésus notre Seigneur, 
${}^{25}livré pour nos fautes et ressuscité pour notre justification.
      <h2 class="intertitle" id="d85e376020">2. Libération et salut par la vie dans le Christ (5 – 8)</h2>
      
         
      \bchapter{}
      \begin{verse}
${}^{1}Nous qui sommes donc devenus justes par la foi, nous voici en paix avec Dieu par notre Seigneur Jésus Christ, 
${}^{2}lui qui nous a donné, par la foi, l’accès à cette grâce dans laquelle nous sommes établis ; et nous mettons notre fierté dans l’espérance d’avoir part à la gloire de Dieu. 
${}^{3}Bien plus, nous mettons notre fierté dans la détresse elle-même, puisque la détresse, nous le savons, produit la persévérance ; 
${}^{4}la persévérance produit la vertu éprouvée ; la vertu éprouvée produit l’espérance ; 
${}^{5}et l’espérance ne déçoit pas, puisque l’amour de Dieu a été répandu dans nos cœurs par l’Esprit Saint qui nous a été donné.
${}^{6}Alors que nous n’étions encore capables de rien, le Christ, au temps fixé par Dieu, est mort pour les impies que nous étions. 
${}^{7}Accepter de mourir pour un homme juste, c’est déjà difficile ; peut-être quelqu’un s’exposerait-il à mourir pour un homme de bien. 
${}^{8}Or, la preuve que Dieu nous aime, c’est que le Christ est mort pour nous, alors que nous étions encore pécheurs. 
${}^{9}À plus forte raison, maintenant que le sang du Christ nous a fait devenir des justes, serons-nous sauvés par lui de la colère de Dieu. 
${}^{10}En effet, si nous avons été réconciliés avec Dieu par la mort de son Fils alors que nous étions ses ennemis, à plus forte raison, maintenant que nous sommes réconciliés, serons-nous sauvés en ayant part à sa vie. 
${}^{11}Bien plus, nous mettons notre fierté en Dieu, par notre Seigneur Jésus Christ, par qui, maintenant, nous avons reçu la réconciliation.
${}^{12}Nous savons que par un seul homme, le péché est entré dans le monde, et que par le péché est venue la mort ; et ainsi, la mort est passée en tous les hommes, étant donné que tous ont péché. 
${}^{13}Avant la loi de Moïse, le péché était déjà dans le monde, mais le péché ne peut être imputé à personne tant qu’il n’y a pas de loi. 
${}^{14}Pourtant, depuis Adam jusqu’à Moïse, la mort a établi son règne, même sur ceux qui n’avaient pas péché par une transgression semblable à celle d’Adam. Or, Adam préfigure celui qui devait venir.
${}^{15}Mais il n'en va pas du don gratuit comme de la faute. En effet, si la mort a frappé la multitude par la faute d’un seul, combien plus la grâce de Dieu s’est-elle répandue en abondance sur la multitude, cette grâce qui est donnée en un seul homme, Jésus Christ. 
${}^{16}Le don de Dieu et les conséquences du péché d’un seul n’ont pas la même mesure non plus : d’une part, en effet, pour la faute d’un seul, le jugement a conduit à la condamnation ; d’autre part, pour une multitude de fautes, le don gratuit de Dieu conduit à la justification. 
${}^{17}Si, en effet, à cause d’un seul homme, par la faute d’un seul, la mort a établi son règne, combien plus, à cause de Jésus Christ et de lui seul, régneront-ils dans la vie, ceux qui reçoivent en abondance le don de la grâce qui les rend justes.
${}^{18}Bref, de même que la faute commise par un seul a conduit tous les hommes à la condamnation, de même l’accomplissement de la justice par un seul a conduit tous les hommes à la justification qui donne la vie. 
${}^{19}En effet, de même que par la désobéissance d’un seul être humain la multitude a été rendue pécheresse, de même par l’obéissance d’un seul la multitude sera-t-elle rendue juste. 
${}^{20}Quant à la loi de Moïse, elle est intervenue pour que se multiplie la faute ; mais là où le péché s’est multiplié, la grâce a surabondé. 
${}^{21}Ainsi donc, de même que le péché a établi son règne de mort, de même la grâce doit établir son règne en rendant juste pour la vie éternelle par Jésus Christ notre Seigneur.
      
         
      \bchapter{}
      \begin{verse}
${}^{1}Que dire alors ? Allons-nous demeurer dans le péché pour que la grâce se multiplie ? 
${}^{2}Pas du tout. Puisque nous sommes morts au péché, comment pourrions-nous vivre encore dans le péché ? 
${}^{3}Ne le savez-vous pas ? Nous tous qui par le baptême avons été unis au Christ Jésus, c’est à sa mort que nous avons été unis par le baptême. 
${}^{4}Si donc, par le baptême qui nous unit à sa mort, nous avons été mis au tombeau avec lui, c’est pour que nous menions une vie nouvelle, nous aussi, comme le Christ qui, par la toute-puissancedu Père, est ressuscité d’entre les morts. 
${}^{5}Car, si nous avons été unis à lui par une mort qui ressemble à la sienne, nous le serons aussi par une résurrection qui ressemblera à la sienne. 
${}^{6}Nous le savons : l’homme ancien qui est en nous a été fixé à la croix avec lui pour que le corps du péché soit réduit à rien, et qu’ainsi nous ne soyons plus esclaves du péché. 
${}^{7}Car celui qui est mort est affranchi du péché. 
${}^{8}Et si nous sommes passés par la mort avec le Christ, nous croyons que nous vivrons aussi avec lui. 
${}^{9}Nous le savons en effet : ressuscité d’entre les morts, le Christ ne meurt plus ; la mort n’a plus de pouvoir sur lui. 
${}^{10}Car lui qui est mort, c'est au péché qu'il est mort une fois pour toutes ; lui qui est vivant, c'est pour Dieu qu'il est vivant. 
${}^{11}De même, vous aussi, pensez que vous êtes morts au péché, mais vivants pour Dieu en Jésus Christ.
${}^{12}Il ne faut donc pas que le péché règne dans votre corps mortel et vous fasse obéir à ses désirs. 
${}^{13}Ne présentez pas au péché les membres de votre corps comme des armes au service de l’injustice ; au contraire, présentez-vous à Dieu comme des vivants revenus d’entre les morts, présentez à Dieu vos membres comme des armes au service de la justice. 
${}^{14}Car le péché n’aura plus de pouvoir sur vous : en effet, vous n’êtes plus sujets de la Loi, vous êtes sujets de la grâce de Dieu.
${}^{15}Alors ? Puisque nous ne sommes pas soumis à la Loi mais à la grâce, allons-nous commettre le péché ? Pas du tout. 
${}^{16}Ne le savez-vous pas ? Celui à qui vous vous présentez comme esclaves pour lui obéir, c’est de celui-là, à qui vous obéissez, que vous êtes esclaves : soit du péché, qui mène à la mort, soit de l’obéissance à Dieu, qui mène à la justice. 
${}^{17}Mais rendons grâce à Dieu : vous qui étiez esclaves du péché, vous avez maintenant obéi de tout votre cœur au modèle présenté par l’enseignement qui vous a été transmis. 
${}^{18}Libérés du péché, vous êtes devenus esclaves de la justice.
${}^{19}J’emploie un langage humain, adapté à votre faiblesse. Vous aviez mis les membres de votre corps au service de l’impureté et du désordre, ce qui mène au désordre ; de la même manière, mettez-les à présent au service de la justice, ce qui mène à la sainteté. 
${}^{20}Quand vous étiez esclaves du péché, vous étiez libres par rapport aux exigences de la justice. 
${}^{21}Qu’avez-vous récolté alors, à commettre des actes dont vous avez honte maintenant ? En effet, ces actes-là aboutissent à la mort. 
${}^{22}Mais maintenant que vous avez été libérés du péché et que vous êtes devenus les esclaves de Dieu, vous récoltez ce qui mène à la sainteté, et cela aboutit à la vie éternelle. 
${}^{23}Car le salaire du péché, c’est la mort ; mais le don gratuit de Dieu, c’est la vie éternelle dans le Christ Jésus notre Seigneur.
      
         
      \bchapter{}
      \begin{verse}
${}^{1}Ne le savez-vous pas, frères – je parle à des gens qui s’y connaissent en matière de loi – : la loi n’a de pouvoir sur un être humain que durant sa vie. 
${}^{2}Ainsi, la femme mariée est liée par la loi à son mari s’il est vivant ; mais si le mari est mort, elle est dégagée de la loi du mari. 
${}^{3}Donc, du vivant de son mari, on la traitera d’adultère si elle appartient à un autre homme ; mais si le mari est mort, elle est libre à l’égard de la loi, si bien qu’elle ne sera pas adultère en appartenant à un autre. 
${}^{4}De même, mes frères, vous aussi, vous avez été mis à mort par rapport à la loi de Moïse en raison du corps crucifié du Christ, pour que vous apparteniez à un autre, Celui qui est ressuscité d’entre les morts, afin que nous portions des fruits pour Dieu. 
${}^{5}En effet, quand nous étions encore des êtres charnels, les passions coupables provoquées par la Loi agissaient dans tous nos membres, pour nous faire porter des fruits de mort. 
${}^{6}Mais maintenant, nous avons été dégagés de la Loi, étant morts à ce qui nous entravait ; ainsi, nous pouvons servir d’une façon nouvelle, celle de l’Esprit, et non plus à la façon ancienne, celle de la lettre de la Loi.
${}^{7}Que dire alors ? La Loi est-elle péché ? Pas du tout ! Mais je n’aurais pas connu le péché s’il n’y avait pas eu la Loi ; en effet, j’aurais ignoré la convoitise si la Loi n’avait pas dit : Tu ne convoiteras pas. 
${}^{8}Se servant de ce commandement, le péché a saisi l’occasion : il a produit en moi toutes sortes de convoitises. Sans la Loi, en effet, le péché est chose morte, 
${}^{9}et moi, jadis, sans la Loi, je vivais ; mais quand le commandement est venu, le péché est devenu vivant, 
${}^{10}et pour moi ce fut la mort. Il se trouve donc que, pour moi, ce commandement qui devait mener à la vie a mené à la mort. 
${}^{11}En effet, le péché a saisi l’occasion ; en se servant du commandement, il m’a séduit et, par lui, il m’a tué. 
${}^{12}Ainsi, la Loi est sainte ; le commandement est saint, juste et bon. 
${}^{13}Est-ce donc quelque chose de bon qui, pour moi, a été la mort ? Pas du tout : c’est le péché ! Pour qu’on voie bien qu’il est le péché, il s’est servi de quelque chose de bon pour causer ma mort ; ainsi, par le commandement, c’est le péché lui-même qui est devenu démesurément pécheur.
${}^{14}Nous savons bien que la Loi est une réalité spirituelle : mais moi, je suis un homme charnel, vendu au péché. 
${}^{15}En effet, ma façon d’agir, je ne la comprends pas, car ce que je voudrais, cela, je ne le réalise pas ; mais ce que je déteste, c’est cela que je fais. 
${}^{16}Or, si je ne veux pas le mal que je fais, je suis d’accord avec la Loi : je reconnais qu’elle est bonne. 
${}^{17}Mais en fait, ce n’est plus moi qui agis, c’est le péché, lui qui habite en moi.
${}^{18}Je sais que le bien n’habite pas en moi, c’est-à-dire dans l’être de chair que je suis. En effet, ce qui est à ma portée, c’est de vouloir le bien, mais pas de l’accomplir. 
${}^{19}Je ne fais pas le bien que je voudrais, mais je commets le mal que je ne voudrais pas. 
${}^{20}Si je fais le mal que je ne voudrais pas, alors ce n’est plus moi qui agis ainsi, mais c’est le péché, lui qui habite en moi. 
${}^{21}Moi qui voudrais faire le bien, je constate donc, en moi, cette loi : ce qui est à ma portée, c’est le mal. 
${}^{22}Au plus profond de moi-même, je prends plaisir à la loi de Dieu. 
${}^{23}Mais, dans les membres de mon corps, je découvre une autre loi, qui combat contre la loi que suit ma raison et me rend prisonnier de la loi du péché présente dans mon corps. 
${}^{24}Malheureux homme que je suis ! Qui donc me délivrera de ce corps qui m’entraîne à la mort ? 
${}^{25}Mais grâce soit rendue à Dieu par Jésus Christ notre Seigneur !
      Ainsi, moi, par ma raison, je suis au service de la loi de Dieu, et, par ma nature charnelle, au service de la loi du péché.
      
         
      \bchapter{}
      \begin{verse}
${}^{1}Ainsi, pour ceux qui sont dans le Christ Jésus, il n’y a plus de condamnation. 
${}^{2}Car la loi de l’Esprit qui donne la vie dans le Christ Jésus t’a libéré de la loi du péché et de la mort. 
${}^{3}En effet, quand Dieu a envoyé son propre Fils dans une condition charnelle semblable à celle des pécheurs pour vaincre le péché, il a fait ce que la loi de Moïse ne pouvait pas faire à cause de la faiblesse humaine : il a condamné le péché dans l’homme charnel. 
${}^{4}Il voulait ainsi que l’exigence de la Loi s’accomplisse en nous, dont la conduite n’est pas selon la chair mais selon l’Esprit.
${}^{5}En effet, ceux qui se conforment à la chair tendent vers ce qui est charnel ; ceux qui se conforment à l’Esprit tendent vers ce qui est spirituel ; 
${}^{6}et la chair tend vers la mort, mais l’Esprit tend vers la vie et la paix. 
${}^{7}Car la tendance de la chair est ennemie de Dieu, elle ne se soumet pas à la loi de Dieu, elle n’en est même pas capable. 
${}^{8}Ceux qui sont sous l’emprise de la chair ne peuvent pas plaire à Dieu. 
${}^{9}Or, vous, vous n’êtes pas sous l’emprise de la chair, mais sous celle de l’Esprit, puisque l’Esprit de Dieu habite en vous. Celui qui n’a pas l’Esprit du Christ ne lui appartient pas. 
${}^{10}Mais si le Christ est en vous, le corps, il est vrai, reste marqué par la mort à cause du péché, mais l’Esprit vous fait vivre, puisque vous êtes devenus des justes. 
${}^{11}Et si l’Esprit de celui qui a ressuscité Jésus d’entre les morts habite en vous, celui qui a ressuscité Jésus, le Christ, d’entre les morts donnera aussi la vie à vos corps mortels par son Esprit qui habite en vous.
${}^{12}Ainsi donc, frères, nous avons une dette, mais elle n’est pas envers la chair pour devoir vivre selon la chair. 
${}^{13}Car si vous vivez selon la chair, vous allez mourir ; mais si, par l’Esprit, vous tuez les agissements de l’homme pécheur, vous vivrez. 
${}^{14}En effet, tous ceux qui se laissent conduire par l’Esprit de Dieu, ceux-là sont fils de Dieu. 
${}^{15}Vous n’avez pas reçu un esprit qui fait de vous des esclaves et vous ramène à la peur ; mais vous avez reçu un Esprit qui fait de vous des fils ; et c’est en lui que nous crions « Abba ! », c’est-à-dire : Père ! 
${}^{16}C’est donc l’Esprit Saint lui-même qui atteste à notre esprit que nous sommes enfants de Dieu. 
${}^{17}Puisque nous sommes ses enfants, nous sommes aussi ses héritiers : héritiers de Dieu, héritiers avec le Christ, si du moins nous souffrons avec lui pour être avec lui dans la gloire.
${}^{18}J’estime, en effet, qu’il n’y a pas de commune mesure entre les souffrances du temps présent et la gloire qui va être révélée pour nous. 
${}^{19}En effet, la création attend avec impatience la révélation des fils de Dieu. 
${}^{20}Car la création a été soumise au pouvoir du néant, non pas de son plein gré, mais à cause de celui qui l’a livrée à ce pouvoir. Pourtant, elle a gardé l’espérance 
${}^{21}d’être, elle aussi, libérée de l’esclavage de la dégradation, pour connaître la liberté de la gloire donnée aux enfants de Dieu. 
${}^{22}Nous le savons bien, la création tout entière gémit, elle passe par les douleurs d’un enfantement qui dure encore. 
${}^{23}Et elle n’est pas seule. Nous aussi, en nous-mêmes, nous gémissons ; nous avons commencé à recevoir l’Esprit Saint, mais nous attendons notre adoption et la rédemption de notre corps. 
${}^{24}Car nous avons été sauvés, mais c’est en espérance ; voir ce qu’on espère, ce n’est plus espérer : ce que l’on voit, comment peut-on l’espérer encore ? 
${}^{25}Mais nous, qui espérons ce que nous ne voyons pas, nous l’attendons avec persévérance.
${}^{26}Bien plus, l’Esprit Saint vient au secours de notre faiblesse, car nous ne savons pas prier comme il faut. L’Esprit lui-même intercède pour nouspar des gémissements inexprimables. 
${}^{27}Et Dieu, qui scrute les cœurs, connaît les intentions de l’Esprit puisque c’est selon Dieu que l’Esprit intercède pour les fidèles.
${}^{28}Nous le savons, quand les hommes aiment Dieu, lui-même fait tout contribuer à leur bien, puisqu'ils sont appelés selon le dessein de son amour. 
${}^{29}Ceux que, d’avance, il connaissait, il les a aussi destinés d’avance à être configurés à l’image de son Fils, pour que ce Fils soit le premier-né d’une multitude de frères. 
${}^{30}Ceux qu’il avait destinés d’avance, il les a aussi appelés ; ceux qu’il a appelés, il en a fait des justes ; et ceux qu’il a rendus justes, il leur a donné sa gloire.
${}^{31}Que dire de plus ? Si Dieu est pour nous, qui sera contre nous ? 
${}^{32}Il n’a pas épargné son propre Fils, mais il l’a livré pour nous tous : comment pourrait-il, avec lui, ne pas nous donner tout ? 
${}^{33}Qui accusera ceux que Dieu a choisis ? Dieu est celui qui rend juste : 
${}^{34}alors, qui pourra condamner ? Le Christ Jésus est mort ; bien plus, il est ressuscité, il est à la droite de Dieu, il intercède pour nous : 
${}^{35}alors, qui pourra nous séparer de l’amour du Christ ? la détresse ? l’angoisse ? la persécution ? la faim ? le dénuement ? le danger ? le glaive ? 
${}^{36}En effet, il est écrit :
        \\C’est pour toi qu’on nous massacre sans arrêt,
        \\qu’on nous traite en brebis d’abattoir.
${}^{37}Mais, en tout cela nous sommes les grands vainqueurs grâce à celui qui nous a aimés. 
${}^{38}J’en ai la certitude : ni la mort ni la vie, ni les anges ni les Principautés célestes, ni le présent ni l’avenir, ni les Puissances, 
${}^{39}ni les hauteurs, ni les abîmes, ni aucune autre créature, rien ne pourra nous séparer de l’amour de Dieu qui est dans le Christ Jésus notre Seigneur.
      <h2 class="intertitle" id="d85e376935">3. Situation et salut d’Israël face aux nations (9 – 11)</h2>
      
         
      \bchapter{}
      \begin{verse}
${}^{1}C'est la vérité que je dis dans le Christ, je ne mens pas, ma conscience m'en rend témoignage dans l'Esprit Saint : 
${}^{2}j’ai dans le cœur une grande tristesse, une douleur incessante. 
${}^{3}Moi-même, pour les Juifs, mes frères de race, je souhaiterais être anathème, séparé du Christ : 
${}^{4}ils sont en effet Israélites, ils ont l’adoption, la gloire, les alliances, la législation, le culte, les promesses de Dieu ; 
${}^{5}ils ont les patriarches, et c’est de leur race que le Christ est né, lui qui est au-dessus de tout, Dieu béni pour les siècles. Amen.
      
         
${}^{6}Cela ne veut pas dire que la parole de Dieu a été mise en échec, car ceux qui sont nés d’Israël ne sont pas tous Israël. 
${}^{7}Et tous ceux qui sont la descendance d’Abraham ne sont pas pour autant ses enfants, car il est écrit : C’est par Isaac qu’une descendance portera ton nom. 
${}^{8}Autrement dit, ce ne sont pas les enfants de la chair qui sont enfants de Dieu, mais ce sont les enfants de la promesse qui sont comptés comme descendance. 
${}^{9}Car telle est la parole de la promesse : À la même époque, je reviendrai, et Sara aura un fils.
${}^{10}Et ce n’est pas tout ; il y a aussi Rébecca : elle ne s’était unie qu’à un seul homme, Isaac notre père. 
${}^{11}Ses enfants n’avaient pas encore été mis au monde, et n’avaient donc fait ni bien ni mal ; or, afin que demeure le projet de Dieu qui relève de son choix 
${}^{12}et ne dépend pas des œuvres mais de celui qui appelle, il fut dit à cette femme : L’aîné servira le plus jeune, 
${}^{13}comme il est écrit : J’ai aimé Jacob, je n’ai pas aimé Ésaü.
${}^{14}Que dire alors ? Y a-t-il de l’injustice en Dieu ? Pas du tout ! 
${}^{15}En effet, il dit à Moïse : À qui je fais miséricorde, je ferai miséricorde ; pour qui j’ai de la tendresse, j’aurai de la tendresse.
${}^{16}Il ne s’agit donc pas du vouloir ni de l’effort humain, mais de Dieu qui fait miséricorde. 
${}^{17}En effet, l’Écriture dit au Pharaon : Si je t’ai suscité, c’est pour montrer en toi ma puissance, et pour que mon nom soit proclamé sur toute la terre. 
${}^{18}Ainsi donc, il fait miséricorde à qui il veut, et il endurcit qui il veut.
${}^{19}Alors tu vas me dire : « Pourquoi Dieu adresse-t-il encore des reproches ? Qui, en effet, a pu s’opposer à sa volonté ? » 
${}^{20}Mais toi, homme, qui es-tu donc, pour entrer en contestation avec Dieu ? L’œuvre dira-t-elle à l’ouvrier : « Pourquoi m’as-tu faite ainsi ? » 
${}^{21}Le potier n’est-il pas maître de son argile, pour faire avec la même pâte un objet pour un usage honorable et un autre pour un usage méprisable ? 
${}^{22}Et si Dieu, voulant manifester sa colère et faire connaître sa puissance, a supporté avec beaucoup de patience des objets de colère voués à la perte, 
${}^{23}s’il l’a fait, n’est-ce pas aussi pour faire connaître la richesse de sa gloire en faveur des objets de miséricorde que, d’avance, il a préparés pour la gloire ? 
${}^{24}Ces objets de miséricorde, c’est nous, qu’il a appelés non seulement d’entre les Juifs, mais aussi d’entre les nations, 
${}^{25}comme précisément il le dit dans le livre du prophète Osée :
       
        \\Celui qu’on appelait « Pas-mon-peuple »,
        \\je l’appellerai « Mon-peuple »,
        \\celle qu’on appelait « Pas-aimée »,
        \\je l’appellerai « Aimée ».
${}^{26}Et, là même où Dieu leur avait dit :
        \\« Vous n’êtes pas mon peuple »,
        \\là ils seront appelés « fils du Dieu vivant ».
${}^{27}Quant à Isaïe, il s’exclame au sujet d’Israël :
        \\Même si le nombre des fils d’Israël
        \\est comme le sable de la mer,
        \\seul le reste d’Israël sera sauvé,
${}^{28}car le Seigneur réalisera sa parole
        \\jusqu’au bout et promptement
        \\sur la terre !
${}^{29}Et comme Isaïe l’a dit par avance :
        \\Si le Seigneur de l’univers
        \\ne nous avait pas laissé une descendance,
        \\nous serions devenus comme Sodome,
        \\nous serions semblables à Gomorrhe.
${}^{30}Que dire alors ? Des païens qui ne cherchaient pas à devenir des justes ont obtenu de le devenir, mais il s’agissait de la justice qui vient de la foi. 
${}^{31}Israël, au contraire, qui cherchait à observer une Loi permettant de devenir juste, n’y est pas parvenu. 
${}^{32}Pourquoi ? Parce qu’au lieu de compter sur la foi, ils comptaient sur les œuvres. Ils ont buté sur la pierre d’achoppement 
${}^{33}dont il est dit dans l’Écriture :
        \\Voici que je pose en Sion une pierre d’achoppement,
        \\un roc qui fait trébucher.
        \\Celui qui met en lui sa foi ne connaîtra pas la honte.
      
         
      \bchapter{}
      \begin{verse}
${}^{1}Frères, le vœu de mon cœur et ma prière à Dieu pour eux, c’est qu’ils obtiennent le salut. 
${}^{2}Car je peux en témoigner : ils ont du zèle pour Dieu, mais un zèle que n’éclaire pas la pleine connaissance. 
${}^{3}En ne reconnaissant pas la justice qui vient de Dieu, et en cherchant à instaurer leur propre justice, ils ne se sont pas soumis à la justice de Dieu. 
${}^{4}Car l’aboutissement de la Loi, c’est le Christ, afin que soit donnée la justice à toute personne qui croit. 
${}^{5}Au sujet de la justice qui vient de la Loi, Moïse écrit : L’homme qui mettra les commandements en pratique y trouvera la vie.
${}^{6}Mais la justice qui vient de la foi parle ainsi : Ne dis pas dans ton cœur : « Qui montera aux cieux ? » – c’est-à-dire pour en faire descendre le Christ.
${}^{7}Ou bien : « Qui descendra au fond de l’abîme ? » – c’est-à-dire pour faire remonter le Christ d’entre les morts.
${}^{8}Mais que dit cette justice ? Tout près de toi est la Parole, elle est dans ta bouche et dans ton cœur. Cette Parole, c’est le message de la foi que nous proclamons.
${}^{9}En effet, si de ta bouche, tu affirmes que Jésus est Seigneur, si, dans ton cœur, tu crois que Dieu l’a ressuscité d’entre les morts, alors tu seras sauvé. 
${}^{10}Car c’est avec le cœur que l’on croit pour devenir juste, c’est avec la bouche que l’on affirme sa foi pour parvenir au salut. 
${}^{11}En effet, l’Écriture dit : Quiconque met en lui sa foi ne connaîtra pas la honte.
${}^{12}Ainsi, entre les Juifs et les païens, il n’y a pas de différence : tous ont le même Seigneur, généreux envers tous ceux qui l’invoquent. 
${}^{13}En effet, quiconque invoquera le nom du Seigneur sera sauvé.
${}^{14}Or, comment l’invoquer, si on n’a pas mis sa foi en lui ? Comment mettre sa foi en lui, si on ne l’a pas entendu ? Comment entendre si personne ne proclame ? 
${}^{15}Comment proclamer sans être envoyé ? Il est écrit : Comme ils sont beaux, les pas des messagers qui annoncent les bonnes nouvelles!
${}^{16}Et pourtant, tous n’ont pas obéi à la Bonne Nouvelle. Isaïe demande en effet : Qui a cru, Seigneur, en nous entendant parler ? 
${}^{17}Or la foi naît de ce que l’on entend ; et ce que l’on entend, c’est la parole du Christ. 
${}^{18}Alors, je pose la question : n’aurait-on pas entendu ? Mais si, bien sûr ! Un psaume le dit :
        \\Sur toute la terre se répand leur message,
        \\et leurs paroles, jusqu’aux limites du monde.
${}^{19}Je pose encore la question : Israël n’aurait-il pas compris ? Moïse, le premier, dit :
        \\Je vais vous rendre jaloux par une nation qui n’en est pas une,
        \\par une nation stupide je vais vous exaspérer.
${}^{20}Et Isaïe a l’audace de dire :
        \\Je me suis laissé trouver par ceux qui ne me cherchaient pas,
        \\je me suis manifesté à ceux qui ne me demandaient rien.
${}^{21}Et à propos d’Israël, il dit :
        \\Tout le jour, j’ai tendu les mains
        \\vers un peuple qui refuse de croire et qui conteste.
      
         
      \bchapter{}
      \begin{verse}
${}^{1}Je pose donc la question : Dieu a-t-il rejeté son peuple ? Pas du tout ! Moi-même, en effet, je suis Israélite, de la descendance d’Abraham, de la tribu de Benjamin. 
${}^{2}Dieu n’a pas rejeté son peuple, que, d’avance, il connaissait. Ne savez-vous pas ce que dit l’Écriture dans l’histoire d’Élie lorsqu’il en appelle à Dieu contre Israël ? Il disait : 
${}^{3}Seigneur, ils ont tué tes prophètes et renversé tes autels ; je suis le seul à être resté, et ils en veulent à ma vie. 
${}^{4}Mais quelle est la réponse divine ? Je me suis réservé sept mille hommes qui n’ont pas fléchi le genou devant Baal. 
${}^{5}De la même manière, il y a donc aussi dans le temps présent un reste choisi par grâce. 
${}^{6}Et si c’est par grâce, ce n’est pas par les œuvres ; autrement, la grâce ne serait plus la grâce. 
${}^{7}Que dire alors ? Ce qu’Israël recherche, il ne l’a pas obtenu ; mais ceux qui ont été choisis l’ont obtenu, tandis que les autres ont été endurcis, 
${}^{8}comme le dit l'Écriture :
        \\Dieu leur a donné un esprit de torpeur,
        \\des yeux pour ne pas voir
        \\et des oreilles pour ne pas entendre,
        \\jusqu’au jour d’aujourd’hui.
${}^{9}Et David ajoute :
        \\Que leur table devienne un piège, une trappe,
        \\une occasion de chute, une juste rétribution ;
${}^{10}que leurs yeux s’obscurcissent pour qu’ils ne voient plus,
        \\fais-leur sans cesse courber le dos.
${}^{11}Je pose encore une question : ceux d’Israël ont-ils trébuché pour vraiment tomber ? Pas du tout ! Mais leur faute procure aux nations païennes le salut, pour qu’ils en deviennent jaloux. 
${}^{12}Or, si leur faute a été richesse pour le monde, si leur amoindrissement a été richesse pour les nations, combien plus le sera leur rassemblement !
${}^{13}Je vous le dis à vous, qui venez des nations païennes : dans la mesure où je suis moi-même apôtre des nations, j’honore mon ministère, 
${}^{14}mais dans l’espoir de rendre jaloux mes frères selon la chair, et d’en sauver quelques-uns. 
${}^{15}Si en effet le monde a été réconcilié avec Dieu quand ils ont été mis à l’écart, qu’arrivera-t-il quand ils seront réintégrés ? Ce sera la vie pour ceux qui étaient morts ! 
${}^{16}Si la partie de la pâte prélevée pour Dieu est sainte, toute la pâte l’est aussi ; si la racine de l’arbre est sainte, les branches le sont aussi. 
${}^{17}De ces branches, quelques-unes ont été coupées, alors que toi, olivier sauvage, tu as été greffé parmi les branches, et tu as part désormais à la sève que donne la racine de l’olivier. 
${}^{18}Alors, ne sois pas plein d’orgueil envers les branches ; malgré tout ton orgueil, ce n’est pas toi qui portes la racine, c’est la racine qui te porte. 
${}^{19}Tu vas me dire : « Des branches ont été coupées pour que moi, je sois greffé ! » 
${}^{20}Fort bien ! Mais c’est à cause de leur manque de foi qu’elles ont été coupées ; tandis que toi, c’est par la foi que tu tiens bon. Ne fais pas le fanfaron, sois plutôt dans la crainte. 
${}^{21}Car si Dieu n’a pas épargné les branches d’origine, il ne t’épargnera pas non plus. 
${}^{22}Observe donc la bonté et la rigueur de Dieu : rigueur pour ceux qui sont tombés, et bonté de Dieu pour toi, si tu demeures dans cette bonté ; autrement, toi aussi tu seras retranché. 
${}^{23}Quant à eux, s’ils ne demeurent pas dans leur manque de foi, ils seront greffés car Dieu est capable de leur redonner leur place en les greffant. 
${}^{24}En effet, toi qui étais par ton origine une branche d’olivier sauvage, tu as été greffé, malgré ton origine, sur un olivier cultivé ; à plus forte raison ceux-ci, qui sont d’origine, seront greffés sur leur propre olivier.
${}^{25}Frères, pour vous éviter de vous fier à votre propre jugement, je ne veux pas vous laisser dans l’ignorance de ce mystère : l’endurcissement d’une partie d’Israël s’est produit pour laisser à l’ensemble des nations le temps d’entrer. 
${}^{26}C’est ainsi qu’Israël tout entier sera sauvé, comme dit l'Écriture :
        \\De Sion viendra le libérateur,
        \\il fera disparaître les impiétés du milieu de Jacob.
        ${}^{27}Telle sera pour eux mon alliance
        \\lorsque j’enlèverai leurs péchés.
${}^{28}Certes, par rapport à l’Évangile, ils sont des adversaires, et cela, à cause de vous ; mais par rapport au choix de Dieu, ils sont des bien-aimés, et cela, à cause de leurs pères. 
${}^{29}Les dons gratuits de Dieu et son appel sont sans repentance. 
${}^{30}Jadis, en effet, vous avez refusé de croire en Dieu, et maintenant, par suite de leur refus de croire, vous avez obtenu miséricorde ; 
${}^{31}de même, maintenant, ce sont eux qui ont refusé de croire, par suite de la miséricorde que vous avez obtenue, mais c’est pour qu’ils obtiennent miséricorde, eux aussi. 
${}^{32}Dieu, en effet, a enfermé tous les hommes dans le refus de croire pour faire à tous miséricorde.
        ${}^{33}Quelle profondeur dans la richesse,
        la sagesse et la connaissance de Dieu !
        \\Ses décisions sont insondables,
        ses chemins sont impénétrables !
       
        ${}^{34}Qui a connu la pensée du Seigneur ?
        Qui a été son conseiller ?
        ${}^{35}Qui lui a donné en premier
        et mériterait de recevoir en retour ?
       
        ${}^{36}Car tout est de lui,
        et par lui, et pour lui.
        \\À lui la gloire pour l’éternité !
        Amen.
      
         
      \bchapter{}
      \begin{verse}
${}^{1}Je vous exhorte donc, frères, par la tendresse de Dieu, à lui présenter votre corps – votre personne tout entière –, en sacrifice vivant, saint, capable de plaire à Dieu : c’est là, pour vous, la juste manière de lui rendre un culte. 
${}^{2}Ne prenez pas pour modèle le monde présent, mais transformez-vous en renouvelant votre façon de penser pour discerner quelle est la volonté de Dieu : ce qui est bon, ce qui est capable de lui plaire, ce qui est parfait.
      
         
${}^{3}Par la grâce qui m’a été accordée, je dis à chacun d’entre vous : n’ayez pas de prétentions déraisonnables, mais pensez à être raisonnables, chacun dans la mesure de la mission que Dieu lui a confiée. 
${}^{4}Prenons une comparaison : en un corps unique, nous avons plusieurs membres, qui n’ont pas tous la même fonction ; 
${}^{5}de même, nous qui sommes plusieurs, nous sommes un seul corps dans le Christ, et membres les uns des autres, chacun pour sa part. 
${}^{6}Et selon la grâce que Dieu nous a accordée, nous avons reçu des dons qui sont différents. Si c’est le don de prophétie, que ce soit à proportion du message confié ; 
${}^{7}si c’est le don de servir, que l’on serve ; si l’on est fait pour enseigner, que l’on enseigne ; 
${}^{8}pour réconforter, que l’on réconforte. Celui qui donne, qu’il soit généreux ; celui qui dirige, qu’il soit empressé ; celui qui pratique la miséricorde, qu’il ait le sourire.
${}^{9}Que votre amour soit sans hypocrisie. Fuyez le mal avec horreur, attachez-vous au bien. 
${}^{10}Soyez unis les uns aux autres par l’affection fraternelle, rivalisez de respect les uns pour les autres. 
${}^{11}Ne ralentissez pas votre élan, restez dans la ferveur de l’Esprit, servez le Seigneur, 
${}^{12}ayez la joie de l’espérance, tenez bon dans l’épreuve, soyez assidus à la prière. 
${}^{13}Partagez avec les fidèles qui sont dans le besoin, pratiquez l’hospitalité avec empressement. 
${}^{14}Bénissez ceux qui vous persécutent ; souhaitez-leur du bien, et non pas du mal. 
${}^{15}Soyez joyeux avec ceux qui sont dans la joie, pleurez avec ceux qui pleurent. 
${}^{16}Soyez bien d’accord les uns avec les autres ; n’ayez pas le goût des grandeurs, mais laissez-vous attirer par ce qui est humble. Ne vous fiez pas à votre propre jugement. 
${}^{17}Ne rendez à personne le mal pour le mal, appliquez-vous à bien agir aux yeux de tous les hommes. 
${}^{18}Autant que possible, pour ce qui dépend de vous, vivez en paix avec tous les hommes. 
${}^{19}Bien-aimés, ne vous faites pas justice vous-mêmes, mais laissez agir la colère de Dieu. Car l’Écriture dit : C’est à moi de faire justice, c’est moi qui rendrai à chacun ce qui lui revient, dit le Seigneur. 
${}^{20}Mais si ton ennemi a faim, donne-lui à manger ; s’il a soif, donne-lui à boire : en agissant ainsi, tu entasseras sur sa tête des charbons ardents. 
${}^{21}Ne te laisse pas vaincre par le mal, mais sois vainqueur du mal par le bien.
      
         
      \bchapter{}
      \begin{verse}
${}^{1}Que chacun soit soumis aux autorités supérieures, car il n’y a d’autorité qu’en dépendance de Dieu, et celles qui existent sont établies sous la dépendance de Dieu ; 
${}^{2}si bien qu’en se dressant contre l’autorité, on est contre l’ordre des choses établi par Dieu, et en prenant cette position, on attire sur soi le jugement.
${}^{3}En effet, ceux qui dirigent ne sont pas à craindre quand on agit bien, mais quand on agit mal. Si tu ne veux pas avoir à craindre l’autorité, fais ce qui est bien, et tu recevras d’elle des éloges. 
${}^{4}Car elle est au service de Dieu pour t’inciter au bien ; mais si tu fais le mal, alors, vis dans la crainte. En effet, ce n’est pas pour rien que l’autorité détient le glaive. Car elle est au service de Dieu : en faisant justice, elle montre la colère de Dieu envers celui qui fait le mal.
${}^{5}C’est donc une nécessité d’être soumis, non seulement pour éviter la colère, mais encore pour obéir à la conscience. 
${}^{6}C’est pour cette raison aussi que vous payez des impôts : ceux qui les perçoivent sont des ministres de Dieu quand ils s’appliquent à cette tâche. 
${}^{7}Rendez à chacun ce qui lui est dû : à celui-ci l’impôt, à un autre la taxe, à celui-ci le respect, à un autre l’honneur.
${}^{8}N’ayez de dette envers personne, sauf celle de l’amour mutuel, car celui qui aime les autres a pleinement accompli la Loi. 
${}^{9}La Loi dit : Tu ne commettras pas d’adultère, tu ne commettras pas de meurtre, tu ne commettras pas de vol, tu ne convoiteras pas. Ces commandements et tous les autres se résument dans cette parole : Tu aimeras ton prochain comme toi-même. 
${}^{10}L’amour ne fait rien de mal au prochain. Donc, le plein accomplissement de la Loi, c’est l’amour.
${}^{11}Vous le savez : c’est le moment, l’heure est déjà venue de sortir de votre sommeil. Car le salut est plus près de nous maintenant qu’à l’époque où nous sommes devenus croyants. 
${}^{12}La nuit est bientôt finie, le jour est tout proche. Rejetons les œuvres des ténèbres, revêtons-nous des armes de la lumière. 
${}^{13}Conduisons-nous honnêtement, comme on le fait en plein jour, sans orgies ni beuveries, sans luxure ni débauches, sans rivalité ni jalousie, 
${}^{14}mais revêtez-vous du Seigneur Jésus Christ ; ne vous abandonnez pas aux préoccupations de la chair pour en satisfaire les convoitises.
      
         
      \bchapter{}
      \begin{verse}
${}^{1}Accueillez celui qui est faible dans la foi, sans critiquer ses raisonnements. 
${}^{2}L’un, à cause de sa foi, s’autorise à manger de tout ; l’autre, étant faible, ne mange que des légumes. 
${}^{3}Que celui qui mange ne méprise pas celui qui ne mange pas, et que celui qui ne mange pas ne juge pas celui qui mange, car Dieu l’a accueilli, lui aussi. 
${}^{4}Toi, qui es-tu pour juger le serviteur d’un autre ? Qu’il tienne debout ou qu’il tombe, cela regarde son maître à lui. Mais il sera debout, car son maître, le Seigneur, a le pouvoir de le faire tenir debout. 
${}^{5}L’un juge qu’il faut faire des différences entre les jours, l’autre juge qu’ils se valent tous : que chacun reste pleinement convaincu de son point de vue. 
${}^{6}Celui qui se préoccupe des jours le fait pour le Seigneur. De même, celui qui mange de tout le fait pour le Seigneur, car il rend grâce à Dieu ; mais celui qui ne mange pas de tout le fait aussi pour le Seigneur et il rend grâce à Dieu. 
${}^{7}En effet, aucun d’entre nous ne vit pour soi-même, et aucun ne meurt pour soi-même : 
${}^{8}si nous vivons, nous vivons pour le Seigneur ; si nous mourons, nous mourons pour le Seigneur. Ainsi, dans notre vie comme dans notre mort, nous appartenons au Seigneur. 
${}^{9}Car, si le Christ a connu la mort, puis la vie, c’est pour devenir le Seigneur et des morts et des vivants.
${}^{10}Alors toi, pourquoi juger ton frère ? Toi, pourquoi mépriser ton frère ? Tous, en effet, nous comparaîtrons devant le tribunal de Dieu. 
${}^{11}Car il est écrit :
        \\Aussi vrai que je suis vivant, dit le Seigneur,
        \\tout genou fléchira devant moi,
        \\et toute langue proclamera la louange de Dieu.
${}^{12}Ainsi chacun de nous rendra compte à Dieu pour soi-même.
${}^{13}Dès lors, cessons de nous juger les uns les autres ; mais jugez plutôt qu’il ne faut rien mettre devant un frère qui le fasse achopper ou trébucher. 
${}^{14}Je le sais, et j’en suis persuadé dans le Seigneur Jésus : aucune chose n’est impure en elle-même, mais si quelqu’un la considère comme impure, pour celui-là elle est impure. 
${}^{15}Car si ton frère a de la peine à cause de ce que tu manges, ta conduite n’est plus conforme à l’amour. Ne va pas faire périr, à cause de ce que tu manges, celui pour qui le Christ est mort. 
${}^{16}Cela dit, ce qui est bien pour vous ne doit pas être occasion de dénigrement. 
${}^{17}En effet, le royaume de Dieu ne consiste pas en des questions de nourriture ou de boisson ; il est justice, paix et joie dans l’Esprit Saint. 
${}^{18}Celui qui sert le Christ de cette manière-là plaît à Dieu, et il est approuvé par les hommes. 
${}^{19}Recherchons donc ce qui contribue à la paix, et ce qui construit les relations mutuelles.
${}^{20}Ne va pas détruire l’œuvre de Dieu pour une question de nourriture. Toutes les choses sont pures, mais c’est un mal de manger quelque chose si cela peut faire tomber un autre. 
${}^{21}Ce qui est bien, c’est de ne pas manger de viande, de ne pas boire de vin, bref, de ne rien prendre qui fasse tomber ton frère. 
${}^{22}La conviction que te donne la foi, garde-la en toi devant Dieu. Heureux celui qui ne se condamne pas lui-même par le choix qu’il fait. 
${}^{23}Mais si quelqu’un mange malgré ses doutes, celui-là est condamné, car il n’agit pas par conviction de foi. Or tout ce qui ne vient pas de la foi est péché.
      
         
      \bchapter{}
      \begin{verse}
${}^{1}Nous les forts, nous devons porter la fragilité des faibles, et non pas faire ce qui nous plaît. 
${}^{2}Que chacun de nous fasse ce qui plaît à son prochain, en vue du bien, dans un but constructif. 
${}^{3}Car le Christ n’a pas fait ce qui lui plaisait, mais, de lui, il est écrit : Sur moi sont retombées les insultes de ceux qui t’insultent. 
${}^{4}Or, tout ce qui a été écrit à l'avance dans les livres saints l’a été pour nous instruire, afin que, grâce à la persévérance et au réconfort des Écritures, nous ayons l’espérance. 
${}^{5}Que le Dieu de la persévérance et du réconfort vous donne d’être d’accord les uns avec les autres selon le Christ Jésus. 
${}^{6}Ainsi, d’un même cœur, d’une seule voix, vous rendrez gloire à Dieu, le Père de notre Seigneur Jésus Christ.
      
         
${}^{7}Accueillez-vous donc les uns les autres, comme le Christ vous a accueillis pour la gloire de Dieu. 
${}^{8}Car je vous le déclare : le Christ s’est fait le serviteur des Juifs, en raison de la fidélité de Dieu, pour réaliser les promesses faites à nos pères ; 
${}^{9}quant aux nations, c'est en raison de sa miséricorde qu'elles rendent gloire à Dieu, comme le dit l’Écriture :
        \\C’est pourquoi je proclamerai ta louange parmi les nations,
        \\je chanterai ton nom.
${}^{10}Il est dit encore :
        \\Réjouissez-vous, nations, avec son peuple !
${}^{11}Et encore :
        \\Louez le Seigneur, toutes les nations ;
        \\que tous les peuples chantent sa louange.
${}^{12}À son tour, Isaïe déclare :
        \\Il paraîtra, le rejeton de Jessé,
        \\celui qui se lève pour commander aux nations ;
        \\en lui les nations mettront leur espérance.
${}^{13}Que le Dieu de l’espérance vous remplisse de toute joie et de paix dans la foi, afin que vous débordiez d’espérance par la puissance de l’Esprit Saint.
${}^{14}Moi-même, je suis convaincu, mes frères, que vous êtes pleins de bonnes qualités, remplis de toute connaissance de Dieu, et capables aussi de vous reprendre les uns les autres. 
${}^{15}Mais je vous ai écrit avec un peu d’audace, comme pour raviver votre mémoire sur certains points, et c’est en raison de la grâce que Dieu m’a donnée. 
${}^{16}Cette grâce, c’est d’être ministre du Christ Jésus pour les nations, avec la fonction sacrée d’annoncer l’Évangile de Dieu, afin que l’offrande des nations soit acceptée par Dieu, sanctifiée dans l’Esprit Saint. 
${}^{17}Je mets donc ma fierté dans le Christ Jésus, pour ce qui est du service de Dieu. 
${}^{18}Car je n’oserais rien dire s’il ne s’agissait de ce que le Christ a mis en œuvre par moi afin d’amener les nations païennes à l’obéissance de la foi, par la parole et l’action, 
${}^{19}la puissance des signes et des prodiges, la puissance de l’Esprit de Dieu. Ainsi, depuis Jérusalem en rayonnant jusqu’à la Dalmatie, j’ai mené à bien l’annonce de l’Évangile du Christ. 
${}^{20}Je l’ai fait en mettant mon honneur à n’évangéliser que là où le nom du Christ n’avait pas encore été prononcé, car je ne voulais pas bâtir sur les fondations posées par un autre, 
${}^{21}mais j’ai agi selon cette parole de l’Écriture :
        \\Ceux à qui on ne l’avait pas annoncé verront ;
        \\ceux qui n’en avaient pas entendu parler comprendront.
${}^{22}C’est précisément ce qui m’a empêché tant de fois d’aller chez vous. 
${}^{23}Mais maintenant je n’ai plus de champ d’action dans les régions où je suis, et j’ai depuis des années le désir d’aller chez vous 
${}^{24}quand je me rendrai en Espagne. En effet, j’espère bien que je vous verrai en passant, et que je recevrai de vous l’aide nécessaire pour me rendre là-bas quand j’aurai d’abord un peu profité de cette rencontre avec vous.
${}^{25}Maintenant, je m’en vais à Jérusalem pour le service des fidèles. 
${}^{26}Car la Macédoine et la Grèce ont décidé un partage fraternel en faveur des pauvres de la communauté de Jérusalem. 
${}^{27}Elles ont pris cette décision en effet, car elles ont une dette envers eux : puisque les nations ont reçu une part des biens spirituels des fidèles de Jérusalem, elles leur sont à leur tour redevables d’une aide matérielle. 
${}^{28}Quand donc j’aurai accompli ce service, après leur avoir remis en bonne et due forme le fruit de ce partage, je m’en irai en Espagne en passant par chez vous. 
${}^{29}Et je le sais bien : quand je me rendrai chez vous, c’est avec la pleine bénédiction du Christ que je viendrai.
${}^{30}Je vous exhorte, frères, par notre Seigneur Jésus Christ et par l’amour de l’Esprit, à soutenir mon combat en priant Dieu pour moi, 
${}^{31}afin que j’échappe à ceux qui, en Judée, refusent de croire, et que mon service à Jérusalem soit bien accepté par les fidèles. 
${}^{32}Alors je pourrai, par la volonté de Dieu, arriver chez vous dans la joie et prendre du repos au milieu de vous.
${}^{33}Que le Dieu de la paix soit avec vous tous. Amen.
      
         
      \bchapter{}
      \begin{verse}
${}^{1}Je vous recommande Phébée notre sœur, ministre de l’Église qui est à Cencrées ; 
${}^{2}accueillez-la dans le Seigneur comme il convient à des fidèles ; aidez-la en toute affaire où elle aurait besoin de vous, car elle a prêté assistance à beaucoup de gens, de même qu’à moi.
${}^{3}Saluez de ma part Prisca et Aquilas, mes compagnons de travail en Jésus Christ, 
${}^{4}eux qui ont risqué leur tête pour me sauver la vie ; je ne suis d’ailleurs pas seul à leur être reconnaissant, toutes les Églises des nations le sont aussi. 
${}^{5}Saluez l’Église qui se rassemble dans leur maison.
      Saluez mon cher Épénète, qui fut le premier à croire au Christ dans la province d’Asie. 
${}^{6}Saluez Marie, qui s’est donné beaucoup de peine pour vous. 
${}^{7}Saluez Andronicos et Junias qui sont de ma parenté. Ils furent mes compagnons de captivité. Ce sont des apôtres bien connus ; ils ont même appartenu au Christ avant moi. 
${}^{8}Saluez Ampliatus, qui m’est cher dans le Seigneur. 
${}^{9}Saluez Urbain, notre compagnon de travail dans le Christ, et mon cher Stakys. 
${}^{10}Saluez Apellès, qui a fait ses preuves dans le Christ. Saluez les gens de chez Aristobule. 
${}^{11}Saluez Hérodion qui est de ma parenté. Saluez les gens de chez Narcisse, ceux qui croient au Seigneur. 
${}^{12}Saluez Tryphène et Tryphose, elles qui se donnent de la peine dans le Seigneur. Saluez la chère Persis, qui s’est donné beaucoup de peine dans le Seigneur. 
${}^{13}Saluez Rufus, choisi par le Seigneur, et sa mère qui est aussi la mienne. 
${}^{14}Saluez Asyncrite, Phlégon, Hermès, Patrobas, Hermas, et les frères qui sont avec eux. 
${}^{15}Saluez Philologue et Julie, Nérée et sa sœur, et Olympas, et tous les fidèles qui sont avec eux. 
${}^{16}Saluez-vous les uns les autres par un baiser de paix. Toutes les Églises du Christ vous saluent.
${}^{17}Je vous exhorte, frères, à faire attention à ceux qui provoquent des divisions et des scandales contrairement à l’enseignement que vous avez reçu : évitez-les ! 
${}^{18}Car les gens de cette espèce ne sont pas au service de notre Seigneur le Christ, mais de leurs propres appétits ; par leurs bonnes paroles et leurs éloges, ils séduisent les cœurs sans malice. 
${}^{19}Votre obéissance est connue de tous, et je m’en réjouis pour vous ; mais je veux que vous soyez avisés en vue du bien, et sans compromission avec le mal. 
${}^{20}Alors, sans délai, le Dieu de la paix écrasera Satan sous vos pieds.
      Que la grâce de notre Seigneur Jésus soit avec vous. 
${}^{21}Timothée, mon compagnon de travail, vous salue, ainsi que Lucius, Jason et Sosipatros, qui sont de ma parenté.
${}^{22}Moi aussi, Tertius, à qui cette lettre a été dictée, je vous salue dans le Seigneur. 
${}^{23}Gaïus vous salue, lui qui me donne l’hospitalité, à moi et à toute l’Église. Éraste, le trésorier de la ville, et notre frère Quartus vous saluent.
        ${}^{25}À Celui qui peut vous rendre forts
        selon mon Évangile qui proclame Jésus Christ :
        \\révélation d’un mystère
        gardé depuis toujours dans le silence,
        ${}^{26}mystère maintenant manifesté
        au moyen des écrits prophétiques,
        selon l’ordre du Dieu éternel,
        \\mystère porté à la connaissance de toutes les nations
        pour les amener à l’obéissance de la foi,
        ${}^{27}à Celui qui est le seul sage, Dieu, par Jésus Christ,
        à lui la gloire pour les siècles. Amen.
