  
  
      
         
      \bchapter{}
${}^{1}Qui aime son fils lui donne souvent le fouet,
        pour qu’il fasse, plus tard, sa joie.
${}^{2}Qui élève bien son fils en retirera des satisfactions ;
        devant ses relations, il en sera fier.
${}^{3}Qui instruit son fils rendra son ennemi jaloux
        et, devant ses amis, il aura de quoi se réjouir.
${}^{4}Qu’un tel père vienne à disparaître, c’est comme s’il n’était pas mort,
        car il laisse après lui quelqu’un qui lui ressemble.
${}^{5}Pendant sa vie, voir son fils était sa joie ;
        au moment de mourir, il n’aura pas de regret.
${}^{6}Il laisse quelqu’un pour le venger de ses ennemis,
        et rendre à ses amis leurs bienfaits.
${}^{7}Qui gâte son fils devra soigner ses blessures ;
        il sera bouleversé à chacun de ses cris.
${}^{8}Un cheval mal dressé devient rétif,
        et un fils à qui tout est permis n’en fait qu’à sa tête.
${}^{9}Cajole un enfant : il te consternera !
        Plaisante avec lui : il te fera pleurer !
${}^{10}Ne ris pas avec lui si tu ne veux pas souffrir avec lui :
        tu finirais par grincer des dents.
${}^{11}Ne lui laisse pas d’indépendance durant sa jeunesse,
        et ne ferme pas les yeux sur ses sottises.
${}^{12}Fais-lui courber l’échine pendant sa jeunesse,
        donne-lui une correction tant qu’il est enfant ;
        \\sinon, il deviendra obstiné et ne t’obéira plus,
        il sera le chagrin de ta vie.
${}^{13}Éduque ton fils, forme-le,
        pour ne pas te heurter un jour à son insolence.
        
           
${}^{14}Mieux vaut un pauvre en bonne santé et de constitution robuste
        qu’un riche torturé par la maladie.
${}^{15}Santé et bonne constitution valent mieux que tout l’or du monde ;
        un corps vigoureux vaut mieux qu’une immense fortune.
${}^{16}Il n’est pas de richesse préférable à la santé du corps,
        ni de bien-être supérieur à la joie de vivre.
${}^{17}Mieux vaut la mort qu’une vie d’amertume,
        et le repos éternel qu’une maladie sans fin.
${}^{18}Des plats succulents devant une bouche sans appétit
        sont comme les aliments déposés sur une tombe.
${}^{19}Et qu’importe à l’idole une offrande,
        elle qui ne mange ni ne sent rien ?
        \\Tel est celui qui a de la fortune et ne peut en profiter :
${}^{20}il l’a sous les yeux et soupire,
        comme un eunuque étreint une jeune fille et soupire.
        \\Tel est celui qui veut établir la justice par la violence.
${}^{21}Ne te laisse pas aller à la tristesse,
        ne te tourmente pas pour tes projets.
${}^{22}La joie du cœur fait vivre l’homme,
        et la gaieté prolonge la durée de ses jours.
${}^{23}Divertis-toi, réconforte ton cœur,
        et chasse loin de toi la tristesse ;
        \\car la tristesse en a perdu beaucoup,
        elle ne sert à rien.
${}^{24}Jalousie et colère abrègent la vie,
        les soucis font vieillir avant l’heure.
${}^{25}Qui a le cœur joyeux a bon appétit,
        et ce qu’il mange lui profite.
      <p class="cantique" id="bib_ct-at_14"><span class="cantique_label">Cantique AT 14</span> = <span class="cantique_ref"><a class="unitex_link" href="#bib_si_31_8">Si 31, 8-11</a></span>
      
         
      \bchapter{}
        <p class="retrait_special1 verset_anchor">
${}^{1}La richesse provoque l’insomnie qui épuise le corps ;
        <p class="retrait_special2 verset_no_anchor">elle cause des soucis qui font perdre le sommeil.
${}^{2}Les soucis de l’existence empêchent de fermer l’œil,
        comme une maladie grave éloigne le sommeil.
${}^{3}Le riche s’est donné de la peine pour faire fortune,
        et, quand il s’arrête, il prend du bon temps.
${}^{4}Le pauvre s’est donné de la peine pour vivre de peu
        et, quand il s’arrête, il tombe dans la misère.
${}^{5}Qui aime l’or ne pourra rester juste,
        qui court après le gain se laisse fourvoyer.
${}^{6}Beaucoup sont tombés pour avoir aimé l’or ;
        leur perte était inévitable.
${}^{7}L’or est un piège pour ceux qui en font une idole ;
        l’insensé s’y fera prendre.
        
           
       
      <div class="box_noborder cantique_chap">
            ${}^{8}Heureux le riche qui fut trouvé sans reproche
            et n’a pas couru après l’or.
            ${}^{9}Qui est-il ? Nous le dirons bienheureux :
            parmi son peuple, il a fait des merveilles !
            ${}^{10}Qui donc fut jugé parfait dans l’épreuve ?
            À lui, la gloire pour toujours\\ !
            \\Qui donc pouvait pécher et n’a point péché,
            faire le mal, et ne l’a pas fait ?
            ${}^{11}Ses biens s’affermiront dans le Seigneur\\,
            et l’assemblée dira ses largesses.
       
${}^{12}Es-tu assis à une table somptueuse ?
        N’en reste pas la bouche ouverte
        et ne dis pas : « Eh bien, que de choses ! »
${}^{13}Souviens-toi que c’est mal d’avoir l’œil avide :
        y a-t-il créature plus avide que l’œil ?
        Voilà pourquoi il pleure à tout propos.
${}^{14}N’étends pas la main vers tout ce qu’il convoite,
        ne te jette pas aussi vite que lui sur le plat.
${}^{15}Pense à ton voisin autant qu’à toi ;
        agis en tout avec réflexion.
${}^{16}Mange ce qui t’est offert, en homme bien élevé ;
        ne mâche pas bruyamment : tu serais insupportable.
${}^{17}Par bonne éducation, arrête-toi avant les autres ;
        ne te montre pas vorace, de peur de choquer.
${}^{18}Et s’il y a beaucoup de monde à table,
        n’étends pas la main le premier.
${}^{19}Qu’il suffit de peu à l’homme bien éduqué !
        Une fois couché, il n’a pas la respiration difficile.
${}^{20}Un estomac léger procure un sommeil réparateur ;
        on se lève le matin et on a l’esprit dispos.
        \\Mais les tourments de l’insomnie, les nausées
        et la colique attendent l’homme glouton.
${}^{21}Si tu as été forcé de trop manger,
        lève-toi, quitte la table et fais une pause.
${}^{22}Mon fils, écoute-moi sans ricaner :
        plus tard, tu comprendras mes paroles.
        \\Sois modéré en tout ce que tu fais,
        et tu ne tomberas jamais malade.
${}^{23}On bénit celui qui reçoit fastueusement,
        il justifie sa réputation de largesse.
${}^{24}Toute la ville dénigre celui qui reçoit chichement,
        il mérite sa réputation de pingre.
${}^{25}Avec le vin ne fais pas le brave,
        car le vin en a perdu beaucoup.
${}^{26}Comme à la forge on éprouve une lame quand on la trempe,
        le vin éprouve les cœurs quand des orgueilleux se querellent.
${}^{27}Pour les hommes, le vin, c’est la vie,
        tant qu’on le boit avec modération.
        \\Qu’est-ce qu’une vie où manque le vin ?
        Il a été créé pour la joie de l’homme.
${}^{28}Le vin est allégresse du cœur et joie de vivre
        pour qui le boit à son heure et avec mesure.
${}^{29}Le vin est amertume de l’âme pour qui le boit avec excès
        au point de s’exalter et de perdre l’équilibre.
${}^{30}L’ivresse décuple la fureur de l’insensé jusqu’au scandale,
        elle diminue sa force et lui attire des coups.
${}^{31}Ne provoque pas ton voisin au cours d’un banquet bien arrosé,
        ne te moque pas de lui quand il est joyeux,
        \\ne lui adresse pas des propos blessants
        et ne le harcèle pas de tes réclamations.
      
         
      \bchapter{}
        <p class="retrait_special1 verset_anchor">
${}^{1}On t’a choisi pour présider un banquet ?
        <p class="retrait_special2 verset_no_anchor">Ne prends pas de grands airs.
        \\Sois un simple convive parmi les autres,
        occupe-toi d’eux, et alors seulement tu iras t’asseoir.
${}^{2}Quand tu auras rempli tout ton office, va prendre place
        afin de te réjouir avec eux
        et recevoir la couronne pour ta parfaite organisation.
        
           
         
${}^{3}Si tu es âgé, prends la parole, car cela te revient,
        mais mesure bien ce que tu dis, ne retarde pas la musique
${}^{4}et, pendant qu’on l’écoute, ne te répands pas en bavardages :
        n’étale pas ta sagesse à contretemps.
${}^{5}Une pierre de grenat enchâssée dans un bijou en or,
        tel est un concert dans un banquet bien arrosé.
${}^{6}Une émeraude enchâssée dans une monture d’or,
        tel est un air de musique sur un bon vin.
        
           
         
${}^{7}Si tu es jeune, ne prends la parole que si tu dois le faire,
        mais pas plus de deux fois, et seulement si on t’interroge.
${}^{8}Résume ton propos : en quelques mots on peut dire beaucoup ;
        montre-toi instruit et discret à la fois.
${}^{9}Avec de grands personnages, ne fais pas l’important ;
        quand un autre a la parole, ne bavarde pas.
${}^{10}Aussi sûr qu’un éclair devance le tonnerre,
        la faveur est d’avance acquise à une personne réservée.
${}^{11}Quand c’est l’heure, lève-toi, ne t’attarde pas,
        rentre chez toi sans flâner.
${}^{12}Là, tu pourras te divertir et faire ce qui te plaît,
        sans risque de pécher par vantardise.
${}^{13}Et, pour tout cela, bénis ton Créateur,
        lui qui t’enivre de ses biens.
        
           
${}^{14}Qui craint le Seigneur recevra son enseignement,
        ceux qui le cherchent dès l’aurore obtiendront sa faveur.
${}^{15}Qui scrute la Loi en sera comblé,
        mais l’hypocrite y trouvera une occasion de chute.
${}^{16}Ceux qui craignent le Seigneur découvriront ce qui est juste
        et feront briller leurs jugements comme la lumière.
${}^{17}Le pécheur n’accepte pas les remarques
        et trouve toujours à justifier ce qui lui plaît.
${}^{18}Un homme réfléchi ne néglige aucun avis,
        l’insolent et l’orgueilleux n’ont peur de rien.
${}^{19}Ne fais rien sans avoir réfléchi,
        et tu n’auras pas à regretter tes actes.
${}^{20}Le chemin accidenté, ne le prends pas,
        de peur de buter sur les pierres.
${}^{21}Ne te fie pas à un chemin uni,
${}^{22}et même avec tes enfants sois sur tes gardes.
${}^{23}En tous tes actes, fie-toi à ta conscience,
        car cela aussi est observance des commandements.
${}^{24}Qui se fie à la Loi est attentif aux commandements,
        qui met sa confiance dans le Seigneur ne connaîtra pas le déclin.
       
      
         
      \bchapter{}
        <p class="retrait_special1 verset_anchor">
${}^{1}À celui qui craint le Seigneur il n’arrive aucun mal :
        <p class="retrait_special2 verset_no_anchor">de chaque épreuve il est délivré.
${}^{2}Un homme sage n’aura pas d’aversion pour la Loi,
        mais qui triche avec elle est comme une barque dans la tempête.
${}^{3}Celui qui est intelligent se fie à la Loi,
        la Loi est pour lui aussi digne de foi qu’un oracle divin.
${}^{4}Prépare ton discours, et tu te feras écouter,
        rassemble tes idées avant d’intervenir.
${}^{5}Le cœur du fou est une roue de chariot,
        et son raisonnement, un essieu qui tourne sur lui-même.
${}^{6}Un ami moqueur est comme un étalon :
        il se met à hennir quel que soit son cavalier.
        
           
${}^{7}Comment se fait-il qu’un jour soit plus important qu’un autre,
        alors que, toute l’année, la lumière des jours vient du soleil ?
${}^{8}C’est le Seigneur qui, dans sa science, les a faits différents ;
        il a diversifié les temps et les fêtes.
${}^{9}Il en a élevé et sanctifié certains ;
        les autres, il les a pris pour faire nombre.
${}^{10}De même, les hommes sont tous tirés du sol,
        et c’est de la terre qu’Adam a été créé ;
${}^{11}mais, dans son vaste savoir, le Seigneur les a distingués :
        à chacun il offre un chemin différent.
${}^{12}Il en est parmi eux qu’il a bénis et élevés,
        il en est qu’il a consacrés, et dont il a fait ses proches.
        \\Il en a maudit et abaissé d’autres,
        il les a rejetés de leur place.
${}^{13}Comme l’argile est dans la main du potier,
        qui la modèle à son gré,
        \\ainsi les hommes sont dans la main de leur Créateur
        qui les rétribue selon son jugement.
${}^{14}Face au mal, le bien ;
        face à la mort, la vie ;
        de même, face à l’homme religieux, le pécheur.
${}^{15}Considère ainsi toutes les œuvres du Très-Haut :
        elles vont deux par deux, l’une en face de l’autre.
      
         
      \bchapter{}
        <p class="retrait_special1 verset_anchor">
${}^{16}Et moi, dernier venu, j’ai été vigilant
        <p class="retrait_special2 verset_no_anchor">comme le grappilleur qui passe après les vendangeurs ;
${}^{17}grâce à la bénédiction du Seigneur j’ai rattrapé le retard
        et, autant qu’un vendangeur, j’ai rempli le pressoir.
${}^{18}Comprenez-le bien : ce n’est pas pour moi seul que j’ai peiné,
        mais pour tous ceux qui recherchent la sagesse.
        
           
${}^{19}Écoutez-moi, grands de ce peuple,
        et vous qui présidez l’assemblée, prêtez l’oreille !
${}^{20}Ni à ton fils ni à ta femme, ni à ton frère ni à ton ami,
        ne donne pouvoir sur toi durant ta vie.
        \\Ne fais don de tes biens à personne :
        tu pourrais t’en repentir et devoir les redemander.
${}^{21}Aussi longtemps que tu vis et qu’il te reste un souffle,
        ne te livre pas au pouvoir d’un mortel.
${}^{22}Car il vaut mieux que tes enfants te sollicitent
        que de dépendre de tes fils.
${}^{23}De toutes tes affaires, garde le contrôle,
        ne laisse pas ternir ta réputation.
${}^{24}Et quand arrivera le dernier des jours de ta vie,
        quand viendra l’heure de ta mort, tu répartiras ton héritage.
${}^{25}Fourrage, bâton et fardeau, voilà pour l’âne ;
        pour le domestique : pain, discipline et travail.
${}^{26}Fais travailler ton serviteur, tu trouveras le repos ;
        laisse ses mains inoccupées, il cherchera la liberté.
${}^{27}La bride et le joug font plier la nuque ;
        pour le mauvais domestique : sévices et châtiments.
${}^{28}Mets-le au travail de peur qu’il ne devienne paresseux,
${}^{29}car la paresse enseigne bien des vices.
${}^{30}Tiens-le à l’ouvrage, selon ce qui lui convient,
        et s’il n’obéit pas, mets-lui des fers aux pieds ;
        \\mais ne dépasse la mesure avec personne,
        et ne fais rien sans discernement.
${}^{31}As-tu un domestique ? Qu’il soit comme un autre toi-même,
        puisque tu l’as acquis dans le sang.
        \\As-tu un domestique ? Traite-le comme un frère,
        puisque tu as besoin de lui comme de toi-même.
${}^{32}Si tu le maltraites et qu’il s’enfuie,
${}^{33}sur quel chemin iras-tu le chercher ?
      
         
      \bchapter{}
        <p class="retrait_special1 verset_anchor">
${}^{1}Les espoirs vains et trompeurs sont le lot des hommes stupides,
        <p class="retrait_special2 verset_no_anchor">et les songes donnent des ailes aux insensés.
${}^{2}S’arrêter à des songes,
        autant saisir une ombre ou poursuivre le vent.
${}^{3}Ce que l’on voit en songe n’est qu’un simple reflet :
        au lieu d’un visage réel, le semblant d’un visage.
${}^{4}De l’impur, que peut-il sortir de pur ?
        Du mensonge, que peut-il sortir de vrai ?
${}^{5}Divinations, présages, songes, autant de balivernes :
        délires d’une femme qui accouche !
${}^{6}À moins qu’ils ne soient envoyés comme une visite du Très-Haut,
        n’y attache pas ton esprit.
${}^{7}Les songes ont égaré bien des gens
        et fait tomber ceux qui avaient mis en eux leur espoir.
${}^{8}Mais la Loi ne trompe pas, elle accomplit ce qu’elle promet,
        la sagesse d’une bouche sincère trouve son accomplissement.
        
           
${}^{9}Un homme qui a voyagé a beaucoup appris,
        celui qui a de l’expérience parle en connaissance de cause.
${}^{10}Qui n’a pas été mis à l’épreuve connaît peu,
${}^{11}mais qui a voyagé est plein de ressources.
${}^{12}J’ai vu beaucoup de choses durant mes voyages,
        j’en sais plus que je ne pourrais dire.
${}^{13}J’ai été maintes fois en danger de mort,
        j’en suis sorti sain et sauf, grâce à mon expérience.
${}^{14}Ceux qui craignent le Seigneur auront la vie,
${}^{15}car ils ont mis leur espérance en celui qui les sauve.
${}^{16}Qui craint le Seigneur n’a rien à redouter,
        il ne s’effraie de rien, car c’est lui son espérance.
${}^{17}Qu’elle est heureuse, l’âme qui craint le Seigneur !
${}^{18}Sur qui prend-elle appui ? Qui est son soutien ?
${}^{19}Ceux qui aiment le Seigneur, le Seigneur les regarde :
        il est bouclier puissant, appui solide,
        \\abri contre le vent brûlant et le soleil de midi,
        protection contre l’obstacle, secours qui préserve de la chute ;
${}^{20}il relève l’âme, illumine le regard,
        donne guérison, vie et bénédiction.
${}^{21}Offrir en sacrifice un bien mal acquis, c’est se moquer ;
${}^{22}les dons des gens sans loi ne sont pas agréés.
${}^{23}Le Très-Haut n’agrée pas les offrandes des impies ;
        ce n’est pas le nombre des sacrifices qui lui fait pardonner les péchés.
${}^{24}Offrir un sacrifice avec les biens du pauvre,
        c’est sacrifier le fils sous les yeux de son père.
${}^{25}La vie des indigents tient à un peu de pain :
        qui le leur enlève est un assassin.
${}^{26}C’est tuer son prochain que lui retirer la subsistance,
${}^{27}c’est verser le sang que priver l’ouvrier de son salaire.
${}^{28}L’un construit, l’autre démolit ;
        qu’ont-ils gagné, sinon des peines ?
${}^{29}L’un prie, l’autre maudit ;
        de qui le Maître écoutera-t-il la voix ?
${}^{30}Si quelqu’un se purifie après avoir touché un cadavre
        et le touche à nouveau,
        à quoi lui aura servi son ablution ?
${}^{31}Ainsi l’homme qui jeûne à cause de ses péchés,
        puis y retourne et recommence :
        \\qui écoutera sa prière ?
        À quoi lui aura servi sa pénitence ?
      
         
      \bchapter{}
        <p class="retrait_special1 verset_anchor">${}^{1}C’est présenter de multiples offrandes
        <p class="retrait_special2 verset_no_anchor">que d’observer la Loi ;
        ${}^{2}c’est offrir un sacrifice de paix
        que s’attacher aux commandements.
        ${}^{3}C’est apporter une offrande de fleur de farine
        que se montrer reconnaissant ;
        ${}^{4}c’est présenter un sacrifice de louange
        que faire l’aumône.
        ${}^{5}On obtient la bienveillance du Seigneur
        en se détournant du mal ;
        \\on offre un sacrifice d’expiation
        en se détournant de l’injustice.
        ${}^{6}Ne te présente pas devant le Seigneur les mains vides.
        ${}^{7}Accomplis tout cela car tel est son commandement.
        ${}^{8}L’offrande de l’homme juste
        est comme la graisse des sacrifices sur l’autel,
        son agréable odeur s’élève devant le Très-Haut.
        ${}^{9}Le sacrifice de l’homme juste est agréé par Dieu
        qui en gardera mémoire\\.
        ${}^{10}Rends gloire au Seigneur sans être regardant :
        ne retranche rien des prémices de ta récolte\\.
        ${}^{11}Chaque fois que tu fais un don, montre un visage joyeux ;
        consacre de bon cœur à Dieu le dixième de ce que tu gagnes.
        ${}^{12}Donne au Très-Haut selon ce qu’il te donne,
        et, sans être regardant, selon tes ressources.
        ${}^{13}Car le Seigneur est celui qui paye de retour ;
        il te rendra sept fois plus que tu n’as donné.
        ${}^{14}N’essaye pas de l’influencer par des présents,
        il ne les acceptera pas ;
        ${}^{15}ne mets pas ta confiance dans un sacrifice injuste.
        \\Car le Seigneur est un juge
        qui se montre impartial envers les personnes.
        ${}^{16}Il ne défavorise pas le pauvre,
        il écoute la prière de l’opprimé.
        ${}^{17}Il ne méprise pas la supplication de l’orphelin,
        ni la plainte répétée de la veuve.
${}^{18}Les larmes de la veuve ne coulent-elles pas sur ses joues,
${}^{19}et son cri n’accuse-t-il pas celui qui la fait pleurer ?
        ${}^{20}Celui dont le service est agréable à Dieu\\sera bien accueilli,
        sa supplication parviendra jusqu’au ciel.
        ${}^{21}La prière du pauvre traverse les nuées ;
        tant qu’elle n’a pas atteint son but, il demeure inconsolable.
        \\Il persévère tant que le Très-Haut n’a pas jeté les yeux sur lui,
        ${}^{22}ni prononcé la sentence en faveur des justes et rendu justice.
        
           
        \\Le Seigneur ne tardera pas,
        il restera impatient,
        \\jusqu’à ce qu’il ait brisé les reins des hommes sans pitié,
${}^{23}tiré vengeance des nations,
        \\supprimé la multitude des insolents
        et brisé le sceptre des injustes,
${}^{24}jusqu’à ce qu’il ait rendu à chacun selon ses œuvres
        et rétribué les actions des hommes selon leurs intentions,
${}^{25}jusqu’à ce qu’il ait jugé la cause de son peuple
        et qu’il l’ait comblé de joie par sa miséricorde.
${}^{26}Qu’elle sera bienvenue, sa miséricorde, au temps du malheur,
        comme les nuages de pluie au temps de la sécheresse !
      <p class="cantique" id="bib_ct-at_15a"><span class="cantique_label">Cantique AT 15a</span> = <span class="cantique_ref"><a class="unitex_link" href="#bib_si_36_1">Si 36, 1-7.13-19</a></span>
      <p class="cantique" id="bib_ct-at_15b"><span class="cantique_label">Cantique AT 15b</span> = <span class="cantique_ref"><a class="unitex_link" href="#bib_si_36_17">Si 36, 17-22</a></span>
      
         
      \bchapter{}
        <p class="retrait_special1 verset_anchor">${}^{1}Prends pitié de nous, Maître et Dieu de tout ;
        <p class="retrait_special1 verset_anchor">${}^{2}répands la crainte sur toutes les nations.
        ${}^{3}Lève la main sur les pays étrangers,
        et qu’ils voient ta puissance !
        
           
         
        ${}^{4}À nos dépens, tu leur montras ta sainteté ;
        à leurs dépens, montre-nous ta grandeur.
        ${}^{5}Qu’ils l’apprennent, comme nous l’avons appris :
        il n’est pas de dieu hors de toi, Seigneur.
        
           
         
        ${}^{6}Renouvelle les prodiges, recommence les merveilles,
        ${}^{7}glorifie ta main et ton bras droit.
${}^{8}\[Réveille ta colère, déverse ta fureur,
${}^{9}détruis l’adversaire, arrache l’ennemi.
${}^{10}Hâte le temps, rappelle-toi le terme\\,
        et que soient racontées tes merveilles !
        
           
         
${}^{11}Qu’un feu vengeur dévore le survivant,
        et que périssent les bourreaux de ton peuple !
${}^{12}Brise les têtes des princes ennemis
        qui disent : « Il n’est rien hors de nous ! »\]
        
           
         
        ${}^{13}Rassemble les tribus de Jacob ;
        ${}^{16}comme aux premiers jours\\, donne-leur ton héritage\\.
        ${}^{17}Prends pitié du peuple porteur de ton nom,
        Israël qui est pour toi un premier-né.
        
           
         
        ${}^{18}Prends compassion de ta Ville sainte,
        Jérusalem, le lieu de ton repos.
        ${}^{19}Remplis Sion de ta louange,
        et ton sanctuaire\\, de ta gloire.
        
           
         
        ${}^{20}Rends témoignage à tes créatures des premiers jours ;
        réveille les prophéties faites\\en ton nom.
        ${}^{21}Donne la récompense à ceux qui t’attendent ;
        que tes prophètes soient reconnus dignes de foi.
        
           
         
        ${}^{22}Écoute\\la prière de tes serviteurs\\,
        selon ta bienveillance\\à l’égard de ton peuple.
        \\Et tous, sur la terre, le sauront :
        tu es « Le Seigneur », le Dieu des siècles !
        
           
${}^{23}L’estomac absorbe toute sorte d’aliments,
        mais tel aliment est meilleur que l’autre.
${}^{24}Comme le palais reconnaît une viande faisandée,
        le cœur avisé reconnaît les paroles mensongères.
${}^{25}Un cœur pervers cause du chagrin,
        mais l’homme d’expérience saura lui rendre la pareille.
${}^{26}Une femme doit accepter n’importe quel mari,
        mais telle fille est préférable à telle autre.
${}^{27}La beauté d’une femme est joie pour les yeux ;
        il n’y a rien de plus désirable pour un homme.
${}^{28}Si elle a sur les lèvres douceur et bonté,
        son mari n’appartient plus au commun des mortels !
${}^{29}Pour qui prend femme, c’est déjà la fortune :
        elle est une aide semblable à lui, une colonne où s’appuyer.
${}^{30}Faute de clôture, un domaine est livré aux pillards ;
        faute d’avoir une femme, on erre à l’aventure en gémissant.
${}^{31}Qui ferait confiance à un agile escroc
        passant de ville en ville ?
        \\Il en va de même pour l’homme qui n’a pas de nid
        et qui s’arrête là où le soir le surprend.
      
         
      \bchapter{}
${}^{1}Tous tes amis diront : « Moi aussi, je suis ton ami ! »,
        mais tel n’est ami que de nom.
${}^{2}N’est-ce pas triste à mourir
        de voir un compagnon ou un ami se changer en ennemi ?
${}^{3}Ô mauvais penchant, d’où as-tu été tiré
        pour couvrir ainsi la terre de fourberie ?
${}^{4}Si l’ami n’est qu’un compagnon de plaisir, au temps de la fête,
        que vienne la détresse, il changera de camp.
${}^{5}Si l’ami a souffert de la faim avec son compagnon,
        au moment du combat, il prendra les armes avec lui.
${}^{6}Que ton cœur n’oublie jamais un ami,
        ne perds pas son souvenir si tu deviens riche.
        
           
${}^{7}Tout conseiller fait valoir son avis,
        mais il en est qui conseillent dans leur propre intérêt.
${}^{8}Méfie-toi d’un donneur de conseils,
        sache d’abord de quoi il a besoin,
        car il donne des conseils intéressés.
        \\Il pourrait jeter sur toi son dévolu
${}^{9}et te dire : « Tu es sur le bon chemin »,
        puis il prendra ses distances pour voir ce qui t’arrive.
${}^{10}Ne consulte pas quelqu’un de sournois ;
        à ceux qui te jalousent cache ton projet.
${}^{11}Ne consulte pas non plus une femme sur sa rivale,
        un lâche sur la guerre,
        \\un négociant sur le taux de change,
        un acheteur sur une vente,
        \\un envieux sur la générosité,
        un homme sans cœur sur la bienfaisance,
        \\un paresseux sur quelque ouvrage que ce soit,
        un travailleur saisonnier sur l’achèvement de la moisson,
        \\un serviteur paresseux sur une grosse besogne.
        Pour un conseil, ne te fie à aucun de ces gens-là.
${}^{12}Mais adresse-toi toujours à un homme religieux,
        dont tu sais qu’il observe les commandements,
        \\qui a un cœur selon ton cœur
        et qui, si tu échoues, partagera ta souffrance.
${}^{13}Puis tiens-t’en au conseil de ton cœur,
        car personne ne t’est plus fidèle que lui.
${}^{14}Bien souvent, l’âme d’un homme l’avertit
        mieux que sept veilleurs en faction sur la hauteur.
${}^{15}Mais, par-dessus tout, supplie le Très-Haut
        de diriger tes pas dans la vérité.
${}^{16}Le point de départ de toute œuvre, c’est la réflexion ;
        toute entreprise est précédée d’un projet.
${}^{17}La racine des pensées, c’est le cœur ;
${}^{18}il en sort quatre rameaux :
        le bien et le mal, la vie et la mort ;
        et celle qui décide de tout, c’est la langue.
${}^{19}Tel homme est habile pour en instruire beaucoup,
        mais pour lui-même il n’est bon à rien.
${}^{20}Tel autre veut faire étalage de sagesse et se rend détestable,
        il finira par mourir de faim :
${}^{21}le Seigneur ne lui donne pas de plaire
        car il est dépourvu de toute sagesse.
${}^{22}Tel, enfin, est sage dans son cœur,
        et les fruits de son intelligence, dans sa bouche, sont dignes de foi.
${}^{23}C’est l’homme sage qui enseigne son peuple :
        les fruits de son intelligence sont dignes de foi !
${}^{24}L’homme sage est comblé de bénédictions ;
        tous ceux qui le voient le proclament heureux.
${}^{25}Les jours d’une vie humaine sont comptés,
        mais les jours d’Israël sont sans nombre.
${}^{26}Le sage, au milieu de son peuple, gagne la confiance,
        et son nom vivra à jamais.
${}^{27}Mon fils, sur ton mode de vie éprouve-toi toi-même,
        vois ce qui te fait du mal et ne te l’accorde pas.
${}^{28}Car tout ne convient pas à tous,
        et tous ne trouvent pas agrément à tout.
${}^{29}Ne sois pas insatiable des plaisirs de la table,
        et ne te précipite pas sur les plats,
${}^{30}car, à trop manger, on se rend malade
        et la gloutonnerie provoque des coliques.
${}^{31}Beaucoup même en sont morts,
        mais qui se modère prolonge sa vie.
      
         
      \bchapter{}
${}^{1}Honore à sa juste valeur le médecin pour ses services :
        le Seigneur l’a créé, lui aussi.
${}^{2}C’est du Très-Haut, en effet, qu’il tient son art de guérir,
        et le roi lui-même lui fait des présents.
${}^{3}La science du médecin lui fait porter la tête haute,
        auprès des grands il est admiré.
${}^{4}Le Seigneur a créé les plantes médicinales,
        l’homme avisé ne les méprise pas.
${}^{5}Le bois n’a-t-il pas jadis adouci l’amertume des eaux,
        pour faire connaître par là sa vertu ?
${}^{6}Le Seigneur lui-même a donné la science à des hommes,
        pour qu’ils le glorifient dans ses merveilles.
${}^{7}Le médecin utilise les plantes pour soigner et ôter la douleur,
${}^{8}le pharmacien en fait des préparations.
        \\Ainsi l’œuvre de Dieu ne se termine pas :
        le bien-être qui vient de lui s’étend sur la face de la terre.
        
           
         
${}^{9}Mon fils, quand tu es malade, ne te décourage pas,
        mais prie le Seigneur, et lui te guérira.
${}^{10}Renonce à ta conduite mauvaise, agis avec droiture,
        et, de tout péché, purifie ton cœur.
${}^{11}Offre un encens d’agréable odeur et un mémorial de fleur de farine,
        présente une offrande généreuse, comme si c’était la dernière.
${}^{12}Puis fais venir le médecin :
        le Seigneur l’a créé, lui aussi ;
        \\qu’il ne s’écarte pas de toi,
        car tu as besoin de lui.
${}^{13}Il est des cas où le rétablissement passe par leurs mains :
${}^{14}eux aussi prieront le Seigneur
        \\pour qu’il leur donne le moyen de te soulager
        et la guérison qui te sauvera la vie.
        
           
         
${}^{15}Celui qui pèche à la face de son Créateur,
        qu’il tombe aux mains du médecin !
        
           
${}^{16}Mon fils, répands tes larmes sur un mort ;
        comme un malheureux déchiré de douleur,
        entonne sur lui un chant de deuil ;
        \\donne à son corps la sépulture qui lui est due
        et ne néglige pas sa tombe.
${}^{17}Pleure amèrement, frappe fort ta poitrine ;
        observe ce deuil comme il convient,
        \\un ou deux jours, pour éviter les calomnies ;
        et puis, console-toi de ta peine.
${}^{18}Car la tristesse hâte la mort,
        la tristesse du cœur abat les forces.
${}^{19}Tant que dure le deuil, la tristesse persiste :
        le cœur maudit une vie de misère.
${}^{20}Ne livre pas ton cœur à la tristesse,
        repousse-la : pense que la vie a une fin.
${}^{21}N’oublie pas qu’il n’y a pas de retour ;
        tu te ferais du tort à toi-même sans être utile au défunt.
${}^{22}« Rappelle-toi, dit-il, mon sort sera aussi le tien ;
        moi hier, toi aujourd’hui. »
${}^{23}Quand un mort repose, laisse aussi reposer sa mémoire ;
        console-toi de lui lorsqu’il a rendu l’âme.
${}^{24}La sagesse du scribe s’acquiert à la faveur du loisir ;
        pour devenir sage, il faut avoir peu d’affaires à mener.
         
${}^{25}Comment deviendrait-il sage, celui qui tient la charrue,
        qui met sa fierté à manier l’aiguillon comme une lance,
        \\qui mène ses bœufs, s’absorbe dans leurs travaux
        et ne parle que de son bétail ?
${}^{26}Il met son cœur à tracer des sillons
        et passe ses nuits à donner du fourrage aux génisses.
         
${}^{27}Il en va de même de l’artisan et du maître d’œuvre,
        qui sont occupés de jour comme de nuit ;
        \\de ceux qui gravent la pierre d’un anneau à cacheter
        et qui s’appliquent à en varier les motifs ;
        \\ils ont à cœur de reproduire le modèle
        et passent des nuits pour achever leur ouvrage.
         
${}^{28}Il en va de même du forgeron, toujours à son enclume ;
        il fixe son attention sur le fer qu’il travaille ;
        \\le souffle du feu fait fondre ses chairs,
        il se démène dans la chaleur du fourneau,
        \\le bruit du marteau lui casse les oreilles,
        ses yeux sont rivés sur le modèle de l’objet ;
        \\il met son cœur à parfaire son œuvre
        et passe des nuits à la rendre belle jusqu’à la perfection.
         
${}^{29}Il en va de même du potier, toujours à son ouvrage ;
        il actionne le tour avec ses pieds,
        \\il est en perpétuel souci de son travail
        et tous ses gestes sont comptés :
${}^{30}de ses mains il façonne l’argile,
        il la malaxe avec ses pieds,
        \\il met son cœur à parfaire le vernis,
        il passe des nuits à nettoyer le four.
         
${}^{31}Tous ces gens-là ont mis leur confiance dans leurs mains,
        et chacun possède la sagesse de son métier.
${}^{32}Sans eux on ne bâtirait pas de ville,
        on n’y habiterait pas, on n’y circulerait pas.
        \\Mais lors des délibérations publiques on ne va pas les chercher,
${}^{33}dans l’assemblée ils n’accèdent pas aux places d’honneur,
        \\ils ne siègent pas comme juges,
        ils ne comprennent pas les dispositions du droit.
        \\Ils n’exposent brillamment ni l’enseignement ni le droit,
        on ne les trouve pas méditant des paraboles.
${}^{34}Mais ils consolident la création originelle,
        et leur prière se rapporte aux travaux de leur métier.
        \\Il en va autrement de celui qui s’applique à la loi du Très-Haut
        \\et la médite :
      <p class="cantique introduce_by_colon" id="bib_ct-at_16"><span class="cantique_label">Cantique AT 16</span> = <span class="cantique_ref"><a class="unitex_link" href="#bib_si_39_13">Si 39, 13-16a</a></span>
      
         
      \bchapter{}
        ${}^{1}il cherchera à connaître la sagesse de tous les anciens
        et se consacrera à la lecture des prophètes.
${}^{2}Il retiendra l’histoire des hommes célèbres,
        il pénétrera dans les détours des paraboles,
${}^{3}il cherchera le sens caché des proverbes,
        il retournera dans sa tête les énigmes des paraboles.
${}^{4}Il aura une place au service des grands,
        il se fera remarquer par les chefs,
        il voyagera dans les pays étrangers,
        car il a l’expérience du bien et du mal que font les hommes.
        ${}^{5}Il s’appliquera de tout son cœur
        à servir dès le matin le Seigneur qui l’a créé.
        \\Il présentera sa supplication devant le Très-Haut,
        il ouvrira la bouche pour la prière
        et il suppliera pour ses péchés.
        ${}^{6}Si le Seigneur souverain le veut\\,
        il sera rempli de l’esprit d’intelligence,
        \\il répandra comme une ondée ses paroles de sagesse,
        et dans la prière il rendra grâce au Seigneur.
        ${}^{7}Il suivra le droit chemin
        dans ses décisions comme dans son savoir,
        et il méditera sur les secrets de Dieu.
        ${}^{8}Il fera connaître l’enseignement qu’il a reçu
        et mettra sa fierté
        dans la loi de l’Alliance prescrite\\par le Seigneur.
        ${}^{9}Beaucoup feront l’éloge de son intelligence ;
        et jamais on ne l’oubliera.
        \\Son souvenir ne disparaîtra pas,
        son nom vivra de génération en génération.
        ${}^{10}Les nations raconteront sa sagesse,
        l’assemblée proclamera ses louanges.
        ${}^{11}S’il prolonge sa vie, son nom passe avant mille autres,
        et s’il meurt, son œuvre est accomplie.
        
           
${}^{12}Je veux encore faire part de mes réflexions,
        car j’en suis plein comme est pleine la lune.
       
        ${}^{13}Écoutez-moi, génération de saints\\ :
        croissez comme la rose plantée au bord des eaux.
        ${}^{14}Comme le Liban\\, exhalez votre parfum,
        et fleurissez comme le lis.
        \\Élevez la voix\\, chantez ensemble\\,
        et bénissez le Seigneur en toutes ses œuvres.
        ${}^{15}Rapportez à son nom la grandeur ;
        rendez-lui grâce par la louange,
        par le chant de vos lèvres et vos cithares.
        \\Et vous direz pour rendre grâce :
        ${}^{16}« Les œuvres du Seigneur sont toutes très bonnes,
        \[et chacun de ses ordres s’accomplira en son temps. »\]
         
${}^{17}Il n’y a pas à demander : « Qu’est-ce que ceci ? Pourquoi cela ? »,
        car toute chose vient à point en son temps.
        \\À sa parole, les eaux se sont figées ;
        par un mot de sa bouche, il a mis les eaux en réserve.
${}^{18}À son ordre, tout s’accomplit selon son bon plaisir,
        et nul ne peut freiner sa force de salut.
${}^{19}Les œuvres de tous les mortels sont devant lui,
        rien ne peut rester caché à ses yeux.
${}^{20}Son regard porte du début à la fin des temps ;
        il n’est rien qui puisse l’étonner.
${}^{21}Il n’y a pas à demander : « Qu’est-ce que ceci ? Pourquoi cela ? »,
        car toute chose créée a son utilité.
         
${}^{22}La bénédiction du Seigneur est comme un fleuve qui déborde,
        comme un torrent qui abreuve la terre.
         
${}^{23}Mais les nations auront sa colère en partage,
        comme au jour où il transforma une terre arrosée en désert de sel.
${}^{24}Ses voies sont droites pour les fidèles,
        et pleines d’obstacles pour les gens sans loi.
${}^{25}Dès l’origine les choses bonnes ont été créées pour les bons,
        et les mauvaises pour les pécheurs.
${}^{26}Ce qui est de première nécessité pour la vie de l’homme,
        c’est l’eau, le feu, le fer, le sel,
        \\la fleur de farine de froment, le lait et le miel,
        le sang de la grappe, l’huile et le vêtement.
${}^{27}Toutes ces choses sont pour le bien de qui est religieux,
        mais se changent en mal pour les pécheurs.
         
${}^{28}Certains vents ont été créés pour punir
        et, dans leur fureur, ils multiplient leurs ravages.
        \\Au temps de la destruction, ils déversent leur violence
        et assouvissent la fureur de Celui qui les a faits.
${}^{29}Feu et grêle, famine et peste,
        tout cela a été créé pour punir.
${}^{30}Les dents des fauves, les scorpions, les vipères,
        l’épée vengeresse qui extermine les impies,
${}^{31}se réjouissent tous d’exécuter ses commandements ;
        \\ils sont sur la terre, prêts en cas de besoin,
        et le moment venu, ne désobéiront pas aux ordres.
         
${}^{32}Voilà pourquoi j’avais ma conviction depuis le début,
        j’ai réfléchi et j’ai écrit :
${}^{33}« Toutes les œuvres du Seigneur sont bonnes,
        il pourvoit à tout besoin quand il le faut. »
${}^{34}Il n’y a pas lieu de dire : « Ceci est pire que cela »,
        car toute chose sera appréciée en son temps.
${}^{35}Et maintenant, chantez de tout votre cœur, à pleine voix,
        et bénissez le nom du Seigneur.
      
         
      \bchapter{}
${}^{1}Une grande inquiétude est la part de tout homme,
        un joug pesant accable les fils d’Adam,
        \\depuis le jour où ils sortent du ventre de leur mère
        jusqu’au jour où ils retournent à la mère universelle.
${}^{2}L’objet de leurs réflexions, la crainte de leur cœur,
        c’est la pensée de ce qui les attend, c’est le jour de leur mort.
${}^{3}Depuis celui qui siège sur un trône de gloire
        jusqu’à celui qui est humilié sur la terre et la cendre,
${}^{4}depuis celui qui porte la pourpre et la couronne
        jusqu’à celui qui est vêtu d’étoffe grossière,
${}^{5}ce n’est que colère, jalousie, trouble, agitation,
        crainte de la mort, rancune et discorde.
        \\Et quand on prend du repos sur son lit,
        le sommeil de la nuit perturbe les idées :
${}^{6}à peine est-on assoupi,
        voilà qu’en rêve, comme de jour, on s’épuise
        \\et, troublé par des visions imaginaires,
        on se prend pour un fuyard échappé du combat.
${}^{7}Au moment de la délivrance on se réveille,
        tout étonné de sa vaine frayeur.
${}^{8}Ainsi en va-t-il pour toute chair, depuis l’homme jusqu’à la bête,
        et pour le pécheur sept fois plus encore :
${}^{9}peste, sang, querelle, épée,
        malheurs, famine, destruction et fléaux !
${}^{10}C’est contre les gens sans loi que tout cela a été créé,
        et c’est à cause d’eux que survint le déluge.
${}^{11}Tout ce qui vient de la terre retourne à la terre
        et ce qui vient des eaux fait retour à la mer.
        
           
${}^{12}Tout cadeau corrupteur part en fumée,
        mais la fidélité se maintient à jamais.
${}^{13}La fortune des escrocs s’épuisera comme un cours d’eau en été,
        elle n’est qu’un grand coup de tonnerre dans l’orage.
${}^{14}Aussi vrai qu’ils se sont réjouis de prendre à pleines mains,
        les corrompus iront à la ruine.
${}^{15}Les rejetons des impies ne donnent pas de rameaux,
        leurs racines impures ne trouveront que rochers arides.
${}^{16}Ou bien, comme le roseau qui prolifère au bord de l’eau,
        on les arrachera avant toute autre plante.
${}^{17}Mais la générosité est un jardin de bénédictions,
        et l’aumône demeure à jamais.
         
${}^{18}Celui qui se suffit à lui-même ou celui qui travaille a une vie agréable,
        mais plus encore celui qui trouve un trésor.
${}^{19}Avoir des enfants et fonder une ville, c’est perpétuer son nom,
        mais on estime plus encore la femme irréprochable.
${}^{20}Le vin et la musique réjouissent le cœur,
        mais plus encore l’amour de la sagesse.
${}^{21}La flûte et la harpe donnent du charme à la mélodie,
        mais plus encore une jolie voix.
${}^{22}Grâce et beauté sont le plaisir des yeux,
        mais plus encore la verdure des champs.
${}^{23}Amis et compagnons, c’est à propos qu’ils se retrouvent,
        mais plus encore la femme avec son mari.
${}^{24}Frères et protecteurs sont utiles aux mauvais jours,
        mais plus encore l’aumône qui délivre.
${}^{25}L’or et l’argent affermissent la démarche,
        mais on apprécie plus encore un bon conseil.
${}^{26}La richesse et la force donnent un cœur confiant,
        mais plus encore, la crainte du Seigneur.
        \\Avec la crainte du Seigneur, rien ne manque ;
        avec elle, nul besoin de chercher secours.
${}^{27}La crainte du Seigneur est un jardin de bénédiction,
        plus que toute gloire elle protège.
${}^{28}Ne vis pas de mendicité, mon fils,
        mieux vaut mourir que mendier !
${}^{29}L’homme qui regarde vers la table d’un autre,
        son existence n’est pas une vie.
        \\Il se souille avec la nourriture d’autrui,
        alors qu’un homme avisé, bien éduqué, s’en gardera.
${}^{30}À la bouche de celui qui n’a honte de rien,
        la nourriture mendiée est douce,
        mais elle brûlera ses entrailles.
      
         
      \bchapter{}
${}^{1}Ô mort, quelle amertume, ta pensée
        pour l’homme qui vit paisible au milieu de ses biens,
        \\pour l’homme qui n’a pas de soucis, à qui tout réussit,
        qui est encore capable de faire bonne chère.
${}^{2}Ô mort, ta sentence est bonne
        pour l’homme dans le besoin, dont la force décline,
        \\pour un grand vieillard qui s’inquiète de tout,
        qui se révolte et perd patience.
${}^{3}Ne redoute pas la sentence de la mort :
        pense à ceux qui t’ont précédé, à ceux qui te suivront.
${}^{4}C’est la sentence du Seigneur pour tout être de chair :
        pourquoi refuser la volonté du Très-Haut ?
        \\Dix ans, cent ans ou mille ans :
        au séjour des morts, nul ne met en question la durée d’une vie.
        
           
${}^{5}Les enfants des pécheurs deviennent des enfants abominables,
        qui fréquentent la société des impies.
${}^{6}L’héritage des pécheurs se perdra,
        la honte s’attache à leurs descendants.
${}^{7}Un père impie sera blâmé par ses enfants :
        à cause de lui ils sont couverts de honte.
${}^{8}Malheur à vous, hommes impies,
        qui avez abandonné la loi du Dieu Très-Haut !
${}^{9}Quand vous vous multipliez, c’est pour la perdition,
        quand vous naissez, vous naissez pour la malédiction ;
        quand vous mourez, la malédiction est votre part.
${}^{10}Tout ce qui vient de la terre, à la terre s’en va ;
        ainsi s’en vont les impies, de la malédiction à la perdition.
${}^{11}Le deuil des hommes n’est que deuil du corps,
        mais pour les pécheurs, leur nom même, qui ne vaut rien, disparaîtra.
${}^{12}Prends soin de ton nom, car il te survivra
        plus que mille monceaux d’or.
${}^{13}Une belle vie, on peut en compter les jours,
        mais une bonne renommée demeure à jamais.
${}^{14}Mes enfants, gardez en paix mes instructions.
        \\Une sagesse cachée, un trésor enfoui :
        à quoi peuvent-ils servir l’un et l’autre ?
${}^{15}Mieux vaut cacher sa sottise
        que dissimuler sa sagesse.
${}^{16}Aussi, apprenez de ma bouche de quoi il faut avoir honte,
        car toute honte n’est pas bonne à entretenir
        et tous ne savent pas juger correctement de tout.
${}^{17}Ayez honte, devant père et mère, de la débauche,
        du mensonge, devant le chef et le puissant,
${}^{18}du délit, devant le juge et le magistrat,
        de la transgression de la loi, devant l’assemblée et le peuple,
        \\de la fourberie, devant l’associé et l’ami,
${}^{19}d’un vol, devant tout le voisinage.
        \\Face à la vérité de Dieu et à l’Alliance,
        tu dois avoir honte de garder tout le pain pour toi,
        \\de donner ou de recevoir avec mépris,
${}^{20}de ne pas répondre à ceux qui te saluent,
        \\d’arrêter ton regard sur une prostituée,
${}^{21}de te détourner d’un parent,
        \\d’enlever à autrui ce qui lui revient ou lui est offert,
        de dévisager une femme mariée,
${}^{22}d’être entreprenant avec sa servante
        – et surtout, n’approche pas de son lit !
        \\Tu dois avoir honte d’insulter tes amis
        – n’insulte pas après avoir donné !
      
         
      \bchapter{}
${}^{1}Tu dois avoir honte de répéter une confidence,
        ou de révéler des secrets.
        \\Alors, tu sauras vraiment ce qui doit te faire honte,
        et tu trouveras grâce devant tout le monde.
        
           
         
        \\Mais voici ce qui ne doit pas te faire honte
        ni pécher par respect humain :
${}^{2}observer la loi du Très-Haut et l’Alliance,
        juger l’impie selon le droit,
${}^{3}tenir des comptes avec un associé ou des gens de passage,
        faire profiter tes amis de ton héritage,
${}^{4}avoir balance et poids exacts,
        acquérir des biens grands ou petits,
${}^{5}faire du profit dans le commerce,
        corriger souvent tes enfants,
        et meurtrir les côtes d’un mauvais serviteur.
${}^{6}Si ta femme est malhonnête, il est bon d’avoir un cadenas ;
        et là où beaucoup de mains ont accès, mets les choses sous clef.
${}^{7}Compte et pèse ce que tu fournis,
        mets par écrit tout ce que tu donnes et reçois.
${}^{8}N’aie pas honte de corriger l’imbécile et le sot,
        ou le grand vieillard qui fait la morale aux jeunes.
        \\Alors, tu seras vraiment instruit
        et approuvé de tous.
        
           
${}^{9}Une fille est pour son père cause secrète d’insomnie,
        elle donne des soucis qui ôtent le sommeil :
        \\quand elle est jeune, elle risque de laisser passer son heure ;
        unie à un mari, elle risque de lui devenir odieuse ;
${}^{10}encore vierge, elle risque d’être déflorée
        et de devenir enceinte dans la maison paternelle ;
        \\une fois mariée, elle pourrait être infidèle,
        ou rester stérile après son union.
${}^{11}Si ta fille est aventureuse, monte bonne garde,
        de peur qu’elle ne fasse de toi la risée de tes ennemis,
        \\la fable de la ville, l’objet des commérages,
        et qu’elle ne te déshonore devant tout le monde.
${}^{12}Ne regarde personne pour sa beauté,
        et ne t’assieds pas au milieu des femmes.
${}^{13}Car les vêtements laissent échapper les mites,
        et la femme, une méchanceté de femme.
${}^{14}Mieux vaut la méchanceté d’un homme
        que les faveurs d’une femme ;
        une femme qui apporte le déshonneur expose à l’insulte.
