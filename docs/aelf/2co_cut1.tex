  
  
    
    \bbook{DEUXIÈME LETTRE AUX CORINTHIENS}{DEUXIÈME LETTRE AUX CORINTHIENS}
      
         
      \bchapter{}
        ${}^{1}Paul, apôtre du Christ Jésus
        par la volonté de Dieu,
        \\et Timothée notre frère,
        \\à l’Église de Dieu qui est à Corinthe,
        \\ainsi qu’à tous les fidèles
        qui sont par toute la Grèce.
        ${}^{2}À vous, la grâce et la paix
        \\de la part de Dieu notre Père
        et du Seigneur Jésus Christ.
        
           
${}^{3}Béni soit Dieu, le Père de notre Seigneur Jésus Christ, le Père plein de tendresse, le Dieu de qui vient tout réconfort. 
${}^{4}Dans toutes nos détresses, il nous réconforte ; ainsi, nous pouvons réconforter tous ceux qui sont dans la détresse, grâce au réconfort que nous recevons nous-mêmes de Dieu. 
${}^{5}En effet, de même que nous avons largement part aux souffrances du Christ, de même, par le Christ, nous sommes largement réconfortés. 
${}^{6}Quand nous sommes dans la détresse, c’est pour que vous obteniez le réconfort et le salut ; quand nous sommes réconfortés, c’est encore pour que vous obteniez le réconfort, et cela vous permet de supporter avec persévérance les mêmes souffrances que nous. 
${}^{7}En ce qui vous concerne, nous avons de solides raisons d’espérer, car, nous le savons, de même que vous avez part aux souffrances, de même vous obtiendrez le réconfort.
${}^{8}Nous ne voulons pas vous le laisser ignorer, frères : la détresse que nous avons connue dans la province d’Asie nous a accablés à l’extrême, au-delà de nos forces, au point que nous ne savions même plus si nous allions rester en vie. 
${}^{9}Mais, si nous nous sommes trouvés sous le coup d’un arrêt de mort, c’était pour que notre confiance ne soit plus en nous-mêmes, mais en Dieu qui ressuscite les morts. 
${}^{10}C’est lui qui nous a arrachés à une mort si terrible et qui nous en arrachera ; en lui nous avons l’espérance qu’il nous en arrachera encore, 
${}^{11}avec l’aide que vous nous apportez en priant pour nous ; ainsi, par l’intervention d’un grand nombre de personnes, la grâce que nous aurons reçue sera pour beaucoup de gens une occasion de rendre grâce à notre sujet.
${}^{12}Ce qui fait notre fierté, c’est le témoignage de notre conscience ; nous avons vécu en ce monde, et particulièrement avec vous, dans la simplicité et la sincérité qui viennent de Dieu, non pas selon une sagesse purement humaine, mais selon la grâce de Dieu. 
${}^{13}Nos lettres ne contiennent vraiment rien d’autre que ce que vous pouvez lire et comprendre. J’espère que vous comprendrez entièrement 
${}^{14}ce que vous avez déjà compris en partie, à savoir : nous sommes pour vous un sujet de fierté, de même que vous le serez pour nous au jour du Seigneur Jésus.
${}^{15}Fort de cette assurance, je voulais d’abord aller chez vous pour que vous receviez une nouvelle grâce, 
${}^{16}puis, en passant par chez vous, me rendre en Macédoine, enfin revenir de Macédoine chez vous, et recevoir votre aide pour aller en Judée. 
${}^{17}Vouloir cela, était-ce faire preuve de légèreté ? Ou bien mes projets ne sont-ils que des projets purement humains, si bien qu’il y aurait chez moi en même temps le « oui » et le « non » ? 
${}^{18}En fait, Dieu en est garant, la parole que nous vous adressons n’est pas « oui et non ». 
${}^{19}Car le Fils de Dieu, le Christ Jésus, que nous avons annoncé parmi vous, Silvain et Timothée, avec moi, n’a pas été « oui et non » ; il n’a été que « oui ». 
${}^{20}Et toutes les promesses de Dieu ont trouvé leur « oui » dans sa personne. Aussi est-ce par le Christ que nous disons à Dieu notre « amen », notre « oui », pour sa gloire. 
${}^{21}Celui qui nous rend solides pour le Christ dans nos relations avec vous, celui qui nous a consacrés, c’est Dieu ; 
${}^{22}il nous a marqués de son sceau, et il a mis dans nos cœurs l’Esprit, première avance sur ses dons.
${}^{23}Quant à moi, j’en prends Dieu à témoin sur ma vie : c’est pour vous ménager que je ne suis pas encore revenu à Corinthe. 
${}^{24}Il ne s’agit pas pour nous d’exercer un pouvoir sur votre foi, mais de contribuer à votre joie, car, par la foi, vous tenez bon.
      
         
      \bchapter{}
      \begin{verse}
${}^{1}J’ai pris la décision de ne pas retourner chez vous dans un climat de tristesse. 
${}^{2}Car si moi je vous attriste, qui peut me réjouir, sinon celui que j’ai attristé ? 
${}^{3}Et si j’ai écrit comme je l’ai fait, c’est précisément pour éviter qu’en arrivant je sois attristé par ceux qui auraient dû me donner de la joie ; car je suis convaincu, en ce qui vous concerne, que ma joie est aussi votre joie à tous. 
${}^{4}Ainsi, c’est dans une grande détresse et le cœur serré que je vous ai écrit, et en versant beaucoup de larmes, non pas pour vous attrister, mais pour que vous sachiez quel immense amour j’ai pour vous.
${}^{5}Si quelqu’un a causé de la tristesse, ce n’est pas à moi seul, mais, sans vouloir exagérer, c’est à vous tous, dans une certaine mesure. 
${}^{6}Pour celui-là, la sanction infligée par la majorité doit suffire, 
${}^{7}si bien que vous devez, au contraire, plutôt lui faire grâce et le réconforter, pour éviter qu’il ne sombre dans une tristesse excessive. 
${}^{8}Je vous exhorte donc à faire prévaloir envers lui une attitude de charité. 
${}^{9}Et voici également pourquoi je vous ai écrit : je voulais vérifier si vous êtes, en tout point, obéissants. 
${}^{10}Quand vous faites grâce à quelqu’un, je le fais, moi aussi ; et moi, quand j’ai fait grâce – si j’ai fait grâce en quelque chose – c’était à cause de vous sous le regard du Christ, 
${}^{11}pour ne pas nous laisser dominer par Satan, dont nous connaissons bien les intentions.
${}^{12}Quand je suis arrivé à Troas pour annoncer l’Évangile du Christ, la porte m’était grande ouverte dans le Seigneur ; 
${}^{13}mais je n’ai pas pu avoir l’esprit tranquille, car je ne trouvais pas Tite mon frère ; alors j’ai fait mes adieux, et je suis parti pour la Macédoine.
${}^{14}Rendons grâce à Dieu qui nous entraîne sans cesse en son cortège triomphal dans le Christ, et qui répand par nous en tout lieu le parfum de sa connaissance. 
${}^{15}Car nous sommes pour Dieu la bonne odeur du Christ, parmi ceux qui accueillent le salut comme parmi ceux qui vont à leur perte ; 
${}^{16}pour les uns, c’est un parfum de mort qui conduit à la mort ; pour les autres, un parfum de vie qui conduit à la vie. Et qui donc est capable de cela ? 
${}^{17}En effet, nous ne sommes pas comme tous ces gens qui sont des trafiquants de la parole de Dieu ; au contraire, c’est avec sincérité, c’est de la part de Dieu, et devant Dieu, que dans le Christ nous parlons.
      
         
      \bchapter{}
      \begin{verse}
${}^{1}Allons-nous, une fois de plus, nous recommander nous-mêmes ? Ou alors avons-nous besoin, comme certains, de lettres de recommandation qu’il faudrait vous présenter, ou obtenir de vous ? 
${}^{2}Notre lettre de recommandation, c’est vous, elle est écrite dans nos cœurs, et tout le monde peut en avoir connaissance et la lire. 
${}^{3}De toute évidence, vous êtes cette lettre du Christ, produite par notre ministère, écrite non pas avec de l’encre, mais avec l’Esprit du Dieu vivant, non pas, comme la Loi, sur des tables de pierre, mais sur des tables de chair, sur vos cœurs.
${}^{4}Et si nous avons une telle confiance en Dieu par le Christ, 
${}^{5}ce n’est pas à cause d’une capacité personnelle que nous pourrions nous attribuer : notre capacité vient de Dieu. 
${}^{6}Lui nous a rendus capables d’être les ministres d’une Alliance nouvelle, fondée non pas sur la lettre mais dans l’Esprit ; car la lettre tue, mais l’Esprit donne la vie. 
${}^{7}Le ministère de la mort, celui de la Loi gravée en lettres sur des pierres, avait déjà une telle gloire que les fils d’Israël ne pouvaient pas fixer le visage de Moïse à cause de la gloire, pourtant passagère, qui rayonnait de son visage. 
${}^{8}Combien plus grande alors sera la gloire du ministère de l’Esprit ! 
${}^{9}Le ministère qui entraînait la condamnation, celui de la Loi, était déjà rayonnant de gloire ; combien plus grande sera la gloire du ministère qui fait de nous des justes ! 
${}^{10}Non, vraiment, ce qui, dans une certaine mesure, a été glorieux ne l’est plus, parce qu’il y a maintenant une gloire incomparable. 
${}^{11}Si, en effet, ce qui était passager a connu un moment de gloire, combien plus ce qui demeure restera-t-il dans la gloire !
${}^{12}Et puisque nous avons une telle espérance, c’est avec grande assurance que nous nous comportons ; 
${}^{13}nous ne sommes pas comme Moïse qui mettait un voile sur son visage pour empêcher les fils d’Israël de voir la fin de ce rayonnement passager. 
${}^{14}Mais leurs pensées se sont endurcies. Jusqu’à ce jour, en effet, le même voile demeure quand on lit l’Ancien Testament ; il n’est pas retiré car c’est dans le Christ qu’il disparaît ; 
${}^{15}et aujourd’hui encore, quand les fils d’Israël lisent les livres de Moïse, un voile couvre leur cœur. 
${}^{16}Quand on se convertit au Seigneur, le voile est enlevé. 
${}^{17}Or, le Seigneur, c’est l’Esprit, et là où l’Esprit du Seigneur est présent, là est la liberté. 
${}^{18}Et nous tous qui n’avons pas de voile sur le visage, nous reflétons la gloire du Seigneur, et nous sommes transformés en son image avec une gloire de plus en plus grande, par l’action du Seigneur qui est Esprit.
      
         
      \bchapter{}
      \begin{verse}
${}^{1}C’est pourquoi, ayant reçu ce ministère par la miséricorde de Dieu, nous ne perdons pas courage : 
${}^{2}nous avons rejeté toute dissimulation honteuse, nous n’agissons pas avec ruse, et nous ne falsifions pas la parole de Dieu. Au contraire, nous manifestons la vérité, et ainsi nous nous recommandons nous-mêmes à toute conscience humaine devant Dieu. 
${}^{3}Et même si l’Évangile que nous annonçons reste voilé, il n’est voilé que pour ceux qui vont à leur perte, 
${}^{4}pour les incrédules dont l’intelligence a été aveuglée par le dieu mauvais de ce monde ; celui-ci les empêche de voir clairement, dans la splendeur de l’Évangile, la gloire du Christ, lui qui est l’image de Dieu. 
${}^{5}En effet, ce que nous proclamons, ce n’est pas nous-mêmes ; c’est ceci : Jésus Christ est le Seigneur ; et nous sommes vos serviteurs, à cause de Jésus. 
${}^{6}Car Dieu qui a dit : Du milieu des ténèbres brillera la lumière, a lui-même brillé dans nos cœurs pour faire resplendir la connaissance de sa gloire qui rayonne sur le visage du Christ.
      
         
${}^{7}Mais ce trésor, nous le portons comme dans des vases d’argile ; ainsi, on voit bien que cette puissance extraordinaire appartient à Dieu et ne vient pas de nous. 
${}^{8}En toute circonstance, nous sommes dans la détresse, mais sans être angoissés ; nous sommes déconcertés, mais non désemparés ; 
${}^{9}nous sommes pourchassés, mais non pas abandonnés ; terrassés, mais non pas anéantis. 
${}^{10}Toujours nous portons, dans notre corps, la mort de Jésus, afin que la vie de Jésus, elle aussi, soit manifestée dans notre corps. 
${}^{11}En effet, nous, les vivants, nous sommes continuellement livrés à la mort à cause de Jésus, afin que la vie de Jésus, elle aussi, soit manifestée dans notre condition charnelle vouée à la mort. 
${}^{12}Ainsi la mort fait son œuvre en nous, et la vie en vous. 
${}^{13}L’Écriture dit : J’ai cru, c’est pourquoi j’ai parlé. Et nous aussi, qui avons le même esprit de foi, nous croyons, et c’est pourquoi nous parlons. 
${}^{14}Car, nous le savons, celui qui a ressuscité le Seigneur Jésus nous ressuscitera, nous aussi, avec Jésus, et il nous placera près de lui avec vous. 
${}^{15}Et tout cela, c’est pour vous, afin que la grâce, plus largement répandue dans un plus grand nombre, fasse abonder l’action de grâce pour la gloire de Dieu.
${}^{16}C’est pourquoi nous ne perdons pas courage, et même si en nous l’homme extérieur va vers sa ruine, l’homme intérieur se renouvelle de jour en jour. 
${}^{17}Car notre détresse du moment présent est légère par rapport au poids vraiment incomparable de gloire éternelle qu’elle produit pour nous. 
${}^{18}Et notre regard ne s’attache pas à ce qui se voit, mais à ce qui ne se voit pas ; ce qui se voit est provisoire, mais ce qui ne se voit pas est éternel.
      
         
      \bchapter{}
      \begin{verse}
${}^{1}Nous le savons, en effet, même si notre corps, cette tente qui est notre demeure sur la terre, est détruit, nous avons un édifice construit par Dieu, une demeure éternelle dans les cieux qui n’est pas l’œuvre des hommes. 
${}^{2}En effet, actuellement nous gémissons dans l’ardent désir de revêtir notre demeure céleste par-dessus l’autre, 
${}^{3}si toutefois le Seigneur ne doit pas nous trouver dévêtus mais vêtus de notre corps. 
${}^{4}En effet, nous qui sommes dans cette tente, notre corps, nous sommes accablés et nous gémissons, car nous ne voudrions pas nous dévêtir, mais revêtir un vêtement par-dessus l’autre, pour que notre être mortel soit absorbé par la vie. 
${}^{5}Celui qui nous a formés pour cela même, c’est Dieu, lui qui nous a donné l’Esprit comme première avance sur ses dons.
${}^{6}Ainsi, nous gardons toujours confiance, tout en sachant que nous demeurons loin du Seigneur, tant que nous demeurons dans ce corps ; 
${}^{7}en effet, nous cheminons dans la foi, non dans la claire vision. 
${}^{8}Oui, nous avons confiance, et nous voudrions plutôt quitter la demeure de ce corps pour demeurer près du Seigneur. 
${}^{9}Mais de toute manière, que nous demeurions dans ce corps ou en dehors, notre ambition, c’est de plaire au Seigneur. 
${}^{10}Car il nous faudra tous apparaître à découvert devant le tribunal du Christ, pour que chacun soit rétribué selon ce qu’il a fait, soit en bien soit en mal, pendant qu’il était dans son corps.
${}^{11}Sachant donc ce qu’est la crainte du Seigneur, nous cherchons à convaincre les hommes, et nous sommes à découvert devant Dieu. J’espère bien être aussi à découvert devant vos consciences. 
${}^{12}Il ne s’agit pas de nous recommander à vous une fois de plus, mais de vous donner l’occasion d’être fiers de nous, pour que vous ayez de quoi répondre à ceux qui mettent leur fierté dans les apparences, et non dans le cœur. 
${}^{13}Si nous avons perdu la tête, c’est pour Dieu ; si nous sommes raisonnables, c’est pour vous. 
${}^{14}En effet, l’amour du Christ nous saisit quand nous pensons qu’un seul est mort pour tous, et qu’ainsi tous ont passé par la mort. 
${}^{15}Car le Christ est mort pour tous, afin que les vivants n’aient plus leur vie centrée sur eux-mêmes, mais sur lui, qui est mort et ressuscité pour eux. 
${}^{16}Désormais nous ne regardons plus personne d’une manière simplement humaine : si nous avons connu le Christ de cette manière, maintenant nous ne le connaissons plus ainsi. 
${}^{17}Si donc quelqu’un est dans le Christ, il est une créature nouvelle. Le monde ancien s’en est allé, un monde nouveau est déjà né. 
${}^{18}Tout cela vient de Dieu : il nous a réconciliés avec lui par le Christ, et il nous a donné le ministère de la réconciliation. 
${}^{19}Car c’est bien Dieu qui, dans le Christ, réconciliait le monde avec lui : il n’a pas tenu compte des fautes, et il a déposé en nous la parole de la réconciliation. 
${}^{20}Nous sommes donc les ambassadeurs du Christ, et par nous c’est Dieu lui-même qui lance un appel : nous le demandons au nom du Christ, laissez-vous réconcilier avec Dieu. 
${}^{21}Celui qui n’a pas connu le péché, Dieu l’a pour nous identifié au péché, afin qu’en lui nous devenions justes de la justice même de Dieu.
      
         
      \bchapter{}
      \begin{verse}
${}^{1}En tant que coopérateurs de Dieu, nous vous exhortons encore à ne pas laisser sans effet la grâce reçue de lui. 
${}^{2}Car il dit dans l’Écriture :
        \\Au moment favorable je t’ai exaucé,
        \\au jour du salut je t’ai secouru.
      Le voici maintenant le moment favorable, le voici maintenant le jour du salut. 
${}^{3}Pour que notre ministère ne soit pas exposé à la critique, nous veillons à ne choquer personne en rien. 
${}^{4}Au contraire, en tout, nous nous recommandons nous-mêmes comme des ministres de Dieu :
        \\par beaucoup d’endurance,
        \\dans les détresses, les difficultés, les angoisses,
        ${}^{5}les coups, la prison, les émeutes,
        \\les fatigues, le manque de sommeil et de nourriture,
        ${}^{6}par la chasteté, la connaissance,
        \\la patience et la bonté,
        \\la sainteté de l’esprit et la sincérité de l’amour,
        ${}^{7}par une parole de vérité,
        \\par une puissance qui vient de Dieu ;
        \\nous nous présentons avec les armes de la justice
        \\pour l’attaque et la défense,
        ${}^{8}dans la gloire et le mépris,
        \\dans la mauvaise et la bonne réputation.
        \\On nous traite d’imposteurs, et nous disons la vérité ;
        ${}^{9}on nous prend pour des inconnus, et nous sommes très connus ;
        \\on nous croit mourants, et nous sommes bien vivants ;
        \\on nous punit, et nous ne sommes pas mis à mort ;
        ${}^{10}on nous croit tristes, et nous sommes toujours joyeux ;
        \\pauvres, et nous faisons tant de riches ;
        \\démunis de tout, et nous possédons tout.
${}^{11}Pour vous, Corinthiens, notre bouche a parlé ouvertement, notre cœur s’est élargi ; 
${}^{12}vous n’êtes pas à l’étroit chez nous, c’est en vous-mêmes que vous êtes à l’étroit. 
${}^{13}Je vous le dis parce que vous êtes mes enfants : payez-nous de retour, élargissez votre cœur, vous aussi.
${}^{14}Ne formez pas d’attelage mal assorti avec des non-croyants : quel point commun peut-il y avoir entre la condition du juste et l’impiété ? quelle communion de la lumière avec les ténèbres ? 
${}^{15}quel accord du Christ avec Satan ? ou quel partage pour un croyant avec un non-croyant ? 
${}^{16}quelle entente y a-t-il entre le sanctuaire de Dieu et les idoles ? Nous, en effet, nous sommes le sanctuaire du Dieu vivant, comme Dieu l’a dit lui-même :
        \\J’habiterai et je marcherai parmi eux,
        \\je serai leur Dieu et ils seront mon peuple.
${}^{17}Sortez donc du milieu de ces gens-là
        \\et séparez-vous, – dit le Seigneur ;
        \\ne touchez à rien d’impur,
        \\et moi je vous accueillerai :
${}^{18}je serai pour vous un père,
        \\et vous serez pour moi des fils et des filles,
        \\– dit le Seigneur souverain de l’univers.
      
         
      \bchapter{}
      \begin{verse}
${}^{1}Ayant reçu de telles promesses, mes bien-aimés, purifions-nous donc de toute souillure de la chair et de l’esprit ; achevons de nous sanctifier dans la crainte de Dieu.
${}^{2}Faites-nous bon accueil : nous n’avons fait de tort à personne, nous n’avons corrompu personne, nous n’avons exploité personne. 
${}^{3}Je ne parle pas pour condamner, car – je l’ai déjà dit – vous êtes dans nos cœurs à la vie, à la mort. 
${}^{4}Grande est l’assurance que j’ai devant vous, grande est ma fierté à votre sujet, je me sens pleinement réconforté, je déborde de joie au milieu de toutes nos détresses.
${}^{5}En fait, à notre arrivée en Macédoine, dans notre faiblesse nous n’avons pas eu le moindre répit mais nous étions dans la détresse à tout moment : au-dehors, des conflits, et au-dedans, des craintes. 
${}^{6}Pourtant, Dieu, lui qui réconforte les humbles, nous a réconfortés par la venue de Tite, 
${}^{7}et non seulement par sa venue, mais par le réconfort qu’il avait trouvé chez vous : il nous a fait part de votre grand désir de nous revoir, de votre désolation, de votre zèle pour moi, et cela m’a donné encore plus de joie. 
${}^{8}En effet, même si je vous ai attristés par ma lettre, je ne le regrette pas ; et même si j’ai pu le regretter – car je vois bien que cette lettre vous a attristés, au moins pour un moment –, 
${}^{9}je me réjouis maintenant, non de ce que vous avez été attristés, mais parce que cette tristesse vous a conduits au repentir. En effet, elle a été vécue selon Dieu, si bien que vous n’avez subi aucun dommage à cause de nous. 
${}^{10}Car une tristesse vécue selon Dieu produit un repentir qui mène au salut, sans causer de regrets, tandis que la tristesse selon le monde produit la mort. 
${}^{11}Mais la tristesse vécue selon Dieu, voyez ce qu’elle a produit chez vous. Quel empressement ! Quelles excuses ! Quelle indignation ! Quelle crainte ! Quel désir ! Quel zèle ! Quelle juste punition ! En tous points, vous avez prouvé que vous étiez irréprochables dans cette affaire. 
${}^{12}Bref, même si je vous ai écrit, ce n’est pas à cause de l’offenseur ni à cause de l’offensé, mais pour rendre manifeste à vos yeux devant Dieu l’empressement que vous avez pour nous. 
${}^{13}Voilà ce qui fait notre réconfort. En plus de ce réconfort, nous nous sommes réjouis, encore bien davantage, en voyant la joie de Tite : son esprit a été pleinement tranquillisé par vous tous. 
${}^{14}Devant lui j’avais montré quelque fierté à votre sujet, et je n’ai pas eu à en rougir ; à vous, nous avons toujours parlé en vérité ; de même, notre fierté devant Tite est apparue fondée en vérité. 
${}^{15}Et sa tendresse à votre égard grandit encore quand il se souvient de votre obéissance à tous, comment vous l’avez accueilli avec crainte et profond respect. 
${}^{16}Quelle joie pour moi d’avoir pleine confiance en vous !
      
         
      \bchapter{}
      \begin{verse}
${}^{1}Frères, nous voulons vous faire connaître la grâce que Dieu a accordée aux Églises de Macédoine. 
${}^{2}Dans les multiples détresses qui les mettaient à l’épreuve, l’abondance de leur joie et leur extrême pauvreté ont débordé en trésors de générosité. 
${}^{3}Ils y ont mis tous leurs moyens, et davantage même, j’en suis témoin ; spontanément, 
${}^{4}avec grande insistance, ils nous ont demandé comme une grâce de pouvoir s’unir à nous pour aider les fidèles de Jérusalem. 
${}^{5}Au-delà même de nos espérances, ils se sont eux-mêmes donnés d’abord au Seigneur, et ensuite à nous, par la volonté de Dieu. 
${}^{6}Et comme Tite avait déjà commencé, chez vous, cette œuvre généreuse, nous lui avons demandé d’aller jusqu’au bout. 
${}^{7}Puisque vous avez tout en abondance, la foi, la Parole, la connaissance de Dieu, toute sorte d’empressement et l’amour qui vous vient de nous, qu’il y ait aussi abondance dans votre don généreux ! 
${}^{8}Ce n’est pas un ordre que je donne, mais je parle de l’empressement des autres pour vérifier l’authenticité de votre charité. 
${}^{9}Vous connaissez en effet le don généreux de notre Seigneur Jésus Christ : lui qui est riche, il s’est fait pauvre à cause de vous, pour que vous deveniez riches par sa pauvreté.
${}^{10}Au sujet de cette collecte, je donne mon avis, car cela vous est utile, à vous qui, dès l’année dernière, avez pris l’initiative non seulement de la réaliser, mais encore de la décider. 
${}^{11}Et maintenant, allez jusqu’au bout de la réalisation : comme vous avez mis votre ardeur à prendre cette décision, ainsi vous irez jusqu’au bout, selon vos moyens. 
${}^{12}Car s’il y a de l’ardeur, on est bien reçu avec ce que l’on a, peu importe ce que l’on n’a pas. 
${}^{13}Il ne s’agit pas de vous mettre dans la gêne en soulageant les autres, il s’agit d’égalité. 
${}^{14}Dans la circonstance présente, ce que vous avez en abondance comblera leurs besoins, afin que, réciproquement, ce qu’ils ont en abondance puisse combler vos besoins, et cela fera l’égalité, 
${}^{15}comme dit l’Écriture à propos de la manne : Celui qui en avait ramassé beaucoup n’eut rien de trop, celui qui en avait ramassé peu ne manqua de rien.
${}^{16}Je rends grâce à Dieu qui a mis dans le cœur de Tite le même empressement à votre égard : 
${}^{17}il a accueilli notre demande, et il a été tellement empressé qu’il est parti chez vous spontanément. 
${}^{18}Nous avons envoyé avec lui le frère dont toutes les Églises chantent la louange à cause de son annonce de l’Évangile 
${}^{19}– ajoutons que ce frère a été désigné par les Églises pour être notre compagnon de voyage, dans cette œuvre de bonté, ce service que nous accomplissons pour la gloire du Seigneur et selon notre ardent désir. 
${}^{20}Nous voulons par là éviter tout reproche à cause des grosses sommes dont nous assurons le service ; 
${}^{21}en effet, nous nous appliquons à bien agir, non seulement aux yeux du Seigneur, mais aussi aux yeux des hommes. 
${}^{22}Nous avons encore envoyé avec eux un autre de nos frères dont nous avons souvent, en bien des cas, vérifié l’empressement, un empressement encore plus fort aujourd’hui à cause de la grande confiance qu’il a en vous. 
${}^{23}En ce qui concerne Tite, c’est mon compagnon et mon collaborateur auprès de vous ; quant à nos frères, ils sont les envoyés des Églises, ils sont la gloire du Christ. 
${}^{24}Donnez-leur donc, à la face des Églises, la preuve de votre amour, de ce qui fait ma fierté à votre sujet.
      
         
      \bchapter{}
      \begin{verse}
${}^{1}Au sujet du service destiné aux fidèles de Jérusalem, je n’ai plus besoin de vous écrire, 
${}^{2}car je connais votre ardeur et, pour vous, j’en tire fierté devant les Macédoniens. Je leur dis que la Grèce se tient prête depuis l’an dernier, et votre zèle a stimulé la plupart d’entre eux. 
${}^{3}Je vous envoie cependant les frères pour que la fierté que nous mettons en vous ne soit pas, sur ce point-là, vidée de son sens ; je vous les envoie pour que vous vous teniez prêts comme je le disais, 
${}^{4}et pour éviter que, si jamais des Macédoniens viennent avec moi et ne vous trouvent pas prêts, cette situation ne tourne à notre honte – sans parler de la vôtre ! 
${}^{5}J’ai donc estimé nécessaire d’inviter les frères à nous devancer chez vous, et à organiser d’avance votre largesse, promise depuis longtemps : ainsi, quand elle sera préparée, ce sera une vraie largesse, et non une mesquinerie. 
${}^{6}Rappelez-vous le proverbe : “À semer trop peu, on récolte trop peu ; à semer largement, on récolte largement”. 
${}^{7}Que chacun donne comme il a décidé dans son cœur, sans regret et sans contrainte, car Dieu aime celui qui donne joyeusement. 
${}^{8}Et Dieu est assez puissant pour vous donner toute grâce en abondance, afin que vous ayez, en toute chose et toujours, tout ce qu’il vous faut, et même que vous ayez en abondance de quoi faire toute sorte de bien. 
${}^{9}L’Écriture dit en effet de l’homme juste :
        \\Il distribue, il donne aux pauvres ;
        \\sa justice demeure à jamais.
${}^{10}Dieu, qui fournit la semence au semeur et le pain pour la nourriture, vous fournira la graine ; il la multipliera, il donnera la croissance à ce que vous accomplirez dans la justice. 
${}^{11}Il vous rendra riches en générosité de toute sorte, ce qui suscitera notre action de grâce envers Dieu. 
${}^{12}Car notre collecte est un ministère qui ne comble pas seulement les besoins des fidèles de Jérusalem, mais déborde aussi en une multitude d’actions de grâce envers Dieu. 
${}^{13}Les fidèles apprécieront ce ministère à sa valeur, et ils rendront gloire à Dieu pour cette soumission avec laquelle vous professez l’Évangile du Christ, et pour la générosité qui vous met en communion avec eux et avec tous. 
${}^{14}En priant pour vous, ils vous manifesteront leur attachement à cause de la grâce incomparable que Dieu vous a faite. 
${}^{15}Rendons grâce à Dieu pour le don ineffable qu’il nous fait.
      
         
      \bchapter{}
      \begin{verse}
${}^{1}Moi-même, Paul, je vous exhorte par la douceur et la bienveillance du Christ, moi si humble quand je suis devant vous, mais plein d’assurance à votre égard quand je n’y suis pas. 
${}^{2}Je vous en prie, ne m’obligez pas à montrer, quand je viendrai, l’assurance et l’audace dont je prétends bien faire preuve contre ceux qui prétendent que nous avons une conduite purement humaine. 
${}^{3}Notre conduite est bien une conduite d’homme, mais nous ne combattons pas de manière purement humaine. 
${}^{4}En effet, les armes de notre combat ne sont pas purement humaines, elles reçoivent de Dieu la puissance qui démolit les forteresses. Nous démolissons les raisonnements fallacieux, 
${}^{5}tout ce qui, de manière hautaine, s’élève contre la connaissance de Dieu, et nous capturons toute pensée pour l’amener à obéir au Christ. 
${}^{6}Nous sommes prêts à sévir contre toute désobéissance, dès que votre obéissance à vous sera parfaite.
${}^{7}Regardez les choses en face. Si quelqu’un est convaincu d’appartenir au Christ, qu’il tienne compte encore de ceci : comme lui-même appartient au Christ, nous également. 
${}^{8}Même si je suis un peu trop fier de l’autorité que le Seigneur nous a donnée sur vous pour construire et non pour démolir, je n’aurai pas à en rougir. 
${}^{9}Je ne veux pas avoir l’air de vous effrayer par mes lettres. 
${}^{10}« Les lettres ont du poids, dit-on, et de la force, mais sa présence physique est sans vigueur, et sa parole est nulle. » 
${}^{11}Celui qui parle ainsi, qu’il tienne bien compte de ceci : tels nous sommes en paroles par nos lettres quand nous ne sommes pas là, tels nous serons encore en actes quand nous serons présents.
${}^{12}Nous n’oserions pas nous égaler ou nous comparer à des gens qui se recommandent eux-mêmes. Lorsqu’ils se prennent eux-mêmes comme unité de mesure et comme norme de comparaison, ils sont sans intelligence. 
${}^{13}Nous n’allons pas nous vanter démesurément, mais nous garderons la mesure du domaine d’activité que Dieu nous a attribué en nous faisant parvenir aussi jusqu’à vous. 
${}^{14}En effet, nous ne dépassons pas nos limites comme ce serait le cas si nous n’étions pas parvenus chez vous ; car, en fait, c’est bien jusqu’à vous que nous sommes arrivés pour annoncer l’Évangile du Christ. 
${}^{15}Nous ne tirons pas du labeur des autres l’occasion de nous vanter démesurément, mais, avec la croissance de votre foi, nous espérons voir honorer de plus en plus notre ministère auprès de vous, sans quitter notre domaine, 
${}^{16}et porter l’Évangile au-delà de chez vous, sans nous vanter de travaux déjà faits sur le domaine des autres. 
${}^{17}Celui qui veut être fier, qu’il mette sa fierté dans le Seigneur. 
${}^{18}Celui dont on reconnaît la valeur n’est pas celui qui se recommande lui-même, c’est celui que le Seigneur recommande.
      
         
      \bchapter{}
      \begin{verse}
${}^{1}Pourriez-vous supporter de ma part un peu de folie ? Oui, de ma part, vous allez le supporter, 
${}^{2}à cause de mon amour jaloux qui est l’amour même de Dieu pour vous. Car je vous ai unis au seul Époux : vous êtes la vierge pure que j’ai présentée au Christ. 
${}^{3}Mais j’ai bien peur qu’à l’exemple d’Ève séduite par la ruse du serpent, votre intelligence des choses ne se corrompe en perdant la simplicité et la pureté qu’il faut avoir à l’égard du Christ. 
${}^{4}En effet, si le premier venu vous annonce un autre Jésus, un Jésus que nous n’avons pas annoncé, si vous recevez un esprit différent de celui que vous avez reçu, ou un Évangile différent de celui que vous avez accueilli, vous le supportez fort bien !
${}^{5}J’estime, moi, que je ne suis inférieur en rien à tous ces super-apôtres. 
${}^{6}Je ne vaux peut-être pas grand-chose pour les discours, mais pour la connaissance de Dieu, c’est différent : nous vous l’avons montré en toute occasion et de toutes les façons. 
${}^{7}Aurais-je commis une faute lorsque, m’abaissant pour vous élever, je vous ai annoncé l’Évangile de Dieu gratuitement ? 
${}^{8}J’ai appauvri d’autres Églises en recevant d’elles l’argent nécessaire pour me mettre à votre service. 
${}^{9}Quand j’étais chez vous, et que je me suis trouvé dans le besoin, je n’ai été à charge de personne ; en effet, pour m’apporter ce dont j’avais besoin, des frères sont venus de Macédoine. En toute occasion, je me suis gardé d’être un poids pour vous, et je m’en garderai toujours. 
${}^{10}Aussi sûrement que la vérité du Christ est en moi, ce motif de fierté ne me sera enlevé dans aucune des régions de la Grèce. 
${}^{11}Pourquoi donc me comporter ainsi ? Serait-ce parce que je ne vous aime pas ? Mais si ! Et Dieu le sait.
${}^{12}Ce que je fais, je le ferai encore, afin d’enlever tout prétexte à ceux qui en cherchent un pour pouvoir se vanter d’être reconnus comme nos égaux. 
${}^{13}Ces sortes de gens sont de faux apôtres, des fraudeurs, qui se déguisent en apôtres du Christ. 
${}^{14}Cela n’a rien d’étonnant : Satan lui-même se déguise en ange de lumière. 
${}^{15}Il n’est donc pas surprenant que ses serviteurs aussi se déguisent en serviteurs de la justice de Dieu ; ils auront une fin conforme à leurs œuvres.
${}^{16}Je le dis de nouveau : que personne ne me prenne pour un insensé ; ou alors, accueillez-moi comme si j’étais un insensé, pour que je puisse à mon tour me vanter un peu. 
${}^{17}Ce que je vais dire, je ne le dirai pas selon le Seigneur, mais comme dans un accès de folie, puisqu’il s’agit de se vanter. 
${}^{18}Tant d’autres se vantent à la manière humaine ; eh bien, je vais, moi aussi, me vanter. 
${}^{19}Vous supportez volontiers les insensés, vous qui êtes sensés ; 
${}^{20}vous supportez d’être traités en esclaves, d’être dévorés, dépouillés, regardés de haut, frappés au visage. 
${}^{21}J’ai honte de le dire : c’est à croire que nous avons été bien faibles avec vous.
      Si certains ont de l’audace – je parle dans un accès de folie –, j’ai de l’audace, moi aussi.
        ${}^{22}Ils sont Hébreux ? Moi aussi.
        \\Ils sont Israélites ? Moi aussi.
        \\Ils sont de la descendance d’Abraham ? Moi aussi.
        ${}^{23}Ils sont ministres du Christ ?
        \\Eh bien – je vais dire une folie – 
        \\moi, je le suis davantage :
        \\dans les fatigues, bien plus ;
        \\dans les prisons, bien plus ;
        \\sous les coups, largement plus ;
        \\en danger de mort, très souvent.
        ${}^{24}Cinq fois, j’ai reçu des Juifs les trente-neuf coups de fouet ;
        ${}^{25}trois fois, j’ai subi la bastonnade ;
        \\une fois, j’ai été lapidé ;
        \\trois fois, j’ai fait naufrage
        \\et je suis resté vingt-quatre heures perdu en pleine mer.
        ${}^{26}Souvent à pied sur les routes,
        \\avec les dangers des fleuves,
        \\les dangers des bandits,
        \\les dangers venant de mes frères de race,
        \\les dangers venant des païens,
        \\les dangers de la ville,
        \\les dangers du désert,
        \\les dangers de la mer,
        \\les dangers des faux frères.
        ${}^{27}J’ai connu la fatigue et la peine,
        \\souvent le manque de sommeil
        \\la faim et la soif,
        \\souvent le manque de nourriture,
        \\le froid et le manque de vêtements,
        ${}^{28}sans compter tout le reste :
        \\ma préoccupation quotidienne, le souci de toutes les Églises.
${}^{29}« Qui donc faiblit, sans que je partage sa faiblesse ? Qui vient à tomber, sans que cela me brûle ? »
${}^{30}S’il faut se vanter, je me vanterai de ce qui fait ma faiblesse. 
${}^{31}Le Dieu et Père du Seigneur Jésus sait que je ne mens pas, lui qui est béni pour les siècles. 
${}^{32}À Damas, le représentant du roi Arétas faisait garder la ville pour s’emparer de moi ; 
${}^{33}on m’a fait descendre par une fenêtre, dans un panier, de l’autre côté du rempart, et j’ai échappé à ses mains.
      
         
      \bchapter{}
      \begin{verse}
${}^{1}Faut-il se vanter ? Ce n’est pas utile. J’en viendrai pourtant aux visions et aux révélations reçues du Seigneur. 
${}^{2}Je sais qu’un fidèle du Christ, voici quatorze ans, a été emporté jusqu’au troisième ciel – est-ce dans son corps ? je ne sais pas ; est-ce hors de son corps ? je ne sais pas ; Dieu le sait – ; 
${}^{3}mais je sais que cet homme dans cet état-là – est-ce dans son corps, est-ce sans son corps ? je ne sais pas, Dieu le sait – 
${}^{4}cet homme-là a été emporté au paradis et il a entendu des paroles ineffables, qu’un homme ne doit pas redire. 
${}^{5}D’un tel homme, je peux me vanter, mais pour moi-même, je ne me vanterai que de mes faiblesses. 
${}^{6}En fait, si je voulais me vanter, ce ne serait pas folie, car je ne dirais que la vérité. Mais j’évite de le faire, pour qu’on n’ait pas de moi une idée plus favorable qu’en me voyant ou en m’écoutant. 
${}^{7}Et ces révélations dont il s’agit sont tellement extraordinaires que, pour m’empêcher de me surestimer, j’ai reçu dans ma chair une écharde, un envoyé de Satan qui est là pour me gifler, pour empêcher que je me surestime. 
${}^{8}Par trois fois, j’ai prié le Seigneur de l’écarter de moi. 
${}^{9}Mais il m’a déclaré : « Ma grâce te suffit, car ma puissance donne toute sa mesure dans la faiblesse. » C’est donc très volontiers que je mettrai plutôt ma fierté dans mes faiblesses, afin que la puissance du Christ fasse en moi sa demeure. 
${}^{10}C’est pourquoi j’accepte de grand cœur pour le Christ les faiblesses, les insultes, les contraintes, les persécutions et les situations angoissantes. Car, lorsque je suis faible, c’est alors que je suis fort.
${}^{11}Me voilà devenu insensé : c’est vous qui m’y avez forcé ! J’aurais dû plutôt être recommandé par vous ; en effet, je n’ai été en rien inférieur à ces super-apôtres, quoique je ne sois rien. 
${}^{12}Les signes auxquels on reconnaît l’apôtre ont été mis en œuvre chez vous : toute cette persévérance, tant de signes, de prodiges, de miracles ! 
${}^{13}Que vous a-t-il manqué par rapport aux autres Églises, sinon que moi, je ne vous ai pas été à charge ? Pardonnez-moi cette injustice !
${}^{14}Me voici prêt à venir chez vous pour la troisième fois, et je ne vous serai pas à charge, car ce que je cherche, ce n’est pas vos biens, mais vous-mêmes. En effet, ce ne sont pas les enfants qui doivent mettre de l’argent de côté pour leurs parents, mais les parents pour leurs enfants. 
${}^{15}Et moi, je serai très heureux de dépenser et de me dépenser tout entier pour vous. Si je vous aime davantage, faut-il qu’en retour je sois moins aimé ? 
${}^{16}Quelques-uns diront que, certes, je n’ai pas été un poids pour vous, mais que je suis un fourbe et que je vous ai pris par ruse. 
${}^{17}Vous ai-je exploités par un de ceux que je vous ai envoyés ? 
${}^{18}J’ai fait appel à Tite, et j’ai envoyé avec lui le frère dont j’ai parlé : Tite vous a-t-il exploités ? N’avons-nous pas marché dans le même esprit ? sur les mêmes traces ?
${}^{19}Depuis un moment, vous pensez que nous vous présentons notre défense. Or, c’est devant Dieu, dans le Christ, que nous parlons. Et tout cela, mes bien-aimés, c’est pour vous construire. 
${}^{20}Car je crains qu’en arrivant, je ne vous trouve pas comme je voudrais, et que vous ne me trouviez pas comme vous voudriez ; je crains qu’il n’y ait rivalité, jalousie, emportements, intrigues, médisance, dénigrement, insolence, désordre ; 
${}^{21}je crains qu’à mon arrivée mon Dieu ne m’humilie à nouveau devant vous, et que je n’aie à pleurer sur bien des gens qui ont été autrefois dans le péché et qui ne se sont pas repentis de l’impureté, de l’inconduite et de la débauche qu’ils ont pratiquées.
      
         
      \bchapter{}
      \begin{verse}
${}^{1}Voici que je vais venir chez vous pour la troisième fois. Toute affaire sera réglée sur la parole de deux ou trois témoins. 
${}^{2}Ceux qui ont été autrefois dans le péché et tous les autres, je les ai déjà prévenus lors de mon deuxième passage, et je les préviens maintenant que je ne suis pas là : si je reviens, j’agirai sans ménagement 
${}^{3}puisque vous cherchez à vérifier si vraiment le Christ parle en moi ; lui, il n’est pas faible à votre égard, mais il montre en vous sa puissance. 
${}^{4}Certes, il a été crucifié du fait de sa faiblesse, mais il est vivant par la puissance de Dieu. Et nous, maintenant, nous sommes faibles en lui ; mais, avec lui, nous serons vraiment vivants par la puissance de Dieu qui s’exercera envers vous.
${}^{5}Mettez-vous donc vous-mêmes à l’épreuve, pour voir si vous êtes dans la foi ; examinez-vous. Peut-être ne reconnaissez-vous pas que Jésus Christ est en vous ? Dans ce cas, vous êtes disqualifiés. 
${}^{6}J’espère que vous reconnaîtrez que nous, nous ne le sommes pas. 
${}^{7}Dans notre prière, nous demandons à Dieu que vous ne commettiez aucun mal ; nous ne le faisons pas pour mettre en évidence notre propre qualification, mais pour que vous, vous fassiez le bien, et que nous soyons, nous, comme disqualifiés. 
${}^{8}Car nous n’avons aucun pouvoir contre la vérité, nous en avons seulement pour la vérité. 
${}^{9}En effet, nous nous réjouissons chaque fois que nous sommes faibles, tandis que vous êtes forts. Ce que nous demandons dans notre prière, c’est que vous avanciez vers la perfection. 
${}^{10}Voici pourquoi je vous écris cela, maintenant que je suis absent : c’est pour n’avoir pas à utiliser avec rigueur, quand je serai présent, le pouvoir que le Seigneur m’a donné en vue de construire et non de démolir.
${}^{11}Enfin, frères, soyez dans la joie, cherchez la perfection, encouragez-vous, soyez d’accord entre vous, vivez en paix, et le Dieu d’amour et de paix sera avec vous. 
${}^{12}Saluez-vous les uns les autres par un baiser de paix. Tous les fidèles vous saluent.
${}^{13}Que la grâce du Seigneur Jésus Christ, l’amour de Dieu et la communion du Saint-Esprit soient avec vous tous.
