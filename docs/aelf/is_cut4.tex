  
  
      <p class="cantique" id="bib_ct-at_24"><span class="cantique_label">Cantique AT 24</span> = <span class="cantique_ref"><a class="unitex_link" href="#bib_is_40_1">Is 40, 1-8</a></span>
      <p class="cantique" id="bib_ct-at_25"><span class="cantique_label">Cantique AT 25</span> = <span class="cantique_ref"><a class="unitex_link" href="#bib_is_40_10">Is 40, 10-17</a></span>
      
         
      \bchapter{}
        ${}^{1}Consolez, consolez mon peuple,
        – dit votre Dieu –
        ${}^{2}parlez au cœur de Jérusalem.
        \\Proclamez que son service\\est accompli,
        que son crime est expié,
        \\qu’elle a reçu de la main du Seigneur
        le double pour toutes ses fautes.
        
           
         
        ${}^{3}Une voix proclame :
        « Dans le désert, préparez le chemin du Seigneur ;
        \\tracez droit, dans les terres arides,
        une route pour notre Dieu.
        ${}^{4}Que tout ravin soit comblé,
        toute montagne et toute\\colline abaissées !
        \\que les escarpements se changent en plaine,
        et les sommets, en large vallée !
        ${}^{5}Alors se révélera la gloire du Seigneur,
        \\et tout être de chair verra
        que la bouche du Seigneur a parlé. »
        
           
         
        ${}^{6}Une voix dit : « Proclame ! »
        Et je dis\\ : « Que vais-je proclamer ? »
        \\Toute chair est comme l’herbe,
        toute sa grâce\\, comme la fleur des champs :
        ${}^{7}l’herbe se dessèche et la fleur se fane
        quand passe sur elle le souffle du Seigneur.
        \\Oui, le peuple est comme l’herbe :
        ${}^{8}l’herbe se dessèche et la fleur se fane,
        \\mais la parole de notre Dieu
        demeure pour toujours.
        
           
        ${}^{9}Monte sur une haute montagne,
        toi qui portes la bonne nouvelle à Sion.
        \\Élève la voix avec force,
        toi qui portes la bonne nouvelle à Jérusalem\\.
        \\Élève la voix\\, ne crains pas.
        Dis aux villes de Juda :
        \\« Voici votre Dieu ! »
         
        ${}^{10}Voici le Seigneur Dieu !
        \\Il vient avec puissance ;
        son bras lui soumet tout\\.
        \\Voici le fruit de son travail avec lui,
        et devant lui, son ouvrage.
         
        ${}^{11}Comme un berger, il fait paître son troupeau :
        son bras rassemble les agneaux,
        \\il les porte sur son cœur\\,
        il mène les brebis qui allaitent.
        ${}^{12}Qui a jaugé les eaux des mers\\dans le creux de sa main,
        et, de ses doigts\\, mesuré les cieux,
        \\évalué en boisseaux la poussière de la terre,
        pesé les montagnes au crochet
        et les collines sur la balance ?
         
        ${}^{13}Qui a mesuré l’esprit du Seigneur ?
        Qui l’a conseillé pour l’instruire ?
        ${}^{14}De qui a-t-il pris conseil pour discerner,
        pour apprendre les chemins du jugement,
        \\pour acquérir le savoir
        et s’instruire des voies de l’intelligence ?
         
        ${}^{15}Voici les nations :
        \\elles sont\\pour lui comme une goutte au bord d’un seau,
        un grain de sable sur le plateau de la balance !
        \\Voici les îles,
        comme une poussière qu’il soulève !
         
        ${}^{16}Le Liban ne pourrait suffire au feu,
        ni ses animaux, suffire à l’holocauste.
        ${}^{17}Toutes les nations, devant lui, sont comme rien,
        moins que vide\\et néant pour lui.
         
${}^{18}À qui pourriez-vous comparer Dieu,
        quelle forme lui donneriez-vous ?
${}^{19}L’idole, c’est un artisan qui l’a fondue ;
        un orfèvre plaque sur elle de l’or
        et fabrique pour elle des chaînettes d’argent.
${}^{20}Le pauvre, pour ses dévotions,
        choisit du bois imputrescible ;
        \\il cherche un artisan habile
        pour fixer une idole qui ne vacille pas.
         
${}^{21}Ne savez-vous pas, n’avez-vous pas entendu,
        ne vous a-t-on pas annoncé dès le commencement,
        \\n’avez-vous pas compris comment la terre a été fondée ?
${}^{22}Il habite au-dessus de la voûte qui couvre la terre
        dont les habitants semblent des sauterelles.
        \\Comme une toile, il a tendu les cieux,
        il les a dépliés comme une tente d’habitation.
         
${}^{23}Il a réduit à rien les grands,
        et à néant, les juges de la terre.
${}^{24}Pas même plantés, pas même semés,
        leur tige n’ayant pas même pris racine en terre,
        \\il souffle sur eux, les voilà qui se dessèchent,
        et le tourbillon les enlève comme de la paille.
         
        ${}^{25}À qui pourriez-vous me comparer,
        qui pourrait être mon égal ?
        – dit le Dieu\\Saint.
        ${}^{26}Levez les yeux et regardez :
        qui a créé tout cela ?
        \\Celui qui déploie toute l’armée des étoiles\\,
        et les appelle chacune par son nom.
        \\Si grande est sa force, et telle est sa puissance
        que pas une seule ne manque.
         
        ${}^{27}Jacob, pourquoi dis-tu,
        Israël, pourquoi affirmes-tu :
        \\« Mon chemin est caché au Seigneur,
        mon droit échappe à mon Dieu » ?
        ${}^{28}Tu ne le sais donc pas, tu ne l’as pas entendu ?
        \\Le Seigneur est le Dieu éternel,
        il crée jusqu’aux extrémités de la terre,
        \\il ne se fatigue pas, ne se lasse pas.
        Son intelligence est insondable.
        ${}^{29}Il rend des forces à l’homme fatigué,
        il augmente la vigueur de celui qui est faible.
        ${}^{30}Les garçons se fatiguent, se lassent,
        et les jeunes gens ne cessent de trébucher,
        ${}^{31}mais ceux qui mettent leur espérance dans le Seigneur
        trouvent des forces nouvelles ;
        \\ils déploient comme des ailes d’aigles,
        ils courent sans se lasser,
        ils marchent sans se fatiguer.
      
         
      \bchapter{}
${}^{1}Vous les îles, faites silence devant moi,
        que les peuples trouvent des forces nouvelles,
        \\qu’ils s’avancent, et qu’ils parlent ;
        approchons ensemble pour le jugement.
${}^{2}Qui a fait surgir de l’Orient
        celui que la victoire rencontre à chaque pas ?
        \\Qui lui donne des nations
        et lui soumet des rois ?
        \\Son épée les réduit en poussière,
        et son arc, en paille qui vole.
${}^{3}Il les poursuit, il passe en toute sécurité ;
        ses pas ne font qu’effleurer le chemin.
${}^{4}Qui a fait cela, qui l’a réalisé ?
        \\Celui qui dès le commencement
        appelle les générations.
        \\Moi, le Seigneur, Je suis le premier
        et, avec les derniers, encore, Je suis.
${}^{5}Les îles ont vu, elles prennent peur,
        les extrémités de la terre frémissent :
        « Ils approchent, ils arrivent ! »
        
           
         
${}^{6}Chacun aide son compagnon,
        il dit à son frère : « Courage ! »
${}^{7}Le ciseleur encourage l’orfèvre,
        le chaudronnier encourage le forgeron;
        \\il dit de la soudure : « Elle est bonne »,
        il la renforce de clous pour qu’elle ne bouge pas.
        
           
        ${}^{8}Toi, Israël, mon serviteur,
        Jacob que j’ai choisi,
        descendance d’Abraham mon ami :
        ${}^{9}aux extrémités de la terre je t’ai saisi,
        du bout du monde\\je t’ai appelé ;
        \\je t’ai dit : Tu es mon serviteur,
        je t’ai choisi, je ne t’ai pas rejeté.
        ${}^{10}Ne crains pas : je suis avec toi ;
        ne sois pas troublé\\ : je suis ton Dieu.
        \\Je t’affermis ; oui, je t’aide,
        je\\te soutiens de ma main victorieuse\\.
${}^{11}Les voici honteux et confus
        tous ceux qui s’enflamment contre toi ;
        \\ils ne seront plus rien, ils périront,
        ceux qui te querellent.
${}^{12}Tu les chercheras, tu ne les trouveras pas,
        ceux qui te combattent ;
        \\ils seront comme rien, comme néant,
        ceux qui te font la guerre.
        ${}^{13}C’est moi, le Seigneur ton Dieu,
        qui saisis ta main droite,
        \\et qui te dis :
        « Ne crains pas, moi, je viens à ton aide. »
        ${}^{14}Ne crains pas, Jacob, pauvre vermisseau\\,
        Israël, pauvre mortel\\.
        \\Je viens à ton aide – oracle du Seigneur ;
        ton rédempteur, c’est le Saint d’Israël.
        ${}^{15}J’ai fait de toi un traîneau à battre le grain\\,
        tout neuf, à double rang de pointes :
        \\tu vas briser les montagnes, les broyer ;
        tu réduiras les collines en menue paille ;
        ${}^{16}tu les vanneras, un souffle les emportera,
        un tourbillon les dispersera.
        \\Mais toi, tu mettras ta joie dans le Seigneur ;
        dans le Saint d’Israël, tu trouveras ta louange.
         
        ${}^{17}Les pauvres et les malheureux cherchent de l’eau,
        et il n’y en a pas ;
        leur langue est desséchée par la soif.
        \\Moi, le Seigneur, je les exaucerai,
        moi\\, le Dieu d’Israël, je ne les abandonnerai pas.
        ${}^{18}Sur les hauteurs dénudées je ferai jaillir des fleuves,
        et des sources au creux des vallées.
        \\Je changerai le désert en lac,
        et la terre aride en fontaines\\.
        ${}^{19}Je planterai dans le désert le cèdre et l’acacia,
        le myrte et l’olivier ;
        \\je mettrai ensemble dans les terres incultes
        le cyprès, l’orme et le mélèze\\,
        ${}^{20}afin que tous\\regardent et reconnaissent,
        afin qu’ils considèrent et comprennent
        \\que la main du Seigneur a fait cela,
        que le Saint d’Israël en est le créateur.
${}^{21}Vous, les dieux, présentez votre défense,
        – dit le Seigneur,
        \\avancez vos arguments,
        – dit le Roi de Jacob.
${}^{22}Qu’ils approchent et nous annoncent
        ce qui doit arriver !
        \\Les événements passés, que furent-ils ?
        \\Faites-en l’annonce : nous y prêterons attention
        et nous en connaîtrons la suite.
        \\Ou bien, parlez-nous de l’avenir !
${}^{23}Annoncez-nous ce qui viendra,
        et nous saurons que vous êtes des dieux !
        \\Allons ! Bien ou mal, mais agissez !
        Cela nous troublerait, nous aurions peur !
${}^{24}Or, vous n’êtes rien, et votre œuvre, moins que néant ;
        abominable, celui qui vous choisit !
         
${}^{25}Du nord j’ai fait surgir un homme, et il est venu ;
        depuis l’orient, il se réclame de mon nom ;
        \\il piétine les gouverneurs comme de la boue,
        comme l’argile foulée par le potier.
${}^{26}Qui l’avait annoncé dès le commencement
        pour que nous le sachions,
        \\dès les temps anciens, pour que nous disions :
        « C’est juste » ?
        \\Mais nul ne l’a annoncé, ne l’a fait entendre ;
        nul n’a entendu vos paroles.
${}^{27}Moi, le premier, je l’ai annoncé à Sion ;
        à Jérusalem, j’ai donné un messager.
${}^{28}J’ai regardé : il n’y a personne,
        \\pas un seul conseiller parmi eux
        qui réponde quand je les interroge !
${}^{29}Voici ce qu’ils sont tous : une malfaisance ;
        leurs œuvres : néant ;
        vent et vide, leurs idoles.
      <p class="cantique" id="bib_ct-at_26"><span class="cantique_label">Cantique AT 26</span> = <span class="cantique_ref"><a class="unitex_link" href="#bib_is_42_10">Is 42, 10-16d</a></span>
      
         
      \bchapter{}
        ${}^{1}Voici mon serviteur que je soutiens\\,
        mon élu qui a toute ma faveur.
        \\J’ai fait reposer sur lui mon esprit ;
        aux nations, il proclamera le droit.
        ${}^{2}Il ne criera pas, il ne haussera pas le ton\\,
        il ne fera pas entendre sa voix au-dehors.
        ${}^{3}Il ne brisera pas le roseau qui fléchit,
        il n’éteindra pas la mèche qui faiblit,
        il proclamera le droit en vérité.
        ${}^{4}Il ne faiblira pas, il ne fléchira pas,
        jusqu’à ce qu’il établisse le droit sur la terre,
        \\et que les îles lointaines
        aspirent à recevoir ses lois\\.
        
           
         
        ${}^{5}Ainsi parle Dieu\\, le Seigneur,
        \\qui crée les cieux et les déploie,
        qui affermit la terre et ce qu’elle produit ;
        \\il donne le souffle au peuple qui l’habite,
        et l’esprit à ceux qui la parcourent :
        ${}^{6}Moi, le Seigneur, je t’ai appelé selon la justice ;
        je te saisis par la main, je te façonne\\,
        \\je fais de toi l’alliance du peuple,
        la lumière des nations :
        ${}^{7}tu ouvriras les yeux des aveugles,
        tu feras sortir les captifs de leur prison,
        et, de leur cachot, ceux qui habitent les ténèbres.
        
           
         
${}^{8}Je suis le Seigneur, tel est mon nom ;
        et je ne céderai pas ma gloire à un autre,
        ni ma louange aux idoles.
${}^{9}Les événements passés, voici qu’ils sont arrivés.
        \\Les nouveaux, c’est moi qui les annonce ;
        avant qu’ils ne germent, je vous les fais connaître.
        
           
        ${}^{10}Chantez au Seigneur un chant nouveau,
        louez-le des extrémités de la terre,
        \\gens de la mer et tout ce qu’elle contient,
        les îles avec leurs habitants.
         
        ${}^{11}Qu’ils poussent des cris, les déserts et leurs villes\\,
        les campements où réside Qédar !
        \\Qu’ils jubilent, les habitants de Séla\\,
        qu’ils acclament du sommet des montagnes !
        ${}^{12}Qu’ils rendent gloire au Seigneur,
        qu’ils publient dans les îles sa louange !
         
        ${}^{13}Le Seigneur, tel un héros, s’élance ;
        tel un guerrier, il excite sa jalousie.
        \\Il jette un cri, il pousse un hurlement ;
        sur ses ennemis, il s’avance en héros.
         
        ${}^{14}« Longtemps, j’ai gardé le silence ;
        je me suis tu, je me suis contenu.
        \\Je gémis comme celle qui enfante,
        je suffoque, je cherche mon souffle\\.
         
        ${}^{15}Je vais dévaster montagnes et collines,
        dessécher toute verdure,
        \\changer les fleuves en terres fermes\\,
        assécher les étangs.
         
        ${}^{16}Alors, je conduirai les aveugles
        sur un chemin qui leur est inconnu ;
        je les mènerai par des sentiers qu’ils ignorent.
        \\Je changerai, pour eux, les ténèbres en lumière ;
        les lieux accidentés, je les aplanirai.
         
        \\Telles sont les paroles que j’accomplis,
        je n’y renonce pas. »
         
${}^{17}Ils reculeront, ils seront couverts de honte,
        ceux qui se fient aux idoles,
        \\ceux qui disent à du métal fondu :
        « C’est vous qui êtes nos dieux ! »
${}^{18}Vous, les sourds, entendez !
        Vous, les aveugles, regardez et voyez !
${}^{19}Qui est aveugle, sinon mon serviteur,
        qui est sourd, sinon le messager que j’envoie ?
        \\Qui est aveugle comme mon familier,
        sourd comme le serviteur du Seigneur ?
${}^{20}Tu as vu beaucoup de choses, sans rien retenir,
        ouvert les oreilles, sans rien entendre !
${}^{21}Le Seigneur prenait plaisir, pour sa justice,
        à magnifier, à faire grandir sa loi ;
${}^{22}or ce peuple est pillé, dépouillé,
        on les a tous mis dans des fosses,
        cachés dans des prisons.
        \\On les a pillés, et nul ne délivre,
        dépouillés, et nul n’ordonne de restituer.
${}^{23}Qui de vous va prêter l’oreille à cela,
        se rendre attentif, être à l’écoute pour l’avenir ?
${}^{24}Qui a livré Jacob à ceux qui l’ont dépouillé,
        livré Israël aux pillards ?
        N’est-ce pas le Seigneur ?
        \\Contre lui nous avons péché,
        en refusant de suivre ses chemins
        et d’obéir à sa loi.
${}^{25}Il a déversé sur Israël l’ardeur de sa colère,
        la violence de la guerre.
        \\La guerre l’a embrasé de toute part,
        et lui n’a pas compris.
        \\Elle l’a consumé,
        et lui n’a rien voulu savoir.
      
         
      \bchapter{}
        ${}^{1}Mais maintenant, ainsi parle le Seigneur,
        \\lui qui t’a créé, Jacob,
        et t’a façonné, Israël :
        \\Ne crains pas, car je t’ai racheté,
        je t’ai appelé par ton nom, tu es à moi.
        ${}^{2}Quand tu traverseras les eaux, je serai avec toi,
        les fleuves ne te submergeront pas.
        \\Quand tu marcheras au milieu du feu, tu ne te brûleras pas,
        la flamme ne te consumera pas.
        ${}^{3}Car je suis le Seigneur ton Dieu,
        le Saint d’Israël, ton Sauveur.
        \\Pour payer ta rançon, j’ai donné l’Égypte,
        en échange de toi, l’Éthiopie et Seba\\.
        ${}^{4}Parce que tu as du prix à mes yeux,
        que tu as de la valeur et que je t’aime,
        \\je donne des humains en échange de toi,
        des peuples en échange de ta vie.
        ${}^{5}Ne crains pas, car je suis avec toi.
        \\Je ferai revenir ta descendance de l’orient ;
        de l’occident je te rassemblerai.
        ${}^{6}Je dirai au nord : « Donne ! »
        et au midi : « Ne retiens pas !
        \\Fais revenir mes fils du pays lointain,
        mes filles des extrémités de la terre,
        ${}^{7}tous ceux qui se réclament de mon nom,
        ceux que j’ai créés, façonnés pour ma gloire,
        ceux que j’ai faits ! »
        
           
${}^{8}Faites sortir le peuple aveugle qui a des yeux,
        les sourds qui ont des oreilles.
${}^{9}Toutes les nations sont rassemblées,
        les peuples sont réunis.
        \\Qui, parmi eux, peut annoncer cela
        et nous rappeler les événements du passé ?
        \\Qu’ils produisent leurs témoins
        pour se justifier ;
        \\qu’on les entende et qu’on puisse dire :
        « C’est vrai ! »
${}^{10}Vous êtes mes témoins – oracle du Seigneur –,
        vous êtes mon serviteur, celui que j’ai choisi
        \\pour que vous sachiez, que vous croyiez en moi
        et compreniez que moi, Je suis.
        \\Avant moi aucun dieu n’a été façonné,
        et après moi il n’y en aura pas.
${}^{11}C’est moi, oui, c’est moi qui suis le Seigneur ;
        en dehors de moi, pas de sauveur.
${}^{12}C’est moi qui annonce, qui sauve et qui proclame,
        et non un dieu étranger parmi vous.
        \\Vous êtes mes témoins – oracle du Seigneur –,
        et moi, je suis Dieu.
${}^{13}Oui, depuis toujours, moi, Je suis :
        personne ne délivre de ma main ;
        \\ce que je fais, qui s’y opposera ?
${}^{14}Ainsi parle le Seigneur,
        votre rédempteur, le Saint d’Israël :
        \\En votre faveur, j’envoie une expédition à Babylone,
        je fais tomber tous les verrous,
        \\et les vaisseaux des Chaldéens
        retentissent de leurs cris.
${}^{15}Je suis le Seigneur, votre Dieu saint,
        le Créateur d’Israël, votre roi !
         
        ${}^{16}Ainsi parle le Seigneur,
        lui qui fit un chemin dans la mer,
        un sentier dans les eaux puissantes,
        ${}^{17}lui qui mit en campagne des chars et des chevaux,
        des troupes et de puissants guerriers ;
        \\les voilà tous\\couchés pour ne plus se relever,
        ils se sont éteints, consumés comme une mèche.
        \\Le Seigneur dit\\ :
        ${}^{18}Ne faites plus mémoire des événements passés,
        ne songez plus aux choses d’autrefois.
        ${}^{19}Voici que je fais une chose nouvelle :
        elle germe déjà, ne la voyez-vous pas ?
        \\Oui, je vais faire passer un chemin dans le désert,
        des fleuves\\dans les lieux arides.
        ${}^{20}Les bêtes sauvages me rendront gloire
        – les chacals et les autruches –
        \\parce que j’aurai fait couler de l’eau dans le désert,
        des fleuves dans les lieux arides,
        \\pour désaltérer mon peuple, celui que j’ai choisi.
        ${}^{21}Ce peuple que je me suis façonné
        redira ma louange.
        ${}^{22}Ce n’est pas moi que tu as invoqué, Jacob,
        car tu t’es fatigué de moi, Israël\\ !
        ${}^{23}Tu ne m’as pas apporté de petit bétail pour l’holocauste,
        tu ne m’as pas glorifié par des sacrifices.
        \\Je ne t’ai pas asservi en réclamant\\des offrandes,
        je ne t’ai pas fatigué en réclamant\\de l’encens.
        ${}^{24}Tu n’as rien dépensé pour m’offrir des aromates,
        tu ne m’as pas rassasié de la graisse de tes sacrifices.
        \\Au contraire, tu m’as asservi par tes péchés,
        tu m’as fatigué par tes fautes.
        ${}^{25}C’est moi, oui, c’est moi qui efface tes crimes,
        à cause de moi-même ;
        \\de tes péchés je ne vais pas me souvenir.
        ${}^{26}Fais-moi un procès, nous comparaîtrons ensemble,
        plaide toi-même pour te justifier.
        ${}^{27}Ton ancêtre Jacob\\a péché,
        tes porte-parole se sont révoltés contre moi.
        ${}^{28}Alors, j’ai dû retirer leur dignité\\aux princes consacrés\\,
        j’ai livré Jacob à l’anathème,
        Israël aux sarcasmes.
      
         
      \bchapter{}
        ${}^{1}Et maintenant, écoute-moi, Jacob mon serviteur,
        Israël que j’ai choisi.
        ${}^{2}Ainsi parle le Seigneur qui t’a fait,
        qui t’a façonné dès le sein maternel
        et reste ton appui :
        \\Sois sans crainte, Jacob mon serviteur,
        Israël\\que j’ai choisi.
        ${}^{3}Je répandrai l’eau sur ce qui est assoiffé,
        des ruisseaux sur la terre desséchée.
        \\Je répandrai mon esprit sur ta postérité,
        ma bénédiction sur tes descendants.
        ${}^{4}Ils grandiront comme\\en un pré verdoyant,
        comme les peupliers au bord des eaux courantes.
        ${}^{5}« Moi, je suis au Seigneur », dira celui-ci,
        et celui-là se réclamera du nom de Jacob.
        \\Un autre encore écrira sur sa main\\ :
        « Je suis\\au Seigneur ! »
        et il prendra le nom d’Israël.
        
           
${}^{6}Ainsi parle le Seigneur, le roi d’Israël,
        son rédempteur, le Seigneur de l’univers :
        \\Je suis le premier et je suis le dernier,
        hors moi, pas de Dieu.
${}^{7}Qui est comme moi ? Qu’il prenne la parole,
        qu’il annonce et m’expose ce qu’il en est !
        \\Qui, d’avance, a fait entendre les choses à venir ?
        Que l’on nous prédise ce qui doit arriver !
${}^{8}Ne tremblez pas, ne craignez pas !
        \\Depuis longtemps, ne te l’ai-je pas fait entendre,
        ne te l’ai-je pas annoncé ?
        \\Vous êtes mes témoins !
        \\Y a-t-il un Dieu en dehors de moi ?
        Il n’est pas d’autre Rocher ; je n’en connais pas.
${}^{9}Ceux qui façonnent des idoles ne sont tous que néant :
        leurs œuvres préférées ne servent à rien ;
        \\ce sont des témoins aveugles et sans intelligence ;
        aussi seront-ils couverts de honte.
${}^{10}Quelqu’un va-t-il façonner un dieu ou fondre une idole
        qui ne sert à rien ?
${}^{11}Voici tous ses adeptes couverts de honte :
        les artisans ne sont que des humains !
        \\Qu’ils se rassemblent tous, qu’ils comparaissent :
        ensemble ils trembleront, ils seront couverts de honte.
${}^{12}Le forgeron fabrique un ciseau sur les braises
        et le façonne à coups de marteau.
        \\Il le fabrique à la force du bras.
        \\Puis il a faim, le voilà sans force ;
        il ne boit pas d’eau, il est épuisé.
${}^{13}Le menuisier prend des mesures,
        il esquisse une idole à la craie,
        \\il la travaille à la gouge,
        il l’esquisse au compas,
        \\il la travaille en prenant pour modèle un homme,
        la beauté d’un être humain,
        \\afin qu’elle habite une maison.
${}^{14}Il a débité des cèdres,
        \\il a pris du rouvre et du chêne,
        qu’il a laissé croître parmi les arbres de la forêt ;
        \\il a planté un pin que la pluie fait grandir.
${}^{15}Pour l’homme, c’est du bois à brûler :
        il le prend pour se chauffer,
        il le brûle aussi pour cuire son pain ;
        \\il en fabrique aussi un dieu, et il se prosterne ;
        il en fait une idole et s’incline devant elle.
${}^{16}Il en brûle la moitié au feu :
        avec cette moitié il fait rôtir la viande,
        la mange et se rassasie.
        \\Il se chauffe aussi et dit :
        « Ah ! Je me chauffe et je vois la flamme ! »
${}^{17}Avec le reste il fait un dieu, son idole,
        il s’incline et se prosterne devant elle, il l’implore,
        il dit : « Délivre-moi, car tu es mon dieu ! »
${}^{18}Ils ne savent pas et ne discernent pas.
        \\Leurs yeux sont empêchés de voir,
        et leurs cœurs, de réfléchir.
${}^{19}Nul ne rentre en soi-même,
        nul n’a de savoir ni de discernement pour se dire :
        \\« J’en ai brûlé la moitié au feu,
        j’ai aussi fait cuire du pain sur les braises,
        j’ai rôti de la viande, je l’ai mangée.
        \\Et, du bois qui reste, je ferais une abomination ?
        Je m’inclinerais devant un morceau de bois ! »
${}^{20}Celui qui se repaît de cendre,
        son cœur abusé l’égare.
        \\Il ne se délivrera pas lui-même, et ne dira pas :
        « N’est-ce pas un leurre que j’ai en main ? »
${}^{21}Souviens-toi de ceci, Jacob :
        toi, Israël, tu es mon serviteur.
        \\Je t’ai façonné, tu es pour moi un serviteur,
        Israël, je ne t’oublierai pas !
${}^{22}J’efface tes révoltes comme des nuages,
        tes péchés comme des nuées.
        \\Reviens à moi, car je t’ai racheté.
${}^{23}Cieux, criez de joie pour l’action du Seigneur.
        Acclamez, profondeurs de la terre !
        \\Montagnes, éclatez en cris de joie,
        vous, forêts, et tous vos arbres !
        \\Car le Seigneur a racheté Jacob,
        en Israël il manifeste sa splendeur.
${}^{24}Ainsi parle le Seigneur, ton rédempteur,
        celui qui t’a façonné dès le sein maternel :
        \\C’est moi, le Seigneur, qui fais toute chose.
        \\Seul, j’ai déployé les cieux,
        j’ai affermi la terre :
        qui était avec moi ?
${}^{25}J’annule les signes des augures,
        je fais divaguer les devins,
        \\je fais reculer les sages
        et délirer leur savoir.
${}^{26}J’accomplis la parole de mon serviteur,
        je réalise le projet de mes messagers
        \\quand je dis de Jérusalem : « Elle sera habitée ! »
        et des villes de Juda : « Elles seront rebâties !
        J’en relèverai les ruines ! »,
${}^{27} et quand je dis à l’abîme : « Dessèche-toi !
        Je vais tarir tes fleuves » ;
${}^{28}de même, quand je dis à Cyrus : « Mon berger »,
        il accomplira tout mon désir ;
        \\il dira de Jérusalem : « Elle sera rebâtie ! »
        et au Temple : « Tu seras rétabli ! »
      <p class="cantique" id="bib_ct-at_27"><span class="cantique_label">Cantique AT 27</span> = <span class="cantique_ref"><a class="unitex_link" href="#bib_is_45_15">Is 45, 15-25</a></span>
      
         
      \bchapter{}
        ${}^{1}Ainsi parle le Seigneur à son messie, à Cyrus,
        qu’il a pris\\par la main
        \\pour lui soumettre les nations et désarmer les rois\\,
        pour lui ouvrir les portes à deux battants,
        car aucune porte ne restera fermée :
${}^{2}Moi, je marcherai devant toi ;
        les terrains bosselés, je les aplanirai ;
        \\les portes de bronze, je les briserai ;
        les verrous de fer, je les ferai sauter.
${}^{3}Je te livrerai les trésors des ténèbres,
        les richesses dissimulées,
        \\pour que tu saches que Je suis le Seigneur,
        celui qui t’appelle par ton nom,
        moi, le Dieu d’Israël.
        ${}^{4}À cause de mon serviteur Jacob, d’Israël mon élu,
        je t’ai appelé par ton nom,
        \\je t’ai donné un titre,
        alors que tu ne me connaissais pas.
        ${}^{5}Je suis le Seigneur, il n’en est pas d’autre :
        hors moi, pas de Dieu.
        \\Je t’ai rendu puissant,
        alors que tu ne me connaissais pas,
        ${}^{6}pour que l’on sache, de l’orient à l’occident,
        qu’il n’y a rien en dehors de moi.
        \\Je suis le Seigneur, il n’en est pas d’autre :
        ${}^{7}je façonne la lumière et je crée les ténèbres,
        \\je fais la paix et je crée le malheur.
        C’est moi, le Seigneur, qui fais tout cela.
        ${}^{8}Cieux, distillez d’en haut votre rosée\\,
        que, des nuages, pleuve la justice,
        \\que la terre s’ouvre, produise le salut\\,
        et qu’alors germe aussi la justice\\.
        \\Moi, le Seigneur, je crée tout cela.
        
           
         
${}^{9}Malheureux qui conteste celui qui l’a façonné,
        tesson parmi des tessons de terre !
        \\L’argile dira-t-elle à celui qui la façonne :
        « Que fais-tu ?
        Ton ouvrage n’a pas de mains ! »
${}^{10}Malheureux qui dit à un père : « Qu’as-tu engendré ? »
        et à une femme : « Qu’as-tu mis au monde ? »
${}^{11}Ainsi parle le Seigneur,
        le Saint d’Israël, celui qui l’a façonné :
        \\« Allez-vous m’interroger sur l’avenir de mes fils
        et me donner des ordres pour l’œuvre de mes mains ?
${}^{12}C’est moi qui ai fait la terre
        et, sur elle, créé l’homme.
        \\Moi, de mes mains j’ai déployé les cieux,
        et donné des ordres à toute leur armée.
${}^{13}C’est moi qui ai fait surgir Cyrus selon la justice
        et j’aplanis tous ses chemins.
        \\C’est lui qui construira ma ville
        et laissera partir mes déportés
        sans paiement ni rançon »,
        \\– dit le Seigneur de l’univers.
        
           
${}^{14}Ainsi parle le Seigneur :
        \\Le labeur de l’Égypte, le commerce de l’Éthiopie,
        et les gens de Seba, hommes de haute taille,
        passeront en ta possession, Jérusalem.
        \\Ils marcheront derrière toi, ils passeront enchaînés.
         
        \\Vers toi, ils se prosterneront ;
        c’est vers toi qu’ils adresseront leurs prières :
        \\« Il n’y a de Dieu qu’en toi ; il n’en est pas d’autre,
        aucun autre dieu ! »
       
        ${}^{15}Vraiment tu es un Dieu qui se cache,
        Dieu d’Israël, Sauveur !
         
        ${}^{16}Ils sont tous humiliés, déshonorés,
        \\ils s’en vont\\, couverts de honte,
        ceux qui fabriquent leurs idoles.
         
        ${}^{17}Israël est sauvé par le Seigneur,
        sauvé pour les siècles.
        \\Vous ne serez ni honteux ni humiliés
        pour la suite des siècles.
         
        ${}^{18}Ainsi parle le Seigneur, le Créateur des cieux,
        \\lui, le Dieu qui fit la terre et la façonna,
        lui qui l’affermit,
        \\qui l’a créée, non pas comme un lieu vide,
        mais qui l’a façonnée pour être habitée :
        \\« Je suis le Seigneur :
        il n’en est pas d’autre !
         
        ${}^{19}Quand j’ai parlé, je ne me cachais pas
        quelque part dans l’obscurité de la terre ;
        \\je n’ai pas dit aux descendants de Jacob :
        Cherchez-moi dans le vide !
        \\Je suis le Seigneur qui profère la justice,
        qui proclame ce qui est droit !
         
        ${}^{20}Rassemblez-vous, venez, approchez tous,
        survivants des nations !
        \\Ils sont dans l’ignorance,
        ceux qui portent leurs idoles de bois,
        \\et qui adressent des prières
        à leur dieu qui ne sauve pas.
         
        ${}^{21}Exposez votre cas, présentez vos preuves\\,
        tenez conseil entre vous :
        \\qui donc l’a d’avance révélé
        et jadis annoncé ?
         
        \\N’est-ce pas moi, le Seigneur ?
        Hors moi, pas de Dieu ;
        \\de Dieu juste et sauveur,
        pas d’autre que moi !
         
        ${}^{22}Tournez-vous vers moi : vous serez sauvés,
        tous les lointains de la terre !
        \\Oui, je suis Dieu : il n’en est pas d’autre !
        ${}^{23}Je le jure par moi-même !
        \\De ma bouche sort la justice,
        la parole irrévocable.
         
        \\Devant moi, tout genou fléchira,
        toute langue en fera le serment :
        ${}^{24}Par le Seigneur seulement – dira-t-elle de moi –
        la justice\\et la force ! »
         
        \\Jusqu’à lui viendront, couverts de honte,
        tous ceux qui s’enflammaient contre lui.
        ${}^{25}Elle obtiendra, par le Seigneur, justice et louange,
        toute la descendance d’Israël.
      
         
      \bchapter{}
${}^{1}Le dieu Bel a fléchi, Nébo s’effondre !
        Leurs effigies sont placées sur des animaux,
        sur des bêtes de somme !
        \\Ces charges que vous portiez vous-mêmes
        sont devenues le fardeau d’animaux fourbus !
${}^{2}Elles s’effondrent, ensemble elles ont fléchi,
        – le fardeau n’a pu être sauvé –,
        \\elles-mêmes s’en vont en captivité.
${}^{3}Écoutez-moi, maison de Jacob,
        tout ce qui reste de la maison d’Israël,
        \\vous qui êtes pris en charge dès avant la naissance
        et portés dès le sein maternel :
${}^{4}jusqu’à votre vieillesse, moi, Je suis ;
        jusqu’à vos cheveux blancs, je vous soutiendrai.
        \\Moi, j’ai agi, c’est moi qui porterai,
        moi qui soutiendrai et délivrerai.
${}^{5}À qui allez-vous me rendre semblable ?
        qui feriez-vous mon égal ?
        \\À qui allez-vous me comparer
        que nous soyons ressemblants ?
${}^{6}Ceux qui versent l’or de leur bourse
        et pèsent de l’argent à la balance
        \\engagent un orfèvre qui en fait un dieu ;
        ils s’inclinent et même ils se prosternent.
${}^{7}Ils l’emportent, ils le chargent sur l’épaule
        et vont le déposer à sa place ;
        il s’y tient, sans pouvoir quitter son lieu.
        \\On a beau crier vers lui, il ne répond pas,
        il ne sauve personne de la détresse.
${}^{8}Rappelez-vous cela et soyez fermes !
        Révoltés, prêtez-y attention !
${}^{9}Rappelez-vous les événements passés, ceux de jadis,
        car Je suis Dieu, il n’en est pas d’autre,
        il n’est de dieu que moi !
${}^{10}Dès le commencement, j’annonce la fin,
        et depuis longtemps, ce qui n’est pas accompli.
        \\Je dis : « Mon projet tiendra ;
        tout mon désir, je l’accomplirai. »
${}^{11}J’appelle depuis l’orient un rapace,
        et, d’une terre lointaine, l’homme de mon projet.
        \\Ce que j’ai dit, je le mènerai à bien ;
        j’ai formé un projet, et je l’accomplirai.
${}^{12}Écoutez-moi, cœurs obstinés
        qui êtes loin de la justice !
${}^{13}Ma justice, je l’ai fait approcher :
        elle n’est pas loin,
        et mon salut ne tardera pas.
        \\Je mettrai le salut en Sion,
        et en Israël ma splendeur.
        
           
      
         
      \bchapter{}
${}^{1}Descends, assieds-toi dans la poussière,
        vierge, fille de Babylone !
        \\Assieds-toi par terre, tu n’as plus de trône,
        fille des Chaldéens,
        \\car on ne t’appellera plus
        « la délicate, la raffinée ».
${}^{2}Prends la meule, mouds la farine,
        relève ton voile, retrousse ta robe,
        découvre tes jambes, traverse les cours d’eau :
${}^{3}que soit découverte ta nudité,
        que l’on voie ta honte.
        \\J’exercerai ma vengeance,
        personne ne m’en empêchera.
        
           
         
${}^{4}– Notre rédempteur se nomme le Seigneur de l’univers,
        le Saint d’Israël.
        
           
         
${}^{5}Assieds-toi donc sans un mot,
        enfonce-toi dans les ténèbres,
        fille des Chaldéens,
        \\car on ne t’appellera plus
        « Souveraine des royaumes ».
${}^{6}J’étais irrité contre mon peuple :
        j’avais profané mon héritage
        et je les avais livrés entre tes mains.
        \\Tu ne leur as montré aucune compassion.
        Sur le vieillard, tu as durement appesanti ton joug.
${}^{7}Tu disais : « Je serai pour toujours,
        perpétuellement souveraine. »
        \\Tu n’as pas pris à cœur ces choses-là,
        ni songé à cette fin.
${}^{8}Maintenant, écoute donc, voluptueuse,
        toi qui trônais avec assurance
        et disais en ton cœur :
        \\« Moi, et rien que moi !
        Je ne serai jamais veuve
        ni ne connaîtrai la privation d’enfants. »
${}^{9}Eh bien, ces deux malheurs fondront sur toi
        d’un seul coup, en un jour :
        \\privation d’enfants et veuvage ;
        tous ces malheurs fondent sur toi,
        \\malgré le nombre de tes sorcelleries,
        malgré la puissance de ta magie.
${}^{10}Tu tirais assurance de ta malice ;
        tu disais : « Personne ne me voit ! »
        \\C’est ta sagesse et ta science qui t’ont égarée.
        \\En ton cœur tu disais :
        « Moi, et rien que moi ! »
${}^{11}Un malheur va fondre sur toi,
        sans que tu puisses le conjurer ;
        \\un désastre te frappera,
        sans que tu puisses y échapper ;
        \\soudain fondra sur toi une tourmente
        que tu ne connais pas.
${}^{12}Reste donc avec ta magie
        et tes nombreuses sorcelleries
        pour lesquelles tu t’es fatiguée dès ta jeunesse :
        \\peut-être pourras-tu en tirer profit,
        et peut-être te rendras-tu redoutable !
${}^{13}Tu t’es épuisée à force de consultations.
        \\Qu’ils se lèvent donc et qu’ils te sauvent,
        ceux qui scrutent le ciel,
        qui observent les étoiles
        \\et, à chaque nouvelle lune,
        font connaître ce qui t’arrivera !
${}^{14}Voici qu’ils sont comme de la paille :
        le feu les brûlera,
        \\ils ne pourront échapper à l’étreinte des flammes ;
        \\ce ne seront pas des braises pour se chauffer,
        ni la flambée devant laquelle on s’assied !
${}^{15}Voilà comment te serviront ceux pour qui tu t’es fatiguée,
        ceux qui trafiquèrent avec toi depuis ta jeunesse ;
        \\chacun s’est fourvoyé de son côté,
        et pas un qui te sauve.
        
           
      
         
      \bchapter{}
${}^{1}Écoutez ceci, maison de Jacob,
        vous qui portez le nom d’Israël,
        \\vous qui êtes issus de Juda,
        vous qui prêtez serment par le nom du Seigneur,
        \\qui faites mémoire du Dieu d’Israël,
        mais sans loyauté ni justice.
${}^{2}Car ils s’appellent « ceux de la Ville sainte »,
        \\et ils s’appuient sur le Dieu d’Israël :
        « Seigneur de l’univers » est son nom.
        
           
         
${}^{3}Les événements passés,
        je les avais annoncés d’avance ;
        \\ils étaient sortis de ma bouche,
        et je les avais fait entendre ;
        \\soudain j’ai agi, et ils sont arrivés.
${}^{4}Sachant que tu es dur,
        que ta nuque est une barre de fer,
        et que ton front est de bronze,
${}^{5}je t’ai annoncé d’avance les événements ;
        avant qu’ils n’arrivent,
        \\je te les ai fait entendre,
        pour que tu n’ailles pas dire :
        \\« C’est ma figurine qui en est l’auteur,
        c’est mon idole, ma statue de métal fondu
        qui les a ordonnés. »
${}^{6}Tu as entendu tout cela : regarde-le ;
        et tu ne l’annoncerais pas ?
        \\Maintenant, je te fais entendre des choses nouvelles,
        secrètes, inconnues de toi.
${}^{7}C’est maintenant qu’elles sont créées
        et non depuis longtemps ;
        \\avant ce jour, tu ne les avais pas entendues ;
        ainsi tu ne pouvais pas dire :
        « Mais oui, je les connaissais ! »
${}^{8}Eh bien non, tu n’as rien entendu,
        non, tu ne savais rien,
        non, autrefois tu n’avais pas ouvert l’oreille ;
        \\je sais bien que tu as trahi encore et encore,
        toi que l’on nomme « Rebelle-dès-le-sein-maternel ».
${}^{9}À cause de mon nom, je suspends ma colère,
        pour mon honneur, je patiente avec toi,
        afin de ne pas t’exterminer.
${}^{10}Ce n’est pas comme de l’argent que je t’ai épuré,
        mais je t’ai éprouvé au creuset du malheur.
${}^{11}C’est à cause de moi, de moi seul que je le fais :
        Comment ! Mon nom serait-il profané ?
        \\Ma gloire, je ne la donnerai pas à un autre.
        
           
${}^{12}Écoute-moi, Jacob,
        toi, Israël, que j’ai appelé !
        \\Moi, Je suis :
        \\je suis le Premier,
        et je suis le Dernier.
${}^{13}C’est ma main qui a fondé la terre,
        ma droite a déployé les cieux.
        \\Je les appelle :
        ensemble ils se présentent.
${}^{14}Tous, réunissez-vous, écoutez !
        Qui parmi eux a annoncé cela ?
        \\Celui que le Seigneur aime
        accomplit sa volonté contre Babylone,
        il sera son bras contre les Chaldéens.
${}^{15}C’est moi, oui, c’est moi qui ai parlé,
        qui l’ai appelé ;
        \\je l’ai fait venir,
        il mènera son entreprise à bien !
${}^{16}Approchez-vous de moi, écoutez ceci :
        \\depuis le commencement,
        je n’ai jamais parlé en secret ;
        \\depuis le temps où cela s’est passé,
        Je suis là.
         
        \\Et maintenant, le Seigneur Dieu,
        avec son esprit, m’envoie.
        ${}^{17}Ainsi parle le Seigneur, ton rédempteur,
        Saint d’Israël :
        \\Je suis le Seigneur ton Dieu,
        je te donne un enseignement utile,
        je te guide sur le chemin où tu marches.
        ${}^{18}Si seulement tu avais prêté attention à mes commandements,
        ta paix serait comme un fleuve,
        ta justice, comme les flots de la mer.
        ${}^{19}Ta postérité serait comme le sable,
        comme les grains de sable\\, ta descendance\\ ;
        \\son nom ne serait ni retranché
        ni effacé devant moi.
${}^{20}Sortez de Babylone ! Vite, quittez les Chaldéens !
        \\Avec des cris de joie,
        annoncez, faites-le entendre,
        \\propagez-le jusqu’aux extrémités de la terre !
        Dites : « Le Seigneur a racheté son serviteur Jacob. »
${}^{21}Ils n’ont pas eu soif dans les lieux arides
        où il les a conduits.
        \\Il a fait sourdre pour eux les eaux du rocher,
        il a fendu le rocher : les eaux ont ruisselé !
         
${}^{22}Pas de paix pour les méchants,
        – dit le Seigneur.
      <p class="cantique" id="bib_ct-at_28"><span class="cantique_label">Cantique AT 28</span> = <span class="cantique_ref"><a class="unitex_link" href="#bib_is_49_7">Is 49, 7-13</a></span>
      
         
      \bchapter{}
        ${}^{1}Écoutez-moi, îles lointaines\\ !
        \\Peuples éloignés, soyez attentifs !
        \\J’étais encore dans le sein maternel
        quand le Seigneur m’a appelé ;
        \\j’étais encore dans les entrailles de ma mère
        quand il a prononcé mon nom.
        ${}^{2}Il a fait de ma bouche une épée tranchante,
        il m’a protégé par l’ombre de sa main ;
        \\il a fait de moi une flèche acérée\\,
        il m’a caché dans son carquois.
        ${}^{3}Il m’a dit :
        « Tu es mon serviteur, Israël,
        en toi je manifesterai ma splendeur. »
        ${}^{4}Et moi, je disais :
        « Je me suis fatigué pour rien,
        c’est pour le néant, c’est en pure perte
        que j’ai usé mes forces. »
        \\Et pourtant, mon droit subsistait auprès du Seigneur,
        ma récompense, auprès de mon Dieu.
        ${}^{5}Maintenant le Seigneur parle,
        lui qui m’a façonné dès le sein de ma mère
        \\pour que je sois son serviteur,
        \\que je lui ramène Jacob,
        que je lui rassemble Israël.
        \\Oui, j’ai de la valeur\\aux yeux du Seigneur,
        c’est mon Dieu qui est ma force.
        ${}^{6}Et il dit :
        \\« C’est trop peu que tu sois mon serviteur
        pour relever les tribus de Jacob,
        ramener les rescapés d’Israël :
        \\je fais de toi la lumière des nations,
        pour que mon salut parvienne
        jusqu’aux extrémités de la terre. »
        
           
        ${}^{7}Ainsi parle le Seigneur,
        rédempteur et saint d’Israël,
        \\au serviteur\\méprisé\\, détesté par les nations\\,
        esclave des puissants :
         
        \\Les rois verront, ils se lèveront,
        \\les grands se prosterneront,
        \\à cause du Seigneur qui est fidèle,
        \\du Saint d’Israël qui t’a choisi.
         
        ${}^{8}Ainsi parle le Seigneur :
        Au temps favorable, je t’ai exaucé,
        \\au jour du salut, je t’ai secouru.
        \\Je t’ai façonné, établi\\,
        \\pour que tu sois l’alliance du peuple,
         
        \\pour relever le pays,
        restituer les héritages dévastés
        ${}^{9}et dire aux prisonniers : « Sortez ! »,
        aux captifs des ténèbres\\ : « Montrez-vous ! »
         
        \\Au long des routes, ils pourront paître ;
        sur les hauteurs dénudées\\seront leurs pâturages.
        ${}^{10}Ils n’auront ni faim ni soif ;
        le vent brûlant\\et le soleil ne les frapperont plus.
         
        \\Lui, plein de compassion, les guidera,
        les conduira vers les eaux vives.
        ${}^{11}De toutes mes montagnes, je ferai un chemin,
        et ma route sera rehaussée.
         
        ${}^{12}Les voici : ils viennent de loin,
        \\les uns du nord et du couchant,
        \\les autres des terres du sud\\.
         
        ${}^{13}Cieux, criez de joie ! Terre, exulte !
        \\Montagnes, éclatez en cris de joie !
        \\Car le Seigneur console son peuple ;
        \\de ses pauvres, il a compassion.
        ${}^{14}Jérusalem\\disait :
        « Le Seigneur m’a abandonnée,
        mon Seigneur m’a oubliée. »
        ${}^{15}Une femme peut-elle oublier son nourrisson,
        ne plus avoir de tendresse pour le fils de ses entrailles ?
        \\Même si elle l’oubliait,
        moi, je ne t’oublierai pas.
${}^{16}Car je t’ai gravée sur les paumes de mes mains,
        j’ai toujours tes remparts devant les yeux.
${}^{17}Ils accourent, tes bâtisseurs ;
        tes démolisseurs, tes dévastateurs, ils s’éloignent de toi.
${}^{18}Lève les yeux alentour et regarde :
        tous, ils se rassemblent et viennent vers toi.
        \\Par ma vie – oracle du Seigneur –,
        \\tous, ils seront comme une parure que tu revêtiras,
        autour de toi, comme la ceinture d’une jeune mariée.
${}^{19}Car tes ruines, tes décombres, ton pays dévasté
        sont désormais trop étroits pour tes habitants,
        et ceux qui te dévoraient s’éloigneront.
${}^{20}Les fils dont tu étais privée
        te diront de nouveau à l’oreille :
        \\« L’espace est trop étroit pour moi,
        fais-moi place, que je m’installe. »
${}^{21}Et tu diras en ton cœur :
        « Qui me les a enfantés, ceux-là ?
        \\Privée d’enfants, j’étais stérile,
        j’étais bannie, rejetée,
        \\et ceux-là, qui les a élevés ?
        \\Quand moi, je restais seule,
        ceux-là, où donc étaient-ils ? »
         
${}^{22}Ainsi parle le Seigneur Dieu :
        \\Voici : de ma main levée, je ferai signe aux nations,
        je dresserai mon étendard vers les peuples.
        \\Ils ramèneront tes fils dans leurs bras,
        tes filles seront portées sur les épaules.
${}^{23}Tu auras pour tuteurs des rois,
        et des princesses pour nourrices.
        \\Face contre terre, ils se prosterneront devant toi,
        ils lècheront la poussière de tes pieds.
        \\Tu sauras que Je suis le Seigneur.
        Ceux qui espèrent en moi ne seront pas confondus.
         
${}^{24}Peut-on reprendre au guerrier sa prise,
        le captif d’un tyran peut-il s’échapper ?
${}^{25}Ainsi parle le Seigneur :
        \\Oui, même le captif du guerrier lui sera repris,
        la prise du tyran lui échappera.
        \\Tes adversaires, moi, je m’en ferai l’adversaire,
        tes fils, moi, je les sauverai.
${}^{26}À ceux qui t’exploitent je ferai manger leur propre chair ;
        ils s’enivreront de leur sang comme d’un vin nouveau,
        \\et tout être de chair saura
        \\que moi, le Seigneur, je suis ton Sauveur,
        ton rédempteur, Force de Jacob.
      
         
      \bchapter{}
${}^{1}Ainsi parle le Seigneur :
        \\Où est donc la lettre de répudiation de votre mère,
        par laquelle je l’ai renvoyée ?
        \\Et quel est celui de mes créanciers
        auquel je vous ai vendus ?
        \\Eh bien, c’est à cause de vos fautes
        que vous avez été vendus,
        \\à cause de vos révoltes
        que votre mère a été renvoyée.
${}^{2}Pourquoi n’y avait-il personne quand je suis venu ?
        À mon appel, pourquoi nul n’a-t-il répondu ?
        \\Ma main serait-elle trop courte pour racheter ?
        n’aurais-je pas la force pour délivrer ?
        \\Eh bien, par ma seule menace je dessèche la mer,
        je change les fleuves en désert ;
        \\leurs poissons pourrissent faute d’eau,
        morts de soif.
${}^{3}Je revêts les cieux de noir,
        je les habille de deuil.
        
           
        ${}^{4}Le Seigneur mon Dieu m’a donné le langage des disciples,
        pour que je puisse, d’une parole,
        soutenir celui qui est épuisé.
        \\Chaque matin, il éveille,
        il éveille mon oreille
        pour qu’en disciple, j’écoute.
        ${}^{5}Le Seigneur mon Dieu m’a ouvert l’oreille,
        et moi, je ne me suis pas révolté,
        je ne me suis pas dérobé.
        ${}^{6}J’ai présenté mon dos à ceux qui me frappaient,
        et mes joues à ceux qui m’arrachaient la barbe\\.
        \\Je n’ai pas caché ma face devant les outrages et les crachats.
        ${}^{7}Le Seigneur mon Dieu vient à mon secours ;
        c’est pourquoi je ne suis pas atteint par les outrages,
        \\c’est pourquoi j’ai rendu ma face dure comme pierre :
        je sais que je ne serai pas confondu.
        ${}^{8}Il est proche, Celui qui me justifie.
        \\Quelqu’un veut-il plaider contre moi ?
        Comparaissons ensemble !
        \\Quelqu’un veut-il m’attaquer en justice ?
        Qu’il s’avance vers moi !
        ${}^{9}Voilà le Seigneur mon Dieu, il prend ma défense ;
        qui donc me condamnera ?
        \\Les voici tous qui s’usent comme un vêtement,
        la teigne les dévorera !
         
${}^{10}Est-il quelqu’un parmi vous qui craint le Seigneur,
        qui écoute la voix de son serviteur ?
        \\S’il a marché dans les ténèbres
        sans la moindre clarté,
        \\qu’il se confie dans le nom du Seigneur,
        qu’il s’appuie sur son Dieu.
${}^{11}Mais vous tous qui allumez un feu,
        formant un cercle de flèches incendiaires,
        \\allez dans le brasier de votre propre feu,
        au milieu des flèches que vous enflammez.
        \\Voici ce que vous réserve ma main :
        vous resterez gisant dans la douleur.
      
         
      \bchapter{}
${}^{1}Écoutez-moi, vous qui tendez vers la justice,
        vous qui recherchez le Seigneur :
        \\regardez le rocher dans lequel vous avez été taillés,
        la carrière d’où vous avez été tirés.
${}^{2}Regardez Abraham votre père,
        et Sara qui vous a enfantés ;
        \\car il était seul quand je l’ai appelé,
        mais je l’ai béni et multiplié.
${}^{3}Oui, le Seigneur console Sion,
        il la console de toutes ses ruines,
        \\il va faire de son désert un Éden,
        de sa steppe un jardin du Seigneur.
        \\On y retrouvera l’allégresse et la joie,
        l’action de grâce et le son de la musique.
        
           
         
${}^{4}Soyez attentifs, vous qui êtes mon peuple ;
        et vous, les nations, prêtez-moi l’oreille !
        \\Car de moi sortira la loi,
        mon droit sera la lumière des peuples !
        <p class="verset_anchor" id="para_bib_is_51_5">Soudain, 
${}^{5}je rendrai proche ma justice,
        mon salut va paraître,
        et mon bras gouvernera les peuples.
        \\Les îles mettront en moi leur espoir,
        elles comptent sur mon bras.
${}^{6}Levez les yeux vers le ciel,
        regardez en bas vers la terre.
        \\Les cieux se dissiperont comme la fumée,
        la terre s’usera comme un vêtement,
        et ses habitants tomberont comme des mouches.
        \\Mais mon salut est pour toujours,
        ma justice ne sera jamais abattue.
${}^{7}Écoutez-moi, vous qui connaissez la justice,
        peuple de ceux qui ont ma loi dans le cœur !
        \\Ne craignez pas l’insulte des hommes,
        ne soyez pas abattus par leurs sarcasmes,
${}^{8}car la teigne les dévorera comme un vêtement,
        les mites les dévoreront comme de la laine.
        \\Mais ma justice est pour toujours,
        et mon salut, de génération en génération.
        
           
         
${}^{9}Éveille-toi, éveille-toi,
        revêts-toi de force, bras du Seigneur !
        \\Éveille-toi comme aux jours anciens,
        au temps des générations d’autrefois.
        \\N’est-ce pas toi qui taillas en pièces Rahab,
        qui transperças le Monstre marin ?
${}^{10}N’est-ce pas toi qui desséchas la mer,
        les eaux du grand abîme,
        \\qui fis des profondeurs de la mer un chemin
        pour que passent les rachetés ?
${}^{11}Ceux qu’a libérés le Seigneur reviennent,
        ils entrent dans Sion avec des cris de fête,
        couronnés de l’éternelle joie.
        \\Allégresse et joie les rejoindront,
        douleur et plainte s’enfuient.
        
           
         
${}^{12}C’est moi, c’est moi qui vous console.
        \\Qui es-tu pour craindre l’homme qui doit mourir,
        un fils d’homme périssable comme l’herbe,
${}^{13}au point d’oublier le Seigneur qui t’a fait,
        qui a tendu les cieux et fondé la terre,
        \\et qui es-tu pour frémir tout au long des jours
        devant la fureur de l’oppresseur quand il s’apprête à détruire ?
        \\Où donc est-elle, la fureur de l’oppresseur ?
${}^{14}Bientôt, le prostré sera libéré,
        il ne mourra pas dans un cachot,
        et le pain ne lui manquera pas !
${}^{15}Moi, je suis le Seigneur, ton Dieu,
        qui soulève la mer et fait mugir ses flots,
        \\– son nom est « Le Seigneur de l’univers ».
${}^{16}J’ai mis dans ta bouche mes paroles,
        je t’ai couvert de l’ombre de ma main,
        \\quand je plantais les cieux et fondais la terre,
        quand j’ai dit à Sion : « Tu es mon peuple. »
        
           
         
${}^{17}Réveille-toi, réveille-toi,
        debout, Jérusalem !
        \\Tu as bu de la main du Seigneur la coupe de sa fureur,
        tu as bu jusqu’à la lie la coupe du vertige !
${}^{18}Personne qui la guide,
        parmi les fils qu’elle a enfantés ;
        \\pas un qui lui prenne la main,
        de tous les fils qu’elle a élevés !
${}^{19}Ce double malheur t’a frappée,
        – qui t’en plaindra ?
        \\Le ravage et la ruine, la famine et l’épée,
        – qui t’en consolera ?
${}^{20}Tes fils, épuisés, gisent à tous les coins de rue,
        comme l’antilope prise au piège,
        \\submergés par la fureur du Seigneur,
        la menace de ton Dieu.
${}^{21}Écoute alors ceci, malheureuse,
        ivre, mais pas de vin :
${}^{22}Ainsi parle ton Maître, le Seigneur, ton Dieu,
        qui défend la cause de son peuple :
        \\Voici que je retire de ta main la coupe du vertige ;
        à la coupe de ma fureur, désormais tu ne boiras plus.
${}^{23}Je la mettrai dans la main de ceux qui t’affligeaient
        et te disaient :
        \\« Courbe-toi, que nous passions ! »
        \\Et tu faisais de ton dos comme un sol,
        comme une rue pour les passants !
        
           
       
      
         
      \bchapter{}
${}^{1}Éveille-toi, éveille-toi,
        \\revêts-toi de force, Sion !
        \\Revêts tes habits de splendeur,
        Jérusalem, ville sainte !
        \\Désormais l’incirconcis et l’impur
        n’entreront plus chez toi.
${}^{2}Secoue ta poussière ! Debout, Jérusalem, ô captive !
        Dénoue les liens de ton cou, ô captive, fille de Sion !
${}^{3}Car ainsi parle le Seigneur :
        \\Vous avez été vendus pour rien,
        c’est sans argent que vous serez rachetés !
${}^{4}Oui, ainsi parle le Seigneur Dieu :
        \\En Égypte, au début,
        mon peuple est descendu en émigré ;
        à la fin, Assour l’a opprimé.
${}^{5}Et maintenant, ici, que me reste-t-il
        – oracle du Seigneur –,
        puisque mon peuple a été enlevé pour rien ?
        \\Ses tyrans triomphent
        – oracle du Seigneur –,
        et tout au long des jours, mon nom est bafoué !
${}^{6}Eh bien ! mon peuple saura quel est mon nom.
        \\Oui, ce jour-là, il saura
        que c’est moi-même qui dis : « Me voici ! »
        
           
        ${}^{7}Comme ils sont beaux sur les montagnes,
        les pas du messager,
        \\celui qui annonce la paix,
        qui porte la bonne nouvelle,
        qui annonce le salut,
        \\et vient dire à Sion :
        « Il règne, ton Dieu ! »
        ${}^{8}Écoutez\\la voix des guetteurs :
        ils élèvent la voix,
        tous ensemble ils crient de joie
        \\car, de leurs propres yeux,
        ils voient le Seigneur qui revient à Sion.
        ${}^{9}Éclatez en cris de joie,
        vous, ruines de Jérusalem,
        \\car le Seigneur console son peuple,
        il rachète Jérusalem !
        ${}^{10}Le Seigneur a montré la sainteté de son bras
        aux yeux de toutes les nations.
        \\Tous les lointains\\de la terre
        ont vu le salut de notre Dieu\\.
         
${}^{11}Allez-vous-en, allez-vous-en, sortez de là,
        ne touchez rien d’impur !
        \\Sortez de Babylone, devenez purs,
        vous qui portez les objets consacrés au Seigneur !
${}^{12}Et ce n’est pas en hâte que vous sortirez,
        vous n’irez pas comme des fuyards,
        \\car il marche devant vous, le Seigneur,
        et celui qui ferme la marche, c’est le Dieu d’Israël !
        ${}^{13}Mon serviteur réussira, dit le Seigneur\\ ;
        il montera, il s’élèvera, il sera exalté !
        ${}^{14}La multitude avait été consternée en le voyant\\,
        car il était si défiguré
        \\qu’il ne ressemblait plus à un homme\\ ;
        il n’avait plus l’apparence d’un fils d’homme\\.
        ${}^{15}Il étonnera\\de même une multitude de nations ;
        devant lui les rois resteront bouche bée,
        \\car ils verront ce que, jamais, on ne leur avait dit,
        ils découvriront ce dont ils n’avaient jamais entendu parler.
      
         
      \bchapter{}
        ${}^{1}Qui aurait cru ce que nous avons entendu ?
        \\Le bras puissant\\du Seigneur, à qui s’est-il révélé ?
        ${}^{2}Devant lui, le serviteur\\a poussé comme une plante chétive,
        une racine dans une terre aride ;
        \\il était sans apparence ni beauté qui attire nos regards,
        son aspect n’avait rien pour nous plaire.
        ${}^{3}Méprisé, abandonné des hommes,
        homme de douleurs, familier de la souffrance,
        \\il était pareil à celui devant qui on se voile la face ;
        et nous l’avons méprisé\\, compté pour rien.
        ${}^{4}En fait, c’étaient nos souffrances qu’il portait,
        nos douleurs dont il était chargé.
        \\Et nous, nous pensions qu’il était frappé,
        meurtri par Dieu, humilié.
        ${}^{5}Or, c’est à cause de nos révoltes qu’il a été transpercé,
        à cause de nos fautes qu’il a été broyé.
        \\Le châtiment qui nous donne la paix a pesé sur lui\\ :
        par ses blessures, nous sommes guéris\\.
        ${}^{6}Nous étions tous errants comme des brebis,
        chacun suivait son propre chemin.
        \\Mais le Seigneur a fait retomber sur lui
        nos fautes à nous tous.
        ${}^{7}Maltraité, il s’humilie,
        il n’ouvre pas la bouche :
        \\comme un agneau conduit à l’abattoir,
        comme une brebis muette devant les tondeurs,
        il n’ouvre pas la bouche.
        ${}^{8}Arrêté, puis jugé\\, il a été supprimé.
        Qui donc s’est inquiété de son sort\\ ?
        \\Il a été retranché de la terre des vivants,
        frappé à mort\\pour les révoltes de son peuple\\.
        ${}^{9}On a placé sa tombe avec les méchants,
        son tombeau avec les riches ;
        \\et pourtant il n’avait pas commis de violence,
        on ne trouvait\\pas de tromperie dans sa bouche.
        ${}^{10}Broyé par la souffrance, il a plu au Seigneur\\.
        S’il remet sa vie\\en sacrifice de réparation,
        \\il verra une descendance, il prolongera ses jours :
        par lui, ce qui plaît au Seigneur réussira.
        
           
         
        ${}^{11}Par suite de ses tourments, il verra la lumière\\,
        la connaissance le comblera.
        \\Le juste, mon serviteur\\, justifiera les multitudes,
        il se chargera de leurs fautes.
        ${}^{12}C’est pourquoi, parmi les grands\\, je lui donnerai sa part,
        avec les puissants il partagera le butin,
        \\car il s’est dépouillé lui-même jusqu’à la mort,
        et il a été compté avec les pécheurs,
        \\alors qu’il portait le péché des multitudes
        et qu’il intercédait pour les pécheurs.
        
           
      
         
      \bchapter{}
        ${}^{1}Crie de joie, femme stérile,
        toi qui n’as pas enfanté ;
        \\jubile, éclate en cris de joie,
        toi qui n’as pas connu les douleurs !
        \\Car les fils de la délaissée seront plus nombreux
        que les fils de l’épouse,
        – dit le Seigneur.
        ${}^{2}Élargis l’espace de ta tente,
        déploie sans hésiter la toile de ta demeure,
        allonge tes cordages, renforce tes piquets !
        ${}^{3}Car tu vas te répandre au nord et au midi\\.
        Ta descendance dépossédera les nations,
        elle peuplera des villes désertées.
        ${}^{4}Ne crains pas,
        tu ne connaîtras plus la honte ;
        \\ne tiens pas compte des outrages,
        tu n’auras plus à rougir,
        \\tu oublieras la honte de ta jeunesse,
        tu ne te rappelleras plus le déshonneur de ton veuvage.
        ${}^{5}Car ton époux, c’est Celui qui t’a faite\\,
        son nom est « Le Seigneur de l’univers ».
        \\Ton rédempteur, c’est le Saint d’Israël,
        il s’appelle « Dieu de toute la terre ».
        ${}^{6}Oui, comme une femme abandonnée, accablée\\,
        le Seigneur te rappelle.
        \\Est-ce que l’on rejette la femme de sa jeunesse ?
        – dit ton Dieu.
        ${}^{7}Un court instant, je t’avais abandonnée,
        mais dans ma grande tendresse, je te ramènerai\\.
        ${}^{8}Quand ma colère a débordé,
        un instant, je t’avais caché ma face.
        \\Mais dans mon éternelle fidélité,
        je te montre ma tendresse,
        – dit le Seigneur, ton rédempteur.
        ${}^{9}Je ferai comme au temps de Noé\\,
        quand j’ai juré que les eaux
        ne submergeraient plus la terre :
        \\de même, je jure de ne plus m’irriter contre toi,
        et de ne plus te menacer.
        ${}^{10}Même si les montagnes s’écartaient,
        si les collines s’ébranlaient,
        \\ma fidélité ne s’écarterait pas de toi,
        mon alliance de paix ne serait pas ébranlée,
        – dit le Seigneur, qui te montre sa tendresse.
        
           
        ${}^{11}Jérusalem\\, malheureuse,
        battue par la tempête, inconsolée,
        \\voici que je vais sertir\\tes pierres
        et poser tes fondations sur des saphirs.
        ${}^{12}Je ferai tes créneaux avec des rubis,
        tes portes en cristal de roche,
        et toute ton enceinte avec des pierres précieuses.
        ${}^{13}Tes fils\\seront tous disciples\\du Seigneur,
        et grande sera leur paix\\.
        ${}^{14}Tu seras établie sur la justice :
        \\loin de toi l’oppression,
        tu n’auras plus à craindre ;
        \\loin de toi la terreur,
        elle ne t’approchera plus.
${}^{15}Si l’on s’en prend à toi,
        moi, je n’y suis pour rien ;
        \\et quiconque s’en prend à toi
        devant toi tombera.
${}^{16}Voici que moi, j’ai créé l’artisan
        qui souffle sur les braises
        et en retire l’arme appropriée ;
        \\c’est moi aussi qui ai créé le destructeur
        pour ravager.
${}^{17}Toute arme forgée contre toi
        restera inefficace ;
        \\toute langue qui te citera au tribunal,
        tu la condamneras.
        \\Telle est la part des serviteurs du Seigneur,
        telle est la justice qui leur viendra de moi,
        – oracle du Seigneur.
      
         
      \bchapter{}
        ${}^{1}Vous tous qui avez soif,
        venez, voici de l’eau !
        \\Même si vous n’avez pas d’argent,
        venez acheter et consommer,
        \\venez acheter du vin et du lait
        sans argent, sans rien payer.
        ${}^{2}Pourquoi dépenser votre argent pour ce qui ne nourrit pas,
        vous fatiguer pour ce qui ne rassasie pas ?
        \\Écoutez-moi bien, et vous mangerez de bonnes choses,
        vous vous régalerez de viandes savoureuses !
        ${}^{3}Prêtez l’oreille ! Venez à moi !
        Écoutez, et vous vivrez.
        \\Je m’engagerai envers vous par une alliance éternelle :
        ce sont les bienfaits garantis à David.
        ${}^{4}Lui, j’en ai fait un témoin pour les peuples,
        pour les peuples, un guide et un chef.
        ${}^{5}Toi, tu appelleras une nation inconnue de toi ;
        une nation qui ne te connaît pas accourra vers toi,
        \\à cause du Seigneur ton Dieu,
        à cause du Saint d’Israël, car il fait ta splendeur.
        
           
         
        ${}^{6}Cherchez le Seigneur tant qu’il se laisse trouver ;
        invoquez-le tant qu’il est proche.
        ${}^{7}Que le méchant abandonne son chemin,
        et l’homme perfide, ses pensées !
        \\Qu’il revienne vers le Seigneur
        qui lui montrera sa miséricorde,
        \\vers notre Dieu
        qui est riche en pardon.
        ${}^{8}Car mes pensées ne sont pas vos pensées,
        et vos chemins ne sont pas mes chemins,
        – oracle du Seigneur.
        ${}^{9}Autant le ciel est élevé au-dessus de la terre,
        autant mes chemins sont élevés au-dessus de vos chemins,
        et mes pensées, au-dessus de vos pensées.
        ${}^{10}La pluie et la neige qui descendent des cieux
        n’y retournent pas sans avoir abreuvé la terre,
        sans l’avoir fécondée et l’avoir fait germer,
        \\donnant la semence au semeur
        et le pain à celui qui doit manger ;
        ${}^{11}ainsi ma parole, qui sort de ma bouche,
        ne me reviendra pas sans résultat,
        \\sans avoir fait ce qui me plaît,
        sans avoir accompli sa mission.
        
           
         
        ${}^{12}Oui, dans la joie vous partirez,
        vous serez conduits dans la paix.
        \\Montagnes et collines, à votre passage, éclateront en cris de joie,
        et tous les arbres de la campagne applaudiront.
        ${}^{13}Au lieu de broussailles poussera le cyprès,
        au lieu de l’ortie poussera le myrte.
        \\Le nom du Seigneur en sera grandi :
        ce signe éternel sera impérissable.
        
           
