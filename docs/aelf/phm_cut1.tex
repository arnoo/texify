  
  
    
    \bbook{LETTRE À PHILÉMON}{LETTRE À PHILÉMON}
      <a class="bib_chap hidden" id="bib_phm_1"/>
${}^{1}Paul, en prison pour le Christ Jésus,
        et Timothée notre frère,
        \\à toi, Philémon, notre collaborateur bien-aimé,
${}^{2}ainsi qu’à notre sœur, Apphia,
        \\à notre compagnon de combat, Archippe,
        \\et à l’Église qui se rassemble dans ta maison.
${}^{3}À vous, la grâce et la paix
        \\de la part de Dieu notre Père
        et du Seigneur Jésus Christ.
${}^{4}À tout moment je rends grâce à mon Dieu, en faisant mémoire de toi dans mes prières, 
${}^{5}car j’entends parler de ton amour et de la foi que tu as pour le Seigneur Jésus et à l’égard de tous les fidèles. 
${}^{6}Je prie pour que ta communion dans la foi devienne efficace par la pleine connaissance de tout le bien qui est en nous, pour le Christ. 
${}^{7}En effet, ta charité m’a déjà apporté beaucoup de joie et de réconfort, car grâce à toi, frère, les cœurs des fidèles ont trouvé du repos.
${}^{8}Certes, j’ai dans le Christ toute liberté de parole pour te prescrire ce qu’il faut faire, 
${}^{9}mais je préfère t’adresser une demande au nom de la charité : moi, Paul, tel que je suis, un vieil homme et, qui plus est, prisonnier maintenant à cause du Christ Jésus, 
${}^{10}j’ai quelque chose à te demander pour Onésime, mon enfant à qui, en prison, j’ai donné la vie dans le Christ. 
${}^{11}Cet Onésime (dont le nom signifie « avantageux ») a été, pour toi, inutile à un certain moment, mais il est maintenant bien utile pour toi comme pour moi. 
${}^{12}Je te le renvoie, lui qui est comme mon cœur. 
${}^{13}Je l’aurais volontiers gardé auprès de moi, pour qu’il me rende des services en ton nom, à moi qui suis en prison à cause de l’Évangile. 
${}^{14}Mais je n’ai rien voulu faire sans ton accord, pour que tu accomplisses ce qui est bien, non par contrainte mais volontiers.
${}^{15}S’il a été éloigné de toi pendant quelque temps, c’est peut-être pour que tu le retrouves définitivement, 
${}^{16}non plus comme un esclave, mais, mieux qu’un esclave, comme un frère bien-aimé : il l’est vraiment pour moi, combien plus le sera-t-il pour toi, aussi bien humainement que dans le Seigneur. 
${}^{17}Si donc tu estimes que je suis en communion avec toi, accueille-le comme si c’était moi. 
${}^{18}S’il t’a fait du tort ou s’il te doit quelque chose, mets cela sur mon compte. 
${}^{19}Moi, Paul, j’écris ces mots de ma propre main : c’est moi qui te rembourserai. Je n’ajouterai pas que toi aussi, tu as une dette envers moi, et cette dette, c’est toi-même. 
${}^{20}Oui, frère, donne-moi cette satisfaction dans le Seigneur, fais que mon cœur trouve du repos dans le Christ.
${}^{21}Confiant dans ton obéissance, je t’écris en sachant que tu feras plus encore que je ne dis. 
${}^{22}En même temps, prévois aussi mon logement, car j’espère que, grâce à vos prières, je vous serai rendu.
${}^{23}Épaphras, mon compagnon de captivité dans le Christ Jésus, te salue, 
${}^{24}ainsi que Marc, Aristarque, Démas et Luc, mes collaborateurs.
${}^{25}Que la grâce du Seigneur Jésus Christ soit avec votre esprit.
