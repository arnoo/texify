  
  
      
         
      \bchapter{}
      \begin{verse}
${}^{1}La vingt-cinquième année de notre déportation, au début de l’année, le dix du mois, quatorze ans après la chute de la ville, en ce jour même, la main du Seigneur se posa sur moi. Il m’emmena là-bas. 
${}^{2}Dans des visions divines, il m’emmena en terre d’Israël ; il me déposa sur une très haute montagne, sur laquelle, au sud, il y avait comme les constructions d’une ville. 
${}^{3}Il m’emmena là-bas ; et voici : il y avait un homme ; son aspect était comme l’aspect du bronze. Il avait à la main une sorte de cordon de lin ainsi qu’une canne à mesurer. Il se tenait à la porte. 
${}^{4}L’homme me dit : « Fils d’homme, regarde de tes yeux, écoute de tes oreilles, sois attentif à tout ce que je te ferai voir, car c’est pour que tu puisses voir cela que tu as été amené ici. Tu raconteras à la maison d’Israël tout ce que tu vas voir. »
      
         
${}^{5}Et voici : il y avait une muraille extérieure, tout autour de l’édifice. L’homme avait dans la main une canne à mesurer de six coudées – cette coudée ancienne valant une coudée nouvelle et un palme. Il mesura l’épaisseur de la construction : une canne ; la hauteur : une canne. 
${}^{6}Il vint vers la porte qui fait face à l’orient, il en monta les marches ; il mesura le seuil de la porte : une canne en profondeur – pour chaque seuil, une canne en profondeur. 
${}^{7}Les loges à l’intérieur des portes : une canne en longueur et une canne en largeur ; entre les loges, cinq coudées. Le seuil de la porte, du côté du vestibule de la porte, depuis l’intérieur : une canne. 
${}^{8}Il mesura le vestibule de la porte : 
${}^{9}huit coudées ; ses piliers : deux coudées, le vestibule de la porte étant situé vers l’intérieur. 
${}^{10}Les loges de la porte orientale étaient au nombre de trois d’un côté et trois de l’autre : mêmes dimensions pour les trois, et mêmes dimensions pour les piliers, de part et d’autre. 
${}^{11}Il mesura la largeur de l’ouverture de la porte : dix coudées ; la profondeur de la porte : treize coudées. 
${}^{12}Il y avait une barrière devant les loges ; cette barrière était d’une coudée de part et d’autre ; les loges avaient six coudées d’un côté et six coudées de l’autre. 
${}^{13}Il mesura la porte, du toit d’une loge au toit de l’autre ; sa largeur était de vingt-cinq coudées, chaque entrée étant en face d’une autre entrée. 
${}^{14}Le vestibule : vingt coudées ; quant au vestibule de la porte, la cour l’entourait. 
${}^{15}De la façade de la porte jusqu’à la façade du vestibule, côté intérieur de la porte : cinquante coudées. 
${}^{16}Il y avait des fenêtres grillagées sur les loges et sur leurs piliers, du côté intérieur de la porte, tout autour ; de même pour le vestibule, des fenêtres tout autour, du côté intérieur. Et sur chaque pilier, des palmiers.
${}^{17}Il me fit entrer dans la cour extérieure. Et voici qu’il y avait des salles et un dallage ; les salles étaient aménagées tout autour de la cour : trente salles sur ce dallage. 
${}^{18}Le dallage, sur le côté des portes, correspondait à la largeur des portes : c’était le dallage inférieur. 
${}^{19}Il mesura la distance, du devant de la porte inférieure jusqu’à la façade extérieure de la cour intérieure : cent coudées. Voilà pour l’orient. Quant au nord, 
${}^{20}il mesura la longueur et la largeur de la porte qui fait face au nord, dans la cour extérieure. 
${}^{21}Ses loges – trois d’un côté et trois de l’autre –, ses piliers et son vestibule étaient de mêmes dimensions que ceux de la première porte ; sa longueur était de cinquante coudées, sa largeur de vingt-cinq coudées. 
${}^{22}Ses fenêtres, son vestibule et ses palmiers étaient de mêmes dimensions que ceux de la porte qui fait face à l’orient ; on y montait par sept marches, devant le vestibule. 
${}^{23}Il y avait une porte sur la cour intérieure face à la porte nord, comme à l’orient. L’homme mesura, d’une porte à l’autre : cent coudées.
${}^{24}Il me fit aller en direction du sud ; et voici : il y avait une porte, en direction du sud. Il mesura ses piliers, son vestibule : mêmes dimensions que les autres. 
${}^{25}Ses fenêtres, tout autour de son vestibule, étaient semblables aux autres fenêtres. Sa longueur : cinquante coudées ; sa largeur : vingt-cinq coudées. 
${}^{26}On y montait par un escalier de sept marches, devant son vestibule. Il y avait des palmiers, de part et d’autre, sur ses piliers. 
${}^{27}La cour intérieure avait une porte en direction du sud ; il mesura d’une porte à l’autre, en direction du sud : cent coudées.
${}^{28}Il me fit entrer dans la cour intérieure par la porte sud et il mesura cette porte : mêmes dimensions que les autres. 
${}^{29}Ses loges, ses piliers et son vestibule : mêmes dimensions que les autres. Il y avait des fenêtres tout autour de son vestibule. Sa longueur : cinquante coudées ; sa largeur : vingt-cinq coudées. 
${}^{30}Des vestibules l’entouraient ; longueur : vingt-cinq coudées ; largeur : cinq coudées. 
${}^{31}Son vestibule donnait sur la cour extérieure ; il y avait des palmiers sur ses piliers ; son escalier avait huit marches.
${}^{32}Il me fit entrer par l’orient dans la cour intérieure. Il mesura la porte : mêmes dimensions que les autres. 
${}^{33}Ses loges, ses piliers et son vestibule : mêmes dimensions que les autres. Il y avait des fenêtres tout autour de son vestibule. Sa longueur : cinquante coudées ; sa largeur : vingt-cinq coudées. 
${}^{34}Son vestibule donnait sur la cour extérieure ; il y avait des palmiers sur ses piliers, de part et d’autre ; son escalier avait huit marches.
${}^{35}Il me fit entrer par la porte nord ; il mesura : mêmes dimensions que les autres, 
${}^{36}ainsi que pour ses loges, ses piliers, son vestibule. Il y avait des fenêtres tout autour. Sa longueur : cinquante coudées ; sa largeur : vingt-cinq coudées. 
${}^{37}Son vestibule donnait sur la cour extérieure ; il y avait des palmiers sur ses piliers, de part et d’autre ; son escalier avait huit marches.
${}^{38}Une salle ouvrait sur le vestibule de la porte, pour qu’on y lave les victimes destinées à l’holocauste. 
${}^{39}Dans le vestibule de la porte, il y avait deux tables d’un côté et deux de l’autre, pour immoler les victimes des holocaustes ainsi que des sacrifices pour la faute et des sacrifices de réparation. 
${}^{40}Puis, du côté extérieur, pour qui montait vers l’entrée de la porte nord, il y avait deux tables et, de l’autre côté du vestibule de la porte, deux tables. 
${}^{41}Quatre tables d’un côté et quatre tables de l’autre côté de la porte : huit tables sur lesquelles on immole. 
${}^{42}Quatre tables en pierres de taille pour l’holocauste ; leur longueur : une coudée et demie ; leur largeur : une coudée et demie ; leur hauteur : une coudée. Sur ces tables, on dépose les instruments pour immoler les victimes des holocaustes et des sacrifices. 
${}^{43}Des rebords d’un palme de largeur étaient aménagés à l’intérieur, sur le pourtour. Sur les tables on met les viandes offertes en présent réservé.
${}^{44}Au-delà de la porte intérieure, dans la cour intérieure, se trouvaient les salles des chantres, l’une sur le côté de la porte nord, avec sa façade au sud ; l’autre sur le côté de la porte sud, avec sa façade au nord. 
${}^{45}Et l’homme me dit : « Cette salle, dont la façade est en direction du sud, est pour les prêtres qui maintiennent les observances de la Maison. 
${}^{46}Et la salle dont la façade est en direction du nord est pour les prêtres qui maintiennent les observances relatives à l’autel ; ce sont les fils de Sadoc qui, parmi les fils de Lévi, s’approchent du Seigneur pour le servir. »
${}^{47}L’homme mesura la cour ; sa longueur : cent coudées ; sa largeur : cent coudées ; c’était un carré. L’autel était devant la Maison. 
${}^{48}Il me fit entrer dans le vestibule de la Maison ; il mesura les piliers du vestibule : cinq coudées d’un côté et cinq coudées de l’autre. Largeur de la porte : trois coudées d’un côté et trois coudées de l’autre. 
${}^{49}Longueur du vestibule : vingt coudées ; largeur : douze coudées ; on pouvait y monter par des marches. Il y avait des colonnes près des piliers, une d’un côté et une de l’autre.
      
         
      \bchapter{}
      \begin{verse}
${}^{1}Il me fit entrer dans la grande salle. Il mesura les piliers : d’un côté il y avait six coudées de large, et de l’autre six coudées de large. 
${}^{2}Largeur de l’entrée : dix coudées ; parois latérales de l’entrée : cinq coudées d’un côté et cinq coudées de l’autre. Il mesura la longueur de la salle : quarante coudées ; sa largeur : vingt coudées. 
${}^{3}Il entra à l’intérieur et mesura le pilier de l’entrée : deux coudées. L’entrée avait six coudées ; les parois latérales de l’entrée, sept coudées. 
${}^{4}Il mesura la longueur de la salle : vingt coudées ; sa largeur : vingt coudées, face à la grande salle. Il me dit : « C’est le Saint des Saints. »
      
         
${}^{5}Il mesura le mur de la Maison : six coudées ; la largeur de la construction latérale était de quatre coudées, tout autour de la Maison. 
${}^{6}Les chambres latérales étaient superposées ; il y en avait trois étages de trente ; elles s’enfonçaient dans le mur de l’édifice des chambres latérales tout autour, pour y être fixées, mais sans qu’elles soient fixées dans le mur de la Maison. 
${}^{7}Ces chambres allaient en s’élargissant, d’un étage à l’autre ; l’augmentation était faite au détriment du mur, d’un étage à l’autre, tout autour de la Maison. C’est pourquoi l’édifice s’élargissait vers le haut. De l’étage inférieur on montait à l’étage intermédiaire vers celui d’en haut.
${}^{8}Et je vis tout autour de l’édifice une élévation, mesurant une canne entière, à la base des chambres annexes ; ce soubassement avait six coudées. 
${}^{9}La largeur du mur appartenant à la construction latérale, à l’extérieur, était de cinq coudées ; quant à l’espace laissé entre les chambres latérales appartenant à la Maison 
${}^{10}et les salles, il était d’une largeur de vingt coudées tout autour de la Maison. 
${}^{11}Comme entrées de la construction latérale vers l’espace laissé libre, il y avait une entrée en direction du nord et une autre en direction du sud ; la largeur de l’espace laissé libre était de cinq coudées tout autour.
${}^{12}Le bâtiment qui faisait face au préau, du côté ouest, avait une largeur de soixante-dix coudées ; le mur du bâtiment avait cinq coudées de largeur sur tout le pourtour ; sa longueur était de quatre-vingt-dix coudées. 
${}^{13}Il mesura la Maison ; sa longueur : cent coudées ; le préau, le bâtiment, ses murs avaient cent coudées de longueur. 
${}^{14}La largeur de la façade de la Maison et du préau, à l’orient, était de cent coudées. 
${}^{15}Il mesura la longueur du bâtiment, du côté du préau qui est derrière, ainsi que ses galeries, de part et d’autre : cent coudées.
      La grande salle à l’intérieur, les vestibules donnant sur la cour, 
${}^{16}les seuils, les fenêtres grillagées, les galeries, tout autour sur trois côtés, face au seuil, étaient de bois tout autour, depuis le sol jusqu’aux fenêtres ; les fenêtres aussi en étaient recouvertes. 
${}^{17}Jusqu’au-dessus de l’entrée, jusqu’à l’intérieur de la Maison, ainsi qu’à l’extérieur et sur tout le pourtour du mur, à l’intérieur et à l’extérieur, des figures 
${}^{18}étaient représentées : des Kéroubim et des palmiers, un palmier entre deux Kéroubim ; chaque Kéroub avait deux faces : 
${}^{19}une face d’homme, tournée vers le palmier, d’un côté, et une face de lion, vers le palmier, de l’autre, tout cela représenté sur le pourtour de la Maison. 
${}^{20}Depuis le sol jusqu’au-dessus de l’entrée, sur le mur de la grande salle, étaient représentés des Kéroubim et des palmiers. 
${}^{21}La grande salle avait des montants carrés.
      Devant le sanctuaire, ce que l’on voyait avait l’aspect 
${}^{22}d’un autel de bois ; sa hauteur : trois coudées ; sa longueur : deux coudées ; ses têtes d’angles, son socle et ses parois étaient en bois. L’homme me dit : « C’est la table qui est devant le Seigneur. » 
${}^{23}La grande salle avait une double porte, et le sanctuaire, 
${}^{24}une double porte ; les portes avaient deux vantaux pivotants : deux pour une porte et deux pour l’autre. 
${}^{25}On avait représenté sur les portes de la grande salle des Kéroubim et des palmiers, comme ceux qu’on avait faits sur les murs. Un auvent de bois s’appuyait sur la façade du vestibule, à l’extérieur. 
${}^{26}Des fenêtres grillagées et des palmiers se trouvaient de part et d’autre, sur les côtés du vestibule, sur les chambres latérales de la Maison et les auvents.
      
         
      \bchapter{}
      \begin{verse}
${}^{1}L’homme me fit sortir dans la cour extérieure, en direction du nord ; puis il me fit entrer dans la salle qui fait face au préau et au bâtiment, vers le nord. 
${}^{2}Vers l’entrée nord, la longueur de la façade était de cent coudées, et la largeur de cinquante coudées. 
${}^{3}Devant les vingt coudées de la cour intérieure et devant le dallage de la cour extérieure, il y avait des galeries superposées sur trois étages. 
${}^{4}Devant les salles, une allée large de dix coudées, vers l’intérieur, et longue de cent coudées ; leurs entrées étaient au nord. 
${}^{5}Les salles supérieures étaient plus étroites car les galeries empiétaient sur elles, plus que sur les salles inférieures et intermédiaires du bâtiment.
${}^{6}Ces salles formaient trois étages et n’avaient pas de colonnes semblables aux colonnes des cours ; c’est pour cela qu’il y avait un rétrécissement par rapport aux salles inférieures et intermédiaires, en partant du sol. 
${}^{7}Le mur extérieur, correspondant aux salles et en face d’elles, en direction de la cour intérieure, avait cinquante coudées de long. 
${}^{8}Car la longueur des salles vers la cour extérieure était de cinquante coudées ; par contre, face à la grande salle, elle était de cent coudées. 
${}^{9}En dessous des mêmes salles, débouchait l’entrée orientale, y donnant accès depuis la cour extérieure. 
${}^{10}Sur la largeur du mur de la cour, en direction de l’orient, face au préau et face au bâtiment, il y avait des salles, 
${}^{11}avec un chemin devant elles ; elles avaient le même aspect que les salles qui étaient en direction du nord : même longueur et même largeur, mêmes sorties, même ordonnance et mêmes entrées. 
${}^{12}De même pour les entrées des salles qui sont en direction du midi : on y accédait par une entrée à l’extrémité du chemin, face au mur de protection situé en direction de l’orient.
${}^{13}L’homme me dit : « Les salles du nord et les salles du midi face au préau sont les salles du sanctuaire ; car c’est là que les prêtres qui s’approchent du Seigneur mangent les choses très saintes. C’est là qu’ils déposent les choses très saintes : l’offrande de céréales, les victimes du sacrifice pour la faute et du sacrifice de réparation. Ce lieu est saint. 
${}^{14}Une fois entrés, les prêtres ne sortiront pas du lieu saint vers la cour extérieure, mais ils déposeront là les vêtements avec lesquels ils officient : ces vêtements sont saints. Ils mettront d’autres vêtements ; ils pourront alors s’approcher des endroits destinés au peuple. »
${}^{15}L’homme compléta les mesures intérieures de la Maison ; il me fit sortir par le chemin de la porte qui fait face à l’orient et la mesura tout autour. 
${}^{16}Il mesura du côté de l’orient avec la canne à mesurer : cinq cents coudées, d’après la canne à mesurer. Il se tourna 
${}^{17}et mesura du côté nord : cinq cents coudées d’après la canne à mesurer. Il se tourna 
${}^{18}et mesura le côté du midi : cinq cents coudées, d’après la canne à mesurer. 
${}^{19}Il se tourna vers le côté occidental ; il mesura : cinq cents coudées, d’après la canne à mesurer. 
${}^{20}Il mesura l’ensemble sur les quatre côtés ; la muraille tout autour était de cinq cents sur la longueur, et de cinq cents sur la largeur. Ce mur séparait le saint du profane.
      
         
      \bchapter{}
      \begin{verse}
${}^{1}L’homme\\me conduisit vers la porte, celle qui fait face à l’orient ; 
${}^{2} et voici que la gloire du Dieu d’Israël arrivait de l’orient. Le bruit qu’elle faisait ressemblait au bruit des grandes eaux\\, et la terre resplendissait de cette gloire.  
${}^{3} Cette vision ressemblait à celle que j’avais eue\\lorsque le Seigneur était venu\\détruire la ville ; elle ressemblait aussi à la vision que j’avais eue quand j’étais au bord du fleuve Kebar\\. Alors je tombai face contre terre.
${}^{4}La gloire du Seigneur entra dans la Maison par la porte qui fait face à l’orient. 
${}^{5}L’esprit m’enleva et me transporta dans la cour intérieure : voici que la gloire du Seigneur remplissait la Maison. 
${}^{6}Et j’entendis une voix qui venait de la Maison, tandis que l’homme\\se tenait près de moi. 
${}^{7}Cette voix me disait : « Fils d’homme, c’est ici le lieu de mon trône, le lieu sur lequel je pose les pieds, et là je demeurerai au milieu des fils d’Israël, pour toujours. La maison d’Israël ne rendra plus impur mon saint nom ; ni elle, ni ses rois avec leurs débauches, ni les cadavres de ses rois avec leurs tombes. 
${}^{8}Ils ont placé leur seuil à côté de mon seuil, les montants de leurs portes à côté des miens, avec un mur entre moi et eux. Ils ont rendu impur mon saint nom par les abominations qu’ils ont commises ; aussi je les ai exterminés dans ma colère. 
${}^{9}Maintenant ils éloigneront de moi leurs débauches ainsi que les cadavres de leurs rois, et je demeurerai au milieu d’eux pour toujours.
${}^{10}Toi, fils d’homme, décris cette Maison à la maison d’Israël, pour qu’ils soient honteux de leurs fautes en mesurant les dimensions de la Maison. 
${}^{11}S’ils sont honteux de tout ce qu’ils ont commis, fais-leur connaître le plan de la Maison, sa disposition, ses sorties, ses entrées, tout son plan et toutes les prescriptions qui la concernent, tout son plan et toutes ses lois. Écris-les sous leurs yeux, afin qu’ils gardent tout son plan et toutes ses prescriptions, et qu’ils les appliquent. 
${}^{12}Telle est la loi de la Maison : au sommet de la montagne, tout son territoire, tout autour, est très saint. Voilà ! Telle est la loi de la Maison. »
${}^{13}Voici les dimensions de l’autel en coudées – cette coudée ancienne valant une coudée nouvelle et un palme. Le creux qui est à la base, mesuré avec cette coudée, a une coudée de large ; il s’étend, jusqu’au rebord qui en fait le tour, sur un empan. Voici la hauteur de l’autel : 
${}^{14}de la base appelée « creux-de-la-terre », jusqu’au socle inférieur, deux coudées ; largeur : une coudée ; depuis le petit socle jusqu’au grand socle : quatre coudées ; largeur : une coudée. 
${}^{15}Le foyer, appelé « montagne-de-Dieu », haut de quatre coudées, a quatre cornes par-dessus. 
${}^{16}Le foyer a douze coudées de long sur douze de large ; il forme un carré par ses quatre côtés. 
${}^{17}Le socle, pour ses quatre côtés, a quatorze coudées de long sur quatorze coudées de large. Le rebord qui l’entoure est d’une demi-coudée, et le creux qui l’entoure est d’une coudée. Les marches font face à l’orient.
${}^{18}L’homme me dit : « Fils d’homme, ainsi parle le Seigneur Dieu : Voici les prescriptions qui concernent l’autel, pour le jour où on le bâtira, en vue d’offrir sur lui l’holocauste et d’y répandre le sang. 
${}^{19}Aux prêtres lévites, ceux de la descendance de Sadoc, qui s’approchent de moi pour me servir – oracle du Seigneur Dieu –, tu donneras un jeune taureau en vue du sacrifice pour la faute. 
${}^{20}Tu prendras de son sang, tu en mettras sur les quatre cornes de l’autel, sur les quatre angles du socle et sur le rebord qui l’entoure ; ainsi, pour l’autel, tu accompliras le sacrifice pour la faute et le rite d’expiation.
${}^{21}Puis tu prendras le taureau du sacrifice pour la faute, et on le brûlera dans un endroit déterminé de la Maison, à l’extérieur du sanctuaire. 
${}^{22}Le deuxième jour, tu amèneras en vue du sacrifice pour la faute un bouc sans défaut, et on accomplira ce sacrifice pour l’autel comme on l’a fait avec le taureau. 
${}^{23}Quand tu auras achevé ce sacrifice, tu amèneras un jeune taureau sans défaut et un bélier sans défaut pris dans le troupeau. 
${}^{24}Tu les amèneras devant le Seigneur ; les prêtres jetteront sur eux du sel et les offriront en holocauste pour le Seigneur. 
${}^{25}Sept jours durant, tu feras quotidiennement le sacrifice du bouc pour la faute. On fera de même pour le jeune taureau et le bélier pris dans le troupeau, tous deux sans défaut. 
${}^{26}Sept jours durant, on fera le rite d’expiation pour l’autel, on le purifiera. C’est ainsi qu’on l’inaugurera. 
${}^{27}On arrivera au terme de ces jours. Le huitième jour et les suivants, les prêtres offriront sur l’autel vos holocaustes et vos sacrifices de paix ; alors je vous serai favorable – oracle du Seigneur Dieu. »
      
         
      \bchapter{}
      \begin{verse}
${}^{1}L’homme me ramena vers la porte extérieure du sanctuaire, celle qui fait face à l’orient ; elle était fermée. 
${}^{2}Le Seigneur me dit : « Cette porte restera fermée ; on ne l’ouvrira pas ; personne n’entrera par là ; car le Seigneur, le Dieu d’Israël, est entré par là ; elle restera fermée. 
${}^{3}Le prince, en sa qualité de prince, pourra s’y asseoir et prendre son repas devant le Seigneur. C’est par le vestibule de la porte qu’il entrera, et il sortira par ce chemin. »
${}^{4}L’homme me fit entrer par la porte du nord, jusqu’à la façade de la Maison. Je regardai : voici que la gloire du Seigneur remplissait la maison du Seigneur. Alors je tombai face contre terre. 
${}^{5}Le Seigneur me dit : « Fils d’homme, sois attentif et regarde de tes yeux, écoute de tes oreilles tout ce que je vais te dire au sujet de toutes les prescriptions relatives à la maison du Seigneur et concernant toutes ses lois ; sois attentif à ceux qui entrent dans la Maison et à tous ceux qui sortent du sanctuaire. 
${}^{6}Tu diras à ces rebelles, à la maison d’Israël : Ainsi parle le Seigneur Dieu : C’en est trop de toutes vos abominations, maison d’Israël ! 
${}^{7}Vous introduisez des étrangers, incirconcis de cœur, incirconcis de chair ; ils sont dans mon sanctuaire, ils profanent ma Maison ; et puis vous présentez ma nourriture – la graisse et le sang. Vous rompez mon alliance par toutes vos abominations ! 
${}^{8}Vous n’avez pas maintenu les observances relatives à mes choses saintes, mais vous avez établi des étrangers, afin qu’ils maintiennent à votre place ces observances, dans mon sanctuaire. 
${}^{9}Ainsi parle le Seigneur Dieu : Aucun étranger, incirconcis de cœur et incirconcis de chair, n’entrera dans mon sanctuaire ; aucun étranger qui réside au milieu des fils d’Israël.
${}^{10}Quant aux Lévites qui se sont éloignés de moi au temps où Israël errait – ils erraient loin de moi, à la poursuite de leurs idoles immondes –, ils porteront le poids de leur péché. 
${}^{11}Aussi, dans mon sanctuaire, ils seront des serviteurs veillant sur les portes de la Maison, et des serviteurs de la Maison : c’est eux qui immoleront les victimes de l’holocauste et du sacrifice pour le peuple ; et c’est eux qui se tiendront devant le peuple pour le servir. 
${}^{12}Parce qu’ils l’ont servi devant les idoles immondes et parce qu’ils firent tomber la maison d’Israël dans le péché, à cause de cela, je lève la main contre eux – oracle du Seigneur Dieu –, ils porteront le poids de leur péché. 
${}^{13}Ils ne s’approcheront pas de moi pour exercer mon sacerdoce ni pour s’approcher de toutes mes choses saintes, des choses très saintes ; ils porteront le poids de leur déshonneur et des abominations qu’ils ont commises. 
${}^{14}Je les établirai pour maintenir les observances dans la Maison, pour tout ce qui a trait à son service et tout ce qui s’y fait.
${}^{15}Les prêtres lévites, fils de Sadoc, ont maintenu les observances de mon sanctuaire, lorsque les fils d’Israël erraient loin de moi : ce sont eux qui s’approcheront de moi pour me servir ; ils se tiendront devant moi pour me présenter la graisse et le sang – oracle du Seigneur Dieu. 
${}^{16}Ce sont eux qui entreront dans mon sanctuaire, eux qui s’approcheront de ma table, pour me servir et qui maintiendront mes observances. 
${}^{17}Quand ils entreront par les portes de la cour intérieure, ils mettront des vêtements de lin. Ils ne porteront pas de laine, quand ils officieront aux portes de la cour intérieure et dans la Maison. 
${}^{18}Ils auront sur la tête des turbans de lin et aux reins des caleçons de lin. Ils n’auront pas de ceinture, à cause de la sueur. 
${}^{19}Quand ils sortiront dans la cour extérieure, vers le peuple, ils ôteront les vêtements dans lesquels ils auront officié et les déposeront dans des salles du sanctuaire. Ils prendront d’autres vêtements afin de ne pas communiquer la sainteté au peuple par leurs vêtements. 
${}^{20}Ils ne se raseront pas la tête, ils ne se laisseront pas pousser les cheveux, mais ils les tailleront soigneusement. 
${}^{21}Aucun prêtre ne boira de vin quand il devra entrer dans la cour intérieure. 
${}^{22}Ils n’épouseront pas de femme veuve ou répudiée, mais seulement des vierges de la maison d’Israël ; ils pourront épouser la veuve d’un prêtre. 
${}^{23}Ils enseigneront à mon peuple à distinguer entre le saint et le profane, et ils lui apprendront à distinguer entre le pur et l’impur. 
${}^{24}En cas de procès, c’est eux qui interviendront en juges ; ils jugeront le cas selon mon droit ; ils observeront mes lois et mes décrets dans toutes mes solennités et ils sanctifieront mes sabbats. 
${}^{25}Le prêtre n’entrera pas chez un homme mort, car il deviendrait impur ; cependant pour un père, une mère, un fils, une fille, un frère, une sœur qui n’a pas appartenu à un homme, un prêtre pourra se rendre impur. 
${}^{26}Après sa purification, on comptera sept jours. 
${}^{27}Puis, le jour où il entrera dans le sanctuaire, dans la cour intérieure, pour servir dans le sanctuaire, il présentera son sacrifice pour la faute – oracle du Seigneur Dieu.
${}^{28}Ils n’auront qu’un seul héritage : leur héritage, c’est moi. Vous ne leur donnerez pas de propriétés en Israël : c’est moi leur propriété. 
${}^{29}Eux auront pour nourriture l’offrande de céréales, la victime du sacrifice pour la faute et celle du sacrifice de réparation. Tout ce qui est voué à l’anathème en Israël sera pour eux. 
${}^{30}Le meilleur de toutes les prémices, toutes les contributions de toutes sortes, parmi toutes celles que vous prélèverez, seront pour les prêtres ; le meilleur de vos fournées, vous le donnerez au prêtre, pour que la bénédiction repose sur votre maison. 
${}^{31}Les prêtres ne mangeront d’aucune bête crevée ou déchirée, que ce soit un oiseau ou du bétail.
      
         
      \bchapter{}
      \begin{verse}
${}^{1}Lorsque le pays vous reviendra en héritage, vous prélèverez une part pour le Seigneur ; elle sera sainte, prise sur le pays ; sa longueur sera de vingt-cinq mille coudées, sa largeur de vingt mille. Ce sera un territoire entièrement saint, de toute part. 
${}^{2}On y réservera, pour le sanctuaire, un carré de cinq cents coudées sur cinq cents, avec, autour, un terrain de cinquante coudées. 
${}^{3}Sur ce que vous aurez prélevé, tu mesureras en longueur vingt-cinq mille coudées et dix mille en largeur ; là sera donc le sanctuaire, le Saint des Saints. 
${}^{4}Cette partie sainte, prise sur le pays, sera pour les prêtres qui desservent le sanctuaire et s’approchent du Seigneur pour le servir ; ils auront ainsi un emplacement pour des maisons et un lieu saint pour le sanctuaire.
${}^{5}Une autre partie de vingt-cinq mille coudées de long et dix mille de large sera pour les Lévites qui desservent la Maison ; ils y posséderont des lieux d’habitation.
${}^{6}Vous donnerez en propriété pour la ville cinq mille coudées de largeur et vingt-cinq mille de longueur, contiguës à la part du sanctuaire ; ce sera pour toute la maison d’Israël.
${}^{7}Pour le prince il y aura un territoire de chaque côté de la part sainte et de la propriété de la ville, longeant la part sainte et la propriété de la ville, du côté occidental vers l’ouest, et du côté oriental vers l’est ; sa longueur correspondra à chacun des lots, depuis la frontière occidentale jusqu’à la frontière orientale 
${}^{8}du pays ; ce sera sa propriété en Israël. Ainsi mes princes n’exploiteront plus mon peuple ; ils donneront le pays à la maison d’Israël, à ses tribus.
${}^{9}Ainsi parle le Seigneur Dieu : C’en est trop, princes d’Israël ! Loin de vous la violence et la dévastation ; pratiquez le droit et la justice ; cessez vos exactions contre mon peuple – oracle du Seigneur Dieu ! 
${}^{10}Ayez des balances justes, un épha juste, un bath juste. 
${}^{11}Que l’épha, pour le grain, et le bath, pour les liquides, soient de même capacité ; que le bath contienne un dixième de homer et l’épha un dixième de homer ; c’est en homer que sera jaugée leur capacité. 
${}^{12}Le sicle vaudra vingt guéras ; vingt sicles, plus vingt-cinq sicles, plus quinze sicles vaudront pour vous une mine.
${}^{13}Voici la contribution que vous prélèverez : un sixième d’épha par homer de blé et un sixième d’épha par homer d’orge. 
${}^{14}Décret concernant l’huile, le bath d’huile : on prélèvera un dixième de bath par kor, dix baths étant un homer, puisque dix baths font un kor. 
${}^{15}On prélèvera un mouton par troupeau de deux cents têtes des pâturages d’Israël lors de l’offrande de céréales, lors de l’holocauste et des sacrifices de paix, pour faire le rite d’expiation sur le peuple – oracle du Seigneur Dieu. 
${}^{16}Tous les gens du pays participeront à cette contribution au profit du prince en Israël.
${}^{17}Au prince incomberont les holocaustes, l’offrande de céréales et la libation, lors des pèlerinages, des nouvelles lunes, des sabbats, lors de toutes les solennités de la maison d’Israël ; c’est lui qui assurera le sacrifice pour la faute, ainsi que l’offrande de céréales, l’holocauste et les sacrifices de paix, pour faire le rite d’expiation en faveur de la maison d’Israël.
${}^{18}Ainsi parle le Seigneur Dieu : Le premier mois, le premier du mois, tu prendras un jeune taureau sans défaut et tu feras le sacrifice pour la faute en faveur du sanctuaire. 
${}^{19}Le prêtre prendra du sang du sacrifice et en mettra sur les montants de la Maison, sur les quatre angles du socle de l’autel et sur le montant de la porte de la cour intérieure. 
${}^{20}Tu feras de même le sept du mois, pour qui a commis une faute par inadvertance ou par ignorance. Vous ferez le rite d’expiation de la Maison.
${}^{21}Le premier mois, le quatorzième jour du mois, ce sera pour vous la Pâque, une fête de sept jours ; on mangera des pains sans levain. 
${}^{22}Ce jour-là, le prince offrira, pour lui-même et pour tout le peuple du pays, un taureau en sacrifice pour la faute. 
${}^{23}Durant les sept jours de la fête, il offrira en holocauste pour le Seigneur sept taureaux et sept béliers sans défaut, chacun des sept jours, et, en sacrifice pour la faute, un bouc par jour. 
${}^{24}Il fera l’offrande d’un épha de farine par taureau et d’un épha par bélier, avec un hine d’huile par épha.
${}^{25}Le septième mois, le quinzième jour du mois, lors de la fête des Tentes, il fera de même, durant les sept jours : même sacrifice pour la faute, même holocauste, même offrande, même présentation d’huile.
       
      
         
      \bchapter{}
      \begin{verse}
${}^{1}Ainsi parle le Seigneur Dieu : La porte de la cour intérieure, qui fait face à l’orient, sera fermée durant les six jours de travail ; mais, le jour du sabbat, elle sera ouverte ; elle sera également ouverte au jour de la nouvelle lune. 
${}^{2}Le prince viendra de l’extérieur, il entrera par le vestibule de la porte et il se tiendra près d’un montant de la porte ; puis les prêtres offriront l’holocauste du prince et ses sacrifices de paix. Le prince se prosternera sur le seuil de la porte puis sortira ; mais la porte ne sera pas refermée avant le soir. 
${}^{3}Les gens du pays se prosterneront devant le Seigneur à l’entrée de cette porte, lors des sabbats et des nouvelles lunes. 
${}^{4}Le jour du sabbat, le prince présentera au Seigneur un holocauste de six agneaux sans défaut et d’un bélier sans défaut. 
${}^{5}Il fera une offrande d’un épha de farine, pour le bélier ; pour les agneaux, l’offrande sera laissée à sa discrétion ; il ajoutera un hine d’huile par épha. 
${}^{6}Le jour de la nouvelle lune, l’holocauste sera d’un jeune taureau sans défaut, de six agneaux et d’un bélier sans défaut. 
${}^{7}Le prince fera aussi l’offrande d’un épha de farine pour le taureau et d’un épha pour le bélier ; pour les agneaux, ce sera selon ses moyens ; il ajoutera un hine d’huile par épha.
${}^{8}Quand le prince entrera, il entrera par le vestibule de la porte, et il sortira par ce chemin. 
${}^{9}Quand les gens du pays viendront devant le Seigneur, lors des solennités, ceux qui entreront par la porte nord, pour se prosterner, sortiront par la porte du midi, et ceux qui entreront par la porte du midi sortiront par la porte du nord. On ne reprendra pas la porte par laquelle on est entré ; on sortira à l’opposé. 
${}^{10}Et le prince, au milieu d’eux, entrera quand ils entreront, et sortira quand ils sortiront.
${}^{11}Lors des pèlerinages et des solennités, l’offrande sera d’un épha de farine pour le taureau et d’un épha pour le bélier ; pour les agneaux, ce sera laissé à sa discrétion ; il ajoutera un hine d’huile par épha. 
${}^{12}Lorsque le prince fera pour le Seigneur un holocauste volontaire, ou un sacrifice de paix volontaire, on lui ouvrira la porte qui fait face à l’orient, et il fera son holocauste et ses sacrifices de paix, comme il fait le jour du sabbat ; puis il sortira et on fermera la porte dès sa sortie.
${}^{13}Avec un agneau de l’année, sans défaut, tu offriras, chaque jour, un holocauste au Seigneur ; tu le feras chaque matin. 
${}^{14}En outre, tu feras, chaque matin, une offrande d’un sixième d’épha de farine et, en huile, d’un tiers de hine, pour humecter la farine. Telle est l’offrande pour le Seigneur. C’est un décret perpétuel, à jamais. 
${}^{15}On présente l’agneau, avec l’offrande et l’huile, chaque matin, pour l’holocauste perpétuel.
${}^{16}Ainsi parle le Seigneur Dieu : Lorsque le prince fait à l’un de ses fils un don pris sur son héritage, ce don appartient à ses fils ; c’est leur propriété héréditaire. 
${}^{17}Lorsque le prince fait à l’un de ses serviteurs un don pris sur son héritage, ce don appartient au serviteur jusqu’à l’année de l’affranchissement ; après quoi il revient au prince. Seule la part de l’héritage donnée aux fils du prince reste en leur possession. 
${}^{18}Le prince ne prendra rien sur l’héritage du peuple en leur extorquant leur propriété ; c’est sur sa propriété qu’il constituera l’héritage de ses fils, afin que personne de mon peuple ne soit dispersé, chacun loin de sa propriété.
${}^{19}Par l’entrée qui est à côté de la porte, l’homme me conduisit vers les salles saintes, tournées vers le nord et destinées aux prêtres. Il y avait au fond un espace, vers l’ouest. 
${}^{20}Il me dit : « C’est le lieu où les prêtres feront bouillir les victimes du sacrifice de réparation et du sacrifice pour le péché, et feront cuire l’offrande de céréales, sans qu’on en fasse rien sortir vers la cour extérieure, ce qui communiquerait la sainteté au peuple. » 
${}^{21}Il me fit sortir vers la cour extérieure et me fit passer près des quatre angles de la cour : il y avait une cour à chaque angle. 
${}^{22}Situées dans les quatre angles de la cour, elles étaient exiguës, longues de quarante coudées et larges de trente ; les quatre cours étaient de mêmes dimensions. 
${}^{23}Une rangée de pierres les entourait toutes quatre, à la base de laquelle des fourneaux avaient été aménagés, tout autour. 
${}^{24}L’homme me dit : « Ce sont les cuisines ; c’est là que les serviteurs de la Maison feront bouillir les sacrifices du peuple. »
      
         
      \bchapter{}
      \begin{verse}
${}^{1}L’homme\\me fit revenir à l’entrée de la Maison, et voici : sous le seuil de la Maison, de l’eau jaillissait vers l’orient, puisque la façade de la Maison était du côté de l’orient. L’eau descendait de dessous le côté droit de la Maison, au sud de l’autel. 
${}^{2}L’homme\\me fit sortir par la porte du nord et me fit faire le tour par l’extérieur, jusqu’à la porte\\qui fait face à l’orient, et là encore l’eau coulait du côté droit. 
${}^{3}L’homme s’éloigna vers l’orient, un cordeau à la main, et il mesura une distance de mille coudées ; alors il me fit traverser l’eau : j’en avais jusqu’aux chevilles. 
${}^{4}Il mesura encore mille coudées et me fit traverser l’eau : j’en avais jusqu’aux genoux. Il mesura encore mille coudées et me fit traverser : j’en avais jusqu’aux reins. 
${}^{5}Il en mesura encore mille : c’était un torrent que je ne pouvais traverser ; l’eau avait grossi, il aurait fallu nager : c’était un torrent infranchissable. 
${}^{6}Alors il me dit : « As-tu vu, fils d’homme ? » Puis il me ramena\\au bord du torrent. 
${}^{7}Quand il m’eut ramené, voici qu’il y avait au bord du torrent, de chaque côté, des arbres en grand nombre. 
${}^{8}Il me dit : « Cette eau coule vers la région\\de l’orient, elle descend dans la vallée du Jourdain\\, et se déverse dans la mer Morte\\, dont elle assainit les eaux. 
${}^{9}En tout lieu où parviendra le torrent\\, tous les animaux pourront vivre et foisonner. Le poisson sera très abondant, car cette eau assainit tout ce qu’elle pénètre\\, et la vie apparaît en tout lieu où arrive le torrent. 
${}^{10}Alors des pêcheurs se tiendront sur la rive depuis Enn-Guèdi jusqu’à Enn-Églaïm ; on y fera sécher les filets. Les espèces de poissons seront aussi nombreuses que celles de la Méditerranée. 
${}^{11}Mais ses marais et ses bassins ne seront pas assainis : ils seront réservés au sel. 
${}^{12}Au bord du torrent, sur les deux rives, toutes sortes d’arbres fruitiers pousseront ; leur feuillage ne se flétrira pas et leurs fruits ne manqueront pas. Chaque mois ils porteront des fruits nouveaux, car cette eau vient du sanctuaire. Les fruits seront une nourriture, et les feuilles un remède. »
      
         
${}^{13}Ainsi parle le Seigneur Dieu : Voici les frontières d’après lesquelles vous partagerez le pays entre les douze tribus d’Israël, avec deux parts pour Joseph. 
${}^{14}Vous l’aurez comme héritage, chacun à part égale, car j’ai juré, la main levée, de le donner à vos pères ; ce pays vous revient en héritage.
${}^{15}Voici la frontière du pays. Du côté nord, depuis la Méditerranée : la route de Hètlone – celle qui va à Cedâd –, 
${}^{16}Hamath, Bérotaï, Sibraïm, qui est entre le territoire de Damas et le territoire de Hamath, Hacer-hat-Tikone qui est vers le territoire de Haurane. 
${}^{17}Ainsi la frontière ira de la mer jusqu’à Haçar-Einane, le territoire de Damas étant au nord ainsi que le territoire de Hamath. Tel est le côté nord. 
${}^{18}Du côté oriental : entre le Haurane et Damas, entre le Galaad et la terre d’Israël, le Jourdain servira de frontière, jusqu’à la mer orientale, vers Tamar. Tel est le côté de l’orient. 
${}^{19}Du côté méridional, au midi : de Tamar jusqu’aux eaux de Mériba de Cadès, jusqu’au Torrent d’Égypte vers la Méditerranée. Tel est le côté du midi, vers le Néguev. 
${}^{20}Et du côté occidental : la Méditerranée, depuis la frontière sud jusqu’en face de l’Entrée-de-Hamath. Tel est le côté occidental.
${}^{21}Vous partagerez le pays entre vous – entre les tribus d’Israël. 
${}^{22}Cela vous reviendra en héritage. Vous le ferez pour vous et pour les immigrés résidant parmi vous, qui ont engendré des fils parmi vous ; ils seront pour vous comme un Israélite de souche au milieu des fils d’Israël ; avec vous ils recevront une part d’héritage, au milieu des tribus d’Israël. 
${}^{23}C’est dans la tribu où réside l’immigré, c’est là que vous lui donnerez sa part d’héritage – oracle du Seigneur Dieu.
      
         
      \bchapter{}
      \begin{verse}
${}^{1}Voici les noms des tribus et leurs territoires. Depuis l’extrême nord, la frontière longe la route de Hètlone : l’Entrée-de-Hamath, Haçar-Einane – le territoire de Damas étant au nord –, elle longe Hamath. Du côté oriental au côté occidental, c’est la part de Dane. 
${}^{2}Le long du territoire de Dane, du côté oriental au côté occidental, c’est la part d’Asher. 
${}^{3}Le long du territoire d’Asher, du côté oriental au côté occidental, c’est la part de Nephtali. 
${}^{4}Le long du territoire de Nephtali, du côté oriental au côté occidental, c’est la part de Manassé. 
${}^{5}Le long du territoire de Manassé, du côté oriental au côté occidental, c’est la part d’Éphraïm. 
${}^{6} Le long du territoire d’Éphraïm, du côté oriental au côté occidental, c’est la part de Roubène. 
${}^{7}Le long du territoire de Roubène, du côté oriental au côté occidental, c’est la part de Juda. 
${}^{8}Le long du territoire de Juda, du côté oriental au côté occidental, sera la part que vous prélèverez : elle aura vingt-cinq mille coudées en largeur et même longueur que l’une des parts, du côté oriental au côté occidental ; le sanctuaire sera au milieu.
      
         
${}^{9}La part que vous prélèverez là-dessus pour le Seigneur aura, en longueur, vingt-cinq mille coudées et, en largeur, vingt mille. 
${}^{10}Voici quelle sera la part sainte. Elle aura pour les prêtres : au nord, vingt-cinq mille coudées ; à l’ouest, une largeur de dix mille ; à l’orient, une largeur de dix mille ; au sud, une longueur de vingt-cinq mille. Le sanctuaire du Seigneur sera au milieu. 
${}^{11}Les prêtres consacrés, fils de Sadoc, ont maintenu mes observances, ils ne se sont pas égarés dans l’erreur des fils d’Israël, comme les Lévites l’ont fait ; 
${}^{12}il leur reviendra une part de la zone prélevée sur le pays, part très sainte, voisine du territoire des Lévites. 
${}^{13}Quant aux Lévites, leur territoire sera contigu à celui des prêtres : vingt-cinq mille coudées en longueur et dix mille en largeur. Chacun aura une longueur de vingt-cinq mille coudées et une largeur de dix mille. 
${}^{14}Rien n’en sera vendu, on ne fera pas d’échange, on n’aliénera pas les premiers fruits du pays : elles sont consacrées au Seigneur.
${}^{15}Les cinq mille coudées qui restent en largeur, le long des vingt-cinq mille, formeront la zone profane de la ville : habitats et pâturages, la ville étant au milieu. 
${}^{16}Voici les dimensions de la ville : côté nord, quatre mille cinq cents coudées ; côté sud, quatre mille cinq cents ; côté est, quatre mille cinq cents, et côté ouest, quatre mille cinq cents. 
${}^{17}Les pâturages de la ville auront : vers le nord, deux cent cinquante coudées ; vers le sud, deux cent cinquante ; vers l’orient, deux cent cinquante et vers l’ouest deux cent cinquante. 
${}^{18}Ce qui restera, contigu en largeur à cette part sainte, aura dix mille coudées vers l’orient et dix mille vers l’ouest ; son produit nourrira ceux qui travaillent dans la ville. 
${}^{19}La main-d’œuvre de la ville qui le cultivera viendra de toutes les tribus d’Israël. 
${}^{20}La part dans son ensemble aura vingt-cinq mille coudées sur vingt-cinq mille. De cette part sainte vous prélèverez un quadrilatère pour le domaine de la ville. 
${}^{21}Le reste sera pour le prince, des deux côtés de la part sainte et du domaine de la ville. Ce qui correspond aux autres lots, le long des vingt-cinq mille coudées de la part prélevée, jusqu’à la frontière orientale, et à l’ouest, le long des vingt-cinq mille coudées, jusqu’à la frontière occidentale, ce sera pour le prince. Il y aura donc, au centre, la part sainte avec le sanctuaire de la Maison. 
${}^{22}Hormis le domaine des Lévites et le domaine de la ville – situés au milieu de ce qui est au prince –, tout l’espace compris entre la frontière de Juda et la frontière de Benjamin sera au prince.
${}^{23}Le reste des tribus : du côté oriental au côté occidental, c’est la part de Benjamin. 
${}^{24}Le long du territoire de Benjamin, du côté oriental au côté occidental, c’est la part de Siméon. 
${}^{25}Le long du territoire de Siméon, du côté oriental au côté occidental, c’est la part d’Issakar. 
${}^{26}Le long du territoire d’Issakar, du côté oriental au côté occidental, c’est la part de Zabulon. 
${}^{27}Le long du territoire de Zabulon, du côté oriental au côté occidental, c’est la part de Gad. 
${}^{28}Le long du territoire de Gad, jusqu’au côté méridional, au sud, la frontière sera : de Tamar, jusqu’aux eaux de Mériba de Cadès, jusqu’au Torrent d’Égypte vers la Méditerranée. 
${}^{29}Tel est le pays qui reviendra en héritage aux tribus d’Israël, et telles seront leurs parts – oracle du Seigneur Dieu.
${}^{30}Voici les issues de la ville. En commençant par le côté nord – de quatre mille cinq cents coudées – 
${}^{31}les portes de la ville seront nommées d’après les tribus d’Israël. Trois portes au nord : la porte de Roubène, la porte de Juda, la porte de Lévi. 
${}^{32}Sur le côté oriental – de quatre mille cinq cents coudées – trois portes : la porte de Joseph, la porte de Benjamin, la porte de Dane. 
${}^{33}Côté méridional – de quatre mille cinq cents coudées – trois portes : la porte de Siméon, la porte d’Issakar, la porte de Zabulon. 
${}^{34}Côté occidental – de quatre mille cinq cents coudées – trois portes : la porte de Gad, la porte d’Asher, la porte de Nephtali. 
${}^{35}Le pourtour aura dix-huit mille coudées. À partir de ce jour, le nom de la ville sera « Le Seigneur-est-là ». 
