\psalm{68}{Dans l’abîme des eaux}
\psalmintro{Du maître de chœur. Sur l’air de «Des lis… ». De David.}

\begin{verse}
\versenb{2}Sauve-m\underline{o}i, mon Dieu: \\*
les eaux m\underline{o}ntent jusqu’à ma gorge! \\

\versenb{3}J’enfonce dans la v\underline{a}se du gouffre, \\*
ri\underline{e}n qui me retienne;~\psalmstar
je descends dans l’ab\underline{î}me des eaux, \\
le fl\underline{o}t m’engloutit. \\

\versenb{4}Je m’épu\underline{i}se à crier, \\*
ma g\underline{o}rge brûle.\psalmstar
Mes ye\underline{u}x se sont usés \\
d’att\underline{e}ndre mon Dieu. \\

\versenb{5}Plus abondants que les cheve\underline{u}x de ma tête, \\*
ceux qui m’en ve\underline{u}lent sans raison;~\psalmstar
ils sont nombre\underline{u}x, mes détracteurs, \\
à me ha\underline{ï}r injustement. \\

Moi qui n’ai ri\underline{e}n volé, \\
que devr\underline{a}i-je rendre?~\psalmstar
\versenb{6}Dieu, tu conn\underline{a}is ma folie, \\*
mes fautes sont à n\underline{u} devant toi. \\

\versenb{7}Qu’ils n’aient pas honte pour m\underline{o}i, ceux qui t’espèrent, \\*
Seigneur, Die\underline{u} de l’univers;~\psalmstar
qu’ils ne rougissent pas de m\underline{o}i, ceux qui te cherchent, \\
Die\underline{u} d’Israël! \\

\versenb{8}C’est pour toi que j’end\underline{u}re l’insulte, \\*
que la honte me co\underline{u}vre le visage: \\
\versenb{9}je suis un étrang\underline{e}r pour mes frères, \\*
un inconnu pour les f\underline{i}ls de ma mère. \\
\versenb{10}L’amour de ta mais\underline{o}n m’a perdu; \\*
on t’insulte, et l’insulte ret\underline{o}mbe sur moi. \\

\versenb{11}Si je pleure et m’imp\underline{o}se un jeûne, \\*
je reç\underline{o}is des insultes; \\
\versenb{12}si je revêts un hab\underline{i}t de pénitence, \\*
je deviens la f\underline{a}ble des gens: \\
\versenb{13}on parle de m\underline{o}i sur les places, \\*
les buveurs de v\underline{i}n me chansonnent. \\

\versenb{14}Et moi, je te pr\underline{i}e, Seigneur: \\*
c’est l’he\underline{u}re de ta grâce;~\psalmstar
dans ton grand amour, Die\underline{u}, réponds-moi, \\
par ta vérit\underline{é} sauve-moi. \\

\versenb{15}Tire-m\underline{o}i de la boue, \\*
sin\underline{o}n je m’enfonce:~\psalmstar
que j’échappe à ce\underline{u}x qui me haïssent, \\
à l’ab\underline{î}me des eaux. \\

\versenb{16}Que les flots ne me subm\underline{e}rgent pas, \\*
que le go\underline{u}ffre ne m’avale,~\psalmstar
que la gue\underline{u}le du puits \\
ne se ferme p\underline{a}s sur moi. \\

\versenb{17}Réponds-m\underline{o}i, Seigneur, \\*
car il est b\underline{o}n, ton amour;~\psalmstar
dans ta gr\underline{a}nde tendresse, \\
r\underline{e}garde-moi. \\

\versenb{18}Ne cache pas ton vis\underline{a}ge à ton serviteur; \\*
je suffoque: v\underline{i}te, réponds-moi.~\psalmstar
\versenb{19}Sois proche de m\underline{o}i, rachète-moi, \\*
paie ma ranç\underline{o}n à l’ennemi. \\

\versenb{20}Toi, tu le s\underline{a}is, on m’insulte: \\*
je suis bafou\underline{é}, déshonoré;~\psalmstar
to\underline{u}s mes oppresseurs \\
sont l\underline{à}, devant toi. \\

\versenb{21}L’insulte m’a broy\underline{é} le cœur, \\*
le m\underline{a}l est incurable;~\psalmstar
j’espérais un seco\underline{u}rs, mais en vain, \\
des consolateurs, je n’en ai p\underline{a}s trouvé. \\

\versenb{22}À mon pain, ils ont mêl\underline{é} du poison; \\*
quand j’avais soif, ils m’ont donn\underline{é} du vinaigre. \\
\versenb{23}$\[$Que leur table devi\underline{e}nne un piège, \\*
un guet-ap\underline{e}ns pour leurs convives! \\
\versenb{24}Que leurs yeux aveugl\underline{é}s ne voient plus, \\*
qu’à tout instant les r\underline{e}ins leur manquent! \\

\versenb{25}Déverse sur e\underline{u}x ta fureur, \\*
que le feu de ta col\underline{è}re les saisisse, \\
\versenb{26}que leur camp devi\underline{e}nne un désert, \\*
que nul n’hab\underline{i}te sous leurs tentes! \\

\versenb{27}Celui que tu frapp\underline{a}is, ils le pourchassent \\*
en comptant les co\underline{u}ps qu’il reçoit. \\
\versenb{28}Charge-les, fa\underline{u}te sur faute; \\*
qu’ils n’aient pas d’acc\underline{è}s à ta justice. \\
\versenb{29}Qu’ils soient rayés du l\underline{i}vre de vie, \\*
retranchés du n\underline{o}mbre des justes.$\]$ \\

\versenb{30}Et moi, humili\underline{é}, meurtri, \\*
que ton salut, Die\underline{u}, me redresse. \\
\versenb{31}Et je louerai le nom de Die\underline{u} par un cantique, \\*
je vais le magnifi\underline{e}r, lui rendre grâce. \\
\versenb{32}Cela plaît au Seigne\underline{u}r plus qu’un taureau, \\*
plus qu’une bête ayant c\underline{o}rnes et sabots. \\

\versenb{33}Les pauvres l’ont v\underline{u}, ils sont en fête: \\*
«Vie et joie, à vo\underline{u}s qui cherchez Dieu ! » \\
\versenb{34}Car le Seigneur éco\underline{u}te les humbles, \\*
il n’oublie pas les si\underline{e}ns emprisonnés. \\
\versenb{35}Que le ciel et la t\underline{e}rre le célèbrent, \\*
les mers et to\underline{u}t leur peuplement! \\

\versenb{36}Car Dieu viendra sauv\underline{e}r Sion \\*
et rebâtir les v\underline{i}lles de Juda. \\
Il en fera une habitati\underline{o}n, un héritage:~\psalmstar
\versenb{37}patrimoine pour les descendants de ses serviteurs, \\*
demeure pour ceux qui \underline{a}iment son nom. \\
\end{verse}

