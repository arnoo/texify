\psalm{41}{Mon âme a soif de Dieu}
\begin{verse}
\psalmintro{Du maître de chœur. Poème. Des fils de Coré.}

\versenb{2}Comme un c\underline{e}rf altéré \\*
ch\underline{e}rche l’eau vive,~\psalmstar
ainsi mon \underline{â}me te cherche \\
t\underline{o}i, mon Dieu. \\

\versenb{3}Mon âme a s\underline{o}if de Dieu, \\*
le Die\underline{u} vivant;~\psalmstar
quand pourr\underline{a}i-je m’avancer, \\
par\underline{a}ître face à Dieu? \\

\versenb{4}Je n’ai d’autre p\underline{a}in que mes larmes, \\*
le jo\underline{u}r, la nuit,~\psalmstar
moi qui chaque jo\underline{u}r entends dire: \\
«Où est-\underline{i}l ton Dieu ? » \\

\versenb{5}Je me souviens, \\*
et mon \underline{â}me déborde:~\psalmstar
en ce temps-là, \\
je franchiss\underline{a}is les portails! \\

Je conduisais vers la mais\underline{o}n de mon Dieu \\
la multit\underline{u}de en fête,~\psalmstar
parm\underline{i} les cris de joie \\
et les acti\underline{o}ns de grâce. \\

\versenb{6}Pourquoi te désol\underline{e}r, ô mon âme, \\*
et gém\underline{i}r sur moi?~\psalmstar
Espère en Dieu! De nouvea\underline{u} je rendrai grâce : \\
il est mon sauveur et mon Dieu! \\

\versenb{7}Si mon \underline{â}me se désole, \\*
je me souvi\underline{e}ns de toi,~\psalmstar
depuis les terres du Jourd\underline{a}in et de l’Hermon, \\
depuis mon h\underline{u}mble montagne. \\

\versenb{8}L’abîme appel\underline{a}nt l’abîme \\*
à la v\underline{o}ix de tes cataractes,~\psalmstar
la masse de tes fl\underline{o}ts et de tes vagues \\
a pass\underline{é} sur moi. \\

\versenb{9}Au long du jo\underline{u}r, le Seigneur \\*
m’env\underline{o}ie son amour;~\psalmstar
et la nuit, son ch\underline{a}nt est avec moi, \\
prière au Die\underline{u} de ma vie. \\

\versenb{10}Je dirai à Die\underline{u}, mon rocher: \\*
«Pourqu\underline{o}i m’oublies-tu ?~\psalmstar
Pourquoi v\underline{a}is-je assombri, \\
press\underline{é} par l’ennemi? » \\

\versenb{11}Outrag\underline{é} par mes adversaires, \\*
je suis meurtr\underline{i} jusqu’aux os,~\psalmstar
moi qui chaque jo\underline{u}r entends dire: \\
«Où est-\underline{i}l ton Dieu ? » \\

\versenb{12}Pourquoi te désol\underline{e}r, ô mon âme, \\*
et gém\underline{i}r sur moi?~\psalmstar
Espère en Dieu! De nouvea\underline{u} je rendrai grâce : \\
il est mon sauve\underline{u}r et mon Dieu! \\
\end{verse}

