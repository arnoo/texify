\psalm{83}{Heureux, les habitants de ta maison}
\psalmintro{Du maître de chœur. Sur la guittith. Des fils de Coré. Psaume.}

\begin{verse}
\versenb{2}De quel amour sont aim\underline{é}es tes demeures, \\*
Seigneur, Die\underline{u} de l’univers! \\

\versenb{3}Mon âme s’épu\underline{i}se à désirer \\*
les parv\underline{i}s du Seigneur;~\psalmstar
mon cœur et ma ch\underline{a}ir sont un cri \\
vers le Die\underline{u} vivant! \\

\versenb{4}L’oiseau lui-même s’est trouv\underline{é} une maison, \\*
et l’hirondelle, un nid pour abrit\underline{e}r sa couvée: \\
tes autels, Seigne\underline{u}r de l’univers, \\
mon R\underline{o}i et mon Dieu! \\

\versenb{5}Heureux les habit\underline{a}nts de ta maison: \\*
ils pourront te chant\underline{e}r encore! \\
\versenb{6}Heureux les hommes dont tu \underline{e}s la force: \\*
des chemins s’o\underline{u}vrent dans leur cœur! \\

\versenb{7}Quand ils traversent la vall\underline{é}e de la soif, \\*
ils la ch\underline{a}ngent en source;~\psalmstar
de quelles bénédicti\underline{o}ns la revêtent \\
les plu\underline{i}es de printemps! \\

\versenb{8}Ils vont de haute\underline{u}r en hauteur, \\*
ils se présentent devant Die\underline{u} à Sion. \\

\versenb{9}Seigneur, Dieu de l’univers, ent\underline{e}nds ma prière; \\*
écoute, Die\underline{u} de Jacob. \\
\versenb{10}Dieu, v\underline{o}is notre bouclier, \\*
regarde le vis\underline{a}ge de ton messie. \\

\versenb{11}Oui, un jo\underline{u}r dans tes parvis \\*
en vaut pl\underline{u}s que mille. \\

J’ai choisi de me ten\underline{i}r sur le seuil, \\
dans la mais\underline{o}n de mon Dieu,~\psalmstar
plut\underline{ô}t que d’habiter \\
parm\underline{i} les infidèles. \\

\versenb{12}Le Seigneur Die\underline{u} est un soleil, \\*
il \underline{e}st un bouclier;~\psalmstar
le Seigneur d\underline{o}nne la grâce, \\
il d\underline{o}nne la gloire. \\

Jamais il ne ref\underline{u}se le bonheur \\
à ceux qui v\underline{o}nt sans reproche. \\

\versenb{13}Seigneur, Die\underline{u} de l’univers, \\*
heureux qui esp\underline{è}re en toi! \\
\end{verse}

