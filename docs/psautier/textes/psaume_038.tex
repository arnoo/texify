\psalm{38}{L’homme n’est qu’un souffle}
\psalmintro{Du maître de chœur. De Yedoutoune. Psaume. De David.}

\begin{verse}
\versenb{2}J’ai dit: « Je garder\underline{a}i mon chemin \\*
sans laisser ma l\underline{a}ngue s’égarer; \\
je garderai un bâill\underline{o}n sur ma bouche, \\
tant que l’impie se tiendr\underline{a} devant moi.» \\

\versenb{3}Je suis resté muet, silencieux; \\*
je me tais\underline{a}is, mais sans profit.~\psalmstar
Mon tourment s’exaspérait, \\
\versenb{4}mon cœur brûl\underline{a}it en moi. \\*
Quand j’y pens\underline{a}is, je m’enflammais, \\
et j’ai laissé parl\underline{e}r ma langue. \\

\versenb{5}Seigneur, fais-moi connaître ma fin, \\*
quel est le n\underline{o}mbre de mes jours: \\
je connaîtrai combi\underline{e}n je suis fragile. \\
\versenb{6}Vois le peu de jo\underline{u}rs que tu m’accordes: \\*
ma durée n’est ri\underline{e}n devant toi. \\

L’homme ici-b\underline{a}s n’est qu’un souffle; \\
\versenb{7}il va, il vient, il n’\underline{e}st qu’une image. \\*
Rien qu’un souffle, to\underline{u}s ses tracas; \\
il amasse, mais qu\underline{i} recueillera? \\

\versenb{8}Maintenant, que puis-je att\underline{e}ndre, Seigneur? \\*
Elle est en t\underline{o}i, mon espérance. \\
\versenb{9}Délivre-moi de to\underline{u}s mes péchés, \\*
épargne-moi les inj\underline{u}res des fous. \\

\versenb{10}Je me suis tu, je n’ouvre p\underline{a}s la bouche, \\*
car c’est t\underline{o}i qui es à l’œuvre. \\
\versenb{11}Éloigne de m\underline{o}i tes coups: \\*
je succombe sous ta m\underline{a}in qui me frappe. \\

\versenb{12}Tu redresses l’homme en corrige\underline{a}nt sa faute,~\psalmdagger
tu ronges comme un v\underline{e}r son désir;~\psalmstar
l’h\underline{o}mme n’est qu’un souffle. \\

\versenb{13}Entends ma prière, Seigneur, éco\underline{u}te mon cri; \\*
ne reste pas so\underline{u}rd à mes pleurs. \\
Je ne suis qu’un h\underline{ô}te chez toi, \\
un passant, comme to\underline{u}s mes pères. \\

\versenb{14}Détourne de moi tes ye\underline{u}x, que je respire \\*
avant que je m’en \underline{a}ille et ne sois plus. \\
\end{verse}

