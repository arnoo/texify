\psalm{9B}{Tu entends le désir des pauvres}
\begin{verse}
\versenb{1}Pourquoi, Seigne\underline{u}r, es-tu si loin? \\*
Pourquoi te cach\underline{e}r aux jours d’angoisse? \\
\versenb{2}L’impie, dans son orgueil, poursu\underline{i}t les malheureux: \\*
ils se font prendre aux r\underline{u}ses qu’il invente. \\

\versenb{3}L’impie se glorifie du dés\underline{i}r de son âme, \\*
l’arrogant blasphème, il br\underline{a}ve le Seigneur; \\
\versenb{4}plein de suffisance, l’imp\underline{i}e ne cherche plus: \\*
«Dieu n’est rien », voil\underline{à} toute sa ruse. \\

\versenb{5}À tout moment, ce qu’il f\underline{a}it réussit;~\psalmdagger
tes sentences le dom\underline{i}nent de très haut.~\psalmstar
(Tous ses advers\underline{a}ires, il les méprise.) \\
\versenb{6}Il s’est dit: « Rien ne pe\underline{u}t m’ébranler, \\*
je suis pour longtemps à l’abr\underline{i} du malheur.» \\

\versenb{7}Sa bouche qui maudit n’est que fra\underline{u}de et violence, \\*
sa langue, mens\underline{o}nge et blessure. \\
\versenb{8}Il se tient à l’aff\underline{û}t près des villages, \\*
il se cache pour tu\underline{e}r l’innocent. \\

Des yeux, il ép\underline{i}e le faible, \\
\versenb{9}il se cache à l’affût, comme un li\underline{o}n dans son fourré; \\*
il se tient à l’affût pour surpr\underline{e}ndre le pauvre, \\
il attire le pauvre, il le pr\underline{e}nd dans son filet. \\

\versenb{10}Il se b\underline{a}isse, il se tapit; \\*
de tout son poids, il t\underline{o}mbe sur le faible. \\
\versenb{11}Il dit en lui-même: « Die\underline{u} oublie ! \\*
il couvre sa face, jam\underline{a}is il ne verra! » \\

\versenb{12}Lève-toi, Seigneur! Die\underline{u}, étends la main ! \\*
N’oublie p\underline{a}s le pauvre! \\
\versenb{13}Pourquoi l’impie brave-t-\underline{i}l le Seigneur \\*
en lui disant: « Viendras-t\underline{u} me chercher ? » \\

\versenb{14}Mais tu as vu: tu regardes le m\underline{a}l et la souffrance, \\*
tu les pr\underline{e}nds dans ta main; \\
sur toi rep\underline{o}se le faible, \\
c’est toi qui viens en \underline{a}ide à l’orphelin. \\

\versenb{15}Brise le bras de l’imp\underline{i}e, du méchant; \\*
alors tu chercheras son impiét\underline{é} sans la trouver. \\
\versenb{16}À tout jamais, le Seigne\underline{u}r est roi: \\*
les païens ont pér\underline{i} sur sa terre. \\

\versenb{17}Tu entends, Seigneur, le dés\underline{i}r des pauvres, \\*
tu rassures leur cœ\underline{u}r, tu les écoutes. \\
\versenb{18}Que justice soit rendue à l’orphelin, \\*
qu’il n’y ait pl\underline{u}s d’opprimé,~\psalmstar
et que tremble le mortel, n\underline{é} de la terre! \\
\end{verse}

