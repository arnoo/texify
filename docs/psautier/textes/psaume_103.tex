\psalm{103}{Quelle profusion dans tes œuvres!}
\begin{verse}
\versenb{1}Bénis le Seigne\underline{u}r, ô mon âme; \\*
Seigneur mon Die\underline{u}, tu es si grand! \\
Revêt\underline{u} de magnificence, \\
\versenb{2}tu as pour mantea\underline{u} la lumière! \\*

Comme une tenture, tu dépl\underline{o}ies les cieux, \\
\versenb{3}tu élèves dans leurs ea\underline{u}x tes demeures; \\*
des nuées, tu te f\underline{a}is un char, \\
tu t’avances sur les \underline{a}iles du vent; \\
\versenb{4}tu prends les v\underline{e}nts pour messagers, \\*
pour serviteurs, les fl\underline{a}mmes des éclairs. \\

\versenb{5}Tu as donné son ass\underline{i}se à la terre: \\*
qu’elle reste inébranl\underline{a}ble au cours des temps. \\
\versenb{6}Tu l’as vêtue de l’ab\underline{î}me des mers: \\*
les eaux couvraient m\underline{ê}me les montagnes; \\
\versenb{7}à ta menace, elles pr\underline{e}nnent la fuite, \\*
effrayées par le tonn\underline{e}rre de ta voix. \\

\versenb{8}Elles passent les montagnes, se r\underline{u}ent dans les vallées \\*
vers le lieu que tu leur \underline{a}s préparé. \\
\versenb{9}Tu leur imposes la lim\underline{i}te à ne pas franchir: \\*
qu’elles ne reviennent jam\underline{a}is couvrir la terre. \\

\versenb{10}Dans les ravins tu fais jaill\underline{i}r des sources \\*
et l’eau chemine aux cre\underline{u}x des montagnes; \\
\versenb{11}elle abreuve les b\underline{ê}tes des champs: \\*
l’âne sauvage y c\underline{a}lme sa soif; \\
\versenb{12}les oiseaux séjo\underline{u}rnent près d’elle: \\*
dans le feuillage on ent\underline{e}nd leurs cris. \\

\versenb{13}De tes demeures tu abre\underline{u}ves les montagnes, \\*
et la terre se rassasie du fru\underline{i}t de tes œuvres; \\
\versenb{14}tu fais pousser les prair\underline{i}es pour les troupeaux, \\*
et les champs pour l’h\underline{o}mme qui travaille. \\

De la terre il t\underline{i}re son pain: \\
\versenb{15}le vin qui réjou\underline{i}t le cœur de l’homme, \\*
l’huile qui adouc\underline{i}t son visage, \\
et le pain qui fortif\underline{i}e le cœur de l’homme. \\

\versenb{16}Les arbres du Seigne\underline{u}r se rassasient, \\*
les cèdres qu’il a plant\underline{é}s au Liban; \\
\versenb{17}c’est là que vient nich\underline{e}r le passereau, \\*
et la cigogne a sa mais\underline{o}n dans les cyprès; \\
\versenb{18}aux chamois, les ha\underline{u}tes montagnes, \\*
aux marmottes, l’abr\underline{i} des rochers. \\

\versenb{19}Tu fis la lune qui m\underline{a}rque les temps \\*
et le soleil qui connaît l’he\underline{u}re de son coucher. \\
\versenb{20}Tu fais descendre les tén\underline{è}bres, la nuit vient: \\*
les animaux dans la for\underline{ê}t s’éveillent; \\
\versenb{21}le lionceau rug\underline{i}t vers sa proie, \\*
il réclame à Die\underline{u} sa nourriture. \\

\versenb{22}Quand paraît le sol\underline{e}il, ils se retirent: \\*
chacun g\underline{a}gne son repaire. \\
\versenb{23}L’homme s\underline{o}rt pour son ouvrage, \\*
pour son trav\underline{a}il, jusqu’au soir. \\

\versenb{24}Quelle profusion dans tes œuvres, Seigneur!~\psalmdagger
Tout cela, ta sag\underline{e}sse l’a fait;~\psalmstar
la terre s’empl\underline{i}t de tes biens. \\

\versenb{25}Voici l’immensit\underline{é} de la mer, \\*
son grouillement innombrable d’animaux gr\underline{a}nds et petits, \\
\versenb{26}ses batea\underline{u}x qui voyagent, \\*
et Léviathan que tu fis pour qu’il s\underline{e}rve à tes jeux. \\

\versenb{27}Tous, ils c\underline{o}mptent sur toi \\*
pour recevoir leur nourrit\underline{u}re au temps voulu. \\
\versenb{28}Tu donnes: e\underline{u}x, ils ramassent ; \\*
tu ouvres la m\underline{a}in: ils sont comblés. \\

\versenb{29}Tu caches ton vis\underline{a}ge: ils s’épouvantent ; \\*
tu reprends leur souffle, ils expirent
et reto\underline{u}rnent à leur poussière. \\
\versenb{30}Tu envoies ton so\underline{u}ffle: ils sont créés ; \\*
tu renouvelles la f\underline{a}ce de la terre. \\

\versenb{31}Gloire au Seigne\underline{u}r à tout jamais! \\*
Que Dieu se réjou\underline{i}sse en ses œuvres! \\
\versenb{32}Il regarde la t\underline{e}rre: elle tremble ; \\*
il touche les mont\underline{a}gnes: elles brûlent. \\

\versenb{33}Je veux chanter au Seigne\underline{u}r tant que je vis; \\*
je veux jouer pour mon Die\underline{u} tant que je dure. \\
\versenb{34}Que mon poème lui s\underline{o}it agréable; \\*
moi, je me réjou\underline{i}s dans le Seigneur. \\
\versenb{35}Que les pécheurs dispar\underline{a}issent de la terre! \\*
Que les imp\underline{i}es n’existent plus! \\

Bénis le Seigne\underline{u}r, ô mon âme! \\
\end{verse}

