\psalm{40}{Je saurai que tu m’aimes}
\psalmintro{Du maître de chœur. Psaume. De David.}

\begin{verse}
\versenb{2}Heureux qui pense au pa\underline{u}vre et au faible: \\*
le Seigneur le sauve au jo\underline{u}r du malheur! \\
\versenb{3}Il le protège et le garde en vie, heure\underline{u}x sur la terre. \\*
Seigneur, ne le livre pas à la merc\underline{i} de l’ennemi! \\
\versenb{4}Le Seigneur le soutient sur son l\underline{i}t de souffrance: \\*
si malade qu’il s\underline{o}it, tu le relèves. \\

\versenb{5}J’avais dit: « Pitié pour m\underline{o}i, Seigneur, \\*
guéris-moi, car j’ai péch\underline{é} contre toi! » \\
\versenb{6}Mes ennemis me cond\underline{a}mnent déjà: \\*
«Quand sera-t-il mort ? son n\underline{o}m, effacé ? » \\
\versenb{7}Si quelqu’un vient me voir, ses prop\underline{o}s sont vides; \\*
il emplit son cœur de pensées méchantes, \\
il sort, et dans la r\underline{u}e il parle. \\

\versenb{8}Unis contre moi, mes ennem\underline{i}s murmurent, \\*
à mon sujet, ils prés\underline{a}gent le pire: \\
\versenb{9}«C’est un mal pernicie\underline{u}x qui le ronge ; \\*
le voilà couché, il ne pourra pl\underline{u}s se lever.» \\
\versenb{10}Même l’ami, qui av\underline{a}it ma confiance \\*
et partageait mon pain, m’a frapp\underline{é} du talon. \\

\versenb{11}Mais toi, Seigneur, prends piti\underline{é} de moi; \\*
relève-moi, je leur rendr\underline{a}i ce qu’ils méritent. \\
\versenb{12}Oui, je saur\underline{a}i que tu m’aimes \\*
si mes ennemis ne chantent p\underline{a}s victoire. \\
\versenb{13}Dans mon innocence tu m’\underline{a}s soutenu \\*
et rétabli pour toujo\underline{u}rs devant ta face. \\

\versenb{14}Béni soit le Seigneur, \\*
Die\underline{u} d’Israël,~\psalmstar
depuis toujours et pour toujours! \\
Am\underline{e}n! Amen ! \\
\end{verse}

