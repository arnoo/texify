\psalm{36}{Les doux posséderont la terre}
\psalmintro{De David.}

\begin{verse}
Ne t’indigne pas à la v\underline{u}e des méchants, \\
n’envie pas les g\underline{e}ns malhonnêtes; \\
\versenb{2}aussi vite que l’h\underline{e}rbe, ils se fanent; \\*
comme la verd\underline{u}re, ils se flétrissent. \\

\versenb{3}Fais confiance au Seigne\underline{u}r, agis bien, \\*
habite la terre et r\underline{e}ste fidèle; \\
\versenb{4}mets ta j\underline{o}ie dans le Seigneur: \\*
il comblera les dés\underline{i}rs de ton cœur. \\

\versenb{5}Dirige ton chem\underline{i}n vers le Seigneur, \\*
fais-lui confiance, et lu\underline{i}, il agira. \\
\versenb{6}Il fera lever comme le jo\underline{u}r ta justice, \\*
et ton droit comme le pl\underline{e}in midi. \\

\versenb{7}Repose-t\underline{o}i sur le Seigneur \\*
et c\underline{o}mpte sur lui. \\
Ne t’indigne pas devant celu\underline{i} qui réussit, \\
devant l’homme qui \underline{u}se d’intrigues. \\

\versenb{8}Laisse ta colère, c\underline{a}lme ta fièvre, \\*
ne t’indigne pas: il n’en viendr\underline{a}it que du mal ; \\
\versenb{9}les méchants ser\underline{o}nt déracinés, \\*
mais qui espère le Seigneur posséder\underline{a} la terre. \\

\versenb{10}Encore un peu de t\underline{e}mps: plus d’impie ; \\*
tu pénètres chez lu\underline{i}: il n’y est plus. \\
\versenb{11}Les doux posséder\underline{o}nt la terre \\*
et jouiront d’une abond\underline{a}nte paix. \\

\versenb{12}L’impie peut intrigu\underline{e}r contre le juste \\*
et grincer des d\underline{e}nts contre lui, \\
\versenb{13}le Seigneur se m\underline{o}que du méchant \\*
car il voit son jo\underline{u}r qui arrive. \\

\versenb{14}L’impie a tiré son épée, il a tend\underline{u} son arc \\*
pour abattre le pauvre et le faible, \\
pour tu\underline{e}r l’homme droit. \\
\versenb{15}Mais l’épée lui entrer\underline{a} dans le cœur, \\*
et son \underline{a}rc se brisera. \\

\versenb{16}Pour le juste, avoir pe\underline{u} de biens \\*
vaut mieux que la fort\underline{u}ne des impies. \\
\versenb{17}Car le bras de l’imp\underline{i}e sera brisé, \\*
mais le Seigneur souti\underline{e}nt les justes. \\

\versenb{18}Il connaît les jo\underline{u}rs de l’homme intègre \\*
qui recevra un hérit\underline{a}ge impérissable. \\
\versenb{19}Pas de honte pour lu\underline{i} aux mauvais jours; \\*
aux temps de famine, il ser\underline{a} rassasié. \\

\versenb{20}Mais oui, les imp\underline{i}es disparaîtront \\*
comme la par\underline{u}re des prés; \\
c’en est fini des ennem\underline{i}s du Seigneur: \\
ils s’en v\underline{o}nt en fumée. \\

\versenb{21}L’impie empr\underline{u}nte et ne rend pas; \\*
le juste a piti\underline{é}: il donne. \\
\versenb{22}Ceux qu’il bénit posséder\underline{o}nt la terre, \\*
ceux qu’il maudit ser\underline{o}nt déracinés. \\

\versenb{23}Quand le Seigneur condu\underline{i}t les pas de l’homme, \\*
ils sont fermes et sa m\underline{a}rche lui plaît. \\
\versenb{24}S’il trébuche, il ne t\underline{o}mbe pas \\*
car le Seigneur le souti\underline{e}nt de sa main. \\

\versenb{25}Jamais, de ma jeun\underline{e}sse à mes vieux jours, \\*
je n’ai vu le juste abandonné
ni ses enfants mendi\underline{e}r leur pain. \\
\versenb{26}Chaque jour il a piti\underline{é}, il prête; \\*
ses descend\underline{a}nts seront bénis. \\

\versenb{27}Évite le mal, f\underline{a}is ce qui est bien, \\*
et tu auras une habitati\underline{o}n pour toujours, \\
\versenb{28}car le Seigneur \underline{a}ime le bon droit, \\*
il n’abandonne p\underline{a}s ses amis. \\

Ceux-là seront préserv\underline{é}s à jamais, \\
les descendants de l’impie ser\underline{o}nt déracinés. \\
\versenb{29}Les justes posséder\underline{o}nt la terre \\*
et toujo\underline{u}rs l’habiteront. \\

\versenb{30}Les lèvres du juste red\underline{i}sent la sagesse \\*
et sa bouche én\underline{o}nce le droit. \\
\versenb{31}La loi de son Die\underline{u} est dans son cœur; \\*
il va, sans cr\underline{a}indre les faux pas. \\

\versenb{32}Les impies gu\underline{e}ttent le juste, \\*
ils cherchent à le f\underline{a}ire mourir. \\
\versenb{33}Mais le Seigneur ne saur\underline{a}it l’abandonner \\*
ni le laisser condamn\underline{e}r par ses juges. \\

\versenb{34}Esp\underline{è}re le Seigneur, \\*
et g\underline{a}rde son chemin: \\
il t’élèvera jusqu’à posséd\underline{e}r la terre; \\
tu verras les imp\underline{i}es déracinés. \\

\versenb{35}J’ai vu l’imp\underline{i}e dans sa puissance \\*
se déployer comme un c\underline{è}dre vigoureux. \\
\versenb{36}Il a passé, voic\underline{i} qu’il n’est plus; \\*
je l’ai cherché, il \underline{e}st introuvable. \\

\versenb{37}Considère l’homme droit, v\underline{o}is l’homme intègre: \\*
un avenir est prom\underline{i}s aux pacifiques. \\
\versenb{38}Les pécheurs seront to\underline{u}s déracinés, \\*
et l’avenir des imp\underline{i}es, anéanti. \\

\versenb{39}Le Seigneur est le sal\underline{u}t pour les justes, \\*
leur abri au t\underline{e}mps de la détresse. \\
\versenb{40}Le Seigneur les \underline{a}ide et les délivre, \\*
il les délivre de l’impie, il les sauve, \\
car ils cherchent en lu\underline{i} leur refuge. \\
\end{verse}

