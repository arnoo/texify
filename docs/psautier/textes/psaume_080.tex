\psalm{80}{«Si mon peuple m’écoutait… »}
\psalmintro{Du maître de chœur. Sur la guittith. D’Asaph.}

\begin{verse}
\versenb{2}Criez de joie pour Die\underline{u}, notre force, \\*
acclamez le Die\underline{u} de Jacob. \\

\versenb{3}Jouez, musiques, frapp\underline{e}z le tambourin, \\*
la harpe et la cith\underline{a}re mélodieuse. \\
\versenb{4}Sonnez du cor pour le m\underline{o}is nouveau, \\*
quand revient le jo\underline{u}r de notre fête. \\

\versenb{5}C’est là, pour Isra\underline{ë}l, une règle, \\*
une ordonnance du Die\underline{u} de Jacob; \\
\versenb{6}Il en fit, pour Jos\underline{e}ph, une loi \\*
quand il marcha contre la t\underline{e}rre d’Égypte. \\

J’entends des mots qui m’ét\underline{a}ient inconnus:~\psalmdagger
\versenb{7}«J’ai ôté le poids qui charge\underline{a}it ses épaules ; \\*
ses mains ont dépos\underline{é} le fardeau. \\

\versenb{8}«Quand tu criais sous l’\underline{o}ppression, je t’ai sauvé ;~\psalmdagger
je répondais, cach\underline{é} dans l’orage, \\
je t’éprouvais près des ea\underline{u}x de Mériba. \\

\versenb{9}«Écoute, je t’adj\underline{u}re, ô mon peuple ; \\*
vas-tu m’écout\underline{e}r, Israël? \\
\versenb{10}Tu n’auras pas chez t\underline{o}i d’autres dieux, \\*
tu ne serviras aucun die\underline{u} étranger. \\

\versenb{11}«C’est moi, le Seigne\underline{u}r ton Dieu,~\psalmdagger
qui t’ai fait monter de la t\underline{e}rre d’Égypte! \\
Ouvre ta bouche, m\underline{o}i, je l’emplirai. \\

\versenb{12}«Mais mon peuple n’a pas écout\underline{é} ma voix, \\*
Israël n’a pas voul\underline{u} de moi. \\
\versenb{13}Je l’ai livré à son cœ\underline{u}r endurci: \\*
qu’il aille et su\underline{i}ve ses vues! \\

\versenb{14}«Ah ! Si mon pe\underline{u}ple m’écoutait, \\*
Israël, s’il \underline{a}llait sur mes chemins! \\
\versenb{15}Aussitôt j’humilier\underline{a}is ses ennemis, \\*
contre ses oppresseurs je tourner\underline{a}is ma main. \\

\versenb{16}«Mes adversaires s’abaisser\underline{a}ient devant lui ; \\*
tel serait leur s\underline{o}rt à jamais! \\
\versenb{17}Je le nourrirais de la fle\underline{u}r du froment, \\*
je le rassasierais avec le mi\underline{e}l du rocher! » \\
\end{verse}

