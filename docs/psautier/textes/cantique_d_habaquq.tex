\psalm[Cantique]{10}{Cantique d'Habaquq (Ha 3)}
\begin{verse}
\versenb{2}Seigneur, j'ai entend\underline{u} parler de toi ;\psalmstar
devant ton œuvre, Seigne\underline{u}r, j'ai craint !\\
Dans le cours des ann\underline{é}es, fais-la revivre,\\
dans le cours des ann\underline{é}es, fais-la connaître !\\

Quand tu frém\underline{i}s de colère,\\
souviens-t\underline{o}i d'avoir pitié.\\

\versenb{3}Dieu vi\underline{e}nt de Téman,\psalmstar
et le saint, du M\underline{o}nt de Paran ;\\
sa majesté co\underline{u}vre les cieux,\\
sa gloire empl\underline{i}t la terre.\\

\versenb{4}Son éclat est par\underline{e}il à la lumière ; \psalmdagger
deux rayons s\underline{o}rtent de ses mains :\\
là se tient cach\underline{é}e sa puissance.\\

\versenb{13}Tu es sorti pour sauv\underline{e}r ton peuple\psalmstar
pour sauv\underline{e}r ton messie.\\

\versenb{15}Tu as foulé, de tes cheva\underline{u}x, la mer\psalmstar
et le remo\underline{u}s des eaux profondes.\\

\versenb{16}J'ai entendu et mes entr\underline{a}illes ont frémi ; \psalmdagger
à cette voix, mes l\underline{è}vres tremblent,\\
la carie pén\underline{è}tre mes os.\\

Et moi je frém\underline{i}s d'être là, \psalmdagger
d'attendre en silence le jo\underline{u}r d'angoisse\\
qui se lèvera sur le peuple dress\underline{é} contre nous.\\

\versenb{17}Le figui\underline{e}r n'a pas fleuri ;\psalmstar
pas de réc\underline{o}lte dans les vignes.\\
Le fruit de l'olivi\underline{e}r a déçu ;\\
dans les champs, pl\underline{u}s de nourriture.\\
L'enclos s'est vid\underline{é} de ses brebis,\\
et l'ét\underline{a}ble, de son bétail.\\

\versenb{18}Et moi, je bondis de j\underline{o}ie dans le Seigneur,\psalmstar
j'exulte en Die\underline{u}, mon Sauveur !\\
Le Seigneur mon Die\underline{u} est ma force ; \psalmdagger
il me donne l'agilit\underline{é} du chamois,\\
il me fait march\underline{e}r dans les hauteurs.
\end{verse}

