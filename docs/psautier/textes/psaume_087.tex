\psalm{87}{Dans cette nuit où je crie}
\psalmintro{Cantique. Psaume. Des fils de Coré. Du maître de chœur. Sur «mahalath ». Pour l’affliction. Poème. De Hémane l’Ézrahite.}

\begin{verse}
\versenb{2}Seigneur, mon Die\underline{u} et mon salut, \\*
dans cette nuit où je cr\underline{i}e en ta présence, \\
\versenb{3}que ma prière parvi\underline{e}nne jusqu’à toi, \\*
ouvre l’or\underline{e}ille à ma plainte. \\

\versenb{4}Car mon âme est rassasi\underline{é}e de malheur, \\*
ma vie est au b\underline{o}rd de l’abîme; \\
\versenb{5}on me voit déjà desc\underline{e}ndre à la fosse,\psalmstar
je suis comme un h\underline{o}mme fini. \\

\versenb{6}Ma place est parm\underline{i} les morts, \\*
avec ceux que l’on a tu\underline{é}s, enterrés, \\
ceux dont tu n’as pl\underline{u}s souvenir, \\
qui sont exclus, et l\underline{o}in de ta main. \\

\versenb{7}Tu m’as mis au plus prof\underline{o}nd de la fosse, \\*
en des lieux englout\underline{i}s, ténébreux; \\
\versenb{8}le poids de ta col\underline{è}re m’écrase, \\*
tu déverses tes fl\underline{o}ts contre moi. \\

\versenb{9}Tu éloignes de m\underline{o}i mes amis, \\*
tu m’as rendu abomin\underline{a}ble pour eux; \\
enfermé, je n’ai p\underline{a}s d’issue: \\
\versenb{10}à force de souffrir, mes ye\underline{u}x s’éteignent. \\*

Je t’appelle, Seigne\underline{u}r, tout le jour, \\
je tends les m\underline{a}ins vers toi: \\
\versenb{11}fais-tu des mir\underline{a}cles pour les morts? \\*
leur ombre se dresse-t-\underline{e}lle pour t’acclamer? \\

\versenb{12}Qui parlera de ton amo\underline{u}r dans la tombe, \\*
de ta fidélité au roya\underline{u}me de la mort? \\
\versenb{13}Connaît-on dans les tén\underline{è}bres tes miracles, \\*
et ta justice, au pa\underline{y}s de l’oubli? \\

\versenb{14}Moi, je crie vers t\underline{o}i, Seigneur; \\*
dès le matin, ma pri\underline{è}re te cherche: \\
\versenb{15}pourquoi me rejet\underline{e}r, Seigneur, \\*
pourquoi me cach\underline{e}r ta face? \\

\versenb{16}Malheureux, frappé à m\underline{o}rt depuis l’enfance, \\*
je n’en peux plus d’endur\underline{e}r tes fléaux; \\
\versenb{17}sur moi, ont déferl\underline{é} tes orages: \\*
tes effrois m’ont rédu\underline{i}t au silence. \\

\versenb{18}Ils me cernent comme l’ea\underline{u} tout le jour, \\*
ensemble ils se ref\underline{e}rment sur moi. \\
\versenb{19}Tu éloignes de moi am\underline{i}s et familiers; \\*
ma compagne, c’\underline{e}st la ténèbre. \\
\end{verse}

