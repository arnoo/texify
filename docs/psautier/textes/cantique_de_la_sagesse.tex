\psalm[Cantique]{14}{Cantique de la Sagesse (Sg 9)}
\begin{verse}
\versenb{1}Dieu de mes pères et Seigne\underline{u}r de tendresse,\psalmstar
par ta parole tu f\underline{i}s l'univers,\\
\versenb{2}tu formas l'h\underline{o}mme par ta Sagesse\psalmstar
pour qu'il dom\underline{i}ne sur tes créatures,\\
\versenb{3}qu'il gouverne le monde avec just\underline{i}ce et sainteté,\psalmstar
qu'il rende, avec droit\underline{u}re, ses jugements.\n\n\versenb{4}Donne-m\underline{o}i la Sagesse,\psalmstar
ass\underline{i}se près de toi.\\

Ne me retranche pas du n\underline{o}mbre de tes fils :\\
\versenb{5}je suis ton serviteur, le f\underline{i}ls de ta servante,\psalmstar
un homme frêle et qui d\underline{u}re peu,\\
trop faible pour comprendre les préc\underline{e}ptes et les lois.\\
\versenb{6}Le plus accompli des enf\underline{a}nts des hommes, \psalmstar
s'il lui manque la Sagesse que tu donnes,\\
sera compt\underline{é} pour rien.\\

\versenb{9}Or la Sag\underline{e}sse est avec toi,\psalmstar
elle qui s\underline{a}it tes œuvres ;\\
elle était là quand tu f\underline{i}s l'univers, \psalmstar
elle connaît ce qui plaît à tes yeux,\\
ce qui est conf\underline{o}rme à tes décrets.\\
\versenb{10}Des cieux très saints, d\underline{a}igne l'envoyer,\psalmstar
fais-la descendre du tr\underline{ô}ne de ta gloire.\n\nQu'elle trav\underline{a}ille à mes côtés\\
et m'appr\underline{e}nne ce qui te plaît.\\
\versenb{11}Car elle sait to\underline{u}t, comprend tout, \psalmstar
guidera mes actes avec prudence,\\*
me garder\underline{a} par sa gloire.
\end{verse}
