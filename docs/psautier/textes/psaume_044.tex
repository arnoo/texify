\psalm{44}{Ton Dieu t’a consacré}
\psalmintro{Du maître de chœur. Sur l’air de «Les lis ». Des fils de Coré. Poème.~Chant~d’amour.}

\begin{verse}
\versenb{2}D’heureuses paroles jaill\underline{i}ssent de mon cœur \\*
quand je dis mes po\underline{è}mes pour le roi \\
d’une langue aussi vive que la pl\underline{u}me du scribe! \\

\versenb{3}Tu es beau, comme aucun des enf\underline{a}nts de l’homme, \\*
la grâce est répand\underline{u}e sur tes lèvres: \\
oui, Dieu te bén\underline{i}t pour toujours. \\

\versenb{4}Guerrier valeureux, porte l’épée de nobl\underline{e}sse et d’honneur! \\*
\versenb{5}Ton honneur, c’est de cour\underline{i}r au combat \\*
pour la justice, la clém\underline{e}nce et la vérité. \\

\versenb{6}Ta main jettera la stupeur, les fl\underline{è}ches qui déchirent; \\*
sous tes coups, les pe\underline{u}ples s’abattront, \\
les ennemis du roi, frapp\underline{é}s en plein cœur. \\

\versenb{7}Ton trône est divin, un tr\underline{ô}ne éternel; \\*
ton sceptre royal est sc\underline{e}ptre de droiture: \\
\versenb{8}tu aimes la justice, tu répro\underline{u}ves le mal. \\*

Oui, Dieu, ton Die\underline{u} t’a consacré \\
d’une onction de joie, comme auc\underline{u}n de tes semblables; \\
\versenb{9}la myrrhe et l’aloès parf\underline{u}ment ton vêtement. \\*

Des palais d’ivoire, la mus\underline{i}que t’enchante. \\
\versenb{10}Parmi tes bien-aimées sont des f\underline{i}lles de roi; \\*
à ta droite, la préférée, sous les \underline{o}rs d’Ophir. \\

\versenb{11}Écoute, ma fille, reg\underline{a}rde et tends l’oreille; \\*
oublie ton peuple et la mais\underline{o}n de ton père: \\
\versenb{12}le roi sera sédu\underline{i}t par ta beauté. \\*

Il est ton Seigneur: prosterne-t\underline{o}i devant lui. \\
\versenb{13}Alors, fille de Tyr, les plus r\underline{i}ches du peuple, \\*
chargés de présents, quêter\underline{o}nt ton sourire. \\

\versenb{14}Fille de roi, elle est l\underline{à}, dans sa gloire, \\*
vêtue d’ét\underline{o}ffes d’or; \\
\versenb{15}on la conduit, toute par\underline{é}e, vers le roi. \\*

Des jeunes filles, ses compagnes, lui f\underline{o}nt cortège; \\
\versenb{16}on les conduit parmi les ch\underline{a}nts de fête: \\*
elles entrent au pal\underline{a}is du roi. \\

\versenb{17}À la place de tes pères se lèver\underline{o}nt tes fils; \\*
sur toute la terre tu feras d’e\underline{u}x des princes. \\

\versenb{18}Je ferai vivre ton nom pour les \underline{â}ges des âges: \\*
que les peuples te rendent grâce, toujo\underline{u}rs, à jamais! \\
\end{verse}

