\psalm{90}{Sous l’abri du Très-Haut}
\begin{verse}
\versenb{1}Quand je me tiens sous l’abr\underline{i} du Très-Haut \\*
et repose à l’\underline{o}mbre du Puissant, \\
\versenb{2}je dis au Seigne\underline{u}r: « Mon refuge, \\*
mon rempart, mon Die\underline{u}, dont je suis sûr! » \\

\versenb{3}C’est lui qui te sauve des filets du chasseur
et de la p\underline{e}ste maléfique;~\psalmstar
\versenb{4}il te co\underline{u}vre et te protège. \\*
Tu trouves sous son \underline{a}ile un refuge: \\
sa fidélité est une arm\underline{u}re, un bouclier. \\

\versenb{5}Tu ne craindras ni les terre\underline{u}rs de la nuit, \\*
ni la flèche qui v\underline{o}le au grand jour, \\
\versenb{6}ni la peste qui r\underline{ô}de dans le noir, \\*
ni le fléau qui fr\underline{a}ppe à midi. \\

\versenb{7}Qu’il en tombe m\underline{i}lle à tes côtés,~\psalmdagger
qu’il en tombe dix m\underline{i}lle à ta droite,~\psalmstar
toi, tu r\underline{e}stes hors d’atteinte. \\

\versenb{8}Il suffit que tu o\underline{u}vres les yeux, \\*
tu verras le sal\underline{a}ire du méchant. \\
\versenb{9}Oui, le Seigne\underline{u}r est ton refuge; \\*
tu as fait du Très-Ha\underline{u}t ta forteresse. \\

\versenb{10}Le malheur ne pourr\underline{a} te toucher, \\*
ni le danger, approch\underline{e}r de ta demeure: \\
\versenb{11}il donne missi\underline{o}n à ses anges \\*
de te garder sur to\underline{u}s tes chemins. \\

\versenb{12}Ils te porter\underline{o}nt sur leurs mains \\*
pour que ton pied ne he\underline{u}rte les pierres; \\
\versenb{13}tu marcheras sur la vip\underline{è}re et le scorpion, \\*
tu écraseras le li\underline{o}n et le Dragon. \\

\versenb{14}«Puisqu’il s’attache à m\underline{o}i, je le délivre ; \\*
je le défends, car il conn\underline{a}ît mon nom. \\
\versenb{15}Il m’appelle, et m\underline{o}i, je lui réponds; \\*
je suis avec lu\underline{i} dans son épreuve. \\

«Je veux le libér\underline{e}r, le glorifier ;~\psalmdagger
\versenb{16}de longs jours, je ve\underline{u}x le rassasier,~\psalmstar
et je ferai qu’il v\underline{o}ie mon salut.» \\
\end{verse}

