\psalm{143}{Heureux qui a pour Dieu le Seigneur}
\psalmintro{De David.}

\begin{verse}
Béni soit le Seigne\underline{u}r, mon rocher!~\psalmdagger
Il exerce mes m\underline{a}ins pour le combat,~\psalmstar
il m’entr\underline{a}îne à la bataille. \\

\versenb{2}Il est mon alli\underline{é}, ma forteresse, \\*
ma citadelle, celu\underline{i} qui me libère; \\
il est le boucli\underline{e}r qui m’abrite, \\
il me donne pouv\underline{o}ir sur mon peuple. \\

\versenb{3}Qu’est-ce que l’homme, \\*
pour que tu le conn\underline{a}isses, Seigneur,~\psalmstar
le fils d’un homme, pour que tu c\underline{o}mptes avec lui? \\
\versenb{4}L’homme est sembl\underline{a}ble à un souffle, \\*
ses jours sont une \underline{o}mbre qui passe. \\

\versenb{5}Seigneur, incline les cie\underline{u}x et descends; \\*
touche les mont\underline{a}gnes: qu’elles brûlent ! \\
\versenb{6}Décoche des écl\underline{a}irs de tous côtés, \\*
tire des flèches et rép\underline{a}nds la terreur. \\

\versenb{7}Des hauteurs, tends-moi la m\underline{a}in, délivre-moi,~\psalmstar
sauve-moi du gouffre des eaux, \\
de l’emprise d’un pe\underline{u}ple étranger: \\
\versenb{8}il dit des par\underline{o}les mensongères, \\*
sa main est une m\underline{a}in parjure. \\

\versenb{9}Pour toi, je chanter\underline{a}i un chant nouveau, \\*
pour toi, je jouerai sur la h\underline{a}rpe à dix cordes, \\
\versenb{10}pour toi qui donnes aux r\underline{o}is la victoire \\*
et sauves de l’épée meurtrière Dav\underline{i}d, ton serviteur. \\

\versenb{11}Délivre-m\underline{o}i, sauve-moi \\*
de l’emprise d’un pe\underline{u}ple étranger: \\
il dit des par\underline{o}les mensongères, \\
sa main est une m\underline{a}in parjure. \\

\versenb{12}Que nos fils soient par\underline{e}ils à des plants \\*
bien ven\underline{u}s dès leur jeune âge;~\psalmstar
nos filles, par\underline{e}illes à des colonnes \\
sculpt\underline{é}es pour un palais! \\

\versenb{13}Nos greniers, rempl\underline{i}s, débordants, \\*
regorger\underline{o}nt de biens;~\psalmstar
les troupeaux, par milli\underline{e}rs, par myriades, \\
emplir\underline{o}nt nos campagnes! \\

\versenb{14}Nos vassaux nous rester\underline{o}nt soumis, \\*
pl\underline{u}s de défaites;~\psalmstar
plus de br\underline{è}ches dans nos murs, \\
plus d’al\underline{e}rtes sur nos places! \\

\versenb{15}Heureux le pe\underline{u}ple ainsi comblé! \\*
Heureux le peuple qui a pour Die\underline{u} «Le Seigneur » ! \\
\end{verse}

