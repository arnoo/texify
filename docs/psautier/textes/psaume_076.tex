\psalm{76}{Dieu oublierait-il d’avoir pitié?}
\psalmintro{Du maître de chœur. D’après Yedoutoune. D’Asaph. Psaume.}

\begin{verse}
\versenb{2}Vers Dieu, je cr\underline{i}e mon appel! \\*
Je crie vers Die\underline{u}: qu’il m’entende ! \\

\versenb{3}Au jour de la détresse, je ch\underline{e}rche le Seigneur;~\psalmdagger
la nuit, je tends les m\underline{a}ins sans relâche, \\
mon âme ref\underline{u}se le réconfort. \\

\versenb{4}Je me souviens de Die\underline{u}, je me plains; \\*
je médite et mon espr\underline{i}t défaille. \\
\versenb{5}Tu refuses à mes ye\underline{u}x le sommeil; \\*
je me trouble, incap\underline{a}ble de parler. \\

\versenb{6}Je pense aux jo\underline{u}rs d’autrefois, \\*
aux ann\underline{é}es de jadis; \\
\versenb{7}la nuit, je me souvi\underline{e}ns de mon chant, \\*
je médite en mon cœur, et mon espr\underline{i}t s’interroge. \\

\versenb{8}Le Seigneur ne fera-t-\underline{i}l que rejeter, \\*
ne sera-t-il jamais pl\underline{u}s favorable? \\
\versenb{9}Son amour a-t-il d\underline{o}nc disparu? \\*
S’est-elle éteinte, d’âge en \underline{â}ge, la parole? \\

\versenb{10}Dieu oublierait-\underline{i}l d’avoir pitié, \\*
dans sa colère a-t-il ferm\underline{é} ses entrailles? \\
\versenb{11}J’ai dit: « Une ch\underline{o}se me fait mal, \\*
la droite du Très-Ha\underline{u}t a changé.» \\

\versenb{12}Je me souviens des expl\underline{o}its du Seigneur, \\*
je rappelle ta merv\underline{e}ille de jadis; \\
\versenb{13}je me red\underline{i}s tous tes hauts faits, \\*
sur tes expl\underline{o}its je médite. \\

\versenb{14}Dieu, la saintet\underline{é} est ton chemin! \\*
Quel dieu est gr\underline{a}nd comme Dieu? \\

\versenb{15}Tu es le Dieu qui accompl\underline{i}s la merveille, \\*
qui fais connaître chez les pe\underline{u}ples ta force: \\
\versenb{16}tu rachetas ton pe\underline{u}ple avec puissance, \\*
les descendants de Jac\underline{o}b et de Joseph. \\

\versenb{17}Les eaux, en te voy\underline{a}nt, Seigneur,~\psalmdagger
les eaux, en te voy\underline{a}nt, tremblèrent, \\
l’abîme lui-m\underline{ê}me a frémi. \\

\versenb{18}Les nuages dévers\underline{è}rent leurs eaux,~\psalmdagger
les nuées donn\underline{è}rent de la voix, \\
la foudre frapp\underline{a}it de toute part. \\

\versenb{19}Au roulement de ta v\underline{o}ix qui tonnait,~\psalmdagger
tes éclairs illumin\underline{è}rent le monde, \\
la terre s’agit\underline{a} et frémit. \\

\versenb{20}Par la mer pass\underline{a}it ton chemin,~\psalmdagger
tes sentiers, par les ea\underline{u}x profondes; \\
et nul n’en conn\underline{a}ît la trace. \\

\versenb{21}Tu as conduit comme un troupea\underline{u} ton peuple \\*
par la main de Mo\underline{ï}se et d’Aaron. \\
\end{verse}

