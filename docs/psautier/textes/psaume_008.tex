\psalm{8}{Qu’il est grand ton nom!}
\psalmintro{Du maître de chœur. Sur la guittith. Psaume. De David.}

\begin{verse}
\versenb{2}Ô Seigne\underline{u}r, notre Dieu, \\*
qu’il est gr\underline{a}nd, ton nom, \\
par to\underline{u}te la terre! \\

Jusqu’aux cieux, ta splende\underline{u}r est chantée \\
\versenb{3}par la bouche des enf\underline{a}nts, des tout-petits: \\*
rempart que tu opp\underline{o}ses à l’adversaire, \\
où l’ennemi se br\underline{i}se en sa révolte. \\

\versenb{4}À voir ton ciel, ouvr\underline{a}ge de tes doigts, \\*
la lune et les ét\underline{o}iles que tu fixas, \\
\versenb{5}qu’est-ce que l’homme pour que tu p\underline{e}nses à lui, \\*
le fils d’un homme, que tu en pr\underline{e}nnes souci? \\

\versenb{6}Tu l’as voulu un peu m\underline{o}indre qu’un dieu, \\*
le couronnant de gl\underline{o}ire et d’honneur; \\
\versenb{7}tu l’établis sur les œ\underline{u}vres de tes mains, \\*
tu mets toute ch\underline{o}se à ses pieds: \\

\versenb{8}les troupeaux de bœ\underline{u}fs et de brebis, \\*
et même les b\underline{ê}tes sauvages, \\
\versenb{9}les oiseaux du ciel et les poiss\underline{o}ns de la mer, \\*
tout ce qui va son chem\underline{i}n dans les eaux. \\

\versenb{10}Ô Seigne\underline{u}r, notre Dieu, \\*
qu’il est gr\underline{a}nd ton nom \\
par to\underline{u}te la terre! \\
\end{verse}

