\psalm{51}{Dieu est fidèle}
\psalmintro{Du maître de chœur. Poème. De David.}

\begin{verse}
\versenb{2}Lorsque Doëg l’Édomite vint dire à Saül: « David est entré dans la maison d’Abimélek. »

\versenb{3}Pourquoi te glorifi\underline{e}r du mal, \\*
t\underline{o}i, l’homme fort?~\psalmstar
Chaque jour, Die\underline{u} est fidèle. \\

\versenb{4}De ta langue affil\underline{é}e comme un rasoir, \\*
tu prép\underline{a}res le crime,~\psalmstar
fo\underline{u}rbe que tu es! \\

\versenb{5}Tu aimes le m\underline{a}l plus que le bien, \\*
et plus que la vérit\underline{é}, le mensonge;~\psalmstar
\versenb{6}tu aimes les par\underline{o}les qui tuent, \\*
l\underline{a}ngue perverse. \\

\versenb{7}Mais Dieu va te ruin\underline{e}r pour toujours, \\*
t’écraser, t’arrach\underline{e}r de ta demeure,~\psalmstar
t’extirper de la t\underline{e}rre des vivants. \\

\versenb{8}Les justes verr\underline{o}nt, ils craindront, \\*
ils rir\underline{o}nt de toi:~\psalmdagger
\versenb{9}«Le voilà d\underline{o}nc cet homme \\*
qui n’a pas mis sa f\underline{o}rce en Dieu!~\psalmstar
Il comptait sur ses gr\underline{a}ndes richesses, \\
il se faisait f\underline{o}rt de son crime! » \\

\versenb{10}Pour moi, comme un b\underline{e}l olivier \\*
dans la mais\underline{o}n de Dieu,~\psalmstar
je compte sur la fidélit\underline{é} de mon Dieu, \\
sans f\underline{i}n, à jamais! \\

\versenb{11}Sans fin, je veux te r\underline{e}ndre grâce, \\*
c\underline{a}r tu as agi.~\psalmstar
J’espère en ton nom dev\underline{a}nt ceux qui t’aiment: \\
ou\underline{i}, il est bon! \\
\end{verse}

