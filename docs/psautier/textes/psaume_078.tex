\psalm{78}{Épargne ceux qui doivent mourir}
\psalmintro{Psaume. D’Asaph.}

\begin{verse}
Dieu, les païens ont envah\underline{i} ton domaine;~\psalmdagger
ils ont souillé ton t\underline{e}mple sacré \\
et mis Jérusal\underline{e}m en ruines. \\

\versenb{2}Ils ont livré les cadavres de tes serviteurs
en pâture aux rap\underline{a}ces du ciel~\psalmstar
et la chair de tes fidèles, aux b\underline{ê}tes de la terre; \\
\versenb{3}ils ont versé le sang comme l’eau
aux alento\underline{u}rs de Jérusalem:~\psalmstar
les morts rest\underline{a}ient sans sépulture. \\

\versenb{4}Nous sommes la ris\underline{é}e des voisins, \\*
la fable et le jou\underline{e}t de l’entourage. \\
\versenb{5}Combien de temps, Seigneur, durer\underline{a} ta colère \\*
et brûlera le fe\underline{u} de ta jalousie? \\

\versenb{6}$\[$Déverse ta fureur
sur les païens qui ne t’ont p\underline{a}s reconnu,~\psalmstar
sur les royaumes qui n’invoquent p\underline{a}s ton nom, \\
\versenb{7}car ils ont dévor\underline{é} Jacob \\*
et ravag\underline{é} son territoire.$\]$ \\

\versenb{8}Ne retiens pas contre nous les péch\underline{é}s de nos ancêtres:~\psalmdagger
que nous vienne bient\underline{ô}t ta tendresse, \\
car nous sommes à bo\underline{u}t de force! \\

\versenb{9}Aide-nous, Dieu notre Sauveur, \\*
pour la gl\underline{o}ire de ton nom!~\psalmstar
Délivre-nous, efface nos fautes, \\
pour la ca\underline{u}se de ton nom! \\

\versenb{10}Pourquoi laisser d\underline{i}re aux païens: \\*
«Où d\underline{o}nc est leur Dieu ? » \\
Que les païens, sous nos ye\underline{u}x, le reconnaissent: \\
il sera vengé, le sang vers\underline{é} de tes serviteurs. \\

\versenb{11}Que monte en ta présence la pl\underline{a}inte du captif! \\*
Ton bras est fort: épargne ceux qui d\underline{o}ivent mourir. \\
\versenb{12}$\[$Rends à nos voisins, sept f\underline{o}is, en plein cœur, \\*
l’outrage qu’ils t’ont f\underline{a}it, Seigneur Dieu.$\]$ \\

\versenb{13}Et nous, ton peuple, le troupea\underline{u} que tu conduis,~\psalmdagger
sans fin nous pourr\underline{o}ns te rendre grâce \\
et d’âge en âge proclam\underline{e}r ta louange. \\
\end{verse}

