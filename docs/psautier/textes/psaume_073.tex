\psalm{73}{Il n’y a plus de prophètes!}
\psalmintro{Poème. D’Asaph.}

\begin{verse}
Pourquoi, Dieu, nous rejet\underline{e}r sans fin? \\
Pourquoi cette colère sur les breb\underline{i}s de ton troupeau? \\

\versenb{2}Rappelle-toi la communauté
que tu acqu\underline{i}s dès l’origine,~\psalmdagger
la tribu que tu revendiqu\underline{a}s pour héritage, \\
la montagne de Sion où tu f\underline{i}s ta demeure. \\

\versenb{3}Dirige tes pas vers ces ru\underline{i}nes sans fin, \\*
l’ennemi dans le sanctuaire a to\underline{u}t saccagé; \\
\versenb{4}dans le lieu de tes assemblées, l’advers\underline{a}ire a rugi \\*
et là, il a plant\underline{é} ses insignes. \\

\versenb{5}On les a vus brandir la cognée, \\*
comme en pl\underline{e}ine forêt,~\psalmstar
\versenb{6}quand ils brisaient les portails
à coups de m\underline{a}sse et de hache. \\

\versenb{7}Ils ont livré au fe\underline{u} ton sanctuaire, \\*
profané et rasé la deme\underline{u}re de ton nom. \\
\versenb{8}Ils ont dit: « All\underline{o}ns ! Détruisons tout ! » \\*
Ils ont brûlé dans le pays les lie\underline{u}x d’assemblées saintes. \\

\versenb{9}Nos signes, nul ne les voit; \\*
il n’y a pl\underline{u}s de prophètes!~\psalmstar
Et pour combien de temps?
Nul d’entre no\underline{u}s ne le sait! \\

\versenb{10}Dieu, combien de temps blasphémer\underline{a} l’adversaire? \\*
L’ennemi en finira-t-il de mépris\underline{e}r ton nom? \\
\versenb{11}Pourquoi reten\underline{i}r ta main, \\*
cacher la f\underline{o}rce de ton bras? \\

\versenb{12}Pourtant, Dieu, mon r\underline{o}i dès l’origine, \\*
vainqueur des combats sur la f\underline{a}ce de la terre, \\
\versenb{13}c’est toi qui fendis la m\underline{e}r par ta puissance, \\*
qui fracassas les têtes des drag\underline{o}ns sur les eaux; \\

\versenb{14}toi qui écrasas la t\underline{ê}te de Léviathan \\*
pour nourrir les m\underline{o}nstres marins; \\
\versenb{15}toi qui ouvris les torr\underline{e}nts et les sources, \\*
toi qui mis à sec des fle\underline{u}ves intarissables. \\

\versenb{16}À toi le jour, à t\underline{o}i la nuit, \\*
toi qui ajustas le sol\underline{e}il et les astres! \\
\versenb{17}C’est toi qui fixas les b\underline{o}rds de la terre; \\*
l’hiver et l’été, c’est t\underline{o}i qui les formas. \\

\versenb{18}Rappelle-toi: l’ennemi a mépris\underline{é} ton nom, \\*
un peuple de fous a blasphém\underline{é} le Seigneur. \\
\versenb{19}Ne laisse pas la Bête égorg\underline{e}r ta Tourterelle, \\*
n’oublie pas sans fin la v\underline{i}e de tes pauvres. \\

\versenb{20}Regarde vers l’Alliance: la gu\underline{e}rre est partout ; \\*
on se cache dans les cav\underline{e}rnes du pays. \\
\versenb{21}Que l’opprimé éch\underline{a}ppe à la honte, \\*
que le pauvre et le malheureux ch\underline{a}ntent ton nom! \\

\versenb{22}Lève-toi, Die\underline{u}, défends ta cause! \\*
Rappelle-toi ces fous qui blasph\underline{è}ment tout le jour. \\
\versenb{23}N’oublie pas le vacarme que f\underline{o}nt tes ennemis, \\*
la clameur de l’ennemi, qui m\underline{o}nte sans fin. \\
\end{verse}

