\psalm{108}{Ils maudissent, toi, tu bénis}
\psalmintro{Du maître de chœur. De David. Psaume.}

\begin{verse}
Die\underline{u} de ma louange, \\
s\underline{o}rs de ton silence! \\

\versenb{2}La bouche de l’impie, la bouche du fourbe, \\*
s’o\underline{u}vrent contre moi:~\psalmstar
ils parlent de moi pour d\underline{i}re des mensonges; \\
\versenb{3}ils me cernent de prop\underline{o}s haineux, \\*
ils m’att\underline{a}quent sans raison. \\

\versenb{4}Pour prix de mon amiti\underline{é}, ils m’accusent, \\*
moi qui ne su\underline{i}s que prière. \\
\versenb{5}Ils me rendent le m\underline{a}l pour le bien, \\*
ils paient mon amiti\underline{é} de leur haine. \\

\versenb{6}«Chargeons un imp\underline{i}e de l’attaquer : \\*
qu’un accusateur se ti\underline{e}nne à sa droite. \\
\versenb{7}À son procès, qu’on le décl\underline{a}re impie, \\*
que sa prière soit compt\underline{é}e comme une faute. \\

\versenb{8}«Que les jours de sa v\underline{i}e soient écourtés, \\*
qu’un autre pr\underline{e}nne sa charge. \\
\versenb{9}Que ses fils devi\underline{e}nnent orphelins, \\*
que sa f\underline{e}mme soit veuve. \\

\versenb{10}«Qu’ils soient errants, vagab\underline{o}nds, ses fils, \\*
qu’ils mendient, expuls\underline{é}s de leurs ruines. \\
\versenb{11}Qu’un usurier sais\underline{i}sse tout son bien, \\*
que d’autres s’emparent du fru\underline{i}t de son travail. \\

\versenb{12}«Que nul ne lui r\underline{e}ste fidèle, \\*
que nul n’ait piti\underline{é} de ses orphelins. \\
\versenb{13}Que soit retranch\underline{é}e sa descendance, \\*
que son nom s’eff\underline{a}ce avec ses enfants. \\

\versenb{14}«Qu’on rappelle au Seigneur les fa\underline{u}tes de ses pères, \\*
que les péchés de sa mère ne soient p\underline{a}s effacés. \\
\versenb{15}Que le Seigneur garde cel\underline{a} devant ses yeux, \\*
et retranche de la t\underline{e}rre leur mémoire! » \\

\versenb{16}Ainsi, celu\underline{i} qui m’accuse \\*
oubl\underline{i}e d’être fidèle:~\psalmstar
il persécute un pa\underline{u}vre, un malheureux, \\
un homme bless\underline{é} à mort. \\

\versenb{17}Puisqu’il \underline{a}ime la malédiction, \\*
qu’elle \underline{e}ntre en lui;~\psalmstar
il ref\underline{u}se la bénédiction, \\
qu’elle s’él\underline{o}igne de lui! \\

\versenb{18}Il a revêtu comme un mantea\underline{u} la malédiction,~\psalmstar
qu’elle entre en lui comme de l’eau, \\
comme de l’hu\underline{i}le dans ses os! \\
\versenb{19}Qu’elle soit l’ét\underline{o}ffe qui l’habille, \\*
la ceinture qui ne le qu\underline{i}tte plus! \\

\versenb{20}C’est ainsi que le Seigneur paier\underline{a} mes accusateurs, \\*
ceux qui profèrent le m\underline{a}l contre moi. \\

\versenb{21}Mais toi, Seigneur Dieu, \\*
agis pour moi à ca\underline{u}se de ton nom.~\psalmstar
Ton amour est fidèle: d\underline{é}livre-moi. \\

\versenb{22}Vois, je suis pa\underline{u}vre et malheureux; \\*
au fond de moi, mon cœur est blessé. \\
\versenb{23}Je m’en vais comme le jo\underline{u}r qui décline, \\*
comme l’ins\underline{e}cte qu’on chasse. \\

\versenb{24}J’ai tant jeûné que mes geno\underline{u}x se dérobent, \\*
je suis amaigr\underline{i}, décharné. \\
\versenb{25}Et moi, on me to\underline{u}rne en dérision, \\*
ceux qui me voient h\underline{o}chent la tête. \\

\versenb{26}Aide-moi, Seigne\underline{u}r mon Dieu: \\*
sauve-m\underline{o}i par ton amour! \\
\versenb{27}Ils connaîtront que l\underline{à} est ta main, \\*
que toi, Seigne\underline{u}r, tu agis. \\

\versenb{28}Ils maudissent, t\underline{o}i, tu bénis,~\psalmstar
ils se sont dressés, ils sont humiliés:
ton servite\underline{u}r est dans la joie. \\
\versenb{29}Qu’ils soient couverts d’infam\underline{i}e, mes accusateurs, \\*
et revêtus du mantea\underline{u} de la honte! \\

\versenb{30}À pleine voix, je rendrai gr\underline{â}ce au Seigneur, \\*
je le louerai parm\underline{i} la multitude, \\
\versenb{31}car il se tient à la dr\underline{o}ite du pauvre \\*
pour le sauver de ce\underline{u}x qui le condamnent. \\
\end{verse}

