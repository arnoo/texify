  
  
          
            \bchapter{Psaume}
            Apprends-nous la mesure de nos jours
${}^{1}Prière. De Moïse, homme de Dieu.
         
        D’âge en \underline{â}ge, Seigneur,
        tu as ét\underline{é} notre refuge.
         
${}^{2}Avant que n\underline{a}issent les montagnes, +
        \\que tu enfantes la \underline{te}rre et le monde, *
        \\de toujours à toujours,
        t\underline{o}i, tu es Dieu.
         
${}^{3}Tu fais retourner l’h\underline{o}mme à la poussière ;
        \\tu as dit : « Retourn\underline{e}z, fils d’Adam ! »
${}^{4}À tes yeux, mille \underline{a}ns sont comme hier,
        \\c’est un jour qui s’en va, une he\underline{u}re dans la nuit.
         
${}^{5}Tu les as balay\underline{é}s : ce n’est qu’un songe ;
        \\dès le matin, c’est une h\underline{e}rbe changeante :
${}^{6}elle fleurit le mat\underline{i}n, elle change ;
        \\le soir, elle est fan\underline{é}e, desséchée.
         
${}^{7}Nous voici anéant\underline{i}s par ta colère ;
        \\ta fure\underline{u}r nous épouvante :
${}^{8}tu étales nos fa\underline{u}tes devant toi,
        \\nos secrets à la lumi\underline{è}re de ta face.
         
${}^{9}Sous tes fureurs tous nos jo\underline{u}rs s’enfuient,
        \\nos années s’évanou\underline{i}ssent dans un souffle.
${}^{10}Le nombre de nos ann\underline{é}es ? Soixante-dix,
        \\quatre-vingts pour les pl\underline{u}s vigoureux !
        \\Leur plus grand nombre n’est que p\underline{e}ine et misère ;
        \\elles s’enfuient, no\underline{u}s nous envolons.
         
        *
         
${}^{11}Qui comprendra la f\underline{o}rce de ta colère ?
        \\Qui peut t’ador\underline{e}r dans tes fureurs ?
${}^{12}Apprends-nous la vraie mes\underline{u}re de nos jours :
        \\que nos cœurs pén\underline{è}trent la sagesse.
         
${}^{13}Reviens, Seigne\underline{u}r, pourquoi tarder ?
        \\Ravise-toi par ég\underline{a}rd pour tes serviteurs.
${}^{14}Rassasie-nous de ton amo\underline{u}r au matin,
        \\que nous passions nos jours dans la j\underline{o}ie et les chants.
${}^{15}Rends-nous en joies tes jo\underline{u}rs de châtiment
        \\et les années où nous connaissi\underline{o}ns le malheur.
         
${}^{16}Fais connaître ton œuvre à tes serviteurs
        \\et ta splende\underline{u}r à leurs fils.
${}^{17}Que vienne sur nous
        la douceur du Seigne\underline{u}r notre Dieu !
        \\Consolide pour nous l’ouvrage de nos mains ;
        oui, consolide l’ouvr\underline{a}ge de nos mains.
      \bchapter{Psaume}
          
            \bchapter{Psaume}
            Sous l’abri du Très-Haut
${}^{1}Quand je me tiens sous l’abr\underline{i} du Très-Haut
        \\et repose à l’\underline{o}mbre du Puissant,
${}^{2}je dis au Seigne\underline{u}r : « Mon refuge,
        \\mon rempart, mon Die\underline{u}, dont je suis sûr ! »
         
        *
         
${}^{3}C’est lui qui te sauve des filets du chasseur
        et de la p\underline{e}ste maléfique ; *
${}^{4}il te co\underline{u}vre et te protège.
        \\Tu trouves sous son \underline{a}ile un refuge :
        \\sa fidélité est une arm\underline{u}re, un bouclier.
         
${}^{5}Tu ne craindras ni les terre\underline{u}rs de la nuit,
        \\ni la flèche qui v\underline{o}le au grand jour,
${}^{6}ni la peste qui r\underline{ô}de dans le noir,
        \\ni le fléau qui fr\underline{a}ppe à midi.
         
${}^{7}Qu’il en tombe m\underline{i}lle à tes côtés, +
        \\qu’il en tombe dix m\underline{i}lle à ta droite, *
        \\toi, tu r\underline{e}stes hors d’atteinte.
         
${}^{8}Il suffit que tu o\underline{u}vres les yeux,
        \\tu verras le sal\underline{a}ire du méchant.
${}^{9}Oui, le Seigne\underline{u}r est ton refuge ;
        \\tu as fait du Très-Ha\underline{u}t ta forteresse.
         
${}^{10}Le malheur ne pourr\underline{a} te toucher,
        \\ni le danger, approch\underline{e}r de ta demeure :
${}^{11}il donne missi\underline{o}n à ses anges
        \\de te garder sur to\underline{u}s tes chemins.
         
${}^{12}Ils te porter\underline{o}nt sur leurs mains
        \\pour que ton pied ne he\underline{u}rte les pierres ;
${}^{13}tu marcheras sur la vip\underline{è}re et le scorpion,
        \\tu écraseras le li\underline{o}n et le Dragon.
         
        *
         
${}^{14}« Puisqu’il s’attache à m\underline{o}i, je le délivre ;
        \\je le défends, car il conn\underline{a}ît mon nom.
${}^{15}Il m’appelle, et m\underline{o}i, je lui réponds ;
        \\je suis avec lu\underline{i} dans son épreuve.
         
        \\« Je veux le libér\underline{e}r, le glorifier ; +
${}^{16}de longs jours, je ve\underline{u}x le rassasier, *
        \\et je ferai qu’il v\underline{o}ie mon salut. »
      \bchapter{Psaume}
          
            \bchapter{Psaume}
            Tes œuvres me comblent de joie
${}^{1}Psaume. Cantique. Pour le jour du sabbat.
         
${}^{2}Qu’il est bon de rendre gr\underline{â}ce au Seigneur,
        \\de chanter pour ton n\underline{o}m, Dieu Très-Haut,
${}^{3}d’annoncer dès le mat\underline{i}n ton amour,
        \\ta fidélit\underline{é}, au long des nuits,
${}^{4}sur la lyre à dix c\underline{o}rdes et sur la harpe,
        \\sur un murm\underline{u}re de cithare.
         
${}^{5}Tes œuvres me c\underline{o}mblent de joie ;
        \\devant l’ouvrage de tes m\underline{a}ins, je m’écrie :
${}^{6}« Que tes œuvres sont gr\underline{a}ndes, Seigneur !
        \\Combien sont prof\underline{o}ndes tes pensées ! »
         
${}^{7}L’homme born\underline{é} ne le sait pas,
        \\l’insensé ne pe\underline{u}t le comprendre :
${}^{8}les impies cr\underline{o}issent comme l’herbe, *
        \\ils fleurissent, ceux qui font le mal,
        mais pour dispar\underline{a}ître à tout jamais.
         
${}^{9}Toi, qui hab\underline{i}tes là-haut,
        \\tu es pour toujo\underline{u}rs le Seigneur.
${}^{10}Vois tes ennemis, Seigneur,
        vois tes ennem\underline{i}s qui périssent, *
        \\et la déroute de ce\underline{u}x qui font le mal.
         
${}^{11}Tu me donnes la fo\underline{u}gue du taureau,
        \\tu me baignes d’hu\underline{i}le nouvelle ;
${}^{12}j’ai vu, j’ai repér\underline{é} mes espions,
        \\j’entends ceux qui vi\underline{e}nnent m’attaquer.
         
${}^{13}Le juste grandir\underline{a} comme un palmier,
        \\il poussera comme un c\underline{è}dre du Liban ;
${}^{14}planté dans les parv\underline{i}s du Seigneur,
        \\il grandira dans la mais\underline{o}n de notre Dieu.
         
${}^{15}Vieillissant, il fructif\underline{i}e encore,
        \\il garde sa s\underline{è}ve et sa verdeur
${}^{16}pour annoncer : « Le Seigne\underline{u}r est droit !
        \\Pas de ruse en Die\underline{u}, mon rocher ! »
      \bchapter{Psaume}
          
            \bchapter{Psaume}
            Depuis toujours, tu es
${}^{1}Le Seigne\underline{u}r est roi ;
        \\il s’est vêt\underline{u} de magnificence,
        \\le Seigneur a revêt\underline{u} sa force.
         
        \\Et la terre tient b\underline{o}n, inébranlable ;
${}^{2}dès l’origine ton tr\underline{ô}ne tient bon,
        \\depuis toujo\underline{u}rs, tu es.
         
${}^{3}Les flots s’él\underline{è}vent, Seigneur,
        \\les flots él\underline{è}vent leur voix,
        \\les flots él\underline{è}vent leur fracas.
         
${}^{4}Plus que la v\underline{o}ix des eaux profondes,
        \\des vagues sup\underline{e}rbes de la mer,
        \\superbe est le Seigne\underline{u}r dans les hauteurs.
         
${}^{5}Tes volontés sont vraim\underline{e}nt immuables :
        \\la sainteté empl\underline{i}t ta maison,
        \\Seigneur, pour la su\underline{i}te des temps.
      \bchapter{Psaume}
          
            \bchapter{Psaume}
            Ton amour m’a soutenu
${}^{1}Dieu qui fais just\underline{i}ce, Seigneur,
        \\Dieu qui fais just\underline{i}ce, parais !
${}^{2}Lève-toi, j\underline{u}ge de la terre ;
        \\aux orgueilleux, r\underline{e}nds ce qu’ils méritent.
         
${}^{3}Combien de temps les imp\underline{i}es, Seigneur,
        \\combien de temps vont-\underline{i}ls triompher ?
${}^{4}Ils parlent haut, ils prof\underline{è}rent l’insolence,
        \\ils se vantent, to\underline{u}s ces malfaisants.
         
${}^{5}C’est ton peuple, Seigne\underline{u}r, qu’ils piétinent,
        \\et ton dom\underline{a}ine qu’ils écrasent ;
${}^{6}ils massacrent la ve\underline{u}ve et l’étranger,
        \\ils assass\underline{i}nent l’orphelin.
         
${}^{7}Ils disent : « Le Seigne\underline{u}r ne voit pas,
        \\le Dieu de Jac\underline{o}b ne sait pas ! »
${}^{8}Sachez-le, espr\underline{i}ts vraiment stupides ;
        \\insensés, comprendrez-vo\underline{u}s un jour ?
         
${}^{9}Lui qui forma l’or\underline{e}ille, il n’entendrait pas ? +
        \\il a façonné l’\underline{œ}il, et il ne verrait pas ?
${}^{10}il a puni des pe\underline{u}ples et ne châtierait plus ?
         
        \\Lui qui donne aux h\underline{o}mmes la connaissance, +
${}^{11}il connaît les pens\underline{é}es de l’homme,
        \\et qu’elles s\underline{o}nt du vent !
         
        *
         
${}^{12}Heureux l’homme que tu chât\underline{i}es, Seigneur,
        \\celui que tu ens\underline{e}ignes par ta loi,
${}^{13}pour le garder en paix aux jo\underline{u}rs de malheur,
        \\tandis que se creuse la f\underline{o}sse de l’impie.
         
${}^{14}Car le Seigneur ne délaisse p\underline{a}s son peuple,
        \\il n’abandonne p\underline{a}s son domaine :
${}^{15}on jugera de nouveau sel\underline{o}n la justice ;
        \\tous les hommes dr\underline{o}its applaudiront.
         
${}^{16}Qui se lèvera pour me déf\underline{e}ndre des méchants ?
        \\Qui m’assistera f\underline{a}ce aux criminels ?
${}^{17}Si le Seigneur ne m’av\underline{a}it secouru,*
        \\j’allais habit\underline{e}r le silence.
         
${}^{18}Quand je dis : « Mon pi\underline{e}d trébuche ! »,
        \\ton amour, Seigne\underline{u}r, me soutient.
${}^{19}Quand d’innombrables souc\underline{i}s m’envahissent,
        \\tu me réconf\underline{o}rtes et me consoles.
         
${}^{20}Es-tu l’allié d’un pouv\underline{o}ir corrompu
        \\qui engendre la misère au mépr\underline{i}s des lois ?
${}^{21}On s’attaque à la v\underline{i}e de l’innocent,
        \\le juste que l’on tue est déclar\underline{é} coupable.
         
${}^{22}Mais le Seigneur ét\underline{a}it ma forteresse,
        \\et Dieu, le roch\underline{e}r de mon refuge.
${}^{23}Il retourne sur e\underline{u}x leur méfait :
        \\pour leur malice, qu’il les réduise au silence,
        qu’il les réduise au silence, le Seigne\underline{u}r notre Dieu.
      \bchapter{Psaume}
          
            \bchapter{Psaume}
            Aujourd’hui écouterez-vous sa parole ?
${}^{1}Venez, crions de j\underline{o}ie pour le Seigneur,
        \\acclamons notre Roch\underline{e}r, notre salut !
${}^{2}Allons jusqu’à lu\underline{i} en rendant grâce,
        \\par nos hymnes de f\underline{ê}te acclamons-le !
         
${}^{3}Oui, le grand Die\underline{u}, c’est le Seigneur,
        \\le grand roi au-dess\underline{u}s de tous les dieux :
${}^{4}il tient en main les profonde\underline{u}rs de la terre,
        \\et les sommets des mont\underline{a}gnes sont à lui ;
${}^{5}à lui la mer, c’est lu\underline{i} qui l’a faite,
        \\et les terres, car ses m\underline{a}ins les ont pétries.
         
${}^{6}Entrez, inclinez-vo\underline{u}s, prosternez-vous,
        \\adorons le Seigne\underline{u}r qui nous a faits.
${}^{7}Oui, il \underline{e}st notre Dieu ; +
        \\nous sommes le pe\underline{u}ple qu’il conduit,
        \\le troupeau guid\underline{é} par sa main.
         
        \\Aujourd’hui écouterez-vo\underline{u}s sa parole ? +
${}^{8}« Ne fermez pas votre cœ\underline{u}r comme au désert,
        \\comme au jour de tentati\underline{o}n et de défi,
${}^{9}où vos pères m’ont tent\underline{é} et provoqué,
        \\et pourtant ils avaient v\underline{u} mon exploit.
         
${}^{10}« Quarante ans leur générati\underline{o}n m’a déçu, +
        \\et j’ai dit : Ce peuple a le cœ\underline{u}r égaré,
        \\il n’a pas conn\underline{u} mes chemins.
${}^{11}Dans ma colère, j’en ai f\underline{a}it le serment :
        \\Jamais ils n’entrer\underline{o}nt dans mon repos. »
      \bchapter{Psaume}
          
            \bchapter{Psaume}
            Il vient pour juger la terre
${}^{1}Chantez au Seigne\underline{u}r un chant nouveau,
        \\chantez au Seigne\underline{u}r, terre entière,
${}^{2}chantez au Seigneur et béniss\underline{e}z son nom !
         
        \\De jour en jour, proclam\underline{e}z son salut,
${}^{3}racontez à tous les pe\underline{u}ples sa gloire,
        \\à toutes les nati\underline{o}ns ses merveilles !
         
${}^{4}Il est grand, le Seigneur, hautem\underline{e}nt loué,
        \\redoutable au-dess\underline{u}s de tous les dieux :
${}^{5}néant, tous les die\underline{u}x des nations !
         
        \\Lui, le Seigne\underline{u}r, a fait les cieux :
${}^{6}devant lui, splende\underline{u}r et majesté,
        \\dans son sanctuaire, puiss\underline{a}nce et beauté.
         
${}^{7}Rendez au Seigneur, fam\underline{i}lles des peuples,
        \\rendez au Seigneur la gl\underline{o}ire et la puissance,
${}^{8}rendez au Seigneur la gl\underline{o}ire de son nom.
         
        \\Apportez votre offrande, entr\underline{e}z dans ses parvis,
${}^{9}adorez le Seigneur, éblouiss\underline{a}nt de sainteté :
        \\tremblez devant lu\underline{i}, terre entière.
         
${}^{10}Allez dire aux nations : « Le Seigne\underline{u}r est roi ! »
        \\Le monde, inébranl\underline{a}ble, tient bon.
        \\Il gouverne les pe\underline{u}ples avec droiture.
         
${}^{11}Joie au ciel ! Ex\underline{u}lte la terre !
        \\Les masses de la m\underline{e}r mugissent,
${}^{12}la campagne tout enti\underline{è}re est en fête.
         
        \\Les arbres des forêts d\underline{a}nsent de joie
${}^{13}devant la face du Seigne\underline{u}r, car il vient,
        \\car il vient pour jug\underline{e}r la terre.
         
        \\Il jugera le m\underline{o}nde avec justice, *
        \\et les peuples sel\underline{o}n sa vérité !
      \bchapter{Psaume}
          
            \bchapter{Psaume}
            Le Très-Haut sur toute la terre
${}^{1}Le Seigneur est roi ! Ex\underline{u}lte la terre !
        \\Joie pour les \underline{î}les sans nombre !
         
${}^{2}Ténèbre et nu\underline{é}e l’entourent,
        \\justice et droit sont l’appu\underline{i} de son trône.
${}^{3}Devant lui s’av\underline{a}nce un feu
        \\qui consume alento\underline{u}r ses ennemis.
         
${}^{4}Quand ses éclairs illumin\underline{è}rent le monde,
        \\la terre le v\underline{i}t et s’affola ;
${}^{5}les montagnes fondaient comme cire dev\underline{a}nt le Seigneur,
        \\devant le Maître de to\underline{u}te la terre.
         
${}^{6}Les cieux ont proclam\underline{é} sa justice,
        \\et tous les peuples ont v\underline{u} sa gloire.
${}^{7}Honte aux serviteurs d’idoles qui se v\underline{a}ntent de vanités !
        \\À genoux devant lu\underline{i}, tous les dieux !
         
        *
         
${}^{8}Pour Sion qui ent\underline{e}nd, grande joie ! *
        \\Les villes de Juda exultent devant tes jugem\underline{e}nts, Seigneur !
         
${}^{9}Tu es, Seigneur, le Très-Haut sur to\underline{u}te la terre : *
        \\tu domines de ha\underline{u}t tous les dieux.
         
${}^{10}Haïssez le mal, vous qui aim\underline{e}z le Seigneur, +
        \\car il garde la v\underline{i}e de ses fidèles *
        \\et les arrache aux m\underline{a}ins des impies.
         
${}^{11}Une lumière est sem\underline{é}e pour le juste,
        \\et pour le cœur s\underline{i}mple, une joie.
${}^{12}Que le Seigneur soit votre j\underline{o}ie, hommes justes ;
        \\rendez grâce en rappel\underline{a}nt son nom très saint.
      \bchapter{Psaume}
          
            \bchapter{Psaume}
            La terre a vu la victoire de Dieu
${}^{1}Psaume.
         
        \\Chantez au Seigne\underline{u}r un chant nouveau,
        \\car il a f\underline{a}it des merveilles ;
        \\par son bras très saint, par sa m\underline{a}in puissante,
        \\il s’est assur\underline{é} la victoire.
         
${}^{2}Le Seigneur a fait conn\underline{a}ître sa victoire
        \\et révélé sa just\underline{i}ce aux nations ;
${}^{3}il s’est rappelé sa fidélit\underline{é}, son amour,
        \\en faveur de la mais\underline{o}n d’Israël ;
        \\la terre tout enti\underline{è}re a vu
        \\la vict\underline{o}ire de notre Dieu.
         
${}^{4}Acclamez le Seigne\underline{u}r, terre entière,
        \\sonnez, chant\underline{e}z, jouez ;
${}^{5}jouez pour le Seigne\underline{u}r sur la cithare,
        \\sur la cithare et t\underline{o}us les instruments ;
${}^{6}au son de la tromp\underline{e}tte et du cor,
        \\acclamez votre r\underline{o}i, le Seigneur !
         
${}^{7}Que résonnent la m\underline{e}r et sa richesse,
        \\le monde et to\underline{u}s ses habitants ;
${}^{8}que les fleuves b\underline{a}ttent des mains,
        \\que les montagnes ch\underline{a}ntent leur joie,
${}^{9}à la face du Seigneur, car il vient
        pour gouvern\underline{e}r la terre, *
        \\pour gouverner le m\underline{o}nde avec justice
        et les pe\underline{u}ples avec droiture !
      \bchapter{Psaume}
          
            \bchapter{Psaume}
            Il est saint !
${}^{1}Le Seigneur est roi : les pe\underline{u}ples s’agitent.
        \\Il trône au-dessus des Kéroubim : la t\underline{e}rre tremble.
         
${}^{2}En Sion le Seigne\underline{u}r est grand :
        \\c’est lui qui dom\underline{i}ne tous les peuples.
         
${}^{3}Ils proclament ton nom, gr\underline{a}nd et redoutable,
        \\c\underline{a}r il est saint !
         
${}^{4}Il est fort, le roi qui \underline{a}ime la justice. +
        \\C’est toi, l’aute\underline{u}r du droit,
        \\toi qui assures en Jacob la just\underline{i}ce et la droiture.
         
${}^{5}Exaltez le Seigne\underline{u}r notre Dieu, +
        \\prosternez-vous au pi\underline{e}d de son trône,
        \\c\underline{a}r il est saint !
         
${}^{6}Moïse et le prêtre Aaron,
        Samu\underline{e}l, le Suppliant, +
        \\tous, ils suppli\underline{a}ient le Seigneur, *
        \\et lu\underline{i} leur répondait.
         
${}^{7}Dans la colonne de nuée, il parl\underline{a}it avec eux ;
        \\ils ont gardé ses volontés, les l\underline{o}is qu’il leur donna.
         
${}^{8}Seigneur notre Dieu, tu leur \underline{a}s répondu : +
        \\avec eux, tu restais un Die\underline{u} patient,
        \\mais tu les puniss\underline{a}is pour leurs fautes.
         
${}^{9}Exaltez le Seigne\underline{u}r notre Dieu, +
        \\prosternez-vous devant sa s\underline{a}inte montagne, *
        \\car il est saint,
        le Seigne\underline{u}r notre Dieu.
      \bchapter{Psaume}
          
            \bchapter{Psaume}
            Rendez-lui grâce : il est fidèle
${}^{1}Psaume. Pour l’action de grâce.
         
        \\Acclamez le Seigne\underline{u}r, terre entière,
${}^{2}servez le Seigne\underline{u}r dans l’allégresse,
        \\venez à lui avec des ch\underline{a}nts de joie !
         
${}^{3}Reconnaissez que le Seigne\underline{u}r est Dieu :
        \\il nous a faits, et nous s\underline{o}mmes à lui,
        \\nous, son pe\underline{u}ple, son troupeau.
         
${}^{4}Venez dans sa mais\underline{o}n lui rendre grâce,
        \\dans sa demeure chant\underline{e}r ses louanges ;
        \\rendez-lui grâce et béniss\underline{e}z son nom !
         
${}^{5}Oui, le Seigne\underline{u}r est bon,
        \\étern\underline{e}l est son amour,
        \\sa fidélité deme\underline{u}re d’âge en âge.
      \bchapter{Psaume}
          
            \bchapter{Psaume}
            Par le chemin le plus parfait
${}^{1}De David. Psaume.
         
        \\Je chanterai just\underline{i}ce et bonté : *
        \\à toi mes h\underline{y}mnes, Seigneur !
${}^{2}J’irai par le chem\underline{i}n le plus parfait ; *
        \\quand viendras-t\underline{u} jusqu’à moi ?
         
        \\Je marcherai d’un cœ\underline{u}r parfait
        avec ce\underline{u}x de ma maison ; *
${}^{3}je n’aurai pas m\underline{ê}me un regard
        pour les prat\underline{i}ques démoniaques.
         
        \\Je haïrai l’acti\underline{o}n du traître
        qui n’aura sur m\underline{o}i nulle prise ; *
${}^{4}loin de m\underline{o}i, le cœur tortueux !
        Le méchant, je ne veux p\underline{a}s le connaître.
         
${}^{5}Qui dénigre en secr\underline{e}t son prochain,
        je le réduir\underline{a}i au silence ; *
        \\le regard hautain, le cœ\underline{u}r ambitieux,
        je ne pe\underline{u}x les tolérer.
         
${}^{6}Mes yeux distinguent les hommes s\underline{û}rs du pays :
        ils sièger\underline{o}nt à mes côtés ; *
        \\qui se conduir\underline{a} parfaitement
        celui-l\underline{à} me servira.
         
${}^{7}Pas de siège, parmi ce\underline{u}x de ma maison,
        pour qui se l\underline{i}vre à la fraude ; *
        \\impossible à qui prof\underline{è}re le mensonge
        de ten\underline{i}r sous mon regard.
         
${}^{8}Chaque matin, je réduir\underline{a}i au silence
        tous les coup\underline{a}bles du pays, *
        \\pour extirper de la v\underline{i}lle du Seigneur
        tous les aute\underline{u}rs de crimes.
      \bchapter{Psaume}
          
            \bchapter{Psaume}
            Ils passent. Tu demeures
${}^{1}Prière du malheureux qui défaille et devant le Seigneur répand sa plainte.
         
${}^{2}Seigneur, ent\underline{e}nds ma prière :
        \\que mon cri parvi\underline{e}nne jusqu’à toi !
${}^{3}Ne me cache p\underline{a}s ton visage
        \\le jour où je su\underline{i}s en détresse !
        \\Le jour où j’app\underline{e}lle, écoute-moi ;
        \\viens v\underline{i}te, réponds-moi !
         
${}^{4}Mes jours s’en v\underline{o}nt en fumée,
        \\mes os comme un brasi\underline{e}r sont en feu ;
${}^{5}mon cœur se dessèche comme l’h\underline{e}rbe fauchée,
        \\j’oublie de mang\underline{e}r mon pain ;
${}^{6}à force de cri\underline{e}r ma plainte,
        \\ma peau c\underline{o}lle à mes os.
         
${}^{7}Je ressemble au corbea\underline{u} du désert,
        \\je suis pareil à la hul\underline{o}tte des ruines :
${}^{8}je v\underline{e}ille la nuit,
        \\comme un oiseau solit\underline{a}ire sur un toit.
${}^{9}Le jour, mes ennem\underline{i}s m’outragent ;
        \\dans leur rage contre m\underline{o}i, ils me maudissent.
         
${}^{10}La cendre est le p\underline{a}in que je mange,
        \\je mêle à ma boiss\underline{o}n mes larmes.
${}^{11}Dans ton indignati\underline{o}n, dans ta colère,
        \\tu m’as sais\underline{i} et rejeté :
${}^{12}l’ombre g\underline{a}gne sur mes jours,
        \\et moi, je me dess\underline{è}che comme l’herbe.
         
        *
         
${}^{13}Mais toi, Seigneur, tu es l\underline{à} pour toujours ;
        \\d’âge en âge on fera mém\underline{o}ire de toi.
${}^{14}Toi, tu montreras ta tendr\underline{e}sse pour Sion ;
        \\il est temps de la prendre en pitié : l’he\underline{u}re est venue.
${}^{15}Tes serviteurs ont piti\underline{é} de ses ruines,
        \\ils aiment j\underline{u}squ’à sa poussière.
         
${}^{16}Les nations craindront le n\underline{o}m du Seigneur,
        \\et tous les rois de la t\underline{e}rre, sa gloire :
${}^{17}quand le Seigneur rebâtir\underline{a} Sion,
        \\quand il apparaîtr\underline{a} dans sa gloire,
${}^{18}il se tournera vers la pri\underline{è}re du spolié,
        \\il n’aura pas mépris\underline{é} sa prière.
         
${}^{19}Que cela soit écrit pour l’\underline{â}ge à venir,
        \\et le peuple à nouveau créé chanter\underline{a} son Dieu :
${}^{20}« Des hauteurs, son sanctuaire, le Seigne\underline{u}r s’est penché ;
        \\du ciel, il reg\underline{a}rde la terre
${}^{21}pour entendre la pl\underline{a}inte des captifs
        \\et libérer ceux qui dev\underline{a}ient mourir. »
         
${}^{22}On publiera dans Sion le n\underline{o}m du Seigneur
        \\et sa louange dans to\underline{u}t Jérusalem,
${}^{23}au rassemblement des roya\underline{u}mes et des peuples
        \\qui viendront serv\underline{i}r le Seigneur.
         
        *
         
${}^{24}Il a brisé ma f\underline{o}rce en chemin,
        \\réduit le n\underline{o}mbre de mes jours.
${}^{25}Et j’ai d\underline{i}t : « Mon Dieu,
        \\ne me prends pas au milie\underline{u} de mes jours ! »
         
        \\Tes années reco\underline{u}vrent tous les temps : +
${}^{26}autrefois tu as fond\underline{é} la terre ;
        \\le ciel est l’ouvr\underline{a}ge de tes mains.
         
${}^{27}Ils passent, mais t\underline{o}i, tu demeures : +
        \\ils s’usent comme un hab\underline{i}t, l’un et l’autre ;
        \\tu les remplaces c\underline{o}mme un vêtement.
         
${}^{28}Toi, tu \underline{e}s le même ;
        \\tes années ne fin\underline{i}ssent pas.
${}^{29}Les fils de tes serviteurs trouver\underline{o}nt un séjour,
        \\et devant toi se maintiendr\underline{a} leur descendance.
      \bchapter{Psaume}
          
            \bchapter{Psaume}
            La tendresse du père pour ses fils
${}^{1}De David.
         
        \\Bénis le Seigne\underline{u}r, ô mon âme,
        \\bénis son nom très s\underline{a}int, tout mon être !
${}^{2}Bénis le Seigne\underline{u}r, ô mon âme,
        \\n’oublie auc\underline{u}n de ses bienfaits !
         
${}^{3}Car il pardonne to\underline{u}tes tes offenses
        \\et te guérit de to\underline{u}te maladie ;
${}^{4}il réclame ta v\underline{i}e à la tombe
        \\et te couronne d’amo\underline{u}r et de tendresse ;
${}^{5}il comble de bi\underline{e}ns tes vieux jours :
        \\tu renouvelles, comme l’\underline{a}igle, ta jeunesse.
         
${}^{6}Le Seigneur fait œ\underline{u}vre de justice,
        \\il défend le dr\underline{o}it des opprimés.
${}^{7}Il révèle ses dess\underline{e}ins à Moïse,
        \\aux enfants d’Isra\underline{ë}l ses hauts faits.
         
${}^{8}Le Seigneur est tendr\underline{e}sse et pitié,
        \\lent à la col\underline{è}re et plein d’amour ;
${}^{9}il n’est pas pour toujo\underline{u}rs en procès,
        \\ne maintient pas sans f\underline{i}n ses reproches ;
${}^{10}il n’agit pas envers no\underline{u}s selon nos fautes,
        \\ne nous rend pas sel\underline{o}n nos offenses.
         
${}^{11}Comme le ciel dom\underline{i}ne la terre,
        \\fort est son amo\underline{u}r pour qui le craint ;
${}^{12}aussi loin qu’est l’ori\underline{e}nt de l’occident,
        \\il met loin de no\underline{u}s nos péchés ;
${}^{13}comme la tendresse du p\underline{è}re pour ses fils,
        \\la tendresse du Seigne\underline{u}r pour qui le craint !
         
${}^{14}Il sait de quoi nous s\underline{o}mmes pétris,
        \\il se souvient que nous s\underline{o}mmes poussière.
${}^{15}L’homme ! ses jo\underline{u}rs sont comme l’herbe ;
        \\comme la fleur des ch\underline{a}mps, il fleurit :
${}^{16}dès que souffle le v\underline{e}nt, il n’est plus,
        \\même la place où il ét\underline{a}it l’ignore.
         
${}^{17}Mais l’amour du Seigneur, sur ceux qui le craignent,
        est de toujo\underline{u}rs à toujours, *
        \\et sa justice pour les enf\underline{a}nts de leurs enfants,
${}^{18}pour ceux qui g\underline{a}rdent son alliance
        \\et se souviennent d’accompl\underline{i}r ses volontés.
${}^{19}Le Seigneur a son tr\underline{ô}ne dans les cieux :
        \\sa royauté s’ét\underline{e}nd sur l’univers.
         
${}^{20}Messagers du Seigneur, bénissez-le,
        invincibles porte\underline{u}rs de ses ordres, *
        \\attentifs au s\underline{o}n de sa parole !
${}^{21}Bénissez-le, arm\underline{é}es du Seigneur,
        \\serviteurs qui exécut\underline{e}z ses désirs !
${}^{22}Toutes les œuvres du Seigne\underline{u}r, bénissez-le,
        \\sur toute l’étend\underline{u}e de son empire !
         
        \\Bénis le Seigne\underline{u}r, ô mon âme !
      \bchapter{Psaume}
          
            \bchapter{Psaume}
            Quelle profusion dans tes œuvres !
${}^{1}Bénis le Seigne\underline{u}r, ô mon âme ;
        \\Seigneur mon Die\underline{u}, tu es si grand !
        \\Revêt\underline{u} de magnificence,
${}^{2}tu as pour mantea\underline{u} la lumière !
         
        \\Comme une tenture, tu dépl\underline{o}ies les cieux,
${}^{3}tu élèves dans leurs ea\underline{u}x tes demeures ;
        \\des nuées, tu te f\underline{a}is un char,
        \\tu t’avances sur les \underline{a}iles du vent ;
${}^{4}tu prends les v\underline{e}nts pour messagers,
        \\pour serviteurs, les fl\underline{a}mmes des éclairs.
         
${}^{5}Tu as donné son ass\underline{i}se à la terre :
        \\qu’elle reste inébranl\underline{a}ble au cours des temps.
${}^{6}Tu l’as vêtue de l’ab\underline{î}me des mers :
        \\les eaux couvraient m\underline{ê}me les montagnes ;
${}^{7}à ta menace, elles pr\underline{e}nnent la fuite,
        \\effrayées par le tonn\underline{e}rre de ta voix.
         
${}^{8}Elles passent les montagnes, se r\underline{u}ent dans les vallées
        \\vers le lieu que tu leur \underline{a}s préparé.
${}^{9}Tu leur imposes la lim\underline{i}te à ne pas franchir :
        \\qu’elles ne reviennent jam\underline{a}is couvrir la terre.
         
${}^{10}Dans les ravins tu fais jaill\underline{i}r des sources
        \\et l’eau chemine aux cre\underline{u}x des montagnes ;
${}^{11}elle abreuve les b\underline{ê}tes des champs :
        \\l’âne sauvage y c\underline{a}lme sa soif ;
${}^{12}les oiseaux séjo\underline{u}rnent près d’elle :
        \\dans le feuillage on ent\underline{e}nd leurs cris.
         
        *
         
${}^{13}De tes demeures tu abre\underline{u}ves les montagnes,
        \\et la terre se rassasie du fru\underline{i}t de tes œuvres ;
${}^{14}tu fais pousser les prair\underline{i}es pour les troupeaux,
        \\et les champs pour l’h\underline{o}mme qui travaille.
         
        \\De la terre il t\underline{i}re son pain :
${}^{15}le vin qui réjou\underline{i}t le cœur de l’homme,
        \\l’huile qui adouc\underline{i}t son visage,
        \\et le pain qui fortif\underline{i}e le cœur de l’homme.
         
${}^{16}Les arbres du Seigne\underline{u}r se rassasient,
        \\les cèdres qu’il a plant\underline{é}s au Liban ;
${}^{17}c’est là que vient nich\underline{e}r le passereau,
        \\et la cigogne a sa mais\underline{o}n dans les cyprès ;
${}^{18}aux chamois, les ha\underline{u}tes montagnes,
        \\aux marmottes, l’abr\underline{i} des rochers.
         
${}^{19}Tu fis la lune qui m\underline{a}rque les temps
        \\et le soleil qui connaît l’he\underline{u}re de son coucher.
${}^{20}Tu fais descendre les tén\underline{è}bres, la nuit vient :
        \\les animaux dans la for\underline{ê}t s’éveillent ;
${}^{21}le lionceau rug\underline{i}t vers sa proie,
        \\il réclame à Die\underline{u} sa nourriture.
         
${}^{22}Quand paraît le sol\underline{e}il, ils se retirent :
        \\chacun g\underline{a}gne son repaire.
${}^{23}L’homme s\underline{o}rt pour son ouvrage,
        \\pour son trav\underline{a}il, jusqu’au soir.
         
        *
         
${}^{24}Quelle profusion dans tes œuvres, Seigneur ! +
        \\Tout cela, ta sag\underline{e}sse l’a fait ; *
        \\la terre s’empl\underline{i}t de tes biens.
         
${}^{25}Voici l’immensit\underline{é} de la mer,
        \\son grouillement innombrable d’animaux gr\underline{a}nds et petits,
${}^{26}ses batea\underline{u}x qui voyagent,
        \\et Léviathan que tu fis pour qu’il s\underline{e}rve à tes jeux.
         
${}^{27}Tous, ils c\underline{o}mptent sur toi
        \\pour recevoir leur nourrit\underline{u}re au temps voulu.
${}^{28}Tu donnes : e\underline{u}x, ils ramassent ;
        \\tu ouvres la m\underline{a}in : ils sont comblés.
         
${}^{29}Tu caches ton vis\underline{a}ge : ils s’épouvantent ;
        \\tu reprends leur souffle, ils expirent
        et reto\underline{u}rnent à leur poussière.
${}^{30}Tu envoies ton so\underline{u}ffle : ils sont créés ;
        \\tu renouvelles la f\underline{a}ce de la terre.
         
${}^{31}Gloire au Seigne\underline{u}r à tout jamais !
        \\Que Dieu se réjou\underline{i}sse en ses œuvres !
${}^{32}Il regarde la t\underline{e}rre : elle tremble ;
        \\il touche les mont\underline{a}gnes : elles brûlent.
         
${}^{33}Je veux chanter au Seigne\underline{u}r tant que je vis ;
        \\je veux jouer pour mon Die\underline{u} tant que je dure.
${}^{34}Que mon poème lui s\underline{o}it agréable ;
        \\moi, je me réjou\underline{i}s dans le Seigneur.
${}^{35}Que les pécheurs dispar\underline{a}issent de la terre !
        \\Que les imp\underline{i}es n’existent plus !
         
        \\Bénis le Seigne\underline{u}r, ô mon âme !
      \bchapter{Psaume}
          
            \bchapter{Psaume}
            Dieu fidèle à sa parole
        \\Alléluia !
         
${}^{1}Rendez grâce au Seigneur, proclam\underline{e}z son nom,
        \\annoncez parmi les pe\underline{u}ples ses hauts faits ;
${}^{2}chantez et jou\underline{e}z pour lui,
        \\redites sans f\underline{i}n ses merveilles ;
${}^{3}glorifiez-vous de son n\underline{o}m très saint :
        \\joie pour les cœ\underline{u}rs qui cherchent Dieu !
         
${}^{4}Cherchez le Seigne\underline{u}r et sa puissance,
        \\recherchez sans tr\underline{ê}ve sa face ;
${}^{5}souvenez-vous des merv\underline{e}illes qu’il a faites,
        \\de ses prodiges, des jugem\underline{e}nts qu’il prononça,
${}^{6}vous, la race d’Abrah\underline{a}m son serviteur,
        \\les fils de Jac\underline{o}b, qu’il a choisis.
         
${}^{7}Le Seigneur, c’est lu\underline{i} notre Dieu :
        \\ses jugements font l\underline{o}i pour l’univers.
${}^{8}Il s’est toujours souven\underline{u} de son alliance,
        \\parole édictée pour m\underline{i}lle générations :
         
${}^{9}promesse f\underline{a}ite à Abraham,
        \\garantie par serm\underline{e}nt à Isaac,
${}^{10}érigée en l\underline{o}i avec Jacob,
        \\alliance étern\underline{e}lle pour Israël.
         
${}^{11}Il a dit : « Je vous donne le pa\underline{y}s de Canaan,
        \\ce sera votre p\underline{a}rt d’héritage. »
         
        *
         
${}^{12}En ces temps-là, on pouv\underline{a}it les compter :
        \\c’était une poign\underline{é}e d’immigrants ;
${}^{13}ils allaient de nati\underline{o}n en nation,
        \\d’un royaume vers un a\underline{u}tre royaume.
         
${}^{14}Mais Dieu ne souffrait p\underline{a}s qu’on les opprime ;
        \\à cause d’eux, il châti\underline{a}it des rois :
${}^{15}« Ne touchez pas à qu\underline{i} m’est consacré,
        \\ne maltraitez p\underline{a}s mes prophètes ! »
         
${}^{16}Il appela sur le pa\underline{y}s la famine,
        \\le privant de to\underline{u}te ressource.
${}^{17}Mais devant eux il envoy\underline{a} un homme,
        \\Joseph, qui fut vend\underline{u} comme esclave.
         
${}^{18}On lui met aux pi\underline{e}ds des entraves,
        \\on lui passe des f\underline{e}rs au cou ;
${}^{19}il souffrait pour la par\underline{o}le du Seigneur,
        \\jusqu’au jour où s’accompl\underline{i}t sa prédiction.
         
${}^{20}Le roi ordonne qu’il s\underline{o}it relâché,
        \\le maître des peuples, qu’il s\underline{o}it libéré.
${}^{21}Il fait de lui le ch\underline{e}f de sa maison,
        \\le maître de to\underline{u}s ses biens,
${}^{22}pour que les princes lui s\underline{o}ient soumis,
        \\et qu’il apprenne la sag\underline{e}sse aux anciens.
         
        *
         
${}^{23}Alors Israël \underline{e}ntre en Égypte,
        \\Jacob émigre au pa\underline{y}s de Cham.
${}^{24}Dieu rend son pe\underline{u}ple nombreux
        \\et plus puissant que to\underline{u}s ses adversaires ;
${}^{25}ceux-là, il les f\underline{a}it se raviser,
        \\haïr son peuple et tromp\underline{e}r ses serviteurs.
         
${}^{26}Mais il envoie son servite\underline{u}r, Moïse,
        \\avec un homme de son ch\underline{o}ix, Aaron,
${}^{27}pour annoncer des s\underline{i}gnes prodigieux,
        \\des miracles au pa\underline{y}s de Cham.
         
${}^{28}Il envoie les ténèbres, to\underline{u}t devient ténèbres :
        \\nul ne rés\underline{i}ste à sa parole ;
${}^{29}il change les ea\underline{u}x en sang
        \\et fait pér\underline{i}r les poissons.
         
${}^{30}Des grenouilles envah\underline{i}ssent le pays
        \\jusque dans les ch\underline{a}mbres du roi.
${}^{31}Il parle, et la verm\underline{i}ne arrive :
        \\des moustiques, sur to\underline{u}te la contrée.
         
${}^{32}Au lieu de la pluie, il d\underline{o}nne la grêle,
        \\des éclairs qui incend\underline{i}ent les champs ;
${}^{33}il frappe les v\underline{i}gnes et les figuiers,
        \\il brise les \underline{a}rbres du pays.
         
${}^{34}Il parle, et les sauter\underline{e}lles arrivent,
        \\des insectes en n\underline{o}mbre infini
${}^{35}qui mangent toute l’h\underline{e}rbe du pays,
        \\qui mangent le fru\underline{i}t de leur sol.
         
${}^{36}Il frappe les fils aîn\underline{é}s du pays,
        \\toute la fle\underline{u}r de la race ;
${}^{37}il fait sortir les siens charg\underline{é}s d’argent et d’or ;
        \\pas un n’a flanch\underline{é} dans leurs tribus !
${}^{38}Et l’Égypte se réjou\underline{i}t de leur départ,
        \\car ils l’av\underline{a}ient terrorisée.
         
        *
         
${}^{39}Il étend une nu\underline{é}e pour les couvrir ;
        \\la nuit, un fe\underline{u} les éclaire.
${}^{40}À leur demande, il fait pass\underline{e}r des cailles,
        \\il les rassasie du p\underline{a}in venu des cieux ;
${}^{41}il ouvre le roch\underline{e}r : l’eau jaillit,
        \\un fleuve co\underline{u}le au désert.
         
${}^{42}Il s’est ainsi souvenu de la par\underline{o}le sacrée
        \\et d’Abrah\underline{a}m, son serviteur ;
${}^{43}il a fait sortir en grande f\underline{ê}te son peuple,
        \\ses élus, avec des cr\underline{i}s de joie !
         
${}^{44}Il leur a donné les t\underline{e}rres des nations,
        \\en héritage, le trav\underline{a}il des peuples,
${}^{45}pourvu qu’ils g\underline{a}rdent ses volontés
        \\et qu’ils obs\underline{e}rvent ses lois.
         
        Alléluia !
      \bchapter{Psaume}
          
            \bchapter{Psaume}
            Avec nos pères, nous avons péché
${}^{1}Alléluia !
         
        \\Rendez grâce au Seigne\underline{u}r : Il est bon !
        \\Étern\underline{e}l est son amour !
${}^{2}Qui dira les hauts f\underline{a}its du Seigneur,
        \\qui célébrer\underline{a} ses louanges ?
${}^{3}Heureux qui prat\underline{i}que la justice,
        \\qui observe le dr\underline{o}it en tout temps !
         
${}^{4}Souviens-toi de m\underline{o}i, Seigneur,
        \\dans ta bienveill\underline{a}nce pour ton peuple ;
        \\toi qui le sa\underline{u}ves, visite-moi :
${}^{5}que je voie le bonhe\underline{u}r de tes élus ;
        \\que j’aie part à la j\underline{o}ie de ton peuple,
        \\à la fiert\underline{é} de ton héritage.
         
        *
         
${}^{6}Avec nos pères, nous av\underline{o}ns péché,
        \\nous avons faill\underline{i} et renié.
${}^{7}En Égypte, nos pères ont méconn\underline{u} tes miracles,
        \\oublié l’abondance de tes grâces
        et résisté au b\underline{o}rd de la mer Rouge.
${}^{8}Mais à cause de son n\underline{o}m, il les sauva,
        \\pour que soit reconn\underline{u}e sa puissance.
         
${}^{9}Il menace la mer Ro\underline{u}ge, elle sèche ;
        \\il les mène à travers les ea\underline{u}x comme au désert.
${}^{10}Il les sauve des m\underline{a}ins de l’oppresseur,
        \\il les rachète aux m\underline{a}ins de l’ennemi.
         
${}^{11}Et les eaux reco\underline{u}vrent leurs adversaires :
        \\pas un d’entre e\underline{u}x n’en réchappe.
${}^{12}Alors ils cr\underline{o}ient à sa parole,
        \\ils ch\underline{a}ntent sa louange.
         
        *
         
${}^{13}Ils s’empressent d’oubli\underline{e}r ce qu’il a fait,
        \\sans attendre de conn\underline{a}ître ses desseins.
${}^{14}Ils se livrent à leur convoit\underline{i}se dans le désert ;
        \\là, ils mettent Die\underline{u} à l’épreuve :
${}^{15}et Dieu leur donne ce qu’ils \underline{o}nt réclamé,
        \\mais ils trouvent ses d\underline{o}ns dérisoires.
         
${}^{16}Dans le camp ils sont jalo\underline{u}x de Moïse
        \\et d’Aaron, le pr\underline{ê}tre du Seigneur.
${}^{17}La terre s’ouvre : elle av\underline{a}le Datane,
        \\elle recouvre la b\underline{a}nde d’Abiram ;
${}^{18}un feu détru\underline{i}t cette bande,
        \\les flammes dév\underline{o}rent ces méchants.
         
${}^{19}À l’Horeb ils fabr\underline{i}quent un veau,
        \\ils adorent un obj\underline{e}t en métal :
${}^{20}ils échangeaient ce qui ét\underline{a}it leur gloire
        \\pour l’image d’un taurea\underline{u}, d’un ruminant.
         
${}^{21}Ils oublient le Die\underline{u} qui les sauve,
        \\qui a fait des prod\underline{i}ges en Égypte,
${}^{22}des miracles au pa\underline{y}s de Cham,
        \\des actions terrifi\underline{a}ntes sur la mer Rouge.
         
${}^{23}Dieu a décid\underline{é} de les détruire.
        \\C’est alors que Mo\underline{ï}se, son élu,
        \\surgit sur la br\underline{è}che, devant lui,
        \\pour empêcher que sa fure\underline{u}r les extermine.
         
        *
         
${}^{24}Ils dédaignent une t\underline{e}rre savoureuse,
        \\ne voulant pas cr\underline{o}ire à sa parole ;
${}^{25}ils récrim\underline{i}nent sous leurs tentes
        \\sans écouter la v\underline{o}ix du Seigneur.
         
${}^{26}Dieu lève la m\underline{a}in contre eux,
        \\jurant de les p\underline{e}rdre au désert,
${}^{27}de perdre leurs descend\underline{a}nts chez les païens,
        \\de les éparpill\underline{e}r sur la terre.
         
${}^{28}Ils se donnent au Ba\underline{a}l de Péor,
        \\ils communient aux rep\underline{a}s des morts ;
${}^{29}ils irritent Dieu par to\underline{u}tes ces pratiques :
        \\un désastre s’ab\underline{a}t sur eux.
         
${}^{30}Mais Pinhas s’est lev\underline{é} en vengeur,
        \\et le dés\underline{a}stre s’arrête :
${}^{31}son action est ten\underline{u}e pour juste
        \\d’âge en \underline{â}ge et pour toujours.
         
${}^{32}Ils provoquent Dieu aux ea\underline{u}x de Mériba,
        \\ils amènent le malhe\underline{u}r sur Moïse ;
${}^{33}comme ils résist\underline{a}ient à son esprit,
        \\ses lèvres ont parl\underline{é} à la légère.
         
        *
         
${}^{34}Refusant de supprim\underline{e}r les peuples
        \\que le Seigneur leur av\underline{a}it désignés,
${}^{35}ils vont se mêl\underline{e}r aux païens,
        \\ils apprennent leur mani\underline{è}re d’agir.
         
${}^{36}Alors ils s\underline{e}rvent leurs idoles,
        \\et pour e\underline{u}x c’est un piège :
${}^{37}ils offrent leurs f\underline{i}ls et leurs filles
        \\en sacrif\underline{i}ce aux démons.
         
${}^{38}Ils versent le s\underline{a}ng innocent,
        \\le sang de leurs f\underline{i}ls et de leurs filles
        \\qu’ils sacrifient aux id\underline{o}les de Canaan,
        \\et la terre en \underline{e}st profanée.
${}^{39}De telles prat\underline{i}ques les souillent ;
        \\ils se prostituent par de t\underline{e}lles actions.
         
${}^{40}Et le Seigneur prend fe\underline{u} contre son peuple :
        \\ses hériti\underline{e}rs lui font horreur ;
${}^{41}il les livre aux m\underline{a}ins des païens :
        \\leurs ennemis devi\underline{e}nnent leurs maîtres ;
${}^{42}ils sont opprim\underline{é}s par l’adversaire :
        \\sa main s’appesant\underline{i}t sur eux.
         
${}^{43}Tant de fois délivrés par Dieu,
        ils s’obst\underline{i}nent dans leur idée, *
        \\ils s’enf\underline{o}ncent dans leur faute.
${}^{44}Et lui reg\underline{a}rde leur détresse
        \\quand il ent\underline{e}nd leurs cris.
         
${}^{45}Il se souvient de son alli\underline{a}nce avec eux ;
        \\dans son amour fid\underline{è}le, il se ravise :
${}^{46}il leur donn\underline{a} de trouver grâce
        \\devant ceux qui les ten\underline{a}ient captifs.
         
        *
         
${}^{47}Sauve-nous, Seign\underline{e}ur notre Dieu,
        \\rassemble-nous du milie\underline{u} des païens,
        \\que nous rendions gr\underline{â}ce à ton saint nom,
        \\fiers de chant\underline{e}r ta louange !
         
        *
         
${}^{48}Béni soit le Seigneur, le Die\underline{u} d’Israël,
        \\depuis toujours et pour la su\underline{i}te des temps !
        Et tout le pe\underline{u}ple dira :
        Am\underline{e}n ! Amen !
