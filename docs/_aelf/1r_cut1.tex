  
  
    
    \bbook{PREMIER LIVRE DES ROIS}{PREMIER LIVRE DES ROIS}
      
         
      \bchapter{}
      \begin{verse}
${}^{1}Le roi David était vieux, avancé en âge ; on le couvrait de vêtements, et cela ne le réchauffait pas. 
${}^{2}Ses serviteurs lui dirent : « Que l’on cherche pour mon seigneur le roi une jeune fille, une vierge. Elle se tiendra devant le roi et prendra soin de lui. Elle se couchera tout contre toi, et cela tiendra chaud à mon seigneur le roi. » 
${}^{3}On chercha une belle jeune fille dans tout le territoire d’Israël. On trouva Abishag la Sunamite, et on la fit venir chez le roi. 
${}^{4}La jeune fille était vraiment très belle ; elle prit soin du roi et fut à son service, mais le roi ne s’unit pas à elle.
      
         
${}^{5}Or, Adonias, fils de Hagguith, cherchait à s’élever. Il disait : « C’est moi qui régnerai ». Il se fit préparer un char, des cavaliers, ainsi que cinquante hommes pour courir devant lui. 
${}^{6}De sa vie, jamais son père ne l’avait attristé en disant : « Pourquoi as-tu agi ainsi ? » De plus, il avait très belle apparence. C’est lui que sa mère avait enfanté après Absalom. 
${}^{7}Il entra en pourparlers avec Joab, fils de Cerouya, et le prêtre Abiatar : tous deux se rangèrent à la suite d’Adonias. 
${}^{8}Mais ni le prêtre Sadoc, ni Benaya, fils de Joad, ni le prophète Nathan, ni Shiméï et Réhi, ni les Braves de David ne furent avec Adonias. 
${}^{9}Adonias offrit en sacrifice des moutons, des bœufs et des veaux gras près de la « Pierre-Qui-Glisse », qui se trouve à côté de la source du Foulon. Il invita tous ses frères, les fils du roi, et tous les hommes de Juda qui étaient au service du roi. 
${}^{10}Mais le prophète Nathan, Benaya, les Braves et Salomon son frère, il ne les invita pas.
${}^{11}Nathan dit à Bethsabée, la mère de Salomon : « N’as-tu pas appris qui est devenu roi ? Adonias, fils de Hagguith ! Et notre seigneur David ne le sait pas ! 
${}^{12}Maintenant, va : laisse-moi te donner un conseil, ainsi tu préserveras ta vie et celle de ton fils Salomon ! 
${}^{13}Va, entre chez le roi David et dis-lui : “N’est-ce pas toi, mon seigneur le roi, qui l’as juré à ta servante : Oui, Salomon ton fils régnera après moi et c’est lui qui s’assiéra sur mon trône ? Pourquoi donc Adonias est-il devenu roi ?” 
${}^{14}Tandis que tu seras là en train de parler avec le roi, moi-même, j’entrerai après toi et je confirmerai tes paroles. »
${}^{15}Et Bethsabée entra chez le roi, jusque dans sa chambre. Le roi était très vieux, et Abishag la Sunamite accomplissait son service auprès du roi. 
${}^{16}Bethsabée se mit à genoux et se prosterna devant le roi. Le roi lui demanda : « Que désires-tu ? » 
${}^{17}Elle lui répondit : « Mon seigneur, c’est toi qui as juré à ta servante, par le Seigneur ton Dieu : “Oui, Salomon ton fils régnera après moi : et c’est lui qui s’assiéra sur mon trône.” 
${}^{18}Or, maintenant voici qu’Adonias est devenu roi ! Cependant, mon seigneur le roi, tu n’en sais rien. 
${}^{19}Il a offert en sacrifice des taureaux, des veaux gras, des moutons en grand nombre. Il a invité tous les fils du roi, ainsi que le prêtre Abiatar, et Joab, le chef de l’armée. Mais Salomon, ton serviteur, il ne l’a pas invité. 
${}^{20}Mon seigneur le roi, c’est sur toi que tout Israël a les yeux fixés, pour que tu leur annonces qui s’assiéra sur le trône de mon seigneur le roi, après lui ! 
${}^{21}Lorsque mon seigneur le roi reposera avec ses pères, que deviendrons-nous, moi et mon fils Salomon ? Des coupables ! »
${}^{22}Tandis qu’elle parlait avec le roi, voici que Nathan, le prophète, entra. 
${}^{23}On annonça au roi : « Voici Nathan, le prophète ! » Il entra chez le roi et se prosterna devant lui, visage contre terre. 
${}^{24}Nathan prit la parole : « Mon seigneur le roi, tu as donc décrété : “Adonias régnera après moi ; c’est lui qui s’assiéra sur mon trône” ! 
${}^{25}Car il est descendu aujourd’hui, et il a offert en sacrifice des taureaux, des veaux gras, des moutons en grand nombre. Il a invité tous les fils du roi, les chefs de l’armée, et le prêtre Abiatar. Ils sont en train de boire et de manger en sa présence, et ils disent : “Vive le roi Adonias !” 
${}^{26}Mais moi, ton serviteur, le prêtre Sadoc, Benaya, fils de Joad, et Salomon ton serviteur, il ne nous a pas invités. 
${}^{27}Est-ce bien de mon seigneur le roi que provient cette affaire ? Pourtant tu n’as pas fait savoir à tes serviteurs qui doit s’asseoir sur le trône de mon seigneur le roi après lui ? »
${}^{28}Le roi David répondit alors : « Appelez-moi Bethsabée ! » Elle entra donc chez le roi et se tint debout devant lui. 
${}^{29}Et le roi fit ce serment : « Par le Seigneur qui est vivant, lui qui a racheté mon âme de toute angoisse, 
${}^{30}oui, je l’ai juré par le Seigneur Dieu d’Israël : Salomon ton fils régnera après moi, et c’est lui qui s’assiéra sur mon trône à ma place. Cela, je le ferai aujourd’hui même ! » 
${}^{31}Bethsabée se mit à genoux, visage contre terre, et se prosterna devant le roi. Elle dit : « Que vive mon seigneur le roi David à jamais ! »
${}^{32}Le roi David reprit : « Appelez-moi le prêtre Sadoc, le prophète Nathan, et Benaya, fils de Joad ». Ils entrèrent chez le roi. 
${}^{33}Et le roi leur dit : « Prenez avec vous les serviteurs de votre maître. Vous placerez mon fils Salomon sur ma propre mule, et vous le ferez descendre à Guihone. 
${}^{34}Là, le prêtre Sadoc et le prophète Nathan lui donneront l’onction comme roi sur Israël. Vous sonnerez du cor et vous direz : “Vive le roi Salomon !” 
${}^{35}Vous remonterez à sa suite, il viendra s’asseoir sur mon trône et c’est lui qui régnera à ma place. Car je l’ai établi comme chef sur Israël et sur Juda. » 
${}^{36}Benaya, fils de Joad, répondit au roi : « Amen ! Et qu’ainsi parle le Seigneur, Dieu de mon seigneur le roi ! 
${}^{37}Comme le Seigneur était avec mon seigneur le roi, qu’il soit ainsi avec Salomon ! Et qu’il élève son trône plus haut que le trône de mon seigneur le roi David. »
${}^{38}Alors descendirent le prêtre Sadoc, Benaya, fils de Joad, les Kerétiens et les Pelétiens. Ils placèrent Salomon sur la mule du roi et le conduisirent à Guihone. 
${}^{39}Le prêtre Sadoc prit dans la Tente la corne d’huile et donna l’onction à Salomon. On sonna du cor et tout le peuple dit : « Vive le roi Salomon ! » 
${}^{40}Tout le peuple remonta derrière lui. Le peuple jouait de la flûte et manifestait une joie débordante, au point que la terre se fendait à leurs voix.
${}^{41}Adonias entendit, et tous ses invités avec lui. Ils finissaient de manger. Joab entendit le son du cor et demanda : « Pourquoi la cité résonne-t-elle de ce tumulte ? » 
${}^{42}Tandis qu’il parlait, arriva Jonathan, fils du prêtre Abiatar. Adonias lui dit : « Viens, car tu es un homme de valeur : ce sont de bonnes nouvelles que tu apportes ! » 
${}^{43}Mais Jonathan répondit à Adonias : « Au contraire ! Notre seigneur le roi David a fait roi Salomon ! 
${}^{44}Le roi a envoyé avec lui le prêtre Sadoc et le prophète Nathan, Benaya, fils de Joad, les Kerétiens et les Pelétiens, et ils l’ont placé sur la mule du roi. 
${}^{45}Le prêtre Sadoc et le prophète Nathan lui ont donné l’onction comme roi à Guihone. Ils en sont remontés, tout joyeux ; la cité était en tumulte ! C’est là le bruit que vous avez entendu ! 
${}^{46}Bien plus : Salomon s’est assis sur le trône royal ! 
${}^{47}Et puis les serviteurs du roi sont venus présenter leurs vœux à notre seigneur le roi David en disant : “Que ton Dieu rende le nom de Salomon plus fameux que ton nom, et qu’il élève son trône au-dessus de ton trône !” Alors le roi s’est prosterné sur son lit. 
${}^{48}Et enfin, c’est ainsi qu’a parlé le roi : “Béni soit le Seigneur, le Dieu d’Israël, qui a donné aujourd’hui un homme pour s’asseoir sur mon trône, et qui a donné à mes yeux de le voir !” »
${}^{49}Tous les invités d’Adonias tremblèrent et se levèrent. Et ils s’enfuirent chacun de son côté. 
${}^{50}Adonias eut peur de Salomon. Il se leva et s’en fut empoigner les cornes de l’autel. 
${}^{51}Cela fut rapporté à Salomon : « Voici qu’Adonias a peur du roi Salomon. Voici qu’il est allé saisir les cornes de l’autel, en disant : “Que le roi Salomon me jure dès aujourd’hui qu’il ne fera pas mourir son serviteur par l’épée !” » 
${}^{52}Salomon répondit : « S’il devient un homme de valeur, pas un de ses cheveux ne tombera à terre. Mais si du mal se trouve en lui, il mourra ! » 
${}^{53}Le roi Salomon envoya des gens le faire descendre de l’autel. Adonias vint se prosterner devant le roi Salomon, et Salomon lui dit : « Va dans ta maison ! »
      
         
      \bchapter{}
      \begin{verse}
${}^{1}Comme les jours de David approchaient de leur fin, il exprima ses volontés à son fils Salomon : 
${}^{2} « Je m’en vais par le chemin de tout le monde. Sois fort, sois un homme courageux\\ ! 
${}^{3} Tu garderas les observances du Seigneur ton Dieu, en marchant dans ses chemins. Tu observeras ses décrets, ses commandements, ses ordonnances et ses édits, selon ce qui est écrit dans la loi de Moïse. Ainsi tu réussiras dans tout ce que tu feras et entreprendras\\, 
${}^{4} et le Seigneur réalisera cette parole qu’il m’a dite : “Si tes fils veillent à suivre leur chemin en marchant devant moi avec loyauté, de tout leur cœur et de toute leur âme, jamais tes descendants ne seront écartés du trône d’Israël.”
${}^{5}Et de plus, tu sais toi-même ce que m’a fait Joab, fils de Cerouya, ce qu’il a fait aux deux chefs des armées d’Israël, Abner, fils de Ner, et Amasa, fils de Jéther : il les a tués ; il a, en temps de paix, versé le sang de la guerre, et mis le sang de la guerre sur le ceinturon de ses reins et les sandales de ses pieds. 
${}^{6}Tu agiras selon ta sagesse et tu ne laisseras pas ses cheveux blancs descendre en paix au séjour des morts. 
${}^{7}Envers les fils de Barzillaï de Galaad, tu agiras avec fidélité. Ils seront parmi tes invités à table, car ils ont agi de la même manière quand ils sont venus à ma rencontre, alors que je fuyais devant ton frère Absalom. 
${}^{8}Mais voici près de toi Shiméï, fils de Guéra, benjaminite de Bakourim. C’est lui qui m’a maudit d’une malédiction terrible, le jour de mon départ pour Mahanaïm. C’est lui aussi qui est venu à ma rencontre au Jourdain, lors de mon retour d’exil, et je lui ai juré par le Seigneur : “Je ne te mettrai pas à mort par l’épée”. 
${}^{9}Cependant, ne le tiens pas pour quitte, car tu es un homme sage : tu sais ce que tu dois lui faire ! Tu feras descendre dans le sang ses cheveux blancs au séjour des morts ! »
${}^{10}David mourut\\, il reposa avec ses pères, et il fut enseveli dans la Cité de David. 
${}^{11} Le règne de David sur Israël avait duré quarante ans : il avait régné sept ans à Hébron, et trente-trois ans à Jérusalem. 
${}^{12} Salomon prit possession du trône de David son père, et sa royauté fut solidement établie.
${}^{13}Adonias, fils de Hagguith, vint trouver Bethsabée, mère de Salomon. Elle lui demanda : « Viens-tu pour la paix ? » Il dit : « Pour la paix ». 
${}^{14}Il poursuivit : « J’ai à te parler. » Elle dit : « Parle ». 
${}^{15}Il reprit : « Tu sais bien, toi, que c’est à moi que revenait la royauté ! Tout Israël me regardait déjà comme son roi. Mais la royauté m’a échappé au profit de mon frère, car c’est du Seigneur qu’elle lui est venue. 
${}^{16}À présent, je n’ai qu’une demande à te faire : ne me repousse pas ! » Elle dit : « Parle » 
${}^{17}Il poursuivit : « Demande, je te prie, au roi Salomon – car il ne te repoussera pas – de me donner pour femme Abishag la Sunamite. 
${}^{18}Bethsabée promit : « Bien. Je parlerai moi-même au roi en ta faveur. » 
${}^{19}Bethsabée se rendit chez le roi Salomon pour lui parler en faveur d’Adonias. Le roi se leva, vint à sa rencontre et se prosterna devant elle. Puis il prit place sur son trône. Il fit installer également un trône pour la mère du roi, et elle prit place à sa droite. 
${}^{20}Elle dit : « Je n’ai qu’une petite demande à te faire : ne me repousse pas ! » Le roi lui dit : « Demande, ma mère, je ne te repousserai pas ! » 
${}^{21}Elle reprit : « Que l’on donne pour femme Abishag la Sunamite à ton frère Adonias. »
${}^{22}Et le roi Salomon répondit à sa mère : « Pourquoi demandes-tu Abishag la Sunamite pour Adonias ? Demande donc pour lui la royauté, puisqu’aussi bien il est mon frère aîné ! Demande pour lui, pour le prêtre Abiatar et pour Joab, fils de Cerouya. » 
${}^{23}Et Salomon fit ce serment par le Seigneur : « Que Dieu amène le malheur sur moi, et pire encore ! C’est au prix de sa vie qu’Adonias a parlé. 
${}^{24}Maintenant, par le Seigneur qui est vivant, lui qui m’a fermement établi, qui m’a fait asseoir sur le trône de David, mon père, et qui, selon sa parole, m’a édifié une maison ; oui, Adonias sera mis à mort aujourd’hui même ! » 
${}^{25}Le roi Salomon envoya donc Benaya, fils de Joad ; celui-ci le frappa et il mourut. 
${}^{26}Au prêtre Abiatar, le roi déclara : « Pars pour Anatoth dans ton domaine, car tu mérites la mort ! Aujourd’hui cependant, je ne te tuerai pas, car tu as porté l’arche du Seigneur Dieu devant David, mon père, et tu as partagé toutes les épreuves que mon père a endurées. » 
${}^{27}Salomon démit Abiatar de sa fonction de prêtre du Seigneur, accomplissant ainsi la parole que le Seigneur avait dite contre la maison d’Éli, à Silo.
${}^{28}La nouvelle en parvint à Joab. Joab en effet avait pris le parti d’Adonias, bien qu’il n’eût pas pris le parti d’Absalom. Joab se réfugia dans la tente du Seigneur et empoigna les cornes de l’autel. 
${}^{29}On rapporta au roi Salomon que Joab s’était réfugié dans la tente du Seigneur et qu’il se trouvait à côté de l’autel. Salomon envoya Benaya, fils de Joad, en lui disant : « Va ! Frappe-le ! » 
${}^{30}Benaya entra dans la tente du Seigneur et dit à Joab : « Ainsi parle le roi : sors d’ici ! » Mais l’autre refusa : « Non ! C’est ici que je mourrai ! » Benaya rapporta la chose au roi : « Ainsi a parlé Joab, ainsi m’a-t-il répondu. » 
${}^{31}Le roi lui répondit : « Fais donc comme il a dit ! Frappe-le à mort et enterre-le ! Tu détourneras ainsi de moi et de la maison de mon père le sang innocent qu’a répandu Joab ! 
${}^{32}Le Seigneur fera retomber son sang sur sa tête, parce qu’il a frappé deux hommes plus justes et meilleurs que lui. Il les a tués par l’épée, à l’insu de David, mon père : Abner, fils de Ner, chef de l’armée d’Israël, et Amasa, fils de Jéther, chef de l’armée de Juda. 
${}^{33}Que leur sang retombe sur la tête de Joab et sur la tête de ses descendants, à jamais ! Mais pour David et pour sa descendance, pour sa maison et pour son trône, que demeure à jamais la paix qui vient du Seigneur ! » 
${}^{34}Benaya, fils de Joad, monta, frappa Joab et le mit à mort. Il fut enseveli dans sa maison, au désert. 
${}^{35}Le roi mit à sa place, comme chef de l’armée, Benaya, fils de Joad ; et il mit à la place d’Abiatar le prêtre Sadoc.
${}^{36}Le roi fit convoquer Shiméï et lui dit : « Construis pour toi une maison à Jérusalem ; tu y demeureras, et n’en sortiras pas pour aller où que ce soit. 
${}^{37}Le jour où tu sortiras et passeras le ravin du Cédron, sache-le bien : à coup sûr tu mourras. Et ton sang sera sur ta tête. » 
${}^{38}Shiméï répondit au roi : « Très bien ! Conformément à la parole de mon seigneur le roi, ainsi se conduira ton serviteur. » Il demeura donc de longs jours à Jérusalem. 
${}^{39}Mais, au bout de trois ans, il arriva que deux serviteurs de Shiméï s’enfuirent chez Akish, fils de Maaka, roi de Gath. On vint en informer Shiméï : « Tes serviteurs se trouvent à Gath ! » 
${}^{40}Shiméï se leva, sella son âne, partit pour Gath, chez Akish, pour rechercher ses serviteurs. Shiméï partit donc et ramena de Gath ses serviteurs. 
${}^{41}On rapporta à Salomon que Shiméï était parti de Jérusalem pour Gath et en était revenu. 
${}^{42}Le roi fit convoquer Shiméï et lui dit : « Ne t’avais-je pas fait prêter serment par le Seigneur, et ne t’avais-je pas averti : “Le jour où tu sortiras pour aller où que ce soit, sache-le bien : à coup sûr tu mourras ?” Et tu m’as répondu : “Très bien ! J’ai entendu !” 
${}^{43}Pourquoi donc n’as-tu pas respecté le serment fait au Seigneur et l’ordre que je t’avais donné ? » 
${}^{44}Le roi dit à Shiméï : « Tu sais – et ton cœur le sait – tout le mal que tu as fait à David, mon père. Le Seigneur va faire retomber sur ta tête le mal que tu as commis. 
${}^{45}Mais le roi Salomon sera béni, et le trône de David, fermement établi devant le Seigneur pour toujours ! » 
${}^{46}Le roi donna un ordre à Benaya, fils de Joad. Celui-ci sortit ; il frappa Shiméï qui mourut. Et dans la main de Salomon fut affermie la royauté.
      
         
      \bchapter{}
      \begin{verse}
${}^{1}Salomon devint le gendre de Pharaon, le roi d’Égypte ; il épousa la fille de Pharaon et la fit venir dans la Cité de David, en attendant d’avoir achevé la construction de sa propre maison, de la maison du Seigneur et du mur d’enceinte de Jérusalem. 
${}^{2}Seulement, le peuple sacrifiait toujours dans les lieux sacrés, car à cette époque on n’avait pas encore construit une maison pour le nom du Seigneur. 
${}^{3}Salomon aimait le Seigneur : il marchait selon les ordres de David, son père. Seulement, il offrait des sacrifices dans les lieux sacrés, et y brûlait de l’encens.
      
         
${}^{4}Le roi Salomon se rendit à Gabaon, qui était alors le lieu sacré le plus important, pour y offrir un sacrifice ; il immola sur l’autel un millier de bêtes en holocauste. 
${}^{5}À Gabaon, pendant la nuit, le Seigneur lui apparut en songe. Dieu lui dit : « Demande ce que je dois te donner. » 
${}^{6}Salomon répondit : « Tu as traité ton serviteur David, mon père, avec une grande fidélité, lui qui a marché en ta présence dans la loyauté, la justice et la droiture de cœur envers toi. Tu lui as gardé cette grande fidélité, tu lui as donné un fils qui est assis maintenant sur son trône. 
${}^{7}Ainsi donc, Seigneur mon Dieu, c’est toi qui m’as fait roi, moi, ton serviteur, à la place de David, mon père ; or, je suis un tout jeune homme, ne sachant comment se comporter, 
${}^{8}et me voilà\\au milieu du peuple que tu as élu ; c’est un peuple nombreux, si nombreux qu’on ne peut ni l’évaluer ni le compter. 
${}^{9}Donne à ton serviteur un cœur attentif pour qu’il sache gouverner ton peuple et discerner le bien et le mal ; sans cela, comment gouverner ton peuple, qui est si important ? » 
${}^{10}Cette demande de Salomon plut au Seigneur, qui lui dit : 
${}^{11}« Puisque c’est cela que tu as demandé, et non pas de longs jours, ni la richesse, ni la mort de tes ennemis, mais puisque tu as demandé le discernement, l’art d’être attentif et de gouverner, 
${}^{12}je fais ce que tu as demandé : je te donne un cœur intelligent et sage, tel que personne n’en a eu avant toi et que personne n’en aura après toi. 
${}^{13}De plus, je te donne même ce que tu n’as pas demandé, la richesse et la gloire, si bien que pendant toute ta vie tu n’auras pas d’égal parmi les rois. 
${}^{14}Et si tu suis mes chemins, en gardant mes décrets et mes commandements comme l’a fait David, ton père, je t’accorderai de longs jours. » 
${}^{15}Salomon s’éveilla : il avait fait un songe ! Il rentra à Jérusalem et se présenta devant l’arche de l’Alliance du Seigneur. Il offrit des holocaustes et des sacrifices de paix, et donna un festin à tous ses serviteurs.
${}^{16}Un jour, deux prostituées vinrent se présenter devant le roi. 
${}^{17} L’une des femmes dit : « De grâce, mon seigneur ! Moi et cette femme, nous habitons la même maison. Et j’ai accouché, alors qu’elle était à la maison. 
${}^{18}Or, trois jours après ma délivrance, cette femme accoucha à son tour. Nous étions ensemble : personne d’autre dans la maison ; il n’y avait que nous deux dans la maison ! 
${}^{19}Une nuit, le fils de cette femme mourut : elle s’était couchée sur lui. 
${}^{20}Elle se leva au milieu de la nuit, prit mon fils qui reposait à mon côté – ta servante dormait – et le coucha contre elle. Et son fils mort, elle le coucha contre moi. 
${}^{21}Au matin, je me levai pour allaiter mon fils : il était mort ! Je l’examinai attentivement au petit jour : ce n’était pas mon fils, celui que j’avais mis au monde. » 
${}^{22}L’autre femme protesta : « Non ! Mon fils est celui qui est vivant, ton fils celui qui est mort. » Mais la première insistait : « Pas du tout ! Ton fils est celui qui est mort, et mon fils celui qui est vivant ! » Elles se disputaient ainsi en présence du roi. 
${}^{23}Le roi dit alors : « Celle-ci affirme : Mon fils, c’est le vivant, et ton fils est le mort. Celle-là affirme : Non ! Ton fils, c’est le mort, et mon fils est le vivant ! » 
${}^{24}Et le roi ajouta : « Donnez-moi une épée ! » On apporta une épée devant le roi. 
${}^{25}Et le roi poursuivit : « Coupez en deux l’enfant vivant, donnez-en la moitié à l’une et la moitié à l’autre. » 
${}^{26}Mais la femme dont le fils était vivant s’adressa au roi – car ses entrailles s’étaient émues à cause de son fils ! – : « De grâce, mon seigneur ! Donnez-lui l’enfant vivant, ne le tuez pas ! » L’autre protestait : « Il ne sera ni à toi ni à moi : coupez-le ! » 
${}^{27}Prenant la parole, le roi déclara : « Donnez à celle-ci l’enfant vivant, ne le tuez pas : c’est elle, sa mère ! » 
${}^{28}Tout Israël apprit le jugement qu’avait rendu le roi. Et l’on regarda le roi avec crainte et respect, car on avait vu que, pour rendre la justice, la sagesse de Dieu était en lui.
      
         
      \bchapter{}
      \begin{verse}
${}^{1}Le roi Salomon régnait sur tout Israël. 
${}^{2}Voici les chefs qu’il avait à son service :
      <p class="retrait1">Azarias, fils de Sadoc, le prêtre ;
      <p class="retrait1char">
${}^{3}Élihoreph et Ahias, fils de Shisha : secrétaires ;
      <p class="retrait1">Josaphat, fils d’Ahiloud : archiviste ;
      <p class="retrait1char">
${}^{4}Benaya, fils de Joad : chef de l’armée ;
      <p class="retrait1">Sadoc et Abiatar : prêtres ;
      <p class="retrait1char">
${}^{5}Azarias, fils de Nathan : chef des préfets ;
      <p class="retrait1">Zaboud, fils de Nathan : prêtre et compagnon du roi ;
      <p class="retrait1char">
${}^{6}Ahishar : maître du palais ;
      <p class="retrait1">Adoniram, fils d’Abda : chef de la corvée.
      <p class="retrait1char">
${}^{7}De plus, Salomon avait douze préfets pour l’ensemble d’Israël. Ils approvisionnaient le roi et sa maison ; un mois par an, chacun son tour, ils assuraient l’approvisionnement.
      <p class="retrait1char">
${}^{8}Voici leurs noms :
      <p class="retrait1">Ben-Hour, dans la montagne d’Éphraïm ;
      <p class="retrait1char">
${}^{9}Ben-Déqer, à Macas, à Shaalvime, à Beth-Shèmesh et à Élone-Beth-Hanane ;
${}^{10}Ben-Hésed, à Aroubboth ; il avait la responsabilité de Soko et de tout le pays de Héfer.
${}^{11}Ben-Abinadab, celle des coteaux de Dor ; il eut pour femme Tafath, fille de Salomon.
${}^{12}Baana, fils d’Ahiloud, avait Taanak et Meguiddo, ainsi que tout Beth-Shéane, à côté de Sartane, en dessous d’Izréel, depuis Beth-Shéane jusqu’à Abel-Mehola, au-delà de Yoqméam.
${}^{13}Ben-Guèber, à Ramoth-de-Galaad, avait les campements de Yaïr, fils de Manassé, en Galaad ; il avait aussi la région d’Argob, dans le Bashane ; soixante grandes villes avec remparts et verrous de bronze.
${}^{14}Ahinadab, fils de Iddo, à Mahanaïm ;
${}^{15}Ahimaas, en Nephtali. Lui aussi prit pour femme une fille de Salomon, Basmath.
${}^{16}Baana, fils de Houshaï, sur la contrée d’Asher et à Bealoth ;
${}^{17}Josaphat, fils de Paroua, sur Issakar ;
${}^{18}Shiméï, fils d’Éla, sur Benjamin ;
${}^{19}Guéber, fils d’Ouri, sur le pays de Galaad, pays de Séhone, roi des Amorites, et d’Og, roi de Bashane ;
      <p class="retrait1">Et il y avait aussi un préfet dans le pays de Juda.
${}^{20}Juda et Israël étaient nombreux, aussi nombreux que le sable au bord de la mer. On mangeait, on buvait et on était dans la joie.
      
         
      \bchapter{}
      \begin{verse}
${}^{1}Salomon dominait sur tous les royaumes, depuis l’Euphrate – le fleuve –, sur le pays des Philistins, et jusqu’à la frontière de l’Égypte. Ils acquittèrent un tribut et servirent Salomon tous les jours de sa vie. 
${}^{2}Les vivres de Salomon comprenaient, pour chaque jour, trente quintaux de semoule et soixante quintaux de farine, 
${}^{3}dix bœufs gras, vingt bœufs de pâturage, cent moutons ; de plus, des cerfs, des gazelles, des chevreuils et des volailles engraissées. 
${}^{4}Car il était maître des régions en deçà du fleuve Euphrate, depuis Tifsa jusqu’à Gaza : maître de tous les rois des régions en deçà du Fleuve. Et il avait la paix sur toutes ses frontières alentour. 
${}^{5}Juda et Israël habitèrent en sécurité, chacun sous sa vigne et sous son figuier, de Dane jusqu’à Bershéba, durant toute la vie de Salomon. 
${}^{6}Salomon avait quarante mille stalles pour les chevaux de ses chars, et douze mille chevaux de selle.
${}^{7}Les préfets mentionnés plus haut approvisionnaient le roi Salomon et ceux qui partageaient la table du roi, un mois chacun, à tour de rôle : ils ne les laissaient manquer de rien. 
${}^{8}Quant à l’orge et au fourrage pour les chevaux et les attelages, chacun les apportait, sur ordre, là où séjournait le roi.
${}^{9}Dieu donna à Salomon une sagesse et une intelligence très grandes, et un cœur aussi vaste que le sable au bord de la mer.
${}^{10}Grande était la sagesse de Salomon,
        \\plus que la sagesse de tous les fils de l’Orient,
        \\plus que toute la sagesse de l’Égypte.
${}^{11}Il fut le plus sage des hommes,
        \\plus encore qu’Étane l’Ézrahite,
        \\et que Hémane, Kalkol et Darda, les fils de Mahol :
        \\son nom était connu de toutes les nations d’alentour.
${}^{12}Il prononça trois mille proverbes
        \\et composa des chants au nombre de mille cinq.
${}^{13}Il parla des arbres, depuis le cèdre du Liban
        \\jusqu’à l’hysope qui pousse sur le mur ;
        \\il parla des quadrupèdes et des oiseaux,
        \\des reptiles et des poissons.
${}^{14}Et l’on venait de tous les peuples
        \\pour entendre la sagesse de Salomon,
        \\on venait de la part de tous les rois de la terre
        \\qui avaient entendu parler de sa sagesse.
${}^{15}Hiram, roi de Tyr, envoya des serviteurs auprès de Salomon, car il avait appris qu’on lui avait donné l’onction comme roi à la place de son père. En effet, Hiram avait toujours été l’ami de David. 
${}^{16}Salomon envoya ce message à Hiram :
${}^{17}« Tu sais que David, mon père, harcelé par les guerres, n’a pas pu bâtir une maison pour le nom du Seigneur son Dieu, jusqu’à ce que le Seigneur eût mis sous ses pieds les ennemis qui l’encerclaient. 
${}^{18} Mais à présent, le Seigneur mon Dieu m’a donné le repos de tous côtés ; je n’ai plus d’opposants ni de dangers à craindre. 
${}^{19} Ainsi, j’ai décidé de bâtir une maison pour le nom du Seigneur mon Dieu, selon la parole du Seigneur à David, mon père :
        \\“Ton fils, celui que je mettrai après toi sur ton trône,
        \\c’est lui qui construira la Maison pour mon nom !”
${}^{20}Maintenant donc, ordonne que l’on coupe pour moi des cèdres du Liban. Mes serviteurs travailleront avec les tiens, et je te donnerai pour leur salaire ce que tu me fixeras ; car tu sais qu’il n’y a personne chez nous qui sache couper les arbres comme les gens de Sidon. » 
${}^{21} Dès qu’il eut entendu le message de Salomon, Hiram se réjouit fort, et il dit : « Béni soit aujourd’hui le Seigneur, qui a donné à David un fils d’une telle sagesse, pour gouverner ce grand peuple ! » 
${}^{22} Puis Hiram envoya dire à Salomon : « J’ai reçu ton message. Je te donnerai du bois de cèdre et du bois de cyprès comme tu le désires. 
${}^{23} Mes serviteurs les feront descendre du Liban à la mer ; et moi, je les assemblerai en radeaux sur la mer à destination de l’endroit que tu m’indiqueras. Là, je les ferai démonter et tu en prendras livraison. De ton côté, tu me donneras des vivres pour ma maison comme je le désire. » 
${}^{24} Ainsi Hiram livrait à Salomon, en bois de cèdre et en bois de cyprès, tout ce qu’il désirait. 
${}^{25} Et Salomon livrait à Hiram vingt mille quintaux\\de blé pour la nourriture de sa maison, plus vingt quintaux\\d’huile d’olives concassées : voilà ce que Salomon livrait à Hiram tous les ans. 
${}^{26} Le Seigneur avait donné la sagesse à Salomon, selon sa promesse. Hiram et Salomon vécurent en paix et ils conclurent une alliance.
${}^{27}Salomon recruta dans tout Israël trente mille hommes de corvée\\. 
${}^{28} Il les envoya au Liban : dix mille à tour de rôle ; ils restaient un mois au Liban, puis deux mois chez eux. Adoniram était chef de la corvée. 
${}^{29} Salomon avait aussi soixante-dix mille porteurs et quatre-vingt mille ouvriers pour les carrières dans la montagne. 
${}^{30} Il y avait encore les chefs, nommés par les préfets pour diriger le travail, au nombre de trois mille trois cents, qui commandaient aux ouvriers. 
${}^{31} Le roi ordonna d’extraire de grands blocs\\, de belle pierre, pour poser en pierres de taille les fondations de la Maison. 
${}^{32} Puis les maçons de Salomon et ceux d’Hiram, ainsi que les gens de Byblos, taillèrent et préparèrent le bois et les pierres pour construire la Maison.
      
         
      \bchapter{}
      \begin{verse}
${}^{1}Quatre cent quatre-vingts ans après la sortie des fils d’Israël du pays d’Égypte, la quatrième année du règne de Salomon sur Israël, au mois de Ziv – qui est le deuxième mois –, il construisit la Maison pour le Seigneur. 
${}^{2}Et la Maison que le roi Salomon construisit pour le Seigneur avait soixante coudées de long, vingt coudées de large et trente coudées de haut. 
${}^{3}Devant la Grande Salle de la Maison, le vestibule était long de vingt coudées dans le sens de la largeur de la Maison, et prolongeait la Maison de dix coudées. 
${}^{4}Il fit à la Maison des fenêtres à embrasures grillagées. 
${}^{5}Il bâtit contre le mur de la Maison un bas-côté tout autour ; ce bas-côté longeait les murs de la Maison, c’est-à-dire de la Grande Salle et du Saint des Saints ; et, dans ce bas-côté, il fit des étages. 
${}^{6}Le bas-côté inférieur avait cinq coudées de large, celui du milieu six coudées, et le troisième étage sept coudées de large. Car Salomon avait ménagé des retraits progressifs tout autour de la Maison, à l’extérieur, pour ne pas entamer les murs de la Maison. 
${}^{7}En effet, la Maison, quand elle fut bâtie, l’avait été en pierres intactes, sorties de la carrière. Tout le temps que dura la construction, on n’entendit dans la Maison ni marteau, ni pic, ni aucun outil de fer. 
${}^{8}Dans le bas-côté, l’entrée, à l’étage du milieu, était sur le flanc droit de la Maison. On y accédait par des escaliers en spirales, et, de là, au troisième étage. 
${}^{9}Salomon construisit la Maison et l’acheva. Il revêtit la Maison d’un lambris en panneaux de cèdre, ornés de moulures. 
${}^{10}Contre toute la Maison, il construisit le bas-côté, qui avait une hauteur de cinq coudées et tenait à la Maison par des poutres de cèdre.
${}^{11}La parole du Seigneur fut adressée à Salomon : 
${}^{12}« Cette Maison que tu es en train de construire, j’y demeurerai si tu marches selon mes décrets, si tu respectes mes ordonnances, si tu gardes tous mes commandements, en y conformant ta conduite ; alors, je maintiendrai pour toi la parole que j’ai dite à David ton père : 
${}^{13}“Je demeurerai au milieu des fils d’Israël, je n’abandonnerai pas mon peuple Israël”. »
${}^{14}Salomon construisit la Maison et l’acheva. 
${}^{15}Il doubla les murs de la Maison, à l’intérieur, avec des planches de cèdre, depuis le sol de la Maison jusqu’en haut des murs. Le plafond, il le couvrit de bois, à l’intérieur, et il recouvrit le sol de la Maison d’un plancher de cyprès. 
${}^{16}Il aménagea, en partant du fond de la Maison, un espace de vingt coudées, en planches de cèdre, depuis le sol jusqu’en haut des murs. Il s’aménagea cet espace intérieur pour en faire la Chambre sainte, le Saint des Saints. 
${}^{17}La Grande Salle devant le Saint des Saints était de quarante coudées. 
${}^{18}Le cèdre destiné à l’intérieur de la Maison était sculpté en forme de coloquintes et de fleurs épanouies. Tout était de cèdre : la pierre n’apparaissait nulle part. 
${}^{19}Et le Saint des Saints, au milieu de la Maison, à l’intérieur, Salomon l’établit pour qu’on y dépose l’arche de l’Alliance du Seigneur. 
${}^{20}Devant le Saint des Saints, qui avait vingt coudées de long, vingt coudées de large et vingt coudées de haut, et qu’il recouvrit d’or fin, il fit un autel revêtu de cèdre. 
${}^{21}Salomon recouvrit d’or fin l’intérieur de la Maison. Il fit passer des chaînes d’or devant le Saint des Saints qu’il recouvrit d’or. 
${}^{22}Et c’est toute la Maison qu’il recouvrit d’or, la Maison tout entière ; et tout l’autel du Saint des Saints, il le recouvrit d’or. 
${}^{23}Dans le Saint des Saints, il fit deux kéroubim en bois d’olivier : leur hauteur était de dix coudées. 
${}^{24}Une aile de kéroub avait cinq coudées, et l’autre aile cinq coudées aussi, – dix coudées de l’extrémité d’une aile à l’extrémité de l’autre. 
${}^{25}Le second kéroub avait lui aussi dix coudées. Même mesure et même forme pour les deux kéroubim. 
${}^{26}La hauteur du premier kéroub était de dix coudées ; de même pour le second kéroub. 
${}^{27}Il plaça les kéroubim au milieu de la Maison, à l’intérieur. Les kéroubim déployaient leurs ailes : l’aile du premier kéroub touchait l’un des murs, et l’aile de l’autre kéroub touchait l’autre mur ; et au milieu de la Maison, leurs ailes se touchaient, aile contre aile. 
${}^{28}Il recouvrit d’or les kéroubim.
${}^{29}Sur les murs de la Maison, tout autour, il sculpta des figures : sculptures de kéroubim, de palmiers et de fleurs épanouies, sur le devant et à l’extérieur. 
${}^{30}Le sol de la Maison, il le recouvrit d’or à l’intérieur et à l’extérieur. 
${}^{31}Pour l’entrée du Saint des Saints, il fit des portes en bois d’olivier – le pilier des montants avait cinq côtés –, 
${}^{32}deux portes en bois d’olivier, sur lesquelles il sculpta des kéroubim, des palmiers et des fleurs épanouies, qu’il recouvrit d’or. De haut en bas, sur les kéroubim et les palmiers, il étendit l’or. 
${}^{33}Il fit de même pour l’entrée de la Grande Salle – des montants en bois d’olivier occupant le quart de l’ensemble. 
${}^{34}Les deux portes étaient en bois de cyprès : une porte avait deux panneaux pivotants, et l’autre, deux courtines pivotantes. 
${}^{35}Il sculpta des kéroubim, des palmiers et des fleurs épanouies, qu’il recouvrit d’or ajusté sur ce qui était modelé. 
${}^{36}Il construisit aussi la cour intérieure : trois rangs de pierres de taille et un rang de madriers de cèdre.
${}^{37}En la quatrième année, au mois de Ziv, furent posées les fondations de la Maison. 
${}^{38}En la onzième année, au mois de Boul – qui est le huitième mois – Salomon acheva la Maison, conformément à tous les plans et toutes les ordonnances. Il la construisit donc en sept ans.
      
         
      \bchapter{}
      \begin{verse}
${}^{1}Quant à sa maison, Salomon mit treize ans pour la construire et l’achever entièrement. 
${}^{2}Il construisit la maison de la Forêt du Liban : cent coudées de long, cinquante coudées de large et trente coudées de haut, sur quatre rangées de colonnes de cèdre surmontées de madriers de cèdre. 
${}^{3}Le plafond de cèdre reposait sur les traverses reliant les colonnes : quarante-cinq traverses, quinze par colonnade. 
${}^{4}Il y avait aussi trois rangées d’embrasures ; chaque baie faisait face à une autre baie, par groupe de trois. 
${}^{5}Toutes les ouvertures et les montants avaient une embrasure carrée ; et le devant de chaque baie faisait face à une autre baie, par groupe de trois. 
${}^{6}Il fit le vestibule des colonnes : cinquante coudées de long et trente coudées de large, et, par-devant, un autre vestibule, des colonnes et, par-devant encore, un auvent. 
${}^{7}Il fit le vestibule du Trône, là où il rendait la justice, le vestibule du Jugement, recouvert de cèdre, d’un bout à l’autre du sol. 
${}^{8}Sa maison, là où il demeurait, était dans une autre cour que celle de la maison du Vestibule, mais elle était de même structure. Pour la fille de Pharaon, que Salomon avait épousée, il fit une maison semblable à ce vestibule. 
${}^{9}Tout cela était en pierres de choix, aux dimensions des pierres de taille, et découpées à la scie, à l’intérieur comme à l’extérieur, des fondations jusqu’aux corniches, et de l’extérieur jusqu’à la grande cour. 
${}^{10}Les fondations étaient en pierres de choix, de grandes pierres : des pierres de dix et de huit coudées. 
${}^{11}Par-dessus les fondations, des pierres de choix, aux dimensions des pierres de taille, et du cèdre. 
${}^{12}Le pourtour de la grande cour était de trois rangées de pierres taillées et d’une rangée de madriers de cèdre, tout comme la cour intérieure de la maison du Seigneur et le Vestibule de la Maison.
      
         
${}^{13}Le roi Salomon envoya chercher Hiram de Tyr. 
${}^{14}Fils d’une veuve de la tribu de Nephtali, et d’un homme de Tyr, artisan en bronze, il était rempli de sagesse, d’intelligence et de connaissance pour faire tout travail du bronze. 
${}^{15}Il moula les deux colonnes de bronze ; la hauteur d’une colonne était de dix-huit coudées. Un fil de douze coudées en aurait fait le tour ; de même pour la seconde colonne. 
${}^{16}Il fit deux chapiteaux de bronze fondu qu’il posa au sommet des colonnes ; la hauteur d’un chapiteau était de cinq coudées, la hauteur de l’autre était aussi de cinq coudées. 
${}^{17}Des filets – une décoration en forme de filets – et des entrelacs – une décoration en forme de chaînettes – ornaient les chapiteaux qui coiffaient les colonnes, sept pour un chapiteau et sept pour l’autre. 
${}^{18}Il fit aussi des grenades : deux rangées, tout autour, sur chaque filet, pour habiller les chapiteaux, au sommet des colonnes ; ainsi fit-il pour les deux chapiteaux. 
${}^{19}Les chapiteaux qui coiffaient les colonnes du Vestibule étaient en forme de lis de quatre coudées. 
${}^{20}Les chapiteaux sur les deux colonnes se trouvaient directement au-dessus du renflement qui dépassait le filet. Les grenades, au nombre de deux cents, étaient disposées en rangées tout autour, sur les deux chapiteaux. 
${}^{21}Il dressa ces colonnes devant le Vestibule de la Grande Salle. Il dressa la colonne de droite et lui donna le nom de Yakine (ce qui signifie : « Il rend stable ») ; il dressa la colonne de gauche et lui donna le nom de Boaz (ce qui signifie : « En lui la force »). 
${}^{22}Sur le sommet des colonnes se trouvait l’ouvrage en forme de lis. Ainsi fut terminé le travail des colonnes.
${}^{23}Il fit la Mer, bassin en métal fondu, de dix coudées de diamètre, car son pourtour était circulaire. Elle avait cinq coudées de haut. Un cordeau de trente coudées en aurait fait le tour. 
${}^{24}En dessous du bord, des coloquintes, tout autour, dix par coudées, encerclaient la Mer. Les coloquintes étaient disposées sur deux rangées, fondues ensemble avec la Mer. 
${}^{25}La Mer était dressée sur douze bœufs : trois faisaient face au nord, trois faisaient face à l’ouest, trois faisaient face au sud, trois faisaient face à l’est. La Mer reposait directement dessus, leurs arrière-trains tournés vers l’intérieur. 
${}^{26}L’épaisseur de la Mer était d’une largeur de paume, son rebord était comme le bord d’une coupe, en forme de fleur de lis. Sa contenance était de deux mille mesures.
${}^{27}Il fit également dix bases de bronze. Chaque base avait quatre coudées de long, quatre coudées de large et trois coudées de haut. 
${}^{28}Voici comment était faite une base : elle avait des panneaux, des panneaux encadrés de bordures. 
${}^{29}Sur les panneaux encadrés de bordures se trouvaient des lions, des bœufs et des kéroubim ; de même sur les bordures supérieures. Au-dessous des lions et des bœufs, des volutes descendaient. 
${}^{30}Chaque base avait quatre roues de bronze et des essieux de bronze. Les quatre pieds, sous la cuve, étaient montés sur des équerres : celles-ci étaient fondues parallèlement aux volutes. 
${}^{31}L’ouverture de chaque base était dans la partie supérieure qu’elle dépassait d’une coudée. L’ouverture était arrondie, en forme de support, d’une coudée et demie. De plus, sur cette ouverture, il y avait des sculptures ; les panneaux en étaient carrés, et non pas arrondis. 
${}^{32}Les quatre roues se trouvaient sous les panneaux de chaque base. Les axes des roues étaient fixés sur la base. La hauteur d’une roue était d’une coudée et demie. 
${}^{33}Les roues étaient comme des roues de chars : leurs axes, leurs jantes, leurs rayons, leurs moyeux, tout était en métal fondu. 
${}^{34}Il y avait quatre équerres, aux quatre coins de la base, et les équerres faisaient corps avec la base. 
${}^{35}Sur le dessus de la base, à une demi-coudée de haut, un cercle en faisait le tour. Près du dessus de la base, les axes et les panneaux faisaient corps avec elle. 
${}^{36}Sur les plaques portant les axes et sur les panneaux, Hiram grava des kéroubim, des lions, des palmiers, à la mesure de l’espace disponible sur chacun, ainsi que des volutes tout autour. 
${}^{37}Ainsi fabriqua-t-il les dix bases : même métal fondu, mêmes mesures, même format pour toutes.
${}^{38}Il fabriqua dix cuves de bronze ; chacune contenait quarante mesures, chacune mesurait quatre coudées, et chacune était posée sur une des dix bases. 
${}^{39}Il disposa ainsi les bases : cinq à droite de la Maison, et cinq à gauche de la Maison. Quant à la Mer, il la plaça du côté droit de la Maison, au sud-est. 
${}^{40}Puis Hiram fit les cuves, les pelles et les bols pour l’aspersion. Hiram acheva tout le travail qu’il avait à faire pour le roi Salomon dans la maison du Seigneur : 
${}^{41}les deux colonnes, les arrondis des chapiteaux sur le sommet des deux colonnes ; les deux filets pour couvrir les deux arrondis des chapiteaux sur le sommet des colonnes ; 
${}^{42}les quatre cents grenades destinées aux deux filets – deux rangées de grenades par filet –, de façon à couvrir les deux arrondis des chapiteaux, jusque sur le flanc des colonnes ; 
${}^{43}les dix bases, les dix cuves sur les bases ; 
${}^{44}la Mer – une seule –, avec les douze bœufs en dessous de la Mer ; 
${}^{45}les vases, les pelles, les bols pour l’aspersion, et tous les objets que Hiram avait fabriqués en bronze poli, pour le roi Salomon dans la maison du Seigneur. 
${}^{46}C’est dans la région du Jourdain, entre Souccoth et Sartane, que le roi les fit mouler dans des couches d’argile.
${}^{47}Salomon mit en place tous ces objets. Il y en avait tant et tant que l’on ne pouvait estimer le poids du bronze employé. 
${}^{48}Salomon fit tous les objets de la maison du Seigneur : l’autel d’or, et la table d’or pour le pain de l’offrande ; 
${}^{49}les chandeliers d’or fin, cinq à droite et cinq à gauche, devant le Saint des Saints ; les fleurs, les lampes, les pincettes en or, 
${}^{50}les récipients, les ciseaux, les bols pour l’aspersion, les gobelets, les brûle-parfums, en or fin ; les gonds en or des portes de la Maison intérieure – le Saint des Saints –, et des portes de la Maison – la Grande Salle. 
${}^{51}Ainsi fut parachevé tout le travail entrepris par le roi Salomon dans la maison du Seigneur. Salomon fit apporter les objets sacrés de David son père : l’argent, l’or et les ustensiles ; il les déposa dans les trésors de la maison du Seigneur.
      
         
      \bchapter{}
      \begin{verse}
${}^{1}Salomon rassembla auprès de lui à Jérusalem les anciens d’Israël et tous les chefs des tribus, les chefs de famille des fils d’Israël, pour aller chercher\\l’arche de l’Alliance du Seigneur dans la Cité de David, c’est-à-dire à Sion. 
${}^{2}Tous les hommes d’Israël se rassemblèrent auprès du roi Salomon au septième mois\\, durant la fête des Tentes\\. 
${}^{3}Quand tous les anciens d’Israël furent arrivés, les prêtres se chargèrent de l’Arche. 
${}^{4}Ils emportèrent l’arche du Seigneur et la tente de la Rencontre avec tous les objets sacrés qui s’y trouvaient ; ce sont les prêtres et les Lévites qui les transportèrent. 
${}^{5}Le roi Salomon et, avec lui, toute la communauté d’Israël qu’il avait convoquée auprès de lui devant l’Arche offrirent en sacrifice des moutons et des bœufs : il y en avait un si grand nombre qu’on ne pouvait ni le compter ni l’évaluer. 
${}^{6}Puis les prêtres transportèrent l’Arche à sa place, dans la Chambre sainte\\que l’on appelle le Saint des Saints, sous les ailes des kéroubim. 
${}^{7}Ceux-ci, en effet, étendaient leurs ailes au-dessus de l’emplacement de l’Arche : ils protégeaient l’Arche et ses barres. 
${}^{8}Les barres étaient si longues que l’on pouvait voir leurs extrémités depuis le sanctuaire, devant la Chambre sainte ; mais on ne les voyait pas de l’extérieur. Elles y sont encore à ce jour. 
${}^{9}Dans l’Arche, il n’y avait rien, sinon les deux tables de pierre que Moïse y avait placées au mont Horeb, quand le Seigneur avait conclu alliance avec les fils d’Israël, à leur sortie du pays d’Égypte. 
${}^{10}Quand les prêtres sortirent du sanctuaire, la nuée remplit la maison du Seigneur, 
${}^{11}et, à cause d’elle, les prêtres durent interrompre le service divin : la gloire du Seigneur remplissait la maison du Seigneur !
${}^{12}Alors Salomon s’écria :
        \\« Le Seigneur déclare demeurer dans la nuée obscure.
        ${}^{13}Et maintenant, je t’ai construit, Seigneur\\,
        \\une maison somptueuse,
        \\un lieu où tu habiteras éternellement. »
${}^{14}Puis le roi se retourna et bénit toute l’assemblée d’Israël ; or toute l’assemblée d’Israël se tenait debout. 
${}^{15}Il dit : « Béni soit le Seigneur, le Dieu d’Israël ! De sa bouche, il a parlé à David mon père et, de sa main, il a accompli ce qu’il avait dit : 
${}^{16}“Depuis le jour où j’ai fait sortir d’Égypte mon peuple Israël, je n’ai choisi aucune ville entre toutes les tribus d’Israël pour y construire une maison où serait mon nom. Mais j’ai choisi David pour qu’il soit le chef de mon peuple Israël.” 
${}^{17}Or David, mon père, avait pris à cœur de construire une maison pour le nom du Seigneur, le Dieu d’Israël. 
${}^{18}Mais le Seigneur a dit à David, mon père : “Tu as pris à cœur de construire une maison pour mon nom, et tu as bien fait de prendre cela à cœur. 
${}^{19}Cependant, ce n’est pas toi qui construiras la maison, mais ton fils, issu de toi : c’est lui qui construira la maison pour mon nom.” 
${}^{20}Le Seigneur a réalisé la parole qu’il avait dite, et j’ai succédé à David, mon père, je me suis assis sur le trône d’Israël, comme l’avait dit le Seigneur, et j’ai construit la maison pour le nom du Seigneur, le Dieu d’Israël. 
${}^{21}Là j’ai fixé un emplacement pour l’Arche où se trouve l’Alliance du Seigneur, l’Alliance qu’il a conclue avec nos pères lorsqu’il les fit sortir du pays d’Égypte. »
${}^{22}Salomon se plaça devant l’autel du Seigneur, en face de toute l’assemblée d’Israël ; il étendit les mains vers le ciel 
${}^{23}et fit cette prière\\ : « Seigneur, Dieu d’Israël, il n’y a pas de Dieu comme toi, ni là-haut dans les cieux, ni sur la terre ici-bas ; car tu gardes ton Alliance et ta fidélité envers\\tes serviteurs, quand ils marchent devant toi de tout leur cœur. 
${}^{24}Tu as gardé pour ton serviteur David, mon père, ce que tu lui avais dit ; et ce que tu lui avais dit de ta bouche, aujourd’hui tu l’as accompli de ta main. 
${}^{25}Et maintenant, Seigneur, Dieu d’Israël, par égard pour ton serviteur David, mon père, garde la parole que tu lui avais dite : “Tes descendants qui siégeront sur le trône d’Israël ne seront pas écartés de ma présence, pourvu que tes fils veillent à suivre leur chemin en marchant devant moi, comme tu as marché devant moi.” 
${}^{26}Maintenant donc, Dieu d’Israël, que se vérifie la parole que tu as dite à ton serviteur David, mon père ! 
${}^{27}Est-ce que, vraiment\\, Dieu habiterait sur la terre ? Les cieux et les hauteurs des cieux ne peuvent te contenir : encore moins cette Maison que j’ai bâtie ! 
${}^{28}Sois attentif à la prière et à la supplication de ton serviteur. Écoute, Seigneur mon Dieu, la prière et le cri qu’il lance aujourd’hui vers toi. 
${}^{29}Que tes yeux soient ouverts nuit et jour sur cette Maison, sur ce lieu dont tu as dit : “C’est ici que sera mon nom.” Écoute donc la prière que ton serviteur fera en ce lieu\\. 
${}^{30}Écoute la supplication de ton serviteur et de ton peuple Israël, lorsqu’ils prieront en ce lieu\\. Toi, dans les cieux où tu habites, écoute et pardonne.
${}^{31}Lorsqu’un homme aura péché contre son prochain et qu’on lui imposera un serment qui peut se retourner contre lui, s’il vient à prêter ce serment devant ton autel dans cette Maison, 
${}^{32}toi, dans les cieux, écoute, agis et juge tes serviteurs. Déclare coupable le coupable : que sa conduite retombe sur sa tête ; déclare juste le juste : traite-le selon sa justice.
${}^{33}Lorsque ton peuple Israël aura été battu devant l’ennemi, pour avoir péché contre toi, s’il revient à toi et célèbre ton nom, s’il prie et te supplie dans cette Maison, 
${}^{34}toi, dans les cieux, écoute, pardonne le péché de ton peuple Israël et fais-les revenir sur le sol que tu as donné à leurs pères.
${}^{35}Lorsque les cieux seront fermés et qu’il n’y aura pas de pluie, parce que les fils d’Israël auront péché contre toi, s’ils prient vers ce lieu et célèbrent ton nom, s’ils se détournent de leur péché, parce que tu les auras humiliés, 
${}^{36}toi, dans les cieux, écoute, pardonne le péché de tes serviteurs et de ton peuple Israël. Tu leur enseigneras le bon chemin par où ils doivent marcher, et tu accorderas la pluie à ta terre, celle que tu as donnée à ton peuple en héritage.
${}^{37}Lorsqu’il y aura la famine dans le pays, lorsqu’il y aura la peste, la rouille et la nigelle du blé, les sauterelles et les criquets, lorsque son ennemi assiégera une ville dans le pays, en tout fléau, en toute maladie, 
${}^{38}quel que soit le motif de la prière ou de la supplication émanant de tout homme ou de tout ton peuple Israël, dès l’instant où chacun reconnaît la plaie de son cœur et qu’il tend les mains vers cette Maison, 
${}^{39}toi, dans les cieux où tu habites, écoute, pardonne et agis. Traite chacun selon toute sa conduite, puisque tu connais son cœur – toi seul, en effet, connais le cœur de tout homme –, 
${}^{40}afin qu’ils te craignent, tous les jours qu’ils vivront devant toi sur le sol que tu as donné à nos pères. 
${}^{42}On entendra parler de ton grand nom, de ta main forte et de ton bras étendu. 
${}^{41}Si donc, à cause de ton nom, un étranger, qui n’est pas de ton peuple Israël, vient d’un pays lointain 
${}^{42}prier dans\\cette Maison, 
${}^{43}toi, dans les cieux où tu habites, écoute-le. Exauce toutes les demandes de l’étranger. Ainsi, tous les peuples de la terre, comme ton peuple Israël, vont reconnaître ton nom et te craindre. Et ils sauront que ton nom est invoqué sur cette Maison que j’ai bâtie.
${}^{44}Lorsque ton peuple partira en guerre contre ses ennemis, dans la direction où tu l’auras envoyé, et qu’il priera le Seigneur, tourné vers la Ville que tu as choisie et vers la Maison que j’ai bâtie pour ton nom, 
${}^{45}toi, dans les cieux, écoute leur prière et leur supplication, et rends-leur justice.
${}^{46}Lorsqu’ils pécheront contre toi – car il n’est pas d’être humain qui ne commette quelque péché – et que tu seras irrité contre eux, alors tu les livreras à la merci de leurs ennemis, et leurs vainqueurs les emmèneront captifs dans un pays ennemi, lointain ou proche. 
${}^{47}Si, au pays où ils auront été emmenés captifs, ils rentrent en eux-mêmes, s’ils se repentent, s’ils élèvent vers toi leur supplication dans le pays de ceux qui les ont faits prisonniers, en disant : “Nous avons péché, nous avons commis une faute, nous avons fait ce qui est mal” ; 
${}^{48}s’ils reviennent à toi de tout leur cœur et de toute leur âme, au pays de leurs ennemis qui les auront emmenés captifs, et s’ils prient vers toi, tournés vers le pays que tu as donné à leurs pères, vers la Ville que tu as choisie et vers la Maison que j’ai bâtie pour ton nom, 
${}^{49}toi, dans les cieux où tu habites, écoute leur prière et leur supplication, et rends-leur justice. 
${}^{50}Pardonne à ton peuple qui a péché contre toi, toutes les rebellions dont il s’est rendu coupable envers toi ; fais-le prendre en pitié par ses vainqueurs, que ceux-ci les prennent en pitié. 
${}^{51}Parce qu’ils sont ton peuple et ton héritage, eux que tu as fait sortir d’Égypte, de cette fournaise à fondre le fer, 
${}^{52}tes yeux sont ouverts à la supplication de ton serviteur et à la supplication de ton peuple Israël, et tu les écoutes toutes les fois qu’ils crient vers toi. 
${}^{53}Car c’est toi qui les as séparés de tous les peuples de la terre, pour qu’ils deviennent ton héritage, comme tu l’as dit par l’intermédiaire de Moïse ton serviteur, quand tu as fait sortir d’Égypte nos pères, Seigneur notre Dieu. »
${}^{54}Quand Salomon eut achevé d’adresser au Seigneur toute cette prière et toute cette supplication, il se releva de devant l’autel du Seigneur, là où il s’était incliné et, les mains tendues vers le ciel, 
${}^{55}il se tint debout et bénit toute l’assemblée d’Israël d’une voix forte, en disant : 
${}^{56}« Béni soit le Seigneur ! Comme il l’avait dit, il a donné à son peuple Israël le pays de son repos ; aucune des promesses qu’il avait faites par l’intermédiaire de son serviteur Moïse n’est restée sans effet. 
${}^{57}Que le Seigneur notre Dieu soit avec nous, comme il a été avec nos pères, qu’il ne nous abandonne pas, qu’il ne nous rejette pas ! 
${}^{58}Qu’il incline nos cœurs vers lui, pour que nous suivions tous ses chemins et que nous gardions les commandements, les décrets et les ordonnances qu’il a donnés à nos pères. 
${}^{59}Ces supplications que j’ai prononcées devant le Seigneur notre Dieu, qu’elles lui restent présentes jour et nuit, afin qu’il rende justice à moi son serviteur et à son peuple Israël, jour après jour. 
${}^{60}Tous les peuples de la terre sauront alors que c’est le Seigneur qui est Dieu\\, il n’y en a pas d’autre. 
${}^{61}Alors, en observant ses décrets et en gardant ses commandements, votre cœur sera tout entier au Seigneur notre Dieu, comme aujourd’hui. »
${}^{62}Le roi et tout Israël avec lui offraient des sacrifices devant le Seigneur. 
${}^{63}Salomon offrit en sacrifice vingt-deux mille bœufs et cent vingt mille moutons, sacrifice de paix qu’il offrait au Seigneur. C’est ainsi que le roi et tous les fils d’Israël firent la dédicace de la maison du Seigneur. 
${}^{64}Ce jour-là, le roi consacra le milieu de la cour qui était devant la maison du Seigneur. C’est là, en effet, qu’il offrit l’holocauste, l’offrande de céréales et les graisses des sacrifices de paix, car l’autel de bronze qui était devant le Seigneur était trop petit pour contenir l’holocauste, l’offrande de céréales et les graisses des sacrifices de paix.
${}^{65}En ce temps-là, Salomon – et tout Israël avec lui – célébra la fête des Tentes : ce fut un grand rassemblement, depuis l’Entrée-de-Hamath jusqu’au Torrent d’Égypte, en présence du Seigneur notre Dieu, durant sept jours. 
${}^{66}Le huitième jour, il renvoya le peuple. Les gens bénirent le roi et s’en allèrent à leurs tentes, joyeux et le cœur content pour tout le bien que le Seigneur avait accordé à David, son serviteur, et à Israël, son peuple.
      
         
      \bchapter{}
      \begin{verse}
${}^{1}Après que Salomon eut achevé de construire la maison du Seigneur et la maison du roi, et tout ce que Salomon avait désiré faire pour son bon plaisir, 
${}^{2}le Seigneur lui apparut une seconde fois, comme il lui était déjà apparu à Gabaon. 
${}^{3}Le Seigneur lui dit : « J’ai entendu la prière et la supplication que tu as présentées devant moi. Je consacre cette Maison que tu as construite pour y mettre mon nom à jamais. Et mes yeux et mon cœur y seront pour toujours.
${}^{4}Pour toi, si tu marches devant moi, comme l’a fait David, ton père, d’un cœur intègre et avec droiture, afin d’agir en tout selon mes commandements, et si tu gardes mes décrets et mes ordonnances, 
${}^{5}alors je maintiendrai le trône de ta royauté sur Israël à jamais, selon ce que j’ai dit à David, ton père : “Aucun des tiens siégeant sur le trône d’Israël ne sera écarté”. 
${}^{6}Mais si vous vous détournez de moi, vous et vos fils, si vous ne gardez plus les commandements et les décrets que j’ai placés devant vous, si vous suivez et servez d’autres dieux, et vous prosternez devant eux, 
${}^{7}alors je retrancherai Israël de la surface de la terre que je lui ai donnée ; la Maison que j’ai consacrée à mon nom, je la rejetterai loin de ma face ; et Israël deviendra la fable et la risée de tous les peuples ; 
${}^{8}cette Maison qui était élevée, quiconque passera près d’elle sifflera de stupeur ; on dira : “Pourquoi donc le Seigneur a-t-il agi de cette manière envers ce pays et envers cette Maison ?” 
${}^{9}On lui répondra : “C’est qu’ils ont abandonné le Seigneur leur Dieu, lui qui avait fait sortir leurs pères du pays d’Égypte. Ils se sont attachés à d’autres dieux, devant lesquels ils se sont prosternés et qu’ils ont servis. Voilà pourquoi le Seigneur a fait venir sur eux tout ce malheur.” »
      
         
${}^{10}Au terme des vingt années pendant lesquelles Salomon avait bâti les deux Maisons, la maison du Seigneur et la maison du roi, 
${}^{11}Hiram, le roi de Tyr, ayant fourni à Salomon du bois de cèdre et de cyprès, et de l’or selon son bon plaisir, le roi Salomon lui donna vingt villes au pays de Galilée. 
${}^{12}Hiram sortit de Tyr pour aller voir les villes que Salomon lui avait données. Mais elles ne plurent pas à ses yeux. Il s’exclama : 
${}^{13}« Quelles villes m’as-tu données là, mon frère ! » Il les surnomma « Terre-de-Rien », nom qu’elles portent encore aujourd’hui. 
${}^{14}Hiram envoya au roi cent vingt lingots d’or.
${}^{15}Voici maintenant un mot sur la corvée que leva le roi Salomon pour la construction de la maison du Seigneur et de sa propre maison, le Terre-Plein et la muraille de Jérusalem, ainsi que Haçor, Meguiddo et Guezer. 
${}^{16}Pharaon, roi d’Égypte, était monté contre Guezer, s’en était emparé, et y avait mis le feu. Les Cananéens qui habitaient la ville, il les avait massacrés. Puis il l’avait donnée en dot à sa fille, la femme de Salomon. 
${}^{17}Salomon rebâtit Guezer et Beth-Horone-le-Bas, 
${}^{18}Baalath et Tamar-du-Désert, dans le pays, 
${}^{19}toutes les villes d’entrepôts appartenant à Salomon, les villes de garnison pour les chars et celles des cavaliers. Et Salomon bâtit ce qu’il désirait, dans Jérusalem, au Liban et dans tout le pays soumis à son autorité. 
${}^{20}Il restait toute une population d’Amorites, de Hittites, de Perizzites, de Hivvites et de Jébuséens, qui n’étaient pas des fils d’Israël. 
${}^{21}Leurs fils qui, après eux, étaient restés dans le pays et que les fils d’Israël n’avaient pu vouer à l’anathème, Salomon les réquisitionna pour la corvée servile, jusqu’à ce jour. 
${}^{22}Mais Salomon ne soumit au servage aucun des fils d’Israël, car ceux-ci étaient des hommes de guerre, ses serviteurs, ses officiers, ses écuyers, les commandants de ses chars et de ses cavaliers. 
${}^{23}Voici le nombre des chefs des préposés aux travaux de Salomon : ils étaient cinq cent cinquante et commandaient au peuple qui effectuait les travaux. 
${}^{24}Quand la fille de Pharaon monta de la Cité de David à la maison que le roi lui avait bâtie, alors Salomon édifia le Terre-Plein.
       
${}^{25}Trois fois par an, Salomon offrait des holocaustes et des sacrifices de paix sur l’autel qu’il avait bâti pour le Seigneur. Et là il brûlait aussi de l’encens devant le Seigneur. Il portait ainsi la Maison à son achèvement. 
${}^{26}Le roi Salomon arma une flotte à Écione-Guéber, près d’Eilath, sur le rivage de la mer des Roseaux, au pays d’Édom. 
${}^{27}Avec les serviteurs de Salomon, Hiram dépêcha sur les navires ses serviteurs, des navigateurs connaissant bien la mer. 
${}^{28}Ils arrivèrent à Ophir et s’y procurèrent de l’or : quatre cent vingt lingots qu’ils rapportèrent au roi Salomon.
      
         
      \bchapter{}
      \begin{verse}
${}^{1}La reine de Saba avait entendu parler de la renommée de Salomon\\, qui faisait honneur au nom du Seigneur\\. Elle vint donc pour le mettre à l’épreuve en lui proposant des énigmes. 
${}^{2} Elle arriva à Jérusalem avec une escorte imposante : des chameaux chargés d’aromates et d’une énorme quantité d’or et de pierres précieuses. Quand elle fut parvenue auprès de Salomon, elle lui exposa les questions qu’elle avait préparées\\, 
${}^{3} mais Salomon trouva réponse à tout et ne fut arrêté par aucune difficulté\\. 
${}^{4} Lorsque la reine de Saba vit toute la sagesse de Salomon, le palais qu’il avait construit, 
${}^{5} les plats servis à sa table, le logement de ses officiers, la tenue du service et l’habillement des serviteurs, ses sommeliers, les holocaustes qu’il offrait à la maison du Seigneur, 
${}^{6} elle en eut le souffle coupé, et elle dit au roi : « Ce que j’ai entendu dire dans mon pays sur toi et sur ta sagesse, c’était donc vrai ! 
${}^{7} Je ne voulais pas croire ce qu’on disait, avant de venir et de voir de mes yeux ; mais voilà qu’on ne m’en avait pas appris la moitié ! Tu surpasses en sagesse et en magnificence la renommée qui était venue jusqu’à moi. 
${}^{8} Heureux tes gens, heureux tes serviteurs que voici, eux qui se tiennent continuellement devant toi et qui entendent ta sagesse ! 
${}^{9} Béni soit le Seigneur ton Dieu, qui t’a montré sa bienveillance en te plaçant sur le trône d’Israël. Parce que le Seigneur aime Israël pour toujours, il t’a établi roi pour exercer le droit et la justice. » 
${}^{10} Elle fit présent au roi de cent vingt lingots\\d’or, d’une grande quantité d’aromates et de pierres précieuses ; il n’est plus jamais venu une quantité d’aromates pareille à celle que la reine de Saba avait donnée au roi Salomon.
${}^{11}La flotte d’Hiram avait donc apporté l’or d’Ophir. Elle en rapporta également du bois de santal, en très grande quantité, et des pierres précieuses. 
${}^{12}Avec ce bois de santal, le roi fit une balustrade pour la maison du Seigneur et la maison du roi ; on en fit aussi des cithares et des harpes pour les chantres. Par la suite, on ne reçut plus jamais de ce bois de santal, et jusqu’à ce jour on n’en a plus revu.
${}^{13}Le roi Salomon offrit à la reine de Saba tout ce qui répondait à ses désirs, en plus des présents qu’il lui faisait avec une munificence digne du roi Salomon. Puis elle s’en retourna dans son pays avec ses serviteurs.
${}^{14}En une seule année, le poids de l’or qui parvenait à Salomon était de six cent soixante-six lingots d’or, 
${}^{15}sans compter les péages des voyageurs, les transactions des commerçants, de tous les rois de l’Occident et des gouverneurs du pays.
${}^{16}Le roi Salomon fit deux cents grands boucliers d’or battu – il utilisait six cents pièces d’or pour un grand bouclier – 
${}^{17}et trois cents petits boucliers d’or battu – il utilisait trois livres d’or pour un petit bouclier. Le roi les plaça dans la maison de la Forêt du Liban. 
${}^{18}Le roi fit aussi un grand trône d’ivoire, qu’il plaqua d’or affiné. 
${}^{19}Ce trône avait six degrés, un dossier à sommet arrondi, et des bras de chaque côté du siège ; deux lions étaient debout près des bras 
${}^{20}et douze lions se tenaient là, sur les six degrés, de chaque côté. Dans aucun royaume on ne fit chose pareille. 
${}^{21}Toutes les coupes du roi Salomon étaient en or, et tous les objets de la maison de la Forêt du Liban, en or fin, non pas en argent : on n’en faisait aucun cas au temps de Salomon. 
${}^{22}Car le roi avait sur la mer une flotte de navires de haut bord naviguant avec la flotte d’Hiram ; une fois tous les trois ans, la flotte arrivait, apportant or et argent, ivoires, singes et paons. 
${}^{23}Le roi Salomon devint le plus grand de tous les rois de la terre en richesse et en sagesse. 
${}^{24}Toute la terre cherchait à rencontrer Salomon face à face, pour entendre la sagesse que Dieu avait mise en son cœur. 
${}^{25}Chacun apportait son offrande : objets d’argent et objets d’or, vêtements, armes et aromates, chevaux et mulets ; et ainsi d’année en année.
${}^{26}Salomon rassembla des chars et des cavaliers. Il avait mille quatre cents chars et douze mille cavaliers. Il les installa dans les villes de garnison et auprès de lui, à Jérusalem. 
${}^{27}À Jérusalem, le roi fit abonder l’argent autant que les pierres, et les cèdres autant que les sycomores dans le Bas-Pays. 
${}^{28}Les chevaux de Salomon provenaient d’Égypte, et le groupe des courtiers du roi les achetaient à Qewé pour un prix convenu. 
${}^{29}Un char provenant d’Égypte atteignait six cents pièces d’argent, et un cheval cent cinquante. Il en était de même pour tous les rois hittites et pour les rois d’Aram, que ces courtiers approvisionnaient.
      
         
      \bchapter{}
      \begin{verse}
${}^{1}Le roi Salomon aima de nombreuses femmes étrangères : outre la fille de Pharaon, des Moabites, des Ammonites, des Édomites, des Sidoniennes, des Hittites. 
${}^{2}Elles étaient de ces nations dont le Seigneur avait dit aux fils d’Israël : « Vous n’entrerez pas chez elles, et elles n’entreront pas chez vous : sûrement, elles détourneraient votre cœur vers leurs dieux. » Mais Salomon s’attacha à elles par amour. 
${}^{3}Il eut sept cents femmes de rang princier et trois cents concubines ; et ses femmes détournèrent son cœur.
${}^{4}Salomon vieillissait ; ses femmes le détournèrent vers d’autres dieux, et son cœur n’était plus tout entier au Seigneur, comme l’avait été celui de son père David. 
${}^{5} Salomon prit part\\au culte d’Astarté, la déesse des Sidoniens, et à celui de Milcom, l’horrible idole\\des Ammonites\\. 
${}^{6} Il fit ce qui est mal aux yeux du Seigneur, et il ne lui obéit pas aussi parfaitement que son père David. 
${}^{7} Il construisit alors, sur la montagne à l’est de Jérusalem, un lieu sacré pour Camosh, l’horrible idole de Moab, et un autre pour Milcom, l’horrible idole des Ammonites. 
${}^{8} Il en fit d’autres pour permettre à toutes ses femmes étrangères de brûler de l’encens et d’offrir des sacrifices à leurs dieux. 
${}^{9} Le Seigneur s’irrita contre Salomon parce qu’il s’était détourné du Seigneur Dieu d’Israël. Pourtant, celui-ci lui était apparu deux fois, 
${}^{10} et lui avait défendu de suivre d’autres dieux ; mais Salomon avait désobéi. 
${}^{11} Le Seigneur lui déclara : « Puisque tu t’es conduit de cette manière, puisque tu n’as pas gardé mon alliance ni observé mes décrets, je vais t’enlever le royaume et le donner à l’un de tes serviteurs. 
${}^{12} Seulement, à cause de ton père David, je ne ferai pas cela durant ta vie ; c’est de la main de ton fils que j’enlèverai le royaume. 
${}^{13} Et encore, je ne lui enlèverai pas tout, je laisserai une tribu à ton fils, à cause de mon serviteur David et de Jérusalem, la ville que j’ai choisie. »
${}^{14}Le Seigneur suscita un adversaire à Salomon : Hadad l’Édomite, qui était de la descendance royale d’Édom. 
${}^{15}Au temps où David était en Édom, Joab, le chef de l’armée, était monté pour ensevelir les morts et avait frappé tous les mâles du pays d’Édom, 
${}^{16}– car Joab et tout Israël étaient restés là pendant six mois, jusqu’à ce qu’ils aient supprimé tous les mâles d’Édom. 
${}^{17}C’est alors que Hadad s’était enfui avec des Édomites, des serviteurs de son père, pour aller en Égypte. Hadad était encore un tout jeune homme. 
${}^{18}Ils partirent de Madiane et arrivèrent à Parane. Ils prirent avec eux des hommes de Parane et arrivèrent en Égypte auprès de Pharaon, roi d’Égypte. Celui-ci lui donna une maison, lui promit le pain et lui attribua une terre. 
${}^{19}Hadad fut en grande faveur auprès de Pharaon, qui lui donna pour épouse la sœur de sa femme, la sœur de Tapnès la reine mère. 
${}^{20}La sœur de Tapnès lui enfanta son fils Guenoubath, et Tapnès l’éleva dans la maison de Pharaon. Guenoubath demeura donc dans la maison de Pharaon, parmi les fils de Pharaon. 
${}^{21}Hadad apprit en Égypte que David reposait avec ses pères, et que Joab, le chef de l’armée, était mort. Il dit à Pharaon : « Laisse-moi partir, que j’aille dans mon pays ! » 
${}^{22}Pharaon lui demanda : « Que te manque-t-il auprès de moi, pour que tu cherches à t’en aller dans ton pays ? » Il répondit : « Rien, mais laisse-moi partir. »
${}^{23}Dieu suscita un autre adversaire à Salomon : Rezone, fils d’Éliada. Il s’était enfui de chez son maître Hadadèzer, roi de Soba ; 
${}^{24}il avait ensuite rallié des hommes autour de lui et il était devenu chef de bande. Mais parce que David les massacrait, ils allèrent à Damas et s’y installèrent. Et ils régnèrent à Damas. 
${}^{25}Rezone fut un adversaire d’Israël durant toute la vie de Salomon. Voici le mal que fit Hadad : il eut Israël en aversion. Il avait régné sur Aram.
${}^{26}Jéroboam, fils de Nebath l’Éphraïmite, était de Seréda ; le nom de sa mère était Seroua ; elle était veuve. Lui était au service de Salomon et il se révolta contre le roi. 
${}^{27}Voici comment il se révolta contre le roi. Salomon construisait le Terre-Plein pour fermer la brèche de la Cité de David son père. 
${}^{28}Ce Jéroboam était quelqu’un de grande valeur. Salomon avait remarqué comment le jeune homme accomplissait son ouvrage, et il fit de lui l’inspecteur des corvées qui pesaient sur la Maison de Joseph. 
${}^{29}Un jour que Jéroboam était sorti de Jérusalem, il fut arrêté en chemin par le prophète Ahias de Silo ; celui-ci portait un manteau neuf, et tous deux étaient seuls dans la campagne. 
${}^{30}Ahias prit le manteau neuf qu’il portait et le déchira en douze morceaux. 
${}^{31}Puis il dit à Jéroboam : « Prends pour toi dix morceaux, car ainsi parle le Seigneur, Dieu d’Israël : Voici que je vais déchirer le royaume en l’arrachant à Salomon, et je te donnerai dix tribus. 
${}^{32}Il lui restera une tribu, à cause de mon serviteur David, et de Jérusalem, la ville que je me suis choisie parmi toutes les tribus d’Israël. 
${}^{33}C’est qu’ils m’ont abandonné et se sont prosternés devant Astarté, la déesse des Sidoniens, devant Camosh, le dieu de Moab, et devant Milcom, le dieu des Ammonites ; ils n’ont pas marché dans mes chemins, pour pratiquer ce qui est droit à mes yeux et respecter mes décrets et mes ordonnances, comme l’a fait David, son père. 
${}^{34}Mais je ne reprendrai pas de sa main tout le royaume, car je le maintiendrai prince tous les jours de sa vie, à cause de David, mon serviteur, que j’ai choisi, lui qui a gardé mes commandements et mes décrets. 
${}^{35}Je reprendrai des mains de son fils la royauté et je te la donnerai sur dix tribus. 
${}^{36}Mais à son fils, je donnerai une seule tribu pour que mon serviteur David ait toujours une lampe qui brille devant moi, à Jérusalem, la ville que je me suis choisie pour y mettre mon nom. 
${}^{37}Toi, je te prendrai, tu régneras sur tout ce que tu désires, et tu seras roi sur Israël. 
${}^{38}Si tu obéis à tout ce que je vais te commander, si tu marches dans mes chemins et si tu pratiques ce qui est droit à mes yeux, en gardant mes décrets et mes commandements, comme l’a fait David mon serviteur, alors je serai avec toi et je construirai pour toi une maison stable, comme celle que j’ai bâtie pour David, et je te donnerai Israël. 
${}^{39}De la sorte, j’humilierai la descendance de David, mais pas pour toujours. »
${}^{40}Salomon chercha à faire mourir Jéroboam. Jéroboam se leva et s’enfuit en Égypte auprès de Shishaq, roi d’Égypte, et il vécut en Égypte jusqu’à la mort de Salomon.
${}^{41}Le reste des actions de Salomon,
        \\tout ce qu’il a fait, et sa sagesse,
        \\cela n’est-il pas écrit dans le livre des Actes de Salomon ?
${}^{42}Le temps que Salomon régna à Jérusalem
        \\sur tout Israël fut de quarante ans.
${}^{43}Puis Salomon reposa avec ses pères,
        \\et il fut enseveli dans la Cité de David son père.
        \\Son fils Roboam régna à sa place.
      
         
      \bchapter{}
      \begin{verse}
${}^{1}Roboam se rendit à Sichem ; c’est à Sichem, en effet, que tout Israël était venu pour le faire roi. 
${}^{2}Jéroboam, fils de Nebath, apprit la nouvelle. À cette époque, il était encore en Égypte. – Il s’était enfui loin du roi Salomon et, depuis, il demeurait en Égypte. 
${}^{3}On envoya chercher Jéroboam, et il vint, ainsi que toute l’assemblée d’Israël. Ils s’adressèrent à Roboam en disant : 
${}^{4}« Ton père a rendu pénible notre joug ; toi, maintenant, allège la pénible servitude imposée par ton père, et le joug pesant qu’il nous a infligé ; alors nous te servirons. » 
${}^{5}Il leur répondit : « Retirez-vous pour trois jours, puis revenez vers moi. » Et le peuple se retira.
${}^{6}Le roi Roboam prit conseil des anciens, ceux qui s’étaient tenus en présence de son père Salomon, de son vivant. Il leur dit : « Quelle réponse conseillez-vous de faire à ce peuple ? » 
${}^{7}Ils lui dirent alors : « Si tu te fais aujourd’hui le serviteur de ce peuple, si tu te mets à leur service, et si, dans ta réponse, tu leur adresses des paroles bienveillantes, ils seront tes serviteurs pour toujours. » 
${}^{8}Mais il négligea le conseil que lui donnaient les anciens ; il prit conseil des jeunes gens qui avaient grandi avec lui et qui se tenaient en sa présence. 
${}^{9}Il leur demanda : « Que conseillez-vous ? Quelle réponse allons-nous faire à ce peuple qui m’a parlé en disant : “Allège donc le joug que nous a infligé ton père” ? » 
${}^{10}Les jeunes gens qui avaient grandi avec lui répondirent : « Voici ce que tu répondras à ce peuple qui t’a parlé en disant : “Ton père a rendu lourd notre joug : mais toi, pour nous, allège-le”. Voici ce que tu leur diras : “Mon petit doigt est plus fort que les reins de mon père. 
${}^{11}S’il est vrai que mon père vous accablait sous un joug pesant, je vais, moi, ajouter encore à votre joug. Mon père vous a corrigés avec des lanières ? Eh bien, moi, je vous corrigerai avec des fouets à pointes de fer !” »
${}^{12}Le troisième jour, Jéroboam revint, ainsi que tout le peuple, auprès de Roboam, conformément à la parole du roi : « Revenez vers moi le troisième jour ». 
${}^{13}Et le roi répondit durement au peuple, en négligeant le conseil donné par les anciens. 
${}^{14}Il parla au peuple en suivant le conseil des jeunes gens : « Mon père a rendu lourd votre joug, je vais, moi, ajouter encore à votre joug. Mon père vous a corrigés avec des lanières ? Eh bien, moi, je vous corrigerai avec des fouets à pointes de fer ! » 
${}^{15}Ainsi, le roi n’écouta pas le peuple ; en effet, la tournure que prenaient les choses venait du Seigneur, pour que s’accomplisse la parole que le Seigneur avait dite par l’intermédiaire d’Ahias de Silo à Jéroboam, fils de Nebath. 
${}^{16}Tout Israël vit que le roi ne les avait pas écoutés. Le peuple rétorqua au roi :
        \\« Quelle part avons-nous chez David ?
        \\Pas d’héritage chez le fils de Jessé !
        \\À tes tentes, Israël !
        \\Maintenant, David, pourvois donc à ta maison ! »
      Et Israël s’en alla dans ses tentes. 
${}^{17}Quant aux fils d’Israël qui demeuraient dans les villes de Juda, Roboam régna sur eux. 
${}^{18}Le roi Roboam envoya Adoram, le chef de la corvée ; mais tout Israël le lapida, et il mourut. Et le roi Roboam se vit contraint de monter sur un char pour s’enfuir à Jérusalem. 
${}^{19}Les dix tribus\\d’Israël rejetèrent la maison de David, et cette situation dure encore aujourd’hui où ceci est écrit\\.
${}^{20}Dès que tout Israël apprit que Jéroboam était revenu, on l’envoya chercher pour qu’il vienne à la réunion du peuple, et on le fit roi sur tout Israël. Il ne restait, pour suivre la Maison de David, que la seule tribu de Juda. 
${}^{21}Roboam arriva à Jérusalem. Il rassembla toute la maison de Juda et la tribu de Benjamin, cent quatre-vingt mille guerriers d’élite, pour combattre la Maison d’Israël, afin de rendre la royauté à Roboam, fils de Salomon. 
${}^{22}Alors, la parole de Dieu fut adressée à Shemaya, homme de Dieu : 
${}^{23}« Parle à Roboam, fils de Salomon, roi de Juda, à toute la Maison de Juda et de Benjamin, et au reste du peuple : 
${}^{24}“Ainsi parle le Seigneur : Ne montez pas, ne faites pas la guerre à vos frères, les fils d’Israël. Retournez chacun chez soi, car je suis moi-même à l’origine de cette affaire”. » Alors ils écoutèrent la parole du Seigneur et ils s’en retournèrent selon la parole du Seigneur. 
${}^{25}Jéroboam fortifia Sichem dans la montagne d’Éphraïm et s’y établit. Puis il en sortit et fortifia Penouël.
${}^{26}Jéroboam se dit : « Maintenant, le royaume risque fort de se rallier de nouveau à la maison de David. 
${}^{27}Si le peuple continue de monter à Jérusalem\\pour offrir des sacrifices dans la maison du Seigneur, le cœur de ce peuple reviendra vers son souverain, Roboam, roi de Juda, et l’on me tuera\\. » 
${}^{28}Après avoir tenu conseil, Jéroboam\\fit fabriquer deux veaux en or, et il déclara au peuple : « Voilà trop longtemps que vous montez à Jérusalem ! Israël, voici tes dieux, qui t’ont fait monter du pays d’Égypte. » 
${}^{29}Il plaça l’un des deux veaux à Béthel, l’autre à Dane, 
${}^{30}et ce fut un grand péché. Le peuple conduisit en procession celui qui allait à Dane. 
${}^{31}Jéroboam y établit un temple à la manière des lieux sacrés\\. Il institua des prêtres pris n’importe où, et qui n’étaient pas des descendants de Lévi. 
${}^{32}Jéroboam célébra la fête le quinzième jour du huitième mois, fête pareille à celle que l’on célébrait en Juda, et il monta à l’autel. Il fit de même à Béthel en offrant des sacrifices aux veaux qu’il avait fabriqués ; il établit à Béthel les prêtres des lieux sacrés qu’il avait institués. 
${}^{33}Il monta à l’autel qu’il avait édifié à Béthel, le quinze du huitième mois, date qu’il avait de lui-même fixée. Il organisa une fête pour les fils d’Israël et il monta à l’autel pour brûler de l’encens.
      
         
      \bchapter{}
      \begin{verse}
${}^{1}Voici qu’un homme de Dieu vint de Juda à Béthel, par ordre du Seigneur. Jéroboam se tenait à l’autel et brûlait de l’encens. 
${}^{2}L’homme interpella l’autel, par ordre du Seigneur, en ces termes : « Autel, autel, ainsi parle le Seigneur ! Voici : un fils va naître pour la Maison de David ; son nom sera Josias ; sur toi, il offrira en sacrifice les prêtres des lieux sacrés, qui, sur toi, brûlent de l’encens. On consumera sur toi des ossements humains. » 
${}^{3}Ce jour-là, il annonça qu’il y aurait un signe. Il dit : « Voici le signe montrant que le Seigneur a parlé : l’autel va se fendre et la cendre qui est dessus se répandra. » 
${}^{4}Dès que le roi entendit la parole que l’homme de Dieu avait proférée contre l’autel de Béthel, Jéroboam tendit la main de dessus l’autel, en disant : « Saisissez-le ! » Mais la main qu’il avait tendue contre l’homme sécha, et il ne pouvait plus la ramener à lui. 
${}^{5}L’autel se fendit, et la cendre se répandit de l’autel, conformément au signe qu’avait donné l’homme de Dieu par ordre du Seigneur. 
${}^{6}Prenant alors la parole, le roi dit à l’homme de Dieu : « Apaise, je te prie, le visage du Seigneur ton Dieu ; intercède pour moi : que ma main revienne à moi ! » L’homme de Dieu apaisa le visage du Seigneur. La main du roi revint à lui et elle fut comme auparavant. 
${}^{7}Le roi s’adressa à l’homme de Dieu : « Viens avec moi à la maison pour te restaurer, et je t’offrirai un cadeau. » 
${}^{8}L’homme de Dieu dit au roi : « Même si tu m’offrais la moitié de ta maison, je n’entrerais pas chez toi, je ne mangerais pas de pain et ne boirais pas d’eau en ce lieu. 
${}^{9}Car il m’a été ordonné par la parole du Seigneur :
        \\Tu ne mangeras pas de pain,
        \\tu ne boiras pas d’eau,
        \\et tu ne retourneras pas
        \\par le chemin que tu as pris pour venir. »
${}^{10}Il prit donc un autre chemin, et ne retourna pas par le chemin qu’il avait pris pour venir à Béthel.
${}^{11}Or à Béthel demeurait un vieux prophète. Ses fils vinrent lui raconter tout ce qu’avait fait l’homme de Dieu à Béthel ce jour-là ; tout ce qu’il avait dit au roi, ils le racontèrent à leur père. 
${}^{12}Leur père leur demanda : « Quel chemin a-t-il pris ? » Ses fils lui montrèrent le chemin qu’avait pris l’homme de Dieu venu de Juda. 
${}^{13}Il dit à ses fils : « Sellez-moi un âne. » Ils lui sellèrent l’âne, et il monta dessus. 
${}^{14}Il partit sur les traces de l’homme de Dieu et le trouva assis sous le térébinthe. Il lui dit : « Es-tu l’homme de Dieu venu de Juda ? » Il répondit : « C’est moi. » 
${}^{15}Il lui dit alors : « Viens avec moi, dans ma maison, manger un morceau de pain. » 
${}^{16}Mais l’autre répondit : « Je ne puis ni retourner avec toi, ni t’accompagner ; je ne mangerai pas de pain et ne boirai pas d’eau avec toi en ce lieu. 
${}^{17}Car il m’a été dit par ordre du Seigneur :
        \\Tu ne mangeras pas de pain,
        \\tu ne boiras pas d’eau là-bas,
        \\tu ne prendras pas au retour le chemin de l’aller. »
${}^{18}Le vieux prophète insista : « Je suis prophète, moi aussi, tout comme toi ! Un ange m’a parlé sur l’ordre du Seigneur. Il m’a dit : Ramène-le avec toi dans ta maison ; qu’il mange du pain et boive de l’eau. » Mais il lui mentait ! 
${}^{19}L’homme de Dieu revint donc avec lui, mangea du pain dans sa maison et but de l’eau. 
${}^{20}Or, tandis qu’ils étaient à table, une parole du Seigneur fut adressée au prophète qui l’avait fait revenir. 
${}^{21}Il interpella l’homme de Dieu venu de Juda : « Ainsi parle le Seigneur : Puisque tu as bravé l’ordre du Seigneur, que tu n’as pas gardé le commandement du Seigneur ton Dieu, 
${}^{22}et puisque tu es revenu, que tu as mangé du pain et bu de l’eau en ce lieu dont il t’avait dit : “N’y mange pas de pain et n’y bois pas d’eau”, ton cadavre n’entrera pas dans le tombeau de tes pères. » 
${}^{23}Après qu’il eut mangé du pain et bu de l’eau, le vieux prophète sella son âne pour l’homme de Dieu qu’il avait fait revenir. 
${}^{24}Celui-ci partit. Un lion le rencontra en chemin et le tua. Son cadavre gisait sur le chemin ; l’âne se tenait à côté de lui, le lion se tenait aussi à côté du cadavre. 
${}^{25}Voici que des passants virent le cadavre qui gisait sur le chemin, et le lion qui se tenait à côté du cadavre. Ils allèrent en parler dans la ville où habitait le vieux prophète. 
${}^{26}Le prophète qui avait détourné l’homme de Dieu de son chemin apprit la nouvelle, et il dit : « C’est l’homme de Dieu qui a bravé l’ordre du Seigneur. Le Seigneur l’a livré au lion qui l’a broyé et l’a tué, conformément à la parole que le Seigneur lui avait dite. » 
${}^{27}Il dit à ses fils : « Sellez-moi un âne ». Ils le sellèrent. 
${}^{28}Il partit et trouva le cadavre gisant sur le chemin. L’âne et le lion se tenaient à côté du cadavre. Le lion n’avait pas dévoré le cadavre, ni rompu l’échine de l’âne. 
${}^{29}Le prophète releva le cadavre de l’homme de Dieu, le disposa sur l’âne et le ramena. Le vieux prophète revint à la ville pour le pleurer et l’ensevelir. 
${}^{30}Il déposa le cadavre dans son propre tombeau. Et on le pleura ainsi : « Hélas, mon frère ! » 
${}^{31}Après l’avoir enseveli, il dit à ses fils : « Quand je mourrai, vous m’ensevelirez dans le tombeau où est enseveli l’homme de Dieu. À côté de ses os, vous déposerez mes os. 
${}^{32}Car elle se réalisera, la parole qu’il a proférée par ordre du Seigneur contre l’autel de Béthel et contre tous les temples des lieux sacrés qui sont dans les villes de Samarie. »
${}^{33}Après ces événements, Jéroboam persévéra dans sa mauvaise conduite ; il continua d’instituer n’importe qui comme prêtres des lieux sacrés : il donnait l’investiture à tous ceux qui le désiraient, pour en faire des prêtres des lieux sacrés. 
${}^{34}Tout cela fit tomber dans le péché la maison de Jéroboam, entraîna sa ruine et provoqua sa disparition de la surface de la terre.
      
         
      \bchapter{}
      \begin{verse}
${}^{1}En ce temps-là, Abiya, fils de Jéroboam, tomba malade. 
${}^{2}Jéroboam dit à sa femme : « Lève-toi, je te prie : déguise-toi, pour que l’on ne sache pas que tu es la femme de Jéroboam. Va à Silo : là se trouve le prophète Ahias, qui a dit de moi que je serais roi sur ce peuple. 
${}^{3}Emporte dix pains, des gâteaux secs, un pot de miel, et va le trouver chez lui. Il te révélera ce qui adviendra de l’enfant. » 
${}^{4}La femme de Jéroboam fit ainsi. Elle se leva, partit pour Silo et se présenta à la maison d’Ahias. Or Ahias ne pouvait plus voir ; il avait le regard fixe à cause de son grand âge. 
${}^{5}Mais le Seigneur avait dit au prophète Ahias : « Voici : la femme de Jéroboam est en route pour te consulter au sujet de son fils qui est malade. Tu lui parleras de telle et telle manière. Et lorsqu’elle se présentera, elle se fera passer pour une étrangère. » 
${}^{6}Dès que le prophète Ahias entendit le bruit de ses pas dans l’entrée, il dit : « Entre, femme de Jéroboam. Pourquoi te fais-tu passer pour une étrangère ? J’ai pour toi un dur message. 
${}^{7}Va dire à Jéroboam : “Ainsi parle le Seigneur, Dieu d’Israël : Je t’ai élevé du milieu du peuple, et je t’ai placé comme chef sur mon peuple Israël ; 
${}^{8}j’ai arraché la royauté à la maison de David et je te l’ai donnée. Mais tu n’as pas été comme mon serviteur David, qui gardait mes commandements et me suivait de tout son cœur, pour ne faire que ce qui est juste à mes yeux. 
${}^{9}Tu as agi plus mal encore que tous ceux qui t’ont précédé : tu es allé te fabriquer d’autres dieux, des idoles de métal fondu, pour provoquer mon indignation, et moi, tu m’as rejeté derrière ton dos. 
${}^{10}C’est pourquoi, voici que je fais venir le malheur sur la maison de Jéroboam. J’exterminerai tous les mâles de la famille de Jéroboam, esclaves ou hommes libres en Israël. Je balaierai les derniers restes de la maison de Jéroboam, comme on balaie à fond les ordures. 
${}^{11}Celui de la famille de Jéroboam qui mourra dans la ville, les chiens le mangeront ; celui qui mourra dans les champs, l’oiseau du ciel le mangera. Car le Seigneur a parlé.” 
${}^{12}Quant à toi, debout ! Retourne dans ta maison. Au moment où tu entreras dans la ville, l’enfant mourra. 
${}^{13}Tout Israël le pleurera, puis on l’ensevelira. Lui seul, en effet, de la famille de Jéroboam, sera mis dans un tombeau, car en lui seul a été trouvé quelque chose de bon pour le Seigneur, le Dieu d’Israël. 
${}^{14}Le Seigneur suscitera pour lui-même un roi sur Israël qui supprimera la maison de Jéroboam. – C’est pour aujourd’hui ! Et même pour maintenant ! – 
${}^{15}Le Seigneur frappera Israël, qui deviendra comme le roseau qui s’agite dans l’eau. Il extirpera Israël du sol fertile qu’il avait donné à leurs pères. Il les dispersera jusqu’au-delà de l’Euphrate, puisqu’en fabriquant leurs poteaux sacrés, ils ont provoqué l’indignation du Seigneur. 
${}^{16}Il livrera Israël à cause des péchés que Jéroboam a commis, et des péchés qu’il a fait commettre à Israël. »
${}^{17}La femme de Jéroboam se leva ; elle partit et revint à Tirsa. Et comme elle arrivait au seuil de la maison, l’enfant mourut. 
${}^{18}On l’ensevelit, et tout Israël le pleura, conformément à la parole que le Seigneur avait dite par l’intermédiaire de son serviteur Ahias le prophète.
${}^{19}Le reste des actions de Jéroboam,
        \\ses combats et son règne,
        \\tout cela est écrit dans le livre des Annales des rois d’Israël.
${}^{20}Le temps que régna Jéroboam fut de vingt-deux ans ;
        \\puis il reposa avec ses pères.
        \\Son fils Nadab régna à sa place.
${}^{21}Roboam, fils de Salomon, régna en Juda. Roboam avait quarante et un ans lorsqu’il devint roi, et il régna dix-sept ans à Jérusalem, la ville que le Seigneur avait choisie parmi toutes les tribus d’Israël pour y mettre son nom. Sa mère s’appelait Naama, l’Ammonite. 
${}^{22}Mais les gens de Juda firent ce qui est mal aux yeux du Seigneur, ils provoquèrent son ardeur jalouse plus que n’avaient fait leurs pères, par les péchés qu’ils avaient commis. 
${}^{23}Eux aussi, ils se construisirent des temples dans les lieux sacrés, des stèles, des poteaux sacrés, sur toute colline élevée et sous tout arbre vert. 
${}^{24}On pratiqua même la prostitution sacrée dans le pays. Ils imitèrent toutes les abominations des nations que le Seigneur avait dépossédées devant les fils d’Israël.
${}^{25}La cinquième année du règne de Roboam, Shishaq, roi d’Égypte, monta contre Jérusalem. 
${}^{26}Il s’empara des trésors de la maison du Seigneur et des trésors de la maison du roi ; il s’empara de tout ; il s’empara aussi de tous les boucliers d’or qu’avait faits Salomon. 
${}^{27}Le roi Roboam fit, pour les remplacer, des boucliers de bronze, et les confia aux chefs des gardes, à la porte de la maison du roi. 
${}^{28}À chaque fois que le roi se rendait à la maison du Seigneur, les gardes prenaient ces boucliers ; puis ils les rapportaient dans la salle des gardes.
${}^{29}Le reste des actions de Roboam, tout ce qu’il a fait,
        \\cela n’est-il pas écrit dans le livre des Annales des rois de Juda ?
${}^{30}Il y eut continuellement la guerre entre Roboam et Jéroboam.
${}^{31}Roboam reposa avec ses pères.
        \\Il fut enseveli avec eux, dans la Cité de David.
        \\Sa mère s’appelait Naama, l’Ammonite.
        \\Son fils Abiam régna à sa place.
      
         
      \bchapter{}
      \begin{verse}
${}^{1}La dix-huitième année du règne de Jéroboam, fils de Nebath, Abiam devint roi sur Juda. 
${}^{2}Il régna trois ans à Jérusalem. Sa mère s’appelait Maaka, elle était la fille d’Absalom. 
${}^{3}Il imita tous les péchés que son père avait commis avant lui, et son cœur ne fut pas tout entier avec le Seigneur son Dieu, comme l’avait été le cœur de David, son aïeul. 
${}^{4}Pourtant, à cause de David, le Seigneur son Dieu lui donna une lampe à Jérusalem, en maintenant son fils après lui et en gardant Jérusalem debout. 
${}^{5}Car David avait fait ce qui est droit aux yeux du Seigneur, et, aucun jour de sa vie, il ne s’était détourné de tout ce qu’il lui commandait, hormis dans l’affaire d’Ourias le Hittite. 
${}^{6}Il y eut toujours la guerre entre Roboam et Jéroboam.
${}^{7}Le reste des actions d’Abiam, tout ce qu’il a fait,
        \\cela n’est-il pas écrit dans le livre des Annales des rois de Juda ?
        \\Il y eut la guerre entre Abiam et Jéroboam.
${}^{8}Abiam reposa avec ses pères,
        \\et on l’ensevelit dans la Cité de David.
        \\Son fils Asa régna à sa place.
${}^{9}La vingtième année du règne de Jéroboam, roi d’Israël, Asa devint roi sur Juda. 
${}^{10}Il régna quarante et un ans à Jérusalem. Sa grand-mère s’appelait Maaka, fille d’Absalom. 
${}^{11}Asa fit ce qui est droit aux yeux du Seigneur, comme David, son ancêtre. 
${}^{12}Il expulsa du pays les prostitués sacrés et supprima toutes les idoles immondes que ses pères avaient fabriquées. 
${}^{13}Et de plus, il destitua Maaka, sa grand-mère, du titre de reine mère, parce qu’elle avait fabriqué une Infamie pour la déesse Ashéra. Asa abattit cette Infamie et la brûla dans le ravin du Cédron. 
${}^{14}Mais les lieux sacrés ne disparurent pas, bien que le cœur d’Asa fût tout entier au Seigneur tous les jours de sa vie. 
${}^{15}Il fit apporter dans la maison du Seigneur les objets sacrés de son père et ceux qu’il avait lui-même consacrés : l’argent, l’or et les ustensiles.
${}^{16}Il y eut la guerre entre Asa et Baasa, roi d’Israël, durant toute leur vie. 
${}^{17}Baasa, roi d’Israël, monta contre le royaume de Juda. Il fortifia la ville de Rama pour empêcher quiconque de communiquer avec Asa, roi de Juda. 
${}^{18}Asa prit tout l’argent et l’or qui restaient dans les trésors de la maison du Seigneur, et aussi les trésors de la maison du roi ; il remit le tout aux mains de ses serviteurs. Le roi Asa envoya ceux-ci chez Ben-Hadad, roi d’Aram, qui était le fils de Tabrimmone, fils de Hézyone, et qui résidait à Damas. Il lui fit dire : 
${}^{19}« Il y a une alliance entre moi et toi, comme entre mon père et ton père ! Voici que je t’envoie en cadeau de l’argent et de l’or. Va, romps ton alliance avec Baasa, roi d’Israël : qu’il s’éloigne de moi ! » 
${}^{20}Ben-Hadad écouta le roi Asa. Il envoya les chefs de ses armées contre les villes d’Israël. Il frappa les villes de Yone, Dane, Abel-Beth-Maaka, toute la région de Kinnèreth, et même tout le pays de Nephtali. 
${}^{21}Lorsque Baasa apprit cela, il cessa aussitôt de fortifier Rama et demeura à Tirsa. 
${}^{22}Alors le roi Asa convoqua tout Juda, sans excepter personne. Ils enlevèrent de Rama les pierres et le bois dont Baasa s’était servi pour la construire, et le roi Asa les réemploya pour fortifier Guéba de Benjamin et Mispa.
${}^{23}Le reste de toutes les actions d’Asa, toute sa bravoure,
        \\tout ce qu’il a fait, et les villes qu’il a construites,
        \\cela n’est-il pas écrit dans le livre des Annales des rois de Juda ?
        \\Hormis ceci : au temps de sa vieillesse, il eut les pieds malades.
${}^{24}Asa reposa avec ses pères,
        \\et il fut enseveli avec eux
        \\dans la Cité de David, son ancêtre.
        \\Son fils Josaphat régna à sa place.
${}^{25}Nadab, fils de Jéroboam, devint roi sur Israël, la deuxième année du règne d’Asa, roi de Juda. Il régna deux ans sur Israël. 
${}^{26}Il fit ce qui est mal aux yeux du Seigneur, et il marcha dans le chemin de son père en imitant le péché que celui-ci avait fait commettre à Israël. 
${}^{27}Baasa, fils d’Ahias, de la maison d’Issakar, conspira contre lui. Baasa le frappa à Guibbetone des Philistins, alors que Nadab et tout Israël assiégeaient Guibbetone. 
${}^{28}C’est en la troisième année d’Asa, roi de Juda, que Baasa le mit à mort, et il régna à sa place. 
${}^{29}Dès qu’il fut roi, il frappa toute la maison de Jéroboam, ne laissant âme qui vive ; il les extermina, conformément à la parole que le Seigneur avait dite par l’intermédiaire de son serviteur Ahias de Silo, 
${}^{30}à cause des péchés que Jéroboam avait commis, et des péchés qu’il avait fait commettre à Israël, provoquant l’indignation du Seigneur, Dieu d’Israël.
${}^{31}Le reste des actions de Nadab, tout ce qu’il a fait,
        \\cela n’est-il pas écrit dans le livre des Annales des rois d’Israël ?
${}^{32}Il y eut la guerre entre Asa et Baasa, roi d’Israël, durant toute leur vie.
${}^{33}En la troisième année du règne d’Asa, roi de Juda, Baasa, fils d’Ahias, devint roi sur Israël à Tirsa pour vingt-quatre années. 
${}^{34}Il fit ce qui est mal aux yeux du Seigneur, et il marcha dans le chemin de Jéroboam en imitant le péché que celui-ci avait fait commettre à Israël.
      
         
      \bchapter{}
      \begin{verse}
${}^{1}La parole du Seigneur fut adressée à Jéhu, fils de Hanani, contre Baasa, en ces termes : 
${}^{2}« Alors que je t’avais tiré de la poussière et placé comme chef sur mon peuple Israël, tu as marché dans le chemin de Jéroboam et fait commettre le péché à mon peuple Israël, pour provoquer mon indignation par leurs péchés. 
${}^{3}Voici que je balaierai les derniers restes de Baasa et de sa maison. Je rendrai ta maison pareille à la maison de Jéroboam, fils de Nebath. 
${}^{4}Celui de la famille de Baasa qui mourra dans la ville, les chiens le mangeront ; celui qui mourra dans les champs, l’oiseau du ciel le mangera. »
${}^{5}Le reste des actions de Baasa,
        \\ce qu’il a fait, sa bravoure,
        \\cela n’est-il pas écrit dans le livre des Annales des rois d’Israël ?
${}^{6}Baasa reposa avec ses pères,
        \\et il fut enseveli à Tirsa.
        \\Son fils Éla régna à sa place.
${}^{7}De plus, par l’intermédiaire du prophète Jéhu, fils de Hanani, la parole du Seigneur fut adressée à Baasa et à sa maison. C’était à cause de tout le mal qu’ils avaient fait aux yeux du Seigneur – car ils avaient provoqué son indignation par l’œuvre de leurs mains, au point de ressembler à la maison de Jéroboam –, et aussi parce qu’ils avaient abattu cette maison.
${}^{8}En la vingt-sixième année du règne d’Asa, roi de Juda, Éla, fils de Baasa, devint roi sur Israël à Tirsa, pour deux ans. 
${}^{9}Son serviteur Zimri, commandant de la moitié des chars, conspira contre lui. Alors qu’Éla était à Tirsa, buvant jusqu’à l’ivresse, dans la maison d’Arsa, le maître du palais, 
${}^{10}Zimri entra, le frappa, et il mourut. C’était la vingt-septième année du règne d’Asa, roi de Juda. Et Zimri régna à la place d’Éla. 
${}^{11}Devenu roi, à peine assis sur le trône, il frappa toute la maison de Baasa. Il ne lui laissa subsister ni mâles, ni proches parents, ni compagnon. 
${}^{12}Zimri extermina la maison de Baasa, conformément à la parole que le Seigneur avait dite contre Baasa par l’intermédiaire de Jéhu le prophète ; 
${}^{13}il fit cela à cause de tous les péchés de Baasa et de son fils Éla, ceux qu’ils avaient commis, et ceux qu’ils avaient fait commettre à Israël, au point de provoquer l’indignation du Seigneur, Dieu d’Israël, par leurs vaines idoles.
${}^{14}Le reste des actions d’Éla, tout ce qu’il a fait,
        \\cela n’est-il pas écrit dans le livre des Annales des rois d’Israël ?
${}^{15}En la vingtième année du règne d’Asa, roi de Juda, Zimri régna sept jours à Tirsa. Le peuple campait alors autour de Guibbetone des Philistins. 
${}^{16}Le peuple qui campait entendit la nouvelle : « Zimri a conspiré ; il a frappé le roi. » Alors, le jour même, dans le camp, tout Israël établit Omri, chef de l’armée, comme roi sur Israël. 
${}^{17}Omri, ayant tout Israël avec lui, monta de Guibbetone, et ils assiégèrent Tirsa. 
${}^{18}Quand Zimri s’aperçut que la ville était prise, il entra dans le donjon de la maison du roi, à laquelle il mit le feu et où il mourut. 
${}^{19}Ce fut à cause des péchés qu’il avait commis : il avait fait ce qui est mal aux yeux du Seigneur, il avait marché dans le chemin de Jéroboam et imité le péché que celui-ci avait commis en faisant pécher tout Israël.
${}^{20}Le reste des actions de Zimri
        \\et le complot qu’il avait tramé,
        \\cela n’est-il pas écrit dans le livre des Annales des rois d’Israël ?
${}^{21}Alors le peuple d’Israël se divisa en deux parties : une partie du peuple suivit Tibni, fils de Guinath, pour le faire roi, et l’autre partie suivit Omri. 
${}^{22}Le peuple qui suivait Omri l’emporta sur celui qui suivait Tibni, fils de Guinath. Tibni mourut, et Omri devint roi.
${}^{23}En la trente et unième année du règne d’Asa, roi de Juda, Omri devint roi sur Israël, pour douze années, et il régna six ans à Tirsa. 
${}^{24}Puis il acheta le mont de Samarie à Sémer, au prix de deux lingots d’argent. Il fortifia la montagne et donna à la ville qu’il avait bâtie le nom de Samarie, du nom de Sémer, le maître de la montagne. 
${}^{25}Omri fit ce qui est mal aux yeux du Seigneur ; il agit plus mal encore que tous ceux qui l’avaient précédé. 
${}^{26}Il marcha dans tous les chemins de Jéroboam, fils de Nebath, en imitant le péché que celui-ci avait fait commettre à Israël, au point de provoquer l’indignation du Seigneur, Dieu d’Israël, par leurs vaines idoles.
${}^{27}Le reste des actions d’Omri,
        \\ce qu’il a fait, ses actes de bravoure,
        \\cela n’est-il pas écrit dans le livre des Annales des rois d’Israël ?
${}^{28}Omri reposa avec ses pères,
        \\et il fut enseveli à Samarie.
        \\Son fils Acab régna à sa place.
${}^{29}Acab, fils d’Omri, devint donc roi sur Israël, la trente-huitième année du règne d’Asa, roi de Juda. Acab, fils d’Omri, régna sur Israël, à Samarie, pendant vingt-deux années. 
${}^{30}Acab, fils d’Omri, fit ce qui est mal aux yeux du Seigneur, plus encore que tous ceux qui l’avaient précédé. 
${}^{31}Son moindre tort fut d’imiter les péchés de Jéroboam, fils de Nebath, car il prit pour femme Jézabel, fille d’Ethbaal, roi des Sidoniens, et il alla servir Baal et se prosterner devant lui. 
${}^{32}Il lui dressa un autel, dans le temple de Baal qu’il avait construit à Samarie. 
${}^{33}Acab fabriqua aussi le Poteau sacré d’Ashéra. Par ses actions Acab ne cessa de provoquer l’indignation du Seigneur, Dieu d’Israël, plus encore que tous les rois d’Israël qui l’avaient précédé. 
${}^{34}C’est de son temps que Hiël de Béthel rebâtit Jéricho : au prix d’Abiram, son premier-né, il en posa les fondations ; au prix de Segoub, son cadet, il en fixa les portes, conformément à la malédiction que le Seigneur avait dite par l’intermédiaire de Josué, fils de Noun.
      
         
      \bchapter{}
      \begin{verse}
${}^{1}Le prophète Élie\\, de Tishbé en Galaad, dit au roi Acab : « Par le Seigneur qui est vivant, par le Dieu d’Israël dont je suis le serviteur\\, pendant plusieurs années il n’y aura pas de rosée ni de pluie, à moins que j’en donne l’ordre. »
${}^{2}La parole du Seigneur lui fut adressée : 
${}^{3} « Va-t’en d’ici, dirige-toi vers l’est, et cache-toi près du torrent de Kérith, qui se jette dans le Jourdain. 
${}^{4} Tu boiras au torrent, et j’ordonne aux corbeaux de t’apporter ta nourriture. » 
${}^{5} Le prophète fit ce que le Seigneur lui avait dit, et alla s’établir près du torrent de Kérith, qui se jette dans le Jourdain. 
${}^{6} Les corbeaux lui apportaient du pain et de la viande, matin et soir, et le prophète buvait au torrent.
${}^{7}Au bout d’un certain temps, il ne tombait plus une goutte de pluie dans tout le pays, et le torrent où buvait le prophète finit par être à sec. 
${}^{8} Alors la parole du Seigneur lui fut adressée : 
${}^{9} « Lève-toi, va à Sarepta, dans le pays de Sidon ; tu y habiteras ; il y a là une veuve que j’ai chargée de te nourrir. » 
${}^{10} Le prophète Élie\\partit pour Sarepta, et il parvint à l’entrée de la ville. Une veuve ramassait du bois ; il l’appela et lui dit : « Veux-tu me puiser, avec ta cruche, un peu d’eau pour que je boive ? » 
${}^{11} Elle alla en puiser. Il lui dit encore : « Apporte-moi aussi un morceau de pain. » 
${}^{12} Elle répondit : « Je le jure par la vie du Seigneur ton Dieu : je n’ai pas de pain. J’ai seulement, dans une jarre, une poignée de farine, et un peu d’huile dans un vase. Je ramasse deux morceaux de bois, je rentre préparer pour moi et pour mon fils ce qui nous reste. Nous le mangerons, et puis nous mourrons. » 
${}^{13} Élie lui dit alors : « N’aie pas peur, va, fais ce que tu as dit. Mais d’abord cuis-moi une petite galette\\et apporte-la moi, ensuite tu en feras pour toi et ton fils. 
${}^{14} Car ainsi parle le Seigneur, Dieu d’Israël :
        \\Jarre de farine point ne s’épuisera,
        \\vase d’huile point ne se videra,
        \\jusqu’au jour où le Seigneur
        \\donnera la pluie pour arroser la terre. »
${}^{15}La femme alla faire ce qu’Élie lui avait demandé, et pendant longtemps, le prophète, elle-même et son fils eurent à manger. 
${}^{16} Et la jarre de farine ne s’épuisa pas, et le vase d’huile ne se vida pas, ainsi que le Seigneur l’avait annoncé par l’intermédiaire d’Élie.
${}^{17}Après cela, le fils de la femme chez qui habitait Élie tomba malade ; le mal fut si violent que l’enfant expira. 
${}^{18} Alors la femme dit à Élie : « Que me veux-tu\\, homme de Dieu ? Tu es venu chez moi pour rappeler mes fautes et faire mourir mon fils ! » 
${}^{19} Élie répondit : « Donne-moi ton fils ! » Il le prit des bras de sa mère, le porta dans sa chambre en haut de la maison et l’étendit sur son lit. 
${}^{20} Puis il invoqua le Seigneur : « Seigneur, mon Dieu, cette veuve chez qui je loge, lui veux-tu du mal jusqu’à faire mourir son fils ? » 
${}^{21} Par trois fois, il s’étendit sur l’enfant en invoquant le Seigneur : « Seigneur, mon Dieu, je t’en supplie, rends la vie à cet enfant ! » 
${}^{22} Le Seigneur entendit la prière d’Élie ; le souffle de l’enfant revint en lui : il était vivant ! 
${}^{23} Élie prit alors l’enfant, de sa chambre il le descendit dans la maison, le remit à sa mère et dit : « Regarde, ton fils est vivant ! » 
${}^{24} La femme lui répondit : « Maintenant je sais que tu es un homme de Dieu, et que, dans ta bouche, la parole du Seigneur est véridique. »
      
         
      \bchapter{}
      \begin{verse}
${}^{1}De nombreux jours s’écoulèrent, et la parole du Seigneur fut adressée à Élie, la troisième année, en ces termes : « Va te présenter devant Acab ; je vais envoyer la pluie sur la surface du sol. » 
${}^{2}Élie partit pour se présenter devant Acab. La famine s’aggravait alors à Samarie. 
${}^{3}Acab appela Abdias, le maître du palais. Or, Abdias craignait beaucoup le Seigneur. 
${}^{4}Ainsi, lorsque Jézabel avait supprimé les prophètes du Seigneur, Abdias avait pris cent prophètes, les avait cachés, cinquante à la fois dans une grotte, et les avait approvisionnés en pain et en eau. 
${}^{5}Acab dit à Abdias : « Va dans le pays, vers toutes les sources d’eau et vers tous les torrents ; peut-être trouverons-nous de l’herbe pour maintenir en vie chevaux et mulets ; et nous n’aurons pas à supprimer une partie des bêtes. » 
${}^{6}Ils se partagèrent le pays pour le parcourir. Acab alla seul par un chemin, et Abdias alla seul par un autre chemin. 
${}^{7}Tandis qu’Abdias était en chemin, voici qu’Élie vint à sa rencontre. Abdias le reconnut et tomba face contre terre. Il dit : « Est-ce bien toi, mon seigneur Élie ? » 
${}^{8}Il lui répondit : « C’est moi ! Va dire à ton maître : “Voici Élie” ! » 
${}^{9}Abdias reprit : « En quoi ai-je péché, pour que tu me livres, moi ton serviteur, aux mains d’Acab, et pour qu’il me fasse mourir ? 
${}^{10}Par la vie du Seigneur ton Dieu ! Il n’y a pas une nation, pas un royaume, où mon seigneur Acab ne t’ait envoyé chercher ! Quand on lui disait : “Il n’est pas ici”, il faisait jurer à ce royaume et à cette nation qu’on ne t’avait pas trouvé. 
${}^{11}Et maintenant, tu me dis : Va dire à ton maître : “Voici Élie” ! 
${}^{12}Mais dès que je t’aurai quitté, l’Esprit du Seigneur t’emportera je ne sais où ; moi, j’irai informer Acab, qui ne te trouvera pas, et il me tuera. Pourtant, ton serviteur craint le Seigneur depuis sa jeunesse ! 
${}^{13}N’a-t-on pas rapporté à mon seigneur Élie ce que j’ai fait lorsque Jézabel tuait les prophètes du Seigneur : comment j’ai caché cent des prophètes du Seigneur, cinquante par cinquante, dans des grottes, et comment je les ai approvisionnés en pain et en eau ? 
${}^{14}Et maintenant tu dis : “Va dire à ton maître : Voici Élie !” Mais il me tuera ! » 
${}^{15}Élie déclara : « Par la vie du Seigneur de l’univers devant qui je me tiens ! Aujourd’hui même, je me présenterai devant lui ! » 
${}^{16}Abdias partit donc à la rencontre d’Acab ; il l’informa, et Acab vint à la rencontre d’Élie. 
${}^{17}Quand Acab vit Élie, il lui dit : « Est-ce bien toi, porte-malheur d’Israël ? » 
${}^{18}Élie répondit : « Ce n’est pas moi qui porte malheur à Israël ; c’est toi et la maison de ton père, parce que vous avez abandonné les commandements du Seigneur et que tu as suivi les Baals. 
${}^{19}Et maintenant, convoque et réunis tout Israël près de moi sur le mont Carmel, avec les quatre cent cinquante prophètes de Baal et les quatre cents prophètes d’Ashéra qui mangent à la table de Jézabel. »
      
         
${}^{20}Acab convoqua tout Israël et réunit les prophètes sur le mont Carmel. 
${}^{21}Élie se présenta devant la foule et dit : « Combien de temps allez-vous danser pour l’un et pour l’autre ? Si c’est le Seigneur qui est Dieu\\, suivez le Seigneur ; si c’est Baal, suivez Baal. » Et la foule ne répondit mot. 
${}^{22}Élie continua : « Moi, je suis le seul qui reste des prophètes du Seigneur, tandis que les prophètes de Baal sont quatre cent cinquante. 
${}^{23}Amenez-nous deux jeunes taureaux ; qu’ils en choisissent un, qu’ils le dépècent et le placent sur le bûcher, mais qu’ils n’y mettent pas le feu. Moi, je préparerai l’autre taureau, je le placerai sur le bûcher, mais je n’y mettrai pas le feu. 
${}^{24}Vous invoquerez le nom de votre dieu, et moi, j’invoquerai le nom du Seigneur : le dieu qui répondra par le feu, c’est lui qui est Dieu. » La foule répondit : « C’est d’accord. » 
${}^{25}Élie dit alors aux prophètes de Baal : « Choisissez votre taureau et commencez, car vous êtes les plus nombreux. Invoquez le nom de votre dieu, mais ne mettez pas le feu. » 
${}^{26}Ils prirent le taureau et le préparèrent, et ils invoquèrent le nom de Baal depuis le matin jusqu’au milieu du jour, en disant : « Ô Baal, réponds-nous ! » Mais il n’y eut ni voix ni réponse ; et ils dansaient devant l’autel qu’ils avaient dressé. 
${}^{27}Au milieu du jour, Élie se moqua d’eux en disant : « Criez plus fort, puisque c’est un dieu : il a des soucis ou des affaires, ou bien il est en voyage ; il dort peut-être, mais il va se réveiller ! » 
${}^{28}Ils crièrent donc plus fort et, selon leur coutume, ils se tailladèrent jusqu’au sang avec des épées et des lances. 
${}^{29}Dans l’après-midi, ils se livrèrent à des transes prophétiques jusqu’à l’heure du sacrifice du soir, mais il n’y eut ni voix, ni réponse, ni le moindre signe. 
${}^{30}Alors Élie dit à la foule : « Approchez. » Et toute la foule s’approcha de lui. Il releva l’autel du Seigneur, qui avait été démoli. 
${}^{31}Il prit douze pierres, selon le nombre des tribus des fils de Jacob à qui le Seigneur avait dit : « Ton nom sera Israël. » 
${}^{32}Avec ces pierres il érigea un autel au Seigneur. Il creusa autour de l’autel une rigole d’une capacité d’environ trente litres\\. 
${}^{33}Il disposa le bois, dépeça le taureau et le plaça sur le bûcher. 
${}^{34}Puis il dit : « Emplissez d’eau quatre cruches, et versez-les sur la victime et sur le bois. » Et l’on fit ainsi. Il dit : « Une deuxième fois ! » et l’on recommença. Il dit : « Une troisième fois ! » et l’on recommença encore. 
${}^{35}L’eau ruissela autour de l’autel, et la rigole elle-même fut remplie d’eau. 
${}^{36}À l’heure du sacrifice du soir, Élie le prophète s’avança et dit : « Seigneur, Dieu d’Abraham, d’Isaac et d’Israël, on saura aujourd’hui que tu es Dieu en Israël, que je suis ton serviteur, et que j’ai accompli toutes ces choses sur ton ordre. 
${}^{37}Réponds-moi, Seigneur, réponds-moi, pour que tout ce peuple sache que c’est toi, Seigneur, qui es Dieu, et qui as retourné leur cœur ! » 
${}^{38}Alors le feu du Seigneur tomba, il dévora la victime et le bois, les pierres et la poussière, et l’eau qui était dans la rigole. 
${}^{39}Tout le peuple en fut témoin ; les gens tombèrent face contre terre et dirent : « C’est le Seigneur qui est Dieu ! C’est le Seigneur qui est Dieu\\ ! » 
${}^{40}Élie leur dit alors : « Saisissez les prophètes de Baal : que pas un seul ne s’échappe ! » Ils les saisirent. Élie les fit descendre au ravin du Qishone, et là il les égorgea.
${}^{41}Le prophète Élie dit au roi Acab : « Monte, tu peux maintenant manger et boire, car j’entends le grondement de la pluie. » 
${}^{42} Acab monta pour aller manger et boire. Élie, de son côté, monta sur le sommet du Carmel, il se courba vers la terre et mit son visage entre ses genoux. 
${}^{43} Il dit à son serviteur : « Monte, et regarde du côté de la mer. » Le serviteur monta, regarda et dit : « Il n’y a rien. » Sept fois de suite, Élie lui dit : « Retourne. » 
${}^{44} La septième fois, le serviteur annonça : « Voilà un nuage qui monte de la mer, gros comme le poing. » Alors Élie dit au serviteur : « Va dire au roi Acab : “Attelle ton char et descends de la montagne, avant d’être arrêté par la pluie.” » 
${}^{45} Peu à peu, le ciel s’obscurcit de nuages, poussés par le vent, et il tomba une grosse pluie. Acab monta sur son char et partit pour la ville de Yizréel. 
${}^{46} La main du Seigneur s’empara du prophète ; Élie retroussa son vêtement et courut en avant d’Acab jusqu’à l’entrée de la ville de Yizréel.
      
         
      \bchapter{}
      \begin{verse}
${}^{1}Le roi Acab avait rapporté à Jézabel comment le prophète Élie avait réagi et comment il avait fait égorger tous les prophètes de Baal. 
${}^{2} Alors Jézabel envoya un messager dire à Élie : « Que les dieux amènent le malheur sur moi, et pire encore\\, si demain, à cette heure même, je ne t’inflige pas le même sort que tu as infligé à ces prophètes. » 
${}^{3} Devant cette menace, Élie se hâta de partir pour sauver sa vie. Arrivé à Bershéba, au royaume de Juda, il y laissa son serviteur. 
${}^{4} Quant à lui, il marcha toute une journée dans le désert. Il vint s’asseoir à l’ombre d’un buisson, et demanda la mort en disant : « Maintenant, Seigneur, c’en est trop ! Reprends ma vie : je ne vaux pas mieux que mes pères. » 
${}^{5} Puis il s’étendit sous le buisson, et s’endormit. Mais voici qu’un ange le toucha et lui dit : « Lève-toi, et mange ! » 
${}^{6} Il regarda, et il y avait près de sa tête une galette\\cuite sur des pierres brûlantes\\et une cruche d’eau. Il mangea, il but, et se rendormit. 
${}^{7} Une seconde fois, l’ange du Seigneur le toucha et lui dit : « Lève-toi, et mange, car il est long, le chemin qui te reste. » 
${}^{8} Élie se leva, mangea et but. Puis, fortifié par cette nourriture, il marcha quarante jours et quarante nuits jusqu’à l’Horeb, la montagne de Dieu.
      
         
${}^{9}Là, il entra dans une\\caverne et y passa la nuit. Et voici que la parole du Seigneur lui fut adressée. Il lui dit : « Que fais-tu là, Élie ? » 
${}^{10}Il répondit : « J’éprouve une ardeur jalouse pour toi, Seigneur, Dieu de l’univers. Les fils d’Israël ont abandonné ton Alliance, renversé tes autels, et tué tes prophètes par l’épée ; moi, je suis le seul à être resté et ils cherchent à prendre ma vie. » 
${}^{11}Le Seigneur dit : « Sors et tiens-toi sur la montagne devant le Seigneur, car il va passer. » À l’approche du Seigneur, il y eut un ouragan, si fort et si violent qu’il fendait les montagnes et brisait les rochers, mais le Seigneur n’était pas dans l’ouragan ; et après l’ouragan, il y eut un tremblement de terre, mais le Seigneur n’était pas dans le tremblement de terre ; 
${}^{12}et après ce tremblement de terre, un feu, mais le Seigneur n’était pas dans ce feu ; et après ce feu, le murmure d’une brise légère\\. 
${}^{13}Aussitôt qu’il l’entendit, Élie se couvrit le visage\\avec son manteau, il sortit et se tint à l’entrée de la caverne. Alors il entendit une voix qui disait : « Que fais-tu là, Élie ? » 
${}^{14}Il répondit : « J’éprouve une ardeur jalouse pour toi, Seigneur, Dieu de l’univers. Les fils d’Israël ont abandonné ton Alliance, renversé tes autels, et tué tes prophètes par l’épée ; moi, je suis le seul à être resté et ils cherchent à prendre ma vie. »
${}^{15}Le Seigneur lui dit : « Repars vers Damas, par le chemin du désert. Arrivé là, tu consacreras par l’onction Hazaël comme roi de Syrie ; 
${}^{16}puis tu consacreras Jéhu, fils de Namsi\\, comme roi d’Israël ; <a class="anchor verset_lettre" id="bib_1r_19_16_b"/>et tu consacreras Élisée, fils de Shafath, d’Abel-Mehola, comme prophète pour te succéder. 
${}^{17}Celui qui échappera à l’épée d’Hazaël, Jéhu le tuera, et celui qui échappera à l’épée de Jéhu, Élisée le tuera. 
${}^{18}Mais je garderai en Israël un reste de sept mille hommes : tous les genoux qui n’auront pas fléchi devant Baal et toutes les bouches qui ne lui auront pas donné de baiser ! »
${}^{19}Élie s’en alla. Il trouva Élisée, fils de Shafath, en train de labourer. Il avait à labourer douze arpents, et il en était au douzième. Élie passa près de lui et jeta vers lui son manteau\\. 
${}^{20}Alors Élisée quitta ses bœufs, courut derrière Élie, et lui dit : « Laisse-moi embrasser mon père et ma mère, puis je te suivrai. » Élie répondit : « Va-t’en, retourne là-bas ! Je n’ai rien fait\\. » 
${}^{21}Alors Élisée s’en retourna ; mais il prit la paire de bœufs pour les immoler, les fit cuire avec le bois de l’attelage, et les donna à manger aux gens. Puis il se leva, partit à la suite d’Élie et se mit à son service.
      
         
      \bchapter{}
      \begin{verse}
${}^{1}Ben-Hadad, roi d’Aram, réunit toute son armée. Il avait avec lui trente-deux rois, des chevaux et des chars. Il monta assiéger Samarie et l’attaqua. 
${}^{2}Il envoya dans la ville des messagers à Acab, roi d’Israël, 
${}^{3}pour lui dire : « Ainsi parle Ben-Hadad : “Ton argent et ton or sont à moi ! À moi, tes femmes et les meilleurs de tes fils !” » 
${}^{4}Le roi d’Israël répondit : « Selon tes ordres, mon seigneur le roi, je suis à toi, moi et tout ce qui m’appartient. »
${}^{5}Les messagers revinrent et dirent à nouveau : « Ainsi parle Ben-Hadad. Je t’ai envoyé ce message : “Ton argent, ton or, tes femmes et tes fils, tu me les donneras !” 
${}^{6}Eh bien, demain à la même heure, je t’enverrai mes serviteurs ; ils fouilleront ta maison et les maisons de tes serviteurs. Ils mettront la main sur tout ce qui réjouit tes yeux et ils l’emporteront ! »
${}^{7}Le roi d’Israël convoqua tous les anciens du pays et dit : « Reconnaissez-le ! Vous voyez bien que cet homme nous veut du mal. Quand il m’a réclamé mes femmes et mes fils, mon argent et mon or, je ne lui ai rien refusé ! » 
${}^{8}Tous les anciens et tout le peuple lui dirent : « Ne l’écoute pas ! N’accepte pas ! » 
${}^{9}Il répondit aux messagers de Ben-Hadad : « Dites à mon seigneur le roi : “Tout ce que tu as fait demander à ton serviteur la première fois, je le ferai. Mais cette exigence, je ne peux la satisfaire.” » Les messagers s’en allèrent et rapportèrent au roi la réponse.
${}^{10}Ben-Hadad lui envoya dire : « Que les dieux amènent le malheur sur moi, et pire encore, s’il reste à Samarie assez de poussière pour que tous les gens qui me suivent en aient une poignée. » 
${}^{11}Le roi d’Israël lui fit répondre : « Dites-lui : “Celui qui boucle son ceinturon, qu’il ne crie pas victoire comme celui qui le détache !” » 
${}^{12}Quand il entendit cette parole, Ben-Hadad, qui était en train de boire avec les rois sous les tentes, dit à ses serviteurs : « À vos postes ! », et ils prirent position contre la ville.
${}^{13}Voici qu’un prophète s’avança au-devant d’Acab, roi d’Israël, et lui dit : « Ainsi parle le Seigneur : As-tu vu cette grande multitude ? Voici que je la livre aujourd’hui dans ta main, et tu reconnaîtras que je suis le Seigneur ! » 
${}^{14}Acab demanda : « Par qui me la livres-tu ? » Il répondit : « Ainsi parle le Seigneur : Par l’élite des chefs de districts. » Acab reprit : « Qui engagera le combat ? » Le prophète répondit : « Toi ! »
${}^{15}Acab passa en revue l’élite des chefs de districts : ils étaient deux cent quarante-deux. Après eux, il passa en revue tout le peuple, tous les fils d’Israël : sept mille hommes. 
${}^{16}Ils firent une sortie à l’heure de midi, tandis que Ben-Hadad, sous les tentes, buvait jusqu’à l’ivresse avec les rois, les trente-deux rois, ses alliés. 
${}^{17}L’élite des chefs de districts sortit d’abord. Ben-Hadad envoya aux nouvelles ; on lui fit ce rapport : « Des hommes sont sortis de Samarie. » 
${}^{18}Il répondit : « S’ils sortent pour la paix, prenez-les vivants. S’ils sortent pour le combat, prenez-les vivants aussi. » 
${}^{19}Ceux qui étaient sortis de la ville, c’était l’élite des chefs de districts, et l’armée venait après eux. 
${}^{20}Et ils frappèrent chacun son homme. Les Araméens s’enfuirent, et Israël les poursuivit. Ben-Hadad, roi d’Aram, se sauva à cheval avec quelques cavaliers. 
${}^{21}Le roi d’Israël sortit, il frappa les chevaux et les chars, infligeant à Aram une grande défaite. 
${}^{22}Le prophète s’avança vers le roi d’Israël. Il lui dit : « Va ! Montre-toi courageux ! Considère bien ce que tu dois faire, car au retour du printemps le roi d’Aram montera contre toi. »
${}^{23}Les serviteurs du roi d’Aram lui dirent : « Leur dieu est un dieu des montagnes, c’est pourquoi ils l’ont emporté sur nous. Mais combattons-les dans la plaine, et, à coup sûr, nous l’emporterons sur eux. 
${}^{24}Fais donc ceci : relève de son poste chacun des rois et mets à leur place des gouverneurs. 
${}^{25}Pour toi, recrute une armée aussi puissante que celle qui est tombée à tes côtés : cheval pour cheval et char pour char ; et livrons-leur bataille dans la plaine. Alors, à coup sûr, nous l’emporterons sur eux. » Il écouta leur avis et fit ainsi.
${}^{26}Au retour du printemps, Ben-Hadad passa donc en revue les Araméens. Puis il monta vers Apheq pour livrer bataille à Israël. 
${}^{27}De leur côté, les fils d’Israël furent passés en revue et approvisionnés, puis ils marchèrent à la rencontre des Araméens. Les fils d’Israël campèrent en face d’eux, disposés comme deux petits troupeaux de chèvres, alors que les Araméens remplissaient le pays. 
${}^{28}L’homme de Dieu s’avança et dit au roi d’Israël : « Ainsi parle le Seigneur : Parce que les Araméens ont dit : “Le Seigneur est un dieu des montagnes, et non un dieu des vallées,” je livrerai entre tes mains cette grande multitude, et vous reconnaîtrez que je suis le Seigneur. » 
${}^{29}Ils campèrent les uns en face des autres, durant sept jours. Le septième jour, le combat s’engagea, et les fils d’Israël battirent les Araméens : cent mille fantassins en un seul jour. 
${}^{30}Ceux qui restaient s’enfuirent dans la ville d’Apheq, mais le rempart s’écroula sur ces vingt-sept mille hommes qui restaient. Et Ben-Hadad s’enfuit. Il entra dans la ville et se réfugia dans une chambre retirée.
${}^{31}Ses serviteurs lui dirent : « Voici ! Nous avons entendu dire que les rois de la maison d’Israël sont des rois qui font miséricorde. Permets que nous mettions de la toile à sac autour de nos reins, et des cordes autour de nos têtes ; puis nous sortirons au-devant du roi d’Israël. Peut-être te laissera-t-il la vie sauve ? » 
${}^{32}Ils serrèrent de la toile à sac sur leurs reins et des cordes autour de leurs têtes. Puis ils se rendirent auprès du roi d’Israël. Ils lui dirent : « Ton serviteur Ben-Hadad a dit : “Pourrais-je avoir la vie sauve ?” » Il répondit : « Est-il encore vivant ? Il est mon frère ! » 
${}^{33}Les hommes virent là un bon présage ; ils se hâtèrent de le prendre au mot et dirent à leur tour : « Ben-Hadad est ton frère. » Acab reprit : « Allez le chercher. » Ben-Hadad sortit vers lui et celui-ci le fit monter sur son char. 
${}^{34}Ben-Hadad lui dit : « Les villes que mon père a prises à ton père, je les restitue, tu auras à Damas des rues pour le commerce, tout comme en possédait mon père à Samarie. Acab répondit : « Et moi, je te laisserai aller si nous faisons alliance. » Il conclut donc avec lui une alliance et le laissa partir.
${}^{35}Par ordre du Seigneur, un des frères-prophètes dit à celui qui l’accompagnait : « Frappe-moi ! » Mais l’autre refusa de le frapper. 
${}^{36}Il lui dit : « Parce que tu n’as pas écouté la voix du Seigneur, dès que tu m’auras quitté, un lion te frappera. » L’autre s’était à peine éloigné que le lion le rencontra et le frappa. 
${}^{37}Le prophète rencontra un autre homme et lui dit : « Frappe-moi ! » L’autre le frappa et le blessa. 
${}^{38}Le prophète alla se poster, guettant le roi sur le chemin. Il s’était rendu méconnaissable, un bandeau sur les yeux. 
${}^{39}Comme le roi passait, il lui cria : « Ton serviteur était sorti pour s’engager dans la bataille. Soudain quelqu’un, s’écartant du combat, m’a amené un homme en disant : “Surveille cet homme ! S’il vient à disparaître, ta vie répondra pour sa vie, ou bien tu paieras la valeur d’un lingot d’argent”. 
${}^{40}Or, ton serviteur s’est occupé ici et là, et l’homme n’y était plus ! » Le roi d’Israël lui dit : « Voilà ta sentence ! Tu l’as rendue toi-même ! » 
${}^{41}Aussitôt l’homme enleva de ses yeux le bandeau, et le roi d’Israël s’aperçut que c’était un des prophètes. 
${}^{42}Celui-ci lui dit : « Ainsi parle le Seigneur : Parce que tu as laissé échapper de ta main l’homme que j’avais voué à l’anathème, ta vie répondra pour sa vie, et ton peuple pour son peuple. » 
${}^{43}Puis le roi d’Israël s’en retourna chez lui, sombre et irrité ; il rentra à Samarie.
      
         
      \bchapter{}
      \begin{verse}
${}^{1}Naboth, de la ville de Yizréel, possédait une vigne à côté du palais d’Acab, roi de Samarie. 
${}^{2} Acab dit un jour à Naboth : « Cède-moi ta vigne ; elle me servira de jardin potager, car elle est juste à côté de ma maison ; je te donnerai en échange une vigne meilleure, ou, si tu préfères, je te donnerai l’argent qu’elle vaut. » 
${}^{3} Naboth répondit à Acab : « Que le Seigneur me préserve de te céder l’héritage de mes pères ! »
${}^{4}Acab retourna chez lui sombre et irrité, parce que Naboth lui avait dit : « Je ne te céderai pas l’héritage de mes pères. » Il se coucha sur son lit, tourna son visage vers le mur, et refusa de manger. 
${}^{5} Sa femme Jézabel vint lui dire : « Pourquoi es-tu de mauvaise humeur ? Pourquoi ne veux-tu pas manger ? » 
${}^{6} Il répondit : « J’ai parlé à Naboth de Yizréel. Je lui ai dit : “Cède-moi ta vigne pour de l’argent, ou, si tu préfères, pour une autre vigne en échange.” Mais il a répondu : “Je ne te céderai pas ma vigne !” » 
${}^{7} Alors sa femme Jézabel lui dit : « Est-ce que tu es le roi d’Israël, oui ou non ? Lève-toi, mange, et retrouve ta bonne humeur : moi, je vais te donner la vigne de Naboth. »
${}^{8}Elle écrivit des lettres au nom d’Acab, elle les scella du sceau royal, et elle les adressa aux anciens et aux notables de la ville où habitait Naboth. 
${}^{9} Elle avait écrit dans ces lettres : « Proclamez un jeûne, faites comparaître Naboth devant le peuple. 
${}^{10} Placez en face de lui deux vauriens\\, qui témoigneront contre lui : “Tu as maudit Dieu et le roi !” Ensuite, faites-le sortir de la ville, lapidez-le, et qu’il meure ! »
${}^{11}Les\\anciens et les notables qui habitaient la ville de Naboth firent ce que Jézabel avait ordonné dans ses lettres. 
${}^{12} Ils proclamèrent un jeûne et firent comparaître Naboth devant le peuple. 
${}^{13} Alors arrivèrent les deux individus qui se placèrent en face de lui et portèrent contre lui ce témoignage : « Naboth a maudit Dieu et le roi. » On fit sortir Naboth de la ville, on le lapida, et il mourut. 
${}^{14} Puis on envoya dire à Jézabel : « Naboth a été lapidé et il est mort. » 
${}^{15} Lorsque Jézabel en fut informée\\, elle dit à Acab : « Va, prends possession de la vigne de ce Naboth qui a refusé de la céder pour de l’argent, car il n’y a plus de Naboth : il est mort. » 
${}^{16} Quand Acab apprit que Naboth était mort, il se rendit à la vigne de Naboth et en prit possession.
${}^{17}La parole du Seigneur fut adressée au prophète Élie de Tishbé : 
${}^{18} « Lève-toi, va trouver Acab, qui règne sur Israël à Samarie. Il est en ce moment dans la vigne de Naboth, où il s’est rendu pour en prendre possession. 
${}^{19} Tu lui diras : “Ainsi parle le Seigneur : Tu as commis un meurtre, et maintenant tu prends possession. C’est pourquoi, ainsi parle le Seigneur : À l’endroit même où les chiens ont lapé le sang de Naboth, les chiens laperont ton sang à toi aussi.” » 
${}^{20} Acab dit à Élie : « Tu m’as donc retrouvé, toi, mon ennemi ! » Élie répondit : « Oui, je t’ai retrouvé. Puisque tu t’es déshonoré\\en faisant ce qui est mal aux yeux du Seigneur, 
${}^{21} je vais faire venir sur toi le malheur : je supprimerai ta descendance, j’exterminerai tous les mâles de ta maison\\, esclaves ou hommes libres en Israël. 
${}^{22} Je ferai à ta maison ce que j’ai fait à celle de Jéroboam,\\fils de Nebath, et à celle de Baasa, fils d’Ahias, tes prédécesseurs\\, car tu as provoqué ma colère et fait pécher Israël. 
${}^{23} Et le Seigneur a encore cette parole contre Jézabel : “Les chiens dévoreront Jézabel sous les murs de la ville de Yizréel !” 
${}^{24} Celui de la maison d’Acab qui mourra dans la ville sera dévoré par les chiens ; celui qui mourra dans la campagne sera dévoré par les oiseaux du ciel. »
${}^{25}On n’a jamais vu personne se déshonorer comme Acab en faisant comme lui ce qui est mal aux yeux du Seigneur, sous l’influence de sa femme Jézabel. 
${}^{26} Il s’est conduit d’une manière abominable en s’attachant aux idoles, comme faisaient les Amorites que le Seigneur avait chassés devant les Israélites.
${}^{27}Quand Acab entendit les paroles prononcées par Élie, il déchira ses habits, se couvrit le corps d’une toile à sac – un vêtement de pénitence\\ – ; et il jeûnait, il gardait la toile à sac pour dormir, et il marchait lentement. 
${}^{28} Alors la parole du Seigneur fut adressée à Élie\\ : 
${}^{29} « Tu vois comment Acab s’est humilié devant moi ! Puisqu’il s’est humilié devant moi, je ne ferai pas venir le malheur de son vivant ; c’est sous le règne de son fils que je ferai venir le malheur sur sa maison. »
      
         
      \bchapter{}
      \begin{verse}
${}^{1}On resta tranquille pendant trois ans, sans guerre entre Aram et Israël. 
${}^{2}Mais la troisième année, Josaphat, roi de Juda, descendit auprès du roi d’Israël. 
${}^{3}Le roi d’Israël dit à ses serviteurs : « Savez-vous que la ville de Ramoth-de-Galaad nous appartient ? Nous restons sans rien dire, au lieu de la reprendre des mains du roi d’Aram ». 
${}^{4}Il dit à Josaphat : « Viendrais-tu avec moi pour combattre à Ramoth-de-Galaad ? » Et Josaphat répondit au roi d’Israël : « Ce sera pour moi comme pour toi, pour mon peuple comme pour ton peuple, pour mes chevaux comme pour tes chevaux. »
      
         
${}^{5}Josaphat dit au roi d’Israël : « Mais consulte d’abord la parole du Seigneur. » 
${}^{6}Le roi d’Israël réunit les prophètes, au nombre d’environ quatre cents. Il leur demanda : « Irai-je à Ramoth-de-Galaad pour combattre, ou dois-je y renoncer ? » Ils dirent : « Monte ! Le Seigneur livrera la ville aux mains du roi. » 
${}^{7}Mais Josaphat reprit : « N’y a-t-il ici aucun autre prophète du Seigneur, par qui nous pourrions le consulter ? » 
${}^{8}Le roi d’Israël répondit à Josaphat : « Il y a encore un homme par qui nous pourrions consulter le Seigneur, mais moi, je le hais, car il ne prophétise rien de bon à mon sujet, mais seulement du mal. C’est Michée, fils de Yimla. » Josaphat répliqua : « Que le roi ne parle pas ainsi ! » 
${}^{9}Le roi d’Israël appela un dignitaire et lui dit : « Vite, fais venir Michée, fils de Yimla ! »
${}^{10}Le roi d’Israël et Josaphat, roi de Juda, siégeaient, en tenue d’apparat, chacun sur son trône, sur l’esplanade à l’entrée de la porte de Samarie. Et, devant eux, tous les prophètes se mettaient à prophétiser. 
${}^{11}Sédécias, fils de Kenahana, s’était fabriqué des cornes de fer. Il disait : « Ainsi parle le Seigneur : Avec cela tu pourfendras Aram jusqu’à l’exterminer. » 
${}^{12}Tous les prophètes prophétisaient de la même manière ; ils disaient : « Monte à Ramoth-de-Galaad ! Tu réussiras ! Le Seigneur livrera la ville aux mains du roi. »
${}^{13}Le messager qui était allé appeler Michée lui dit : « Voici les paroles des prophètes : d’une seule voix, ils annoncent du bien pour le roi. Que ta parole soit donc conforme à celle de chacun : annonce du bien ! » 
${}^{14}Michée répondit : « Par le Seigneur qui est vivant ! Ce que le Seigneur me dira, c’est cela que j’annoncerai ! » 
${}^{15}Il entra chez le roi qui lui dit : « Michée, irons-nous à Ramoth-de-Galaad pour combattre, ou devons-nous y renoncer ? » Il répondit : « Monte ! Tu réussiras. Le Seigneur livrera la ville aux mains du roi. » 
${}^{16}Le roi lui rétorqua : « Combien de fois devrai-je t’adjurer de me dire seulement la vérité au nom du Seigneur ? » 
${}^{17}Michée dit alors :
        \\« J’ai vu tout Israël dispersé sur les montagnes,
        \\comme des brebis sans berger.
        \\Le Seigneur a dit :
        \\“Ces gens n’ont plus de maître ;
        \\qu’ils retournent en paix, chacun dans sa maison”. »
${}^{18}Le roi d’Israël dit à Josaphat : « Ne te l’avais-je pas dit ? Il ne prophétise à mon sujet rien de bon, mais seulement du mal ! » 
${}^{19}Michée reprit : « Eh bien ! Écoute la parole du Seigneur ! J’ai vu le Seigneur qui siégeait sur son trône ; toute l’armée des cieux se tenait près de lui, à sa droite et à sa gauche. 
${}^{20}Le Seigneur demanda : “Qui séduira Acab, pour qu’il monte et qu’il tombe à Ramoth-de-Galaad ?” L’un répondit ceci, l’autre répondit cela. 
${}^{21}Alors un esprit s’avança et se tint en présence du Seigneur. Il dit : “Moi, je le séduirai”. Le Seigneur reprit : “De quelle manière ?” 
${}^{22}Il répondit : “J’avancerai, je deviendrai esprit de mensonge dans la bouche de tous ses prophètes.” Le Seigneur déclara : “Tu le séduiras, tu l’auras même en ton pouvoir. Avance, et fais comme tu as dit.” » 
${}^{23}Michée continua : « Maintenant donc, voici que le Seigneur a mis un esprit de mensonge dans la bouche de tous tes prophètes qui sont là, voici que le Seigneur annonce contre toi le malheur. »
       
${}^{24}Sédécias, fils de Kenahana, s’approcha et frappa Michée sur la joue, en disant : « Par où l’esprit du Seigneur s’est-il échappé de moi pour te parler ? » 
${}^{25}Michée répondit : « Eh bien ! Le jour où tu fuiras dans une chambre retirée pour te cacher, tu le verras. » 
${}^{26}Le roi d’Israël donna cet ordre : « Saisis-toi de Michée et remets-le aux mains d’Amone, gouverneur de la ville, et à Joas, le fils du roi. 
${}^{27}Tu diras : “Ainsi parle le roi : Mettez cet homme en prison, nourrissez-le de rations réduites de pain et d’eau jusqu’à ce que je revienne sain et sauf”. » 
${}^{28}Michée reprit : « Si vraiment tu reviens sain et sauf, c’est que le Seigneur n’a pas parlé par ma bouche ! » Il ajouta : « Vous, tous les peuples, écoutez ! »
${}^{29}Le roi d’Israël et Josaphat, roi de Juda, montèrent à Ramoth-de-Galaad. 
${}^{30}Le roi d’Israël dit à Josaphat : « Je vais me déguiser pour marcher au combat, mais toi, revêts ta tenue. » Le roi d’Israël se déguisa pour marcher au combat. 
${}^{31}Le roi d’Aram avait donné cet ordre à ses trente-deux commandants de chars : « Vous n’attaquerez ni petit ni grand, mais uniquement le roi d’Israël ! » 
${}^{32}Lorsque les commandants de chars virent Josaphat, ils dirent : « C’est sûrement le roi d’Israël. » Et ils se dirigèrent de son côté pour l’attaquer. Mais Josaphat poussa un cri. 
${}^{33}Alors les commandants de chars virent que ce n’était pas le roi d’Israël et ils se détournèrent de lui. 
${}^{34}Un homme tira de l’arc au hasard et atteignit le roi d’Israël entre les attaches et la cuirasse. Le roi dit au conducteur de son char : « Tourne bride et fais-moi sortir du champ de bataille, car je me sens mal ! » 
${}^{35}Le combat, ce jour-là, devint très violent. On maintenait le roi debout sur son char, face aux Araméens. Et le soir, il mourut. Le sang de sa blessure coulait au fond du char. 
${}^{36}Au coucher du soleil, un cri se propagea dans le campement : « Chacun à sa ville et chacun à sa terre ! » 
${}^{37}Le roi était mort ; il fut ramené à Samarie, et c’est à Samarie que fut enseveli le roi. 
${}^{38}On lava le char à grande eau à l’étang de Samarie. Les chiens lapèrent le sang et les prostituées s’y baignèrent, conformément à la parole que le Seigneur avait dite.
${}^{39}Le reste des actions d’Acab, tout ce qu’il a fait,
        \\la maison d’ivoire qu’il a édifiée
        \\et toutes les villes qu’il a construites,
        \\cela n’est-il pas écrit dans le livre des Annales des rois d’Israël ?
${}^{40}Acab reposa avec ses pères.
        \\Son fils Ocozias régna à sa place.
${}^{41}Josaphat, fils d’Asa, devint roi sur Juda, la quatrième année du règne d’Acab, roi d’Israël. 
${}^{42}Josaphat avait trente-cinq ans lorsqu’il devint roi, et il régna vingt-cinq ans à Jérusalem. Le nom de sa mère était Azouba ; elle était fille de Shilki. 
${}^{43}Il marcha dans tous les chemins d’Asa, son père ; il ne s’en détourna pas, faisant ce qui est droit aux yeux du Seigneur. 
${}^{44}Toutefois les lieux sacrés ne disparurent pas : le peuple offrait encore des sacrifices et brûlait de l’encens dans les lieux sacrés. 
${}^{45}Josaphat fut en paix avec le roi d’Israël.
${}^{46}Le reste des actions de Josaphat,
        \\la bravoure dont il fit preuve,
        \\les guerres qu’il livra,
        \\cela n’est-il pas écrit dans le livre des Annales des rois de Juda ?
${}^{47}Les derniers prostitués sacrés qui restaient du temps de son père Asa, il les balaya du pays. 
${}^{48}Il n’y avait pas de roi en Édom, mais un préfet du roi. 
${}^{49}Josaphat fit des navires de haut bord pour aller à Ophir chercher de l’or, mais il n’y alla pas car les navires se brisèrent à Écione-Guéber. 
${}^{50}Alors, Ocozias, fils d’Acab, dit à Josaphat : « Mes serviteurs iront avec tes serviteurs sur les navires ». Mais Josaphat refusa !
${}^{51}Josaphat reposa avec ses pères.
        \\Il fut enseveli avec eux dans la Cité de David, son ancêtre.
        \\Son fils Joram régna à sa place.
${}^{52}Ocozias, fils d’Acab, devint roi sur Israël, à Samarie, la dix-septième année du règne de Josaphat, roi de Juda. Il régna deux ans sur Israël. 
${}^{53}Il fit ce qui est mal aux yeux du Seigneur. Il marcha dans le chemin de son père, dans le chemin de sa mère, et dans le chemin de Jéroboam, fils de Nebath, qui avait fait commettre à Israël des péchés. 
${}^{54}Il servit Baal et se prosterna devant lui. Il provoqua l’indignation du Seigneur, Dieu d’Israël, tout comme l’avait fait son père.
