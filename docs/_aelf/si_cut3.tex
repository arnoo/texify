  
  
      
         
      \bchapter{}
${}^{1}Le gouvernant sage éduque son peuple,
        et le pouvoir d’un homme intelligent s’exerce selon les règles.
${}^{2}Tel le gouvernant, tels ses ministres ;
        tel celui qui dirige la cité, tels les habitants.
${}^{3}Un roi ignorant conduit son peuple à la ruine,
        une ville s’édifie sur l’intelligence de ses chefs.
${}^{4}Le gouvernement de la terre est dans la main du Seigneur,
        qui, le moment venu, suscite l’homme providentiel.
${}^{5}La réussite de l’homme est dans la main du Seigneur,
        sur la personne du magistrat il fait reposer sa gloire.
        
           
${}^{6}Ne garde pas rancune au prochain, quels que soient ses torts,
        et ne fais rien dans un mouvement de violence.
${}^{7}Aux yeux du Seigneur et de l’homme, l’orgueil est odieux ;
        pour tous deux l’injustice est une faute.
${}^{8}La souveraineté passe d’un peuple à un autre,
        à cause des injustices, des violences et de l’argent.
        \\Il n’y a pire criminel que l’avare :
        il va jusqu’à vendre son âme.
${}^{9}De quoi pourrait s’enorgueillir celui qui est terre et poussière,
        alors que ses entrailles pourrissent déjà de son vivant ?
${}^{10}Longue maladie défie le médecin ;
        tel est roi aujourd’hui qui mourra demain.
${}^{11}À sa mort, l’homme reçoit en héritage
        larves, bêtes et vers.
${}^{12}L’orgueil de l’homme commence quand il s’écarte du Seigneur,
        quand son cœur s’éloigne de celui qui l’a créé,
${}^{13}car l’orgueil commence avec le péché,
        et qui s’y attache provoque un déluge d’abominations.
        \\Voilà pourquoi le Seigneur inflige aux orgueilleux d’éclatantes punitions,
        avant de les détruire complètement.
${}^{14}Le Seigneur a renversé les princes de leurs trônes
        et installé les doux à leur place.
${}^{15}Le Seigneur a déraciné les nations païennes,
        et, à leur place, il a planté les humbles.
${}^{16}Le Seigneur a dévasté les territoires des païens,
        il les a ravagés jusqu’aux fondements de la terre.
${}^{17}Il en a asséché pour exterminer leurs habitants,
        il a effacé de la terre jusqu’à leur souvenir.
${}^{18}L’orgueil n’a pas été créé pour les humains,
        ni les fureurs de la colère pour les enfants de la femme.
${}^{19}Quelle race est digne d’honneur ? La race des hommes.
        Quelle race est digne d’honneur ? Ceux qui craignent le Seigneur.
        \\Quelle race est méprisable ? La race des hommes.
        Quelle race est méprisable ? Ceux qui transgressent les commandements.
${}^{20}Comme un chef est à l’honneur au milieu de ses frères,
        ainsi sont aux yeux du Seigneur ceux qui le craignent.
${}^{21}La faveur divine commence avec la crainte du Seigneur,
        la disgrâce commence avec l’obstination et l’orgueil.
${}^{22}Le riche, le noble et le pauvre
        n’ont de fierté que dans la crainte du Seigneur.
${}^{23}Il est injuste de mépriser un homme intelligent qui est pauvre,
        et il ne convient pas d’honorer un pécheur.
${}^{24}Les grands, les juges, les puissants sont couverts de gloire,
        mais nul ne surpasse celui qui craint le Seigneur.
${}^{25}Un serviteur sage commandera des personnes libres,
        et l’homme de bon sens n’y trouvera pas à redire.
${}^{26}Ne joue pas au sage quand tu es au travail,
        et ne te vante pas quand tu es dans la gêne.
${}^{27}Mieux vaut un travailleur dans l’abondance
        qu’un flâneur qui se vante et manque de pain.
${}^{28}Mon fils, reste modeste dans ton opinion sur toi-même,
        et estime-toi à ta juste valeur.
${}^{29}Qui justifierait celui qui se fait tort à lui-même ?
        Et qui estimera celui qui se méprise ?
${}^{30}On honore le pauvre pour son savoir,
        et le riche pour sa richesse.
${}^{31}Si quelqu’un est honoré dans la pauvreté,
        combien plus le serait-il dans la richesse !
        \\Mais s’il est méprisé dans la richesse,
        combien plus le serait-il dans la pauvreté !
      
         
      \bchapter{}
${}^{1}La sagesse de l’humble lui tient la tête haute,
        et le fait asseoir au milieu des grands.
${}^{2}Ne fais pas l’éloge d’un homme parce qu’il est beau,
        et ne repousse personne à cause de son apparence.
${}^{3}L’abeille est un des plus petits êtres qui volent,
        mais ce qu’elle produit est d’une douceur exquise.
${}^{4}Ne mets pas ta fierté dans le vêtement que tu portes,
        et ne t’exalte pas au jour de ta gloire.
        \\Car les œuvres du Seigneur sont étonnantes,
        et son action, cachée aux hommes.
${}^{5}Beaucoup de souverains ont été déposés,
        et un inconnu a porté le diadème,
${}^{6}Bien des princes ont subi les derniers outrages,
        et des gens illustres ont été livrés aux mains d’autrui.
        
           
${}^{7}Ne blâme pas avant de t’être informé,
        réfléchis d’abord, et tu pourras critiquer.
${}^{8}Ne réponds pas avant d’avoir écouté,
        et n’interromps pas celui qui parle.
${}^{9}Ne t’enflamme pas dans une affaire qui ne te regarde pas,
        ne prends pas parti dans une querelle de pécheurs.
${}^{10}Mon fils, ne te lance pas dans beaucoup d’entreprises :
        si tu les multiplies, tu ne t’en tireras pas sans dommage.
        \\Même en courant, tu n’aboutiras à rien
        et tu ne pourras pas t’en sortir par la fuite.
${}^{11}Un tel se dépense, se donne du mal, s’active,
        et n’en est que plus dépourvu.
${}^{12}Tel autre est faible, il a besoin de soutien,
        il manque de moyens et ne surabonde que de misère.
        \\Mais les yeux du Seigneur l’ont regardé avec bienveillance,
        pour le relever de son abaissement
${}^{13}et redresser sa tête,
        à l’étonnement de tous.
${}^{14}Bonheur et malheur, vie et mort,
        richesse et pauvreté viennent du Seigneur.
${}^{15}Sagesse, science et connaissance de la Loi viennent du Seigneur,
        amour et pratique des bonnes œuvres viennent de lui.
${}^{16}L’égarement et les ténèbres ont été créés pour les pécheurs,
        ceux qui se targuent de leur malice vieillissent avec elle.
${}^{17}Le Seigneur maintient ses dons à ceux qui sont religieux ;
        sa bienveillance, à jamais, les conduit.
${}^{18}Tel s’enrichit à force d’être économe et regardant,
        mais voici ce qu’il y gagne :
${}^{19}quand il dit : « Enfin le repos !
        Maintenant je vais jouir de mes biens »,
        \\il ignore combien de temps cela va durer :
        il devra laisser ses biens à d’autres et mourra.
${}^{20}Reste attaché à ton métier, sois assidu,
        continue ton ouvrage jusqu’à la vieillesse.
${}^{21}Ne sois pas admiratif devant les œuvres du pécheur,
        fais confiance au Seigneur et persévère dans ta besogne.
        \\Car il est facile au Seigneur
        d’enrichir le pauvre à l’improviste, en un instant.
${}^{22}La récompense de qui est religieux, c’est la bénédiction du Seigneur :
        en peu de temps, il fait fleurir cette bénédiction.
${}^{23}Ne dis pas : « De quoi pourrais-je avoir besoin ?
        Quels biens puis-je avoir encore ? »
${}^{24}Ne dis pas non plus : « J’ai tout ce qu’il me faut.
        Quel mal pourrait encore m’atteindre ? »
${}^{25}Au jour du bonheur, on oublie le malheur ;
        au jour du malheur, on oublie le bonheur.
${}^{26}Mais au jour de la mort, il est facile, pour le Seigneur,
        de rendre à l’homme selon sa conduite.
${}^{27}La peine d’un moment fait oublier le bien-être,
        et les œuvres d’un homme se révèlent à ses derniers instants.
${}^{28}Avant sa mort, ne déclare personne heureux,
        car c’est au terme de sa vie que l’on connaît un homme.
${}^{29}N’introduis pas n’importe qui dans ta maison,
        car un escroc a plus d’un tour dans son sac.
${}^{30}Comme la perdrix qui sert d’appât dans une cage,
        tel est le cœur de l’orgueilleux :
        \\comme un espion,
        il guette et attend ta chute.
${}^{31}Pour tendre un piège, il tourne le bien en mal ;
        il jette le discrédit sur ce qui est estimable.
${}^{32}Une étincelle allume un grand brasier :
        le pécheur à l’affût va jusqu’au meurtre.
${}^{33}Prends garde au malveillant, car il prépare un mauvais coup :
        il peut te discréditer à jamais.
${}^{34}Si tu introduis chez toi l’étranger, il y jettera le trouble
        et te rendra étranger aux gens de ta maison.
      
         
      \bchapter{}
${}^{1}Si tu fais le bien, sache à qui tu le fais,
        et l’on te saura gré de ton geste.
${}^{2}Fais du bien à qui est religieux, et tu seras payé de retour,
        sinon par lui, du moins par le Très-Haut.
${}^{3}Pas de bonheur possible pour qui s’obstine dans le mal
        et refuse de faire l’aumône.
${}^{4}Donne à qui est religieux,
        mais ne viens pas en aide au pécheur.
${}^{5}À celui qui est humble fais du bien, mais ne donne pas à l’impie :
        refuse-lui son pain, ne lui donne rien ;
        \\il pourrait devenir plus fort que toi,
        et tu recevrais alors double mal
        pour tout le bien que tu lui aurais fait.
${}^{6}Car le Très-Haut lui-même déteste les pécheurs,
        il rendra aux impies ce qu’ils méritent,
        il se les réserve pour le jour du châtiment.
${}^{7}Donne à l’homme de bien,
        mais ne viens pas en aide au pécheur.
        
           
${}^{8}Dans le bonheur on ne peut pas reconnaître l’ami,
        dans le malheur l’ennemi ne sera plus masqué.
${}^{9}Quand un homme est heureux, ses ennemis sont dépités ;
        quand il est malheureux, même l’ami s’éloigne.
${}^{10}Ne fais jamais confiance à ton ennemi,
        car sa méchanceté se cache comme le bronze sous l’oxyde.
${}^{11}Même s’il se fait humble et s’avance avec un air penché,
        prends garde, méfie-toi de lui.
        \\Traite-le comme un miroir qu’il faut polir,
        sache que la rouille ne le dissimulera pas toujours.
${}^{12}Ne le mets pas près de toi :
        il te renverserait pour prendre ta place.
        \\Ne le fais pas asseoir à ta droite,
        il convoiterait ton siège.
        \\Tu comprendrais enfin mes paroles
        et, plein de regrets, tu te rappellerais mes conseils.
${}^{13}Qui aura pitié du charmeur mordu par un serpent
        et de ceux qui vont au-devant des fauves ?
${}^{14}Ainsi pour qui se lie à un pécheur
        et se rend complice de ses péchés.
${}^{15}Il restera un moment à tes côtés,
        mais dès que tu auras le dos tourné, il ne se contiendra plus.
${}^{16}L’ennemi n’a que douceur sur les lèvres,
        mais dans son cœur il cherche à te précipiter à la fosse.
        \\L’ennemi peut avoir les larmes aux yeux,
        mais s’il en a l’occasion, il sera insatiable de ton sang.
${}^{17}S’il t’arrive malheur, il sera le premier auprès de toi,
        et, sous prétexte de t’aider, il te fera un croc-en-jambe.
${}^{18}Il hochera la tête, se frottera les mains
        et ne fera que ricaner, laissant voir son vrai visage.
      
         
      \bchapter{}
${}^{1}Qui touche à du goudron se salit,
        qui fréquente un orgueilleux lui devient semblable.
${}^{2}Ne te charge pas d’un fardeau trop pesant,
        ne t’associe pas à plus fort ou plus riche que toi.
        \\Comment au pot de terre associer le pot de fer ?
        Au premier choc, celui-là se brise.
${}^{3}Le riche commet une injustice, et c’est lui qui se fâche !
        Le pauvre subit l’injustice, et il doit encore demander pardon.
${}^{4}Tant que tu lui es utile, le riche se sert de toi ;
        mais quand tu seras dans le besoin, il te laissera tomber.
${}^{5}Si tu as quelque bien, il vivra avec toi,
        pour te dépouiller sans remords.
${}^{6}A-t-il besoin de toi ? Il t’abuse,
        il te fait bon visage et te laisse espérer.
        \\Il t’adresse des compliments et dit :
        « De quoi as-tu besoin ? »
${}^{7}Il te rend confus par ses festins,
        jusqu’à ce qu’il t’ait mis sur la paille deux ou trois fois,
        \\et pour finir il se moquera de toi ;
        s’il te voit par la suite, il passe son chemin.
        \\Après quoi, il te regardera de haut, te laissera tomber
        et hochera la tête à ton sujet.
${}^{8}Fais attention : ne te laisse pas abuser
        ni humilier par ta naïveté.
        
           
${}^{9}Si un grand t’invite, dérobe-toi :
        il t’invitera de plus belle.
${}^{10}Ne t’impose pas, de peur d’être repoussé,
        ne te tiens pas trop loin, de peur d’être oublié.
${}^{11}Ne t’avise pas de lui parler d’égal à égal,
        ne te fie pas à son bavardage,
        \\car par ce flot de paroles il te met à l’épreuve :
        il te scrute, même quand il te sourit.
${}^{12}Il divulguera impitoyablement tes propos
        et ne t’épargnera ni les coups ni les chaînes.
${}^{13}Fais attention et prends bien garde,
        car tu frôles ta propre ruine.
${}^{14}Si tu entends cela, reste vigilant dans ton sommeil,
        aime le Seigneur ta vie durant
        et invoque-le pour ton salut.
${}^{15}Tout animal aime son semblable
        et tout homme, son pareil.
${}^{16}Toute chair s’unit selon son espèce,
        et l’homme s’attache à son semblable.
${}^{17}Quoi de commun entre le loup et l’agneau,
        entre qui est pécheur et qui est religieux ?
${}^{18}Quelle paix possible entre l’hyène et le chien ?
        Quelle paix entre le riche et le pauvre ?
${}^{19}L’âne sauvage du désert est la proie des lions,
        comme le pauvre est la pâture des riches.
${}^{20}L’orgueilleux déteste l’humiliation,
        comme le riche déteste le pauvre.
${}^{21}Que chancelle un riche, il est soutenu par ses amis ;
        que tombe un pauvre, il est repoussé par les siens.
${}^{22}Quand le riche fait un faux pas, il trouve beaucoup d’appuis ;
        s’il dit des sottises, on lui donne raison.
        \\Mais quand un petit trébuche, on lui fait des reproches ;
        même s’il a des choses sensées à dire,
        on ne lui en donne pas l’occasion.
${}^{23}Quand le riche prend la parole, tous font silence
        et portent son discours aux nues.
        \\Si le pauvre veut s’exprimer, on demande : « Qui est-ce ? »
        et s’il bute sur un mot, on l’enfonce.
${}^{24}La richesse est bonne tant qu’elle est sans péché ;
        la pauvreté est un mal, au dire de l’impie.
${}^{25}Le cœur de l’homme modèle son visage
        soit en bien, soit en mal.
${}^{26}À cœur content, joyeux visage ;
        mais pour forger des proverbes, on s’épuise à réfléchir !
       
      <p class="cantique"><span class="cantique_label"><a href="#bib_ct-at_13">Cantique AT 13</a></span> = <span class="cantique_ref"><a class="unitex_link" href="#bib_si_14_20">Si 14,20-21</a>;<a class="unitex_link" href="#bib_si_15_3">15,3-5</a></span>
      <p class="cantique" id="bib_ct-at_13bis"><span class="cantique_label">Cantique AT 13bis</span> = <span class="cantique_ref"><a class="unitex_link" href="#bib_si_14_20">Si 14, 20-27</a></span>
      
         
      \bchapter{}
${}^{1}Heureux l’homme qui n’a pas failli en paroles
        et n’est pas tourmenté du regret de ses fautes.
${}^{2}Heureux celui que sa conscience ne condamne pas
        et qui n’a pas perdu l’espoir.
        
           
${}^{3}La richesse ne sied pas à l’homme mesquin,
        et, pour l’homme envieux, quelle fortune suffirait ?
${}^{4}Qui amasse en se privant amasse pour autrui :
        avec ses biens, d’autres vivront dans le plaisir.
${}^{5}Celui qui est dur pour lui-même, pour qui sera-t-il bon ?
        Il ne jouira même pas de sa fortune.
${}^{6}Nul n’est pire que celui qui se ronge d’envie :
        il reçoit le salaire de sa perversion.
${}^{7}S’il fait le bien, c’est par mégarde ;
        il finit toujours par laisser voir sa méchanceté.
${}^{8}Il est vil, celui qui a l’œil envieux,
        qui détourne son regard et méprise les gens.
${}^{9}L’œil de l’avare n’est pas satisfait de ce qui lui revient ;
        une avidité malsaine dessèche l’âme.
${}^{10}L’avare est chiche de pain,
        il en manque, lui-même, à sa table.
         
${}^{11}Mon fils, fais-toi plaisir avec ce que tu as
        et présente au Seigneur des offrandes dignes de lui.
${}^{12}Souviens-toi que la mort ne tardera pas,
        et que l’heure fixée ne t’a pas été révélée.
${}^{13}Avant de mourir, fais plaisir à tes amis ;
        sois généreux, donne à la mesure de tes moyens.
${}^{14}Ne te prive pas d’un jour de bonheur,
        ne refuse pas ta part de satisfactions.
${}^{15}Ne devras-tu pas laisser à d’autres le fruit de tes peines ?
        Le produit de ton labeur ne sera-t-il pas tiré au sort ?
${}^{16}Donne, reçois, distrais-toi,
        car au séjour des morts ne se trouve aucun plaisir.
${}^{17}Toute chair se fripe comme un vêtement ;
        c’est la loi depuis toujours : Tu dois mourir.
${}^{18}Comme parmi les feuilles d’un arbre touffu
        les unes tombent et les autres poussent,
        \\ainsi font les générations de chair et de sang :
        l’une meurt, et l’autre naît.
${}^{19}Toute œuvre corruptible disparaît,
        et son auteur s’en va avec elle.
        ${}^{20}Heureux qui s’exerce à la sagesse
        et réfléchit avec intelligence,
        ${}^{21}qui médite en son cœur ses chemins
        et pénètre dans ses secrets.
        ${}^{22}Il suit sa piste, comme un chasseur,
        il guette son passage,
        ${}^{23}Il regarde par sa fenêtre,
        il écoute à sa porte.
        ${}^{24}Il se tient près de sa maison
        et fixe dans ses murs le piquet de la tente
        ${}^{25}qu’il dresse tout près de chez elle ;
        il campe ainsi au séjour du bonheur.
        ${}^{26}Sous la protection de la Sagesse, il place ses enfants ;
        sous ses branches, il demeure ;
        ${}^{27}à son ombre, il est protégé de la chaleur ;
        il s’établit dans sa gloire.
       
      <div class="box_other filet_bleu">
          <h3 class="intertitle cantique_chap" id="bib_ct-at_13">Cantique AT 13</h3><a class="cantique_chap" href="#bib_si_14">14</a>
            <a class="cantique_verset" href="#bib_si_14_20"><span class="cantique_verset_in">20</span></a>Heureux qui s’exerce à la sagesse
            et réfléchit avec intelligence,
            <a class="cantique_verset" href="#bib_si_14_21"><span class="cantique_verset_in">21</span></a>qui médite en son cœur ses chemins
            et pénètre dans ses secrets.
            <a class="cantique_verset" href="#bib_si_15_3"><span class="cantique_verset_in">3</span></a>Elle-même le nourrit du pain de l’intelligence
            et lui donne à boire l’eau de la sagesse.
            <a class="cantique_verset" href="#bib_si_15_4"><span class="cantique_verset_in">4</span></a>Il s’appuie sur elle et ne chancelle pas ;
            il s’attache à elle et n’en rougit pas.
            <a class="cantique_verset" href="#bib_si_15_5"><span class="cantique_verset_in">5</span></a>C’est elle qui l’élève au-dessus de ses proches ;
            au milieu de l’assemblée, il ouvrira la bouche.
      
         
      \bchapter{}
        ${}^{1}Ainsi agit celui qui craint le Seigneur.
        Celui qui est maître de la Loi atteindra la Sagesse.
        ${}^{2}Elle vient à sa rencontre comme une mère ;
        comme une épouse vierge\\, elle l’accueillera.
        ${}^{3}Elle-même le nourrit du pain de l’intelligence
        et lui donne à boire l’eau de la sagesse.
        ${}^{4}Il s’appuie sur elle et ne chancelle pas ;
        il s’attache à elle et n’en rougit pas.
        ${}^{5}C’est elle qui l’élève au-dessus de ses proches ;
        au milieu de l’assemblée, il ouvrira la bouche.
        ${}^{6}Il trouvera la joie et recevra une couronne d’allégresse,
        il aura pour héritage une renommée éternelle.
${}^{7}Jamais les insensés n’atteindront la Sagesse,
        les pécheurs ne la verront pas.
${}^{8}Elle se tient loin de l’orgueil,
        et les menteurs ne songent pas à elle.
${}^{9}La louange sonne faux dans la bouche du pécheur,
        parce que le Seigneur ne l’y a pas placée.
${}^{10}C’est la Sagesse qui fait jaillir la louange,
        et le Seigneur la mène à bien.
        
           
${}^{11}Ne dis pas : « C’est le Seigneur qui m’a dévoyé »,
        car il ne fait pas ce qu’il a en horreur.
${}^{12}Ne dis pas : « C’est lui qui m’a égaré »,
        car il n’a que faire du pécheur.
${}^{13}Tout ce qui est abominable est détesté du Seigneur
        et ne peut être aimé de ceux qui le craignent.
${}^{14}C’est lui qui, au commencement, a créé l’homme
        et l’a laissé à son libre arbitre.
        ${}^{15}Si tu le veux, tu peux observer les commandements,
        il dépend de ton choix de rester fidèle.
        ${}^{16}Le Seigneur a mis devant toi l’eau et le feu :
        étends la main vers ce que tu préfères.
        ${}^{17}La vie et la mort sont proposées aux hommes,
        l’une ou l’autre leur est donnée selon leur choix.
        ${}^{18}Car la sagesse du Seigneur est grande,
        fort est son pouvoir, et il voit tout.
        ${}^{19}Ses regards sont tournés vers ceux qui le craignent,
        il connaît toutes les actions des hommes.
        ${}^{20}Il n’a commandé à personne d’être impie,
        il n’a donné à personne la permission de pécher.
      
         
      \bchapter{}
${}^{1}Ne souhaite pas beaucoup d’enfants
        qui ne seraient bons à rien,
        \\et ne mets pas ta joie dans tes fils,
        s’ils sont impies.
${}^{2}Aussi nombreux qu’ils soient, ne t’en réjouis pas
        s’ils n’ont pas la crainte du Seigneur.
${}^{3}Ne t’attends pas à ce qu’ils aient longue vie
        et ne te fie pas à leur nombre,
        \\car tu te lamenteras pour un deuil prématuré
        et tu seras surpris par leur fin.
        \\Un seul vaut mieux que mille ;
        il est préférable de mourir sans enfants
        que d’avoir des enfants impies.
${}^{4}Il suffit d’une personne intelligente pour que la ville prospère,
        mais la race des sans-loi sera anéantie.
${}^{5}J’ai vu de mes yeux bien des exemples,
        et entendu de mes oreilles des cas plus graves encore :
${}^{6}un feu embraser une assemblée de pécheurs,
        et la colère de Dieu s’enflammer contre un peuple rebelle.
${}^{7}Il n’a pas pardonné aux géants d’autrefois,
        qui, sûrs de leur puissance, s’étaient révoltés.
${}^{8}Il n’a pas épargné les habitants de Sodome :
        leur orgueil lui faisait horreur.
${}^{9}Il n’a pas eu pitié d’un peuple allant à sa perte
        qui se glorifiait dans ses péchés.
        \\Voilà comment il a traité des peuples au cœur endurci ;
        même de ses nombreux fidèles il n’a pas eu pitié.
${}^{10}C’est ainsi qu’il fit avec les six cent mille Israélites
        qui s’étaient ligués contre lui par dureté de cœur.
        \\Il châtie et prend pitié, il frappe et il guérit :
        le Seigneur veille avec compassion et rigueur.
${}^{11}Un seul aurait eu la nuque raide,
        que son impunité aurait été surprenante,
        \\car dans le Seigneur, il y a miséricorde mais aussi colère :
        il pardonne avec magnificence, mais répand aussi sa fureur.
${}^{12}Telle sa miséricorde, telle sa sévérité :
        il jugera les hommes sur leurs actes.
${}^{13}Le pécheur ne s’échappera pas avec son butin,
        la patience de qui est religieux ne sera pas déçue.
${}^{14}Il sera tenu compte de toute action charitable,
        et chacun sera traité selon ses œuvres.
${}^{15}Le Seigneur a endurci Pharaon si bien qu’il ne le reconnut pas,
        afin que les actes divins soient manifestés sous le ciel.
${}^{16}À toute créature sa miséricorde se révèle,
        il a donné en partage aux fils d’Adam
        et sa lumière et les ténèbres.
        
           
${}^{17}Ne dis pas : « Je me cacherai du Seigneur :
        là-haut, qui se souviendra de moi ?
        \\Au milieu de la foule je ne serai pas reconnu :
        que suis-je dans l’immense création ? »
${}^{18}Car voici : le ciel et les cieux par-dessus le ciel,
        l’abîme et la terre seront ébranlés lors de sa visite.
        \\C’est par sa volonté que l’univers fut
        et qu’il se maintient.
${}^{19}Les montagnes aussi bien que les fondements de la terre
        seront saisis d’un même effroi sous son regard.
${}^{20}Mais le cœur ne réfléchit pas à tout cela.
        Qui prête attention à la manière dont Dieu agit ?
${}^{21}Comme la tempête que l’homme ne peut prévoir,
        la plupart des œuvres de Dieu sont tenues cachées.
${}^{22}Ne dis pas : « Les œuvres de sa justice, qui les annonce ?
        Et qui les attend ? Car elle est loin, l’Alliance,
        et l’examen de toute chose n’aura lieu qu’à la fin. »
${}^{23}C’est manquer de bon sens que penser ainsi ;
        l’homme égaré et stupide ne conçoit que des folies.
${}^{24}Écoute-moi, mon fils, et acquiers le savoir,
        applique ton cœur à ce que je dis.
${}^{25}Je te révélerai peu à peu ce que j’ai à t’apprendre,
        je te transmettrai le savoir avec exactitude.
${}^{26}Par un décret du Seigneur, ses œuvres existent dès l’origine ;
        dès leur création, il les a séparées pour les répartir.
${}^{27}Il a mis en ordre sa création pour toujours
        et réglé pour la suite des temps l’activité des astres :
        \\ils ne connaissent ni peine ni fatigue
        et n’interrompent jamais leur tâche.
${}^{28}Aucun ne heurte son voisin,
        et, jamais, ils n’enfreindront la parole divine.
${}^{29}Après cela, le Seigneur a regardé la terre
        et l’a comblée de ses bienfaits.
${}^{30}De toute espèce d’êtres vivants il en a couvert la surface,
        et c’est à la terre qu’ils retourneront.
      
         
      \bchapter{}
        ${}^{1}Le Seigneur a créé l’homme en le tirant de la terre,
        et il l’a fait retourner à la terre.
        ${}^{2}Il a donné aux humains des jours comptés, un temps déterminé,
        il a remis en leur pouvoir ce qui est sur la terre.
        ${}^{3}Il les a revêtus d’une force pareille à la sienne,
        il les a faits à son image.
        ${}^{4}Il a mis en tout vivant la crainte de l’être humain,
        pour que celui-ci commande en maître
        aux bêtes sauvages et aux oiseaux.
        ${}^{5}Les humains ont reçu du Seigneur l’usage des cinq sens ;
        il leur a donné en partage un sixième sens, l’intelligence,
        et un septième, la parole, qui permet d’interpréter ses œuvres.
        ${}^{6}Aux humains il a donné
        du jugement, une langue, des yeux,
        des oreilles, et un cœur pour réfléchir.
        ${}^{7}Il les a remplis de savoir et d’intelligence,
        il leur a fait connaître le bien et le mal.
        ${}^{8}Il a posé son regard sur leur cœur,
        leur montrant la grandeur de ses œuvres ;
        il leur a donné de se glorifier à jamais de ses merveilles.
        ${}^{9}Ils raconteront la grandeur de ses œuvres,
        ${}^{10}ils célébreront le Nom très saint.
        ${}^{11}Il leur a aussi accordé le savoir,
        il leur a donné en héritage la loi de vie,
        afin qu’ils comprennent, dès maintenant, qu’ils sont mortels.
        ${}^{12}Il a établi avec eux une Alliance éternelle,
        et il leur a fait connaître ses jugements.
        ${}^{13}Leurs yeux ont vu la grandeur de sa gloire,
        leurs oreilles ont entendu la majesté de sa voix.
        ${}^{14}Il leur a dit : « Gardez-vous de toute injustice »,
        et à chacun il a donné des commandements
        au sujet du prochain.
        
           
        ${}^{15}Leurs chemins sont toujours à découvert devant lui,
        ils n’échappent jamais à ses regards.
${}^{16}Leurs chemins les mènent au mal dès l’enfance,
        et, de leurs cœurs de pierre, ils ne peuvent faire des cœurs de chair.
${}^{17}Quand le Seigneur a réparti les peuples sur toute la terre,
        \\il a mis un chef à la tête de chaque nation,
        mais la part du Seigneur, c’est Israël ;
${}^{18}c’est son premier-né, qu’il nourrit et qu’il forme ;
        il lui donne en partage la lumière de l’amour
        et ne l’abandonne jamais.
${}^{19}Toutes les actions des hommes sont, devant Dieu, en plein soleil :
        il a constamment les yeux sur leur conduite.
${}^{20}Leurs injustices ne lui échappent pas,
        tous leurs péchés sont à découvert devant le Seigneur.
${}^{21}Mais le Seigneur est bon et connaît ses créatures,
        loin de les abandonner ou de les perdre, il les épargne.
${}^{22}L’aumône d’un homme est marquée d’un sceau devant Dieu
        qui veille sur une bonne action comme sur la prunelle de l’œil
        et accorde le repentir à ses fils et à ses filles.
${}^{23}À la fin il se lèvera et fera les comptes,
        il rendra à chacun ce qui lui revient.
        ${}^{24}À ceux qui se repentent, Dieu ouvre le chemin du retour ;
        il réconforte ceux qui manquent de persévérance.
        ${}^{25}Convertis-toi au Seigneur, et renonce à tes péchés ;
        mets-toi devant lui pour prier,
        et diminue tes occasions de chute.
        ${}^{26}Reviens vers le Très-Haut et détourne-toi de l’injustice,
        – c’est lui qui conduit des ténèbres à la lumière de la vie – ;
        les actions abominables, déteste-les.
        ${}^{27}Qui pourra célébrer le Très-Haut dans le séjour des morts,
        remplacer les vivants qui lui rendent gloire ?
        ${}^{28}La louange est enlevée au mort puisqu’il n’est plus ;
        c’est le vivant, le bien-portant, qui célébrera le Seigneur.
        ${}^{29}Qu’elle est grande, la miséricorde du Seigneur,
        qu’il est grand, son pardon
        pour ceux qui se convertissent à lui !
${}^{30}Les humains n’ont pas la toute-puissance
        car un fils d’homme n’est pas immortel.
${}^{31}Quoi de plus lumineux que le soleil ? Et pourtant il disparaît.
        À plus forte raison, celui qui médite le mal,
        l’être de chair et de sang.
${}^{32}Dieu passe en revue les astres dans les hauteurs du ciel,
        et, à plus forte raison, les humains qui sont tous terre et cendre.
      
         
      \bchapter{}
${}^{1}Celui qui vit à jamais a créé l’univers entier.
${}^{2}Seul le Seigneur sera reconnu juste,
        il n’y en a pas d’autre que lui.
        \\Il tient le gouvernail du monde avec la paume de sa main,
        tout obéit à sa volonté,
        \\car, par sa puissance, il est roi de l’univers ;
        les choses saintes, il les sépare des profanes.
        
           
         
${}^{4}À personne, il n’a donné d’énumérer ses œuvres.
        Qui fera le tour de ses grandeurs ?
${}^{5}Qui mesurera la puissance de sa majesté ?
        Et qui pourrait raconter ses miséricordes ?
${}^{6}Rien à en retrancher, rien à y ajouter.
        On ne peut faire le tour des merveilles du Seigneur.
${}^{7}Quand l’homme a fini, c’est à peine s’il commence ;
        s’arrête-t-il, il mesure son indigence.
        
           
         
${}^{8}Qu’est-ce que l’homme, et à quoi est-il bon ?
        Quel sens a le bien qu’il fait, quel sens a le mal ?
${}^{9}La durée de sa vie est de cent ans tout au plus.
        Le moment du repos éternel est imprévisible pour chacun.
${}^{10}Une goutte d’eau dans la mer, un grain de sable :
        voilà ses courtes années face à l’éternité.
${}^{11}C’est pourquoi le Seigneur est patient avec les humains
        et répand sur eux sa miséricorde.
        
           
         
${}^{12}Il voit et sait combien leur fin est misérable ;
        c’est pourquoi il pardonne sans compter.
${}^{13}L’homme a pitié de son prochain ;
        le Seigneur, lui, a pitié de toute créature.
        \\Il corrige, il instruit, il enseigne ;
        comme un berger, il fait revenir son troupeau.
${}^{14}Il a pitié de ceux qui accueillent son enseignement
        et qui observent avec empressement ses décrets.
        
           
${}^{15}Mon fils, quand tu fais le bien, n’y joins pas le reproche,
        et quand tu donnes, n’ajoute pas des paroles amères.
${}^{16}La rosée ne rafraîchit-elle pas dans la chaleur brûlante ?
        De même, une parole peut faire plus de bien qu’un cadeau.
${}^{17}Ne vois-tu pas qu’une parole vaut mieux qu’un cadeau somptueux ?
        L’homme généreux unit l’une à l’autre.
${}^{18}Le sot montre sa mesquinerie par ses reproches,
        et le cadeau de l’envieux fait monter les larmes aux yeux.
${}^{19}Avant de parler, informe-toi ;
        avant d’être malade, soigne-toi.
${}^{20}Avant le jugement, fais ton examen de conscience :
        lors de la visite du Seigneur, tu trouveras le pardon.
${}^{21}Avant même de tomber malade, humilie-toi
        et, quand tu as péché, montre du repentir.
         
${}^{22}Ne manque pas d’accomplir tes vœux au moment fixé
        et n’attends pas la mort pour t’en acquitter.
${}^{23}Prends tes dispositions avant de faire un vœu,
        ne sois pas de ceux qui tentent le Seigneur ;
${}^{24}n’oublie pas la colère qui sévira au dernier jour,
        quand il châtiera en détournant sa face.
         
${}^{25}Au temps de l’abondance, souviens-toi du temps de famine ;
        aux jours de richesse, rappelle-toi la pauvreté et le dénuement.
${}^{26}Du matin au soir les situations changent,
        tout passe si vite devant le Seigneur.
${}^{27}Un homme sage est toujours sur ses gardes ;
        aux jours où le péché menace, il évite toute faute.
${}^{28}Quiconque est intelligent reconnaît la sagesse
        et rend hommage à celui qui l’a trouvée.
${}^{29}Ceux qui parlent avec intelligence sont eux-mêmes des sages ;
        ils répandent, comme une ondée, de justes proverbes.
        \\Mieux vaut faire confiance au Maître unique
        que s’attacher d’un cœur mort à ce qui est mort.
${}^{30}Ne te laisse pas entraîner par tes passions,
        et réfrène tes appétits.
${}^{31}Si tu donnes pleine satisfaction à tes désirs,
        tu feras de toi la risée de tes ennemis.
${}^{32}Ne mets pas ta joie dans une vie de plaisir,
        ne t’endette pas par de telles dépenses.
${}^{33}Ne tombe pas dans la misère
        à force d’emprunter pour courir les festins,
        \\quand tu n’as plus un sou en poche.
        Ce serait en vouloir à ta propre vie.
      
         
      \bchapter{}
${}^{1}Un travailleur qui boit ne deviendra jamais riche ;
        celui qui néglige les petites choses peu à peu tombera.
${}^{2}Le vin et les femmes égarent même les hommes intelligents.
        Celui qui fréquente les prostituées perdra toute retenue ;
${}^{3}la pourriture et les vers s’empareront de lui,
        et sa témérité lui coûtera la vie.
        
           
${}^{4}Qui fait trop vite confiance est une tête légère ;
        qui commet un péché se fait tort à lui-même.
${}^{5}Celui qui trouve sa joie dans le mal sera condamné,
        mais celui qui domine ses plaisirs couronne sa propre vie.
${}^{6}Celui qui maîtrise sa langue vivra sans conflit ;
        qui déteste le bavardage se soustrait au mal.
${}^{7}Ne répète jamais les on-dit :
        tu n’y perdras rien.
${}^{8}Ne colporte de racontars ni devant ton ami ni devant ton ennemi,
        et ne révèle rien, sauf si ton silence te rendait complice.
${}^{9}Certes, on t’écouterait, mais on se méfierait de toi,
        et on en viendrait à te haïr.
${}^{10}As-tu entendu quelque chose ? Sois un tombeau !
        Courage ! Tu ne vas pas éclater.
${}^{11}Pour une parole qu’il retient, voilà le sot dans les douleurs,
        comme une femme prête à accoucher !
${}^{12}Telle une flèche enfoncée dans la chair de la cuisse,
        la parole est insupportable aux entrailles d’un sot.
${}^{13}Questionne ton ami : peut-être n’a-t-il rien fait,
        et, s’il a fait quelque chose, il ne recommencera pas.
${}^{14}Questionne ton prochain : peut-être n’a-t-il rien dit,
        et, s’il a parlé, il ne le fera plus.
${}^{15}Questionne ton ami, car la calomnie est monnaie courante,
        et ne va pas croire tout ce que l’on dit.
${}^{16}On peut déraper bien malgré soi :
        qui n’a jamais péché par sa langue ?
${}^{17}Questionne ton prochain avant d’en venir aux menaces
        et tiens compte de la loi du Très-Haut,
        sans garder de ressentiment.
${}^{18}La faveur divine commence avec la crainte du Seigneur,
        et la sagesse gagne son affection.
${}^{19}Connaître les commandements du Seigneur
        est une leçon de vie ;
        \\ceux qui agissent pour lui plaire
        cueilleront les fruits de l’arbre d’immortalité.
         
${}^{20}Toute sagesse est crainte du Seigneur,
        toute sagesse comporte la pratique de sa Loi
        et la connaissance de son pouvoir souverain.
${}^{21}Le serviteur qui dit à son maître : « Je ne ferai pas ce qui te plaît »,
        même s’il le fait ensuite, exaspère celui qui le nourrit.
${}^{22}La science du mal n’est pas sagesse,
        et le discernement n’est pas dans le conseil des pécheurs :
${}^{23}c’est du vice, et le vice est abominable ;
        qui manque de sagesse est toujours insensé ;
${}^{24}mieux vaut peu d’intelligence avec la crainte de Dieu
        qu’un esprit très subtil qui transgresse la Loi.
${}^{25}Il y a une habileté consommée qui est injustice ;
        certains recourent à l’intrigue pour faire valoir leur droit.
        Sage est celui qui rend de justes sentences.
${}^{26}Un tel marche, courbé dans le noir ;
        au fond de lui, il est plein de fausseté.
${}^{27}Il cache son visage et fait la sourde oreille ;
        s’il n’est pas démasqué, il prendra le pas sur toi.
${}^{28}S’il s’abstient de pécher, c’est par manque de force ;
        à la première occasion, il commettra le mal.
${}^{29}On reconnaît un homme à son aspect ;
        à l’expression du visage, on reconnaît quelqu’un d’avisé.
${}^{30}La manière dont quelqu’un s’habille, sa façon de rire
        et sa démarche révèlent ce qu’il est.
