  
  
    
    \bbook{LETTRE AUX PHILIPPIENS}{LETTRE AUX PHILIPPIENS}
      
         
      \bchapter{}
        ${}^{1}Paul et Timothée, serviteurs du Christ Jésus,
        \\à tous ceux qui sont sanctifiés dans le Christ Jésus
        et habitent à Philippes,
        \\ainsi qu’aux responsables et aux ministres de l’Église.
        ${}^{2}À vous, la grâce et la paix
        \\de la part de Dieu notre Père
        et du Seigneur Jésus Christ.
        
           
${}^{3}Je rends grâce à mon Dieu chaque fois que je fais mémoire de vous. 
${}^{4}À tout moment, chaque fois que je prie pour vous tous, c’est avec joie que je le fais, 
${}^{5}à cause de votre communion avec moi, dès le premier jour jusqu’à maintenant, pour l’annonce de l’Évangile. 
${}^{6}J’en suis persuadé, celui qui a commencé en vous un si beau travail le continuera jusqu’à son achèvement au jour où viendra le Christ Jésus.
${}^{7}Il est donc juste que j’aie de telles dispositions à l’égard de vous tous, car je vous porte dans mon cœur, vous qui communiez tous à la grâce qui m’est faite dans mes chaînes comme dans la défense de l’Évangile et son annonce ferme. 
${}^{8}Oui, Dieu est témoin de ma vive affection pour vous tous dans la tendresse du Christ Jésus. 
${}^{9}Et, dans ma prière, je demande que votre amour vous fasse progresser de plus en plus dans la pleine connaissance et en toute clairvoyance 
${}^{10}pour discerner ce qui est important. Ainsi, serez-vous purs et irréprochables pour le jour du Christ, 
${}^{11}comblés du fruit de la justice qui s’obtient par Jésus Christ, pour la gloire et la louange de Dieu.
${}^{12}Je veux que vous le sachiez, frères : ce qui m’arrive a plutôt fait progresser l’annonce de l’Évangile ; 
${}^{13}ainsi donc, dans tout le prétoire et partout ailleurs, mes chaînes manifestent mon attachement au Christ, 
${}^{14}et la plupart des frères, chez qui mes chaînes suscitent une ferme confiance dans le Seigneur, trouvent une audace nouvelle pour dire sans crainte la Parole. 
${}^{15}Les uns proclament le Christ en esprit de jalousie et de rivalité ; d’autres le font avec une intention bienveillante. 
${}^{16}Ceux-ci annoncent le Christ par amour, sachant que je suis ici pour défendre l’Évangile ; 
${}^{17}ceux-là le font en intrigants, sans intention pure, pensant aviver ainsi l’épreuve de mes chaînes.
${}^{18}Qu’importe ! De toute façon, que ce soit avec des arrière-pensées ou avec sincérité, le Christ est annoncé, et de cela je me réjouis. Bien plus, je me réjouirai encore, 
${}^{19}car je sais que cela tournera à mon salut, grâce à votre prière et à l’assistance de l’Esprit de Jésus Christ. 
${}^{20}C’est ce que j’attends avec impatience, et c’est ce que j’espère. Je n’aurai à rougir de rien ; au contraire, je garderai toute mon assurance, maintenant comme toujours ; soit que je vive, soit que je meure, le Christ sera glorifié dans mon corps. 
${}^{21}En effet, pour moi, vivre c’est le Christ, et mourir est un avantage. 
${}^{22}Mais si, en vivant en ce monde, j’arrive à faire un travail utile, je ne sais plus comment choisir. 
${}^{23}Je me sens pris entre les deux : je désire partir pour être avec le Christ, car c’est bien préférable ; 
${}^{24}mais, à cause de vous, demeurer en ce monde est encore plus nécessaire. 
${}^{25}De cela, je suis convaincu. Je sais donc que je resterai, et que je continuerai à être avec vous tous, pour votre progrès et votre joie dans la foi. 
${}^{26}Ainsi, à travers ce qui m’arrive, vous aurez d’autant plus de fierté dans le Christ Jésus, du fait de mon retour parmi vous.
${}^{27}Quant à vous, ayez un comportement digne de l’Évangile du Christ. Ainsi, soit que je vienne vous voir, soit que je reste absent, j’entendrai dire de vous que vous tenez bon dans un seul esprit, que vous luttez ensemble, d’une seule âme, pour la foi en l’Évangile, 
${}^{28}et que vous ne vous laissez pas intimider par les adversaires : ce sera pour eux la preuve de leur perte et pour vous celle du salut. Et tout cela vient de Dieu 
${}^{29}qui, pour le Christ, vous a fait la grâce non seulement de croire en lui mais aussi de souffrir pour lui. 
${}^{30}Ce combat que vous soutenez, vous m’avez vu le mener moi aussi, et vous entendez maintenant que je le mène encore.
      <p class="cantique" id="bib_ct-nt_5"><span class="cantique_label">Cantique NT 5</span> = <span class="cantique_ref"><a class="unitex_link" href="#bib_ph_2_6">Ph 2, 6-11</a></span>
      
         
      \bchapter{}
      \begin{verse}
${}^{1}S’il est vrai que, dans le Christ, on se réconforte les uns les autres, si l’on s’encourage avec amour, si l’on est en communion dans l’Esprit, si l’on a de la tendresse et de la compassion, 
${}^{2}alors, pour que ma joie soit complète, ayez les mêmes dispositions, le même amour, les mêmes sentiments ; recherchez l’unité. 
${}^{3}Ne soyez jamais intrigants ni vaniteux, mais ayez assez d’humilité pour estimer les autres supérieurs à vous-mêmes. 
${}^{4}Que chacun de vous ne soit pas préoccupé de ses propres intérêts ; pensez aussi à ceux des autres.
      
         
${}^{5}Ayez en vous les dispositions qui sont dans le Christ Jésus :
       
        \\Le Christ Jésus\\,
        ${}^{6}ayant la condition de Dieu,
        \\ne retint pas jalousement
        le rang qui l’égalait à Dieu.
         
        ${}^{7}Mais il s’est anéanti,
        \\prenant la condition de serviteur,
        devenant semblable aux hommes.
         
        \\Reconnu homme à son aspect,
        ${}^{8}il s’est abaissé,
        \\devenant obéissant jusqu’à la mort,
        et la mort de la croix.
         
        ${}^{9}C’est pourquoi Dieu l’a exalté :
        \\il l’a doté du Nom
        qui est au-dessus de tout nom,
         
        ${}^{10}afin qu’au nom de Jésus
        tout genou fléchisse
        \\au ciel, sur terre et aux enfers,
         
        ${}^{11}et que toute langue proclame\\ :
        « Jésus Christ est Seigneur »
        \\à la gloire de Dieu le Père.
${}^{12}Ainsi, mes bien-aimés, vous qui avez toujours obéi, travaillez à votre salut avec crainte et profond respect ; ne le faites pas seulement quand je suis là, mais encore bien plus maintenant que je n’y suis pas. 
${}^{13}Car c’est Dieu qui agit pour produire en vous la volonté et l’action, selon son projet bienveillant. 
${}^{14}Faites tout sans récriminer et sans discuter ; 
${}^{15}ainsi vous serez irréprochables et purs, vous qui êtes des enfants de Dieu sans tache au milieu d’une génération tortueuse et pervertie où vous brillez comme les astres dans l’univers, 
${}^{16}en tenant ferme la parole de vie. Alors je serai fier de vous quand viendra le jour du Christ : je n’aurai pas couru pour rien ni peiné pour rien. 
${}^{17}Et si je dois verser mon sang pour l’ajouter au sacrifice que vous offrez à Dieu par votre foi, je m’en réjouis et je partage votre joie à tous. 
${}^{18}Et vous, de même, réjouissez-vous et partagez ma joie.
${}^{19}Dans le Seigneur Jésus, j’ai l’espoir de vous envoyer bientôt Timothée, pour que j’aie, moi aussi, la satisfaction de recevoir de vos nouvelles. 
${}^{20}Je n’ai en effet personne d’autre qui partage véritablement avec moi le souci de ce qui vous concerne. 
${}^{21}Car tous les autres se préoccupent de leurs propres affaires, non pas de celles de Jésus Christ. 
${}^{22}Mais lui, vous savez que sa valeur est éprouvée : comme un fils avec son père, il s’est mis avec moi au service de l’Évangile. 
${}^{23}J’espère donc vous l’envoyer dès que je verrai clair sur ma situation. 
${}^{24}J’ai d’ailleurs confiance dans le Seigneur que je viendrai moi-même bientôt.
${}^{25}J’ai aussi jugé nécessaire de vous envoyer Épaphrodite, mon frère, mon compagnon de travail et de combat. Il était votre envoyé, pour me rendre les services dont j’avais besoin, 
${}^{26}mais il avait un grand désir de vous revoir tous, et il se tourmentait parce que vous aviez appris sa maladie. 
${}^{27}Car il a été malade, et bien près de la mort, mais Dieu a eu pitié de lui, et pas seulement de lui, mais aussi de moi, en m’évitant d’avoir tristesse sur tristesse. 
${}^{28}Je m’empresse donc de vous le renvoyer : ainsi vous retrouverez votre joie en le voyant, et moi je serai moins triste. 
${}^{29}Dans le Seigneur, faites-lui donc un accueil vraiment joyeux, et tenez de telles personnes en grande estime : 
${}^{30}c’est pour l’œuvre du Christ qu’il a failli mourir ; il a risqué sa vie pour accomplir, à votre place, les services que vous ne pouviez me rendre vous-mêmes.
      
         
      \bchapter{}
      \begin{verse}
${}^{1}Enfin, mes frères, soyez dans la joie du Seigneur. Vous écrire les mêmes choses ne m’est pas pénible, et pour vous c’est plus sûr.
${}^{2}Prenez garde à ces chiens, prenez garde à ces mauvais ouvriers, avec leur fausse circoncision, prenez garde. 
${}^{3}Car c’est nous qui sommes les vrais circoncis, nous qui rendons notre culte par l’Esprit de Dieu, nous qui mettons notre fierté dans le Christ Jésus et qui ne plaçons pas notre confiance dans ce qui est charnel. 
${}^{4}J’aurais pourtant, moi aussi, des raisons de placer ma confiance dans la chair. Si un autre pense avoir des raisons de le faire, moi, j’en ai bien davantage : 
${}^{5}circoncis à huit jours, de la race d’Israël, de la tribu de Benjamin, Hébreu, fils d’Hébreux ; pour l’observance de la loi de Moïse, j’étais pharisien ; 
${}^{6}pour ce qui est du zèle, j’étais persécuteur de l’Église ; pour la justice que donne la Loi, j’étais devenu irréprochable.
${}^{7}Mais tous ces avantages que j’avais, je les ai considérés, à cause du Christ, comme une perte. 
${}^{8}Oui, je considère tout cela comme une perte à cause de ce bien qui dépasse tout : la connaissance du Christ Jésus, mon Seigneur. À cause de lui, j’ai tout perdu ; je considère tout comme des ordures, afin de gagner un seul avantage, le Christ, 
${}^{9}et, en lui, d’être reconnu juste, non pas de la justice venant de la loi de Moïse mais de celle qui vient de la foi au Christ, la justice venant de Dieu, qui est fondée sur la foi. 
${}^{10}Il s’agit pour moi de connaître le Christ, d’éprouver la puissance de sa résurrection et de communier aux souffrances de sa Passion, en devenant semblable à lui dans sa mort, 
${}^{11}avec l’espoir de parvenir à la résurrection d’entre les morts.
${}^{12}Certes, je n’ai pas encore obtenu cela, je n’ai pas encore atteint la perfection, mais je poursuis ma course pour tâcher de saisir, puisque j’ai moi-même été saisi par le Christ Jésus. 
${}^{13}Frères, quant à moi, je ne pense pas avoir déjà saisi cela. Une seule chose compte : oubliant ce qui est en arrière, et lancé vers l’avant, 
${}^{14}je cours vers le but en vue du prix auquel Dieu nous appelle là-haut dans le Christ Jésus. 
${}^{15}Nous tous qui sommes adultes dans la foi, nous devons avoir ces dispositions-là ; et, si vous en avez d’autres, là-dessus encore Dieu vous éclairera. 
${}^{16}En tout cas, du point où nous sommes arrivés, marchons dans la même direction.
${}^{17}Frères, ensemble imitez-moi, et regardez bien ceux qui se conduisent selon l’exemple que nous vous donnons. 
${}^{18}Car je vous l’ai souvent dit, et maintenant je le redis en pleurant : beaucoup de gens se conduisent en ennemis de la croix du Christ. 
${}^{19}Ils vont à leur perte. Leur dieu, c’est leur ventre, et ils mettent leur gloire dans ce qui fait leur honte ; ils ne pensent qu’aux choses de la terre. 
${}^{20}Mais nous, nous avons notre citoyenneté dans les cieux, d’où nous attendons comme sauveur le Seigneur Jésus Christ, 
${}^{21}lui qui transformera nos pauvres corps à l’image de son corps glorieux, avec la puissance active qui le rend même capable de tout mettre sous son pouvoir.
      
         
      \bchapter{}
      \begin{verse}
${}^{1}Ainsi, mes frères bien-aimés pour qui j’ai tant d’affection, vous, ma joie et ma couronne, tenez bon dans le Seigneur, mes bien-aimés.
      
         
${}^{2}J’exhorte Évodie, j’exhorte aussi Syntykhè, à se mettre d’accord dans le Seigneur. 
${}^{3}Oui, je te le demande à toi aussi, mon vrai compagnon d’effort, viens-leur en aide, à elles qui ont lutté avec moi pour l’annonce de l’Évangile, ainsi que Clément et mes autres collaborateurs, dont les noms se trouvent au livre de vie.
${}^{4}Soyez toujours dans la joie du Seigneur ; je le redis : soyez dans la joie. 
${}^{5}Que votre bienveillance soit connue de tous les hommes. Le Seigneur est proche. 
${}^{6}Ne soyez inquiets de rien, mais, en toute circonstance, priez et suppliez, tout en rendant grâce, pour faire connaître à Dieu vos demandes. 
${}^{7}Et la paix de Dieu, qui dépasse tout ce qu’on peut concevoir, gardera vos cœurs et vos pensées dans le Christ Jésus.
${}^{8}Enfin, mes frères, tout ce qui est vrai et noble, tout ce qui est juste et pur, tout ce qui est digne d’être aimé et honoré, tout ce qui s’appelle vertu et qui mérite des éloges, tout cela, prenez-le en compte. 
${}^{9}Ce que vous avez appris et reçu, ce que vous avez vu et entendu de moi, mettez-le en pratique. Et le Dieu de la paix sera avec vous.
${}^{10}J’ai éprouvé une grande joie dans le Seigneur à voir maintenant refleurir vos bonnes dispositions pour moi : elles étaient bien vivantes, mais vous n’aviez pas occasion de les montrer. 
${}^{11}Ce ne sont pas les privations qui me font parler ainsi, car j’ai appris à me contenter de ce que j’ai. 
${}^{12}Je sais vivre de peu, je sais aussi être dans l’abondance. J’ai été formé à tout et pour tout : à être rassasié et à souffrir la faim, à être dans l’abondance et dans les privations. 
${}^{13}Je peux tout en celui qui me donne la force.
${}^{14}Cependant, vous avez bien fait de vous montrer solidaires quand j’étais dans la gêne. 
${}^{15}Vous, les Philippiens, vous le savez : dans les premiers temps de l’annonce de l’Évangile, au moment où je quittais la Macédoine, je n’ai eu ma part dans les recettes et dépenses d’aucune Église, excepté la vôtre. 
${}^{16}À Thessalonique déjà, vous m’avez envoyé, et même deux fois, ce dont j’avais besoin. 
${}^{17}Je ne recherche pas les dons ; ce que je recherche, c’est le bénéfice qui s’ajoutera à votre compte. 
${}^{18}J’ai d’ailleurs tout reçu, je suis dans l’abondance ; je suis comblé depuis qu’Épaphrodite m’a remis votre envoi : c’est comme une offrande d’agréable odeur, un sacrifice digne d’être accepté et de plaire à Dieu. 
${}^{19}Et mon Dieu comblera tous vos besoins selon sa richesse, magnifiquement, dans le Christ Jésus. 
${}^{20}Gloire à Dieu notre Père pour les siècles des siècles. Amen.
${}^{21}Saluez chacun des fidèles dans le Christ Jésus. Les frères qui sont avec moi vous saluent. 
${}^{22}Tous les fidèles vous saluent, spécialement ceux qui font partie du personnel de l’empereur.
${}^{23}La grâce du Seigneur Jésus Christ soit avec votre esprit. Amen.
