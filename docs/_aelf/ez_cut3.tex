  
  
      
         
      \bchapter{}
      \begin{verse}
${}^{1}La parole du Seigneur me fut adressée : 
${}^{2}« Fils d’homme, dirige ton regard vers les fils d’Ammone et prophétise contre eux. 
${}^{3}Tu diras aux fils d’Ammone : Écoutez la parole du Seigneur Dieu.
      Ainsi parle le Seigneur Dieu : Parce que tu as ri de mon sanctuaire qui a été profané, de la terre d’Israël qui a été dévastée, de la maison de Juda qui est partie en exil, 
${}^{4}eh bien ! voici : je te livre en possession aux fils de l’Orient ; ils établiront chez toi leurs campements, ils feront chez toi leur demeure. Ce sont eux qui mangeront tes fruits, eux qui boiront ton lait. 
${}^{5}Je changerai Rabba en pâturage à chameaux, et les villes du pays des fils d’Ammone en bercail à brebis. Alors vous saurez que Je suis le Seigneur.
${}^{6}Ainsi parle le Seigneur Dieu : Parce que tu as battu des mains et tapé du pied, que tu as eu une joie profonde, un mépris total pour ce qui arrivait à la terre d’Israël, 
${}^{7}eh bien ! voici : j’étends la main contre toi, je vais te livrer aux nations pour être pillé, je te retrancherai d’entre les peuples, je te ferai disparaître d’entre les pays, je t’anéantirai. Alors tu sauras que Je suis le Seigneur.
${}^{8}Ainsi parle le Seigneur Dieu : Parce que Moab et Séïr ont dit : “Voilà que la maison de Juda est devenue comme toutes les nations”, 
${}^{9}eh bien ! voici : je vais dégarnir de leurs villes les contreforts de Moab, ces villes qui sont les joyaux des extrémités du pays : Beth-Yeshimoth, Baal-Meone et Qiryataïm. 
${}^{10}Je les livre en possession aux fils de l’Orient, et cela outre les fils d’Ammone, pour qu’on ne se souvienne plus des fils d’Ammone parmi les nations. 
${}^{11}Et je ferai justice de Moab. Alors on saura que Je suis le Seigneur.
${}^{12}Ainsi parle le Seigneur Dieu : Parce qu’Édom a agi par vengeance contre la maison de Juda, et qu’il s’est rendu gravement coupable en se vengeant d’elle, 
${}^{13}eh bien ! ainsi parle le Seigneur Dieu : Je vais étendre la main contre Édom, en retrancher hommes et bêtes. J’en ferai une ruine depuis Témane, et l’on tombera par l’épée jusqu’à Dedane. 
${}^{14}J’aurai ma revanche sur Édom par la main d’Israël, mon peuple. Il agira contre Édom selon ma colère et ma fureur. Alors ils connaîtront ma revanche – oracle du Seigneur Dieu.
${}^{15}Ainsi parle le Seigneur Dieu : Parce que les Philistins ont agi par vengeance, prenant leur revanche, le mépris au cœur, avec la volonté de détruire dans une haine perpétuelle, 
${}^{16}eh bien ! ainsi parle le Seigneur Dieu : Voici que je vais étendre la main contre les Philistins, je retrancherai les Kerétiens, je ferai disparaître ce qui reste du littoral de la mer. 
${}^{17}Je prendrai sur eux une grande revanche, je les châtierai dans ma fureur. Alors ils sauront que Je suis le Seigneur quand j’aurai ma revanche sur eux. »
      
         
      \bchapter{}
      \begin{verse}
${}^{1}La onzième année de la première déportation, le premier du mois, la parole du Seigneur me fut adressée :
${}^{2}« Fils d’homme, parce que Tyr a dit de Jérusalem :
        “Ha ! Ha ! La voilà brisée, la porte des peuples !
        À mon tour de me remplir, elle est ruinée !”
${}^{3}C’est pourquoi, ainsi parle le Seigneur Dieu :
        Tyr, me voici contre toi.
        \\Je soulève contre toi des nations nombreuses
        comme la mer soulève ses vagues.
${}^{4}Elles détruiront les remparts de Tyr, elles abattront ses tours ;
        je balaierai sa poussière, je mettrai à nu son rocher.
${}^{5}Elle deviendra au milieu de la mer un séchoir à filets
        – oui, moi, j’ai parlé, oracle du Seigneur Dieu –,
        elle sera pillée par les nations,
${}^{6}et ses filles dans la campagne seront tuées par l’épée.
        Alors on saura que Je suis le Seigneur.
         
${}^{7}Ainsi parle le Seigneur Dieu :
        Je ferai venir du nord contre Tyr
        \\Nabucodonosor, roi de Babylone, le roi des rois ;
        il viendra avec des chevaux, des chars,
        des cavaliers, tout un rassemblement de gens.
${}^{8}Par l’épée il tuera tes filles dans la campagne ;
        il bâtira contre toi des retranchements.
        \\il élèvera contre toi un remblai
        et contre toi dressera un mur de boucliers.
${}^{9}De son bélier, il donnera des coups contre tes remparts ;
        avec ses pioches, il démolira tes tours.
${}^{10}La foule de ses chevaux te couvrira de poussière ;
        le bruit des coursiers, des roues et des chars
        ébranlera tes remparts
        \\lorsqu’il entrera dans tes portes
        comme on entre dans une ville ouverte par une brèche.
${}^{11}Du sabot de ses chevaux, il foulera toutes tes rues,
        il tuera ton peuple par l’épée,
        et tes stèles colossales tomberont par terre.
${}^{12}Ils prendront en butin tes richesses,
        ils pilleront tes marchandises,
        \\ils abattront tes remparts
        et ils démoliront tes luxueuses maisons ;
        \\ils jetteront au fond de l’eau tes pierres,
        tes boiseries et même ta poussière.
${}^{13}Je ferai cesser le bruit de tes chansons,
        et la voix de tes cithares ne se fera plus entendre.
${}^{14}Je mettrai à nu ton rocher,
        tu deviendras un séchoir à filets,
        tu ne seras plus rebâtie.
        \\Je suis le Seigneur, j’ai parlé
        – oracle du Seigneur Dieu.
         
${}^{15}Ainsi parle le Seigneur Dieu à la ville de Tyr :
        Est-ce que le bruit de ta chute, le râle des victimes,
        \\le carnage qui s’accomplira au milieu de toi,
        ne vont pas ébranler les îles ?
${}^{16}Tous les princes de la mer descendront de leurs trônes,
        ils ôteront leurs manteaux,
        de leurs vêtements chamarrés ils se dépouilleront ;
        \\ils se revêtiront d’effroi, et par terre ils s’assiéront,
        ils trembleront sans cesse et se désoleront sur toi.
${}^{17}Ils entonneront sur toi une complainte. Ils diront :
        Comment a-t-elle disparu,
        celle dont les habitants venaient de la mer,
        la ville fameuse dont la force était sur les mers,
        et dont les habitants répandaient partout la terreur ?
${}^{18}Maintenant, au jour de ta chute, les îles tremblent,
        les îles de la mer sont épouvantées au jour de ta fin.
       
${}^{19}Ainsi parle le Seigneur Dieu : Quand je ferai de toi une ville en ruine, pareille aux villes inhabitées, quand je ferai monter contre toi l’Abîme et que les grandes eaux te recouvriront, 
${}^{20}je te ferai descendre avec ceux qui sont descendus dans la fosse, vers les gens d’autrefois ; je te ferai habiter le pays des profondeurs, semblable à des ruines éternelles, avec ceux qui sont descendus dans la fosse : tu ne seras plus habitée. Je donnerai de la splendeur à la terre des vivants, 
${}^{21}et de toi je ferai un objet d’épouvante ; tu ne seras plus. On te cherchera mais on ne te trouvera plus jamais – oracle du Seigneur Dieu. »
      
         
      \bchapter{}
      \begin{verse}
${}^{1}La parole du Seigneur me fut adressée : 
${}^{2}« Écoute, fils d’homme, entonne une complainte sur Tyr. 
${}^{3}Tu diras à la ville de Tyr :
        \\Toi qui habites aux portes de la mer,
        \\toi qui fais du commerce avec les peuples,
        avec les îles nombreuses,
        \\ainsi parle le Seigneur Dieu :
        \\Tyr, toi qui disais : “Je suis d’une beauté parfaite”,
${}^{4}ton territoire s’étend au cœur des mers,
        tes constructeurs ont rendu parfaite ta beauté.
${}^{5}En cyprès de Senir, ils ont construit
        toutes les parois de ton bateau,
        \\ils ont pris un cèdre du Liban
        pour ériger ton grand mât.
${}^{6}En chêne du Bashane, ils ont fait les rangées de tes rames ;
        ton habitacle, ils l’ont fait d’ivoire
        enchâssé dans du buis des îles de Kittim ;
${}^{7}Ta voile, en lin chamarré venu d’Égypte,
        te servait de pavillon.
        \\Ta bâche était en pourpre violette
        et en pourpre rouge des îles d’Élisha.
${}^{8}Tu avais comme rameurs
        les habitants de Sidon et d’Arvad ;
        \\Tyr, les plus habiles de chez toi
        étaient tes navigateurs.
${}^{9}Les anciens de Guebal et ses gens habiles
        étaient chez toi pour réparer tes avaries.
        \\Tous les navires de la mer et leurs matelots
        venaient chez toi pour négocier tes marchandises.
${}^{10}Les Perses, les gens de Loud et de Pouth
        servaient dans ton armée,
        ils étaient tes hommes de guerre ;
        \\ils suspendaient chez toi boucliers et casques,
        ils faisaient ta force.
${}^{11}Les fils d’Arvad et ton armée étaient autour de toi
        sur tes remparts,
        et les Gammadiens sur tes tours.
        \\Ils suspendaient leurs carquois à tes murs d’enceinte ;
        ils rendaient parfaite ta beauté.
       
${}^{12}Tarsis échangeait avec toi toutes sortes de biens en abondance : ses gens te fournissaient de l’argent, du fer, de l’étain et du plomb, contre tes produits. 
${}^{13}Yavane, Toubal et Mèshek commerçaient avec toi ; ils te fournissaient comme marchandises des êtres humains et des objets de bronze. 
${}^{14}De Beth-Togarma, on te fournissait contre tes produits des chevaux, des cavaliers et des mulets. 
${}^{15}Les fils de Dedane commerçaient avec toi ; avec toi, des îles nombreuses faisaient des échanges et t’apportaient en paiement des cornes d’ivoire, des troncs d’ébène. 
${}^{16}Aram échangeait avec toi ce que tu fabriquais en abondance. On te fournissait contre tes produits des escarboucles, des étoffes de pourpre rouge, des étoffes chamarrées, du lin, du corail, des rubis. 
${}^{17}Juda et le pays d’Israël eux-mêmes commerçaient avec toi ; ils te fournissaient du blé de Minnith, du millet, du miel, de l’huile et du baume. 
${}^{18}Damas échangeait avec toi, contre ce que tu fabriquais en abondance, toutes sortes de biens : du vin de Helbone et de la laine de Çahar. 
${}^{19}Contre tes produits, Wedane et Yavane-Méouzzal te fournissaient en fer forgé, en casse, en roseau aromatique destinés à tes marchés. 
${}^{20}Dedane faisait avec toi commerce de couvertures de cheval. 
${}^{21}L’Arabie et tous les princes de Qédar eux-mêmes faisaient des échanges avec toi ; ils te payaient en agneaux, béliers et boucs. 
${}^{22}Les commerçants de Saba et de Raéma commerçaient avec toi ; contre tes produits, ils te fournissaient les meilleurs de tous les aromates, toutes sortes de pierres précieuses et de l’or. 
${}^{23}Harrane, Kanné et Éden, les commerçants de Saba, Assour, Kilmad commerçaient avec toi. 
${}^{24}Eux faisaient commerce avec toi de vêtements d’apparat, de manteaux de pourpre violette et d’étoffes chamarrées, de brocarts chatoyants, de cordes solidement tressées, qu’on trouvait dans tes souks. 
${}^{25}Les navires de Tarsis naviguaient pour ton commerce.
       
        \\Tu as été remplie, chargée lourdement au cœur des mers.
${}^{26}Sur les grandes eaux, tu fus conduite par tes rameurs ;
        le vent d’est te brisa au cœur des mers.
${}^{27}Tes biens, tes produits, tes marchandises,
        tes matelots, tes navigateurs,
        ceux qui réparent tes avaries,
        ceux qui négocient tes marchandises,
        tous les hommes de guerre,
        tous ceux qui s’assemblent chez toi,
        \\couleront au cœur des mers, le jour de ton naufrage.
${}^{28}Au cri poussé par tes navigateurs,
        les rivages trembleront.
${}^{29}Alors tous les rameurs descendront de leurs navires ;
        les matelots et tous les gens de mer resteront à terre.
${}^{30}Ils feront entendre leur voix à ton sujet,
        ils crieront amèrement,
        \\se mettront de la poussière sur la tête,
        se rouleront dans la cendre.
${}^{31}Ils se raseront le crâne à cause de toi,
        ils se revêtiront de toile à sac.
        \\Ils pleureront sur toi ;
        dans l’amertume de leur cœur, ils se lamenteront amèrement.
${}^{32}Dans leur douleur, ils entonneront une complainte sur toi,
        sur toi ils élèveront une plainte :
        \\“Qui est comme Tyr,
        réduite au silence au milieu de la mer ?”
${}^{33}En exportant tes produits sur les mers,
        tu rassasiais de nombreux peuples ;
        \\l’abondance de tes biens et de tes marchandises
        enrichissait les rois de la terre.
${}^{34}Maintenant te voilà brisée par les mers,
        dans les profondeurs des eaux ;
        \\tes marchandises et la foule assemblée chez toi ont sombré.
${}^{35}Tous les habitants des îles sont pris de stupeur à cause de toi,
        leurs rois sont horrifiés, les visages bouleversés.
${}^{36}Ceux qui font du commerce parmi les peuples sifflent sur toi :
        tu es devenue un objet d’épouvante.
        Tu ne seras jamais plus. »
      
         
      \bchapter{}
      \begin{verse}
${}^{1}La parole du Seigneur me fut adressée : 
${}^{2} « Fils d’homme, tu diras au prince de la ville\\de Tyr : Ainsi parle le Seigneur Dieu :
        \\Ton cœur s’est exalté
        et tu as dit : “Je suis un dieu\\,
        \\j’habite une résidence divine,
        au cœur des mers.”
        \\Pourtant, tu es un homme et non un dieu\\,
        toi qui prends tes pensées pour des pensées divines.
        ${}^{3}Tu serais donc plus sage que Daniel\\,
        il n’y aurait pas de secret trop profond pour toi ?
        ${}^{4}Par ta sagesse et ton intelligence
        tu as fait fortune,
        tu as accumulé l’or et l’argent dans tes trésors.
        ${}^{5}Par ton génie du commerce,
        tu as multiplié ta fortune,
        et à cause de cette fortune ton cœur s’est exalté.
        ${}^{6}C’est pourquoi, ainsi parle le Seigneur Dieu :
        \\Parce que tu prends tes pensées pour des pensées divines,
        ${}^{7}je fais venir contre toi des barbares,
        une nation redoutable\\.
        \\Ils tireront l’épée contre ta belle sagesse,
        ils profaneront ta splendeur.
        ${}^{8}Ils te feront descendre dans la fosse
        et tu mourras au cœur des mers,
        d’une mort violente\\.
        ${}^{9}Oseras-tu dire encore devant tes meurtriers :
        “Je suis dieu\\” ?
        \\Sous la main de ceux qui te transperceront,
        tu seras un homme et non un dieu.
        ${}^{10}Tu mourras de la mort des païens\\incirconcis,
        par la main des barbares.
        \\Oui, moi, j’ai parlé,
        – oracle du Seigneur Dieu. »
       
${}^{11}La parole du Seigneur me fut adressée : 
${}^{12}« Fils d’homme, entonne une complainte sur le roi de Tyr. Tu lui diras : Ainsi parle le Seigneur Dieu :
        \\Toi, le sceau d’une œuvre exemplaire,
        plein de sagesse, d’une beauté parfaite,
${}^{13}tu étais en Éden, dans le jardin de Dieu,
        entouré de murs en pierres précieuses :
        \\sardoine, topaze et jaspe, chrysolithe, cornaline et onyx,
        saphir, escarboucle et émeraude ;
        \\l’or ouvragé de tes tambourins et de tes flûtes
        a été préparé au jour de ta création.
${}^{14}Toi, Kéroub choisi, le protecteur,
        je t’avais établi sur la sainte montagne de Dieu ;
        au milieu des pierres étincelantes, tu allais et venais.
${}^{15}Tu fus intègre dans ta conduite depuis le jour de ta création,
        jusqu’à ce que soit découverte en toi la perfidie :
${}^{16}en multipliant tes affaires,
        tu t’es rempli de violence, et tu as péché.
        \\Aussi, je te réduis au rang de profane,
        loin de la montagne de Dieu ;
        \\je te fais périr, Kéroub protecteur,
        loin des pierres étincelantes.
${}^{17}Ton cœur s’est exalté à cause de ta beauté,
        tu laissas ta splendeur corrompre ta sagesse.
        \\Je te jette par terre,
        je te donne en spectacle aux rois.
${}^{18}Par tes multiples péchés,
        par la perversion de tes affaires,
        tu as profané ton sanctuaire.
        \\Aussi je fais sortir du milieu de toi
        un feu qui te dévore ;
        \\sur la terre, je te réduis en cendres,
        devant les yeux de tous ceux qui te regardent.
${}^{19}Tous ceux qui te connaissent, parmi les peuples,
        sont pris de stupeur à cause de toi.
        \\Tu deviens un objet d’épouvante ;
        tu ne seras plus, à jamais. »
${}^{20}La parole du Seigneur me fut adressée : 
${}^{21}« Fils d’homme, dirige ton regard vers Sidon, et prophétise contre elle. 
${}^{22}Tu diras : Ainsi parle le Seigneur Dieu :
        \\Me voici contre toi, Sidon,
        je serai glorifié au milieu de toi.
        \\Alors on saura que Je suis le Seigneur,
        lorsque j’exécuterai le jugement contre elle.
        \\Alors, je manifesterai en elle ma sainteté.
${}^{23}En elle j’enverrai la peste,
        il y aura du sang dans ses rues ;
        \\on tombera au milieu d’elle, transpercé par l’épée
        qui se lèvera contre elle de tous côtés.
        \\Alors on saura que Je suis le Seigneur.
       
${}^{24}Contre la maison d’Israël, parmi tous ceux qui alentour la méprisent, il n’y aura plus de ronces qui blessent ou d’épines piquantes. Alors on saura que Je suis le Seigneur.
       
${}^{25}Ainsi parle le Seigneur Dieu : Quand je rassemblerai la maison d’Israël du milieu des peuples où elle a été dispersée, je manifesterai en elle ma sainteté aux yeux des nations ; elle habitera sur le sol que j’ai donné à mon serviteur Jacob. 
${}^{26}Ils y habiteront en sécurité, ils bâtiront des maisons, planteront des vignes ; ils habiteront en sécurité, lorsque j’exécuterai le jugement contre ceux d’alentour qui les méprisent. Alors on saura que Je suis le Seigneur leur Dieu. »
      
         
      \bchapter{}
      \begin{verse}
${}^{1}La dixième année de la première déportation, le dixième mois, le douze du mois, la parole du Seigneur me fut adressée : 
${}^{2}« Fils d’homme, dirige ton regard vers Pharaon, roi d’Égypte, et prophétise contre lui et contre l’Égypte tout entière. 
${}^{3}Parle. Tu diras : Ainsi parle le Seigneur Dieu :
        \\Me voici contre toi, Pharaon, roi d’Égypte,
        grand dragon tapi parmi les bras du Nil ;
        \\tu as dit : il est à moi, le Nil,
        c’est moi qui l’ai fait.
${}^{4}Je mettrai des crochets à tes mâchoires,
        je collerai à tes écailles les poissons de tes Nils ;
        \\je te tirerai hors de tes Nils,
        et tous les poissons de tes Nils colleront à tes écailles.
${}^{5}Je te rejetterai dans le désert,
        toi et tous les poissons de tes Nils.
        \\Tu retomberas en plein champ,
        tu ne seras ni ramassé ni recueilli.
        \\Je te donnerai en pâture aux bêtes de la terre
        et aux oiseaux du ciel.
${}^{6}Alors tous les habitants de l’Égypte
        sauront que Je suis le Seigneur.
        Ils furent un appui de roseau pour la maison d’Israël :
${}^{7}lorsqu’on te saisissait, Pharaon, tu te brisais dans la main,
        tu transperçais toute la main ;
        \\pour ceux qui s’appuyaient sur toi,
        tu te cassais, tu faisais défaillir leurs reins.
${}^{8}C’est pourquoi, ainsi parle le Seigneur Dieu : Je fais venir sur toi l’épée ; je retrancherai de toi hommes et bêtes. 
${}^{9}Le pays d’Égypte deviendra désolation et ruines. Alors on saura que Je suis le Seigneur.
      Parce que Pharaon a dit : “Le Nil est à moi, c’est moi qui l’ai fait”, 
${}^{10}eh bien, me voici contre toi et contre tes Nils : je ferai du pays d’Égypte des ruines et une désolation, de Migdol à Assouan et jusqu’à la frontière de l’Éthiopie. 
${}^{11}Le pied des hommes n’y passera pas ; le pied des bêtes n’y passera pas. Le pays restera inhabité durant quarante ans. 
${}^{12}Je ferai du pays d’Égypte une désolation au milieu de pays dévastés ; au milieu de villes en ruine, ses villes seront une désolation, durant quarante ans. Et je vais disperser les Égyptiens parmi les nations, les disséminer parmi les pays.
${}^{13}Mais – ainsi parle le Seigneur Dieu – au bout de quarante ans, je rassemblerai les Égyptiens d’entre les peuples où ils auront été dispersés. 
${}^{14}Je changerai la destinée des Égyptiens ; je les ferai revenir au pays de Patros, leur pays d’origine. Ils formeront un royaume modeste. 
${}^{15}Il sera plus modeste que les autres royaumes ; il ne s’élèvera plus au-dessus des nations. Je l’amoindrirai pour qu’il ne domine plus les nations. 
${}^{16}Il n’offrira plus à la maison d’Israël une sécurité qui la poussait à pécher en se tournant vers l’Égypte. Alors on saura que Je suis le Seigneur Dieu. »
       
${}^{17}La vingt-septième année de la première déportation, le premier mois, le premier du mois, la parole du Seigneur me fut adressée : 
${}^{18}« Fils d’homme, Nabucodonosor, roi de Babylone, a engagé son armée dans un grand effort contre Tyr : tous les crânes sont pelés, toutes les épaules écorchées, mais ni lui ni son armée n’ont retiré aucun salaire de Tyr pour l’effort qu’ils ont engagé contre la ville. 
${}^{19}C’est pourquoi, ainsi parle le Seigneur Dieu : Je vais donner le pays d’Égypte à Nabucodonosor, roi de Babylone ; il enlèvera ses richesses, prendra son butin, la pillera complètement. L’Égypte servira de salaire à son armée. 
${}^{20}En compensation de l’effort qu’il a fourni, je lui donne le pays d’Égypte, parce que lui et son armée ont travaillé pour moi – oracle du Seigneur Dieu. 
${}^{21}En ce jour, je ferai grandir la puissance de la maison d’Israël ; quant à toi, fils d’homme, je te donnerai d’ouvrir la bouche au milieu d’eux. Alors ils sauront que Je suis le Seigneur. »
      
         
      \bchapter{}
      \begin{verse}
${}^{1}La parole du Seigneur me fut adressée : 
${}^{2}« Fils d’homme, prophétise. Tu diras : Ainsi parle le Seigneur Dieu :
        \\Hurlez pour ce jour de malheur !
${}^{3}Car le jour est proche,
        il est proche le jour du Seigneur.
        \\Ce sera un jour chargé de nuages ;
        ce sera le temps des nations.
${}^{4}L’épée viendra en Égypte ;
        on tremblera en Éthiopie
        \\quand les morts tomberont en Égypte,
        quand on s’emparera de ses richesses,
        et qu’on démolira ses fondations.
${}^{5}Koush, Pouth, Loud, tout ce mélange de peuples,
        \\Koub et les gens du pays allié
        tomberont avec eux sous l’épée.
         
${}^{6}Ainsi parle le Seigneur :
        \\Ils tomberont, les soutiens de l’Égypte ;
        l’orgueil de sa force s’effondrera ;
        \\de Migdol à Assouan, on tombera par l’épée
        – oracle du Seigneur Dieu.
${}^{7}Ces lieux seront dévastés, au milieu de pays dévastés ; ses villes seront au milieu de villes en ruine. 
${}^{8}Alors on saura que Je suis le Seigneur, lorsque je mettrai le feu à l’Égypte et que tous ceux qui l’aident seront brisés. 
${}^{9}En ce jour-là, des messagers s’en iront de ma part en bateau pour inquiéter la tranquille Éthiopie ; chez elle on tremblera au jour de l’Égypte. Oui, cela vient.
       
${}^{10}Ainsi parle le Seigneur Dieu :
        \\Je supprimerai les richesses de l’Égypte,
        par la main de Nabucodonosor, roi de Babylone.
${}^{11}Lui-même et avec lui son peuple, la plus redoutable des nations,
        ont été amenés pour détruire le pays.
        \\Ils tireront l’épée contre l’Égypte,
        ils rempliront de victimes le pays.
${}^{12}Les bras du Nil, je les mettrai à sec ;
        je livrerai le pays à la main de méchants,
        \\par la main d’étrangers je dévasterai le pays
        et ce qu’il contient.
        \\Je suis le Seigneur, j’ai parlé.
         
${}^{13}Ainsi parle le Seigneur Dieu :
        \\Je ferai périr les idoles immondes,
        je supprimerai de Memphis les faux dieux.
        \\Au pays d’Égypte il n’y aura plus de prince ;
        je répandrai la crainte dans le pays d’Égypte.
${}^{14}Je dévasterai Patros,
        je mettrai le feu à Tanis,
        j’exécuterai le jugement contre Thèbes.
${}^{15}Je déverserai ma fureur sur Sine,
        place forte de l’Égypte,
        \\je ferai disparaître les richesses de Thèbes.
${}^{16}Je mettrai le feu à l’Égypte,
        Sine se tordra de douleur,
        \\Thèbes sera démantelée,
        et Memphis assaillie en plein jour.
${}^{17}Les jeunes hommes des villes d’One et de Pi-Bèseth
        tomberont par l’épée ;
        \\et voici que les jeunes filles s’en iront captives.
${}^{18}À Daphné, le jour se changera en ténèbres
        lorsque je briserai les sceptres de l’Égypte
        et que l’orgueil de sa force sera supprimé.
        \\Un nuage la recouvrira,
        et ses filles s’en iront captives.
${}^{19}J’exécuterai mon jugement contre l’Égypte.
        Alors on saura que Je suis le Seigneur. »
       
${}^{20}La onzième année de la première déportation, le premier mois, le sept du mois, la parole du Seigneur me fut adressée : 
${}^{21}« Fils d’homme, le bras de Pharaon, roi d’Égypte, je le brise, et on ne le soigne pas : personne n’y applique de remède, ne lui met de bandage, personne ne le soigne pour qu’il retrouve la force de tenir l’épée. 
${}^{22}C’est pourquoi, ainsi parle le Seigneur Dieu : Me voici contre Pharaon, roi d’Égypte. Je lui briserai les bras, celui qui est valide et celui qui est déjà brisé, et je ferai tomber l’épée de sa main. 
${}^{23}Je vais disperser les Égyptiens parmi les nations, les disséminer parmi les pays. 
${}^{24}Mais je fortifierai les bras du roi de Babylone et je mettrai mon épée dans sa main ; je briserai les bras de Pharaon qui poussera les gémissements d’un homme blessé à mort. 
${}^{25}Je fortifierai les bras du roi de Babylone, mais les bras de Pharaon tomberont. Alors on saura que Je suis le Seigneur, quand je mettrai mon épée dans la main du roi de Babylone et qu’il la brandira contre le pays d’Égypte. 
${}^{26}Je vais disperser les Égyptiens parmi les nations, les disséminer parmi les pays. Alors on saura que Je suis le Seigneur. »
      
         
      \bchapter{}
      \begin{verse}
${}^{1}La onzième année de la première déportation, le troisième mois, le premier du mois, la parole du Seigneur me fut adressée :
${}^{2}« Fils d’homme, dis à Pharaon, roi d’Égypte,
        et à sa multitude :
        \\À qui te comparer dans ta grandeur ?
${}^{3}Voici : un cèdre du Liban avait une belle ramure,
        des branchages produisant de l’ombre,
        \\et une taille si élevée
        que son sommet était au milieu des nuages.
${}^{4}Les eaux l’ont fait grandir ;
        l’Abîme qui lui a donné de croître
        \\faisait couler ses fleuves
        autour du lieu où il était planté,
        \\et dirigeait ses canaux
        vers tous les arbres de la campagne.
${}^{5}Ainsi sa taille était-elle plus élevée
        que celle de tous les arbres de la campagne,
        \\ses surgeons s’étaient multipliés,
        ses branches, allongées,
        grâce aux eaux abondantes qui coulaient vers lui.
${}^{6}Dans ses rameaux nichaient tous les oiseaux du ciel,
        sous ses branches toutes les bêtes sauvages mettaient bas,
        et à son ombre habitaient de nombreuses nations.
${}^{7}Il était beau par sa grandeur,
        par l’ampleur de son branchage ;
        ses racines s’étendaient jusqu’aux eaux abondantes.
${}^{8}Les cèdres ne l’égalaient pas
        dans le jardin de Dieu,
        \\les cyprès n’étaient pas comparables à ses branches,
        ni les platanes à ses rameaux ;
        \\aucun arbre dans le jardin de Dieu
        ne lui était comparable en beauté.
${}^{9}Je l’avais rendu beau
        par l’abondance de ses branches ;
        \\tous les arbres d’Éden, dans le jardin de Dieu,
        le jalousaient.
${}^{10}C’est pourquoi, ainsi parle le Seigneur Dieu : Parce qu’il a haussé sa taille, parce que son sommet atteint les nuages, que son cœur s’est élevé avec orgueil, 
${}^{11}je le livre aux mains du tyran des nations qui le traitera selon sa méchanceté. Je l’ai chassé.
${}^{12}Des étrangers, parmi les plus redoutables des nations, l’ont abattu, puis abandonné. Son branchage est tombé sur les montagnes et dans toutes les vallées, ses branches ont été brisées dans tous les ravins du pays, tous les gens du pays ont quitté son ombre, et l’ont abandonné.
${}^{13}Tous les oiseaux du ciel se posent sur ses débris,
        toutes les bêtes sauvages gîtent dans ses branches.
${}^{14}Cela est arrivé pour qu’aucun arbre bien arrosé ne hausse sa taille, que son sommet n’atteigne les nuages, et que, bien abreuvé, il ne se tienne avec orgueil au-dessus des autres. Car tous, ils sont livrés à la mort, au pays des profondeurs, au milieu des fils d’hommes, auprès de ceux qui descendent à la fosse.
${}^{15}Ainsi parle le Seigneur Dieu : Le jour où le cèdre est descendu au séjour des morts, j’ai fait prendre le deuil ; sur lui j’ai refermé l’Abîme, j’ai arrêté ses fleuves, et les grandes eaux ont été retenues ; à cause de lui j’ai assombri le Liban ; j’ai fait dépérir tous les arbres de la campagne à cause de lui. 
${}^{16}J’ai fait trembler les nations au bruit de sa chute, lorsque je le fis descendre au séjour des morts avec ceux qui descendent à la fosse. Alors dans le pays des profondeurs, tous les arbres d’Éden ont eu leur revanche, les arbres de choix, les meilleurs du Liban, tous ceux qui étaient abreuvés d’eau. 
${}^{17}Ceux-là aussi sont descendus avec lui au séjour des morts, auprès de ceux qui ont été transpercés par l’épée. Ils étaient son bras et habitaient à son ombre au milieu des nations. 
${}^{18}À qui donc comparer ta gloire et ta grandeur parmi les arbres d’Éden ? On t’a fait descendre avec les arbres d’Éden au pays des profondeurs ; tu reposeras au milieu des incirconcis, avec ceux qui ont été transpercés par l’épée. Tel sera Pharaon et toute sa multitude – oracle du Seigneur Dieu. »
      
         
      \bchapter{}
      \begin{verse}
${}^{1}La douzième année de la première déportation, le douzième mois, le premier du mois, la parole du Seigneur me fut adressée : 
${}^{2}« Fils d’homme, entonne une complainte sur Pharaon, roi d’Égypte. Tu lui diras :
        \\Tu étais comparable au lionceau des nations,
        tu étais comme un dragon dans les mers,
        \\tu faisais jaillir tes fleuves,
        tes pattes troublaient l’eau,
        tu agitais les fleuves.
${}^{3}Ainsi parle le Seigneur Dieu :
        \\J’étendrai sur toi mon filet
        lorsque s’assembleront des peuples nombreux ;
        ils te tireront dans ses mailles.
${}^{4}Je te rejetterai à terre,
        en plein champ je te lancerai,
        \\je ferai que se posent sur toi tous les oiseaux du ciel,
        je rassasierai de toi les bêtes de toute la terre.
${}^{5}Je mettrai ta chair sur les montagnes,
        je remplirai les vallées de ta pourriture,
${}^{6}j’abreuverai la terre de ton fumier
        et de ton sang répandu sur les montagnes,
        les ravins en seront remplis.
${}^{7}Lorsque tu t’éteindras, je voilerai les cieux,
        j’obscurcirai les étoiles,
        \\je voilerai d’une nuée le soleil,
        la lune ne laissera plus luire sa lumière.
${}^{8}Tous les luminaires des cieux, je les obscurcirai à cause de toi,
        je répandrai les ténèbres sur ton pays
        – oracle du Seigneur Dieu.
${}^{9}Je bouleverserai le cœur de peuples nombreux quand je ferai savoir ton écroulement aux nations, à ces pays que tu ne connais pas. 
${}^{10}À ton sujet, je frapperai de stupeur des peuples nombreux, leurs rois seront horrifiés à cause de toi quand je brandirai mon épée devant leur visage. Ils ne cesseront de trembler, chacun pour sa vie, au jour de ta chute.
${}^{11}Car ainsi parle le Seigneur Dieu :
        \\L’épée du roi de Babylone pénétrera en toi.
${}^{12}Je ferai tomber ta multitude sous l’épée des guerriers.
        \\Ce sont les plus redoutables des nations,
        elles anéantiront l’orgueil de l’Égypte,
        et toute sa multitude sera exterminée.
${}^{13}Je ferai périr tout son bétail
        au bord des grandes eaux ;
        \\le pied de l’homme ne les troublera plus,
        les sabots du bétail ne les troubleront plus.
${}^{14}Alors j’apaiserai les eaux de l’Égypte
        et je ferai couler ses fleuves comme de l’huile,
        – oracle du Seigneur Dieu.
${}^{15}Quand j’aurai fait du pays d’Égypte une désolation,
        quand j’aurai vidé le pays de ce qu’il contient
        \\et que je frapperai tous ceux qui y habitent,
        alors on saura que Je suis le Seigneur. »
${}^{16}C’est une complainte. Les filles des nations la chanteront. Qu’elles chantent cette complainte sur l’Égypte et sur toute sa multitude ! Oui, qu’elles chantent cette complainte – oracle du Seigneur Dieu.
       
${}^{17}La douzième année de la première déportation, le quinze du mois, la parole du Seigneur me fut adressée : 
${}^{18}« Fils d’homme, lamente-toi sur la multitude de l’Égypte ; fais-la descendre, elle et les filles des nations majestueuses, dans le pays des profondeurs avec ceux qui descendent à la fosse. 
${}^{19}Serais-tu plus agréable que les autres ? Descends, repose avec les incirconcis. 
${}^{20}Ils tombent au milieu des victimes de l’épée. L’épée a été tirée. Entraînez l’Égypte et toute sa multitude ! 
${}^{21}Du milieu du séjour des morts, les plus puissants guerriers avec ceux qui les aident lui diront : “Ils sont descendus, ils reposent, les incirconcis, victimes de l’épée.” »
       
${}^{22}Là se trouvent Assour avec toute sa troupe ; autour de lui, ses tombeaux. Tous, ils sont des victimes, tombées sous l’épée. 
${}^{23}Leurs tombeaux ont été placés au tréfonds de la fosse. Voici la troupe d’Assour autour de sa tombe. Tous, ils sont des victimes, tombées sous l’épée, eux qui répandaient la terreur sur la terre des vivants.
${}^{24}Là se trouvent Élam et toute sa multitude autour de sa tombe. Tous, ils sont des victimes, tombées sous l’épée. Ils sont descendus incirconcis au pays des profondeurs, eux qui répandaient la terreur sur la terre des vivants. Ils portent leur déshonneur avec ceux qui sont descendus à la fosse. 
${}^{25}Au milieu des victimes, on a fait reposer Élam, parmi toute la multitude ; autour de lui, ses tombeaux. Tous ces incirconcis ont été victimes de l’épée, car ils répandaient la terreur sur la terre des vivants. Ils portent leur déshonneur avec ceux qui sont descendus à la fosse. Élam est placé au milieu des victimes.
${}^{26}Là se trouvent Mèshek-Toubal et toute sa multitude ; autour de lui, ses tombeaux. Tous ces incirconcis ont été victimes de l’épée, car ils répandaient la terreur sur la terre des vivants. 
${}^{27}Ils ne reposent pas avec les guerriers tombés parmi les incirconcis, eux qui descendirent au séjour des morts, les armes à la main. On a placé leurs épées sous leurs têtes, leurs boucliers sont sur leurs ossements. Oui, ces guerriers terrifiaient la terre des vivants ! 
${}^{28}Mais toi, c’est au milieu des incirconcis que tu seras mis en pièces, et tu reposeras avec les victimes de l’épée.
${}^{29}Là se trouve Édom, ses rois et tous ses princes. Eux qui étaient vaillants, ils ont été placés avec les victimes de l’épée ; ils reposent avec les incirconcis et avec ceux qui sont descendus à la fosse.
${}^{30}Là se trouvent tous les chefs du Nord et tous les Sidoniens qui sont descendus avec les victimes. Eux dont la vaillance provoquait la terreur, ils sont remplis de honte. Ils reposent, incirconcis, avec les victimes de l’épée ; ils portent leur déshonneur avec ceux qui sont descendus à la fosse.
${}^{31}Pharaon les verra et il se consolera à cause de toute cette multitude. Mais ils seront victimes de l’épée, Pharaon et toute son armée – oracle du Seigneur Dieu. 
${}^{32}Oui, je l’ai laissé répandre la terreur sur la terre des vivants, mais on le fera reposer au milieu des incirconcis avec les victimes de l’épée, lui, Pharaon et toute sa multitude – oracle du Seigneur Dieu. »
