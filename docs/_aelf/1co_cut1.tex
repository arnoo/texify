  
  
    
    \bbook{PREMIÈRE LETTRE AUX CORINTHIENS}{PREMIÈRE LETTRE AUX CORINTHIENS}
      
         
      \bchapter{}
        ${}^{1}Paul, appelé par la volonté de Dieu
        pour être apôtre du Christ Jésus,
        \\et Sosthène notre frère,
        ${}^{2}à l’Église de Dieu qui est à Corinthe,
        \\à ceux qui ont été sanctifiés dans le Christ Jésus
        et sont appelés à être saints
        \\avec tous ceux qui, en tout lieu,
        invoquent le nom de notre Seigneur Jésus Christ,
        leur Seigneur et le nôtre.
        ${}^{3}À vous, la grâce et la paix,
        \\de la part de Dieu notre Père
        et du Seigneur Jésus Christ.
        
           
${}^{4}Je ne cesse de rendre grâce à Dieu à votre sujet, pour la grâce qu’il vous a donnée dans le Christ Jésus ; 
${}^{5}en lui vous avez reçu toutes les richesses, toutes celles de la parole et de la connaissance de Dieu. 
${}^{6}Car le témoignage rendu au Christ s’est établi fermement parmi vous. 
${}^{7}Ainsi, aucun don de grâce ne vous manque, à vous qui attendez de voir se révéler notre Seigneur Jésus Christ. 
${}^{8}C’est lui qui vous fera tenir fermement jusqu’au bout, et vous serez sans reproche au jour de notre Seigneur Jésus Christ. 
${}^{9}Car Dieu est fidèle, lui qui vous a appelés à vivre en communion avec son Fils, Jésus Christ notre Seigneur.
${}^{10}Frères, je vous exhorte au nom de notre Seigneur Jésus Christ : ayez tous un même langage ; qu’il n’y ait pas de division entre vous, soyez en parfaite harmonie de pensées et d’opinions. 
${}^{11}Il m’a été rapporté à votre sujet, mes frères, par les gens de chez Chloé, qu’il y a entre vous des rivalités. 
${}^{12}Je m’explique. Chacun de vous prend parti en disant : « Moi, j’appartiens à Paul », ou bien : « Moi, j’appartiens à Apollos », ou bien : « Moi, j’appartiens à Pierre », ou bien : « Moi, j’appartiens au Christ ». 
${}^{13}Le Christ est-il donc divisé ? Est-ce Paul qui a été crucifié pour vous ? Est-ce au nom de Paul que vous avez été baptisés ? 
${}^{14}Je remercie Dieu de n’avoir baptisé aucun de vous, sauf Crispus et Gaïus : 
${}^{15}ainsi on ne pourra pas dire que vous avez été baptisés en mon nom. 
${}^{16}En fait, j’ai aussi baptisé Stéphanas et les gens de sa maison ; et je ne sais plus si j’ai baptisé quelqu’un d’autre. 
${}^{17}Le Christ, en effet, ne m’a pas envoyé pour baptiser, mais pour annoncer l’Évangile, et cela sans avoir recours au langage de la sagesse humaine, ce qui rendrait vaine la croix du Christ.
${}^{18}Car le langage de la croix est folie pour ceux qui vont à leur perte, mais pour ceux qui vont vers leur salut, pour nous, il est puissance de Dieu. 
${}^{19}L’Écriture dit en effet :
        \\Je mènerai à sa perte la sagesse des sages,
        \\et l’intelligence des intelligents, je la rejetterai.
${}^{20}Où est-il, le sage ? Où est-il, le scribe ? Où est-il, le raisonneur d’ici-bas ? La sagesse du monde, Dieu ne l’a-t-il pas rendue folle ? 
${}^{21}Puisque, en effet, par une disposition de la sagesse de Dieu, le monde, avec toute sa sagesse, n’a pas su reconnaître Dieu, il a plu à Dieu de sauver les croyants par cette folie qu’est la proclamation de l’Évangile. 
${}^{22}Alors que les Juifs réclament des signes miraculeux, et que les Grecs recherchent une sagesse, 
${}^{23}nous, nous proclamons un Messie crucifié, scandale pour les Juifs, folie pour les nations païennes. 
${}^{24}Mais pour ceux que Dieu appelle, qu’ils soient Juifs ou Grecs, ce Messie, ce Christ, est puissance de Dieu et sagesse de Dieu. 
${}^{25}Car ce qui est folie de Dieu est plus sage que les hommes, et ce qui est faiblesse de Dieu est plus fort que les hommes.
${}^{26}Frères, vous qui avez été appelés par Dieu, regardez bien : parmi vous, il n’y a pas beaucoup de sages aux yeux des hommes, ni de gens puissants ou de haute naissance. 
${}^{27}Au contraire, ce qu’il y a de fou dans le monde, voilà ce que Dieu a choisi, pour couvrir de confusion les sages ; ce qu’il y a de faible dans le monde, voilà ce que Dieu a choisi, pour couvrir de confusion ce qui est fort ; 
${}^{28}ce qui est d’origine modeste, méprisé dans le monde, ce qui n’est pas, voilà ce que Dieu a choisi, pour réduire à rien ce qui est ; 
${}^{29}ainsi aucun être de chair ne pourra s’enorgueillir devant Dieu. 
${}^{30}C’est grâce à Dieu, en effet, que vous êtes dans le Christ Jésus, lui qui est devenu pour nous sagesse venant de Dieu, justice, sanctification, rédemption. 
${}^{31}Ainsi, comme il est écrit : Celui qui veut être fier, qu’il mette sa fierté dans le Seigneur.
      
         
      \bchapter{}
      \begin{verse}
${}^{1}Frères, quand je suis venu chez vous, je ne suis pas venu vous annoncer le mystère de Dieu avec le prestige du langage ou de la sagesse. 
${}^{2}Parmi vous, je n’ai rien voulu connaître d’autre que Jésus Christ, ce Messie crucifié. 
${}^{3}Et c’est dans la faiblesse, craintif et tout tremblant, que je me suis présenté à vous. 
${}^{4}Mon langage, ma proclamation de l’Évangile, n’avaient rien d’un langage de sagesse qui veut convaincre ; mais c’est l’Esprit et sa puissance qui se manifestaient, 
${}^{5}pour que votre foi repose, non pas sur la sagesse des hommes, mais sur la puissance de Dieu.
${}^{6}Pourtant, c’est bien de sagesse que nous parlons devant ceux qui sont adultes dans la foi, mais ce n’est pas la sagesse de ce monde, la sagesse de ceux qui dirigent ce monde et qui vont à leur destruction. 
${}^{7}Au contraire, ce dont nous parlons, c’est de la sagesse du mystère de Dieu, sagesse tenue cachée, établie par lui dès avant les siècles, pour nous donner la gloire. 
${}^{8}Aucun de ceux qui dirigent ce monde ne l’a connue, car, s’ils l’avaient connue, ils n’auraient jamais crucifié le Seigneur de gloire. 
${}^{9}Mais ce que nous proclamons, c’est, comme dit l’Écriture :
        \\ce que l’œil n’a pas vu, ce que l’oreille n’a pas entendu,
        \\ce qui n’est pas venu à l’esprit de l’homme,
        \\ce que Dieu a préparé pour ceux dont il est aimé.
${}^{10}Et c’est à nous que Dieu, par l’Esprit, en a fait la révélation. Car l’Esprit scrute le fond de toutes choses, même les profondeurs de Dieu. 
${}^{11}Qui donc, parmi les hommes, sait ce qu’il y a dans l’homme, sinon l’esprit de l’homme qui est en lui ? De même, personne ne connaît ce qu’il y a en Dieu, sinon l’Esprit de Dieu. 
${}^{12}Or nous, ce n’est pas l’esprit du monde que nous avons reçu, mais l’Esprit qui vient de Dieu, et ainsi nous avons conscience des dons que Dieu nous a accordés. 
${}^{13}Nous disons cela avec un langage que nous n’apprenons pas de la sagesse humaine, mais que nous apprenons de l’Esprit ; nous comparons entre elles les réalités spirituelles. 
${}^{14}L’homme, par ses seules capacités, n’accueille pas ce qui vient de l’Esprit de Dieu ; pour lui ce n’est que folie, et il ne peut pas comprendre, car c’est par l’Esprit qu’on examine toute chose. 
${}^{15}Celui qui est animé par l’Esprit soumet tout à examen, mais lui, personne ne peut l’y soumettre. 
${}^{16}Car il est écrit : Qui a connu la pensée du Seigneur et qui pourra l’instruire ? Eh bien nous, nous avons la pensée du Christ !
      
         
      \bchapter{}
      \begin{verse}
${}^{1}Frères, quand je me suis adressé à vous, je n’ai pas pu vous parler comme à des spirituels, mais comme à des êtres seulement charnels, comme à des petits enfants dans le Christ. 
${}^{2}C’est du lait que je vous ai donné, et non de la nourriture solide ; vous n’auriez pas pu en manger, et encore maintenant vous ne le pouvez pas, 
${}^{3}car vous êtes encore des êtres charnels. Puisqu’il y a entre vous des jalousies et des rivalités, n’êtes-vous pas toujours des êtres charnels, et n’avez-vous pas une conduite tout humaine ? 
${}^{4}Quand l’un de vous dit : « Moi, j’appartiens à Paul », et un autre : « Moi, j’appartiens à Apollos », n’est-ce pas une façon d’agir tout humaine ?
      
         
${}^{5}Mais qui donc est Apollos ? qui est Paul ? Des serviteurs par qui vous êtes devenus croyants, et qui ont agi selon les dons du Seigneur à chacun d’eux. 
${}^{6}Moi, j’ai planté, Apollos a arrosé ; mais c’est Dieu qui donnait la croissance. 
${}^{7}Donc celui qui plante n’est pas important, ni celui qui arrose ; seul importe celui qui donne la croissance : Dieu. 
${}^{8}Celui qui plante et celui qui arrose ne font qu’un, mais chacun recevra son propre salaire suivant la peine qu’il se sera donnée. 
${}^{9}Nous sommes des collaborateurs de Dieu, et vous êtes un champ que Dieu cultive, une maison que Dieu construit. 
${}^{10}Selon la grâce que Dieu m’a donnée, moi, comme un bon architecte, j’ai posé la pierre de fondation. Un autre construit dessus. Mais que chacun prenne garde à la façon dont il contribue à la construction. 
${}^{11}La pierre de fondation, personne ne peut en poser d’autre que celle qui s’y trouve : Jésus Christ. 
${}^{12}Que l’on construise sur la pierre de fondation avec de l’or, de l’argent, des pierres précieuses, ou avec du bois, du foin ou du chaume, 
${}^{13}l’ouvrage de chacun sera mis en pleine lumière. En effet, le jour du jugement le manifestera, car cette révélation se fera par le feu, et c’est le feu qui permettra d’apprécier la qualité de l’ouvrage de chacun. 
${}^{14}Si quelqu’un a construit un ouvrage qui résiste, il recevra un salaire ; 
${}^{15}si l’ouvrage est entièrement brûlé, il en subira le préjudice. Lui-même sera sauvé, mais comme au travers du feu.
${}^{16}Ne savez-vous pas que vous êtes un sanctuaire de Dieu, et que l’Esprit de Dieu habite en vous ? 
${}^{17}Si quelqu’un détruit le sanctuaire de Dieu, cet homme, Dieu le détruira, car le sanctuaire de Dieu est saint, et ce sanctuaire, c’est vous.
${}^{18}Que personne ne s’y trompe : si quelqu’un parmi vous pense être un sage à la manière d’ici-bas, qu’il devienne fou pour devenir sage. 
${}^{19}Car la sagesse de ce monde est folie devant Dieu. Il est écrit en effet : C’est lui qui prend les sages au piège de leur propre habileté. 
${}^{20}Il est écrit encore : Le Seigneur le sait : les raisonnements des sages n’ont aucune valeur ! 
${}^{21}Ainsi, il ne faut pas mettre sa fierté en tel ou tel homme. Car tout vous appartient, 
${}^{22}que ce soit Paul, Apollos, Pierre, le monde, la vie, la mort, le présent, l’avenir : tout est à vous, 
${}^{23}mais vous, vous êtes au Christ, et le Christ est à Dieu.
      
         
      \bchapter{}
      \begin{verse}
${}^{1}Que l’on nous regarde donc comme des auxiliaires du Christ et des intendants des mystères de Dieu. 
${}^{2}Or, tout ce que l’on demande aux intendants, c’est d’être trouvés dignes de confiance. 
${}^{3}Pour ma part, je me soucie fort peu d’être soumis à votre jugement, ou à celui d’une autorité humaine ; d’ailleurs, je ne me juge même pas moi-même. 
${}^{4}Ma conscience ne me reproche rien, mais ce n’est pas pour cela que je suis juste : celui qui me soumet au jugement, c’est le Seigneur. 
${}^{5}Ainsi, ne portez pas de jugement prématuré, mais attendez la venue du Seigneur, car il mettra en lumière ce qui est caché dans les ténèbres, et il rendra manifestes les intentions des cœurs. Alors, la louange qui revient à chacun lui sera donnée par Dieu.
${}^{6}Frères, j’ai pris pour vous ces comparaisons qui s’appliquent à Apollos et à moi-même ; ainsi, vous pourrez apprendre de nous à ne pas aller au-delà de ce qui est écrit, afin qu’aucun de vous n’aille se gonfler d’orgueil en prenant le parti de l’un contre l’autre. 
${}^{7}Qui donc t’a mis à part ? As-tu quelque chose sans l’avoir reçu ? Et si tu l’as reçu, pourquoi te vanter comme si tu ne l’avais pas reçu ? 
${}^{8}Vous voilà déjà comblés, vous voilà déjà riches, vous voilà devenus rois sans nous ! Ah ! si seulement vous étiez devenus rois, pour que nous aussi, nous le soyons avec vous ! 
${}^{9}Mais nous, les Apôtres, il me semble que Dieu nous a exposés en dernier comme en vue d’une mise à mort, car nous sommes donnés en spectacle au monde, aux anges et aux hommes. 
${}^{10}Nous, nous sommes fous à cause du Christ, et vous, vous êtes raisonnables dans le Christ ; nous sommes faibles, et vous êtes forts ; vous êtes à l’honneur, et nous, dans le mépris. 
${}^{11}Maintenant encore, nous avons faim, nous avons soif, nous sommes dans le dénuement, maltraités, nous n’avons pas de domicile, 
${}^{12}nous travaillons péniblement de nos mains. On nous insulte, nous bénissons. On nous persécute, nous le supportons. 
${}^{13}On nous calomnie, nous réconfortons. Jusqu’à présent, nous sommes pour ainsi dire l’ordure du monde, le rebut de l’humanité.
${}^{14}Je ne vous écris pas cela pour vous faire honte, mais pour vous reprendre comme mes enfants bien-aimés. 
${}^{15}Car, dans le Christ, vous pourriez avoir dix mille guides, vous n’avez pas plusieurs pères : par l’annonce de l’Évangile, c’est moi qui vous ai donné la vie dans le Christ Jésus. 
${}^{16}Aussi, je vous en prie : imitez-moi. 
${}^{17}C’est pour cela que je vous ai envoyé Timothée, qui est mon enfant bien-aimé et fidèle dans le Seigneur ; il vous rappellera les voies que je trace dans le Christ Jésus, telles que je les enseigne partout dans toutes les Églises. 
${}^{18}Pensant que je n’allais pas venir chez vous, quelques-uns se sont gonflés d’orgueil. 
${}^{19}Or je viendrai bientôt chez vous, si le Seigneur le veut, et je prendrai connaissance, non pas de ce que disent ces gens gonflés d’orgueil, mais des actes dont ils sont capables. 
${}^{20}Car le royaume de Dieu ne consiste pas dans la parole, mais dans la capacité d’agir. 
${}^{21}Que préférez-vous : que je vienne chez vous muni d’un bâton, ou avec amour et en esprit de douceur ?
      
         
      \bchapter{}
      \begin{verse}
${}^{1}On entend dire partout qu’il y a chez vous un cas d’inconduite, une inconduite telle qu’on n’en voit même pas chez les païens : il s’agit d’un homme qui vit avec la femme de son père. 
${}^{2}Et, malgré cela, vous êtes gonflés d’orgueil au lieu d’en pleurer et de chasser de votre communauté celui qui commet cet acte. 
${}^{3}Quant à moi, qui suis absent de corps mais présent d’esprit, j’ai déjà jugé, comme si j’étais présent, l’homme qui agit de la sorte : 
${}^{4}au nom du Seigneur Jésus, lors d’une réunion où je serai spirituellement avec vous, dans la puissance de notre Seigneur Jésus, 
${}^{5}il faut livrer cet individu au pouvoir de Satan, pour la perdition de son être de chair ; ainsi, son esprit pourra être sauvé au jour du Seigneur. 
${}^{6}Vraiment, vous n’avez pas de quoi être fiers : <a class="anchor verset_lettre" id="bib_1co_5_6_b"/>ne savez-vous pas qu’un peu de levain suffit pour que fermente toute la pâte ? 
${}^{7}Purifiez-vous donc des vieux ferments, et vous serez une pâte nouvelle, vous qui êtes le pain de la Pâque, celui qui n’a pas fermenté. Car notre agneau pascal a été immolé : c’est le Christ. 
${}^{8}Ainsi, célébrons la Fête, non pas avec de vieux ferments, non pas avec ceux de la perversité et du vice, mais avec du pain non fermenté, celui de la droiture et de la vérité. 
${}^{9}Je vous ai écrit dans ma lettre de ne pas fréquenter les débauchés. 
${}^{10}Cela ne concernait pas de façon générale les débauchés qui sont dans ce monde, ni les profiteurs, les escrocs ou les idolâtres – autrement, vous seriez obligés de sortir du monde ! 
${}^{11}En réalité, ce que je vous écrivais, c’est de ne pas fréquenter celui qui porte le nom de frère, mais qui est débauché, ou profiteur, idolâtre, ou diffamateur, ivrogne, ou escroc : il ne faut même pas prendre un repas avec un homme comme celui-là. 
${}^{12}Est-ce à moi de juger ceux du dehors ? Et ceux du dedans, n’est-ce pas à vous de les juger ? 
${}^{13}Quant à ceux du dehors, c’est Dieu qui les jugera. Ôtez donc du milieu de vous l’homme mauvais.
      
         
      
         
      \bchapter{}
      \begin{verse}
${}^{1}Lorsque l’un d’entre vous a un désaccord avec un autre, comment ose-t-il aller en procès devant des juges païens plutôt que devant les fidèles ? 
${}^{2}Ne savez-vous pas que les fidèles jugeront le monde ? Et si c’est vous qui devez juger le monde, seriez-vous indignes de juger des affaires de moindre importance ? 
${}^{3}Ne savez-vous pas que nous jugerons des anges ? À plus forte raison les affaires de cette vie ! 
${}^{4}Et quand vous avez de telles affaires, vous prenez comme juges des gens qui n’ont pas d’autorité dans l’Église ! 
${}^{5}Je vous le dis à votre honte. N’y aurait-il parmi vous aucun homme assez sage pour servir d’arbitre entre ses frères ? 
${}^{6}Pourtant, un frère est en procès avec son frère, et cela devant des gens qui ne sont pas croyants ! 
${}^{7}C’est déjà un échec pour vous d’avoir des litiges entre vous. Pourquoi ne pas plutôt supporter l’injustice ? Pourquoi ne pas plutôt vous laisser dépouiller ? 
${}^{8}Au contraire, c’est vous qui commettez l’injustice et qui dépouillez les autres, et cela, vous le faites à des frères ! 
${}^{9}Ne savez-vous pas que ceux qui commettent l’injustice ne recevront pas le royaume de Dieu en héritage ? Ne vous y trompez pas : ni les débauchés, les idolâtres, les adultères, ni les dépravés et les sodomites, 
${}^{10}ni les voleurs et les profiteurs, ni les ivrognes, les diffamateurs et les escrocs, aucun de ceux-là ne recevra le royaume de Dieu en héritage. 
${}^{11}Voilà ce qu’étaient certains d’entre vous. Mais vous avez été lavés, vous avez été sanctifiés, vous êtes devenus des justes, au nom du Seigneur Jésus Christ et par l’Esprit de notre Dieu.
      
         
${}^{12}« Tout m’est permis », dit-on, mais je dis : « Tout n'est pas bon ». « Tout m’est permis », mais moi, je ne permettrai à rien de me dominer. 
${}^{13}Les aliments sont pour le ventre, et le ventre pour les aliments ; or Dieu fera disparaître et ceux-ci et celui-là. <a class="anchor verset_lettre" id="bib_1co_6_13_b"/>Le corps n’est pas pour la débauche, il est pour le Seigneur, et le Seigneur est pour le corps ; 
${}^{14}et Dieu, par sa puissance, a ressuscité le Seigneur et nous ressuscitera nous aussi. 
${}^{15}Ne le savez-vous pas ? Vos corps sont les membres du Christ. Vais-je donc prendre les membres du Christ pour en faire les membres d’une prostituée ? Absolument pas ! 
${}^{16}Ne le savez-vous pas ? Celui qui s’unit à une prostituée ne fait avec elle qu’un seul corps. Car il est dit : Tous deux ne feront plus qu’un. 
${}^{17}Celui qui s’unit au Seigneur ne fait avec lui qu’un seul esprit. 
${}^{18}Fuyez la débauche. Tous les péchés que l’homme peut commettre sont extérieurs à son corps ; mais l’homme qui se livre à la débauche commet un péché contre son propre corps. 
${}^{19}Ne le savez-vous pas ? Votre corps est un sanctuaire de l’Esprit Saint, lui qui est en vous et que vous avez reçu de Dieu ; vous ne vous appartenez plus à vous-mêmes, 
${}^{20}car vous avez été achetés à grand prix. Rendez donc gloire à Dieu dans votre corps.
      
         
      \bchapter{}
      \begin{verse}
${}^{1}Au sujet de ce que vous dites dans votre lettre, certes, il est bon pour l’homme de ne pas toucher la femme. 
${}^{2}Cependant, étant donné les occasions de débauche, que chacun ait sa femme à lui, et que chacune ait son propre mari. 
${}^{3}Que le mari remplisse son devoir d’époux envers sa femme, et de même la femme envers son mari. 
${}^{4}Ce n’est pas la femme qui dispose de son propre corps, c’est son mari ; et de même, ce n’est pas le mari qui dispose de son propre corps, c’est sa femme. 
${}^{5}Ne vous refusez pas l’un à l’autre, si ce n’est d’un commun accord et temporairement, pour prendre le temps de prier et pour vous retrouver ensuite ; autrement, Satan vous tenterait, profitant de votre incapacité à vous maîtriser. 
${}^{6}Ce que je dis là est une concession, et non un ordre. 
${}^{7}Je voudrais bien que tout le monde soit comme moi-même, mais chacun a reçu de Dieu un don qui lui est personnel : l’un celui-ci, l’autre celui-là.
${}^{8}À ceux qui ne sont pas mariés et aux veuves, je déclare qu’il est bon pour eux de rester comme je suis. 
${}^{9}Mais s’ils ne peuvent pas se maîtriser, qu’ils se marient, car mieux vaut se marier que brûler de désir. 
${}^{10}À ceux qui sont mariés, je donne cet ordre – il ne vient pas de moi, mais du Seigneur – : que la femme ne se sépare pas de son mari ; 
${}^{11}et même si elle est séparée, qu’elle reste seule, ou qu’elle se réconcilie avec son mari ; et que le mari ne renvoie pas sa femme.
${}^{12}Aux autres, je déclare ceci – moi-même et non le Seigneur – : si un de nos frères a une femme non croyante, et que celle-ci soit d’accord pour vivre avec lui, qu’il ne la renvoie pas. 
${}^{13}Et si une femme a un mari non croyant, et que celui-ci soit d’accord pour vivre avec elle, qu’elle ne renvoie pas son mari. 
${}^{14}En effet le mari non croyant se trouve sanctifié par sa femme, et la femme non croyante se trouve sanctifiée par son mari croyant. Autrement, vos enfants ne seraient pas purifiés, et en fait ils sont sanctifiés. 
${}^{15}Mais si le non-croyant se sépare, qu’il le fasse : en de telles circonstances, notre frère ou notre sœur n’est pas réellement lié ; c’est pour vivre dans la paix que Dieu vous a appelés. 
${}^{16}Toi la femme, comment savoir si tu sauveras ton mari ? Et toi l’homme, comment savoir si tu sauveras ta femme ?
${}^{17}Pourtant, chacun doit continuer à vivre dans la situation que le Seigneur lui a donnée en partage, et où il était quand Dieu l’a appelé. C’est la règle que j’établis dans toutes les Églises. 
${}^{18}Celui qui avait la circoncision quand il a été appelé, qu’il ne la fasse pas disparaître ; celui qui n’avait pas la circoncision quand il a été appelé, qu’il ne se fasse pas circoncire. 
${}^{19}Avoir la circoncision, ce n’est rien ; ne pas l’avoir, ce n’est rien : ce qu’il faut, c’est garder les commandements de Dieu. 
${}^{20}Chacun doit rester dans la situation où il a été appelé. 
${}^{21}Toi qui étais esclave quand tu as été appelé, ne t’en inquiète pas ; même si tu as la possibilité de devenir libre, tire plutôt profit de ta situation. 
${}^{22}En effet, l’esclave qui a été appelé par le Seigneur est un affranchi du Seigneur ; de même, l’homme libre qui a été appelé est un esclave du Christ. 
${}^{23}Vous avez été achetés à grand prix, ne devenez pas esclaves des hommes. 
${}^{24}Frères, chacun doit rester devant Dieu dans la situation où il a été appelé.
${}^{25}Au sujet du célibat, je n’ai pas un ordre du Seigneur, mais je donne mon avis, moi qui suis devenu digne de confiance grâce à la miséricorde du Seigneur. 
${}^{26}Je pense que le célibat est une chose bonne, étant donné les nécessités présentes ; oui, c’est une chose bonne de vivre ainsi. 
${}^{27}Tu es marié ? ne cherche pas à te séparer de ta femme. Tu n’as pas de femme ? ne cherche pas à te marier. 
${}^{28}Si cependant tu te maries, ce n’est pas un péché ; et si une jeune fille se marie, ce n’est pas un péché. Mais ceux qui font ce choix y trouveront les épreuves correspondantes, et c’est cela que moi, je voudrais vous éviter. 
${}^{29}Frères, je dois vous le dire : le temps est limité. Dès lors, que ceux qui ont une femme soient comme s’ils n’avaient pas de femme, 
${}^{30}ceux qui pleurent, comme s’ils ne pleuraient pas, ceux qui ont de la joie, comme s’ils n’en avaient pas, ceux qui font des achats, comme s’ils ne possédaient rien, 
${}^{31}ceux qui profitent de ce monde, comme s’ils n’en profitaient pas vraiment. Car il passe, ce monde tel que nous le voyons. 
${}^{32}J’aimerais vous voir libres de tout souci. Celui qui n’est pas marié a le souci des affaires du Seigneur, il cherche comment plaire au Seigneur. 
${}^{33}Celui qui est marié a le souci des affaires de ce monde, il cherche comment plaire à sa femme, et il se trouve divisé. 
${}^{34}La femme sans mari, ou celle qui reste vierge, a le souci des affaires du Seigneur, afin d’être sanctifiée dans son corps et son esprit. Celle qui est mariée a le souci des affaires de ce monde, elle cherche comment plaire à son mari. 
${}^{35}C’est dans votre intérêt que je dis cela ; ce n’est pas pour vous tendre un piège, mais pour vous proposer ce qui est bien, afin que vous soyez attachés au Seigneur sans partage. 
${}^{36}Si un jeune homme pense qu’il risque de ne pas respecter une jeune fille, s’il est plein d’ardeur et que l’issue devienne inévitable, qu’il fasse comme il veut : ils peuvent se marier, ce n’est pas un péché. 
${}^{37}Mais s’il tient ferme intérieurement, s’il ne subit aucune contrainte, s’il est maître de sa propre volonté et a pris dans son cœur la décision de ne pas s’unir à cette jeune fille, il fera bien. 
${}^{38}Ainsi, celui qui se marie fait bien, et celui qui ne se marie pas fera mieux encore. 
${}^{39}La femme reste liée aussi longtemps que son mari est en vie. Mais si son mari meurt, elle est libre d’épouser celui qu’elle veut, mais seulement s’il est croyant. 
${}^{40}Pourtant elle sera plus heureuse si elle reste comme elle est ; c’est là mon opinion, et je pense avoir, moi aussi, l’Esprit de Dieu.
      
         
      \bchapter{}
      \begin{verse}
${}^{1}Au sujet des viandes qui ont été offertes aux idoles, nous savons bien que nous avons tous la connaissance nécessaire ; <a class="anchor verset_lettre" id="bib_1co_8_1_b"/>mais la connaissance rend orgueilleux, tandis que l’amour fait œuvre constructive. 
${}^{2}Si quelqu’un pense être arrivé à connaître quelque chose, il ne connaît pas encore comme il faudrait ; 
${}^{3}mais si quelqu’un aime Dieu, celui-là est vraiment connu de lui. 
${}^{4}Quant à manger ces viandes offertes aux idoles, le pouvons-nous ? Nous savons que, dans le monde, une idole n’est rien du tout ; il n’y a de dieu que le Dieu unique. 
${}^{5}Bien qu’il y ait en effet, au ciel et sur la terre, ce qu’on appelle des dieux – et il y a une quantité de « dieux » et de « seigneurs » –, 
${}^{6}pour nous, au contraire,
        \\il n’y a qu’un seul Dieu, le Père,
        de qui tout vient et vers qui nous allons ;
        \\et un seul Seigneur, Jésus Christ,
        par qui tout vient et par qui nous vivons.
${}^{7}Mais tout le monde n’a pas cette connaissance : certains, habitués jusqu’ici aux idoles, croient vénérer les idoles en mangeant de cette viande, et leur conscience, qui est faible, s’en trouve souillée. 
${}^{8}Ce n’est pas un aliment qui nous rapprochera de Dieu. Si nous n’en mangeons pas, nous n’avons rien de moins, et si nous en mangeons, nous n’avons rien de plus. 
${}^{9}Mais prenez garde que l’usage de votre droit ne soit une occasion de chute pour les faibles. 
${}^{10}En effet, si l’un d’eux te voit, toi qui as cette connaissance, attablé dans le temple d’une idole, cet homme qui a la conscience faible ne sera-t-il pas encouragé à manger de la viande offerte aux idoles ? 
${}^{11}Et la connaissance que tu as va faire périr le faible, ce frère pour qui le Christ est mort. 
${}^{12}Ainsi, en péchant contre vos frères, et en blessant leur conscience qui est faible, vous péchez contre le Christ lui-même. 
${}^{13}C’est pourquoi, si une question d’aliments doit faire tomber mon frère, je ne mangerai plus jamais de viande, pour ne pas faire tomber mon frère.
      
         
      \bchapter{}
      \begin{verse}
${}^{1}Ne suis-je pas libre ? Ne suis-je pas apôtre ? N’ai-je pas vu Jésus notre Seigneur ? Et vous, n’êtes-vous pas mon œuvre dans le Seigneur ? 
${}^{2}Si pour d’autres je ne suis pas apôtre, pour vous en tout cas je le suis ; le sceau qui authentifie mon apostolat, c’est vous, dans le Seigneur. 
${}^{3}Ma défense devant ceux qui enquêtent sur mon compte, la voici. 
${}^{4}N’aurions-nous pas le droit de manger et de boire ? 
${}^{5}N’aurions-nous pas le droit d’emmener avec nous une femme croyante, comme les autres apôtres, les frères du Seigneur et Pierre ? 
${}^{6}Ou bien serais-je le seul avec Barnabé à ne pas avoir le droit d’être dispensé de travail ? 
${}^{7}Arrive-t-il qu’on serve dans l’armée à ses propres frais ? qu’on plante une vigne sans manger de ses fruits ? qu’on garde un troupeau sans boire du lait de ce troupeau ? 
${}^{8}Est-ce que je parle seulement au niveau humain ? La Loi ne dit-elle pas la même chose ? 
${}^{9}En effet, dans la loi de Moïse il est écrit : Tu ne muselleras pas le bœuf qui foule le grain. Dieu s’inquiète-t-il des bœufs ? 
${}^{10}ou bien le dit-il en réalité à cause de nous ? Oui, c’est pour nous que cela fut écrit, puisque le laboureur doit avoir un espoir quand il laboure, et celui qui foule le grain doit espérer en avoir sa part. 
${}^{11}Si nous avons semé pour vous des biens spirituels, serait-ce trop de récolter chez vous des biens matériels ? 
${}^{12}Si d’autres ont quelque droit sur vous, n’en avons-nous pas encore plus qu’eux ? Mais nous n’avons pas fait usage de ce droit ; au contraire, nous supportons tout pour ne pas créer d’obstacle à l’Évangile du Christ. 
${}^{13}Ne le savez-vous pas ? Ceux qui assurent le culte du temple sont nourris par le temple ; ceux qui servent à l’autel ont leur part de ce qui est offert sur l’autel. 
${}^{14}De même aussi, le Seigneur a prescrit à ceux qui annoncent l’Évangile de vivre de la proclamation de l’Évangile. 
${}^{15}Mais moi, je n’ai jamais fait usage d’aucun de ces droits. Et je n’écris pas cela pour les réclamer. Plutôt mourir ! Personne ne m’enlèvera ce motif de fierté. 
${}^{16}En effet, annoncer l’Évangile, ce n’est pas là pour moi un motif de fierté, c’est une nécessité qui s’impose à moi. Malheur à moi si je n’annonçais pas l’Évangile ! 
${}^{17}Certes, si je le fais de moi-même, je mérite une récompense. Mais je ne le fais pas de moi-même, c’est une mission qui m’est confiée. 
${}^{18}Alors quel est mon mérite ? C'est d'annoncer l'Évangile sans rechercher aucun avantage matériel, et sans faire valoir mes droits de prédicateur de l'Évangile.
${}^{19}Oui, libre à l’égard de tous, je me suis fait l’esclave de tous afin d’en gagner le plus grand nombre possible. 
${}^{20}Et avec les Juifs, j’ai été comme un Juif, pour gagner les Juifs. Avec ceux qui sont sujets de la Loi, j’ai été comme un sujet de la Loi, moi qui ne le suis pas, pour gagner les sujets de la Loi. 
${}^{21}Avec les sans-loi, j’ai été comme un sans-loi, moi qui ne suis pas sans loi de Dieu, mais sous la loi du Christ, pour gagner les sans-loi. 
${}^{22}Avec les faibles, j’ai été faible, pour gagner les faibles. Je me suis fait tout à tous pour en sauver à tout prix quelques-uns. 
${}^{23}Et tout cela, je le fais à cause de l’Évangile, pour y avoir part, moi aussi.
${}^{24}Vous savez bien que, dans le stade, tous les coureurs participent à la course, mais un seul reçoit le prix. Alors, vous, courez de manière à l’emporter. 
${}^{25}Tous les athlètes à l’entraînement s’imposent une discipline sévère ; ils le font pour recevoir une couronne de laurier qui va se faner, et nous, pour une couronne qui ne se fane pas. 
${}^{26}Moi, si je cours, ce n’est pas sans fixer le but ; si je fais de la lutte, ce n’est pas en frappant dans le vide. 
${}^{27}Mais je traite durement mon corps, j’en fais mon esclave, pour éviter qu’après avoir proclamé l’Évangile à d’autres, je sois moi-même disqualifié.
      
         
      \bchapter{}
      \begin{verse}
${}^{1}Frères, je ne voudrais pas vous laisser ignorer que, lors de la sortie d’Égypte, nos pères étaient tous sous la protection de la nuée, et que tous ont passé à travers la mer. 
${}^{2}Tous, ils ont été unis à Moïse par un baptême dans la nuée et dans la mer ; 
${}^{3}tous, ils ont mangé la même nourriture spirituelle ; 
${}^{4}tous, ils ont bu la même boisson spirituelle ; car ils buvaient à un rocher spirituel qui les suivait, et ce rocher, c’était le Christ. 
${}^{5}Cependant, la plupart n’ont pas su plaire à Dieu : leurs ossements, en effet, jonchèrent le désert. 
${}^{6}Ces événements devaient nous servir d’exemple, pour nous empêcher de désirer ce qui est mal comme l’ont fait ces gens-là. 
${}^{7}Ne devenez pas idolâtres, comme certains d’entre eux, selon qu’il est écrit : Le peuple s’est assis pour manger et boire, et ils se sont levés pour s’amuser. 
${}^{8}Ne nous livrons pas à la débauche, comme l’ont fait certains d’entre eux : il en est tombé vingt-trois mille en un seul jour. 
${}^{9}Ne mettons pas le Christ à l’épreuve, comme l’ont fait certains d’entre eux : ils ont péri mordus par les serpents. 
${}^{10}Cessez de récriminer comme l’ont fait certains d’entre eux : ils ont été exterminés. 
${}^{11}Ce qui leur est arrivé devait servir d’exemple, et l’Écriture l’a raconté pour nous avertir, nous qui nous trouvons à la fin des temps. 
${}^{12}Ainsi donc, celui qui se croit solide, qu’il fasse attention à ne pas tomber. 
${}^{13}L’épreuve qui vous a atteints n’a pas dépassé la mesure humaine. Dieu est fidèle : il ne permettra pas que vous soyez éprouvés au-delà de vos forces. Mais avec l’épreuve il donnera le moyen d’en sortir et la force de la supporter.
      
         
${}^{14}Aussi, mes bien-aimés, fuyez le culte des idoles. 
${}^{15}Je vous parle comme à des personnes raisonnables ; jugez vous-mêmes de ce que je dis. 
${}^{16}La coupe de bénédiction que nous bénissons, n’est-elle pas communion au sang du Christ ? Le pain que nous rompons, n’est-il pas communion au corps du Christ ? 
${}^{17}Puisqu’il y a un seul pain, la multitude que nous sommes est un seul corps, car nous avons tous part à un seul pain. 
${}^{18}Voyez ce qui se passe chez les Israélites : ceux qui mangent les victimes offertes sur l’autel de Dieu, ne sont-ils pas en communion avec lui ? 
${}^{19}Je ne prétends pas que la viande offerte aux idoles ou que les idoles elles-mêmes représentent quoi que ce soit. 
${}^{20}Mais je dis que les sacrifices des païens sont offerts aux démons, et non à Dieu, et je ne veux pas que vous soyez en communion avec les démons. 
${}^{21}Vous ne pouvez pas boire à la coupe du Seigneur et en même temps à celle des démons ; vous ne pouvez pas prendre part à la table du Seigneur et en même temps à celle des démons. 
${}^{22}Voulons-nous provoquer l’ardeur jalouse du Seigneur ? Sommes-nous plus forts que lui ?
${}^{23}« Tout est permis », dit-on, mais je dis : « Tout n'est pas bon. » « Tout est permis », mais tout n’est pas constructif. 
${}^{24}Que personne ne cherche son propre intérêt, mais celui d’autrui. 
${}^{25}Tout ce qui se vend au marché, mangez-en sans poser de questions par motif de conscience. 
${}^{26}Car il est écrit : Au Seigneur, la terre et tout ce qui la remplit. 
${}^{27}Si vous êtes invités par quelqu’un qui n’est pas croyant, et que vous vouliez vous rendre chez lui, mangez tout ce qu’on vous sert sans poser de questions par motif de conscience. 
${}^{28}Mais si quelqu’un vous dit : « Cela, c’est de la viande offerte en sacrifice », n’en mangez pas, à cause de celui qui vous a prévenus et par motif de conscience ; 
${}^{29}je ne parle pas de votre conscience à vous, mais de celle d’autrui. Pourquoi en effet ma liberté serait-elle jugée par la conscience d’un autre ? 
${}^{30}Si je participe à un repas dans l’action de grâce, pourquoi me blâmer pour cette nourriture dont je rends grâce ? 
${}^{31}Tout ce que vous faites : manger, boire, ou toute autre action, faites-le pour la gloire de Dieu. 
${}^{32}Ne soyez un obstacle pour personne, ni pour les Juifs, ni pour les païens, ni pour l’Église de Dieu. 
${}^{33}Ainsi, moi-même, en toute circonstance, je tâche de m’adapter à tout le monde, sans chercher mon intérêt personnel, mais celui de la multitude des hommes, pour qu’ils soient sauvés.
      
         
      \bchapter{}
       
      \begin{verse}
${}^{1}Imitez-moi, comme moi aussi j’imite le Christ.
${}^{2}Je vous félicite de vous souvenir si bien de moi, et de garder les traditions que je vous ai transmises. 
${}^{3}Mais je veux que vous le sachiez : la tête de tout homme, c’est le Christ ; la tête de la femme, c’est l’homme ; la tête du Christ, c’est Dieu. 
${}^{4}Tout homme qui prie ou prophétise ayant quelque chose sur la tête fait honte à sa tête. 
${}^{5}Toute femme qui prie ou prophétise sans avoir la tête couverte fait honte à sa tête : c’est exactement comme si elle était rasée. 
${}^{6}En effet, si elle ne se couvre pas, qu’elle aille jusqu’à se faire tondre ; et si c’est une honte pour la femme d’être tondue ou rasée, qu’elle se couvre. 
${}^{7}L’homme, lui, ne doit pas se couvrir la tête, puisqu’il est image et gloire de Dieu, et la femme est la gloire de l’homme. 
${}^{8}Ce n’est pas l’homme, en effet, qui a été tiré de la femme, mais la femme qui a été tirée de l’homme, 
${}^{9}et ce n’est pas l’homme qui a été créé à cause de la femme, mais la femme à cause de l’homme. 
${}^{10}C’est pourquoi la femme doit avoir sur la tête un signe de sa dignité, à cause des anges. 
${}^{11}D’ailleurs, dans le Seigneur la femme n’est pas sans l’homme, ni l’homme sans la femme. 
${}^{12}En effet, de même que la femme a été tirée de l’homme, de même l’homme vient au monde par la femme, et tout cela vient de Dieu. 
${}^{13}Jugez-en par vous-mêmes : est-il convenable qu’une femme prie Dieu sans avoir la tête couverte ? 
${}^{14}La nature elle-même ne vous enseigne-t-elle pas que, pour un homme, il est déshonorant d’avoir les cheveux longs, 
${}^{15}alors que, pour une femme, c’est une gloire, car la chevelure lui a été donnée pour s’en draper ? 
${}^{16}Et si quelqu’un croit devoir ergoter, nous, nous n’avons pas cette manière de faire, et les Églises de Dieu non plus.
${}^{17}Puisque j’en suis à vous faire des recommandations, je ne vous félicite pas pour vos réunions : elles vous font plus de mal que de bien. 
${}^{18}Tout d’abord, quand votre Église se réunit, j’entends dire que, parmi vous, il existe des divisions, et je crois que c’est assez vrai, 
${}^{19}car il faut bien qu’il y ait parmi vous des groupes qui s’opposent, afin qu’on reconnaisse ceux d’entre vous qui ont une valeur éprouvée. 
${}^{20}Donc, lorsque vous vous réunissez tous ensemble, ce n’est plus le repas du Seigneur que vous prenez ; 
${}^{21}en effet, chacun se précipite pour prendre son propre repas, et l’un reste affamé, tandis que l’autre a trop bu. 
${}^{22}N’avez-vous donc pas de maisons pour manger et pour boire ? Méprisez-vous l’Église de Dieu au point d’humilier ceux qui n’ont rien ? Que puis-je vous dire ? vous féliciter ? Non, pour cela je ne vous félicite pas !
${}^{23}J’ai moi-même reçu ce qui vient du Seigneur, et je vous l’ai transmis : la nuit où il était livré, le Seigneur Jésus prit du pain, 
${}^{24}puis, ayant rendu grâce, il le rompit, et dit : « Ceci est mon corps, qui est pour vous. Faites cela en mémoire de moi. » 
${}^{25}Après le repas, il fit de même avec la coupe, en disant : « Cette coupe est la nouvelle Alliance en mon sang. Chaque fois que vous en boirez, faites cela en mémoire de moi. » 
${}^{26}Ainsi donc, chaque fois que vous mangez ce pain et que vous buvez cette coupe, vous proclamez la mort du Seigneur, jusqu’à ce qu’il vienne.
${}^{27}Et celui qui aura mangé le pain ou bu la coupe du Seigneur d’une manière indigne devra répondre du corps et du sang du Seigneur. 
${}^{28}On doit donc s’examiner soi-même avant de manger de ce pain et de boire à cette coupe. 
${}^{29}Celui qui mange et qui boit mange et boit son propre jugement s’il ne discerne pas le corps du Seigneur. 
${}^{30}C’est pour cela qu’il y a chez vous beaucoup de malades et d’infirmes et qu’un certain nombre sont endormis dans la mort. 
${}^{31}Si nous avions du discernement envers nous-mêmes, nous ne serions pas jugés. 
${}^{32}Mais lorsque nous sommes jugés par le Seigneur, c’est une correction que nous recevons, afin de ne pas être condamnés avec le monde. 
${}^{33}Ainsi donc, mes frères, quand vous vous réunissez pour ce repas, attendez-vous les uns les autres ; 
${}^{34}si quelqu’un a faim, qu’il mange à la maison, pour que vos réunions ne vous attirent pas le jugement du Seigneur. Quant au reste, je le réglerai quand je viendrai.
      
         
      \bchapter{}
      \begin{verse}
${}^{1}Frères, au sujet des dons spirituels, je ne veux pas vous laisser dans l’ignorance. 
${}^{2}Vous le savez bien : quand vous étiez païens, vous étiez entraînés sans contrôle vers les idoles muettes. 
${}^{3}C’est pourquoi je vous le rappelle : Si quelqu’un parle sous l’action de l’Esprit de Dieu, <a class="anchor verset_lettre" id="bib_1co_12_3_b"/>il ne dira jamais : « Jésus est anathème » ; et personne n’est capable de dire : « Jésus est Seigneur » sinon dans l’Esprit Saint.
${}^{4}Les dons de la grâce sont variés, mais c’est le même Esprit. 
${}^{5}Les services sont variés, mais c’est le même Seigneur. 
${}^{6}Les activités sont variées, mais c’est le même Dieu qui agit en tout et en tous. 
${}^{7}À chacun est donnée la manifestation de l’Esprit en vue du bien. 
${}^{8}À celui-ci est donnée, par l’Esprit, une parole de sagesse ; à un autre, une parole de connaissance, selon le même Esprit ; 
${}^{9}un autre reçoit, dans le même Esprit, un don de foi ; un autre encore, dans l’unique Esprit, des dons de guérison ; 
${}^{10}à un autre est donné d’opérer des miracles, à un autre de prophétiser, à un autre de discerner les inspirations ; à l’un, de parler diverses langues mystérieuses ; à l’autre, de les interpréter. 
${}^{11}Mais celui qui agit en tout cela, c’est l’unique et même Esprit : il distribue ses dons, comme il le veut, à chacun en particulier.
${}^{12}Prenons une comparaison : le corps ne fait qu’un, il a pourtant plusieurs membres ; et tous les membres, malgré leur nombre, ne forment qu’un seul corps. Il en est ainsi pour le Christ. 
${}^{13}C’est dans un unique Esprit, en effet, que nous tous, Juifs ou païens, esclaves ou hommes libres, nous avons été baptisés pour former un seul corps. Tous, nous avons été désaltérés par un unique Esprit.
${}^{14}Le corps humain se compose non pas d’un seul, mais de plusieurs membres. 
${}^{15}Le pied aurait beau dire : « Je ne suis pas la main, donc je ne fais pas partie du corps », il fait cependant partie du corps. 
${}^{16}L’oreille aurait beau dire : « Je ne suis pas l’œil, donc je ne fais pas partie du corps », elle fait cependant partie du corps. 
${}^{17}Si, dans le corps, il n’y avait que les yeux, comment pourrait-on entendre ? S’il n’y avait que les oreilles, comment pourrait-on sentir les odeurs ? 
${}^{18}Mais, dans le corps, Dieu a disposé les différents membres comme il l’a voulu. 
${}^{19}S’il n’y avait en tout qu’un seul membre, comment cela ferait-il un corps ? 
${}^{20}En fait, il y a plusieurs membres, et un seul corps. 
${}^{21}L’œil ne peut pas dire à la main : « Je n’ai pas besoin de toi » ; la tête ne peut pas dire aux pieds : « Je n’ai pas besoin de vous ».
${}^{22}Bien plus, les parties du corps qui paraissent les plus délicates sont indispensables. 
${}^{23}Et celles qui passent pour moins honorables, ce sont elles que nous traitons avec plus d’honneur ; celles qui sont moins décentes, nous les traitons plus décemment ; 
${}^{24}pour celles qui sont décentes, ce n’est pas nécessaire. Mais en organisant le corps, Dieu a accordé plus d’honneur à ce qui en est dépourvu. 
${}^{25}Il a voulu ainsi qu’il n’y ait pas de division dans le corps, mais que les différents membres aient tous le souci les uns des autres. 
${}^{26}Si un seul membre souffre, tous les membres partagent sa souffrance ; si un membre est à l’honneur, tous partagent sa joie.
${}^{27}Or, vous êtes corps du Christ et, chacun pour votre part, vous êtes membres de ce corps. 
${}^{28}Parmi ceux que Dieu a placés ainsi dans l’Église, il y a premièrement des apôtres, deuxièmement des prophètes, troisièmement ceux qui ont charge d’enseigner ; ensuite, il y a les miracles, puis les dons de guérison, d’assistance, de gouvernement, le don de parler diverses langues mystérieuses. 
${}^{29}Tout le monde évidemment n’est pas apôtre, tout le monde n’est pas prophète, ni chargé d’enseigner ; tout le monde n’a pas à faire des miracles, 
${}^{30}à guérir, à dire des paroles mystérieuses, ou à les interpréter. 
${}^{31}Recherchez donc avec ardeur les dons les plus grands.
      Et maintenant, je vais vous indiquer le chemin par excellence.
      
         
      \bchapter{}
        ${}^{1}J’aurais beau parler toutes les langues
        des hommes et des anges,
        \\si je n’ai pas la charité, s’il me manque l’amour,
        \\je ne suis qu’un cuivre qui résonne,
        une cymbale retentissante.
        ${}^{2}J’aurais beau être prophète,
        \\avoir toute la science des mystères
        et toute la connaissance de Dieu,
        \\j’aurais beau avoir toute la foi
        jusqu’à transporter les montagnes,
        \\s’il me manque l’amour,
        je ne suis rien.
        ${}^{3}J’aurais beau distribuer toute ma fortune aux affamés,
        j’aurais beau me faire brûler vif,
        \\s’il me manque l’amour,
        cela ne me sert à rien.
        ${}^{4}L’amour prend patience ;
        l’amour rend service ;
        \\l’amour ne jalouse pas ;
        il ne se vante pas, ne se gonfle pas d’orgueil ;
        ${}^{5}il ne fait rien d’inconvenant ;
        il ne cherche pas son intérêt ;
        \\il ne s’emporte pas ;
        il n’entretient pas de rancune ;
        ${}^{6}il ne se réjouit pas de ce qui est injuste,
        mais il trouve sa joie dans ce qui est vrai ;
        ${}^{7}il supporte tout, il fait confiance en tout,
        il espère tout, il endure tout.
        ${}^{8}L’amour ne passera jamais.
        
           
      Les prophéties seront dépassées, le don des langues cessera, la connaissance actuelle sera dépassée. 
${}^{9}En effet, notre connaissance est partielle, nos prophéties sont partielles. 
${}^{10}Quand viendra l’achèvement, ce qui est partiel sera dépassé. 
${}^{11}Quand j’étais petit enfant, je parlais comme un enfant, je pensais comme un enfant, je raisonnais comme un enfant. Maintenant que je suis un homme, j’ai dépassé ce qui était propre à l’enfant. 
${}^{12}Nous voyons actuellement de manière confuse, comme dans un miroir ; ce jour-là, nous verrons face à face. Actuellement, ma connaissance est partielle ; ce jour-là, je connaîtrai parfaitement, comme j’ai été connu. 
${}^{13}Ce qui demeure aujourd’hui, c’est la foi, l’espérance et la charité ; mais la plus grande des trois, c’est la charité.
      
         
      \bchapter{}
      \begin{verse}
${}^{1}Efforcez-vous d’atteindre la charité. Recherchez avec ardeur les dons spirituels, surtout celui de prophétie. 
${}^{2}En effet, celui qui parle en langues ne parle pas pour les hommes, mais pour Dieu : personne ne comprend, car, sous l’effet de l’inspiration, il dit des choses mystérieuses. 
${}^{3}Mais celui qui prophétise parle pour les hommes : il est constructif, il réconforte, il encourage. 
${}^{4}Celui qui parle en langues ne construit que lui-même, tandis que celui qui prophétise construit l’assemblée de l’Église. 
${}^{5}Je souhaiterais que vous parliez tous en langues, mais, plus encore, que vous prophétisiez. Car prophétiser vaut mieux que parler en langues, à moins qu’on n’interprète ce qui a été dit en langues : ainsi, on aide à la construction de l’Église.
${}^{6}D’ailleurs, frères, si je viens chez vous et que je parle en langues, en quoi vous serai-je utile si ma parole ne vous apporte ni révélation, ni connaissance, ni prophétie, ni enseignement ? 
${}^{7}Ainsi des objets inanimés comme la flûte ou la cithare, quand ils produisent des sons, s’ils ne donnent pas des notes distinctes, comment reconnaître l’air joué par l’instrument ? 
${}^{8}Et si la trompette produit des sons confus, qui va se préparer au combat ? 
${}^{9}Vous de même, si votre langue ne produit pas un message intelligible, comment reconnaître ce qui est dit ? Vous serez de ceux qui parlent pour le vent. 
${}^{10}Il y a dans le monde je ne sais combien d’espèces de langages, et personne n’en est dépourvu. 
${}^{11}Mais si je ne connais pas le sens de ce langage, je serai un barbare pour celui qui parle et il le sera pour moi. 
${}^{12}Alors, vous, puisque vous êtes avides d’inspirations, cherchez à progresser, mais en vue de construire l’Église.
${}^{13}Dès lors, celui qui parle en langues, qu’il prie pour être capable d’interpréter. 
${}^{14}Si je prie en langues, mon esprit, assurément, est en prière, mais mon intelligence reste sans fruit. 
${}^{15}Que vais-je donc faire ? Je vais prier selon l’inspiration, mais prier aussi avec l’intelligence, je vais chanter selon l’inspiration, mais chanter aussi avec l’intelligence. 
${}^{16}Car si tu prononces une prière de bénédiction selon l’inspiration seulement, alors celui qui est là et n’y connaît rien, comment va-t-il répondre « Amen » à ton action de grâce, puisqu’il ne comprend pas ce que tu dis ? 
${}^{17}Toi, bien sûr, tu fais une belle action de grâce, mais ce n’est pas constructif pour l’autre. 
${}^{18}Je parle en langues plus que vous tous, et j’en rends grâce à Dieu ; 
${}^{19}mais, quand l’Église est rassemblée, je préfère dire cinq paroles avec mon intelligence de manière à instruire les autres, plutôt que d’en dire dix mille en langues.
${}^{20}Frères, pour le bon sens, ne soyez pas des enfants ; pour le mal, oui, soyez des petits enfants, mais pour le bon sens, soyez des adultes. 
${}^{21}Dans la Loi, il est écrit ceci :
        \\C’est par des gens de langue étrangère,
        \\par des lèvres d’étrangers, que je parlerai à ce peuple ;
        \\et même ainsi ils ne m’écouteront pas, dit le Seigneur.
${}^{22}Cela veut dire que parler en langues est un signe non pour les croyants, mais pour ceux qui ne croient pas, alors que la prophétie est un signe non pour ceux qui ne croient pas, mais pour les croyants. 
${}^{23}Quand donc l’Église tout entière est rassemblée, si tous parlent en langues, et qu’il arrive des gens qui ne sont pas initiés ou ne sont pas croyants, ne vont-ils pas dire que vous délirez ? 
${}^{24}Si au contraire tous prophétisent, et qu’il arrive un non-croyant ou un non-initié, il se sent mis en question par tous, comme soumis à examen par tous, 
${}^{25}les secrets de son cœur sont mis au grand jour : il tombera face contre terre pour se prosterner devant Dieu et proclamer : « Vraiment, Dieu est parmi vous ! »
${}^{26}Alors, frères, quand vous vous réunissez, et que chacun apporte un cantique, ou un enseignement, ou une révélation, ou une intervention en langues, ou une interprétation, il faut que tout serve à construire l’Église. 
${}^{27}Et si on parle en langues, qu’il y en ait deux à le faire, trois tout au plus, chacun à son tour, et qu’il y ait quelqu’un pour interpréter. 
${}^{28}Mais s’il n’y a pas d’interprète, qu’on se taise dans l’assemblée, qu’on parle pour soi-même et pour Dieu. 
${}^{29}Quant aux prophètes, que deux ou trois prennent la parole, et que les autres exercent le discernement. 
${}^{30}Mais si quelqu’un d’autre dans l’assistance reçoit une révélation, que le premier se taise. 
${}^{31}Vous pouvez tous prophétiser, l’un après l’autre, pour que tous en retirent instruction et réconfort. 
${}^{32}Les inspirations des prophètes sont sous le contrôle des prophètes, 
${}^{33}car Dieu n’est pas un Dieu de désordre, mais de paix.
      Comme cela se fait dans toutes nos Églises, 
${}^{34}que les femmes gardent le silence dans les assemblées, car elles n’ont pas la permission de parler ; mais qu’elles restent dans la soumission, comme le dit la Loi. 
${}^{35}Et si elles veulent obtenir un éclaircissement, qu’elles interrogent leur mari à la maison. Car pour une femme c’est une honte de parler dans l’assemblée. 
${}^{36}La parole de Dieu serait-elle venue de chez vous ? Ne serait-elle arrivée que chez vous ?
${}^{37}Si quelqu’un pense être prophète ou inspiré par l’Esprit, qu’il reconnaisse dans ce que je vous écris un commandement du Seigneur. 
${}^{38}S’il ne le reconnaît pas, lui-même ne sera pas reconnu. 
${}^{39}Ainsi, mes frères, recherchez le don de prophétie, et n’empêchez pas de parler en langues, 
${}^{40}mais que tout se passe dans la dignité et dans l’ordre.
      
         
      \bchapter{}
      \begin{verse}
${}^{1}Frères, je vous rappelle la Bonne Nouvelle que je vous ai annoncée ; cet Évangile, vous l’avez reçu ; c’est en lui que vous tenez bon, 
${}^{2}c’est par lui que vous serez sauvés si vous le gardez tel que je vous l’ai annoncé ; autrement, c’est pour rien que vous êtes devenus croyants. 
${}^{3}Avant tout, je vous ai transmis ceci, que j’ai moi-même reçu : le Christ est mort pour nos péchés conformément aux Écritures, 
${}^{4}et il fut mis au tombeau ; il est ressuscité le troisième jour conformément aux Écritures, 
${}^{5}il est apparu à Pierre, puis aux Douze ; 
${}^{6}ensuite il est apparu à plus de cinq cents frères à la fois – la plupart sont encore vivants, et quelques-uns sont endormis dans la mort –, 
${}^{7}ensuite il est apparu à Jacques, puis à tous les Apôtres. 
${}^{8}Et en tout dernier lieu, il est même apparu à l’avorton que je suis. 
${}^{9}Car moi, je suis le plus petit des Apôtres, je ne suis pas digne d’être appelé Apôtre, puisque j’ai persécuté l’Église de Dieu. 
${}^{10}Mais ce que je suis, je le suis par la grâce de Dieu, et sa grâce, venant en moi, n’a pas été stérile. Je me suis donné de la peine plus que tous les autres ; à vrai dire, ce n’est pas moi, c’est la grâce de Dieu avec moi. 
${}^{11}Bref, qu’il s’agisse de moi ou des autres, voilà ce que nous proclamons, voilà ce que vous croyez.
      
         
${}^{12}Nous proclamons que le Christ est ressuscité d’entre les morts ; alors, comment certains d’entre vous peuvent-ils affirmer qu’il n’y a pas de résurrection des morts ? 
${}^{13}S’il n’y a pas de résurrection des morts, le Christ non plus n’est pas ressuscité. 
${}^{14}Et si le Christ n’est pas ressuscité, notre proclamation est sans contenu, votre foi aussi est sans contenu ; 
${}^{15}et nous faisons figure de faux témoins de Dieu, pour avoir affirmé, en témoignant au sujet de Dieu, qu’il a ressuscité le Christ, alors qu’il ne l’a pas ressuscité si vraiment les morts ne ressuscitent pas. 
${}^{16}Car si les morts ne ressuscitent pas, le Christ non plus n’est pas ressuscité. 
${}^{17}Et si le Christ n’est pas ressuscité, votre foi est sans valeur, vous êtes encore sous l’emprise de vos péchés ; 
${}^{18}et donc, ceux qui se sont endormis dans le Christ sont perdus. 
${}^{19}Si nous avons mis notre espoir dans le Christ pour cette vie seulement, nous sommes les plus à plaindre de tous les hommes.
${}^{20}Mais non ! le Christ est ressuscité d’entre les morts, lui, premier ressuscité parmi ceux qui se sont endormis. 
${}^{21}Car, la mort étant venue par un homme, c’est par un homme aussi que vient la résurrection des morts. 
${}^{22}En effet, de même que tous les hommes meurent en Adam, de même c’est dans le Christ que tous recevront la vie, 
${}^{23}mais chacun à son rang : en premier, le Christ, et ensuite, lors du retour du Christ, ceux qui lui appartiennent. 
${}^{24}Alors, tout sera achevé, quand le Christ remettra le pouvoir royal à Dieu son Père, après avoir anéanti, parmi les êtres célestes, toute Principauté, toute Souveraineté et Puissance. 
${}^{25}Car c’est lui qui doit régner jusqu’au jour où Dieu aura mis sous ses pieds tous ses ennemis. 
${}^{26}Et le dernier ennemi qui sera anéanti, c’est la mort, 
${}^{27}car il a tout mis sous ses pieds. Mais quand le Christ dira : « Tout est soumis désormais », c’est évidemment à l’exclusion de Celui qui lui aura soumis toutes choses. 
${}^{28}Et, quand tout sera mis sous le pouvoir du Fils, lui-même se mettra alors sous le pouvoir du Père qui lui aura tout soumis, et ainsi, Dieu sera tout en tous.
${}^{29}Autrement, que feront-ils, ceux qui se font baptiser pour les morts ? Si vraiment les morts ne ressuscitent pas, pourquoi se faire baptiser pour eux ? 
${}^{30}Et pourquoi nous aussi courons-nous des dangers à chaque instant ? 
${}^{31}Chaque jour, j’affronte la mort, et cela, frères, est votre fierté, que je partage dans le Christ Jésus notre Seigneur. 
${}^{32}S’il n’y avait eu que de l’humain dans mon combat contre les bêtes à Éphèse, à quoi cela m’aurait-il servi ? Si les morts ne ressuscitent pas, mangeons et buvons, car demain nous mourrons. 
${}^{33}Ne vous y trompez pas : Les mauvaises compagnies corrompent les bonnes mœurs. 
${}^{34}Reprenez donc vos esprits, et ne péchez pas : en effet, certains d’entre vous n’ont pas la connaissance de Dieu. Je vous le dis, à votre honte.
${}^{35}Mais quelqu’un pourrait dire : « Comment les morts ressuscitent-ils ? avec quelle sorte de corps reviennent-ils ? » 
${}^{36}– Réfléchis donc ! Ce que tu sèmes ne peut reprendre vie sans mourir d’abord ; 
${}^{37}et ce que tu sèmes, ce n’est pas le corps de la plante qui va pousser, mais c’est une simple graine : du blé, par exemple, ou autre chose. 
${}^{38}Et Dieu lui donne un corps comme il l’a voulu : à chaque semence un corps particulier. 
${}^{39}Il y a plusieurs sortes de chair : autre est celle des hommes, et autre celle des bêtes, autre celle des oiseaux, et autre celle des poissons. 
${}^{40}Il y a des corps célestes et des corps terrestres, mais autre est l’éclat des célestes, autre celui des terrestres ; 
${}^{41}autre est l’éclat du soleil, autre l’éclat de la lune, autre l’éclat des étoiles ; et chaque étoile a même un éclat différent. 
${}^{42}Ainsi en est-il de la résurrection des morts. Ce qui est semé périssable ressuscite impérissable ; 
${}^{43}ce qui est semé sans honneur ressuscite dans la gloire ; ce qui est semé faible ressuscite dans la puissance ; 
${}^{44}ce qui est semé corps physique ressuscite corps spirituel ; car s’il existe un corps physique, il existe aussi un corps spirituel. 
${}^{45}L’Écriture dit : Le premier homme, Adam, devint un être vivant ; le dernier Adam – le Christ – est devenu l’être spirituel qui donne la vie. 
${}^{46}Ce qui vient d’abord, ce n’est pas le spirituel, mais le physique ; ensuite seulement vient le spirituel. 
${}^{47}Pétri d’argile, le premier homme vient de la terre ; le deuxième homme, lui, vient du ciel. 
${}^{48}Comme Adam est fait d’argile, ainsi les hommes sont faits d’argile ; comme le Christ est du ciel, ainsi les hommes seront du ciel. 
${}^{49}Et de même que nous aurons été à l’image de celui qui est fait d’argile, de même nous serons à l’image de celui qui vient du ciel.
${}^{50}Je le déclare, frères : la chair et le sang sont incapables de recevoir en héritage le royaume de Dieu, et ce qui est périssable ne reçoit pas en héritage ce qui est impérissable. 
${}^{51}C’est un mystère que je vous annonce : nous ne mourrons pas tous, mais tous nous serons transformés, 
${}^{52}et cela en un instant, en un clin d’œil, quand, à la fin, la trompette retentira. Car elle retentira, et les morts ressusciteront, impérissables, et nous, nous serons transformés. 
${}^{53}Il faut en effet que cet être périssable que nous sommes revête ce qui est impérissable ; il faut que cet être mortel revête l’immortalité. 
${}^{54}Et quand cet être périssable aura revêtu ce qui est impérissable, quand cet être mortel aura revêtu l’immortalité, alors se réalisera la parole de l’Écriture :
        \\La mort a été engloutie dans la victoire.
        ${}^{55}Ô Mort, où est ta victoire ?
        \\Ô Mort, où est-il, ton aiguillon ?
${}^{56}L’aiguillon de la mort, c’est le péché ; ce qui donne force au péché, c’est la Loi. 
${}^{57}Rendons grâce à Dieu qui nous donne la victoire par notre Seigneur Jésus Christ. 
${}^{58}Ainsi, mes frères bien-aimés, soyez fermes, soyez inébranlables, prenez une part toujours plus active à l’œuvre du Seigneur, car vous savez que, dans le Seigneur, la peine que vous vous donnez n’est pas perdue.
      
         
      \bchapter{}
      \begin{verse}
${}^{1}Au sujet de la collecte pour les fidèles de Jérusalem, faites, vous aussi, comme je l’ai prescrit aux Églises de Galatie. 
${}^{2}Le premier jour de la semaine, chacun mettra de côté ce qu’il a réussi à épargner, afin que l’on n’attende pas mon arrivée pour faire la collecte. 
${}^{3}Quand je serai là, ce sont les personnes jugées aptes par vous que j’enverrai avec des lettres porter à Jérusalem votre don généreux. 
${}^{4}S’il est bon que j’y aille aussi, ces personnes iront avec moi.
      
         
${}^{5}Je viendrai chez vous après avoir traversé la Macédoine ; je ne ferai que traverser la Macédoine, 
${}^{6}mais, chez vous, je vais peut-être séjourner et, même, rester pour l’hiver, afin que vous m’aidiez à me rendre où je voudrais aller. 
${}^{7}Cette fois-ci, en effet, je ne veux pas vous voir seulement en passant ; j’espère bien rester quelque temps chez vous si le Seigneur le permet. 
${}^{8}Mais je resterai à Éphèse jusqu’à la Pentecôte, 
${}^{9}car une porte s’est ouverte toute grande à mon activité, et les adversaires sont nombreux.
${}^{10}Si Timothée vient, veillez à ce qu’il n’ait rien à craindre chez vous, car il travaille à l’œuvre du Seigneur, tout comme moi. 
${}^{11}Que personne donc ne le méprise. Aidez-le à revenir en paix auprès de moi, car je l’attends avec les frères. 
${}^{12}Au sujet d’Apollos notre frère, je l’ai fortement exhorté à venir chez vous avec les frères ; mais la volonté n’y était vraiment pas pour qu’il vienne maintenant. Il viendra donc quand ce sera le moment favorable.
${}^{13}Veillez, tenez bon dans la foi, soyez des hommes, soyez forts. 
${}^{14}Que tout chez vous se passe dans l’amour.
${}^{15}Frères, voici encore une exhortation : vous savez que Stéphanas et les gens de sa maison ont été dans votre province les premiers à croire, et se sont engagés au service des fidèles ; 
${}^{16}à votre tour, soyez soumis à de tels hommes et à tous ceux qui collaborent et peinent avec eux. 
${}^{17}Je suis heureux de la présence de Stéphanas, de Fortunatus et d’Akhaïcos, eux qui ont suppléé à votre absence ; 
${}^{18}en effet, ils ont tranquillisé mon esprit et le vôtre. Sachez donc apprécier de tels hommes.
${}^{19}Les Églises de la province d’Asie vous saluent. Aquilas et Prisca vous saluent bien dans le Seigneur, avec l’Église qui se rassemble dans leur maison. 
${}^{20}Tous les frères vous saluent. Saluez-vous les uns les autres par un baiser de paix.
${}^{21}La salutation est de ma main à moi, Paul. 
${}^{22}Si quelqu’un n’aime pas le Seigneur, qu’il soit anathème. « Marana tha ! » (Notre Seigneur, viens !) 
${}^{23}Que la grâce du Seigneur Jésus soit avec vous. 
${}^{24}Je vous aime tous dans le Christ Jésus.
