  
  
    
    \bbook{PREMIER LIVRE DE SAMUEL}{PREMIER LIVRE DE SAMUEL}
      
         
      \bchapter{}
      \begin{verse}
${}^{1}Il y avait un homme de la ville\\de Rama\\, dans la montagne d’Éphraïm ; il s’appelait Elcana, fils de Yéroham, fils d’Éliou, fils de Tohou, fils de Souf ; c’était un Éphratéen. 
${}^{2}Cet homme avait deux femmes. L’une s’appelait Anne, l’autre Peninna. Peninna avait des enfants, mais Anne n’en avait pas. 
${}^{3}Chaque année, Elcana montait de sa ville au sanctuaire de Silo pour se prosterner devant le Seigneur de l’univers\\et lui offrir un sacrifice. C’est à Silo que résidaient, comme prêtres du Seigneur, les deux fils d’Éli, Hofni et Pinhas.
${}^{4}Un jour, Elcana offrait le sacrifice ; il distribua des parts de la victime à sa femme Peninna, à tous ses fils et à toutes ses filles. 
${}^{5} Mais à Anne, il donna une part de choix car il aimait Anne, que pourtant le Seigneur avait rendue stérile. 
${}^{6} Sa rivale cherchait, par des paroles blessantes, à la mettre en colère parce que le Seigneur l’avait rendue stérile. 
${}^{7} Cela recommençait tous les ans, quand Anne montait au sanctuaire du Seigneur : Peninna cherchait à la mettre en colère. Anne pleura et ne voulut rien manger. 
${}^{8} Son mari Elcana lui dit : « Anne, pourquoi pleures-tu ? Pourquoi ne manges-tu pas ? Pourquoi ton cœur est-il triste ? Et moi, est-ce que je ne compte pas à tes yeux plus que dix fils ? »
       
${}^{9}Anne se leva, après qu’ils eurent mangé et bu\\. Le prêtre Éli était assis sur son siège, à l’entrée du sanctuaire du Seigneur. 
${}^{10} Anne, pleine d’amertume, se mit à prier le Seigneur et pleura abondamment. 
${}^{11} Elle fit un vœu en disant : « Seigneur de l’univers ! Si tu veux bien regarder l’humiliation\\de ta servante, te souvenir de moi, ne pas m’oublier, et me donner un fils\\, je le donnerai au Seigneur pour toute sa vie, et le rasoir ne passera pas sur sa tête\\. »
${}^{12}Tandis qu’elle prolongeait sa prière devant le Seigneur, Éli observait sa bouche. 
${}^{13} Anne parlait dans son cœur : seules ses lèvres remuaient, et l’on n’entendait pas sa voix. Éli pensa qu’elle était ivre 
${}^{14} et lui dit : « Combien de temps vas-tu rester ivre ? Cuve donc ton vin ! » 
${}^{15} Anne répondit : « Non, mon seigneur, je ne suis qu’une femme affligée, je n’ai bu ni vin ni boisson forte ; j’épanche mon âme devant le Seigneur. 
${}^{16} Ne prends pas ta servante pour une vaurienne\\ : c’est l’excès de mon chagrin et de mon dépit qui m’a fait prier\\aussi longtemps. » 
${}^{17} Éli lui répondit : « Va en paix, et que le Dieu d’Israël t’accorde ce que tu lui as demandé. » 
${}^{18} Anne dit alors : « Que ta servante trouve grâce devant toi ! » Elle s’en alla\\, elle se mit à manger, et son visage n’était plus le même.
${}^{19}Le lendemain, Elcana et les siens se levèrent de bon matin. Après s’être prosternés devant le Seigneur, ils s’en retournèrent chez eux, à Rama. Elcana s’unit à Anne sa femme, et le Seigneur se souvint d’elle. 
${}^{20}Anne conçut et, le temps venu\\, elle enfanta un fils ; elle lui donna le nom de Samuel (c’est-à-dire : Dieu exauce)\\car, disait-elle : « Je l’ai demandé au Seigneur. » 
${}^{21}Elcana, son mari, monta au sanctuaire avec toute sa famille pour offrir au Seigneur le sacrifice annuel et s’acquitter du vœu pour la naissance de l’enfant\\. 
${}^{22}Mais Anne n’y monta pas. Elle dit à son mari : « Quand l’enfant sera sevré, je l’emmènerai : il sera présenté au Seigneur, et il restera là pour toujours. » 
${}^{23}Son mari Elcana lui répondit : « Fais ce qui est bon à tes yeux ; reste ici jusqu’à ce que tu l’aies sevré. Toutefois, que le Seigneur réalise sa parole ! » La femme resta donc et allaita son fils jusqu’à ce qu’elle l’eût sevré.
       
${}^{24}Lorsque Samuel fut sevré\\, Anne, sa mère, le conduisit à la maison du Seigneur, à Silo ; l’enfant était encore tout jeune. Anne avait pris avec elle un taureau de trois ans\\, un sac de farine et une outre de vin. 
${}^{25} On offrit le taureau en sacrifice, et on amena l’enfant au prêtre Éli. 
${}^{26} Anne lui dit alors : « Écoute-moi\\, mon seigneur, je t’en prie ! Aussi vrai que tu es vivant, je suis cette femme qui se tenait ici près de toi pour prier le Seigneur. 
${}^{27} C’est pour obtenir cet enfant que je priais, et le Seigneur me l’a donné en réponse à ma demande. 
${}^{28} À mon tour je le donne au Seigneur pour qu’il en dispose\\. Il demeurera à la disposition du Seigneur tous les jours de sa vie. » Alors ils se prosternèrent devant le Seigneur.
      <p class="cantique" id="bib_ct-at_3"><span class="cantique_label">Cantique AT 3</span> = <span class="cantique_ref"><a class="unitex_link" href="#bib_1s_2_1">1 S 2, 1-10</a></span>
      
         
      \bchapter{}
       
      \begin{verse}
${}^{1}Et Anne fit cette prière\\ :
       
        \\Mon cœur exulte à cause du Seigneur ;
        \\mon front s’est relevé grâce à mon Dieu !
        \\Face à mes ennemis, s’ouvre ma bouche :
        \\oui, je me réjouis de ton salut !
         
        ${}^{2}Il n’est pas de Saint pareil au Seigneur.
        – Pas d’autre Dieu\\que toi !
        \\Pas de Rocher pareil à notre Dieu !
         
        ${}^{3}Assez de paroles hautaines,
        \\pas d’insolence à la bouche.
        \\Le Seigneur est le Dieu qui sait,
        \\qui pèse nos actes.
         
        ${}^{4}L’arc des forts est brisé,
        \\mais le faible se revêt de vigueur\\.
        ${}^{5}Les plus comblés s’embauchent pour du pain,
        \\et les affamés se reposent.
        \\Quand la stérile enfante sept fois,
        \\la femme aux fils nombreux dépérit\\.
         
        ${}^{6}Le Seigneur fait mourir et vivre ;
        \\il fait descendre à l’abîme et en ramène.
        ${}^{7}le Seigneur rend pauvre et riche ;
        \\il abaisse et il élève.
         
        ${}^{8}De la poussière, il relève le faible,
        \\il retire le malheureux de la cendre
        \\pour qu’il siège parmi les princes,
        \\et reçoive un trône de gloire.
         
        \\Au Seigneur, les colonnes de la terre :
        \\sur elles, il a posé le monde.
        ${}^{9}Il veille sur les pas de ses fidèles,
        \\et les méchants périront dans les ténèbres.
        \\La force ne rend pas l’homme vainqueur :
        ${}^{10}les adversaires du Seigneur seront brisés.
         
        \\Le Très-Haut\\tonnera dans les cieux ;
        \\le Seigneur jugera la terre entière.
        \\Il donnera la puissance à son roi,
        \\il relèvera le front de son messie\\.
${}^{11}Elcana repartit chez lui à Rama, tandis que l’enfant demeurait au service du Seigneur, en présence du prêtre Éli. 
${}^{12}Or les fils d’Éli étaient des vauriens qui ne connaissaient pas le Seigneur. 
${}^{13}À l’égard du peuple, la manière d’agir de ces prêtres-là était la suivante : chaque fois que l’on offrait un sacrifice, le servant du prêtre arrivait au moment où l’on faisait cuire la viande, ayant en main la fourchette à trois dents. 
${}^{14}Il piquait dans la cuve, le pot, le chaudron ou la marmite, et tout ce que ramenait la fourchette, le prêtre le prenait pour lui. C’est ainsi qu’ils procédaient envers tous ceux d’Israël qui venaient là-bas, à Silo. 
${}^{15}De surcroît, avant même que l’on fasse fumer la graisse, le servant du prêtre venait dire à l’homme qui offrait le sacrifice : « Donne pour le prêtre de la viande à rôtir ! Il n’acceptera pas de toi de la viande cuite mais seulement de la viande crue. » 
${}^{16}Si l’homme lui disait : « Qu’on fasse d’abord fumer la graisse, et ensuite prends ce que tu désires », alors il répondait : « Non ! tu dois me le donner maintenant, sinon je le prendrai de force. » 
${}^{17}Le péché des jeunes gens était très grand devant le Seigneur car ces hommes traitaient avec mépris l’offrande destinée au Seigneur.
${}^{18}Samuel assurait le service en présence du Seigneur ; l’enfant portait un pagne de lin. 
${}^{19}Sa mère lui faisait chaque année un petit manteau qu’elle lui apportait quand elle montait avec son mari pour offrir le sacrifice annuel. 
${}^{20}Éli bénissait Elcana et sa femme en disant : « Que le Seigneur t’accorde par cette femme une descendance, en échange de l’enfant qu’elle a mis à la disposition du Seigneur ! » Puis ils s’en retournaient chez Elcana. 
${}^{21}Et le Seigneur intervint en faveur d’Anne : elle devint enceinte et elle enfanta trois fils et deux filles. Quant au jeune Samuel, il grandissait auprès du Seigneur.
${}^{22}Éli était devenu très vieux. Il entendait raconter tout ce que faisaient ses fils à l’égard de tout Israël et aussi qu’ils couchaient avec les femmes qui étaient en fonction à l’entrée de la tente de la Rencontre. 
${}^{23}Il leur dit : « Pourquoi faites-vous de pareilles choses, ces mauvaises choses que j’entends dire par tout le peuple ? 
${}^{24}Non, mes fils, elle n’est pas belle, la rumeur que j’entends colporter par le peuple du Seigneur. 
${}^{25}Si un homme pèche contre un autre homme, Dieu sera l’arbitre. Mais si c’est contre le Seigneur qu’un homme pèche, qui interviendra pour lui ? » Ils n’écoutèrent pas la voix de leur père – en effet, le Seigneur voulait les faire mourir. 
${}^{26}Quant au jeune Samuel, il continuait de grandir en taille, aussi agréable au Seigneur qu’aux hommes.
${}^{27}Un homme de Dieu vint trouver Éli. Il lui dit : « Ainsi parle le Seigneur : Ne me suis-je donc pas révélé à la maison de ton père lorsqu’en Égypte elle appartenait à la maison de Pharaon ? 
${}^{28}J’ai choisi ton père parmi toutes les tribus d’Israël pour qu’il soit mon prêtre, pour qu’il monte à mon autel, fasse brûler l’encens et porte l’éphod en ma présence. J’ai donné à la maison de ton père toutes les nourritures offertes par les fils d’Israël. 
${}^{29}Pourquoi piétinez-vous mon sacrifice et mon offrande que j’ai prescrits dans la Demeure ? Pourquoi honores-tu tes fils plus que moi, au point de vous engraisser avec le meilleur de toutes les offrandes d’Israël, mon peuple ? 
${}^{30}C’est pourquoi – oracle du Seigneur, le Dieu d’Israël – certes, j’avais bien dit : “Ta maison et la maison de ton père marcheront en ma présence pour toujours”, mais maintenant – oracle du Seigneur –, quelle horreur ! Oui, j’honore seulement ceux qui m’honorent, mais ceux qui me dédaignent tombent dans le mépris. 
${}^{31}Voici venir des jours où je briserai ton bras et le bras de la maison de ton père, si bien qu’il n’y aura plus de vieillard dans ta maison. 
${}^{32}Tu contempleras un rival dans la Demeure et tout le bien qu’il fera à Israël ; mais dans ta maison, il n’y aura plus jamais de vieillard. 
${}^{33}Cependant, je laisserai l’un des tiens auprès de mon autel, pour que tes yeux se consument, et que ton âme languisse, alors que tous ceux qui auront proliféré dans ta maison mourront dans la force de l’âge.
${}^{34}Le signe en sera pour toi ce qui va arriver à tes deux fils Hofni et Pinhas : ils mourront tous deux le même jour. 
${}^{35}Puis, je susciterai pour moi un prêtre fidèle qui agira selon mon cœur et mon désir. Je bâtirai pour lui une maison stable, et il marchera en présence de mon messie pour toujours. 
${}^{36}Alors, tout ce qui subsistera de ta maison viendra se prosterner devant lui pour une piécette d’argent et une couronne de pain. Il dira : “Attache-moi, je t’en prie, à une fonction sacerdotale, pour que j’aie un morceau de pain à manger !” »
      
         
      \bchapter{}
      \begin{verse}
${}^{1}Le jeune Samuel assurait le service du Seigneur en présence du prêtre\\Éli. La parole du Seigneur était rare en ces jours-là, et la vision, peu répandue.
${}^{2}Un jour, Éli était couché à sa place habituelle – sa vue avait baissé et il ne pouvait plus bien voir. 
${}^{3} La lampe de Dieu n’était pas encore éteinte. Samuel était couché dans le temple du Seigneur, où se trouvait l’arche de Dieu. 
${}^{4} Le Seigneur appela Samuel, qui répondit : « Me voici ! » 
${}^{5} Il courut vers le prêtre\\Éli, et il dit : « Tu m’as appelé, me voici.\\ » Éli répondit : « Je n’ai pas appelé. Retourne te coucher. » L’enfant\\alla se coucher. 
${}^{6} De nouveau, le Seigneur appela Samuel. Et Samuel se leva. Il alla auprès d’Éli, et il dit : « Tu m’as appelé, me voici. » Éli répondit : « Je n’ai pas appelé, mon fils. Retourne te coucher. » 
${}^{7} Samuel ne connaissait pas encore le Seigneur, et la parole du Seigneur ne lui avait pas encore été révélée.
${}^{8}De nouveau, le Seigneur appela Samuel. Celui-ci se leva. Il alla auprès d’Éli, et il dit : « Tu m’as appelé, me voici. » Alors Éli comprit que c’était le Seigneur qui appelait l’enfant, 
${}^{9} et il lui dit : « Va te recoucher, et s’il t’appelle, tu diras : “Parle, Seigneur, ton serviteur écoute.” » Samuel alla se recoucher à sa place habituelle.
${}^{10}Le Seigneur vint, il se tenait là et il appela comme les autres fois : « Samuel ! Samuel ! » Et Samuel répondit : « Parle, ton serviteur écoute. »
${}^{11}Le Seigneur dit à Samuel : « Voici que je vais accomplir une chose en Israël à faire tinter les deux oreilles de qui l’apprendra. 
${}^{12}Ce jour-là, je réaliserai contre Éli toutes les paroles prononcées au sujet de sa maison, du début à la fin. 
${}^{13}Je lui ai annoncé que j’allais juger sa maison pour toujours, à cause de cette faute : sachant que ses fils méprisaient Dieu, il ne les a pas repris ! 
${}^{14}Voilà pourquoi, je le jure à la maison d’Éli : ni sacrifice, ni offrande, rien ne pourra jamais effacer la faute de la maison d’Éli. »
${}^{15}Samuel resta couché jusqu’au matin, puis il ouvrit les portes de la maison du Seigneur. Mais Samuel craignait de rapporter à Éli la vision. 
${}^{16} Éli appela Samuel et dit : « Samuel, mon fils ! » Il répondit : « Me voici. » 
${}^{17}Éli ajouta : « Quelle est la parole qu’il t’a adressée ? Ne me la cache pas, je t’en prie. Que Dieu amène le malheur sur toi, et pire encore, si tu me caches le moindre mot de toute la parole qu’il t’a adressée ! » 
${}^{18}Samuel lui rapporta toutes les paroles sans rien lui cacher. Alors Éli déclara : « C’est le Seigneur. Qu’il fasse ce qui est bon à ses yeux ! »
${}^{19}Samuel grandit. Le Seigneur était avec lui, et il ne laissa aucune de ses paroles sans effet\\. 
${}^{20}Tout Israël, depuis Dane jusqu’à Bershéba, reconnut que Samuel était vraiment un prophète du Seigneur. 
${}^{21}Le Seigneur continua de se manifester dans le temple de Silo, car c’est à Silo que le Seigneur se révélait par sa parole à Samuel.
      
         
      \bchapter{}
       
      \begin{verse}
${}^{1}Et la parole de Samuel s’adressa à tout Israël.
      Israël sortit pour aller combattre les Philistins. Israël campa près d’Ébène-Ézèr, tandis que les Philistins étaient campés à Apheq. 
${}^{2} Les Philistins se déployèrent contre Israël, et le combat s’engagea. Dans cette bataille rangée en rase campagne, Israël fut battu par les Philistins, qui tuèrent environ quatre mille hommes, 
${}^{3} et le peuple revint au camp. Les anciens d’Israël dirent alors : « Pourquoi le Seigneur nous a-t-il fait battre\\aujourd’hui par les Philistins ? Allons prendre à Silo l’arche de l’Alliance du Seigneur ; qu’elle vienne au milieu de nous, et qu’elle nous sauve de la main de nos ennemis. » 
${}^{4} Le peuple envoya des gens à Silo ; ils en rapportèrent l’arche de l’Alliance du Seigneur des armées\\qui siège sur les Kéroubim. Les deux fils du prêtre Éli, Hofni et Pinhas, étaient là auprès de l’arche de Dieu. 
${}^{5} Quand l’Arche\\arriva au camp, tout Israël poussa une grande ovation qui fit résonner la terre. 
${}^{6} Les Philistins entendirent le bruit\\et dirent : « Que signifie cette grande ovation\\dans le camp des Hébreux ? » Ils comprirent alors que l’arche du Seigneur était arrivée dans le camp. 
${}^{7} Alors ils eurent peur, car ils se disaient : « Dieu est arrivé au camp des Hébreux. » Puis ils dirent : « Malheur à nous ! Les choses ont bien changé depuis hier\\. 
${}^{8} Malheur à nous ! Qui nous délivrera de la main de ces dieux puissants ? Ce sont eux qui ont frappé les Égyptiens de toutes sortes de calamités dans le désert\\. 
${}^{9} Soyez forts, Philistins, soyez des hommes courageux\\, pour ne pas être asservis aux Hébreux comme ils vous ont été asservis : soyez courageux\\et combattez ! » 
${}^{10} Les Philistins livrèrent bataille, Israël fut battu et chacun s’enfuit à ses tentes. Ce fut un très grand désastre : en Israël trente mille soldats tombèrent. 
${}^{11} L’arche de Dieu fut prise, et les deux fils d’Éli, Hofni et Pinhas, moururent.
${}^{12}Un homme de Benjamin quitta en courant le champ de bataille et parvint à Silo le jour même ; il avait les vêtements déchirés et la tête couverte de terre. 
${}^{13}Lorsqu’il arriva, Éli était assis sur son siège, près du chemin ; il était aux aguets, car son cœur tremblait pour l’arche de Dieu. L’homme arriva donc dans la ville pour annoncer la nouvelle, et toute la ville se mit à pousser des cris. 
${}^{14}Éli entendit la clameur et se demanda : « Que veut dire ce bruit de foule ? » Mais l’homme vint en toute hâte porter la nouvelle à Éli. 
${}^{15}Or celui-ci était âgé de quatre-vingt-dix-huit ans ; il avait le regard fixe et ne pouvait plus voir. 
${}^{16}L’homme dit à Éli : « C’est moi qui viens du champ de bataille, je me suis enfui du champ de bataille aujourd’hui même. » Éli demanda : « Que s’est-il passé, mon fils ? » 
${}^{17}Le messager répondit : « Israël a fui devant les Philistins ; de plus, le peuple a subi une grande défaite ; même tes deux fils, Hofni et Pinhas, sont morts, et l’arche de Dieu a été prise… » 
${}^{18}À cette mention de l’arche de Dieu, Éli tomba de son siège à la renverse, sur le côté de la porte ; il se brisa la nuque et mourut : l’homme, en effet, était âgé et lourd. C’est lui qui avait jugé Israël pendant quarante ans.
${}^{19}Sa belle-fille, la femme de Pinhas, était enceinte et sur le point d’accoucher. En apprenant ces nouvelles : la prise de l’arche de Dieu, la mort de son beau-père et de son mari, elle s’accroupit et accoucha, car les douleurs l’avaient saisie. 
${}^{20}Comme elle était près de mourir, les femmes qui se tenaient auprès d’elle lui dirent : « Sois sans crainte, car c’est un fils que tu as mis au monde ! » Mais elle ne répondit pas et n’y prêta pas attention. 
${}^{21}Elle appela l’enfant Ikabod en disant : « La gloire est bannie d’Israël », par allusion à la prise de l’arche de Dieu, et à la mort de son beau-père et de son mari. 
${}^{22}Elle avait dit : « La gloire est bannie d’Israël », parce que l’arche de Dieu avait été prise.
      
         
      \bchapter{}
      \begin{verse}
${}^{1}Les Philistins avaient donc pris l’arche de Dieu. Ils la firent venir d’Ébène-Ézèr à Ashdod. 
${}^{2}Ils prirent l’arche de Dieu pour l’introduire dans la maison du dieu Dagone ; ils la placèrent à côté de Dagone. 
${}^{3}Mais lorsque les gens d’Ashdod se levèrent tôt le lendemain, voici que Dagone était tombé face contre terre devant l’arche du Seigneur. Ils prirent Dagone et le remirent à sa place. 
${}^{4}Ils se levèrent tôt le lendemain matin, et voici que Dagone était tombé face contre terre devant l’arche du Seigneur ; la tête de Dagone et les deux paumes de ses mains, coupées, se trouvaient sur le seuil. De Dagone, seul le corps était resté à sa place. 
${}^{5}Voilà pourquoi, aujourd’hui encore, à Ashdod, les prêtres de Dagone et tous ceux qui entrent dans la maison de Dagone évitent de fouler le seuil.
${}^{6}La main du Seigneur pesa lourdement sur les gens d’Ashdod. Il fit chez eux des ravages, il frappa de tumeurs Ashdod et son territoire. 
${}^{7}Lorsque les gens d’Ashdod virent ce qu’il en était, ils dirent : « Que l’arche du Dieu d’Israël ne reste pas chez nous, car sa main s’est faite dure contre nous et contre Dagone notre dieu ! » 
${}^{8}Ils invitèrent donc tous les princes des Philistins à se réunir chez eux et ils dirent : « Qu’allons-nous faire de l’arche du Dieu d’Israël ? » Les princes répondirent : « C’est dans la ville de Gath que doit être transférée l’Arche ! » Et l’on transféra l’arche du Dieu d’Israël.
${}^{9}Or, après qu’on l’eut transférée, la main du Seigneur fut sur la ville, causant une très grande panique. Le Seigneur frappa les gens de la ville du plus petit au plus grand : ils eurent des éruptions de tumeurs. 
${}^{10}Ils envoyèrent l’arche de Dieu à Éqrone. Mais, dès que l’arche de Dieu y arriva, tous les gens d’Éqrone s’écrièrent : « Ils ont transféré chez moi l’arche du Dieu d’Israël pour me faire mourir, moi et mon peuple ! » 
${}^{11}Ils invitèrent tous les princes des Philistins à se réunir et ils dirent : « Renvoyez l’arche du Dieu d’Israël, qu’elle retourne à l’endroit où elle était et qu’elle ne me fasse pas mourir, moi et mon peuple ! » En effet, il y avait dans toute la ville une panique de mort : la main de Dieu pesait très lourdement sur elle. 
${}^{12}Les gens qui ne mouraient pas étaient affligés de tumeurs, et le cri de détresse de la ville monta vers le ciel.
      
         
      \bchapter{}
      \begin{verse}
${}^{1}L’arche du Seigneur demeura en territoire philistin pendant sept mois. 
${}^{2}Puis les Philistins convoquèrent prêtres et devins, en disant : « Qu’allons-nous faire de l’arche du Seigneur ? Indiquez-nous comment la renvoyer à l’endroit où elle était. » 
${}^{3}Ils répondirent : « Si vous renvoyez l’arche du Dieu d’Israël, ne la renvoyez pas sans rien, mais ne manquez pas d’y joindre une offrande de réparation. Alors, vous serez guéris et vous saurez pourquoi sa main ne s’écartait pas de vous. » 
${}^{4}Ils demandèrent : « Quelle offrande de réparation faut-il y joindre ? » Ils répondirent : « D’après le nombre des princes Philistins : cinq tumeurs en or et cinq rats en or, car c’est un même fléau qui vous a tous atteints, vous et vos princes. 
${}^{5}Vous ferez donc des images de vos tumeurs et des images des rats qui dévastent votre pays, et vous rendrez gloire au Dieu d’Israël. Peut-être sa main se fera-t-elle plus légère sur vous, sur vos dieux et sur votre pays. 
${}^{6}À quoi bon alourdir votre cœur, comme l’ont fait les Égyptiens et Pharaon ? Quand Dieu se fut joué d’eux, n’ont-ils pas renvoyé les fils d’Israël ? Et ils sont partis. 
${}^{7}Maintenant, prenez et préparez un chariot neuf ainsi que deux vaches qui allaitent et qui n’ont pas encore porté le joug ; vous attellerez les vaches au chariot et vous les séparerez de leurs petits que vous ramènerez à l’étable. 
${}^{8}Puis, vous prendrez l’arche du Seigneur et vous la placerez sur le chariot. Quant aux objets d’or que vous lui remettrez en offrande de réparation, vous les déposerez dans le coffre, à côté de l’Arche. Vous la renverrez, et elle partira. 
${}^{9}Vous verrez alors : si elle prend la route de son territoire en montant vers Beth-Shèmesh, c’est bien Dieu qui nous a fait ce grand mal. Sinon, nous saurons que ce n’est pas sa main qui nous a touchés : c’est par accident que cela nous est arrivé. »
${}^{10}Ainsi firent les gens. Ils prirent deux vaches qui allaitaient, ils les attelèrent au chariot et retinrent leurs petits à l’étable. 
${}^{11}Puis ils déposèrent l’arche du Seigneur sur le chariot, ainsi que le coffre avec les rats en or et les images de leurs tumeurs. 
${}^{12}Les vaches allèrent droit leur chemin sur la route de Beth-Shèmesh. Elles avançaient en meuglant, mais gardèrent le même chemin sans se détourner ni à droite ni à gauche, les princes des Philistins marchant derrière elles jusqu’à la limite de Beth-Shèmesh. 
${}^{13}Les gens de Beth-Shèmesh faisaient la moisson des blés dans la vallée. Levant les yeux, ils aperçurent l’Arche et se réjouirent de la voir. 
${}^{14}Le chariot arriva dans le champ de Josué de Beth-Shèmesh et il s’y arrêta. Il y avait là une grande pierre. On fendit le bois du chariot et on offrit les vaches en holocauste au Seigneur. 
${}^{15}Les Lévites avaient descendu l’arche du Seigneur avec le coffre contenant les objets en or et placé à côté d’elle ; ils déposèrent le tout sur la grande pierre. Ce jour-là, les gens de Beth-Shèmesh offrirent des holocaustes et firent des sacrifices pour le Seigneur. 
${}^{16}Et ce même jour, les cinq princes des Philistins, ayant vu cela, s’en retournèrent à Éqrone. 
${}^{17}Voici quelles étaient les tumeurs en or remises par les Philistins en offrande de réparation au Seigneur : une pour Ashdod, une pour Gaza, une pour Ascalon, une pour Gath, une pour Éqrone ; 
${}^{18}et les rats en or, selon le nombre de toutes les villes des Philistins relevant des cinq princes – depuis la ville fortifiée jusqu’au village sans murailles. La Grande Pierre en est témoin, sur laquelle on avait déposé l’arche du Seigneur ; elle se trouve, aujourd’hui encore, dans le champ de Josué de Beth-Shèmesh.
${}^{19}Le Seigneur frappa les gens de Beth-Shèmesh, parce qu’ils avaient regardé dans l’arche du Seigneur. Il en frappa soixante-dix parmi le peuple. Et le peuple prit le deuil, parce que le Seigneur l’avait durement frappé. 
${}^{20}Les gens de Beth-Shèmesh dirent : « Qui pourra se tenir devant le Seigneur, ce Dieu saint ? » et : « Chez qui le Seigneur montera-t-il, loin de nous ? » 
${}^{21}Alors, ils envoyèrent des messagers aux habitants de Qiryath-Yearim pour leur dire : « Les Philistins ont ramené l’arche du Seigneur. Descendez ! Faites-la monter chez vous ! »
      
         
      \bchapter{}
      \begin{verse}
${}^{1}Les gens de Qiryath-Yearim vinrent donc et firent monter l’arche du Seigneur. Ils la firent entrer dans la maison d’Abinadab, sur la colline, et ils consacrèrent son fils Éléazar pour qu’il garde l’arche du Seigneur.
      
         
${}^{2}Depuis le jour où l’Arche s’installa à Qiryath-Yearim, de nombreux jours s’étaient écoulés, vingt ans déjà, lorsque toute la maison d’Israël se mit à soupirer après le Seigneur. 
${}^{3}Alors Samuel, s’adressant à toute la maison d’Israël, déclara : « Si c’est de tout votre cœur que vous revenez au Seigneur, écartez du milieu de vous les dieux de l’étranger et les Astartés, attachez vos cœurs au Seigneur, servez-le, lui seul, et il vous délivrera de la main des Philistins. » 
${}^{4}Alors les fils d’Israël écartèrent les Baals et les Astartés ; ils ne servirent plus que le Seigneur seul.
${}^{5}Samuel dit : « Rassemblez tout Israël à Mispa, et je prierai pour vous auprès du Seigneur. » 
${}^{6}Ils se rassemblèrent donc à Mispa. Ils puisèrent de l’eau qu’ils répandirent devant le Seigneur. Ce jour-là, ils jeûnèrent, et ils déclarèrent en ce lieu : « Nous avons péché contre le Seigneur. » Et Samuel jugea les fils d’Israël à Mispa.
${}^{7}Les Philistins apprirent que les fils d’Israël s’étaient rassemblés à Mispa, et les princes des Philistins montèrent pour attaquer Israël. Les fils d’Israël, en l’apprenant, eurent peur des Philistins. 
${}^{8}Ils dirent à Samuel : « Ne reste pas muet, ne nous abandonne pas, et ne cesse pas de crier vers le Seigneur notre Dieu, pour qu’il nous sauve de la main des Philistins ! » 
${}^{9}Samuel prit un agneau de lait et l’offrit tout entier en holocauste au Seigneur. Samuel cria vers le Seigneur en faveur d’Israël, et le Seigneur lui répondit. 
${}^{10}Pendant que Samuel offrait l’holocauste, les Philistins s’approchèrent pour combattre Israël. Mais, ce jour-là, le Seigneur tonna d’une grande voix contre les Philistins ; il les frappa de panique, et ils furent battus devant Israël. 
${}^{11}Alors les hommes d’Israël sortirent de Mispa et poursuivirent les Philistins de leurs coups jusqu’au-dessous de Beth-Kar. 
${}^{12}Samuel prit une pierre et la plaça entre Mispa et le lieu-dit « La Dent ». Il lui donna le nom d’Ébène-Ézèr (c’est-à-dire : Pierre du Secours), en disant : « Le Seigneur nous a secourus jusqu’ici. »
${}^{13}Les Philistins durent s’incliner. Ils ne recommencèrent plus à envahir le territoire d’Israël. La main du Seigneur fut contre eux durant toute la vie de Samuel. 
${}^{14}Alors les villes d’Israël dont les Philistins s’étaient emparés lui furent restituées, d’Éqrone à Gath, et Israël délivra leurs territoires de la main des Philistins. Il y eut la paix entre Israël et les Amorites. 
${}^{15}Samuel jugea Israël tous les jours de sa vie. 
${}^{16}Il allait d’année en année faire le tour des villes de Béthel, Guilgal et Mispa ; en tous ces lieux, il jugeait Israël. 
${}^{17}Il revenait ensuite à Rama, où il avait sa maison. C’est là qu’il jugea Israël, et là qu’il bâtit un autel au Seigneur.
      
         
      \bchapter{}
      \begin{verse}
${}^{1}Quand Samuel fut devenu vieux, il établit ses fils juges en Israël. 
${}^{2}Son fils aîné s’appelait Joël, et le second, Abiya ; ils jugeaient à Bershéba. 
${}^{3}Mais ses fils ne marchèrent pas sur ses traces. Attirés par le gain, ils acceptèrent des cadeaux et firent dévier le droit.
${}^{4}Tous les anciens d’Israël se réunirent et vinrent trouver Samuel à Rama. 
${}^{5}Ils lui dirent : « Tu es devenu vieux, et tes fils ne marchent pas sur tes traces. Maintenant donc, établis, pour nous gouverner\\, un roi comme en ont toutes les nations. » 
${}^{6}Samuel fut mécontent parce qu’ils avaient dit : « Donne-nous un roi pour nous gouverner\\ », et il se mit à prier le Seigneur. 
${}^{7}Or, le Seigneur lui répondit : « Écoute la voix du peuple en tout ce qu’ils te diront. Ce n’est pas toi qu’ils rejettent, c’est moi qu’ils rejettent : ils ne veulent pas que je règne sur eux. 
${}^{8}Tout comme ils ont agi depuis le jour où je les ai fait monter d’Égypte jusqu’à aujourd’hui, m’abandonnant pour servir d’autres dieux, de même agissent-ils envers toi. 
${}^{9}Maintenant donc, écoute leur voix, mais avertis-les solennellement et fais-leur connaître les droits du roi qui régnera sur eux. »
${}^{10}Samuel rapporta toutes les paroles du Seigneur au peuple qui lui demandait un roi. 
${}^{11} Et il dit : « Tels seront les droits du roi qui va régner sur vous. Vos fils, il les prendra, il les affectera à ses chars et à ses chevaux, et ils courront devant son char. 
${}^{12} Il les utilisera comme officiers de millier et comme officiers de cinquante hommes\\ ; il les fera labourer et moissonner à son profit, fabriquer ses armes de guerre et les pièces de ses chars. 
${}^{13} Vos filles, il les prendra pour la préparation de ses parfums, pour sa cuisine et pour sa boulangerie. 
${}^{14} Les meilleurs de vos champs, de vos vignes et de vos oliveraies, il les prendra pour les donner à ses serviteurs. 
${}^{15} Sur vos cultures et vos vignes il prélèvera la dîme, pour la donner à ses dignitaires\\et à ses serviteurs. 
${}^{16} Les meilleurs de vos serviteurs, de vos servantes et de vos jeunes gens, ainsi que vos ânes, il les prendra et les fera travailler pour lui. 
${}^{17} Sur vos troupeaux, il prélèvera la dîme, et vous-mêmes deviendrez ses esclaves. 
${}^{18} Ce jour-là, vous pousserez des cris à cause du roi que vous aurez choisi, mais, ce jour-là, le Seigneur ne vous répondra pas ! »
${}^{19}Le peuple refusa d’écouter Samuel et dit : « Non ! il nous faut un roi ! 
${}^{20}Nous serons, nous aussi, comme toutes les nations\\ ; notre roi nous gouvernera, il marchera à notre tête et combattra avec nous. » 
${}^{21}Samuel écouta toutes les paroles du peuple et les répéta aux oreilles du Seigneur. 
${}^{22}Et le Seigneur lui dit : « Écoute-les, et qu’un roi règne sur eux ! » Alors Samuel dit aux hommes d’Israël : « Allez ! chacun dans sa ville ! »
      
         
      \bchapter{}
      \begin{verse}
${}^{1}Il y avait dans la tribu\\de Benjamin un homme appelé Kish, fils d’Abiel, fils de Ceror, fils de Becorath, fils d’Afiah, fils d’un Benjaminite. C’était un homme de valeur\\. 
${}^{2}Il avait un fils appelé Saül, qui était jeune et beau. Aucun fils d’Israël n’était plus beau que lui, et il dépassait tout le monde de plus d’une tête\\. 
${}^{3}Les ânesses appartenant à Kish, père de Saül, s’étaient égarées. Kish dit à son fils Saül : « Prends donc avec toi l’un des serviteurs\\, et pars à la recherche des ânesses. » 
${}^{4} Ils traversèrent\\la montagne d’Éphraïm, ils traversèrent le pays de Shalisha sans les trouver ; ils traversèrent le pays de Shaalim : elles n’y étaient pas\\ ; ils traversèrent le pays de Benjamin sans les trouver.
${}^{5}Comme ils arrivaient au pays de Souf, Saül dit au serviteur qui l’accompagnait : « Allons, retournons, de peur que mon père ne se fasse du souci pour nous et en oublie les ânesses. » 
${}^{6}L’autre lui dit : « Mais il y a justement dans cette ville un homme de Dieu. C’est un homme respecté. Tout ce qu’il dit se produit à coup sûr. Allons-y maintenant ! Peut-être nous renseignera-t-il sur le chemin que nous suivons. » 
${}^{7}Saül dit à son serviteur : « Soit, allons-y ! Mais qu’apporterons-nous à cet homme ? Il n’y a plus de pain dans nos sacs, ni rien de convenable à offrir à l’homme de Dieu. Qu’avons-nous au juste ? » 
${}^{8}Le serviteur reprit la parole et répondit à Saül : « Il se trouve que j’ai dans la main un peu d’argent ; je le donnerai à l’homme de Dieu, et il nous renseignera sur notre chemin. » 
${}^{9}– Autrefois en Israël, quand on allait consulter Dieu, on disait : « Venez, allons chez le voyant ! », car celui qu’on appelle aujourd’hui « prophète », on l’appelait alors « voyant ». 
${}^{10}Saül dit à son serviteur : « Tu as bien parlé. Viens, allons-y ! » Et ils allèrent à la ville où se trouvait l’homme de Dieu.
${}^{11}Comme ils gravissaient la montée qui mène à la ville, ils trouvèrent des jeunes filles qui sortaient puiser de l’eau. Ils leur demandèrent : « Le voyant est-il par ici ? » 
${}^{12}Elles leur répondirent : « Oui, juste devant toi ! Mais maintenant dépêche-toi, car, s’il est venu aujourd’hui à la ville, c’est qu’il y a aujourd’hui un sacrifice pour le peuple sur le lieu sacré. 
${}^{13}À votre arrivée dans la ville, vous allez sûrement le trouver avant qu’il monte au lieu sacré pour manger, car le peuple ne mangera pas avant son arrivée. C’est lui en effet qui doit bénir le sacrifice ; après quoi, les invités pourront manger. Et maintenant, montez ! Car lui, vous le trouverez tout de suite. »
${}^{14}Ils montèrent donc à la ville. Comme ils pénétraient à l’intérieur de la ville, voici que Samuel sortit à leur rencontre pour monter au lieu sacré. 
${}^{15}Or, un jour avant l’arrivée de Saül, le Seigneur avait révélé ceci à l’oreille de Samuel : 
${}^{16}« Demain, à la même heure, je t’enverrai un homme du pays de Benjamin. Tu lui donneras l’onction comme chef de mon peuple Israël : c’est lui qui sauvera mon peuple de la main des Philistins. Oui, j’ai vu mon peuple ; oui, son cri est parvenu jusqu’à moi. » 
${}^{17}Quand Samuel aperçut Saül, le Seigneur l’avertit : « Voilà l’homme dont je t’ai parlé ; c’est lui qui exercera le pouvoir sur mon peuple. »
${}^{18}Saül aborda Samuel à l’entrée de la ville et lui dit : « Indique-moi, je t’en prie, où est la maison du voyant. » 
${}^{19}Samuel répondit à Saül : « C’est moi le voyant. Monte devant moi au lieu sacré. Vous mangerez aujourd’hui avec moi. Demain matin, je te laisserai partir et je te renseignerai sur tout ce qui te préoccupe. 
${}^{20}Tes ânesses égarées depuis trois jours, cesse de t’en préoccuper, car elles sont retrouvées. À qui donc appartient tout ce qu’il y a de précieux en Israël ? N’est-ce pas à toi et à toute la maison de ton père ? » 
${}^{21}Saül répondit : « Ne suis-je pas un Benjaminite, appartenant à l’une des plus petites tribus d’Israël ? Et ma famille n’est-elle pas la dernière de toutes les familles de la tribu de Benjamin ? Pourquoi donc me parles-tu ainsi ? »
${}^{22}Samuel prit Saül et son serviteur, et les fit entrer dans la salle. Il leur donna une place en tête des invités qui étaient une trentaine. 
${}^{23}Samuel dit au cuisinier : « Donne la part que je t’ai donnée, celle dont je t’ai dit : Mets-la de côté ! » 
${}^{24}Le cuisinier présenta le gigot avec le morceau qui est au-dessus. Il déposa le tout devant Saül, en disant : « Voilà ! Ce qui a été réservé est devant toi : mange ! Cela t’a été gardé pour cette fête, quand on a dit : Je convoque le peuple. » Saül mangea donc avec Samuel ce jour-là. 
${}^{25}Puis ils descendirent du lieu sacré à la ville, et Samuel s’entretint avec Saül sur la terrasse.
${}^{26}Le lendemain, ils se levèrent tôt. Dès que monta l’aurore, Samuel appela Saül sur la terrasse et lui dit : « Lève-toi ! Je vais te laisser partir. » Saül se leva, et ils sortirent tous deux au-dehors, lui et Samuel. 
${}^{27}Comme ils descendaient à la limite de la ville, Samuel dit à Saül : « Dis au serviteur de passer devant nous – et ce dernier passa devant – et toi, arrête-toi un instant, que je te fasse entendre la parole de Dieu. »
      
         
      \bchapter{}
      \begin{verse}
${}^{1}Alors, Samuel prit la fiole d’huile et la répandit sur la tête de Saül ; puis il l’embrassa et lui dit : « N’est-ce pas le Seigneur qui te donne l’onction comme chef sur son héritage ? 
${}^{2}Aujourd’hui, quand tu m’auras quitté, tu trouveras deux hommes près du tombeau de Rachel, sur la frontière de Benjamin, à Cilçah, et ils te diront : “Elles sont retrouvées, les ânesses que tu étais allé chercher. Mais maintenant ton père a oublié l’affaire des ânesses, il se fait du souci pour vous et se dit : Que faire pour mon fils ?” 
${}^{3}De là, poussant plus loin, tu arriveras au chêne de Tabor, où viendront te trouver trois hommes montant vers Dieu à Béthel, l’un portant trois chevreaux, l’autre portant trois couronnes de pain, le troisième une outre de vin. 
${}^{4}Ils te salueront et te donneront deux pains que tu recevras de leurs mains. 
${}^{5}Après cela, tu arriveras à Guibéa de Dieu, où il y a des postes de garde philistins. Et là, en entrant dans la ville, tu tomberas sur un groupe de prophètes qui descendent du lieu sacré, précédés de harpes, tambourins, flûtes et cithares ; ils seront en état de transe prophétique. 
${}^{6}Alors l’Esprit du Seigneur s’emparera de toi, tu seras saisi de transe prophétique avec eux et tu seras changé en un autre homme. 
${}^{7}Quand se produiront pour toi de tels signes, agis selon ce qui se présentera, car Dieu est avec toi. 
${}^{8}Tu descendras avant moi à Guilgal. Et moi, je descendrai te rejoindre pour offrir des holocaustes et faire des sacrifices de paix. Pendant sept jours tu attendras, jusqu’à ce que je vienne te rejoindre. Alors je te ferai savoir comment tu dois agir. »
      
         
${}^{9}Or, dès que Saül eut tourné le dos en quittant Samuel, Dieu changea son cœur, et tous ces signes se produisirent le jour même. 
${}^{10}À l’entrée de Guibéa, voici qu’un groupe de prophètes vint à sa rencontre. L’Esprit de Dieu s’empara de lui, et il fut saisi de transe prophétique au milieu d’eux. 
${}^{11}Alors tous ceux qui le connaissaient de longue date virent qu’il prophétisait avec les prophètes. Et les gens se dirent l’un à l’autre : « Qu’est-il donc arrivé au fils de Kish ? Saül aussi est-il parmi les prophètes ? » 
${}^{12}Un homme de cet endroit intervint pour dire : « Et qui est leur père ? ». Voilà comment est né le proverbe : « Saül aussi est-il parmi les prophètes ? »
${}^{13}Lorsqu’il fut sorti de sa transe prophétique, il se rendit au lieu sacré. 
${}^{14}L’oncle de Saül lui demanda, ainsi qu’à son serviteur : « Où êtes-vous allés ? » Il répondit : « À la recherche des ânesses. Mais nous n’avons rien vu et nous nous sommes rendus chez Samuel. » 
${}^{15}L’oncle de Saül dit : « Raconte-moi donc ce que Samuel vous a dit. » 
${}^{16}Saül répondit à son oncle : « Il nous a simplement annoncé que les ânesses étaient retrouvées. » Cependant, à propos de la royauté, il ne lui raconta pas ce qu’avait dit Samuel.
${}^{17}Samuel convoqua le peuple auprès du Seigneur, à Mispa. 
${}^{18}Il dit aux fils d’Israël : « Ainsi parle le Seigneur, le Dieu d’Israël : C’est moi qui ai fait monter Israël d’Égypte, qui vous ai délivrés de la main des Égyptiens et de tous les royaumes qui vous opprimaient. 
${}^{19}Mais vous, aujourd’hui, vous avez rejeté votre Dieu, lui qui vous a sauvés de tous vos malheurs et de toutes vos angoisses, et vous lui avez dit : “C’est un roi que tu établiras sur nous ! Et maintenant, présentez-vous devant le Seigneur par tribus et par clans. »
${}^{20}Samuel fit approcher toutes les tribus d’Israël, et la tribu de Benjamin fut désignée par le sort. 
${}^{21}Il fit approcher la tribu de Benjamin par familles, et la famille de Matri fut désignée. Puis Saül fils de Kish fut désigné. On le chercha, mais sans le trouver. 
${}^{22}On interrogea encore le Seigneur : « Y a-t-il encore quelqu’un qui soit venu ici ? » Et le Seigneur dit : « Voici qu’il est caché parmi les bagages ! » 
${}^{23}On courut le tirer de là, et il se présenta au milieu du peuple ; il dépassait tout le monde de plus d’une tête. 
${}^{24}Samuel dit à tout le peuple : « Avez-vous vu celui que le Seigneur a choisi ? Il n’a pas son pareil dans tout le peuple. » Et tout le peuple fit une ovation, en criant : « Vive le roi ! » 
${}^{25}Samuel exposa au peuple le droit de la royauté ; il l’écrivit dans un livre qu’il déposa devant le Seigneur. Puis Samuel renvoya tout le peuple, chacun chez soi. 
${}^{26}Saül aussi s’en alla chez lui, à Guibéa. Les hommes de valeur, dont Dieu avait touché le cœur, partirent avec lui. 
${}^{27}Quant aux vauriens, ils dirent : « Comment celui-là nous sauverait-il ? » Ils le méprisèrent et ne lui apportèrent pas d’offrandes. Mais Saül fit comme s’il n’avait rien entendu.
      
         
      \bchapter{}
      \begin{verse}
${}^{1}Nahash l’Ammonite monta prendre position contre Yabesh-de-Galaad. Alors tous les gens de Yabesh dirent à Nahash : « Conclus avec nous une alliance, et nous te servirons. » 
${}^{2}Mais Nahash l’Ammonite leur fit cette réponse : « Voici comment je la conclurai : en vous crevant à tous l’œil droit ! J’infligerai cette infamie à tout Israël. » 
${}^{3}Les anciens de Yabesh lui dirent : « Laisse-nous un répit de sept jours, que nous puissions envoyer des messagers dans tout le territoire d’Israël, et si personne ne vient à notre secours, nous nous rendrons à toi. » 
${}^{4}Les messagers arrivèrent à Guibéa de Saül et rapportèrent ces paroles aux oreilles du peuple. Alors tout le peuple éclata en sanglots. 
${}^{5}Or voici que Saül revenait des champs derrière ses bœufs et il demanda : « Qu’a donc le peuple ? Qu’a-t-il à pleurer ainsi ? » On lui répéta les paroles des gens de Yabesh. 
${}^{6}L’Esprit de Dieu s’empara de Saül quand il entendit ces paroles, et il s’enflamma d’une grande colère. 
${}^{7}Il prit une paire de bœufs, les dépeça et en envoya les morceaux dans tout le territoire d’Israël par l’entremise de messagers qui dirent : « Celui qui ne partira pas au combat derrière Saül et Samuel, voilà ce qui arrivera à ses bœufs ! » La terreur du Seigneur s’abattit sur le peuple, et ils marchèrent tous comme un seul homme. 
${}^{8}Saül les passa en revue à Bézèq : les fils d’Israël étaient trois cent mille, et les hommes de Juda, trente mille. 
${}^{9}On dit aux messagers qui étaient venus : « Transmettez ceci aux gens de Yabesh-de-Galaad : Demain, quand le soleil sera au plus chaud, vous aurez du secours. » Les messagers vinrent informer les gens de Yabesh, qui s’en réjouirent. 
${}^{10}Ceux-ci dirent aux Ammonites : « Demain, nous nous rendrons à vous, et vous nous traiterez comme bon vous semble. »
${}^{11}Or, le lendemain, Saül disposa le peuple en trois colonnes. Ils pénétrèrent dans le camp aux dernières heures de la nuit et frappèrent les Ammonites jusqu’à l’heure la plus chaude du jour. Alors les survivants se dispersèrent, et il n’en resta pas deux ensemble. 
${}^{12}Le peuple dit à Samuel : « Qui donc disait : “Saül régnerait-il sur nous ?” Livrez-nous ces gens-là, que nous les mettions à mort ! » 
${}^{13}Mais Saül déclara : « On ne mettra personne à mort en un tel jour, car, aujourd’hui même, le Seigneur a réalisé le salut en Israël. »
${}^{14}Samuel dit au peuple : « Venez, allons à Guilgal ! Nous y renouvellerons la royauté. » 
${}^{15}Tout le peuple alla donc à Guilgal. Là, à Guilgal, on fit roi Saül en présence du Seigneur ; on offrit des sacrifices de paix en présence du Seigneur. Et là, Saül et tous les gens d’Israël se livrèrent à de grandes réjouissances.
      
         
      \bchapter{}
      \begin{verse}
${}^{1}Samuel déclara à tout Israël : « Voyez comme j’ai écouté jusqu’au bout votre appel : j’ai fait régner sur vous un roi. 
${}^{2}Et maintenant, voici que le roi marche devant vous. Quant à moi, je suis devenu vieux, j’ai blanchi, et mes fils que voici sont avec vous. Moi qui ai marché devant vous depuis ma jeunesse jusqu’à ce jour, 
${}^{3}me voici ! Témoignez contre moi en face du Seigneur et de son messie : De qui ai-je pris le bœuf ? De qui ai-je pris l’âne ? Qui ai-je exploité ? Qui ai-je maltraité ? De la main de qui ai-je reçu un pot-de-vin pour fermer les yeux sur son cas ? – Je restituerai. » 
${}^{4}Ils répondirent : « Tu ne nous as pas exploités, tu ne nous as pas maltraités et tu n’as rien pris de la main de personne. » 
${}^{5}Il leur dit : « Le Seigneur est témoin contre vous, et son messie est témoin aujourd’hui, que vous n’avez rien trouvé en ma main. » Ils dirent : « Il est témoin… » 
${}^{6}Et Samuel dit au peuple : « Il est témoin, le Seigneur, lui qui a suscité Moïse et Aaron, et qui a fait monter vos pères du pays d’Égypte ! 
${}^{7}Et maintenant, tenez-vous prêts : que j’entre en jugement avec vous, devant le Seigneur, pour toutes les justes actions que le Seigneur a accomplies envers vous et vos pères. 
${}^{8}Alors que Jacob était venu en Égypte, vos pères ont crié vers le Seigneur, et le Seigneur envoya Moïse et Aaron qui les ont fait sortir d’Égypte et les ont installés en ce lieu où nous sommes. 
${}^{9}Mais ils ont oublié le Seigneur leur Dieu, et lui les a vendus au pouvoir de Sissera, chef de l’armée du roi de Haçor, au pouvoir des Philistins et au pouvoir du roi de Moab, qui leur ont fait la guerre. 
${}^{10}Alors, ils ont crié vers le Seigneur : “Nous avons péché en abandonnant le Seigneur pour servir les Baals et les Astartés. Maintenant, délivre-nous de la main de nos ennemis, et nous te servirons !” 
${}^{11}Le Seigneur envoya Yeroubbaal, Baraq, Jephté et Samuel ; il vous a délivrés de la main de vos ennemis d’alentour, et vous avez pu habiter le pays en sécurité. 
${}^{12}Mais quand vous avez vu Nahash, roi des fils d’Ammone, venir vous attaquer, vous m’avez dit : “Non, c’est un roi qui doit régner sur nous” – alors que votre roi, c’est le Seigneur votre Dieu. 
${}^{13}Et maintenant, voici le roi que vous avez choisi, celui que vous avez demandé, et voici que le Seigneur vous l’a donné. 
${}^{14}Puissiez-vous craindre le Seigneur, le servir, écouter sa voix, sans vous révolter contre les ordres du Seigneur et, vous-mêmes avec le roi qui règne sur vous, puissiez-vous suivre le Seigneur votre Dieu ! 
${}^{15}Mais si vous n’écoutez pas la voix du Seigneur, si vous vous révoltez contre les ordres du Seigneur, la main du Seigneur sera contre vous, comme elle fut contre vos pères.
${}^{16}Maintenant donc, tenez-vous prêts et voyez cette grande chose que le Seigneur va accomplir sous vos yeux. 
${}^{17}N’est-ce pas aujourd’hui la moisson des blés ? Je vais invoquer le Seigneur, et il fera tonner et pleuvoir. Comprenez donc et voyez à quel point vous avez mal agi aux yeux du Seigneur en demandant pour vous-mêmes un roi. » 
${}^{18}Samuel invoqua le Seigneur, et le Seigneur fit tonner et pleuvoir ce jour-là. Tout le peuple éprouva une grande crainte à l’égard du Seigneur et de Samuel. 
${}^{19}Tout le peuple dit à Samuel : « Prie le Seigneur ton Dieu en faveur de tes serviteurs, afin que nous ne mourions pas. Oui, nous avons ajouté à tous nos péchés ce mal de demander pour nous un roi. »
${}^{20}Samuel répondit au peuple : « Ne craignez pas ! Certes, vous avez fait tout ce mal. Mais cessez de vous écarter du Seigneur ; servez-le de tout votre cœur. 
${}^{21}Ne vous écartez pas du Seigneur : ce serait suivre les idoles de néant qui ne servent à rien et ne délivrent pas, car elles ne sont que néant. 
${}^{22}Le Seigneur, lui, ne rejettera pas son peuple, à cause de son grand nom ; en effet, il a voulu faire de vous son peuple. 
${}^{23}Et moi-même, quelle horreur pour moi si je péchais contre le Seigneur, si je cessais de prier en votre faveur ! Je vous enseignerai le chemin du bien et de la droiture. 
${}^{24}Seulement, vous, craignez le Seigneur et servez-le en vérité, de tout votre cœur : voyez ce qu’il a fait de grand parmi vous. 
${}^{25}Mais si vous persistez à faire le mal, vous et votre roi, vous périrez. »
      
         
      \bchapter{}
      \begin{verse}
${}^{1}On ignore l’âge de Saül lorsqu’il devint roi, et il régna deux ans sur Israël. 
${}^{2}Saül choisit trois mille hommes en Israël : il y en eut deux mille avec Saül à Mikmas et dans la montagne de Béthel, et mille avec Jonathan, son fils, à Guibéa de Benjamin. Quant au reste du peuple, il le renvoya, chacun à ses tentes.
${}^{3}Jonathan détruisit le poste de garde des Philistins, qui était à Guéba, et les Philistins l’apprirent. Alors Saül fit sonner du cor dans tout le pays pour dire : « Que les Hébreux l’apprennent ! » 
${}^{4}Tout Israël l’apprit et disait : « Saül a détruit le poste de garde des Philistins ; et même, Israël est devenu odieux aux Philistins ! » Alors le peuple se regroupa derrière Saül, à Guilgal. 
${}^{5}Les Philistins se rassemblèrent en vue de combattre Israël : trente mille chars, six mille cavaliers et une troupe aussi nombreuse que les grains de sable au bord de la mer. Ils montèrent établir leur camp à Mikmas, à l’est de Beth-Awen. 
${}^{6}Les hommes d’Israël virent le danger, tant le peuple était menacé de près. Ils se cachèrent dans les grottes, les trous, les rochers, les souterrains et les citernes. 
${}^{7}Des Hébreux passèrent aussi le Jourdain pour gagner le pays de Gad et de Galaad.
      Saül était encore à Guilgal, et tout le peuple qui était derrière lui tremblait de peur. 
${}^{8}Il attendit sept jours le rendez-vous de Samuel, mais Samuel ne vint pas à Guilgal. Le peuple, quittant Saül, commençait à se disperser. 
${}^{9}Alors Saül dit : « Amenez-moi les animaux pour l’holocauste et les sacrifices de paix ! » Et il offrit l’holocauste. 
${}^{10}Or, comme il achevait d’offrir l’holocauste, voici que Samuel arriva. Saül sortit à sa rencontre pour le saluer. 
${}^{11}Samuel lui dit : « Qu’as-tu fait ? » Saül répondit : « Quand j’ai vu que le peuple se dispersait en me quittant, que toi-même tu ne venais pas au rendez-vous et que les Philistins étaient rassemblés à Mikmas, 
${}^{12}je me suis dit : Maintenant, les Philistins vont descendre pour m’attaquer à Guilgal, sans que j’aie apaisé le Seigneur ! Alors, me faisant violence, j’ai offert l’holocauste. » 
${}^{13}Samuel dit à Saül : « Tu as agi comme un insensé ! Tu n’as pas observé le commandement du Seigneur ton Dieu, ce qu’il t’avait ordonné. Autrement, le Seigneur aurait établi ta royauté sur Israël pour toujours. 
${}^{14}Mais maintenant ta royauté ne tiendra pas. Le Seigneur a cherché un homme selon son cœur et l’a institué chef de son peuple, puisque tu n’as pas observé ce que t’avait ordonné le Seigneur. » 
${}^{15}Puis Samuel se leva pour monter de Guilgal à Guibéa de Benjamin, et le reste du peuple monta derrière Saül à la rencontre de la troupe en armes.
      Quand ils furent arrivés de Guilgal à Guibéa de Benjamin, Saül passa en revue les gens qui se trouvaient avec lui : environ six cents hommes.
${}^{16}Saül, son fils Jonathan et les gens qui se trouvaient avec eux étaient restés à Guéba de Benjamin, tandis que les Philistins avaient dressé leur camp à Mikmas. 
${}^{17}Les troupes de choc sortirent du camp des Philistins en trois groupes. Le premier prit la direction d’Ofra, au pays de Shoual, 
${}^{18}le deuxième, la direction de Beth-Horone, le troisième, la direction de la frontière qui surplombe le Val des Hyènes, du côté du désert.
${}^{19}On ne trouvait pas de forgeron dans tout le pays d’Israël, car les Philistins s’étaient dit : « Il ne faut pas que les Hébreux fabriquent des épées ou des lances. » 
${}^{20}Tout le monde en Israël descendait donc chez les Philistins pour affûter chacun son soc, sa houe, sa hache ou son pic. 
${}^{21}Le prix imposé était : deux tiers de sicle pour les socs, les houes, les haches et la remise en état des aiguillons. 
${}^{22}Il arriva donc qu’au jour du combat, personne, dans la troupe qui était avec Saül et Jonathan, n’avait en main ni épée ni lance. On en trouva cependant pour Saül et pour son fils Jonathan.
${}^{23}Un poste de Philistins vint s’établir au passage de Mikmas.
      
         
      \bchapter{}
      \begin{verse}
${}^{1}Un jour, Jonathan, fils de Saül, dit à son écuyer : « Viens ! Passons vers le poste des Philistins qui est là, sur l’autre versant ! » Mais il n’avertit pas son père. 
${}^{2}Saül se tenait à la limite de Guibéa sous le grenadier de Migrone et, avec lui, il y avait environ six cents hommes. 
${}^{3}Ahiyya, fils d’Ahitoub frère d’Ikabod, fils de Pinhas, fils d’Éli, était prêtre du Seigneur à Silo : il portait l’éphod. Le peuple ne savait pas que Jonathan était parti. 
${}^{4}Parmi les passages que Jonathan avait recherchés pour passer vers le poste des Philistins, il en est un avec une dent de rocher sur un versant et une dent de rocher sur l’autre. L’une est appelée Boceç, et l’autre, Senné. 
${}^{5}L’une se dresse au nord, en face de Mikmas, l’autre au sud, en face de Guéba.
${}^{6}Jonathan dit à son écuyer : « Viens ! Passons vers le poste de ces incirconcis. Peut-être le Seigneur agira-t-il en notre faveur, car rien n’empêche le Seigneur de donner le salut, que l’on soit peu ou beaucoup. » 
${}^{7}Son écuyer lui répondit : « Fais tout ce que tu as dans le cœur. Vas-y ! Et moi, je suis avec toi, selon ton cœur. » 
${}^{8}Jonathan reprit : « Voici que nous passons vers ces hommes : ils vont nous repérer. 
${}^{9}S’ils nous disent : “Halte ! Attendez que nous vous ayons rejoints”, alors nous resterons sur place, nous ne monterons pas vers eux. 
${}^{10}Mais s’ils nous disent : “Montez vers nous !”, alors nous monterons, car le Seigneur les aura livrés entre nos mains. Ce sera pour nous le signe. » 
${}^{11}Tous deux se firent repérer par le poste des Philistins, lesquels se dirent : « Voici des Hébreux qui sortent des trous où ils s’étaient cachés ! » 
${}^{12}Les hommes du poste interpellèrent Jonathan et son écuyer, en leur disant : « Montez vers nous ! Nous avons quelque chose à vous apprendre. » Jonathan dit à son écuyer : « Monte derrière moi : le Seigneur les a livrés au pouvoir d’Israël ! » 
${}^{13}Jonathan, suivi de son écuyer, monta en s’aidant des mains et des pieds. Alors les Philistins tombèrent devant Jonathan et, derrière lui, son écuyer les mettait à mort. 
${}^{14}Ce premier coup porté par Jonathan et son écuyer frappa une vingtaine d’hommes, sur l’étendue d’un demi-arpent. 
${}^{15}Ce fut la terreur dans le camp, dans la campagne et parmi tout le peuple. Le poste et les troupes de choc furent terrifiés eux aussi. La terre se mit à trembler, et ce fut une terreur de Dieu.
${}^{16}À Guibéa de Benjamin, les guetteurs de Saül observaient, et voici que le tumulte se propageait çà et là. 
${}^{17}Alors Saül dit à la troupe qui était avec lui : « Faites donc l’appel et voyez qui est parti de chez nous. » On fit l’appel : Jonathan et son écuyer manquaient. 
${}^{18}Saül dit à Ahiyya : « Fais approcher l’arche de Dieu. » En effet l’arche de Dieu était, ce jour-là, avec les fils d’Israël. 
${}^{19}Mais pendant que Saül parlait au prêtre, le tumulte dans le camp des Philistins ne cessait d’augmenter. Alors Saül dit au prêtre : « Retire ta main, cesse de consulter Dieu. » 
${}^{20}Puis, sur un cri de ralliement, Saül et toute la troupe qui était avec lui marchèrent au combat. Or, les Philistins avaient tiré l’épée l’un contre l’autre, et c’était la panique la plus totale. 
${}^{21}Quant aux Hébreux qui appartenaient de longue date aux Philistins, qui étaient montés avec eux au camp et se tenaient tout autour, même ceux-là passèrent du côté d’Israël qui était avec Saül et Jonathan. 
${}^{22}Tous les hommes d’Israël qui s’étaient cachés dans la montagne d’Éphraïm apprirent la déroute des Philistins ; ils se mirent, eux aussi, à leur poursuite pour les combattre. 
${}^{23}Ce jour-là, le Seigneur donna le salut à Israël. Le combat s’étendit au-delà de Beth-Awen.
${}^{24}Ce jour-là, les hommes d’Israël avaient été accablés parce que Saül avait proféré à l’adresse du peuple cette imprécation : « Maudit soit l’homme qui prendra de la nourriture avant le soir, avant que je me sois vengé de mes ennemis ! » Et personne dans le peuple n’avait goûté de nourriture.
${}^{25}Tout le monde entra dans la forêt. Or il y avait du miel à la surface du sol. 
${}^{26}Quand le peuple entra dans la forêt, voici qu’il y coulait du miel ! Cependant, nul n’y toucha pour en manger, car le peuple avait peur du serment. 
${}^{27}Mais Jonathan n’avait pas entendu son père imposer au peuple le serment. Il tendit le bâton qu’il avait à la main et en trempa le bout dans ce miel végétal ; il ramena la main à sa bouche, et son regard s’éclaircit. 
${}^{28}Mais quelqu’un dans le peuple prit la parole et dit : « Ton père a imposé au peuple ce serment solennel : “Maudit soit l’homme qui prendra de la nourriture aujourd’hui !”, alors que le peuple est épuisé. »
${}^{29}Jonathan répondit : « Mon père a porté malheur au pays. Voyez comme mon regard s’est éclairci parce que j’ai goûté un peu de ce miel ! 
${}^{30}À plus forte raison, si le peuple s’était nourri aujourd’hui sur le butin trouvé chez l’ennemi, le coup porté aux Philistins n’aurait-il pas été plus rude encore ? »
${}^{31}Ce jour-là, ils battirent les Philistins depuis Mikmas jusqu’à Ayyalone, et le peuple fut complètement épuisé. 
${}^{32}Il se jeta sur le butin. On prit du petit et du gros bétail, avec de jeunes bêtes ; on les égorgea à même le sol, et le peuple mangea au-dessus du sang. 
${}^{33}On rapporta la chose à Saül : « Voici que le peuple est en train de pécher contre le Seigneur en mangeant au-dessus du sang ! » Saül dit : « Vous avez trahi. Roulez vers moi une grosse pierre tant qu’il fait jour ! » 
${}^{34}Puis il déclara : « Dispersez-vous parmi le peuple et dites : Que chacun amène vers moi son bœuf ou son mouton. Vous les égorgerez et les mangerez ici. Mais vous ne pécherez pas contre le Seigneur en mangeant auprès du sang. » Alors, à la nuit tombée, tous les gens amenèrent le bœuf que chacun possédait et que l’on égorgea à cet endroit. 
${}^{35}Saül bâtit un autel au Seigneur. C’était la première fois qu’il bâtissait un autel au Seigneur.
${}^{36}Saül dit : « Descendons à la poursuite des Philistins pendant la nuit, pillons chez eux jusqu’au lever du jour et n’en laissons subsister aucun. » Ils répondirent : « Fais tout ce qui te semble bon. » Le prêtre dit : « Ici, approchons-nous de Dieu. » 
${}^{37}Alors Saül consulta Dieu : « Dois-je descendre à la poursuite des Philistins ? Les livreras-tu aux mains d’Israël ? » Mais, ce jour-là, Dieu ne lui donna pas de réponse. 
${}^{38}Alors Saül dit : « Avancez ici, vous tous, les chefs du peuple ; comprenez et voyez en quoi consiste le péché commis aujourd’hui. 
${}^{39}Oui, par le Seigneur vivant, le sauveur d’Israël, même si Jonathan, mon fils, est en cause, il mourra ! » Mais dans tout le peuple, personne ne lui répondit. 
${}^{40}Puis il dit à tout Israël : « Vous serez d’un côté. Moi et mon fils Jonathan, nous serons de l’autre côté. » Le peuple répondit : « Fais ce qui te semble bon. »
${}^{41}Alors Saül s’adressa au Seigneur : « Dieu d’Israël, fais connaître la vérité ! » Jonathan et Saül furent désignés par le sort, et le peuple fut mis hors de cause. 
${}^{42}Saül dit : « Jetez les sorts entre moi et mon fils Jonathan. » Et Jonathan fut désigné. 
${}^{43}Saül dit à Jonathan : « Révèle-moi ce que tu as fait. » Jonathan le lui révéla en disant : « Oui, j’ai goûté un peu de miel au bout du bâton que j’avais à la main. Me voici : que je meure ! » 
${}^{44}Saül déclara : « Que Dieu amène le malheur sur moi, et pire encore, si tu ne meurs pas, Jonathan ! » 
${}^{45}Mais le peuple dit à Saül : « Est-ce que Jonathan devrait mourir, lui qui a remporté cette grande victoire en Israël ? Quelle horreur ! Par le Seigneur vivant, il ne tombera pas à terre un seul cheveu de sa tête, car c’est avec Dieu qu’il a agi en ce jour. » Le peuple racheta Jonathan, et celui-ci ne mourut pas. 
${}^{46}Quant à Saül, il renonça à poursuivre les Philistins, et les Philistins s’en allèrent chez eux.
${}^{47}Quand Saül se fut emparé de la royauté sur Israël, il mena la guerre contre tous ses ennemis alentour : Moab, les fils d’Ammone, Édom, les rois de Soba et les Philistins. De quelque côté qu’il se tournât, il leur faisait du mal. 
${}^{48}Il déploya sa force et battit Amalec. Il délivra Israël de la main de ceux qui le pillaient.
${}^{49}Les fils de Saül étaient Jonathan, Yishwi et Malki-Shoua. Ses deux filles s’appelaient, l’aînée, Mérab et, la cadette, Mikal. 
${}^{50}La femme de Saül s’appelait Ahinoam, fille d’Ahimaaç. Le chef de son armée s’appelait Abner, fils de Ner qui était l’oncle de Saül. 
${}^{51}Kish, le père de Saül, et Ner, le père d’Abner, étaient fils d’Abiel.
${}^{52}Il y eut une guerre acharnée contre les Philistins durant tous les jours de Saül. Aussi, quand Saül remarquait un homme de valeur ou quelqu’un de courageux, il se l’attachait.
      
         
      \bchapter{}
      \begin{verse}
${}^{1}Samuel dit à Saül : « C’est moi que le Seigneur a envoyé pour te donner l’onction comme roi sur son peuple, sur Israël. Et maintenant, écoute la voix, écoute les paroles du Seigneur. 
${}^{2}Ainsi parle le Seigneur des armées : Je vais demander compte à Amalec de ce qu’il a fait à Israël en lui barrant la route, lorsqu’il montait d’Égypte. 
${}^{3}Maintenant donc, va ! Tu frapperas Amalec ; et vous devrez vouer à l’anathème tout ce qui lui appartient. Tu ne l’épargneras pas. Tu mettras tout à mort : l’homme comme la femme, l’enfant comme le nourrisson, le bœuf comme le mouton, le chameau comme l’âne. »
${}^{4}Saül convoqua le peuple et le passa en revue à Telaïm. Il y avait deux cent mille fantassins et, en plus, dix mille hommes de Juda.
${}^{5}Parvenu à la ville d’Amalec, Saül se mit en embuscade dans le lit du torrent. 
${}^{6}Il dit aux Qénites : « Allez, écartez-vous, sortez du milieu des Amalécites, de peur que je ne vous traite comme eux, alors que vous, vous avez agi avec fidélité envers tous les fils d’Israël quand ils montaient d’Égypte. » Et les Qénites s’écartèrent du milieu d’Amalec.
${}^{7}Saül frappa Amalec depuis Havila jusqu’à l’entrée de Shour qui est en face de l’Égypte. 
${}^{8}Il captura vivant Agag, le roi d’Amalec, et voua à l’anathème tout le peuple qu’il passa au fil de l’épée. 
${}^{9}Mais Saül et le peuple épargnèrent Agag, ainsi que le meilleur du petit et du gros bétail, les bêtes grasses, les agneaux et tout ce qu’il y avait de bon : ils ne voulurent pas les vouer à l’anathème. Par contre, tout ce qui était sans valeur et de mauvaise qualité, c’est cela qu’ils vouèrent à l’anathème.
${}^{10}La parole du Seigneur fut adressée à Samuel : 
${}^{11}« Je me repens d’avoir fait régner Saül comme roi car il s’est détourné de moi et n’a pas accompli mes paroles. » Alors Samuel fut bouleversé et cria vers le Seigneur toute la nuit.
${}^{12}Samuel se leva de bon matin pour rencontrer Saül. On vint lui annoncer : « Saül est allé au village de Carmel, il s’est dressé une stèle, il a fait demi-tour et, poussant plus loin, il est descendu à Guilgal. » 
${}^{13}Samuel arriva auprès de Saül, et Saül lui dit : « Sois béni du Seigneur ! J’ai accompli la parole du Seigneur. » 
${}^{14}Mais Samuel dit : « Quels sont ces bêlements qui frappent mes oreilles, et ces beuglements que j’entends ? » 
${}^{15}Saül répondit : « On a ramené ces animaux de chez les Amalécites : c’est que le peuple a épargné le meilleur du petit et du gros bétail en vue de le sacrifier au Seigneur ton Dieu, et ce qui restait, nous l’avons voué à l’anathème. »
${}^{16}Samuel dit à Saül : « Assez ! Je vais t’apprendre ce que le Seigneur m’a dit pendant la nuit. » Saül lui dit : « Parle. » 
${}^{17} Alors Samuel déclara : « Toi qui reconnaissais ta petitesse, n’es-tu pas devenu le chef des tribus d’Israël, puisque le Seigneur t’a donné l’onction comme roi sur Israël ? 
${}^{18} Il t’a envoyé en campagne et t’a donné cet ordre : “Va, et voue à l’anathème ces impies d’Amalécites, fais-leur la guerre jusqu’à l’extermination.” 
${}^{19} Pourquoi n’as-tu pas obéi à la voix du Seigneur ? Pourquoi t’es-tu jeté sur le butin. Pourquoi as-tu fait ce qui est mal aux yeux du Seigneur ? » 
${}^{20} Saül répondit à Samuel : « Mais j’ai obéi à la voix du Seigneur ! Je suis allé là où il m’envoyait, j’ai ramené Agag, roi d’Amalec, et j’ai voué Amalec à l’anathème. 
${}^{21} Dans le butin, le peuple a choisi le meilleur de ce qui était voué à l’anathème, petit et gros bétail, pour l’offrir en sacrifice au Seigneur ton Dieu, à Guilgal. » 
${}^{22} Samuel répliqua :
        \\« Le Seigneur aime-t-il les holocaustes et les sacrifices
        \\autant que l’obéissance à sa parole ?
        \\Oui, l’obéissance vaut mieux que le sacrifice,
        \\la docilité vaut mieux que la graisse des béliers.
        ${}^{23}La révolte est un péché comme la divination ;
        \\la rébellion est une faute comme la consultation des idoles.
        \\Parce que tu as rejeté la parole du Seigneur,
        \\lui aussi t’a rejeté : tu ne seras plus roi ! »
       
${}^{24}Saül dit à Samuel : « J’ai péché en transgressant l’ordre du Seigneur et tes paroles : c’est que j’ai eu peur du peuple et je lui ai obéi. 
${}^{25}Maintenant, je t’en prie, enlève mon péché, reviens avec moi, que je me prosterne devant le Seigneur. » 
${}^{26}Mais Samuel répondit à Saül : « Je ne reviendrai pas avec toi. Puisque tu as rejeté la parole du Seigneur, le Seigneur t’a rejeté pour que tu ne sois plus roi sur Israël. » 
${}^{27}Comme Samuel se détournait pour partir, Saül saisit le pan de son manteau, qui fut arraché. 
${}^{28}Alors Samuel lui dit : « Aujourd’hui, le Seigneur t’a arraché la royauté sur Israël et il l’a donnée à ton prochain qui vaut mieux que toi. » 
${}^{29}Pourtant, celui qui est la Splendeur d’Israël ne se dément pas ni ne se repent : n’étant pas un homme, il n’a pas à se repentir. 
${}^{30}Saül dit : « J’ai péché, mais daigne pourtant m’honorer devant les anciens de mon peuple et devant Israël. Reviens avec moi, pour que je me prosterne devant le Seigneur ton Dieu. » 
${}^{31}Samuel revint à la suite de Saül qui se prosterna devant le Seigneur.
${}^{32}Samuel dit : « Amenez-moi Agag, roi d’Amalec. » Agag vint à lui, tout heureux ; il se disait : « Vraiment, l’amertume de la mort s’est écartée. » 
${}^{33}Mais Samuel lui dit : « De même que, par ton épée, des femmes ont été privées de leurs enfants, de même, parmi les femmes, ta mère sera privée de son enfant ! » Et Samuel exécuta Agag devant le Seigneur, à Guilgal.
${}^{34}Samuel s’en alla à Rama, et Saül remonta chez lui à Guibéa de Saül. 
${}^{35}Samuel ne revit plus Saül jusqu’au jour de sa mort. En effet, Samuel avait pris le deuil à cause de Saül, et le Seigneur s’était repenti d’avoir fait régner Saül sur Israël.
      
         
      \bchapter{}
      \begin{verse}
${}^{1}Le Seigneur dit à Samuel : « Combien de temps encore seras-tu en deuil à cause de Saül ? Je l’ai rejeté pour qu’il ne règne plus sur Israël. Prends une corne que tu rempliras d’huile, et pars ! Je t’envoie auprès de Jessé de Bethléem, car j’ai vu parmi ses fils mon roi. » 
${}^{2} Samuel répondit : « Comment faire\\ ? Saül va le savoir, et il me tuera. » Le Seigneur reprit : « Emmène avec toi une génisse, et tu diras que tu viens\\offrir un sacrifice au Seigneur. 
${}^{3} Tu convoqueras Jessé au sacrifice ; je t’indiquerai moi-même ce que tu dois faire et tu me consacreras par l’onction celui que je te désignerai\\. »
${}^{4}Samuel fit ce qu’avait dit le Seigneur. Quand il parvint à Bethléem, les anciens de la ville allèrent à sa rencontre en tremblant, et demandèrent : « Est-ce pour la paix que tu viens ? » 
${}^{5} Samuel répondit : « Oui, pour la paix. Je suis venu offrir un sacrifice au Seigneur. Purifiez-vous, et vous viendrez avec moi au sacrifice. » Il purifia Jessé et ses fils, et les convoqua au sacrifice.
${}^{6}Lorsqu’ils arrivèrent et que Samuel aperçut Éliab, il se dit : « Sûrement, c’est lui le messie, lui qui recevra l’onction\\du Seigneur ! » 
${}^{7} Mais le Seigneur dit à Samuel : « Ne considère pas son apparence ni sa haute taille, car je l’ai écarté. Dieu ne regarde pas comme les hommes : les hommes regardent l’apparence, mais le Seigneur regarde le cœur. » 
${}^{8} Jessé appela Abinadab et le présenta à Samuel, qui dit : « Ce n’est pas lui non plus que le Seigneur a choisi. » 
${}^{9} Jessé présenta Shamma, mais Samuel dit : « Ce n’est pas lui non plus que le Seigneur a choisi. » 
${}^{10} Jessé présenta ainsi à Samuel ses sept fils, et Samuel lui dit : « Le Seigneur n’a choisi aucun de ceux-là. »
${}^{11}Alors Samuel dit à Jessé : « N’as-tu pas d’autres garçons ? » Jessé répondit : « Il reste encore le plus jeune, il est en train de garder le troupeau. » Alors Samuel dit à Jessé : « Envoie-le chercher : nous ne nous mettrons pas à table tant qu’il ne sera pas arrivé\\. » 
${}^{12} Jessé le fit donc venir : le garçon était roux, il avait de beaux yeux, il était beau. Le Seigneur dit alors : « Lève-toi, donne-lui l’onction : c’est lui ! » 
${}^{13} Samuel prit la corne pleine d’huile, et lui donna l’onction au milieu de ses frères. L’Esprit du Seigneur s’empara de David à partir de ce jour-là\\. Quant à Samuel, il se mit en route et s’en revint à Rama.
${}^{14}L’Esprit du Seigneur se détourna de Saül, et un esprit mauvais, envoyé par le Seigneur, se mit à le tourmenter. 
${}^{15}Les serviteurs de Saül lui dirent : « Voici qu’un mauvais esprit de Dieu te tourmente. 
${}^{16}Un seul mot de notre maître, et les serviteurs qui sont devant toi chercheront un bon joueur de cithare ; ainsi, quand un mauvais esprit de Dieu viendra sur toi, cet homme jouera de son instrument, et cela te fera du bien. » 
${}^{17}Saül répondit à ses serviteurs : « Voyez donc s’il n’y a pas pour moi un homme qui soit un bon musicien, et amenez-le-moi. » 
${}^{18}L’un des garçons prit la parole et dit : « J’ai vu, justement, un fils de Jessé, de Bethléem, qui sait jouer. C’est un homme de valeur, un vaillant guerrier ; il parle avec intelligence ; c’est un bel homme et le Seigneur est avec lui ! » 
${}^{19}Alors Saül envoya à Jessé des messagers pour lui dire : « Envoie-moi ton fils David, qui est avec le troupeau. » 
${}^{20}Jessé prit un âne qu’il chargea de pains, ainsi qu’une outre de vin et un chevreau, et il envoya son fils David les porter à Saül. 
${}^{21}David arriva auprès de Saül et se tint à sa disposition. Saül aima beaucoup David qui devint son écuyer. 
${}^{22}Saül envoya dire à Jessé : « Que David se tienne donc à ma disposition, car il a trouvé grâce à mes yeux. » 
${}^{23}Ainsi, lorsque l’Esprit de Dieu venait sur Saül, David prenait la cithare et en jouait. Alors Saül se calmait et se trouvait bien : l’esprit mauvais s’écartait de lui.
      
         
      \bchapter{}
      \begin{verse}
${}^{1}Les Philistins rassemblèrent leurs armées pour la guerre ; ils se rassemblèrent à Soko de Juda et ils établirent leur camp entre Soko et Azéqa, à Éfès-Dammim. 
${}^{2}Saül et les hommes d’Israël se rassemblèrent et établirent leur camp dans le Val du Térébinthe, puis se rangèrent en ordre de bataille face aux Philistins. 
${}^{3}Les Philistins se tenaient sur la montagne d’un côté, Israël se tenait sur la montagne de l’autre côté ; entre eux il y avait la vallée.
${}^{4}Alors sortit des rangs philistins un champion qui s’appelait Goliath. Originaire de Gath, il mesurait six coudées et un empan. 
${}^{5}Il avait un casque de bronze sur la tête, il était revêtu d’une cuirasse à écailles ; la cuirasse pesait cinq mille sicles de bronze. 
${}^{6}Il avait des jambières de bronze et un javelot de bronze entre les épaules. 
${}^{7}Le bois de sa lance était comme le rouleau d’un métier à tisser, et sa pointe pesait six cents sicles de fer. Et devant lui marchait le porte-bouclier.
${}^{8}Il s’arrêta et cria vers les lignes d’Israël. Il leur dit : « À quoi bon sortir pour vous ranger en ordre de bataille ? Ne suis-je pas, moi, le Philistin, et vous, les esclaves de Saül ? Choisissez-vous un homme, et qu’il descende vers moi ! 
${}^{9}S’il est le plus fort en luttant avec moi et qu’il m’abatte, nous deviendrons vos esclaves. Mais si je suis le plus fort et que je l’abatte, vous deviendrez nos esclaves, vous nous serez asservis. » 
${}^{10}Le Philistin ajouta : « Moi, aujourd’hui, je lance un défi aux lignes d’Israël : donnez-moi un homme, et nous lutterons l’un contre l’autre ! » 
${}^{11}Saül et tout Israël entendirent les paroles du Philistin ; ils en furent consternés, ils éprouvèrent une grande crainte.
${}^{12}David était fils de cet Éphratéen de Bethléem en Juda, nommé Jessé et qui avait huit fils. Or, au temps de Saül, cet homme était un vieillard avancé en âge. 
${}^{13}Les trois fils aînés de Jessé s’en étaient allés : ils avaient suivi Saül à la guerre. Les trois fils de Jessé partis à la guerre se nommaient : le premier-né Éliab, le deuxième Abinadab, et le troisième Shamma. 
${}^{14}David était le plus jeune. Les trois aînés avaient donc suivi Saül ; 
${}^{15}quant à David, il allait chez Saül et en revenait pour faire paître le troupeau de son père à Bethléem. 
${}^{16}Le Philistin s’avançait matin et soir ; il se présenta ainsi pendant quarante jours. 
${}^{17}Jessé dit à son fils David : « Prends donc pour tes frères cette mesure d’épis grillés, avec les dix pains que voici, et cours les porter au camp à tes frères. 
${}^{18}Ces dix fromages, tu les porteras à l’officier de millier ; tu verras si tes frères sont en bonne santé, et tu m’en rapporteras le signe que tout va bien. 
${}^{19}Saül, tes frères et tous les hommes d’Israël sont en train de combattre les Philistins dans le Val du Térébinthe. »
${}^{20}David se leva de bon matin, laissa le troupeau à un gardien, et partit avec les provisions, comme Jessé le lui avait ordonné. Il arriva au milieu du camp lorsque l’armée, sortant pour se mettre en ligne, poussait le cri de guerre. 
${}^{21}Israël et les Philistins se rangèrent ligne contre ligne. 
${}^{22}David se déchargea de ses bagages, les laissa aux mains du gardien des bagages et courut vers la ligne de front. Une fois arrivé, il demanda à ses frères s’ils étaient en bonne santé.
${}^{23}Comme il parlait avec eux, voici que monta des lignes philistines le champion appelé Goliath, le Philistin de Gath, qui reprit les mêmes paroles, et David l’entendit. 
${}^{24}En voyant l’homme, tous ceux d’Israël s’enfuirent devant lui, terrifiés. 
${}^{25}Ils disaient : « Avez-vous vu cet homme qui monte contre nous ? C’est pour défier Israël qu’il monte ! Celui qui l’abattra, le roi le fera riche, très riche ; il lui donnera sa fille, et il affranchira sa famille de toute charge en Israël. »
${}^{26}David demanda à ceux qui se tenaient près de lui : « Que fera-t-on pour récompenser l’homme qui abattra ce Philistin et relèvera le défi lancé à Israël ? Qui est-il, en effet, ce Philistin incirconcis, pour avoir défié les armées du Dieu vivant ? » 
${}^{27}Les gens répondirent avec les mêmes paroles : « Ainsi fera-t-on pour récompenser l’homme qui l’abattra… » 
${}^{28}Éliab, son frère aîné, l’entendit qui parlait avec les gens. Il se mit en colère contre David et dit : « Pourquoi donc es-tu descendu ? À qui as-tu laissé ton maigre troupeau dans le désert ? Je connais, moi, ton arrogance et la malice de ton cœur : c’est pour voir la bataille que tu es descendu ! » 
${}^{29}David répondit : « Qu’est-ce que j’ai fait encore ? On n’a plus le droit de parler ! » 
${}^{30}David se détourna de lui et s’adressa à un autre. Il répéta sa demande, et les gens lui firent la même réponse qu’auparavant. 
${}^{31}Mais les paroles de David attirèrent l’attention et furent rapportées à Saül qui le fit venir.
${}^{32}David dit à Saül : « Que personne ne perde courage à cause de ce Philistin. Moi, ton serviteur, j’irai me battre\\avec lui. » 
${}^{33}Saül répondit à David : « Tu ne peux pas marcher contre ce Philistin pour lutter avec lui, car tu n’es qu’un enfant, et lui, c’est un homme de guerre depuis sa jeunesse. »
${}^{34}David dit à Saül : « Quand ton serviteur était berger du troupeau de son père, si un lion ou bien un ours venait emporter une brebis du troupeau, 
${}^{35}je partais à sa poursuite, je le frappais et la délivrais de sa gueule. S’il m’attaquait, je le saisissais par la crinière et je le frappais à mort. 
${}^{36}Ton serviteur a frappé et le lion et l’ours. Eh bien ! ce Philistin incirconcis sera comme l’un d’eux puisqu’il a défié les armées du Dieu vivant ! » 
${}^{37}David insista : « Le Seigneur, qui m’a délivré des griffes du lion et de l’ours\\, me délivrera des mains de ce Philistin. » Alors Saül lui dit : « Va, et que le Seigneur soit avec toi ! » 
${}^{38}Saül revêtit David de ses propres vêtements. Il lui mit sur la tête un casque de bronze et le revêtit d’une cuirasse. 
${}^{39}David se mit à la ceinture l’épée de Saül par-dessus ses vêtements. Il fut incapable de marcher car il n’était pas entraîné. Et David dit à Saül : « Je ne peux pas marcher avec tout cela car je ne suis pas entraîné. » Et il s’en débarrassa.
${}^{40}David prit en main son bâton, il se choisit dans le torrent cinq cailloux bien lisses et les mit dans son sac de berger, dans une poche ; puis, la fronde à la main, il s’avança vers le Philistin. 
${}^{41} Le Philistin se mit en marche et, précédé de son porte-bouclier, approcha de David. 
${}^{42} Lorsqu’il le vit, il le regarda avec mépris car c’était un jeune garçon ; il était roux et de belle apparence. 
${}^{43} Le Philistin lui dit : « Suis-je donc un chien, pour que tu viennes contre moi avec un bâton\\ ? » Puis il le maudit en invoquant\\ses dieux. 
${}^{44} Il dit à David : « Viens vers moi, que je te donne en pâture\\aux oiseaux du ciel et aux bêtes sauvages ! » 
${}^{45} David lui répondit : « Tu viens contre moi avec épée, lance et javelot, mais moi, je viens contre toi avec le nom du Seigneur des armées\\, le Dieu des troupes d’Israël que tu as défié. 
${}^{46} Aujourd’hui le Seigneur va te livrer entre mes mains\\, je vais t’abattre, te trancher la tête, donner aujourd’hui même les cadavres de l’armée philistine aux oiseaux du ciel et aux bêtes de la terre. Toute la terre saura qu’il y a un Dieu pour Israël, 
${}^{47} et tous ces gens rassemblés sauront que le Seigneur ne donne la victoire ni par l’épée ni par la lance, mais que le Seigneur est maître du combat, et qu’il vous livre entre nos mains\\. »
${}^{48}Goliath s’était dressé, s’était mis en marche et s’approchait à la rencontre de David. Celui-ci s’élança\\et courut vers les lignes des ennemis\\à la rencontre du Philistin. 
${}^{49} Il plongea la main dans son sac, et en retira un caillou qu’il lança avec sa fronde. Il atteignit le Philistin au front, le caillou s’y enfonça, et Goliath tomba face contre terre. 
${}^{50} Ainsi David triompha du Philistin avec une fronde et un caillou : quand il frappa le Philistin et le mit à mort, il n’avait pas d’épée à la main. 
${}^{51} Mais David courut ; arrivé près du Philistin, il lui prit son épée, qu’il tira du fourreau, et le tua en lui coupant la tête. Quand les Philistins virent que leur héros était mort, ils prirent la fuite.
${}^{52}Les hommes d’Israël et de Juda se levèrent en poussant le cri de guerre ; ils poursuivirent les Philistins jusqu’à l’entrée de la vallée et jusqu’aux portes d’Éqrone. Des Philistins, blessés à mort, tombèrent sur la route de Shaaraïm, jusqu’à Gath et jusqu’à Éqrone. 
${}^{53}Puis, les fils d’Israël revinrent de leur poursuite acharnée contre les Philistins et se mirent à piller leur camp. 
${}^{54}David saisit la tête du Philistin et l’apporta à Jérusalem. Quant à ses armes, il les déposa dans sa propre tente.
${}^{55}Lorsque Saül avait vu David sortir à la rencontre du Philistin, il avait demandé au chef de l’armée, Abner : « De qui ce garçon est-il le fils, Abner ? » Et Abner lui avait répondu : « Par ta vie, ô roi, je ne le sais pas. » 
${}^{56}Le roi lui avait dit : « Informe-toi : de qui ce jeune homme est-il le fils ? » 
${}^{57}Quand David fut de retour, après avoir abattu le Philistin, Abner le retint et le fit venir devant Saül ; David avait à la main la tête du Philistin.
${}^{58}Saül lui demanda : « Mon garçon, de qui es-tu le fils ? » Et David lui répondit : « Je suis le fils de ton serviteur Jessé, de Bethléem. »
      
         
      \bchapter{}
      \begin{verse}
${}^{1}Or dès que David eut fini de parler à Saül, Jonathan s’attacha de toute son âme à David et il l’aima comme lui-même. 
${}^{2}Ce jour-là, Saül retint David et ne lui permit pas de retourner chez son père. 
${}^{3}Et Jonathan conclut une alliance avec David, car il l’aimait comme lui-même. 
${}^{4}Jonathan se dépouilla du manteau qu’il portait et le donna à David, ainsi que ses vêtements, et même son épée, son arc et son ceinturon. 
${}^{5}Dans ses expéditions, David réussissait partout où l’envoyait Saül, et Saül le mit à la tête des hommes de guerre. Il était bien vu de tout le peuple et même des serviteurs de Saül.
      
         
${}^{6}Au retour de l’armée\\, lorsque David revint après avoir tué le Philistin, les femmes sortirent de toutes les villes d’Israël à la rencontre du roi Saül pour chanter et danser au son des tambourins, des cris de joie et des cymbales\\. 
${}^{7} Les femmes dansaient\\en se renvoyant ce refrain :
        \\« Saül a tué ses milliers,
        \\et David, ses dizaines de milliers. »
${}^{8}Saül le prit très mal et fut très irrité. Il disait : « À David on attribue les dizaines de milliers, et à moi les milliers ; il ne lui manque plus que la royauté ! » 
${}^{9} Depuis ce jour-là\\, Saül regardait David avec méfiance.
${}^{10}Le lendemain, un mauvais esprit envoyé par Dieu s’empara de Saül qui fut saisi de transe prophétique au milieu de la maison. David jouait de son instrument comme chaque jour, et Saül avait sa lance à la main. 
${}^{11}Saül la lança en se disant : « Je vais clouer David au mur ! » Mais par deux fois David échappa à Saül.
${}^{12}Saül eut peur de David. En effet, le Seigneur était avec David et s’était écarté de Saül. 
${}^{13}Quant à Saül, il écarta David de sa présence en le nommant officier de millier. David partait en campagne et revenait à la tête du peuple. 
${}^{14}Il réussissait dans toutes ses expéditions : le Seigneur était avec lui. 
${}^{15}En voyant à quel point il réussissait, Saül le redouta. 
${}^{16}Mais tous, en Israël et Juda, aimaient David, car c’était lui qui partait en campagne et revenait à leur tête.
${}^{17}Saül dit à David : « Voici ma fille aînée Mérab. C’est elle que je te donnerai pour femme. Seulement, sois pour moi un bon guerrier, menant les combats du Seigneur. » Saül s’était dit : « Ne portons pas la main sur lui, mais que des Philistins le fassent ! » 
${}^{18}David dit à Saül : « Qui suis-je, quel est mon lignage, quel est le clan de mon père en Israël, pour que je devienne le gendre du roi ? » 
${}^{19}Mais au moment de donner à David Mérab, fille de Saül, on la donna pour femme à Adriël de Mehola.
${}^{20}Or Mikal, fille de Saül, aimait David. Saül en fut informé, et cela lui parut une bonne chose. 
${}^{21}Il se disait : « Je la lui donnerai, elle sera pour lui un piège, et la main des Philistins sera contre lui. » Saül déclara donc à David une deuxième fois : « Aujourd’hui, tu deviendras mon gendre. » 
${}^{22}Saül donna cet ordre à ses serviteurs : « Parlez à David en secret. Dites-lui : Voici que tu plais au roi, et tous ses serviteurs t’aiment. Deviens donc le gendre du roi ! » 
${}^{23}Les serviteurs de Saül redirent ces paroles à l’oreille de David qui déclara : « Est-ce peu de chose, à vos yeux, de devenir le gendre du roi ? Or, moi, je suis quelqu’un de pauvre et de peu d’importance. » 
${}^{24}Les serviteurs de Saül lui rapportèrent ces paroles en disant : « Voilà comment David a parlé. » 
${}^{25}Alors Saül reprit : « Vous direz ceci à David : Pour sa fille, le roi ne veut pas d’autre paiement que cent prépuces de Philistins, afin de tirer vengeance de ses ennemis. » Saül comptait ainsi faire tomber David aux mains des Philistins.
${}^{26}Les serviteurs rapportèrent ces paroles à David, et la chose lui parut bonne pour devenir le gendre du roi. Le délai n’était pas encore accompli 
${}^{27}que David se mettait en route, lui et ses hommes, et qu’il abattait deux cents Philistins. David rapporta leurs prépuces, il les remit tous au roi, afin de pouvoir devenir le gendre du roi. Alors Saül lui donna pour femme sa fille Mikal.
${}^{28}Voyant et comprenant que le Seigneur était avec David et que Mikal, sa propre fille, l’aimait, 
${}^{29}Saül eut encore plus peur de David et il éprouva contre lui une hostilité de tous les jours. 
${}^{30}Les princes des Philistins repartirent en campagne. Mais, à chacune de leurs campagnes, David remportait plus de succès que tous les serviteurs de Saül. Son nom devint très célèbre.
      
         
      \bchapter{}
      \begin{verse}
${}^{1}Saül dit à son fils Jonathan et à tous ses serviteurs son intention de faire mourir David. Mais Jonathan, le fils de Saül, aimait beaucoup David 
${}^{2} et il alla le prévenir : « Mon père Saül cherche à te faire mourir. Demain matin, sois sur tes gardes, mets-toi à l’abri, dissimule-toi. 
${}^{3} Moi, je sortirai et je me tiendrai à côté de mon père dans le champ où tu seras. Je parlerai de toi à mon père, je verrai ce qu’il en est et je te le ferai savoir. »
${}^{4}Jonathan fit à son père Saül l’éloge de David ; il dit : « Que le roi ne commette pas de faute contre son serviteur David, car lui n’a commis aucune faute envers toi. Au contraire, ses exploits sont une très bonne chose pour toi. 
${}^{5} Il a risqué sa vie, il a frappé à mort Goliath\\le Philistin, et le Seigneur a donné une grande victoire à tout Israël : tu l’as vu et tu en as été heureux. Pourquoi donc commettre une faute contre la vie d’un innocent\\, en faisant mourir David sans motif ? » 
${}^{6} Saül écouta Jonathan et fit ce serment : « Par le Seigneur vivant, il ne sera pas mis à mort ! » 
${}^{7} Alors Jonathan appela David et lui répéta tout ce que le roi avait dit. Puis il le conduisit à Saül, et David reprit sa place comme avant.
${}^{8}La guerre avait repris. David partit combattre les Philistins. Il leur porta un coup très dur, et ils s’enfuirent devant lui. 
${}^{9}Un mauvais esprit envoyé par le Seigneur vint sur Saül, alors qu’il était assis dans sa maison. Il tenait sa lance à la main, et David jouait de son instrument. 
${}^{10}Saül chercha à clouer David au mur avec sa lance, mais David esquiva le coup de Saül, qui ficha sa lance dans le mur. David prit la fuite et fut sauvé cette nuit-là.
${}^{11}Saül envoya des émissaires à la maison de David, pour le surveiller et le mettre à mort au matin. Mais Mikal, la femme de David, l’avertit : « Si tu ne te sauves pas cette nuit, demain tu seras mis à mort ! »
${}^{12}Et Mikal fit descendre David par la fenêtre. Il partit, prit la fuite et fut sauvé. 
${}^{13}Mikal saisit l’idole domestique, la plaça sur le lit ; elle mit à l’endroit de la tête une touffe de poils de chèvre et recouvrit le reste d’un vêtement.
${}^{14}Saül envoya des émissaires pour se saisir de David, mais elle dit : « Il est malade. » 
${}^{15}Alors Saül renvoya les émissaires voir David, en leur disant : « Apportez-le-moi, dans son lit, pour qu’il soit mis à mort. » 
${}^{16}Mais quand les émissaires furent entrés, il n’y avait, sur le lit, que l’idole, avec une touffe de poils de chèvre à l’endroit de la tête ! 
${}^{17}Saül dit à Mikal : « Pourquoi m’as-tu ainsi trompé ? Tu as laissé partir mon ennemi, et il est sauvé ! » Mikal répondit à Saül : « C’est lui qui m’a dit : “Laisse-moi partir. Pourquoi faudrait-il que je sois la cause de ta mort ?” »
${}^{18}David prit donc la fuite et fut sauvé. Il arriva chez Samuel à Rama et lui rapporta tout ce que Saül lui avait fait subir. Puis ils allèrent, lui et Samuel, habiter à Nayoth. 
${}^{19}On rapporta à Saül : « Voici que David est à Nayoth-de-Rama ! » 
${}^{20}Saül envoya des émissaires pour se saisir de David. Ils virent le groupe des prophètes, en état de transe prophétique, avec Samuel debout, à leur tête. Un esprit de Dieu vint sur les émissaires de Saül, et ils entrèrent en transe, eux aussi. 
${}^{21}On le rapporta à Saül qui envoya d’autres émissaires ; ils entrèrent en transe, eux aussi. Une troisième fois, Saül envoya des émissaires qui entrèrent en transe à leur tour.
${}^{22}Alors il se rendit lui-même à Rama et parvint à la grande citerne qui est à Sèkou. Il demanda : « Où se trouvent Samuel et David ? » On lui répondit : « À Nayoth-de-Rama ! » 
${}^{23}Comme il se rendait là-bas, à Nayoth-de-Rama, sur lui aussi vint un esprit de Dieu. Saisi de transe prophétique, il continua à marcher jusqu’à ce qu’il parvienne à Nayoth-de-Rama. 
${}^{24}Lui aussi, il retira ses vêtements, lui aussi fut en transe devant Samuel ; puis il s’écroula, nu, restant ainsi toute la journée et toute la nuit. Voilà pourquoi l’on dit : « Saül est-il aussi parmi les prophètes ? »
      
         
      \bchapter{}
      \begin{verse}
${}^{1}David s’enfuit de Nayoth-de-Rama et vint dire devant Jonathan : « Qu’ai-je fait ? Quel est mon péché ? Quelle est ma faute à l’égard de ton père, pour qu’il en veuille à ma vie ? » 
${}^{2}Jonathan lui répondit : « Quelle horreur ! Non, tu ne mourras pas. Il se trouve que mon père ne fait aucune chose, importante ou non, sans m’en informer. Alors, pourquoi m’aurait-il caché celle-là ? C’est impossible. » 
${}^{3}Mais David reprit, en faisant un serment : « Ton père sait très bien que j’ai trouvé grâce à tes yeux. Il s’est dit : “Que Jonathan ne sache rien, de peur qu’il ne soit affligé.” Mais, par le Seigneur vivant et par ta propre vie, il n’y a qu’un seul pas entre la mort et moi ! »
${}^{4}Jonathan dit à David : « À quoi penses-tu ? Je le ferai pour toi. » 
${}^{5}David répondit à Jonathan : « Demain, ce sera la nouvelle lune, et moi, je devrais prendre place auprès du roi pour le repas. Mais tu me laisseras partir et, après-demain, je me cacherai dans la campagne jusqu’au soir. 
${}^{6}Si ton père remarque mon absence, tu diras : “David a insisté auprès de moi pour faire un saut à Bethléem, sa ville, car on y célèbre le sacrifice annuel pour tout le clan.” 
${}^{7}S’il dit : “C’est bien”, je suis en paix, moi, ton serviteur. Mais s’il s’enflamme de colère, sache qu’il est résolu au pire. 
${}^{8}Agis avec fidélité envers ton serviteur, puisque tu m’as fait entrer dans une alliance du Seigneur avec toi. Mais si je suis coupable, fais-moi mourir toi-même. Pourquoi m’obliger à venir devant ton père ? » 
${}^{9}Jonathan répondit : « Quelle horreur pour toi ! Mais si vraiment j’apprenais que mon père est résolu au pire, comment ne pas t’en informer ? » 
${}^{10}David dit à Jonathan : « Qui m’informera si ton père te répond avec dureté ? » 
${}^{11}Jonathan lui répondit : « Viens ! Sortons dans la campagne. » Et tous deux sortirent dans la campagne.
${}^{12}Alors, Jonathan dit à David : « Par le Seigneur, le Dieu d’Israël ! Demain et après-demain, à cette heure-ci, je sonderai les intentions de mon père. Si tout va bien pour David, et si je n’envoie pas de message pour te le révéler, 
${}^{13}que le Seigneur amène le malheur sur Jonathan, et pire encore ! Mais s’il fait trouver bon à mon père de te mettre à mal, je te le révélerai et je te laisserai aller en paix. Que le Seigneur soit avec toi comme il fut avec mon père ! 
${}^{14}Tant que je vivrai, puisses-tu agir envers moi avec fidélité, comme le Seigneur, pour que je ne meure pas ! 
${}^{15}Puisses-tu ne jamais retirer ta fidélité de ma maison, même quand le Seigneur retranchera de la surface du sol chacun des ennemis de David. » 
${}^{16}Ainsi Jonathan conclut-il une alliance avec la maison de David, en disant : « Le Seigneur demandera des comptes à David – ou plutôt à ses ennemis. » 
${}^{17}Puis Jonathan fit encore prêter serment à David, par l’amitié qu’il lui portait, car il l’aimait comme lui-même.
${}^{18}Jonathan lui dit : « Demain, c’est la nouvelle lune ; on remarquera ton absence à cause de ta place inoccupée. 
${}^{19}Après-demain, tu descendras vite. Tu arriveras à l’endroit où tu étais caché l’autre jour et tu te placeras à côté de la butte. 
${}^{20}Quant à moi, je tirerai trois flèches de ce côté, comme pour viser une cible. 
${}^{21}J’enverrai le garçon en disant : “Va, retrouve les flèches !” Puis, si je lui dis : “Les flèches sont en arrière de toi ; ramasse-les”, alors tu pourras revenir, tu seras en paix ; par le Seigneur vivant, tout ira bien. 
${}^{22}Mais si je dis au jeune homme : “Les flèches sont au-devant de toi”, alors tu t’en iras, car c’est le Seigneur qui te fait partir. 
${}^{23}La parole que nous venons d’échanger, toi et moi, le Seigneur en est le garant pour toujours, entre toi et moi. »
${}^{24}David se cachait donc dans la campagne. Quand arriva la nouvelle lune, le roi prit place à table pour le repas. 
${}^{25}Le roi s’assit à sa place, comme les autres fois, sur le siège placé contre le mur. Jonathan se mit en face, Abner s’assit à côté de Saül, mais la place de David resta inoccupée. 
${}^{26}Saül n’en parla pas, ce jour-là, car il se disait : « Quelque chose est arrivé qui l’aura rendu impur. Sans doute, n’est-il pas pur. » 
${}^{27}Or, le lendemain de la nouvelle lune, le deuxième jour, la place de David restait inoccupée. Saül dit à son fils Jonathan : « Pourquoi le fils de Jessé n’est-il pas venu à table, ni hier, ni aujourd’hui ? » 
${}^{28}Jonathan répondit à Saül : « David m’a demandé avec insistance de le laisser aller jusqu’à Bethléem. 
${}^{29}Il m’a dit : “Laisse-moi partir, je t’en prie, car nous avons, dans la ville, un sacrifice pour le clan. Mon frère lui-même m’a ordonné de m’y rendre. Maintenant, si j’ai trouvé grâce à tes yeux, laisse-moi m’échapper pour que j’aille voir mes frères !” Voilà pourquoi il n’est pas venu à la table du roi. »
${}^{30}Saül s’enflamma de colère contre Jonathan. Il lui dit : « Fils rebelle et dévoyé ! Je sais bien, moi, que tu as pris parti pour le fils de Jessé, à ta honte et à la honte de la nudité de ta mère. 
${}^{31}Oui, aussi longtemps que le fils de Jessé sera vivant sur la terre, ni toi, ni ta royauté ne tiendront. Et maintenant, envoie quelqu’un pour le prendre : qu’on me l’amène, car il mérite la mort ! » 
${}^{32}Jonathan répondit à Saül son père. Il lui dit : « Pourquoi le mettre à mort ? Qu’a-t-il fait ? » 
${}^{33}Saül jeta sa lance contre lui pour le frapper. Alors Jonathan comprit que son père avait résolu de mettre à mort David. 
${}^{34}Jonathan se leva de table, enflammé de colère. Il ne mangea rien, ce deuxième jour de la nouvelle lune, car il était affligé au sujet de David et parce que son père l’avait insulté, lui, Jonathan.
${}^{35}Au matin, Jonathan sortit dans la campagne pour la rencontre avec David. Un jeune garçon l’accompagnait. 
${}^{36}Il dit au garçon : « Cours, retrouve-moi les flèches que je vais tirer ! » Le garçon courut. Or Jonathan avait tiré une flèche de manière à le dépasser. 
${}^{37}Le garçon parvint à l’endroit où se trouvait la flèche qu’avait tirée Jonathan, et celui-ci cria en direction du garçon : « Est-ce que la flèche n’est pas au-devant de toi ? » 
${}^{38}Puis Jonathan cria au garçon : « Vite, dépêche-toi, ne t’arrête pas ! » Le garçon de Jonathan ramassa la flèche et revint auprès de son maître. 
${}^{39}Le garçon ne savait rien, mais David et Jonathan, eux, savaient de quoi il s’agissait.
${}^{40}Jonathan confia ses armes à son garçon et lui dit : « Va les reporter à la ville. » 
${}^{41}Le garçon étant parti, David, à côté de la butte, se leva. Il tomba face contre terre et se prosterna trois fois. Ils s’embrassèrent et pleurèrent l’un avec l’autre, jusqu’à ce que David eût dominé ses larmes. 
${}^{42}Alors Jonathan dit à David : « Va en paix, puisque nous avons tous deux prêté serment au nom du Seigneur en disant : Que le Seigneur soit entre toi et moi, entre ta descendance et la mienne, pour toujours ! »
      
         
      \bchapter{}
      \begin{verse}
${}^{1}David se leva et s’en alla, tandis que Jonathan rentrait dans la ville.
      
         
${}^{2}David arriva à Nob chez le prêtre Ahimélek. Celui-ci vint en tremblant à la rencontre de David et lui dit : « Pourquoi es-tu seul, sans personne avec toi ? » 
${}^{3}David répondit au prêtre Ahimélek : « Le roi m’a donné un ordre et m’a dit : “Que personne ne sache rien de l’affaire pour laquelle je t’envoie et que je t’ai ordonnée.” Mes compagnons, je leur ai fixé un point de rencontre à tel endroit. 
${}^{4}Maintenant, qu’as-tu sous la main ? Donne-moi cinq pains ou bien ce que tu pourras trouver. » 
${}^{5}Le prêtre répondit à David : « Je n’ai pas de pain ordinaire sous la main, mais il y a le pain consacré. Les hommes pourront en manger s’ils se sont gardés de rapports avec les femmes. » 
${}^{6}David répondit au prêtre : « Assurément, les femmes nous ont été interdites, comme précédemment quand je partais en campagne ; sur ce point, les hommes étaient en état de sainteté. Cette expédition est profane, certes, mais aujourd’hui elle sera sanctifiée de ce fait. » 
${}^{7}Le prêtre lui donna alors du pain consacré. En effet, il n’y avait là pas d’autre pain que le pain disposé devant le Seigneur, celui que l’on retire, pour le remplacer, le jour même, par du pain chaud.
${}^{8}Cependant, le même jour, un des serviteurs de Saül se trouvait là, retenu devant le Seigneur. Il s’appelait Doëg l’Édomite, et c’était le plus important des bergers de Saül.
${}^{9}David dit à Ahimélek : « N’as-tu pas ici, sous la main, une lance ou une épée ? Je n’ai pris avec moi ni épée ni bagages, tant la mission du roi était urgente. » 
${}^{10}Le prêtre répondit : « Il y a l’épée de Goliath, le Philistin que tu as abattu dans le Val du Térébinthe : la voici, enveloppée dans le manteau, derrière l’éphod. Si tu veux la prendre, prends-la ; il n’y en a pas d’autre ici. » David lui dit : « Elle n’a pas sa pareille, donne-la-moi. »
${}^{11}David se leva et s’enfuit, ce jour-là, loin de Saül. Il arriva chez Akish, roi de Gath. 
${}^{12}Les serviteurs d’Akish dirent à leur maître : « Cet homme n’est-il pas David, le roi du pays ? N’est-ce pas celui pour qui l’on dansait en se renvoyant ce refrain :
        \\“Saül a tué ses milliers,
        \\et David, ses dizaines de milliers” ? »
${}^{13}David fut impressionné par ces paroles, si bien qu’il eut très peur d’Akish, roi de Gath. 
${}^{14}Alors, à leurs yeux, il fit semblant de perdre la raison ; il se mit à divaguer au milieu d’eux, à tracer des signes sur les battants de la porte, à baver dans sa barbe. 
${}^{15}Akish dit à ses serviteurs : « Vous voyez bien que cet homme est fou. Pourquoi me l’amenez-vous ? 
${}^{16}Est-ce que je manque de fous pour que vous ameniez celui-ci faire le fou devant moi ? Et cet individu entrerait dans ma maison ? »
      
         
      \bchapter{}
      \begin{verse}
${}^{1}David partit de là et se sauva dans la grotte d’Adoullam. L’ayant appris, ses frères, avec toute la maison de son père, descendirent le rejoindre à cet endroit. 
${}^{2}Alors se rassemblèrent autour de lui tous les gens en détresse, tous les endettés et tous les mécontents, et il devint leur chef. Il y eut ainsi avec lui environ quatre cents hommes.
${}^{3}David partit de là pour Mispé de Moab. Il dit au roi de Moab : « Permets que mon père et ma mère se retirent chez vous, en attendant que je sache ce que Dieu fera de moi. » 
${}^{4}Il les conduisit devant le roi de Moab, et ils restèrent chez le roi tout le temps que David se tint dans son refuge fortifié.
${}^{5}Le prophète Gad dit à David : « Ne reste pas dans ce refuge. Va-t’en, rentre au pays de Juda ! » David s’en alla et parvint à la forêt de Hèreth.
${}^{6}Saül apprit qu’on avait reconnu David et ceux qui l’accompagnaient. Saül se tenait sur la hauteur à Guibéa, sous le tamaris, la lance à la main, et tous ses serviteurs debout auprès de lui. 
${}^{7}Saül leur dit : « Écoutez bien, fils de Benjamin ! Le fils de Jessé vous donnerait-il, à vous tous, des champs et des vignes ? Vous nommerait-il, vous tous, officiers de millier ou officiers de centaine, 
${}^{8}pour que vous ayez tous conspiré contre moi, pour que personne ne m’ait averti quand mon fils faisait alliance avec le fils de Jessé ? Non, aucun de vous ne s’est tourmenté à mon sujet, et personne ne m’a averti quand mon fils dressait mon serviteur contre moi pour me tendre une embuscade, comme aujourd’hui. » 
${}^{9}Doëg l’Édomite prit la parole, lui qui se tenait là avec les serviteurs de Saül. Il dit : « J’ai vu le fils de Jessé : il est venu à Nob, chez Ahimélek, fils d’Ahitoub. 
${}^{10}Ahimélek a consulté pour lui le Seigneur, il lui a fourni des vivres et lui a donné l’épée de Goliath le Philistin. »
${}^{11}Alors le roi fit convoquer le prêtre Ahimélek, fils d’Ahitoub, et toute la maison de son père, c’est-à-dire les prêtres qui étaient à Nob. Ils vinrent tous auprès du roi. 
${}^{12}Saül dit : « Écoute bien, fils d’Ahitoub ! » Et celui-ci déclara : « Me voici, mon seigneur. » 
${}^{13}Saül lui dit : « Pourquoi avez-vous conspiré contre moi, toi et le fils de Jessé, quand tu lui as fourni du pain et une épée, quand tu as consulté Dieu en sa faveur, pour qu’il se dresse contre moi et me tende une embuscade, comme aujourd’hui ? » 
${}^{14}Ahimélek répondit au roi : « Y a-t-il parmi tous tes serviteurs quelqu’un de sûr comme David, le gendre du roi, lui qui est affecté à ta garde personnelle et qui est honoré dans ta maison ? 
${}^{15}Est-ce aujourd’hui que j’ai commencé à consulter Dieu pour lui ? Quelle horreur ! Que le roi ne charge en rien son serviteur ni la maison de mon père, car ton serviteur ignorait absolument tout de cette affaire. » 
${}^{16}Mais le roi lui dit : « Tu vas mourir, Ahimélek, tu vas mourir, toi et toute la maison de ton père. » 
${}^{17}Le roi dit aux gardes qui étaient debout près de lui : « Tournez-vous et mettez à mort les prêtres du Seigneur, puisqu’ils prêtent main forte, eux aussi, à David : ils le savaient en fuite et ne m’ont pas averti ! » Mais les serviteurs du roi refusèrent de lever la main pour frapper les prêtres du Seigneur.
${}^{18}Le roi dit alors à Doëg : « Tourne-toi et frappe les prêtres ! » Doëg l’Édomite se retourna, et c’est lui qui frappa les prêtres. Il fit mourir, ce jour-là, quatre-vingt-cinq hommes qui portaient l’éphod de lin. 
${}^{19}Et la ville des prêtres, Nob, on la passa au fil de l’épée : hommes et femmes, enfants et nourrissons, bœufs, ânes et moutons, au fil de l’épée ! 
${}^{20}Un seul des fils d’Ahimélek, fils d’Ahitoub, put se sauver. Il s’appelait Abiatar. Il prit la fuite pour rejoindre David. 
${}^{21}Abiatar rapporta à David que Saül avait massacré les prêtres du Seigneur, 
${}^{22}et David lui dit : « Je savais, ce jour-là, que Doëg l’Édomite étant présent, il irait sûrement informer Saül. C’est à cause de moi que les événements ont tourné contre toute personne de la maison de ton père. 
${}^{23}Reste avec moi, ne crains rien : il en voudra à ma vie, celui qui en voudra à la tienne ! Tu es en sûreté auprès de moi. »
      
         
      \bchapter{}
      \begin{verse}
${}^{1}On rapporta à David cette nouvelle : « Voici que les Philistins sont en guerre contre Qeïla ; ils pillent les aires à grain ! » 
${}^{2}David consulta le Seigneur : « Dois-je partir ? Est-ce que je battrai ces Philistins ? » Le Seigneur dit à David : « Pars, tu battras les Philistins et tu sauveras Qeïla. » 
${}^{3}Mais les hommes de David lui dirent : « Déjà, nous avons peur ici, en Juda : ce sera bien pire, si nous allons à Qeïla contre les lignes des Philistins ! » 
${}^{4}À nouveau, David consulta le Seigneur, et le Seigneur lui répondit : « Lève-toi ! Descends à Qeïla car je livre les Philistins entre tes mains. » 
${}^{5}David partit alors pour Qeïla avec ses hommes et attaqua les Philistins. Il emmena leurs troupeaux. Il leur porta un coup très dur. David sauva ainsi les habitants de Qeïla.
${}^{6}Or quand Abiatar, fils d’Ahimélek, s’était enfui auprès de David à Qeïla, il avait emporté l’éphod avec lui. 
${}^{7}Saül fut averti de l’entrée de David à Qeïla et il dit : « Dieu l’a remis en mon pouvoir, car il s’est enfermé lui-même, en entrant dans une ville munie de portes et de verrous. » 
${}^{8}Saül appela donc à la guerre tout le peuple pour descendre à Qeïla assiéger David et ses hommes. 
${}^{9}Quand David sut que Saül préparait contre lui un mauvais coup, il dit au prêtre Abiatar : « Apporte l’éphod. » 
${}^{10}David dit alors : « Seigneur, Dieu d’Israël, ton serviteur vient d’apprendre que Saül projetait de venir à Qeïla pour détruire la ville à cause de moi. 
${}^{11}Les notables de Qeïla vont-ils me livrer entre ses mains ? Saül descendra-t-il, comme ton serviteur vient de l’apprendre ? Seigneur, Dieu d’Israël, daigne en informer ton serviteur ! » Et le Seigneur dit : « Saül descendra. »
${}^{12}David dit : « Les notables de Qeïla vont-ils nous livrer, moi et mes hommes, entre les mains de Saül ? » Le Seigneur dit : « Ils vous livreront. » 
${}^{13}Alors David se leva, lui et ses hommes – ils étaient environ six cents. Ils sortirent de Qeïla et s’en allèrent à l’aventure. Mais on informa Saül que David s’était échappé de Qeïla, si bien qu’il renonça à son expédition.
${}^{14}David alla demeurer au désert dans des refuges ; il demeura dans la montagne, au désert de Zif. Pendant tout ce temps, Saül ne cessa de rechercher David, mais Dieu ne le livra pas entre ses mains. 
${}^{15}David s’aperçut que Saül s’était mis en campagne pour lui ôter la vie. David était dans le désert de Zif, à Horesha.
${}^{16}Jonathan, fils de Saül, se mit en route et alla trouver David à Horesha. Il l’encouragea au nom de Dieu, 
${}^{17}et lui dit : « Sois sans crainte : la main de mon père Saül ne te trouvera pas. C’est toi qui régneras sur Israël, et moi, je serai ton second ; d’ailleurs, Saül, mon père, le sait bien. » 
${}^{18}Ils conclurent tous deux une alliance devant le Seigneur. David demeura à Horesha, et Jonathan s’en alla chez lui.
${}^{19}Des gens de Zif montèrent auprès de Saül à Guibéa pour lui dire : « Est-ce que David ne se cache pas chez nous, dans les refuges de Horesha, sur la colline de Hakila, au sud de la steppe ? 
${}^{20}Et maintenant, si tel est ton désir, ô roi, descends ; c’est à nous de le livrer aux mains du roi. » 
${}^{21}Saül leur dit : « Soyez bénis du Seigneur, vous qui avez eu pitié de moi ! 
${}^{22}Allez donc, vérifiez bien, tâchez de savoir en quel endroit il est passé et si quelqu’un l’a vu là-bas. On me dit, en effet, qu’il est plein de ruse. 
${}^{23}Tâchez de reconnaître toutes les cachettes où il peut se cacher. Vous reviendrez me voir quand vous aurez vérifié, et j’irai avec vous. Alors, s’il est dans le pays, je le chercherai parmi tous les clans de Juda. »
${}^{24}Précédant Saül, ils se mirent en route pour Zif. David et ses hommes étaient dans le désert de Maône, dans la plaine au sud de la steppe. 
${}^{25}Saül et ses hommes allèrent à sa recherche. David en fut informé ; il descendit à la Roche et demeura dans le désert de Maône. Mais Saül, apprenant cela, poursuivit David au désert de Maône. 
${}^{26}Saül marchait d’un côté de la montagne ; David et ses hommes, de l’autre côté. David précipita sa marche pour s’éloigner de Saül, mais Saül et ses hommes l’encerclaient déjà, lui et les siens, pour les capturer. 
${}^{27}C’est alors qu’un messager vint dire à Saül : « Viens vite, car les Philistins ont fait irruption dans le pays. » 
${}^{28}Abandonnant la poursuite de David, Saül fit demi-tour pour affronter les Philistins. C’est pourquoi on a appelé ce lieu la « Roche-de-la-Séparation ».
      
         
      \bchapter{}
      \begin{verse}
${}^{1}De là, David monta aux refuges d’Enn-Guèdi où il s’établit. 
${}^{2}Or, quand Saül revint de la poursuite des Philistins, on l’en informa : « Voici que David est au désert d’Enn-Guèdi. » 
${}^{3}Saül prit trois mille hommes, choisis dans tout Israël, et partit à la recherche de David et de ses gens en face du Rocher des Bouquetins. 
${}^{4}Il arriva aux parcs à moutons qui sont en bordure de la route ; il y a là une grotte, où Saül entra pour se soulager\\. Or, David et ses hommes se trouvaient au fond de la grotte. 
${}^{5}Les hommes de David lui dirent : « Voici le jour dont le Seigneur t’a dit : “Je livrerai ton ennemi entre tes mains, tu en feras ce que tu voudras.” » David vint couper furtivement le pan du manteau de Saül. 
${}^{6}Alors le cœur lui battit d’avoir coupé le pan du manteau de Saül. 
${}^{7}Il dit à ses hommes : « Que le Seigneur me préserve de faire une chose pareille à mon maître\\, qui a reçu l’onction du Seigneur : porter la main sur lui, qui est le messie\\du Seigneur. » 
${}^{8}Par ses paroles, David retint ses hommes. Il leur interdit de se jeter sur Saül. Alors Saül quitta la grotte et continua sa route.
${}^{9}David se leva, sortit de la grotte, et lui cria : « Mon seigneur le roi ! » Saül regarda derrière lui. David s’inclina jusqu’à terre\\et se prosterna, 
${}^{10} puis il lui cria : « Pourquoi écoutes-tu les gens qui te disent : “David te veut du mal” ? 
${}^{11} Aujourd’hui même, tes yeux ont vu comment le Seigneur t’avait livré entre mes mains dans la grotte ; pourtant, j’ai refusé de te tuer, je t’ai épargné et j’ai dit : “Je ne porterai pas la main sur mon seigneur le roi qui a reçu l’onction du Seigneur.” 
${}^{12} Regarde, père, regarde donc : voici dans ma main le pan de ton manteau. Puisque j’ai pu le couper, et que pourtant je ne t’ai pas tué, reconnais qu’il n’y a en moi ni méchanceté ni révolte. Je n’ai pas commis de faute contre toi, alors que toi, tu traques ma vie pour me l’enlever. 
${}^{13} C’est le Seigneur qui sera juge entre toi et moi, c’est le Seigneur qui me vengera de toi, mais ma main ne te touchera pas ! 
${}^{14} Comme dit le vieux proverbe : “Des méchants sort la méchanceté.” C’est pourquoi ma main ne te touchera pas. 
${}^{15} Après qui donc le roi d’Israël s’est-il mis en campagne ? Après qui cours-tu donc ? Après un chien crevé, après une puce ? 
${}^{16} Que le Seigneur soit notre arbitre, qu’il juge entre toi et moi, qu’il examine et défende ma cause, et qu’il me rende justice, en me délivrant de ta main ! »
${}^{17}Lorsque David eut fini de parler, Saül s’écria : « Est-ce bien ta voix que j’entends, mon fils David ? » Et Saül se mit à crier et à pleurer. 
${}^{18}Puis il dit à David : « Toi, tu es juste, et plus que moi : car toi, tu m’as fait du bien, et moi, je t’ai fait du mal. 
${}^{19}Aujourd’hui tu as montré toute ta bonté envers moi : le Seigneur m’avait livré entre tes mains\\, et tu ne m’as pas tué ! 
${}^{20}Quand un homme surprend son ennemi, va-t-il le laisser partir tranquillement ? Que le Seigneur te récompense pour le bien que tu m’as fait aujourd’hui. 
${}^{21}Je sais maintenant que tu régneras certainement, et que la royauté d’Israël tiendra bon en ta main. 
${}^{22}Alors, jure-moi par le Seigneur que tu ne supprimeras pas ma descendance après moi, que tu ne feras pas disparaître mon nom de la maison de mon père. » 
${}^{23}David le jura à Saül. Puis Saül rentra chez lui, tandis que David et ses hommes remontaient à leur refuge.
      
         
      \bchapter{}
      \begin{verse}
${}^{1}Samuel mourut. Tout Israël se rassembla pour le pleurer. On l’ensevelit chez lui, à Rama. David se mit en route et descendit au désert de Parane.
${}^{2}Il y avait à Maône quelqu’un dont l’exploitation se trouvait à Carmel. Cet homme, très riche, possédait trois mille moutons et un millier de chèvres. Il se trouvait à Carmel pour la tonte de son troupeau. 
${}^{3}L’homme s’appelait Nabal, et sa femme, Abigaïl. La femme était intelligente et belle, tandis que l’homme était dur et malfaisant. Il était du clan de Caleb.
${}^{4}David, au désert, apprit que Nabal faisait tondre son troupeau. 
${}^{5}Il envoya dix serviteurs en leur disant : « Montez à Carmel. Vous entrerez chez Nabal et vous le saluerez de ma part. 
${}^{6}Vous direz : “Pour l’année qui vient, paix à toi, paix à ta maison, paix à tout ce qui t’appartient ! 
${}^{7}J’ai appris que l’on faisait la tonte chez toi. Sache maintenant ceci : tes bergers étaient avec nous, et nous ne les avons pas molestés ; rien n’a disparu de chez eux, tout le temps de leur séjour à Carmel. 
${}^{8}Interroge-les : ils t’informeront. Que mes serviteurs trouvent grâce à tes yeux, puisque nous sommes venus en ce jour de fête ! Alors, je t’en prie, donne à tes serviteurs et à ton fils David ce que ta main trouvera.” »
${}^{9}Les serviteurs de David entrèrent chez Nabal et lui répétèrent toutes ces paroles de la part de David, puis restèrent en silence. 
${}^{10}Nabal répondit aux serviteurs de David : « Qui est David et qui est le fils de Jessé ? Ils sont nombreux aujourd’hui, les serviteurs évadés de chez leur maître ! 
${}^{11}Et je prendrais de mon pain, de mon eau, de mes bêtes que j’ai fait abattre pour mes tondeurs, et je les donnerais à des gens dont je ne sais même pas d’où ils viennent ? » 
${}^{12}Les serviteurs de David rebroussèrent chemin ; ils s’en retournèrent et, arrivés auprès de David, lui rapportèrent toutes ces paroles. 
${}^{13}David dit à ses hommes : « Que chacun de vous prenne son épée. » Et chacun d’eux prit son épée. David aussi prit la sienne. Quatre cents hommes environ montèrent à la suite de David ; deux cents restèrent près des bagages.
${}^{14}Cependant, Abigaïl, la femme de Nabal, avait été avertie. L’un des bergers lui avait dit : « Voici que David a envoyé des messagers depuis le désert, pour saluer notre maître ; mais lui s’est emporté contre eux. 
${}^{15}Et pourtant, ces hommes étaient très bons pour nous : nous n’avons pas été molestés, et rien n’a disparu de chez nous, tout le temps où nous avons parcouru avec eux la campagne. 
${}^{16}Ils étaient pour nous un rempart, de nuit comme de jour, tout le temps où nous avons été avec eux à faire paître les troupeaux. 
${}^{17}Maintenant, tâche de voir ce que tu dois faire, car le malheur est décidé contre notre maître Nabal et toute sa maison. C’est un vaurien : on ne peut même pas lui parler ! »
       
${}^{18}Abigaïl se dépêcha de prendre deux cents pains, deux outres de vin, cinq moutons tout préparés, cinq boisseaux d’épis grillés, cent gâteaux de raisins secs et deux cents gâteaux de figues qu’elle chargea sur des ânes. 
${}^{19}Elle dit aux serviteurs : « Passez devant moi, je vous suis. » Cependant, elle n’avertit pas Nabal, son mari.
${}^{20}Alors que, sur son âne, elle descendait à l’abri de la montagne, voici que David et ses hommes descendaient dans sa direction : elle les rencontra. 
${}^{21}Or, David s’était dit : « C’est donc en pure perte que j’ai protégé tout ce que possédait ce Nabal dans le désert, et que rien n’a disparu de ce qu’il possédait ! Il m’a rendu le mal pour le bien. 
${}^{22}Que Dieu amène le malheur sur David – ou plutôt sur ses ennemis –, et pire encore, si je laisse subsister parmi tous les siens, d’ici demain matin, un seul mâle ! » 
${}^{23}Apercevant David, Abigaïl descendit vite de son âne, elle se jeta devant David, face contre terre, et se prosterna. 
${}^{24}S’étant jetée à ses pieds, elle dit : « C’est moi, c’est ma faute, mon seigneur ! Permets à ta servante de te parler ! Écoute donc les paroles de ta servante. 
${}^{25}De grâce, que mon seigneur ne prête pas attention à ce vaurien de Nabal : il porte bien son nom ! Son nom est “le Fou”, et la folie l’accompagne. Mais moi, ta servante, je n’avais pas vu les serviteurs de mon seigneur, ceux que tu avais envoyés. 
${}^{26}Et maintenant, par la vie du Seigneur et par ta propre vie, puisque le Seigneur t’a empêché d’en venir au sang et de te sauver par ta propre main, qu’ils deviennent comme Nabal, tes ennemis et ceux qui veulent du mal à mon seigneur ! 
${}^{27}Et maintenant, ce présent que ta servante apporte à mon seigneur, qu’il soit remis aux serviteurs qui t’accompagnent. 
${}^{28}Pardonne, je te prie, l’offense de ta servante. Nul doute, en effet : le Seigneur fera à mon seigneur une maison stable, parce que toi, mon seigneur, tu as mené les combats du Seigneur et que, de toute ta vie, on ne trouvera pas de mal en toi. 
${}^{29}Un homme s’est levé pour te poursuivre et s’en prendre à ta vie, mais la vie de mon seigneur sera gardée précieusement avec celles des vivants, auprès du Seigneur ton Dieu, tandis que la vie de tes ennemis, le Seigneur la placera dans le creux de sa fronde pour la lancer au loin. 
${}^{30}Aussi, lorsque le Seigneur aura fait à mon seigneur tout le bien qu’il a prédit à ton sujet et qu’il t’aura institué chef sur Israël, 
${}^{31}ce ne sera pas un obstacle pour toi, ni un remords au cœur de mon seigneur, d’avoir versé le sang inutilement, en voulant te sauver par ta propre main. Et quand le Seigneur aura fait du bien à mon seigneur, tu te souviendras de ta servante. »
${}^{32}David dit à Abigaïl : « Béni soit le Seigneur, Dieu d’Israël, qui t’a envoyée en ce jour à ma rencontre. 
${}^{33}Bénie soit ton intelligence, et bénie sois-tu, toi qui m’as retenu aujourd’hui d’en venir au sang et de me sauver par ma propre main ! 
${}^{34}Mais, par le Seigneur vivant, par le Dieu d’Israël qui m’a empêché de te faire du mal, si tu n’étais pas venue aussi vite à ma rencontre, il ne serait pas resté à Nabal un seul mâle, avant que le matin se lève ! » 
${}^{35}David reçut de la main d’Abigaïl ce qu’elle lui avait apporté. Puis il lui dit : « Remonte en paix chez toi. Tu le vois : je t’ai écoutée, je t’ai fait grâce. »
${}^{36}Quand Abigaïl revint chez Nabal, celui-ci donnait un festin dans sa maison, un vrai festin de roi. Nabal avait le cœur en joie, mais comme il était complètement ivre, Abigaïl ne l’informa de rien avant que le matin se lève. 
${}^{37}Le lendemain matin, après que Nabal eut cuvé son vin, sa femme l’informa de ce qui s’était passé. Alors le cœur de Nabal défaillit dans sa poitrine, et lui-même fut comme pétrifié. 
${}^{38}Au bout d’une dizaine de jours, le Seigneur frappa Nabal qui mourut.
       
${}^{39}David apprit que Nabal était mort et il dit : « Béni soit le Seigneur qui a défendu ma cause, après l’insulte reçue de Nabal, et qui a empêché son serviteur de faire le mal. Quant à la méchanceté de Nabal, le Seigneur l’a fait retomber sur sa tête. » Puis David envoya dire à Abigaïl qu’il la prendrait pour femme. 
${}^{40}Les serviteurs de David vinrent donc chez Abigaïl à Carmel et lui dirent : « David nous a envoyés chez toi afin de te prendre pour sa femme. » 
${}^{41}Elle se leva, puis se prosterna face contre terre et dit : « Voici ta servante, comme une esclave prête à laver les pieds des serviteurs de mon seigneur. » 
${}^{42}Se relevant en toute hâte, Abigaïl monta sur son âne et, accompagnée de cinq de ses servantes, elle suivit les messagers de David et devint sa femme.
${}^{43}David avait aussi épousé Ahinoam de Yizréel. Elles furent toutes les deux ses femmes. 
${}^{44}Or Saül avait donné sa fille Mikal, femme de David, en mariage à Palti, fils de Laïsh, qui était de Gallim.
      
         
      \bchapter{}
      \begin{verse}
${}^{1}Les gens de Zif vinrent trouver Saül à Guibéa pour lui dire : « Est-ce que David ne se cache pas sur la colline de Hakila, en face de la steppe ? » 
${}^{2}Saül se mit en route, il descendit vers le désert de Zif avec trois mille hommes, l’élite d’Israël, pour y traquer David. 
${}^{3}Saül campa sur la colline de Hakila qui est en face de la steppe, au bord de la route. David, qui se tenait dans le désert, s’aperçut que Saül le poursuivait jusqu’au désert. 
${}^{4}Il envoya des espions et fut certain de l’arrivée de Saül. 
${}^{5}David se mit en route et parvint à l’endroit où campait Saül. Il vit l’endroit où étaient couchés Saül et Abner, fils de Ner, le chef de son armée. Saül était couché au milieu du camp, et la troupe campait autour de lui.
${}^{6}David prit la parole et dit à Ahimélek le Hittite ainsi qu’à Abishaï, fils de Cerouya, le frère de Joab : « Qui veut descendre avec moi au camp, jusqu’à Saül ? » Abishaï répondit : « Moi, je descendrai avec toi. » 
${}^{7}David et Abishaï arrivèrent de nuit, près de la troupe. Or, Saül était couché, endormi, au milieu du camp, sa lance plantée en terre près de sa tête ; Abner et ses hommes\\étaient couchés autour de lui.
${}^{8}Alors Abishaï dit à David : « Aujourd’hui Dieu a livré ton ennemi entre tes mains. Laisse-moi donc le clouer à terre avec sa propre lance, d’un seul coup, et je n’aurai pas à m’y reprendre à deux fois. » 
${}^{9}Mais David dit à Abishaï : « Ne le tue pas ! Qui pourrait demeurer impuni après avoir porté la main sur celui qui a reçu l’onction\\du Seigneur ? » 
${}^{10}Puis David ajouta : « Par la vie du Seigneur ! C’est le Seigneur seul qui le frappera, soit que son jour arrive et qu’il meure, soit qu’il descende au combat et qu’il y périsse. 
${}^{11}Que le Seigneur me préserve de porter la main sur lui, le messie du Seigneur ! Et maintenant, prends donc la lance qui est près de sa tête avec la gourde d’eau, et allons-nous en. » 
${}^{12}David prit la lance et la gourde d’eau qui étaient près de la tête de Saül, et ils s’en allèrent. Personne ne vit rien, personne ne le sut, personne ne s’éveilla : ils dormaient tous, car le Seigneur avait fait tomber sur eux un sommeil mystérieux.
${}^{13}David passa sur l’autre versant de la montagne et s’arrêta sur le sommet, au loin, à bonne distance. 
${}^{14}Alors David cria en direction de la troupe et d’Abner, fils de Ner : « Ne vas-tu pas répondre, Abner ? » Celui-ci répondit : « Qui es-tu, toi qui appelles le roi ? » 
${}^{15}David dit à Abner : « N’es-tu pas un homme, toi qui es sans égal en Israël ? Alors pourquoi n’as-tu pas veillé sur le roi, ton maître ? Quelqu’un du peuple est venu pour tuer le roi, ton maître. 
${}^{16}Ce n’est pas bien, ce que tu as fait là. Par la vie du Seigneur, vous méritez la mort pour n’avoir pas veillé sur votre maître, le messie du Seigneur. Maintenant, regarde où sont la lance du roi et la gourde d’eau qui étaient près de sa tête. » 
${}^{17}Saül reconnut la voix de David et dit : « Est-ce bien ta voix, mon fils David ? » David répondit : « C’est ma voix, mon seigneur le roi. » 
${}^{18}Et il continua : « Pourquoi mon seigneur poursuit-il son serviteur ? Qu’ai-je donc fait et quel mal ai-je commis ? 
${}^{19}Que mon seigneur le roi daigne écouter maintenant les paroles de son serviteur. Si c’est le Seigneur qui t’a excité contre moi, qu’il soit apaisé par l’agréable odeur d’une offrande ! Mais si ce sont des êtres humains, maudits soient-ils devant le Seigneur ! Aujourd’hui ils m’ont expulsé, dépossédé de l’héritage du Seigneur, en disant : “Va servir d’autres dieux !” 
${}^{20}Et maintenant, que mon sang ne soit pas répandu sur la terre, loin de la face du Seigneur. En effet, le roi d’Israël s’est mis en campagne pour chercher une simple puce, comme on chasse la perdrix dans les montagnes. »
${}^{21}Saül dit : « J’ai péché. Reviens, mon fils David ! Je ne te ferai plus de mal, puisque ma vie a été aujourd’hui si précieuse à tes yeux. Oui, j’ai agi comme un fou et je me suis lourdement trompé. » 
${}^{22}David répondit : « Voici la lance du roi\\. Qu’un jeune garçon traverse et vienne la prendre ! 
${}^{23}Le Seigneur rendra à chacun selon sa justice et sa fidélité. Aujourd’hui, le Seigneur t’avait livré entre mes mains, mais je n’ai pas voulu porter la main sur le messie du Seigneur. 
${}^{24}Et de même que ta vie aujourd’hui a eu beaucoup de valeur à mes yeux, de même ma vie en aura beaucoup aux yeux du Seigneur, qui me délivrera de toute angoisse. » 
${}^{25}Saül dit à David : « Béni sois-tu, mon fils David ! Oui, quoi que tu entreprennes, tu réussiras. » Puis David reprit son chemin, et Saül retourna chez lui.
      
         
      \bchapter{}
      \begin{verse}
${}^{1}David se dit en lui-même : « C’est sûr, un jour ou l’autre, je périrai par la main de Saül. Mieux vaut donc pour moi m’échapper définitivement dans le pays des Philistins. Saül renoncera désormais à me chercher dans tout le territoire d’Israël ; ainsi j’échapperai à sa main ! » 
${}^{2}David se mit en route avec les six cents hommes qui l’accompagnaient et passa chez Akish, fils de Maok, roi de Gath. 
${}^{3}David s’installa auprès d’Akish, à Gath, lui et ses hommes, chacun avec sa famille. David y était avec ses deux femmes : Ahinoam de Yizréel et Abigaïl de Carmel – la femme de Nabal. 
${}^{4}On avertit Saül que David s’était enfui à Gath, et Saül cessa de le chercher.
${}^{5}David dit à Akish : « Si j’ai trouvé grâce à tes yeux, qu’on me donne un lieu où je puisse habiter, dans une ville à l’écart. Pourquoi ton serviteur habiterait-il auprès de toi, dans la ville royale ? » 
${}^{6}Aussitôt, Akish lui donna Ciqlag. C’est pourquoi Ciqlag a appartenu aux rois de Juda, jusqu’à ce jour. 
${}^{7}La durée du séjour de David en territoire philistin fut d’un an et quatre mois.
${}^{8}David entreprit, avec ses hommes, des incursions chez les Gueshourites, les Guirzites et les Amalécites, ces peuplades qui, depuis toujours, occupent le territoire jusqu’à l’entrée de Shour et jusqu’au pays d’Égypte. 
${}^{9}David dévastait le pays, ne laissant en vie ni homme ni femme, s’emparant du petit et du gros bétail, des ânes, des chameaux, ainsi que des vêtements. Puis, à son retour, il se rendait chez Akish. 
${}^{10}Quand Akish demandait : « Où avez-vous fait une incursion aujourd’hui ? », David répondait : « Contre le Néguev de Juda » ou « Contre le Néguev des Yerahmëélites » ou « Dans le Néguev des Qénites ». 
${}^{11}David ne laissait ramener vivant à Gath ni homme ni femme. Il se disait : « Ils pourraient parler contre nous et dire : “Voilà ce que David a fait”. » Telle fut sa manière d’agir tout le temps qu’il séjourna dans le pays philistin. 
${}^{12}Akish faisait confiance à David. Il se disait : « David s’est rendu vraiment trop odieux à son peuple Israël : il restera mon serviteur à jamais. »
      
         
      \bchapter{}
      \begin{verse}
${}^{1}En ces jours-là, les Philistins rassemblèrent leurs armées et entrèrent en campagne pour combattre Israël. Akish dit à David : « Tu sais sûrement que tu vas partir avec moi à la guerre, toi et tes hommes. » 
${}^{2}David répondit à Akish : « Certes, tu sais, toi-même, ce que fera ton serviteur ! » Et Akish lui dit : « Eh bien, je te prends pour toujours comme garde du corps ! »
      
         
${}^{3}Samuel était mort, tout Israël l’avait pleuré et l’avait enseveli dans sa ville, à Rama. Or Saül avait écarté du pays les nécromanciens et les devins.
${}^{4}Les Philistins se rassemblèrent et vinrent camper à Shounem. Saül rassembla tout Israël. Ils campèrent à Gelboé. 
${}^{5}Quand Saül vit le camp des Philistins, il fut effrayé ; son cœur se mit à battre violemment. 
${}^{6}Saül interrogea le Seigneur, mais le Seigneur ne lui répondit ni par les songes, ni par les sorts, ni par les prophètes. 
${}^{7}Alors Saül dit à ses serviteurs : « Cherchez-moi une femme experte en nécromancie ; j’irai chez elle pour la consulter. » Ses serviteurs lui dirent : « Il y a une nécromancienne à Enn-Dor. »
${}^{8}Saül se déguisa, mit d’autres vêtements et partit, accompagné de deux hommes. Ils arrivèrent, de nuit, chez la femme. Saül lui dit : « Interroge pour moi l’esprit des morts et fais monter pour moi celui que je te dirai. » 
${}^{9}La femme lui répondit : « Allons ! Tu sais toi-même ce que Saül a fait : il a supprimé du pays la nécromancie et la divination. Et toi, pourquoi veux-tu me tendre un piège qui me fera mourir ? » 
${}^{10}Mais Saül lui fit ce serment : « Par la vie du Seigneur, tu ne cours aucun risque dans cette affaire. » 
${}^{11}La femme lui dit : « Qui ferai-je monter pour toi du séjour des morts ? » Il répondit : « Fais monter pour moi Samuel. » 
${}^{12}La femme vit Samuel et poussa un grand cri. Elle dit à Saül : « Pourquoi m’as-tu trompée ? Tu es Saül ! » 
${}^{13}Le roi lui dit : « Ne crains pas. Mais que vois-tu ? » La femme dit à Saül : « Je vois comme un dieu montant de la terre. » 
${}^{14}Saül demanda : « Quelle est son apparence ? » Elle répondit : « Celui qui monte est un vieillard ; il est enveloppé d’un manteau. » Saül comprit alors que c’était Samuel. Il s’inclina, face contre terre, et se prosterna.
${}^{15}Samuel dit à Saül : « Pourquoi as-tu troublé mon repos en me faisant monter ? » Saül dit : « Je suis dans une grande angoisse. Les Philistins me font la guerre, et Dieu s’est écarté de moi. Il ne me répond plus, ni par l’intermédiaire des prophètes, ni par les songes. Aussi t’ai-je appelé pour que tu m’indiques ce que je dois faire. » 
${}^{16}Samuel dit : « Et pourquoi m’interroges-tu, alors que le Seigneur s’est écarté de toi et qu’il est devenu ton adversaire ? 
${}^{17}Le Seigneur a fait comme il l’avait dit par mon intermédiaire : le Seigneur t’a arraché la royauté et l’a donnée à David, à ton prochain. 
${}^{18}Puisque tu n’as pas obéi à la voix du Seigneur et que tu n’as pas traité Amalec selon l’ardeur de sa colère, eh bien, le Seigneur te traite aujourd’hui de cette manière. 
${}^{19}Et avec toi le Seigneur livrera aussi Israël aux mains des Philistins. Demain, toi et tes fils, vous me rejoindrez. Même l’armée d’Israël, le Seigneur la livrera aux mains des Philistins. »
${}^{20}Aussitôt, Saül s’effondra par terre, de toute sa hauteur, tant les paroles de Samuel l’avaient effrayé. Il était aussi sans force, n’ayant rien mangé de toute la journée ni de toute la nuit. 
${}^{21}La femme s’approcha de Saül et vit qu’il était tout bouleversé. Elle lui dit : « Tu le vois, ta servante t’a obéi : j’ai risqué ma vie et j’ai obéi aux ordres que tu m’as donnés. 
${}^{22}Mais maintenant, daigne écouter, toi aussi, la voix de ta servante : laisse-moi te servir un morceau de pain et mange ! Tu retrouveras des forces et tu pourras aller ton chemin. » 
${}^{23}Il refusa et dit : « Je ne mangerai pas. » Mais ses serviteurs insistèrent, ainsi que la femme. Alors, il leur obéit, se leva de terre et s’assit sur le divan. 
${}^{24}La femme avait chez elle un veau à l’engrais. Elle se hâta de l’abattre. Ensuite, elle prit de la farine, la pétrit et fit cuire des pains sans levain. 
${}^{25}Elle apporta le tout devant Saül et ses serviteurs. Ils mangèrent. Puis, s’étant levés, ils repartirent au cours de cette même nuit.
      
         
      \bchapter{}
      \begin{verse}
${}^{1}Les Philistins rassemblèrent toutes leurs armées à Apheq, tandis qu’Israël campait à la source qui est dans la plaine de Yizréel. 
${}^{2}Les princes des Philistins défilaient avec leurs troupes de centaines et de milliers ; David et ses hommes, avec Akish, défilaient en dernier. 
${}^{3}Les princes des Philistins demandèrent : « Ces Hébreux, qui sont-ils ? » Akish leur dit alors : « C’est David, le serviteur de Saül, roi d’Israël : il est avec moi depuis un an ou deux, et je n’ai pas trouvé la moindre chose à lui reprocher, du jour de son ralliement jusqu’à maintenant ! » 
${}^{4}Mais les princes des Philistins s’emportèrent contre lui. Ils lui dirent : « Renvoie cet homme ! Et qu’il retourne dans le lieu que tu lui as assigné ! Mais qu’il ne descende pas avec nous au combat : il se changerait en adversaire. Avec quoi celui-là pourrait-il regagner la faveur de son maître, sinon avec les têtes des hommes que voici ? 
${}^{5}N’est-il pas ce David pour qui l’on dansait en se renvoyant ce refrain :
        \\“Saül a tué ses milliers,
        \\et David, ses dizaines de milliers” ? »
       
${}^{6}Akish appela David et lui dit : « Par la vie du Seigneur, tu es un homme droit. Je prends plaisir à partir en campagne et à revenir avec toi, car depuis le jour où tu es venu chez moi jusqu’à maintenant, je n’ai rien trouvé de mauvais en toi. Mais tu déplais aux princes. 
${}^{7}Retourne donc et va en paix : ainsi tu ne feras rien qui soit mauvais aux yeux des princes des Philistins. » 
${}^{8}David dit à Akish : « Mais qu’ai-je donc fait ? Qu’as-tu trouvé en ton serviteur depuis le jour où je me suis présenté à toi jusqu’à maintenant ? Pourquoi ne puis-je aller combattre les ennemis de mon seigneur le roi ? » 
${}^{9}Akish répondit à David : « À mes yeux, tu es bon comme un ange de Dieu, je le sais. Seulement, les princes des Philistins ont dit : “Il ne montera pas avec nous au combat !” 
${}^{10}Alors, lève-toi de bon matin, toi et les serviteurs de ton maître qui sont venus avec toi. Oui, levez-vous de bon matin et, dès qu’il fera jour, partez ! »
${}^{11}David se leva tôt, lui et ses hommes, pour partir de bon matin et retourner au pays des Philistins. Quant aux Philistins, ils se dirigèrent vers la plaine de Yizréel.
      
         
      \bchapter{}
      \begin{verse}
${}^{1}Lorsque David et ses hommes arrivèrent à Ciqlag le troisième jour, des Amalécites venaient de faire une incursion dans le Néguev et à Ciqlag. Ils avaient dévasté Ciqlag et l’avaient incendiée. 
${}^{2}Ils avaient réduit en captivité les femmes qui s’y trouvaient, sans tuer personne, ni petit ni grand, et les avaient tous emmenés, en continuant leur chemin. 
${}^{3}David et ses hommes arrivèrent donc à la ville ; ils virent qu’elle était incendiée, et que leurs femmes, leurs fils et leurs filles étaient emmenés en captivité. 
${}^{4}Alors, David et les gens qui étaient avec lui poussèrent des cris et pleurèrent jusqu’à n’avoir plus la force de pleurer. 
${}^{5}Les deux femmes de David avaient été emmenées captives : Ahinoam de Yizréel, et Abigaïl, femme de Nabal de Carmel. 
${}^{6}David fut dans une grande angoisse car les gens parlaient de le lapider, tous étant remplis d’amertume à cause de leurs fils et de leurs filles. Mais David reprit courage par le Seigneur son Dieu.
${}^{7}David dit au prêtre Abiatar, fils d’Ahimélek : « Apporte-moi donc l’éphod. » Abiatar apporta l’éphod à David. 
${}^{8}David interrogea le Seigneur : « Si je poursuis cette bande de pillards, pourrai-je les atteindre ? » Et il lui dit : « Poursuis ! Sûrement, tu atteindras ! Sûrement, tu délivreras ! » 
${}^{9}David partit, ainsi que les six cents hommes qui étaient avec lui, et ils arrivèrent au torrent de Besor. Les autres étaient restés à Ciqlag. 
${}^{10}David continua la poursuite avec quatre cents hommes, tandis que deux cents restaient sur place, étant trop fatigués pour traverser le torrent de Besor.
${}^{11}On trouva dans la campagne un Égyptien. On l’amena auprès de David. Puis on lui donna du pain qu’il mangea, et on lui fit boire de l’eau. 
${}^{12}On lui donna encore du gâteau de figues et deux gâteaux de raisins secs. Après avoir mangé, il retrouva ses esprits. En effet, il n’avait rien mangé ni bu depuis trois jours et trois nuits. 
${}^{13}David lui demanda : « À qui appartiens-tu et d’où es-tu ? » Il répondit : « Je suis un jeune Égyptien, esclave d’un Amalécite. Mon maître m’a abandonné il y a trois jours, parce que j’étais malade. 
${}^{14}C’est nous qui avons fait une incursion au Néguev des Kerétiens, contre le Néguev de Juda et contre celui de Caleb, et nous avons incendié Ciqlag. » 
${}^{15}David lui dit : « Veux-tu me conduire jusqu’à cette bande ? » Il répondit : « Jure-moi par Dieu que tu ne me feras pas mourir et ne me livreras pas entre les mains de mon maître. Alors je te conduirai jusqu’à cette bande. » 
${}^{16}Il le conduisit donc. Voici que les Amalécites étaient éparpillés sur toute l’étendue du pays, mangeant, buvant, faisant la fête avec l’énorme butin qu’ils avaient pris au pays des Philistins et au pays de Juda. 
${}^{17}David les combattit depuis l’aube jusqu’au soir du lendemain. Aucun d’eux n’en réchappa, sinon quatre cents jeunes gens qui s’enfuirent à dos de chameau. 
${}^{18}David délivra tous ceux qu’Amalec avait pris ; il délivra ainsi ses deux femmes. 
${}^{19}Il ne manqua personne, ni petit ni grand, aucun fils ni aucune fille, ni la moindre chose du butin, de tout ce qui leur avait été pris. David ramena le tout. 
${}^{20}David prit tout le petit et le gros bétail. Ceux qui précédaient ce troupeau pour le conduire disaient : « Voilà le butin de David ! »
${}^{21}David arriva près des deux cents hommes, trop fatigués pour le suivre et qui étaient restés au torrent de Besor. Ils se portèrent à la rencontre de David et de sa troupe. David s’avança avec sa troupe et les salua. 
${}^{22}Mais parmi les hommes qui avaient accompagné David, ce furent tous les méchants et les vauriens qui prirent la parole et dirent : « Puisqu’ils ne sont pas venus avec nous, on ne leur donnera rien du butin que nous avons récupéré, si ce n’est à chacun sa femme et ses enfants. Qu’ils les emmènent et qu’ils s’en aillent ! » 
${}^{23}Mais David déclara : « Non, vous ne ferez pas cela, mes frères, avec ce que le Seigneur nous a donné. Il nous a gardés, il a livré entre nos mains la bande qui nous avait attaqués. 
${}^{24}Qui pourrait vous écouter sur ce point ? En effet, comme est la part de celui qui descend au combat, ainsi est la part de celui qui reste aux bagages : ils partageront entre eux. » 
${}^{25}À partir de ce jour, David en fit pour Israël une règle, un droit, qui vaut encore aujourd’hui.
${}^{26}Arrivé à Ciqlag, David envoya des parts de butin aux anciens de Juda, ses proches, avec ce message : « Pour vous, comme en bénédiction, voici une part du butin pris sur les ennemis du Seigneur. »
${}^{27}Il en envoya aussi à ceux de Béthel, à ceux de Ramoth-du-Néguev, à ceux de Yattir, 
${}^{28}à ceux d’Aroër, à ceux de Sifemoth, à ceux d’Eshtemoa, 
${}^{29}à ceux de Rakal, à ceux des villes des Yerahmëélites, à ceux des villes des Qénites, 
${}^{30}à ceux de Horma, à ceux de Bor-Ashane, à ceux d’Atak, 
${}^{31}à ceux d’Hébron et de tous les lieux où David et ses hommes étaient passés.
      
         
      \bchapter{}
      \begin{verse}
${}^{1}Les Philistins livraient bataille à Israël. Les hommes d’Israël s’enfuirent devant les Philistins et tombèrent, frappés à mort, sur le mont Gelboé. 
${}^{2}Les Philistins rattrapèrent Saül et ses fils. Ils frappèrent Jonathan, Abinadab et Malki-Shoua, les fils de Saül. 
${}^{3}Le poids du combat se porta vers Saül. Les tireurs d’arc le surprirent, et il fut gravement blessé par les tireurs. 
${}^{4}Saül dit à son écuyer : « Tire ton épée et transperce-moi, de peur que ces incirconcis ne viennent me transpercer et se jouer de moi. » Mais son écuyer refusa, tant il avait peur. Alors Saül prit son épée et se jeta sur elle. 
${}^{5}Quand l’écuyer vit que Saül était mort, il se jeta, lui aussi, sur son épée et mourut avec lui. 
${}^{6}Ainsi, ce jour-là, Saül, ses trois fils et son écuyer, avec tous ses hommes, moururent ensemble. 
${}^{7}Les hommes d’Israël qui se trouvaient de l’autre côté de la vallée, et ceux qui étaient de l’autre côté du Jourdain, virent que leurs troupes avaient pris la fuite et que Saül et ses fils étaient morts. Ils abandonnèrent leurs villes et s’enfuirent. Alors les Philistins vinrent s’y installer.
${}^{8}Le lendemain, les Philistins, venus pour dépouiller les morts, trouvèrent Saül et ses trois fils, gisant sur le mont Gelboé. 
${}^{9}Ils lui coupèrent la tête et le dépouillèrent de ses armes. Puis ils envoyèrent, à la ronde, dans le pays des Philistins, porter la bonne nouvelle dans la maison de leurs idoles et parmi le peuple. 
${}^{10}Ils déposèrent ses armes dans la maison des Astartés et clouèrent son corps sur le rempart de Beth-Shéane.
${}^{11}Les habitants de Yabesh-de-Galaad apprirent ce que les Philistins avaient fait à Saül. 
${}^{12}Alors tous les hommes de valeur se mirent en route et marchèrent toute la nuit. Ils enlevèrent du rempart de Beth-Shéane le corps de Saül et ceux de ses fils. Ils revinrent à Yabesh et y brûlèrent les corps. 
${}^{13}Ils prirent ensuite leurs ossements, les ensevelirent sous le tamaris de Yabesh et jeûnèrent pendant sept jours.
