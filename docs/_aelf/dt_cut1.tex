  
  
    
    \bbook{DEUTÉRONOME}{DEUTÉRONOME}
      
         
      \bchapter{}
      \begin{verse}
${}^{1}Voici les paroles que Moïse adressa à tout Israël dans le désert, au-delà du Jourdain, dans la Araba, en face de Souf – entre Parane, Tofel, Labane, Hacéroth et Di Zahab. 
${}^{2}Du mont Horeb à Cadès-Barnéa, il y a onze jours de marche par la route du mont Séïr. 
${}^{3}Or, en la quarantième année après la sortie d’Égypte, le premier jour du onzième mois, Moïse parla aux fils d’Israël : il leur rapporta tout ce que le Seigneur lui avait ordonné de transmettre. 
${}^{4}Il avait vaincu Séhone, roi des Amorites, qui résidait à Heshbone, et vaincu, à Édréï, Og, roi de Bashane, qui résidait à Ashtaroth. 
${}^{5}Puis, au-delà du Jourdain, au pays de Moab, Moïse commença à exposer la Loi que voici :
      
         
${}^{6}Le Seigneur notre Dieu nous a parlé à l’Horeb. Il nous a dit : « Vous êtes restés assez longtemps sur cette montagne. 
${}^{7}Repartez, pénétrez dans la montagne des Amorites et allez chez tous leurs voisins, dans la Araba, la Montagne, le Bas-Pays, le Néguev et au bord de la mer Méditerranée, dans le pays des Cananéens et au Liban, jusqu’au grand fleuve, l’Euphrate. 
${}^{8}Voyez : Je mets ce pays devant vos yeux. Allez donc prendre possession du pays que le Seigneur a juré à vos pères Abraham, Isaac et Jacob, de leur donner à eux ainsi qu’à leurs descendants. »
${}^{9}Moïse poursuivit : Je vous ai dit en ce temps-là : « Je ne puis à moi seul vous porter. 
${}^{10}Le Seigneur votre Dieu vous a déjà multipliés, et aujourd’hui vous voici nombreux comme les étoiles du ciel. 
${}^{11}Le Seigneur, le Dieu de vos pères, vous multipliera encore mille fois autant, il vous bénira comme il vous l’a dit. 
${}^{12}Comment porterais-je à moi seul le poids écrasant de vos conflits ? 
${}^{13}Choisissez, dans chacune de vos tribus, des hommes sages, intelligents et expérimentés, et j’en ferai vos chefs. » 
${}^{14}Alors vous m’avez répondu : « Ta proposition est excellente. » 
${}^{15}J’ai donc pris, parmi les chefs de vos tribus, des hommes sages et expérimentés, et je les ai établis sur vous comme chefs : officiers de millier, officiers de centaine, officiers de cinquantaine et officiers de dizaine, et aussi des scribes pour vos tribus.
${}^{16}J’ai donné ces ordres à vos juges en ce temps-là : « Vous entendrez les causes de vos frères et vous trancherez selon la justice les litiges entre eux, ou entre ton frère et l’immigré qui réside chez lui. 
${}^{17}Lorsque vous jugerez, vous n’agirez pas avec partialité : vous écouterez aussi bien le petit que le grand ; vous n’aurez peur de personne, car le jugement appartient à Dieu. Si l’affaire vous paraît trop difficile, vous me la soumettrez, et je l’entendrai. » 
${}^{18}Je vous ai prescrit en ce temps-là tout ce que vous aviez à faire.
${}^{19}Nous avons alors quitté l’Horeb et nous avons marché à travers le grand et terrible désert que vous avez vu, sur la route de la montagne des Amorites, comme le Seigneur notre Dieu nous l’avait ordonné. Ainsi, nous sommes arrivés à Cadès-Barnéa. 
${}^{20}Et je vous ai dit : « Vous êtes parvenus jusqu’à la montagne des Amorites, que le Seigneur notre Dieu nous donne. 
${}^{21}Vois ! Le Seigneur ton Dieu a mis ce pays devant tes yeux : monte en prendre possession, comme te l’a dit le Seigneur, le Dieu de tes pères. Ne crains pas ! Ne t’effraie pas ! » 
${}^{22}Alors, vous vous êtes tous approchés de moi et vous m’avez dit : « Envoyons donc des hommes en éclaireurs : qu’ils explorent pour nous le pays et nous renseignent sur la route par laquelle nous monterons et les villes où nous entrerons. » 
${}^{23}Cette proposition me parut excellente et j’ai pris, parmi vous, douze hommes, un par tribu. 
${}^{24}Ils ont pris la direction de la montagne et y sont montés. Arrivés aux gorges d’Eshkol, ils ont parcouru le pays en espionnant. 
${}^{25}Ils ont pris dans leurs mains des fruits du pays et sont descendus nous les apporter. Ils nous ont renseignés en disant : « C’est un bon pays, que le Seigneur notre Dieu nous donne ! »
${}^{26}Mais vous n’avez pas voulu y monter ; vous avez été rebelles à l’ordre du Seigneur votre Dieu, 
${}^{27}vous avez calomnié le Seigneur sous vos tentes, en disant : « C’est parce que le Seigneur nous hait qu’il nous a fait sortir du pays d’Égypte, c’est pour nous livrer entre les mains des Amorites et nous exterminer ! 
${}^{28}Où donc allons-nous monter ? Nos frères ont fait fondre notre courage en disant : “C’est un peuple plus nombreux et de plus haute taille que nous ; les villes sont grandes et fortifiées jusqu’aux cieux ; nous y avons même vu les fils des géants Anaqites !” »
${}^{29}Alors je vous ai dit : « N’ayez pas peur ! Ne les craignez pas ! 
${}^{30}Le Seigneur votre Dieu, qui marche devant vous, combattra lui-même pour vous, exactement comme il a agi avec vous en Égypte, sous vos yeux. 
${}^{31}Tu l’as vu aussi dans le désert : le Seigneur ton Dieu t’a porté, comme un homme porte son fils, tout au long de la route que vous avez parcourue jusqu’à votre arrivée en ce lieu. 
${}^{32}Et, malgré cela, vous n’avez pas mis votre foi dans le Seigneur votre Dieu, 
${}^{33}lui qui marchait devant vous sur la route pour chercher le lieu de vos campements : il était dans le feu durant la nuit pour éclairer vos pas sur le chemin, et dans la nuée durant le jour. »
${}^{34}Le Seigneur a entendu le bruit de vos paroles, il s’est irrité et a déclaré sous serment : 
${}^{35}« Pas un seul de ces hommes, personne de cette génération mauvaise ne verra le bon pays que j’ai juré de donner à vos pères, 
${}^{36}à l’exception de Caleb, fils de Yefounnè : lui, il le verra ; à lui je le donnerai, ainsi qu’à ses fils, ce pays qu’il a foulé, car il a suivi le Seigneur sans réserve. » 
${}^{37}Même contre moi, le Seigneur s’est mis en colère, à cause de vous. Il a dit : « Toi non plus, tu n’y entreras pas ! 
${}^{38}Mais Josué, fils de Noun, qui se tient à tes côtés, lui, y entrera. Affermis-le, car c’est lui qui conduira Israël dans son héritage. 
${}^{39}Et vos enfants dont vous disiez qu’ils seraient pris en butin, vos fils qui ne savent pas encore discerner le bien du mal, eux, ils y entreront. C’est à eux que je donnerai le pays, c’est eux qui en prendront possession. 
${}^{40}Quant à vous, faites demi-tour, partez au désert en direction de la mer des Roseaux. »
${}^{41}Alors, vous m’avez répondu : « Nous avons péché contre le Seigneur ! Eh bien, nous, montons et combattons, exactement comme le Seigneur notre Dieu nous l’avait ordonné. » Chacun s’est mis en tenue de combat et vous avez pensé qu’il était facile de prendre la direction de la montagne et d’y monter. 
${}^{42}Mais le Seigneur m’a déclaré : « Tu leur diras : Ne montez pas ! N’allez pas au combat, car je ne suis pas au milieu de vous. Ne vous mettez pas en condition d’être battus par vos ennemis ! » 
${}^{43}Je vous ai parlé, mais vous n’avez pas écouté, vous avez été rebelles à l’ordre du Seigneur, vous avez été présomptueux, vous avez pris la direction de la montagne et vous y êtes montés ! 
${}^{44}Mais les Amorites qui habitent cette montagne sont sortis à votre rencontre et, comme un essaim d’abeilles, ils vous ont poursuivis ; ils vous ont mis en pièces, de Séïr jusqu’à Horma. 
${}^{45}À votre retour, vous avez pleuré devant le Seigneur. Mais le Seigneur n’a pas écouté votre voix, il ne vous a pas prêté l’oreille. 
${}^{46}Alors vous êtes restés à Cadès pendant de longs jours, pendant autant de jours que vous y aviez séjourné auparavant.
      
         
      \bchapter{}
      \begin{verse}
${}^{1}Ensuite nous avons fait demi-tour et nous sommes partis vers le désert en direction de la mer des Roseaux, comme le Seigneur me l’avait dit. Il nous fallut bien des jours pour contourner la montagne de Séïr.
      
         
${}^{2}Puis le Seigneur m’a dit : 
${}^{3}« Vous tournez autour de cette montagne depuis assez longtemps. Dirigez-vous vers le nord ! 
${}^{4}Donne cet ordre au peuple : Vous allez traverser le territoire de vos frères, les fils d’Ésaü qui habitent en Séïr. Ils auront peur de vous, mais prenez bien garde ! 
${}^{5}Ne les attaquez pas, car je ne vous donnerai pas une parcelle de leur pays, pas même de quoi y poser le pied. En effet, c’est à Ésaü que j’ai donné en possession la montagne de Séïr. 
${}^{6}La nourriture que vous mangerez, vous la leur paierez à prix d’argent, et même l’eau que vous boirez, vous l’achèterez. 
${}^{7}Car le Seigneur ton Dieu a béni l’œuvre de tes mains ; il a veillé sur ta marche à travers ce grand désert : voilà quarante ans que le Seigneur ton Dieu est avec toi, et tu n’as jamais manqué de rien. »
${}^{8}Nous avons donc passé loin de nos frères, les fils d’Ésaü, qui habitent à Séïr, par la route de la Araba qui vient d’Eilath et d’Écione-Guéber ; puis nous avons tourné et pris la route en direction du désert de Moab. 
${}^{9}Et le Seigneur me dit : « N’attaque pas les Moabites, n’engage pas le combat contre eux : de ce pays je ne te donnerai rien en possession, car c’est aux fils de Loth que j’ai donné Ar en possession. » 
${}^{10}Les Eymites, autrefois, y habitaient ; c’était un peuple grand et nombreux, de haute taille comme les géants Anaqites ; 
${}^{11}on les considérait comme des Refaïtes, mais les Moabites les appelaient Eymites. 
${}^{12}En Séïr, autrefois, habitaient les Horites ; mais les fils d’Ésaü les avaient dépossédés, chassés, exterminés, pour y habiter à leur place. Israël a fait de même pour le pays qu’il possède, celui que le Seigneur lui a donné.
${}^{13}« Maintenant, debout ! Traversez le torrent de Zèred. » Nous avons donc traversé le torrent de Zèred. 
${}^{14}De Cadès-Barnéa au passage du torrent de Zèred, notre marche a duré trente-huit ans, jusqu’à ce que disparaisse toute la génération des hommes de guerre, ainsi que le Seigneur l’avait juré ; 
${}^{15}et même la main du Seigneur s’était levée contre eux pour les rayer du camp jusqu’à leur totale disparition.
${}^{16}Et lorsque la mort eut fait disparaître du peuple tous ces hommes de guerre, 
${}^{17}le Seigneur me parla ainsi : 
${}^{18}« Aujourd’hui, tu vas traverser Ar, territoire de Moab, 
${}^{19}et tu te trouveras face aux fils d’Ammone. Ne les attaque pas, n’engage pas le combat contre eux : du pays des fils d’Ammone, je ne te donnerai rien en possession, car c’est aux fils de Loth que je l’ai donné en possession. »
${}^{20}On le considérait également comme un pays de Refaïtes. Les Refaïtes y avaient habité autrefois, et les Ammonites les appelaient Zamzoummites. 
${}^{21}C’était un peuple grand et nombreux, de haute taille comme les géants Anaqites. Mais le Seigneur les avait exterminés devant les Ammonites, et ceux-ci les avaient dépossédés pour y habiter à leur place. 
${}^{22}Le Seigneur avait agi de même pour les fils d’Ésaü qui habitent en Séïr ; devant eux, il avait exterminé les Horites. Les fils d’Ésaü les avaient dépossédés de Séïr et, jusqu’à ce jour, ils y habitent à leur place. 
${}^{23}Il en fut de même pour les Avvites qui habitaient les villages jusqu’à Gaza : les Kaftorites venus de Kaftor les avaient exterminés pour habiter à leur place.
${}^{24}« Debout ! Levez le camp et passez le torrent de l’Arnon ! Vois ! J’ai livré entre tes mains Séhone l’Amorite, roi de Heshbone, ainsi que son pays. Commence à prendre possession de ton héritage. Engage le combat. 
${}^{25}En ce jour, je commence à répandre la terreur et la crainte de toi à la face des peuples, sous tous les cieux ; au bruit de ton approche, ils trembleront et frémiront devant toi. »
${}^{26}Alors Moïse poursuivit : Du désert de Qedémoth, j’ai envoyé des messagers à Séhone, roi de Heshbone, avec ces paroles de paix : 
${}^{27}« Je voudrais traverser ton pays : je resterai sur la route, je ne dévierai ni à droite ni à gauche. 
${}^{28}La nourriture que je mangerai, tu me la vendras à prix d’argent, et l’eau que je boirai, tu me la fourniras aussi à prix d’argent. Je voudrais simplement passer à pied, 
${}^{29}ainsi que me l’ont permis les fils d’Ésaü qui habitent en Séïr et les Moabites qui vivent dans le pays d’Ar. Je pourrai donc traverser le Jourdain et arriver au pays que nous donne le Seigneur notre Dieu. »
${}^{30}Mais Séhone, roi de Heshbone, n’a pas voulu nous laisser traverser son pays ; en effet, le Seigneur ton Dieu avait rendu son esprit inflexible et avait endurci son cœur, pour le livrer entre tes mains ; il en est encore ainsi aujourd’hui. 
${}^{31}Et le Seigneur m’a dit : « Vois, j’ai commencé à te livrer Séhone et son pays ; toi, commence à prendre possession de son pays. » 
${}^{32}Séhone sortit alors à notre rencontre avec tout son peuple pour combattre à Yahça. 
${}^{33}Le Seigneur notre Dieu livra Séhone entre nos mains, et nous l’avons vaincu, ainsi que ses fils et tout son peuple.
${}^{34}Nous avons conquis toutes ses villes en ce temps-là, et voué chaque ville à l’anathème : hommes, femmes et enfants ; nous n’avons laissé aucun survivant. 
${}^{35}Nous avons gardé pour seul butin le bétail et les dépouilles prises dans les villes que nous avions conquises. 
${}^{36}Depuis Aroër qui est sur le bord du torrent de l’Arnon, et la ville qui est au fond de la vallée, pas une ville jusqu’à Galaad ne nous fut inaccessible : le Seigneur notre Dieu livra tout entre nos mains. 
${}^{37}Mais tu ne t’es pas approché du pays des fils d’Ammone, tout le long du torrent du Yabboq, des villes de la montagne, d’aucun des lieux que le Seigneur notre Dieu avait interdits.
      
         
      \bchapter{}
      \begin{verse}
${}^{1}Nous avons obliqué pour monter en direction du Bashane, mais Og, roi de Bashane, est sorti à notre rencontre, lui et tout son peuple, pour combattre à Édréï.
${}^{2}Le Seigneur m’a dit : « Ne le crains pas, car je l’ai livré entre tes mains, lui, son peuple et son pays : tu le traiteras comme tu as traité Séhone, roi des Amorites, qui habitait Heshbone. » 
${}^{3}Et le Seigneur notre Dieu a aussi livré entre nos mains Og, roi de Bashane, et tout son peuple. Nous les avons battus sans leur laisser aucun survivant. 
${}^{4}Nous avons conquis toutes leurs villes en ce temps-là. Pas une cité qui ne leur fut prise ; il y en avait soixante dans toute la région d’Argob située dans le Bashane où régnait Og. 
${}^{5}Toutes ces villes étaient fortifiées avec de hautes murailles et des doubles portes verrouillées. Il y avait en outre un très grand nombre de villages. 
${}^{6}Nous les avons vouées à l’anathème, comme nous l’avions fait pour les villes de Séhone, roi de Heshbone ; nous avons voué chaque ville à l’anathème : hommes, femmes et enfants. 
${}^{7}Et nous avons gardé comme butin tout le bétail et les dépouilles prises dans ces villes.
${}^{8}En ce temps-là, nous avons donc pris aux deux rois des Amorites ce pays qui se trouve au-delà du Jourdain, depuis le torrent de l’Arnon jusqu’au mont Hermon, 
${}^{9}que les gens de Sidon appelaient Siryone, et les Amorites, Senir. 
${}^{10}Nous avons pris toutes les villes du Plateau, tout le Galaad et tout le Bashane jusqu’à Salka et Édréï, villes du Bashane où régnait Og. 
${}^{11}Og, roi de Bashane, était le seul survivant des géants Refaïtes. Son lit était un lit de fer – ne serait-ce pas celui qu’on voit à Rabba des Ammonites ? – un lit qui fait bien neuf coudées de long et quatre coudées de large.
${}^{12}C’est de ce pays que nous avons pris possession en ce temps-là. Le territoire depuis Aroër sur le torrent de l’Arnon, et la moitié des monts de Galaad avec ses villes, je les ai donnés aux gens de Roubène et de Gad. 
${}^{13}Le reste du Galaad et tout le Bashane, royaume d’Og, je les ai donnés à la moitié de la tribu de Manassé : toute la région de l’Argob et tout le Bashane qu’on appelle le pays des Refaïtes. 
${}^{14}Yaïr, fils de Manassé, prit toute la région d’Argob jusqu’à la frontière des Gueshourites et des Maakatites. Il a donné son nom à ces contrées du Bashane qu’on appelle encore aujourd’hui « les campements de Yaïr ». 
${}^{15}C’est à Makir que j’ai donné le Galaad. 
${}^{16}Aux gens de Roubène et de Gad j’ai donné le territoire qui s’étend du Galaad jusqu’au torrent de l’Arnon – le milieu du torrent servant de frontière – et jusqu’au torrent du Yabboq, frontière des Ammonites ; 
${}^{17}je leur ai donné la Araba, le Jourdain servant de frontière, de Kinnèreth à la mer de la Araba, la mer Morte, au bas des pentes du Pisga, à l’est.
${}^{18}Voici ce que je vous ai ordonné en ce temps-là : « C’est le Seigneur votre Dieu qui vous a donné ce pays pour que vous le possédiez en héritage. Vous y passerez devant vos frères, les fils d’Israël, vous, les guerriers, en tenue de combat ; 
${}^{19}seuls vos femmes, vos enfants et votre cheptel qui, je le sais, sera nombreux, habiteront dans les villes que je vous ai données ; 
${}^{20}cela, jusqu’au jour où le Seigneur accordera le repos à vos frères comme à vous-mêmes ; alors ils posséderont, eux aussi, le pays que leur donne le Seigneur votre Dieu au-delà du Jourdain. Vous retournerez chacun dans le territoire que je vous ai donné en héritage. »
${}^{21}À Josué, en ce temps-là, j’ai ordonné : « Tu vois de tes propres yeux tout ce que le Seigneur votre Dieu a fait à ces deux rois. Le Seigneur traitera de même tous les royaumes que tu vas traverser. 
${}^{22}N’ayez pas peur, car c’est le Seigneur votre Dieu qui combat pour vous ! »
${}^{23}J’ai imploré le Seigneur en ce temps-là : 
${}^{24}« Seigneur mon Dieu, c’est toi qui as commencé à montrer à ton serviteur ta grandeur et la force de ta main. Y a-t-il au ciel et sur la terre un dieu dont les œuvres et la puissance égalent les tiennes ? 
${}^{25}Permets-moi, je t’en prie, de passer et de voir le bon pays d’au-delà du Jourdain, cette belle montagne et le Liban ! »
${}^{26}Mais, à cause de vous, le Seigneur s’est mis en colère contre moi, et il ne m’a pas écouté. Il m’a dit : « Assez ! Ne recommence pas à me parler de cette affaire ! 
${}^{27}Monte au sommet du Pisga, porte ton regard vers l’ouest, le nord, le sud et l’est ; regarde de tous tes yeux, car le Jourdain que voici, tu ne le passeras pas ! 
${}^{28}Donne tes ordres à Josué, fortifie-le et affermis-le, car c’est lui qui passera le Jourdain à la tête de ce peuple, c’est lui qui les mettra en possession de leur héritage, ce pays que tu vois. »
${}^{29}Alors nous sommes restés dans la vallée, en face de Beth-Péor.
      
         
      \bchapter{}
      \begin{verse}
${}^{1}Maintenant, Israël, écoute les décrets et les ordonnances que je vous enseigne pour que vous les mettiez en pratique. Ainsi vous vivrez, vous entrerez, pour en prendre possession, dans le pays que vous donne le Seigneur, le Dieu de vos pères. 
${}^{2} Vous n’ajouterez rien à ce que je vous ordonne, et vous n’y enlèverez rien, mais vous garderez les commandements du Seigneur votre Dieu tels que je vous les prescris.
${}^{3}Vous voyez de vos propres yeux ce que le Seigneur a fait à Baal-Péor : tous les hommes qui avaient suivi le Baal de Péor, le Seigneur ton Dieu les a retranchés de toi et les a exterminés. 
${}^{4}Mais vous qui êtes restés attachés au Seigneur votre Dieu, vous êtes tous vivants aujourd’hui !
${}^{5}Voyez, je vous enseigne les décrets et les ordonnances que le Seigneur mon Dieu m’a donnés pour vous, afin que vous les mettiez en pratique dans le pays où vous allez entrer pour en prendre possession. 
${}^{6} Vous les garderez, vous les mettrez en pratique ; ils seront votre sagesse et votre intelligence aux yeux de tous les peuples. Quand ceux-ci entendront parler de tous ces décrets, ils s’écrieront : « Il n’y a pas un peuple sage et intelligent comme cette grande nation ! » 
${}^{7} Quelle est en effet la grande nation dont les dieux soient aussi proches que le Seigneur notre Dieu est proche de nous chaque fois que nous l’invoquons ? 
${}^{8} Et quelle est la grande nation dont les décrets et les ordonnances soient aussi justes que toute cette Loi que je vous donne aujourd’hui ? 
${}^{9} Mais prends garde à toi : garde-toi de jamais oublier ce que tes yeux ont vu ; ne le laisse pas sortir de ton cœur un seul jour. Enseigne-le à tes fils, et aux fils de tes fils.
${}^{10}Le jour où tu étais debout en présence du Seigneur ton Dieu au mont Horeb, ce jour-là le Seigneur m’avait dit : « Rassemble le peuple auprès de moi, je leur ferai entendre mes paroles pour que, tout au long de leur vie sur la terre, ils apprennent à me craindre et qu’ils l’apprennent aussi à leurs fils. » 
${}^{11}Vous vous êtes donc approchés et tenus debout, au pied de la montagne. Et la montagne était en feu, embrasée jusqu’en plein ciel, parmi les ténèbres des nuages et de la nuée obscure. 
${}^{12}Alors, le Seigneur vous a parlé du milieu du feu ; le son de ses paroles, vous l’entendiez, mais vous n’avez vu aucune forme ; rien qu’une voix ! 
${}^{13}Il vous a révélé son Alliance, les Dix Paroles qu’il vous a ordonné de mettre en pratique. Et il les a écrites sur deux tables de pierre. 
${}^{14}Et à moi, en ce temps-là, le Seigneur ordonna de vous enseigner les décrets et les ordonnances, afin que vous les mettiez en pratique dans le pays où vous allez passer pour en prendre possession.
${}^{15}Prenez bien garde à vous-mêmes : vous n’avez vu aucune forme le jour où le Seigneur vous a parlé à l’Horeb du milieu du feu. 
${}^{16}N’allez pas vous corrompre en vous fabriquant une idole, une statue de quelque forme que ce soit, représentant homme ou femme, 
${}^{17}bête qui marche sur la terre, oiseau qui vole dans le ciel, 
${}^{18}bestiole qui rampe sur le sol, poisson qui vit dans les eaux sous la terre.
${}^{19}Prends garde lorsque tu lèves les yeux vers le ciel et que tu vois le soleil, la lune et les étoiles, toute l’armée des cieux ! Ne te laisse pas égarer, ne te prosterne pas devant eux pour les servir. Ceux-là, le Seigneur les a donnés en partage à tous les peuples qui sont sous le ciel. 
${}^{20}Tandis que vous, le Seigneur vous a pris et fait sortir de l’Égypte, cette fournaise à fondre le fer. Il vous a pris pour que vous deveniez son peuple, son héritage, comme vous l’êtes encore aujourd’hui.
${}^{21}Mais le Seigneur s’est irrité contre moi à cause de vous et il a juré que je ne passerais pas le Jourdain, que je n’entrerais pas dans le bon pays que le Seigneur te donne en héritage. 
${}^{22}Oui, je vais mourir ici, sur cette terre, je ne passerai pas le Jourdain, tandis que vous, vous le passerez et vous posséderez ce bon pays-là ! 
${}^{23}Gardez-vous d’oublier l’Alliance que le Seigneur votre Dieu a conclue avec vous : ne vous faites pas une idole, pas une image de quoi que ce soit ; c’est le commandement du Seigneur ton Dieu ! 
${}^{24}Car le Seigneur ton Dieu est un feu dévorant, c’est un Dieu jaloux !
${}^{25}Quand tu auras engendré des fils et des petits-fils, quand vous aurez vieilli dans ce pays, alors, si vous vous corrompez en fabriquant une idole, une image de quoi que ce soit, si vous faites ce qui est mal aux yeux du Seigneur ton Dieu, au point de l’indigner, 
${}^{26}alors – j’en prends aujourd’hui à témoin contre vous les cieux et la terre – aussitôt vous disparaîtrez totalement de la surface du pays dont vous allez prendre possession en passant le Jourdain ; vous n’y prolongerez pas vos jours, car vous serez totalement détruits ! 
${}^{27}Le Seigneur vous dispersera parmi les peuples ; il ne restera de vous qu’un petit nombre parmi les nations où le Seigneur vous aura conduits. 
${}^{28}Là-bas, vous servirez des dieux, ouvrages de mains humaines, en bois et en pierre, qui ne voient pas, n’entendent pas, ne mangent pas, ne sentent pas.
${}^{29}Alors, de là-bas, tu rechercheras le Seigneur ton Dieu, et tu le trouveras si tu le cherches de tout ton cœur et de toute ton âme. 
${}^{30}Quand tu seras dans la détresse, quand tout cela te sera arrivé, dans la suite des temps, tu reviendras alors jusqu’au Seigneur ton Dieu et tu écouteras sa voix. 
${}^{31}Car le Seigneur ton Dieu est un Dieu miséricordieux : il ne t’abandonnera pas, il ne te détruira pas, il n’oubliera pas l’Alliance jurée à tes pères.
${}^{32}Interroge donc les temps anciens qui t’ont précédé, depuis le jour où Dieu créa l’homme sur la terre : d’un bout du monde à l’autre\\, est-il arrivé quelque chose d’aussi grand, a-t-on jamais connu rien de pareil ? 
${}^{33}Est-il un peuple qui ait entendu comme toi la voix de Dieu parlant du milieu du feu, et qui soit resté en vie ? 
${}^{34}Est-il un dieu qui ait entrepris de se choisir une nation, de venir la prendre au milieu d’une autre, à travers des épreuves, des signes, des prodiges et des combats, à main forte et à bras étendu, et par des exploits terrifiants – comme tu as vu le Seigneur ton Dieu le faire pour toi en Égypte ?
${}^{35}Il t’a été donné de voir tout cela pour que tu saches que c’est le Seigneur qui est Dieu, il n’y en a pas d’autre\\. 
${}^{36} Du haut du ciel, il t’a fait entendre sa voix pour t’instruire ; sur la terre, il t’a fait voir son feu impressionnant, et tu as entendu ce qu’il te disait du milieu du feu. 
${}^{37} Parce qu’il a aimé tes pères et qu’il a choisi leur descendance, en personne il t’a fait sortir d’Égypte par sa grande force, 
${}^{38} pour chasser devant toi des nations plus grandes et plus puissantes, te faire entrer dans leur pays et te le donner en héritage, comme cela se réalise aujourd’hui.
${}^{39}Sache donc aujourd’hui, et médite cela en ton cœur\\ : c’est le Seigneur qui est Dieu, là-haut dans le ciel comme ici-bas sur la terre ; il n’y en a pas d’autre. 
${}^{40} Tu garderas les décrets et les commandements du Seigneur que je te donne aujourd’hui, afin d’avoir, toi et tes fils, bonheur et longue vie sur la terre que te donne le Seigneur ton Dieu, tous les jours.
${}^{41}Alors, Moïse réserva trois villes au-delà du Jourdain, à l’orient, 
${}^{42}pour servir de refuge au meurtrier qui aurait tué involontairement son prochain, sans jamais avoir eu de haine contre lui. En se réfugiant dans l’une de ces villes, le meurtrier aura la vie sauve. 
${}^{43}Ces villes étaient Bècèr au désert dans le pays du Plateau pour les gens de Roubène, Ramoth-de-Galaad pour les gens de Gad, et Golane en Bashane pour ceux de Manassé.
${}^{44}Voici la Loi que Moïse exposa devant les fils d’Israël. 
${}^{45}Ce sont les édits, les décrets et les ordonnances proclamés par Moïse aux fils d’Israël, à leur sortie d’Égypte. 
${}^{46}Cela se passait au-delà du Jourdain, dans la vallée en face de Beth-Péor, au pays de Séhone, roi des Amorites, qui habitait à Heshbone. Moïse et les fils d’Israël l’avaient vaincu à leur sortie d’Égypte et 
${}^{47}ils avaient pris possession de son pays ainsi que du pays d’Og, roi de Bashane. Séhone et Og étaient les deux rois des Amorites qui habitaient au-delà du Jourdain, au soleil levant. 
${}^{48}Leurs territoires s’étendaient d’Aroër, au bord des gorges de l’Arnon, jusqu’à la montagne de Sihone, c’est-à-dire l’Hermon, 
${}^{49}et comprenaient aussi toute la Araba au-delà du Jourdain, à l’est, jusqu’à la mer de la Araba, au bas des pentes du Pisga.
      
         
      \bchapter{}
      \begin{verse}
${}^{1}Moïse convoqua tout Israël et leur dit :
      \begin{verse}Écoute, Israël, les décrets et les ordonnances que je proclame à vos oreilles aujourd’hui. Vous les apprendrez et vous veillerez à les mettre en pratique. 
${}^{2}Le Seigneur notre Dieu a conclu une alliance avec nous à l’Horeb : 
${}^{3}ce n’est pas avec nos pères que le Seigneur a conclu cette alliance, mais bien avec nous, nous-mêmes qui sommes ici aujourd’hui, tous vivants. 
${}^{4}C’est face à face que le Seigneur a parlé avec vous sur la montagne, du milieu du feu. 
${}^{5}Moi, je me tenais entre le Seigneur et vous en ce temps-là, pour vous transmettre la parole du Seigneur, car vous aviez peur du feu et vous n’étiez pas montés sur la montagne. Le Seigneur a dit :
${}^{6}« Je suis le Seigneur ton Dieu, qui t’ai fait sortir du pays d’Égypte, de la maison d’esclavage.
${}^{7}Tu n’auras pas d’autres dieux que moi.
${}^{8}Tu ne feras aucune idole, aucune image de ce qui est là-haut dans les cieux, ou en bas sur la terre, ou dans les eaux par-dessous la terre. 
${}^{9} Tu ne te prosterneras pas devant ces images pour leur rendre un culte. Car moi, le Seigneur ton Dieu, je suis un Dieu jaloux : chez ceux qui me haïssent, je punis la faute des pères sur les fils, jusqu’à la troisième et la quatrième génération ; 
${}^{10} mais ceux qui m’aiment et observent mes commandements, je leur montre ma fidélité jusqu’à la millième génération.
${}^{11}Tu n’invoqueras pas le nom du Seigneur ton Dieu pour le mal, car le Seigneur ne laissera pas impuni celui qui invoque son nom pour le mal.
${}^{12}Observe le jour du sabbat, en le sanctifiant, selon l’ordre du Seigneur ton Dieu. 
${}^{13} Pendant six jours tu travailleras et tu feras tout ton ouvrage, 
${}^{14} mais le septième jour est le jour du repos, sabbat en l’honneur du Seigneur ton Dieu. Tu ne feras aucun ouvrage, ni toi, ni ton fils, ni ta fille, ni ton serviteur, ni ta servante, ni ton bœuf, ni ton âne, ni aucune de tes bêtes, ni l’immigré qui réside dans ta ville. Ainsi, comme toi-même, ton serviteur et ta servante se reposeront. 
${}^{15} Tu te souviendras que tu as été esclave au pays d’Égypte, et que le Seigneur ton Dieu t’en a fait sortir à main forte et à bras étendu. C’est pourquoi le Seigneur ton Dieu t’a ordonné de célébrer le jour du sabbat.
${}^{16}Honore ton père et ta mère, comme te l’a ordonné le Seigneur ton Dieu, afin d’avoir longue vie et bonheur sur la terre que te donne le Seigneur ton Dieu.
${}^{17}Tu ne commettras pas de meurtre.
${}^{18}Tu ne commettras pas d’adultère.
${}^{19}Tu ne commettras pas de vol.
${}^{20}Tu ne porteras pas de faux témoignage contre ton prochain.
${}^{21}Tu ne convoiteras pas la femme de ton prochain, tu ne désireras ni sa maison ni son champ, ni son serviteur ni sa servante, ni son bœuf ou son âne : rien de ce qui lui appartient. »
${}^{22}Ces paroles, le Seigneur les a dites à toute l’assemblée de son peuple sur la montagne, du milieu du feu, des nuages et de la nuée obscure ; il les a dites d’une voix puissante et n’a rien ajouté. Ensuite il les a écrites sur deux tables de pierre, qu’il m’a données.
${}^{23}Or, quand vous avez entendu la voix sortant des ténèbres, tandis que la montagne était embrasée par le feu, vous vous êtes approchés de moi, vous, tous les chefs de tribus, et vous, les anciens, 
${}^{24}et vous m’avez dit : « Voici que le Seigneur notre Dieu nous a montré sa gloire et sa grandeur ; nous avons entendu sa voix du milieu du feu ; aujourd’hui, nous avons vu que Dieu peut parler à l’homme et lui laisser la vie. 
${}^{25}Et maintenant, pourquoi mourir, dévorés par ce grand feu ? Si nous continuons à entendre la voix du Seigneur notre Dieu, nous allons mourir ! 
${}^{26}Est-il jamais arrivé à un être de chair d’entendre, comme nous, le Dieu vivant parler du milieu du feu et, malgré tout, de rester en vie ? 
${}^{27}Toi, Moïse, approche donc pour écouter tout ce que dira le Seigneur notre Dieu : tu nous répéteras toutes les paroles du Seigneur notre Dieu ; nous les écouterons et nous les mettrons en pratique. »
${}^{28}Le Seigneur a entendu le son de votre voix lorsque vous me parliez. Il m’a dit : « J’ai entendu résonner la voix de ce peuple lorsqu’il te parlait : ils ont bien fait de dire tout cela ! 
${}^{29}Si seulement ils avaient à cœur de me craindre et de garder mes commandements, chaque jour, pour leur bonheur et celui de leurs fils, à jamais ! 
${}^{30}Va leur dire : “Retournez à vos tentes !” 
${}^{31}Mais toi, reste ici près de moi : je vais te dire tous les commandements, les décrets et les ordonnances que tu leur enseigneras pour qu’ils les mettent en pratique dans le pays que je leur donne en possession. »
${}^{32}Vous veillerez à agir comme vous l’a ordonné le Seigneur votre Dieu, sans dévier ni à droite ni à gauche. 
${}^{33}En tout, vous suivrez le chemin que le Seigneur votre Dieu vous a tracé : alors vous vivrez, vous aurez bonheur et longue vie dans le pays dont vous allez prendre possession.
      
         
      \bchapter{}
      \begin{verse}
${}^{1}Voici le commandement\\, les décrets et les ordonnances que le Seigneur votre Dieu m’a prescrit de vous enseigner, pour que vous les mettiez en pratique dans le pays dont vous allez prendre possession quand vous aurez passé le Jourdain. 
${}^{2} Tu craindras le Seigneur ton Dieu. Tous les jours de ta vie, toi, ainsi que ton fils et le fils de ton fils, tu observeras tous ses décrets et ses commandements, que je te prescris aujourd’hui, et tu auras longue vie. 
${}^{3} Israël, tu écouteras, tu veilleras à mettre en pratique ce qui t’apportera bonheur et fécondité, dans un pays ruisselant de lait et de miel\\, comme te l’a dit le Seigneur, le Dieu de tes pères.
${}^{4}Écoute, Israël : le Seigneur notre Dieu est l’Unique. 
${}^{5} Tu aimeras le Seigneur ton Dieu de tout ton cœur, de toute ton âme et de toute ta force. 
${}^{6} Ces paroles que je te donne aujourd’hui resteront dans ton cœur. 
${}^{7} Tu les rediras à tes fils, tu les répéteras sans cesse, à la maison ou en voyage, que tu sois couché ou que tu sois levé ; 
${}^{8} tu les attacheras à ton poignet comme un signe, elles seront un bandeau\\sur ton front\\, 
${}^{9} tu les inscriras à l’entrée\\de ta maison et aux portes de ta ville\\.
${}^{10}Quand le Seigneur ton Dieu te fera entrer dans le pays qu’il a juré à tes pères, Abraham, Isaac et Jacob, de te donner ; quand tu auras des villes grandes et belles que tu n’as pas bâties, 
${}^{11} des maisons pleines de richesses que tu n’y as pas entassées, des citernes que tu n’as pas creusées, des vignes et des oliveraies que tu n’as pas plantées ; quand tu auras bien mangé et te seras rassasié : 
${}^{12} alors garde-toi d’oublier le Seigneur, lui qui t’a fait sortir d’Égypte, de la maison d’esclavage. 
${}^{13} Tu craindras le Seigneur ton Dieu, tu le serviras, c’est par son nom que tu prêteras serment.
${}^{14}Vous ne suivrez pas d’autres dieux, ces dieux des nations qui vous entourent, 
${}^{15}car le Seigneur ton Dieu est un Dieu jaloux au milieu de toi. La colère du Seigneur ton Dieu s’enflammerait contre toi et il te ferait disparaître de la face de la terre. 
${}^{16}Vous ne mettrez pas le Seigneur votre Dieu à l’épreuve, comme vous l’avez fait à Massa. 
${}^{17}Vous observerez avec soin les commandements du Seigneur votre Dieu, les édits et les décrets qu’il t’a prescrits. 
${}^{18}Tu feras ce qui est droit et bon aux yeux du Seigneur, afin d’être heureux, et d’entrer, pour en prendre possession, dans le bon pays que le Seigneur a juré de donner à tes pères, 
${}^{19}en chassant tous tes ennemis devant toi, comme l’a dit le Seigneur.
${}^{20}Et demain, quand ton fils te demandera : « Quels sont donc ces édits, ces décrets et ces ordonnances que le Seigneur notre Dieu vous a prescrits ? », 
${}^{21}alors tu diras à ton fils : « Nous étions esclaves de Pharaon, en Égypte, et le Seigneur nous a fait sortir d’Égypte par la force de sa main. 
${}^{22}Sous nos yeux, le Seigneur a accompli des signes et des prodiges grands et funestes contre l’Égypte, Pharaon et toute sa maison. 
${}^{23}Mais nous, il nous a fait sortir de là pour nous faire entrer dans le pays qu’il voulait nous donner, celui qu’il avait promis par serment à nos pères. 
${}^{24}Alors le Seigneur nous a commandé de mettre en pratique tous ces décrets, pour que nous craignions le Seigneur notre Dieu : ainsi, nous serons toujours heureux et il nous gardera en vie comme nous le sommes aujourd’hui. 
${}^{25}Et nous serons justes si nous veillons à mettre en pratique tous ces commandements, en présence du Seigneur notre Dieu, comme il nous l’a prescrit. »
      
         
      \bchapter{}
      \begin{verse}
${}^{1}Quand le Seigneur ton Dieu te fera entrer dans le pays dont tu vas prendre possession, il expulsera devant toi des nations nombreuses, le Hittite, le Guirgashite, l’Amorite, le Cananéen, le Perizzite, le Hivvite et le Jébuséen, sept nations plus nombreuses et plus puissantes que toi. 
${}^{2}Le Seigneur ton Dieu te les livrera et tu les battras. Alors tu les voueras à l’anathème. Tu ne concluras pas d’alliance avec elles. Tu ne leur feras pas grâce. 
${}^{3}Tu ne contracteras pas de mariage avec elles, tu ne donneras pas ta fille à leur fils, tu ne prendras pas leur fille pour la donner à ton fils, 
${}^{4}car cela détournerait ton fils de me suivre ; ils serviraient d’autres dieux. La colère du Seigneur s’enflammerait contre vous, et il aurait vite fait de t’exterminer. 
${}^{5}Voici ce que vous leur ferez : vous démolirez leurs autels, vous briserez leurs stèles, vous abattrez leurs poteaux sacrés et vous brûlerez leurs idoles. 
${}^{6}Car tu es un peuple consacré au Seigneur ton Dieu : c’est toi qu’il a choisi pour être son peuple, son domaine particulier parmi tous les peuples de la terre.
      
         
${}^{7}Si le Seigneur s’est attaché à vous, s’il vous a choisis, ce n’est pas que vous soyez le plus nombreux de tous les peuples, car vous êtes le plus petit de tous. 
${}^{8} C’est par amour pour vous, et pour tenir le serment fait à vos pères, que le Seigneur vous a fait sortir par la force de sa main, et vous a rachetés\\de la maison d’esclavage et de la main de Pharaon, roi d’Égypte.
${}^{9}Tu sauras donc que c’est le Seigneur ton Dieu qui est Dieu, le Dieu vrai qui garde son Alliance et sa fidélité\\pour mille générations à ceux qui l’aiment et gardent ses commandements. 
${}^{10} Mais il riposte à ses adversaires en les faisant périr, et sa riposte est immédiate. 
${}^{11} Tu garderas donc le commandement, les décrets et les ordonnances que je te prescris aujourd’hui de mettre en pratique.
${}^{12}Et parce que vous aurez écouté ces ordonnances, que vous les aurez gardées et mises en pratique, le Seigneur ton Dieu te gardera l’Alliance et la fidélité qu’il a jurées à tes pères. 
${}^{13}Il t’aimera, il te bénira, il te multipliera, il bénira le fruit de ton sein et le fruit de ton sol, ton froment, ton vin nouveau, ton huile fraîche, la portée de tes vaches et de tes brebis, sur la terre qu’il a juré à tes pères de te donner. 
${}^{14}Béni seras-tu plus que tous les peuples ! Pas de stérilité chez toi, ni pour les hommes ni pour les femmes, ni pour le bétail. 
${}^{15}Le Seigneur détournera de toi toute maladie ; ces funestes épidémies que tu as connues en Égypte, il ne te les infligera pas, mais il en frappera tous ceux qui te haïssent. 
${}^{16}Tu dévoreras tous ces peuples que le Seigneur ton Dieu te livre ; ton œil sera sans pitié pour eux, et tu ne serviras pas leurs dieux, car ce serait pour toi un piège.
${}^{17}Peut-être diras-tu en ton cœur : « Ces nations sont plus nombreuses que moi, comment pourrais-je les déposséder ? » 
${}^{18}Ne les crains pas ! Rappelle-toi donc ce que le Seigneur ton Dieu a fait à Pharaon et à toute l’Égypte, 
${}^{19}les grandes épreuves que tes yeux ont vues, les signes et les prodiges, la main forte et le bras étendu par lesquels le Seigneur ton Dieu t’a fait sortir ! Le Seigneur ton Dieu en fera autant à tous les peuples dont tu as peur. 
${}^{20}Bien plus, le Seigneur ton Dieu enverra des frelons jusqu’à ce que périssent les survivants qui se seraient cachés. 
${}^{21}Tu ne trembleras pas devant eux, car le Seigneur ton Dieu, au milieu de toi, est un Dieu grand et terrible. 
${}^{22}Le Seigneur ton Dieu expulsera ces nations devant toi peu à peu : tu ne pourras les exterminer sur-le-champ, de peur que les bêtes sauvages ne se multiplient contre toi. 
${}^{23}Mais le Seigneur ton Dieu te les livrera, et les frappera d’une grande panique jusqu’à ce qu’elles soient détruites. 
${}^{24}Il livrera leurs rois entre tes mains et tu feras disparaître leur nom de sous les cieux : pas un ne tiendra devant toi, tu les extermineras jusqu’au dernier.
${}^{25}Les idoles de leurs dieux, vous les brûlerez. Tu ne convoiteras ni l’or ni l’argent qui les recouvrent, et tu ne les prendras pas, de peur qu’ils ne soient pour toi un piège ; car, pour le Seigneur ton Dieu, c’est une abomination. 
${}^{26}Tu ne feras pas entrer dans ta maison une Abomination ; tu serais, comme elle, anathème ; tu l’auras en horreur et en abomination, car c’est un anathème.
      
         
      \bchapter{}
      \begin{verse}
${}^{1}Tous les commandements que je vous prescris aujourd’hui, vous veillerez à les mettre en pratique, afin que vous viviez, deveniez de plus en plus nombreux et entriez en possession du pays que le Seigneur a juré de donner à vos pères. 
${}^{2}Souviens-toi de la longue marche que tu as faite\\pendant quarante années dans le désert ; le Seigneur ton Dieu te l’a imposée pour te faire passer par la pauvreté ; il voulait t’éprouver et savoir ce que tu as dans le cœur : allais-tu garder ses commandements, oui ou non ? 
${}^{3}Il t’a fait passer par la pauvreté, il t’a fait sentir la faim, et il t’a donné à manger la manne – cette nourriture que ni toi ni tes pères n’aviez connue – pour que tu saches que l’homme ne vit pas seulement de pain, mais de tout ce qui vient de la bouche du Seigneur. 
${}^{4}Ton vêtement ne s’est pas usé sur toi, et ton pied ne s’est pas enflé, au cours de ces quarante années ! 
${}^{5}Tu le sauras en ton cœur : comme un homme éduque son fils, ainsi le Seigneur ton Dieu fait ton éducation. 
${}^{6}Tu garderas les commandements du Seigneur ton Dieu pour marcher sur ses chemins et pour le craindre.
      
         
${}^{7}Le Seigneur ton Dieu te conduit vers un pays fertile : pays de rivières abondantes, de sources profondes jaillissant dans les vallées et les montagnes, 
${}^{8} pays de blé et d’orge, de raisin, de grenades et de figues, pays d’olives, d’huile et de miel ; 
${}^{9} pays où le pain ne te manquera pas et où tu ne seras privé de rien ; pays dont les pierres contiennent du fer, et dont les montagnes sont des mines de cuivre. 
${}^{10} Tu mangeras et tu seras rassasié, tu béniras le Seigneur ton Dieu pour ce pays fertile qu’il t’a donné. 
${}^{11} Garde-toi d’oublier le Seigneur ton Dieu, de négliger ses commandements, ses ordonnances et ses décrets, que je te donne aujourd’hui. 
${}^{12} Quand tu auras mangé et seras rassasié, quand tu auras bâti de belles maisons et que tu les habiteras, 
${}^{13} quand tu auras vu se multiplier ton gros et ton petit bétail, ton argent, ton or et tous tes biens, 
${}^{14} n’en tire pas orgueil, et n’oublie pas le Seigneur ton Dieu qui t’a fait sortir du pays d’Égypte, de la maison d’esclavage. 
${}^{15} C’est lui qui t’a fait traverser ce désert, vaste et terrifiant, pays des serpents brûlants et des scorpions, pays de la sécheresse et de la soif. C’est lui qui, pour toi, a fait jaillir l’eau de la roche la plus dure. 
${}^{16} C’est lui qui, dans le désert, t’a donné la manne – cette nourriture inconnue de tes pères – pour te faire passer par la pauvreté et pour t’éprouver avant de te rendre heureux\\.
${}^{17}Garde-toi de dire en ton cœur : « C’est ma force, c’est la vigueur de ma main qui m’ont procuré cette richesse. » 
${}^{18}Souviens-toi du Seigneur ton Dieu : car c’est lui qui t’a donné la force d’acquérir cette richesse, en confirmant ainsi l’Alliance qu’il avait jurée à tes pères, comme on le voit\\aujourd’hui. 
${}^{19}Si jamais tu en viens à oublier le Seigneur ton Dieu, si tu suis d’autres dieux, si tu les sers et si tu te prosternes devant eux – je l’atteste aujourd’hui contre vous –, à coup sûr vous périrez : 
${}^{20}comme les nations que le Seigneur aura fait périr devant vous, ainsi vous périrez, pour n’avoir pas écouté la voix du Seigneur votre Dieu.
      
         
      \bchapter{}
      \begin{verse}
${}^{1}Écoute, Israël ! Te voilà aujourd’hui sur le point de passer le Jourdain, pour aller déposséder des nations plus grandes et plus puissantes que toi, et prendre des villes immenses dont les fortifications montent jusqu’au ciel. 
${}^{2}C’est un peuple grand et de haute stature que les géants Anaqites ! Toi, tu le connais, tu as entendu dire : « Qui peut tenir devant les fils d’Anaq ? » 
${}^{3}Or tu sais aujourd’hui que le Seigneur ton Dieu passera devant toi, comme un feu dévorant : c’est lui qui les exterminera ; c’est lui qui les abaissera devant toi ; alors tu les déposséderas et tu les feras périr aussitôt, comme te l’a dit le Seigneur.
${}^{4}Lorsque le Seigneur ton Dieu les aura chassés devant toi, ne dis pas en ton cœur : « C’est à cause de ma justice que le Seigneur m’a fait entrer dans ce pays pour en prendre possession. » Car c’est à cause de la méchanceté de ces nations, que le Seigneur les dépossède devant toi. 
${}^{5}Ce n’est pas en raison de ta justice ni à cause de la droiture de ton cœur que tu entreras dans leur pays pour en prendre possession ; c’est en raison de leur méchanceté que le Seigneur ton Dieu dépossède ces nations devant toi, et pour tenir la parole qu’il a jurée à tes pères, Abraham, Isaac et Jacob. 
${}^{6}Sache bien que ce n’est pas à cause de ta justice que le Seigneur ton Dieu te donne à posséder ce bon pays, car tu es un peuple à la nuque raide.
${}^{7}Souviens-toi. N’oublie pas que tu as irrité le Seigneur ton Dieu dans le désert. Depuis le jour où vous êtes sortis d’Égypte jusqu’à ce que vous arriviez en ce lieu, vous avez été rebelles au Seigneur.
${}^{8}Au mont Horeb vous avez irrité le Seigneur, et le Seigneur s’est mis dans une telle colère qu’il voulait vous exterminer. 
${}^{9} J’étais monté sur la montagne pour recevoir les tables de pierre, les tables de l’Alliance que le Seigneur a conclue avec vous. Je suis resté dans la montagne quarante jours et quarante nuits sans manger ni boire\\. 
${}^{10} Le Seigneur m’a donné les deux tables de pierre écrites du doigt de Dieu, et portant toutes les paroles qu’il vous avait dites du milieu du feu, sur la montagne, le jour de l’assemblée. 
${}^{11} C’est donc au bout de quarante jours et de quarante nuits que le Seigneur m’a donné ces deux tables de pierre, les tables de l’Alliance. 
${}^{12} Le Seigneur me dit : « Lève-toi, descends vite d’ici, car ton peuple, que tu as fait sortir d’Égypte, s’est corrompu. Ils n’ont pas mis longtemps à s’écarter du chemin que je leur avais ordonné de suivre\\ : ils se sont fait une idole en métal fondu. » 
${}^{13} Le Seigneur me dit encore : « Je vois que ce peuple est un peuple à la nuque raide.  
${}^{14} Laisse-moi faire, je vais les anéantir, effacer leur nom de sous les cieux, et je ferai de toi une nation plus puissante et plus nombreuse que ce peuple\\ ! »
${}^{15}Je redescendis\\de la montagne, qui était toujours en feu ; je tenais dans les mains les deux tables de l’Alliance. 
${}^{16}Et je vis que vous veniez de pécher contre le Seigneur votre Dieu. Vous vous étiez fait un veau en métal fondu, vous n’aviez pas mis longtemps à vous écarter du chemin que le Seigneur vous avait ordonné de suivre\\. 
${}^{17}Je pris les deux tables ; de mes deux mains, je les jetai et je les brisai sous vos yeux. 
${}^{18}Je tombai à terre\\devant le Seigneur, et, comme la première fois, je fus quarante jours et quarante nuits sans manger ni boire\\, à cause de tous les péchés que vous aviez commis : vous aviez fait ce qui est mal aux yeux du Seigneur et ainsi vous l’aviez exaspéré. 
${}^{19}Je redoutais cette colère, cette fureur du Seigneur, irrité au point de vouloir vous anéantir. Et, cette fois encore, le Seigneur m’écouta. 
${}^{20}Contre Aaron également, le Seigneur se mit en grande colère, au point de vouloir l’anéantir ; à ce moment-là j’ai intercédé aussi pour Aaron. 
${}^{21}Quant à votre péché, ce veau que vous aviez fait, je l’ai pris, je l’ai brûlé, je l’ai broyé, je l’ai réduit en fine poussière, et j’en ai jeté la poussière dans le torrent qui descend de la montagne.
${}^{22}À Tabeéra, à Massa, à Qibroth-ha-Taawa, vous avez irrité le Seigneur. 
${}^{23}Et quand le Seigneur vous renvoya de Cadès-Barnéa, en disant : « Montez prendre possession du pays que je vous ai donné ! », vous avez été rebelles à l’ordre du Seigneur votre Dieu, vous n’avez pas cru en lui, vous n’avez pas écouté sa voix. 
${}^{24}Rebelles au Seigneur, vous l’avez été depuis le jour où je vous connais !
${}^{25}Je suis donc tombé à terre devant le Seigneur et, durant ces quarante jours et ces quarante nuits, je restai prostré, car le Seigneur avait dit qu’il allait vous anéantir. 
${}^{26}J’ai intercédé auprès du Seigneur et j’ai dit : « Seigneur mon Dieu, ne détruis pas ton peuple, ton héritage, lui que tu as racheté dans ta grandeur et que tu as fait sortir d’Égypte par ta main puissante. 
${}^{27}Souviens-toi de tes serviteurs, Abraham, Isaac et Jacob ; ne regarde pas l’endurcissement de ce peuple, ni sa méchanceté, ni son péché. 
${}^{28}Que, dans le pays d’où tu nous as fait sortir, l’on ne dise pas : “Le Seigneur n’a pas été capable de les faire entrer dans le pays dont il leur avait parlé ; c’est à cause de sa haine contre eux qu’il les a fait sortir pour les faire mourir dans le désert.” 
${}^{29}Pourtant, c’est bien eux, ton peuple, ton héritage, que tu as fait sortir par ta grande force et ton bras étendu. »
      
         
      \bchapter{}
      \begin{verse}
${}^{1}En ce temps-là, le Seigneur me dit : « Taille deux tables de pierre, semblables aux premières, et monte vers moi sur la montagne ; tu feras aussi une arche de bois. 
${}^{2}J’écrirai sur ces tables les paroles qui étaient sur les premières, celles que tu as brisées, et tu les placeras dans l’arche. »
${}^{3}J’ai donc fait une arche en bois d’acacia, j’ai taillé deux tables de pierre, semblables aux premières, et je suis monté sur la montagne, les deux tables dans les mains. 
${}^{4}Il écrivit sur les tables la même inscription que la première fois, ces Dix Paroles que le Seigneur vous avait dites sur la montagne, du milieu du feu, le jour de l’assemblée. Puis le Seigneur me les donna. 
${}^{5}Je suis redescendu de la montagne, j’ai placé les tables dans l’arche que j’avais faite, et elles y sont restées, comme le Seigneur me l’avait ordonné.
${}^{6}Les Israélites levèrent le camp et quittèrent les puits des Bené-Yaaqane pour Moséra. C’est là que mourut Aaron et qu’il fut mis au tombeau. Éléazar, son fils, lui succéda comme prêtre. 
${}^{7}De là, ils partirent pour la Goudgoda, et de la Goudgoda vers Yotbata, un pays de torrents d’eau.
${}^{8}En ce temps-là, le Seigneur mit à part les descendants de Lévi, pour porter l’arche de son Alliance, se tenir en sa présence, assurer le service divin, et bénir le peuple\\au nom du Seigneur, comme ils l’ont fait jusqu’à ce jour. 
${}^{9}C’est pourquoi Lévi n’a pas reçu sa part d’héritage comme ses frères. C’est le Seigneur qui est son héritage, comme l’a dit le Seigneur notre Dieu.
${}^{10}Quant à moi, je me suis tenu sur la montagne, comme la première fois, quarante jours et quarante nuits. Cette fois encore, le Seigneur m’écouta ; le Seigneur consentit à ne pas t’exterminer. 
${}^{11}Alors le Seigneur me dit : « Lève-toi ! Va devant le peuple, donne le signal du départ : qu’ils entrent en possession du pays que j’ai juré à leurs pères de leur donner. »
${}^{12}Et maintenant, sais-tu, Israël, ce que le Seigneur ton Dieu te demande ? Craindre le Seigneur ton Dieu, suivre tous ses chemins, aimer le Seigneur ton Dieu, le servir de tout ton cœur et de toute ton âme, 
${}^{13}garder les commandements et les décrets du Seigneur que je te donne aujourd’hui pour ton bien.
${}^{14}C’est au Seigneur ton Dieu qu’appartiennent les cieux et les hauteurs des cieux, la terre et tout ce qu’elle renferme. 
${}^{15} Et pourtant, c’est uniquement à tes pères que le Seigneur ton Dieu s’est attaché par amour. Après eux, entre tous les peuples, c’est leur descendance qu’il a choisie, ce qu’il fait encore aujourd’hui avec vous. 
${}^{16} Pratiquez la circoncision du cœur, n’ayez plus la nuque raide, 
${}^{17} car le Seigneur votre Dieu est le Dieu des dieux et le Seigneur des seigneurs, le Dieu grand, vaillant et redoutable, qui est impartial et ne se laisse pas acheter. 
${}^{18} C’est lui qui rend justice\\à l’orphelin et à la veuve, qui aime l’immigré, et qui lui donne nourriture et vêtement. 
${}^{19} Aimez donc l’immigré, car au pays d’Égypte vous étiez des immigrés.
${}^{20}Tu craindras le Seigneur ton Dieu, tu le serviras, c’est à lui que tu resteras attaché, c’est par son nom que tu prêteras serment. 
${}^{21} Il est ton Dieu, c’est lui que tu dois louer : il a fait pour toi ces choses grandes et redoutables que tu as vues de tes yeux. 
${}^{22} Quand tes pères sont arrivés en Égypte, ils n’étaient que soixante-dix ; mais à présent le Seigneur votre Dieu vous a rendus aussi nombreux que les étoiles du ciel.
      
         
      \bchapter{}
      \begin{verse}
${}^{1}Tu aimeras donc le Seigneur ton Dieu et tu garderas ses observances, ses décrets, ses ordonnances et ses commandements, chaque jour. 
${}^{2}Aujourd’hui, contrairement à vos fils à qui il n’a pas été donné de connaître et de voir, vous, vous connaissez les leçons du Seigneur votre Dieu, sa grandeur, sa main forte et son bras étendu ; 
${}^{3}les signes et les œuvres qu’il a accomplis en Égypte, contre Pharaon, roi d’Égypte, et tout son pays : 
${}^{4}ce qu’il a fait à l’armée égyptienne, à ses chevaux et à ses chars, en faisant déferler sur eux les eaux de la mer des Roseaux tandis qu’ils vous poursuivaient – le Seigneur les a supprimés ; vous le constatez encore aujourd’hui ; 
${}^{5}vous connaissez ce qu’il a fait pour vous dans le désert jusqu’à ce que vous arriviez en ce lieu ; 
${}^{6}ce qu’il a fait à Datane et à Abiram, les fils d’Éliab, fils de Roubène, quand la terre ouvrit sa bouche et les engloutit au milieu de tout Israël, avec leurs familles, leurs tentes et tous les gens qui les suivaient. 
${}^{7}Ce sont bien vos propres yeux qui ont vu toute la grande œuvre accomplie par le Seigneur !
      
         
${}^{8}Vous garderez donc tous les commandements que je te prescris aujourd’hui, afin d’être forts, de prendre possession du pays où vous allez entrer 
${}^{9}et de prolonger vos jours sur la terre que le Seigneur a juré à vos pères de leur donner, à eux et à leur descendance, un pays ruisselant de lait et de miel.
${}^{10}Car le pays où tu entres pour en prendre possession n’est pas comme le pays d’Égypte d’où vous êtes sortis : après avoir semé, il te fallait arroser avec ton pied, comme on arrose un jardin potager. 
${}^{11}Le pays où vous allez passer pour le posséder est un pays de montagnes et de vallées, qui boit la pluie du ciel. 
${}^{12}C’est un pays dont le Seigneur ton Dieu prend soin : les yeux du Seigneur ton Dieu sont fixés sur lui constamment, du début à la fin de l’année. 
${}^{13}Assurément, si vous écoutez bien mes commandements, ceux que je vous prescris aujourd’hui, si vous aimez le Seigneur votre Dieu, et le servez de tout votre cœur et de toute votre âme, 
${}^{14}je donnerai à votre pays la pluie en son temps, pluie d’automne et pluie de printemps, et tu récolteras ton froment, ton vin nouveau et ton huile fraîche, 
${}^{15}je mettrai dans ton champ de l’herbe pour ton bétail. Tu mangeras et tu seras rassasié.
${}^{16}Prenez bien garde que votre cœur ne soit séduit, que vous ne vous détourniez pour servir d’autres dieux et vous prosterner devant eux : 
${}^{17}la colère du Seigneur s’enflammerait contre vous ; il fermerait les cieux, et il n’y aurait plus de pluie, la terre ne donnerait plus son fruit, et vous disparaîtriez rapidement de ce bon pays que le Seigneur vous donne.
${}^{18}Les paroles que je vous donne, vous les mettrez dans votre cœur, dans votre âme. Vous les attacherez à votre poignet comme un signe, elles seront un bandeau sur votre front\\. 
${}^{19}Vous les apprendrez à vos fils, vous les leur direz quand tu seras assis dans ta maison et quand tu marcheras sur la route, quand tu seras couché et quand tu seras debout. 
${}^{20}Tu les inscriras à l’entrée de ta maison et aux portes de ta ville : 
${}^{21}ainsi vos jours et ceux de vos fils seront nombreux sur la terre que le Seigneur a juré à vos pères de leur donner. Que ces jours durent aussi longtemps que les cieux au-dessus de la terre !
${}^{22}Car si vous gardez bien, pour les mettre en pratique, tous ces commandements que je vous prescris : aimer le Seigneur votre Dieu, marcher dans tous ses chemins et vous attacher à lui, 
${}^{23}alors, le Seigneur dépossédera toutes ces nations devant vous, et vous posséderez des nations plus grandes et plus puissantes que vous. 
${}^{24}Tout lieu que foulera la plante de vos pieds vous appartiendra ; votre territoire s’étendra depuis le désert et le Liban, depuis le fleuve, l’Euphrate, jusqu’à la Méditerranée. 
${}^{25}Personne ne vous résistera, le Seigneur votre Dieu répandra frayeur et terreur sur toute la surface du pays que vous foulerez, comme il vous l’a dit.
${}^{26}Vois ! Aujourd’hui je vous propose la bénédiction ou la malédiction : 
${}^{27}la bénédiction si vous écoutez les commandements du Seigneur votre Dieu, que je vous donne aujourd’hui ; 
${}^{28}la malédiction si vous n’écoutez pas les commandements du Seigneur votre Dieu, si vous vous écartez du chemin que je vous prescris aujourd’hui, pour suivre d’autres dieux que vous ne connaissez pas.
${}^{29}Lorsque le Seigneur ton Dieu t’aura fait entrer dans le pays où tu arrives pour en prendre possession, tu placeras la bénédiction sur le mont Garizim et la malédiction sur le mont Ébal. 
${}^{30}Ces monts ne sont-ils pas au-delà du Jourdain, derrière la route du couchant, dans le pays du Cananéen qui habite dans la Araba, en face de Guilgal, à côté des chênes de Moré ? 
${}^{31}Car vous allez passer le Jourdain, pour entrer en possession du pays que le Seigneur votre Dieu vous donne ; vous en prendrez possession et vous y demeurerez. 
${}^{32}Vous veillerez à mettre en pratique tous les décrets et les ordonnances que je vous présente aujourd’hui.
      
         
      \bchapter{}
      \begin{verse}
${}^{1}Voici les décrets et les ordonnances que vous veillerez à mettre en pratique dans le pays que le Seigneur, le Dieu de tes pères, t’a donné en possession ; vous les mettrez en pratique aussi longtemps que vous vivrez sur ce sol.
      
         
${}^{2}Vous ferez disparaître complètement tous les lieux de culte où les nations que vous dépossédez ont servi leurs dieux, sur les hautes montagnes et sur les collines, ainsi que sous tous les arbres verts. 
${}^{3}Vous démolirez leurs autels, vous briserez leurs stèles ; leurs poteaux sacrés, vous les brûlerez, les idoles de leurs dieux, vous les abattrez, et de chacun de ces lieux vous supprimerez leur nom. 
${}^{4}Vous n’agirez pas ainsi à l’égard du Seigneur votre Dieu ; 
${}^{5}c’est uniquement au lieu choisi par le Seigneur votre Dieu parmi toutes vos tribus pour y mettre son nom et y demeurer, c’est là que vous le chercherez, c’est là que tu viendras. 
${}^{6}Vous y apporterez vos holocaustes et vos sacrifices, vos dîmes et ce que votre main aura prélevé, vos offrandes votives et vos offrandes volontaires, les premiers-nés de votre gros et de votre petit bétail. 
${}^{7}Vous mangerez là en présence du Seigneur votre Dieu ; vous serez heureux dans tout ce que vous entreprendrez, vous et votre maisonnée, parce que le Seigneur ton Dieu t’aura béni.
${}^{8}Vous n’imiterez en rien notre manière d’agir ici, aujourd’hui : chacun fait tout ce que bon lui semble. 
${}^{9}En effet, jusqu’à présent, vous n’êtes pas encore parvenus au lieu du repos, l’héritage que le Seigneur ton Dieu te donne. 
${}^{10}Mais vous allez passer le Jourdain et vous habiterez dans le pays que le Seigneur votre Dieu vous donne en héritage : il vous procurera le repos en vous dégageant de tous vos ennemis d’alentour, et vous habiterez en sécurité. 
${}^{11}C’est au lieu choisi par le Seigneur votre Dieu pour y faire demeurer son nom, c’est là que vous apporterez tout ce que je vous prescris : vos holocaustes et vos sacrifices, vos dîmes et ce que votre main aura prélevé, ainsi que le meilleur de toutes les offrandes votives promises au Seigneur. 
${}^{12}Vous vous réjouirez alors en présence du Seigneur votre Dieu, vous, vos fils et vos filles, vos serviteurs et vos servantes, ainsi que le lévite qui demeure dans vos villes car il n’a ni part ni héritage avec vous.
${}^{13}Garde-toi d’offrir tes holocaustes en tout lieu qui te semblerait bon ; 
${}^{14}c’est uniquement au lieu choisi par le Seigneur en l’une de tes tribus, c’est là que tu offriras tes holocaustes ; c’est là que tu feras tout ce que je te commande. 
${}^{15}Cependant, chaque fois que tu le désires, en chacune de tes villes, tu pourras abattre des bêtes et en manger, selon ce que t’aura donné la bénédiction du Seigneur ton Dieu ; l’homme impur et celui qui est pur en mangeront, comme si c’était de la gazelle ou du cerf. 
${}^{16}Toutefois, vous ne consommerez pas le sang : tu le répandras à terre comme de l’eau. 
${}^{17}Dans tes villes, tu ne pourras pas manger la dîme de ton froment, de ton vin nouveau, de ton huile fraîche, ni les premiers-nés de ton gros et de ton petit bétail, ni aucune offrande votive que tu auras promise, ni aucune de tes offrandes volontaires, ni ce que ta main aura prélevé. 
${}^{18}C’est uniquement en présence du Seigneur ton Dieu, dans le lieu choisi par le Seigneur ton Dieu, que tu en mangeras, toi, ton fils et ta fille, ton serviteur et ta servante, ainsi que le lévite qui habite dans tes villes ; tu te réjouiras en présence du Seigneur ton Dieu de tout ce que tu entreprendras. 
${}^{19}Garde-toi de négliger le lévite au long des jours où tu vivras sur ton sol.
${}^{20}Quand le Seigneur ton Dieu aura élargi ton territoire, selon sa parole, et que tu diras : « Je mangerais bien de la viande », alors, si tu le désires, tu pourras en manger autant que tu voudras. 
${}^{21}Si tu es trop loin du lieu que le Seigneur ton Dieu a choisi pour y mettre son nom, tu pourras abattre une bête parmi le gros ou le petit bétail que t’aura donné le Seigneur ; tu le feras comme je te l’ai ordonné ; tu en mangeras dans tes villes autant que tu en désireras. 
${}^{22}Tout comme on mange de la gazelle ou du cerf, ainsi tu en mangeras ; l’homme impur et celui qui est pur en mangeront l’un et l’autre. 
${}^{23}Cependant, garde-toi fermement de consommer le sang, car le sang, c’est la vie, et tu ne dois pas manger la vie avec la chair. 
${}^{24}Tu ne consommeras pas le sang, tu le répandras à terre comme de l’eau. 
${}^{25}Tu ne le consommeras pas, et ainsi tu seras heureux, toi et tes fils après toi, car tu auras fait ce qui est droit aux yeux du Seigneur.
${}^{26}Toutefois, les offrandes saintes que tu dois faire, et tes offrandes votives, tu iras les apporter au lieu choisi par le Seigneur. 
${}^{27}Tu offriras tes holocaustes, chair et sang, sur l’autel du Seigneur ton Dieu. Le sang de tes autres sacrifices sera répandu sur l’autel du Seigneur ton Dieu, mais la chair, tu la mangeras.
${}^{28}Observe, écoute toutes les paroles des commandements que je te donne ; ainsi tu seras heureux pour toujours, toi et tes fils après toi, car tu auras fait ce qui est bon et ce qui est droit aux yeux du Seigneur ton Dieu.
${}^{29}Quand le Seigneur ton Dieu aura retranché, pour les déposséder devant toi, les nations où tu te rendras, quand donc tu les auras dépossédées et que tu habiteras dans leur pays, 
${}^{30}garde-toi de te laisser prendre au piège à leur suite ; après qu’elles auront été exterminées de devant toi, ne recherche pas leurs dieux en disant : « Comment ces nations servaient-elles leurs dieux ? Que je fasse de même, moi aussi ! » 
${}^{31}Non, tu n’agiras pas ainsi à l’égard du Seigneur ton Dieu. Car tout ce que le Seigneur a en abomination, tout ce qu’il déteste, elles le font pour leurs dieux : pour leurs dieux, elles vont même jusqu’à consumer par le feu leurs fils et leurs filles !
      
         
      \bchapter{}
      \begin{verse}
${}^{1}Tout ce que je vous commande, vous veillerez à le mettre en pratique. Tu n’y ajouteras rien, tu n’en retrancheras rien.
${}^{2}S’il surgit au milieu de toi un prophète ou un faiseur de songes, qui te propose un signe ou un prodige 
${}^{3}– même si se réalise le signe ou le prodige qu’il t’a annoncé en disant : « Allons à la suite d’autres dieux que tu ne connais pas, et servons-les ! » –, 
${}^{4}tu n’écouteras pas les paroles de ce prophète ou de ce faiseur de songes. En effet, le Seigneur votre Dieu vous met à l’épreuve : il veut savoir si vous aimez vraiment le Seigneur votre Dieu de tout votre cœur et de toute votre âme. 
${}^{5}C’est le Seigneur votre Dieu que vous devez suivre, c’est lui que vous craindrez ; ses commandements, vous les garderez ; sa voix, vous l’écouterez ; c’est lui que vous servirez ; c’est à lui que vous vous attacherez. 
${}^{6}Quant à ce prophète ou ce faiseur de songes, il sera mis à mort, car il a prêché la révolte contre le Seigneur votre Dieu, lui qui vous a fait sortir du pays d’Égypte, et qui t’a racheté de la maison d’esclavage ; cet homme voulait t’égarer, loin du chemin que le Seigneur ton Dieu t’a ordonné de suivre. Tu ôteras le mal du milieu de toi.
${}^{7}Si ton frère, fils de ta mère, ton fils ou ta fille, ta femme bien-aimée ou l’ami qui est un autre toi-même, cherche en secret à te séduire en disant : « Allons servir d’autres dieux ! » – des dieux que ni tes pères ni toi ne connaissiez, 
${}^{8}ces dieux des peuples proches ou éloignés de toi d’une extrémité de la terre à l’autre –, 
${}^{9}tu ne l’approuveras pas, tu ne l’écouteras pas, tu ne porteras pas sur lui un regard de pitié, tu ne l’épargneras pas, tu ne l’excuseras pas. 
${}^{10}Bien plus, tu devras le tuer : tu seras le premier à lever la main contre lui pour le mettre à mort ; ensuite le peuple tout entier l’achèvera de ses mains. 
${}^{11}Tu le lapideras jusqu’à ce que mort s’ensuive, parce qu’il a cherché à t’égarer loin du Seigneur ton Dieu, lui qui t’a fait sortir du pays d’Égypte, de la maison d’esclavage. 
${}^{12}Tout Israël l’apprendra et sera saisi de crainte ; alors on cessera de commettre ce genre de mal au milieu de toi.
${}^{13}Si tu entends dire que, dans l’une des villes que le Seigneur ton Dieu te donne pour y habiter, 
${}^{14}des hommes, des mécréants sortis du milieu de toi, ont égaré les habitants de leur ville en disant : « Allons servir d’autres dieux ! », des dieux que vous ne connaissez pas, 
${}^{15}tu t’informeras, tu feras une enquête, tu interrogeras avec soin. Si c’est vrai, si le fait est établi, si une telle abomination a été commise au milieu de toi, 
${}^{16}tu devras passer au fil de l’épée les habitants de cette ville ; tu voueras celle-ci à l’anathème, avec tout ce qu’elle contient ; même son bétail, tu le passeras au fil de l’épée. 
${}^{17}Tout le butin, tu le rassembleras au milieu de la place et tu mettras le feu à la ville ainsi qu’à tout le butin, en offrande totale au Seigneur ton Dieu. Cette ville deviendra pour toujours une ruine ; elle ne sera plus rebâtie. 
${}^{18}Ta main ne gardera rien de cet anathème, afin que le Seigneur revienne de l’ardeur de sa colère, qu’il te montre sa tendresse, te fasse miséricorde, et qu’il te multiplie comme il l’a juré à tes pères. 
${}^{19}Il en sera ainsi pourvu que tu écoutes la voix du Seigneur ton Dieu, gardant tous ses commandements que je te donne aujourd’hui et faisant ce qui est droit aux yeux du Seigneur ton Dieu.
      
         
      \bchapter{}
      \begin{verse}
${}^{1}Vous êtes des fils pour le Seigneur votre Dieu. Vous ne vous ferez pas d’entailles, vous ne vous couperez pas les cheveux sur le front, en signe de deuil. 
${}^{2}Car tu es un peuple consacré au Seigneur ton Dieu, c’est toi que le Seigneur a choisi pour être son peuple, son domaine particulier parmi tous les peuples qui sont sur la surface de la terre.
      
         
${}^{3}Tu ne mangeras rien d’abominable ! 
${}^{4}Voici les animaux dont vous pourrez manger : bœuf, mouton, chevreau, 
${}^{5}cerf, gazelle, chevreuil, bouquetin, daim, antilope, mouflon. 
${}^{6}De tout animal qui a le sabot fourchu, fendu en deux ongles, et qui rumine, vous pourrez manger.
${}^{7}Toutefois, parmi les ruminants et les animaux à sabot fourchu et fendu, vous ne pourrez manger de ceux-ci : le chameau, le lièvre et le daman, car ils ruminent mais n’ont pas le sabot fourchu ; vous les tiendrez pour impurs. 
${}^{8}Ni le porc, car il a le sabot fourchu et fendu mais il ne rumine pas ; vous le tiendrez pour impur. Vous ne mangerez pas de leur chair et vous ne toucherez pas à leurs cadavres.
${}^{9}Voici ce dont vous pourrez manger parmi tout ce qui vit dans l’eau : tout ce qui possède nageoires et écailles, vous pourrez en manger. 
${}^{10}Mais vous ne mangerez pas de ce qui ne possède ni nageoires ni écailles : vous le tiendrez pour impur.
${}^{11}Vous pourrez manger de tout oiseau pur, 
${}^{12}mais voici les oiseaux dont vous ne pourrez pas manger : l’aigle, le gypaète, l’orfraie, 
${}^{13}le busard, le vautour et les différentes espèces de milans, 
${}^{14}toutes les espèces de corbeaux, 
${}^{15}l’autruche, le chat-huant, la mouette et les différentes espèces d’éperviers, 
${}^{16}le hibou, la chouette, l’ibis, 
${}^{17}la hulotte, le vautour blanc, le cormoran, 
${}^{18}la cigogne et les différentes espèces de hérons, la huppe, la chauve-souris. 
${}^{19}Vous tiendrez tous les insectes ailés pour impurs, vous n’en mangerez pas. 
${}^{20}Vous pourrez manger de tout volatile pur.
${}^{21}Vous ne pourrez manger d’aucune bête crevée. Tu la donneras à l’immigré qui réside dans ta ville : lui peut en manger, ou bien vends-la à un étranger. Toi, en effet, tu es un peuple consacré au Seigneur ton Dieu.
      Tu ne feras pas cuire un chevreau dans le lait de sa mère.
${}^{22}Tu prélèveras chaque année la dîme de tout ce que tes semailles auront produit dans tes champs. 
${}^{23}La dîme de ton froment, de ton vin nouveau et de ton huile fraîche, les premiers-nés de ton gros et de ton petit bétail, tu les mangeras devant le Seigneur ton Dieu, au lieu qu’il aura choisi pour y faire demeurer son nom ; ainsi, tu apprendras à craindre le Seigneur ton Dieu, tous les jours. 
${}^{24}Si le chemin est trop long pour toi, si tu ne peux apporter la dîme au lieu choisi par le Seigneur pour y mettre son nom, parce que ce lieu est trop éloigné, et que le Seigneur ton Dieu t’aura béni, 
${}^{25}tu convertiras la dîme en argent, tu serreras l’argent dans ta main et tu iras au lieu choisi par le Seigneur ton Dieu ; 
${}^{26}tu échangeras cet argent contre tout ce que tu voudras : gros bétail, petit bétail, vin, boisson forte, tout ce dont tu auras envie. Là, tu mangeras devant le Seigneur ton Dieu et tu te réjouiras, toi et ta maisonnée. 
${}^{27}Et le lévite qui habite dans ta ville, tu ne le délaisseras pas, puisqu’il n’a ni part ni héritage avec toi.
${}^{28}Au bout de trois ans, tu prélèveras toutes les dîmes de tes récoltes de cette année-là et tu les déposeras aux portes de ta ville. 
${}^{29}Alors viendront le lévite – puisqu’il n’a ni part ni héritage avec toi –, l’immigré, l’orphelin et la veuve qui résident dans ta ville ; ils mangeront et seront rassasiés. Ainsi le Seigneur ton Dieu bénira toute œuvre de tes mains.
      
         
      \bchapter{}
      \begin{verse}
${}^{1}Au bout de sept ans, tu feras la remise des dettes. 
${}^{2}Voici comment se fera cette remise : tout possesseur d’une créance fera remise à son prochain de ce qu’il lui aura prêté ; il n’exercera pas de poursuite contre son prochain ou son frère, puisqu’on aura proclamé la remise des dettes en l’honneur du Seigneur. 
${}^{3}Contre l’étranger tu pourras exercer des poursuites mais, en ce qui concerne ton frère, tu feras la remise de sa dette.
${}^{4}De toute manière, il n’y aura pas de malheureux chez toi. Le Seigneur, en effet, te comblera de bénédictions dans le pays que le Seigneur ton Dieu te donne en héritage pour que tu en prennes possession. 
${}^{5}Ceci, à condition que tu écoutes bien la voix du Seigneur ton Dieu, en veillant à pratiquer tout ce commandement que je te donne aujourd’hui. 
${}^{6}Alors le Seigneur ton Dieu te bénira comme il te l’a dit : tu prêteras à de nombreuses nations mais tu n’emprunteras pas, tu domineras de nombreuses nations mais aucune ne te dominera.
${}^{7}Se trouve-t-il chez toi un malheureux parmi tes frères, dans l’une des villes de ton pays que le Seigneur ton Dieu te donne ? Tu n’endurciras pas ton cœur, tu ne fermeras pas la main à ton frère malheureux, 
${}^{8}mais tu lui ouvriras tout grand la main et lui prêteras largement de quoi suffire à ses besoins. 
${}^{9}Garde-toi de tenir en ton cœur ces propos pervers : « Voici bientôt la septième année, l’année de la remise des dettes », en regardant méchamment ton frère malheureux sans rien lui donner ; il en appellerait au Seigneur contre toi, et tu serais chargé d’un péché ! 
${}^{10}Tu lui donneras largement, ce n’est pas à contrecœur que tu lui donneras. Pour ce geste, le Seigneur ton Dieu te bénira dans toutes tes actions et dans toutes tes entreprises.
${}^{11}Certes, le malheureux ne disparaîtra pas de ce pays. Aussi je te donne ce commandement : tu ouvriras tout grand ta main pour ton frère quand il est, dans ton pays, pauvre et malheureux.
${}^{12}Quand, parmi tes frères hébreux, un homme ou une femme se sera vendu à toi, il te servira durant six ans. La septième année, tu le renverras libre de chez toi 
${}^{13}et, en ce cas, tu ne le renverras pas les mains vides : 
${}^{14}tu le couvriras de cadeaux avec le produit de ton petit bétail, de ton aire à grain et de ton pressoir ; tu lui donneras à la mesure de la bénédiction du Seigneur ton Dieu. 
${}^{15}Tu te souviendras que tu as été esclave au pays d’Égypte et que le Seigneur ton Dieu t’a racheté : voilà pourquoi je te donne aujourd’hui ce commandement.
${}^{16}Mais s’il te dit : « Je ne veux pas sortir de chez toi ! », parce qu’il t’aime, toi et ta maison, et qu’il est heureux chez toi, 
${}^{17}tu prendras un poinçon, tu lui en perceras l’oreille contre la porte, et il sera ton serviteur pour toujours. Envers ta servante, tu feras de même.
${}^{18}Qu’il ne te semble pas pénible de le renvoyer libre de chez toi : durant six ans de service, il t’aura rapporté deux fois ce que gagne un salarié ! Ainsi, le Seigneur ton Dieu te bénira en tout ce que tu feras.
${}^{19}Tout premier-né mâle qui naîtra dans ton gros ou ton petit bétail, tu le consacreras au Seigneur ton Dieu. Tu ne feras pas travailler le premier-né de ta vache, tu ne tondras pas le premier-né de ta brebis. 
${}^{20}C’est devant le Seigneur ton Dieu, au lieu choisi par le Seigneur, que tu le mangeras, chaque année, toi et ceux de ta maison.
${}^{21}Mais si l’animal a une tare, s’il est boiteux ou aveugle, s’il a n’importe quel autre défaut grave, tu ne le sacrifieras pas au Seigneur ton Dieu ; 
${}^{22}tu le mangeras dans ta ville ; l’homme impur et celui qui est pur en mangeront l’un et l’autre, comme si c’était de la gazelle ou du cerf. 
${}^{23}Il n’y a que le sang que tu ne consommeras pas, tu le répandras sur la terre comme de l’eau.
      
         
      \bchapter{}
      \begin{verse}
${}^{1}Observe le mois des Épis et célèbre la Pâque pour le Seigneur ton Dieu, car c’est au mois des Épis que le Seigneur ton Dieu t’a fait sortir d’Égypte, durant la nuit. 
${}^{2}Tu feras le sacrifice de la Pâque pour le Seigneur ton Dieu avec du petit et du gros bétail, dans le lieu choisi par le Seigneur ton Dieu pour y faire demeurer son nom. 
${}^{3}Tu n’y mangeras pas de pain levé ; mais là, pendant sept jours, tu mangeras des pains sans levain – un pain de misère – car c’est en toute hâte que tu es sorti du pays d’Égypte : ceci, pour faire mémoire, tous les jours de ta vie, de ce jour où tu sortis du pays d’Égypte. 
${}^{4}Pendant sept jours, on ne verra pas chez toi de levain, sur tout ton territoire ; rien de la chair que tu auras sacrifiée le soir du premier jour ne sera conservé jusqu’au lendemain matin.
${}^{5}Tu ne pourras pas sacrifier la Pâque dans n’importe laquelle des villes que te donne le Seigneur ton Dieu, 
${}^{6}mais c’est au lieu choisi par le Seigneur ton Dieu pour y faire demeurer son nom que tu sacrifieras la Pâque, le soir, au coucher du soleil, au moment précis où tu sortis d’Égypte. 
${}^{7}Tu la feras cuire et tu la mangeras au lieu choisi par le Seigneur ton Dieu, puis tu t’en retourneras au matin, et tu regagneras tes tentes. 
${}^{8}Pendant six jours, tu mangeras des pains sans levain ; le septième jour, ce sera la clôture de la fête pour le Seigneur ton Dieu, et tu ne feras aucun travail.
${}^{9}Tu compteras sept semaines : dès que la faucille commence à couper les épis, tu commenceras à compter les sept semaines. 
${}^{10}Puis tu célébreras la fête des Semaines en l’honneur du Seigneur ton Dieu, avec l’offrande volontaire que fera ta main ; ton offrande sera à la mesure de la bénédiction du Seigneur ton Dieu. 
${}^{11}Tu te réjouiras en présence du Seigneur ton Dieu, au lieu choisi par le Seigneur ton Dieu pour y faire demeurer son nom, et avec toi se réjouiront ton fils et ta fille, ton serviteur et ta servante, le lévite qui réside dans ta ville, l’immigré, l’orphelin et la veuve qui sont au milieu de toi. 
${}^{12}Tu te souviendras que tu as été esclave au pays d’Égypte, et tu veilleras à pratiquer ces décrets.
${}^{13}La fête des Tentes, tu la célébreras pendant sept jours, lorsque tu auras rentré le produit de ton aire à grain et de ton pressoir. 
${}^{14}Tu la fêteras dans la joie, toi, ton fils et ta fille, ton serviteur et ta servante, le lévite et l’immigré, l’orphelin et la veuve qui résident dans ta ville. 
${}^{15}Durant sept jours, tu célébreras la fête en l’honneur du Seigneur ton Dieu au lieu choisi par le Seigneur ; car le Seigneur ton Dieu t’aura béni dans toutes tes récoltes et dans tous tes travaux, et tu seras rempli de joie.
${}^{16}Trois fois par an – à la fête des Pains sans levain, à la fête des Semaines et à la fête des Tentes –, tous les hommes paraîtront devant la face du Seigneur ton Dieu, au lieu qu’il aura choisi. Ils ne paraîtront pas les mains vides devant la face du Seigneur, 
${}^{17}mais chacun fera, de sa main, un don à la mesure de la bénédiction que le Seigneur ton Dieu aura donnée.
${}^{18}Dans toutes les villes que te donne le Seigneur ton Dieu, tu établiras des juges et des scribes sur tes tribus ; ils jugeront le peuple en de justes jugements. 
${}^{19}Tu ne feras pas dévier le droit, tu n’agiras pas avec partialité, et tu n’accepteras pas de présent, car le présent aveugle les sages et compromet la cause des justes. 
${}^{20}C’est la justice, rien que la justice, que tu rechercheras, afin de vivre et de prendre possession du pays que te donne le Seigneur ton Dieu.
${}^{21}Tu ne planteras pas de poteau sacré, de quelque bois que ce soit, à côté de l’autel du Seigneur ton Dieu, que tu auras bâti à ton usage, 
${}^{22}et tu ne dresseras pas de stèle : le Seigneur ton Dieu déteste cela.
      
         
      \bchapter{}
      \begin{verse}
${}^{1}Tu ne sacrifieras pas au Seigneur ton Dieu un bœuf ou un mouton qui ait une tare ou un défaut quelconque, car c’est une abomination pour le Seigneur ton Dieu.
${}^{2}Supposons qu’au milieu de toi, dans l’une des villes que le Seigneur ton Dieu te donne, un homme ou une femme fasse ce qui est mal aux yeux du Seigneur ton Dieu en transgressant son alliance, 
${}^{3}en allant servir d’autres dieux et se prosterner devant eux, devant le soleil, la lune ou toute l’armée des cieux – toutes choses que je n’ai pas commandées. 
${}^{4}Si on te le rapporte, si tu l’entends dire, tu feras une enquête approfondie. Si c’est vrai, si le fait est établi, si une telle abomination a été commise en Israël, 
${}^{5}tu feras sortir aux portes de ta ville cet homme ou cette femme coupable de cette action mauvaise ; tu lapideras l’homme ou la femme jusqu’à ce que mort s’ensuive.
${}^{6}C’est sur les déclarations de deux ou trois témoins que l’on pourra mettre à mort celui qui doit mourir ; on ne pourra pas mettre à mort sur la déclaration d’un seul témoin. 
${}^{7}Les témoins seront les premiers à lever la main contre le condamné pour le mettre à mort ; ensuite le peuple tout entier l’achèvera de ses mains. Tu ôteras le mal du milieu de toi.
${}^{8}Si une cause te semble trop difficile à juger, qu’il s’agisse de sang versé, de litige ou de blessures, et que cette affaire soit contestée au tribunal de ta ville, tu te lèveras et tu monteras au lieu choisi par le Seigneur ton Dieu ; 
${}^{9}tu iras trouver les prêtres lévites et le juge en fonction ces jours-là. Tu les consulteras et ils te feront connaître la sentence. 
${}^{10}Tu te conformeras à la sentence qu’ils t’auront fait connaître en ce lieu choisi par le Seigneur, et tu auras soin d’agir selon toutes leurs instructions. 
${}^{11}Suivant l’arrêt qu’ils auront rendu et la sentence qu’ils auront prononcée, tu agiras, sans dévier ni à droite ni à gauche de la parole qu’ils t’auront dite. 
${}^{12}Mais l’homme qui agit avec présomption, n’écoutant ni le prêtre qui se tient là pour le service du Seigneur ton Dieu, ni le juge, cet homme-là mourra. Tu ôteras le mal d’Israël. 
${}^{13}Le peuple tout entier l’apprendra et sera saisi de crainte ; il n’agira plus avec présomption.
${}^{14}Lorsque tu seras entré dans le pays que le Seigneur ton Dieu te donne, que tu en auras pris possession et que tu y habiteras, si tu dis : « Je veux établir sur moi un roi, comme toutes les nations d’alentour », 
${}^{15}tu devras établir sur toi un roi choisi par le Seigneur ton Dieu ; c’est parmi tes frères que tu prendras un roi pour l’établir sur toi ; tu ne pourras pas te donner un roi étranger qui ne serait pas l’un de tes frères.
${}^{16}Mais qu’il n’aille pas multiplier le nombre de ses chevaux, qu’il ne ramène pas le peuple en Égypte pour avoir beaucoup de chevaux, alors que le Seigneur vous a dit : « Vous n’en reprendrez plus jamais le chemin ! » 
${}^{17}Qu’il ne multiplie pas le nombre de ses femmes, de peur que son cœur ne dévie. Son argent et son or, qu’il ne les multiplie pas à l’excès !
${}^{18}Quand il montera sur son trône royal, il écrira pour lui-même, sur un livre, une copie de cette Loi en présence des prêtres lévites. 
${}^{19}Elle restera auprès de lui. Il la lira tous les jours de sa vie, afin d’apprendre à craindre le Seigneur son Dieu en gardant, pour les mettre en pratique, toutes les paroles de cette Loi et tous ces décrets. 
${}^{20}Alors, son cœur ne s’élèvera pas au-dessus de ses frères, il ne déviera du commandement ni à droite ni à gauche. Ainsi, lui et ses fils régneront de longs jours en Israël.
      
         
      \bchapter{}
      \begin{verse}
${}^{1}Les prêtres lévites, la tribu de Lévi tout entière, n’auront ni part ni héritage avec Israël : ils se nourriront des sacrifices par le feu, offerts au Seigneur, et de son patrimoine. 
${}^{2}Cette tribu n’aura pas d’héritage au milieu de ses frères, mais le Seigneur sera son héritage, comme il le lui a dit.
${}^{3}Voici quels seront les droits des prêtres sur le peuple, sur ceux qui offrent en sacrifice un bœuf ou un mouton : on donnera au prêtre l’épaule, les joues et la panse. 
${}^{4}Tu lui donneras les prémices de ton froment, de ton vin nouveau et de ton huile fraîche, ainsi que les prémices de la toison de ton petit bétail. 
${}^{5}Car c’est lui, le prêtre, que le Seigneur ton Dieu a choisi parmi toutes les tribus pour se tenir en sa présence et servir en son nom, lui et ses fils, tous les jours.
${}^{6}Si le lévite vient de l’une de tes villes, de tout endroit en Israël où il séjourne, et qu’il désire entrer au lieu choisi par le Seigneur, 
${}^{7}il servira au nom du Seigneur son Dieu comme tous ses frères lévites qui se tiennent là, en présence du Seigneur. 
${}^{8}Il aura pour se nourrir une part égale à la leur, outre ce que chacun pourra tirer de la vente des biens paternels.
${}^{9}Lorsque tu seras entré dans le pays que le Seigneur ton Dieu te donne, tu n’apprendras pas à commettre les abominations que commettent ces nations-là. 
${}^{10}On ne trouvera chez toi personne qui fasse passer son fils ou sa fille par le feu, personne qui scrute les présages, ou pratique astrologie, incantation, enchantement, 
${}^{11}personne qui use de magie, interroge les spectres et les esprits, ou consulte les morts. 
${}^{12}Car quiconque fait cela est en abomination pour le Seigneur, et c’est à cause de telles abominations que le Seigneur ton Dieu dépossède les nations devant toi. 
${}^{13}Toi, tu seras parfait à l’égard du Seigneur ton Dieu. 
${}^{14}Ces nations que tu vas déposséder écoutent les astrologues et ceux qui scrutent les présages. Mais à toi, ce n’est pas cela que t’a donné le Seigneur ton Dieu.
${}^{15}Au milieu de vous, parmi vos frères, le Seigneur votre Dieu fera se lever un prophète comme moi, et vous l’écouterez. 
${}^{16} C’est bien ce que vous avez demandé au Seigneur votre Dieu, au mont\\Horeb, le jour de l’assemblée, quand vous disiez : « Je ne veux plus entendre la voix du Seigneur mon Dieu, je ne veux plus voir cette grande flamme, je ne veux pas mourir ! » 
${}^{17} Et le Seigneur me dit alors : « Ils ont bien fait de dire cela. 
${}^{18} Je ferai se lever au milieu de leurs frères un prophète comme toi ; je mettrai dans sa bouche mes paroles, et il leur dira tout ce que je lui prescrirai. 
${}^{19} Si quelqu’un n’écoute pas les paroles que ce prophète prononcera en mon nom, moi-même je lui en demanderai compte. 
${}^{20} Mais un prophète qui aurait la présomption\\de dire en mon nom une parole que je ne lui aurais pas prescrite, ou qui parlerait au nom d’autres dieux, ce prophète-là mourra. »
${}^{21}Peut-être te demanderas-tu : « Comment reconnaîtrons-nous que le Seigneur n’a pas dit cette parole ? » 
${}^{22}Si le prophète parle au nom du Seigneur, et que la parole reste sans effet et ne s’accomplit pas, alors le Seigneur n’a pas dit cette parole : le prophète l’a dite avec présomption. Tu ne dois pas en avoir peur !
      
         
      \bchapter{}
      \begin{verse}
${}^{1}Lorsque le Seigneur ton Dieu aura retranché les nations dont il te donne le pays, lorsque tu les auras dépossédées et que tu habiteras leurs villes et leurs maisons, 
${}^{2}alors, tu mettras à part trois villes au milieu du pays que le Seigneur ton Dieu te donne en possession. 
${}^{3}Tu entretiendras leurs routes d’accès et tu diviseras en trois le territoire de ton pays, celui que le Seigneur ton Dieu t’aura donné en héritage. Ainsi, tout meurtrier pourra se réfugier dans l’une de ces villes.
${}^{4}Voici dans quel cas un meurtrier pourra s’y réfugier et avoir la vie sauve : un homme qui frappe son prochain par méprise, sans avoir jamais éprouvé de haine contre lui. 
${}^{5}Par exemple, un homme va dans la forêt avec un autre pour abattre des arbres : sa main brandit la hache pour couper du bois, mais le fer s’échappe du manche et atteint son compagnon qui en meurt ; cet homme pourra se réfugier dans l’une de ces villes et il aura la vie sauve. 
${}^{6}Autrement, le vengeur du sang, dans sa fureur, poursuivrait le meurtrier, le rattraperait – si la route est longue – et le frapperait à mort. En effet, ce meurtrier n’encourt pas la peine de mort, puisqu’il n’a jamais éprouvé de haine contre son compagnon. 
${}^{7}C’est pourquoi je te le commande : « Tu mettras à part trois villes. »
${}^{8}Et si le Seigneur ton Dieu élargit ton territoire, comme il l’a juré à tes pères, et te donne tout le pays qu’il leur a promis 
${}^{9}– pourvu que tu observes et pratiques tout ce commandement que je te prescris aujourd’hui : aimer le Seigneur et marcher tous les jours sur ses chemins – si donc le Seigneur élargit ton territoire, tu ajouteras encore trois villes aux trois premières. 
${}^{10}Ainsi, le sang innocent ne sera pas versé au milieu du pays que le Seigneur ton Dieu te donne en héritage : ce sang retomberait sur toi.
${}^{11}Mais si un homme éprouve de la haine contre son prochain, le guette, se dresse contre lui et le frappe à mort, puis se réfugie dans l’une de ces villes, 
${}^{12}alors, les anciens de sa ville enverront quelqu’un pour se saisir de lui, ils le livreront entre les mains du vengeur du sang et il mourra. 
${}^{13}Tu n’auras pas pour lui un regard de pitié. Tu feras disparaître d’Israël l’effusion du sang innocent, et cela pour ton bien.
${}^{14}Tu ne déplaceras pas les bornes de ton voisin, celles qu’ont posées les premiers arrivés dans le pays que tu as reçu en héritage, ce pays que le Seigneur ton Dieu te donne en possession.
${}^{15}Il ne suffira pas qu’un seul témoin se lève contre un homme coupable d’un crime, d’une faute, d’un péché, quels qu’ils soient. Pour instruire l’affaire, il faudra la déclaration de deux ou trois témoins.
${}^{16}Lorsqu’un témoin à charge se lève contre un homme pour l’accuser de quelque forfait, 
${}^{17}les deux hommes qui sont en litige se tiendront en présence du Seigneur, devant les prêtres et les juges qui seront en fonction ces jours-là. 
${}^{18}Les juges feront une enquête approfondie. S’il s’agit d’un faux témoin, s’il a accusé faussement son frère, 
${}^{19}vous le traiterez comme il avait l’intention de traiter son frère. Tu ôteras le mal du milieu de toi. 
${}^{20}Les autres en entendront parler et seront saisis de crainte. Alors, on cessera de commettre ce genre d’action mauvaise au milieu de toi. 
${}^{21}Tu n’auras pas un regard de pitié : vie pour vie, œil pour œil, dent pour dent, main pour main, pied pour pied.
      
         
      \bchapter{}
      \begin{verse}
${}^{1}Lorsque tu partiras en guerre contre tes ennemis et que tu verras des chevaux, des chars, un peuple plus nombreux que toi, tu ne les craindras pas, car le Seigneur ton Dieu est avec toi, lui qui t’a fait monter du pays d’Égypte.
${}^{2}Quand vous serez sur le point de combattre, le prêtre s’avancera et parlera au peuple. 
${}^{3}Il lui dira : « Écoute, Israël ! Vous êtes aujourd’hui sur le point de combattre vos ennemis. Que votre courage ne faiblisse pas ! N’ayez pas peur, ne vous affolez pas, ne tremblez pas devant eux, 
${}^{4}car le Seigneur votre Dieu marche avec vous, afin de combattre pour vous et de vous sauver ! »
${}^{5}Ensuite les scribes parleront au peuple en ces termes : « Y a-t-il un homme qui a construit une maison neuve et ne l’a pas encore inaugurée ? Qu’il s’en aille et retourne chez lui, de peur qu’il ne meure au combat et qu’un autre n’inaugure sa maison. 
${}^{6}Y a-t-il un homme qui a planté une vigne et n’en a pas profité ? Qu’il s’en aille et retourne chez lui, de peur qu’il ne meure au combat et qu’un autre n’en profite. 
${}^{7}Y a-t-il un homme qui a choisi une fiancée et ne l’a pas encore épousée ? Qu’il s’en aille et retourne chez lui, de peur qu’il ne meure au combat et qu’un autre ne l’épouse. »
${}^{8}Les scribes diront encore au peuple : « Y a-t-il un homme qui a peur et dont le courage faiblit ? Qu’il s’en aille et retourne chez lui, de peur que le courage de ses frères à cause de lui ne fonde comme le sien. » 
${}^{9}Quand les scribes auront fini de parler, ils placeront des officiers à la tête du peuple.
${}^{10}Lorsque tu t’approcheras d’une ville pour la combattre, tu lui proposeras la paix. 
${}^{11}Si elle accepte la paix et t’ouvre ses portes, toute la population qui s’y trouve sera astreinte à la corvée et te servira. 
${}^{12}Mais si elle refuse la paix et engage le combat, tu l’assiégeras. 
${}^{13}Le Seigneur ton Dieu la livrera entre tes mains, et tu passeras tous les hommes au fil de l’épée. 
${}^{14}Quant aux femmes, aux enfants, au bétail, tout ce qui se trouve dans la ville, tout le butin, tu t’en saisiras ; tu te nourriras du butin pris aux ennemis que le Seigneur ton Dieu t’aura livrés. 
${}^{15}Tu agiras ainsi envers toutes les villes très éloignées de toi, villes qui n’appartiennent pas aux nations que voici.
${}^{16}Dans les seules villes des peuples que le Seigneur ton Dieu te donne en héritage, tu ne laisseras subsister aucun être vivant. 
${}^{17}En effet, tu dois vouer à l’anathème le Hittite, l’Amorite, le Cananéen, le Perizzite, le Hivvite et le Jébuséen, selon l’ordre du Seigneur ton Dieu, 
${}^{18}afin qu’ils ne vous apprennent pas à pratiquer toutes les abominations qu’ils pratiquent envers leurs dieux : ce serait pécher contre le Seigneur votre Dieu.
${}^{19}Lorsque tu assiégeras une ville pendant de longs jours, en combattant contre elle pour la conquérir, tu ne brandiras pas la hache sur les arbres pour les abattre, car c’est eux qui te nourriront. Tu ne les couperas donc pas : l’arbre des champs est-il un homme pour que tu le traites en assiégé ? 
${}^{20}Seul l’arbre que tu reconnaîtras comme n’étant pas un arbre fruitier, tu pourras l’abattre, le couper, et tu en feras des machines de guerre contre la ville que tu combats, jusqu’à sa reddition.
      
         
      \bchapter{}
      \begin{verse}
${}^{1}Lorsque, sur la terre que le Seigneur ton Dieu te donne en possession, on trouvera, gisant dans la campagne, une victime dont l’assassin est inconnu, 
${}^{2}tes anciens et tes juges sortiront, ils mesureront la distance entre la victime et les villes proches. 
${}^{3}On déterminera la ville la plus proche de la victime. Alors, les anciens de cette ville prendront une génisse qui n’a jamais servi, qui n’a jamais été sous le joug. 
${}^{4}Ils la feront descendre vers un torrent intarissable, dans un endroit ni labouré ni ensemencé. Et là, dans le torrent, ils briseront la nuque de la génisse.
${}^{5}Alors s’avanceront les prêtres, fils de Lévi, car c’est eux que le Seigneur ton Dieu a choisis pour le servir et bénir en son nom, c’est sur leur déclaration que se règlent tout litige et tout échange de coups. 
${}^{6}Puis, tous les anciens de cette ville qui se sont approchés de la victime se laveront les mains au-dessus de la génisse dont la nuque a été brisée dans le torrent. 
${}^{7}Et ils déclareront : « Nos mains n’ont pas répandu le sang de cet homme et nos yeux n’ont rien vu. 
${}^{8}Absous, ô Seigneur, ton peuple Israël que tu as racheté. Ne laisse pas au milieu de ton peuple Israël le sang innocent. » Et ils seront absous du sang. 
${}^{9}Toi, tu feras disparaître d’Israël l’effusion du sang innocent ; ainsi tu feras ce qui est droit aux yeux du Seigneur.
${}^{10}Lorsque tu partiras en guerre contre tes ennemis et que le Seigneur ton Dieu les aura livrés entre tes mains, il t’arrivera de faire des prisonniers. 
${}^{11}Peut-être remarqueras-tu, parmi les captives, une femme de belle apparence, dont tu t’éprendras ; si tu veux l’épouser 
${}^{12}et la faire entrer dans ta maison, alors elle se rasera la tête, se coupera les ongles, 
${}^{13}retirera son manteau de captive et habitera dans ta maison. Elle pleurera son père et sa mère pendant un mois ; après quoi, tu t’approcheras d’elle, tu l’épouseras et elle deviendra ta femme. 
${}^{14}S’il arrive que tu ne la désires plus, tu la laisseras partir où elle voudra ; tu ne pourras absolument pas la vendre ni en tirer profit, puisque tu lui as fait violence.
${}^{15}Lorsqu’un homme a deux femmes, il peut arriver qu’il aime l’une et n’aime pas l’autre, et que toutes les deux lui donnent des fils. Si l’aîné est le fils de la femme qu’il n’aime pas, 
${}^{16}le jour où cet homme partagera son héritage entre ses fils, il ne pourra pas traiter comme un premier-né le fils de la femme qu’il aime, au détriment de l’aîné, fils de la femme qu’il n’aime pas. 
${}^{17}C’est bien l’aîné, fils de la femme qu’il n’aime pas, qu’il reconnaîtra comme tel, en lui donnant double part de ses biens, car c’est lui les prémices de sa virilité, c’est à lui qu’appartient le droit d’aînesse.
${}^{18}Lorsqu’un homme a un fils rebelle et obstiné qui n’écoute ni son père ni sa mère, s’ils lui font la leçon et qu’il ne les écoute toujours pas, 
${}^{19}alors, son père et sa mère se saisiront de lui et le conduiront auprès des anciens de la ville, au lieu désigné. 
${}^{20}Et ils diront aux anciens de la ville : « Notre fils que voici est rebelle et obstiné, il ne nous écoute pas, c’est un débauché, un buveur ! » 
${}^{21}Alors tous les hommes de la ville le lapideront jusqu’à ce que mort s’ensuive. Ainsi tu ôteras le mal du milieu de toi. Tout Israël l’apprendra et sera dans la crainte.
${}^{22}Lorsqu’un homme ayant commis une faute passible de mort a été condamné à mort et pendu à un arbre, 
${}^{23}on ne laissera pas son cadavre sur l’arbre durant la nuit. Tu devras le mettre au tombeau le jour même, car un pendu est une malédiction de Dieu. Ainsi tu ne rendras pas impur le sol que le Seigneur ton Dieu te donne en héritage.
      
         
      \bchapter{}
      \begin{verse}
${}^{1}Si tu vois errer le bœuf ou le mouton de ton frère, tu ne t’en détourneras pas, tu dois le ramener à ton frère. 
${}^{2}Si ton frère n’est pas de ton voisinage et si tu ne le connais pas, recueille la bête dans ta maison ; qu’elle reste avec toi jusqu’à ce que ton frère vienne la réclamer ; alors, tu la lui rendras. 
${}^{3}Tu agiras de même en ce qui concerne son âne, son manteau, tout objet que ton frère aurait perdu et que tu aurais trouvé. Tu ne pourras pas te détourner. 
${}^{4}Si tu vois l’âne ou le bœuf de ton frère tomber sur le chemin, tu ne t’en détourneras pas : tu dois l’aider à relever la bête.
${}^{5}Une femme ne portera pas un costume d’homme, et un homme ne revêtira pas un vêtement de femme : quiconque fait cela est une abomination pour le Seigneur ton Dieu.
${}^{6}Lorsque tu trouves, en chemin, sur un arbre ou par terre, un nid avec des oisillons ou des œufs que la mère est en train de couver, tu ne retireras pas la mère de ses petits. 
${}^{7}Tu devras laisser la mère s’en aller et tu n’emporteras que les petits. Ainsi tu seras heureux et tu auras de longs jours.
${}^{8}Lorsque tu bâtis une maison neuve, construis une balustrade autour du toit : si quelqu’un venait à en tomber, tu ne serais pas responsable du sang répandu sur ta maison.
${}^{9}Tu ne sèmeras pas dans ta vigne une autre espèce de plante, de peur que le tout ne soit consacré : à la fois la graine semée et le produit de la vigne. 
${}^{10}Tu ne laboureras pas avec un bœuf et un âne attelés ensemble. 
${}^{11}Tu ne t’habilleras pas d’un tissu mêlé, fait de laine et de lin ensemble.
${}^{12}Tu mettras des franges aux quatre côtés du vêtement dont tu te couvriras.
${}^{13}Lorsqu’un homme a pris une femme, s’est uni à elle, puis se met à la détester, 
${}^{14}s’il l’accuse d’actions scandaleuses et lui fait une mauvaise réputation en disant : « Cette femme, je l’ai prise, je me suis approché d’elle, mais je ne l’ai pas trouvée vierge », 
${}^{15}alors le père et la mère de la jeune femme produiront les signes de sa virginité et les présenteront aux anciens à la porte de la ville.
${}^{16}Le père de la jeune femme dira aux anciens : « Ma fille, je l’ai donnée pour épouse à cet homme, mais il s’est mis à la détester ; 
${}^{17}et maintenant il l’accuse d’actions scandaleuses en disant : “Je ne l’ai pas trouvée vierge !” Or voici la preuve de sa virginité. » Et les parents déploieront le drap des noces devant les anciens de la ville. 
${}^{18}Alors, les anciens de cette ville se saisiront de l’homme pour le punir. 
${}^{19}Ils lui infligeront une amende de cent pièces d’argent qu’ils donneront au père de la jeune femme, car cet homme a sali la réputation d’une vierge d’Israël. Elle restera sa femme et il ne pourra la renvoyer, tant qu’il vivra.
${}^{20}Mais si la chose se révèle exacte, si on ne peut montrer la preuve de la virginité de la jeune femme, 
${}^{21}on l’amènera à la porte de la maison de son père. Les hommes de la ville la lapideront jusqu’à ce que mort s’ensuive, car elle a commis une infamie en Israël, en se prostituant dans la maison de son père. Tu ôteras le mal du milieu de toi.
${}^{22}Lorsqu’on trouvera un homme couché avec une femme mariée, ils mourront tous deux, l’homme qui a couché avec la femme, et la femme également. Tu ôteras le mal du milieu d’Israël.
${}^{23}Lorsqu’une jeune fille vierge est fiancée à un homme, si un autre homme la rencontre dans la ville et couche avec elle, 
${}^{24}vous les amènerez tous les deux à la porte de cette ville et vous les lapiderez jusqu’à ce que mort s’ensuive : la jeune fille, parce que, étant dans la ville, elle n’a pas crié au secours ; l’homme, parce qu’il a abusé de la femme de son prochain. Tu ôteras le mal du milieu de toi.
${}^{25}Mais si c’est en pleine campagne que l’homme rencontre la jeune fiancée, qu’il la violente et couche avec elle, l’homme, seul, mourra. 
${}^{26}À la jeune fille, tu ne feras rien. Elle n’a pas commis un péché qui mérite la mort. Le cas est le même que si un homme se jette sur son prochain et l’assassine. 
${}^{27}Puisque c’est en pleine campagne que l’homme a rencontré la jeune fiancée, elle a eu beau crier, il n’y avait personne pour la secourir.
${}^{28}Lorsqu’un homme rencontre une jeune fille vierge qui n’est pas fiancée, la saisit et couche avec elle, si on les prend sur le fait, 
${}^{29}l’homme donnera cinquante pièces d’argent au père de la jeune fille. Puisqu’il a abusé d’elle, elle deviendra sa femme et il ne pourra la renvoyer tant qu’il vivra.
      
         
      \bchapter{}
      \begin{verse}
${}^{1}Un homme ne prendra pas la femme de son père ; il ne portera pas atteinte aux droits de son père.
      
         
${}^{2}L’homme aux testicules écrasés et l’homme à la verge coupée n’entreront pas dans l’assemblée du Seigneur.
${}^{3}Un bâtard n’entrera pas dans l’assemblée du Seigneur ; même à la dixième génération, il n’y entrera pas.
${}^{4}L’Ammonite et le Moabite n’entreront pas dans l’assemblée du Seigneur, même à la dixième génération ; ils n’y entreront jamais. 
${}^{5}Cela, parce qu’ils ne sont pas venus au-devant de vous avec le pain et l’eau, sur la route, quand vous sortiez d’Égypte, et parce que le Moabite a soudoyé contre toi, pour te maudire, Balaam, fils de Béor, de Petor en Aram-des-deux-fleuves. 
${}^{6}Mais le Seigneur ton Dieu a refusé d’écouter Balaam ; le Seigneur ton Dieu a changé pour toi la malédiction en bénédiction, car il t’aime, le Seigneur ton Dieu. 
${}^{7}Jamais, tant que tu vivras, tu ne t’inquiéteras de leur prospérité ni de leur bonheur.
${}^{8}Tu ne prendras pas en abomination l’Édomite, car c’est ton frère. Et tu ne prendras pas en abomination l’Égyptien, car tu as été un immigré dans son pays. 
${}^{9}À la troisième génération, leurs fils pourront entrer dans l’assemblée du Seigneur.
${}^{10}Lorsque tu partiras en campagne contre tes ennemis, tu te garderas de toute souillure. 
${}^{11}S’il y a parmi vous un homme qui n’est pas pur, à la suite d’une pollution nocturne, il sortira du camp et n’y rentrera pas. 
${}^{12}Mais, à l’approche du soir, il se baignera dans l’eau et, au coucher du soleil, il rentrera dans le camp.
${}^{13}En dehors du camp, tu disposeras d’un coin où te mettre à l’écart. 
${}^{14}Dans ton équipement, tu auras une petite pioche, et quand tu t’accroupiras à l’écart, tu t’en serviras pour creuser ; puis tu recouvriras tes excréments.
${}^{15}Car le Seigneur ton Dieu va et vient à l’intérieur du camp pour te protéger et te livrer tes ennemis. Ton camp sera donc saint, il ne faut pas que le Seigneur voie quelque chose d’inconvenant ; sinon, il se détournerait de toi.
${}^{16}Tu ne livreras pas à son maître un esclave qui s’est sauvé de chez son maître et s’est réfugié auprès de toi. 
${}^{17}Il habitera avec toi, chez toi, au lieu qu’il aura choisi, dans l’une de tes villes, où bon lui semblera. Tu ne l’exploiteras pas.
${}^{18}Il n’y aura pas de courtisane sacrée parmi les filles d’Israël, ni de prostitué sacré parmi les fils d’Israël. 
${}^{19}Tu n’apporteras jamais dans la maison du Seigneur ton Dieu, pour t’acquitter d’un vœu quelconque, le gain d’une prostituée ou le salaire d’un « chien », car tous deux sont une abomination pour le Seigneur ton Dieu.
${}^{20}Tu ne feras à ton frère aucun prêt à intérêt : qu’il s’agisse d’argent, de nourriture ou de quoi que ce soit qui puisse rapporter des intérêts.
${}^{21}À un étranger, tu pourras prêter à intérêt, mais pas à ton frère, afin que le Seigneur ton Dieu te bénisse dans toutes tes entreprises sur la terre où tu vas entrer pour en prendre possession. 
${}^{22}Lorsque tu fais un vœu au Seigneur ton Dieu, tu ne tarderas pas à l’accomplir, car le Seigneur ton Dieu t’en demanderait certainement compte ; tu te chargerais d’un péché. 
${}^{23}Mais si tu t’abstiens de prononcer un vœu, tu ne te chargeras pas d’un péché. 
${}^{24}Ce qui sort de tes lèvres, veille à le mettre en pratique, exécute le vœu que tu as fait spontanément au Seigneur ton Dieu et que tu as formulé de ta propre bouche.
${}^{25}Lorsque tu entreras dans la vigne de ton voisin, tu pourras manger du raisin autant que tu veux, jusqu’à satiété, mais tu ne pourras pas en emporter dans ton panier. 
${}^{26}Lorsque tu entreras dans le champ de blé mûr de ton voisin, tu pourras cueillir des épis avec la main, mais tu n’y mettras pas la faucille !
      
         
      \bchapter{}
      \begin{verse}
${}^{1}Lorsqu’un homme prend une femme et l’épouse, et qu’elle cesse de trouver grâce à ses yeux, parce qu’il découvre en elle une tare, il lui écrira une lettre de répudiation et la lui remettra en la renvoyant de sa maison. 
${}^{2}Si cette femme, après avoir quitté la maison, épouse un autre homme, 
${}^{3}si ce deuxième homme se met lui aussi à la détester, lui écrit une lettre de répudiation, la lui remet et la renvoie de sa maison – ou encore si ce deuxième homme vient à mourir –, 
${}^{4}son premier mari ne peut la reprendre pour femme, du fait qu’elle aura été rendue impure. Car elle serait une abomination devant le Seigneur. Tu n’entraîneras pas dans le péché le pays que le Seigneur ton Dieu te donne en héritage.
      
         
${}^{5}Lorsqu’un homme vient de prendre femme, il n’ira pas à l’armée, on ne le chargera d’aucune affaire ; il sera exempté de tout pour rester à la maison pendant un an ; il fera la joie de la femme qu’il a épousée.
${}^{6}On ne prendra pas en gage le moulin, ni même une seule meule, car ce serait prendre en gage la vie même.
${}^{7}S’il se trouve qu’un homme enlève un de ses frères parmi les fils d’Israël, qu’il veuille en tirer profit et le vende comme esclave, l’auteur du rapt mourra. Tu ôteras le mal du milieu de toi.
${}^{8}En cas de lèpre, veille à observer scrupuleusement et à mettre en pratique tout ce que vous enseigneront les prêtres lévites ; veillez à agir selon ce que je leur ai commandé. 
${}^{9}Souviens-toi de ce que le Seigneur ton Dieu a fait à Miryam, sur la route, quand vous êtes sortis d’Égypte.
${}^{10}Lorsque tu fais à ton prochain un prêt quelconque, tu n’entreras pas dans sa maison pour lui prendre un gage. 
${}^{11}Tu resteras dehors et l’homme à qui tu prêtes sortira pour te l’apporter. 
${}^{12}Si c’est un pauvre, tu ne te coucheras pas en gardant son gage. 
${}^{13}Tu devras le lui rapporter au coucher du soleil : il se couchera dans son manteau et te bénira. Et tu seras juste devant le Seigneur ton Dieu.
${}^{14}Tu n’exploiteras pas un salarié pauvre et malheureux, que ce soit l’un de tes frères, ou un immigré qui réside dans ton pays, dans ta ville. 
${}^{15}Le jour même, tu lui donneras son salaire. Que le soleil ne se couche pas sur cette dette, car c’est un pauvre, il attend impatiemment son dû. Ainsi, il ne criera pas contre toi vers le Seigneur, et tu ne te chargeras pas d’un péché.
${}^{16}Les pères ne seront pas mis à mort à la place des fils, les fils ne seront pas mis à mort à la place des pères : chacun sera mis à mort pour son propre péché.
${}^{17}Tu ne feras pas dévier le droit\\de l’immigré ni celui de l’orphelin, et tu ne feras pas saisir comme gage le manteau de la veuve. 
${}^{18} Souviens-toi que tu as été esclave en Égypte et que le Seigneur ton Dieu t’a racheté\\. Voilà pourquoi je te donne ce commandement\\.
${}^{19}Lorsque tu feras ta moisson\\, si tu oublies une gerbe dans ton champ, tu ne retourneras pas la chercher. Laisse-la\\pour l’immigré, l’orphelin et la veuve, afin que le Seigneur ton Dieu te bénisse dans tous tes travaux\\. 
${}^{20} Lorsque tu auras récolté tes olives\\, tu ne retourneras pas chercher ce qui reste. Laisse-le pour l’immigré, l’orphelin et la veuve. 
${}^{21} Lorsque tu vendangeras ta vigne, tu ne retourneras pas grappiller ce qui reste. Laisse-le pour l’immigré, l’orphelin et la veuve. 
${}^{22} Souviens-toi que tu as été esclave au pays d’Égypte. Voilà pourquoi je te donne ce commandement\\.
      
         
      \bchapter{}
      \begin{verse}
${}^{1}Lorsque des hommes ont un litige, ils iront en justice et seront jugés : l’innocent sera déclaré innocent, et le coupable, coupable.
${}^{2}Si le coupable mérite d’être frappé, le juge le fera se coucher par terre et lui fera donner, en sa présence, le nombre de coups proportionné à sa faute. 
${}^{3}On pourra lui donner quarante coups, mais pas davantage, de peur de provoquer une blessure grave et d’avilir ainsi ton frère à tes yeux.
${}^{4}Tu ne mettras pas de muselière au bœuf qui foule le grain.
${}^{5}Lorsque des frères habitent ensemble, si l’un d’eux meurt sans avoir de fils, l’épouse du défunt ne pourra pas appartenir à quelqu’un d’étranger à la famille ; son beau-frère viendra vers elle et la prendra pour femme ; il accomplira ainsi envers elle son devoir de beau-frère. 
${}^{6}Le premier-né qu’elle mettra au monde perpétuera le nom du frère défunt ; ainsi, ce nom ne sera pas effacé d’Israël.
${}^{7}Mais si l’homme ne désire pas épouser sa belle-sœur, celle-ci ira trouver les anciens à la porte de la ville et leur dira : « Mon beau-frère refuse de perpétuer le nom de son frère en Israël ; il ne veut pas accomplir envers moi son devoir de beau-frère. » 
${}^{8}Les anciens de la ville le convoqueront et lui parleront. Il se tiendra devant eux et dira : « Je ne veux pas épouser ma belle-sœur. » 
${}^{9}Alors sa belle-sœur s’avancera vers lui, sous les yeux des anciens ; elle lui retirera la sandale du pied et lui crachera au visage ; puis elle déclarera : « C’est ainsi que l’on traite l’homme qui ne rebâtit pas la maison de son frère. » 
${}^{10}Et dorénavant, en Israël, on l’appellera : « Maison du déchaussé ».
${}^{11}Lorsque deux fils d’Israël se battent ensemble, et que la femme de l’un d’eux s’approche pour soustraire son mari aux coups de l’autre, si elle étend la main et saisit ce dernier par les testicules, 
${}^{12}tu lui couperas la main. Tu n’auras pas un regard de pitié.
${}^{13}Dans ta sacoche, tu n’auras pas deux poids différents, un grand et un petit. 
${}^{14}Dans ta maison, tu n’auras pas deux boisseaux différents, un grand et un petit. 
${}^{15}Tu auras un poids exact et juste, un boisseau exact et juste, afin d’avoir de longs jours sur la terre que te donne le Seigneur ton Dieu. 
${}^{16}Car tout homme qui agit de manière malhonnête est une abomination pour le Seigneur ton Dieu.
${}^{17}Souviens-toi de ce que t’a fait subir Amalec, sur la route, quand vous êtes sortis d’Égypte. 
${}^{18}Il t’a rejoint sur la route et a massacré tous ceux qui traînaient à l’arrière, alors que tu étais fourbu, exténué. Il n’a pas craint Dieu ! 
${}^{19}Aussi, quand le Seigneur ton Dieu t’aura dégagé de tous tes ennemis d’alentour et t’aura procuré le repos, dans le pays que le Seigneur ton Dieu te donne en héritage pour en prendre possession, alors tu effaceras le souvenir d’Amalec de dessous les cieux. N’oublie pas !
      
         
      \bchapter{}
      \begin{verse}
${}^{1}Lorsque tu seras entré dans le pays que te donne en héritage le Seigneur ton Dieu, quand tu le posséderas et y habiteras, 
${}^{2}tu prendras une part des prémices de tous les fruits de ton sol, les fruits que tu auras tirés de ce pays que te donne le Seigneur ton Dieu, et tu les mettras dans une corbeille. Tu te rendras au lieu que le Seigneur ton Dieu aura choisi pour y faire demeurer son nom. 
${}^{3}Tu iras trouver le prêtre en fonction ces jours-là et tu lui diras : « Je le déclare aujourd’hui au Seigneur ton Dieu : je suis entré dans le pays que le Seigneur a juré à nos pères de nous donner. »
${}^{4}Le prêtre recevra de tes mains la corbeille et la déposera devant l’autel du Seigneur ton Dieu. 
${}^{5} Tu prononceras ces paroles devant le Seigneur ton Dieu :
      « Mon père était un Araméen nomade, qui descendit en Égypte : il y vécut en immigré avec son petit clan. C’est là qu’il est devenu une grande nation, puissante et nombreuse. 
${}^{6} Les Égyptiens nous ont maltraités, et réduits à la pauvreté ; ils nous ont imposé un dur esclavage. 
${}^{7} Nous avons crié vers le Seigneur, le Dieu de nos pères. Il a entendu notre voix, il a vu que nous étions dans la misère, la peine et l’oppression. 
${}^{8} Le Seigneur nous a fait sortir d’Égypte à main forte et à bras étendu, par des actions terrifiantes, des signes et des prodiges. 
${}^{9} Il nous a conduits dans ce lieu et nous a donné ce pays, un pays ruisselant de lait et de miel. 
${}^{10} Et maintenant voici que j’apporte les prémices des fruits du sol que tu m’as donné, Seigneur. »
      Ensuite tu les déposeras devant le Seigneur ton Dieu et tu te prosterneras devant lui. 
${}^{11}Alors tu te réjouiras pour tous les biens que le Seigneur ton Dieu t’a donnés, à toi et à ta maison. Avec toi se réjouiront le lévite, et l’immigré qui réside chez toi.
${}^{12}La troisième année, l’année de la dîme, après avoir achevé de prélever toute la dîme de ta récolte, tu la distribueras au lévite, à l’immigré, à l’orphelin et à la veuve ; ils mangeront à satiété dans ta ville. 
${}^{13}Alors, en présence du Seigneur ton Dieu, tu diras :
      « J’ai retiré de ma maison ce qui est consacré et je l’ai distribué au lévite, à l’immigré, à l’orphelin et à la veuve, selon le commandement que tu m’as donné. Je n’ai pas transgressé tes commandements, je n’ai rien oublié. 
${}^{14}Quand j’étais en deuil, je n’ai pas mangé de ce qui est consacré ; je ne l’ai pas retiré de ma maison pour un usage impur, je ne l’ai pas donné à un mort. J’ai écouté la voix du Seigneur mon Dieu. J’ai agi selon tous tes commandements. 
${}^{15}De ta sainte demeure, du haut du ciel, regarde : bénis ton peuple Israël et la terre que tu nous as donnée selon le serment fait à nos pères ; c’est un pays ruisselant de lait et de miel. »
${}^{16}Aujourd’hui le Seigneur ton Dieu te commande de mettre en pratique ces décrets et ces ordonnances. Tu veilleras à les pratiquer de tout ton cœur et de toute ton âme.
${}^{17}Aujourd’hui tu as obtenu du Seigneur cette déclaration : lui sera ton Dieu ; toi, tu suivras ses chemins, tu garderas ses décrets, ses commandements et ses ordonnances, tu écouteras sa voix.
${}^{18}Aujourd’hui le Seigneur a obtenu de toi cette déclaration : tu seras son peuple, son domaine particulier, comme il te l’a dit, tu devras garder tous ses commandements. 
${}^{19} Il te fera dépasser en prestige, renommée et gloire toutes les nations qu’il a faites, et tu seras un peuple consacré au Seigneur ton Dieu, comme il l’a dit.
      
         
      \bchapter{}
      \begin{verse}
${}^{1}Moïse, avec tous les anciens d’Israël, ordonna au peuple : « Gardez tout le commandement que je vous donne aujourd’hui. 
${}^{2}Le jour où vous passerez le Jourdain pour vous rendre au pays que le Seigneur ton Dieu te donne, tu dresseras de grandes pierres et tu les enduiras de chaux. 
${}^{3}Tu écriras sur ces pierres toutes les paroles de cette Loi, lors de ton passage, pour entrer dans le pays que le Seigneur ton Dieu te donne, un pays ruisselant de lait et de miel, comme te l’a déclaré le Seigneur, le Dieu de tes pères. 
${}^{4}Quand vous aurez passé le Jourdain, vous dresserez ces pierres sur le mont Ébal, comme je vous le commande aujourd’hui, et tu les enduiras de chaux. 
${}^{5}Là, tu bâtiras un autel au Seigneur ton Dieu, un autel de pierres que tu n’auras pas travaillées avec le fer. 
${}^{6}C’est avec des pierres brutes que tu bâtiras l’autel du Seigneur ton Dieu ; sur cet autel, tu offriras des holocaustes au Seigneur ton Dieu. 
${}^{7}Tu offriras aussi des sacrifices de paix, et là, tu mangeras, tu te réjouiras en présence du Seigneur ton Dieu. 
${}^{8}Puis, tu écriras sur les pierres toutes les paroles de cette Loi, bien lisiblement. »
${}^{9}Et Moïse, avec les prêtres lévites, parla à tout Israël et dit : « Fais silence, Israël, écoute ! Aujourd’hui, tu es devenu un peuple pour le Seigneur ton Dieu. 
${}^{10}Tu écouteras la voix du Seigneur ton Dieu ; tu mettras en pratique ses commandements et ses décrets que je te donne aujourd’hui. »
${}^{11}Ce jour-là, Moïse ordonna au peuple : 
${}^{12}« Se tiendront sur le mont Garizim pour bénir le peuple quand vous aurez passé le Jourdain : Siméon, Lévi, Juda, Issakar, Joseph et Benjamin. 
${}^{13}Se tiendront sur le mont Ébal pour la malédiction : Roubène, Gad, Asher, Zabulon, Dane et Nephtali. »
${}^{14}Devant tous les hommes d’Israël, les Lévites proclameront à haute voix :
${}^{15}« Maudit soit l’homme qui fabrique une idole ou une statue en métal fondu et la place en un lieu secret. Abomination pour le Seigneur que cette œuvre sortie des mains d’un artisan ! » Et tout le peuple répondra et dira : « Amen. »
${}^{16}« Maudit qui méprise son père et sa mère ! » Et tout le peuple dira : « Amen. »
${}^{17}« Maudit qui déplace les bornes du terrain de son voisin ! » Et tout le peuple dira : « Amen. »
${}^{18}« Maudit qui fait perdre son chemin à l’aveugle ! » Et tout le peuple dira : « Amen. »
${}^{19}« Maudit qui fait dévier le droit de l’immigré, de l’orphelin, de la veuve ! » Et tout le peuple dira : « Amen. »
${}^{20}« Maudit qui couche avec la femme de son père, car il porte atteinte aux droits de son père ! » Et tout le peuple dira : « Amen. »
${}^{21}« Maudit qui couche avec quelque bête que ce soit ! » Et tout le peuple dira : « Amen. »
${}^{22}« Maudit qui couche avec sa sœur, fille de son père ou fille de sa mère ! » Et tout le peuple dira : « Amen. »
${}^{23}« Maudit qui couche avec sa belle-mère ! » Et tout le peuple dira : « Amen. »
${}^{24}« Maudit qui frappe son prochain en cachette ! » Et tout le peuple dira : « Amen. »
${}^{25}« Maudit qui accepte un cadeau pour frapper à mort un innocent ! » Et tout le peuple dira : « Amen. »
${}^{26}« Maudit qui n’accomplira pas les paroles de cette Loi et ne les mettra pas en pratique ! » Et tout le peuple dira : « Amen. »
      
         
      \bchapter{}
      \begin{verse}
${}^{1}« Si tu écoutes attentivement la voix du Seigneur ton Dieu, si tu veilles à mettre en pratique tous ses commandements que moi je te donne aujourd’hui, alors le Seigneur ton Dieu te placera plus haut que toutes les nations de la terre. 
${}^{2}Toutes les bénédictions que voici viendront sur toi et t’atteindront, parce que tu auras écouté la voix du Seigneur ton Dieu :
${}^{3}Béni seras-tu dans la ville !
        Béni seras-tu dans les champs !
${}^{4}Bénis seront le fruit de tes entrailles, de ton sol, de ton bétail,
        tes vaches pleines et tes brebis mères.
${}^{5}Bénis seront ton panier
        et ta huche à pain !
${}^{6}Béni seras-tu quand tu entreras !
        Béni seras-tu quand tu sortiras !
${}^{7}Des ennemis qui se dresseront contre toi, le Seigneur fera des vaincus devant toi : par un seul chemin, ils sortiront à ta rencontre ; par sept chemins, ils fuiront devant toi.
${}^{8}Le Seigneur ordonnera que la bénédiction soit avec toi, dans tes greniers et en toutes tes entreprises, et il te bénira dans le pays que le Seigneur ton Dieu te donne. 
${}^{9}Le Seigneur t’établira pour lui peuple consacré, comme il te l’a juré, car tu garderas les commandements du Seigneur ton Dieu et tu suivras ses chemins ; 
${}^{10}tous les peuples de la terre verront que le nom du Seigneur est proclamé sur toi, et ils auront peur de toi. 
${}^{11}Le Seigneur te comblera de biens en surabondance : il fera fructifier ta famille, ton bétail et ton sol sur la terre qu’il a juré à tes pères de te donner. 
${}^{12}Le Seigneur ouvrira pour toi son beau trésor, le ciel pour donner la pluie à ton pays au temps favorable et bénir ainsi toute œuvre de ta main. Tu prêteras à beaucoup de nations, et toi, tu n’emprunteras pas. 
${}^{13}Le Seigneur te mettra à la tête, et non pas en queue ; tu ne feras que monter, tu ne descendras pas, si tu écoutes les commandements du Seigneur ton Dieu que je te donne aujourd’hui à garder et à mettre en pratique. 
${}^{14}Tu ne t’écarteras ni à droite ni à gauche de tout ce que je vous ordonne aujourd’hui : vous ne suivrez pas d’autres dieux pour les servir.
${}^{15}« Mais si tu n’écoutes pas la voix du Seigneur ton Dieu, si tu ne veilles pas à mettre en pratique tous ses commandements et ses décrets que moi je te donne aujourd’hui, alors, toutes les malédictions que voici viendront sur toi et t’atteindront :
${}^{16}Maudit seras-tu dans la ville !
        Maudit seras-tu dans les champs !
${}^{17}Maudits seront ton panier
        et ta huche à pain !
${}^{18}Maudits seront le fruit de tes entrailles
        et le fruit de ton sol,
        tes vaches pleines et tes brebis mères !
${}^{19}Maudit seras-tu quand tu entreras !
        Maudit seras-tu quand tu sortiras !
${}^{20}Le Seigneur enverra contre toi malédiction, panique et menace dans toutes tes entreprises, jusqu’à ce que tu sois exterminé et que tu disparaisses au plus vite, à cause de la perversité de tes actions car tu m’auras abandonné.
${}^{21}Le Seigneur te fera attraper une peste qui finira par t’éliminer de la terre où tu entres pour en prendre possession. 
${}^{22}Le Seigneur te frappera de consomption, de fièvre, d’inflammation, de brûlures, de sécheresse, de la rouille et de la nigelle du blé, qui te poursuivront jusqu’à ce que tu disparaisses.
${}^{23}Le ciel, au-dessus de ta tête, sera de bronze, et la terre, sous tes pieds, sera de fer. 
${}^{24}Le Seigneur fera tomber sur ton pays une pluie de cendre et de poussière ; du ciel, elle descendra sur toi, jusqu’à ce que tu sois exterminé. 
${}^{25}Le Seigneur fera de toi un vaincu devant tes ennemis : par un seul chemin, tu sortiras à leur rencontre ; par sept chemins, tu fuiras devant eux. Tu feras horreur à tous les royaumes de la terre. 
${}^{26}Ton cadavre sera la proie des oiseaux du ciel et des bêtes de la terre... et personne pour les inquiéter ! 
${}^{27}Le Seigneur te frappera des furoncles d’Égypte, d’abcès, de gale, de pustules, et rien ne pourra t’en guérir. 
${}^{28}Le Seigneur te frappera de démence, de cécité et d’égarement d’esprit. 
${}^{29}En plein midi, tu iras tâtonnant comme tâtonne un aveugle dans les ténèbres, tu ne réussiras pas à trouver ta route ; chaque jour tu ne seras qu’exploité, spolié... et personne pour te sauver !
${}^{30}La fiancée que tu as choisie, un autre la possédera ; la maison que tu as construite, tu ne l’habiteras pas ; la vigne que tu as plantée, tu n’en profiteras pas. 
${}^{31}Ton bœuf sera abattu sous tes yeux, et tu n’en mangeras pas ; ton âne te sera enlevé, et il ne te reviendra pas ; tes brebis seront livrées à tes ennemis... et personne pour te sauver. 
${}^{32}Tes fils et tes filles seront livrés à un autre peuple ; tes yeux se consumeront à les guetter chaque jour, et tu ne pourras rien faire. 
${}^{33}Le fruit de ton sol et le produit de ton travail, un peuple que tu ne connais pas les mangera : chaque jour, tu ne seras qu’exploité, maltraité. 
${}^{34}Au spectacle que tu auras sous les yeux, tu deviendras fou ! 
${}^{35}Le Seigneur te frappera aux genoux et aux cuisses de furoncles malins dont tu ne pourras guérir : tu en auras des pieds à la tête.
${}^{36}Le Seigneur te mènera, avec le roi que tu auras établi sur toi, vers une nation inconnue de toi et de tes pères, et là tu serviras d’autres dieux : du bois et de la pierre ! 
${}^{37}Tu deviendras objet de stupeur, fable et risée de tous les peuples vers qui le Seigneur ton Dieu t’aura conduit ! 
${}^{38}Tu verras germer beaucoup de graines dans les champs, mais tu récolteras fort peu, car la sauterelle aura tout dévasté. 
${}^{39}Tu planteras des vignes et tu les cultiveras, mais tu n’auras pas de vin à boire, ni même de vendange à faire, car le ver aura tout mangé ! 
${}^{40}Tu auras des oliviers sur tout ton territoire, mais pas d’huile pour t’enduire le corps, car tes oliviers seront abattus. 
${}^{41}Tu mettras au monde des fils et des filles, mais ils ne resteront pas avec toi, car ils s’en iront en captivité. 
${}^{42}Tous tes arbres et le fruit de ton sol, le grillon en prendra possession.
${}^{43}L’immigré qui réside chez toi s’élèvera au-dessus de toi, de plus en plus haut, et toi, tu descendras de plus en plus bas. 
${}^{44}Lui te prêtera, et toi, tu n’auras rien à lui prêter. Il sera en tête, et toi, en queue.
${}^{45}Toutes ces malédictions viendront sur toi, te poursuivront et t’atteindront jusqu’à ce que tu sois exterminé, parce que tu n’auras pas écouté la voix du Seigneur ton Dieu, tu n’auras pas gardé les commandements et les décrets qu’il t’a donnés. 
${}^{46}Ces malédictions seront un signe et un prodige sur toi et sur ta descendance, à jamais.
${}^{47}Au lieu de servir le Seigneur ton Dieu dans la joie et l’allégresse de ton cœur au temps de l’abondance, 
${}^{48}tu serviras l’ennemi envoyé par le Seigneur dans la faim, la soif, la nudité, la privation de tout. L’ennemi te mettra un joug de fer sur la nuque, jusqu’à t’exterminer.
${}^{49}Le Seigneur lancera contre toi, comme un aigle qui plane, une nation venue de loin, du bout du monde, une nation dont tu ne comprendras pas la langue, 
${}^{50}une nation au visage dur, sans respect pour le vieillard et sans pitié pour l’enfant. 
${}^{51}Elle mangera le fruit de ton bétail et de ton sol, jusqu’à ce que tu sois exterminé : elle ne te laissera rien de ton froment, de ton vin nouveau et de ton huile fraîche, de tes vaches pleines et de tes brebis mères, jusqu’à ce qu’elle te fasse disparaître. 
${}^{52}Elle t’assiégera dans toutes tes villes jusqu’à ce que s’écroulent dans tout le pays les hauts remparts et les fortifications, dans lesquels tu mets ta confiance ; elle t’assiégera dans toutes tes villes, dans tout ce pays que le Seigneur ton Dieu te donne.
${}^{53}Dans l’angoisse et la détresse auxquelles l’ennemi t’aura réduit, tu mangeras le fruit de tes entrailles, la chair des fils et des filles, que le Seigneur ton Dieu t’a donnés. 
${}^{54}L’homme le plus délicat, le plus raffiné, jettera un regard mauvais sur son frère, sur sa femme bien-aimée, sur ceux des fils qui lui restent : 
${}^{55}il aura peur d’avoir à partager avec l’un d’eux la chair de ses fils qu’il mangera sans en rien laisser, dans l’angoisse et la détresse auxquelles l’ennemi t’aura réduit dans toutes tes villes. 
${}^{56}La femme la plus délicate, la plus raffinée, celle qui n’essaie même pas de poser par terre la plante du pied tant elle est raffinée et délicate, jettera un regard mauvais sur l’homme qu’elle aime, sur son fils et sa fille, 
${}^{57}sur le fœtus sorti de son ventre, sur les enfants qu’elle a mis au monde : privée de tout, elle les mangera en cachette, dans l’angoisse et la détresse auxquelles l’ennemi t’aura réduit dans tes villes.
${}^{58}Si tu ne veilles pas à mettre en pratique toutes les paroles de cette Loi, paroles écrites dans ce livre, pour que tu craignes ce nom glorieux et redoutable : “Le Seigneur, ton Dieu”, 
${}^{59}alors le Seigneur te frappera de manière stupéfiante, toi et ta descendance, il te frappera de plaies graves et tenaces, de maladies pernicieuses et durables. 
${}^{60}Il fera revenir sur toi toutes les épidémies d’Égypte dont tu avais grand-peur, et elles ne te lâcheront pas. 
${}^{61}Et même toutes les maladies et les plaies qui ne sont pas mentionnées dans ce livre de la Loi, le Seigneur les déchaînera contre toi jusqu’à ce que tu sois exterminé. 
${}^{62}Vous avez été aussi nombreux que les étoiles du ciel, et il ne restera de vous qu’un petit nombre de gens, parce que tu n’auras pas écouté la voix du Seigneur ton Dieu.
${}^{63}Ainsi donc, de même que le Seigneur se plaisait à vous rendre heureux et nombreux, de même il se plaira à vous faire périr, à vous exterminer, et vous serez arrachés de la terre où tu vas entrer pour en prendre possession. 
${}^{64}Le Seigneur te dispersera parmi tous les peuples, d’une extrémité de la terre à l’autre, et là tu serviras d’autres dieux que ni toi ni tes pères vous ne connaissez : du bois et de la pierre ! 
${}^{65}Et, parmi ces nations, tu ne seras jamais tranquille, pas une place où reposer tes pieds. Là-bas, le Seigneur te donnera un cœur inquiet, un regard éteint, une existence qui s’épuise. 
${}^{66}Ta vie sera en suspens devant toi ; nuit et jour, tu trembleras ; tu n’auras plus de prise sur ta vie. 
${}^{67}Le matin, tu diras : “Qui me donnera d’être au soir ?”, et le soir : “Qui me donnera d’être au matin ?”, tellement ton cœur tremblera au spectacle de ce qui se déroulera sous tes yeux.
${}^{68}Et le Seigneur te fera retourner sur des bateaux en Égypte, par une route dont je t’avais dit : “Tu ne la reverras plus jamais !” Et là vous vous mettrez vous-mêmes en vente pour être les serviteurs et les servantes de tes ennemis, mais il n’y aura pas d’acheteur ! »
${}^{69}Voici les paroles de l’alliance que le Seigneur ordonna à Moïse de conclure avec les fils d’Israël au pays de Moab, en plus de l’alliance qu’il avait conclue avec eux au mont Horeb.
      
         
      \bchapter{}
      \begin{verse}
${}^{1}Moïse convoqua tout Israël et il leur dit :
      \begin{verse}Vous avez vu de vos propres yeux tout ce que le Seigneur a fait dans le pays d’Égypte à Pharaon, à tous ses serviteurs et à tout son pays : 
${}^{2}les grandes épreuves que vous avez vues de vos yeux, ces signes et ces prodiges grandioses. 
${}^{3}Mais, jusqu’à ce jour, le Seigneur ne vous a pas donné un cœur pour connaître, des yeux pour voir, des oreilles pour entendre.
${}^{4}Pendant quarante ans, je vous ai fait marcher dans le désert : les habits que vous portiez ne se sont pas usés, ni les sandales, à vos pieds. 
${}^{5}Vous n’avez pas eu de pain à manger, ni de vin ou de boisson forte à boire : cela pour que vous reconnaissiez que je suis le Seigneur votre Dieu. 
${}^{6}Ensuite, vous êtes arrivés en ce lieu ; Séhone, roi de Heshbone, et Og, roi de Bashane, sont sortis à notre rencontre pour combattre, et nous les avons vaincus. 
${}^{7}Nous avons pris leur pays et l’avons donné en héritage aux tribus de Roubène et de Gad, ainsi qu’à une moitié de la tribu de Manassé.
${}^{8}Vous garderez les paroles de cette alliance, et vous les mettrez en pratique pour réussir dans toutes vos actions.
${}^{9}Aujourd’hui, vous vous tenez tous debout devant le Seigneur votre Dieu : vos chefs, vos tribus, vos anciens, vos scribes, tous les hommes d’Israël, 
${}^{10}vos petits enfants, vos femmes et l’immigré qui réside à l’intérieur de ton camp, aussi bien celui qui coupe ton bois et que celui qui puise ton eau ; 
${}^{11}tu es là pour passer dans l’alliance du Seigneur ton Dieu, cette alliance, avec son poids de malédictions, le Seigneur ton Dieu la conclut avec toi aujourd’hui, 
${}^{12}pour te constituer aujourd’hui comme son peuple et être lui-même ton Dieu, ainsi qu’il te l’a déclaré et qu’il l’a juré à tes pères, Abraham, Isaac et Jacob. 
${}^{13}Cette alliance, avec son poids de malédictions, je ne la conclus pas avec vous seulement, 
${}^{14}je la conclus aussi bien avec celui qui se trouve présent ici aujourd’hui avec nous devant le Seigneur notre Dieu, qu’avec celui qui ne se trouve pas présent ici aujourd’hui avec nous.
${}^{15}Vous savez, vous, comment nous avons habité au pays d’Égypte et comment nous avons passé au milieu des nations dont vous avez traversé le territoire. 
${}^{16}Vous avez vu les horreurs et les idoles immondes qu’elles possèdent : du bois et de la pierre, de l’argent et de l’or ! 
${}^{17}Qu’il n’y ait donc pas chez vous homme, femme, clan ou tribu dont le cœur se détourne aujourd’hui du Seigneur notre Dieu pour aller servir les dieux de ces nations ; qu’il n’y ait chez vous aucune racine produisant du poison ou de l’absinthe. 
${}^{18}Mais si quelqu’un, alors qu’il entend ces paroles de malédictions, se bénit lui-même en disant : « La paix soit avec moi, même si je marche selon mon cœur endurci ! Car il est vrai, le proverbe : Terre abreuvée n’a plus soif ! », 
${}^{19}à celui-là le Seigneur ne pourra pardonner. Alors la colère du Seigneur et son ardeur jalouse s’enflammeront contre cet homme, toutes les malédictions décrites dans ce livre s’abattront sur lui, et le Seigneur effacera son nom sous les cieux. 
${}^{20}Pour son malheur, le Seigneur le mettra à part de toutes les tribus d’Israël, conformément à toutes les malédictions de l’alliance écrite dans ce livre de la Loi.
${}^{21}La génération suivante, celle de vos fils qui se lèveront après vous, et l’étranger venu d’un pays lointain, voyant les plaies de ce pays et les maladies infligées par le Seigneur, diront : 
${}^{22}« Toute la terre n’est plus que soufre, sel et feu ; elle n’est plus ensemencée ; rien ne germe et rien ne pousse, pas une herbe. C’est une catastrophe comme Sodome et Gomorrhe, comme Adma et Seboïm, que le Seigneur a ravagées dans sa colère et sa fureur. » 
${}^{23}Et toutes les nations demanderont : « Pourquoi le Seigneur a-t-il ainsi traité ce pays ? Pourquoi une colère d’une telle ardeur ? » 
${}^{24}Et on répondra : « C’est parce qu’ils ont abandonné l’alliance du Seigneur, le Dieu de leurs pères, cette alliance qu’il avait conclue avec eux quand il les fit sortir du pays d’Égypte. 
${}^{25}Ils sont allés servir d’autres dieux, ils se sont prosternés devant eux, des dieux qu’ils ne connaissaient pas et que le Seigneur ne leur avait pas donnés en partage. 
${}^{26}La colère du Seigneur s’est enflammée à ce point contre ce pays qu’il a fait venir sur lui toutes les malédictions décrites dans ce livre. 
${}^{27}Avec colère, fureur et grande irritation, le Seigneur les a arrachés de leur terre et les a jetés sur une autre terre ; ainsi en est-il aujourd’hui. »
${}^{28}Au Seigneur notre Dieu sont les choses cachées, mais les choses révélées sont pour nous et nos fils à jamais, afin que nous mettions en pratique toutes les paroles de cette Loi.
      
         
      \bchapter{}
      \begin{verse}
${}^{1}Lorsque toutes ces paroles se seront réalisées pour toi, cette bénédiction et cette malédiction que j’ai mises devant toi, tu les feras revenir en ton cœur, au milieu de toutes les nations où le Seigneur ton Dieu t’aura exilé. 
${}^{2}Tu reviendras au Seigneur ton Dieu, toi et tes fils, tu écouteras sa voix de tout ton cœur et de toute ton âme, tu observeras tout ce que je te commande aujourd’hui. 
${}^{3}Alors le Seigneur changera ton sort\\, il te montrera sa tendresse\\, et il te rassemblera de nouveau du milieu de tous les peuples où il t’aura dispersé. 
${}^{4}Serais-tu exilé au bout du monde, là même le Seigneur ton Dieu ira te prendre, et il te rassemblera. 
${}^{5}Le Seigneur ton Dieu te fera rentrer au pays que tes pères ont possédé, et tu le posséderas ; il te rendra heureux et nombreux, plus encore que tes pères.
${}^{6}Le Seigneur ton Dieu te circoncira le cœur, à toi et à ta descendance, pour que tu aimes le Seigneur ton Dieu de tout ton cœur et de toute ton âme, afin de vivre ; 
${}^{7}et le Seigneur ton Dieu fera tomber toutes les malédictions sur tes ennemis et sur ceux qui te haïssent et te persécutent. 
${}^{8}Quant à toi, tu reviendras et tu écouteras la voix du Seigneur, tu mettras en pratique tous ses commandements que je te donne aujourd’hui. 
${}^{9}Le Seigneur te comblera de bonheur en toutes tes œuvres : il fera fructifier ta famille, ton bétail et ton sol ; oui, de nouveau le Seigneur prendra plaisir à ton bonheur, comme il prenait plaisir au bonheur de tes pères, 
${}^{10}pourvu que tu écoutes la voix du Seigneur ton Dieu, en observant ses commandements et ses décrets inscrits dans ce livre de la Loi, et que tu reviennes au Seigneur ton Dieu de tout ton cœur et de toute ton âme. 
${}^{11}Car cette loi\\que je te prescris aujourd’hui n’est pas au-dessus de tes forces ni hors de ton atteinte. 
${}^{12}Elle n’est pas dans les cieux, pour que tu dises : « Qui montera aux cieux nous la chercher ? Qui nous la fera entendre, afin que nous la mettions en pratique ? » 
${}^{13}Elle n’est pas au-delà des mers, pour que tu dises : « Qui se rendra au-delà des mers nous la chercher ? Qui nous la fera entendre, afin que nous la mettions en pratique ? » 
${}^{14}Elle est tout près de toi, cette Parole, elle est dans ta bouche et dans ton cœur, afin que tu la mettes en pratique.
${}^{15}Vois ! Je mets aujourd’hui devant toi ou bien la vie et le bonheur, ou bien la mort et le malheur.  
${}^{16} Ce que je te commande aujourd’hui, c’est d’aimer le Seigneur ton Dieu, de marcher dans ses chemins, de garder ses commandements, ses décrets et ses ordonnances. Alors, tu vivras et te multiplieras ; le Seigneur ton Dieu te bénira dans le pays dont tu vas prendre possession. 
${}^{17} Mais si tu détournes ton cœur, si tu n’obéis pas, si tu te laisses entraîner à te prosterner devant d’autres dieux et à les servir, 
${}^{18} je vous le déclare aujourd’hui : certainement vous périrez, vous ne vivrez pas de longs jours sur la terre dont vous allez prendre possession quand vous aurez passé le Jourdain.
${}^{19}Je prends aujourd’hui à témoin contre vous le ciel et la terre : je mets devant toi la vie ou la mort, la bénédiction ou la malédiction. Choisis donc la vie, pour que vous viviez, toi et ta descendance, 
${}^{20} en aimant le Seigneur ton Dieu, en écoutant sa voix, en vous attachant à lui ; c’est là que se trouve ta vie, une longue vie sur la terre que le Seigneur a juré de donner à tes pères, Abraham, Isaac et Jacob.
      
         
      \bchapter{}
      \begin{verse}
${}^{1}Moïse prononça\\ces paroles devant tout Israël :
      \begin{verse}
${}^{2}« Maintenant que j’ai cent vingt ans, je ne peux plus être votre chef\\. Le Seigneur m’a dit : “Ce Jourdain, tu ne le passeras pas !” 
${}^{3} C’est le Seigneur votre Dieu qui passera devant vous ; il anéantira les nations que vous rencontrerez, et vous donnera leur territoire\\. Et c’est Josué qui passera le Jourdain à votre tête, comme l’a dit le Seigneur. 
${}^{4} Le Seigneur traitera les nations comme il a traité les rois des Amorites, Séhone et Og, et leur pays, tous ceux qu’il a exterminés. 
${}^{5} Le Seigneur vous les livrera, et vous les traiterez exactement comme je vous l’ai ordonné. 
${}^{6} Soyez forts et courageux\\, ne craignez pas, n’ayez pas peur devant eux : le Seigneur votre Dieu marche lui-même avec vous ; il ne vous lâchera pas, il ne vous abandonnera pas. »
${}^{7}Alors Moïse appela Josué, et lui dit en présence de tout Israël : « Sois fort et courageux : c’est toi qui vas entrer avec ce peuple dans le pays que le Seigneur a promis par serment à ses pères\\, c’est toi qui vas remettre au peuple son héritage. 
${}^{8} C’est le Seigneur qui marchera devant toi, c’est lui qui sera avec toi ; il ne te lâchera pas, il ne t’abandonnera pas. Ne crains pas, ne t’effraie pas\\ ! »
${}^{9}Moïse mit cette Loi par écrit et la donna aux prêtres, fils de Lévi, qui portent l’arche de l’Alliance du Seigneur, ainsi qu’à tous les anciens d’Israël. 
${}^{10}Et Moïse leur donna cet ordre : « Au bout de sept ans, au temps fixé pour l’année de la remise des dettes, à la fête des Tentes, 
${}^{11}tout Israël viendra voir la face du Seigneur ton Dieu, au lieu qu’il aura choisi ; alors tu liras cette Loi, devant tout Israël qui l’écoutera. 
${}^{12}Tu rassembleras le peuple, hommes, femmes, enfants, ainsi que l’immigré qui réside dans ta ville, pour qu’ils entendent et qu’ils apprennent à craindre le Seigneur votre Dieu, en veillant à mettre en pratique toutes les paroles de cette Loi. 
${}^{13}Et leurs fils, qui ne la connaissent pas, entendront et apprendront à craindre le Seigneur votre Dieu, tous les jours où vous vivrez sur la terre dont vous allez prendre possession en passant le Jourdain. »
${}^{14}Alors le Seigneur dit à Moïse : « Voici qu’approche le jour de ta mort. Appelle Josué et présentez-vous dans la tente de la Rencontre : je lui donnerai mes ordres. » Moïse et Josué allèrent donc se présenter dans la tente de la Rencontre. 
${}^{15}Et le Seigneur apparut dans la tente, dans une colonne de nuée. La colonne de nuée se dressait à l’entrée de la tente.
${}^{16}Alors le Seigneur dit à Moïse : « Voici que tu vas te coucher avec tes pères, et ce peuple se lèvera : il se prostituera en servant les dieux de l’étranger, ceux de la terre où il va entrer ; il m’abandonnera, il va rompre mon alliance, celle que j’ai conclue avec lui. 
${}^{17}Ce jour-là, ma colère s’enflammera contre lui. Je les abandonnerai, je leur cacherai ma face. Alors il sera bon à dévorer, malheurs et détresses sans nombre l’atteindront. Et, ce jour-là, il dira : “Si ces malheurs m’ont atteint, n’est-ce pas parce que mon Dieu n’est plus au milieu de moi ?” 
${}^{18}Mais moi, ce jour-là, je cacherai ma face, je la cacherai à cause de tout le mal qu’il aura fait en se tournant vers d’autres dieux.
${}^{19}Et maintenant, écrivez un cantique à votre usage. Toi, apprends-le aux fils d’Israël, mets-le sur leurs lèvres, afin que ce cantique me serve de témoin contre les fils d’Israël. 
${}^{20}En effet, quand j’aurai fait entrer ce peuple sur le sol ruisselant de lait et de miel, que j’ai juré de donner à leurs pères, il mangera, il se rassasiera, il engraissera, puis il se tournera vers d’autres dieux, il les servira, il me méprisera, il rompra mon alliance. 
${}^{21}Quand malheurs et détresses sans nombre l’auront atteint, ce cantique servira de témoin contre lui, car sa descendance l’aura encore sur les lèvres. Oui, je connais bien le projet qu’il forme aujourd’hui, avant même que je le fasse entrer dans le pays que j’ai juré de donner. » 
${}^{22}Ce jour-là, Moïse écrivit ce cantique et l’apprit aux fils d’Israël.
${}^{23}Puis le Seigneur donna cet ordre à Josué, fils de Noun : « Sois fort et courageux, car c’est toi qui feras entrer les fils d’Israël dans le pays que j’ai juré de leur donner ; et moi, je serai avec toi. »
${}^{24}Quand Moïse eut achevé d’écrire entièrement les paroles de cette Loi dans un livre, 
${}^{25}il donna ses ordres aux Lévites, qui portent l’arche de l’Alliance : 
${}^{26}« Prenez ce livre de la Loi et posez-le à côté de l’arche de l’Alliance du Seigneur votre Dieu ; là, il servira de témoin contre toi. 
${}^{27}En effet, moi je connais ton esprit rebelle et ta nuque raide. Si, aujourd’hui, alors que je suis encore vivant parmi vous, vous vous êtes révoltés contre le Seigneur, qu’en sera-t-il après ma mort ? 
${}^{28}Rassemblez auprès de moi tous les anciens de vos tribus et vos scribes : que je prononce ces paroles à leurs oreilles ! Que je prenne à témoin contre eux le ciel et la terre ! 
${}^{29}Oui, je le sais : après ma mort, vous allez vous pervertir et vous écarter du chemin que je vous ai prescrit. Alors, dans la suite des temps, le malheur s’avancera vers vous, parce que vous aurez fait ce qui est mal aux yeux du Seigneur, au point de l’irriter par toutes vos actions. »
${}^{30}Alors, aux oreilles de toute l’assemblée d’Israël, Moïse prononça ce cantique dans son entier :
       
      <p class="cantique" id="bib_ct-at_2"><span class="cantique_label">Cantique AT 2</span> = <span class="cantique_ref"><a class="unitex_link" href="#bib_dt_32_1">Dt 32, 1-12</a></span>
      <p class="cantique" id="bib_ct-at_2bis"><span class="cantique_label">Cantique AT 2bis</span> = <span class="cantique_ref"><a class="unitex_link" href="#bib_dt_32_6">Dt 32, 6bc.18-20ab.21.26.27.29.30.35cd-36.39-41</a></span>
      
         
      \bchapter{}
      \begin{verse}
${}^{1}Écoutez, cieux, je vais parler !
        \\que la terre entende les paroles de ma bouche !
        ${}^{2}Mon enseignement ruissellera comme la pluie,
        \\ma parole descendra comme la rosée,
        \\comme l’ondée sur la verdure,
        \\comme l’averse sur l’herbe.
         
        ${}^{3}C’est le nom du Seigneur que j’invoque ;
        \\à notre Dieu, reportez la grandeur.
        ${}^{4}Il est le Rocher : son œuvre est parfaite ;
        \\tous ses chemins ne sont que justice.
        \\Dieu de vérité, non pas de perfidie,
        \\il est juste, il est droit.
         
        ${}^{5}Ils l’ont déshonoré, ses fils perdus,
        \\génération fourbe et tortueuse.
        ${}^{6}Est-ce là, ce que tu rends au Seigneur,
        \\peuple stupide et sans sagesse ?
        \\N’est-ce pas lui, ton père, qui t’a créé,
        \\lui qui t’a fait et affermi ?
         
        ${}^{7}Rappelle-toi les jours de jadis,
        \\pénètre le cours des âges.
        \\Interroge ton père, il t’instruira ;
        \\les anciens te le diront.
         
        ${}^{8}Quand le Très-Haut dota les nations,
        \\quand il sépara les fils d’Adam,
        \\il fixa les frontières des peuples
        \\d’après le nombre des fils d’Israël.
        ${}^{9}Mais le lot du Seigneur, ce fut son peuple,
        \\Jacob, sa part d’héritage.
         
        ${}^{10}Il le trouve au pays du désert,
        \\chaos de hurlements sauvages.
        \\Il l’entoure, il l’élève, il le garde
        \\comme la prunelle de son œil.
         
        ${}^{11}Tel un aigle qui éveille sa nichée
        \\et plane au-dessus de ses petits,
        \\il déploie son envergure, il le prend,
        \\il le porte sur ses ailes.
        ${}^{12}Le Seigneur seul l’a conduit :
        \\pas de dieu étranger auprès de lui.
       
${}^{13}Il le fait chevaucher les hauteurs de la terre
        \\et le nourrit du produit des campagnes,
        \\il lui donne à goûter le miel de la roche
        \\et l’huile suintant du rocher le plus dur,
         
${}^{14}le beurre des vaches et le lait des brebis,
        \\la graisse des agneaux, les béliers de Bashane et les boucs,
        \\il lui donne la fine fleur du froment,
        \\et le sang de la grappe que tu bois fermenté.
         
${}^{15}Mais Israël a engraissé, il a regimbé ;
        \\– tu as engraissé, tu as grossi, tu as épaissi ! –
        \\il a rejeté le Dieu qui l’avait fait,
        \\il a déshonoré son Rocher, son salut.
         
${}^{16}Ils le rendent jaloux avec des étrangers
        \\et, par des abominations, ils l’exaspèrent.
${}^{17}Ils sacrifient à des démons qui ne sont pas Dieu,
        \\à des dieux qu’ils ne connaissent pas,
        \\de tout nouveaux venus
        \\que leurs pères n’ont pas redoutés.
         
        ${}^{18}Tu dédaignes le Rocher qui t’a mis au monde ;
        \\le Dieu qui t’a engendré, tu l’oublies.
        ${}^{19}Le Seigneur l’a vu : il réprouve
        \\ses fils et ses filles qui l’ont exaspéré.
         
        ${}^{20}Il dit : « Je vais leur cacher ma face
        \\et je verrai quel sera leur avenir\\ ;
        \\oui, c’est une engeance pervertie,
        \\ce sont des enfants sans foi.
         
        ${}^{21}Eux m’ont rendu jaloux par un dieu qui n’est pas dieu\\,
        \\exaspéré par leurs vaines idoles ;
        \\moi, je vais les rendre jaloux par un peuple
        qui n’est pas un peuple\\,
        \\les exaspérer par une nation stupide.
         
${}^{22}Oui, un feu a jailli de ma colère,
        \\il brûle jusqu’au fond du séjour des morts,
        \\il dévore la terre et ce qu’elle produit,
        \\il embrase les fondements des montagnes.
         
${}^{23}J’amoncellerai sur eux les malheurs,
        \\j’épuiserai contre eux mes flèches.
${}^{24}Quand ils seront épuisés par la faim,
        \\dévorés par la fièvre et par la peste,
        \\je lâcherai contre eux la dent des animaux,
        \\le venin des serpents.
         
${}^{25}Au-dehors, l’épée supprimera les enfants,
        \\et au-dedans régnera l’épouvante.
        \\Ils périront ensemble : jeune homme, jeune fille,
        \\enfant tout petit, vieillard aux cheveux blancs.
         
        ${}^{26}J’ai dit : “Je les réduirai en menue paille ;
        \\j’effacerai leur souvenir parmi les hommes !”
        ${}^{27}Mais il y a l’arrogance de l’ennemi !
        \\J’ai peur d’une méprise chez l’adversaire.
        \\On dirait : “C’est notre main qui a le dessus !
        \\Non, le Seigneur n’y est pour rien !” »
         
${}^{28}Cette nation a perdu le jugement,
        \\ils sont incapables de comprendre.
        ${}^{29}S’ils étaient des sages, ils comprendraient,
        \\ils discerneraient leur avenir :
        ${}^{30}Se peut-il que, par un seul, mille hommes soient poursuivis,
        \\et que, par deux, dix mille soient mis en fuite,
        \\sans que leur Rocher les ait vendus,
        \\que le Seigneur les ait livrés ?
${}^{31}Mais leur Rocher n’est pas comme notre Rocher,
        \\et nos ennemis en sont témoins.
         
${}^{32}Car leur vigne provient d’une vigne de Sodome,
        des vignobles de Gomorrhe,
        \\leurs raisins sont des raisins empoisonnés,
        des grappes amères ;
${}^{33}leur vin est un venin de serpent,
        un violent poison d’aspic.
         
${}^{34}« Ce que je tiens en réserve,
        \\ce qui est scellé dans mes trésors,
        n’est-ce pas ce que je dis :
        ${}^{35}à moi la vengeance et la rétribution,
        \\au temps où leur pied trébuchera :
        \\oui, proche est le jour de leur ruine,
        \\imminent, le sort qui les attend. »
         
        ${}^{36}Car le Seigneur fera justice à son peuple,
        \\il prendra en pitié ses serviteurs,
        \\quand il verra que leurs mains faiblissent,
        \\qu’il n’y a plus ni esclave ni homme libre.
         
${}^{37}Alors il dira : « Où sont leurs dieux,
        \\le rocher où ils cherchaient refuge,
${}^{38}leurs dieux qui mangeaient la graisse de leurs sacrifices
        \\et buvaient le vin de leurs libations ?
        \\Qu’ils se lèvent et viennent à votre aide !
        \\Que vous trouviez un lieu où vous cacher !
         
        ${}^{39}Voyez-le, maintenant, c’est moi, et moi seul ;
        \\pas d’autre dieu que moi ;
        \\c’est moi qui fais mourir et vivre\\,
        \\si j’ai frappé, c’est moi qui guéris,
        \\et personne ne délivre de ma main.
         
        ${}^{40}Car je lève la main vers les cieux,
        \\et je dis : Aussi vrai que je vis à jamais,
        ${}^{41}si j’aiguise l’éclair de mon épée,
        \\si ma main brandit le jugement,
        \\je tournerai la vengeance contre mes rivaux,
        \\je réglerai leur compte à mes adversaires ;
         
${}^{42}j’enivrerai mes flèches de sang,
        \\et mon épée dévorera la chair :
        \\voyez le sang des blessés et des captifs
        \\et les chevelures dénouées de l’ennemi ! »
         
${}^{43}Nations, acclamez son peuple !
        \\Dieu vengera le sang de ses serviteurs,
        \\il retournera la vengeance contre ses adversaires,
        \\il purifiera et sa terre et son peuple.
       
${}^{44}Moïse, accompagné de Josué, fils de Noun, vint prononcer toutes les paroles de ce cantique aux oreilles du peuple.
${}^{45}Quand Moïse eut achevé de dire toutes ces paroles à tout Israël, 
${}^{46}il ajouta : « Prenez à cœur toutes ces paroles ; je les établis aujourd’hui comme un témoin contre vous ; et ordonnez à vos fils de veiller à mettre en pratique toutes les paroles de cette Loi ! 
${}^{47}Car ce n’est pas une parole creuse, extérieure à vous : c’est votre vie. C’est par cette parole que vous prolongerez vos jours sur le sol dont vous allez prendre possession en passant le Jourdain. »
${}^{48}En ce jour même, le Seigneur parla à Moïse. Il dit : 
${}^{49}« Monte sur cette montagne de la chaîne des Abarim, sur le mont Nébo, au pays de Moab en face de Jéricho, et regarde le pays de Canaan que je donne en propriété aux fils d’Israël. 
${}^{50}Puis, tu mourras sur la montagne où tu seras monté, et tu seras réuni aux tiens, comme ton frère Aaron est mort sur la montagne de Hor et a été réuni aux siens. 
${}^{51}Puisque vous m’avez été infidèles parmi les fils d’Israël, aux eaux de Mériba de Cadès dans le désert de Cine, puisque vous avez méconnu ma sainteté au milieu des fils d’Israël, 
${}^{52}c’est donc de la montagne d’en face que tu verras le pays, mais tu n’y entreras pas, dans ce pays que je donne aux fils d’Israël. »
      
         
      \bchapter{}
      \begin{verse}
${}^{1}Voici la bénédiction que Moïse, l’homme de Dieu, prononça sur les fils d’Israël avant de mourir. 
${}^{2}Il dit :
        \\Le Seigneur est venu du Sinaï.
        \\Il s’est levé pour eux du côté de Séïr,
        \\il a resplendi depuis le mont de Parane
        \\et il est arrivé à Mériba de Cadès :
        \\à sa droite brillait pour eux le feu de la loi !
         
${}^{3}Toi, l’ami des peuples,
        \\tous les saints sont dans ta main
        \\et ils se sont prosternés à tes pieds.
        \\Chacun recueille tes paroles.
         
${}^{4}Moïse nous a prescrit une Loi,
        \\c’est un héritage pour l’assemblée de Jacob.
${}^{5}Et il y eut en Israël un roi,
        \\quand se réunirent les chefs du peuple
        \\et les tribus d’Israël, tous ensemble.
         
${}^{6}Que vive Roubène, qu’il ne meure pas !
        \\Que vive le petit nombre de ses gens !
         
${}^{7}Voici ce qu’il dit pour Juda :
        \\Écoute, Seigneur, la voix de Juda,
        \\fais-le revenir vers son peuple ;
        \\ses mains prendront sa défense
        \\et tu lui viendras en aide contre ses ennemis.
         
${}^{8}Pour Lévi, il dit :
        \\Tes Toummim et tes Ourim
        \\appartiennent à l’homme qui t’est fidèle,
        \\celui que tu as éprouvé à Massa,
        \\que tu as querellé aux eaux de Mériba,
${}^{9}lui qui a dit de son père et de sa mère :
        \\« Je ne les ai pas vus ! »,
        \\lui qui ne reconnaît pas ses frères
        \\et ignore ses fils !
        \\Car ils ont gardé ta parole
        \\et maintenu ton alliance,
${}^{10}ils enseignent tes ordonnances à Jacob
        \\et ta Loi à Israël ;
        \\ils présentent l’encens à tes narines
        \\et l’holocauste sur ton autel.
${}^{11}Bénis, Seigneur, sa vaillance
        \\et accepte l’œuvre de ses mains,
        \\brise les reins de ses agresseurs :
        \\qu’ils ne se relèvent plus, ceux qui le haïssent !
         
${}^{12}Pour Benjamin, il dit :
        \\Bien-aimé du Seigneur,
        \\il demeure en confiance près de lui ;
        \\Dieu le protège chaque jour ;
        \\il demeure entre ses collines.
         
${}^{13}Pour Joseph, il dit :
        \\Son pays est béni du Seigneur.
        \\À lui, le meilleur don du ciel, la rosée,
        \\et l’abîme qui se creuse dans les profondeurs,
${}^{14}les meilleurs produits du soleil
        \\et les meilleurs fruits de la lune,
${}^{15}les dons excellents des antiques montagnes
        \\et les dons les meilleurs des collines éternelles,
${}^{16}la meilleure part de ce qui remplit le pays
        \\et la faveur de Celui qui demeure dans le buisson !
        \\Que tout cela couronne la tête de Joseph,
        \\le front de celui qui est consacré parmi ses frères !
${}^{17}Il est ton taureau premier-né : honneur à lui !
        \\Ses cornes sont des cornes de buffle,
        \\il en frappe les peuples,
        \\tous ensemble jusqu’aux lointains de la terre.
        \\Telles sont les myriades d’Éphraïm,
        \\tels sont les milliers de Manassé.
         
${}^{18}Pour Zabulon, il dit :
        \\Réjouis-toi, Zabulon, en tes expéditions,
        \\et toi, Issakar, sous tes tentes !
${}^{19}Ils convoquent des peuples sur la montagne,
        \\ils y offrent des offrandes justes,
        \\ils s’approprient les richesses de la mer
        \\et les trésors cachés dans le sable.
         
${}^{20}Pour Gad, il dit :
        \\Béni soit celui qui met Gad au large !
        \\Gad s’est installé comme une lionne,
        \\déchiquetant épaule et crâne ;
${}^{21}il a jeté son dévolu sur les prémices,
        \\la part réservée au prince ;
        \\il a rejoint les chefs du peuple,
        \\il a accompli la justice du Seigneur
        \\et ses jugements en faveur d’Israël.
         
${}^{22}Pour Dane, il dit :
        \\Dane est un lionceau
        \\qui s’élance de Bashane.
         
${}^{23}Pour Nephtali, il dit :
        \\Nephtali est rassasié de la faveur du Seigneur
        \\et comblé de sa bénédiction :
        \\qu’il prenne possession de l’ouest et du midi !
         
${}^{24}Pour Asher, il dit :
        \\Asher, qu’il soit béni entre les fils !
        \\Qu’il soit favorisé parmi ses frères !
        \\Qu’il baigne son pied dans l’huile !
${}^{25}Que tes verrous soient de fer et de bronze !
        \\Que ta force dure autant que tes jours !
         
${}^{26}Nul n’est semblable à Dieu, Israël.
        \\Pour venir à ton aide, il chevauche les cieux
        \\et les nuées, dans son triomphe.
         
${}^{27}Le Dieu des temps anciens est un refuge,
        \\son bras depuis toujours est à l’œuvre ici-bas ;
        \\il a chassé l’ennemi devant toi
        \\et il a dit : « Extermine ! »
         
${}^{28}Israël repose en confiance,
        \\la source de Jacob est mise à part,
        \\dans un pays de froment et de vin nouveau,
        \\le ciel même y répand la rosée.
         
${}^{29}Heureux es-tu, Israël !
        \\Qui est semblable à toi,
        \\peuple sauvé par le Seigneur,
        \\lui, le bouclier qui te protège,
        \\l’épée qui te mène au triomphe ?
        \\Tes ennemis s’inclineront devant toi,
        \\et toi, tu marcheras sur les hauteurs de leur pays.
      
         
      \bchapter{}
      \begin{verse}
${}^{1}Moïse monta des steppes de Moab au mont Nébo, au sommet du Pisga, qui est\\en face de Jéricho. Le Seigneur lui fit voir tout le pays : Galaad jusqu’à Dane, 
${}^{2} tout Nephtali, le pays d’Éphraïm et de Manassé, tout le pays de Juda jusqu’à la Méditerranée, 
${}^{3} le Néguev, la région du Jourdain\\, la vallée de Jéricho ville des Palmiers, jusqu’à Soar. 
${}^{4} Le Seigneur lui dit : « Ce pays que tu vois, j’ai juré à Abraham, à Isaac et à Jacob de le donner à leur descendance. Je te le fais voir, mais tu n’y entreras pas. »
${}^{5}Moïse, le serviteur du Seigneur, mourut là, au pays de Moab, selon la parole du Seigneur. 
${}^{6} On l’enterra dans la vallée qui est en face de Beth-Péor, au pays de Moab. Mais aujourd’hui encore, personne ne sait où se trouve son tombeau. 
${}^{7} Moïse avait cent vingt ans quand il mourut ; sa vue n’avait pas baissé, sa vitalité n’avait pas diminué. 
${}^{8} Les fils d’Israël pleurèrent Moïse dans les steppes de Moab, pendant trente jours. C’est alors que s’achevèrent les jours\\du deuil de Moïse. 
${}^{9} Josué, fils de Noun, était rempli de l’esprit de sagesse, parce que Moïse lui avait imposé les mains. Les fils d’Israël lui obéirent, ils firent ce que le Seigneur avait prescrit à Moïse.
${}^{10}Il ne s’est plus levé en Israël un prophète comme Moïse, lui que le Seigneur rencontrait\\face à face. 
${}^{11} Que de signes et de prodiges le Seigneur l’avait envoyé accomplir en Égypte, devant Pharaon, tous ses serviteurs et tout son pays ! 
${}^{12} Avec quelle main puissante, quel pouvoir redoutable, Moïse avait agi aux yeux de tout Israël !
