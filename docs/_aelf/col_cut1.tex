  
  
    
    \bbook{LETTRE AUX COLOSSIENS}{LETTRE AUX COLOSSIENS}
      <p class="cantique" id="bib_ct-nt_6"><span class="cantique_label">Cantique NT 6</span> = <span class="cantique_ref"><a class="unitex_link" href="#bib_col_1_12">Col 1, 12-20</a></span>
      
         
      \bchapter{}
        ${}^{1}Paul, apôtre du Christ Jésus
        par la volonté de Dieu,
        \\et Timothée notre frère,
        ${}^{2}aux frères sanctifiés par la foi dans le Christ
        qui habitent Colosses.
        \\À vous, la grâce et la paix
        \\de la part de Dieu notre Père.
        
           
${}^{3}Nous rendons grâce à Dieu, le Père de notre Seigneur Jésus Christ, en priant pour vous à tout moment. 
${}^{4}Nous avons entendu parler de votre foi dans le Christ Jésus et de l’amour que vous avez pour tous les fidèles 
${}^{5}dans l’espérance de ce qui vous est réservé au ciel ; vous en avez déjà reçu l’annonce par la parole de vérité, l’Évangile 
${}^{6}qui est parvenu jusqu’à vous. Lui qui porte du fruit et progresse dans le monde entier, il fait de même chez vous, depuis le jour où vous avez reçu l’annonce et la pleine connaissance de la grâce de Dieu dans la vérité. 
${}^{7}Cet enseignement vous a été donné par Épaphras, notre cher compagnon de service, qui est pour vous un ministre du Christ digne de foi ; 
${}^{8}il nous a fait savoir de quel amour l’Esprit vous anime.
${}^{9}Depuis le jour où nous en avons entendu parler, nous ne cessons pas de prier pour vous. Nous demandons à Dieu de vous combler de la pleine connaissance de sa volonté, en toute sagesse et intelligence spirituelle. 
${}^{10}Ainsi votre conduite sera digne du Seigneur, et capable de lui plaire en toutes choses ; par tout le bien que vous ferez, vous porterez du fruit et vous progresserez dans la vraie connaissance de Dieu. 
${}^{11}Vous serez fortifiés en tout par la puissance de sa gloire, qui vous donnera toute persévérance et patience.
      Dans la joie, 
${}^{12}vous rendrez grâce à Dieu le Père, qui vous a rendus capables d’avoir part à l’héritage des saints, dans la lumière.
       
        ${}^{13}Nous arrachant au pouvoir des ténèbres,
        \\il nous a placés dans le Royaume de son Fils bien-aimé\\ :
        ${}^{14}en lui nous avons la rédemption\\,
        le pardon des péchés.
         
        ${}^{15}Il est l’image du Dieu invisible,
        \\le premier-né, avant toute créature\\ :
        ${}^{16}en lui, tout fut créé,
        dans le ciel et sur la terre.
         
        \\Les êtres visibles et invisibles,
        \\Puissances, Principautés,
        Souverainetés\\, Dominations\\,
        \\tout est créé par lui et pour lui.
         
        ${}^{17}Il est avant toute chose,
        \\et tout subsiste en lui.
         
        ${}^{18}Il est aussi la tête du corps, la tête\\de l’Église :
        \\c’est lui le commencement,
        le premier-né d’entre les morts,
        \\afin qu’il ait en tout la primauté.
         
        ${}^{19}Car Dieu a jugé bon
        qu’habite en lui toute plénitude
        ${}^{20}et que tout, par le Christ,
        lui soit enfin réconcilié\\,
         
        \\faisant la paix par le sang de sa Croix,
        \\la paix\\pour tous les êtres
        sur la terre et dans le ciel.
       
${}^{21}Et vous, vous étiez jadis étrangers à Dieu, et même ses ennemis, par vos pensées et vos actes mauvais. 
${}^{22}Mais maintenant, Dieu vous a réconciliés avec lui, dans le corps du Christ, son corps de chair, par sa mort, afin de vous introduire en sa présence, saints, immaculés, irréprochables. 
${}^{23}Cela se réalise si vous restez solidement fondés dans la foi, sans vous détourner de l’espérance que vous avez reçue en écoutant l’Évangile proclamé à toute créature sous le ciel. De cet Évangile, moi, Paul, je suis devenu ministre.
${}^{24}Maintenant je trouve la joie dans les souffrances que je supporte pour vous ; ce qui reste à souffrir des épreuves du Christ dans ma propre chair, je l’accomplis pour son corps qui est l’Église. 
${}^{25}De cette Église, je suis devenu ministre, et la mission que Dieu m’a confiée, c’est de mener à bien pour vous l’annonce de sa parole, 
${}^{26}le mystère qui était caché depuis toujours à toutes les générations, mais qui maintenant a été manifesté à ceux qu’il a sanctifiés. 
${}^{27}Car Dieu a bien voulu leur faire connaître en quoi consiste la gloire sans prix de ce mystère parmi toutes les nations : le Christ est parmi vous, lui, l’espérance de la gloire ! 
${}^{28}Ce Christ, nous l’annonçons : nous avertissons tout homme, nous instruisons chacun en toute sagesse, afin de l’amener à sa perfection dans le Christ. 
${}^{29}C’est pour cela que je m’épuise à combattre, avec la force du Christ dont la puissance agit en moi.
      
         
      \bchapter{}
      \begin{verse}
${}^{1}Je veux en effet que vous sachiez quel dur combat je mène pour vous, et aussi pour les fidèles de Laodicée et pour tant d’autres qui ne m’ont jamais vu personnellement. 
${}^{2}Je combats pour que leurs cœurs soient remplis de courage et pour que, rassemblés dans l’amour, ils accèdent à la plénitude de l’intelligence dans toute sa richesse, et à la vraie connaissance du mystère de Dieu. Ce mystère, c’est le Christ, 
${}^{3}en qui se trouvent cachés tous les trésors de la sagesse et de la connaissance.
${}^{4}Je vous dis cela pour que personne ne vous égare par des arguments trop habiles. 
${}^{5}Car si je suis absent physiquement, je suis toutefois spirituellement avec vous, et je me réjouis de voir l’ordre qu’il y a chez vous et la fermeté de votre foi au Christ.
${}^{6}Menez donc votre vie dans le Christ Jésus, le Seigneur, tel que vous l’avez reçu. 
${}^{7}Soyez enracinés, édifiés en lui, restez fermes dans la foi, comme on vous l’a enseigné ; soyez débordants d’action de grâce. 
${}^{8}Prenez garde à ceux qui veulent faire de vous leur proie par une philosophie vide et trompeuse, fondée sur la tradition des hommes, sur les forces qui régissent le monde, et non pas sur le Christ.
${}^{9}Car en lui, dans son propre corps, habite toute la plénitude de la divinité. 
${}^{10}En lui, vous êtes pleinement comblés, car il domine toutes les Puissances de l’univers. 
${}^{11}En lui, vous avez reçu une circoncision qui n’est pas celle que pratiquent les hommes, mais celle qui réalise l’entier dépouillement de votre corps de chair ; telle est la circoncision qui vient du Christ. 
${}^{12}Dans le baptême, vous avez été mis au tombeau avec lui et vous êtes ressuscités avec lui par la foi en la force de Dieu qui l’a ressuscité d’entre les morts. 
${}^{13}Vous étiez des morts, parce que vous aviez commis des fautes et n’aviez pas reçu de circoncision dans votre chair. Mais Dieu vous a donné la vie avec le Christ : il nous a pardonné toutes nos fautes. 
${}^{14}Il a effacé le billet de la dette qui nous accablait en raison des prescriptions légales pesant sur nous : il l’a annulé en le clouant à la croix. 
${}^{15}Ainsi, Dieu a dépouillé les Puissances de l’univers ; il les a publiquement données en spectacle et les a traînées dans le cortège triomphal du Christ.
${}^{16}Alors, que personne ne vous juge pour des questions de nourriture et de boisson, ou à propos de fête, de nouvelle lune ou de sabbat : 
${}^{17}tout cela n’est que l’ombre de ce qui devait venir, mais la réalité, c’est le Christ. 
${}^{18}Ne vous laissez pas frustrer de votre récompense par ceux qui veulent vous humilier par un culte des anges et qui s’évadent dans des visions et se laissent vainement gonfler d’orgueil par des idées purement humaines. 
${}^{19}Ces gens-là ne sont pas en union avec la tête, avec Celui par qui tout le corps poursuit sa croissance en Dieu, grâce aux articulations et aux ligaments qui maintiennent sa cohésion.
${}^{20}Si, avec le Christ, vous êtes morts aux forces qui régissent le monde, pourquoi subir des prescriptions légales comme si votre vie dépendait encore du monde : 
${}^{21}« Ne prends pas ceci, ne goûte pas cela, ne touche pas cela », 
${}^{22}alors que toutes ces choses sont faites pour disparaître quand on s’en sert ! Ce ne sont là que des préceptes et des enseignements humains, 
${}^{23}qui ont des airs de sagesse, de religion personnelle, d’humilité et de rigueur pour le corps, mais ne sont d’aucune valeur pour maîtriser la chair.
      
         
      \bchapter{}
      \begin{verse}
${}^{1}Si donc vous êtes ressuscités avec le Christ, recherchez les réalités d’en haut : c’est là qu’est le Christ, assis à la droite de Dieu. 
${}^{2}Pensez aux réalités d’en haut, non à celles de la terre. 
${}^{3}En effet, vous êtes passés par la mort, et votre vie reste cachée avec le Christ en Dieu. 
${}^{4}Quand paraîtra le Christ, votre vie, alors vous aussi, vous paraîtrez avec lui dans la gloire.
      
         
${}^{5}Faites donc mourir en vous ce qui n’appartient qu’à la terre : débauche, impureté, passion, désir mauvais, et cette soif de posséder, qui est une idolâtrie. 
${}^{6}Voilà ce qui provoque la colère de Dieu contre ceux qui lui désobéissent, 
${}^{7}voilà quelle était votre conduite autrefois lorsque, vous aussi, vous viviez dans ces désordres. 
${}^{8}Mais maintenant, vous aussi, débarrassez-vous de tout cela : colère, emportement, méchanceté, insultes, propos grossiers sortis de votre bouche. 
${}^{9}Plus de mensonge entre vous : vous vous êtes débarrassés de l’homme ancien qui était en vous et de ses façons d’agir, 
${}^{10}et vous vous êtes revêtus de l’homme nouveau qui, pour se conformer à l’image de son Créateur, se renouvelle sans cesse en vue de la pleine connaissance. 
${}^{11}Ainsi, il n’y a plus le païen et le Juif, le circoncis et l’incirconcis, il n’y a plus le barbare ou le primitif, l’esclave et l’homme libre ; mais il y a le Christ : il est tout, et en tous.
${}^{12}Puisque vous avez été choisis par Dieu, que vous êtes sanctifiés, aimés par lui, revêtez-vous de tendresse et de compassion, de bonté, d’humilité, de douceur et de patience. 
${}^{13}Supportez-vous les uns les autres, et pardonnez-vous mutuellement si vous avez des reproches à vous faire. Le Seigneur vous a pardonné : faites de même. 
${}^{14}Par-dessus tout cela, ayez l’amour, qui est le lien le plus parfait. 
${}^{15}Et que, dans vos cœurs, règne la paix du Christ à laquelle vous avez été appelés, vous qui formez un seul corps. Vivez dans l’action de grâce. 
${}^{16}Que la parole du Christ habite en vous dans toute sa richesse ; instruisez-vous et reprenez-vous les uns les autres en toute sagesse ; par des psaumes, des hymnes et des chants spirituels, chantez à Dieu, dans vos cœurs, votre reconnaissance. 
${}^{17}Et tout ce que vous dites, tout ce que vous faites, que ce soit toujours au nom du Seigneur Jésus, en offrant par lui votre action de grâce à Dieu le Père.
${}^{18}Vous les femmes, soyez soumises à votre mari ; dans le Seigneur, c’est ce qui convient. 
${}^{19}Et vous les hommes, aimez votre femme, ne soyez pas désagréables avec elle. 
${}^{20}Vous les enfants, obéissez en toute chose à vos parents ; cela est beau dans le Seigneur. 
${}^{21}Et vous les parents, n’exaspérez pas vos enfants ; vous risqueriez de les décourager.
${}^{22}Vous les esclaves, obéissez en toute chose à vos maîtres d’ici-bas, non pas seulement sous leurs yeux, par souci de plaire aux hommes, mais dans la simplicité de votre cœur, en craignant le Seigneur. 
${}^{23}Quel que soit votre travail, faites-le de bon cœur, comme pour le Seigneur et non pour plaire à des hommes : 
${}^{24}vous savez bien qu’en retour vous recevrez du Seigneur votre héritage. C’est le Christ, le Seigneur, que vous servez. 
${}^{25}Celui qui fait le mal récoltera le mal qu’il aura fait, car Dieu est impartial.
      
         
      \bchapter{}
      \begin{verse}
${}^{1}Vous les maîtres, assurez à vos esclaves la justice et l’équité, sachant que, vous aussi, vous avez un Maître dans le ciel.
      
         
${}^{2}Soyez assidus à la prière ; qu’elle vous tienne vigilants dans l’action de grâce. 
${}^{3}Priez en même temps pour nous, afin que Dieu ouvre une porte à notre parole et que nous annoncions le mystère du Christ, pour lequel je suis en prison ; 
${}^{4}ainsi, je le manifesterai comme je me dois d’en parler. 
${}^{5}Conduisez-vous avec sagesse envers ceux du dehors, en tirant parti du moment favorable. 
${}^{6}Que vos paroles soient toujours bienveillantes, qu’elles ne manquent pas de sel, vous saurez ainsi répondre à chacun comme il faut.
${}^{7}Vous serez informés de tout ce qui me concerne, par Tychique, le frère bien-aimé, fidèle ministre et mon compagnon de service, dans le Seigneur. 
${}^{8}Je l’envoie spécialement auprès de vous afin que vous ayez de nos nouvelles et qu’il réconforte vos cœurs ; 
${}^{9}je l’envoie avec Onésime, le frère fidèle et bien-aimé, qui est de chez vous ; ils vous informeront de tout ce qui se passe ici.
${}^{10}Vous avez les salutations d’Aristarque, mon compagnon de captivité, et celles de Marc, le cousin de Barnabé – vous avez reçu des instructions à son sujet : s’il vient chez vous, accueillez-le – ; 
${}^{11}vous avez aussi les salutations de Jésus appelé Justus : ces trois-là sont les seuls d’origine juive à travailler avec moi au règne de Dieu, ils ont été pour moi une consolation. 
${}^{12}Vous avez les salutations d’Épaphras, qui est de chez vous : un serviteur du Christ Jésus, qui mène sans cesse pour vous le combat de la prière, afin que vous teniez debout, comme des gens parfaits et pleinement accordés à la volonté de Dieu ; 
${}^{13}je lui rends ce témoignage qu’il se donne beaucoup de peine pour vous, pour ceux de Laodicée et ceux de Hiérapolis. 
${}^{14}Vous avez la salutation de Luc, le médecin bien-aimé, et de Démas.
${}^{15}Saluez les frères de Laodicée, et aussi Nympha et l’Église qui se rassemble dans sa maison. 
${}^{16}Et quand on aura lu cette lettre chez vous, faites en sorte qu’on la lise aussi dans l’Église de Laodicée ; lisez aussi vous-mêmes celle qui vous viendra de Laodicée. 
${}^{17}Enfin dites à Archippe : « Veille à bien accomplir le ministère que tu as reçu dans le Seigneur. »
${}^{18}La salutation est de ma main à moi, Paul. Souvenez-vous que je suis en prison. La grâce soit avec vous.
