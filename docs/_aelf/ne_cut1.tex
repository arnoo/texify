  
  
    
    \bbook{NÉHÉMIE}{NÉHÉMIE}
      
         
      \bchapter{}
      \begin{verse}
${}^{1}Paroles de Néhémie, fils de Hakalya. Au mois de Kisléou, la vingtième année du règne d’Artaxerxès, alors que je me trouvais à Suse-la-Citadelle, 
${}^{2}Hanani, l’un de mes frères, arriva de Juda avec quelques hommes. Je les interrogeai au sujet des Juifs rescapés, ceux qui avaient échappé à la captivité, et sur Jérusalem. 
${}^{3}Ils me répondirent : « Ceux qui ont échappé à la captivité et qui sont restés là-bas dans la province sont dans une grande détresse et dans la honte ; le rempart de Jérusalem n’est que brèches, et ses portes ont été dévastées par le feu. »
      
         
${}^{4}Dès que j’entendis ces paroles, je me suis assis et j’ai pleuré ; je fus dans le deuil durant plusieurs jours, jeûnant et priant devant le Dieu du ciel. 
${}^{5}Je dis alors : « Ah ! Seigneur, Dieu du ciel, Dieu grand et redoutable qui garde l’alliance et la fidélité à ceux qui l’aiment et observent ses commandements, 
${}^{6}que ton oreille soit attentive, et tes yeux ouverts, pour écouter la prière de ton serviteur.
      Aujourd’hui, devant ta face je prie jour et nuit pour les fils d’Israël, tes serviteurs : je confesse les péchés des fils d’Israël, nos péchés contre toi ; moi-même et la maison de mon père, nous avons péché ! 
${}^{7}Nous avons vraiment mal agi envers toi ; nous n’avons pas observé les commandements, les décrets et les ordonnances que tu avais prescrits à Moïse, ton serviteur.
${}^{8}Souviens-toi de la parole que Moïse ton serviteur a prononcée sur ton ordre : “Si vous êtes infidèles, moi, je vous disperserai parmi les peuples ; 
${}^{9}mais si vous revenez à moi, si vous observez mes commandements et les mettez en pratique, quand bien même certains auraient été chassés jusqu’à l’extrémité des cieux, je les rassemblerai et je les ramènerai au lieu que j’ai choisi pour y faire habiter mon nom.”
${}^{10}Ce sont tes serviteurs et ton peuple que tu as rachetés par ta grande puissance et par la force de ta main. 
${}^{11}Ah ! Seigneur, que ton oreille soit attentive à la prière de ton serviteur, et à la prière de tes serviteurs qui se plaisent à craindre ton nom. Fais qu’aujourd’hui ton serviteur réussisse et trouve miséricorde en face de cet homme. »
      J’étais en effet échanson du roi.
      
         
      \bchapter{}
      \begin{verse}
${}^{1}La vingtième année du règne d’Artaxerxès, au mois de Nissane, je présentai le vin et l’offris au roi. Je n’avais jamais montré de tristesse devant lui, 
${}^{2}mais ce jour-là, le roi me dit : « Pourquoi ce visage triste ? Tu n’es pourtant pas malade ! Tu as donc du chagrin ? » Rempli de crainte, je répondis : 
${}^{3}« Que le roi vive toujours ! Comment n’aurais-je pas l’air triste, quand la ville où sont enterrés mes pères a été dévastée, et ses portes, dévorées par le feu ? » 
${}^{4}Le roi me dit alors : « Que veux-tu donc me demander ? » Je fis une prière au Dieu du ciel, et je répondis au roi : 
${}^{5}« Si tel est le bon plaisir du roi, et si tu es satisfait de ton serviteur, laisse-moi aller en Juda, dans la ville où sont enterrés mes pères, et je la rebâtirai. » 
${}^{6}Le roi, qui avait la reine à côté de lui, me demanda : « Combien de temps durera ton voyage ? Quand reviendras-tu ? » Je lui indiquai une date qu’il approuva, et il m’autorisa à partir. 
${}^{7}Je dis encore : « Si tel est le bon plaisir du roi, qu’on me donne des lettres pour les gouverneurs de la province qui est à l’ouest de l’Euphrate, afin qu’ils facilitent mon passage jusqu’en Juda ; 
${}^{8}et aussi une lettre pour Asaph, l’inspecteur des forêts royales, afin qu’il me fournisse du bois de charpente pour les portes de la citadelle qui protégera la maison de Dieu\\, le rempart de la ville, et la maison où je vais m’installer. » Le roi me l’accorda, car la main bienfaisante\\de mon Dieu était sur moi.
${}^{9}Je me rendis auprès des gouverneurs de Transeuphratène et je leur remis les lettres du roi. Le roi m’avait fait escorter par des officiers de l’armée et des cavaliers. 
${}^{10}Sânballath le Horonite et Tobie, le fonctionnaire ammonite, l’apprirent, et ils se montrèrent fort mécontents que quelqu’un soit venu s’enquérir de ce qui était bon pour les fils d’Israël.
${}^{11}Je suis donc arrivé à Jérusalem, j’y suis resté trois jours. 
${}^{12}Puis je me suis levé, de nuit, accompagné de quelques hommes, mais je n’avais confié à personne ce que mon Dieu m’avait inspiré d’accomplir en faveur de Jérusalem ; je n’avais avec moi aucune autre bête de somme que ma propre monture. 
${}^{13}Pendant la nuit, je sortis par la porte de la Vallée, je me rendis devant la source du Dragon, puis à la porte du Fumier : j’inspectai attentivement les remparts de Jérusalem, qui n’étaient que brèches et dont les portes avaient été dévorées par le feu. 
${}^{14}Je poursuivis mon chemin vers la porte de la Source et le réservoir du Roi, et je ne trouvai plus de passage pour la bête que je chevauchais. 
${}^{15}Je remontai donc de nuit par le ravin, inspectant toujours attentivement le rempart, je rentrai par la porte de la Vallée et je m’en revins.
${}^{16}Les magistrats ne surent pas où j’étais allé ni ce que j’avais fait. Jusque-là je n’avais rien révélé aux Juifs, prêtres, notables, magistrats, ni aux autres qui étaient chargés des travaux. 
${}^{17}Je leur dis alors : « Vous voyez la détresse où nous sommes : Jérusalem est en ruines, ses portes ont été dévastées par le feu. Venez ! Allons rebâtir le rempart de Jérusalem, et nous ne serons plus un sujet de honte ! » 
${}^{18}Je leur révélai comment la main bienfaisante de mon Dieu avait été sur moi, et aussi comment le roi m’avait parlé. Ils s’écrièrent : « Mettons-nous à reconstruire ! » Et, avec courage, ils se préparèrent à cette belle œuvre.
${}^{19}Sânballath le Horonite, Tobie, le fonctionnaire ammonite, et Guèshem l’Arabe l’apprirent. Ils se moquèrent de nous et nous méprisèrent. Ils nous dirent : « Qu’êtes-vous en train de faire ? Allez-vous vous révolter contre le roi ? » 
${}^{20}Mais je leur répliquai en ces termes : « C’est le Dieu du ciel lui-même qui nous fera réussir. Nous, ses serviteurs, nous allons nous mettre à reconstruire. Mais pour vous, il n’y aura ni part, ni droit, ni souvenir dans Jérusalem. »
      
         
      \bchapter{}
      \begin{verse}
${}^{1}Élyashib, le grand prêtre, ainsi que ses frères les prêtres, se mirent donc à reconstruire la porte des Brebis ; ils la consacrèrent et en posèrent les battants ; puis ils continuèrent jusqu’à la tour des Cent, la consacrèrent, et ils continuèrent jusqu’à la tour de Hananéel. 
${}^{2}À la suite construisirent les gens de Jéricho. À la suite construisit Zakkour, fils d’Imri. 
${}^{3}Les fils de Ha-Senaa construisirent la porte des Poissons ; ils en firent la charpente, posèrent ses battants, ses verrous et ses barres. 
${}^{4}À leur suite travailla Merémoth, fils de Ouriya, fils de Haqqos. À leur suite travailla Meshoullam, fils de Bèrèkya, fils de Meshézabéel. À leur suite travailla Sadoc, fils de Baana. 
${}^{5}À leur suite travaillèrent les gens de Teqoa, mais leurs notables refusèrent de se plier au service de leurs maîtres. 
${}^{6}C’est à la porte Vieille que travaillèrent Yoyada, fils de Paséah, et Meshoullam, fils de Besodya ; ils en firent la charpente, posèrent ses battants, ses verrous et ses barres. 
${}^{7}À leur suite, travaillèrent Melatya le Gabaonite et Yadone le Méronite, ainsi que des hommes de Gabaon et de Mispa, à côté du siège du gouverneur de Transeuphratène. 
${}^{8}À leur suite travailla Ouzziël, fils de Harhaya, tous deux orfèvres, et à sa suite travailla Hananya, fils de parfumeurs : ils restaurèrent Jérusalem jusqu’à la Muraille large. 
${}^{9}À leur suite travailla Refaya, fils de Hour, chef de la moitié du district de Jérusalem. 
${}^{10}À leur suite travailla Yedaya, fils de Haroumaf, en face de sa maison. À sa suite travailla Hattoush, fils de Hashabneya. 
${}^{11}C’est au secteur suivant que travaillèrent Malkiya, fils de Harim, et Hashoub, fils de Pahath-Moab, ainsi qu’à la tour des Fours. 
${}^{12}À la suite travailla Shalloum, fils de Ha-Lohesh, chef d’une moitié du district de Jérusalem, lui et ses filles. 
${}^{13}C’est à la porte de la Vallée que travaillèrent Hanoun et les habitants de Zanoah : ils la construisirent, posèrent ses battants, ses verrous et ses barres ; ils restaurèrent encore mille coudées de mur, jusqu’à la porte du Fumier. 
${}^{14}C’est à la porte du Fumier que travailla Malkiya, fils de Récab, chef du district de Bet-ha-Kérem ; il la construisit, posa ses battants, ses verrous et ses barres. 
${}^{15}C’est à la porte de la Source que travailla Shalloum, fils de Kol-Hozé, chef du district de Mispa : il la construisit lui-même, la couvrit d’un toit, en posa les battants, les verrous et les barres. Il refit aussi le mur du réservoir du canal, attenant au jardin du roi, jusqu’aux marches qui descendent de la Cité de David.
${}^{16}Après lui travailla Néhémie fils d’Azbouq, chef de la moitié du district de Beth-Sour, jusqu’en face des tombeaux de David, jusqu’au réservoir artificiel et jusqu’à la maison des Vaillants. 
${}^{17}Après lui travaillèrent les Lévites dont Rehoum, fils de Bani. À sa suite, Hashabya, chef de la moitié du district de Qéïla, travailla pour son district. 
${}^{18}Après lui travaillèrent leurs frères dont Binnouï, fils de Hénadad, chef de la moitié du district de Qéïla. 
${}^{19}À sa suite travailla Ézer, fils de Josué, chef de Mispa, sur un autre secteur, face à la montée de l’Arsenal, à l’angle. 
${}^{20}Après lui travailla avec ardeur Baruc, fils de Zabbaï, sur un autre secteur, depuis l’angle jusqu’à l’entrée de la maison d’Élyashib, le grand prêtre. 
${}^{21}Après lui travailla Merémoth, fils d’Ouriya, fils de Haqqos, sur un autre secteur, depuis l’entrée de la maison d’Élyashib jusqu’à son extrémité. 
${}^{22}Et après lui travaillèrent les prêtres des alentours. 
${}^{23}Après eux travailla Benjamin ainsi que Hashshoub, juste en face de leur maison. Après lui travailla Azarya, fils de Maaséya, fils d’Ananya, à côté de sa maison. 
${}^{24}Après lui travailla Binnouï, fils de Hénadad, sur un autre secteur, depuis la maison d’Azarya jusqu’à l’angle et jusqu’au coin. 
${}^{25}Palal, fils d’Ouzaï, travailla en face de l’angle et de la tour qui fait saillie sur la maison du roi, celle d’en haut, près de la cour de garde. Et après lui Pedaya, fils de Paréosh, 
${}^{26}– les servants habitaient l’Ophel – jusque devant la porte des Eaux, à l’est, et devant la tour en saillie. 
${}^{27}Après lui travaillèrent les gens de Teqoa sur un autre secteur, depuis l’endroit qui fait face à la grande tour en saillie jusqu’au mur de l’Ophel. 
${}^{28}Depuis le dessus de la porte des Chevaux travaillèrent les prêtres, chacun en face de sa maison. 
${}^{29}Après eux travailla Sadoc, fils d’Immer, en face de sa maison. Après lui travailla Shemaya, fils de Shekanya, gardien de la porte de l’Orient. 
${}^{30}Après lui travailla Hananya, fils de Shèlèmya, ainsi que Hanoun, sixième fils de Çalaf, sur un autre secteur. Après lui travailla Meshoullam, fils de Bèrèkya, en face de son logement. 
${}^{31}Après lui travailla Malkiya, l’orfèvre, jusqu’à la maison des servants et des marchands, en face de la porte de la Revue, jusqu’à la chambre haute de l’angle. 
${}^{32}Entre la chambre haute de l’angle et la porte des Brebis travaillèrent les orfèvres et les marchands.
${}^{33}Lorsque Sânballath apprit que nous reconstruisions le rempart, il fut saisi de colère et se montra très irrité. Il se moqua des Juifs, 
${}^{34}et s’écria devant ses frères et devant les troupes de Samarie : « Que font donc ces misérables Juifs ? Ne sont-ils pas en train de réparer pour leur compte ? Vont-ils offrir des sacrifices ? Vont-ils terminer en un jour ? Feront-ils revivre ces pierres à partir de monceaux de décombres ? Elles sont calcinées ! » 
${}^{35}Tobie l’Ammonite se tenait à ses côtés ; il dit : « À construire comme cela, si un renard y montait, il démolirait leur muraille de pierres ! »
${}^{36}Écoute, ô notre Dieu, comme nous sommes méprisés ! Fais retomber leur insulte sur leur tête. Livre-les en butin dans un pays de captivité ! 
${}^{37}Ne pardonne pas leur faute et que leur péché ne soit pas effacé devant toi : car ils ont offensé les bâtisseurs !
${}^{38}Nous avons donc reconstruit le rempart qui fut entièrement réparé jusqu’à mi-hauteur. Le peuple mit tout son cœur à le faire.
      
         
      \bchapter{}
      \begin{verse}
${}^{1}Lorsque Sânballath, Tobie, les Arabes, les Ammonites et les Ashdodites apprirent que la réparation des remparts de Jérusalem avançait car les brèches commençaient à être comblées, leur colère fut très grande. 
${}^{2}Ils se liguèrent tous ensemble pour venir attaquer Jérusalem et y jeter la confusion. 
${}^{3}Alors nous avons invoqué notre Dieu, et à cause d’eux nous avons mis en place une garde de jour et de nuit contre eux. 
${}^{4}Mais les gens de Juda disaient : « Elle fléchit, la force du porteur : il y a trop de décombres ! Et nous, jamais nous ne pourrons reconstruire le rempart ! » 
${}^{5}Et nos ennemis déclaraient : « Ils ne se rendront compte de rien, ils ne verront rien, et alors nous surgirons parmi eux, nous les tuerons et nous mettrons fin à l’entreprise ! »
${}^{6}Mais arrivèrent des Juifs qui habitaient près d’eux et qui déjà dix fois nous avaient avertis de toutes les localités où ils habitaient : « Il faut que vous reveniez vers nous. » 
${}^{7}Je disposai le peuple en contrebas, dans l’espace derrière le rempart, aux endroits découverts ; je le disposai par clans, avec leurs épées, leurs lances et leurs arcs. 
${}^{8}Je regardai et, debout, je dis aux notables, aux magistrats et au reste du peuple : « Ne les craignez pas ! Souvenez-vous du Seigneur, grand et redoutable, et combattez pour vos frères, vos fils, vos filles, vos femmes et vos maisons ! »
${}^{9}Nos ennemis apprirent que nous étions avertis et que Dieu avait déjoué leur plan ; alors nous sommes tous retournés au rempart, chacun à son travail. 
${}^{10}Mais, à partir de ce jour, la moitié seulement de mes hommes participèrent au travail. Les autres étaient munis de lances, de boucliers, d’arcs et de cuirasses, les chefs se tenant derrière tous les gens de Juda. 
${}^{11}Ceux qui construisaient le rempart et ceux qui portaient et chargeaient les matériaux travaillaient d’une main et, de l’autre, tenaient une arme de jet. 
${}^{12}Chacun des bâtisseurs, tandis qu’il bâtissait, portait son épée attachée aux reins. Le sonneur de cor se tenait à côté de moi. 
${}^{13}Je dis aux notables, aux magistrats et au reste du peuple : « Le chantier est considérable et étendu. Or nous sommes dispersés sur le rempart, loin les uns des autres. 
${}^{14}Où que vous soyez, quand vous entendrez le son du cor, vous vous rassemblerez autour de nous. Notre Dieu combattra pour nous. » 
${}^{15}Ainsi étions-nous au travail depuis le lever de l’aurore jusqu’à l’apparition des étoiles, la moitié d’entre nous tenant des lances. 
${}^{16}C’est aussi en ce temps-là que je dis au peuple : « Chacun, avec son serviteur, passera la nuit à l’intérieur de Jérusalem, pour nous servir de garde pendant la nuit et travailler pendant le jour. » 
${}^{17}Mais ni moi, ni mes frères, ni mes serviteurs, ni les hommes de garde qui me suivaient ne quittions nos vêtements ; chacun gardait son arme.
      
         
      \bchapter{}
      \begin{verse}
${}^{1}Il s’éleva alors une grande plainte des gens du peuple et de leurs femmes contre leurs frères juifs. 
${}^{2}Les uns disaient : « Nos fils, nos filles et nous-mêmes, nous sommes nombreux. Prenons du froment et mangeons, ainsi nous survivrons. » 
${}^{3}D’autres déclaraient : « Nous devons donner en gages nos champs, nos vignes et nos maisons pour avoir du froment pendant la famine. » 
${}^{4}D’autres disaient encore : « Pour acquitter l’impôt royal, nous avons dû emprunter de l’argent sur nos champs et nos vignes. 
${}^{5}Pourtant, nous sommes de la même chair que nos frères, nos enfants valent les leurs, et nous devons livrer en esclavage nos fils et nos filles ; il en est, parmi nos filles, qui déjà sont asservies ! Nous n’y pouvons rien, puisque nos champs et nos vignes sont à d’autres. »
${}^{6}Je fus saisi d’une grande colère quand j’entendis leur plainte et de telles paroles. 
${}^{7}Je pris la décision d’admonester notables et magistrats ; je leur dis : « Quel est ce fardeau que vous faites peser sur vos frères ? » Et je convoquai contre eux une grande assemblée. 
${}^{8}Je leur dis : « Nous avons, dans la mesure de nos moyens, racheté nos frères juifs vendus aux nations. Mais vous, vous vendez vos frères, et c’est à nous qu’ils sont vendus ! » Ils gardèrent le silence et ne trouvèrent rien à répliquer. 
${}^{9}Je poursuivis : « Ce que vous faites là n’est pas bien. Ne devez-vous pas marcher dans la crainte de notre Dieu, pour éviter l’insulte des nations, nos ennemis ? 
${}^{10}Moi aussi, mes frères et mes serviteurs, nous leur avons prêté de l’argent et du froment. Eh bien ! remettons-leur cette dette. 
${}^{11}Rendez-leur immédiatement leurs champs, leurs vignes, leurs oliviers et leurs maisons, et remettez-leur ce qu’ils vous doivent en argent, froment, vin nouveau et huile fraîche que vous leur avez prêtés. » 
${}^{12}Ils répondirent : « Nous le rendrons, et nous n’exigerons plus rien d’eux ; nous agirons comme tu l’as dit. » Je convoquai les prêtres et je leur fis jurer d’agir selon cette parole. 
${}^{13}Alors je secouai le pli de mon vêtement, en disant : « Que Dieu secoue de la sorte, hors de sa maison et de ses biens, quiconque ne tiendra pas cette parole : qu’il soit ainsi secoué et dépouillé ! » Toute l’assemblée répondit : « Amen ! » et loua le Seigneur. Et le peuple agit selon ce qui avait été dit.
${}^{14}Bien plus, depuis le jour où je fus institué gouverneur au pays de Juda, depuis la vingtième année du roi Artaxerxès jusqu’à sa trente-deuxième année, pendant douze ans, ni moi ni mes frères n’avons prélevé l’impôt destiné à la subsistance du gouverneur. 
${}^{15}Or les gouverneurs précédents pressuraient le peuple : ils lui prenaient chaque jour, pour le pain et le vin, quarante pièces d’argent ; leurs serviteurs aussi opprimaient le peuple. Moi, au contraire, par crainte de Dieu, je n’ai pas agi ainsi. 
${}^{16}Je me suis également tenu au travail de ce rempart, et nous n’avons acquis aucun champ ! Tous mes serviteurs étaient là, rassemblés à la tâche. 
${}^{17}Les Juifs et les magistrats qui mangeaient à ma table étaient au nombre de cent cinquante, sans compter ceux qui nous venaient des nations environnantes. 
${}^{18}Chaque jour, on apprêtait pour moi un bœuf, six moutons de choix et des volailles ; tous les dix jours, on apportait toute sorte de vins en abondance. Malgré cela, je n’ai jamais réclamé ce qui me revenait en tant que gouverneur, car une lourde charge pesait sur ce peuple.
${}^{19}Souviens-toi, mon Dieu, pour mon bonheur, de tout ce que j’ai fait pour ce peuple !
      
         
      \bchapter{}
      \begin{verse}
${}^{1}Sânballath, Tobie, Guèshem l’Arabe et nos autres ennemis apprirent que j’avais reconstruit le rempart et qu’il n’y restait plus une seule brèche – à cette date toutefois, je n’avais pas encore posé les battants des portes. 
${}^{2}Alors Sânballath, ainsi que Guèshem, me fit dire : « Viens, rencontrons-nous à Kefirim, dans la vallée d’Ono. » Mais ils projetaient de me faire du mal. 
${}^{3}Je leur envoyai donc des messagers avec cette réponse : « Je suis occupé à un grand travail et je ne puis descendre : pourquoi le travail cesserait-il ? Devrais-je le quitter pour descendre vers vous ? » 
${}^{4}Quatre fois ils m’adressèrent la même invitation, et je leur fis la même réponse.
${}^{5}Une cinquième fois encore, Sânballath m’envoya son serviteur, porteur d’une lettre ouverte, 
${}^{6}où il était écrit : « On entend dire parmi les nations – et Gashmou le confirme – que toi et les Juifs songeriez à vous révolter. Ce serait pour cette raison que tu reconstruis le rempart ; et, selon ces dires, c’est toi qui deviendrais leur roi. 
${}^{7}Tu aurais même mis en place des prophètes pour proclamer à ton sujet dans Jérusalem : “Il y a un roi en Juda !” Et maintenant ces bruits-là vont parvenir aux oreilles du roi. Viens donc à présent, que nous tenions conseil ensemble. » 
${}^{8}Mais je lui fis répondre : « Il n’y a rien de vrai dans tes paroles ; tout cela, c’est toi-même qui l’inventes ! » 
${}^{9}Eux tous, en effet, voulaient nous effrayer, en se disant : « Leurs mains vont abandonner l’ouvrage, il ne se fera jamais. » Et maintenant, mon Dieu, fortifie mes mains ! 
${}^{10}Je me rendis donc chez Shemaya, fils de Delaya, fils de Mehétabéel, qui avait un empêchement. Il déclara : « Rencontrons-nous dans la maison de Dieu, à l’intérieur du temple. Fermons bien les portes du Temple, car on va venir pour te tuer ; oui, cette nuit, on viendra te tuer ! » 
${}^{11}Mais je répondis : « Un homme comme moi prendrait-il la fuite ? Et quel homme tel que moi pourrait entrer dans le Temple et rester en vie ? Non, je n’y entrerai pas ! » 
${}^{12}Je reconnus que ce n’était pas Dieu qui l’avait envoyé, mais qu’il avait ainsi prophétisé à mon sujet parce que Tobie et Sânballath l’avaient soudoyé. 
${}^{13}Et on l’avait soudoyé, afin que, pris de frayeur, je fasse cela et commette un péché. Ils en auraient eu prétexte à me faire mauvaise réputation. Ainsi auraient-ils pu m’outrager !
${}^{14}Souviens-toi, mon Dieu, de Tobie et de Sânballath, pour ce qu’ils ont fait, et aussi de Noadya, la prophétesse, et des autres prophètes qui voulaient m’effrayer.
${}^{15}Le rempart fut achevé en cinquante-deux jours, le 25 du mois d’Éloul. 
${}^{16}Lorsque tous nos ennemis l’apprirent, et que toutes les nations autour de nous furent saisies de crainte, ils furent abaissés à leurs propres yeux. Ils reconnurent dans ce travail l’action de notre Dieu. 
${}^{17}À cette même époque, les notables de Juda multiplièrent leurs lettres à l’adresse de Tobie, tandis que celles de Tobie leur parvenaient. 
${}^{18}Car il avait en Juda beaucoup d’alliés, étant le gendre de Shekanya, fils d’Arah, et son fils Yohanane ayant épousé la fille de Meshoullam, fils de Bérékya. 
${}^{19}Ils vantaient même, en ma présence, ses bonnes actions et lui rapportaient mes paroles. Et Tobie envoyait des lettres pour m’effrayer.
      
         
      \bchapter{}
      \begin{verse}
${}^{1}Lorsque le rempart fut reconstruit et que j’eus posé les battants des portes, les portiers, les chantres et les Lévites furent installés dans leurs fonctions. 
${}^{2}Je mis Jérusalem sous l’autorité de mon frère Hanani et de Hananya, commandant de la citadelle, qui était un homme de confiance et qui craignait Dieu plus que beaucoup d’autres. 
${}^{3}Je leur dis : « Les portes de Jérusalem ne seront pas ouvertes tant que le soleil ne chauffera pas ; et les gens seront encore debout quand on devra clore et verrouiller les battants. On instituera une garde pour les habitants de Jérusalem, chacun ayant son tour de garde, chacun devant sa maison. »
      
         
${}^{4}La ville était vaste en tous sens, elle était grande, mais la population y était peu nombreuse et les maisons n’étaient pas reconstruites. 
${}^{5}Mon Dieu m’inspira de rassembler les notables, les magistrats et le peuple pour en faire le recensement. Je trouvai le livre du recensement de ceux qui étaient revenus les premiers, et j’y trouvai les relevés suivants :
${}^{6}Voici les gens de la province qui sont montés de la captivité de l’exil. Ils avaient été déportés par Nabucodonosor, roi de Babylone, et revinrent à Jérusalem et en Juda, chacun dans sa ville. 
${}^{7}Ils arrivèrent avec Zorobabel, Josué, Néhémie, Azarya, Raamya, Nahamani, Mordokaï, Bilshane, Mispéreth, Bigwaï, Nehoum, Baana. Nombre des hommes du peuple d’Israël : 
${}^{8}les fils de Paréosh : 2 172 ; 
${}^{9}les fils de Shefatya : 372 ; 
${}^{10}les fils d’Arah : 652 ; 
${}^{11}les fils de Pahat-Moab, c’est-à-dire les fils de Josué et de Yoab : 2 818 ; 
${}^{12}les fils d’Élam : 1 254 ; 
${}^{13}les fils de Zattou : 845 ; 
${}^{14}les fils de Zakkaï : 760 ; 
${}^{15}les fils de Binnouï : 648 ; 
${}^{16}les fils de Bébaï : 628 ; 
${}^{17}les fils d’Azgad : 2 322 ; 
${}^{18}les fils d’Adoniqam : 667 ; 
${}^{19}les fils de Bigwaï : 2 067 ; 
${}^{20}les fils d’Adine : 655 ; 
${}^{21}les fils d’Ater, c’est-à-dire de Hizqiya : 98 ; 
${}^{22}les fils de Hashoum : 328 ; 
${}^{23}les fils de Béçaï : 324 ; 
${}^{24}les fils de Harif : 112 ; 
${}^{25}les fils de Gabaon : 95 ; 
${}^{26}les hommes de Bethléem et de Netofa : 188 ; 
${}^{27}les hommes d’Anatoth : 128 ; 
${}^{28}les hommes de Beth-Azmaveth : 42 ; 
${}^{29}les hommes de Qiryath-Yearim, Kefira et Beéroth : 743 ; 
${}^{30}les hommes de Rama et de Guéba : 621 ; 
${}^{31}les hommes de Mikmas : 122 ; 
${}^{32}les hommes de Béthel et de Aï : 123 ; 
${}^{33}les hommes de l’autre Nébo : 52 ; 
${}^{34}les fils de l’autre Élam : 1 254 ; 
${}^{35}les fils de Harim : 320 ; 
${}^{36}les fils de Jéricho : 345 ; 
${}^{37}les fils de Lod, Hadid et Ono : 721 ; 
${}^{38}les fils de Senaa : 3 930. 
${}^{39}Les prêtres : les fils de Yedaya, c’est-à-dire la maison de Josué : 973 ; 
${}^{40}les fils d’Immer : 1 052 ; 
${}^{41}les fils de Pashehour : 1 247 ; 
${}^{42}les fils de Harim : 1 017.
${}^{43}Les Lévites : les fils de Josué, c’est-à-dire Qadmiel, Binnouï et Hodva : 74.
${}^{44}Les chantres : les fils d’Asaph : 148.
${}^{45}Les portiers : les fils de Shalloum, les fils d’Ater, les fils de Talmone, les fils d’Aqqoub, les fils de Hatita, les fils de Shobaï : 138.
${}^{46}Les servants : les fils de Ciha, les fils de Hasoufa, les fils de Tabbaoth, 
${}^{47}les fils de Qéros, les fils de Sia, les fils de Padone, 
${}^{48}les fils de Lebana, les fils de Hagaba, les fils de Shalmaï, 
${}^{49}les fils de Hanane, les fils de Guiddel, les fils de Gahar, 
${}^{50}les fils de Reaya, les fils de Recine, les fils de Neqoda, 
${}^{51}les fils de Gazzam, les fils d’Ouzza, les fils de Paséah, 
${}^{52}les fils de Bésaï, les fils de Méounim, les fils de Nefishsim, 
${}^{53}les fils de Baqbouq, les fils de Haqoufa, les fils de Harhour, 
${}^{54}les fils de Baçlith, les fils de Mehida, les fils de Harsha, 
${}^{55}les fils de Barqos, les fils de Sissera, les fils de Témah, 
${}^{56}les fils de Neciah, les fils de Hatifa.
${}^{57}Les fils des serviteurs de Salomon : les fils de Sotaï, les fils de Soféreth, les fils de Perida, 
${}^{58}les fils de Yaala, les fils de Darqone, les fils de Guiddel, 
${}^{59}les fils de Shefatya, les fils de Hattil, les fils de Pokéreth-ha-Cebayim, les fils d’Amone.
${}^{60}Total des servants et des fils des serviteurs de Salomon : 392.
${}^{61}Voici ceux qui sont montés de Tel-Mélah, Tel-Harsha, Keroub-Addone et Immer, mais qui n’ont pu établir que leur famille et leur race étaient bien d’Israël : 
${}^{62}les fils de Delaya, les fils de Tobie, les fils de Neqoda : 642. 
${}^{63}Et parmi les prêtres, les fils de Hobaya, les fils de Haqqos, les fils de Barzillaï, celui qui avait pris pour femme l’une des filles de Barzillaï le Galaadite et avait été appelé de ce nom. 
${}^{64}Ceux-là recherchèrent leur registre généalogique, mais on ne le trouva pas. Ils furent alors déclarés impurs et exclus du sacerdoce. 
${}^{65}Et le gouverneur leur interdit de manger des aliments très saints jusqu’à ce que le prêtre se tienne devant Dieu pour le consulter par les Ourim et les Toummim.
${}^{66}L’assemblée tout entière était de 42 360 personnes, 
${}^{67}sans compter leurs serviteurs et leurs servantes, au nombre de 7 337. Il y avait aussi 245 chanteurs et chanteuses. 
${}^{68}Les chameaux étaient au nombre de 435, et les ânes de 6 720.
${}^{69}Certains chefs de famille firent des dons pour les travaux. Le gouverneur versa au trésor mille pièces d’or, cinquante coupes et cinq cent trente tuniques sacerdotales. 
${}^{70}Des chefs de famille versèrent au trésor des travaux 20 000 pièces d’or et 2 200 pièces d’argent. 
${}^{71}Quant aux dons faits par le reste du peuple, ils se montèrent à 20 000 pièces d’or, 2 000 pièces d’argent et 67 tuniques sacerdotales.
${}^{72}Alors s’installèrent prêtres, Lévites, portiers, chantres, une partie du peuple, les servants et tout Israël, dans leurs villes.
      Ainsi quand arriva le septième mois, les fils d’Israël étaient établis dans leurs villes.
      
         
      \bchapter{}
      \begin{verse}
${}^{1}Tout le peuple se rassembla comme un seul homme\\sur la place située devant la porte des Eaux. On demanda au scribe Esdras d’apporter le livre de la loi de Moïse, que le Seigneur avait prescrite\\à Israël. 
${}^{2} Alors le prêtre Esdras apporta la Loi en présence de l’assemblée, composée des hommes, des femmes, et de tous les enfants en âge de comprendre. C’était le premier jour du septième mois.
${}^{3}Esdras, tourné vers la place de la porte des Eaux, fit la lecture dans le livre, depuis le lever du jour jusqu’à midi, en présence des hommes, des femmes, et de tous les enfants en âge de comprendre : tout le peuple écoutait la lecture de la Loi.
${}^{4}Le scribe Esdras se tenait sur une tribune de bois, construite tout exprès. Près de lui se tenaient : à sa droite, Mattitya, Shèma, Anaya, Ouriya, Hilqiya et Maaséya, et, à sa gauche, Pedaya, Mishaël, Malkiya, Hashoum, Hashbaddana, Zacharie et Meshoullam. 
${}^{5}Esdras ouvrit le livre ; tout le peuple le voyait, car il dominait l’assemblée. Quand il ouvrit le livre, tout le monde se mit debout. 
${}^{6}Alors Esdras bénit le Seigneur, le Dieu très grand, et tout le peuple, levant les mains, répondit : « Amen ! Amen ! » Puis ils s’inclinèrent et se prosternèrent devant le Seigneur, le visage contre terre. 
${}^{7}Josué, Bani, Shérébya, Yamine, Aqqoub, Shabbetaï, Hodiya, Maaséya, Qelita, Azarya, Yozabad, Hanane, Pelaya, qui étaient lévites, expliquaient la Loi au peuple, pendant que le peuple demeurait debout sur place. 
${}^{8}Esdras lisait\\un passage\\dans le livre de la loi de Dieu, puis les Lévites\\traduisaient, donnaient le sens, et l’on pouvait comprendre.
${}^{9}Néhémie le gouverneur, Esdras qui était prêtre et scribe, et les Lévites qui donnaient les explications, dirent à tout le peuple : « Ce jour est consacré au Seigneur votre Dieu ! Ne prenez pas le deuil, ne pleurez pas ! » Car ils pleuraient tous\\en entendant les paroles de la Loi. 
${}^{10}Esdras leur dit encore : « Allez, mangez des viandes savoureuses, buvez des boissons aromatisées, et envoyez une part\\à celui qui n’a rien de prêt. Car ce jour est consacré à notre Dieu ! Ne vous affligez pas : la joie du Seigneur est votre rempart ! » 
${}^{11}Les Lévites calmaient tout le peuple en disant : « Cessez de pleurer\\, car ce jour est saint. Ne vous affligez pas ! »
${}^{12}Puis tout le peuple se dispersa pour aller manger, boire, envoyer des parts à ceux qui n’avaient rien de prêt\\, et se livrer à de grandes réjouissances ; en effet, ils avaient compris les paroles qu’on leur avait fait entendre.
${}^{13}Le deuxième jour, les chefs de famille de tout le peuple, les prêtres et les Lévites se rassemblèrent autour du scribe Esdras, pour scruter les paroles de la Loi. 
${}^{14}Dans la Loi que le Seigneur avait prescrite par l’intermédiaire de Moïse, ils trouvèrent écrit que les fils d’Israël devaient habiter dans des huttes durant la fête du septième mois, 
${}^{15}et qu’ils devaient l’annoncer et le faire publier dans toutes leurs villes et à Jérusalem, en ces termes : « Sortez dans la montagne et rapportez des rameaux d’olivier, d’olivier sauvage, de myrte, de palmier et d’autres arbres touffus, pour faire des huttes, comme il est écrit. » 
${}^{16}Le peuple sortit donc : ils rapportèrent des rameaux et se firent des huttes, chacun sur son toit, dans leurs propres cours, dans les cours de la maison de Dieu, ainsi que sur la place de la porte des Eaux et sur la place de la porte d’Éphraïm.
${}^{17}Toute l’assemblée – ceux qui étaient revenus de la captivité – fit donc des huttes et habita dans ces huttes. Les fils d’Israël n’avaient rien fait de tel depuis les jours de Josué, fils de Noun, jusqu’à ce jour. Ce fut une très grande joie. 
${}^{18}On lut dans le livre de la loi de Dieu chaque jour, depuis le premier jour jusqu’au dernier. Sept jours durant, on célébra la fête. Le huitième jour eut lieu, selon la coutume, la clôture de la fête.
      
         
      \bchapter{}
      \begin{verse}
${}^{1}Le vingt-quatrième jour du mois, les fils d’Israël se rassemblèrent pour un jeûne, revêtus de toile à sac et couverts de poussière. 
${}^{2}Les fils d’Israël\\s’étaient séparés de tous les étrangers. Debout, ils confessèrent leurs péchés et les fautes de leurs pères. 
${}^{3}Debout et sans quitter leur place, ils passèrent trois heures\\à lire le livre de la loi du Seigneur leur Dieu, et trois heures\\à confesser leurs péchés en se prosternant devant le Seigneur leur Dieu. 
${}^{4}Sur l’estrade des Lévites, Josué se leva avec Bani, Qadmiel, Shebanya, Bounni, Shérébya, Bani, Kenani ; ils crièrent à pleine voix devant le Seigneur leur Dieu ; 
${}^{5}et les Lévites Josué, Qadmiel, Bani, Hashabneya, Shérébya, Hodiya, Shebanya, Petahya dirent au peuple :
        \\« Levez-vous, bénissez le Seigneur votre Dieu !
        \\Depuis toujours et pour toujours,
        \\qu’il soit béni, ton nom glorieux
        qui est au-dessus de toute bénédiction et de toute louange !
        ${}^{6}C’est toi qui es le Seigneur, toi seul !
        \\C’est toi qui as fait les cieux,
        les cieux des cieux et toutes les étoiles,
        \\la terre et tout ce qu’elle porte,
        les mers et tout ce qu’elles contiennent.
        \\C’est toi qui fais vivre tout cela,
        et les astres devant toi se prosternent.
        ${}^{7}C’est toi le Seigneur Dieu,
        toi qui as choisi Abram,
        \\qui l’as fait sortir d’Our en Chaldée
        et lui as donné le nom d’Abraham.
        ${}^{8}Tu as trouvé son cœur fidèle devant toi,
        et tu as fait avec lui une Alliance
        \\pour donner à sa descendance le pays des Cananéens,
        des Hittites, des Amorites, des Perizzites,
        des Jébuséens et des Guirgashites.
        \\Et tu as tenu parole,
        car tu es juste.
        ${}^{9}Tu as vu la misère de nos pères en Égypte,
        tu as entendu leurs cris près de la mer des Roseaux.
        ${}^{10}Tu as fait des signes et des prodiges contre Pharaon,
        contre tous ses serviteurs
        et tout le peuple de son pays,
        \\car tu connaissais leur arrogance envers nos pères\\.
        Tu t’es fait un nom qui est toujours le tien.
        ${}^{11}Devant eux tu as ouvert la mer,
        ils sont passés à pied sec au milieu de la mer.
        \\Tu as repoussé leurs poursuivants dans les flots
        comme on jette une pierre dans les eaux impétueuses.
        ${}^{12}Tu les guidais le jour par une colonne de nuée,
        la nuit par une colonne de feu
        \\pour éclairer le chemin qu’ils devaient prendre.
        ${}^{13}Tu es descendu sur le mont Sinaï
        et, des cieux, tu leur as parlé ;
        \\tu leur as donné des ordonnances justes, des lois sûres,
        de bons décrets et commandements ;
        ${}^{14}tu leur as fait connaître le sabbat, que tu as consacré,
        tu leur as prescrit des commandements,
        \\des décrets et une Loi
        par ton serviteur Moïse.
        ${}^{15}Quand ils ont eu faim,
        tu leur as donné le pain venu du ciel ;
        \\quand ils ont eu soif,
        tu as fait jaillir l’eau du rocher.
        \\Tu leur as dit d’aller prendre possession du pays
        que tu avais fait serment de leur donner.
        ${}^{16}Mais eux aussi, nos pères, furent arrogants,
        ils ont raidi leur nuque,
        \\ils n’ont pas obéi à tes commandements,
        ${}^{17}ils ont refusé d’obéir,
        \\ils n’ont pas gardé le souvenir des merveilles
        que tu avais faites pour eux ;
        \\ils ont raidi leur nuque, ils se sont révoltés
        et se sont donné un chef
        afin de retourner à leur esclavage.
        \\Mais tu es le Dieu qui pardonne,
        tendre et miséricordieux,
        \\lent à la colère et plein d’amour :
        tu ne les as pas abandonnés !
        ${}^{18}Ils se sont même fabriqué un veau en métal fondu,
        et ils ont déclaré :
        \\“Voici ton Dieu, qui t’a fait monter d’Égypte !”
        \\Et malgré tous leurs grands blasphèmes,
        ${}^{19}même alors, dans ton immense tendresse,
        tu ne les as pas abandonnés dans le désert ;
        \\la colonne de nuée ne se retira pas loin d’eux,
        elle les guidait sur le chemin pendant le jour,
        \\et la colonne de feu pendant la nuit
        les éclairait sur le chemin qu’ils devaient prendre.
        ${}^{20}Tu leur as donné ton esprit bienfaisant pour les instruire,
        tu n’as pas refusé la manne à leur bouche
        et tu leur as donné l’eau pour leur soif.
${}^{21}Pendant quarante ans, dans le désert, tu as pris soin d’eux
        et ils ne manquèrent de rien ;
        \\leurs vêtements ne s’usèrent pas
        et leurs pieds n’enflèrent pas.
${}^{22}Tu leur as livré des royaumes et des peuples,
        que tu as répartis à leurs frontières :
        \\ils ont pris possession du pays de Séhone, roi de Heshbone,
        et du pays d’Og, roi du Bashane.
${}^{23}Tu as multiplié leurs fils comme les étoiles du ciel
        et tu les as introduits dans le pays
        \\où tu avais dit à leurs pères d’entrer
        pour en prendre possession.
${}^{24}Les fils y sont entrés,
        ils ont pris possession du pays,
        \\et tu as abaissé devant eux ses habitants, les Cananéens ;
        tu les as livrés entre leurs mains,
        \\eux, leurs rois et les peuples du pays,
        pour qu’ils les traitent à leur gré.
${}^{25}Ils s’emparèrent de villes fortifiées
        et d’une terre fertile ;
        \\ils prirent possession de maisons
        remplies de tous les biens,
        \\de citernes déjà creusées, de vignes, d’oliviers,
        d’arbres fruitiers à profusion :
        \\ils mangèrent, ils se rassasièrent, ils engraissèrent,
        et, par ta grande bonté, ils furent dans les délices.
${}^{26}Mais voici qu’ils ont été rebelles,
        ils se sont révoltés contre toi,
        \\ils ont rejeté ta Loi derrière leur dos ;
        \\ils ont tué les prophètes
        qui les adjuraient de revenir à toi ;
        \\ils ont été coupables de grands blasphèmes.
${}^{27}Alors tu les as livrés aux mains de leurs adversaires
        qui les opprimaient.
        \\Au temps de leur oppression, ils criaient vers toi,
        et toi, du ciel, tu les entendais :
        \\dans ton immense tendresse tu leur donnais des sauveurs
        qui les sauvaient de la main de leurs adversaires.
${}^{28}Mais, dès qu’ils avaient un répit,
        ils recommençaient à faire le mal devant toi,
        \\et tu les abandonnais aux mains de leurs ennemis
        qui les dominaient.
        \\Eux, de nouveau, criaient vers toi,
        et toi, du ciel, tu les entendais.
        \\Que de fois, dans ta tendresse,
        tu les as délivrés !
${}^{29}Tu les adjurais de revenir à ta Loi :
        \\mais eux se montraient arrogants,
        ils n’obéissaient pas à tes commandements.
        \\Ainsi ont-ils péché contre tes ordonnances,
        où l’homme qui les observe trouve la vie.
        \\Ils ont présenté une épaule rebelle,
        ils ont raidi leur nuque et n’ont pas obéi.
${}^{30}De longues années, tu les as supportés ;
        \\par ton esprit, tu les adjurais
        par la voix de tes prophètes,
        mais ils n’ont pas prêté l’oreille.
        \\Alors tu les as livrés
        aux mains des peuples des autres pays.
${}^{31}Dans ton immense miséricorde,
        tu ne les as pas exterminés,
        \\tu ne les as pas abandonnés,
        car tu es un Dieu tendre et miséricordieux.
${}^{32}Et maintenant, ô notre Dieu,
        toi le Dieu grand, puissant et redoutable,
        qui gardes l’alliance et la fidélité,
        \\que ne soit pas tenue pour peu de chose devant toi
        toute l’affliction qui nous a atteints,
        \\nous, nos rois, nos princes, nos prêtres,
        nos prophètes, nos pères et tout ton peuple,
        \\depuis le temps des rois d’Assour
        jusqu’à ce jour.
${}^{33}Toi, tu es juste en tout ce qui nous est advenu,
        car tu as agi avec vérité,
        \\tandis que nous, nous avons commis le mal.
${}^{34}Oui, nos rois, nos chefs, nos prêtres et nos pères
        n’ont pas mis ta Loi en pratique,
        \\ils n’ont pas prêté attention à tes commandements,
        ni aux exigences dont tu témoignais envers eux.
${}^{35}Au temps de la royauté,
        \\alors que tu leur avais accordé une grande prospérité
        dans le pays vaste et fertile que tu avais mis devant eux,
        \\ils ne t’ont pas servi,
        ils ne se sont pas détournés de leurs actions mauvaises.
${}^{36}Aujourd’hui, voici que nous sommes asservis !
        \\Ce pays que tu avais donné à nos pères
        pour jouir de ses fruits et de ses biens,
        \\nous y sommes en servitude.
${}^{37}Ses produits abondants profitent aux rois
        que tu nous as imposés à cause de nos péchés,
        \\et qui disposent à leur gré de nos personnes
        et de notre bétail.
        \\Nous sommes dans une grande détresse. »
      
         
      \bchapter{}
      \begin{verse}
${}^{1}En conséquence, nous prenons un engagement ferme, par écrit. Sur ce texte scellé figurent nos chefs, nos Lévites et nos prêtres.
${}^{2}Sur les documents scellés figurent :
      le gouverneur Néhémie, fils de Hakalya, et Cidqiya,
${}^{3}Seraya, Azarya, Jérémie, 
${}^{4}Pashehour, Amarya, Malkiya, 
${}^{5}Hattoush, Shebanya, Mallouk, 
${}^{6}Harim, Merémoth, Abdias, 
${}^{7}Daniel, Guinnetone, Baruc, 
${}^{8}Meshoullam, Abiya, Miyamine, 
${}^{9}Maazya, Bilgaï, Shemaya : ce sont les prêtres.
${}^{10}Puis les Lévites : Josué, fils d’Azanya, Binnouï, des fils de Hénadad, Qadmiel, 
${}^{11}et leurs frères Shebanya, Hodiya, Qelita, Pelaya, Hanane, 
${}^{12}Mika, Rehob, Hashabya, 
${}^{13}Zakkour, Shérébya, Shebanya, 
${}^{14}Hodiya, Bani, Beninou.
${}^{15}Les chefs du peuple : Paréosh, Pahath-Moab, Élam, Zattou, Bani, 
${}^{16}Bounni, Azgad, Bébaï, 
${}^{17}Adonias, Bigwaï, Adine, 
${}^{18}Ater, Ézékias, Azzour, 
${}^{19}Hodiya, Hashoum, Béçaï, 
${}^{20}Harif, Anatoth, Nébaï, 
${}^{21}Magpiash, Meshoullam, Hézir, 
${}^{22}Meshézabéel, Sadoc, Yaddoua, 
${}^{23}Pelatya, Hanane, Anaya, 
${}^{24}Osée, Hananya, Hashshoub, 
${}^{25}Hallohesh, Pilha, Shobèq, 
${}^{26}Rehoum, Hashabna, Maaséya, 
${}^{27}Ahiya, Hanane, Anane, 
${}^{28}Mallouk, Harim, Baana.
${}^{29}Le reste du peuple, prêtres, Lévites, portiers, chantres, servants, tous ceux qui se sont séparés des peuples des autres pays pour se tourner vers la loi de Dieu, ainsi que leurs femmes, leurs fils et leurs filles, tous ceux qui peuvent comprendre, 
${}^{30}se joignent à leurs frères et aux notables ; ils s’engagent, par promesse et serment, à marcher selon la loi de Dieu, donnée par l’intermédiaire de Moïse, le serviteur de Dieu, en gardant et observant tous les commandements du Seigneur notre Dieu, ses ordonnances et ses décrets.
${}^{31}En conséquence, nous ne donnerons pas nos filles aux gens du pays et nous ne prendrons pas leurs filles pour nos fils. 
${}^{32}Si les gens du pays apportent pour les vendre, le jour du sabbat, des marchandises ou quelque denrée que ce soit, nous ne leur achèterons rien ni pendant le sabbat ni un jour de fête. Et, la septième année, nous renoncerons aux produits du sol et remettrons les dettes de toute sorte.
${}^{33}Nous nous sommes fixé pour règle de donner un tiers de sicle par an pour le culte dans la maison de notre Dieu, 
${}^{34}pour le pain de l’offrande de céréales, l’offrande perpétuelle et l’holocauste perpétuel, pour les sabbats, les nouvelles lunes, les solennités, et pour les choses consacrées, pour les sacrifices d’expiation en faveur d’Israël, pour tout le service de la maison de notre Dieu.
${}^{35}Nous – les prêtres, les Lévites et le peuple – nous avons aussi réglé par le sort la question de l’offrande de bois que l’on doit apporter à la maison de notre Dieu, chaque famille à son tour, aux temps fixés, chaque année, pour allumer le feu sur l’autel du Seigneur notre Dieu, comme il est écrit dans la Loi. 
${}^{36}De même, on doit apporter chaque année à la maison de notre Dieu les prémices de notre sol, les prémices de tous les fruits de tout arbre, 
${}^{37}ainsi que les premiers-nés de nos fils et de notre bétail, comme il est écrit dans la Loi. Les premiers-nés de notre gros et de notre petit bétail qu’on apporte à la maison de notre Dieu sont destinés aux prêtres en fonction dans la maison de notre Dieu. 
${}^{38}De plus, les premières de nos fournées, nos contributions, fruits de tout arbre, vin nouveau, huile fraîche, nous les apporterons aux prêtres, dans les salles de la maison de notre Dieu ; la dîme de notre sol ira aux Lévites, et ce sont les Lévites eux-mêmes qui percevront la dîme dans toutes les villes où nous travaillons. 
${}^{39}Un prêtre, fils d’Aaron, accompagnera les Lévites quand ils percevront la dîme ; les Lévites prélèveront le dixième de la dîme pour la maison de notre Dieu, et l’apporteront dans les salles de la maison du Trésor. 
${}^{40}C’est, en effet, dans ces salles que les fils d’Israël et les fils de Lévi apportent la contribution en froment, vin nouveau et huile fraîche. Là se trouvent aussi les objets du sanctuaire, les prêtres en fonction, les portiers et les chantres. Ainsi, nous ne négligerons pas la maison de notre Dieu.
      
         
      \bchapter{}
      \begin{verse}
${}^{1}Alors les princes du peuple s’établirent à Jérusalem. Le reste du peuple tira au sort pour faire venir un homme sur dix habiter à Jérusalem, la Ville sainte, tandis que les neuf autres resteraient dans leurs villes. 
${}^{2}De plus, le peuple bénit tous les hommes qui furent volontaires pour habiter à Jérusalem.
${}^{3}Voici les chefs de la province qui s’établirent à Jérusalem. – Dans les autres villes de Juda habitaient les Israélites, les prêtres, les Lévites, les servants et les fils des serviteurs de Salomon, chacun en sa propriété, dans leurs propres villes. 
${}^{4}À Jérusalem donc habitaient des fils de Juda et des fils de Benjamin.
      Ce sont, parmi les fils de Juda : Ataya, fils d’Ouzziya, fils de Zacharie, fils d’Amarya, fils de Shefatya, fils de Mahalaléel, d’entre les fils de Pérès ; 
${}^{5}Maaséya, fils de Baruc, fils de Kol-Hozé, fils de Hazaya, fils d’Adaya, fils de Yoyarib, fils de Zacharie, fils du Silonite. 
${}^{6}Le total des fils de Pérès habitant à Jérusalem était de 468 hommes d’armes.
${}^{7}Voici les fils de Benjamin : Sallou, fils de Meshoullam, fils de Yoëd, fils de Pedaya, fils de Qolaya, fils de Maaséya, fils d’Itiel, fils d’Isaïe, 
${}^{8}et après lui, Gabbaï, Sallaï ; ils étaient 928. 
${}^{9}Joël, fils de Zikri, était leur inspecteur, et Juda, fils de Hassenoua, était le second personnage de la ville.
${}^{10}Parmi les prêtres : Yedaya, fils de Yoyarib Yakine, 
${}^{11}Seraya, fils d’Hilqiya, fils de Meshoullam, fils de Sadoc, fils de Merayoth, fils d’Ahitoub, recteur de la maison de Dieu, 
${}^{12}et ses frères, chargés du service de la Maison ; ils étaient 822. Adaya, fils de Yeroham, fils de Pelalya, fils d’Amci, fils de Zacharie, fils de Pashehour, fils de Malkiya, 
${}^{13}et ses frères, chefs de famille ; ils étaient 242. Et Amashesaï, fils d’Azarel, fils d’Ahzaï, fils de Meshillémoth, fils d’Immer, 
${}^{14}et ses frères, vaillants guerriers ; ils étaient 128. Zabdiel, fils de Ha-Guedolim, était leur inspecteur.
${}^{15}Parmi les Lévites : Shemaya, fils de Hashshoub, fils d’Azriqam, fils de Hashabya, fils de Bounni ; 
${}^{16}Shabbetaï et Yozabad, ceux des chefs lévitiques responsables des travaux extérieurs de la maison de Dieu ; 
${}^{17}Mattanya, fils de Mika, fils de Zabdi, fils d’Asaph, le chef de chœur, entonnait la prière de louange ; Baqbouqya, le second parmi ses frères ; Abda, fils de Shammoua, fils de Galal, fils de Yedoutoune. 
${}^{18}Le total des Lévites dans la Ville sainte était de 284.
       
${}^{19}Les portiers : Aqqoub, Talmone avec leurs frères, montaient la garde aux portes ; ils étaient 172.
${}^{20}Le reste d’Israël, avec les prêtres et les Lévites, se trouvaient dans toutes les villes de Juda, chacun dans son héritage. 
${}^{21}Les servants habitaient l’Ophel ; Ciha et Guishpa étaient à la tête des servants. 
${}^{22}L’inspecteur des Lévites de Jérusalem était Ouzzi, fils de Bani, fils de Hashabya, fils de Mattanya, fils de Mika ; il faisait partie des fils d’Asaph, les chantres chargés du service de la maison de Dieu. 
${}^{23}Il y avait en effet une prescription du roi à leur sujet, et un règlement concernant les chantres pour chaque jour. 
${}^{24}Petahya, fils de Meshézabéel, l’un des fils de Zérah, fils de Juda, était au côté du roi pour toutes les affaires du peuple.
${}^{25}Dans les villages des campagnes habitèrent des fils de Juda : à Qiryath-ha-Arba et ses dépendances, à Dibone et ses dépendances, à Yeqqabcéel et ses villages, 
${}^{26}à Yéshoua, à Molada, à Beth-Pèleth, 
${}^{27}à Haçar-Shoual, à Bershéba et ses dépendances, 
${}^{28}à Ciqlag, à Mekona et ses dépendances, 
${}^{29}à Enn-Rimmone, à Soréa, à Yarmouth, 
${}^{30}Zanoah, Adoullam et leurs villages, Lakish et ses campagnes, Azéqa et ses dépendances. Ils s’établirent donc de Bershéba jusqu’à la vallée de Hinnome.
${}^{31}Des fils de Benjamin, depuis Guéba, s’établirent à Mikmas, Aya, Béthel et ses dépendances, 
${}^{32}à Anatoth, Nob, Ananya, 
${}^{33}Haçor, Rama, Guittaïm, 
${}^{34}Hadid, Seboïm, Neballath, 
${}^{35}Lod et Ono, la vallée des Artisans. 
${}^{36}Parmi les Lévites, certains des régions de Juda allèrent en Benjamin.
      
         
      \bchapter{}
      \begin{verse}
${}^{1}Voici les prêtres et les Lévites qui montèrent avec Zorobabel, fils de Salathiel, et Josué : Seraya, Jérémie, Esdras, 
${}^{2}Amarya, Mallouk, Hattoush, 
${}^{3}Shekanya, Rehoum, Merémoth, 
${}^{4}Iddo, Guinnetoï, Abiya, 
${}^{5}Miyamine, Maadya, Bilga, 
${}^{6}Shemaya ; et Yoyarib, Yedaya, 
${}^{7}Sallou, Amoq, Hilqiya, Yedaya. Tels étaient les chefs des prêtres, et leurs frères, au temps de Josué.
${}^{8}Les Lévites étaient Josué, Binnouï, Qadmiel, Shérébya, Juda, Mattanya qui, avec ses frères, était chargé des chants de louange, 
${}^{9}tandis que Baqbouqya, Ounni, leurs frères, leur faisaient face pour les offices.
${}^{10}Josué engendra Joakim ; Joakim engendra Élyashib ; Élyashib, Yoyada ; 
${}^{11}Yoyada engendra Jonathan ; et Jonathan engendra Yaddoua.
${}^{12}Au temps de Joakim, les prêtres chefs des familles étaient : pour la famille de Seraya, Meraya ; pour celle de Jérémie, Hananya ; 
${}^{13}pour celle d’Esdras, Meshoullam ; pour celle d’Amarya, Yehohanane ; 
${}^{14}pour celle de Melikou, Jonathan ; pour celle de Shebanya, Joseph ; 
${}^{15}pour celle de Harim, Adna ; pour celle de Merayoth, Helqaï ; 
${}^{16}pour celle d’Iddo, Zacharie ; pour celle de Guinnetone, Meshoullam ; 
${}^{17}pour celle d’Abiya, Zikri ; pour celle de Minyamine, ... ; pour celle de Moadya, Piltaï ; 
${}^{18}pour celle de Bilga, Shammoua ; pour celle de Shemaya, Jonathan ; 
${}^{19}pour celle de Yoyarib, Mattenaï ; pour celle de Yedaya, Ouzzi ; 
${}^{20}pour celle de Sallaï, Qallaï ; pour celle d’Amoq, Éber ; 
${}^{21}pour celle de Hilqiya, Hashabya ; pour celle de Yedaya, Netanel.
${}^{22}Au temps d’Élyashib, de Yoyada, de Yohanane et de Yaddoua, les Lévites, chefs de famille, ainsi que les prêtres furent enregistrés jusqu’au règne de Darius le Perse. 
${}^{23}Les fils de Lévi, chefs des familles, furent inscrits au livre des Annales jusqu’au temps de Yohanane, fils d’Élyashib. 
${}^{24}Les chefs des Lévites étaient Hashabya, Shérébya, Josué, Binnouï, Qadmiel, et leurs frères qui leur faisaient face pour louer et rendre grâce selon les prescriptions de David, l’homme de Dieu, chacun d’après leur tour de service. 
${}^{25}Mattanya, Baqbouqya, Abdias, Meshoullam, Talmone et Aqqoub, gardiens-portiers, montaient la garde aux entrepôts près des portes. 
${}^{26}Ceux-là vivaient au temps de Joakim, fils de Josué, fils de Yoçadaq, et au temps de Néhémie le gouverneur et d’Esdras le prêtre-scribe.
${}^{27}Pour la dédicace du rempart de Jérusalem, on alla chercher les Lévites partout où ils habitaient ; on les fit venir à Jérusalem pour célébrer la dédicace dans l’allégresse par des actions de grâce et des chants, avec cymbales, harpes et cithares. 
${}^{28}Les fils des chantres se rassemblèrent donc, depuis la région qui entoure Jérusalem, et depuis les villages des Netofatites, 
${}^{29}depuis Beth-ha-Guilgal, et les campagnes de Guéba et d’Azmaveth ; les chantres, en effet, s’étaient bâti des villages tout autour de Jérusalem. 
${}^{30}Prêtres et Lévites se purifièrent eux-mêmes, puis ils purifièrent le peuple, les portes et le rempart.
${}^{31}Je fis alors monter les princes de Juda sur le rempart et j’organisai deux grands chœurs. Le premier chœur marcha sur le sommet du rempart, vers la droite, en direction de la porte du Fumier. 
${}^{32}Derrière eux marchaient Hoshaya et la moitié des princes de Juda, 
${}^{33}ainsi qu’Azarya, Esdras et Meshoullam, 
${}^{34}Juda, Benjamin, Shemaya et Jérémie, 
${}^{35}choisis parmi les prêtres et munis de trompettes ; puis Zacharie, fils de Jonathan, fils de Shemaya, fils de Mattanya, fils de Mikaya, fils de Zakkour, fils d’Asaph, 
${}^{36}avec ses frères Shemaya, Azarel, Milalaï, Guilalaï, Maaï, Netanel, Juda, Hanani, munis des instruments de musique de David, l’homme de Dieu. Esdras, le scribe, marchait devant eux. 
${}^{37}Près de la porte de la Source, droit devant eux, ils gravirent les marches de la Cité de David, par la montée du rempart, au-dessus de la maison de David, jusqu’à la porte des Eaux, à l’est.
${}^{38}Le second chœur marcha vers la gauche ; je le suivis, avec la moitié du peuple, par la montée du rempart, au-dessus de la tour des Fours et jusqu’à la partie large du rempart. 
${}^{39}Puis on alla au-dessus de la porte d’Éphraïm, la porte Vieille, la porte des Poissons, la tour de Hananéel et la tour des Cent, jusqu’à la porte des Brebis. On fit halte à la porte de la Garde.
${}^{40}Les deux chœurs prirent ensuite place dans la maison de Dieu. – J’avais avec moi la moitié des magistrats, 
${}^{41}ainsi que les prêtres Élyaqim, Maaséya, Minyamine, Mikaya, Élyoénaï, Zacharie, Hananya, munis de trompettes, 
${}^{42}et Maaséya, Shemaya, Éléazar, Ouzzi, Yehohanane, Malkiya, Élam et Ézer. Les chantres se firent entendre sous la direction de Yizrahya.
${}^{43}On offrit ce jour-là de grands sacrifices et les gens se livrèrent à la joie, car Dieu leur avait donné un grand sujet de joie ; femmes et enfants se réjouirent aussi. Et la joie de Jérusalem s’entendit de loin.
${}^{44}En ce jour-là, on préposa aux salles prévues pour les provisions, contributions, prémices et dîmes, des hommes qui y rassembleraient, de la campagne entourant les villes, les parts que la Loi attribue aux prêtres et aux Lévites. Car Juda mettait sa joie dans les prêtres et les Lévites en fonction. 
${}^{45}Ce sont eux qui assuraient le service de leur Dieu et le service des purifications rituelles, tandis que les chantres et les portiers suivaient les prescriptions de David et de son fils Salomon. 
${}^{46}Car autrefois, au temps de David et d’Asaph, il y avait des chefs pour les chantres et il y avait des chants de louange et d’action de grâce envers Dieu. 
${}^{47}Donc au temps de Zorobabel et au temps de Néhémie, tout Israël versait aux chantres et aux portiers les parts qui leur revenaient pour chaque jour, et donnait aux Lévites les offrandes consacrées ; et les Lévites en donnaient aux fils d’Aaron.
      
         
      \bchapter{}
      \begin{verse}
${}^{1}En ce temps-là, on lut dans le livre de Moïse en présence du peuple et l’on y trouva écrit que l’Ammonite et le Moabite n’entreront pas dans l’assemblée de Dieu, et cela jamais. 
${}^{2}Car ils ne sont pas venus au-devant des fils d’Israël avec le pain et l’eau. Ils ont soudoyé contre eux Balaam pour les maudire, mais notre Dieu changea la malédiction en bénédiction. 
${}^{3}Dès qu’on eut entendu la Loi, on exclut d’Israël tous les étrangers.
${}^{4}Auparavant, le prêtre Élyashib avait été chargé des salles de la maison de notre Dieu. C’était un proche de Tobie. 
${}^{5}Il lui avait aménagé une grande salle, où l’on mettait auparavant les offrandes de céréales, l’encens, les objets du culte, la dîme de froment, de vin nouveau et d’huile fraîche, c’est-à-dire ce qui est prescrit pour les Lévites, les chantres et les portiers, de même que la contribution pour les prêtres.
${}^{6}Durant tout ce temps, j’étais absent de Jérusalem, car dans la trente-deuxième année d’Artaxerxès, roi de Babylone, j’étais revenu auprès du roi. Au bout d’un certain temps, je demandai au roi un congé 
${}^{7}et je revins à Jérusalem. J’appris alors qu’Élyashib avait mal agi à propos de Tobie en lui aménageant une salle dans les cours de la maison de Dieu. 
${}^{8}Cela me déplut fort : je jetai donc hors de la salle toutes les affaires de Tobie, 
${}^{9}et j’ordonnai qu’on purifie les salles ; puis j’y rapportai les objets de la maison de Dieu, les offrandes de céréales et l’encens.
${}^{10}J’appris également que les parts des Lévites n’avaient pas été données, et que les Lévites et les chantres chargés du service s’étaient enfuis chacun vers son champ. 
${}^{11}Je fis donc des reproches aux magistrats et je leur dis : « Pourquoi la maison de Dieu est-elle à l’abandon ? » Puis je les réunis et les rétablis à leur poste. 
${}^{12}Alors tout Juda apporta la dîme du froment, du vin nouveau et de l’huile fraîche pour les mettre dans les réserves. 
${}^{13}Je préposai aux réserves le prêtre Shèlèmya, le scribe Sadoc, et Pedaya, l’un des Lévites, et, auprès d’eux, Hanane, fils de Zakkour, fils de Mattanya, car ils étaient considérés comme dignes de confiance ; c’est à eux qu’il revenait de faire les distributions à leurs frères.
${}^{14}À cause de cela, souviens-toi de moi, mon Dieu ; n’efface pas la fidélité avec laquelle j’ai agi pour la maison de mon Dieu et pour ses offices.
${}^{15}En ce temps-là, je vis en Juda des gens qui foulaient au pressoir, le jour du sabbat. D’autres apportaient des gerbes de blé, les chargeaient sur des ânes, avec du vin, des raisins, des figues et toute sorte de fardeaux, pour les apporter à Jérusalem le jour du sabbat. En ce jour où ils vendaient leurs denrées, je fis un avertissement. 
${}^{16}À Jérusalem, les Tyriens, qui habitaient là, faisaient venir du poisson et des marchandises de toute sorte pour les vendre aux Judéens le jour du sabbat. 
${}^{17}Je fis donc des reproches aux notables de Juda, et je leur dis : « Vous agissez bien mal, en profanant le jour du sabbat ! 
${}^{18}N’est-ce pas ainsi que vos pères ont agi ? Alors notre Dieu a fait venir tout ce malheur sur nous et sur cette ville. Et vous, en profanant le sabbat, vous augmentez sa colère contre Israël. » 
${}^{19}Aussi, dès que l’ombre eut gagné les portes de Jérusalem, juste avant le sabbat, j’ordonnai de fermer les battants et je dis de ne les rouvrir qu’après le sabbat. Je postai quelques-uns de mes serviteurs aux portes, pour qu’aucun fardeau n’entre le jour du sabbat. 
${}^{20}Une fois ou deux, des commerçants et des marchands en toute sorte de marchandises passèrent la nuit hors de Jérusalem. 
${}^{21}Mais je les avertis, et leur dis : « Pourquoi passer la nuit aux abords du rempart ? Si vous recommencez, je porterai la main sur vous ! » Depuis lors, ils ne sont plus venus le jour du sabbat. 
${}^{22}J’ordonnai aux Lévites de se purifier et de venir surveiller les portes, pour que l’on sanctifie le jour du sabbat.
      À cause de cela aussi, souviens-toi de moi, mon Dieu, et prends-moi en pitié, en ta grande fidélité !
${}^{23}En ce temps-là également, je vis des Juifs qui avaient épousé des femmes ashdodites, ammonites ou moabites. 
${}^{24}Quant à leurs enfants, la moitié parlait l’ashdodien ou la langue de tel ou tel peuple, mais aucun ne savait plus parler le judéen. 
${}^{25}Je leur fis des reproches et je les maudis. J’en frappai plusieurs, je leur arrachai les cheveux et leur fit prêter serment devant Dieu, en leur disant : « Vous ne donnerez pas vos filles à leurs fils, vous ne prendrez pour femmes aucune de leurs filles, ni pour vos fils ni pour vous-mêmes ! 
${}^{26}N’est-ce pas en cela qu’a péché Salomon, roi d’Israël ? Parmi tant de nations, aucun roi ne fut pareil à lui ; il était aimé de son Dieu, et Dieu le fit roi sur tout Israël. Même lui, les femmes étrangères l’entraînèrent à pécher ! 
${}^{27}Faudra-t-il entendre dire que vous agissez tout aussi mal : trahir notre Dieu en prenant pour femmes des étrangères ? »
${}^{28}L’un des fils de Juda, fils du grand prêtre Élyashib, était le gendre de Sânballath le Horonite. Je le chassai loin de moi.
${}^{29}Souviens-toi de ces gens, mon Dieu : ils ont souillé le sacerdoce et l’alliance des prêtres et des Lévites.
       
${}^{30}Je les purifiai donc de tout étranger. Je remis en vigueur les règles à observer par les prêtres et les Lévites, chacun selon sa fonction. 
${}^{31}Je fis de même pour l’offrande du bois aux dates fixées, ainsi que pour les prémices.
      Souviens-toi de moi, mon Dieu, pour mon bonheur !
