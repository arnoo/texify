  
  
    
    \bbook{DEUXIÈME LETTRE AUX THESSALONICIENS}{DEUXIÈME LETTRE AUX THESSALONICIENS}
      
         
      \bchapter{}
        ${}^{1}Paul, Silvain et Timothée,
        \\à l’Église de Thessalonique
        qui est en Dieu notre Père
        et dans le Seigneur Jésus Christ.
        ${}^{2}À vous, la grâce et la paix
        \\de la part de Dieu notre Père
        et du Seigneur Jésus Christ.
        
           
${}^{3}Frères, à tout moment nous devons rendre grâce à Dieu à votre sujet, et c’est bien de le faire, étant donné les grands progrès de votre foi, et l’amour croissant que tous et chacun, vous avez les uns pour les autres. 
${}^{4}C’est pourquoi nous-mêmes sommes fiers de vous au milieu des Églises de Dieu, à cause de votre endurance et de votre foi dans toutes les persécutions et les détresses que vous supportez. 
${}^{5}Il y a là un signe du juste jugement de Dieu ; ainsi vous deviendrez dignes de son Royaume pour lequel vous souffrez. 
${}^{6}C’est justice, en effet, que Dieu rende la détresse à ceux qui vous l’infligent, 
${}^{7}et qu’il vous accorde, à vous qui subissez la détresse, le soulagement avec nous lorsque, du haut du ciel, le Seigneur Jésus se révélera avec les anges, messagers de sa puissance, 
${}^{8}dans le feu flamboyant ; alors il fera justice contre ceux qui ignorent Dieu et à ceux qui n’obéissent pas à l’Évangile de notre Seigneur Jésus. 
${}^{9}Ceux-là subiront comme châtiment la ruine éternelle, loin de la face du Seigneur et de la gloire de sa force, 
${}^{10}quand il viendra en ce jour-là pour être glorifié dans ses saints et admiré en tous ceux qui ont cru ; or vous, vous avez cru à notre témoignage. 
${}^{11}C’est pourquoi nous prions pour vous à tout moment afin que notre Dieu vous trouve dignes de l’appel qu’il vous a adressé ; par sa puissance, qu’il vous donne d’accomplir tout le bien que vous désirez, et qu’il rende active votre foi. 
${}^{12}Ainsi, le nom de notre Seigneur Jésus sera glorifié en vous, et vous en lui, selon la grâce de notre Dieu et du Seigneur Jésus Christ.
      
         
      \bchapter{}
      \begin{verse}
${}^{1}Frères, nous avons une demande à vous faire à propos de la venue de notre Seigneur Jésus Christ et de notre rassemblement auprès de lui : 
${}^{2}si l'on nous attribue une inspiration, une parole ou une lettre prétendant que le jour du Seigneur est arrivé, n'allez pas aussitôt perdre la tête, ne vous laissez pas effrayer.
${}^{3}Ne laissez personne vous égarer d’aucune manière. Car il faut que vienne d’abord l’apostasie, et que se révèle l’Homme de l’impiété, le fils de perdition, 
${}^{4}celui qui s’oppose, et qui s’élève contre tout ce que l’on nomme Dieu ou que l’on vénère, et qui va jusqu’à siéger dans le temple de Dieu en se faisant passer lui-même pour Dieu. 
${}^{5}Ne vous souvenez-vous pas que je vous en ai parlé quand j’étais encore chez vous ? 
${}^{6}Maintenant vous savez ce qui le retient, de sorte qu’il ne se révélera qu’au temps fixé pour lui. 
${}^{7}Car le mystère d’iniquité est déjà à l’œuvre ; il suffit que soit écarté celui qui le retient à présent. 
${}^{8}Alors sera révélé l’Impie, que le Seigneur Jésus supprimera par le souffle de sa bouche et fera disparaître par la manifestation de sa venue. 
${}^{9}La venue de l’Impie, elle, se fera par la force de Satan avec une grande puissance, des signes et des prodiges trompeurs, 
${}^{10}avec toute la séduction du mal, pour ceux qui se perdent du fait qu’ils n’ont pas accueilli l’amour de la vérité, ce qui les aurait sauvés. 
${}^{11}C’est pourquoi Dieu leur envoie une force d’égarement qui les fait croire au mensonge ; 
${}^{12}ainsi seront jugés tous ceux qui n’ont pas cru à la vérité, mais qui se sont complus dans le mal.
${}^{13}Quant à nous, à tout moment nous devons rendre grâce à Dieu à votre sujet, frères, vous qui êtes aimés du Seigneur, puisque Dieu vous a choisis en premier pour être sauvés par l’Esprit qui sanctifie et par la foi en la vérité. 
${}^{14}C’est à cela que Dieu vous a appelés par notre proclamation de l’Évangile, pour que vous entriez en possession de la gloire de notre Seigneur Jésus Christ.
${}^{15}Ainsi donc, frères, tenez bon, et gardez ferme les traditions que nous vous avons enseignées, soit de vive voix, soit par lettre. 
${}^{16}Que notre Seigneur Jésus Christ lui-même, et Dieu notre Père qui nous a aimés et nous a pour toujours donné réconfort et bonne espérance par sa grâce, 
${}^{17}réconfortent vos cœurs et les affermissent en tout ce que vous pouvez faire et dire de bien.
      
         
      \bchapter{}
      \begin{verse}
${}^{1}Priez aussi pour nous, frères, afin que la parole du Seigneur poursuive sa course, et que, partout, on lui rende gloire comme chez vous. 
${}^{2}Priez pour que nous échappions aux gens pervers et mauvais, car tout le monde n’a pas la foi. 
${}^{3}Le Seigneur, lui, est fidèle : il vous affermira et vous protégera du Mal. 
${}^{4}Et, dans le Seigneur, nous avons toute confiance en vous : vous faites et continuerez à faire ce que nous vous ordonnons. 
${}^{5}Que le Seigneur conduise vos cœurs dans l’amour de Dieu et l’endurance du Christ.
      
         
${}^{6}Frères, au nom du Seigneur Jésus Christ, nous vous ordonnons d’éviter tout frère qui mène une vie désordonnée et ne suit pas la tradition que vous avez reçue de nous. 
${}^{7}Vous savez bien, vous, ce qu’il faut faire pour nous imiter. Nous n’avons pas vécu parmi vous de façon désordonnée ; 
${}^{8}et le pain que nous avons mangé, nous ne l’avons pas reçu gratuitement. Au contraire, dans la peine et la fatigue, nuit et jour, nous avons travaillé pour n’être à la charge d’aucun d’entre vous. 
${}^{9}Bien sûr, nous avons le droit d’être à charge, mais nous avons voulu être pour vous un modèle à imiter. 
${}^{10}Et quand nous étions chez vous, nous vous donnions cet ordre : si quelqu’un ne veut pas travailler, qu’il ne mange pas non plus. 
${}^{11}Or, nous apprenons que certains d’entre vous mènent une vie déréglée, affairés sans rien faire. 
${}^{12}À ceux-là, nous adressons dans le Seigneur Jésus Christ cet ordre et cet appel : qu’ils travaillent dans le calme pour manger le pain qu’ils auront gagné. 
${}^{13}Vous, frères, ne vous lassez pas de faire le bien.
${}^{14}Si quelqu’un n’obéit pas à ce que nous disons dans cette lettre, signalez-le ; ne le fréquentez pas, pour qu’il soit couvert de confusion ; 
${}^{15}mais ne le considérez pas comme un ennemi, réprimandez-le plutôt comme un frère.
${}^{16}Que le Seigneur de la paix vous donne lui-même la paix, en tout temps et de toute manière. Que le Seigneur soit avec vous tous.
${}^{17}La salutation est de ma main à moi, Paul. Je signe de cette façon toutes mes lettres, c’est mon écriture. 
${}^{18}Que la grâce de notre Seigneur Jésus Christ soit avec vous tous.
