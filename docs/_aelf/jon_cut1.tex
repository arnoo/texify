  
  
    
    \bbook{JONAS}{JONAS}
      
         
      \bchapter{}
      \begin{verse}
${}^{1}La parole du Seigneur fut adressée à Jonas, fils d’Amittaï : 
${}^{2} « Lève-toi, va à Ninive, la grande ville païenne\\, et proclame que sa méchanceté est montée jusqu’à moi. » 
${}^{3} Jonas se leva, mais pour s’enfuir à Tarsis, loin de la face du Seigneur. Descendu à Jaffa, il trouva un navire en partance pour Tarsis. Il paya son passage et s’embarqua pour s’y rendre, loin de la face du Seigneur. 
${}^{4} Mais le Seigneur lança sur la mer un vent violent, et il s’éleva une grande tempête, au point que le navire menaçait de se briser. 
${}^{5} Les matelots prirent peur ; ils crièrent chacun vers son dieu et, pour s’alléger, lancèrent la cargaison à la mer. Or, Jonas était descendu dans la cale du navire, il s’était couché et dormait d’un sommeil mystérieux. 
${}^{6} Le capitaine alla le trouver et lui dit : « Qu’est-ce que tu fais ? Tu dors ? Lève-toi ! Invoque ton dieu. Peut-être que ce dieu s’occupera de nous pour nous empêcher de périr. » 
${}^{7} Et les matelots se disaient entre eux : « Tirons au sort pour savoir à qui nous devons ce malheur. » Ils tirèrent au sort, et le sort tomba sur Jonas. 
${}^{8} Ils lui demandèrent : « Dis-nous donc d’où nous vient ce malheur. Quel est ton métier ? D’où viens-tu ? Quel est ton pays ? De quel peuple es-tu ? » 
${}^{9} Jonas leur répondit : « Je suis Hébreu, moi ; je crains le Seigneur, le Dieu du ciel, qui a fait la mer et la terre ferme. » 
${}^{10} Les matelots\\furent saisis d’une grande peur et lui dirent : « Qu’est-ce que tu as fait là ? » Car ces hommes savaient, d’après ce qu’il leur avait dit, qu’il fuyait la face du Seigneur. 
${}^{11} Ils lui demandèrent : « Qu’est-ce que nous devons faire de toi, pour que la mer se calme autour de nous ? » Car la mer était de plus en plus furieuse. 
${}^{12} Il leur répondit : « Prenez-moi, jetez-moi à la mer, pour que la mer se calme autour de vous. Car, je le reconnais, c’est à cause de moi que cette grande tempête vous assaille. » 
${}^{13} Les matelots\\ramèrent pour regagner la terre, mais sans y parvenir, car la mer était de plus en plus furieuse autour d’eux. 
${}^{14} Ils invoquèrent alors le Seigneur\\ : « Ah ! Seigneur, ne nous fais pas mourir à cause de cet homme, et ne nous rends pas responsables de la mort d’un innocent\\, car toi, tu es le Seigneur : ce que tu as voulu, tu l’as fait. » 
${}^{15} Puis ils prirent Jonas et le jetèrent à la mer. Alors la fureur de la mer tomba. 
${}^{16} Les hommes furent saisis par la crainte du Seigneur ; ils lui offrirent un sacrifice accompagné de vœux.
      
         
      <p class="cantique" id="bib_ct-at_42bis"><span class="cantique_label">Cantique AT 42 bis</span> = <span class="cantique_ref"><a class="unitex_link" href="#bib_jon_2_3">Jon 2, 3-10</a></span>
      
         
      \bchapter{}
      \begin{verse}
${}^{1}Le Seigneur donna l’ordre à un grand poisson d’engloutir Jonas. Jonas demeura dans les entrailles du poisson trois jours et trois nuits. 
${}^{2} Depuis les entrailles du poisson, il pria le Seigneur son Dieu. 
${}^{3} Il disait :
      
         
       
        \\Dans ma détresse, je crie vers le Seigneur,
        et lui me répond ;
        \\du ventre des enfers j’appelle :
        tu écoutes ma voix.
         
        ${}^{4}Tu m’as jeté au plus profond du cœur des mers,
        et le flot m’a cerné ;
        \\tes ondes et tes vagues ensemble
        ont passé sur moi.
         
        ${}^{5}Et je dis : me voici rejeté
        de devant tes yeux ;
        \\pourrai-je revoir encore
        ton temple saint ?
         
        ${}^{6}Les eaux m’ont assailli jusqu’à l’âme,
        l’abîme m’a cerné ;
        \\les algues m’enveloppent la tête,
        ${}^{7}à la racine des montagnes.
         
        \\Je descendis aux pays dont les verrous
        m’enfermaient pour toujours ;
        \\mais tu retires ma vie de la fosse,
        Seigneur mon Dieu.
         
        ${}^{8}Quand mon âme en moi défaillait,
        je me souvins du Seigneur ;
        \\et ma prière parvint jusqu’à toi
        dans ton temple saint.
         
        ${}^{9}Les servants de vaines idoles
        perdront leur faveur.
        ${}^{10}Mais moi, au son de l’action de grâce,
        je t’offrirai des sacrifices ;
        \\j’accomplirai les vœux que j’ai faits :
        au Seigneur appartient le salut.
       
${}^{11}Alors le Seigneur parla au poisson, et celui-ci rejeta Jonas sur la terre ferme.
      
         
      \bchapter{}
      \begin{verse}
${}^{1}La parole du Seigneur fut adressée de nouveau à Jonas : 
${}^{2} « Lève-toi, va à Ninive, la grande ville païenne\\, proclame le message que je te donne sur elle. » 
${}^{3} Jonas se leva et partit pour Ninive, selon la parole du Seigneur. Or, Ninive était une ville extraordinairement grande\\ : il fallait trois jours pour la traverser. 
${}^{4} Jonas la parcourut une journée à peine en proclamant : « Encore quarante jours, et Ninive sera détruite ! » 
${}^{5} Aussitôt, les gens de Ninive crurent en Dieu. Ils annoncèrent un jeûne, et tous, du plus grand au plus petit, se vêtirent de toile à sac. 
${}^{6} La chose arriva jusqu’au roi de Ninive. Il se leva de son trône, quitta son manteau, se couvrit d’une toile à sac, et s’assit sur la cendre. 
${}^{7} Puis il fit crier dans Ninive ce décret du roi et de ses grands : « Hommes et bêtes, gros et petit bétail, ne goûteront à rien, ne mangeront pas et ne boiront pas\\. 
${}^{8} Hommes et bêtes, on se couvrira de toile à sac, on criera vers Dieu de toute sa force, chacun se détournera de sa conduite mauvaise et de ses actes de violence. 
${}^{9} Qui sait si Dieu ne se ravisera pas et ne se repentira pas, s’il ne reviendra pas de l’ardeur de sa colère ? Et alors nous ne périrons pas ! » 
${}^{10} En voyant leur réaction, et comment ils se détournaient de leur conduite mauvaise, Dieu renonça au châtiment dont il les avait menacés\\.
      
         
      
         
      \bchapter{}
      \begin{verse}
${}^{1}Jonas trouva la chose très mauvaise et se mit en colère. 
${}^{2} Il fit cette prière au Seigneur : « Ah ! Seigneur, je l’avais bien dit lorsque j’étais encore dans mon pays ! C’est pour cela que je m’étais d’abord enfui à Tarsis. Je savais bien que tu es un Dieu tendre et miséricordieux, lent à la colère et plein d’amour, renonçant au châtiment. 
${}^{3} Eh bien, Seigneur, prends ma vie ; mieux vaut pour moi mourir que vivre. » 
${}^{4} Le Seigneur lui dit : « As-tu vraiment raison de te mettre en colère ? » 
${}^{5} Jonas sortit de Ninive\\et s’assit à l’est de la ville. Là, il fit une hutte et s’assit dessous, à l’ombre, pour voir ce qui allait arriver dans la ville. 
${}^{6} Le Seigneur Dieu donna l’ordre à un arbuste\\, un ricin, de pousser au-dessus de Jonas pour donner de l’ombre à sa tête et le délivrer ainsi de sa mauvaise humeur\\. Jonas se réjouit d’une grande joie à cause du ricin. 
${}^{7} Mais le lendemain, à l’aube, Dieu donna l’ordre à un ver de piquer le ricin, et celui-ci se dessécha. 
${}^{8} Au lever du soleil, Dieu donna l’ordre au vent d’est de brûler\\ ; Jonas fut frappé d’insolation. Se sentant défaillir, il demanda la mort et ajouta : « Mieux vaut pour moi mourir que vivre. » 
${}^{9} Dieu dit à Jonas : « As-tu vraiment raison de te mettre en colère au sujet de ce ricin ? » Il répondit : « Oui, j’ai bien raison de me mettre en colère jusqu’à souhaiter la mort. » 
${}^{10} Le Seigneur répliqua : « Toi, tu as pitié de ce ricin, qui ne t’a coûté aucun travail et que tu n’as pas fait grandir, qui a poussé en une nuit, et en une nuit a disparu. 
${}^{11} Et moi, comment n’aurais-je pas pitié de Ninive, la grande ville, où, sans compter une foule d’animaux, il y a plus de cent vingt mille êtres humains qui ne distinguent pas encore\\leur droite de leur gauche ? »
      
         
