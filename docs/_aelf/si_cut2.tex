  
  
      
         
      \bchapter{}
        ${}^{1}Toute sagesse vient du Seigneur,
        et demeure auprès de lui pour toujours.
        ${}^{2}Le sable des mers, les gouttes de la pluie,
        et les jours de l’éternité\\,
        qui pourra en faire le compte ?
        ${}^{3}La hauteur du ciel, l’étendue de la terre,
        la profondeur de l’abîme\\,
        qui pourra les évaluer ?
        ${}^{4}Avant toute chose fut créée la sagesse ;
        et depuis toujours, la profondeur de l’intelligence.
        ${}^{5}La source de la sagesse,
        c’est la parole de Dieu au plus haut des cieux.
        Ses chemins sont les commandements éternels.
        ${}^{6}La racine de la sagesse, qui en a eu la révélation,
        et ses subtilités, qui en a eu la connaissance ?
        ${}^{7}La science de la sagesse, à qui fut-elle manifestée,
        et qui a profité de sa grande expérience ?
        ${}^{8}Il n’y a qu’un seul être sage et très redoutable,
        celui qui siège sur son trône.
        <p class="verset_anchor" id="para_bib_si_1_9">C’est le Seigneur, 
${}^{9}lui qui a créé la sagesse ;
        il l’a vue et mesurée,
        il l’a répandue sur toutes ses œuvres,
        ${}^{10}parmi tous les vivants, dans la diversité de ses dons,
        \\et\\ceux qui aiment Dieu en ont été comblés.
        \\L’amour du Seigneur est une éminente sagesse ;
        Dieu en accorde une part à ceux dont il veut se laisser voir.
        
           
${}^{11}La crainte du Seigneur est gloire et fierté,
        joie et couronne d’allégresse.
${}^{12}La crainte du Seigneur réjouira le cœur ;
        elle procure plaisir, joie et longue vie.
        \\La crainte du Seigneur est un don du Seigneur ;
        car elle fait persévérer sur les voies de l’amour.
${}^{13}Celui qui craint le Seigneur connaîtra une fin heureuse ;
        au jour de sa mort, il sera béni.
         
${}^{14}La sagesse commence avec la crainte du Seigneur :
        elle est formée en chaque fidèle dès le sein maternel.
${}^{15}Elle a bâti son nid chez les humains, fondation d’éternité :
        elle sera fidèle envers leurs descendants.
         
${}^{16}La sagesse s’accomplit dans la crainte du Seigneur ;
        elle les enivre de ses fruits.
${}^{17}Elle remplira leurs maisons de biens désirables,
        et leurs greniers, de ses produits.
         
${}^{18}La sagesse est couronnée par la crainte du Seigneur,
        elle fait refleurir la paix et le bien-être.
        \\L’un et l’autre sont des dons de Dieu qui mènent au bonheur,
        une juste fierté épanouit ceux qui aiment Dieu.
         
${}^{19}La sagesse répand comme une ondée la science et la connaissance avisée,
        elle exalte la gloire de ceux qui la possèdent.
${}^{20}La sagesse s’enracine dans la crainte du Seigneur,
        et sa ramure est longue vie.
${}^{21}La crainte du Seigneur éloigne les péchés,
        et qui s’attache à elle détourne la fureur.
${}^{22}Une injuste colère ne peut être justifiée :
        le poids de cette colère entraîne la chute.
${}^{23}Qui a de la patience résistera autant qu’il le faut
        et, plus tard, la joie lui sera rendue.
${}^{24}Autant qu’il le faut, il gardera pour lui ses paroles ;
        l’éloge de sa perspicacité sera sur toutes les lèvres.
         
${}^{25}Dans les trésors de la sagesse sont les proverbes du savoir ;
        le pécheur, lui, a la religion en horreur.
${}^{26}Désires-tu la sagesse ? Garde les commandements,
        et le Seigneur la conduira vers toi.
${}^{27}Car la crainte du Seigneur est sagesse et instruction ;
        la douceur et la fidélité attirent sa faveur.
         
${}^{28}Ne te dérobe pas à la crainte du Seigneur,
        ne viens pas à lui avec un cœur double.
${}^{29}Quand tu parles aux gens, ne sois pas hypocrite ;
        veille à tes lèvres.
${}^{30}Ne t’élève pas, de peur de tomber
        et d’attirer sur toi le déshonneur :
        \\le Seigneur dévoilerait tes secrets
        et te jetterait à terre au milieu de l’assemblée,
        \\pour n’être pas venu à la crainte du Seigneur
        et parce que ton cœur regorge de fausseté.
      
         
      \bchapter{}
        ${}^{1}Mon fils,
        si tu viens te mettre au service du Seigneur,
        prépare-toi à subir l’épreuve ;
        ${}^{2}fais-toi un cœur droit, et tiens bon ;
        ne t’agite pas à l’heure de l’adversité.
        ${}^{3}Attache-toi au Seigneur, ne l’abandonne pas,
        afin d’être comblé dans tes derniers jours.
        ${}^{4}Toutes les adversités, accepte-les ;
        dans les revers de ta pauvre vie, sois patient ;
        ${}^{5}car l’or est vérifié par le feu,
        et les hommes agréables à Dieu, par le creuset de l’humiliation.
        Dans les maladies comme dans le dénuement, aie foi en lui.
        ${}^{6}Mets ta confiance en lui, et il te viendra en aide ;
        rends tes chemins droits, et mets en lui ton espérance.
        
           
         
        ${}^{7}Vous qui craignez le Seigneur, comptez sur sa miséricorde,
        ne vous écartez pas du chemin, de peur de tomber.
        ${}^{8}Vous qui craignez le Seigneur, ayez confiance en lui,
        et votre récompense ne saurait vous échapper.
        ${}^{9}Vous qui craignez le Seigneur, espérez le bonheur,
        la joie éternelle et la miséricorde :
        ce qu’il donne en retour est un don éternel, pour la joie.
        
           
         
        ${}^{10}Considérez les générations passées et voyez :
        \\Celui qui a mis sa confiance dans le Seigneur,
        a-t-il été déçu ?
        \\Celui qui a persévéré dans la crainte du Seigneur,
        a-t-il été abandonné ?
        \\Celui qui l’a invoqué,
        a-t-il été méprisé ?
        ${}^{11}Car le Seigneur est tendre et miséricordieux,
        il pardonne les péchés,
        et il sauve au moment de la détresse\\.
        
           
         
${}^{12}Malheur aux cœurs lâches et aux mains négligentes,
        au pécheur qui suit deux sentiers.
${}^{13}Malheur au cœur négligent, qui ne fait pas confiance :
        il ne sera pas protégé.
${}^{14}Malheur à vous qui avez perdu la persévérance :
        que ferez-vous lors de la visite du Seigneur ?
        
           
         
${}^{15}Ceux qui craignent le Seigneur ne désobéiront pas à ses paroles,
        ceux qui l’aiment suivront ses chemins.
${}^{16}Ceux qui craignent le Seigneur chercheront à lui plaire,
        ceux qui l’aiment se rassasieront de sa loi.
${}^{17}Ceux qui craignent le Seigneur prépareront leur cœur
        et s’humilieront devant lui, disant :
${}^{18}« Nous voulons tomber dans les mains du Seigneur,
        et non dans celles des hommes.
        \\Car telle est sa grandeur,
        telle est aussi sa miséricorde. »
        
           
      
         
      \bchapter{}
${}^{1}Mes enfants, écoutez-moi, qui suis votre père,
        et agissez en conséquence, afin d’être sauvés.
        ${}^{2}Le Seigneur glorifie le père dans ses enfants,
        il renforce l’autorité de la mère sur ses fils.
        ${}^{3}Celui qui honore son père
        obtient le pardon de ses péchés,
        ${}^{4}celui qui glorifie sa mère
        est comme celui qui amasse un trésor.
        ${}^{5}Celui qui honore son père aura de la joie dans ses enfants,
        au jour de sa prière il sera exaucé.
        ${}^{6}Celui qui glorifie son père verra de longs jours,
        celui qui obéit au Seigneur donne du réconfort à sa mère.
${}^{7}Celui qui craint le Seigneur honorera son père
        et servira ses parents comme des maîtres.
${}^{8}Honore ton père en acte et en parole,
        afin que sa bénédiction vienne sur toi.
${}^{9}Car la bénédiction d’un père affermit la maison de ses enfants,
        mais la malédiction d’une mère en sape les fondations.
${}^{10}Ne te glorifie pas en rabaissant ton père,
        car l’abaissement de ton père n’est pas une gloire pour toi.
${}^{11}La gloire d’un homme vient de la notoriété de son père,
        une mère méprisée fait la honte de ses enfants.
        ${}^{12}Mon fils, soutiens ton père dans sa vieillesse,
        ne le chagrine pas pendant sa vie.
        ${}^{13}Même si son esprit l’abandonne, sois indulgent,
        ne le méprise pas, toi qui es en pleine force.
        ${}^{14}Car ta miséricorde envers ton père ne sera pas oubliée,
        et elle relèvera ta maison si elle est ruinée par le péché\\.
${}^{15}Au jour de la détresse, le Seigneur se souviendra de toi ;
        comme givre au soleil, ainsi fondront tes péchés.
${}^{16}Qui abandonne son père est un blasphémateur ;
        qui met en rage sa mère est maudit du Seigneur.
        
           
        ${}^{17}Mon fils, accomplis toute chose dans l’humilité,
        et tu seras aimé plus qu’un bienfaiteur.
        ${}^{18}Plus tu es grand, plus il faut t’abaisser :
        tu trouveras grâce devant le Seigneur.
        ${}^{19}Beaucoup sont haut placés et glorieux,
        mais c’est aux humbles que le Seigneur révèle ses secrets.
        ${}^{20}Grande est la puissance du Seigneur,
        et les humbles lui rendent gloire.
        ${}^{21}Ne cherche pas ce qui est trop difficile pour toi,
        ne scrute pas ce qui est au-dessus de tes forces.
        ${}^{22}Médite ce qu’on t’a prescrit :
        tu n’as pas à t’occuper des choses cachées.
        ${}^{23}Ne sois pas curieux de ce qui te dépasse :
        déjà ce qu’on t’a enseigné est au-delà de l’esprit humain.
        ${}^{24}Leur présomption a égaré bien des gens,
        leur manque de jugement a fait dévier leurs pensées.
${}^{25}Si tes yeux n’avaient pas de prunelles, tu serais privé de lumière ;
        alors, si tu es dénué de science, ne te vante pas !
${}^{26}Un cœur endurci finira dans le malheur,
        celui qui aime le danger s’y perdra.
${}^{27}Un cœur endurci sera écrasé de peines,
        le pécheur entasse péché sur péché.
        ${}^{28}La condition de l’orgueilleux est sans remède,
        car la racine du mal est en lui.
        ${}^{29}Qui est sensé médite les maximes de la sagesse\\ ;
        l’idéal du sage, c’est une oreille qui écoute.
${}^{30}L’eau éteint la flamme du feu,
        et l’aumône obtient le pardon des péchés.
${}^{31}Celui qui sait rendre les bienfaits pense à l’avenir ;
        s’il vient à tomber, il trouvera un soutien.
      
         
      \bchapter{}
${}^{1}Mon fils, ne retire pas au pauvre ce qu’il lui faut pour vivre,
        ne fais pas attendre le regard d’un indigent.
${}^{2}Ne fais pas souffrir un affamé,
        n’exaspère pas un homme qui est dans la misère.
${}^{3}N’ajoute pas au trouble d’un cœur irrité,
        ne fais pas attendre ton aumône à celui qui en a besoin.
${}^{4}Ne repousse pas celui qui supplie dans la détresse,
        ne détourne pas du pauvre ton visage.
${}^{5}Ne détourne pas du miséreux ton regard,
        ne donne pas à un homme l’occasion de te maudire.
${}^{6}Car s’il te maudit dans l’amertume de son âme,
        celui qui l’a créé entendra sa prière.
${}^{7}Rends-toi aimable à toute l’assemblée,
        et baisse la tête devant celui qui commande.
${}^{8}Penche l’oreille vers le pauvre,
        et réponds avec douceur à son salut de paix.
${}^{9}Délivre l’opprimé du pouvoir de l’oppresseur,
        et ne sois pas timide quand tu rends la justice.
${}^{10}Sois comme un père pour les orphelins,
        et pour leur mère sois comme un mari.
        \\Alors tu seras comme un fils du Très-Haut,
        il t’aimera plus que ta propre mère.
        
           
        ${}^{11}La sagesse conduit ses fils à la grandeur,
        elle prend soin de ceux qui la cherchent.
        ${}^{12}L’aimer, c’est aimer la vie ;
        ceux qui la cherchent dès l’aurore seront comblés de bonheur ;
        ${}^{13}celui qui la possède obtiendra la gloire en héritage ;
        là où il entre, le Seigneur donne sa bénédiction.
        ${}^{14}Ceux qui rendent un culte à la sagesse célèbrent le Dieu saint,
        ceux qui l’aiment sont aimés du Seigneur ;
        ${}^{15}celui qui l’écoute jugera les nations,
        celui qui s’attache à elle sera en sécurité dans sa demeure.
        ${}^{16}S’il se confie en elle, il en prendra possession,
        et tous ses descendants la recevront en héritage.
        ${}^{17}Pour commencer, elle le conduira par des chemins sinueux,
        elle fera venir sur lui la peur et l’appréhension,
        \\elle le tourmentera par la sévérité\\de son éducation,
        jusqu’à ce qu’elle puisse lui faire confiance ;
        elle l’éprouvera par ses exigences.
        ${}^{18}Puis elle reviendra tout droit vers lui,
        elle le comblera de bonheur
        en lui dévoilant ses secrets.
        ${}^{19}Mais s’il s’égare loin d’elle, elle l’abandonnera
        et le laissera aller à sa perte.
${}^{20}Sois sur tes gardes, évite le mal,
        et n’aie pas honte d’être toi-même.
${}^{21}Car il y a une honte qui conduit au péché,
        et une honte qui est gloire et grâce.
${}^{22}Ne te laisse pas impressionner, allant contre ta conscience,
        par l’apparence des personnes.
        Ne cède pas aux pressions qui te feraient tomber.
${}^{23}Ne t’interdis pas de parler quand il le faut.
        et ne cache pas ta sagesse.
${}^{24}La sagesse se reconnaît à la parole,
        et l’instruction, aux propos de la langue.
${}^{25}Ne parle pas contre la vérité,
        rougis plutôt de ton ignorance.
${}^{26}N’aie pas honte de confesser tes péchés,
        ne lutte pas contre le courant du fleuve.
${}^{27}Ne t’aplatis pas devant un sot,
        ne te laisse pas intimider par le puissant.
${}^{28}Lutte pour la vérité jusqu’à la mort,
        et le Seigneur Dieu combattra pour toi.
${}^{29}Ne te montre pas hardi dans ton langage,
        mou et négligent dans tes actes.
${}^{30}Ne sois pas comme un lion chez toi,
        cherchant à faire illusion à ceux de ta maison.
${}^{31}Que ta main ne soit pas tendue pour prendre,
        et fermée lorsqu’il faut rendre.
      
         
      \bchapter{}
        ${}^{1}Ne t’appuie pas sur tes richesses,
        ne dis pas : « Elles me suffisent. »
        ${}^{2}Ne te laisse pas entraîner par ton instinct et ta force
        à suivre les désirs de ton cœur.
        ${}^{3}Ne dis pas : « Qui m’en imposera ? »,
        car le Seigneur ne manquerait pas de te châtier.
        ${}^{4}Ne dis pas : « J’ai péché, et rien ne m’est arrivé »,
        car le Seigneur sait attendre longtemps.
        ${}^{5}Ne sois pas assuré du pardon
        au point d’entasser péché sur péché.
        ${}^{6}Ne dis pas : « Sa miséricorde est grande,
        il pardonnera bien tous mes péchés »,
        \\car, en lui, il y a pitié mais aussi colère ;
        son indignation s’abattra sur les pécheurs.
        ${}^{7}Ne tarde pas à te retourner vers le Seigneur,
        ne remets pas ta décision de jour en jour ;
        \\car brusquement éclatera la colère du Seigneur,
        et à l’heure du châtiment, tu seras anéanti.
        ${}^{8}Ne t’appuie pas sur des richesses injustement acquises :
        elles ne te serviront de rien au jour de l’adversité.
        
           
${}^{9}Ne disperse pas à tous les vents
        et ne t’engage pas dans tous les sentiers,
        comme fait le pécheur à la langue double.
${}^{10}Tiens fermement tes convictions
        et n’aie qu’une parole.
${}^{11}Sois prompt à écouter,
        mais pour donner ta réponse, prends ton temps.
${}^{12}Si tu as quelque compétence, réponds à ton prochain ;
        sinon, que ta main soit sur ta bouche.
${}^{13}La gloire comme le déshonneur sont dans la parole ;
        la langue de l’homme peut être sa ruine.
${}^{14}Ne te fais pas traiter de médisant
        et ne tends pas de pièges avec ta langue :
        \\comme le voleur est couvert de honte,
        la langue double sera durement condamnée.
${}^{15}Ne te mets en faute ni dans les grandes ni dans les petites choses,
      
         
      \bchapter{}
${}^{1}et d’ami ne deviens pas ennemi.
        \\Car une mauvaise réputation héritera la honte et l’opprobre,
        comme il en va pour le pécheur à la langue double.
${}^{2}Ne te laisse pas emporter par ta passion,
        de peur de finir dépecé comme un bœuf ;
${}^{3}tu dévorerais ton propre feuillage, tu perdrais tes fruits
        et tu te retrouverais comme du bois sec.
${}^{4}Une passion mauvaise perd celui qui la possède
        et fait de lui la risée de ses ennemis.
        
           
        ${}^{5}La parole agréable attire de nombreux amis,
        le langage aimable attire de nombreuses gentillesses.
        ${}^{6}De bonnes relations,
        tu peux en avoir avec beaucoup de monde ;
        \\mais des conseillers ?
        n’en choisis qu’un seul entre mille !
        ${}^{7}Si tu veux acquérir un ami,
        acquiers-le en le mettant à l’épreuve ;
        n’aie pas trop vite confiance en lui.
        ${}^{8}Il y a celui qui est ton ami quand cela lui convient,
        mais qui ne reste pas avec toi au jour de ta détresse.
        ${}^{9}Il y a celui qui d’ami se transforme en ennemi,
        et qui va divulguer, pour ta confusion, ce qui l’oppose à toi.
        ${}^{10}Il y a celui qui est ton ami pour partager tes repas,
        mais qui ne reste pas avec toi au jour de ta détresse.
        ${}^{11}Quand tout va bien pour toi,
        il est comme un autre toi-même
        et commande avec assurance à tes domestiques ;
        ${}^{12}mais si tu deviens pauvre, il est contre toi,
        et il se cache pour t’éviter.
        ${}^{13}Tes ennemis, tiens-les à distance ;
        avec tes amis, sois sur tes gardes.
        ${}^{14}Un ami fidèle, c’est un refuge assuré,
        celui qui le trouve a trouvé un trésor.
        ${}^{15}Un ami fidèle n’a pas de prix,
        sa valeur est inestimable.
        ${}^{16}Un ami fidèle est un élixir de vie
        que découvriront ceux qui craignent le Seigneur.
        ${}^{17}Celui qui craint le Seigneur choisit bien ses amis\\,
        car son compagnon lui ressemblera.
${}^{18}Mon fils, dès ta jeunesse, accueille l’instruction,
        et jusqu’à l’âge des cheveux blancs tu trouveras la sagesse.
${}^{19}Comme celui qui laboure et fait les semailles, cultive-la
        et attends ses bons fruits.
        \\Tu peineras un peu pour la travailler,
        mais bientôt, tu mangeras de ses produits.
${}^{20}Aux ignorants, elle semble terriblement dure,
        et qui n’a pas d’intelligence n’y persévère pas :
${}^{21}elle pèse sur lui comme une lourde pierre,
        et il ne tarde pas à la rejeter.
${}^{22}Car la sagesse mérite bien son nom :
        elle n’est pas accessible au grand nombre.
         
${}^{23}Écoute, mon fils, et reçois mon avis ;
        ne rejette pas mon conseil.
${}^{24}Engage tes pieds dans les entraves de la sagesse
        et ton cou dans son carcan.
${}^{25}Incline ton épaule pour la porter
        et ne te rebiffe pas contre ses liens.
${}^{26}Viens à elle de toute ton âme,
        et de toute ta force suis ses chemins.
${}^{27}Mets-toi sur sa trace et cherche-la : elle se fera connaître,
        et, quand tu l’auras saisie, ne la laisse pas s’échapper.
${}^{28}Pour finir, tu trouveras en elle ton repos,
        et elle deviendra ta joie.
${}^{29}Alors, ses entraves seront pour toi une puissante protection,
        et son carcan, un vêtement de gloire.
${}^{30}Son joug est une parure d’or,
        ses liens sont un ruban de pourpre.
${}^{31}Tu la porteras comme un vêtement de gloire,
        tu la ceindras comme une couronne d’allégresse.
         
${}^{32}Si tu le veux, mon fils, tu deviendras instruit ;
        à force d’application, tu auras du savoir-faire.
${}^{33}Si tu prends plaisir à écouter, tu apprendras,
        et si tu prêtes l’oreille, tu deviendras sage.
${}^{34}Tiens-toi dans la compagnie des anciens ;
        si tu trouves un sage, attache-toi à lui.
${}^{35}Écoute volontiers tout discours qui vient de Dieu,
        et ne néglige aucun proverbe sensé.
${}^{36}Si tu vois quelqu’un de bon sens, cours vers lui dès le matin,
        et que tes pieds usent le seuil de sa porte.
${}^{37}Réfléchis aux préceptes du Seigneur,
        applique-toi toujours à étudier ses commandements ;
        \\lui-même affermira ton cœur,
        et ton désir de sagesse sera comblé.
      
         
      \bchapter{}
${}^{1}Ne fais pas le mal,
        et aucun mal ne t’arrivera.
${}^{2}Éloigne-toi de l’injuste,
        et il s’écartera de toi.
${}^{3}Mon fils, ne sème pas dans les sillons de l’injustice,
        de peur d’en récolter sept fois plus.
${}^{4}Ne demande pas au Seigneur une situation en vue,
        ni au roi une place d’honneur.
${}^{5}Ne joue pas au juste devant le Seigneur,
        ni au sage devant le roi.
${}^{6}Ne brigue pas la fonction de juge,
        si tu n’es pas armé pour briser l’injustice ;
        \\tu pourrais te laisser intimider par un puissant,
        et compromettre ton intégrité.
${}^{7}Ne te rends pas coupable envers le peuple de la ville,
        ne perds pas la considération publique.
${}^{8}Ne commets pas une deuxième fois le même péché,
        car pour le premier tu ne resteras pas impuni.
${}^{9}Ne dis pas :
        \\« Le Dieu Très-Haut verra l’abondance de mes offrandes ;
        quand je les présenterai devant lui, il les acceptera. »
${}^{10}Ne te décourage pas dans ta prière,
        et ne néglige pas de faire l’aumône.
${}^{11}Ne te moque pas d’un homme qui est dans la peine,
        car le Dieu qui humilie est aussi celui qui élève.
${}^{12}Ne cultive pas le mensonge contre ton frère,
        et pas davantage contre ton ami.
${}^{13}Ne consens jamais au mensonge :
        l’habitude de mentir ne produit rien de bon.
${}^{14}Ne pérore pas devant les anciens,
        et ne rabâche pas dans ta prière.
${}^{15}Ne prends pas en dégoût les travaux pénibles,
        ni le travail des champs institué par le Très-Haut.
${}^{16}Ne te joins pas à la compagnie des pécheurs,
        souviens-toi que la colère ne tardera pas.
${}^{17}Humilie-toi profondément,
        car l’impie aura pour châtiment le feu et les vers.
        
           
${}^{18}Ne vends pas un ami pour de l’argent,
        ni un frère de sang pour l’or d’Ophir.
${}^{19}Ne te détourne pas d’une épouse sage et bonne,
        car sa valeur surpasse l’or.
${}^{20}Ne maltraite pas l’esclave qui travaille fidèlement,
        ni le journalier qui se dévoue sans compter.
${}^{21}Aime de tout ton cœur l’esclave intelligent,
        ne lui refuse pas la liberté.
${}^{22}As-tu du bétail ? Prends-en soin ;
        s’il te rapporte, ne t’en défais pas.
${}^{23}As-tu des enfants ? Éduque-les ;
        dès le jeune âge, apprends-leur à obéir.
${}^{24}As-tu des filles ? Veille à leur vertu,
        et montre-leur un visage sérieux.
${}^{25}Donne ta fille en mariage, tu auras conclu une affaire importante,
        mais donne-la à un homme de bon sens.
${}^{26}As-tu une femme selon ton cœur ? Surtout, ne la renvoie pas !
        Mais si elle n’est pas aimée, méfie-toi d’elle !
${}^{27}De tout ton cœur, honore ton père,
        et n’oublie pas les douleurs de ta mère.
${}^{28}Souviens-toi que tu leur dois d’être né ;
        comment leur rendras-tu ce qu’ils ont fait pour toi ?
${}^{29}De toute ton âme, révère le Seigneur
        et vénère ses prêtres.
${}^{30}De toute ta force, aime ton Créateur
        et ne délaisse pas ceux qui le servent.
${}^{31}Crains le Seigneur et honore le prêtre,
        donne-lui la part qui est prescrite :
        \\prémices, sacrifice de réparation, épaules des bêtes immolées,
        sacrifice de consécration et prémices saintes.
${}^{32}Tends aussi ta main au pauvre,
        pour être pleinement béni.
${}^{33}Que ta générosité s’étende à tous les vivants ;
        même envers les morts sois généreux.
${}^{34}Ne te détourne pas de ceux qui pleurent,
        afflige-toi avec les affligés.
${}^{35}N’hésite pas à visiter un malade :
        en agissant ainsi, tu seras aimé.
${}^{36}Quoi que tu fasses, souviens-toi que ta vie a une fin,
        et jamais tu ne pécheras.
      
         
      \bchapter{}
${}^{1}N’entre pas en conflit avec un homme puissant,
        de peur de tomber entre ses mains.
${}^{2}Ne te querelle pas avec un homme riche,
        de peur de ne pas faire le poids,
        \\car avec de l’or on a perdu bien des gens,
        il a même fait fléchir le cœur des rois.
${}^{3}Ne te dispute pas avec un bavard :
        tu ne ferais qu’entasser du bois sur son feu.
${}^{4}Ne plaisante pas avec quelqu’un de mal élevé,
        de peur d’entendre insulter tes parents.
${}^{5}Ne fais pas de reproches au pécheur repentant :
        souviens-toi que nous sommes tous passibles de châtiment.
${}^{6}Ne méprise pas un vieillard,
        car certains d’entre nous prennent de l’âge.
${}^{7}Ne te réjouis pas de la mort de quelqu’un :
        souviens-toi que tous nous devrons mourir.
        
           
         
${}^{8}Ne néglige pas la conversation des sages,
        reviens souvent à leurs proverbes,
        \\car auprès d’eux tu acquerras l’instruction
        et l’art de servir les grands.
${}^{9}Ne fuis pas la conversation des vieillards
        – eux-mêmes ont appris de leurs pères –
        \\car auprès d’eux tu acquerras l’intelligence
        et l’art de répondre en temps voulu.
        
           
         
${}^{10}N’attise pas les ardeurs du pécheur,
        de peur de te brûler au feu de sa flamme.
${}^{11}Ne tiens pas tête à l’effronté :
        il pourrait te prendre au piège de ton propre discours.
${}^{12}Ne prête pas à plus fort que toi,
        et si tu le fais, considère ton bien comme perdu.
${}^{13}Ne te porte pas garant au-delà de tes moyens,
        et si tu le fais, prépare-toi à payer.
${}^{14}Ne sois pas en procès avec un juge,
        car on s’en remettra à son jugement.
${}^{15}Ne fais pas route avec l’aventurier,
        de peur qu’il ne t’épuise,
        \\car il n’en fera qu’à sa tête
        et sa folie vous perdra tous les deux.
${}^{16}Ne te dispute pas avec un coléreux,
        et ne t’engage pas avec lui dans un lieu désert,
        \\car le sang compte pour rien à ses yeux ;
        là où il n’y a plus de secours, il t’abattra.
        
           
         
${}^{17}Ne prends pas conseil d’un insensé,
        car il ne pourra tenir sa langue.
${}^{18}Devant un étranger, ne fais rien qui doive rester caché,
        car tu ignores ce qu’il en tirera.
${}^{19}N’ouvre pas ton cœur au premier venu,
        il ne t’en saurait aucun gré.
        
           
      
         
      \bchapter{}
${}^{1}Ne sois pas jaloux de la femme de ton cœur :
        tu lui donnerais l’idée de te faire du mal.
${}^{2}Ne livre pas ton âme au pouvoir d’une femme :
        elle prendrait de l’ascendant sur toi.
${}^{3}Ne va pas voir une femme de mauvaise vie :
        tu tomberais dans ses filets.
${}^{4}Ne fréquente pas une chanteuse,
        de peur de te laisser prendre à ses artifices.
${}^{5}Ne fixe pas ton regard sur une jeune fille,
        de peur d’être puni à cause d’elle.
${}^{6}Ne te livre pas aux prostituées :
        tu y perdrais ton patrimoine.
${}^{7}Ne promène pas tes regards dans les rues de la ville
        et ne rôde pas dans les coins déserts.
${}^{8}Détourne les yeux d’une jolie femme,
        ne fixe pas ton regard sur une belle étrangère.
        \\La beauté féminine en a égaré beaucoup,
        et l’amour s’y enflamme comme un feu.
${}^{9}Ne t’assieds jamais près d’une femme mariée,
        et dans un banquet ne t’enivre pas avec elle,
        \\de peur que tu t’en éprennes
        et que ta passion entraîne ta perte.
        
           
${}^{10}N’abandonne pas un vieil ami,
        car un nouvel ami ne le vaudra pas.
        \\Ami nouveau, vin nouveau :
        qu’il vieillisse, tu le boiras avec plaisir.
${}^{11}N’envie pas la réussite du pécheur,
        tu ne sais pas comment il finira.
${}^{12}Ne te réjouis pas, avec les impies, de leur succès,
        souviens-toi qu’ils ne mourront pas impunis.
${}^{13}Tiens-toi loin de l’homme capable de tuer,
        et tu n’auras pas à craindre la mort.
        \\Mais si tu viens à lui, évite tout faux pas,
        de peur qu’il ne t’enlève la vie :
        \\sache que tu t’avances au milieu de pièges,
        et que tu t’exposes aux créneaux d’une ville assiégée.
${}^{14}Fréquente tes voisins autant que tu le peux,
        et demande conseil aux sages.
${}^{15}Aime à t’entretenir avec les gens intelligents,
        et consacre tes conversations à la loi du Très-Haut.
${}^{16}Recherche des hommes justes pour prendre avec eux ton repas,
        mets ta fierté dans la crainte du Seigneur.
${}^{17}Pour son ouvrage on loue la main de l’artiste,
        et le chef d’un peuple pour la sagesse de sa parole.
${}^{18}Le bavard est redouté dans la ville,
        l’homme emporté par ses discours se fait détester.
