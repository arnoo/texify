  
  
    
    \bbook{ÉZÉKIEL}{ÉZÉKIEL}
      
         
      \bchapter{}
      \begin{verse}
${}^{1}La trentième année, le quatrième mois, le cinq du mois, je me trouvais à Babylone au milieu des exilés près du fleuve Kebar ; les cieux s’ouvrirent et j’eus des visions divines.
${}^{2}Le cinq du mois, la cinquième année de la déportation\\du roi Jékonias, 
${}^{3} la parole de Dieu fut adressée à Ézékiel, fils du prêtre Bouzi, dans le pays des Chaldéens, au bord du fleuve Kebar. La main du Seigneur se posa sur lui.
${}^{4}J’ai vu : un vent de tempête venant du nord, un gros nuage, un feu jaillissant et, autour, une clarté ; au milieu, comme un scintillement de vermeil du milieu du feu. 
${}^{5}Au milieu, la forme de quatre Vivants ; elle paraissait une forme humaine. 
${}^{6}Ils avaient chacun quatre faces et chacun quatre ailes. 
${}^{7}Leurs jambes étaient droites ; leurs pieds, pareils aux sabots d’un veau, étincelaient comme scintille le bronze poli. 
${}^{8}Des mains humaines, sous leurs ailes, étaient tournées dans les quatre directions, ainsi que leurs visages et leurs ailes à tous les quatre. 
${}^{9}Leurs ailes étaient jointes l’une à l’autre ; ils ne se tournaient pas en marchant : ils allaient chacun droit devant soi. 
${}^{10}La forme de leurs visages, c’était visage d’homme et, vers la droite, visage de lion pour tous les quatre, visage de taureau à gauche pour tous les quatre, et visage d’aigle pour tous les quatre. 
${}^{11}Leurs ailes étaient déployées vers le haut ; deux se rejoignaient l’une l’autre, et deux couvraient leur corps. 
${}^{12}Chacun allait droit devant soi ; là où l’esprit voulait aller, ils allaient. Ils avançaient sans s’écarter. 
${}^{13}Ils avaient une forme de vivants. Leur aspect était celui de brandons enflammés, une certaine apparence de torches allait et venait entre les Vivants. Il y avait la clarté du feu, et des éclairs sortant du feu. 
${}^{14}Et les Vivants s’élançaient en tous sens : leur aspect était celui de l’éclair.
${}^{15}J’ai vu les Vivants : il y avait une roue à terre, à côté de chaque Vivant, pour leurs quatre visages. 
${}^{16}Ces roues et leurs éléments scintillaient comme de la chrysolithe. Toutes les quatre avaient même forme. L’aspect de leurs éléments était tel que les roues paraissaient imbriquées l’une dans l’autre. 
${}^{17}Quand elles avançaient, elles allaient dans les quatre directions ; elles avançaient sans s’écarter. 
${}^{18}Leur pourtour était grand et effrayant, rempli de scintillements autour de chacune des quatre roues. 
${}^{19}Quand les Vivants avançaient, les roues avançaient à côté d’eux ; quand les Vivants s’élevaient de terre, les roues s’élevaient. 
${}^{20}Là où l’esprit voulait aller, ils allaient, et les roues s’élevaient avec eux : l’esprit du Vivant était dans les roues ! 
${}^{21}Quand ils avançaient, elles avançaient ; quand ils s’arrêtaient, elles s’arrêtaient ; et quand ils s’élevaient de terre, les roues s’élevaient avec eux : l’esprit du Vivant était dans les roues !
${}^{22}La forme au-dessus de la tête des Vivants était un firmament ; scintillant comme un cristal éblouissant, il s’étendait sur leurs têtes, bien au-dessus. 
${}^{23}Sous le firmament, leurs ailes étaient déployées l’une vers l’autre ; chacun en avait deux qui lui couvraient le corps.
${}^{24}J’entendis le bruit de leurs ailes, pareil, quand ils marchaient, au bruit des grandes eaux, pareil à la voix du Puissant, une rumeur comme celle d’une armée. Lorsqu’ils s’arrêtaient, ils laissaient retomber leurs ailes. 
${}^{25}On entendit un bruit venant de plus haut que le firmament qui était au-dessus de leurs têtes. 
${}^{26}Au-dessus de ce firmament\\, il y avait une forme de trône, qui ressemblait à du saphir\\ ; et, sur ce\\trône, quelqu’un qui avait l’aspect d’un être humain\\, au-dessus, tout en haut. 
${}^{27}Puis j’ai vu comme un scintillement de vermeil, comme l’aspect d’un feu qui l’enveloppait tout autour, à partir de ce qui semblait être ses reins et au-dessus. À partir de ce qui semblait être ses reins et au-dessous, j’ai vu comme l’aspect d’un feu et, autour, une clarté. 
${}^{28}Comme l’arc apparaît dans la nuée un jour de pluie, ainsi cette clarté à l’entour : c’était l’aspect, la forme de la gloire du Seigneur. À cette vue, je tombai face contre terre\\, et j’entendis une voix qui me parlait.
      
         
      \bchapter{}
      \begin{verse}
${}^{1}Elle me dit : « Fils d’homme, tiens-toi debout, je vais te parler. » 
${}^{2}À cette parole, l’esprit vint en moi et me fit tenir debout. J’écoutai celui qui me parlait. 
${}^{3}Il me dit : « Fils d’homme, je t’envoie vers les fils d’Israël, vers une nation rebelle qui s’est révoltée contre moi. Jusqu’à ce jour, eux et leurs pères se sont soulevés contre moi. 
${}^{4}Les fils ont le visage dur, et le cœur obstiné ; c’est à eux que je t’envoie. Tu leur diras : “Ainsi parle le Seigneur Dieu...” 
${}^{5}Alors, qu’ils écoutent ou qu’ils n’écoutent pas – c’est une engeance de rebelles ! – ils sauront qu’il y a un prophète au milieu d’eux. 
${}^{6}Et toi, fils d’homme, ne les crains pas, ne crains pas leurs paroles. Ils sont pour toi épines et ronces, tu es assis sur des scorpions. Ne crains pas leurs paroles ; devant eux ne t’effraie pas – c’est une engeance de rebelles ! 
${}^{7}Tu leur diras mes paroles, qu’ils écoutent ou qu’ils n’écoutent pas – c’est une engeance de rebelles ! 
${}^{8}Et toi, fils d’homme, écoute ce que je te dis. Ne sois pas rebelle comme cette engeance de rebelles. Ouvre la bouche, et mange ce que je te donne. »
${}^{9}Alors j’ai vu : une main tendue vers moi, tenant un livre en forme de rouleau. 
${}^{10} Elle le déroula devant moi ; ce rouleau était écrit au-dedans et au-dehors\\, rempli de\\lamentations, plaintes et clameurs.
      
         
      \bchapter{}
      \begin{verse}
${}^{1}Le Seigneur me dit : « Fils d’homme, ce qui est devant toi\\, mange-le, mange ce rouleau\\ ! Puis, va ! Parle à la maison d’Israël. » 
${}^{2} J’ouvris la bouche, il me fit manger le rouleau 
${}^{3} et il me dit : « Fils d’homme, remplis ton ventre, rassasie tes entrailles avec ce rouleau que je te donne. » Je le mangeai, et dans ma bouche il fut doux comme du miel\\.
${}^{4}Il me dit alors : « Debout, fils d’homme ! Va vers la maison d’Israël, et dis-lui mes paroles. 
${}^{5}Ce n’est pas à un peuple au parler obscur et à la langue difficile que tu es envoyé, c’est à la maison d’Israël. 
${}^{6}Ce n’est pas à des peuples nombreux, au parler obscur et à la langue difficile, dont tu ne comprendrais pas les paroles, que tu es envoyé – si je t’envoyais vers eux, ils t’écouteraient ! 
${}^{7}Mais la maison d’Israël ne voudra pas t’écouter, parce qu’ils ne veulent pas m’écouter. La maison d’Israël tout entière a le front endurci et le cœur obstiné. 
${}^{8}Et voici que je rends ton visage aussi dur que leur visage, ton front aussi dur que leur front. 
${}^{9}Comme un diamant plus dur que le roc, ainsi je rends ton front. Ne les crains pas, devant eux ne t’effraie pas – c’est une engeance de rebelles ! » 
${}^{10}Puis il me dit : « Fils d’homme, toutes les paroles que je te dirai, reçois-les dans ton cœur, écoute de toutes tes oreilles. 
${}^{11}Va, rends-toi vers les exilés, vers les fils de ton peuple, et tu leur parleras. Qu’ils écoutent ou qu’ils n’écoutent pas, tu leur diras : Ainsi parle le Seigneur Dieu. »
${}^{12}Alors l’esprit me souleva et j’entendis derrière moi le bruit d’une grande clameur : « Bénie soit la gloire du Seigneur depuis son lieu ! » 
${}^{13}Puis j’entendis le bruit que faisaient les ailes des Vivants, battant l’une contre l’autre, et le bruit des roues à côté d’eux, et le bruit d’une grande clameur. 
${}^{14}Alors l’esprit me souleva, il me saisit ; j’allais, plein d’amertume et l’esprit enfiévré ; la main du Seigneur pesait durement sur moi. 
${}^{15}J’arrivai à Tel-Aviv chez les exilés, chez ceux qui résident près du fleuve Kebar. J’y restai sept jours, frappé de stupeur, au milieu d’eux.
${}^{16}Au bout des sept jours, la parole du Seigneur me fut adressée : 
${}^{17} « Fils d’homme, je fais de toi un guetteur pour la maison d’Israël\\. Lorsque tu entendras une parole de ma bouche, tu les avertiras de ma part. 
${}^{18} Si je dis au méchant : “Tu vas mourir”, et que tu ne l’avertisses pas, si tu ne lui dis pas d’abandonner\\sa conduite mauvaise afin qu’il vive, lui, le méchant, mourra de son péché, mais à toi, je demanderai compte de son sang. 
${}^{19} Au contraire, si tu avertis le méchant, et qu’il ne se détourne pas de sa méchanceté et de sa conduite mauvaise, lui mourra de son péché, mais toi, tu auras sauvé ta vie. 
${}^{20} Si le juste se détourne de sa justice et fait le mal, je le ferai trébucher\\ : il mourra. Parce que tu ne l’auras pas averti, il mourra de son péché, et l’on ne se souviendra plus de la justice qu’il avait pratiquée ; mais à toi je demanderai compte de son sang\\. 
${}^{21} Au contraire, si tu avertis le juste de ne pas pécher, et qu’en effet il ne pèche pas, c’est certain, il vivra parce qu’il aura été averti, et toi, tu auras sauvé ta vie. »
