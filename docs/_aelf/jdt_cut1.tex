  
  
    
    \bbook{JUDITH}{JUDITH}
      
         
      \bchapter{}
      \begin{verse}
${}^{1}C’était l’an douze du règne de Nabucodonosor, roi des Assyriens à Ninive la grande ville. En ce temps-là, Arphaxad, roi des Mèdes à Ecbatane, 
${}^{2}entoura cette ville d’un mur d’enceinte en pierres de taille larges de trois coudées et longues de six, donnant au rempart une hauteur de soixante-dix coudées et une largeur de cinquante. 
${}^{3}Sur les portes, il dressa des tours de cent coudées de haut sur soixante de large à leurs fondations ; 
${}^{4}les portes elles-mêmes s’élevaient à soixante-dix coudées sur quarante de large, pour permettre les sorties de ses forces d’élite et des fantassins en ordre de bataille.
${}^{5}En ce temps-là, le roi Nabucodonosor fit la guerre au roi Arphaxad dans la Grande Plaine, c’est-à-dire la plaine située sur le territoire de Ragau. 
${}^{6}Tous les habitants de la région montagneuse se rallièrent à lui, ainsi que tous ceux des vallées de l’Euphrate, du Tigre et de l’Hydaspe, et ceux de la plaine soumise au roi d’Élam, Ariok ; de très nombreux peuples vinrent se ranger en ordre de bataille aux côtés des fils de Khéléoud. 
${}^{7}Nabucodonosor, roi des Assyriens, envoya aussi des messagers à tous les habitants de la Perside ; et à tous ceux qui habitaient vers l’ouest, aux habitants de Cilicie, de Damascène, du Liban et de l’Anti-Liban ; et à tous les habitants du littoral, 
${}^{8}aux peuples du Carmel et du Galaad, de la Haute-Galilée et de la grande plaine d’Esdrelon, 
${}^{9}à tous ceux de Samarie et de ses villes ; et au-delà du Jourdain jusqu’à Jérusalem, Batanée, Khélous, Cadès ; et au-delà du Torrent d’Égypte, jusqu’à Taphnès et Ramessès, tout le territoire de Guesèm, 
${}^{10}jusqu’au-dessus de Tanis et de Memphis, à tous les habitants de l’Égypte jusqu’aux abords du territoire de l’Éthiopie.
${}^{11}Mais tous les habitants de toute la terre méprisèrent la parole de Nabucodonosor, roi des Assyriens, et ne se rangèrent pas à ses côtés pour combattre, car ils ne le craignaient pas, ils le considéraient comme un homme isolé. Ils renvoyèrent donc ses messagers les mains vides, et sans honneurs. 
${}^{12}Nabucodonosor fut pris d’une violente fureur contre toute cette partie de la terre. Il jura par son trône et son diadème de châtier et d’anéantir par l’épée tous les territoires de Cilicie, de Damascène et de Syrie, ainsi que tous les habitants de la région de Moab, les fils d’Ammone, toute la Judée et tous les habitants d’Égypte jusqu’aux abords du territoire des deux mers.
${}^{13}En la dix-septième année, il se rangea en bataille, avec son armée, contre le roi Arphaxad, et fut le plus fort au combat. Il mit en déroute toute l’armée d’Arphaxad, toute sa cavalerie et tous ses chars. 
${}^{14}Il se rendit maître de ses villes, parvint jusqu’à Ecbatane, s’empara de ses fortins, pilla ses avenues, changea en objet de honte ce qui faisait sa splendeur. 
${}^{15}Il se saisit d’Arphaxad dans les montagnes de Ragau, l’abattit à coups de javelots et d’épieux, puis l’acheva. 
${}^{16}Ensuite, il s’en retourna à Ninive, lui, les siens et toute la troupe mêlée qui s’était jointe à lui, foule immense d’hommes de guerre. Il demeura là avec son armée, à se reposer et faire la fête pendant cent vingt jours.
      
         
      \bchapter{}
      \begin{verse}
${}^{1}La dix-huitième année, le vingt-deuxième jour du premier mois, il fut question, dans la demeure de Nabucodonosor, roi des Assyriens, du châtiment qu’il exercerait sur toute la terre, comme il l’avait dit. 
${}^{2}Il convoqua tous les officiers de sa maison et les grands de sa cour, tint avec eux un conseil secret et, de sa propre bouche, il voua totalement la terre à la malédiction. 
${}^{3}Ils décidèrent d’exterminer tout être qui n’avait pas obéi à la parole de sa bouche.
${}^{4}Lorsque le conseil fut terminé, Nabucodonosor, roi des Assyriens, fit appeler Holopherne, général en chef de son armée, le second du royaume après lui. Il lui dit : 
${}^{5}« Ainsi parle le grand roi, le seigneur de toute la terre : Toi, dès que tu te seras éloigné de ma présence, tu prendras avec toi des hommes conscients de leur vigueur, jusqu’à cent vingt mille fantassins et une multitude de chevaux avec douze mille hommes pour les monter. 
${}^{6}Tu partiras en expédition pour affronter toute la terre située à l’ouest, parce que ses habitants ont désobéi à la parole de ma bouche. 
${}^{7}Envoie des messagers pour qu’on prépare, en signe d’allégeance, la terre et l’eau, car, dans ma fureur, je vais partir en expédition contre eux. Toute la face de la terre, je la couvrirai des pieds de mes soldats, je la livrerai à leur razzia. 
${}^{8}Les blessés de ces pays rempliront les ravins ; tous les torrents et les fleuves déborderont, gorgés de cadavres. 
${}^{9}Leurs captifs, je les emmènerai jusqu’aux extrémités de la terre. 
${}^{10}Alors, pars ! Commence par occuper pour moi tous leurs territoires. Ils se rendront à toi et tu me les réserveras pour le jour de leur mise en accusation. 
${}^{11}Quant à ceux qui désobéiront, ils n’échapperont pas à ton regard : livre-les au massacre et à la razzia dans tout le territoire sous ton contrôle. 
${}^{12}Par ma vie et par la force de ma royauté, j’ai dit ! Et j’accomplirai tout cela de ma main. 
${}^{13}Et toi, ne transgresse en rien les paroles de ton seigneur, mais exécute-les rigoureusement, selon ce que je t’ai prescrit, et sans tarder. »
${}^{14}Holopherne s’éloigna de la présence de son seigneur et il convoqua tous les princes, les généraux et les officiers de l’armée d’Assour. 
${}^{15}Il compta des hommes d’élite pour les mettre en ordre de bataille, comme le lui avait commandé son seigneur, jusqu’à concurrence de cent vingt mille hommes et de douze mille archers à cheval. 
${}^{16}Il les disposa de la manière dont on range une multitude pour la guerre. 
${}^{17}Il prit des chameaux, des ânes et des mulets en très grande quantité, pour porter leur bagage ; des brebis, des bœufs et des chèvres innombrables, pour le ravitaillement. 
${}^{18}Il prit aussi des provisions en quantité pour chaque homme, de l’or et de l’argent en abondance, provenant de la maison du roi.
${}^{19}Puis il partit en expédition, lui et toute son armée, pour précéder le roi Nabucodonosor et couvrir de leurs chars, de leurs cavaliers et de leurs fantassins d’élite toute la face de la terre, à l’ouest. 
${}^{20}Une immense mêlée accompagnait leur expédition, nombreuse comme les sauterelles et comme la poussière de la terre : c’était une multitude impossible à dénombrer.
${}^{21}Ils s’éloignèrent à trois jours de marche de Ninive, jusqu’en bordure de la plaine de Bektileth. Ils établirent leur camp hors de Bektileth, non loin de la montagne située au nord de la Haute-Cilicie. 
${}^{22}De là, avec toute son armée, fantassins, cavaliers et chars, Holopherne s’engagea dans la région montagneuse. 
${}^{23}Il pourfendit Phoud et Loud, pilla tous les fils de Rassis et les fils d’Ismaël qui vivaient en bordure du désert, au sud du pays de Khéleône. 
${}^{24}Il longea l’Euphrate, traversa la Mésopotamie jusqu’aux abords de la mer, en rasant toutes les villes hautes qui surplombaient le torrent d’Abrona. 
${}^{25}Il s’empara du territoire de la Cilicie, tailla en pièces tous ceux qui lui résistaient et parvint à la frontière méridionale de Japhet, en bordure de l’Arabie. 
${}^{26}Il encercla tous les fils de Madiane, mit le feu à leurs campements et pilla leurs parcs à bétail. 
${}^{27}Ensuite, il descendit dans la plaine de Damascène, à l’époque de la moisson des blés. Il mit le feu à tous leurs champs, fit exterminer le petit et le gros bétail, dépouilla leurs villes, écrasa les moissons de leurs plaines et frappa tous leurs jeunes gens du tranchant de l’épée.
${}^{28}À son approche, crainte et tremblement fondirent sur les habitants du littoral, ceux de Tyr et de Sidon, les habitants de Sour et d’Okina, et tous les habitants de Jamnia ; les habitants d’Azôt et d’Ascalon étaient dans l’épouvante.
      
         
      \bchapter{}
      \begin{verse}
${}^{1}Ils lui envoyèrent des messagers, porteurs de paroles de paix, pour dire : 
${}^{2}« Nous sommes esclaves du grand roi Nabucodonosor, et nous voici à tes pieds sous ton regard ; traite-nous comme bon te semble. 
${}^{3}Nos fermes, notre territoire tout entier, tous nos champs de blé, notre petit et gros bétail, tous les parcs à bétail de nos campements sont à ta disposition. Utilise-les comme il te plaira. 
${}^{4}Nos villes mêmes, et leurs habitants, te sont asservis ; marche à leur rencontre comme tu l’entends. » 
${}^{5}Ces hommes se présentèrent donc devant Holopherne et lui rapportèrent ces paroles.
${}^{6}Holopherne descendit vers le littoral, lui et son armée ; il établit des garnisons dans les villes hautes et y leva des hommes d’élite pour lui prêter main-forte. 
${}^{7}Les habitants de ces villes et de la région d’alentour l’accueillirent avec des couronnes et des chœurs de danse, au son des tambourins. 
${}^{8}Mais il rasa tout leur territoire et abattit leurs bosquets sacrés, car on lui avait donné pour tâche d’exterminer tous les dieux de la terre, afin que toutes les nations rendent un culte à Nabucodonosor, à lui seul, et que toute langue et toute tribu l’invoquent comme un dieu.
${}^{9}Ensuite, il se rendit en face d’Esdrelon, près de Dotaïa, localité située devant la grande chaîne de montagnes de Judée. 
${}^{10}Il établit son campement à mi-chemin entre Guéba et Scythopolis, et il resta là tout un mois afin de rassembler tout le bagage de son armée.
      
         
      \bchapter{}
      \begin{verse}
${}^{1}Les fils d’Israël, habitants de la Judée, apprirent tout ce qu’Holopherne, le général en chef de Nabucodonosor, roi des Assyriens, avait fait aux nations, et la manière dont il avait dépouillé tous leurs sanctuaires et les avait livrés à l’anéantissement. 
${}^{2}Ils furent saisis d’une grande, très grande crainte devant lui, et ils furent bouleversés pour Jérusalem et pour le Temple du Seigneur leur Dieu. 
${}^{3}En effet, leur retour de captivité était encore tout récent ; le peuple entier de Judée venait à peine de se regrouper ; le mobilier, l’autel et la demeure de Dieu, qui avaient été profanés, venaient d’être à nouveau consacrés.
${}^{4}Ils envoyèrent des messagers dans tout le territoire de Samarie et à Kona, Bethorone, Belmaïne, Jéricho, et jusqu’à Khoba, Ésora et le Val de Salem. 
${}^{5}Ils occupèrent tous les sommets des hautes montagnes, fortifièrent les villages qui s’y trouvaient et en firent des dépôts pour les provisions en vue de préparer la guerre, car leurs champs venaient d’être moissonnés. 
${}^{6}Le grand prêtre Joakim, qui résidait alors à Jérusalem, écrivit aux habitants de Béthulie et de Bétomesthaïm, ville située en face d’Esdrelon et de la plaine de Dothaïne, 
${}^{7}pour leur dire de bloquer les cols de la région montagneuse, seule voie d’accès vers la Judée. Il leur serait facile, en effet, d’arrêter ceux qui passeraient, car le passage étroit ne se laissait franchir que par deux hommes à la fois. 
${}^{8}Les fils d’Israël agirent selon les ordres du grand prêtre Joakim et du Conseil des anciens, qui représentait tout le peuple d’Israël et siégeait à Jérusalem.
${}^{9}Avec une ardeur soutenue, tous les hommes d’Israël crièrent vers Dieu ; avec une ardeur soutenue, ils s’humilièrent. 
${}^{10}Eux-mêmes, leurs femmes, leurs tout-petits et leurs troupeaux, ainsi que tout immigré, journalier ou esclave acheté mirent des sacs sur leurs reins. 
${}^{11}Tous les Israélites de Jérusalem, hommes, femmes et enfants, se jetèrent sur le sol devant le Temple, la tête couverte de cendres, et déployèrent leurs sacs devant le Seigneur. 
${}^{12}Ils enveloppèrent d’un sac l’autel lui-même et crièrent d’un seul cœur vers le Dieu d’Israël, le suppliant ardemment de ne pas livrer leurs tout-petits à la razzia, leurs femmes au rapt, les villes de leur héritage à l’anéantissement, le Lieu saint à la profanation et à l’outrage pour la plus grande joie des nations. 
${}^{13}Et le Seigneur entendit leur voix, il regarda leur détresse. Le peuple observa un jeûne pendant de nombreux jours, dans toute la Judée et à Jérusalem, devant le Lieu saint du Seigneur souverain de l’univers.
${}^{14}Le grand prêtre Joakim, tous les prêtres qui se tenaient devant le Seigneur, et ceux qui assuraient le service liturgique, les reins enveloppés de toile à sac, offraient l’holocauste perpétuel, les offrandes votives et les dons volontaires du peuple. 
${}^{15}Leurs bonnets étaient couverts de cendre ; de toutes leurs forces ils criaient vers le Seigneur, le suppliant de visiter avec bonté toute la maison d’Israël.
      
         
      \bchapter{}
      \begin{verse}
${}^{1}On informa Holopherne, général en chef de l’armée d’Assour, que les fils d’Israël se préparaient au combat, qu’ils avaient fermé les défilés de la région montagneuse, fortifié tous les hauts sommets et mis des pièges dans les plaines. 
${}^{2}Il entra dans une violente colère et convoqua tous les chefs de Moab, les généraux d’Ammone et tous les gouverneurs du littoral. 
${}^{3}Il leur dit : « Fils de Canaan, informez-moi. Quel est ce peuple établi dans la région montagneuse et quelles sont les villes qu’il habite ? Quelle est l’importance de son armée et en quoi consistent sa force et sa vigueur ? Quel roi est à leur tête pour commander ses troupes ? 
${}^{4}Et pourquoi ont-ils dédaigné de venir à ma rencontre, à la différence de tous ceux qui habitent à l’ouest ? »
${}^{5}Akhior, le commandant de tous les fils d’Ammone, répondit : « Que mon seigneur daigne écouter les paroles de la bouche de ton serviteur : je t’informerai de la vérité sur le peuple qui habite cette région montagneuse, non loin de toi. Il ne sortira pas de mensonge de la bouche de ton serviteur. 
${}^{6}Ce peuple, ce sont des gens qui descendent des Chaldéens. 
${}^{7}Ils allèrent d’abord s’établir en Mésopotamie, parce qu’ils ne voulaient pas suivre les dieux de leurs pères qui étaient nés en Chaldée. 
${}^{8}Ils s’étaient écartés, en effet, du chemin de leurs ancêtres et se prosternaient devant le Dieu du ciel, le Dieu qu’ils avaient appris à connaître. On les expulsa loin des dieux de Chaldée et ils s’exilèrent en Mésopotamie, où ils s’établirent pour de longs jours.
${}^{9}Puis leur Dieu leur dit de quitter ce lieu où ils s’étaient établis et d’aller au pays de Canaan. Ils s’y installèrent et furent comblés d’or, d’argent et de troupeaux en surabondance. 
${}^{10}Ils descendirent ensuite en Égypte, car une famine avait recouvert la face de la terre de Canaan. Ils s’y établirent et y restèrent aussi longtemps qu’on assura leur subsistance. Là, ils devinrent une grande multitude, une race que nul ne pouvait dénombrer. 
${}^{11}Mais le roi d’Égypte se joua d’eux en les astreignant à la corvée des briques. On les humilia, on en fit des esclaves.
${}^{12}Alors, ils crièrent vers leur Dieu, qui frappa tout le pays d’Égypte de fléaux incurables, si bien que les Égyptiens les expulsèrent loin de leur présence. 
${}^{13}Dieu assécha devant eux la mer Rouge 
${}^{14}et les conduisit sur la route du Sinaï et de Cadès-Barné. Ils expulsèrent tous les habitants du désert. 
${}^{15}Ils occupèrent le pays des Amorites et mirent leur vigueur à exterminer tous les Heshbonites. Après avoir traversé le Jourdain, ils reçurent en possession toute la région montagneuse, 
${}^{16}expulsant devant eux le Cananéen, le Perizzite et le Jébuséen, Sichem et tous les Guirgashites. C’est là qu’ils s’installèrent pour de longs jours.
${}^{17}Aussi longtemps qu’ils ne commirent pas de péchés devant leur Dieu, la prospérité fut avec eux, car avec eux est un Dieu qui déteste l’injustice. 
${}^{18}Mais lorsqu’ils s’écartèrent du chemin qui leur était fixé, ils furent complètement exterminés en de nombreux combats ; ils furent aussi emmenés captifs sur une terre qui n’était pas la leur. Le temple de leur Dieu fut rasé et leurs villes tombèrent au pouvoir de leurs adversaires.
${}^{19}Or, maintenant qu’ils sont revenus vers leur Dieu, ils sont remontés de la terre où ils avaient été dispersés, ils ont repris possession de Jérusalem où se trouve leur sanctuaire, ils se sont installés dans la région montagneuse, car elle était déserte.
${}^{20}Maintenant donc, maître et seigneur, s’il y a dans ce peuple quelque égarement, s’ils pèchent contre leur Dieu, nous examinerons avec soin s’il y a bien chez eux cette occasion de chute, puis nous monterons et nous les combattrons. 
${}^{21}Mais si cette nation n’a pas transgressé sa Loi, que mon seigneur les évite, de peur que leur Seigneur et leur Dieu ne les couvre de son bouclier. Nous serions alors accablés d’outrages devant toute la terre. »
${}^{22}Quand Akhior eut fini de prononcer ces paroles, toute la foule rassemblée en cercle autour de la tente se mit à récriminer. Les grands de l’entourage d’Holopherne et tous les habitants du littoral et de Moab parlaient de le mettre en pièces : 
${}^{23}« Non, nous n’avons rien à craindre des fils d’Israël. C’est un peuple sans puissance ni force, incapable de tenir dans une bataille un peu rude. 
${}^{24}Allons donc ! Montons, et ton armée n’en fera qu’une bouchée, ô notre maître Holopherne. »
      
         
      \bchapter{}
      \begin{verse}
${}^{1}Quand fut calmé le tumulte des hommes qui entouraient le conseil, Holopherne, général en chef de l’armée d’Assour, dit à Akhior devant tout le peuple des tribus étrangères et tous les fils de Moab : 
${}^{2}« Qui es-tu, toi, Akhior, et qui sont les mercenaires d’Éphraïm, pour que tu fasses le prophète chez nous, comme aujourd’hui, et pour que tu nous dises de ne pas combattre la race d’Israël ? Tu prétends que leur Dieu les couvrira de son bouclier ? Qui donc est Dieu, sinon Nabucodonosor ? C’est lui qui enverra sa force et les exterminera de la face de la terre, et leur Dieu ne les délivrera pas ! 
${}^{3}Mais nous, serviteurs de Nabucodonosor, nous les frapperons comme s’ils n’étaient qu’un seul homme, et ils ne résisteront pas à la force de nos chevaux. 
${}^{4}Nous les submergerons chez eux, leurs montagnes s’enivreront de leur sang, leurs plaines seront remplies de leurs cadavres ; loin de pouvoir tenir pied devant nous, ils périront du premier au dernier – parole du roi Nabucodonosor, seigneur de toute la terre. Car il a parlé, et les paroles de son discours ne resteront pas vaines.
${}^{5}Toi, donc, Akhior, mercenaire d’Ammone, qui as tenu ce discours, en ce jour où tu manques à tes devoirs, tu ne verras plus mon visage à partir d’aujourd’hui, jusqu’à ce que j’aie châtié cette race venue d’Égypte. 
${}^{6}Alors, le fer de mon armée et la lance de mes officiers transperceront tes flancs et tu tomberas parmi les blessés, quand je me tournerai contre la race d’Israël. 
${}^{7}Mes serviteurs vont t’emmener dans la région montagneuse et te laisser dans une des villes contrôlant les cols. 
${}^{8}Tu périras mais pas avant d’être exterminé avec ces gens. 
${}^{9}Et si vraiment tu espères en ton cœur qu’ils ne seront pas pris, que ton visage n’ait pas l’air abattu ! J’ai parlé, et aucune de mes paroles ne sera sans effet. »
${}^{10}Holopherne donna l’ordre aux serviteurs qui se tenaient près de lui dans sa tente de se saisir d’Akhior, de l’emmener à Béthulie et de le livrer aux mains des fils d’Israël. 
${}^{11}Les serviteurs le saisirent donc, le menèrent hors du camp vers la plaine, puis ils passèrent d’une région à l’autre, du centre de la plaine vers la montagne, et ils parvinrent près des sources en contrebas de Béthulie.
${}^{12}Quand les hommes de la ville au sommet de la montagne les aperçurent, ils prirent leurs armes et firent une sortie à l’extérieur de la ville, et tous les hommes armés de frondes prirent le contrôle du col, en lançant leurs pierres sur les serviteurs d’Holopherne. 
${}^{13}Ceux-ci se glissèrent en contrebas de la montagne, ligotèrent Akhior et le laissèrent là, jeté au pied de la montagne. Puis ils s’en retournèrent vers leur seigneur.
${}^{14}Les fils d’Israël descendirent alors de leur ville, s’arrêtèrent près d’Akhior, le délièrent, l’emmenèrent à Béthulie et le présentèrent devant les chefs de la ville, 
${}^{15}qui étaient en ce temps-là Ozias, fils de Michée, de la tribu de Siméon, Khabris, fils de Gothoniel, et Kharmis, fils de Melkhiel. 
${}^{16}Ceux-ci convoquèrent tous les anciens de la ville. Tous les jeunes gens et les femmes accoururent aussi à l’assemblée. On plaça Akhior au milieu de tout le peuple et Ozias l’interrogea sur ce qui était arrivé. 
${}^{17}Dans sa réponse, il leur rapporta les paroles dites au conseil d’Holopherne, toutes les paroles qu’il avait lui-même prononcées au milieu des chefs des fils d’Assour, ainsi que les paroles grandiloquentes d’Holopherne à l’encontre de la maison d’Israël.
${}^{18}Le peuple se jeta sur le sol, se prosterna devant Dieu et s’écria : 
${}^{19}« Seigneur, Dieu du ciel, considère leur arrogance et prends en pitié l’humiliation de notre race. En ce jour, tourne ton regard vers le visage de ceux qui te sont consacrés. »
${}^{20}On réconforta Akhior et on le félicita vivement. 
${}^{21}Au sortir de l’assemblée, Ozias le prit chez lui et offrit un banquet aux anciens. Durant toute cette nuit-là, on implora le secours du Dieu d’Israël.
      
         
      \bchapter{}
      \begin{verse}
${}^{1}Le lendemain, Holopherne donna l’ordre à toute son armée et à toute la troupe qui était venue lui prêter main-forte de se mettre en marche vers Béthulie, de s’emparer des cols de la région montagneuse et d’engager le combat avec les fils d’Israël. 
${}^{2}Ce jour même, tous les guerriers se mirent en marche : une armée sur pied de guerre, composée de cent soixante-dix mille fantassins et douze mille cavaliers, sans compter l’équipement et la foule considérable des hommes qui les accompagnaient à pied. 
${}^{3}Ils établirent leur cantonnement dans le vallon proche de Béthulie, près de la source, et se déployèrent en largeur depuis Dothaïn jusqu’à Belbaïm, et en longueur depuis Béthulie jusqu’à Kyamone, en face d’Esdrelon.
${}^{4}Quand les fils d’Israël virent cette multitude, ils furent profondément bouleversés. Ils se dirent l’un à l’autre : « Maintenant, ceux-ci vont brouter toute la surface de la terre. Ni les hautes montagnes, ni les ravins, ni les collines ne résisteront à leur charge. » 
${}^{5}Chacun prit ses armes de combat ; ils allumèrent des feux sur les tours et restèrent à veiller toute cette nuit-là.
${}^{6}Le deuxième jour, Holopherne fit sortir toute sa cavalerie en face des fils d’Israël qui étaient à Béthulie. 
${}^{7}Il explora les cols donnant accès à leur ville ; il repéra leurs sources d’eau, s’en empara, y laissa des postes d’hommes de guerre, puis retourna lui-même vers sa troupe. 
${}^{8}Alors, s’approchèrent de lui tous les chefs des fils d’Ésaü, tous les commandants du peuple de Moab et les généraux du littoral. Ils lui dirent : 
${}^{9}« Que notre maître daigne écouter cette parole ; ainsi ton armée ne subira aucun dommage. 
${}^{10}En effet, ce peuple des fils d’Israël ne compte pas sur ses lances, mais sur la hauteur des montagnes où il habite. Et il n’est certes pas facile d’accéder aux sommets de leurs montagnes. 
${}^{11}Aussi, maître, ne va pas leur faire la guerre comme dans une bataille rangée, et il ne tombera pas un seul homme de ta troupe. 
${}^{12}Reste dans ton camp en y gardant tous les hommes de ton armée, et que tes esclaves prennent le contrôle de la source d’eau qui jaillit au pied de la montagne, 
${}^{13}car c’est là que tous les habitants de Béthulie viennent puiser leur eau. La soif les affaiblira et ils livreront la ville. Quant à nous et nos troupes, nous monterons sur les sommets des montagnes voisines et nous y installerons des avant-postes, afin que pas un homme ne sorte de la ville. 
${}^{14}Ils dépériront de faim, eux, leurs femmes et leurs enfants ; avant même que l’épée ne les touche, ils s’écrouleront en pleine rue, là où ils habitent. 
${}^{15}Ainsi, tu leur feras payer chèrement leur révolte et leur refus de venir à ta rencontre pacifiquement. »
${}^{16}Ces paroles plurent à Holopherne et à tous ses officiers, et il ordonna d’agir en conséquence. 
${}^{17}Alors, le détachement des fils d’Ammone se mit en route, et avec lui cinq mille fils d’Assour. Ils prirent position dans le vallon et s’emparèrent des points d’eau et des sources des fils d’Israël. 
${}^{18}Les fils d’Ésaü montèrent avec les fils d’Ammone. Ils prirent position dans la région montagneuse en face de Dothaïn. Ils envoyèrent aussi des hommes vers le sud et vers l’est, face à Egrebel, localité proche de Khous, qui surplombe le torrent de Mokhmour. Le reste de l’armée assyrienne prit position dans la plaine et recouvrit toute la surface du pays. Leurs tentes et leur bagage formaient un immense campement, composé d’une foule considérable.
${}^{19}Alors, les fils d’Israël crièrent vers le Seigneur leur Dieu, car ils sentaient fondre leur courage. En effet, leurs ennemis les avaient encerclés complètement, leur ôtant toute possibilité de fuir. 
${}^{20}Tout le camp d’Assour – ses fantassins, ses chars et ses cavaliers – maintint l’encerclement pendant trente-quatre jours. Quant à tous les habitants de Béthulie, l’eau vint à manquer dans tous leurs récipients, 
${}^{21}et toutes les citernes se vidèrent. Il n’y eut pas un seul jour où ils purent boire à leur soif, car la boisson était rationnée. 
${}^{22}Les tout-petits dépérissaient. Les femmes et les jeunes gens, épuisés par la soif, s’écroulaient en pleine rue dans la ville et dans les passages de ses portes ; ils avaient perdu toute énergie.
${}^{23}Alors, la troupe tout entière, les jeunes gens, les femmes et les enfants s’ameutèrent contre Ozias et les chefs de la ville. Ils se mirent à crier et à hurler devant tous les anciens : 
${}^{24}« Que Dieu soit juge entre vous et nous, car vous nous avez causé une grande injustice en refusant d’engager des pourparlers de paix avec les fils d’Assour. 
${}^{25}Maintenant, il n’y a personne pour nous porter secours. Au contraire, Dieu nous a vendus pour que nous tombions entre leurs mains, que nous soyons terrassés par la soif devant eux et que nous subissions de lourdes pertes. 
${}^{26}Faites-les donc venir maintenant et livrez la ville entière au pillage de la troupe d’Holopherne et de toute son armée ! 
${}^{27}Car, pour nous, mieux vaut devenir leur proie : nous serons esclaves, mais nous serons vivants, et nous ne verrons pas de nos propres yeux mourir nos tout-petits, nous ne verrons pas nos femmes et nos enfants rendre l’âme. 
${}^{28}Nous prenons à témoin devant vous le ciel et la terre, ainsi que notre Dieu, le Seigneur de nos pères, qui nous châtie selon nos fautes et selon les péchés de nos pères, afin que ne s’accomplisse pas en ce jour même tout ce que nous venons d’évoquer. »
${}^{29}Une plainte immense et unanime s’éleva du milieu de l’assemblée. Ils crièrent d’une voix forte vers le Seigneur leur Dieu. 
${}^{30}Ozias leur dit alors : « Courage, frères, tenons encore cinq jours ; d’ici là, le Seigneur notre Dieu fera revenir sur nous sa miséricorde. Il ne nous abandonnera pas jusqu’au bout ! 
${}^{31}Mais si ces jours s’écoulent sans qu’il nous vienne du secours, alors j’agirai selon vos paroles. » 
${}^{32}Ayant dit cela, il dispersa la troupe, chacun dans ses quartiers : les hommes se rendirent sur les remparts et les tours de la ville. Quant aux femmes et aux enfants, ils furent renvoyés dans leurs maisons. La ville était plongée dans une profonde humiliation.
      
         
      \bchapter{}
      \begin{verse}
${}^{1}En ces jours-là, Judith apprit ce qui s’était passé. Elle était fille de Merari, fils d’Ox, fils de Joseph, fils d’Oziel, fils d’Helkias, fils d’Ananias, fils de Gédéon, fils de Raphaïn, fils d’Akitob, fils d’Élie, fils de Khelkias, fils d’Éliab, fils de Nathanaël, fils de Salamiel, fils de Sarasadaï, fils d’Israël.
${}^{2}Judith avait pour mari Manassé, de la même tribu qu’elle et du même clan ; il mourut à l’époque de la moisson de l’orge\\. 
${}^{3} Il surveillait les moissonneurs dans les champs quand il fut frappé d’insolation\\. Il s’alita et mourut dans sa ville de Béthulie. On l’ensevelit avec ses pères dans le champ situé entre Dothaïn et Balamone.
${}^{4}Judith vécut chez elle dans le veuvage trois ans et quatre mois. 
${}^{5} Elle s’était fait une tente sur sa terrasse. Elle avait mis sur ses reins une toile à sac et elle portait des vêtements de veuve par-dessus. 
${}^{6} Elle jeûnait tous les jours de son veuvage, excepté la veille et le jour du sabbat, la veille et le jour de la nouvelle lune, et les jours de fête et de réjouissance de la maison d’Israël. 
${}^{7} Elle était belle\\et très séduisante. Manassé, son mari, lui avait laissé de l’or et de l’argent, des esclaves, garçons et filles, des troupeaux et des champs, dont elle gardait la disposition. 
${}^{8} Elle ne donnait prise à aucune critique, car elle craignait Dieu profondément.
${}^{9}Elle apprit donc que le peuple, découragé par la pénurie d’eau, avait adressé de dures critiques au chef de la ville. Elle apprit aussi tout ce qu’Ozias leur avait dit, comment il leur avait juré de livrer la ville aux Assyriens au bout de cinq jours. 
${}^{10}Alors, elle envoya sa suivante, celle qui était préposée à tous ses biens, inviter Ozias, Khabris et Kharmis, les anciens de la ville.
${}^{11}Ils vinrent chez elle, et elle leur dit : « Écoutez-moi, chefs des habitants de Béthulie : elle n’est pas droite, la parole que vous avez prononcée aujourd’hui devant le peuple, non plus que ce serment prêté entre Dieu et vous, quand vous vous êtes engagés à livrer la ville à nos ennemis, si le Seigneur ne nous portait secours dans le délai fixé. 
${}^{12}Allons ! Qui donc êtes-vous pour mettre en ce jour Dieu à l’épreuve, et pour vous dresser au-dessus de lui parmi les fils des hommes ? 
${}^{13}En réalité, vous qui scrutez les intentions du Seigneur souverain de l’univers, vous n’y comprendrez jamais rien ! 
${}^{14}La profondeur du cœur de l’homme, vous ne la découvrirez pas ; les raisonnements de son esprit, vous ne les saisirez pas. Comment donc pourrez-vous sonder le Dieu qui a fait tout cela, comprendre sa pensée et reconnaître son projet ? Non, frères, n’irritez pas le Seigneur notre Dieu ! 
${}^{15}Car même s’il n’a pas l’intention de nous porter secours dans les cinq jours, il a le pouvoir, lui, de nous protéger aux jours qu’il voudra, comme de nous exterminer devant nos ennemis.
${}^{16}Vous donc, n’exigez pas de gages au sujet des volontés du Seigneur notre Dieu. Car il n’est pas comme un homme, Dieu, pour qu’on lui adresse des menaces ; il n’est pas comme un fils d’homme, pour qu’on le soumette à l’arbitrage. 
${}^{17}C’est pourquoi, en attendant avec patience le salut qui vient de lui, invoquons-le à notre secours. Il écoutera notre voix, si cela lui plaît. 
${}^{18}En effet, il ne s’est pas trouvé dans notre génération, il n’y a aujourd’hui ni tribu, ni clan, ni bourg, ni ville qui se soient prosternés devant des dieux faits de main d’homme. Cela s’est produit autrefois 
${}^{19}et, pour cette raison, nos pères furent livrés à l’épée et à la razzia et ils subirent une lourde défaite devant nos ennemis. 
${}^{20}Quant à nous, nous ne connaissons pas d’autre Dieu que lui. C’est pourquoi nous gardons l’espoir qu’il ne nous méprisera pas, non plus que ceux de notre race.
${}^{21}De plus, si nous nous laissons prendre, toute la Judée s’effondrera de même, le Lieu saint sera pillé et notre sang devra répondre de sa profanation. 
${}^{22}Le meurtre de nos frères, la captivité de notre terre et la dévastation de notre héritage retomberont sur nos têtes au milieu des nations où nous connaîtrons l’esclavage, et nous serons un scandale et une honte face à nos conquérants. 
${}^{23}Notre esclavage n’aboutira pas à un retour en grâce, mais le Seigneur notre Dieu en fera notre déshonneur. 
${}^{24}Maintenant donc, frères, montrons à nos frères que leur vie est suspendue à la nôtre, et que le Lieu saint, la demeure de Dieu et l’autel dépendent de nous.
${}^{25}Plus encore, rendons grâce au Seigneur notre Dieu, qui nous met à l’épreuve comme nos pères. 
${}^{26}Rappelez-vous comment il agit avec Abraham, comment il mit Isaac à l’épreuve, et tout ce qui arriva à Jacob, en Mésopotamie de Syrie, lorsqu’il gardait le petit bétail de Laban, le frère de sa mère. 
${}^{27}De même qu’il les fit passer par le feu de l’épreuve pour scruter leurs cœurs, le Seigneur ne cherche pas à nous punir. S’il flagelle ceux qui s’approchent de lui, c’est pour leur donner un avertissement. »
${}^{28}Ozias dit à Judith : « Dans tout ce que tu as dit, tu as parlé avec un cœur perspicace et personne ne s’opposera à tes paroles. 
${}^{29}En effet, ce n’est pas d’aujourd’hui que ta sagesse est manifeste. Depuis le début de ta vie, tout le peuple a pu connaître ton intelligence, les qualités foncières de ton cœur. 
${}^{30}Mais le peuple a eu si soif qu’il nous a contraints d’agir comme nous le leur avons dit et de nous engager par un serment, que nous ne transgresserons pas. 
${}^{31}Maintenant, toi qui es une femme pieuse, supplie donc le Seigneur pour nous ; alors, il enverra la pluie pour remplir nos citernes et nous ne serons plus épuisés. »
${}^{32}Judith leur répondit : « Écoutez-moi bien, car je vais accomplir une action dont le souvenir parviendra aux fils de notre race, de génération en génération. 
${}^{33}Vous, tenez-vous cette nuit près de la porte de la ville, et moi, je sortirai avec ma suivante. Avant la date où vous avez dit que vous livreriez la ville à nos ennemis, le Seigneur visitera Israël par ma main. 
${}^{34}Mais ne cherchez pas à connaître mon action : je ne vous dirai rien avant que soit achevé ce que j’ai à faire. »
${}^{35}Ozias et les chefs de la ville lui dirent : « Va en paix et que le Seigneur Dieu marche devant toi pour châtier nos ennemis. » 
${}^{36}Ils quittèrent la tente de Judith et rejoignirent leurs postes.
      
         
      \bchapter{}
      \begin{verse}
${}^{1}Judith se jeta face contre terre, répandit de la cendre sur sa tête et ne garda que le sac dont elle était revêtue. C’était précisément l’heure où, à Jérusalem, on présentait l’encens du soir dans la demeure de Dieu. Elle cria d’une voix forte vers le Seigneur :
${}^{2}« Seigneur, Dieu de mon père Siméon,
        \\tu as mis dans sa main l’épée
        \\pour punir des étrangers
        \\qui avaient souillé une vierge
        en dénouant sa ceinture,
        \\qui l’avaient déshonorée
        en mettant sa cuisse à nu,
        \\qui l’avaient outragée
        en profanant son sein.
        \\Toi, tu avais dit : “C’est inadmissible”,
        et pourtant, ils le firent.
${}^{3}En représailles, tu as livré leurs chefs au massacre.
        \\Par tromperie, tu as livré au sang
        la couche qui avait rougi de leur tromperie.
        \\Tu as frappé les esclaves à côté des princes
        et les princes sur leurs trônes.
${}^{4}Tu as livré leurs femmes au rapt,
        leurs filles à la captivité
        \\et tout leur butin en partage
        aux fils que tu aimes,
        \\eux qui, brûlant d’ardeur pour toi,
        pris de dégoût pour la souillure de leur sang,
        t’avaient appelé au secours.
        \\Ô Dieu, mon Dieu,
        moi, une veuve, exauce-moi aussi.
${}^{5}Car c’est toi qui as fait le passé,
        le présent et l’avenir ;
        \\ce qui est maintenant et ce qui viendra,
        ton esprit l’a conçu ;
        \\ce que tu avais dans l’esprit
        est advenu.
${}^{6}Les œuvres que tu as voulues
        se sont présentées en disant : “Nous voici !”
        \\Car tous tes chemins sont préparés,
        et ton jugement, c’est ta prescience qui le fonde.
${}^{7}Car innombrable est devenue l’armée des Assyriens ;
        de leurs chevaux et cavaliers, ils se sont prévalus ;
        du bras de leurs fantassins, ils ont tiré orgueil ;
        \\ils ont mis leur espoir dans le bouclier,
        dans le javelot, l’arc et la fronde.
        \\Ils n’ont pas reconnu que tu es le Seigneur,
        le briseur de guerres.
${}^{8}Ton nom est “Le Seigneur”.
        Toi, écrase leur vigueur par ta puissance,
        rabaisse leur force dans ta fureur,
        \\car ils ont décidé de profaner ton Lieu saint,
        de souiller la Tente où repose ton nom de gloire,
        d’abattre par le fer la puissance de ton autel.
${}^{9}Regarde leur arrogance,
        envoie ta colère sur leurs têtes,
        \\mets dans ma main de veuve
        la force d’accomplir ce que j’ai dans l’esprit.
${}^{10}Par la tromperie de mes lèvres,
        \\frappe l’esclave avec le chef,
        le chef avec son officier.
        \\Renverse leur superbe
        par la main d’une femme.
${}^{11}Car ce n’est pas dans le nombre
        que réside ta force,
        \\ni ton pouvoir en des hommes vigoureux.
        \\Mais tu es le Dieu des humbles,
        secours des opprimés,
        \\protecteur des faibles,
        refuge des délaissés,
        sauveur des désespérés.
${}^{12}Oui, toi, Dieu de mon père,
        Dieu de l’héritage d’Israël,
        \\maître du ciel et de la terre,
        créateur des eaux,
        \\roi de tout ce que tu as créé,
        oui, exauce ma supplication !
${}^{13}Donne-moi un langage trompeur
        pour blesser et meurtrir
        \\ceux qui ont tramé de cruels projets
        contre ton alliance, ta Demeure consacrée,
        le mont Sion et le domaine de tes fils.
${}^{14}À ta nation entière, à toutes les tribus,
        \\donne de reconnaître et de comprendre
        que tu es, toi, le Dieu de toute puissance et force :
        \\en dehors de toi, il n’en est pas d’autre
        pour couvrir de son bouclier la race d’Israël. »
      
         
      \bchapter{}
      \begin{verse}
${}^{1}Judith acheva de crier vers le Dieu d’Israël et de prononcer toutes ces paroles. 
${}^{2}Elle se releva de sa prostration, appela sa suivante et redescendit à l’intérieur de sa demeure, là où elle se tenait les jours de sabbats et de fêtes. 
${}^{3}Elle retira la toile à sac dont elle était revêtue et ôta ses habits de veuve. Elle prit de l’eau pour se baigner entièrement et elle s’enduisit d’une huile au lourd parfum. Elle coiffa sa chevelure. Elle ajusta sa ceinture, puis revêtit ses habits de fête, ceux qu’elle avait portés du vivant de son mari Manassé. 
${}^{4}Elle chaussa des sandales, mit ses anneaux de chevilles, ses bracelets, ses bagues, ses boucles d’oreilles, et toute sa parure. Elle se fit très belle afin de séduire les regards de tous les hommes qui la verraient. 
${}^{5}Puis, elle donna à sa suivante une outre de vin et un flacon d’huile. Elle remplit une besace de galettes de farine d’orge, de gâteaux de fruits secs et de pains blancs. Elle empaqueta tous ses paniers et en chargea sa suivante. 
${}^{6}Toutes deux sortirent alors et se rendirent à la porte de la ville de Béthulie. Elles y trouvèrent postés Ozias et les anciens de la ville, Khabris et Kharmis.
${}^{7}Quand ils virent Judith le visage transformé, et portant d’autres vêtements, sa beauté les plongea dans la plus grande admiration. Ils lui dirent :
${}^{8}« Que le Dieu de nos pères te tienne en grâce,
        qu’il te donne de mener à bien ton projet,
        \\pour l’orgueil des fils d’Israël,
        l’exaltation de Jérusalem. »
${}^{9}Judith se prosterna devant Dieu, puis elle leur dit : « Ordonnez que l’on ouvre pour moi la porte de la ville, et je sortirai pour mener à bien ce dont vous venez de parler avec moi. » Ils donnèrent l’ordre aux jeunes gens de lui ouvrir, comme elle l’avait dit. 
${}^{10}C’est ce qu’ils firent, et Judith sortit, accompagnée de sa jeune esclave. Les hommes de la ville la suivaient du regard, aussi longtemps qu’elle descendait la montagne et traversait le vallon, jusqu’à ce qu’il ne fût plus possible de l’observer.
${}^{11}Toutes deux marchaient droit devant elles dans le vallon. Les Assyriens d’un avant-poste se portèrent à leur rencontre. 
${}^{12}Ils se saisirent de Judith et l’interrogèrent : « De quel peuple es-tu ? D’où viens-tu et où vas-tu ? » Elle répondit : « Je suis une fille des Hébreux et je m’enfuis de chez eux, car ils sont sur le point de vous être livrés en pâture. 
${}^{13}Quant à moi, je viens voir Holopherne, le général en chef de votre armée, pour lui donner des renseignements sûrs. Je lui montrerai le chemin par où passer pour se rendre maître de toute la région montagneuse, sans qu’un seul homme ne manque à l’appel, sans qu’une seule vie ne se perde. » 
${}^{14}En l’entendant parler ainsi, les hommes la dévisageaient, et sa beauté les plongeait dans une vive admiration. Ils lui dirent : 
${}^{15}« Tu auras sauvé ta vie en te hâtant de descendre au-devant de notre seigneur ! Maintenant, va vers sa tente. Quelques-uns d’entre nous t’escorteront pour te remettre entre ses mains. 
${}^{16}Lorsque tu te tiendras en face de lui, que ton cœur soit sans crainte. Répète-lui ce que tu viens de dire, et il te traitera bien. » 
${}^{17}Ils détachèrent alors cent de leurs hommes qui l’encadrèrent, elle et sa suivante, et les conduisirent jusqu’à la tente d’Holopherne.
${}^{18}Il se fit un attroupement dans tout le camp, car la nouvelle de sa présence s’était répandue dans les tentes. On fit cercle autour d’elle, comme elle se tenait à l’extérieur de la tente d’Holopherne, en attendant de lui être annoncée. 
${}^{19}On admirait sa beauté, on admirait à travers elle les fils d’Israël, et l’on se disait l’un à l’autre : « Qui regardera de haut ce peuple où l’on trouve de telles femmes ? Vraiment, il n’est pas bon d’en laisser subsister un seul homme : les survivants seraient capables de subjuguer toute la terre. »
${}^{20}Ceux qui étaient étendus auprès d’Holopherne ainsi que tous ses officiers sortirent enfin. Ils introduisirent Judith dans la tente. 
${}^{21}Holopherne se reposait sur son lit, sous un voile tissé de pourpre, d’or, d’émeraudes et de pierres précieuses. 
${}^{22}On la lui annonça, et il se présenta sur le seuil de la tente, précédé de flambeaux d’argent. 
${}^{23}Lorsque Judith s’approcha d’Holopherne et de ses officiers, tous admirèrent la beauté de son visage. Se jetant face contre terre, elle se prosterna devant lui, mais les serviteurs la relevèrent.
      
         
      \bchapter{}
      \begin{verse}
${}^{1}Holopherne lui dit : « Courage, femme, que ton cœur ne craigne rien, car moi, je n’ai jamais fait de mal à quiconque choisit de servir Nabucodonosor, roi de toute la terre. 
${}^{2}Encore maintenant, si ton peuple, qui habite la région montagneuse, ne m’avait méprisé, je n’aurais pas brandi ma lance contre lui : ce sont eux qui l’ont voulu. 
${}^{3}Maintenant, dis-moi pour quelle raison tu t’es enfuie de chez eux pour venir à nous. À vrai dire, tu es venue trouver ton salut. Courage ! Tu resteras en vie cette nuit, ainsi qu’à l’avenir. 
${}^{4}Personne ne te fera de tort. Au contraire, tu seras bien traitée, comme il en va pour les serviteurs de mon seigneur, le roi Nabucodonosor. »
${}^{5}Judith lui répondit : « Accueille les paroles de ta servante ; que ton esclave puisse s’exprimer en ta présence. Cette nuit, je ne proférerai aucun mensonge devant mon seigneur. 
${}^{6}Et si tu suis les conseils de ton esclave, Dieu mènera une action à bon terme avec toi, et mon seigneur n’échouera pas dans ses projets. 
${}^{7}Vive Nabucodonosor, roi de toute la terre ! Vive sa force ! Car il t’a envoyé pour corriger tous les vivants. Et non seulement les hommes le servent grâce à toi, mais encore les bêtes sauvages, les troupeaux et les oiseaux du ciel vivront, grâce à ta vigueur, pour Nabucodonosor et pour toute sa maison. 
${}^{8}Oui, nous avons entendu parler de ta sagesse et de la subtilité de ton esprit : on rapporte par toute la terre que tu es le seul héros de tout le royaume, puissant par ton savoir-faire et admirable dans la conduite de la guerre.
${}^{9}Venons-en au discours prononcé par Akhior devant ton conseil ; nous en avons appris les termes, car les hommes de Béthulie ont épargné Akhior et celui-ci leur a communiqué tout ce qu’il avait dit en ta présence. 
${}^{10}Eh bien, maître et seigneur, ne néglige pas ce discours, mais garde-le dans ton cœur, car il est véridique. Oui vraiment, ceux de notre race ne sont jamais punis, l’épée n’a pas de pouvoir sur eux, à moins qu’ils ne pèchent contre leur Dieu. 
${}^{11}Or maintenant, afin que mon seigneur ne connaisse ni revers ni échec, la mort va fondre sur leurs têtes, car le péché s’est emparé d’eux, ce péché par lequel ils excitent la colère de leur Dieu chaque fois qu’ils commettent un écart. 
${}^{12}Comme les vivres leur manquent et que toute eau se fait rare, ils ont projeté d’abattre leurs troupeaux et résolu de consommer tout ce que Dieu, par ses lois, leur a défendu de manger. 
${}^{13}Même les prémices du blé et les dîmes du vin et de l’huile, consacrées et gardées pour les prêtres qui se tiennent à Jérusalem en présence de notre Dieu, ils ont jugé bon de les utiliser sans en rien laisser, alors qu’il n’est permis à personne dans le peuple d’y toucher, même de la main. 
${}^{14}Or, à Jérusalem, tout le monde l’a fait. C’est pourquoi ils y ont envoyé des gens pour en ramener la permission nécessaire de la part du Conseil des anciens. 
${}^{15}Lorsque cette permission leur sera communiquée et qu’ils passeront à l’acte, ce jour-là ils te seront livrés pour que tu les extermines.
${}^{16}C’est pourquoi moi, ta servante, sachant tout cela, je me suis enfuie de chez eux et Dieu m’a envoyée pour réaliser avec toi des actions qui frapperont de stupeur toute la terre ainsi que tous ceux qui l’apprendront. 
${}^{17}En effet, ta servante est remplie de la crainte de Dieu, elle sert nuit et jour le Dieu du ciel. Dorénavant je resterai donc près de toi, mon seigneur. La nuit, ta servante sortira dans le ravin, je prierai Dieu, et il me dira quand les gens de mon peuple auront commis leurs péchés. 
${}^{18}Je reviendrai alors t’en faire le rapport, tu feras une sortie avec toute ton armée et nul d’entre eux ne te résistera. 
${}^{19}Ensuite, je te conduirai à travers la Judée jusqu’aux portes de Jérusalem. Je placerai ton char au milieu de la ville, tu conduiras ses habitants comme des brebis qui n’ont pas de berger, et pas un chien ne grognera contre toi. Tout cela m’a été dit et annoncé pour que je le sache à l’avance, et j’ai été envoyée pour te l’annoncer à mon tour. »
${}^{20}Les paroles de Judith plurent à Holopherne et à tous ses officiers. Remplis d’admiration pour sa sagesse, ils s’exclamèrent : 
${}^{21}« D’une extrémité de la terre à l’autre, il ne se trouve pas de femme semblable à celle-ci par la beauté du visage et l’intelligence des paroles. » 
${}^{22}Holopherne lui dit : « Dieu a bien fait de t’envoyer en avant de ton peuple, pour que la force soit en nos mains, et la ruine sur ceux qui méprisent mon seigneur. 
${}^{23}Quant à toi, tu es aussi jolie à voir qu’habile en tes discours. Si tu agis comme tu l’as dit, ton Dieu sera mon Dieu, et toi, tu résideras dans la demeure du roi Nabucodonosor, ta renommée s’étendra sur toute la terre. »
      
         
      \bchapter{}
      \begin{verse}
${}^{1}Il ordonna de l’introduire là où se trouvait son argenterie et commanda de lui présenter des plats préparés pour lui, et de lui donner à boire de son vin. 
${}^{2}Mais Judith lui dit : « Je n’en mangerai pas, de peur qu’il n’y ait là pour moi une occasion de chute : ce que j’ai fait apporter subviendra à mes besoins. » 
${}^{3}Holopherne lui dit : « Et si tes provisions viennent à manquer, où pourrons-nous en chercher pour t’en procurer de semblables ? Car il n’y a personne de ta race parmi nous. » 
${}^{4}Judith lui répondit : « Par ta vie, mon seigneur, ta servante n’aura pas épuisé ses provisions, avant que le Seigneur ne réalise par ma main ce qu’il a projeté. » 
${}^{5}Les officiers d’Holopherne la conduisirent alors à sa tente. Elle dormit jusqu’au milieu de la nuit et se leva au moment du tour de garde du matin. 
${}^{6}Elle envoya dire à Holopherne : « Que mon seigneur donne des ordres afin de permettre à sa servante de sortir pour la prière. » 
${}^{7}Holopherne commanda donc à ses gardes de ne pas l’en empêcher.
      Elle demeura trois jours dans le camp. La nuit, elle se rendait dans le ravin de Béthulie et se baignait à la source où se trouvait le poste de garde. 
${}^{8}Quand elle en remontait, elle priait le Seigneur Dieu d’Israël d’orienter son chemin pour le relèvement des fils de son peuple. 
${}^{9}Et une fois rentrée, elle demeurait dans sa tente en état de pureté, jusqu’à ce qu’on lui apporte sa nourriture, le soir.
${}^{10}Le quatrième jour, Holopherne organisa un banquet réservé à ses propres serviteurs. Il n’adressa d’invitation à aucun de ses subordonnés. 
${}^{11}Il dit à Bagoas, le préposé à sa chambre et à toutes ses affaires : « Va donc convaincre la femme de chez les Hébreux, qui est auprès de toi, afin qu’elle vienne manger et boire avec nous. 
${}^{12}Car, pour nous, ce serait perdre la face que de laisser passer une telle femme sans avoir eu de relation avec elle. Si nous ne parvenons pas à l’attirer, elle se moquera de nous ! » 
${}^{13}Bagoas sortit de chez Holopherne et se rendit chez Judith. Il lui dit : « Que cette jolie esclave n’hésite pas à venir chez mon seigneur pour y être honorée en sa présence, pour boire du vin et se réjouir avec nous, et pour devenir aujourd’hui même comme l’une des filles des Assyriens qui se tiennent dans la demeure de Nabucodonosor. » 
${}^{14}Judith lui répondit : « Qui suis-je, moi, pour contredire mon seigneur ? Tout ce qui plaît à ses yeux, je me hâterai de le faire et ce sera pour moi un motif d’allégresse jusqu’au jour de ma mort. »
${}^{15}Elle se leva, mit son plus beau vêtement et toute sa parure féminine. Sa servante la précéda et étendit par terre devant Holopherne les toisons que Judith avait reçues de Bagoas pour son usage quotidien, afin qu’elle puisse s’y étendre pour manger. 
${}^{16}Judith entra et s’allongea. Le cœur d’Holopherne en fut transporté, son âme troublée, et il fut pris d’un violent désir de s’unir à elle car, depuis le jour où il l’avait vue, il guettait l’occasion de la séduire. 
${}^{17}Il lui dit : « Bois donc ! Viens te réjouir avec nous ! » 
${}^{18}Elle dit : « Je boirai volontiers, seigneur, car depuis ma naissance, vivre n’a jamais été pour moi plus exaltant qu’en ce jour. » 
${}^{19}Elle prit ce que lui avait préparé sa servante, puis mangea et but en face de lui. 
${}^{20}Holopherne était sous son charme ; il but du vin en très grande quantité, comme il n’en avait jamais bu en un jour depuis sa naissance.
      <p class="cantique" id="bib_ct-at_6bis"><span class="cantique_label">Cantique AT 6bis</span> = <span class="cantique_ref"><a class="unitex_link" href="#bib_jdt_13_18">Jdt 13, 18b-20</a></span>
      
         
      \bchapter{}
      \begin{verse}
${}^{1}Quand il se fit tard, les serviteurs d’Holopherne se hâtèrent de partir. Bagoas ferma la tente de l’extérieur et renvoya de la présence de son seigneur tous ceux qui se tenaient là. Ils allèrent se coucher, brisés qu’ils étaient tous par les excès du banquet. 
${}^{2}Judith fut laissée seule dans la tente, avec Holopherne effondré sur son lit, noyé dans le vin. 
${}^{3}Elle dit à sa servante de se tenir prête à l’extérieur de la chambre à coucher et de guetter le moment où elle sortirait, comme chaque jour. Elle avait dit en effet qu’elle sortirait pour sa prière, et en avait également prévenu Bagoas. 
${}^{4}Tous s’étaient donc retirés et personne, du plus petit jusqu’au plus grand, n’était resté dans la chambre à coucher.
      Debout près du lit, Judith se dit en son cœur : « Seigneur, Dieu de toute puissance, en cette heure, tourne ton regard vers les œuvres de mes mains, pour l’exaltation de Jérusalem. 
${}^{5}Car maintenant c’est le moment de ressaisir ton héritage et de réaliser mon projet pour écraser les ennemis qui se sont dressés contre nous. » 
${}^{6}Elle s’avança vers le montant du lit, proche de la tête d’Holopherne, elle en détacha son sabre, 
${}^{7}elle s’approcha du lit, empoigna la chevelure d’Holopherne et dit : « Rends-moi forte en ce jour, Seigneur, Dieu d’Israël. »
${}^{8}Par deux fois, elle le frappa au cou, de toute sa vigueur, et en détacha la tête. 
${}^{9}Puis, elle fit rouler le corps en bas de la couche et détacha le voile des colonnes. Peu après, elle sortit, confia la tête d’Holopherne à sa suivante 
${}^{10}qui la mit dans sa besace à provisions, et elles sortirent toutes deux ensemble, comme elles avaient coutume de le faire, pour aller prier.
      Les deux femmes traversèrent le camp, contournèrent le ravin, gravirent la montagne de Béthulie et parvinrent aux portes de la ville. 
${}^{11}De loin, Judith cria aux gardiens des portes : « Ouvrez, ouvrez donc la porte ! Il est avec nous, Dieu, notre Dieu, pour déployer encore sa vigueur en Israël et sa force contre ses ennemis, comme il l’a fait aujourd’hui aussi. » 
${}^{12}Dès que les hommes de la ville entendirent sa voix, ils se hâtèrent de descendre vers la porte et appelèrent les anciens de la ville. 
${}^{13}Tous les gens accoururent, du plus petit jusqu’au plus grand, car le retour de Judith leur paraissait incroyable ; ils ouvrirent la porte et accueillirent les deux femmes ; ils allumèrent un feu pour faire de la lumière et firent cercle autour d’elles. 
${}^{14}Judith leur dit d’une voix forte :
        \\« Louez Dieu, louez-le,
        \\louez Dieu car il n’a pas écarté sa miséricorde
        de la maison d’Israël
        \\mais cette nuit, par ma main,
        il a écrasé nos ennemis. »
${}^{15}Puis elle retira la tête de sa besace, la leur montra et dit : « Voici la tête d’Holopherne, général en chef de l’armée d’Assour, et voici le voile sous lequel il gisait dans son ivresse ! Le Seigneur l’a frappé par la main d’une femme. 
${}^{16}Oui, vive le Seigneur, qui m’a gardée dans le chemin où j’ai marché, car mon visage n’a séduit cet homme que pour sa perte : il n’a pas commis avec moi le péché qui m’aurait souillée et déshonorée. » 
${}^{17}Tout le peuple était transporté ; il s’inclina et se prosterna devant Dieu. Tous s’écrièrent, unanimes :
        \\« Tu es béni, toi notre Dieu,
        \\toi qui, en ce jour, as réduit à néant
        \\les ennemis de ton peuple. »
       
${}^{18}Et Ozias, l’un des chefs de la ville, dit à Judith :
        \\« Bénie sois-tu, ma fille, par le Dieu Très-Haut,
        plus que toutes les femmes de la terre ;
        \\et béni soit le Seigneur Dieu,
        Créateur du ciel et de la terre.
        \\Car le Seigneur t’a dirigée
        pour frapper à la tête le chef de nos ennemis.
         
        ${}^{19}Jamais l’espérance\\dont tu as fait preuve
        ne s’éloignera du cœur des hommes,
        \\mais ils se rappelleront éternellement
        la puissance\\de Dieu.
         
        ${}^{20}Oui, Dieu fasse que tu sois exaltée à jamais,
        qu’il te visite de ses bienfaits,
        \\car tu n’as pas épargné ta propre vie
        pour la cause de notre race humiliée,
        \\tu es sortie pour empêcher notre ruine,
        marchant avec droiture devant notre Dieu. »
       
      Et tout le peuple dit : « Amen ! Amen ! »
       
      
         
      \bchapter{}
      \begin{verse}
${}^{1}Judith leur déclara : « Écoutez-moi donc, frères : prenez cette tête et suspendez-la aux créneaux de votre rempart. 
${}^{2}Ensuite, quand l’aurore brillera et que le soleil paraîtra sur la terre, vous prendrez chacun vos armes de combat ; et vous, tous les hommes vigoureux, vous ferez une sortie hors de la ville. Vous vous donnerez un guide, comme pour descendre dans la plaine, en direction de l’avant-poste des fils d’Assour. Mais vous ne descendrez pas. 
${}^{3}Les fils d’Assour saisiront leur armement, regagneront leur camp, éveilleront les généraux de l’armée et accourront autour de la tente d’Holopherne, mais ils ne le trouveront pas. La peur, alors, fondra sur eux et ils s’enfuiront devant vous. 
${}^{4}Vous les poursuivrez, vous-mêmes et les habitants de tout le territoire d’Israël, et vous les abattrez sur les chemins de leur fuite. 
${}^{5}Mais auparavant, appelez-moi Akhior, l’Ammonite, afin qu’il voie et reconnaisse celui qui a méprisé la maison d’Israël, celui par qui il avait été envoyé vers nous comme un homme voué à la mort. »
${}^{6}On invita donc Akhior à sortir de chez Ozias. Sitôt arrivé, lorsqu’il vit la tête d’Holopherne dans la main d’un homme parmi l’assemblée du peuple, il tomba face contre terre, le souffle coupé. 
${}^{7}On le releva. Il se jeta aux pieds de Judith, se prosterna devant elle et lui dit : « Bénie sois-tu dans toutes les tentes de Juda et dans toutes les nations. Tous ceux qui entendront prononcer ton nom seront bouleversés. 
${}^{8}Informe-moi donc maintenant de tout ce que tu as fait durant ces jours-ci. » Et Judith lui fit, au milieu du peuple, le récit complet de ses actions depuis le jour de son départ jusqu’à ce moment où elle leur parlait. 
${}^{9}Quand elle eut terminé, le peuple poussa des cris retentissants et remplit la ville de cris de joie. 
${}^{10}En voyant tout ce que le Dieu d’Israël avait accompli, Akhior crut en Dieu, sans réserve. Il se fit circoncire, et fut admis définitivement dans la maison d’Israël.
${}^{11}Au lever de l’aurore, on suspendit la tête d’Holopherne au rempart. Les hommes de Béthulie prirent chacun leurs armes et firent par cohortes une sortie en direction des cols de la montagne. 
${}^{12}Quand les fils d’Assour les virent, ils envoyèrent prévenir leurs chefs ; ceux-ci allèrent chercher les généraux, les commandants et tous les officiers. 
${}^{13}Ils parvinrent à la tente d’Holopherne et dirent à celui qui était préposé à toutes ses affaires : « Réveille donc notre seigneur, car les esclaves ont eu l’audace de descendre contre nous pour attaquer, et pour se faire exterminer jusqu’au dernier. » 
${}^{14}Bagoas entra donc et frappa à l’ouverture de la tente, supposant qu’Holopherne dormait avec Judith. 
${}^{15}Comme personne ne semblait rien entendre, il écarta le rideau, entra dans la chambre à coucher et trouva, jeté sur le marchepied, le cadavre sans la tête. 
${}^{16}Il poussa de grands cris, avec pleurs, gémissements, cris aigus, et déchira ses vêtements. 
${}^{17}Puis il entra dans la tente où logeait Judith, mais ne la trouva pas. Il bondit alors vers le peuple en hurlant : 
${}^{18}« Les esclaves se sont révoltés ! À elle seule, une femme de chez les Hébreux a couvert de honte la maison du roi Nabucodonosor. Voilà Holopherne à terre ! Sa tête a disparu ! » 
${}^{19}À ces mots, les chefs de l’armée d’Assour, l’âme toute bouleversée, déchirèrent leurs tuniques, ils poussèrent des hurlements et de grands cris au milieu du camp.
      
         
      \bchapter{}
      \begin{verse}
${}^{1}Lorsqu’ils les entendirent, ceux qui se trouvaient dans les tentes furent mis hors d’eux-mêmes par ce qui s’était passé. 
${}^{2}Crainte et tremblement fondirent sur eux. Plus personne ne resta en place : ce fut la débandade générale ; ils s’enfuirent par tous les chemins de la plaine et de la région montagneuse. 
${}^{3}Ceux qui se trouvaient dans les campements de cette région, encerclant Béthulie, prirent la fuite, eux aussi.
      Alors tous ceux des fils d’Israël qui étaient capables de combattre foncèrent en bandes sur l’ennemi. 
${}^{4}Ozias envoya des messagers à Bétomasthaïm, à Khoba et à Kola et dans tout le territoire d’Israël pour leur annoncer l’issue des événements, afin qu’ils foncent tous sur l’adversaire et l’anéantissent. 
${}^{5}Lorsqu’ils entendirent les messagers, les fils d’Israël, tous ensemble, fondirent sur l’ennemi et le mirent en pièces, jusqu’à Khoba. Les habitants de Jérusalem survinrent également, ainsi que ceux de toute la région montagneuse, car on les avait informés de ce qui s’était passé dans le camp de leurs ennemis. Puis ce furent les gens de Galaad et de Galilée qui les prirent en tenaille et les frappèrent durement jusqu’à proximité de Damas et de sa région. 
${}^{6}Le reste des habitants de Béthulie fondit sur le camp d’Assour, le pilla et s’enrichit considérablement. 
${}^{7}De retour du carnage, les fils d’Israël se rendirent maîtres de ce qui restait. Ceux des villages et des fermes de la montagne, ainsi que de la plaine, s’emparèrent d’un large butin, car il y en avait une quantité vraiment considérable.
${}^{8}Le grand prêtre Joakim et le Conseil des anciens des fils d’Israël qui habitaient à Jérusalem vinrent contempler les bienfaits du Seigneur en faveur d’Israël, voir Judith et la saluer. 
${}^{9}Lorsqu’ils entrèrent chez elle, tous la bénirent d’une seule voix et lui dirent :
        \\« Tu es la gloire de Jérusalem,
        tu es l’orgueil d’Israël,
        tu es la fierté de notre race.
${}^{10}Tout cela, tu l’as fait de ta main ;
        en Israël, tu as fait ce qui est bien,
        et Dieu y a trouvé sa joie.
        \\Sois bénie par le Seigneur,
        souverain de l’univers,
        pour la durée des siècles. »
       
      Et tout le peuple dit : « Amen. »
       
${}^{11}Tout le peuple ramassa le butin du camp des Assyriens pendant trente jours. On donna à Judith la tente d’Holopherne, avec toute l’argenterie, les lits, les récipients et toutes ses affaires. Elle les prit, chargea elle-même sa mule, attela ses chariots, y entassa le tout. 
${}^{12}Les femmes d’Israël accoururent toutes pour voir Judith et la bénir. Certaines d’entre elles formèrent un chœur pour l’honorer. Judith prit dans ses mains des bâtons garnis de feuillage et les donna aux femmes de son cortège. 
${}^{13}Elle et ses compagnes se couronnèrent d’olivier. Judith précéda tout le peuple, menant la danse, en tête de toutes les femmes, et tous les hommes d’Israël l’accompagnaient, en armes et couronnés, chantant des hymnes. 
${}^{14}Au milieu de tout Israël, Judith entonna cette action de grâce, louange reprise par tout le peuple.
       
      <p class="cantique" id="bib_ct-at_7"><span class="cantique_label">Cantique AT 7</span> = <span class="cantique_ref"><a class="unitex_link" href="#bib_jdt_16_1">Jdt 16, 1.2a.13-15</a></span>
      
         
      \bchapter{}
      \begin{verse}
${}^{1}Elle dit :
      
         
       
        \\Chantez\\pour mon Dieu sur les tambourins.
        Jouez pour le Seigneur sur les cymbales.
        \\Joignez pour lui l’hymne\\à la louange.
        Exaltez-le ! Invoquez son nom !
         
        ${}^{2}Le Seigneur est un Dieu briseur de guerres ;
        son nom est « Le Seigneur ».
        \\\[Il a établi son camp au milieu de son peuple
        pour m’arracher à la main de mes persécuteurs.
${}^{3}Assour est venu des montagnes du nord,
        son armée est venue par dizaines de milliers ;
        \\leur multitude obstruait les torrents,
        leurs chevaux recouvraient les collines.
${}^{4}Il voulait incendier mon territoire,
        faire périr mes jeunes gens par l’épée,
        \\jeter à terre mes nourrissons,
        livrer au rapt mes tout-petits
        et m’enlever mes jeunes filles.
         
${}^{5}Mais le Seigneur souverain de l’univers les a confondus
        par la main d’une femme.
${}^{6}Non, ce n’est pas devant des jeunes gens
        que leur homme fort a succombé,
        \\ce ne sont pas des fils de Titans qui l’ont frappé,
        ni d’immenses Géants qui l’ont attaqué,
        \\mais c’est Judith, fille de Merari :
        par la beauté de son visage, elle l’a paralysé.
         
${}^{7}Elle quitta ses habits de veuve
        pour le relèvement des affligés d’Israël ;
        \\elle parfuma son visage,
${}^{8}retint ses cheveux d’un bandeau
        et mit une robe de lin, pour le séduire.
${}^{9}Sa sandale ravit son regard,
        sa beauté captiva son âme
        et le sabre lui trancha la gorge.
${}^{10}Les Perses frémirent devant son audace ;
        devant sa hardiesse, les Mèdes furent bouleversés.
         
${}^{11}Alors, les humbles de mon peuple
        poussèrent des cris retentissants.
        \\Les faibles de mon peuple hurlèrent,
        et l’ennemi fut terrifié ;
        \\ils élevèrent la voix,
        et l’ennemi tourna bride.
${}^{12}Des fils de frêles jeunes filles les transpercèrent,
        ils les blessèrent, comme rejetons de déserteurs,
        et les voilà tués dans la bataille de mon Seigneur.\]
         
        ${}^{13}Je chanterai pour mon Dieu un chant nouveau.
        \\Seigneur, tu es grand, tu es glorieux,
        admirable de force, invincible.
        ${}^{14}Que ta création, tout entière, te serve !
        Tu dis, et elle existe.
        \\Tu envoies ton souffle : elle est créée\\.
        Nul ne résiste à ta voix.
        ${}^{15}Si les bases des montagnes croulent dans les eaux,
        si les rochers, devant ta face, fondent comme cire,
        \\tu feras grâce à ceux qui te craignent.
${}^{16}\[Oui, tout sacrifice d’agréable odeur est peu de chose ;
        encore moins, toute graisse pour l’holocauste en ton honneur ;
        \\mais celui qui craint le Seigneur est grand, plus que tout.
${}^{17}Malheur aux nations qui se dressent contre ma race :
        le Seigneur souverain de l’univers
        les châtiera au jour du jugement ;
        \\qu’il livre leur chair au feu et aux vers :
        dans cette épreuve, ils pleureront à jamais. »\]
       
${}^{18}En entrant à Jérusalem, tous se prosternèrent devant Dieu, et lorsque le peuple se fut purifié, ils présentèrent leurs holocaustes, leurs offrandes volontaires et leurs dons. 
${}^{19}Judith y ajouta tout le mobilier d’Holopherne que le peuple lui avait donné, et le voile qu’elle avait elle-même emporté de la chambre à coucher, elle le consacra totalement à Dieu. 
${}^{20}Pendant trois mois, le peuple fut en liesse à Jérusalem devant le Lieu saint. Et Judith demeura avec son peuple.
${}^{21}Ces jours écoulés, chacun reprit la route vers la terre de son héritage et Judith revint à Béthulie où elle demeura dans son domaine. De son temps, elle devint célèbre dans tout le pays. 
${}^{22}Beaucoup d’hommes la désirèrent, mais aucun ne s’unit à elle, durant tous les jours de sa vie, depuis la mort de son mari Manassé, depuis qu’il avait été réuni aux siens.
${}^{23}Elle atteignit un âge très avancé et vieillit dans la maison de son mari, jusqu’à l’âge de cent cinq ans. Après avoir affranchi sa suivante, elle mourut à Béthulie, et on l’ensevelit dans la grotte où reposait son mari Manassé. 
${}^{24}La maison d’Israël fit pour elle un deuil de sept jours. Avant de mourir, elle avait réparti ses biens entre tous les proches de son mari Manassé et ceux de sa famille à elle. 
${}^{25}Plus personne ne vint effrayer les fils d’Israël du vivant de Judith et longtemps après sa mort.
