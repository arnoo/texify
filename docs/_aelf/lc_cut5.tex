  
  
${}^{28}Après avoir ainsi parlé, Jésus partit en avant pour monter à Jérusalem.
${}^{29}Lorsqu’il approcha de Bethphagé et de Béthanie, près de l’endroit appelé mont des Oliviers, il envoya deux de ses disciples, 
${}^{30}en disant : « Allez à ce village d’en face. À l’entrée, vous trouverez un petit âne attaché, sur lequel personne ne s’est encore assis. Détachez-le et amenez-le. 
${}^{31}Si l’on vous demande : “Pourquoi le détachez-vous ?” vous répondrez : “Parce que le Seigneur en a besoin.” » 
${}^{32}Les envoyés partirent et trouvèrent tout comme Jésus leur avait dit. 
${}^{33}Alors qu’ils détachaient le petit âne, ses maîtres leur demandèrent : « Pourquoi détachez-vous l’âne ? » 
${}^{34}Ils répondirent : « Parce que le Seigneur en a besoin. » 
${}^{35}Ils amenèrent l’âne auprès de Jésus, jetèrent leurs manteaux dessus, et y firent monter Jésus. 
${}^{36}À mesure que Jésus avançait, les gens étendaient leurs manteaux sur le chemin. 
${}^{37}Alors que déjà Jésus approchait de la descente du mont des Oliviers, toute la foule des disciples, remplie de joie, se mit à louer Dieu à pleine voix pour tous les miracles qu’ils avaient vus, 
${}^{38}et ils disaient :
        \\« Béni soit celui qui vient,
        le Roi, au nom du Seigneur.
        \\Paix dans le ciel et gloire au plus haut des cieux ! »
${}^{39}Quelques pharisiens, qui se trouvaient dans la foule, dirent à Jésus : « Maître, réprimande tes disciples ! » 
${}^{40}Mais il prit la parole en disant : « Je vous le dis : si eux se taisent, les pierres crieront. »
${}^{41}Lorsque Jésus fut près de Jérusalem, voyant la ville, il pleura sur elle, en disant : 
${}^{42}« Ah ! si toi aussi, tu avais reconnu en ce jour ce qui donne la paix ! Mais maintenant cela est resté caché à tes yeux. 
${}^{43}Oui, viendront pour toi des jours où tes ennemis construiront des ouvrages de siège contre toi, t’encercleront et te presseront de tous côtés ; 
${}^{44}ils t’anéantiront, toi et tes enfants qui sont chez toi, et ils ne laisseront pas chez toi pierre sur pierre, parce que tu n’as pas reconnu le moment où Dieu te visitait. »
      <h2 class="intertitle" id="d85e357224">1. Ministère de Jésus à Jérusalem (19,45 – 21)</h2>
${}^{45}Entré dans le Temple, Jésus se mit à en expulser les vendeurs. Il leur déclarait : 
${}^{46}« Il est écrit : Ma maison sera une maison de prière. Or vous, vous en avez fait une caverne de bandits. »
${}^{47}Et il était chaque jour dans le Temple pour enseigner. Les grands prêtres et les scribes, ainsi que les notables, cherchaient à le faire mourir, 
${}^{48}mais ils ne trouvaient pas ce qu’ils pourraient faire ; en effet, le peuple tout entier, suspendu à ses lèvres, l’écoutait.
      
         
      \bchapter{}
      \begin{verse}
${}^{1}Un de ces jours-là où Jésus, dans le Temple, enseignait le peuple et proclamait la Bonne Nouvelle, survinrent les grands prêtres et les scribes avec les anciens. 
${}^{2}Ils lui demandèrent : « Dis-nous par quelle autorité tu fais cela ? Ou alors qui est celui qui t’a donné cette autorité ? » 
${}^{3}Il leur répliqua : « Moi aussi, je vais vous poser une question. Dites-moi : 
${}^{4}Le baptême de Jean venait-il du ciel ou des hommes ? » 
${}^{5}Ils firent entre eux ce raisonnement : « Si nous disons : “Du ciel”, il va dire : “Pourquoi n’avez-vous pas cru à sa parole ?” 
${}^{6}Si nous disons : “Des hommes”, le peuple tout entier va nous lapider, car il est persuadé que Jean est un prophète. » 
${}^{7}Et ils répondirent qu’ils ne savaient pas d’où il venait. 
${}^{8}Alors Jésus leur déclara : « Eh bien, moi non plus, je ne vous dis pas par quelle autorité je fais cela. »
      
         
${}^{9}Il se mit à dire au peuple la parabole que voici : « Un homme planta une vigne, loua celle-ci à des vignerons et partit en voyage pour un temps assez long. 
${}^{10}Le moment venu, il envoya un serviteur auprès des vignerons afin que ceux-ci lui remettent ce qui lui revenait du fruit de la vigne. Mais les vignerons, après l’avoir frappé, renvoyèrent le serviteur les mains vides. 
${}^{11}Le maître persista et envoya un autre serviteur ; celui-là aussi, après l’avoir frappé et humilié, ils le renvoyèrent les mains vides. 
${}^{12}Le maître persista encore et il envoya un troisième serviteur ; mais après l’avoir blessé, ils le jetèrent dehors. 
${}^{13}Le maître de la vigne dit alors : “Que vais-je faire ? Je vais envoyer mon fils bien-aimé : peut-être que lui, ils le respecteront !” 
${}^{14}En le voyant, les vignerons se firent l’un à l’autre ce raisonnement : “Voici l’héritier. Tuons-le, pour que l’héritage soit à nous.” 
${}^{15}Et, après l’avoir jeté hors de la vigne, ils le tuèrent. Que leur fera donc le maître de la vigne ? 
${}^{16}Il viendra, fera périr ces vignerons et donnera la vigne à d’autres. » Les auditeurs dirent à Jésus : « Pourvu que cela n’arrive pas ! » 
${}^{17}Mais lui, posant son regard sur eux, leur dit : « Que signifie donc ce qui est écrit ?
        \\La pierre qu’ont rejetée les bâtisseurs
        \\est devenue la pierre d’angle.
${}^{18}Tout homme qui tombera sur cette pierre s’y brisera ; celui sur qui elle tombera, elle le réduira en poussière ! »
${}^{19}À cette heure-là, les scribes et les grands prêtres cherchèrent à mettre la main sur Jésus ; mais ils eurent peur du peuple. Ils avaient bien compris, en effet, qu’il avait dit cette parabole à leur intention.
${}^{20}Ils se mirent alors à le surveiller et envoyèrent des espions qui jouaient le rôle d’hommes justes pour prendre sa parole en défaut, afin de le livrer à l’autorité et au pouvoir du gouverneur. 
${}^{21}Ceux-ci l’interrogèrent en disant : « Maître, nous le savons : tu parles et tu enseignes avec droiture, tu es impartial et tu enseignes le chemin de Dieu selon la vérité. 
${}^{22}Nous est-il permis, oui ou non, de payer l’impôt à César, l’empereur ? » 
${}^{23}Mais Jésus, percevant leur fourberie, leur dit : 
${}^{24}« Montrez-moi une pièce d’argent. De qui porte-t-elle l’effigie et l’inscription ? – De César », répondirent-ils. 
${}^{25}Il leur dit : « Alors rendez à César ce qui est à César, et à Dieu ce qui est à Dieu. » 
${}^{26}Ils furent incapables de le prendre en défaut devant le peuple en le faisant parler et, tout étonnés de sa réponse, ils gardèrent le silence.
${}^{27}Quelques sadducéens – ceux qui soutiennent qu’il n’y a pas de résurrection – s’approchèrent de Jésus 
${}^{28}et l’interrogèrent : « Maître, Moïse nous a prescrit : Si un homme a un frère qui meurt en laissant une épouse mais pas d’enfant, il doit épouser la veuve pour susciter une descendance à son frère. 
${}^{29}Or, il y avait sept frères : le premier se maria et mourut sans enfant ; 
${}^{30}de même le deuxième, 
${}^{31}puis le troisième épousèrent la veuve, et ainsi tous les sept : ils moururent sans laisser d’enfants. 
${}^{32}Finalement la femme mourut aussi. 
${}^{33}Eh bien, à la résurrection, cette femme-là, duquel d’entre eux sera-t-elle l’épouse, puisque les sept l’ont eue pour épouse ? »
${}^{34}Jésus leur répondit : « Les enfants de ce monde prennent femme et mari. 
${}^{35}Mais ceux qui ont été jugés dignes d’avoir part au monde à venir et à la résurrection d’entre les morts ne prennent ni femme ni mari, 
${}^{36}car ils ne peuvent plus mourir : ils sont semblables aux anges, ils sont enfants de Dieu et enfants de la résurrection. 
${}^{37}Que les morts ressuscitent, Moïse lui-même le fait comprendre dans le récit du buisson ardent, quand il appelle le Seigneur le Dieu d’Abraham, Dieu d’Isaac, Dieu de Jacob. 
${}^{38}Il n’est pas le Dieu des morts, mais des vivants. Tous, en effet, vivent pour lui. »
${}^{39}Alors certains scribes prirent la parole pour dire : « Maître, tu as bien parlé. » 
${}^{40}Et ils n’osaient plus l’interroger sur quoi que ce soit.
${}^{41}Jésus leur demanda : « Comment peut-on dire que le Christ est fils de David ? 
${}^{42}David lui-même dit, en effet, dans le livre des Psaumes :
        \\Le Seigneur a dit à mon Seigneur :
        Siège à ma droite
${}^{43}jusqu’à ce que j’aie placé tes ennemis
        comme un escabeau sous tes pieds.
${}^{44}David l’appelle donc Seigneur : comment peut-il être son fils ? »
${}^{45}Comme tout le peuple l’écoutait, il dit à ses disciples : 
${}^{46}« Méfiez-vous des scribes qui tiennent à se promener en vêtements d’apparat et qui aiment les salutations sur les places publiques, les sièges d’honneur dans les synagogues et les places d’honneur dans les dîners. 
${}^{47}Ils dévorent les biens des veuves et, pour l’apparence, ils font de longues prières : ils seront d’autant plus sévèrement jugés. »
      
         
      \bchapter{}
      \begin{verse}
${}^{1}Levant les yeux, il vit les gens riches qui mettaient leurs offrandes dans le Trésor. 
${}^{2}Il vit aussi une veuve misérable y mettre deux petites pièces de monnaie. 
${}^{3}Alors il déclara : « En vérité, je vous le dis : cette pauvre veuve a mis plus que tous les autres. 
${}^{4}Car tous ceux-là, pour faire leur offrande, ont pris sur leur superflu mais elle, elle a pris sur son indigence : elle a mis tout ce qu’elle avait pour vivre. »
      
         
${}^{5}Comme certains parlaient du Temple, des belles pierres et des ex-voto qui le décoraient, Jésus leur déclara : 
${}^{6}« Ce que vous contemplez, des jours viendront où il n’en restera pas pierre sur pierre : tout sera détruit. » 
${}^{7}Ils lui demandèrent : « Maître, quand cela arrivera-t-il ? Et quel sera le signe que cela est sur le point d’arriver ? » 
${}^{8}Jésus répondit : « Prenez garde de ne pas vous laisser égarer, car beaucoup viendront sous mon nom, et diront : “C’est moi”, ou encore : “Le moment est tout proche.” Ne marchez pas derrière eux ! 
${}^{9}Quand vous entendrez parler de guerres et de désordres, ne soyez pas terrifiés : il faut que cela arrive d’abord, mais ce ne sera pas aussitôt la fin. »
${}^{10}Alors Jésus ajouta : « On se dressera nation contre nation, royaume contre royaume. 
${}^{11}Il y aura de grands tremblements de terre et, en divers lieux, des famines et des épidémies ; des phénomènes effrayants surviendront, et de grands signes venus du ciel.
${}^{12}Mais avant tout cela, on portera la main sur vous et l’on vous persécutera ; on vous livrera aux synagogues et aux prisons, on vous fera comparaître devant des rois et des gouverneurs, à cause de mon nom. 
${}^{13}Cela vous amènera à rendre témoignage. 
${}^{14}Mettez-vous donc dans l’esprit que vous n’avez pas à vous préoccuper de votre défense. 
${}^{15}C’est moi qui vous donnerai un langage et une sagesse à laquelle tous vos adversaires ne pourront ni résister ni s’opposer. 
${}^{16}Vous serez livrés même par vos parents, vos frères, votre famille et vos amis, et ils feront mettre à mort certains d’entre vous. 
${}^{17}Vous serez détestés de tous, à cause de mon nom. 
${}^{18}Mais pas un cheveu de votre tête ne sera perdu. 
${}^{19}C’est par votre persévérance que vous garderez votre vie.
${}^{20}Quand vous verrez Jérusalem encerclée par des armées, alors sachez que sa dévastation approche. 
${}^{21}Alors, ceux qui seront en Judée, qu’ils s’enfuient dans les montagnes ; ceux qui seront à l’intérieur de la ville, qu’ils s’en éloignent ; ceux qui seront à la campagne, qu’ils ne rentrent pas en ville, 
${}^{22}car ce seront des jours où justice sera faite pour que soit accomplie toute l’Écriture. 
${}^{23}Quel malheur pour les femmes qui seront enceintes et celles qui allaiteront en ces jours-là, car il y aura un grand désarroi dans le pays, une grande colère contre ce peuple. 
${}^{24}Ils tomberont sous le tranchant de l’épée, ils seront emmenés en captivité dans toutes les nations ; Jérusalem sera foulée aux pieds par des païens, jusqu’à ce que leur temps soit accompli.
${}^{25}Il y aura des signes dans le soleil, la lune et les étoiles. Sur terre, les nations seront affolées et désemparées par le fracas de la mer et des flots. 
${}^{26}Les hommes mourront de peur dans l’attente de ce qui doit arriver au monde, car les puissances des cieux seront ébranlées. 
${}^{27}Alors, on verra le Fils de l’homme venir dans une nuée, avec puissance et grande gloire. 
${}^{28}Quand ces événements commenceront, redressez-vous et relevez la tête, car votre rédemption approche. »
${}^{29}Et il leur dit cette parabole : « Voyez le figuier et tous les autres arbres. 
${}^{30}Regardez-les : dès qu’ils bourgeonnent, vous savez que l’été est tout proche. 
${}^{31}De même, vous aussi, lorsque vous verrez arriver cela, sachez que le royaume de Dieu est proche. 
${}^{32}Amen, je vous le dis : cette génération ne passera pas sans que tout cela n’arrive. 
${}^{33}Le ciel et la terre passeront, mes paroles ne passeront pas.
${}^{34}Tenez-vous sur vos gardes, de crainte que votre cœur ne s’alourdisse dans les beuveries, l’ivresse et les soucis de la vie, et que ce jour-là ne tombe sur vous à l’improviste 
${}^{35}comme un filet ; il s’abattra, en effet, sur tous les habitants de la terre entière. 
${}^{36}Restez éveillés et priez en tout temps : ainsi vous aurez la force d’échapper à tout ce qui doit arriver, et de vous tenir debout devant le Fils de l’homme. »
${}^{37}Il passait ses journées dans le Temple à enseigner ; mais ses nuits, il sortait les passer en plein air, à l’endroit appelé mont des Oliviers.
${}^{38}Et tout le peuple, dès l’aurore, venait à lui dans le Temple pour l’écouter.
      <h2 class="intertitle" id="d85e357959">2. La Passion et la mort (22 – 23)</h2>
      
         
      \bchapter{}
      \begin{verse}
${}^{1}La fête des pains sans levain, qu’on appelle la Pâque, approchait. 
${}^{2}Les grands prêtres et les scribes cherchaient par quel moyen supprimer Jésus, car ils avaient peur du peuple.
${}^{3}Satan entra en Judas, appelé Iscariote, qui était au nombre des Douze. 
${}^{4}Judas partit s’entretenir avec les grands prêtres et les chefs des gardes, pour voir comment leur livrer Jésus. 
${}^{5}Ils se réjouirent et ils décidèrent de lui donner de l’argent. 
${}^{6}Judas fut d’accord, et il cherchait une occasion favorable pour le leur livrer à l’écart de la foule.
${}^{7}Arriva le jour des pains sans levain, où il fallait immoler l’agneau pascal. 
${}^{8}Jésus envoya Pierre et Jean, en leur disant : « Allez faire les préparatifs pour que nous mangions la Pâque. »
${}^{9}Ils lui dirent : « Où veux-tu que nous fassions les préparatifs ? » 
${}^{10}Jésus leur répondit : « Voici : quand vous entrerez en ville, un homme portant une cruche d’eau viendra à votre rencontre ; suivez-le dans la maison où il pénétrera. 
${}^{11}Vous direz au propriétaire de la maison : “Le maître te fait dire : Où est la salle où je pourrai manger la Pâque avec mes disciples ?” 
${}^{12}Cet homme vous indiquera, à l’étage, une grande pièce aménagée. Faites-y les préparatifs. » 
${}^{13}Ils partirent donc, trouvèrent tout comme Jésus leur avait dit, et ils préparèrent la Pâque.
${}^{14}Quand l’heure fut venue, Jésus prit place à table, et les Apôtres avec lui. 
${}^{15}Il leur dit : « J’ai désiré d’un grand désir manger cette Pâque avec vous avant de souffrir ! 
${}^{16}Car je vous le déclare : jamais plus je ne la mangerai jusqu’à ce qu’elle soit pleinement accomplie dans le royaume de Dieu. » 
${}^{17}Alors, ayant reçu une coupe et rendu grâce, il dit : « Prenez ceci et partagez entre vous. 
${}^{18}Car je vous le déclare : désormais, jamais plus je ne boirai du fruit de la vigne jusqu’à ce que le royaume de Dieu soit venu. » 
${}^{19}Puis, ayant pris du pain et rendu grâce, il le rompit et le leur donna, en disant : « Ceci est mon corps, donné pour vous. Faites cela en mémoire de moi. » 
${}^{20}Et pour la coupe, après le repas, il fit de même, en disant : « Cette coupe est la nouvelle Alliance en mon sang répandu pour vous.
${}^{21}Et cependant, voici que la main de celui qui me livre est à côté de moi sur la table. 
${}^{22}En effet, le Fils de l’homme s’en va selon ce qui a été fixé. Mais malheureux cet homme-là par qui il est livré ! » 
${}^{23}Les Apôtres commencèrent à se demander les uns aux autres quel pourrait bien être, parmi eux, celui qui allait faire cela.
${}^{24}Ils en arrivèrent à se quereller : lequel d’entre eux, à leur avis, était le plus grand ? 
${}^{25}Mais il leur dit : « Les rois des nations les commandent en maîtres, et ceux qui exercent le pouvoir sur elles se font appeler bienfaiteurs. 
${}^{26}Pour vous, rien de tel ! Au contraire, que le plus grand d’entre vous devienne comme le plus jeune, et le chef, comme celui qui sert. 
${}^{27}Quel est en effet le plus grand : celui qui est à table, ou celui qui sert ? N’est-ce pas celui qui est à table ? Eh bien moi, je suis au milieu de vous comme celui qui sert.
${}^{28}Vous, vous avez tenu bon avec moi dans mes épreuves. 
${}^{29}Et moi, je dispose pour vous du Royaume, comme mon Père en a disposé pour moi. 
${}^{30}Ainsi vous mangerez et boirez à ma table dans mon Royaume, et vous siégerez sur des trônes pour juger les douze tribus d’Israël.
${}^{31}Simon, Simon, voici que Satan vous a réclamés pour vous passer au crible comme le blé. 
${}^{32}Mais j’ai prié pour toi, afin que ta foi ne défaille pas. Toi donc, quand tu seras revenu, affermis tes frères. » 
${}^{33}Pierre lui dit : « Seigneur, avec toi, je suis prêt à aller en prison et à la mort. » 
${}^{34}Jésus reprit : « Je te le déclare, Pierre : le coq ne chantera pas aujourd’hui avant que toi, par trois fois, tu aies nié me connaître. »
${}^{35}Puis il leur dit : « Quand je vous ai envoyés sans bourse, ni sac, ni sandales, avez-vous donc manqué de quelque chose ? » 
${}^{36}Ils lui répondirent : « Non, de rien. » Jésus leur dit : « Eh bien maintenant, celui qui a une bourse, qu’il la prenne, de même celui qui a un sac ; et celui qui n’a pas d’épée, qu’il vende son manteau pour en acheter une. 
${}^{37}Car, je vous le déclare : il faut que s’accomplisse en moi ce texte de l’Écriture : Il a été compté avec les impies. De fait, ce qui me concerne va trouver son accomplissement. » 
${}^{38}Ils lui dirent : « Seigneur, voici deux épées. » Il leur répondit : « Cela suffit. »
${}^{39}Jésus sortit pour se rendre, selon son habitude, au mont des Oliviers, et ses disciples le suivirent. 
${}^{40}Arrivé en ce lieu, il leur dit : « Priez, pour ne pas entrer en tentation. » 
${}^{41}Puis il s’écarta à la distance d’un jet de pierre environ. S’étant mis à genoux, il priait en disant : 
${}^{42}« Père, si tu le veux, éloigne de moi cette coupe ; cependant, que soit faite non pas ma volonté, mais la tienne. » 
${}^{43}Alors, du ciel, lui apparut un ange qui le réconfortait. 
${}^{44}Entré en agonie, Jésus priait avec plus d’insistance, et sa sueur devint comme des gouttes de sang qui tombaient sur la terre. 
${}^{45}Puis Jésus se releva de sa prière et rejoignit ses disciples qu’il trouva endormis, accablés de tristesse. 
${}^{46}Il leur dit : « Pourquoi dormez-vous ? Relevez-vous et priez, pour ne pas entrer en tentation. »
${}^{47}Il parlait encore, quand parut une foule de gens. Celui qui s’appelait Judas, l’un des Douze, marchait à leur tête. Il s’approcha de Jésus pour lui donner un baiser. 
${}^{48}Jésus lui dit : « Judas, c’est par un baiser que tu livres le Fils de l’homme ? » 
${}^{49}Voyant ce qui allait se passer, ceux qui entouraient Jésus lui dirent : « Seigneur, et si nous frappions avec l’épée ? » 
${}^{50}L’un d’eux frappa le serviteur du grand prêtre et lui trancha l’oreille droite. 
${}^{51}Mais Jésus dit : « Restez-en là ! » Et, touchant l’oreille de l’homme, il le guérit.
${}^{52}Jésus dit alors à ceux qui étaient venus l’arrêter, grands prêtres, chefs des gardes du Temple et anciens : « Suis-je donc un bandit, pour que vous soyez venus avec des épées et des bâtons ? 
${}^{53}Chaque jour, j’étais avec vous dans le Temple, et vous n’avez pas porté la main sur moi. Mais c’est maintenant votre heure et le pouvoir des ténèbres. »
${}^{54}S’étant saisis de Jésus, ils l’emmenèrent et le firent entrer dans la résidence du grand prêtre. Pierre suivait à distance. 
${}^{55}On avait allumé un feu au milieu de la cour, et tous étaient assis là. Pierre vint s’asseoir au milieu d’eux. 
${}^{56}Une jeune servante le vit assis près du feu ; elle le dévisagea et dit : « Celui-là aussi était avec lui. » 
${}^{57}Mais il nia : « Non, je ne le connais pas. » 
${}^{58}Peu après, un autre dit en le voyant : « Toi aussi, tu es l’un d’entre eux. » Pierre répondit : « Non, je ne le suis pas. » 
${}^{59}Environ une heure plus tard, un autre insistait avec force : « C’est tout à fait sûr ! Celui-là était avec lui, et d’ailleurs il est Galiléen. » 
${}^{60}Pierre répondit : « Je ne sais pas ce que tu veux dire. » Et à l’instant même, comme il parlait encore, un coq chanta. 
${}^{61}Le Seigneur, se retournant, posa son regard sur Pierre. Alors Pierre se souvint de la parole que le Seigneur lui avait dite : « Avant que le coq chante aujourd’hui, tu m’auras renié trois fois. » 
${}^{62}Il sortit et, dehors, pleura amèrement.
${}^{63}Les hommes qui gardaient Jésus se moquaient de lui et le rouaient de coups. 
${}^{64}Ils lui avaient voilé le visage, et ils l’interrogeaient : « Fais le prophète ! Qui est-ce qui t’a frappé ? » 
${}^{65}Et ils proféraient contre lui beaucoup d’autres blasphèmes.
${}^{66}Lorsqu’il fit jour, se réunit le collège des anciens du peuple, grands prêtres et scribes, et on emmena Jésus devant leur conseil suprême. 
${}^{67}Ils lui dirent : « Si tu es le Christ, dis-le nous. » Il leur répondit : « Si je vous le dis, vous ne me croirez pas ; 
${}^{68}et si j’interroge, vous ne répondrez pas. 
${}^{69}Mais désormais le Fils de l’homme sera assis à la droite de la Puissance de Dieu. » 
${}^{70}Tous lui dirent alors : « Tu es donc le Fils de Dieu ? » Il leur répondit : « Vous dites vous-mêmes que je le suis. » 
${}^{71}Ils dirent alors : « Pourquoi nous faut-il encore un témoignage ? Nous-mêmes, nous l’avons entendu de sa bouche. »
      
         
      \bchapter{}
      \begin{verse}
${}^{1}L’assemblée tout entière se leva, et on l’emmena chez Pilate.
      \begin{verse}
${}^{2}On se mit alors à l’accuser : « Nous avons trouvé cet homme en train de semer le trouble dans notre nation : il empêche de payer l’impôt à l’empereur, et il dit qu’il est le Christ, le Roi. » 
${}^{3}Pilate l’interrogea : « Es-tu le roi des Juifs ? » Jésus répondit : « C’est toi-même qui le dis. » 
${}^{4}Pilate s’adressa aux grands prêtres et aux foules : « Je ne trouve chez cet homme aucun motif de condamnation. » 
${}^{5}Mais ils insistaient avec force : « Il soulève le peuple en enseignant dans toute la Judée ; après avoir commencé en Galilée, il est venu jusqu’ici. » 
${}^{6}À ces mots, Pilate demanda si l’homme était Galiléen. 
${}^{7}Apprenant qu’il relevait de l’autorité d’Hérode, il le renvoya devant ce dernier, qui se trouvait lui aussi à Jérusalem en ces jours-là.
${}^{8}À la vue de Jésus, Hérode éprouva une joie extrême : en effet, depuis longtemps il désirait le voir à cause de ce qu’il entendait dire de lui, et il espérait lui voir faire un miracle. 
${}^{9}Il lui posa bon nombre de questions, mais Jésus ne lui répondit rien. 
${}^{10}Les grands prêtres et les scribes étaient là, et ils l’accusaient avec véhémence. 
${}^{11}Hérode, ainsi que ses soldats, le traita avec mépris et se moqua de lui : il le revêtit d’un manteau de couleur éclatante et le renvoya à Pilate. 
${}^{12}Ce jour-là, Hérode et Pilate devinrent des amis, alors qu’auparavant il y avait de l’hostilité entre eux.
${}^{13}Alors Pilate convoqua les grands prêtres, les chefs et le peuple. 
${}^{14}Il leur dit : « Vous m’avez amené cet homme en l’accusant d’introduire la subversion dans le peuple. Or, j’ai moi-même instruit l’affaire devant vous et, parmi les faits dont vous l’accusez, je n’ai trouvé chez cet homme aucun motif de condamnation. 
${}^{15}D’ailleurs, Hérode non plus, puisqu’il nous l’a renvoyé. En somme, cet homme n’a rien fait qui mérite la mort. 
${}^{16}Je vais donc le relâcher après lui avoir fait donner une correction. »
${}^{18}Ils se mirent à crier tous ensemble : « Mort à cet homme ! Relâche-nous Barabbas. » 
${}^{19}Ce Barabbas avait été jeté en prison pour une émeute survenue dans la ville, et pour meurtre. 
${}^{20}Pilate, dans son désir de relâcher Jésus, leur adressa de nouveau la parole. 
${}^{21}Mais ils vociféraient : « Crucifie-le ! Crucifie-le ! » 
${}^{22}Pour la troisième fois, il leur dit : « Quel mal a donc fait cet homme ? Je n’ai trouvé en lui aucun motif de condamnation à mort. Je vais donc le relâcher après lui avoir fait donner une correction. » 
${}^{23}Mais ils insistaient à grands cris, réclamant qu’il soit crucifié ; et leurs cris s’amplifiaient.
${}^{24}Alors Pilate décida de satisfaire leur requête. 
${}^{25}Il relâcha celui qu’ils réclamaient, le prisonnier condamné pour émeute et pour meurtre, et il livra Jésus à leur bon plaisir.
${}^{26}Comme ils l’emmenaient, ils prirent un certain Simon de Cyrène, qui revenait des champs, et ils le chargèrent de la croix pour qu’il la porte derrière Jésus.
${}^{27}Le peuple, en grande foule, le suivait, ainsi que des femmes qui se frappaient la poitrine et se lamentaient sur Jésus. 
${}^{28}Il se retourna et leur dit : « Filles de Jérusalem, ne pleurez pas sur moi ! Pleurez plutôt sur vous-mêmes et sur vos enfants ! 
${}^{29}Voici venir des jours où l’on dira : “Heureuses les femmes stériles, celles qui n’ont pas enfanté, celles qui n’ont pas allaité !” 
${}^{30}Alors on dira aux montagnes : “Tombez sur nous”, et aux collines : “Cachez-nous.” 
${}^{31}Car si l’on traite ainsi l’arbre vert, que deviendra l’arbre sec ? »
${}^{32}Ils emmenaient aussi avec Jésus deux autres, des malfaiteurs, pour les exécuter.
${}^{33}Lorsqu’ils furent arrivés au lieu dit : Le Crâne (ou Calvaire), là ils crucifièrent Jésus, avec les deux malfaiteurs, l’un à droite et l’autre à gauche. 
${}^{34}Jésus disait : « Père, pardonne-leur : ils ne savent pas ce qu’ils font. » Puis, ils partagèrent ses vêtements et les tirèrent au sort.
${}^{35}Le peuple restait là à observer. Les chefs tournaient Jésus en dérision et disaient : « Il en a sauvé d’autres : qu’il se sauve lui-même, s’il est le Messie de Dieu, l’Élu ! » 
${}^{36}Les soldats aussi se moquaient de lui ; s’approchant, ils lui présentaient de la boisson vinaigrée, 
${}^{37}en disant : « Si tu es le roi des Juifs, sauve-toi toi-même ! » 
${}^{38}Il y avait aussi une inscription au-dessus de lui :
      « Celui-ci est le roi des Juifs. »
${}^{39}L’un des malfaiteurs suspendus en croix l’injuriait : « N’es-tu pas le Christ ? Sauve-toi toi-même, et nous aussi ! » 
${}^{40}Mais l’autre lui fit de vifs reproches : « Tu ne crains donc pas Dieu ! Tu es pourtant un condamné, toi aussi ! 
${}^{41}Et puis, pour nous, c’est juste : après ce que nous avons fait, nous avons ce que nous méritons. Mais lui, il n’a rien fait de mal. » 
${}^{42}Et il disait : « Jésus, souviens-toi de moi quand tu viendras dans ton Royaume. » 
${}^{43}Jésus lui déclara : « Amen, je te le dis : aujourd’hui, avec moi, tu seras dans le Paradis. »
${}^{44}C’était déjà environ la sixième heure (c’est-à-dire : midi) ; l’obscurité se fit sur toute la terre jusqu’à la neuvième heure, 
${}^{45}car le soleil s’était caché. Le rideau du Sanctuaire se déchira par le milieu. 
${}^{46}Alors, Jésus poussa un grand cri : « Père, entre tes mains je remets mon esprit. » Et après avoir dit cela, il expira.
${}^{47}À la vue de ce qui s’était passé, le centurion rendit gloire à Dieu : « Celui-ci était réellement un homme juste. » 
${}^{48}Et toute la foule des gens qui s’étaient rassemblés pour ce spectacle, observant ce qui se passait, s’en retournaient en se frappant la poitrine.
${}^{49}Tous ses amis, ainsi que les femmes qui le suivaient depuis la Galilée, se tenaient plus loin pour regarder.
${}^{50}Alors arriva un membre du Conseil, nommé Joseph ; c’était un homme bon et juste, 
${}^{51}qui n’avait donné son accord ni à leur délibération, ni à leurs actes. Il était d’Arimathie, ville de Judée, et il attendait le règne de Dieu. 
${}^{52}Il alla trouver Pilate et demanda le corps de Jésus. 
${}^{53}Puis il le descendit de la croix, l’enveloppa dans un linceul et le mit dans un tombeau taillé dans le roc, où personne encore n’avait été déposé. 
${}^{54}C’était le jour de la Préparation de la fête, et déjà brillaient les lumières du sabbat.
${}^{55}Les femmes qui avaient accompagné Jésus depuis la Galilée suivirent Joseph. Elles regardèrent le tombeau pour voir comment le corps avait été placé. 
${}^{56}Puis elles s’en retournèrent et préparèrent aromates et parfums. Et, durant le sabbat, elles observèrent le repos prescrit.
      <h2 class="intertitle" id="d85e358797">3. Après la Résurrection (24)</h2>
      
         
      \bchapter{}
      \begin{verse}
${}^{1}Le premier jour de la semaine, à la pointe de l’aurore, les femmes se rendirent au tombeau, portant les aromates qu’elles avaient préparés. 
${}^{2}Elles trouvèrent la pierre roulée sur le côté du tombeau. 
${}^{3}Elles entrèrent, mais ne trouvèrent pas le corps du Seigneur Jésus. 
${}^{4}Alors qu’elles étaient désemparées, voici que deux hommes se tinrent devant elles en habit éblouissant. 
${}^{5}Saisies de crainte, elles gardaient leur visage incliné vers le sol. Ils leur dirent : « Pourquoi cherchez-vous le Vivant parmi les morts ? 
${}^{6}Il n’est pas ici, il est ressuscité. Rappelez-vous ce qu’il vous a dit quand il était encore en Galilée : 
${}^{7}“Il faut que le Fils de l’homme soit livré aux mains des pécheurs, qu’il soit crucifié et que, le troisième jour, il ressuscite.” » 
${}^{8}Alors elles se rappelèrent les paroles qu’il avait dites.
${}^{9}Revenues du tombeau, elles rapportèrent tout cela aux Onze et à tous les autres. 
${}^{10}C’étaient Marie Madeleine, Jeanne, et Marie mère de Jacques ; les autres femmes qui les accompagnaient disaient la même chose aux Apôtres. 
${}^{11}Mais ces propos leur semblèrent délirants, et ils ne les croyaient pas. 
${}^{12}Alors Pierre se leva et courut au tombeau ; mais en se penchant, il vit les linges, et eux seuls. Il s’en retourna chez lui, tout étonné de ce qui était arrivé.
${}^{13}Le même jour, deux disciples faisaient route vers un village appelé Emmaüs, à deux heures de marche de Jérusalem, 
${}^{14}et ils parlaient entre eux de tout ce qui s’était passé. 
${}^{15}Or, tandis qu’ils s’entretenaient et s’interrogeaient, Jésus lui-même s’approcha, et il marchait avec eux. 
${}^{16}Mais leurs yeux étaient empêchés de le reconnaître. 
${}^{17}Jésus leur dit : « De quoi discutez-vous en marchant ? » Alors, ils s’arrêtèrent, tout tristes. 
${}^{18}L’un des deux, nommé Cléophas, lui répondit : « Tu es bien le seul étranger résidant à Jérusalem qui ignore les événements de ces jours-ci. » 
${}^{19}Il leur dit : « Quels événements ? » Ils lui répondirent : « Ce qui est arrivé à Jésus de Nazareth, cet homme qui était un prophète puissant par ses actes et ses paroles devant Dieu et devant tout le peuple : 
${}^{20}comment les grands prêtres et nos chefs l’ont livré, ils l’ont fait condamner à mort et ils l’ont crucifié. 
${}^{21}Nous, nous espérions que c’était lui qui allait délivrer Israël. Mais avec tout cela, voici déjà le troisième jour qui passe depuis que c’est arrivé. 
${}^{22}À vrai dire, des femmes de notre groupe nous ont remplis de stupeur. Quand, dès l’aurore, elles sont allées au tombeau, 
${}^{23}elles n’ont pas trouvé son corps ; elles sont venues nous dire qu’elles avaient même eu une vision : des anges, qui disaient qu’il est vivant. 
${}^{24}Quelques-uns de nos compagnons sont allés au tombeau, et ils ont trouvé les choses comme les femmes l’avaient dit ; mais lui, ils ne l’ont pas vu. » 
${}^{25}Il leur dit alors : « Esprits sans intelligence ! Comme votre cœur est lent à croire tout ce que les prophètes ont dit ! 
${}^{26}Ne fallait-il pas que le Christ souffrît cela pour entrer dans sa gloire ? » 
${}^{27}Et, partant de Moïse et de tous les Prophètes, il leur interpréta, dans toute l’Écriture, ce qui le concernait.
${}^{28}Quand ils approchèrent du village où ils se rendaient, Jésus fit semblant d’aller plus loin. 
${}^{29}Mais ils s’efforcèrent de le retenir : « Reste avec nous, car le soir approche et déjà le jour baisse. » Il entra donc pour rester avec eux. 
${}^{30}Quand il fut à table avec eux, ayant pris le pain, il prononça la bénédiction et, l’ayant rompu, il le leur donna. 
${}^{31}Alors leurs yeux s’ouvrirent, et ils le reconnurent, mais il disparut à leurs regards. 
${}^{32}Ils se dirent l’un à l’autre : « Notre cœur n’était-il pas brûlant en nous, tandis qu’il nous parlait sur la route et nous ouvrait les Écritures ? »
${}^{33}À l’instant même, ils se levèrent et retournèrent à Jérusalem. Ils y trouvèrent réunis les onze Apôtres et leurs compagnons, qui leur dirent : 
${}^{34}« Le Seigneur est réellement ressuscité : il est apparu à Simon-Pierre. » 
${}^{35}À leur tour, ils racontaient ce qui s’était passé sur la route, et comment le Seigneur s’était fait reconnaître par eux à la fraction du pain.
${}^{36}Comme ils en parlaient encore, lui-même fut présent au milieu d’eux, et leur dit : « La paix soit avec vous ! » 
${}^{37}Saisis de frayeur et de crainte, ils croyaient voir un esprit. 
${}^{38}Jésus leur dit : « Pourquoi êtes-vous bouleversés ? Et pourquoi ces pensées qui surgissent dans votre cœur ? 
${}^{39}Voyez mes mains et mes pieds : c’est bien moi ! Touchez-moi, regardez : un esprit n’a pas de chair ni d’os comme vous constatez que j’en ai. » 
${}^{40}Après cette parole, il leur montra ses mains et ses pieds. 
${}^{41}Dans leur joie, ils n’osaient pas encore y croire, et restaient saisis d’étonnement. Jésus leur dit : « Avez-vous ici quelque chose à manger ? » 
${}^{42}Ils lui présentèrent une part de poisson grillé 
${}^{43}qu’il prit et mangea devant eux.
${}^{44}Puis il leur déclara : « Voici les paroles que je vous ai dites quand j’étais encore avec vous : Il faut que s’accomplisse tout ce qui a été écrit à mon sujet dans la loi de Moïse, les Prophètes et les Psaumes. » 
${}^{45}Alors il ouvrit leur intelligence à la compréhension des Écritures. 
${}^{46}Il leur dit : « Ainsi est-il écrit que le Christ souffrirait, qu’il ressusciterait d’entre les morts le troisième jour, 
${}^{47}et que la conversion serait proclamée en son nom, pour le pardon des péchés, à toutes les nations, en commençant par Jérusalem. 
${}^{48}À vous d’en être les témoins. 
${}^{49}Et moi, je vais envoyer sur vous ce que mon Père a promis. Quant à vous, demeurez dans la ville jusqu’à ce que vous soyez revêtus d’une puissance venue d’en haut. »
${}^{50}Puis Jésus les emmena au dehors, jusque vers Béthanie ; et, levant les mains, il les bénit. 
${}^{51}Or, tandis qu’il les bénissait, il se sépara d’eux et il était emporté au ciel.
${}^{52}Ils se prosternèrent devant lui, puis ils retournèrent à Jérusalem, en grande joie. 
${}^{53}Et ils étaient sans cesse dans le Temple à bénir Dieu.
