  
  
${}^{51}Comme s’accomplissait le temps où il allait être enlevé au ciel, Jésus, le visage déterminé, prit la route de Jérusalem.
${}^{52}Il envoya, en avant de lui, des messagers ; ceux-ci se mirent en route et entrèrent dans un village de Samaritains pour préparer sa venue. 
${}^{53}Mais on refusa de le recevoir, parce qu’il se dirigeait vers Jérusalem. 
${}^{54}Voyant cela, les disciples Jacques et Jean dirent : « Seigneur, veux-tu que nous ordonnions qu’un feu tombe du ciel et les détruise ? » 
${}^{55}Mais Jésus, se retournant, les réprimanda. 
${}^{56}Puis ils partirent pour un autre village.
${}^{57}En cours de route, un homme dit à Jésus : « Je te suivrai partout où tu iras. » 
${}^{58}Jésus lui déclara : « Les renards ont des terriers, les oiseaux du ciel ont des nids ; mais le Fils de l’homme n’a pas d’endroit où reposer la tête. »
${}^{59}Il dit à un autre : « Suis-moi. » L’homme répondit : « Seigneur, permets-moi d’aller d’abord enterrer mon père. » 
${}^{60}Mais Jésus répliqua : « Laisse les morts enterrer leurs morts. Toi, pars, et annonce le règne de Dieu. »
${}^{61}Un autre encore lui dit : « Je te suivrai, Seigneur ; mais laisse-moi d’abord faire mes adieux aux gens de ma maison. » 
${}^{62}Jésus lui répondit : « Quiconque met la main à la charrue, puis regarde en arrière, n’est pas fait pour le royaume de Dieu. »
      
         
      \bchapter{}
      \begin{verse}
${}^{1}Après cela, parmi les disciples le Seigneur en désigna encore soixante-douze, et il les envoya deux par deux, en avant de lui, en toute ville et localité où lui-même allait se rendre. 
${}^{2}Il leur dit : « La moisson est abondante, mais les ouvriers sont peu nombreux. Priez donc le maître de la moisson d’envoyer des ouvriers pour sa moisson. 
${}^{3}Allez ! Voici que je vous envoie comme des agneaux au milieu des loups. 
${}^{4}Ne portez ni bourse, ni sac, ni sandales, et ne saluez personne en chemin. 
${}^{5}Mais dans toute maison où vous entrerez, dites d’abord : “Paix à cette maison.” 
${}^{6}S’il y a là un ami de la paix, votre paix ira reposer sur lui ; sinon, elle reviendra sur vous. 
${}^{7}Restez dans cette maison, mangeant et buvant ce que l’on vous sert ; car l’ouvrier mérite son salaire. Ne passez pas de maison en maison. 
${}^{8}Dans toute ville où vous entrerez et où vous serez accueillis, mangez ce qui vous est présenté. 
${}^{9}Guérissez les malades qui s’y trouvent et dites-leur : “Le règne de Dieu s’est approché de vous.” 
${}^{10}Mais dans toute ville où vous entrerez et où vous ne serez pas accueillis, allez sur les places et dites : 
${}^{11}“Même la poussière de votre ville, collée à nos pieds, nous l’enlevons pour vous la laisser. Toutefois, sachez-le : le règne de Dieu s’est approché.” 
${}^{12}Je vous le déclare : au dernier jour, Sodome sera mieux traitée que cette ville.
${}^{13}Malheureuse es-tu, Corazine ! Malheureuse es-tu, Bethsaïde ! Car, si les miracles qui ont eu lieu chez vous avaient eu lieu à Tyr et à Sidon, il y a longtemps que leurs habitants auraient fait pénitence, avec le sac et la cendre. 
${}^{14}D’ailleurs, Tyr et Sidon seront mieux traitées que vous lors du Jugement. 
${}^{15}Et toi, Capharnaüm, seras-tu élevée jusqu’au ciel ? Non ! Jusqu’au séjour des morts tu descendras !
${}^{16}Celui qui vous écoute m’écoute ; celui qui vous rejette me rejette ; et celui qui me rejette rejette celui qui m’a envoyé. »
${}^{17}Les soixante-douze disciples revinrent tout joyeux, en disant : « Seigneur, même les démons nous sont soumis en ton nom. »
${}^{18}Jésus leur dit : « Je regardais Satan tomber du ciel comme l’éclair. 
${}^{19}Voici que je vous ai donné le pouvoir d’écraser serpents et scorpions, et sur toute la puissance de l’Ennemi : absolument rien ne pourra vous nuire. 
${}^{20}Toutefois, ne vous réjouissez pas parce que les esprits vous sont soumis ; mais réjouissez-vous parce que vos noms se trouvent inscrits dans les cieux. »
${}^{21}À l’heure même, Jésus exulta de joie sous l’action de l’Esprit Saint, et il dit : « Père, Seigneur du ciel et de la terre, je proclame ta louange : ce que tu as caché aux sages et aux savants, tu l’as révélé aux tout-petits. Oui, Père, tu l’as voulu ainsi dans ta bienveillance. 
${}^{22}Tout m’a été remis par mon Père. Personne ne connaît qui est le Fils, sinon le Père ; et personne ne connaît qui est le Père, sinon le Fils et celui à qui le Fils veut le révéler. »
${}^{23}Puis il se tourna vers ses disciples et leur dit en particulier : « Heureux les yeux qui voient ce que vous voyez ! 
${}^{24}Car, je vous le déclare : beaucoup de prophètes et de rois ont voulu voir ce que vous-mêmes voyez, et ne l’ont pas vu, entendre ce que vous entendez, et ne l’ont pas entendu. »
${}^{25}Et voici qu’un docteur de la Loi se leva et mit Jésus à l’épreuve en disant : « Maître, que dois-je faire pour avoir en héritage la vie éternelle ? » 
${}^{26}Jésus lui demanda : « Dans la Loi, qu’y a-t-il d’écrit ? Et comment lis-tu ? » 
${}^{27}L’autre répondit : « Tu aimeras le Seigneur ton Dieu de tout ton cœur, de toute ton âme, de toute ta force et de toute ton intelligence, et ton prochain comme toi-même. »
${}^{28}Jésus lui dit : « Tu as répondu correctement. Fais ainsi et tu vivras. »
${}^{29}Mais lui, voulant se justifier, dit à Jésus : « Et qui est mon prochain ? »
${}^{30}Jésus reprit la parole : « Un homme descendait de Jérusalem à Jéricho, et il tomba sur des bandits ; ceux-ci, après l’avoir dépouillé et roué de coups, s’en allèrent, le laissant à moitié mort. 
${}^{31}Par hasard, un prêtre descendait par ce chemin ; il le vit et passa de l’autre côté. 
${}^{32}De même un lévite arriva à cet endroit ; il le vit et passa de l’autre côté. 
${}^{33}Mais un Samaritain, qui était en route, arriva près de lui ; il le vit et fut saisi de compassion. 
${}^{34}Il s’approcha, et pansa ses blessures en y versant de l’huile et du vin ; puis il le chargea sur sa propre monture, le conduisit dans une auberge et prit soin de lui. 
${}^{35}Le lendemain, il sortit deux pièces d’argent, et les donna à l’aubergiste, en lui disant : “Prends soin de lui ; tout ce que tu auras dépensé en plus, je te le rendrai quand je repasserai.” 
${}^{36}Lequel des trois, à ton avis, a été le prochain de l’homme tombé aux mains des bandits ? » 
${}^{37}Le docteur de la Loi répondit : « Celui qui a fait preuve de pitié envers lui. » Jésus lui dit : « Va, et toi aussi, fais de même. »
${}^{38}Chemin faisant, Jésus entra dans un village. Une femme nommée Marthe le reçut. 
${}^{39}Elle avait une sœur appelée Marie qui, s’étant assise aux pieds du Seigneur, écoutait sa parole. 
${}^{40}Quant à Marthe, elle était accaparée par les multiples occupations du service. Elle intervint et dit : « Seigneur, cela ne te fait rien que ma sœur m’ait laissé faire seule le service ? Dis-lui donc de m’aider. » 
${}^{41}Le Seigneur lui répondit : « Marthe, Marthe, tu te donnes du souci et tu t’agites pour bien des choses. 
${}^{42}Une seule est nécessaire. Marie a choisi la meilleure part, elle ne lui sera pas enlevée. »
      
         
      \bchapter{}
      \begin{verse}
${}^{1}Il arriva que Jésus, en un certain lieu, était en prière.
      \begin{verse}Quand il eut terminé, un de ses disciples lui demanda :
      « Seigneur, apprends-nous à prier, comme Jean le Baptiste, lui aussi, l’a appris à ses disciples. » 
${}^{2}Il leur répondit : « Quand vous priez, dites :
        \\Père,
        \\que ton nom soit sanctifié,
        \\que ton règne vienne.
        ${}^{3}Donne-nous le pain
        \\dont nous avons besoin pour chaque jour
        ${}^{4}Pardonne-nous nos péchés,
        \\car nous-mêmes, nous pardonnons aussi
        \\à tous ceux qui ont des torts envers nous.
        \\Et ne nous laisse pas entrer en tentation. »
${}^{5}Jésus leur dit encore : « Imaginez que l’un de vous ait un ami et aille le trouver au milieu de la nuit pour lui demander : “Mon ami, prête-moi trois pains, 
${}^{6}car un de mes amis est arrivé de voyage chez moi, et je n’ai rien à lui offrir.” 
${}^{7}Et si, de l’intérieur, l’autre lui répond : “Ne viens pas m’importuner ! La porte est déjà fermée ; mes enfants et moi, nous sommes couchés. Je ne puis pas me lever pour te donner quelque chose.” 
${}^{8}Eh bien ! je vous le dis : même s’il ne se lève pas pour donner par amitié, il se lèvera à cause du sans-gêne de cet ami, et il lui donnera tout ce qu’il lui faut. 
${}^{9}Moi, je vous dis : Demandez, on vous donnera ; cherchez, vous trouverez ; frappez, on vous ouvrira. 
${}^{10}En effet, quiconque demande reçoit ; qui cherche trouve ; à qui frappe, on ouvrira. 
${}^{11}Quel père parmi vous, quand son fils lui demande un poisson, lui donnera un serpent au lieu du poisson ? 
${}^{12}ou lui donnera un scorpion quand il demande un œuf ? 
${}^{13}Si donc vous, qui êtes mauvais, vous savez donner de bonnes choses à vos enfants, combien plus le Père du ciel donnera-t-il l’Esprit Saint à ceux qui le lui demandent ! »
${}^{14}Jésus expulsait un démon qui rendait un homme muet. Lorsque le démon fut sorti, le muet se mit à parler, et les foules furent dans l’admiration. 
${}^{15}Mais certains d’entre eux dirent : « C’est par Béelzéboul, le chef des démons, qu’il expulse les démons. » 
${}^{16}D’autres, pour le mettre à l’épreuve, cherchaient à obtenir de lui un signe venant du ciel. 
${}^{17}Jésus, connaissant leurs pensées, leur dit : « Tout royaume divisé contre lui-même devient désert, ses maisons s’écroulent les unes sur les autres. 
${}^{18}Si Satan, lui aussi, est divisé contre lui-même, comment son royaume tiendra-t-il ? Vous dites en effet que c’est par Béelzéboul que j’expulse les démons. 
${}^{19}Mais si c’est par Béelzéboul que moi, je les expulse, vos disciples, par qui les expulsent-ils ? Dès lors, ils seront eux-mêmes vos juges. 
${}^{20}En revanche, si c’est par le doigt de Dieu que j’expulse les démons, c’est donc que le règne de Dieu est venu jusqu’à vous. 
${}^{21}Quand l’homme fort, et bien armé, garde son palais, tout ce qui lui appartient est en sécurité. 
${}^{22}Mais si un plus fort survient et triomphe de lui, il lui enlève son armement auquel il se fiait, et il distribue tout ce dont il l’a dépouillé. 
${}^{23}Celui qui n’est pas avec moi est contre moi ; celui qui ne rassemble pas avec moi disperse. 
${}^{24}Quand l’esprit impur est sorti de l’homme, il parcourt des lieux arides en cherchant où se reposer. Et il ne trouve pas. Alors il se dit : “Je vais retourner dans ma maison, d’où je suis sorti.” 
${}^{25}En arrivant, il la trouve balayée et bien rangée. 
${}^{26}Alors il s’en va, et il prend d’autres esprits encore plus mauvais que lui, au nombre de sept ; ils entrent et s’y installent. Ainsi, l’état de cet homme-là est pire à la fin qu’au début. »
${}^{27}Comme Jésus disait cela, une femme éleva la voix au milieu de la foule pour lui dire : « Heureuse la mère qui t’a porté en elle, et dont les seins t’ont nourri ! » 
${}^{28}Alors Jésus lui déclara : « Heureux plutôt ceux qui écoutent la parole de Dieu, et qui la gardent ! »
${}^{29}Comme les foules s’amassaient, Jésus se mit à dire : « Cette génération est une génération mauvaise : elle cherche un signe, mais en fait de signe il ne lui sera donné que le signe de Jonas. 
${}^{30}Car Jonas a été un signe pour les habitants de Ninive ; il en sera de même avec le Fils de l’homme pour cette génération. 
${}^{31}Lors du Jugement, la reine de Saba se dressera en même temps que les hommes de cette génération, et elle les condamnera. En effet, elle est venue des extrémités de la terre pour écouter la sagesse de Salomon, et il y a ici bien plus que Salomon. 
${}^{32}Lors du Jugement, les habitants de Ninive se lèveront en même temps que cette génération, et ils la condamneront ; en effet, ils se sont convertis en réponse à la proclamation faite par Jonas, et il y a ici bien plus que Jonas.
${}^{33}Personne, après avoir allumé une lampe, ne la met dans une cachette ou bien sous le boisseau : on la met sur le lampadaire pour que ceux qui entrent voient la lumière.
${}^{34}La lampe de ton corps, c’est ton œil. Quand ton œil est limpide, ton corps tout entier est aussi dans la lumière ; mais quand ton œil est mauvais, ton corps aussi est dans les ténèbres. 
${}^{35}Examine donc si la lumière qui est en toi n’est pas ténèbres ; 
${}^{36}si ton corps tout entier est dans la lumière sans aucune part de ténèbres, alors il sera dans la lumière tout entier, comme lorsque la lampe t’illumine de son éclat. »
${}^{37}Pendant que Jésus parlait, un pharisien l’invita pour le repas de midi. Jésus entra chez lui et prit place. 
${}^{38}Le pharisien fut étonné en voyant qu’il n’avait pas fait d’abord les ablutions précédant le repas. 
${}^{39}Le Seigneur lui dit : « Bien sûr, vous les pharisiens, vous purifiez l’extérieur de la coupe et du plat, mais à l’intérieur de vous-mêmes vous êtes remplis de cupidité et de méchanceté. 
${}^{40}Insensés ! Celui qui a fait l’extérieur n’a-t-il pas fait aussi l’intérieur ? 
${}^{41}Donnez plutôt en aumône ce que vous avez, et alors tout sera pur pour vous. 
${}^{42}Quel malheur pour vous, pharisiens, parce que vous payez la dîme sur toutes les plantes du jardin, comme la menthe et la rue et vous passez à côté du jugement et de l’amour de Dieu. Ceci, il fallait l’observer, sans abandonner cela. 
${}^{43}Quel malheur pour vous, pharisiens, parce que vous aimez le premier siège dans les synagogues, et les salutations sur les places publiques. 
${}^{44}Quel malheur pour vous, parce que vous êtes comme ces tombeaux qu’on ne voit pas et sur lesquels on marche sans le savoir. »
${}^{45}Alors un docteur de la Loi prit la parole et lui dit : « Maître, en parlant ainsi, c’est nous aussi que tu insultes. » 
${}^{46}Jésus reprit : « Vous aussi, les docteurs de la Loi, malheureux êtes-vous, parce que vous chargez les gens de fardeaux impossibles à porter, et vous-mêmes, vous ne touchez même pas ces fardeaux d’un seul doigt.
${}^{47}Quel malheur pour vous, parce que vous bâtissez les tombeaux des prophètes, alors que vos pères les ont tués. 
${}^{48}Ainsi vous témoignez que vous approuvez les actes de vos pères, puisque eux-mêmes ont tué les prophètes, et vous, vous bâtissez leurs tombeaux. 
${}^{49}C’est pourquoi la Sagesse de Dieu elle-même a dit : Je leur enverrai des prophètes et des apôtres ; parmi eux, ils en tueront et en persécuteront. 
${}^{50}Ainsi cette génération devra rendre compte du sang de tous les prophètes qui a été versé depuis la fondation du monde, 
${}^{51}depuis le sang d’Abel jusqu’au sang de Zacharie, qui a péri entre l’autel et le sanctuaire. Oui, je vous le déclare : on en demandera compte à cette génération.
${}^{52}Quel malheur pour vous, docteurs de la Loi, parce que vous avez enlevé la clé de la connaissance ; vous-mêmes n’êtes pas entrés, et ceux qui voulaient entrer, vous les en avez empêchés. »
${}^{53}Quand Jésus fut sorti de la maison, les scribes et les pharisiens commencèrent à s’acharner contre lui et à le harceler de questions ; 
${}^{54}ils lui tendaient des pièges pour traquer la moindre de ses paroles.
      
         
      \bchapter{}
      \begin{verse}
${}^{1}Comme la foule s’était rassemblée par milliers au point qu’on s’écrasait, Jésus, s’adressant d’abord à ses disciples, se mit à dire : « Méfiez-vous du levain des pharisiens, c’est-à-dire de leur hypocrisie.
${}^{2}Tout ce qui est couvert d’un voile sera dévoilé, tout ce qui est caché sera connu. 
${}^{3}Aussi tout ce que vous aurez dit dans les ténèbres sera entendu en pleine lumière, ce que vous aurez dit à l’oreille dans le fond de la maison sera proclamé sur les toits.
${}^{4}Je vous le dis, à vous mes amis : Ne craignez pas ceux qui tuent le corps, et après cela ne peuvent rien faire de plus. 
${}^{5}Je vais vous montrer qui vous devez craindre : craignez celui qui, après avoir tué, a le pouvoir d’envoyer dans la géhenne. Oui, je vous le dis : c’est celui-là que vous devez craindre. 
${}^{6}Est-ce que l’on ne vend pas cinq moineaux pour deux sous ? Or pas un seul n’est oublié au regard de Dieu. 
${}^{7}À plus forte raison les cheveux de votre tête sont tous comptés. Soyez sans crainte : vous valez plus qu’une multitude de moineaux.
${}^{8}Je vous le dis : Quiconque se sera déclaré pour moi devant les hommes, le Fils de l’homme aussi se déclarera pour lui devant les anges de Dieu. 
${}^{9}Mais celui qui m’aura renié en face des hommes sera renié à son tour en face des anges de Dieu.
${}^{10}Quiconque dira une parole contre le Fils de l’homme, cela lui sera pardonné ; mais si quelqu’un blasphème contre l’Esprit Saint, cela ne lui sera pas pardonné.
${}^{11}Quand on vous traduira devant les gens des synagogues, les magistrats et les autorités, ne vous inquiétez pas de la façon dont vous vous défendrez ni de ce que vous direz. 
${}^{12}Car l’Esprit Saint vous enseignera à cette heure-là ce qu’il faudra dire. »
${}^{13}Du milieu de la foule, quelqu’un demanda à Jésus : « Maître, dis à mon frère de partager avec moi notre héritage. »
${}^{14}Jésus lui répondit : « Homme, qui donc m’a établi pour être votre juge ou l’arbitre de vos partages ? » 
${}^{15}Puis, s’adressant à tous : « Gardez-vous bien de toute avidité, car la vie de quelqu’un, même dans l’abondance, ne dépend pas de ce qu’il possède. »
${}^{16}Et il leur dit cette parabole : « Il y avait un homme riche, dont le domaine avait bien rapporté. 
${}^{17}Il se demandait : “Que vais-je faire ? Car je n’ai pas de place pour mettre ma récolte.” 
${}^{18}Puis il se dit : “Voici ce que je vais faire : je vais démolir mes greniers, j’en construirai de plus grands et j’y mettrai tout mon blé et tous mes biens. 
${}^{19}Alors je me dirai à moi-même : Te voilà donc avec de nombreux biens à ta disposition, pour de nombreuses années. Repose-toi, mange, bois, jouis de l’existence.” 
${}^{20}Mais Dieu lui dit : “Tu es fou : cette nuit même, on va te redemander ta vie. Et ce que tu auras accumulé, qui l’aura ?” 
${}^{21}Voilà ce qui arrive à celui qui amasse pour lui-même, au lieu d’être riche en vue de Dieu. »
${}^{22}Puis il dit à ses disciples : « C’est pourquoi, je vous dis : À propos de votre vie, ne vous souciez pas de ce que vous mangerez, ni, à propos de votre corps, de quoi vous allez le vêtir. 
${}^{23}En effet, la vie vaut plus que la nourriture, et le corps plus que le vêtement. 
${}^{24}Observez les corbeaux : ils ne font ni semailles ni moisson, ils n’ont ni réserves ni greniers, et Dieu les nourrit. Vous valez tellement plus que les oiseaux ! 
${}^{25}D’ailleurs qui d’entre vous, en se faisant du souci, peut ajouter une coudée à la longueur de sa vie ? 
${}^{26}Si donc vous n’êtes pas capables de la moindre chose, pourquoi vous faire du souci pour le reste ? 
${}^{27}Observez les lis : comment poussent-ils ? Ils ne filent pas, ils ne tissent pas. Or je vous le dis : Salomon lui-même, dans toute sa gloire, n’était pas habillé comme l’un d’entre eux. 
${}^{28}Si Dieu revêt ainsi l’herbe qui aujourd’hui est dans le champ et demain sera jetée dans le feu, il fera tellement plus pour vous, hommes de peu de foi ! 
${}^{29}Ne cherchez donc pas ce que vous allez manger et boire ; ne soyez pas anxieux. 
${}^{30}Tout cela, les nations du monde le recherchent, mais votre Père sait que vous en avez besoin. 
${}^{31}Cherchez plutôt son Royaume, et cela vous sera donné par surcroît. 
${}^{32}Sois sans crainte, petit troupeau : votre Père a trouvé bon de vous donner le Royaume.
${}^{33}Vendez ce que vous possédez et donnez-le en aumône. Faites-vous des bourses qui ne s’usent pas, un trésor inépuisable dans les cieux, là où le voleur n’approche pas, où la mite ne détruit pas. 
${}^{34}Car là où est votre trésor, là aussi sera votre cœur.
${}^{35}Restez en tenue de service, votre ceinture autour des reins, et vos lampes allumées. 
${}^{36}Soyez comme des gens qui attendent leur maître à son retour des noces, pour lui ouvrir dès qu’il arrivera et frappera à la porte. 
${}^{37}Heureux ces serviteurs-là que le maître, à son arrivée, trouvera en train de veiller. Amen, je vous le dis : c’est lui qui, la ceinture autour des reins, les fera prendre place à table et passera pour les servir. 
${}^{38}S’il revient vers minuit ou vers trois heures du matin et qu’il les trouve ainsi, heureux sont-ils ! 
${}^{39}Vous le savez bien : si le maître de maison avait su à quelle heure le voleur viendrait, il n’aurait pas laissé percer le mur de sa maison. 
${}^{40}Vous aussi, tenez-vous prêts : c’est à l’heure où vous n’y penserez pas que le Fils de l’homme viendra. »
${}^{41}Pierre dit alors : « Seigneur, est-ce pour nous que tu dis cette parabole, ou bien pour tous ? » 
${}^{42}Le Seigneur répondit : « Que dire de l’intendant fidèle et sensé à qui le maître confiera la charge de son personnel pour distribuer, en temps voulu, la ration de nourriture ? 
${}^{43}Heureux ce serviteur que son maître, en arrivant, trouvera en train d’agir ainsi ! 
${}^{44}Vraiment, je vous le déclare : il l’établira sur tous ses biens. 
${}^{45}Mais si le serviteur se dit en lui-même : “Mon maître tarde à venir”, et s’il se met à frapper les serviteurs et les servantes, à manger, à boire et à s’enivrer, 
${}^{46}alors quand le maître viendra, le jour où son serviteur ne s’y attend pas et à l’heure qu’il ne connaît pas, il l’écartera et lui fera partager le sort des infidèles.
${}^{47}Le serviteur qui, connaissant la volonté de son maître, n’a rien préparé et n’a pas accompli cette volonté, recevra un grand nombre de coups. 
${}^{48}Mais celui qui ne la connaissait pas, et qui a mérité des coups pour sa conduite, celui-là n’en recevra qu’un petit nombre. À qui l’on a beaucoup donné, on demandera beaucoup ; à qui l’on a beaucoup confié, on réclamera davantage.
${}^{49}Je suis venu apporter un feu sur la terre, et comme je voudrais qu’il soit déjà allumé ! 
${}^{50}Je dois recevoir un baptême, et quelle angoisse est la mienne jusqu’à ce qu’il soit accompli ! 
${}^{51}Pensez-vous que je sois venu mettre la paix sur la terre ? Non, je vous le dis, mais bien plutôt la division. 
${}^{52}Car désormais cinq personnes de la même famille seront divisées : trois contre deux et deux contre trois ; 
${}^{53}ils se diviseront : le père contre le fils et le fils contre le père, la mère contre la fille et la fille contre la mère, la belle-mère contre la belle-fille et la belle-fille contre la belle-mère. »
${}^{54}S’adressant aussi aux foules, Jésus disait : « Quand vous voyez un nuage monter au couchant, vous dites aussitôt qu’il va pleuvoir, et c’est ce qui arrive. 
${}^{55}Et quand vous voyez souffler le vent du sud, vous dites qu’il fera une chaleur torride, et cela arrive. 
${}^{56}Hypocrites ! Vous savez interpréter l’aspect de la terre et du ciel ; mais ce moment-ci, pourquoi ne savez-vous pas l’interpréter ?
${}^{57}Et pourquoi aussi ne jugez-vous pas par vous-mêmes ce qui est juste ? 
${}^{58}Ainsi, quand tu vas avec ton adversaire devant le magistrat, pendant que tu es en chemin mets tout en œuvre pour t’arranger avec lui, afin d’éviter qu’il ne te traîne devant le juge, que le juge ne te livre à l’huissier, et que l’huissier ne te jette en prison. 
${}^{59}Je te le dis : tu n’en sortiras pas avant d’avoir payé jusqu’au dernier centime. »
      
         
      \bchapter{}
      \begin{verse}
${}^{1}À ce moment, des gens qui se trouvaient là rapportèrent à Jésus l’affaire des Galiléens que Pilate avait fait massacrer, mêlant leur sang à celui des sacrifices qu’ils offraient. 
${}^{2}Jésus leur répondit : « Pensez-vous que ces Galiléens étaient de plus grands pécheurs que tous les autres Galiléens, pour avoir subi un tel sort ? 
${}^{3}Eh bien, je vous dis : pas du tout ! Mais si vous ne vous convertissez pas, vous périrez tous de même. 
${}^{4}Et ces dix-huit personnes tuées par la chute de la tour de Siloé, pensez-vous qu’elles étaient plus coupables que tous les autres habitants de Jérusalem ? 
${}^{5}Eh bien, je vous dis : pas du tout ! Mais si vous ne vous convertissez pas, vous périrez tous de même. »
${}^{6}Jésus disait encore cette parabole : « Quelqu’un avait un figuier planté dans sa vigne. Il vint chercher du fruit sur ce figuier, et n’en trouva pas. 
${}^{7}Il dit alors à son vigneron : “Voilà trois ans que je viens chercher du fruit sur ce figuier, et je n’en trouve pas. Coupe-le. À quoi bon le laisser épuiser le sol ?” 
${}^{8}Mais le vigneron lui répondit : “Maître, laisse-le encore cette année, le temps que je bêche autour pour y mettre du fumier. 
${}^{9}Peut-être donnera-t-il du fruit à l’avenir. Sinon, tu le couperas.” »
${}^{10}Jésus était en train d’enseigner dans une synagogue, le jour du sabbat. 
${}^{11}Voici qu’il y avait là une femme, possédée par un esprit qui la rendait infirme depuis dix-huit ans ; elle était toute courbée et absolument incapable de se redresser. 
${}^{12}Quand Jésus la vit, il l’interpella et lui dit : « Femme, te voici délivrée de ton infirmité. » 
${}^{13}Et il lui imposa les mains. À l’instant même elle redevint droite et rendait gloire à Dieu. 
${}^{14}Alors le chef de la synagogue, indigné de voir Jésus faire une guérison le jour du sabbat, prit la parole et dit à la foule : « Il y a six jours pour travailler ; venez donc vous faire guérir ces jours-là, et non pas le jour du sabbat. » 
${}^{15}Le Seigneur lui répliqua : « Hypocrites ! Chacun de vous, le jour du sabbat, ne détache-t-il pas de la mangeoire son bœuf ou son âne pour le mener boire ? 
${}^{16}Alors cette femme, une fille d’Abraham, que Satan avait liée voici dix-huit ans, ne fallait-il pas la délivrer de ce lien le jour du sabbat ? » 
${}^{17}À ces paroles de Jésus, tous ses adversaires furent remplis de honte, et toute la foule était dans la joie à cause de toutes les actions éclatantes qu’il faisait.
${}^{18}Jésus disait donc : « À quoi le règne de Dieu est-il comparable, à quoi vais-je le comparer ? 
${}^{19}Il est comparable à une graine de moutarde qu’un homme a prise et jetée dans son jardin. Elle a poussé, elle est devenue un arbre, et les oiseaux du ciel ont fait leur nid dans ses branches. »
${}^{20}Il dit encore : « À quoi pourrai-je comparer le règne de Dieu ? 
${}^{21}Il est comparable au levain qu’une femme a pris et enfoui dans trois mesures de farine, jusqu’à ce que toute la pâte ait levé. »
${}^{22}Tandis qu’il faisait route vers Jérusalem, Jésus traversait villes et villages en enseignant.
${}^{23}Quelqu’un lui demanda : « Seigneur, n’y a-t-il que peu de gens qui soient sauvés ? » Jésus leur dit : 
${}^{24}« Efforcez-vous d’entrer par la porte étroite, car, je vous le déclare, beaucoup chercheront à entrer et n’y parviendront pas. 
${}^{25}Lorsque le maître de maison se sera levé pour fermer la porte, si vous, du dehors, vous vous mettez à frapper à la porte, en disant : “Seigneur, ouvre-nous”, il vous répondra : “Je ne sais pas d’où vous êtes.” 
${}^{26}Alors vous vous mettrez à dire : “Nous avons mangé et bu en ta présence, et tu as enseigné sur nos places.” 
${}^{27}Il vous répondra : “Je ne sais pas d’où vous êtes. Éloignez-vous de moi, vous tous qui commettez l’injustice.” 
${}^{28}Là, il y aura des pleurs et des grincements de dents, quand vous verrez Abraham, Isaac et Jacob, et tous les prophètes dans le royaume de Dieu, et que vous-mêmes, vous serez jetés dehors. 
${}^{29}Alors on viendra de l’orient et de l’occident, du nord et du midi, prendre place au festin dans le royaume de Dieu. 
${}^{30}Oui, il y a des derniers qui seront premiers, et des premiers qui seront derniers. »
${}^{31}À ce moment-là, quelques pharisiens s’approchèrent de Jésus pour lui dire : « Pars, va-t’en d’ici : Hérode veut te tuer. »
${}^{32}Il leur répliqua : « Allez dire à ce renard : voici que j’expulse les démons et je fais des guérisons aujourd’hui et demain, et, le troisième jour, j’arrive au terme. 
${}^{33}Mais il me faut continuer ma route aujourd’hui, demain et le jour suivant, car il ne convient pas qu’un prophète périsse en dehors de Jérusalem.
${}^{34}Jérusalem, Jérusalem, toi qui tues les prophètes et qui lapides ceux qui te sont envoyés, combien de fois ai-je voulu rassembler tes enfants comme la poule rassemble ses poussins sous ses ailes, et vous n’avez pas voulu ! 
${}^{35}Voici que votre temple est abandonné à vous-mêmes. Je vous le déclare : vous ne me verrez plus jusqu’à ce que vienne le jour où vous direz :
      Béni soit celui qui vient au nom du Seigneur ! »
      
         
      \bchapter{}
      \begin{verse}
${}^{1}Un jour de sabbat, Jésus était entré dans la maison d’un chef des pharisiens pour y prendre son repas, et ces derniers l’observaient.
${}^{2}Or voici qu’il y avait devant lui un homme atteint d’hydropisie. 
${}^{3}Prenant la parole, Jésus s’adressa aux docteurs de la Loi et aux pharisiens pour leur demander : « Est-il permis, oui ou non, de faire une guérison le jour du sabbat ? » 
${}^{4}Ils gardèrent le silence. Tenant alors le malade, Jésus le guérit et le laissa aller. 
${}^{5}Puis il leur dit : « Si l’un de vous a un fils ou un bœuf qui tombe dans un puits, ne va-t-il pas aussitôt l’en retirer, même le jour du sabbat ? » 
${}^{6}Et ils furent incapables de trouver une réponse.
${}^{7}Jésus dit une parabole aux invités lorsqu’il remarqua comment ils choisissaient les premières places, et il leur dit : 
${}^{8}« Quand quelqu’un t’invite à des noces, ne va pas t’installer à la première place, de peur qu’il ait invité un autre plus considéré que toi. 
${}^{9}Alors, celui qui vous a invités, toi et lui, viendra te dire : “Cède-lui ta place” ; et, à ce moment, tu iras, plein de honte, prendre la dernière place. 
${}^{10}Au contraire, quand tu es invité, va te mettre à la dernière place. Alors, quand viendra celui qui t’a invité, il te dira : “Mon ami, avance plus haut”, et ce sera pour toi un honneur aux yeux de tous ceux qui seront à la table avec toi. 
${}^{11}En effet, quiconque s’élève sera abaissé ; et qui s’abaisse sera élevé. »
${}^{12}Jésus disait aussi à celui qui l’avait invité : « Quand tu donnes un déjeuner ou un dîner, n’invite pas tes amis, ni tes frères, ni tes parents, ni de riches voisins ; sinon, eux aussi te rendraient l’invitation et ce serait pour toi un don en retour. 
${}^{13}Au contraire, quand tu donnes une réception, invite des pauvres, des estropiés, des boiteux, des aveugles ; 
${}^{14}heureux seras-tu, parce qu’ils n’ont rien à te donner en retour : cela te sera rendu à la résurrection des justes. »
${}^{15}En entendant parler Jésus, un des convives lui dit : « Heureux celui qui participera au repas dans le royaume de Dieu ! »
${}^{16}Jésus lui dit : « Un homme donnait un grand dîner, et il avait invité beaucoup de monde. 
${}^{17}À l’heure du dîner, il envoya son serviteur dire aux invités : “Venez, tout est prêt.” 
${}^{18}Mais ils se mirent tous, unanimement, à s’excuser. Le premier lui dit : “J’ai acheté un champ, et je suis obligé d’aller le voir ; je t’en prie, excuse-moi.” 
${}^{19}Un autre dit : “J’ai acheté cinq paires de bœufs, et je pars les essayer ; je t’en prie, excuse-moi.” 
${}^{20}Un troisième dit : “Je viens de me marier, et c’est pourquoi je ne peux pas venir.” 
${}^{21}De retour, le serviteur rapporta ces paroles à son maître. Alors, pris de colère, le maître de maison dit à son serviteur : “Dépêche-toi d’aller sur les places et dans les rues de la ville ; les pauvres, les estropiés, les aveugles et les boiteux, amène-les ici.” 
${}^{22}Le serviteur revint lui dire : “Maître, ce que tu as ordonné est exécuté, et il reste encore de la place.” 
${}^{23}Le maître dit alors au serviteur : “Va sur les routes et dans les sentiers, et fais entrer les gens de force, afin que ma maison soit remplie. 
${}^{24}Car, je vous le dis, aucun de ces hommes qui avaient été invités ne goûtera de mon dîner.” »
${}^{25}De grandes foules faisaient route avec Jésus ; il se retourna et leur dit : 
${}^{26}« Si quelqu’un vient à moi sans me préférer à son père, sa mère, sa femme, ses enfants, ses frères et sœurs, et même à sa propre vie, il ne peut pas être mon disciple. 
${}^{27}Celui qui ne porte pas sa croix pour marcher à ma suite ne peut pas être mon disciple.
${}^{28}Quel est celui d’entre vous qui, voulant bâtir une tour, ne commence par s’asseoir pour calculer la dépense et voir s’il a de quoi aller jusqu’au bout ? 
${}^{29}Car, si jamais il pose les fondations et n’est pas capable d’achever, tous ceux qui le verront vont se moquer de lui : 
${}^{30}“Voilà un homme qui a commencé à bâtir et n’a pas été capable d’achever !” 
${}^{31}Et quel est le roi qui, partant en guerre contre un autre roi, ne commence par s’asseoir pour voir s’il peut, avec dix mille hommes, affronter l’autre qui marche contre lui avec vingt mille ? 
${}^{32}S’il ne le peut pas, il envoie, pendant que l’autre est encore loin, une délégation pour demander les conditions de paix. 
${}^{33}Ainsi donc, celui d’entre vous qui ne renonce pas à tout ce qui lui appartient ne peut pas être mon disciple.
${}^{34}C’est une bonne chose que le sel ; mais si le sel lui-même se dénature, avec quoi lui rendra-t-on sa saveur ? 
${}^{35}Il ne peut servir ni pour la terre, ni pour le fumier : on le jette dehors ! Celui qui a des oreilles pour entendre, qu’il entende ! »
      
         
      \bchapter{}
      \begin{verse}
${}^{1}Les publicains et les pécheurs venaient tous à Jésus pour l’écouter. 
${}^{2}Les pharisiens et les scribes récriminaient contre lui : « Cet homme fait bon accueil aux pécheurs, et il mange avec eux ! »
${}^{3}Alors Jésus leur dit cette parabole : 
${}^{4}« Si l’un de vous a cent brebis et qu’il en perd une, n’abandonne-t-il pas les quatre-vingt-dix-neuf autres dans le désert pour aller chercher celle qui est perdue, jusqu’à ce qu’il la retrouve ? 
${}^{5}Quand il l’a retrouvée, il la prend sur ses épaules, tout joyeux, 
${}^{6}et, de retour chez lui, il rassemble ses amis et ses voisins pour leur dire : “Réjouissez-vous avec moi, car j’ai retrouvé ma brebis, celle qui était perdue !” 
${}^{7}Je vous le dis : C’est ainsi qu’il y aura de la joie dans le ciel pour un seul pécheur qui se convertit, plus que pour quatre-vingt-dix-neuf justes qui n’ont pas besoin de conversion.
${}^{8}Ou encore, si une femme a dix pièces d’argent et qu’elle en perd une, ne va-t-elle pas allumer une lampe, balayer la maison, et chercher avec soin jusqu’à ce qu’elle la retrouve ? 
${}^{9}Quand elle l’a retrouvée, elle rassemble ses amies et ses voisines pour leur dire : “Réjouissez-vous avec moi, car j’ai retrouvé la pièce d’argent que j’avais perdue !” 
${}^{10}Ainsi je vous le dis : Il y a de la joie devant les anges de Dieu pour un seul pécheur qui se convertit. »
${}^{11}Jésus dit encore : « Un homme avait deux fils. 
${}^{12}Le plus jeune dit à son père : “Père, donne-moi la part de fortune qui me revient.” Et le père leur partagea ses biens. 
${}^{13}Peu de jours après, le plus jeune rassembla tout ce qu’il avait, et partit pour un pays lointain où il dilapida sa fortune en menant une vie de désordre. 
${}^{14}Il avait tout dépensé, quand une grande famine survint dans ce pays, et il commença à se trouver dans le besoin. 
${}^{15}Il alla s’engager auprès d’un habitant de ce pays, qui l’envoya dans ses champs garder les porcs. 
${}^{16}Il aurait bien voulu se remplir le ventre avec les gousses que mangeaient les porcs, mais personne ne lui donnait rien. 
${}^{17}Alors il rentra en lui-même et se dit : “Combien d’ouvriers de mon père ont du pain en abondance, et moi, ici, je meurs de faim ! 
${}^{18}Je me lèverai, j’irai vers mon père, et je lui dirai : Père, j’ai péché contre le ciel et envers toi. 
${}^{19}Je ne suis plus digne d’être appelé ton fils. Traite-moi comme l’un de tes ouvriers.” 
${}^{20}Il se leva et s’en alla vers son père. Comme il était encore loin, son père l’aperçut et fut saisi de compassion ; il courut se jeter à son cou et le couvrit de baisers. 
${}^{21}Le fils lui dit : “Père, j’ai péché contre le ciel et envers toi. Je ne suis plus digne d’être appelé ton fils.” 
${}^{22}Mais le père dit à ses serviteurs : “Vite, apportez le plus beau vêtement pour l’habiller, mettez-lui une bague au doigt et des sandales aux pieds, 
${}^{23}allez chercher le veau gras, tuez-le, mangeons et festoyons, 
${}^{24}car mon fils que voilà était mort, et il est revenu à la vie ; il était perdu, et il est retrouvé.” Et ils commencèrent à festoyer. 
${}^{25}Or le fils aîné était aux champs. Quand il revint et fut près de la maison, il entendit la musique et les danses. 
${}^{26}Appelant un des serviteurs, il s’informa de ce qui se passait. 
${}^{27}Celui-ci répondit : “Ton frère est arrivé, et ton père a tué le veau gras, parce qu’il a retrouvé ton frère en bonne santé.” 
${}^{28}Alors le fils aîné se mit en colère, et il refusait d’entrer. Son père sortit le supplier. 
${}^{29}Mais il répliqua à son père : “Il y a tant d’années que je suis à ton service sans avoir jamais transgressé tes ordres, et jamais tu ne m’as donné un chevreau pour festoyer avec mes amis. 
${}^{30}Mais, quand ton fils que voilà est revenu après avoir dévoré ton bien avec des prostituées, tu as fait tuer pour lui le veau gras !” 
${}^{31}Le père répondit : “Toi, mon enfant, tu es toujours avec moi, et tout ce qui est à moi est à toi. 
${}^{32}Il fallait festoyer et se réjouir ; car ton frère que voilà était mort, et il est revenu à la vie ; il était perdu, et il est retrouvé !” »
      
         
      \bchapter{}
      \begin{verse}
${}^{1}Jésus disait encore aux disciples : « Un homme riche avait un gérant qui lui fut dénoncé comme dilapidant ses biens. 
${}^{2}Il le convoqua et lui dit : “Qu’est-ce que j’apprends à ton sujet ? Rends-moi les comptes de ta gestion, car tu ne peux plus être mon gérant.” 
${}^{3}Le gérant se dit en lui-même : “Que vais-je faire, puisque mon maître me retire la gestion ? Travailler la terre ? Je n’en ai pas la force. Mendier ? J’aurais honte. 
${}^{4}Je sais ce que je vais faire, pour qu’une fois renvoyé de ma gérance, des gens m’accueillent chez eux.” 
${}^{5}Il fit alors venir, un par un, ceux qui avaient des dettes envers son maître. Il demanda au premier : “Combien dois-tu à mon maître ?” 
${}^{6}Il répondit : “Cent barils d’huile.” Le gérant lui dit : “Voici ton reçu ; vite, assieds-toi et écris cinquante.” 
${}^{7}Puis il demanda à un autre : “Et toi, combien dois-tu ?” Il répondit : “Cent sacs de blé.” Le gérant lui dit : “Voici ton reçu, écris quatre-vingts.” 
${}^{8}Le maître fit l’éloge de ce gérant malhonnête car il avait agi avec habileté ; en effet, les fils de ce monde sont plus habiles entre eux que les fils de la lumière.
${}^{9}Eh bien moi, je vous le dis : Faites-vous des amis avec l’argent malhonnête, afin que, le jour où il ne sera plus là, ces amis vous accueillent dans les demeures éternelles. 
${}^{10}Celui qui est digne de confiance dans la moindre chose est digne de confiance aussi dans une grande. Celui qui est malhonnête dans la moindre chose est malhonnête aussi dans une grande. 
${}^{11}Si donc vous n’avez pas été dignes de confiance pour l’argent malhonnête, qui vous confiera le bien véritable ? 
${}^{12}Et si, pour ce qui est à autrui, vous n’avez pas été dignes de confiance, ce qui vous revient, qui vous le donnera ?
${}^{13}Aucun domestique ne peut servir deux maîtres : ou bien il haïra l’un et aimera l’autre, ou bien il s’attachera à l’un et méprisera l’autre. Vous ne pouvez pas servir à la fois Dieu et l’argent. »
${}^{14}Quand ils entendaient tout cela, les pharisiens, eux qui aimaient l’argent, tournaient Jésus en dérision. 
${}^{15}Il leur dit alors : « Vous, vous êtes de ceux qui se font passer pour justes aux yeux des gens, mais Dieu connaît vos cœurs ; en effet, ce qui est prestigieux pour les gens est une chose abominable aux yeux de Dieu.
${}^{16}La Loi et les Prophètes vont jusqu’à Jean le Baptiste ; depuis lors, le royaume de Dieu est annoncé, et chacun met toute sa force pour y entrer. 
${}^{17}Il est plus facile au ciel et à la terre de disparaître qu’à un seul petit trait de la Loi de tomber.
${}^{18}Tout homme qui renvoie sa femme et en épouse une autre commet un adultère ; et celui qui épouse une femme renvoyée par son mari commet un adultère.
${}^{19}« Il y avait un homme riche, vêtu de pourpre et de lin fin, qui faisait chaque jour des festins somptueux. 
${}^{20}Devant son portail gisait un pauvre nommé Lazare, qui était couvert d’ulcères. 
${}^{21}Il aurait bien voulu se rassasier de ce qui tombait de la table du riche ; mais les chiens, eux, venaient lécher ses ulcères. 
${}^{22}Or le pauvre mourut, et les anges l’emportèrent auprès d’Abraham. Le riche mourut aussi, et on l’enterra. 
${}^{23}Au séjour des morts, il était en proie à la torture ; levant les yeux, il vit Abraham de loin et Lazare tout près de lui. 
${}^{24}Alors il cria : “Père Abraham, prends pitié de moi et envoie Lazare tremper le bout de son doigt dans l’eau pour me rafraîchir la langue, car je souffre terriblement dans cette fournaise. 
${}^{25}– Mon enfant, répondit Abraham, rappelle-toi : tu as reçu le bonheur pendant ta vie, et Lazare, le malheur pendant la sienne. Maintenant, lui, il trouve ici la consolation, et toi, la souffrance. 
${}^{26}Et en plus de tout cela, un grand abîme a été établi entre vous et nous, pour que ceux qui voudraient passer vers vous ne le puissent pas, et que, de là-bas non plus, on ne traverse pas vers nous.” 
${}^{27}Le riche répliqua : “Eh bien ! père, je te prie d’envoyer Lazare dans la maison de mon père. 
${}^{28}En effet, j’ai cinq frères : qu’il leur porte son témoignage, de peur qu’eux aussi ne viennent dans ce lieu de torture !” 
${}^{29}Abraham lui dit : “Ils ont Moïse et les Prophètes : qu’ils les écoutent ! 
${}^{30}– Non, père Abraham, dit-il, mais si quelqu’un de chez les morts vient les trouver, ils se convertiront.” 
${}^{31}Abraham répondit : “S’ils n’écoutent pas Moïse ni les Prophètes, quelqu’un pourra bien ressusciter d’entre les morts : ils ne seront pas convaincus.” »
      
         
      \bchapter{}
      \begin{verse}
${}^{1}Jésus disait à ses disciples : « Il est inévitable que surviennent des scandales, des occasions de chute ; mais malheureux celui par qui cela arrive ! 
${}^{2}Il vaut mieux qu’on lui attache au cou une meule en pierre et qu’on le précipite à la mer, plutôt qu’il ne soit une occasion de chute pour un seul des petits que voilà. 
${}^{3}Prenez garde à vous-mêmes !
      Si ton frère a commis un péché, fais-lui de vifs reproches, et, s’il se repent, pardonne-lui. 
${}^{4}Même si sept fois par jour il commet un péché contre toi, et que sept fois de suite il revienne à toi en disant : “Je me repens”, tu lui pardonneras. »
${}^{5}Les Apôtres dirent au Seigneur : « Augmente en nous la foi ! » 
${}^{6}Le Seigneur répondit : « Si vous aviez de la foi, gros comme une graine de moutarde, vous auriez dit à l’arbre que voici : “Déracine-toi et va te planter dans la mer”, et il vous aurait obéi.
${}^{7}« Lequel d’entre vous, quand son serviteur aura labouré ou gardé les bêtes, lui dira à son retour des champs : “Viens vite prendre place à table” ? 
${}^{8}Ne lui dira-t-il pas plutôt : “Prépare-moi à dîner, mets-toi en tenue pour me servir, le temps que je mange et boive. Ensuite tu mangeras et boiras à ton tour” ? 
${}^{9}Va-t-il être reconnaissant envers ce serviteur d’avoir exécuté ses ordres ? 
${}^{10}De même vous aussi, quand vous aurez exécuté tout ce qui vous a été ordonné, dites : “Nous sommes de simples serviteurs : nous n’avons fait que notre devoir.” »
${}^{11}Jésus, marchant vers Jérusalem, traversait la région située entre la Samarie et la Galilée.
${}^{12}Comme il entrait dans un village, dix lépreux vinrent à sa rencontre. Ils s’arrêtèrent à distance 
${}^{13}et lui crièrent : « Jésus, maître, prends pitié de nous. » 
${}^{14}À cette vue, Jésus leur dit : « Allez vous montrer aux prêtres. » En cours de route, ils furent purifiés. 
${}^{15}L’un d’eux, voyant qu’il était guéri, revint sur ses pas, en glorifiant Dieu à pleine voix. 
${}^{16}Il se jeta face contre terre aux pieds de Jésus en lui rendant grâce. Or, c’était un Samaritain. 
${}^{17}Alors Jésus prit la parole en disant : « Tous les dix n’ont-ils pas été purifiés ? Les neuf autres, où sont-ils ? 
${}^{18}Il ne s’est trouvé parmi eux que cet étranger pour revenir sur ses pas et rendre gloire à Dieu ! » 
${}^{19}Jésus lui dit : « Relève-toi et va : ta foi t’a sauvé. »
${}^{20}Comme les pharisiens demandaient à Jésus quand viendrait le règne de Dieu, il prit la parole et dit : « La venue du règne de Dieu n’est pas observable. 
${}^{21}On ne dira pas : “Voilà, il est ici !” ou bien : “Il est là !” En effet, voici que le règne de Dieu est au milieu de vous. » 
${}^{22}Puis il dit aux disciples : « Des jours viendront où vous désirerez voir un seul des jours du Fils de l’homme, et vous ne le verrez pas. 
${}^{23}On vous dira : “Voilà, il est là-bas !” ou bien : “Voici, il est ici !” N’y allez pas, n’y courez pas. 
${}^{24}En effet, comme l’éclair qui jaillit illumine l’horizon d’un bout à l’autre, ainsi le Fils de l’homme, quand son jour sera là. 
${}^{25}Mais auparavant, il faut qu’il souffre beaucoup et qu’il soit rejeté par cette génération. 
${}^{26}Comme cela s’est passé dans les jours de Noé, ainsi en sera-t-il dans les jours du Fils de l’homme. 
${}^{27}On mangeait, on buvait, on prenait femme, on prenait mari, jusqu’au jour où Noé entra dans l’arche et où survint le déluge qui les fit tous périr. 
${}^{28}Il en était de même dans les jours de Loth : on mangeait, on buvait, on achetait, on vendait, on plantait, on bâtissait ; 
${}^{29}mais le jour où Loth sortit de Sodome, du ciel tomba une pluie de feu et de soufre qui les fit tous périr ; 
${}^{30}cela se passera de la même manière le jour où le Fils de l’homme se révélera. 
${}^{31}En ce jour-là, celui qui sera sur sa terrasse, et aura ses affaires dans sa maison, qu’il ne descende pas pour les emporter ; et de même celui qui sera dans son champ, qu’il ne retourne pas en arrière. 
${}^{32}Rappelez-vous la femme de Loth. 
${}^{33}Qui cherchera à conserver sa vie la perdra. Et qui la perdra la sauvegardera. 
${}^{34}Je vous le dis : Cette nuit-là, deux personnes seront dans le même lit : l’une sera prise, l’autre laissée. 
${}^{35}Deux femmes seront ensemble en train de moudre du grain : l’une sera prise, l’autre laissée. » 
${}^{37}Prenant alors la parole, les disciples lui demandèrent : « Où donc, Seigneur ? » Il leur répondit : « Là où sera le corps, là aussi se rassembleront les vautours. »
      
         
      \bchapter{}
      \begin{verse}
${}^{1}Jésus disait à ses disciples une parabole sur la nécessité pour eux de toujours prier sans se décourager : 
${}^{2}« Il y avait dans une ville un juge qui ne craignait pas Dieu et ne respectait pas les hommes. 
${}^{3}Dans cette même ville, il y avait une veuve qui venait lui demander : “Rends-moi justice contre mon adversaire.” 
${}^{4}Longtemps il refusa ; puis il se dit : “Même si je ne crains pas Dieu et ne respecte personne, 
${}^{5}comme cette veuve commence à m’ennuyer, je vais lui rendre justice pour qu’elle ne vienne plus sans cesse m’assommer.” » 
${}^{6}Le Seigneur ajouta : « Écoutez bien ce que dit ce juge dépourvu de justice ! 
${}^{7}Et Dieu ne ferait pas justice à ses élus, qui crient vers lui jour et nuit ? Les fait-il attendre ? 
${}^{8}Je vous le déclare : bien vite, il leur fera justice. Cependant, le Fils de l’homme, quand il viendra, trouvera-t-il la foi sur la terre ? »
      
         
${}^{9}À l’adresse de certains qui étaient convaincus d’être justes et qui méprisaient les autres, Jésus dit la parabole que voici : 
${}^{10}« Deux hommes montèrent au Temple pour prier. L’un était pharisien, et l’autre, publicain (c’est-à-dire un collecteur d’impôts). 
${}^{11}Le pharisien se tenait debout et priait en lui-même : “Mon Dieu, je te rends grâce parce que je ne suis pas comme les autres hommes – ils sont voleurs, injustes, adultères –, ou encore comme ce publicain. 
${}^{12}Je jeûne deux fois par semaine et je verse le dixième de tout ce que je gagne.” 
${}^{13}Le publicain, lui, se tenait à distance et n’osait même pas lever les yeux vers le ciel ; mais il se frappait la poitrine, en disant : “Mon Dieu, montre-toi favorable au pécheur que je suis !” 
${}^{14}Je vous le déclare : quand ce dernier redescendit dans sa maison, c’est lui qui était devenu un homme juste, plutôt que l’autre. Qui s’élève sera abaissé ; qui s’abaisse sera élevé. »
${}^{15}Des gens présentaient à Jésus même les nourrissons, afin qu’il pose la main sur eux. En voyant cela, les disciples les écartaient vivement. 
${}^{16}Mais Jésus les fit venir à lui en disant : « Laissez les enfants venir à moi, et ne les empêchez pas, car le royaume de Dieu est à ceux qui leur ressemblent. 
${}^{17}Amen, je vous le dis : celui qui n’accueille pas le royaume de Dieu à la manière d’un enfant n’y entrera pas. »
${}^{18}Un notable lui demanda : « Bon maître, que dois-je faire pour avoir la vie éternelle en héritage ? » 
${}^{19}Jésus lui dit : « Pourquoi dire que je suis bon ? Personne n’est bon, sinon Dieu seul. 
${}^{20}Tu connais les commandements :
        \\Ne commets pas d’adultère,
        \\ne commets pas de meurtre,
        \\ne commets pas de vol,
        \\ne porte pas de faux témoignage,
        \\honore ton père et ta mère. »
${}^{21}L’homme répondit : « Tout cela, je l’ai observé depuis ma jeunesse. » 
${}^{22}À ces mots Jésus lui dit : « Une seule chose te fait encore défaut : vends tout ce que tu as, distribue-le aux pauvres et tu auras un trésor dans les cieux. Puis viens, suis-moi. » 
${}^{23}Mais entendant ces paroles, l’homme devint profondément triste, car il était très riche.
${}^{24}Le voyant devenu si triste, Jésus dit : « Comme il est difficile à ceux qui possèdent des richesses de pénétrer dans le royaume de Dieu ! 
${}^{25}Car il est plus facile à un chameau de passer par un trou d’aiguille qu’à un riche d’entrer dans le royaume de Dieu. » 
${}^{26}Ceux qui l’entendaient lui demandèrent : « Mais alors, qui peut être sauvé ? » 
${}^{27}Jésus répondit : « Ce qui est impossible pour les hommes est possible pour Dieu. »
${}^{28}Alors Pierre lui dit : « Voici que nous-mêmes, après avoir quitté ce qui nous appartenait, nous t’avons suivi. » 
${}^{29}Jésus déclara : « Amen, je vous le dis : nul n’aura quitté, à cause du royaume de Dieu, une maison, une femme, des frères, des parents, des enfants, 
${}^{30}sans qu’il reçoive bien davantage en ce temps-ci et, dans le monde à venir, la vie éternelle. »
${}^{31}Prenant les Douze auprès de lui, il leur dit : « Voici que nous montons à Jérusalem, et que va s’accomplir tout ce qui a été écrit par les prophètes sur le Fils de l’homme. 
${}^{32}En effet, il sera livré aux nations païennes, accablé de moqueries, maltraité, couvert de crachats ; 
${}^{33}après l’avoir flagellé, on le tuera et, le troisième jour, il ressuscitera. » 
${}^{34}Eux ne comprirent rien à cela : c’était une parole dont le sens leur était caché, et ils ne saisissaient pas de quoi Jésus parlait.
${}^{35}Alors que Jésus approchait de Jéricho, un aveugle mendiait, assis au bord de la route. 
${}^{36}Entendant la foule passer devant lui, il s’informa de ce qu’il y avait. 
${}^{37}On lui apprit que c’était Jésus le Nazaréen qui passait. 
${}^{38}Il s’écria : « Jésus, fils de David, prends pitié de moi ! » 
${}^{39}Ceux qui marchaient en tête le rabrouaient pour le faire taire. Mais lui criait de plus belle : « Fils de David, prends pitié de moi ! » 
${}^{40}Jésus s’arrêta et il ordonna qu’on le lui amène. Quand il se fut approché, Jésus lui demanda : 
${}^{41}« Que veux-tu que je fasse pour toi ? » Il répondit : « Seigneur, que je retrouve la vue. » 
${}^{42}Et Jésus lui dit : « Retrouve la vue ! Ta foi t’a sauvé. » 
${}^{43}À l’instant même, il retrouva la vue, et il suivait Jésus en rendant gloire à Dieu. Et tout le peuple, voyant cela, adressa une louange à Dieu.
      
         
      \bchapter{}
      \begin{verse}
${}^{1}Entré dans la ville de Jéricho, Jésus la traversait. 
${}^{2}Or, il y avait un homme du nom de Zachée ; il était le chef des collecteurs d’impôts, et c’était quelqu’un de riche. 
${}^{3}Il cherchait à voir qui était Jésus, mais il ne le pouvait pas à cause de la foule, car il était de petite taille. 
${}^{4}Il courut donc en avant et grimpa sur un sycomore pour voir Jésus qui allait passer par là. 
${}^{5}Arrivé à cet endroit, Jésus leva les yeux et lui dit : « Zachée, descends vite : aujourd’hui il faut que j’aille demeurer dans ta maison. » 
${}^{6}Vite, il descendit et reçut Jésus avec joie. 
${}^{7}Voyant cela, tous récriminaient : « Il est allé loger chez un homme qui est un pécheur. » 
${}^{8}Zachée, debout, s’adressa au Seigneur : « Voici, Seigneur : je fais don aux pauvres de la moitié de mes biens, et si j’ai fait du tort à quelqu’un, je vais lui rendre quatre fois plus. » 
${}^{9}Alors Jésus dit à son sujet : « Aujourd’hui, le salut est arrivé pour cette maison, car lui aussi est un fils d’Abraham. 
${}^{10}En effet, le Fils de l’homme est venu chercher et sauver ce qui était perdu. »
      
         
${}^{11}Comme on l’écoutait, Jésus ajouta une parabole : il était près de Jérusalem et ses auditeurs pensaient que le royaume de Dieu allait se manifester à l’instant même. 
${}^{12}Voici donc ce qu’il dit : « Un homme de la noblesse partit dans un pays lointain pour se faire donner la royauté et revenir ensuite. 
${}^{13}Il appela dix de ses serviteurs, et remit à chacun une somme de la valeur d’une mine ; puis il leur dit : “Pendant mon voyage, faites de bonnes affaires.” 
${}^{14}Mais ses concitoyens le détestaient, et ils envoyèrent derrière lui une délégation chargée de dire : “Nous ne voulons pas que cet homme règne sur nous.” 
${}^{15}Quand il fut de retour après avoir reçu la royauté, il fit convoquer les serviteurs auxquels il avait remis l’argent, afin de savoir ce que leurs affaires avaient rapporté. 
${}^{16}Le premier se présenta et dit : “Seigneur, la somme que tu m’avais remise a été multipliée par dix.” 
${}^{17}Le roi lui déclara : “Très bien, bon serviteur ! Puisque tu as été fidèle en si peu de chose, reçois l’autorité sur dix villes.” 
${}^{18}Le second vint dire : “La somme que tu m’avais remise, Seigneur, a été multipliée par cinq.” 
${}^{19}À celui-là encore, le roi dit : “Toi, de même, sois à la tête de cinq villes.” 
${}^{20}Le dernier vint dire : “Seigneur, voici la somme que tu m’avais remise ; je l’ai gardée enveloppée dans un linge. 
${}^{21}En effet, j’avais peur de toi, car tu es un homme exigeant, tu retires ce que tu n’as pas mis en dépôt, tu moissonnes ce que tu n’as pas semé.” 
${}^{22}Le roi lui déclara : “Je vais te juger sur tes paroles, serviteur mauvais : tu savais que je suis un homme exigeant, que je retire ce que je n’ai pas mis en dépôt, que je moissonne ce que je n’ai pas semé ; 
${}^{23}alors pourquoi n’as-tu pas mis mon argent à la banque ? À mon arrivée, je l’aurais repris avec les intérêts.” 
${}^{24}Et le roi dit à ceux qui étaient là : “Retirez-lui cette somme et donnez-la à celui qui a dix fois plus.” 
${}^{25}On lui dit : “Seigneur, il a dix fois plus ! 
${}^{26}– Je vous le déclare : on donnera à celui qui a ; mais celui qui n’a rien se verra enlever même ce qu’il a. 
${}^{27}Quant à mes ennemis, ceux qui n’ont pas voulu que je règne sur eux, amenez-les ici et égorgez-les devant moi.” »
