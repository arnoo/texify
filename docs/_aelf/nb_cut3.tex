  
  
      
         
      \bchapter{}
      \begin{verse}
${}^{1}Les fils d’Israël repartirent : ils allèrent camper dans les steppes de Moab, au-delà du Jourdain, à la hauteur de Jéricho.
      
         
${}^{2}Balaq, fils de Cippor, vit tout ce qu’Israël avait fait subir aux Amorites. 
${}^{3}Moab eut grand peur d’un peuple aussi nombreux. Oui, Moab fut saisi d’effroi devant les fils d’Israël. 
${}^{4}Alors Moab dit aux anciens de Madiane : « Maintenant, cette multitude va tout brouter aux alentours, comme un bœuf broute l’herbe des champs ! »
      Balaq, fils de Cippor, régnait sur Moab en ce temps-là. 
${}^{5}Il envoya donc des messagers à Balaam, fils de Béor, qui était à Petor au bord du Fleuve, son pays d’origine, pour l’appeler ; il lui faisait dire : « Voici un peuple qui est sorti d’Égypte, le voici répandu dans tout le pays, il s’est installé en face de moi ! 
${}^{6}Viens donc, je t’en prie, et maudis-moi ce peuple car il est plus puissant que moi. Peut-être alors pourrai-je le battre et le chasser du pays, car, je le sais, celui que tu bénis est béni, et celui que tu maudis est maudit. »
${}^{7}Les anciens de Moab et les anciens de Madiane s’en allèrent donc, munis de cadeaux pour le devin. Ils arrivèrent chez Balaam à qui ils répétèrent les paroles de Balaq. 
${}^{8}Balaam leur dit : « Passez la nuit ici, et je vous rendrai réponse suivant ce que le Seigneur m’aura dit. » Les princes de Moab restèrent donc chez Balaam. 
${}^{9}Dieu vint auprès de Balaam et dit : « Qui sont ces hommes chez toi ? » 
${}^{10}Balaam répondit à Dieu : « Balaq, fils de Cippor, roi de Moab, m’a envoyé dire : 
${}^{11}Voici un peuple sorti d’Égypte, qui s’est répandu dans tout le pays. Viens donc et maudis-le pour moi ! Peut-être alors pourrai-je combattre contre lui et le chasser ! » 
${}^{12}Mais Dieu dit à Balaam : « Tu n’iras pas avec eux ! Tu ne maudiras pas ce peuple, car il est béni ! » 
${}^{13}Balaam se leva de bon matin et dit aux princes de Balaq : « Retournez dans votre pays, car le Seigneur a refusé de me laisser partir avec vous. » 
${}^{14}Les princes de Moab se levèrent, retournèrent chez Balaq et lui dirent : « Balaam n’a pas voulu venir avec nous ! »
${}^{15}Mais Balaq envoya encore d’autres princes, plus nombreux et plus importants que les premiers. 
${}^{16}Ils allèrent donc auprès de Balaam et lui dirent : « Ainsi parle Balaq, fils de Cippor : Ne refuse pas, je t’en prie, de venir chez moi. 
${}^{17}Je te comblerai de beaucoup d’honneurs, et tout ce que tu me diras, je le ferai. Viens donc et maudis-moi ce peuple ! » 
${}^{18}Balaam répondit aux serviteurs de Balaq : « Même si Balaq me donnait plein sa maison d’argent et d’or, je ne pourrais transgresser la parole du Seigneur mon Dieu en aucune chose, petite ou grande. 
${}^{19}Maintenant, je vous en prie, restez ici cette nuit, vous aussi, car je sais que le Seigneur va encore me parler. » 
${}^{20}Dieu vint auprès de Balaam pendant la nuit et lui dit : « Puisque ces hommes sont venus t’appeler, lève-toi, pars avec eux. Seulement, ce que je te dirai, c’est cela que tu feras. » 
${}^{21}Balaam se leva de bon matin, sella son ânesse et partit avec les princes de Moab.
${}^{22}Mais, comme il partait, la colère de Dieu s’enflamma et l’ange du Seigneur se posta sur le chemin en adversaire, tandis qu’il s’en allait, monté sur son ânesse et accompagné de ses deux serviteurs. 
${}^{23}L’ânesse vit l’ange du Seigneur posté sur le chemin, son épée dégainée à la main ; elle quitta le chemin et prit par les champs. Balaam frappa l’ânesse pour la ramener sur le chemin. 
${}^{24}Alors, l’ange du Seigneur se plaça dans un chemin creux qui passait dans les vignes, entre deux murets. 
${}^{25}L’ânesse vit l’ange du Seigneur et se serra contre le mur, serrant ainsi le pied de Balaam contre le mur, et Balaam se remit à la frapper.
${}^{26}L’ange du Seigneur les dépassa encore une fois et se plaça dans un passage étroit où il n’était possible de dévier ni à droite ni à gauche. 
${}^{27}L’ânesse vit l’ange du Seigneur et se coucha sous Balaam qui s’enflamma de colère et la frappa de sa cravache.
${}^{28}Alors le Seigneur ouvrit la bouche de l’ânesse qui dit à Balaam : « Que t’ai-je fait pour que tu me frappes par trois fois ? » 
${}^{29}Et Balaam dit à l’ânesse : « C’est que tu t’es moquée de moi ! Ah ! si j’avais à la main une épée, à l’instant je te tuerais ! » 
${}^{30}Et l’ânesse dit à Balaam : « Ne suis-je pas ton ânesse, celle que depuis toujours tu ne cesses de monter ? Ai-je l’habitude d’agir ainsi à ton égard ? » Et lui répondit : « Non ! »
${}^{31}Alors le Seigneur dessilla les yeux de Balaam qui vit l’ange du Seigneur posté sur le chemin, son épée dégainée à la main. Balaam s’inclina et se prosterna sur son front. 
${}^{32}L’ange du Seigneur lui dit : « Pourquoi as-tu frappé ton ânesse par trois fois ? Tu le vois, je suis venu moi-même en adversaire car, à mon gré, ce voyage était précipité. 
${}^{33}L’ânesse m’a vu, elle, et, par trois fois, elle s’est détournée de moi. Si elle ne s’était pas détournée de moi, c’est toi qu’à l’instant j’aurais tué. Mais elle, je l’aurais laissé vivre. » 
${}^{34}Balaam dit à l’ange du Seigneur : « J’ai péché. Je ne savais pas que tu étais là, posté devant moi sur le chemin. Mais, maintenant, si c’est mal à tes yeux, je vais m’en retourner. » 
${}^{35}Mais l’ange du Seigneur dit à Balaam : « Pars avec ces hommes ! Mais tu diras seulement ce que je te dirai de dire ! » Balaam partit donc avec les princes de Balaq.
${}^{36}Balaq apprit que Balaam arrivait et il sortit à sa rencontre. Il alla à Ar-Moab sur la frontière tracée par l’Arnon, à la limite de son territoire. 
${}^{37}Balaq dit à Balaam : « Ne t’avais-je pas envoyé assez de monde pour t’appeler ? Pourquoi n’es-tu pas venu chez moi ? Vraiment, ne serais-je pas capable de te combler d’honneurs ? » 
${}^{38}Et Balaam dit à Balaq : « Me voici arrivé chez toi. Mais maintenant, vais-je dire n’importe quoi ? La parole que Dieu mettra dans ma bouche, c’est elle que je proclamerai. »
${}^{39}Balaam accompagna Balaq et ils allèrent à Qiryath-Houçoth. 
${}^{40}Balaq offrit en sacrifice du gros et du petit bétail, et en remit des parts à Balaam et aux princes qui étaient avec lui. 
${}^{41}Le lendemain matin, Balaq emmena Balaam et le fit monter à Bamoth-Baal d’où il pouvait voir une partie du peuple.
      
         
      \bchapter{}
      \begin{verse}
${}^{1}Balaam dit à Balaq : « Construis-moi ici sept autels et prépare-moi ici sept taureaux et sept béliers ! » 
${}^{2}Balaq fit ce qu’avait dit Balaam ; puis Balaq et Balaam offrirent un taureau et un bélier sur chaque autel. 
${}^{3}Balaam dit à Balaq : « Tiens-toi debout près de ton holocauste. Moi, je m’en irai plus loin. Peut-être le Seigneur viendra-t-il à ma rencontre, et ce qu’il m’aura fait voir, je te le communiquerai. » Balaam s’en alla sur une hauteur. 
${}^{4}Dieu vint à la rencontre de Balaam qui lui dit : « J’ai préparé les sept autels et j’ai offert un taureau et un bélier sur chacun d’eux. » 
${}^{5}Alors le Seigneur mit une parole dans la bouche de Balaam, puis il dit : « Retourne vers Balaq. C’est ainsi que tu lui parleras. »
${}^{6}Balaam retourna donc vers Balaq qui se tenait debout près de son holocauste avec tous les princes de Moab.
${}^{7}Et il prononça ces paroles énigmatiques :
        \\« Balaq m’a fait venir d’Aram,
        le roi de Moab m’a fait venir des monts d’Orient :
        \\“Viens, maudis pour moi Jacob !
        Viens, menace Israël !”
${}^{8}Comment vais-je maudire
        celui que Dieu n’a pas maudit ?
        \\Comment vais-je menacer
        celui que le Seigneur n’a pas menacé ?
${}^{9}Quand, du sommet des rochers, je le regarde,
        quand, du haut des collines, je le contemple,
        \\je vois un peuple qui demeure à part
        et n’est pas recensé parmi les nations.
${}^{10}Qui a pu dénombrer la poussière de Jacob,
        qui a pu compter la multitude d’Israël ?
        \\Que moi-même je meure de la mort du juste,
        que la fin de ma vie soit pareille à la sienne ! »
${}^{11}Balaq dit à Balaam « Que m’as-tu fait là ? C’est pour maudire mes ennemis que je t’ai choisi, or voici que tu les couvres de bénédictions ! » 
${}^{12}Balaam répondit : « Ce que le Seigneur met dans ma bouche, c’est cela que je dois veiller à dire. » 
${}^{13}Alors Balaq reprit : « Viens donc avec moi en un autre lieu d’où tu verras le peuple, mais tu n’en verras qu’une partie, tu ne le verras pas tout entier. De là-bas, maudis-le pour moi ! » 
${}^{14}Il l’emmena vers le Champ des Guetteurs, au sommet du Pisga, et construisit sept autels. Il offrit un taureau et un bélier sur chacun d’eux. 
${}^{15}Balaam dit à Balaq : « Tiens-toi debout ici près de ton holocauste ; quant à moi, Dieu viendra à ma rencontre. » 
${}^{16}Alors le Seigneur vint à la rencontre de Balaam, il mit une parole dans sa bouche, puis il dit : « Retourne vers Balaq. C’est ainsi que tu lui parleras. » 
${}^{17}Il revint auprès de Balaq qui se tenait debout près de son holocauste avec les princes de Moab. Balaq lui demanda : « Qu’a dit le Seigneur ? »
${}^{18}Et Balaam prononça ces paroles énigmatiques :
        \\« Lève-toi, Balaq, écoute !
        \\Prête-moi l’oreille, fils de Cippor !
${}^{19}Dieu n’est pas homme pour mentir,
        un fils d’Adam pour se rétracter.
        \\Va-t-il dire et ne pas agir,
        prononcer une parole et ne pas l’exécuter ?
${}^{20}Voici que je prends le parti de bénir ;
        il a béni et je n’y reviendrai pas !
${}^{21}Il n’a pas aperçu d’action mauvaise en Jacob,
        il n’a pas vu d’oppression en Israël.
        \\Le Seigneur son Dieu est avec lui,
        chez lui retentit l’ovation royale.
${}^{22}Dieu l’a fait sortir d’Égypte :
        sa vigueur fut pour lui comme celle du buffle !
${}^{23}Pas de présage en Jacob,
        pas de divination en Israël :
        \\aussi, au temps voulu, sera dit à Jacob
        – à Israël – ce que Dieu accomplit.
${}^{24}Voici qu’un peuple se lèvera comme une lionne,
        comme un lion il se dressera.
        \\Il ne se couchera pas sans avoir dévoré sa proie,
        sans avoir bu le sang des victimes ! »
${}^{25}Alors Balaq dit à Balaam : « Si tu ne peux pas le maudire, au moins ne le bénis pas ! » 
${}^{26}Mais Balaam répondit à Balaq : « Ne te l’ai-je pas dit : “Tout ce que dira le Seigneur, je le ferai” ? »
${}^{27}Balaq dit à Balaam : « Viens donc ! Je vais te mener en un autre lieu. Peut-être plaira-t-il à Dieu que, de là-bas, tu le maudisses pour moi ! » 
${}^{28}Balaq emmena Balaam au sommet du Péor qui fait face à la steppe et la domine. 
${}^{29}Balaam dit à Balaq : « Construis-moi ici sept autels et prépare-moi ici sept taureaux et sept béliers. » 
${}^{30}Balaq fit comme avait dit Balaam. Il offrit un taureau et un bélier sur chaque autel.
      
         
      \bchapter{}
      \begin{verse}
${}^{1}Balaam vit qu’aux yeux du Seigneur c’était bien de bénir Israël et il n’alla pas, comme les autres fois, à la recherche de présages ; il tourna son visage vers le désert. 
${}^{2}Levant les yeux, il vit Israël qui campait, rangé par tribus. L’esprit de Dieu fut sur lui, 
${}^{3}et il prononça ces paroles énigmatiques :
        \\« Oracle de Balaam, fils de Béor,
        oracle de l’homme au regard pénétrant\\,
        ${}^{4}oracle de celui qui entend les paroles de Dieu.
        \\Il voit ce que le Puissant lui fait voir,
        il tombe en extase\\, et ses yeux s’ouvrent.
         
        ${}^{5}Que tes tentes sont belles, Jacob,
        et tes demeures, Israël !
        ${}^{6}Elles s’étendent comme des vallées,
        comme des jardins au bord d’un fleuve ;
        \\le Seigneur les a plantées comme des aloès,
        comme des cèdres au bord des eaux !
        ${}^{7}Un héros sortira de la descendance de Jacob\\,
        il dominera sur des peuples nombreux\\.
        \\Son règne\\sera plus grand que celui de Gog,
        sa royauté sera exaltée.
         
${}^{8}Dieu a fait sortir Israël d’Égypte :
        sa vigueur fut pour lui comme celle du buffle !
        \\Israël dévore les nations qui l’attaquent,
        il leur brise les os,
        il frappe de ses flèches.
${}^{9}Puis il s’accroupit, il se couche,
        comme un lion, comme une lionne.
        Qui le fera se relever ?
        \\Béni soit celui qui te bénira,
        maudit soit celui qui te maudira ! »
${}^{10}Alors la colère de Balaq s’enflamma contre Balaam ; il tapa des mains et dit à Balaam : « C’est pour maudire mes ennemis que je t’ai appelé ; or voici que tu les couvres de bénédictions, et cela par trois fois ! 
${}^{11}Maintenant, déguerpis et va-t’en chez toi ! J’avais dit que je te comblerais d’honneurs. Mais voilà : le Seigneur te refuse les honneurs ! » 
${}^{12}Balaam dit à Balaq : « N’avais-je pas dit aux messagers que tu m’as envoyés : 
${}^{13}“Même si Balaq me donnait plein sa maison d’argent et d’or, je ne pourrais transgresser la parole du Seigneur en amenant, de moi-même, bonheur ou malheur. Ce que le Seigneur dira, je le dirai.” 
${}^{14}Et maintenant je vais rejoindre mon peuple. Viens, que je t’avertisse du traitement que ce peuple infligera à ton peuple dans les temps à venir. »
        ${}^{15}Balaam prononça encore ces paroles énigmatiques :
        \\« Oracle de Balaam, fils de Béor,
        oracle de l’homme au regard pénétrant\\,
        ${}^{16}oracle de celui qui entend les paroles de Dieu,
        qui possède la science du Très-Haut.
        \\Il voit ce que le Puissant lui fait voir,
        il tombe en extase\\, et ses yeux s’ouvrent\\.
        ${}^{17}Ce héros\\, je le vois – mais pas pour maintenant –
        je l’aperçois – mais pas de près :
        \\Un astre se lève\\, issu de Jacob,
        un sceptre se dresse, issu d’Israël.
        \\Il brise les flancs de Moab,
        il décime tous les fils de Seth ;
${}^{18}il prendra possession d’Édom,
        possession de Séïr, son ennemi.
        \\Israël déploiera sa puissance,
${}^{19}et de Jacob surgira un dominateur
        qui fera périr tout survivant de la ville. »
       
${}^{20}Balaam vit ensuite Amalec et il prononça ces paroles énigmatiques :
        \\« La première des nations, Amalec !
        Mais sa fin, c’est sa ruine ! »
       
${}^{21}Puis il vit les Qénites et il prononça ces paroles énigmatiques :
        \\« Ta demeure est solide,
        ton nid, posé sur un rocher !
${}^{22}Mais Caïn sera la proie des flammes.
        Combien de temps Assour te tiendra-t-il captif ? »
       
${}^{23}Balaam prononça encore ces paroles énigmatiques :
        \\« Ah ! Qui donc pourra survivre
        quand Dieu en disposera,
${}^{24}quand des navires viendront de Kittim
        pour opprimer Assour et opprimer Éber ?
        \\Car lui aussi court à sa ruine. »
       
${}^{25}Balaam se leva et s’en alla. Il s’en retourna chez lui, tandis que Balaq, lui aussi, s’en allait par son propre chemin.
      
         
      \bchapter{}
      \begin{verse}
${}^{1}Israël alla habiter Shittim, et le peuple commença à se livrer à la prostitution avec les filles de Moab. 
${}^{2}Elles invitèrent le peuple aux sacrifices offerts à leurs dieux. Le peuple mangea et il se prosterna devant leurs dieux. 
${}^{3}Israël s’attacha à Baal-Péor, et la colère du Seigneur s’enflamma contre Israël.
${}^{4}Le Seigneur dit à Moïse : « Saisis tous les chefs du peuple et qu’on les écartèle pour le Seigneur, devant le soleil. Alors la colère ardente du Seigneur se détournera d’Israël ! » 
${}^{5}Moïse dit aux juges d’Israël : « Que chacun de vous tue, parmi ses hommes, ceux qui se sont attachés à Baal-Péor ! »
${}^{6}Et voici qu’un homme, un des fils d’Israël, arriva. Il amena à ses frères une Madianite, et cela sous les yeux de Moïse, sous les yeux de toute la communauté des fils d’Israël, alors qu’ils pleuraient à l’entrée de la tente de la Rencontre. 
${}^{7}Voyant cela, Pinhas, fils d’Éléazar, fils du prêtre Aaron, se leva du milieu de la communauté, saisit en main une lance 
${}^{8}et entra dans l’alcôve à la suite de l’homme d’Israël. Il les transperça tous deux, l’homme d’Israël, et la femme, au bas-ventre. Alors s’arrêta le fléau qui accablait les fils d’Israël. 
${}^{9}Vingt-quatre mille moururent de ce fléau.
${}^{10}Le Seigneur parla à Moïse. Il dit : 
${}^{11}« Pinhas, fils d’Éléazar, fils du prêtre Aaron, a détourné ma fureur des fils d’Israël, parce qu’il a été animé, au milieu d’eux, de la même ardeur jalouse que moi. Aussi, je n’ai pas exterminé les fils d’Israël dans mon ardeur jalouse. 
${}^{12}Parle donc ainsi : Voici que moi, je lui donne mon alliance de paix. 
${}^{13}Ce sera pour lui et pour sa descendance après lui une alliance qui lui assurera un sacerdoce perpétuel, parce qu’il a été animé d’une ardeur jalouse pour son Dieu et a accompli le rite d’expiation sur les fils d’Israël. »
${}^{14}L’homme d’Israël qui fut frappé – frappé avec la Madianite – se nommait Zimri, fils de Salou, responsable d’une famille de Siméonites ; 
${}^{15}et la femme madianite qui fut frappée se nommait Kozbi, fille de Sour. Ce dernier était chef de clans d’une tribu en Madiane.
${}^{16}Le Seigneur parla encore à Moïse. Il dit : 
${}^{17}« Attaquez les Madianites et battez-les. 
${}^{18}Car ils vous ont attaqués avec leurs intrigues, les intrigues dont ils ont usé contre vous dans l’affaire de Péor et dans l’affaire de Kozbi, leur sœur, fille d’un responsable de Madiane, celle qui fut frappée le jour du fléau dans l’affaire de Péor. »
${}^{19}Voici ce qui arriva après le fléau.
      
         
      \bchapter{}
      \begin{verse}
${}^{1}Le Seigneur dit à Moïse et à Éléazar, fils du prêtre Aaron : 
${}^{2}« De toute la communauté des fils d’Israël, faites le dénombrement, par familles, de tous ceux qui ont vingt ans et plus, ceux qui, en Israël, pourraient aller au combat. » 
${}^{3}Moïse et le prêtre Éléazar leur parlèrent donc dans les steppes de Moab, au bord du Jourdain, à la hauteur de Jéricho. Ils dirent : 
${}^{4}« Tous ceux qui ont vingt ans et plus, on va les recenser comme le Seigneur l’a ordonné à Moïse. »
      Voici quels étaient les fils d’Israël sortis du pays d’Égypte : 
${}^{5}Roubène, le fils aîné d’Israël. Fils de Roubène : de Hénoch est issu le clan des Hénochites ; de Pallou, le clan des Pallouites ; 
${}^{6}de Hesrone, le clan des Hesronites ; de Karmi, le clan des Karmites. 
${}^{7}Tels étaient les clans des Roubénites. On recensa 43 730 hommes.
${}^{8}Fils de Pallou : Éliab. 
${}^{9}Fils d’Éliab : Nemouël, Datane, Abiram. C’étaient ce Datane et cet Abiram, délégués de leur communauté, qui s’étaient révoltés contre Moïse et Aaron. Ils appartenaient à la communauté de Coré quand ils se révoltèrent contre le Seigneur. 
${}^{10}La terre, ouvrant sa gueule, les engloutit ainsi que Coré, lorsque sa communauté fut livrée à la mort, le feu ayant dévoré les deux cent cinquante hommes. Ils sont un signe. 
${}^{11}Mais les fils de Coré ne moururent pas.
${}^{12}Fils de Siméon par clans : de Nemouël est issu le clan des Nemouélites ; de Yamine, le clan des Yaminites ; de Yakine, le clan des Yakinites ; 
${}^{13}de Zèrah, le clan des Zarhites ; de Saül, le clan des Saülites. 
${}^{14}Tels étaient les clans des Siméonites : 22 200 hommes.
${}^{15}Fils de Gad par clans : de Cefone est issu le clan des Cefonites ; de Haggui, le clan des Hagguites ; de Shouni, le clan des Shounites ; 
${}^{16}d’Ozni, le clan des Oznites ; d’Éri, le clan des Érites ; 
${}^{17}d’Arod, le clan des Arodites ; d’Aréli, le clan des Arélites. 
${}^{18}Tels étaient les clans des fils de Gad, selon leur recensement : ils étaient 40 500 hommes.
${}^{19}Fils de Juda : Er et Onane ; mais Er et Onane moururent au pays de Canaan. 
${}^{20}Voici donc les fils de Juda par clans : de Shéla est issu le clan des Shélanites ; de Pérès, le clan des Parsites ; de Zèrah, le clan des Zarhites. 
${}^{21}Et voici les fils de Pérès : de Hesrone est issu le clan des Hesronites ; de Hamoul, le clan des Hamoulites. 
${}^{22}Tels étaient les clans de Juda, selon leur recensement : ils étaient 76 500 hommes.
${}^{23}Fils d’Issakar par clans : de Tola est issu le clan des Tolaïtes ; de Pouwa, le clan des Pounites ; 
${}^{24}de Yashoub, le clan des Yashoubites ; de Shimrone, le clan des Shimronites. 
${}^{25}Tels étaient les clans d’Issakar, selon leur recensement : ils étaient 64 300 hommes.
${}^{26}Fils de Zabulon par clans : de Sèred est issu le clan des Sardites ; d’Élone, le clan des Élonites ; de Yahleël, le clan des Yahleélites. 
${}^{27}Tels étaient les clans des Zabulonites, selon leur recensement : ils étaient 60 500 hommes.
${}^{28}Fils de Joseph par clans : Manassé et Éphraïm.
${}^{29}Fils de Manassé : de Makir est issu le clan des Makirites. Makir engendra Galaad ; de Galaad est issu le clan des Galaadites. 
${}^{30}Voici les fils de Galaad : de Ièzer est issu le clan des Ièzerites ; de Héleq, le clan des Helqites ; 
${}^{31}d’Asriël, le clan des Asriélites ; de Shèkem, le clan des Shikemites ; 
${}^{32}de Shemida, le clan des Shemidaïtes ; de Héfer, le clan des Héferites. 
${}^{33}Celofehad, fils de Héfer, n’eut pas de fils mais seulement des filles. Les filles de Celofehad se nommaient Mahla, Noa, Hogla, Milka et Tirça. 
${}^{34}Tels étaient les clans de Manassé. On recensa 52 700 hommes.
${}^{35}Voici les fils d’Éphraïm par clans : de Shoutèlah est issu le clan des Shoutalhites ; de Bèker, le clan des Bakrites ; de Tahane, le clan des Tahanites. 
${}^{36}Et voici les fils de Shoutèlah : d’Érane est issu le clan des Éranites. 
${}^{37}Tels étaient les clans des fils d’Éphraïm, selon leur recensement : 32 500 hommes. Tels étaient les fils de Joseph selon leurs clans.
${}^{38}Fils de Benjamin par clans : de Bèla est issu le clan des Baléïtes ; d’Ashbel, le clan des Ashbélites ; d’Ahiram, le clan des Ahiramites ; 
${}^{39}de Shefoufâm, le clan des Shoufamites ; de Houfam, le clan des Houfamites. 
${}^{40}Les fils de Bèla furent : Ard et Naaman, le clan des Ardites et, issu de Naaman, le clan des Naamites. 
${}^{41}Tels étaient les fils de Benjamin selon leurs clans. On recensa 45 600 hommes.
${}^{42}Voici les fils de Dane par clans : de Shouham est issu le clan des Shouhamites ; tels étaient les clans de Dane selon leurs clans. 
${}^{43}Tous les clans des Shouhamites, selon leur recensement, comptaient 64 400 hommes.
${}^{44}Fils d’Asher par clans : de Yimna est issu le clan de Yimna ; de Yishwi, le clan des Yishwites ; de Beria, le clan des Beriites. 
${}^{45}Issus des fils de Beria : de Hèber, le clan des Hébrites ; de Malkiël, le clan des Malkiélites. 
${}^{46}La fille d’Asher se nommait Sarah. 
${}^{47}Tels étaient les clans des fils d’Asher, selon leur recensement : 53 400 hommes.
${}^{48}Fils de Nephtali par clans : de Yahceël est issu le clan des Yahceélites ; de Gouni, le clan des Gounites ; 
${}^{49}de Yécer, le clan des Yiçrites ; de Shillem, le clan des Shillémites. 
${}^{50}Tels étaient les clans de Nephtali selon leurs clans. On recensa 45 400 hommes.
${}^{51}Tel était le nombre des recensés des fils d’Israël : 601 730 hommes.
${}^{52}Le Seigneur parla à Moïse. Il dit : 
${}^{53}« C’est entre eux que le pays sera partagé : les parts d’héritage seront proportionnées au nombre des personnes. 
${}^{54}Aux clans les plus importants, tu donneras une part d’héritage plus importante ; aux plus petits, une part plus petite : chacun recevra une part d’héritage selon le nombre de ses recensés. 
${}^{55}C’est seulement par tirage au sort que le pays sera partagé : ils recevront des parts d’héritage selon le nombre de personnes de leurs tribus patriarcales. 
${}^{56}On partagera donc l’héritage par tirage au sort entre les clans plus importants et les plus petits. »
${}^{57}Et voici les Lévites recensés par clans : de Guershone est issu le clan des Guershonites ; de Qehath, le clan des Qehatites ; de Merari, le clan des Merarites. 
${}^{58}Voici les clans de Lévi : clan des Libnites, clan des Hébronites, clan des Mahlites, clan des Moushites, clan des Coréïtes. Qehath engendra Amram. 
${}^{59}La femme d’Amram se nommait Yokèbed ; fille de Lévi, elle fut enfantée à Lévi en Égypte. À son tour, elle enfanta à Amram Aaron, Moïse et leur sœur Miryam. 
${}^{60}Aaron engendra Nadab, Abihou, Éléazar et Itamar. 
${}^{61}Nadab et Abihou moururent pour avoir présenté un feu profane devant le Seigneur. 
${}^{62}Parmi les Lévites, on recensa tous les mâles âgés d’un mois et plus : 23 000 hommes. Ils n’avaient pas été recensés parmi les fils d’Israël puisque, parmi les fils d’Israël, aucun héritage ne leur avait été attribué.
${}^{63}Voilà ceux qui furent recensés par Moïse et le prêtre Éléazar, lorsqu’ils recensèrent les fils d’Israël dans les steppes de Moab, au bord du Jourdain, à la hauteur de Jéricho. 
${}^{64}Parmi eux, il ne restait personne de ceux qui avaient été recensés par Moïse et le prêtre Aaron, lors du recensement des fils d’Israël dans le désert du Sinaï. 
${}^{65}En effet, le Seigneur leur avait dit : « Ils mourront, oui, ils mourront dans le désert ! », et aucun d’eux n’était resté en vie, sauf Caleb, fils de Yefounnè, et Josué, fils de Noun.
      
         
      \bchapter{}
      \begin{verse}
${}^{1}Se présentèrent alors les filles de Celofehad, fils de Héfer, fils de Galaad, fils de Makir, fils de Manassé. Elles appartenaient à l’un des clans de Manassé, fils de Joseph. Voici les noms des filles de Celofehad : Mahla, Noa, Hogla, Milka et Tirça. 
${}^{2}Elles se tinrent devant Moïse, devant le prêtre Éléazar, devant les responsables et toute la communauté à l’entrée de la tente de la Rencontre. Elles dirent : 
${}^{3}« Notre père est mort dans le désert. Il ne faisait pas partie de la communauté de ceux qui se liguèrent contre le Seigneur, la communauté de Coré ; il est mort à cause de sa propre faute mais il n’avait pas de fils. 
${}^{4}Pourquoi le nom de notre père devrait-il disparaître du clan parce qu’il n’a pas eu de fils ? Donne-nous donc une propriété parmi les frères de notre père ! »
${}^{5}Moïse présenta leur requête devant le Seigneur. 
${}^{6}Et le Seigneur parla à Moïse. Il dit : 
${}^{7}« Les filles de Celofehad ont bien raison ! Oui, tu leur donneras en héritage une propriété parmi les frères de leur père. Tu leur transmettras l’héritage de leur père. 
${}^{8}Et tu parleras aux fils d’Israël. Tu diras : “Si un homme meurt sans avoir de fils, vous transmettrez son héritage à sa fille. 
${}^{9}S’il n’a pas de fille, vous donnerez l’héritage à ses frères. 
${}^{10}S’il n’a pas de frères, vous donnerez l’héritage aux frères de son père. 
${}^{11}Et si son père n’a pas de frères, vous donnerez l’héritage au plus proche parent de son clan, il en prendra possession.” C’est pour les fils d’Israël une règle de droit, comme le Seigneur l’a ordonné à Moïse. »
${}^{12}Le Seigneur dit à Moïse : « Monte sur la montagne des Abarim que voici ; regarde la terre que j’ai donnée aux fils d’Israël ! 
${}^{13}Tu la regarderas, puis tu seras réuni aux tiens, toi aussi, comme l’a été Aaron, ton frère. 
${}^{14}En effet, vous vous êtes rebellés contre ma parole au désert de Cine, quand la communauté m’a cherché querelle, alors que l’eau jaillissante aurait dû manifester à leurs yeux ma sainteté. Cela se passait aux eaux de Mériba de Cadès, les eaux de la querelle de Cadès, dans le désert de Cine. »
${}^{15}Alors Moïse parla au Seigneur. Il dit : 
${}^{16}« Que le Seigneur, Dieu des esprits qui animent tout être de chair, établisse à la tête de la communauté un homme 
${}^{17}qui parte en campagne et revienne à leur tête, qui les fasse sortir et rentrer. Ainsi la communauté du Seigneur ne sera pas comme du petit bétail sans berger. » 
${}^{18}Le Seigneur dit à Moïse : « Prends Josué, fils de Noun, un homme habité par l’esprit. Tu poseras la main sur lui, 
${}^{19}puis tu le placeras devant le prêtre Éléazar et devant toute la communauté, et tu lui donneras tes ordres sous leurs yeux. 
${}^{20}Tu mettras en lui une part de ton rayonnement pour que toute la communauté des fils d’Israël l’écoute. 
${}^{21}Il se tiendra devant le prêtre Éléazar qui le soumettra, devant le Seigneur, au jugement des Ourim. À sa parole, tous les fils d’Israël sortiront ; à sa parole, ils rentreront, lui et tous les fils d’Israël avec lui, toute la communauté. »
${}^{22}Moïse fit comme le Seigneur le lui avait ordonné : il prit Josué et le plaça devant le prêtre Éléazar et devant toute la communauté. 
${}^{23}Il posa les mains sur lui et lui donna ses ordres. Et il en fut comme le Seigneur l’avait dit par l’intermédiaire de Moïse.
      
         
      \bchapter{}
      \begin{verse}
${}^{1}Le Seigneur parla à Moïse. Il dit : 
${}^{2}« Donne cet ordre aux fils d’Israël. Tu leur diras : Les présents qui me sont réservés, mes vivres, sous forme de nourriture offerte en agréable odeur pour moi, vous aurez soin de me les apporter à la date prescrite. 
${}^{3}Et tu leur diras encore : Voici la nourriture offerte que vous apporterez au Seigneur : des agneaux de l’année, sans défaut, deux par jour, en holocauste perpétuel. 
${}^{4}Tu présenteras le premier agneau au matin et le second au coucher du soleil, 
${}^{5}et, comme offrande, un dixième d’épha de fleur de farine pétrie dans un quart de mesure d’huile d’olive vierge. 
${}^{6}C’est là un holocauste perpétuel, tel qu’il fut fait au mont Sinaï, une nourriture offerte, en agréable odeur pour le Seigneur. 
${}^{7}La libation qui l’accompagne sera d’un quart de mesure pour le premier agneau. C’est dans le sanctuaire qu’on versera pour le Seigneur la libation de boisson forte. 
${}^{8}Tu offriras le second agneau au coucher du soleil. Tu feras l’offrande de céréales et la libation comme le matin. C’est une nourriture offerte, en agréable odeur pour le Seigneur.
${}^{9}Le jour du sabbat, tu offriras deux agneaux de l’année, sans défaut, avec deux dixièmes de fleur de farine pétrie à l’huile, et la libation qui accompagne. 
${}^{10}C’est l’holocauste du sabbat, à offrir chaque sabbat, en plus de l’holocauste perpétuel et de la libation qui l’accompagne.
${}^{11}Au commencement de chaque mois, vous apporterez au Seigneur pour l’holocauste deux taureaux, un bélier et sept agneaux de l’année, sans défaut. 
${}^{12}On ajoutera, comme offrande de céréales, trois dixièmes de fleur de farine pétrie à l’huile pour chaque taureau, deux dixièmes de farine pétrie à l’huile pour l’unique bélier 
${}^{13}et un seul dixième de farine pétrie à l’huile pour chaque agneau. C’est un holocauste d’agréable odeur, une nourriture offerte, pour le Seigneur. 
${}^{14}Et les libations qui accompagnent seront d’une demie mesure de vin par taureau, d’un tiers de mesure pour le bélier et d’un quart de mesure par agneau. Tel est l’holocauste du mois, à offrir chaque mois, tous les mois de l’année. 
${}^{15}On ajoutera un bouc offert au Seigneur en sacrifice pour la faute, en plus de l’holocauste perpétuel et de sa libation.
${}^{16}Le premier mois, le quatorzième jour du mois, c’est la Pâque pour le Seigneur. 
${}^{17}Le quinzième jour de ce mois, c’est fête. Pendant sept jours, on mangera des pains sans levain. 
${}^{18}Le premier jour, il y aura une assemblée sainte. Vous n’accomplirez aucun travail, aucun labeur. 
${}^{19}Vous apporterez en nourriture offerte, comme holocauste au Seigneur : deux taureaux, un bélier et sept agneaux de l’année ; vous les choisirez sans défaut. 
${}^{20}L’offrande qui les accompagne sera de la fleur de farine pétrie à l’huile ; vous en présenterez trois dixièmes par taureau et deux dixièmes pour le bélier. 
${}^{21}Tu en présenteras un seul dixième pour chacun des sept agneaux. 
${}^{22}On ajoutera un bouc en sacrifice pour la faute, afin d’accomplir sur vous le rite d’expiation. 
${}^{23}C’est indépendamment de l’holocauste du matin qui fait partie de l’holocauste perpétuel que vous ferez cela. 
${}^{24}Vous ferez ainsi chaque jour pendant sept jours : l’offrande du pain, la nourriture offerte, en agréable odeur pour le Seigneur. Cela s’ajoutera à l’holocauste perpétuel et à sa libation. 
${}^{25}Le septième jour, vous tiendrez une assemblée sainte, vous n’accomplirez aucun travail, aucun labeur.
${}^{26}Le jour des Prémices, quand vous apporterez au Seigneur la nouvelle offrande de céréales pour la fête des Semaines, vous tiendrez une assemblée sainte et vous n’accomplirez aucun travail, aucun labeur. 
${}^{27}Vous apporterez au Seigneur pour l’holocauste d’agréable odeur deux taureaux, un bélier et sept agneaux de l’année. 
${}^{28}L’offrande de céréales qui les accompagne sera de la fleur de farine pétrie à l’huile : trois dixièmes par taureau, deux dixièmes pour le bélier, 
${}^{29}un seul dixième pour chacun des sept agneaux. 
${}^{30}On ajoutera un bouc pour accomplir sur vous le rite d’expiation. 
${}^{31}Indépendamment de l’holocauste perpétuel accompagné de son offrande de céréales, vous les présenterez avec les libations. Vous choisirez des bêtes sans défaut.
      
         
      \bchapter{}
      \begin{verse}
${}^{1}« Le septième mois, le premier jour du mois, vous tiendrez une assemblée sainte. Vous ne ferez aucun travail, aucun labeur. Ce sera pour vous un jour d’ovations. 
${}^{2}Vous ferez un holocauste d’agréable odeur pour le Seigneur : un taureau, un bélier et sept agneaux de l’année, sans défaut. 
${}^{3}L’offrande de céréales qui les accompagne sera de la fleur de farine pétrie à l’huile : trois dixièmes par taureau, deux dixièmes pour le bélier, 
${}^{4}un dixième pour chacun des sept agneaux. 
${}^{5}On ajoutera un bouc en sacrifice pour la faute, afin d’accomplir sur vous le rite d’expiation, 
${}^{6}indépendamment de l’holocauste mensuel et de son offrande de céréales, de l’holocauste perpétuel et de son offrande de céréales avec les libations, selon le droit. C’est une nourriture offerte, en agréable odeur pour le Seigneur.
${}^{7}Le dix de ce septième mois, vous tiendrez une assemblée sainte, vous ferez pénitence et vous ne ferez aucun travail. 
${}^{8}Vous apporterez au Seigneur pour l’holocauste d’agréable odeur un taureau, un bélier et sept agneaux de l’année ; vous les choisirez sans défaut. 
${}^{9}L’offrande de céréales qui les accompagne sera de la fleur de farine pétrie à l’huile : trois dixièmes par taureau, deux dixièmes pour le bélier, 
${}^{10}un seul dixième pour chacun des sept agneaux. 
${}^{11}On ajoutera un bouc en sacrifice pour la faute, indépendamment du sacrifice pour la faute du jour du Grand Pardon, et de l’holocauste perpétuel accompagné de son offrande de céréales, avec les libations.
${}^{12}Le quinzième jour du septième mois, vous tiendrez une assemblée sainte, et vous ne ferez aucun travail, aucun labeur. Vous célébrerez la fête du Seigneur pendant sept jours. 
${}^{13}Le premier jour, vous apporterez au Seigneur pour l’holocauste, comme nourriture offerte en agréable odeur, treize taureaux, deux béliers, quatorze agneaux de l’année ; ils seront sans défaut. 
${}^{14}L’offrande de céréales qui les accompagne sera de la fleur de farine pétrie à l’huile : trois dixièmes pour chacun des treize taureaux, deux dixièmes pour chacun des deux béliers, 
${}^{15}un seul dixième pour chacun des quatorze agneaux. 
${}^{16}On ajoutera un bouc en sacrifice pour la faute, indépendamment de l’holocauste perpétuel accompagné de son offrande de céréales, avec sa libation.
${}^{17}Le deuxième jour : douze taureaux, deux béliers et quatorze agneaux de l’année, sans défaut. 
${}^{18}L’offrande de céréales et les libations, qui accompagnent le sacrifice des taureaux, des béliers et des agneaux, seront à proportion de leur nombre, selon le droit. 
${}^{19}On ajoutera un bouc en sacrifice pour la faute, indépendamment de l’holocauste perpétuel, accompagné de son offrande de céréales, avec les libations.
${}^{20}Le troisième jour : onze taureaux, deux béliers et quatorze agneaux de l’année, sans défaut. 
${}^{21}L’offrande de céréales et les libations, qui accompagnent le sacrifice des taureaux, des béliers et des agneaux, seront à proportion de leur nombre, selon le droit. 
${}^{22}On ajoutera un bouc en sacrifice pour la faute, indépendamment de l’holocauste perpétuel, accompagné de son offrande de céréales, avec sa libation.
${}^{23}Le quatrième jour : dix taureaux, deux béliers et quatorze agneaux de l’année, sans défaut. 
${}^{24}L’offrande de céréales et les libations, qui accompagnent le sacrifice des taureaux, des béliers et des agneaux, seront à proportion de leur nombre, selon le droit. 
${}^{25}On ajoutera un bouc en sacrifice pour la faute, indépendamment de l’holocauste perpétuel, accompagné de son offrande de céréales, avec sa libation.
${}^{26}Le cinquième jour : neuf taureaux, deux béliers et quatorze agneaux de l’année, sans défaut. 
${}^{27}L’offrande de céréales et les libations, qui accompagnent le sacrifice des taureaux, des béliers et des agneaux, seront à proportion de leur nombre, selon le droit. 
${}^{28}On ajoutera un bouc en sacrifice pour la faute, indépendamment de l’holocauste perpétuel, accompagné de son offrande de céréales, avec sa libation.
${}^{29}Le sixième jour : huit taureaux, deux béliers et quatorze agneaux de l’année, sans défaut. 
${}^{30}L’offrande de céréales et les libations, qui accompagnent le sacrifice des taureaux, des béliers et des agneaux, seront à proportion de leur nombre, selon le droit. 
${}^{31}On ajoutera un bouc en sacrifice pour la faute, indépendamment de l’holocauste perpétuel, accompagné de son offrande de céréales, avec ses libations.
${}^{32}Le septième jour : sept taureaux, deux béliers et quatorze agneaux de l’année, sans défaut. 
${}^{33}L’offrande de céréales et les libations, qui accompagnent le sacrifice des taureaux, des béliers et des agneaux, seront à proportion de leur nombre, selon le droit. 
${}^{34}On ajoutera un bouc en sacrifice pour la faute, indépendamment de l’holocauste perpétuel, accompagné de son offrande de céréales, avec sa libation.
${}^{35}Le huitième jour, aura lieu pour vous la clôture de la fête ; vous ne ferez aucun travail, aucun labeur. 
${}^{36}Vous apporterez au Seigneur pour l’holocauste, comme nourriture offerte en agréable odeur, un taureau, un bélier, sept agneaux de l’année ; ils seront sans défaut. 
${}^{37}L’offrande de céréales et les libations, qui accompagnent le sacrifice du taureau, du bélier et des agneaux, seront à proportion de leur nombre, selon le droit. 
${}^{38}On ajoutera un bouc en sacrifice pour la faute, indépendamment de l’holocauste perpétuel, accompagné de son offrande de céréales, avec sa libation.
${}^{39}Outre vos offrandes votives et vos offrandes volontaires, voilà donc ce que vous ferez pour le Seigneur, lors de vos solennités, en ce qui concerne vos holocaustes, offrandes de céréales, libations et sacrifices de paix. »
      
         
      \bchapter{}
      \begin{verse}
${}^{1}Moïse rapporta aux fils d’Israël tout ce que le Seigneur lui avait ordonné.
      
         
${}^{2}Puis il parla aux chefs de tribu des fils d’Israël. Il dit : « Voici ce que le Seigneur a ordonné : 
${}^{3}L’homme qui fait un vœu au Seigneur, ou prend par serment un engagement qui l’oblige lui-même, ne reniera pas sa parole ; il agira en tout selon ce qu’il a dit. 
${}^{4}Quand une femme, jeune encore et vivant chez son père, fait un vœu au Seigneur et prend un engagement, 
${}^{5}si son père apprend l’existence de ce vœu et de cet engagement par lequel elle s’oblige elle-même, et que cependant il ne lui dise rien, alors tous ses vœux restent valides et tout engagement qui l’oblige elle-même reste valide. 
${}^{6}Mais si son père, le jour même où il apprend cela, fait opposition, alors tous ses vœux et l’engagement qui l’oblige elle-même ne sont plus valides ; le Seigneur lui pardonnera puisque son père a fait opposition.
${}^{7}Si elle appartient à un homme par le mariage et si elle est liée par des vœux qu’elle s’est imposés ou qu’elle a tenu des propos inconsidérés qui l’obligent elle-même, 
${}^{8}et que son mari apprenne cela, si, le jour même où il l’apprend, il ne lui dit rien, alors ses vœux restent valides et l’engagement qui l’oblige elle-même reste valide. 
${}^{9}Mais si, le jour même où son mari l’apprend, il fait opposition, alors il annule le vœu qu’elle s’est imposée et les propos inconsidérés qui l’obligent elle-même ; le Seigneur pardonnera à la femme.
${}^{10}Le vœu d’une veuve ou d’une femme répudiée, tout engagement qui l’oblige elle-même, reste valide pour elle. 
${}^{11}Si c’est dans la maison de son mari qu’une femme fait un vœu ou prend par serment un engagement qui l’oblige elle-même, 
${}^{12}et que son mari apprenne cela, ne lui dise rien et ne fasse pas opposition, tous ses vœux restent valides et tout engagement qui l’oblige elle-même reste valide. 
${}^{13}Mais si son mari les annule le jour même où il l’apprend, rien de ce qui a passé ses lèvres en ce qui concerne ses vœux et ce qui l’oblige elle-même ne sera valide : son mari les a annulés, et le Seigneur pardonnera à la femme. 
${}^{14}Tout vœu et tout engagement par serment à faire pénitence sera validé par le mari, ou annulé par lui. 
${}^{15}Si, d’un jour à l’autre, son mari ne lui a rien dit, il rend valides tous ses vœux et tous les engagements qu’elle s’est imposés ; il les rend valides puisqu’il ne lui a rien dit le jour même où il l’a appris. 
${}^{16}Mais s’il tarde à les annuler après l’avoir appris, c’est lui qui portera le poids de sa faute à elle. »
${}^{17}Voilà donc les décrets que le Seigneur a prescrits à Moïse, concernant les relations entre un mari et sa femme, entre un père et sa fille, quand, jeune encore, elle vit chez son père.
      
         
      \bchapter{}
      \begin{verse}
${}^{1}Le Seigneur parla à Moïse. Il dit : 
${}^{2}« Exerce la vengeance des fils d’Israël contre les Madianites. Ensuite tu seras réuni aux tiens. » 
${}^{3}Alors, Moïse parla au peuple. Il dit : « Équipez, parmi vous, des hommes pour l’armée ! Ils iront combattre Madiane pour lui infliger la vengeance du Seigneur. 
${}^{4}Vous enverrez à l’armée mille hommes par tribu, mille hommes de chacune des tribus d’Israël. »
${}^{5}On engagea donc, parmi les clans d’Israël, mille hommes par tribu, soit douze mille hommes équipés pour l’armée. 
${}^{6}Moïse envoya donc à l’armée mille hommes par tribu et, avec eux, Pinhas, fils du prêtre Éléazar. Celui-ci se joignit à l’armée, ayant à portée de main les objets du sanctuaire et les trompettes pour accompagner les ovations. 
${}^{7}Ils partirent en guerre contre Madiane, comme le Seigneur l’avait ordonné à Moïse, et ils tuèrent tous les hommes. 
${}^{8}Ils tuèrent aussi les rois de Madiane, les ajoutant à leurs victimes : Évi, Rèqem, Sour, Hour et Rèba, les cinq rois de Madiane. Ils tuèrent par l’épée Balaam, fils de Béor. 
${}^{9}Puis les fils d’Israël emmenèrent comme captives les femmes de Madiane avec leurs petits enfants et ils se livrèrent au pillage en s’emparant de tout leur bétail, de tous leurs troupeaux, de toutes leurs richesses. 
${}^{10}À toutes les villes qu’habitaient les Madianites et à tous leurs campements, ils mirent le feu. 
${}^{11}Ils prirent donc tout le butin, tout ce qu’ils avaient capturé : hommes et bétail. 
${}^{12}Les captifs, les prises de guerre, le butin, ils les amenèrent à Moïse, au prêtre Éléazar et à la communauté des fils d’Israël qui avaient leur camp dans les steppes de Moab au bord du Jourdain, à la hauteur de Jéricho.
${}^{13}Moïse, le prêtre Éléazar et tous les chefs de la communauté sortirent à leur rencontre, hors du camp. 
${}^{14}Moïse s’irrita contre les commandants des forces armées, officiers de millier, officiers de centaine, qui revenaient de cette expédition guerrière. 
${}^{15}Il leur dit : « Vous avez donc laissé vivre toutes les femmes ? 
${}^{16}Or ce sont elles qui ont engagé les fils d’Israël à être infidèles au Seigneur, dans l’affaire de Balaam, comme dans l’affaire de Péor : ce qui a provoqué un fléau dans la communauté du Seigneur. 
${}^{17}Maintenant, tuez donc tous les petits garçons ; et toutes les femmes qui ont partagé la couche d’un homme, tuez-les ! 
${}^{18}Mais toutes les petites filles, elles qui n’ont pas partagé la couche d’un homme, vous pouvez les garder vivantes pour vous. 
${}^{19}Quant à vous, campez pendant sept jours hors du camp. Que vous ayez tué quelqu’un ou touché un mort, tous vous vous purifierez le troisième et le septième jour, vous et vos captifs. 
${}^{20}De même, tout vêtement, objets en peau, ouvrages en poil de chèvre, objets en bois, vous les purifierez. »
${}^{21}Le prêtre Éléazar dit aux hommes de l’armée qui étaient allés au combat : « Voici la disposition de la loi que le Seigneur a prescrite à Moïse : 
${}^{22}l’or, l’argent, le bronze, le fer, l’étain, le plomb, 
${}^{23}bref toute chose qui supporte le feu, vous la passerez par le feu, et elle sera pure ; on la purifiera aussi par l’eau lustrale. Mais toute chose qui ne supporte pas le feu, vous la passerez par l’eau. 
${}^{24}Vous laverez aussi vos vêtements, le septième jour, et vous serez purs. Après quoi, vous rentrerez dans le camp. »
${}^{25}Le Seigneur parla à Moïse. Il dit : 
${}^{26}« Toi, le prêtre Éléazar et les chefs de famille de la communauté, vous ferez le compte de ce qui a été pris ou capturé, homme ou bétail. 
${}^{27}Ce qui a été pris, tu le diviseras en deux parts, l’une pour ceux qui, mobilisés, sont partis au combat et l’autre pour toute la communauté. 
${}^{28}Sur la part des hommes de guerre qui sont partis au combat, tu prélèveras une taxe pour le Seigneur : elle sera d’un être vivant sur cinq cents, qu’il s’agisse d’hommes, de bœufs, d’ânes ou de moutons. 
${}^{29}Vous la prendrez sur la moitié qu’ils ont reçue, et tu donneras au prêtre Éléazar ce qui a été prélevé pour le Seigneur. 
${}^{30}De la moitié qui revient aux fils d’Israël tu prendras un être vivant sur cinquante, qu’il s’agisse d’hommes, de bœufs, d’ânes, de moutons, de tout bétail. Tu les donneras aux Lévites préposés à la garde de la demeure du Seigneur. »
${}^{31}Moïse et le prêtre Éléazar firent comme le Seigneur l’avait ordonné à Moïse. 
${}^{32}Les prises de guerre, ce qui restait après le pillage effectué par les troupes en campagne, étaient de 675 000 moutons, 
${}^{33}72 000 bœufs, 
${}^{34}61 000 ânes ; 
${}^{35}quant aux personnes humaines, les femmes qui n’avaient pas partagé la couche d’un homme, il y en avait en tout 32 000. 
${}^{36}La moitié attribuée à ceux qui étaient partis au combat se montait à 337 500 moutons, 
${}^{37}dont une taxe pour le Seigneur de 675 bêtes, 
${}^{38}36 000 bœufs, dont une taxe pour le Seigneur de 72 bêtes, 
${}^{39}30 500 ânes, dont une taxe pour le Seigneur de 61 bêtes ; 
${}^{40}16 000 êtres humains dont une taxe pour le Seigneur de 32 personnes. 
${}^{41}Moïse donna la taxe prélevée pour le Seigneur au prêtre Éléazar, comme le Seigneur l’avait ordonné à Moïse. 
${}^{42}La moitié attribuée aux fils d’Israël, distincte de celle que Moïse avait retirée pour les combattants, 
${}^{43}cette moitié attribuée à la communauté était de : 337 500 moutons, 
${}^{44}36 000 bœufs, 
${}^{45}30 500 ânes, 
${}^{46}16 000 êtres humains. 
${}^{47}Sur la moitié attribuée aux fils d’Israël, Moïse prit un cinquantième des hommes et du bétail, et il les donna aux Lévites préposés à la garde de la demeure du Seigneur, comme le Seigneur l’avait ordonné à Moïse.
${}^{48}Ceux qui avaient commandé les clans en campagne, officiers de millier, officiers de centaine, s’approchèrent de Moïse. 
${}^{49}Ils lui dirent : « Tes serviteurs ont fait le compte des hommes de guerre qui étaient sous nos ordres. Il n’en manque pas un seul ! 
${}^{50}Nous allons donc apporter, en présent réservé au Seigneur, ce que chacun a trouvé : objet d’or, chaînette, bracelet, anneau, boucle d’oreille, breloque, pour faire sur nous le rite d’expiation en présence du Seigneur. » 
${}^{51}Moïse et le prêtre Éléazar reçurent l’or qu’ils leur présentaient, tous objets ouvragés. 
${}^{52}Tout cet or, part prélevée pour le Seigneur, faisait 16 750 sicles, provenant des officiers de millier et des officiers de centaine. 
${}^{53}Les hommes de l’armée avaient pillé, chacun pour soi. 
${}^{54}Moïse et le prêtre Éléazar reçurent l’or des officiers de millier et des officiers de centaine. Ils l’apportèrent dans la tente de la Rencontre pour servir de mémorial aux fils d’Israël en présence du Seigneur.
      
         
      \bchapter{}
      \begin{verse}
${}^{1}Les fils de Roubène et les fils de Gad possédaient des troupeaux nombreux qui étaient de grande qualité. Ils virent le pays de Yazèr et le pays de Galaad, et voici que la région convenait bien pour des troupeaux ! 
${}^{2}Les fils de Gad et les fils de Roubène vinrent donc trouver Moïse, le prêtre Éléazar et les chefs de la communauté, et ils dirent : 
${}^{3}« Ataroth, Dibone, Yazèr, Nimra, Heshbone, Élalé, Sebam, Nébo, Béone, 
${}^{4}sont des villes faisant partie d’un pays que le Seigneur a frappé devant la communauté d’Israël. Ce pays convient bien pour des troupeaux ; or, tes serviteurs ont des troupeaux. 
${}^{5}Si nous avons trouvé grâce à tes yeux, ajoutèrent-ils, que ce pays soit donné en propriété à tes serviteurs. Ne nous fais pas passer le Jourdain. »
${}^{6}Mais Moïse répondit aux fils de Gad et aux fils de Roubène : « Vos frères partiraient en guerre, et vous, vous resteriez ici ? 
${}^{7}Pourquoi donc découragez-vous les fils d’Israël de passer dans le pays que le Seigneur leur donne ? 
${}^{8}C’est ainsi que nos pères ont agi quand je les ai envoyés de Cadès-Barnéa pour voir le pays. 
${}^{9}Ils sont montés jusqu’au torrent d’Eshkol et, après avoir vu le pays, ils ont découragé les fils d’Israël d’entrer dans le pays que le Seigneur leur a donné. 
${}^{10}Le Seigneur s’enflamma de colère, ce jour-là. Il fit ce serment : 
${}^{11}“Jamais les hommes qui sont montés d’Égypte – tous ceux de plus de vingt ans – ne verront la terre que j’ai juré de donner à Abraham, à Isaac et à Jacob, car ils ne m’ont pas suivi sans réserve, 
${}^{12}à l’exception de Caleb, fils de Yefounnè, le Qenizzite, et de Josué, fils de Noun, car ils ont suivi le Seigneur sans réserve !” 
${}^{13}Le Seigneur s’est enflammé de colère contre Israël et il l’a fait errer dans le désert pendant quarante ans, jusqu’à ce que disparaisse toute la génération qui avait fait le mal aux yeux du Seigneur. 
${}^{14}Et voici que vous, engeance de pécheurs, vous vous insurgez à la suite de vos pères, en attisant encore l’ardeur de la colère du Seigneur contre Israël ! 
${}^{15}Si vous vous détournez de lui, le Seigneur prolongera encore le séjour au désert, et vous aurez détruit tout ce peuple ! »
${}^{16}S’approchant de Moïse, ils dirent : « Nous allons construire ici des parcs à moutons pour nos troupeaux et des villes pour nos petits enfants. 
${}^{17}Nous-mêmes alors, nous prendrons les armes, prêts à marcher à la tête des fils d’Israël jusqu’à ce que nous les ayons fait entrer dans le lieu qui leur est destiné. Nos petits enfants resteront dans les villes fortifiées, où ils seront à l’abri des habitants du pays. 
${}^{18}Nous ne reviendrons pas dans nos maisons avant que chacun des fils d’Israël ait pris possession de sa part d’héritage. 
${}^{19}Car nous ne posséderons aucune part d’héritage au-delà du Jourdain, ni plus loin, puisque la nôtre se trouve de ce côté-ci, à l’orient du Jourdain. »
${}^{20}Moïse leur dit : « Si vous faites comme vous le dites, si vous prenez les armes en présence du Seigneur, 
${}^{21}si tous vos guerriers passent le Jourdain en présence du Seigneur jusqu’à ce qu’il ait dépossédé ses ennemis en les chassant devant lui, 
${}^{22}et si le pays est conquis en présence du Seigneur avant que vous ne reveniez chez vous, alors vous serez quittes envers le Seigneur et envers Israël, et ce pays-ci vous sera laissé en propriété en présence du Seigneur. 
${}^{23}Mais si vous n’agissez pas ainsi, alors vous pécherez contre le Seigneur et, sachez-le, votre péché vous rattrapera. 
${}^{24}Construisez donc des villes pour vos petits enfants et des parcs pour vos moutons. Mais, la parole sortie de votre bouche, accomplissez-la. »
${}^{25}Les fils de Gad et les fils de Roubène répondirent à Moïse : « Tes serviteurs feront comme mon seigneur l’a ordonné. 
${}^{26}Nos petits enfants et nos femmes, nos troupeaux et tout notre bétail resteront ici, dans les villes de Galaad. 
${}^{27}Tes serviteurs, tous équipés pour l’armée, passeront en présence du Seigneur pour aller au combat, comme mon seigneur le dit. »
${}^{28}Moïse donna des ordres à leur sujet au prêtre Éléazar, à Josué, fils de Noun, et aux chefs de famille des tribus des fils d’Israël. 
${}^{29}Il leur dit : « Si les fils de Gad et les fils de Roubène passent avec vous le Jourdain, tous armés pour le combat, en présence du Seigneur, et si le pays en face de vous est conquis, alors vous leur donnerez le pays de Galaad en propriété. 
${}^{30}Mais s’ils ne passent pas en armes avec vous, alors ils n’auront de propriété que parmi vous au pays de Canaan. »
${}^{31}Les fils de Gad et les fils de Roubène répondirent : « Ce que le Seigneur a dit à tes serviteurs, oui, nous le ferons. 
${}^{32}Nous-mêmes, en présence du Seigneur, nous passerons en armes au pays de Canaan. Et nous aurons notre propriété, notre part d’héritage, de ce côté-ci du Jourdain. »
${}^{33}Moïse donna aux fils de Gad et aux fils de Roubène, ainsi qu’à la demi-tribu de Manassé, fils de Joseph, le royaume de Séhone, roi des Amorites, et le royaume de Og, roi de Bashane, soit le pays avec les villes à l’intérieur du territoire et les villes frontalières du pays. 
${}^{34}Les fils de Gad construisirent Dibone, Ataroth, Aroër, 
${}^{35}Atroth-Shofane, Yazèr, Yogboha, 
${}^{36}Beth-Nimra et Beth-Harane, des villes fortifiées ainsi que des parcs à moutons. 
${}^{37}Les fils de Roubène reconstruisirent Heshbone, Élalé, Qiryataïm, 
${}^{38}Nébo, Baal-Méone – dont on changea le nom – et Sibma. Ils donnèrent des noms aux villes qu’ils avaient reconstruites.
${}^{39}Les fils de Makir, fils de Manassé, allèrent s’emparer du Galaad. Ils dépossédèrent les Amorites qui s’y trouvaient, 
${}^{40}et Moïse donna le Galaad à Makir, fils de Manassé, qui s’y installa. 
${}^{41}Yaïr, fils de Manassé, alla s’emparer de leurs campements. Il les appela : « Campements de Yaïr ». 
${}^{42}Nobah alla s’emparer de Qenath et de ses dépendances. Et il lui donna son propre nom : Nobah.
      
         
      \bchapter{}
      \begin{verse}
${}^{1}Voici les étapes faites par les fils d’Israël, quand ils sortirent du pays d’Égypte en ordre militaire sous la conduite de Moïse et d’Aaron. 
${}^{2}Moïse inscrivit les points de départ de leurs étapes, selon la parole du Seigneur. Voici donc leurs étapes et leurs points de départ.
${}^{3}Ils partirent de Ramsès le premier mois, le quinzième jour de ce premier mois. Le lendemain de la Pâque, les fils d’Israël sortirent librement sous les yeux de toute l’Égypte, 
${}^{4}tandis que les Égyptiens enterraient tous ceux que le Seigneur avait frappés parmi eux, tous leurs premiers-nés. Le Seigneur avait fait justice de leurs dieux.
${}^{5}Partis de Ramsès, les fils d’Israël campèrent à Souccoth.
${}^{6}Partis de Souccoth, ils campèrent à Étam, aux confins du désert.
${}^{7}Partis d’Étam, ils revinrent sur Pi-Hahiroth, en face de Baal-Cefone, et campèrent devant Migdol.
${}^{8}Partis de devant Hahiroth, ils gagnèrent le désert en traversant la mer et, après trois jours de marche dans le désert d’Étam, ils campèrent à Mara.
${}^{9}Partis de Mara, ils arrivèrent à Élim où se trouvent douze sources et soixante-dix palmiers ; c’est là qu’ils campèrent.
${}^{10}Partis d’Élim, ils campèrent près de la mer des Roseaux.
${}^{11}Partis de la mer des Roseaux, ils campèrent dans le désert de Sine.
${}^{12}Partis du désert de Sine, ils campèrent à Dofqa.
${}^{13}Partis de Dofqa, ils campèrent à Aloush.
${}^{14}Partis d’Aloush, ils campèrent à Refidim où le peuple n’eut pas d’eau à boire.
${}^{15}Partis de Refidim, ils campèrent dans le désert du Sinaï.
${}^{16}Partis du désert du Sinaï, ils campèrent à Qibroth-ha-Taawa.
${}^{17}Partis de Qibroth-ha-Taawa, ils campèrent à Hacéroth.
${}^{18}Partis de Hacéroth, ils campèrent à Ritma.
${}^{19}Partis de Ritma, ils campèrent à Rimmone-Pèrès.
${}^{20}Partis de Rimmone-Pèrès, ils campèrent à Libna.
${}^{21}Partis de Libna, ils campèrent à Rissa.
${}^{22}Partis de Rissa, ils campèrent à Qehélata.
${}^{23}Partis de Qehélata, ils campèrent au mont Shèfer.
${}^{24}Partis du mont Shèfer, ils campèrent à Harada.
${}^{25}Partis de Harada, ils campèrent à Maqehéloth.
${}^{26}Partis de Maqehéloth, ils campèrent à Tahath.
${}^{27}Partis de Tahath, ils campèrent à Tèrah.
${}^{28}Partis de Tèrah, ils campèrent à Mitqa.
${}^{29}Partis de Mitqa, ils campèrent à Hashmona.
${}^{30}Partis de Hashmona, ils campèrent à Mosséroth.
${}^{31}Partis de Mosséroth, ils campèrent à Bené-Yaaqane.
${}^{32}Partis de Bené-Yaaqane, ils campèrent à Hor-Guidgad.
${}^{33}Partis de Hor-Guidgad, ils campèrent à Yotbata.
${}^{34}Partis de Yotbata, ils campèrent à Abrona.
${}^{35}Partis d’Abrona, ils campèrent à Écione-Guéber.
${}^{36}Partis d’Écione-Guéber, ils campèrent dans le désert de Cine, c’est-à-dire à Cadès.
${}^{37}Partis de Cadès, ils campèrent à Hor-la-Montagne, aux confins du pays d’Édom. 
${}^{38}Selon la parole du Seigneur, le prêtre Aaron monta à Hor-la-Montagne et c’est là qu’il mourut, quarante ans après la sortie des fils d’Israël du pays d’Égypte, au cinquième mois, le premier du mois. 
${}^{39}Aaron avait cent vingt-trois ans lorsqu’il mourut à Hor-la-Montagne. 
${}^{40}Un Cananéen, le roi d’Arad, apprit l’arrivée des fils d’Israël ; il habitait le Néguev, dans le pays de Canaan.
${}^{41}Partis de Hor-la-Montagne, ils campèrent à Salmona.
${}^{42}Partis de Salmona, ils campèrent à Pounone.
${}^{43}Partis de Pounone, ils campèrent à Oboth.
${}^{44}Partis d’Oboth, ils campèrent aux Ruines-des-Abarim, à la frontière de Moab.
${}^{45}Partis des Ruines, ils campèrent à Dibone-Gad.
${}^{46}Partis de Dibone-Gad, ils campèrent à Almone-Diblataïma.
${}^{47}Partis d’Almone-Diblataïma, ils campèrent dans les monts Abarim, en face de Nébo.
${}^{48}Partis des monts Abarim, ils campèrent dans les steppes de Moab, au bord du Jourdain, à la hauteur de Jéricho. 
${}^{49}Ils campèrent au bord du Jourdain depuis Beth-Yeshimoth jusqu’à Abel-Shittim, dans les steppes de Moab.
${}^{50}Le Seigneur parla à Moïse dans les steppes de Moab, au bord du Jourdain, à la hauteur de Jéricho. Il dit : 
${}^{51}« Parle aux fils d’Israël. Tu leur diras : Quand vous passerez le Jourdain, en direction du pays de Canaan, 
${}^{52}dépossédez tous les habitants de ce pays en les chassant devant vous et détruisez toutes leurs idoles. Vous détruirez toutes leurs statues en métal fondu et vous ferez disparaître tous leurs lieux sacrés. 
${}^{53}Prenez possession du pays et habitez-le, car c’est à vous que j’ai donné ce pays pour que vous le possédiez. 
${}^{54}Vous vous partagerez le pays par tirage au sort selon vos clans : aux plus nombreux vous attribuerez une part d’héritage plus grande, aux moins nombreux, une plus petite ; chacun recevra ce qui lui sera dévolu par le sort. Vous partagerez selon vos tribus patriarcales. 
${}^{55}Si vous ne dépossédez pas les habitants du pays en les chassant devant vous, ceux d’entre eux que vous y laisserez deviendront comme des épines dans vos yeux, des aiguillons dans vos flancs : ils vous opprimeront dans le pays que vous habiterez. 
${}^{56}Et c’est vous que je traiterai comme j’avais pensé les traiter ! »
      
         
      \bchapter{}
      \begin{verse}
${}^{1}Le Seigneur parla à Moïse. Il dit : 
${}^{2}« Ordonne ceci aux fils d’Israël. Tu leur diras : Vous allez entrer dans le pays de Canaan. C’est le pays qui vous est échu en héritage, le pays de Canaan, à l’intérieur des frontières que voici.
${}^{3}Votre frontière sud, c’est-à-dire du Néguev, se trouve du côté du désert de Cine qui confine à Édom : à l’est, cette frontière sud part de l’extrémité de la mer Morte ; 
${}^{4}la frontière oblique alors au sud vers la montée des Aqrabbim et passe à Cine pour aboutir, venant du sud, à Cadès-Barnéa, pour repartir vers Haçar-Addar et passer par Asmone ; 
${}^{5}contournant Asmone, la frontière oblique vers le Torrent d’Égypte et aboutit à la mer.
${}^{6}Votre frontière ouest sera la mer Méditerranée ; celle-ci sera votre frontière maritime.
${}^{7}Et voici quelle sera votre frontière nord : de la Méditerranée, vous tracerez une ligne vers Hor-la-Montagne. 
${}^{8}De Hor-la-Montagne, vous tracerez une ligne en direction de l’Entrée-de-Hamath, et la frontière aboutira à Cedâd ; 
${}^{9}puis elle ira jusqu’à Zifrone ; enfin elle aboutira à Haçar-Einane. Telle sera votre frontière nord.
${}^{10}Ensuite vous tracerez une ligne pour votre frontière orientale depuis Haçar-Einane jusqu’à Shefâm. 
${}^{11}La frontière descendra de Shefâm à Ribla, à l’est de Aïn, puis elle descendra jusqu’à atteindre les collines de la mer de Kinnèreth, à l’est. 
${}^{12}Enfin, la frontière descend vers le Jourdain et aboutit à la mer Morte. Tel sera votre pays, limité tout autour par ses frontières. »
${}^{13}Moïse donna cet ordre aux fils d’Israël. Il dit : « Voici le pays que vous vous partagerez en héritage par tirage au sort, celui que le Seigneur a prescrit de donner aux neuf tribus et à la demi-tribu. 
${}^{14}En effet, la tribu des fils de Roubène, par familles, et la tribu des fils de Gad, par familles, ont déjà reçu leurs parts d’héritage, ainsi que la demi-tribu de Manassé. 
${}^{15}Ces deux tribus et cette demi-tribu ont reçu leurs parts d’héritage de l’autre côté du Jourdain, à la hauteur de Jéricho, à l’est, à l’orient. »
${}^{16}Le Seigneur parla à Moïse. Il dit : 
${}^{17}« Voici les noms des hommes qui partageront entre vous le pays : le prêtre Éléazar et Josué, fils de Noun ; 
${}^{18}vous prendrez aussi un responsable par tribu, un seul, pour procéder au partage du pays.
${}^{19}Et voici les noms de ces hommes :
      pour la tribu de Juda : Caleb, fils de Yefounnè ;
${}^{20}pour la tribu des fils de Siméon : Shemouël, fils d’Ammihoud ;
${}^{21}pour la tribu de Benjamin : Élidad, fils de Kislone ;
${}^{22}pour la tribu des fils de Dane, le responsable : Bouqqi, fils de Yogli ;
${}^{23}quant aux fils de Joseph, pour la tribu des fils de Manassé, le responsable : Hanniël, fils d’Éfod ;
${}^{24}et pour la tribu des fils d’Éphraïm, le responsable : Qemouël, fils de Shiftane ;
${}^{25}pour la tribu des fils de Zabulon, le responsable : Éliçafane, fils de Parnak ;
${}^{26}pour la tribu des fils d’Issakar, le responsable : Paltiël, fils d’Azzane ;
${}^{27}pour la tribu des fils d’Asher, le responsable : Ahihoud, fils de Shelomi ;
${}^{28}pour la tribu des fils de Nephtali, le responsable : Pedahel, fils d’Ammihoud. »
${}^{29}C’est à ceux-là que le Seigneur a ordonné de partager l’héritage, entre les fils d’Israël, au pays de Canaan.
      
         
      \bchapter{}
      \begin{verse}
${}^{1}Le Seigneur parla à Moïse dans les steppes de Moab, au bord du Jourdain, à la hauteur de Jéricho. Il dit : 
${}^{2}« Ordonne aux fils d’Israël de donner aux Lévites, sur la part d’héritage qu’ils ont en propriété, des villes pour y habiter ; vous donnerez aussi aux Lévites les pâturages qui entourent ces villes. 
${}^{3}Les villes leur appartiendront pour qu’ils y habitent, et les pâturages seront pour leur bétail, pour leurs biens matériels, pour tous leurs animaux. 
${}^{4}Les pâturages des villes que vous donnerez aux Lévites s’étendront depuis le mur extérieur de la ville sur mille coudées alentour. 
${}^{5}Vous mesurerez, à l’extérieur de la ville, du côté est, deux mille coudées ; du côté sud, deux mille coudées ; du côté ouest, deux mille coudées ; du côté nord, deux mille coudées, la ville étant au centre. Ce sera pour eux les pâturages des villes. 
${}^{6}Les villes que vous donnerez aux Lévites sont les villes de refuge au nombre de six, données par vous pour que les meurtriers puissent s’y enfuir ; vous leur donnerez, en outre, quarante-deux villes. 
${}^{7}Au total, vous donnerez aux Lévites quarante-huit villes, les villes et leurs pâturages. 
${}^{8}Les villes que vous donnerez proviendront de la propriété des fils d’Israël : à celui qui a beaucoup, vous prendrez beaucoup ; à celui qui a peu, vous prendrez peu. Chacun donnera des villes aux Lévites en proportion de la part d’héritage qu’il aura reçue. »
      
         
${}^{9}Le Seigneur parla à Moïse. Il dit : 
${}^{10}« Parle ainsi aux fils d’Israël : Quand vous aurez passé le Jourdain pour entrer dans le pays de Canaan, 
${}^{11}choisissez-vous des villes qui seront pour vous des villes de refuge. Le meurtrier qui, par inadvertance, aura frappé quelqu’un à mort s’y enfuira. 
${}^{12}Ces villes vous serviront de refuge contre le vengeur du sang. Ainsi, le meurtrier ne sera pas mis à mort avant d’avoir comparu en jugement devant la communauté. 
${}^{13}Les villes que vous donnerez aux Lévites seront pour vous des villes de refuge au nombre de six : 
${}^{14}vous donnerez trois villes situées au-delà du Jourdain et vous donnerez trois villes situées au pays de Canaan. Elles seront les villes de refuge.
${}^{15}Ces six villes serviront de refuge aux fils d’Israël ainsi qu’à l’immigré et à l’hôte de passage, afin que s’y enfuie quiconque aura, par inadvertance, frappé quelqu’un à mort. 
${}^{16}S’il l’a frappé avec un objet en fer, et causé la mort, c’est un meurtrier ; et le meurtrier doit être mis à mort. 
${}^{17}Si de sa main il l’a frappé avec une pierre qui peut provoquer la mort, et a effectivement causé la mort, c’est un meurtrier ; et le meurtrier doit être mis à mort. 
${}^{18}Ou encore si de sa main il l’a frappé avec un objet en bois qui peut provoquer la mort, et a effectivement causé la mort, c’est un meurtrier ; et le meurtrier doit être mis à mort. 
${}^{19}C’est le vengeur du sang qui mettra à mort le meurtrier ; dès qu’il tombera sur lui, il le mettra à mort.
${}^{20}Si par haine un homme a bousculé quelqu’un ou a lancé quelque chose contre lui avec malveillance, et qu’ainsi il a causé la mort, 
${}^{21}ou si avec hostilité il l’a frappé de la main, et qu’ainsi il a causé la mort, celui qui a frappé sera mis à mort ; c’est un meurtrier. Dès qu’il tombera sur le meurtrier, le vengeur du sang le mettra à mort.
${}^{22}Mais si c’est par hasard et sans hostilité qu’un homme a bousculé quelqu’un, ou lui a lancé quelque objet sans malveillance, 
${}^{23}ou encore si, sans le voir, il l’a atteint avec une pierre qui peut provoquer la mort – et qu’en faisant tomber cette pierre, il a effectivement causé la mort, tandis qu’il n’était pas son ennemi et ne lui voulait pas de mal –, 
${}^{24}alors, la communauté jugera entre celui qui a frappé et le vengeur du sang, selon le droit. 
${}^{25}La communauté soustraira le meurtrier à la main du vengeur du sang. La communauté le renverra dans sa ville de refuge, là où il s’était enfui. Il devra y rester jusqu’à la mort du grand prêtre qui a reçu l’onction avec l’huile sainte.
${}^{26}Mais si le meurtrier sort des limites de la ville de refuge où il s’est enfui, 
${}^{27}et que le vengeur du sang, le trouvant hors des limites de sa ville de refuge, le tue, le vengeur n’aura pas de sang sur les mains. 
${}^{28}En effet, le meurtrier devra rester dans sa ville de refuge jusqu’à la mort du grand prêtre. Après la mort du grand prêtre, le meurtrier retournera sur la terre qui lui appartient. 
${}^{29}Telles seront pour vous les règles de droit, d’âge en âge, où que vous habitiez.
${}^{30}Quiconque frappe à mort une personne, c’est sur la déposition de plusieurs témoins qu’il sera tué. Mais la déposition d’un seul témoin ne pourra faire condamner quelqu’un à mort. 
${}^{31}Vous n’accepterez pas de rançon en échange de la vie d’un meurtrier. Il est passible de mort, il doit être mis à mort. 
${}^{32}Vous n’accepterez pas non plus de rançon pour permettre à celui qui s’est enfui dans sa ville de refuge de revenir habiter sur sa terre avant la mort du grand prêtre. 
${}^{33}Vous ne profanerez pas la terre où vous êtes, car le sang, lui, profane la terre. On ne peut faire l’expiation pour le sang répandu sur la terre que par le sang de celui qui l’a répandu. 
${}^{34}Vous ne rendrez pas impure la terre où vous habitez et au milieu de laquelle je demeure, car Je suis le Seigneur, moi qui demeure au milieu des fils d’Israël. »
      
         
      \bchapter{}
      \begin{verse}
${}^{1}Alors s’approchèrent les chefs de famille du clan des fils de Galaad, fils de Makir, fils de Manassé, l’un des clans des fils de Joseph. Ils prirent la parole devant Moïse et devant les responsables, chefs de famille des fils d’Israël. 
${}^{2}Ils dirent : « Le Seigneur a prescrit à mon seigneur de donner le pays en héritage aux fils d’Israël en le partageant par tirage au sort, et mon seigneur a reçu l’ordre du Seigneur de donner la part d’héritage de notre frère Celofehad à ses filles. 
${}^{3}Or, si l’une d’elles épousait un membre d’une autre tribu d’Israël, sa part d’héritage serait retranchée de l’héritage de nos pères, tandis qu’elle s’ajouterait à l’héritage de la tribu dans laquelle l’épouse serait entrée ; cette part serait alors retranchée de notre héritage attribué par le tirage au sort. 
${}^{4}Quand pour les fils d’Israël arrivera le jubilé, sa part d’héritage serait ajoutée à l’héritage de la tribu dans laquelle elle serait entrée, et donc cette part d’héritage serait retranchée de l’héritage de la tribu de nos pères. »
${}^{5}Alors Moïse donna l’ordre suivant aux fils d’Israël, selon la parole du Seigneur. Il dit : « Ceux de la tribu des fils de Joseph ont raison. 
${}^{6}Voici donc ce qu’a ordonné le Seigneur en ce qui concerne les filles de Celofehad. Il a dit : Elles pourront épouser qui bon leur semblera, à condition d’épouser quelqu’un d’un clan de la tribu de leur père. 
${}^{7}Ainsi les parts d’héritage des fils d’Israël ne passeront pas d’une tribu à l’autre. En effet, chacun des fils d’Israël restera attaché à l’héritage de la tribu de ses pères. 
${}^{8}Toute fille qui possède une part d’héritage provenant des tribus des fils d’Israël doit épouser quelqu’un d’un clan de la tribu de son père, afin que chacun des fils d’Israël possède la part d’héritage de ses pères. 
${}^{9}Ainsi la part d’héritage ne passera pas d’une tribu à l’autre, car les tribus des fils d’Israël resteront attachées chacune à sa part d’héritage. »
${}^{10}Les filles de Celofehad firent comme le Seigneur l’avait ordonné à Moïse. 
${}^{11}Mahla, Tirça, Hogla, Milka et Noa, les filles de Celofehad, épousèrent des fils de leurs oncles. 
${}^{12}Elles épousèrent donc des hommes des clans des fils de Manassé, fils de Joseph, et leur part d’héritage resta dans la tribu à laquelle appartenait le clan de leur père.
       
${}^{13}Voilà les commandements et les ordonnances que le Seigneur prescrivit aux fils d’Israël par l’intermédiaire de Moïse, dans les steppes de Moab, au bord du Jourdain, à la hauteur de Jéricho.
