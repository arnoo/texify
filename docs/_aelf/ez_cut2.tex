  
  
${}^{22}La main du Seigneur se posa sur moi. Il me dit : « Lève-toi, sors dans la vallée ; là, je te parlerai. » 
${}^{23}Je me levai et je sortis dans la vallée ; voici que la gloire du Seigneur se tenait là, pareille à la gloire que j’avais vue au bord du fleuve Kebar, et je tombai face contre terre. 
${}^{24}Alors l’esprit vint en moi, il me fit tenir debout. Il me parla et me dit : « Va t’enfermer dans ta maison. 
${}^{25}Fils d’homme, voici qu’on va te mettre des liens, des gens te ligoteront, et tu ne pourras plus sortir au milieu d’eux. 
${}^{26}Je ferai coller ta langue à ton palais ; tu seras muet ; tu ne seras plus pour eux l’homme qui leur fait des reproches – c’est une engeance de rebelles ! 
${}^{27}Mais lorsque je te parlerai, j’ouvrirai ta bouche, et tu leur diras : “Ainsi parle le Seigneur Dieu. Celui qui écoute, qu’il écoute ; celui qui n’écoute pas, qu’il n’écoute pas. C’est une engeance de rebelles !”
      
         
      \bchapter{}
      \begin{verse}
${}^{1}Et toi, fils d’homme, prends une brique, mets-la devant toi, et grave dessus une ville, Jérusalem. 
${}^{2}Mets le siège devant la ville : bâtis contre elle des retranchements, élève un remblai, installe des camps et place des béliers tout autour. 
${}^{3}Prends alors une plaque de fer, et mets-la, telle une muraille de fer, entre toi et la ville. Puis dirige ton regard vers elle. Elle sera assiégée : tu auras mis le siège devant elle. C’est un signe pour la maison d’Israël.
${}^{4}Couche-toi sur le côté gauche et prends sur toi la faute de la maison d’Israël. Autant de jours que tu seras couché, tu porteras leur faute. 
${}^{5}Et moi, je t’impose un nombre de jours égal au nombre des années de leur faute ; pendant 390 jours, tu porteras la faute de la maison d’Israël. 
${}^{6}Quand ces jours seront achevés, tu te coucheras de nouveau mais sur le côté droit, et tu porteras la faute de la maison de Juda durant quarante jours. Je te fixe un jour par année. 
${}^{7}Vers Jérusalem assiégée, tu dirigeras ton regard et ton bras dégagé ; tu prophétiseras contre elle. 
${}^{8}Voici que je te mets des liens ; tu ne te retourneras pas d’un côté sur l’autre, jusqu’à ce que tu aies achevé les jours où tu fais le siège.
${}^{9}Prends du blé, de l’orge, des fèves, des lentilles, du millet et de l’épeautre : mets-les dans un même récipient ; tu t’en feras du pain. Tu en mangeras pendant les jours où tu seras couché sur le côté, soit 390 jours. 
${}^{10}Ce sera la nourriture que tu mangeras ; tu en pèseras des rations de vingt sicles par jour ; tu les mangeras à intervalles réguliers. 
${}^{11}L’eau que tu boiras te sera mesurée : tu en boiras un sixième de hine à intervalles réguliers. 
${}^{12}Tu mangeras une galette d’orge ; tu la feras cuire sur des excréments humains, devant leurs yeux. » 
${}^{13}Et le Seigneur dit : « C’est ainsi que les fils d’Israël mangeront leur pain impur, parmi les nations où je les disperserai. » 
${}^{14}Je répondis : « Ah ! Seigneur mon Dieu, jamais je n’ai été impur ! Depuis mon enfance jusqu’à aujourd’hui, jamais je n’ai mangé d’animal crevé ou déchiré ; aucune viande immonde n’est jamais entrée dans ma bouche. » 
${}^{15}Il me dit : « Eh bien ! je t’accorde de la bouse de vache, au lieu d’excréments humains ; tu cuiras ton pain dessus. » 
${}^{16}Puis il me dit : « Fils d’homme, voici que je vais supprimer dans Jérusalem les réserves de pain : ils mangeront dans l’angoisse un pain rationné, ils boiront avec épouvante de l’eau mesurée. 
${}^{17}Ainsi, le pain et l’eau manquant, les uns et les autres seront frappés d’épouvante ; ils pourriront dans leur péché.
      
         
      \bchapter{}
      \begin{verse}
${}^{1}Et toi, fils d’homme, prends une lame tranchante ; tu t’en serviras comme d’un rasoir de barbier ; tu te raseras la tête et la barbe. Puis tu prendras une balance et tu partageras ce que tu auras coupé. 
${}^{2}Tu en brûleras un tiers par la flamme au milieu de la ville, quand seront accomplis les jours du siège. Tu prendras le deuxième tiers que tu frapperas par l’épée tout autour de la ville. Le dernier tiers, tu le disperseras au vent ; je vais tirer l’épée contre eux. 
${}^{3}Mais tu en prendras une petite quantité que tu serreras dans le pan de ton manteau. 
${}^{4}Tu en prendras encore, que tu jetteras dans le feu et que tu brûleras. De là, un feu sortira vers toute la maison d’Israël.
${}^{5}Ainsi parle le Seigneur Dieu : Voilà bien Jérusalem ! Au milieu des nations, je l’avais placée, environnée de pays étrangers. 
${}^{6}Elle s’est rebellée contre mes ordonnances avec plus de perversité que les nations, et contre mes décrets, plus que les pays qui l’entourent. Ils rejettent mes ordonnances, ils ne marchent pas selon mes décrets. 
${}^{7}C’est pourquoi, ainsi parle le Seigneur Dieu : Vos débordements sont pires que ceux des peuples qui vous entourent, car vous n’avez pas marché selon mes décrets, vous n’avez pas exécuté mes ordonnances, vous n’avez même pas exécuté les ordonnances des nations qui vous entourent. 
${}^{8}C’est pourquoi, ainsi parle le Seigneur Dieu : Me voici, à mon tour, contre toi ; chez toi, je rendrai la justice sous les yeux des nations. 
${}^{9}À cause de toutes tes abominations, j’agirai contre toi comme je n’ai jamais agi, et comme je ne le ferai jamais plus. 
${}^{10}C’est pourquoi des pères dévoreront leurs fils, au milieu de toi, et des fils dévoreront leurs pères. J’exécuterai le jugement contre toi, et tout ce qui restera de toi, je le disperserai à tout vent.
${}^{11}C’est pourquoi, par ma vie – oracle du Seigneur Dieu – puisque tu as rendu mon sanctuaire impur par toutes tes horreurs et toutes tes abominations, moi aussi, je vais tout raser ; je n’aurai pas un regard de pitié ; moi non plus, je n’épargnerai personne. 
${}^{12}Au milieu de toi, un tiers de tes habitants mourra par la peste ou sera anéanti par la famine ; autour de toi, un tiers tombera par l’épée ; le dernier tiers, je le disperserai à tous les vents, et je tirerai l’épée derrière eux. 
${}^{13}Ma colère ira jusqu’au bout, sur eux ma fureur sera assouvie et j’aurai ma revanche. Alors ils sauront que Je suis le Seigneur : j’ai parlé dans mon ardeur jalouse, allant jusqu’au bout de ma fureur contre eux.
${}^{14}Je ferai de toi une ruine, un objet de raillerie parmi les nations qui t’entourent, et aux yeux de tous les passants. 
${}^{15}Tu seras, pour les nations qui t’entourent, un objet de raillerie et de sarcasmes, une leçon, un objet de stupéfaction, quand je ferai justice de toi avec colère et fureur, et dans ma fureur je te châtierai. Je suis le Seigneur, j’ai parlé.
${}^{16}Je lancerai contre vous les flèches redoutables de la famine qui vous détruiront ; je les enverrai vous détruire. Je rendrai plus lourde sur vous la famine et je supprimerai vos réserves de pain. 
${}^{17}J’enverrai contre vous la famine et les bêtes féroces qui te priveront de tes enfants ; la peste et le sang passeront chez toi, et je ferai venir l’épée contre toi. Je suis le Seigneur, j’ai parlé. »
      
         
      \bchapter{}
      \begin{verse}
${}^{1}La parole du Seigneur me fut adressée : 
${}^{2}« Fils d’homme, dirige ton regard vers les montagnes d’Israël et prophétise contre elles. 
${}^{3}Tu diras : Montagnes d’Israël, écoutez la parole du Seigneur Dieu. Ainsi parle le Seigneur Dieu aux montagnes et aux collines, aux ravins et aux vallées. Me voici : je fais venir contre vous l’épée, je ferai disparaître vos lieux sacrés. 
${}^{4}Vos autels seront abattus, vos colonnes à encens, brisées ; je ferai tomber, devant vos idoles immondes, vos morts, transpercés. 
${}^{5}Je mettrai les cadavres des fils d’Israël devant leurs idoles immondes, et je disperserai vos ossements autour de vos autels. 
${}^{6}Partout où vous habitez, les villes seront ruinées, les lieux sacrés, dévastés, si bien que vos autels seront ruinés, voués à la malédiction, vos idoles immondes seront supprimées et disparaîtront, vos colonnes à encens seront brisées, vos ouvrages, effacés. 
${}^{7}On tombera, transpercé, au milieu de vous. Alors vous saurez que Je suis le Seigneur.
${}^{8}Mais j’en épargnerai, quand il y aura pour vous des rescapés de l’épée au milieu des nations, quand vous serez dispersés parmi les pays étrangers. 
${}^{9}Alors vos rescapés se souviendront de moi, parmi les nations où ils auront été emmenés captifs, eux dont j’aurai brisé le cœur prostitué qui s’est détourné de moi, et leurs yeux qui se prostituaient aux idoles immondes. Le dégoût leur montera au visage pour tout le mal qu’ils auront commis avec toutes leurs abominations. 
${}^{10}Alors ils sauront que Je suis le Seigneur : ce n’est pas en vain que je parle de leur infliger ces maux.
${}^{11}Ainsi parle le Seigneur Dieu : Bats des mains, tape du pied et dis : “Hélas !” pour tout le mal abominable de la maison d’Israël ; elle va tomber par l’épée, par la famine et par la peste. 
${}^{12}Qui est au loin mourra par la peste, qui est proche tombera par l’épée ; et le reste – les assiégés – mourra de faim. J’irai jusqu’au bout de ma fureur contre eux. 
${}^{13}Alors vous saurez que Je suis le Seigneur, quand leurs morts, transpercés, seront couchés au milieu de leurs idoles immondes, autour de leurs autels, sur toute colline élevée, au sommet de toute montagne, sous tout arbre verdoyant et tous les térébinthes touffus, sur les lieux mêmes où ils offraient un parfum d’apaisement à toutes leurs idoles immondes. 
${}^{14}J’étendrai la main contre eux et je ferai de ce pays une solitude désolée depuis le désert jusqu’à Ribla, partout où ils habitent. Alors ils sauront que Je suis le Seigneur. »
      
         
      \bchapter{}
${}^{1}La parole du Seigneur me fut adressée :
${}^{2}« Fils d’homme, tu diras :
        Ainsi parle le Seigneur Dieu à la terre d’Israël :
        C’est la fin ! La fin arrive aux quatre coins du pays.
${}^{3}Maintenant, c’est la fin pour toi ;
        contre toi j’enverrai ma colère,
        \\je te jugerai selon ta conduite,
        je ferai retomber sur toi toutes tes abominations.
${}^{4}Pour toi, je n’aurai pas un regard de pitié,
        je n’épargnerai personne ;
        \\je ferai retomber sur toi ta conduite,
        et tes abominations resteront au milieu de toi.
        \\Alors vous saurez que Je suis le Seigneur.
        
           
         
${}^{5}Ainsi parle le Seigneur Dieu :
        Malheur, le malheur unique, voici qu’il arrive !
${}^{6}La fin arrive, elle arrive, la fin ;
        elle s’éveille pour toi ; la voici qui arrive.
${}^{7}Le terme arrive pour toi qui habites le pays ;
        le temps arrive, le jour est proche.
        Panique, au lieu de joie, pour les montagnes !
${}^{8}Maintenant, d’ici peu, je vais déverser ma fureur contre toi ;
        j’irai jusqu’au bout de ma colère contre toi ;
        \\je te jugerai selon ta conduite
        et je ferai retomber sur toi toutes tes abominations.
${}^{9}Je n’aurai pas un regard de pitié,
        je n’épargnerai personne ;
        \\je ferai retomber sur toi ta conduite,
        et tes abominations resteront au milieu de toi.
        \\Alors vous saurez que Je suis le Seigneur, celui qui frappe.
        
           
         
${}^{10}Voici le jour.
        Voici qu’il arrive, le terme, il se met en route.
        Le gourdin fleurit, l’arrogance bourgeonne.
${}^{11}La violence s’est dressée, le gourdin du crime.
        Il ne reste rien d’eux, rien de leurs débordements,
        rien de leur grondement ;
        plus de répit pour eux.
${}^{12}Le temps arrive, le jour est tout proche.
        Que l’acheteur ne se réjouisse pas,
        que le vendeur ne se désole pas !
        \\Contre tout débordement, l’ardeur de la colère !
${}^{13}Celui qui vend ne reviendra plus à ses ventes,
        aussi longtemps qu’il vivra ;
        la vision sur les débordements ne reviendra plus.
        \\Chacun vivra dans sa faute,
        sans reprendre vigueur.
${}^{14}On sonnera de la trompette,
        tout sera prêt, et personne pour aller au combat.
        \\Contre tout débordement, l’ardeur de la colère !
${}^{15}Dans les rues, l’épée ;
        à la maison, peste et famine.
        \\Qui est aux champs mourra par l’épée ;
        qui est en ville, famine et peste le dévoreront.
${}^{16}Les rescapés s’échapperont ;
        \\ils iront vers les montagnes, tels les colombes des vallées,
        tous gémissant, chacun par sa faute.
${}^{17}Toutes les mains faibliront,
        tous les genoux fondront en eau.
${}^{18}Ils se revêtiront de toile à sac,
        un frisson les saisira.
        \\Sur tous les visages, la honte,
        toutes les têtes seront rasées.
${}^{19}Ils jetteront leur argent dans les rues,
        leur or sera pour eux une souillure.
        \\Leur argent ni leur or ne pourront les sauver
        au jour des fureurs du Seigneur.
        \\Leurs gosiers ne seront plus rassasiés,
        ils ne rempliront plus leur ventre.
        \\Argent et or les ont fait trébucher dans la faute.
${}^{20}Ils mettaient leur orgueil dans leur splendide parure,
        ils en ont fait leurs images abominables, leurs horreurs ;
        c’est pourquoi j’en ferai leur souillure.
${}^{21}Tout cela, je le livrerai au pillage entre les mains des étrangers,
        en butin, aux méchants du pays ;
        ils le profaneront.
${}^{22}Je détournerai d’eux mon visage,
        on profanera mon trésor.
        \\Des brigands y viendront
        et le profaneront.
${}^{23}Fabrique une chaîne :
        le pays est rempli de jugements sanguinaires,
        la ville est remplie de violence.
${}^{24}Je ferai venir les pires des nations,
        elles s’empareront des maisons.
        \\Je ferai cesser l’orgueil des puissants
        et leurs sanctuaires seront profanés.
${}^{25}L’angoisse arrive ;
        ils chercheront la paix : rien !
${}^{26}Il arrivera désastre sur désastre,
        mauvaise nouvelle sur mauvaise nouvelle ;
        \\ils quémanderont une vision au prophète ;
        l’instruction manquera au prêtre,
        le conseil, aux anciens.
${}^{27}Le roi portera le deuil,
        le prince se vêtira de désolation,
        les mains des gens du pays trembleront.
        \\J’agirai envers eux d’après leur conduite,
        je les jugerai selon leurs jugements.
        \\Alors ils sauront que Je suis le Seigneur. »
        
           
      
         
      \bchapter{}
      \begin{verse}
${}^{1}La sixième année de la première déportation, le sixième mois, le cinq du mois, j’étais assis dans ma maison, et les anciens de Juda étaient assis devant moi ; là s’abattit sur moi la main du Seigneur Dieu.
${}^{2}J’ai vu : il y avait quelqu’un qui avait l’aspect d’un homme. À partir de ce qui semblait être ses reins et au-dessous, c’était du feu ; à partir de ses reins et au-dessus, c’était une sorte d’éclat, comme un scintillement de vermeil. 
${}^{3}Il étendit comme une main et me saisit par une mèche de cheveux. L’esprit me souleva entre ciel et terre. Il m’emmena jusqu’à Jérusalem, en des visions divines, à l’entrée de la porte intérieure, celle qui est tournée vers le nord ; là se trouve le siège de l’idole de la jalousie, qui provoque l’ardeur jalouse de Dieu. 
${}^{4}Voici que la gloire du Dieu d’Israël était là, pareille à la vision que j’avais eue dans la vallée.
${}^{5}Il me dit : « Fils d’homme, lève donc les yeux en direction du nord. » Je levai les yeux en direction du nord, et voici : au nord de la porte, il y avait un autel ; dans le passage se trouvait cette idole de la jalousie. 
${}^{6}Il me dit : « Fils d’homme, vois-tu ce qu’ils font ? Vois-tu les grandes abominations que la maison d’Israël commet ici pour m’éloigner de mon sanctuaire ? Tu vas voir encore de grandes abominations ! »
${}^{7}Il m’emmena jusqu’à la porte de la cour. Je regardai : il y avait un trou dans le mur. 
${}^{8}Il me dit : « Fils d’homme, perce donc le mur. » Je perçai le mur : il y eut alors une ouverture. 
${}^{9}Il me dit : « Entre, et regarde les affreuses abominations qu’ils commettent ici. » 
${}^{10}J’entrai et je regardai : il y avait là quantité d’images de reptiles et de bêtes – une horreur ! – et toutes les idoles immondes de la maison d’Israël, gravées tout autour sur le mur. 
${}^{11}Soixante-dix hommes, des anciens de la maison d’Israël, se tenaient debout devant elles – Yaazanyahou, fils de Shafane, était parmi eux. Chacun avait son encensoir à la main, et le parfum d’un nuage d’encens montait. 
${}^{12}Il me dit : « Tu as vu, fils d’homme, ce que font les anciens de la maison d’Israël dans l’obscurité, chacun dans la pièce où se trouve l’effigie de son idole ? Ils disent : “Le Seigneur ne peut pas nous voir ; le Seigneur a abandonné le pays.” » 
${}^{13}Et il me dit : « Tu verras encore les grandes abominations qu’ils commettent. »
${}^{14}Il m’emmena à l’entrée de la porte de la maison du Seigneur, celle qui est tournée vers le nord : voici que des femmes étaient assises là, pleurant la déesse Tammouz. 
${}^{15}Il me dit : « Tu as vu, fils d’homme ? Tu verras encore des abominations plus grandes que celles-ci. »
${}^{16}Il m’emmena vers la cour intérieure de la maison du Seigneur : voici qu’à l’entrée du sanctuaire du Seigneur, entre le Vestibule et l’autel, il y avait environ vingt-cinq hommes tournant le dos au sanctuaire du Seigneur, et le visage vers l’orient. Ils se prosternaient en direction de l’orient, vers le soleil. 
${}^{17}Et il me dit : « Tu as vu, fils d’homme ? Est-ce trop peu pour la maison de Juda de commettre les abominations qu’ils commettent ici ? Oui, ils remplissent le pays de violence, ils provoquent encore mon indignation : les voici qui élèvent le rameau jusqu’à leur nez ! 
${}^{18}À mon tour d’agir avec fureur ; je n’aurai pas un regard de pitié, je n’épargnerai personne. Ils auront beau crier à mes oreilles d’une voix forte, je ne les écouterai pas. »
      
         
      \bchapter{}
      \begin{verse}
${}^{1}J’entendis le Seigneur Dieu me crier d’une voix forte\\ :
      \begin{verse}« Ils sont tout proches, les châtiments de Jérusalem\\, et chacun tient à la main son arme de mort. » 
${}^{2} Alors six hommes s’avancèrent, venant de la porte supérieure, celle qui est du côté nord. Chacun tenait à la main son arme de destruction. Au milieu d’eux, un homme, vêtu de lin\\, portant à la ceinture une écritoire de scribe. Ils s’avancèrent, et s’arrêtèrent près de l’autel de bronze. 
${}^{3} La gloire du Dieu d’Israël s’éleva au-dessus des Kéroubim où elle reposait, et se dirigea vers le seuil de la maison du Seigneur\\. Alors le Seigneur appela l’homme vêtu de lin, portant à la ceinture une écritoire de scribe. 
${}^{4} Il lui dit : « Passe à travers la ville, à travers Jérusalem, et marque d’une croix\\au front\\ceux qui gémissent et qui se lamentent sur toutes les abominations qu’on y commet. » 
${}^{5} Puis j’entendis le Seigneur dire aux autres\\ : « Passez derrière lui à travers la ville, et frappez. N’ayez pas un regard de pitié, n’épargnez personne : 
${}^{6} vieillards et jeunes gens, jeunes filles, enfants, femmes, tuez-les, exterminez-les. Mais tous ceux qui sont marqués au front\\, ne les touchez pas\\. Commencez l’extermination par mon sanctuaire. » Ils commencèrent donc par les vieillards qui adoraient les idoles\\à l’entrée de la maison du Seigneur\\. 
${}^{7} Le Seigneur ajouta : « Rendez impure cette Maison, emplissez les cours de cadavres, puis sortez ! » Ils sortirent donc et frappèrent à travers la ville.
${}^{8}Or pendant qu’ils frappaient, j’étais resté seul ; je tombai face contre terre et je criai : « Ah ! Seigneur Dieu, vas-tu exterminer tout ce qui reste d’Israël en déversant ta fureur sur Jérusalem ? » 
${}^{9}Il me dit : « La faute de la maison d’Israël et de Juda est grande, immense. Le pays est rempli de sang, la ville est pleine de perversité. Ils disent : “Le Seigneur a abandonné le pays, le Seigneur ne voit plus rien.” 
${}^{10}Eh bien ! moi, je n’aurai pas un regard de pitié, je n’épargnerai personne. Je ferai retomber leur conduite sur leur tête. » 
${}^{11}Et voici que l’homme vêtu de lin, portant à la ceinture une écritoire, revint et rendit compte en ces termes : « J’ai fait comme tu me l’avais ordonné. »
      
         
      \bchapter{}
      \begin{verse}
${}^{1}J’ai vu : sur le firmament qui était au-dessus de la tête des Kéroubim, on voyait, apparaissant sur eux, comme une pierre de saphir dont la forme ressemblait à celle d’un trône. 
${}^{2}Le Seigneur dit à l’homme vêtu de lin : « Entre par les espaces du cercle sous le Kéroub ; prends à pleines mains des braises ardentes par les espaces entre les Kéroubim, et répands-les sur la ville. » L’homme y entra sous mes yeux.
${}^{3}Les Kéroubim se tenaient à droite de la Maison lorsque l’homme entra, et la nuée remplissait la cour intérieure. 
${}^{4}La gloire du Seigneur s’éleva de dessus le Kéroub vers le seuil de la Maison ; la Maison fut remplie de la nuée tandis que la cour était remplie par l’éclat de la gloire du Seigneur. 
${}^{5}Le bruit des ailes des Kéroubim s’entendait jusque dans la cour extérieure, comme la voix du Dieu-Puissant lorsqu’il parle. 
${}^{6}Quand il ordonna à l’homme vêtu de lin : « Prends du feu par les espaces du cercle, par les espaces entre les Kéroubim », l’homme y entra, il se tint près de la roue. 
${}^{7}Le Kéroub étendit la main, par les espaces entre les Kéroubim, vers le feu qui est dans les espaces entre les Kéroubim ; il en préleva et le mit dans la main de l’homme vêtu de lin. Celui-ci le saisit et sortit. 
${}^{8}Alors apparut sous les ailes des Kéroubim comme une main humaine.
${}^{9}J’ai vu : quatre roues près des Kéroubim, une roue près de chaque Kéroub. Les roues scintillaient comme une pierre de chrysolithe. 
${}^{10}Toutes les quatre paraissaient avoir même forme ; elles étaient comme imbriquées l’une dans l’autre. 
${}^{11}Quand ils avançaient, ils allaient dans les quatre directions. Ils avançaient sans s’écarter, car ils avançaient du côté où était dirigée la tête. Ils avançaient sans s’écarter. 
${}^{12}Tout leur corps, leur dos, leurs mains, leurs ailes, ainsi que les roues, étaient remplis de scintillements, autour des quatre roues. 
${}^{13}À ces roues fut donné le nom de Galgal (c’est-à-dire : cercle) ; je l’ai entendu. 
${}^{14}Ils avaient chacun quatre faces : la première était une face de Kéroub ; la deuxième, une face d’homme ; la troisième, une face de lion ; la quatrième, une face d’aigle. 
${}^{15}Les Kéroubim s’élevèrent – c’était le Vivant que j’avais vu près du fleuve Kebar. 
${}^{16}Quand les Kéroubim avançaient, les roues avançaient à côté d’eux ; quand les Kéroubim déployaient leurs ailes pour s’élever de terre, les roues ne s’écartaient pas mais demeuraient à côté d’eux. 
${}^{17}Quand ils s’arrêtaient, elles s’arrêtaient ; quand ils s’élevaient, elles s’élevaient avec eux : l’esprit du Vivant était en elles !
${}^{18}La gloire du Seigneur quitta le seuil de la Maison et s’arrêta au-dessus des Kéroubim. 
${}^{19} Ceux-ci déployèrent leurs ailes ; je les vis partir en s’élevant de terre, et les roues avec eux. Ils s’arrêtèrent à l’entrée de la porte orientale de la maison du Seigneur ; la gloire du Dieu d’Israël était au-dessus d’eux. 
${}^{20} C’étaient les Vivants\\que j’avais vus au-dessous du Dieu d’Israël, près du fleuve Kebar, et je reconnus que c’étaient des Kéroubim. 
${}^{21} Chacun avait quatre faces et quatre ailes, et une forme de mains humaines sous ses ailes. 
${}^{22} Leurs faces étaient semblables aux faces que j’avais vues près du fleuve Kebar ; tel était leur aspect. Chacun allait droit devant lui.
      
         
      \bchapter{}
      \begin{verse}
${}^{1}L’esprit me souleva et m’emmena jusqu’à la porte orientale de la maison du Seigneur, celle qui fait face à l’orient. À l’entrée de la porte, il y avait vingt-cinq hommes, parmi lesquels je vis Yaazanya, fils d’Azzour, et Pelatyahou, fils de Benayahou, des princes du peuple. 
${}^{2}Il me dit : « Fils d’homme, ces hommes-là préparent leur mauvais coup, conseillent le mal dans cette ville : 
${}^{3}“On n’est pas près de bâtir des maisons, disent-ils ; voici la marmite, et nous sommes la viande !” 
${}^{4}C’est pourquoi, prophétise contre eux, prophétise, fils d’homme ! »
${}^{5}L’esprit du Seigneur tomba sur moi et il me dit : « Parle ! Ainsi parle le Seigneur : C’est ainsi que vous avez parlé, maison d’Israël ; ce qui vous monte à l’esprit, moi je le connais. 
${}^{6}Elles sont nombreuses dans cette ville, vos victimes ; elles remplissent les rues. 
${}^{7}C’est pourquoi, ainsi parle le Seigneur Dieu : Vos victimes, que vous avez mises au milieu de la ville, ce sont elles, la viande, et la ville, c’est la marmite. Quant à vous, je vous en ferai sortir. 
${}^{8}L’épée, vous la craignez ? L’épée, je la ferai venir contre vous – oracle du Seigneur Dieu. 
${}^{9}Je vous ferai sortir de la ville, je vous livrerai aux mains des étrangers ; de vous, je ferai justice. 
${}^{10}Vous tomberez par l’épée sur le territoire d’Israël, je vous jugerai. Alors vous saurez que Je suis le Seigneur. 
${}^{11}Cette ville ne sera pas pour vous une marmite, et dedans vous ne serez pas comme de la viande : c’est sur le territoire d’Israël que je vous jugerai. 
${}^{12}Alors vous saurez que Je suis le Seigneur, dont vous n’avez pas suivi les décrets ni observé les coutumes, mais vous avez observé les coutumes des peuples qui vous entourent. »
${}^{13}Or, comme je prophétisais, le fils de Benayahou, Pelatyahou (c’est-à-dire : Rescapé-de-Dieu) mourut. Je tombai face contre terre et criai d’une voix forte : « Ah ! Seigneur Dieu, vas-tu exterminer le reste d’Israël ? »
${}^{14}Alors la parole du Seigneur me fut adressée : 
${}^{15}« Fils d’homme, c’est à chacun de tes frères, à tes parents et à toute la maison d’Israël que les habitants de Jérusalem disent : “Restez loin du Seigneur, c’est à nous que le pays fut donné en possession.” 
${}^{16}C’est pourquoi tu diras : Ainsi parle le Seigneur Dieu : Oui, je les ai éloignés parmi les nations ; oui, je les ai dispersés dans les pays étrangers. Mais j’ai été pour eux comme un sanctuaire, dans les pays où ils sont allés. 
${}^{17}C’est pourquoi tu diras : Ainsi parle le Seigneur Dieu : Je vous rassemblerai du milieu des peuples, je vous réunirai de tous les pays où vous avez été dispersés ; puis je vous donnerai la terre d’Israël. 
${}^{18}Ils y entreront, ils en supprimeront toutes les horreurs et toutes les abominations.
${}^{19}Je leur donnerai un cœur loyal, je mettrai en eux un esprit nouveau : j’enlèverai de leur chair le cœur de pierre, et je leur donnerai un cœur de chair, 
${}^{20}afin qu’ils suivent mes décrets, qu’ils gardent mes coutumes et qu’ils les observent. Alors ils seront mon peuple, et moi je serai leur Dieu. 
${}^{21}Quant à ceux dont le cœur s’est attaché aux horreurs et aux abominations, je ferai retomber leur conduite sur leur tête – oracle du Seigneur Dieu. »
${}^{22}Alors les Kéroubim levèrent leurs ailes, les roues auprès d’eux ; la gloire du Dieu d’Israël était au-dessus d’eux. 
${}^{23}La gloire du Seigneur s’éleva du milieu de la ville et s’arrêta sur la montagne qui est à l’est de la ville. 
${}^{24}L’esprit me souleva et m’emmena chez les Chaldéens, vers les exilés ; ce fut en vision, dans l’esprit de Dieu. Et au-dessus de moi s’éleva la vision que j’avais vue. 
${}^{25}Je racontai aux exilés tout ce que le Seigneur m’avait fait voir.
      
         
      \bchapter{}
      \begin{verse}
${}^{1}La parole du Seigneur me fut adressée : 
${}^{2} « Fils d’homme, tu habites au milieu d’une engeance de rebelles\\ ; ils ont des yeux pour voir, et ne voient pas ; des oreilles pour entendre, et n’entendent pas, car c’est une engeance de rebelles. 
${}^{3} Toi, fils d’homme, prépare-toi un sac d’exilé ; sous leurs yeux, pars en plein jour, comme un exilé ; sous leurs yeux, pars de ta maison vers un autre lieu ; peut-être verront-ils qu’ils sont une engeance de rebelles. 
${}^{4} Tu sortiras ton sac, comme un sac d’exilé, en plein jour, sous leurs yeux. Toi-même, tu sortiras le soir, sous leurs yeux, comme s’en vont les exilés. 
${}^{5} Sous leurs yeux, tu feras un trou dans le mur, et tu sortiras\\par là. 
${}^{6} Sous leurs yeux, tu chargeras ton sac\\sur ton épaule, et tu le sortiras dans l’obscurité ; tu voileras ton visage, et tu ne verras plus le pays : j’ai fait de toi un signe\\pour la maison d’Israël. »
${}^{7}Je fis ce qui m’avait été ordonné : en plein jour, je sortis mon sac, comme un sac d’exilé ; puis le soir, je fis un trou dans le mur, à la main ; je sortis mon sac\\dans l’obscurité, et sous leurs yeux je le chargeai sur mon épaule.
       
${}^{8}Au matin, la parole du Seigneur me fut adressée : 
${}^{9}« Fils d’homme, la maison d’Israël, cette engeance de rebelles, t’a bien demandé : “Qu’est-ce que tu fais là ?” 
${}^{10}Réponds : Ainsi parle le Seigneur Dieu : Cet oracle concerne le prince qui est à Jérusalem et toute la maison d’Israël qui s’y trouve. 
${}^{11}Tu diras : Je suis pour vous un signe. Ce que j’ai fait, c’est cela même qui leur sera fait : ils partiront en exil, en captivité ; 
${}^{12}le prince qui est au milieu d’eux chargera son sac sur son épaule, il sortira dans l’obscurité\\ ; on percera le mur pour le faire sortir ; il voilera son visage, si bien qu’il ne verra plus de ses yeux le pays. 
${}^{13}J’étendrai sur lui mon filet, et il sera pris au piège. Je l’amènerai à Babylone, au pays des Chaldéens ; il ne verra pas ce pays, il y mourra. 
${}^{14}Tout son entourage, sa garde et tous ses bataillons, je les disséminerai à tous les vents, et je tirerai l’épée derrière eux. 
${}^{15}Alors ils sauront que Je suis le Seigneur, quand je les aurai dispersés parmi les nations, et disséminés dans les pays étrangers. 
${}^{16}Cependant je laisserai un certain nombre d’entre eux qui échapperont à l’épée, à la famine et à la peste, pour raconter parmi les nations où ils se rendront toutes leurs abominations, afin qu’elles sachent que Je suis le Seigneur. »
${}^{17}La parole du Seigneur me fut adressée : 
${}^{18}« Fils d’homme, tu mangeras ton pain en tremblant, tu boiras ton eau dans l’inquiétude et l’angoisse. 
${}^{19}Tu diras aux gens du pays : Ainsi parle le Seigneur Dieu au sujet des habitants de Jérusalem qui sont sur le sol d’Israël : ils mangeront leur pain dans l’angoisse, ils boiront leur eau dans l’épouvante ; ainsi leur terre sera dévastée, privée de ce qui l’emplit, privée de la violence de tous ses habitants. 
${}^{20}Les villes habitées seront détruites, le pays deviendra un lieu désolé. Alors vous saurez que Je suis le Seigneur. »
${}^{21}La parole du Seigneur me fut adressée : 
${}^{22}« Fils d’homme, que voulez-vous dire en appliquant à la terre d’Israël ce proverbe : “Plus se prolongent les jours, moins s’accomplissent les visions” ? 
${}^{23}Eh bien ! tu diras : Ainsi parle le Seigneur Dieu : Je mettrai fin à ce proverbe ; on ne le répétera plus en Israël. À l’inverse, tu leur diras : Les jours approchent où toute vision s’accomplira. 
${}^{24}Il n’y aura plus ni visions illusoires ni présages trompeurs, au milieu de la maison d’Israël. 
${}^{25}Je suis le Seigneur, je parlerai. La parole que je dirai se réalisera, et sans délai ! C’est de vos jours, engeance de rebelles, que je réaliserai la parole que je vais dire – oracle du Seigneur Dieu. »
${}^{26}La parole du Seigneur me fut adressée : 
${}^{27}« Fils d’homme, voici ce que dit la maison d’Israël : “La vision de ce voyant n’est pas pour demain ; il prophétise pour des temps éloignés.” 
${}^{28}C’est pourquoi tu leur diras : Ainsi parle le Seigneur Dieu : Il n’y a plus de délai pour aucune de mes paroles. La parole que je dirai va se réaliser – oracle du Seigneur Dieu. »
      
         
      \bchapter{}
      \begin{verse}
${}^{1}La parole du Seigneur me fut adressée : 
${}^{2}« Fils d’homme, prophétise contre les prophètes d’Israël, prophétise. Tu diras à ceux qui prophétisent de leur propre initiative : Écoutez la parole du Seigneur ! 
${}^{3}Ainsi parle le Seigneur Dieu : Quel malheur pour les prophètes insensés qui suivent leur propre inspiration sans avoir rien vu ! 
${}^{4}Des chacals dans des ruines, tels furent tes prophètes, Israël. 
${}^{5}Vous n’êtes pas montés sur les brèches, vous n’avez pas construit de mur autour de la maison d’Israël pour qu’elle puisse tenir ferme dans le combat, au jour du Seigneur. 
${}^{6}Ils ont des visions illusoires, et des présages trompeurs, eux qui disent “Oracle du Seigneur”, sans que le Seigneur les ait envoyés ; et ils attendent la confirmation de leur parole ! 
${}^{7}Ne sont-elles pas illusoires, vos visions, et mensongers, vos présages, quand vous dites “Oracle du Seigneur”, alors que moi, je n’ai pas parlé ? 
${}^{8}C’est pourquoi ainsi parle le Seigneur Dieu : Ce que vous dites est illusoire, ce que vous voyez est mensonge. C’est pourquoi me voici contre vous – oracle du Seigneur Dieu. 
${}^{9}Ma main pèsera sur les prophètes aux visions illusoires et aux présages mensongers : ils ne seront pas admis au conseil de mon peuple ; ils ne seront pas inscrits dans le livre de la maison d’Israël ; ils ne pénétreront pas sur le sol d’Israël. Alors vous saurez que Je suis le Seigneur Dieu.
${}^{10}Ils ont égaré mon peuple en disant “Paix !”, alors qu’il n’y a pas de paix, et quand mon peuple bâtit une paroi, eux l’enduisent de badigeon. 
${}^{11}À cause de cela, dis aux badigeonneurs : Viendra une pluie torrentielle, tomberont des grêlons, se déchaînera un vent de tempête, 
${}^{12}et voilà le mur abattu ! Ne vous dira-t-on pas : “Où est le badigeon dont vous l’avez enduit ?” 
${}^{13}C’est pourquoi, ainsi parle le Seigneur Dieu : Je vais déchaîner un vent de tempête dans ma fureur, il y aura une pluie torrentielle dans ma colère, et des grêlons dans ma fureur, pour détruire. 
${}^{14}J’abattrai le mur que vous avez enduit de badigeon, je le jetterai à terre et ses fondations seront mises à nu. Il tombera et vous périrez dessous. Alors vous saurez que Je suis le Seigneur. 
${}^{15}J’irai jusqu’au bout de ma fureur contre le mur et contre ceux qui le badigeonnent. Je vous dirai : Plus de mur ! Plus de badigeonneurs, 
${}^{16}ces prophètes d’Israël qui prophétisent sur Jérusalem et qui ont pour elle une vision de paix, alors qu’il n’y a pas de paix – oracle du Seigneur Dieu !
${}^{17}Et toi, fils d’homme, dirige ton regard vers les filles de ton peuple qui prophétisent de leur propre initiative, et prophétise contre elles. 
${}^{18}Tu diras : Ainsi parle le Seigneur Dieu : Quel malheur pour celles qui cousent des cordelettes à tous les poignets, qui fabriquent des voiles pour les têtes de diverses tailles, afin de capturer des vies ! Vous capturez la vie des gens de mon peuple, et voulez conserver la vôtre ? 
${}^{19}Vous me profanez devant mon peuple pour quelques poignées d’orge et quelques morceaux de pain, en faisant mourir des gens qui ne doivent pas mourir, et en faisant vivre ceux qui ne doivent pas vivre ; ainsi vous mentez à mon peuple qui écoute le mensonge. 
${}^{20}C’est pourquoi, ainsi parle le Seigneur Dieu : Voici, je m’en prends à vos cordelettes, avec lesquelles vous capturez les vies comme des oiseaux. Je les déchirerai sur vos bras, et je libérerai les vies que vous avez capturées comme des oiseaux. 
${}^{21}Je déchirerai vos voiles, j’arracherai mon peuple à vos mains, pour qu’il ne soit plus une proie dans vos mains. Alors vous saurez que Je suis le Seigneur.
${}^{22}Vous avez troublé le cœur du juste, alors que moi je ne l’avais pas inquiété ; vous avez fortifié les mains du méchant pour qu’il ne renonce pas à sa mauvaise conduite afin de retrouver la vie ; 
${}^{23}à cause de cela, vous n’aurez plus de visions illusoires, et ne ferez plus de présages. J’arracherai mon peuple à vos mains. Alors vous saurez que Je suis le Seigneur. »
      
         
      \bchapter{}
      \begin{verse}
${}^{1}Quelques-uns des anciens d’Israël vinrent me trouver et s’assirent devant moi. 
${}^{2}La parole du Seigneur me fut adressée : 
${}^{3}« Fils d’homme, ces gens-là ont exalté en leur cœur leurs idoles immondes ; ils placent devant eux ce qui les fait trébucher. Vais-je me laisser consulter par eux ? 
${}^{4}Parle donc. Tu leur diras : Ainsi parle le Seigneur Dieu : Tout homme de la maison d’Israël qui exalte en son cœur ses idoles immondes, ou qui place devant lui ce qui le fait trébucher, s’il vient ensuite trouver le prophète, moi, le Seigneur, je lui répondrai en personne, d’après le nombre de ses idoles immondes, 
${}^{5}afin de saisir au cœur la maison d’Israël ; à mon égard, tous sont devenus des étrangers à cause de leurs idoles immondes. 
${}^{6}C’est pourquoi tu diras à la maison d’Israël : Ainsi parle le Seigneur Dieu : Retournez-vous ! Détournez-vous de vos idoles immondes ; de toutes vos abominations détournez vos visages !
${}^{7}Soit un homme de la maison d’Israël, ou un immigré établi en Israël : il se sépare de moi, il exalte en son cœur ses idoles immondes, il place devant lui ce qui le fait trébucher et il vient demander au prophète de me consulter pour lui. Moi, le Seigneur, je lui répondrai en personne. 
${}^{8}Je porterai mon regard sur cet homme, j’en ferai un exemple proverbial, je le retrancherai de mon peuple. Alors vous saurez que Je suis le Seigneur. 
${}^{9}Et le prophète, s’il se laisse séduire et prononce une parole, c’est moi, le Seigneur, qui l’aurai séduit ; j’étendrai la main contre lui et je le supprimerai du milieu de mon peuple Israël. 
${}^{10}Ils porteront leur faute, la faute de celui qui consulte comme la faute du prophète, 
${}^{11}pour que la maison d’Israël ne s’égare plus loin de moi et ne soit plus rendue impure par tous ses crimes. Alors ils seront mon peuple, et moi je serai leur Dieu – oracle du Seigneur Dieu. »
${}^{12}La parole du Seigneur me fut adressée : 
${}^{13}« Fils d’homme, si un pays péchait contre moi en commettant l’infidélité, si j’étendais la main contre lui, détruisais sa réserve de pain, lui envoyais la famine, en retranchais hommes et bêtes, 
${}^{14}et si dans ce pays il y avait ces trois hommes, Noé, Daniel et Job, alors eux, par leur justice, sauveraient leur vie – oracle du Seigneur Dieu.
${}^{15}Si je faisais passer des bêtes féroces dans ce pays pour le priver de ses enfants et qu’il devienne un lieu désolé où nul ne passe, à cause de ces bêtes, 
${}^{16}si dans ce pays il y avait ces trois hommes, alors, par ma vie – oracle du Seigneur Dieu – ils ne pourraient sauver ni fils ni filles : eux seuls seraient sauvés, et le pays deviendrait un lieu désolé.
${}^{17}Ou bien, si je faisais venir l’épée contre ce pays, si je disais : “Que passe l’épée dans ce pays, qu’elle en retranche hommes et bêtes”, 
${}^{18}et si dans ce pays il y avait ces trois hommes, alors, par ma vie – oracle du Seigneur Dieu – ils ne pourraient sauver ni fils ni filles : eux seuls seraient sauvés.
${}^{19}Ou bien, si j’envoyais la peste contre ce pays, si je déversais ma fureur contre lui pour en retrancher dans le sang hommes et bêtes, 
${}^{20}et si dans ce pays il y avait Noé, Daniel et Job, alors, par ma vie – oracle du Seigneur Dieu – ils ne pourraient sauver aucun fils ni aucune fille, mais eux, par leur justice, sauveraient leur vie. »
${}^{21}Ainsi parle le Seigneur Dieu : « Même si j’ai envoyé mes quatre terribles châtiments, épée, famine, bêtes féroces et peste, contre Jérusalem pour en retrancher hommes et bêtes, 
${}^{22}voilà qu’en elle subsiste un reste. On a fait sortir de la ville des fils et des filles ; les voici qui sortent vers vous. Vous pourrez voir leur conduite et leurs actes ; alors vous vous consolerez du malheur que j’ai fait venir sur Jérusalem, de tout ce que j’ai fait venir contre elle. 
${}^{23}Ils vous consoleront, car vous verrez leur conduite et leurs actes. Alors vous saurez que ce n’est pas en vain que j’ai accompli tout ce que j’ai fait dans la ville – oracle du Seigneur Dieu. »
      
         
      \bchapter{}
      \begin{verse}
${}^{1}La parole du Seigneur me fut adressée : 
${}^{2}« Fils d’homme, pour quelle raison le bois de la vigne vaudrait-il mieux que tous les autres bois ? Pourquoi ses branches seraient-elles meilleures que celles des arbres de la forêt ?
${}^{3}En tire-t-on du bois
        pour en faire un ouvrage ?
        \\En tire-t-on une cheville
        pour y suspendre un objet ?
${}^{4}Voilà qu’on le jette au feu pour le consumer :
        le feu consume ses deux extrémités,
        \\le milieu est brûlé ;
        peut-il servir à quelque ouvrage ?
${}^{5}Déjà, lorsqu’il était intact,
        on n’en faisait nul ouvrage ;
        \\une fois que le feu l’a consumé et brûlé,
        pourrait-on encore en faire quelque ouvrage ?
${}^{6}C’est pourquoi, ainsi parle le Seigneur Dieu :
        Comme je jette au feu le bois de la vigne
        \\pour le consumer, de préférence aux bois de la forêt,
        ainsi je jette au feu les habitants de Jérusalem.
${}^{7}Je porte mon regard sur eux.
        Ils sont sortis du feu, mais le feu les consumera.
        \\Alors vous saurez que Je suis le Seigneur,
        lorsque j’aurai posé mon regard sur eux.
${}^{8}Je ferai de ce pays un lieu désolé
        à cause de l’infidélité qu’ils ont commise
        – oracle du Seigneur Dieu. »
      
         
      \bchapter{}
      \begin{verse}
${}^{1}La parole du Seigneur me fut adressée : 
${}^{2} « Fils d’homme, fais connaître à Jérusalem ses abominations. 
${}^{3} Tu diras : Ainsi parle le Seigneur Dieu à Jérusalem : Par tes origines et ta naissance, tu es du pays de Canaan. Ton père était un Amorite\\, et ta mère, une Hittite\\. 
${}^{4} À ta naissance, le jour où tu es née, on ne t’a pas coupé le cordon, on ne t’a pas plongée dans l’eau pour te nettoyer, on ne t’a pas frottée de sel, ni enveloppée de langes. 
${}^{5} Aucun regard de pitié pour toi, personne\\pour te donner le moindre de ces soins, par compassion. On t’a jetée en plein champ, avec dégoût\\, le jour de ta naissance.
${}^{6}Je suis passé près de toi, et je t’ai vue te débattre dans ton sang. Quand tu étais dans ton sang, je t’ai dit : “Je veux que tu vives\\ !” 
${}^{7} Je t’ai fait croître comme l’herbe des champs. Tu as poussé, tu as grandi, tu es devenue femme\\, ta poitrine s’est formée, ta chevelure s’est développée. Mais tu étais complètement nue.
${}^{8}Je suis passé près de toi, et je t’ai vue : tu avais atteint l’âge des amours. J’étendis sur toi le pan de mon manteau\\et je couvris ta nudité. Je me suis engagé envers toi par serment, je suis entré en alliance avec toi – oracle du Seigneur Dieu – et tu as été à moi. 
${}^{9} Je t’ai plongée dans l’eau, je t’ai nettoyée de ton sang, je t’ai parfumée avec de l’huile. 
${}^{10} Je t’ai revêtue d’habits chamarrés, je t’ai chaussée de souliers en cuir fin, je t’ai donné une ceinture de lin précieux\\, je t’ai couverte de soie. 
${}^{11} Je t’ai parée de joyaux : des bracelets à tes poignets, un collier à ton cou, 
${}^{12} un anneau à ton nez, des boucles à tes oreilles, et sur ta tête un diadème magnifique. 
${}^{13} Tu étais parée d’or et d’argent, vêtue de lin précieux, de soie et d’étoffes chamarrées. La fleur de farine, le miel et l’huile étaient ta nourriture. Tu devins de plus en plus belle et digne de la royauté. 
${}^{14} Ta renommée se répandit parmi les nations, à cause de ta beauté, car elle était parfaite, grâce à ma splendeur dont je t’avais revêtue\\ – oracle du Seigneur Dieu.
${}^{15}Mais tu t’es fiée à ta beauté, tu t’es prostituée en usant de ta renommée, tu as prodigué tes faveurs à tout passant : tu as été à n’importe qui\\. 
${}^{16}Tu as pris de tes vêtements, tu as fait des lieux sacrés aux riches couleurs, et tu t’y es prostituée – cela ne s’était jamais fait et ne sera plus. 
${}^{17}Tu as pris tes bijoux d’or et d’argent que je t’avais donnés ; tu t’es fabriqué des idoles masculines, tu t’es prostituée avec elles. 
${}^{18}Tu as pris tes vêtements chamarrés et tu les en as recouvertes. Mon huile et mon encens, tu les as déposés devant elles. 
${}^{19}Mon pain que je t’avais donné, la fleur de farine, l’huile et le miel dont je te nourrissais, tu les as déposés devant elles, en parfum d’apaisement. Il en fut ainsi – oracle du Seigneur Dieu. 
${}^{20}Tu as pris tes fils et tes filles que tu m’avais enfantés, et tu les as sacrifiés pour qu’elles s’en nourrissent. Était-ce donc trop peu que ta prostitution ? 
${}^{21}Tu as égorgé mes fils et tu les as livrés en les faisant passer par le feu pour elles. 
${}^{22}Dans toutes tes abominations et tes prostitutions, tu ne t’es pas souvenue des jours de ta jeunesse, quand tu étais complètement nue et que tu te débattais dans ton sang.
${}^{23}Et après tant de méchanceté – quel malheur, quel malheur pour toi ! oracle du Seigneur Dieu –, 
${}^{24}tu t’es bâti un podium, tu t’es fait une estrade sur toutes les places. 
${}^{25}À l’entrée de chaque chemin, tu t’es construit une estrade pour y souiller ta beauté et livrer ton corps à tout passant ; ainsi tu as multiplié tes prostitutions. 
${}^{26}Tu t’es prostituée aux fils de l’Égypte, tes voisins au corps puissant ; tu as multiplié tes prostitutions au point de provoquer mon indignation. 
${}^{27}Voici donc que j’ai étendu la main contre toi ; je t’ai coupé les vivres, je t’ai livrée à la merci de tes ennemies, les filles des Philistins, déshonorées elles-mêmes par ta conduite de débauchée. 
${}^{28}Tu t’es prostituée aux fils d’Assour, faute d’être rassasiée ; tu t’es prostituée à eux, et tu n’as pas été davantage rassasiée. 
${}^{29}Alors tu as multiplié tes prostitutions au pays des marchands, chez les Chaldéens, et cette fois encore tu n’as pas été rassasiée. 
${}^{30}Comme ton cœur était lâche ! – oracle du Seigneur Dieu. Quand tu faisais tout cela, agissant comme une prostituée effrénée, 
${}^{31}te bâtissant un podium à l’entrée de chaque chemin, quand tu faisais une estrade sur toutes les places, tu n’as même pas agi comme une prostituée car tu dédaignais le salaire. 
${}^{32}La femme adultère, au lieu de son mari, prend des étrangers. 
${}^{33}À toutes les prostituées, on fait un cadeau. Mais c’est toi qui faisais des cadeaux à tous tes amants ; tu les payais, pour qu’ils viennent vers toi, de tous côtés, se prostituer avec toi. 
${}^{34}Toi, tu as fait le contraire de ce que font les prostituées : on ne courait pas après toi, on ne te donnait pas de salaire ; c’est toi, au contraire, qui en donnais.
${}^{35}Eh bien, prostituée, écoute la parole du Seigneur : 
${}^{36}Ainsi parle le Seigneur Dieu : Puisque ton argent a été dilapidé, et ta nudité, dévoilée dans tes prostitutions avec tes amants, avec toutes tes idoles immondes, abominables, et à cause du sang de tes fils que tu leur as livrés, 
${}^{37}eh bien, voici que je rassemble tous les amants à qui tu as plu, tous ceux que tu as aimés, comme tous ceux que tu as haïs ; de tous côtés je les rassemble contre toi, je dévoile ta nudité devant eux, et ils voient toute ta nudité. 
${}^{38}Je t’inflige le châtiment des femmes adultères et des femmes sanguinaires : je répands ton sang avec fureur et jalousie. 
${}^{39}Je te livre entre leurs mains ; ils nivelleront ton podium et démoliront tes estrades ; ils t’arracheront tes vêtements et te prendront tes bijoux ; ils te laisseront complètement nue. 
${}^{40}Puis ils dresseront l’assemblée contre toi, ils te lapideront et de leurs épées te démembreront ; 
${}^{41}ils incendieront tes maisons ; ils feront justice de toi, sous les yeux d’une multitude de femmes. Je mettrai fin à ta vie de prostituée ; tu ne pourras même plus donner de salaire. 
${}^{42}J’assouvirai ma fureur contre toi. Puis ma jalousie se détournera de toi, je m’apaiserai, je ne serai plus irrité. 
${}^{43}Parce que tu ne t’es pas souvenue des jours de ta jeunesse et que tu m’as provoqué par ta conduite, voici qu’à mon tour je la fais retomber sur ta tête – oracle du Seigneur Dieu. N’as-tu pas pratiqué la débauche avec toutes tes abominations ?
${}^{44}Tous les faiseurs de proverbes en diront un à ton sujet : “Telle mère, telle fille.” 
${}^{45}Tu es bien la fille de ta mère qui détestait son mari et ses fils ; tu es bien la sœur de tes sœurs qui ont détesté leur mari et leurs fils. Votre mère était une Hittite et votre père un Amorite. 
${}^{46}Ta sœur aînée, c’est Samarie, qui habite à ta gauche avec ses filles. Ta sœur cadette, c’est Sodome, qui habite à ta droite avec ses filles. 
${}^{47}Il n’a pas suffi que tu imites leur conduite ni que tu commettes leurs abominations : tu t’es montrée plus corrompue qu’elles dans toute ta conduite. 
${}^{48}Par ma vie ! – oracle du Seigneur Dieu – ta sœur Sodome et ses filles n’en ont pas fait autant que toi et tes filles.
${}^{49}Voici quelle fut la faute de Sodome, ta sœur : orgueil, voracité, insouciance désinvolte ; oui, telles furent ses fautes et celles de ses filles ; elles ne fortifiaient pas la main du pauvre et du malheureux. 
${}^{50}Elles ont été pleines d’orgueil et ont commis devant moi ce qui est abominable ; c’est pourquoi je les ai rejetées, comme tu l’as vu. 
${}^{51}Quant à Samarie, elle n’a pas commis la moitié de tes péchés. Tu as multiplié bien plus qu’elle tes abominations. Tes sœurs, tu les as fait paraître justes par toutes les abominations que tu as commises. 
${}^{52}À ton tour, porte ton déshonneur, toi qui as innocenté tes sœurs : tes péchés, qui t’ont rendue plus abominable qu’elles, les font paraître plus justes que toi. Sois donc pleine de honte, et porte ton déshonneur, toi qui fais paraître justes tes sœurs. 
${}^{53}Je changerai leur destinée : la destinée de Sodome et de ses filles, la destinée de Samarie et de ses filles ; puis je changerai ta propre destinée au milieu d’elles, 
${}^{54}pour que tu portes ton déshonneur, et que tu sois déshonorée par tout ce que tu as fait : c’est ainsi que tu les consoleras !
${}^{55}Tes sœurs, Sodome et ses filles, reviendront à leur état ancien ; Samarie et ses filles reviendront à leur état ancien. Toi et tes filles, vous reviendrez à votre état ancien. 
${}^{56}De Sodome, ta sœur, n’as-tu pas fait des gorges chaudes, au jour de ton orgueil, 
${}^{57}avant que soit découverte ta méchanceté ? Comme elle, tu es maintenant l’objet de la raillerie des filles d’Aram, de toutes celles d’alentour et des filles des Philistins, qui te narguent de partout. 
${}^{58}Tu portes le poids de ta débauche et de tes abominations – oracle du Seigneur.
${}^{59}Car ainsi parle le Seigneur Dieu : Je vais agir avec toi comme tu as agi, toi qui as méprisé le serment et rompu l’alliance. 
${}^{60}Cependant, moi, je me ressouviendrai de mon alliance\\, celle que j’ai conclue avec toi au temps de ta jeunesse, et j’établirai pour toi une alliance éternelle. 
${}^{61}Tu te souviendras de ta conduite, et tu seras déshonorée, quand tu recueilleras tes sœurs, tes aînées et tes cadettes – c’est-à-dire Sodome et Samarie\\ – et quand je te les donnerai pour filles, sans que cela découle de ton alliance. 
${}^{62}Moi, je rétablirai mon alliance avec toi. Alors tu sauras que Je suis le Seigneur. 
${}^{63}Ainsi tu te souviendras, tu seras couverte de honte. Dans ton déshonneur, tu n’oseras pas ouvrir la bouche quand je te pardonnerai tout ce que tu as fait – oracle du Seigneur Dieu. »
      
         
      \bchapter{}
      \begin{verse}
${}^{1}La parole du Seigneur me fut adressée : 
${}^{2}« Fils d’homme, propose une énigme et raconte une parabole à la maison d’Israël.
${}^{3}Tu leur diras : Ainsi parle le Seigneur Dieu :
        \\Le grand aigle,
        aux grandes ailes,
        \\à l’envergure immense,
        au plumage épais et chamarré,
        vint au Liban.
        \\Il s’empara de la cime du cèdre,
${}^{4}cueillit le sommet de sa ramure ;
        \\il l’emporta au pays des marchands,
        et dans une ville de trafiquants le déposa.
${}^{5}Puis il prit une des semences du pays,
        la mit dans un champ prêt aux semailles :
        \\au bord des eaux abondantes,
        telle une pousse de saule, il la planta.
${}^{6}La semence germa,
        devint une vigne florissante,
        à la taille basse,
        \\tournant ses pampres vers l’aigle,
        étendant ses racines sous lui.
        \\C’était une vigne,
        elle donna des sarments
        et lança ses branches.
         
${}^{7}Il y eut encore un grand aigle,
        aux grandes ailes,
        au plumage abondant.
        \\Et voici que cette vigne
        dirigea vers lui ses racines,
        \\et vers lui tendit ses pampres
        pour qu’il l’arrose,
        \\loin des terrasses
        où elle était plantée.
${}^{8}C’est dans un champ fertile,
        au bord des eaux abondantes,
        \\qu’elle était plantée,
        pour produire des rameaux,
        \\porter du fruit,
        devenir une vigne magnifique.
         
${}^{9}Tu diras : Ainsi parle le Seigneur Dieu :
        \\Réussira-t-elle ?
        \\L’aigle ne va-t-il pas arracher ses racines,
        ôter son fruit pour qu’il sèche ?
        \\Ses pousses cueillies sécheront toutes.
        \\Nul besoin d’un bras puissant
        ni d’un peuple nombreux
        pour la déraciner !
${}^{10}La voici plantée : réussira-t-elle ?
        \\Dès que l’atteindra le vent d’est,
        ne va-t-elle pas se dessécher ?
        \\Sur la terrasse même où elle poussait,
        elle séchera ! »
       
${}^{11}La parole du Seigneur me fut adressée : 
${}^{12}« Parle donc à l’engeance de rebelles : Ne savez-vous pas ce que cela signifie ? Tu diras : Voici que le roi de Babylone est venu à Jérusalem ; il a enlevé son roi et ses princes ; il les a amenés chez lui, à Babylone. 
${}^{13}Il a pris quelqu’un de souche royale et a conclu une alliance avec lui ; il lui a fait prêter serment, après avoir enlevé les puissants du pays 
${}^{14}pour que le royaume reste modeste, sans s’élever, qu’il garde son alliance et qu’elle subsiste. 
${}^{15}Mais ce prince s’est révolté contre lui : il a envoyé des messagers en Égypte pour se faire donner des chevaux et un grand nombre de soldats. Réussira-t-il ? En réchappera-t-il, celui qui a fait cela ? Il a rompu l’alliance, et il en réchapperait ? 
${}^{16}Par ma vie ! – oracle du Seigneur Dieu –, là où réside le roi qui l’a fait régner, dont il a méprisé le serment et rompu l’alliance, c’est là, en plein milieu de Babylone, qu’il mourra. 
${}^{17}Avec sa grande armée, avec ses troupes nombreuses, Pharaon ne pourra agir en sa faveur lors de la guerre, quand on élèvera un remblai et que l’on bâtira des retranchements, pour retrancher tant de vies humaines. 
${}^{18}Il a méprisé le serment et rompu l’alliance ! Pourtant, il lui avait donné la main, et il a fait tout cela : il n’en réchappera pas !
${}^{19}C’est pourquoi, ainsi parle le Seigneur Dieu : Par ma vie ! mon serment qu’il a méprisé, mon alliance qu’il a rompue, je les ferai retomber sur sa tête. 
${}^{20}J’étendrai sur lui mon filet, et il sera pris au piège. Je l’amènerai à Babylone ; là-bas, j’entrerai en jugement contre lui à cause de son infidélité envers moi. 
${}^{21}Et toute son élite, celle de tous ses bataillons, elle tombera par l’épée ; ceux qui resteront seront éparpillés à tous les vents. Alors vous saurez que Je suis le Seigneur, j’ai parlé. »
       
        ${}^{22}Ainsi parle le Seigneur Dieu :
        \\« À la cime du grand cèdre,
        je prendrai une tige\\ ;
        \\au sommet de sa ramure,
        j’en cueillerai une toute jeune,
        \\et je la planterai moi-même
        sur une montagne très élevée.
        ${}^{23}Sur la haute montagne d’Israël je la planterai.
        \\Elle portera des rameaux, et produira du fruit,
        elle deviendra un cèdre magnifique.
        \\En dessous d’elle habiteront tous les passereaux
        et toutes sortes d’oiseaux,
        à l’ombre de ses branches ils habiteront\\.
        ${}^{24}Alors tous les arbres des champs sauront
        que Je suis le Seigneur :
        \\je renverse l’arbre élevé
        et relève l’arbre renversé,
        \\je fais sécher l’arbre vert
        et reverdir l’arbre sec.
        \\Je suis le Seigneur, j’ai parlé,
        et je le ferai. »
      
         
      \bchapter{}
      \begin{verse}
${}^{1}La parole du Seigneur me fut adressée : 
${}^{2} « Qu’avez-vous donc, dans le pays d’Israël, à répéter ce proverbe :
        \\“Les pères mangent du raisin vert,
        \\et les dents des fils en sont irritées\\” ?
${}^{3}Par ma vie\\ ! – oracle du Seigneur Dieu – vous n’aurez plus à répéter ce proverbe en Israël. 
${}^{4} En effet, toutes les vies m’appartiennent, la vie du père aussi bien que celle du fils, elles m’appartiennent. Celui qui a péché, c’est lui qui mourra.
${}^{5}L’homme qui est juste, qui observe le droit et la justice, 
${}^{6} qui ne va pas aux festins sur les montagnes, ne lève pas les yeux vers les idoles immondes de la maison d’Israël, ne rend pas impure la femme de son prochain, ne s’approche pas d’une femme en état de souillure\\ ; 
${}^{7} l’homme qui n’exploite personne, qui restitue ce qu’on lui a laissé en gage, ne commet pas de fraude, donne son pain à celui qui a faim et couvre d’un vêtement celui qui est nu ; 
${}^{8} l’homme qui ne prête pas à intérêt, ne pratique pas l’usure, qui détourne sa main du mal, tranche équitablement entre deux adversaires, 
${}^{9} qui marche selon mes décrets et observe mes ordonnances pour agir avec vérité\\ : un tel homme est juste, c’est certain, il vivra, – oracle du Seigneur Dieu.
${}^{10}Mais si cet homme a un fils violent et sanguinaire, coupable d’une de ces fautes, 
${}^{11}– alors que lui n’en a commis aucune – un fils qui, de plus, va aux festins sur les montagnes et rend impure la femme de son prochain, 
${}^{12}qui exploite le pauvre et le malheureux, qui commet des fraudes, ne restitue pas un gage, lève les yeux vers les idoles immondes et se livre à l’abomination, 
${}^{13}qui prête à intérêt et pratique l’usure, ce fils-là\\vivra-t-il ? Il ne vivra pas ; il s’est livré à toutes ces abominations : il sera mis à mort, et son sang, qu’il soit sur lui\\ !
${}^{14}Mais voici : un homme a un fils qui voit tous les péchés qu’a commis son père, il les voit sans les imiter, 
${}^{15}il ne va pas aux festins sur les montagnes, ne lève pas les yeux vers les idoles immondes de la maison d’Israël, ne rend pas impure la femme de son prochain, 
${}^{16}il n’exploite personne, ne prend pas de gages, ne commet pas de fraude, donne son pain à celui qui a faim et couvre d’un vêtement celui qui est nu, 
${}^{17}il détourne sa main du mal, ne prélève pas d’intérêt et ne pratique pas l’usure, il accomplit mes ordonnances et marche selon mes décrets. Ce fils ne mourra pas à cause des fautes de son père, c’est certain, il vivra. 
${}^{18}Mais son père, s’il a pratiqué la violence, commis des fraudes et n’a pas bien agi au milieu de son peuple : il mourra en raison de sa faute. 
${}^{19}Or vous dites : “Pourquoi le fils ne porte-t-il pas la faute de son père ?” Le fils a pratiqué le droit et la justice, il a observé tous mes décrets et les a pratiqués : c’est certain, il vivra.
${}^{20}Celui qui a péché, c’est lui qui mourra ! Le fils ne portera pas la faute de son père, ni le père, la faute de son fils : la justice sera la part du juste, la méchanceté, celle du méchant.
${}^{21}Mais le méchant, s’il se détourne de tous les péchés qu’il a commis, s’il observe tous mes décrets, s’il pratique le droit et la justice, c’est certain, il vivra, il ne mourra pas. 
${}^{22} On ne se souviendra d’aucun des crimes qu’il a commis, il vivra à cause de la justice qu’il a pratiquée\\. 
${}^{23} Prendrais-je donc plaisir\\à la mort du méchant – oracle du Seigneur Dieu –, et non pas plutôt à ce qu’il se détourne de sa conduite et qu’il vive ?
${}^{24}Mais le juste, s’il se détourne de sa justice et fait le mal en imitant toutes les abominations du méchant, il le ferait et il vivrait ? Toute la justice qu’il avait pratiquée, on ne s’en souviendra plus\\ : à cause de son infidélité et de son péché, il mourra !
${}^{25}Et pourtant vous dites : “La conduite du Seigneur n’est pas la bonne\\”. Écoutez donc, fils d’Israël : est-ce ma conduite qui n’est pas la bonne ? N’est-ce pas plutôt la vôtre ? 
${}^{26} Si le juste se détourne de sa justice, commet le mal, et meurt dans cet état, c’est à cause de son mal qu’il mourra. 
${}^{27} Si le méchant se détourne de sa méchanceté pour pratiquer le droit et la justice, il sauvera sa vie. 
${}^{28} Il a ouvert les yeux et s’est détourné de ses crimes. C’est certain, il vivra, il ne mourra pas.
${}^{29}Et pourtant la maison d’Israël répète : “La conduite du Seigneur est étrange”. Est-ce ma conduite qui est étrange, maison d’Israël ? N’est-ce pas votre conduite qui est étrange ? 
${}^{30} C’est pourquoi – oracle du Seigneur Dieu – je vous jugerai chacun selon sa conduite\\, maison d’Israël. Retournez-vous ! Détournez-vous de vos crimes, et vous ne trébucherez plus dans la faute\\. 
${}^{31} Rejetez tous les crimes que vous avez commis, faites-vous un cœur nouveau et un esprit nouveau. Pourquoi vouloir mourir, maison d’Israël ? 
${}^{32} Je ne prends plaisir à la mort de personne, – oracle du Seigneur Dieu – : convertissez-vous\\, et vous vivrez. »
      
         
      \bchapter{}
      \begin{verse}
${}^{1}« Et toi, entonne une complainte sur les princes d’Israël. 
${}^{2}Tu diras :
        \\Ta mère ? Une lionne,
        parmi les lions ;
        \\couchée au milieu des lionceaux,
        elle nourrissait ses petits.
${}^{3}Elle éleva un de ses petits
        qui devint un jeune lion ;
        \\il apprit à déchirer sa proie,
        il dévora des hommes.
${}^{4}Des nations en entendirent parler,
        il fut pris dans leur fosse ;
        \\avec des crochets, on l’emmena au pays d’Égypte.
${}^{5}Alors la lionne vit son attente déçue,
        sans espoir ;
        \\elle prit un autre de ses petits,
        et en fit un jeune lion.
${}^{6}Il rôdait parmi les lions,
        devenu un jeune lion ;
        \\il apprit à déchirer sa proie,
        il dévora des hommes.
${}^{7}Il démolit leurs palais,
        il détruisit leurs villes ;
        \\le pays, avec ce qu’il contient, fut terrorisé
        au bruit de son rugissement.
${}^{8}On dressa contre lui des nations d’alentour,
        venues de leurs provinces ;
        \\elles étendirent sur lui leur filet,
        il fut pris dans leur fosse.
${}^{9}Avec des crochets ils le mirent en cage,
        ils le menèrent au roi de Babylone,
        \\ils le menèrent en lieu sûr,
        pour qu’on n’entende plus sa voix
        sur les montagnes d’Israël.
${}^{10}Ta mère était comme une vigne
        plantée au bord des eaux.
        \\Elle était féconde et touffue,
        car les eaux étaient abondantes.
${}^{11}Elle eut des tiges vigoureuses
        qui devinrent des sceptres royaux ;
        \\elle grandit en taille
        et s’éleva au milieu des branchages ;
        \\on l’admira pour sa hauteur
        et le grand nombre de ses pampres.
${}^{12}Mais elle a été arrachée avec fureur
        et jetée à terre ;
        \\le vent d’est a desséché son fruit ;
        elle a été brisée,
        \\ses tiges vigoureuses ont séché,
        le feu les a dévorées.
${}^{13}Et maintenant, elle est plantée au désert,
        dans un pays de soif et d’aridité.
${}^{14}Un feu est sorti de ses tiges,
        il a dévoré ses sarments et ses fruits.
        \\Plus de tiges vigoureuses,
        plus de sceptre royal ! »
       
      Ce chant est une complainte, qu’il serve de complainte !
      
         
      \bchapter{}
      \begin{verse}
${}^{1}La septième année de la première déportation, le cinquième mois, le dix du mois, quelques anciens d’Israël vinrent consulter le Seigneur. Ils s’assirent devant moi. 
${}^{2}La parole du Seigneur me fut adressée : 
${}^{3}« Fils d’homme, parle aux anciens d’Israël. Tu leur diras : Ainsi parle le Seigneur Dieu : Est-ce pour me consulter que vous venez ? Par ma vie ! Je ne me laisserai pas consulter par vous – oracle du Seigneur Dieu. 
${}^{4}Ne dois-tu pas les juger, oui, les juger, fils d’homme ? Fais-leur connaître les abominations de leurs pères.
${}^{5}Tu leur diras : Ainsi parle le Seigneur Dieu : Le jour où j’ai choisi Israël, où j’ai levé ma main en faveur de la descendance de la maison de Jacob, je me suis fait connaître d’eux au pays d’Égypte ; j’ai levé ma main pour eux en disant : “Je suis le Seigneur votre Dieu.” 
${}^{6}Ce jour-là, j’ai levé ma main pour eux, jurant de les faire sortir du pays d’Égypte vers un pays que j’avais exploré pour eux, un pays ruisselant de lait et de miel, le plus beau de tous les pays. 
${}^{7}Je leur ai dit : “Que chacun rejette les horreurs qui attirent ses yeux ! Ne vous rendez pas impurs avec les idoles immondes de l’Égypte. Je suis le Seigneur votre Dieu.” 
${}^{8}Mais ils se révoltèrent contre moi et refusèrent de m’écouter : aucun ne rejeta les horreurs qui attiraient ses yeux ; ils ne renoncèrent pas aux idoles immondes de l’Égypte. Alors j’ai dit : “Je déverserai ma fureur contre eux, j’irai jusqu’au bout de ma colère envers eux, en plein pays d’Égypte.” 
${}^{9}Cependant j’ai agi à cause de mon nom, pour qu’il ne soit pas profané devant les nations parmi lesquelles ils vivaient. Sous les yeux de ces nations, je me fis connaître à eux en les faisant sortir du pays d’Égypte. 
${}^{10}Je les fis donc sortir du pays d’Égypte et je les menai au désert.
${}^{11}Je leur ai donné mes décrets et leur ai fait connaître mes ordonnances, celles que l’homme doit pratiquer pour en vivre. 
${}^{12}Je leur donnai aussi mes sabbats comme signe entre moi et eux, pour que l’on sache que Je suis le Seigneur, qui les sanctifie. 
${}^{13}Mais, au désert, la maison d’Israël se révolta contre moi ; ils ne marchèrent pas selon mes décrets, ils rejetèrent mes ordonnances, celles que l’homme doit pratiquer pour en vivre, et ils profanèrent grandement mes sabbats. Alors j’ai dit : “Je déverserai ma fureur contre eux au désert, pour les exterminer.” 
${}^{14}Cependant j’ai agi à cause de mon nom, pour qu’il ne soit pas profané devant les nations, sous les yeux desquelles je les avais fait sortir. 
${}^{15}Mais moi, encore une fois, je levai ma main contre eux au désert, jurant de ne pas les faire entrer dans le pays que je leur avais donné, un pays ruisselant de lait et de miel, le plus beau de tous les pays. 
${}^{16}Car ils avaient rejeté mes ordonnances, n’avaient pas marché selon mes décrets et avaient profané mes sabbats : leur cœur marchait à la suite de leurs idoles immondes. 
${}^{17}Pourtant j’ai eu pour eux un regard de pitié, je n’ai pas voulu les détruire, je ne les ai pas exterminés dans le désert.
${}^{18}Au désert, j’ai dit à leurs fils : Ne marchez pas selon les décrets de vos pères, n’observez pas leurs ordonnances, ne vous rendez pas impurs avec leurs idoles immondes. 
${}^{19}Je suis le Seigneur, votre Dieu. Marchez selon mes décrets, observez mes ordonnances et mettez-les en pratique. 
${}^{20}Sanctifiez mes sabbats ; qu’ils soient un signe entre moi et vous, pour qu’on sache que Je suis le Seigneur votre Dieu. 
${}^{21}Cependant les fils se révoltèrent contre moi, ils ne marchèrent pas selon mes décrets, n’observèrent ni ne pratiquèrent mes ordonnances, celles que l’homme doit pratiquer pour en vivre, et ils profanèrent mes sabbats. Alors j’ai dit : je déverserai ma fureur contre eux et, dans le désert, j’irai jusqu’au bout de ma colère envers eux. 
${}^{22}Mais j’ai retiré ma main, et j’ai agi à cause de mon nom, pour qu’il ne soit pas profané devant les nations sous les yeux desquelles je les avais fait sortir.
${}^{23}Au désert, j’ai levé ma main contre eux, jurant de les disperser parmi les nations et de les disséminer dans les pays étrangers. 
${}^{24}Car ils n’avaient pas pratiqué mes ordonnances, ils avaient rejeté mes décrets et avaient profané mes sabbats ; leurs regards s’étaient attachés aux idoles immondes de leurs pères. 
${}^{25}Je suis même allé jusqu’à leur laisser leurs décrets qui n’étaient pas bons et leurs ordonnances dont ils ne pourraient pas vivre ; 
${}^{26}je les ai rendus impurs par leurs dons, quand ils faisaient passer par le feu tous les premiers-nés. C’était pour les frapper de stupeur, afin qu’ils sachent que Je suis le Seigneur.
${}^{27}C’est pourquoi, parle à la maison d’Israël, fils d’homme. Tu leur diras : Ainsi parle le Seigneur Dieu : Sans cesse, vos pères m’ont outragé par les infidélités qu’ils ont commises. 
${}^{28}Je les ai fait entrer dans le pays que j’avais juré, la main levée, de leur donner, et ils ont regardé chaque colline élevée et chaque arbre touffu ; c’est là qu’ils ont offert leurs sacrifices, là qu’ils m’ont provoqué en apportant leurs présents réservés, là qu’ils ont déposé leurs parfums d’apaisement, là qu’ils ont versé leurs libations. 
${}^{29}Alors je leur ai dit : Qu’est donc ce lieu sacré, ce “bamah” où vous allez ? Et ils lui ont donné jusqu’à ce jour le nom de “bamah” (c’est-à-dire : “Qu’est-ce donc ?”).
${}^{30}C’est pourquoi, dis à la maison d’Israël : Ainsi parle le Seigneur Dieu : Alors ! vous vous rendez impurs en suivant le chemin de vos pères, en vous prostituant avec leurs horreurs ! 
${}^{31}Quand vous apportez vos dons, quand vous faites passer vos fils par le feu, vous vous rendez impurs jusqu’à ce jour avec toutes vos idoles immondes ! Et moi, je me laisserais consulter par vous, maison d’Israël ? Par ma vie – oracle du Seigneur Dieu –, je ne me laisserai pas consulter par vous !
${}^{32}Ce qui vous monte à l’esprit ne se réalisera pas, lorsque vous dites : “Nous voulons être comme les nations, comme les clans des autres pays, nous voulons servir le bois et la pierre.” 
${}^{33}Par ma vie – oracle du Seigneur Dieu –, je régnerai sur vous, à main forte et à bras étendu, en répandant ma fureur ! 
${}^{34}Je vous ferai sortir d’entre les peuples ; de ces pays où vous avez été dispersés, je vous rassemblerai à main forte et à bras étendu, en répandant ma fureur ; 
${}^{35}je vous mènerai au désert des peuples et là, face à face, j’entrerai en jugement contre vous. 
${}^{36}Comme je suis entré en jugement contre vos pères au désert du pays d’Égypte, ainsi entrerai-je en jugement contre vous – oracle du Seigneur Dieu. 
${}^{37}Je vous ferai passer sous ma houlette, je vous introduirai dans les liens de l’alliance. 
${}^{38}J’ôterai de chez vous les rebelles, ceux qui se sont révoltés contre moi : je les ferai sortir du pays où ils séjournent, mais ils n’entreront pas dans le pays d’Israël. Alors vous saurez que Je suis le Seigneur.
${}^{39}Quant à vous, maison d’Israël, ainsi parle le Seigneur Dieu : Que chacun aille servir ses idoles immondes, et ensuite, on verra bien si vous ne m’écouterez pas ! Alors vous ne profanerez plus mon saint nom par vos dons et vos idoles immondes. 
${}^{40}Car c’est sur ma sainte montagne, sur la haute montagne d’Israël – oracle du Seigneur Dieu –, c’est là que me servira toute la maison d’Israël, établie tout entière dans le pays. C’est là que j’accueillerai et viendrai chercher vos contributions, le meilleur de vos portions en tout ce que vous me consacrez. 
${}^{41}Comme un parfum d’apaisement, je vous accueillerai quand je vous aurai fait sortir d’entre les peuples, et je vous rassemblerai des pays où vous aurez été dispersés. Ainsi, par vous, se manifestera ma sainteté aux yeux des nations. 
${}^{42}Alors vous saurez que Je suis le Seigneur, lorsque je vous ferai entrer sur le sol d’Israël, dans ce pays que j’ai juré, la main levée, de donner à vos pères. 
${}^{43}C’est là que vous vous souviendrez de votre conduite et de tous les actes par lesquels vous vous êtes rendus impurs ; le dégoût vous montera au visage à cause de tous les méfaits que vous avez commis. 
${}^{44}Alors vous saurez que Je suis le Seigneur, quand j’agirai envers vous à cause de mon nom, et non pas d’après votre mauvaise conduite et vos actes corrompus, maison d’Israël – oracle du Seigneur Dieu. »
      
         
      \bchapter{}
      \begin{verse}
${}^{1}La parole du Seigneur me fut adressée : 
${}^{2}« Fils d’homme, dirige ton regard vers le midi ; invective le sud, prophétise contre la forêt du Néguev. 
${}^{3}Tu diras à la forêt du Néguev : Écoute la parole du Seigneur. Ainsi parle le Seigneur Dieu : Voici que je vais allumer en toi un feu : il dévorera tout arbre vert et tout arbre sec ; la flamme ardente ne s’éteindra pas, et tous les visages en seront brûlés, depuis le Néguev jusqu’au nord. 
${}^{4}Alors tout être de chair verra que moi, le Seigneur, j’ai allumé cette flamme, et elle ne s’éteindra pas. » 
${}^{5}Et je dis : « Ah ! Seigneur mon Dieu, ils disent de moi : “Ne voilà-t-il pas qu’il débite des paraboles !” »
${}^{6}La parole du Seigneur me fut adressée : 
${}^{7}« Fils d’homme, dirige ton regard vers Jérusalem, invective les sanctuaires, prophétise contre la terre d’Israël. 
${}^{8}Tu diras à la terre d’Israël : Ainsi parle le Seigneur : Me voici contre toi ; je vais tirer mon épée du fourreau et retrancher de chez toi le juste et le méchant. 
${}^{9}C’est pour retrancher le juste et le méchant que mon épée va sortir de son fourreau, contre tout être de chair, du Néguev jusqu’au nord. 
${}^{10}Et tout être de chair saura que Je suis le Seigneur : j’ai tiré mon épée du fourreau, elle n’y rentrera plus.
${}^{11}Fils d’homme, pousse des gémissements ; rempli d’amertume, les reins brisés, gémis sous leurs yeux. 
${}^{12}Et s’ils te disent : “Pourquoi ces gémissements ?”, tu leur diras : “À cause d’une nouvelle qui arrive, tous les cœurs vont défaillir, toute main faiblir ; tous les esprits seront abattus, tous les genoux fondront en eau. Voici, cela vient, c’est fait – oracle du Seigneur Dieu”. »
${}^{13}La parole du Seigneur me fut adressée : 
${}^{14}« Fils d’homme, prophétise. Tu diras : Ainsi parle le Seigneur :
        \\L’épée, oui, l’épée est affûtée,
        bien fourbie,
${}^{15}affûtée pour accomplir le massacre,
        fourbie pour jeter des éclairs...
${}^{16}On l’a donnée à fourbir,
        à saisir à pleine main :
        \\elle est affûtée, l’épée, et fourbie
        pour être mise dans la main du tueur.
${}^{17}Crie, hurle, fils d’homme,
        car elle est tirée contre mon peuple,
        \\contre tous les princes d’Israël,
        ils sont précipités sur l’épée avec mon peuple.
        \\Aussi, frappe-toi la cuisse,
${}^{18}car c’est une épreuve,
        – oracle du Seigneur Dieu.
${}^{19}Et toi, fils d’homme, prophétise
        et bats des mains.
        \\L’épée par deux fois, par trois fois,
        l’épée transperce,
        \\la grande épée transperce des victimes,
        tout alentour !
${}^{20}Afin de faire trembler les cœurs,
        de multiplier les embûches,
        \\à toutes les portes j’ai placé
        le massacre par l’épée.
        \\Ah ! Elle est faite pour jeter des éclairs,
        fourbie pour le massacre.
${}^{21}Tranche à droite, prends position à gauche,
        là où tu dois faire face.
${}^{22}Moi aussi, je bats des mains,
        et j’assouvirai ma fureur.
        \\Je suis le Seigneur, j’ai parlé. »
${}^{23}La parole du Seigneur me fut adressée : 
${}^{24}« Et toi, fils d’homme, trace deux chemins pour que vienne l’épée du roi de Babylone. Que ces deux chemins partent du même pays. À leur entrée, tu mettras un signe indiquant la direction d’une ville. 
${}^{25}Tu traceras un chemin pour que l’épée vienne vers Rabba des fils d’Ammone, et vers Juda, à Jérusalem, la ville forte. 
${}^{26}Le roi de Babylone, en effet, s’est arrêté au carrefour, à l’entrée des deux chemins, pour scruter les présages. Il secoue les flèches, consulte les idoles, examine le foie. 
${}^{27}Dans sa main droite, le présage désigne Jérusalem : qu’on mette en place les béliers, qu’on donne l’ordre de la tuerie, qu’on pousse le cri de guerre, qu’on place les béliers contre les portes, qu’on élève un remblai, qu’on bâtisse des retranchements. 
${}^{28}Aux yeux des habitants de Jérusalem, ce ne sera que vain présage ; ils ont pour eux les serments prononcés, mais le roi leur rappellera leur faute, et ils seront faits captifs.
${}^{29}C’est pourquoi, ainsi parle le Seigneur Dieu : Vous avez provoqué le rappel de votre faute, quand vos crimes ont été découverts, de sorte que vos péchés sont apparus dans toutes vos actions ; vous avez attiré l’attention sur vous : c’est pourquoi vous serez capturés par la force. 
${}^{30}Quant à toi, vil malfaiteur, prince d’Israël dont le jour approche avec la dernière de tes fautes, 
${}^{31}ainsi parle le Seigneur Dieu : On ôtera le turban, on enlèvera la couronne ; tout sera bouleversé : ce qui est bas sera élevé, ce qui est élevé sera abaissé. 
${}^{32}Ruine ! ruine ! j’en ferai une ruine comme il n’y en a jamais eu, jusqu’à ce que vienne celui à qui appartient le jugement et à qui je le remettrai.
${}^{33}Et toi, fils d’homme, prophétise. Tu diras : Ainsi parle le Seigneur Dieu au sujet des fils d’Ammone et de leurs railleries. Tu diras :
        \\Épée ! Épée !
        Tu es dégainée pour le massacre,
        \\fourbie pour dévorer,
        pour jeter des éclairs,
${}^{34}trancher le cou aux vils malfaiteurs
        dont le jour approche
        avec la dernière de leurs fautes !
        \\Illusoires, les visions à ton sujet,
        et trompeurs, les présages !
${}^{35}Remets l’épée au fourreau. Dans le lieu où tu as été créé, au pays de tes origines, je te jugerai. 
${}^{36}Je déverserai sur toi mon indignation, je soufflerai contre toi le feu de mon emportement, et je te livrerai entre les mains de gens stupides, artisans d’extermination. 
${}^{37}Tu seras la proie du feu, ton sang coulera au milieu du pays ; on ne se souviendra plus de toi : Je suis le Seigneur, j’ai parlé. »
      
         
      \bchapter{}
      \begin{verse}
${}^{1}La parole du Seigneur me fut adressée : 
${}^{2}« Et toi, fils d’homme, ne dois-tu pas juger, oui, juger la ville sanguinaire et lui faire connaître toutes ses abominations ? 
${}^{3}Tu diras : Ainsi parle le Seigneur Dieu : Ô ville qui répands le sang au milieu d’elle pour faire venir son heure, tu fabriques des idoles immondes chez toi, et tu en es impure ! 
${}^{4}Par le sang que tu as répandu, tu es devenue coupable ; par les idoles immondes que tu as fabriquées, tu t’es rendue impure : ainsi tu as rapproché ton jour, te voici parvenue au terme de tes années. C’est pourquoi je fais de toi un objet de raillerie pour les nations, de risée pour tous les pays. 
${}^{5}Proches ou lointains, ils se riront de toi, car ton nom est impur, et la panique te submerge.
${}^{6}Les princes d’Israël répandent chez toi le sang, chacun selon la force de son bras. 
${}^{7}Chez toi, on dédaigne père et mère ; au milieu de toi, on fait violence à l’immigré ; chez toi, on exploite l’orphelin et la veuve. 
${}^{8}Tu méprises les objets saints du culte, tu profanes mes sabbats. 
${}^{9}Il y a chez toi des gens qui calomnient pour que l’on répande le sang. Chez toi, on va aux festins sur les montagnes ; au milieu de toi, on pratique la débauche. 
${}^{10}Chez toi, on découvre la nudité de son père ; chez toi on abuse de la femme en état d’impureté. 
${}^{11}L’un commet l’abomination avec la femme de son prochain ; l’autre, dans la débauche, rend impure sa belle-fille ; chez toi, un autre abuse de sa sœur, la fille de son père. 
${}^{12}Chez toi, on accepte des présents, pour répandre le sang. Tu prélèves l’intérêt, tu pratiques l’usure, tu tires profit de ton prochain par la violence ; et moi, tu m’oublies – oracle du Seigneur Dieu !
${}^{13}Voici que je vais battre des mains à cause des profits que tu as faits, et du sang qui coule au milieu de toi. 
${}^{14}Ton cœur tiendra-t-il, tes mains seront-elles fermes, les jours où je m’en prendrai à toi ? Je suis le Seigneur, j’ai parlé, et je le ferai. 
${}^{15}Je te disperserai parmi les nations, je te disséminerai dans les pays étrangers, je ferai disparaître l’impureté de chez toi. 
${}^{16}Tu t’es profanée toi-même aux yeux des nations, mais tu sauras que Je suis le Seigneur. »
       
${}^{17}La parole du Seigneur me fut adressée : 
${}^{18}« Fils d’homme, la maison d’Israël est devenue pour moi comme des scories. Qu’ils soient de l’argent, du bronze, de l’étain, du fer ou du plomb, au milieu de la fournaise, tous, ils sont devenus des scories. 
${}^{19}C’est pourquoi, ainsi parle le Seigneur Dieu : Puisque vous êtes tous des scories, eh bien ! je vais vous entasser au milieu de Jérusalem. 
${}^{20}Comme on entasse argent, bronze, fer, plomb et étain au milieu d’une fournaise pour y attiser le feu et les faire fondre, ainsi, dans ma colère et dans ma fureur, je vous entasserai et je vous ferai fondre. 
${}^{21}Je vous rassemblerai et j’attiserai contre vous le feu de mon emportement, je vous ferai fondre au milieu de la ville. 
${}^{22}Comme de l’argent en fusion au milieu de la fournaise, ainsi serez-vous fondus au milieu de la ville. Alors vous saurez que Je suis le Seigneur : j’ai déversé ma fureur sur vous. »
${}^{23}La parole du Seigneur me fut adressée : 
${}^{24}« Fils d’homme, dis à Jérusalem : Tu es une terre qui n’a pas été purifiée, qui n’a pas reçu de pluie au jour de l’indignation. 
${}^{25}Il y a une conjuration de ses prophètes au milieu d’elle ; ils sont comme un lion rugissant qui déchire sa proie ; ils ont dévoré les gens, pris richesses et bijoux, multiplié les veuves, au milieu d’elle. 
${}^{26}Ses prêtres ont violé ma loi et profané les objets saints du culte ; ils n’ont pas séparé le saint et le profane, ils n’ont pas enseigné à distinguer l’impur et le pur ; ils ont détourné les yeux de mes sabbats, et j’ai été profané parmi eux. 
${}^{27}Ses princes, au milieu d’elle, sont comme des loups qui déchirent leur proie et répandent le sang, prêts à faire périr les gens pour en tirer profit. 
${}^{28}Ses prophètes enduisent tout de badigeon avec leurs visions fallacieuses et leurs présages trompeurs. Ils disent : “Ainsi parle le Seigneur Dieu”, alors que le Seigneur n’a pas parlé. 
${}^{29}Les gens du pays pratiquent la violence et commettent des fraudes, ils exploitent le pauvre et le malheureux, ils font violence à l’immigré, au mépris du droit. 
${}^{30}J’ai cherché parmi eux quelqu’un qui relève le mur et se tienne devant moi, debout sur la brèche, pour défendre le pays et m’empêcher de le détruire, et je n’ai trouvé personne. 
${}^{31}Alors j’ai déversé sur eux mon indignation ; dans le feu de mon emportement, je les ai exterminés. J’ai fait retomber leur conduite sur leur tête – oracle du Seigneur Dieu. »
      
         
      \bchapter{}
      \begin{verse}
${}^{1}La parole du Seigneur me fut adressée : 
${}^{2}« Fils d’homme, il était une fois deux femmes, filles de la même mère. 
${}^{3}Elles se prostituèrent en Égypte ; dès leur jeunesse, elles se prostituèrent. C’est là qu’on a pressé leurs seins, là qu’on a caressé leur poitrine virginale. 
${}^{4}Voici leurs noms : Ohola, l’aînée, Oholiba sa sœur. Alors elles furent à moi et enfantèrent des fils et des filles. Quant à leurs noms, Ohola, c’est Samarie, Oholiba, c’est Jérusalem. 
${}^{5}Or Ohola se prostitua alors qu’elle m’appartenait. Elle brûla de désir pour ses amants, les fils d’Assour, des voisins 
${}^{6}vêtus de pourpre, gouverneurs et magistrats, tous jeunes, séduisants, cavaliers accomplis. 
${}^{7}Elle s’adonna à la prostitution avec toute l’élite des fils d’Assour. Chez tous ceux pour qui elle brûlait, elle se rendit impure au contact de toutes leurs idoles immondes. 
${}^{8}Elle ne renonça pas à ses prostitutions commencées en Égypte, quand elle était toute jeune et que l’on couchait avec elle, que l’on caressait sa poitrine virginale en l’entraînant dans la prostitution. 
${}^{9}C’est pourquoi je l’ai livrée aux mains de ses amants, aux mains des fils d’Assour pour qui elle brûlait : 
${}^{10}eux dévoilèrent sa nudité, prirent ses fils et ses filles ; elle-même, ils la firent périr par l’épée. Elle fut célèbre parmi les femmes, car on avait fait justice d’elle.
${}^{11}Sa sœur Oholiba vit tout cela, mais elle fut corrompue davantage, elle brûla plus encore, ses prostitutions furent pires que celles de sa sœur. 
${}^{12}Elle brûla pour les fils d’Assour, gouverneurs et magistrats, ses voisins superbement vêtus, cavaliers accomplis, tous jeunes et séduisants. 
${}^{13}Je vis qu’elle s’était rendue impure, et que toutes deux avaient pris le même chemin. 
${}^{14}Oholiba ajouta encore à ses prostitutions : elle vit des hommes gravés sur le mur, des images de Chaldéens colorées au vermillon ; 
${}^{15}leurs reins étaient serrés dans une ceinture, leur tête coiffée de foulards flottants ; tous avaient l’allure d’écuyers et ressemblaient aux fils de Babylone en Chaldée, leur pays d’origine ; 
${}^{16}elle brûla pour eux au premier regard et leur envoya des messagers en Chaldée. 
${}^{17}Alors les fils de Babylone vinrent à elle pour partager la couche des amours, et la rendirent impure par leurs prostitutions. Mais après qu’ils l’eurent rendue impure, son cœur se détacha d’eux. 
${}^{18}Elle a dévoilé ses prostitutions, elle a dévoilé sa nudité. Alors mon cœur s’est détaché d’elle comme il s’était détaché de sa sœur. 
${}^{19}Elle a multiplié ses prostitutions, se souvenant des jours de sa jeunesse quand elle se prostituait au pays d’Égypte ; 
${}^{20}elle y brûlait pour des partenaires dont la vigueur était celle des ânes, et le rut celui des étalons.
${}^{21}Tu recherchais la débauche de ta jeunesse, quand les Égyptiens s’en prenaient à tes seins, à ta poitrine juvénile. 
${}^{22}C’est pourquoi, Oholiba, ainsi parle le Seigneur Dieu : Voici que je dresse contre toi tes amants ; ceux dont ton cœur s’est détaché, je les amène de partout contre toi, 
${}^{23}les fils de Babylone et tous les Chaldéens, Peqod, Shoa et Qoa, tous les fils d’Assour avec eux, tous jeunes et séduisants, gouverneurs et magistrats, écuyers renommés, tous montant des chevaux. 
${}^{24}Du nord viendront contre toi chars et chariots, avec une cohorte de peuples. De tous côtés, ils t’opposeront le grand et le petit bouclier ainsi que le casque. Je leur confierai le jugement, et ils te jugeront selon leur droit. 
${}^{25}J’exercerai ma jalousie contre toi, et ils te traiteront avec fureur, ils te couperont le nez et les oreilles, et ceux des tiens qui resteront tomberont par l’épée ; ils prendront tes fils et tes filles, et ce qui restera de toi sera dévoré par le feu. 
${}^{26}Ils t’arracheront tes vêtements et te prendront tes bijoux. 
${}^{27}Je mettrai fin à ta débauche, à ta prostitution qui a commencé en Égypte. Tu ne lèveras plus les yeux vers eux et tu ne te souviendras plus de l’Égypte. 
${}^{28}Car ainsi parle le Seigneur Dieu : Voici que je te livre aux mains de ceux que tu as pris en haine, aux mains de ceux dont ton cœur s’est détaché. 
${}^{29}Ils te traiteront avec haine, ils s’empareront de tout le fruit de ton travail, ils te laisseront complètement nue. Ainsi sera dévoilée la nudité de tes fornications, de ta débauche et de ta prostitution. 
${}^{30}Ils te feront cela parce que tu t’es prostituée en suivant les nations et tu t’es rendue impure avec leurs idoles immondes. 
${}^{31}Tu as imité la conduite de ta sœur ; je mettrai sa coupe dans ta main.
       
${}^{32}Ainsi parle le Seigneur Dieu :
        \\La coupe de ta sœur, tu la boiras,
        elle est profonde et large ;
        \\elle sera l’occasion de rire et de moquerie
        tant sa contenance est grande.
${}^{33}Tu seras remplie d’ivresse et de douleur.
        \\C’est une coupe de désolation et de dévastation,
        la coupe de ta sœur Samarie !
${}^{34}Tu la boiras, tu la videras ;
        tu en rongeras les tessons
        et tu te déchireras les seins.
        \\Oui, moi, j’ai parlé,
        – oracle du Seigneur Dieu.
       
${}^{35}C’est pourquoi, ainsi parle le Seigneur Dieu : Parce que tu m’as oublié et que tu m’as rejeté derrière ton dos, à ton tour de porter le poids de ta débauche et de tes prostitutions. »
       
${}^{36}Le Seigneur me dit : « Fils d’homme, ne dois-tu pas juger Ohola et Oholiba ? Expose-leur leurs abominations ! 
${}^{37}Elles ont commis l’adultère, elles ont du sang sur les mains ; avec leurs idoles immondes elles ont commis l’adultère. Les fils qu’elles m’avaient enfantés, elles les ont fait passer par le feu, pour en nourrir les idoles. 
${}^{38}Elles m’ont fait encore ceci : elles ont rendu impur mon sanctuaire, ce jour-là ; elles ont profané mes sabbats. 
${}^{39}Quand elles immolaient leurs fils à leurs idoles immondes, ce jour-là, elles sont allées à mon sanctuaire et l’ont profané. Voilà ce qu’elles ont fait au milieu de ma maison. 
${}^{40}Bien plus, elles ont appelé des hommes venant de loin, et vers qui un messager avait été envoyé. Voici qu’ils sont venus : pour eux tu t’es baignée, tu t’es fardé les yeux, tu t’es parée, 
${}^{41}tu t’es mise sur un lit d’apparat, devant lequel une table était dressée où l’on avait mis mon encens et mon huile. 
${}^{42}On entendait le bruit d’une foule animée, insouciante ; il y avait là une multitude d’hommes, de buveurs venant du désert ; ils ont mis des bracelets aux poignets des femmes et sur leur tête un magnifique diadème. 
${}^{43}Et je me disais : Elle est usée par les adultères, et c’est avec celle-là qu’on se livre à la prostitution ? 
${}^{44}On est venu chez elle comme chez une prostituée ! C’est ainsi qu’on vient chez Ohola et Oholiba, ces femmes débauchées. 
${}^{45}Mais des hommes justes les jugeront selon le droit concernant les femmes adultères et celles qui répandent le sang. Car elles ont commis l’adultère et elles ont du sang sur les mains. »
${}^{46}Ainsi parle le Seigneur Dieu : « Dresse contre elles une assemblée. Qu’on les livre à l’épouvante et au pillage ; 
${}^{47}que l’assemblée les lapide et les frappe de l’épée ; que l’on tue leurs fils et leurs filles, et qu’on mette le feu à leurs maisons. 
${}^{48}Je ferai cesser la débauche du pays. Ce sera un avertissement pour toutes les femmes, elles n’imiteront plus votre débauche. 
${}^{49}On fera retomber sur vous votre débauche : vous porterez le poids des fautes commises avec vos idoles immondes. Alors vous saurez que Je suis le Seigneur Dieu. »
      
         
      \bchapter{}
      \begin{verse}
${}^{1}La neuvième année de la première déportation, le dixième mois, le dix du mois, la parole du Seigneur me fut adressée : 
${}^{2}« Fils d’homme, note par écrit la date d’aujourd’hui, d’aujourd’hui même, car aujourd’hui le roi de Babylone s’est porté contre Jérusalem. 
${}^{3}Raconte une parabole à cette engeance de rebelles. Tu leur diras : Ainsi parle le Seigneur Dieu :
        \\Prépare la marmite, prépare-la ;
        verses-y de l’eau.
${}^{4}Rassemble dedans les morceaux,
        tous les bons morceaux : cuisse et épaule ;
        remplis-la des meilleurs os.
${}^{5}Prends le meilleur mouton,
        entasse du bois dessous ;
        \\fais tout bouillir à gros bouillons,
        que même les os soient cuits.
         
${}^{6}C’est pourquoi ainsi parle le Seigneur Dieu :
        \\Malheur à la ville sanguinaire,
        marmite rouillée,
        dont la rouille ne s’enlève pas !
        \\On enlèvera morceau après morceau,
        sans tirer au sort.
${}^{7}Car le sang qui est au milieu d’elle,
        sur la roche nue elle l’a versé,
        \\elle ne l’a pas répandu sur la terre
        ni recouvert de poussière.
${}^{8}Pour faire monter la fureur, pour prendre ma revanche,
        j’ai mis son sang sur la roche nue, sans le recouvrir.
${}^{9}C’est pourquoi ainsi parle le Seigneur Dieu :
        Malheur à la ville sanguinaire !
        Moi aussi, je vais dresser un grand bûcher.
${}^{10}Amoncelle du bois, allume le feu,
        cuis, recuis la viande,
        \\ajoute les épices,
        que les os soient brûlés !
${}^{11}Puis, mets la marmite vide sur les braises
        pour qu’elle chauffe ;
        \\que le bronze rougisse,
        que les impuretés fondent à l’intérieur
        et que la rouille soit consumée !
${}^{12}Que d’efforts ! Et pourtant, la masse de rouille de cette marmite ne s’en ira pas au feu. 
${}^{13}Ton impureté, c’est la débauche, car j’ai voulu te purifier, mais tu ne t’es pas laissé purifier de ton impureté. On ne te purifiera plus, jusqu’à ce que j’aie assouvi ma fureur contre toi. 
${}^{14}Je suis le Seigneur, j’ai parlé. Elle vient, ma fureur, et j’agis. Je ne fléchirai pas, je serai sans pitié, je ne me repentirai pas. On te jugera selon ta conduite et selon tes actes – oracle du Seigneur Dieu. »
${}^{15}La parole du Seigneur me fut adressée : 
${}^{16} « Fils d’homme, je vais te prendre subitement la joie de tes yeux. Tu ne feras pas de lamentation, tu ne pleureras pas, tu ne laisseras pas couler tes larmes. 
${}^{17} Soupire en silence, ne prends pas le deuil ; enroule ton turban sur ta tête, chausse tes sandales\\, ne voile pas tes lèvres\\, ne prends pas le repas funéraire\\. »
${}^{18}Le matin, je parlais encore au peuple, et le soir ma femme mourut. Le lendemain matin, je fis ce qui m’avait été ordonné. 
${}^{19} Les gens me dirent : « Vas-tu nous expliquer ce que tu fais là ? Qu’est-ce que cela veut dire pour nous ? » 
${}^{20} Je leur répondis : « La parole du Seigneur m’a été adressée : 
${}^{21} Dis à la maison d’Israël : Ainsi parle le Seigneur Dieu : Je vais profaner mon sanctuaire, votre orgueil et votre force, la joie de vos yeux, la passion de votre cœur. Vos fils et vos filles, que vous avez laissés à Jérusalem\\, tomberont par l’épée. 
${}^{22} Vous ferez alors comme je viens de faire : vous ne voilerez pas vos lèvres, vous ne prendrez pas le repas funéraire, 
${}^{23} vous mettrez vos turbans, et vous chausserez vos sandales. Vous ne ferez pas de lamentation, vous ne pleurerez pas. Mais vous pourrirez dans vos péchés, et vous gémirez tous ensemble. 
${}^{24} Ézékiel sera pour vous un signe\\ : tout ce qu’il a fait, vous le ferez. Et quand cela arrivera, vous saurez que Je suis le Seigneur Dieu.
${}^{25}Et toi, fils d’homme, le jour où je leur prendrai leur force, leur allégresse et leur parure, la joie de leurs yeux, le délice de leur vie, leurs fils et leurs filles, 
${}^{26}ce jour-là, arrivera vers toi un rescapé pour faire entendre la nouvelle. 
${}^{27}Ce jour-là, ta bouche s’ouvrira pour parler au rescapé : tu parleras et tu ne seras plus muet ; tu seras pour eux un signe. Alors ils sauront que Je suis le Seigneur. »
