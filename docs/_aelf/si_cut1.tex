  
  
    
    \bbook{BEN SIRA}{BEN SIRA}
      <sup>(1)</sup> La Loi, les Prophètes et les livres qui leur font suite nous ont transmis de nombreuses et grandes leçons, et il faut, à ce sujet, louer Israël pour son enseignement et sa sagesse. Or, il ne suffit pas d’acquérir le savoir par la lecture ; <sup>(5)</sup> il faut encore, une fois gagné par la passion d’apprendre, se rendre utile aux autres tant par la parole que par l’écrit. Aussi, mon grand-père Jésus, après s’être adonné sans réserve à la lecture de la Loi, des Prophètes <sup>(10)</sup> et des autres livres de nos Pères, et avoir acquis en ce domaine une grande compétence, a-t-il été amené à écrire lui-même sur l’enseignement et la sagesse. De la sorte, ceux qui ont la passion d’apprendre s’y appliqueront à leur tour et progresseront plus encore dans la vie selon la Loi.
      <sup>(15)</sup> Vous êtes donc invités à faire la lecture de cet ouvrage avec une bienveillante attention et à vous montrer indulgents s’il vous semble que, <sup>(20)</sup> en dépit de nos efforts de traduction, nous avons échoué à rendre telle ou telle expression. En effet, ce qui est exprimé à l’origine en hébreu n’a plus la même force une fois traduit dans une autre langue. D’ailleurs, non seulement pour cet ouvrage, mais aussi pour la Loi elle-même, les Prophètes <sup>(25)</sup> et les autres livres, la traduction présente des différences considérables avec l’original.
      C’est la trente-huitième année du règne de Ptolémée Évergète que je me suis rendu en Égypte. Au cours de mon séjour, j’ai trouvé une copie de cette importante instruction. <sup>(30)</sup> J’ai jugé alors qu’il était de la plus haute nécessité de mettre tout mon zèle et tous mes efforts à traduire ce livre. J’ai donc consacré beaucoup de veilles et de science, pendant cette période, pour mener à terme cet ouvrage et le publier à l’intention de ceux qui, à l’étranger, ont la passion d’apprendre <sup>(35)</sup> et veulent réformer leurs mœurs afin de vivre selon la Loi.
