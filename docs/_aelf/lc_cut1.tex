  
  
    
    \bbook{ÉVANGILE SELON SAINT LUC}{ÉVANGILE SELON SAINT LUC}
      <p class="cantique" id="bib_ct-nt_1"><span class="cantique_label">Cantique NT 1</span> = <span class="cantique_ref"><a class="unitex_link" href="#bib_lc_1_47">Lc 1, 47-55</a></span>
      <p class="cantique" id="bib_ct-nt_2"><span class="cantique_label">Cantique NT 2</span> = <span class="cantique_ref"><a class="unitex_link" href="#bib_lc_1_68">Lc 1, 68-79</a></span>
      
         
      \bchapter{}
      \begin{verse}
${}^{1}Beaucoup ont entrepris de composer un récit des événements qui se sont accomplis parmi nous, 
${}^{2}d’après ce que nous ont transmis ceux qui, dès le commencement, furent témoins oculaires et serviteurs de la Parole. 
${}^{3}C’est pourquoi j’ai décidé, moi aussi, après avoir recueilli avec précision des informations concernant tout ce qui s’est passé depuis le début, d’écrire pour toi, excellent Théophile, un exposé suivi, 
${}^{4}afin que tu te rendes bien compte de la solidité des enseignements que tu as entendus.
      
         
${}^{5}Il y avait, au temps d’Hérode le Grand, roi de Judée, un prêtre du groupe d’Abia, nommé Zacharie. Sa femme aussi était descendante d’Aaron ; elle s’appelait Élisabeth. 
${}^{6}Ils étaient l’un et l’autre des justes devant Dieu : ils suivaient tous les commandements et les préceptes du Seigneur de façon irréprochable. 
${}^{7}Ils n’avaient pas d’enfant, car Élisabeth était stérile et, de plus, ils étaient l’un et l’autre avancés en âge.
${}^{8}Or, tandis que Zacharie, durant la période attribuée aux prêtres de son groupe, assurait le service du culte devant Dieu, 
${}^{9}il fut désigné par le sort, suivant l’usage des prêtres, pour aller offrir l’encens dans le sanctuaire du Seigneur. 
${}^{10}Toute la multitude du peuple était en prière au dehors, à l’heure de l’offrande de l’encens. 
${}^{11}L’ange du Seigneur lui apparut, debout à droite de l’autel de l’encens. 
${}^{12}À sa vue, Zacharie fut bouleversé et la crainte le saisit.
${}^{13}L’ange lui dit : « Sois sans crainte, Zacharie, car ta supplication a été exaucée : ta femme Élisabeth mettra au monde pour toi un fils, et tu lui donneras le nom de Jean. 
${}^{14}Tu seras dans la joie et l’allégresse, et beaucoup se réjouiront de sa naissance, 
${}^{15}car il sera grand devant le Seigneur. Il ne boira pas de vin ni de boisson forte, et il sera rempli d’Esprit Saint dès le ventre de sa mère ; 
${}^{16}il fera revenir de nombreux fils d’Israël au Seigneur leur Dieu ; 
${}^{17}il marchera devant, en présence du Seigneur, avec l’esprit et la puissance du prophète Élie, pour faire revenir le cœur des pères vers leurs enfants, ramener les rebelles à la sagesse des justes, et préparer au Seigneur un peuple bien disposé. » 
${}^{18}Alors Zacharie dit à l’ange : « Comment vais-je savoir que cela arrivera ? Moi, en effet, je suis un vieillard et ma femme est avancée en âge. » 
${}^{19}L’ange lui répondit : « Je suis Gabriel et je me tiens en présence de Dieu. J’ai été envoyé pour te parler et pour t’annoncer cette bonne nouvelle. 
${}^{20}Mais voici que tu seras réduit au silence et, jusqu’au jour où cela se réalisera, tu ne pourras plus parler, parce que tu n’as pas cru à mes paroles ; celles-ci s’accompliront en leur temps. »
${}^{21}Le peuple attendait Zacharie et s’étonnait qu’il s’attarde dans le sanctuaire. 
${}^{22}Quand il sortit, il ne pouvait pas leur parler, et ils comprirent que, dans le sanctuaire, il avait eu une vision. Il leur faisait des signes et restait muet. 
${}^{23}Lorsqu’il eut achevé son temps de service liturgique, il repartit chez lui. 
${}^{24}Quelque temps plus tard, sa femme Élisabeth conçut un enfant. Pendant cinq mois, elle garda le secret. Elle se disait : 
${}^{25}« Voilà ce que le Seigneur a fait pour moi, en ces jours où il a posé son regard pour effacer ce qui était ma honte devant les hommes. »
${}^{26}Le sixième mois, l’ange Gabriel fut envoyé par Dieu dans une ville de Galilée, appelée Nazareth, 
${}^{27}à une jeune fille vierge, accordée en mariage à un homme de la maison de David, appelé Joseph ; et le nom de la jeune fille était Marie. 
${}^{28}L’ange entra chez elle et dit : « Je te salue, Comblée-de-grâce, le Seigneur est avec toi. » 
${}^{29}À cette parole, elle fut toute bouleversée, et elle se demandait ce que pouvait signifier cette salutation. 
${}^{30}L’ange lui dit alors : « Sois sans crainte, Marie, car tu as trouvé grâce auprès de Dieu. 
${}^{31}Voici que tu vas concevoir et enfanter un fils ; tu lui donneras le nom de Jésus. 
${}^{32}Il sera grand, il sera appelé Fils du Très-Haut ; le Seigneur Dieu lui donnera le trône de David son père ; 
${}^{33}il régnera pour toujours sur la maison de Jacob, et son règne n’aura pas de fin. »
${}^{34}Marie dit à l’ange : « Comment cela va-t-il se faire puisque je ne connais pas d’homme ? » 
${}^{35}L’ange lui répondit : « L’Esprit Saint viendra sur toi, et la puissance du Très-Haut te prendra sous son ombre ; c’est pourquoi celui qui va naître sera saint, il sera appelé Fils de Dieu. 
${}^{36}Or voici que, dans sa vieillesse, Élisabeth, ta parente, a conçu, elle aussi, un fils et en est à son sixième mois, alors qu’on l’appelait la femme stérile. 
${}^{37}Car rien n’est impossible à Dieu. » 
${}^{38}Marie dit alors : « Voici la servante du Seigneur ; que tout m’advienne selon ta parole. » Alors l’ange la quitta.
${}^{39}En ces jours-là, Marie se mit en route et se rendit avec empressement vers la région montagneuse, dans une ville de Judée. 
${}^{40}Elle entra dans la maison de Zacharie et salua Élisabeth. 
${}^{41}Or, quand Élisabeth entendit la salutation de Marie, l’enfant tressaillit en elle. Alors, Élisabeth fut remplie d’Esprit Saint, 
${}^{42}et s’écria d’une voix forte : « Tu es bénie entre toutes les femmes, et le fruit de tes entrailles est béni. 
${}^{43}D’où m’est-il donné que la mère de mon Seigneur vienne jusqu’à moi ? 
${}^{44}Car, lorsque tes paroles de salutation sont parvenues à mes oreilles, l’enfant a tressailli d’allégresse en moi. 
${}^{45}Heureuse celle qui a cru à l’accomplissement des paroles qui lui furent dites de la part du Seigneur. »
${}^{46}Marie dit alors :
       
        \\« Mon âme exalte le Seigneur,
        ${}^{47}exulte mon esprit\\en Dieu, mon Sauveur !
         
        ${}^{48}Il s’est penché sur son humble servante\\ ;
        \\désormais tous les âges me diront bienheureuse.
        ${}^{49}Le Puissant fit pour moi des merveilles\\ ;
        \\Saint est son nom !
         
        ${}^{50}Sa miséricorde\\s’étend\\d’âge en âge
        \\sur ceux qui le craignent.
        ${}^{51}Déployant\\la force de son bras,
        \\il disperse les superbes\\.
         
        ${}^{52}Il renverse les puissants de leurs trônes,
        \\il élève les humbles.
        ${}^{53}Il comble de biens les affamés,
        \\renvoie les riches les mains\\vides.
         
        ${}^{54}Il relève Israël son serviteur,
        \\il se souvient de son amour\\,
        ${}^{55}de la promesse faite à nos pères,
        \\en faveur d’Abraham et sa descendance à jamais. »
       
${}^{56}Marie resta avec Élisabeth environ trois mois, puis elle s’en retourna chez elle.
${}^{57}Quand fut accompli le temps où Élisabeth devait enfanter, elle mit au monde un fils. 
${}^{58}Ses voisins et sa famille apprirent que le Seigneur lui avait montré la grandeur de sa miséricorde, et ils se réjouissaient avec elle. 
${}^{59}Le huitième jour, ils vinrent pour la circoncision de l’enfant. Ils voulaient l’appeler Zacharie, du nom de son père. 
${}^{60}Mais sa mère prit la parole et déclara : « Non, il s’appellera Jean. » 
${}^{61}On lui dit : « Personne dans ta famille ne porte ce nom-là ! » 
${}^{62}On demandait par signes au père comment il voulait l’appeler. 
${}^{63}Il se fit donner une tablette sur laquelle il écrivit : « Jean est son nom. » Et tout le monde en fut étonné. 
${}^{64}À l’instant même, sa bouche s’ouvrit, sa langue se délia : il parlait et il bénissait Dieu. 
${}^{65}La crainte saisit alors tous les gens du voisinage et, dans toute la région montagneuse de Judée, on racontait tous ces événements. 
${}^{66}Tous ceux qui les apprenaient les conservaient dans leur cœur et disaient : « Que sera donc cet enfant ? » En effet, la main du Seigneur était avec lui.
${}^{67}Zacharie, son père, fut rempli d’Esprit Saint et prononça ces paroles prophétiques :
       
        ${}^{68}« Béni soit le Seigneur, le Dieu d’Israël,
        \\qui visite et rachète son peuple\\.
         
        ${}^{69}Il a fait surgir la force qui nous sauve
        \\dans la maison de David, son serviteur,
        ${}^{70}comme il l’avait dit\\par la bouche des saints,
        \\par ses prophètes\\, depuis les temps anciens :
         
        ${}^{71}salut qui nous arrache à l’ennemi,
        \\à la main de tous nos oppresseurs,
         
        ${}^{72}amour qu’il montre envers nos pères,
        \\mémoire de son alliance sainte\\,
         
        ${}^{73}serment juré à notre père Abraham
        \\de nous rendre sans crainte,
         
        ${}^{74}afin que, délivrés de la main des ennemis,
        ${}^{75}nous le servions dans la justice et la sainteté,
        \\en sa présence, tout au long de nos jours.
         
        ${}^{76}Toi aussi\\, petit enfant, tu seras appelé
        prophète du Très-Haut ;
        \\tu marcheras devant, à la face du Seigneur,
        et tu prépareras ses chemins
         
        ${}^{77}pour donner à son peuple de connaître\\le salut
        \\par la rémission de ses péchés,
         
        ${}^{78}grâce à la tendresse, à l’amour\\de notre Dieu,
        \\quand nous visite l’astre d’en haut,
         
        ${}^{79}pour illuminer ceux qui habitent les ténèbres
        et l’ombre de la mort,
        \\pour conduire\\nos pas
        au chemin de la paix. »
       
${}^{80}L’enfant grandissait et son esprit se fortifiait. Il alla vivre au désert jusqu’au jour où il se fit connaître à Israël.
      <p class="cantique" id="bib_ct-nt_3"><span class="cantique_label">Cantique NT 3</span> = <span class="cantique_ref"><a class="unitex_link" href="#bib_lc_2_29">Lc 2, 29-32</a></span>
      
         
      \bchapter{}
      \begin{verse}
${}^{1}En ces jours-là, parut un édit de l’empereur Auguste, ordonnant de recenser toute la terre – 
${}^{2}ce premier recensement eut lieu lorsque Quirinius était gouverneur de Syrie. – 
${}^{3}Et tous allaient se faire recenser, chacun dans sa ville d’origine.
${}^{4}Joseph, lui aussi, monta de Galilée, depuis la ville de Nazareth, vers la Judée, jusqu’à la ville de David appelée Bethléem. Il était en effet de la maison et de la lignée de David. 
${}^{5}Il venait se faire recenser avec Marie, qui lui avait été accordée en mariage et qui était enceinte. 
${}^{6}Or, pendant qu’ils étaient là, le temps où elle devait enfanter fut accompli. 
${}^{7}Et elle mit au monde son fils premier-né ; elle l’emmaillota et le coucha dans une mangeoire, car il n’y avait pas de place pour eux dans la salle commune.
${}^{8}Dans la même région, il y avait des bergers qui vivaient dehors et passaient la nuit dans les champs pour garder leurs troupeaux. 
${}^{9}L’ange du Seigneur se présenta devant eux, et la gloire du Seigneur les enveloppa de sa lumière. Ils furent saisis d’une grande crainte. 
${}^{10}Alors l’ange leur dit : « Ne craignez pas, car voici que je vous annonce une bonne nouvelle, qui sera une grande joie pour tout le peuple : 
${}^{11}Aujourd’hui, dans la ville de David, vous est né un Sauveur qui est le Christ, le Seigneur. 
${}^{12}Et voici le signe qui vous est donné : vous trouverez un nouveau-né emmailloté et couché dans une mangeoire. » 
${}^{13}Et soudain, il y eut avec l’ange une troupe céleste innombrable, qui louait Dieu en disant :
        ${}^{14}« Gloire à Dieu au plus haut des cieux,
        \\et paix sur la terre aux hommes, qu’Il aime. »
${}^{15}Lorsque les anges eurent quitté les bergers pour le ciel, ceux-ci se disaient entre eux : « Allons jusqu’à Bethléem pour voir ce qui est arrivé, l’événement que le Seigneur nous a fait connaître. » 
${}^{16}Ils se hâtèrent d’y aller, et ils découvrirent Marie et Joseph, avec le nouveau-né couché dans la mangeoire. 
${}^{17}Après avoir vu, ils racontèrent ce qui leur avait été annoncé au sujet de cet enfant. 
${}^{18}Et tous ceux qui entendirent s’étonnaient de ce que leur racontaient les bergers. 
${}^{19}Marie, cependant, retenait tous ces événements et les méditait dans son cœur. 
${}^{20}Les bergers repartirent ; ils glorifiaient et louaient Dieu pour tout ce qu’ils avaient entendu et vu, selon ce qui leur avait été annoncé.
${}^{21}Quand fut arrivé le huitième jour, celui de la circoncision, l’enfant reçut le nom de Jésus, le nom que l’ange lui avait donné avant sa conception.
${}^{22}Quand fut accompli le temps prescrit par la loi de Moïse pour la purification, les parents de Jésus l’amenèrent à Jérusalem pour le présenter au Seigneur, 
${}^{23}selon ce qui est écrit dans la Loi : Tout premier-né de sexe masculin sera consacré au Seigneur. 
${}^{24}Ils venaient aussi offrir le sacrifice prescrit par la loi du Seigneur : un couple de tourterelles ou deux petites colombes.
${}^{25}Or, il y avait à Jérusalem un homme appelé Syméon. C’était un homme juste et religieux, qui attendait la Consolation d’Israël, et l’Esprit Saint était sur lui. 
${}^{26}Il avait reçu de l’Esprit Saint l’annonce qu’il ne verrait pas la mort avant d’avoir vu le Christ, le Messie du Seigneur. 
${}^{27}Sous l’action de l’Esprit, Syméon vint au Temple. Au moment où les parents présentaient l’enfant Jésus pour se conformer au rite de la Loi qui le concernait, 
${}^{28}Syméon reçut l’enfant dans ses bras, et il bénit Dieu en disant :
       
        ${}^{29}« Maintenant, ô Maître souverain,
        \\tu peux laisser ton serviteur s’en aller
        \\en paix, selon ta parole.
         
        ${}^{30}Car mes yeux ont vu le salut
        ${}^{31}que tu préparais à la face des peuples\\ :
         
        ${}^{32}lumière qui se révèle aux nations
        \\et donne gloire à ton peuple Israël. »
       
${}^{33}Le père et la mère de l’enfant s’étonnaient de ce qui était dit de lui. 
${}^{34}Syméon les bénit, puis il dit à Marie sa mère : « Voici que cet enfant provoquera la chute et le relèvement de beaucoup en Israël. Il sera un signe de contradiction 
${}^{35}– et toi, ton âme sera traversée d’un glaive – : ainsi seront dévoilées les pensées qui viennent du cœur d’un grand nombre. »
${}^{36}Il y avait aussi une femme prophète, Anne, fille de Phanuel, de la tribu d’Aser. Elle était très avancée en âge ; après sept ans de mariage, 
${}^{37}demeurée veuve, elle était arrivée à l’âge de quatre-vingt-quatre ans. Elle ne s’éloignait pas du Temple, servant Dieu jour et nuit dans le jeûne et la prière. 
${}^{38}Survenant à cette heure même, elle proclamait les louanges de Dieu et parlait de l’enfant à tous ceux qui attendaient la délivrance de Jérusalem.
${}^{39}Lorsqu’ils eurent achevé tout ce que prescrivait la loi du Seigneur, ils retournèrent en Galilée, dans leur ville de Nazareth.
${}^{40}L’enfant, lui, grandissait et se fortifiait, rempli de sagesse, et la grâce de Dieu était sur lui.
${}^{41}Chaque année, les parents de Jésus se rendaient à Jérusalem pour la fête de la Pâque. 
${}^{42}Quand il eut douze ans, ils montèrent en pèlerinage suivant la coutume. 
${}^{43}À la fin de la fête, comme ils s’en retournaient, le jeune Jésus resta à Jérusalem à l’insu de ses parents. 
${}^{44}Pensant qu’il était dans le convoi des pèlerins, ils firent une journée de chemin avant de le chercher parmi leurs parents et connaissances. 
${}^{45}Ne le trouvant pas, ils retournèrent à Jérusalem, en continuant à le chercher. 
${}^{46}C’est au bout de trois jours qu’ils le trouvèrent dans le Temple, assis au milieu des docteurs de la Loi : il les écoutait et leur posait des questions, 
${}^{47}et tous ceux qui l’entendaient s’extasiaient sur son intelligence et sur ses réponses. 
${}^{48}En le voyant, ses parents furent frappés d’étonnement, et sa mère lui dit : « Mon enfant, pourquoi nous as-tu fait cela ? Vois comme ton père et moi, nous avons souffert en te cherchant ! » 
${}^{49}Il leur dit : « Comment se fait-il que vous m’ayez cherché ? Ne saviez-vous pas qu’il me faut être chez mon Père ? » 
${}^{50}Mais ils ne comprirent pas ce qu’il leur disait.
${}^{51}Il descendit avec eux pour se rendre à Nazareth, et il leur était soumis. Sa mère gardait dans son cœur tous ces événements. 
${}^{52}Quant à Jésus, il grandissait en sagesse, en taille et en grâce, devant Dieu et devant les hommes.
