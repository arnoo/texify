  
  
          
            \bchapter{Psaume}
            Longtemps, j’ai cherché à savoir
${}^{1}Psaume. D’Asaph.
         
        \\Vraiment, Dieu est b\underline{o}n pour Israël,
        \\pour les h\underline{o}mmes au cœur pur.
         
${}^{2}Un rien, et je p\underline{e}rdais pied,
        \\un peu plus, et je fais\underline{a}is un faux pas ;
${}^{3}car j’étais jalo\underline{u}x des superbes,
        \\je voyais le succ\underline{è}s des impies.
         
${}^{4}Jusqu’à leur mort, ils ne m\underline{a}nquent de rien,
        \\ils jouissent d’une sant\underline{é} parfaite ;
${}^{5}ils échappent aux souffr\underline{a}nces des hommes,
        \\aux coups qui fr\underline{a}ppent les mortels.
         
${}^{6}Ainsi, l’orgu\underline{e}il est leur collier,
        \\la violence, l’hab\underline{i}t qui les couvre ;
${}^{7}leurs yeux qui br\underline{i}llent de bien-être
        \\trahissent les env\underline{i}es de leur cœur.
         
${}^{8}Ils ricanent, ils pr\underline{ô}nent le mal,
        \\de très haut, ils pr\underline{ô}nent la force ;
${}^{9}leur bouche accap\underline{a}re le ciel,
        \\et leur langue parco\underline{u}rt la terre.
         
${}^{10}Ainsi, le peuple se déto\underline{u}rne
        \\vers la source d’une t\underline{e}lle abondance.
${}^{11}Ils disent : « Comment Die\underline{u} saurait-il ?
        \\le Très-Haut, qu\underline{e} peut-il savoir ? »
         
${}^{12}Voyez comme s\underline{o}nt les impies :
        \\tranquilles, ils am\underline{a}ssent des fortunes.
         
        *
         
${}^{13}Vraiment, c’est en vain que j’ai gard\underline{é} mon cœur pur,
        \\lavé mes mains en s\underline{i}gne d’innocence !
${}^{14}Me voici frapp\underline{é} chaque jour,
        \\châti\underline{é} dès le matin.
         
${}^{15}Si j’avais dit : « Je vais parl\underline{e}r comme eux »,
        \\j’aurais trahi la r\underline{a}ce de tes fils.
${}^{16}Longtemps, j’ai cherch\underline{é} à savoir,
        \\je me suis donn\underline{é} de la peine.
         
${}^{17}Mais quand j’entrai dans la deme\underline{u}re de Dieu,
        \\je compris quel ser\underline{a}it leur avenir.
${}^{18}Vraiment, tu les as m\underline{i}s sur la pente :
        \\déjà tu les entr\underline{a}înes vers la ruine.
         
${}^{19}Comment vont-ils soud\underline{a}in au désastre,
        \\anéantis, achev\underline{é}s par la terreur ?
${}^{20}À ton réveil, Seigneur, tu ch\underline{a}sses leur image,
        \\comme un songe au sort\underline{i}r du sommeil.
         
        *
         
${}^{21}Oui, mon cœ\underline{u}r s’aigrissait,
        \\j’avais les r\underline{e}ins transpercés.
${}^{22}Moi, stup\underline{i}de, comme une bête,
        \\je ne savais pas, mais j’ét\underline{a}is avec toi.
         
${}^{23}Moi, je suis toujo\underline{u}rs avec toi,
        \\avec toi qui as sais\underline{i} ma main droite.
${}^{24}Tu me conduis sel\underline{o}n tes desseins ;
        \\puis tu me prendr\underline{a}s dans la gloire.
         
${}^{25}Qui donc est pour m\underline{o}i dans le ciel
        \\si je n’ai, même avec toi, aucune j\underline{o}ie sur la terre ?
${}^{26}Ma chair et mon cœ\underline{u}r sont usés :
        \\ma part, le roc de mon cœur, c’est Die\underline{u} pour toujours.
         
${}^{27}Qui s’éloigne de t\underline{o}i périra :
        \\tu détruis ce\underline{u}x qui te délaissent.
${}^{28}Pour moi, il est bon d’être pr\underline{o}che de Dieu ;
        \\j’ai pris refuge aupr\underline{è}s de mon Dieu
        \\pour annoncer les œ\underline{u}vres du Seigneur
        \\aux p\underline{o}rtes de Sion.
      \bchapter{Psaume}
          
            \bchapter{Psaume}
            Il n’y a plus de prophètes !
${}^{1}Poème. D’Asaph.
         
        \\Pourquoi, Dieu, nous rejet\underline{e}r sans fin ?
        \\Pourquoi cette colère sur les breb\underline{i}s de ton troupeau ?
         
${}^{2}Rappelle-toi la communauté
        que tu acqu\underline{i}s dès l’origine, +
        \\la tribu que tu revendiqu\underline{a}s pour héritage,
        \\la montagne de Sion où tu f\underline{i}s ta demeure.
         
${}^{3}Dirige tes pas vers ces ru\underline{i}nes sans fin,
        \\l’ennemi dans le sanctuaire a to\underline{u}t saccagé ;
${}^{4}dans le lieu de tes assemblées, l’advers\underline{a}ire a rugi
        \\et là, il a plant\underline{é} ses insignes.
         
${}^{5}On les a vus brandir la cognée,
        comme en pl\underline{e}ine forêt, *
${}^{6}quand ils brisaient les portails
        à coups de m\underline{a}sse et de hache.
         
${}^{7}Ils ont livré au fe\underline{u} ton sanctuaire,
        \\profané et rasé la deme\underline{u}re de ton nom.
${}^{8}Ils ont dit : « All\underline{o}ns ! Détruisons tout ! »
        \\Ils ont brûlé dans le pays les lie\underline{u}x d’assemblées saintes.
         
${}^{9}Nos signes, nul ne les voit ;
        il n’y a pl\underline{u}s de prophètes ! *
        \\Et pour combien de temps ?
        Nul d’entre no\underline{u}s ne le sait !
         
        *
         
${}^{10}Dieu, combien de temps blasphémer\underline{a} l’adversaire ?
        \\L’ennemi en finira-t-il de mépris\underline{e}r ton nom ?
${}^{11}Pourquoi reten\underline{i}r ta main,
        \\cacher la f\underline{o}rce de ton bras ?
         
${}^{12}Pourtant, Dieu, mon r\underline{o}i dès l’origine,
        \\vainqueur des combats sur la f\underline{a}ce de la terre,
${}^{13}c’est toi qui fendis la m\underline{e}r par ta puissance,
        \\qui fracassas les têtes des drag\underline{o}ns sur les eaux ;
         
${}^{14}toi qui écrasas la t\underline{ê}te de Léviathan
        \\pour nourrir les m\underline{o}nstres marins ;
${}^{15}toi qui ouvris les torr\underline{e}nts et les sources,
        \\toi qui mis à sec des fle\underline{u}ves intarissables.
         
${}^{16}À toi le jour, à t\underline{o}i la nuit,
        \\toi qui ajustas le sol\underline{e}il et les astres !
${}^{17}C’est toi qui fixas les b\underline{o}rds de la terre ;
        \\l’hiver et l’été, c’est t\underline{o}i qui les formas.
         
        *
         
${}^{18}Rappelle-toi : l’ennemi a mépris\underline{é} ton nom,
        \\un peuple de fous a blasphém\underline{é} le Seigneur.
${}^{19}Ne laisse pas la Bête égorg\underline{e}r ta Tourterelle,
        \\n’oublie pas sans fin la v\underline{i}e de tes pauvres.
         
${}^{20}Regarde vers l’Alliance : la gu\underline{e}rre est partout ;
        \\on se cache dans les cav\underline{e}rnes du pays.
${}^{21}Que l’opprimé éch\underline{a}ppe à la honte,
        \\que le pauvre et le malheureux ch\underline{a}ntent ton nom !
         
${}^{22}Lève-toi, Die\underline{u}, défends ta cause !
        \\Rappelle-toi ces fous qui blasph\underline{è}ment tout le jour.
${}^{23}N’oublie pas le vacarme que f\underline{o}nt tes ennemis,
        \\la clameur de l’ennemi, qui m\underline{o}nte sans fin.
      \bchapter{Psaume}
          
            \bchapter{Psaume}
            C’est Dieu qui jugera
${}^{1}Du maître de chœur. Sur l’air de « Ne détruis pas ». Psaume. D’Asaph. Cantique.
         
${}^{2}À toi, Die\underline{u}, nous rendons grâce ; +
        \\nous rendons grâce, et ton n\underline{o}m est proche :
        \\on procl\underline{a}me tes merveilles !
         
${}^{3}« Oui, au mom\underline{e}nt que j’ai fixé,
        \\moi, je juger\underline{a}i avec droiture.
${}^{4}Que s’effondrent la t\underline{e}rre et ses habitants :
        \\moi seul en ai pos\underline{é} les colonnes !
         
${}^{5}« Aux arrogants, je d\underline{i}s : Plus d’arrogance !
        \\et aux impies : Ne levez p\underline{a}s votre front !
${}^{6}Ne levez pas votre fr\underline{o}nt contre le ciel,
        \\ne parlez pas en le pren\underline{a}nt de haut ! »
         
${}^{7}Ce n’est pas du lev\underline{a}nt ni du couchant,
        \\ni du désert, que vi\underline{e}nt le relèvement.
${}^{8}Non, c’est Die\underline{u} qui jugera :
        \\il abaisse les uns, les a\underline{u}tres il les relève.
         
${}^{9}Le Seigneur tient en m\underline{a}in une coupe
        \\où fermente un v\underline{i}n capiteux ;
        \\il le verse, et tous les imp\underline{i}es de la terre
        \\le boir\underline{o}nt jusqu’à la lie.
         
${}^{10}Et moi, j’annoncer\underline{a}i toujours
        dans mes hymnes au Die\underline{u} de Jacob : +
${}^{11}« Je briserai le fr\underline{o}nt des impies, *
        \\et le front du j\underline{u}ste s’élèvera ! »
      \bchapter{Psaume}
          
            \bchapter{Psaume}
            Quand Dieu se lève pour juger
${}^{1}Du maître de chœur. Sur les instruments à corde. Psaume. D’Asaph. Cantique.
         
${}^{2}Dieu s’est fait conn\underline{a}ître en Juda ;
        \\en Israël, son n\underline{o}m est grand.
${}^{3}À Salem il a fix\underline{é} sa tente,
        \\et sa deme\underline{u}re à Sion.
${}^{4}Ici, il a bris\underline{é} les traits de l’arc,
        \\l’épée, le boucli\underline{e}r et la guerre.
         
${}^{5}Magnifique, t\underline{o}i, tu resplendis
        \\au-dessus d’une mont\underline{a}gne de butin.
${}^{6}Les voici dépouill\underline{é}s, ces guerriers,
        \\endormis, tous ces br\underline{a}ves aux mains inertes.
${}^{7}Sous ta menace, Die\underline{u} de Jacob,
        \\le char et le chev\underline{a}l se sont figés.
         
${}^{8}Toi, tu \underline{e}s le redoutable !
        \\Qui tiendra sous les co\underline{u}ps de ta fureur ?
${}^{9}Des cieux, tu pron\underline{o}nces le verdict ;
        \\la terre a pe\underline{u}r et se tait
${}^{10}quand Dieu se l\underline{è}ve pour juger,
        \\pour sauver tous les h\underline{u}mbles de la terre.
         
${}^{11}La colère des h\underline{o}mmes te rend gloire
        \\quand les surviv\underline{a}nts te font cortège.
${}^{12}Faites des vœux et tenez vos promesses
        au Seigne\underline{u}r votre Dieu ; *
        \\vous qui l’entourez,
        portez votre offr\underline{a}nde au redoutable.
${}^{13}Il éteint le so\underline{u}ffle des princes,
        \\lui, redoutable aux r\underline{o}is de la terre !
      \bchapter{Psaume}
          
            \bchapter{Psaume}
            Dieu oublierait-il d’avoir pitié ?
${}^{1}Du maître de chœur. D’après Yedoutoune. D’Asaph. Psaume.
         
${}^{2}Vers Dieu, je cr\underline{i}e mon appel !
        \\Je crie vers Die\underline{u} : qu’il m’entende !
         
${}^{3}Au jour de la détresse, je ch\underline{e}rche le Seigneur ; +
        \\la nuit, je tends les m\underline{a}ins sans relâche,
        \\mon âme ref\underline{u}se le réconfort.
         
${}^{4}Je me souviens de Die\underline{u}, je me plains ;
        \\je médite et mon espr\underline{i}t défaille.
${}^{5}Tu refuses à mes ye\underline{u}x le sommeil ;
        \\je me trouble, incap\underline{a}ble de parler.
         
${}^{6}Je pense aux jo\underline{u}rs d’autrefois,
        \\aux ann\underline{é}es de jadis ;
${}^{7}la nuit, je me souvi\underline{e}ns de mon chant,
        \\je médite en mon cœur, et mon espr\underline{i}t s’interroge.
         
${}^{8}Le Seigneur ne fera-t-\underline{i}l que rejeter,
        \\ne sera-t-il jamais pl\underline{u}s favorable ?
${}^{9}Son amour a-t-il d\underline{o}nc disparu ?
        \\S’est-elle éteinte, d’âge en \underline{â}ge, la parole ?
         
${}^{10}Dieu oublierait-\underline{i}l d’avoir pitié,
        \\dans sa colère a-t-il ferm\underline{é} ses entrailles ?
${}^{11}J’ai dit : « Une ch\underline{o}se me fait mal,
        \\la droite du Très-Ha\underline{u}t a changé. »
         
${}^{12}Je me souviens des expl\underline{o}its du Seigneur,
        \\je rappelle ta merv\underline{e}ille de jadis ;
${}^{13}je me red\underline{i}s tous tes hauts faits,
        \\sur tes expl\underline{o}its je médite.
         
        *
         
${}^{14}Dieu, la saintet\underline{é} est ton chemin !
        \\Quel dieu est gr\underline{a}nd comme Dieu ?
         
${}^{15}Tu es le Dieu qui accompl\underline{i}s la merveille,
        \\qui fais connaître chez les pe\underline{u}ples ta force :
${}^{16}tu rachetas ton pe\underline{u}ple avec puissance,
        \\les descendants de Jac\underline{o}b et de Joseph.
         
${}^{17}Les eaux, en te voy\underline{a}nt, Seigneur, +
        \\les eaux, en te voy\underline{a}nt, tremblèrent,
        \\l’abîme lui-m\underline{ê}me a frémi.
         
${}^{18}Les nuages dévers\underline{è}rent leurs eaux, +
        \\les nuées donn\underline{è}rent de la voix,
        \\la foudre frapp\underline{a}it de toute part.
         
${}^{19}Au roulement de ta v\underline{o}ix qui tonnait, +
        \\tes éclairs illumin\underline{è}rent le monde,
        \\la terre s’agit\underline{a} et frémit.
         
${}^{20}Par la mer pass\underline{a}it ton chemin, +
        \\tes sentiers, par les ea\underline{u}x profondes ;
        \\et nul n’en conn\underline{a}ît la trace.
         
${}^{21}Tu as conduit comme un troupea\underline{u} ton peuple
        \\par la main de Mo\underline{ï}se et d’Aaron.
      \bchapter{Psaume}
          
            \bchapter{Psaume}
            Qu’ils n’oublient pas…
${}^{1}Poème. D’Asaph.
         
        \\Écoute ma l\underline{o}i, ô mon peuple,
        \\tends l’oreille aux par\underline{o}les de ma bouche.
${}^{2}J’ouvrirai la bouche pour une p\underline{a}rabole,
        \\je publierai ce qui fut cach\underline{é} dès l’origine.
         
${}^{3}Nous avons entend\underline{u} et nous savons
        \\ce que nos pères nous \underline{o}nt raconté ;
${}^{4}nous le redirons à l’\underline{â}ge qui vient,
        \\sans rien cach\underline{e}r à nos descendants :
        \\les titres de gl\underline{o}ire du Seigneur,
        \\sa puissance et les merv\underline{e}illes qu’il a faites.
         
${}^{5}Il fixa une r\underline{è}gle en Jacob,
        \\il établit en Isra\underline{ë}l une loi,
        \\loi qui ordonn\underline{a}it à nos pères
        \\d’enseigner ces ch\underline{o}ses à leur fils,
${}^{6}pour que l’âge suiv\underline{a}nt les connaisse,
        \\et leur descend\underline{a}nce à venir.
         
        \\Qu’ils se lèvent et les rac\underline{o}ntent à leurs fils
${}^{7}pour qu’ils placent en Die\underline{u} leur espoir
        \\et n’oublient pas les expl\underline{o}its du Seigneur
        \\mais obs\underline{e}rvent ses commandements.
         
${}^{8}Qu’ils ne soient pas, c\underline{o}mme leurs pères,
        \\une génération indoc\underline{i}le et rebelle,
        \\génération de cœ\underline{u}rs inconstants
        \\et d’esprits infid\underline{è}les à Dieu.
         
        *
         
${}^{9}Les fils d’Éphra\underline{ï}m, archers d’élite,
        \\se sont enfuis, le jo\underline{u}r du combat :
${}^{10}ils n’ont pas gardé l’alli\underline{a}nce de Dieu,
        \\ils refusaient de su\underline{i}vre sa loi ;
${}^{11}ils avaient oubli\underline{é} ses exploits,
        \\les merveilles dont ils f\underline{u}rent les témoins.
         
${}^{12}Devant leurs pères il accompl\underline{i}t un miracle
        \\en Égypte, au pa\underline{y}s de Tanis :
${}^{13}il fend la mer, il les f\underline{a}it passer,
        \\dressant les ea\underline{u}x comme une digue ;
${}^{14}le jour, il les condu\underline{i}t par la nuée,
        \\et la nuit, par la lumi\underline{è}re d’un feu.
         
${}^{15}Il fend le roch\underline{e}r du désert,
        \\les désaltère aux ea\underline{u}x profondes ;
${}^{16}de la roche, il t\underline{i}re des ruisseaux
        \\qu’il fait déval\underline{e}r comme un fleuve.
         
${}^{17}Mais ils péchaient enc\underline{o}re contre lui,
        \\dans les lieux arides ils brav\underline{a}ient le Très-Haut ;
${}^{18}ils tentaient le Seigne\underline{u}r dans leurs cœurs,
        \\ils réclamèrent de mang\underline{e}r à leur faim.
         
${}^{19}Ils s’en prennent à Die\underline{u} et demandent :
        \\« Dieu peut-il apprêter une t\underline{a}ble au désert ?
${}^{20}Sans doute, il a frapp\underline{é} le rocher :
        \\l’eau a jailli, elle co\underline{u}le à flots !
        \\Mais pourra-t-il nous donn\underline{e}r du pain
        \\et procurer de la vi\underline{a}nde à son peuple ? »
         
${}^{21}Alors le Seigneur entendit et s’emporta,
        il s’enflamma de fure\underline{u}r contre Jacob, *
        \\sa colère mont\underline{a} contre Israël,
${}^{22}car ils n’avaient pas f\underline{o}i en Dieu,
        \\ils ne croyaient p\underline{a}s qu’il les sauverait.
         
${}^{23}Il commande aux nu\underline{é}es là-haut,
        \\il ouvre les écl\underline{u}ses du ciel :
${}^{24}pour les nourrir il fait pleuv\underline{o}ir la manne,
        \\il leur donne le from\underline{e}nt du ciel ;
${}^{25}chacun se nourr\underline{i}t du pain des Forts,
        \\il les pourvoit de v\underline{i}vres à satiété.
         
${}^{26}Dans le ciel, il po\underline{u}sse le vent d’est
        \\et lance le grand v\underline{e}nt du midi.
${}^{27}Sur eux il fait pleuvoir une nu\underline{é}e d’oiseaux,
        \\autant de viande que de s\underline{a}ble au bord des mers.
         
${}^{28}Elle s’abat au milie\underline{u} de leur camp
        \\tout auto\underline{u}r de leurs demeures.
${}^{29}Ils mangent, ils s\underline{o}nt rassasiés,
        \\Dieu content\underline{a}it leur envie.
         
${}^{30}Mais leur envie n’était pas satisfaite,
        ils avaient enc\underline{o}re la bouche pleine, *
${}^{31}quand s’éleva la col\underline{è}re de Dieu :
        \\il frappe les plus vaill\underline{a}nts d’entre eux
        \\et terrasse la jeun\underline{e}sse d’Israël.
         
${}^{32}Et pourtant ils péch\underline{a}ient encore,
        \\ils n’avaient pas f\underline{o}i en ses merveilles.
${}^{33}D’un souffle il ach\underline{è}ve leurs jours,
        \\et leurs ann\underline{é}es en un moment.
         
${}^{34}Quand Dieu les frapp\underline{a}it, ils le cherchaient,
        \\ils revenaient et se tourn\underline{a}ient vers lui :
${}^{35}ils se souvenaient que Die\underline{u} est leur rocher,
        \\et le Dieu Très-Ha\underline{u}t, leur rédempteur.
         
${}^{36}Mais de leur bo\underline{u}che ils le trompaient,
        \\de leur l\underline{a}ngue ils lui mentaient.
${}^{37}Leur cœur n’était pas const\underline{a}nt envers lui ;
        \\ils n’étaient pas fid\underline{è}les à son alliance.
         
${}^{38}Et lui, m\underline{i}séricordieux,
        \\au lieu de détru\underline{i}re, il pardonnait ;
        \\maintes fois, il ret\underline{i}nt sa colère
        \\au lieu de réveill\underline{e}r sa violence.
${}^{39}Il se rappelait : ils ne s\underline{o}nt que chair,
        \\un souffle qui s’en v\underline{a} sans retour.
         
${}^{40}Que de fois au dés\underline{e}rt ils l’ont bravé,
        \\offensé d\underline{a}ns les solitudes !
${}^{41}De nouveau ils t\underline{e}ntaient Dieu,
        \\ils attristaient le S\underline{a}int d’Israël.
${}^{42}Ils avaient oubli\underline{é} ce jour
        \\où sa main les sauv\underline{a} de l’adversaire.
         
        *
         
${}^{43}Par ses signes il frapp\underline{a} l’Égypte,
        \\et le pays de Tan\underline{i}s par ses prodiges.
${}^{44}Il transforme en s\underline{a}ng l’eau des fleuves
        \\et les ruisseaux, pour qu’ils ne b\underline{o}ivent pas.
${}^{45}Il leur envoie une verm\underline{i}ne qui les ronge,
        \\des grenouilles qui inf\underline{e}stent tout.
         
${}^{46}Il livre les réc\underline{o}ltes aux sauterelles
        \\et le fruit de leur trav\underline{a}il aux insectes.
${}^{47}Il ravage leurs v\underline{i}gnes par les grêlons
        \\et leurs figui\underline{e}rs par le gel.
         
${}^{48}Il abandonne le bét\underline{a}il à la grêle
        \\et les troupea\underline{u}x à la foudre.
${}^{49}Il lâche sur eux le feu de sa colère,
        indignation, fure\underline{u}r, effroi, *
        \\il envoie des \underline{a}nges de malheur.
         
${}^{50}Il ouvre la route à sa colère,
        il abandonne leur \underline{â}me à la mort, *
        \\et livre leur v\underline{i}e à la peste.
${}^{51}Il frappe tous les fils aîn\underline{é}s de l’Égypte,
        \\sous les tentes de Cham, la fle\underline{u}r de sa race.
         
${}^{52}Tel un berger, il condu\underline{i}t son peuple,
        \\il pousse au dés\underline{e}rt son troupeau.
${}^{53}Il les guide et les déf\underline{e}nd, il les rassure ;
        \\leurs ennemis sont englout\underline{i}s par la mer.
         
${}^{54}Il les fait entrer dans son dom\underline{a}ine sacré,
        \\la montagne acqu\underline{i}se par sa main.
${}^{55}Il chasse des nations devant eux,
        il délimite leurs p\underline{a}rts d’héritage *
        \\et il installe sous leurs tentes les trib\underline{u}s d’Israël.
         
${}^{56}Mais ils bravaient, ils tent\underline{a}ient le Dieu Très-Haut,
        \\ils refusaient d’observ\underline{e}r ses lois ;
${}^{57}ils déviaient comme leurs p\underline{è}res, ils désertaient,
        \\trahissaient comme un \underline{a}rc infidèle.
${}^{58}Leurs hauts lie\underline{u}x le provoquaient,
        \\leurs idoles excit\underline{a}ient sa jalousie.
         
${}^{59}Dieu a entend\underline{u}, il s’emporte,
        \\il écarte tout à f\underline{a}it Israël ;
${}^{60}il quitte la deme\underline{u}re de Silo,
        \\la tente qu’il avait dress\underline{é}e chez les hommes ;
${}^{61}il laisse captur\underline{e}r sa gloire,
        \\et sa puissance par des m\underline{a}ins ennemies.
         
${}^{62}Il livre son pe\underline{u}ple à l’épée,
        \\contre son hérit\underline{a}ge, il s’emporte :
${}^{63}le feu a dévor\underline{é} les jeunes gens,
        \\les jeunes filles n’ont pas conn\underline{u} la joie des noces ;
${}^{64}les prêtres sont tomb\underline{é}s sous l’épée,
        \\les veuves n’ont pas chant\underline{é} leur lamentation.
         
${}^{65}Le Seigneur, tel un dorme\underline{u}r qui s’éveille,
        \\tel un guerrier que le v\underline{i}n ragaillardit,
${}^{66}frappe l’ennem\underline{i} à revers
        \\et le livre pour toujo\underline{u}rs à la honte.
         
        *
         
${}^{67}Il écarte la mais\underline{o}n de Joseph,
        \\ne choisit pas la trib\underline{u} d’Éphraïm.
${}^{68}Il choisit la trib\underline{u} de Juda,
        \\la montagne de Si\underline{o}n, qu’il aime.
${}^{69}Il a bâti comme le ci\underline{e}l son temple ;
        \\comme la terre, il l’a fond\underline{é} pour toujours.
         
${}^{70}Il choisit Dav\underline{i}d son serviteur ;
        \\il le prend dans les p\underline{a}rcs à moutons ;
${}^{71}il l’appelle à quitt\underline{e}r ses brebis *
        \\pour en faire le berger de Jacob, son peuple,
        d’Isra\underline{ë}l, son héritage.
         
${}^{72}Berger au cœ\underline{u}r intègre,
        \\sa main prud\underline{e}nte les conduit.
      \bchapter{Psaume}
          
            \bchapter{Psaume}
            Épargne ceux qui doivent mourir
${}^{1}Psaume. D’Asaph.
         
        \\Dieu, les païens ont envah\underline{i} ton domaine ; +
        \\ils ont souillé ton t\underline{e}mple sacré
        \\et mis Jérusal\underline{e}m en ruines.
         
${}^{2}Ils ont livré les cadavres de tes serviteurs
        en pâture aux rap\underline{a}ces du ciel *
        \\et la chair de tes fidèles, aux b\underline{ê}tes de la terre ;
${}^{3}ils ont versé le sang comme l’eau
        aux alento\underline{u}rs de Jérusalem : *
        \\les morts rest\underline{a}ient sans sépulture.
         
${}^{4}Nous sommes la ris\underline{é}e des voisins,
        \\la fable et le jou\underline{e}t de l’entourage.
${}^{5}Combien de temps, Seigneur, durer\underline{a} ta colère
        \\et brûlera le fe\underline{u} de ta jalousie ?
         
${}^{6}\[Déverse ta fureur
        sur les païens qui ne t’ont p\underline{a}s reconnu, *
        \\sur les royaumes qui n’invoquent p\underline{a}s ton nom,
${}^{7}car ils ont dévor\underline{é} Jacob
        \\et ravag\underline{é} son territoire.\]
         
${}^{8}Ne retiens pas contre nous les péch\underline{é}s de nos ancêtres : +
        \\que nous vienne bient\underline{ô}t ta tendresse,
        \\car nous sommes à bo\underline{u}t de force !
         
${}^{9}Aide-nous, Dieu notre Sauveur,
        pour la gl\underline{o}ire de ton nom ! *
        \\Délivre-nous, efface nos fautes,
        pour la ca\underline{u}se de ton nom !
         
${}^{10}Pourquoi laisser d\underline{i}re aux païens :
        \\« Où d\underline{o}nc est leur Dieu ? »
        \\Que les païens, sous nos ye\underline{u}x, le reconnaissent :
        \\il sera vengé, le sang vers\underline{é} de tes serviteurs.
         
${}^{11}Que monte en ta présence la pl\underline{a}inte du captif !
        \\Ton bras est fort : épargne ceux qui d\underline{o}ivent mourir.
${}^{12}\[Rends à nos voisins, sept f\underline{o}is, en plein cœur,
        \\l’outrage qu’ils t’ont f\underline{a}it, Seigneur Dieu.\]
         
${}^{13}Et nous, ton peuple, le troupea\underline{u} que tu conduis, +
        \\sans fin nous pourr\underline{o}ns te rendre grâce
        \\et d’âge en âge proclam\underline{e}r ta louange.
      \bchapter{Psaume}
          
            \bchapter{Psaume}
            Fais-nous revenir
${}^{1}Du maître de chœur. Sur l’air de « Les lis de la charte ». D’Asaph. Psaume.
         
${}^{2}Berger d’Isra\underline{ë}l, écoute,
        \\toi qui conduis Jos\underline{e}ph, ton troupeau :
        \\resplendis au-dess\underline{u}s des Kéroubim,
${}^{3}devant Éphraïm, Benjam\underline{i}n, Manassé !
        \\Rév\underline{e}ille ta vaillance
        \\et vi\underline{e}ns nous sauver.
         
${}^{4}Dieu, fais-no\underline{u}s revenir ; *
        que ton visage s’éclaire,
        et nous ser\underline{o}ns sauvés !
         
${}^{5}Seigneur, Die\underline{u} de l’univers, *
        \\vas-tu longtemps encore
        opposer ta colère aux pri\underline{è}res de ton peuple,
${}^{6}le nourrir du p\underline{a}in de ses larmes, *
        \\l’abreuver de l\underline{a}rmes sans mesure ?
${}^{7}Tu fais de nous la c\underline{i}ble des voisins :
        \\nos ennemis ont vraim\underline{e}nt de quoi rire !
         
${}^{8}Dieu, fais-no\underline{u}s revenir ; *
        que ton visage s’éclaire,
        et nous ser\underline{o}ns sauvés !
         
${}^{9}La vigne que tu as pr\underline{i}se à l’Égypte,
        \\tu la replantes en chass\underline{a}nt des nations.
${}^{10}Tu déblaies le s\underline{o}l devant elle,
        \\tu l’enracines pour qu’elle empl\underline{i}sse le pays.
         
${}^{11}Son ombre couvr\underline{a}it les montagnes,
        \\et son feuillage, les c\underline{è}dres géants ;
${}^{12}elle étendait ses sarm\underline{e}nts jusqu’à la mer,
        \\et ses rej\underline{e}ts, jusqu’au Fleuve.
         
${}^{13}Pourquoi as-tu perc\underline{é} sa clôture ?
        \\Tous les passants y grapp\underline{i}llent en chemin ;
${}^{14}le sanglier des for\underline{ê}ts la ravage
        \\et les bêtes des ch\underline{a}mps la broutent.
         
${}^{15}Dieu de l’univ\underline{e}rs, reviens !
         
        \\Du haut des cieux, reg\underline{a}rde et vois :
        \\visite cette v\underline{i}gne, protège-la,
${}^{16}celle qu’a plant\underline{é}e ta main puissante,
        \\le rejeton qui te d\underline{o}it sa force.
${}^{17}La voici détru\underline{i}te, incendiée ;
        \\que ton visage les men\underline{a}ce, ils périront !
         
${}^{18}Que ta main souti\underline{e}nne ton protégé,
        \\le fils de l’homme qui te d\underline{o}it sa force.
${}^{19}Jamais plus nous n’ir\underline{o}ns loin de toi :
        \\fais-nous vivre et invoqu\underline{e}r ton nom !
         
${}^{20}Seigneur, Dieu de l’univers,
        fais-no\underline{u}s revenir ; *
        que ton visage s’éclaire,
        et nous ser\underline{o}ns sauvés.
      \bchapter{Psaume}
          
            \bchapter{Psaume}
            « Si mon peuple m’écoutait… »
${}^{1}Du maître de chœur. Sur la guittith. D’Asaph.
         
${}^{2}Criez de joie pour Die\underline{u}, notre force,
        \\acclamez le Die\underline{u} de Jacob.
         
${}^{3}Jouez, musiques, frapp\underline{e}z le tambourin,
        \\la harpe et la cith\underline{a}re mélodieuse.
${}^{4}Sonnez du cor pour le m\underline{o}is nouveau,
        \\quand revient le jo\underline{u}r de notre fête.
         
${}^{5}C’est là, pour Isra\underline{ë}l, une règle,
        \\une ordonnance du Die\underline{u} de Jacob ;
${}^{6}Il en fit, pour Jos\underline{e}ph, une loi
        \\quand il marcha contre la t\underline{e}rre d’Égypte.
         
        \\J’entends des mots qui m’ét\underline{a}ient inconnus : +
${}^{7}« J’ai ôté le poids qui charge\underline{a}it ses épaules ;
        \\ses mains ont dépos\underline{é} le fardeau.
         
${}^{8}« Quand tu criais sous l’\underline{o}ppression, je t’ai sauvé ; +
        \\je répondais, cach\underline{é} dans l’orage,
        \\je t’éprouvais près des ea\underline{u}x de Mériba.
         
${}^{9}« Écoute, je t’adj\underline{u}re, ô mon peuple ;
        \\vas-tu m’écout\underline{e}r, Israël ?
${}^{10}Tu n’auras pas chez t\underline{o}i d’autres dieux,
        \\tu ne serviras aucun die\underline{u} étranger.
         
${}^{11}« C’est moi, le Seigne\underline{u}r ton Dieu, +
        \\qui t’ai fait monter de la t\underline{e}rre d’Égypte !
        \\Ouvre ta bouche, m\underline{o}i, je l’emplirai.
         
${}^{12}« Mais mon peuple n’a pas écout\underline{é} ma voix,
        \\Israël n’a pas voul\underline{u} de moi.
${}^{13}Je l’ai livré à son cœ\underline{u}r endurci :
        \\qu’il aille et su\underline{i}ve ses vues !
         
${}^{14}« Ah ! Si mon pe\underline{u}ple m’écoutait,
        \\Israël, s’il \underline{a}llait sur mes chemins !
${}^{15}Aussitôt j’humilier\underline{a}is ses ennemis,
        \\contre ses oppresseurs je tourner\underline{a}is ma main.
         
${}^{16}« Mes adversaires s’abaisser\underline{a}ient devant lui ;
        \\tel serait leur s\underline{o}rt à jamais !
${}^{17}Je le nourrirais de la fle\underline{u}r du froment,
        \\je le rassasierais avec le mi\underline{e}l du rocher ! »
      \bchapter{Psaume}
          
            \bchapter{Psaume}
            Rendez justice au faible
${}^{1}Psaume. D’Asaph.
         
        \\Dans l’assemblée divine, Die\underline{u} préside ;
        \\entouré des die\underline{u}x, il juge.
         
${}^{2}« Combien de temps jugerez-vo\underline{u}s sans justice,
        \\soutiendrez-vous la ca\underline{u}se des impies ?
         
${}^{3}« Rendez justice au f\underline{a}ible, à l’orphelin ;
        \\faites droit à l’indig\underline{e}nt, au malheureux.
         
${}^{4}« Libérez le f\underline{a}ible et le pauvre,
        \\arrachez-le aux m\underline{a}ins des impies. »
         
${}^{5}Mais non, sans sav\underline{o}ir, sans comprendre, +
        \\ils vont au milie\underline{u} des ténèbres :
        \\les fondements de la terre en s\underline{o}nt ébranlés.
         
${}^{6}« Je l’ai dit : Vous \underline{ê}tes des dieux,
        \\des fils du Très-Ha\underline{u}t, vous tous !
         
${}^{7}« Pourtant, vous mourr\underline{e}z comme des hommes,
        \\comme les princes, to\underline{u}s, vous tomberez ! »
         
${}^{8}Lève-toi, Dieu, j\underline{u}ge la terre,
        \\car toutes les nati\underline{o}ns t’appartiennent.
      \bchapter{Psaume}
          
            \bchapter{Psaume}
            Ils ont fait alliance contre toi
${}^{1}Cantique. Psaume. D’Asaph.
         
${}^{2}Dieu, ne garde p\underline{a}s le silence,
        \\ne sois pas immob\underline{i}le et muet.
${}^{3}Vois tes ennem\underline{i}s qui grondent,
        \\tes adversaires qui l\underline{è}vent la tête.
         
${}^{4}Contre ton peuple, ils tr\underline{a}ment un complot,
        \\ils intriguent c\underline{o}ntre les tiens.
${}^{5}Ils disent : « Venez ! retranchons-l\underline{e}s des nations :
        \\que soit oublié le n\underline{o}m d’Israël ! »
         
${}^{6}Oui, tous ens\underline{e}mble ils intriguent ;
        \\ils ont fait alli\underline{a}nce contre toi,
${}^{7}ceux d’Éd\underline{o}m et d’Ismaël,
        \\ceux de Mo\underline{a}b et d’Agar ;
         
${}^{8}Guébal, Amm\underline{o}ne, Amalec,
        \\la Philistie, avec les g\underline{e}ns de Tyr ;
${}^{9}même Asho\underline{u}r s’est joint à eux
        \\pour appuy\underline{e}r les fils de Loth.
         
        *
         
${}^{10}Traite-les comme tu f\underline{i}s de Madiane,
        \\de Sissera et Yabine au torr\underline{e}nt de Qishone :
${}^{11}ils ont été anéant\underline{i}s à Enn-Dor,
        \\ils ont servi de fumi\underline{e}r pour la terre.
         
${}^{12}Supprime leurs chefs comme Or\underline{e}b et Zéèb,
        \\tous leurs princes, comme Zèb\underline{a} et Salmounna,
${}^{13}eux qui dis\underline{a}ient : « À nous,
        \\à nous le dom\underline{a}ine de Dieu ! »
         
${}^{14}Dieu, rends-les pareils au br\underline{i}n de paille,
        \\à la graine qui tourbill\underline{o}nne dans le vent.
${}^{15}Comme un feu dévore la forêt,
        comme une flamme embr\underline{a}se les montagnes, *
${}^{16}oui, poursuis-les de tes ouragans,
        et que tes or\underline{a}ges les épouvantent !
         
${}^{17}Que leur front soit marqu\underline{é} d’infamie,
        \\et qu’ils cherchent ton n\underline{o}m, Seigneur !
${}^{18}Frappés pour toujours d’épouv\underline{a}nte et de honte,
        \\qu’ils pér\underline{i}ssent, déshonorés !
         
${}^{19}Et qu’ils le sachent : +
        \\toi seul, tu as pour n\underline{o}m Le Seigneur,
        \\le Très-Haut sur to\underline{u}te la terre !
      \bchapter{Psaume}
          
            \bchapter{Psaume}
            Heureux, les habitants de ta maison
${}^{1}Du maître de chœur. Sur la guittith. Des fils de Coré. Psaume.
         
${}^{2}De quel amour sont aim\underline{é}es tes demeures,
        \\Seigneur, Die\underline{u} de l’univers !
         
${}^{3}Mon âme s’épu\underline{i}se à désirer
        les parv\underline{i}s du Seigneur ; *
        \\mon cœur et ma ch\underline{a}ir sont un cri
        vers le Die\underline{u} vivant !
         
${}^{4}L’oiseau lui-même s’est trouv\underline{é} une maison,
        \\et l’hirondelle, un nid pour abrit\underline{e}r sa couvée :
        \\tes autels, Seigne\underline{u}r de l’univers,
        \\mon R\underline{o}i et mon Dieu !
         
${}^{5}Heureux les habit\underline{a}nts de ta maison :
        \\ils pourront te chant\underline{e}r encore !
${}^{6}Heureux les hommes dont tu \underline{e}s la force :
        \\des chemins s’o\underline{u}vrent dans leur cœur !
         
${}^{7}Quand ils traversent la vall\underline{é}e de la soif,
        ils la ch\underline{a}ngent en source ; *
        \\de quelles bénédicti\underline{o}ns la revêtent
        les plu\underline{i}es de printemps !
         
${}^{8}Ils vont de haute\underline{u}r en hauteur,
        \\ils se présentent devant Die\underline{u} à Sion.
         
${}^{9}Seigneur, Dieu de l’univers, ent\underline{e}nds ma prière ;
        \\écoute, Die\underline{u} de Jacob.
${}^{10}Dieu, v\underline{o}is notre bouclier,
        \\regarde le vis\underline{a}ge de ton messie.
         
${}^{11}Oui, un jo\underline{u}r dans tes parvis
        \\en vaut pl\underline{u}s que mille.
         
        \\J’ai choisi de me ten\underline{i}r sur le seuil,
        dans la mais\underline{o}n de mon Dieu, *
        \\plut\underline{ô}t que d’habiter
        parm\underline{i} les infidèles.
         
${}^{12}Le Seigneur Die\underline{u} est un soleil,
        il \underline{e}st un bouclier ; *
        \\le Seigneur d\underline{o}nne la grâce,
        il d\underline{o}nne la gloire.
         
        \\Jamais il ne ref\underline{u}se le bonheur
        \\à ceux qui v\underline{o}nt sans reproche.
         
${}^{13}Seigneur, Die\underline{u} de l’univers,
        \\heureux qui esp\underline{è}re en toi !
      \bchapter{Psaume}
          
            \bchapter{Psaume}
            La gloire habitera notre terre
${}^{1}Du maître de chœur. Des fils de Coré. Psaume.
         
${}^{2}Tu as aimé, Seigne\underline{u}r, cette terre,
        \\tu as fait revenir les déport\underline{é}s de Jacob ;
${}^{3}tu as ôté le péch\underline{é} de ton peuple,
        \\tu as couvert to\underline{u}te sa faute ;
${}^{4}tu as mis fin à to\underline{u}tes tes colères,
        \\tu es revenu de ta gr\underline{a}nde fureur.
         
${}^{5}Fais-nous revenir, Die\underline{u}, notre salut,
        \\oublie ton ressentim\underline{e}nt contre nous.
${}^{6}Seras-tu toujours irrit\underline{é} contre nous,
        \\maintiendras-tu ta col\underline{è}re d’âge en âge ?
         
${}^{7}N’est-ce pas toi qui reviendr\underline{a}s nous faire vivre
        \\et qui seras la j\underline{o}ie de ton peuple ?
${}^{8}Fais-nous voir, Seigne\underline{u}r, ton amour,
        \\et donne-no\underline{u}s ton salut.
         
        *
         
${}^{9}J’écoute : que dir\underline{a} le Seigneur Dieu ? +
        \\Ce qu’il dit, c’est la paix
        pour son pe\underline{u}ple et ses fidèles ; *
        \\qu’ils ne reviennent jam\underline{a}is à leur folie !
${}^{10}Son salut est proche de ce\underline{u}x qui le craignent,
        \\et la gloire habiter\underline{a} notre terre.
         
${}^{11}Amour et vérit\underline{é} se rencontrent,
        \\justice et p\underline{a}ix s’embrassent ;
${}^{12}la vérité germer\underline{a} de la terre
        \\et du ciel se pencher\underline{a} la justice.
         
${}^{13}Le Seigneur donner\underline{a} ses bienfaits,
        \\et notre terre donner\underline{a} son fruit.
${}^{14}La justice marcher\underline{a} devant lui,
        \\et ses pas tracer\underline{o}nt le chemin.
      \bchapter{Psaume}
          
            \bchapter{Psaume}
            Unifie mon cœur
${}^{1}Prière. De David.
         
        \\Écoute, Seigne\underline{u}r, réponds-moi,
        \\car je suis pa\underline{u}vre et malheureux.
${}^{2}Veille sur moi qui suis fid\underline{è}le, ô mon Dieu,
        \\sauve ton serviteur qui s’appu\underline{i}e sur toi.
         
${}^{3}Prends pitié de m\underline{o}i, Seigneur,
        \\toi que j’app\underline{e}lle chaque jour.
${}^{4}Seigneur, réjou\underline{i}s ton serviteur :
        \\vers toi, j’él\underline{è}ve mon âme !
         
${}^{5}Toi qui es b\underline{o}n et qui pardonnes,
        \\plein d’amour pour tous ce\underline{u}x qui t’appellent,
${}^{6}écoute ma pri\underline{è}re, Seigneur,
        \\entends ma v\underline{o}ix qui te supplie.
         
${}^{7}Je t’appelle au jo\underline{u}r de ma détresse,
        \\et toi, Seigne\underline{u}r, tu me réponds.
${}^{8}Aucun parmi les die\underline{u}x n’est comme toi,
        \\et rien n’ég\underline{a}le tes œuvres.
         
        *
         
${}^{9}Toutes les nations, que tu as faites,
        viendront se prostern\underline{e}r devant toi *
        \\et rendre gloire à ton n\underline{o}m, Seigneur,
${}^{10}car tu es grand et tu f\underline{a}is des merveilles,
        \\toi, Die\underline{u}, le seul.
         
${}^{11}Montre-moi ton chem\underline{i}n, Seigneur, +
        \\que je marche suiv\underline{a}nt ta vérité ;
        \\unifie mon cœur pour qu’il cr\underline{a}igne ton nom.
         
${}^{12}Je te rends grâce de tout mon cœur, Seigne\underline{u}r mon Dieu,
        \\toujours je rendrai gl\underline{o}ire à ton nom ;
${}^{13}il est grand, ton amo\underline{u}r pour moi :
        \\tu m’as tiré de l’ab\underline{î}me des morts.
         
        *
         
${}^{14}Mon Dieu, des orgueilleux se l\underline{è}vent contre moi, +
        \\des puissants se sont ligu\underline{é}s pour me perdre :
        \\ils n’ont pas souc\underline{i} de toi.
         
${}^{15}Toi, Seigneur,
        Dieu de tendr\underline{e}sse et de pitié, *
        \\lent à la colère,
        plein d’amo\underline{u}r et de vérité !
         
${}^{16}Reg\underline{a}rde vers moi,
        \\prends piti\underline{é} de moi.
        \\Donne à ton servite\underline{u}r ta force,
        \\et sauve le f\underline{i}ls de ta servante.
         
${}^{17}Accomplis un s\underline{i}gne en ma faveur ; +
        \\alors mes ennem\underline{i}s, humiliés, *
        \\verront que toi, Seigneur,
        tu m’\underline{a}ides et me consoles.
      \bchapter{Psaume}
          
            \bchapter{Psaume}
            En toi, ville de Dieu, toutes nos sources
${}^{1}Des fils de Coré. Psaume. Cantique.
         
        \\Elle est fondée sur les mont\underline{a}gnes saintes. +
         
${}^{2}Le Seigneur aime les p\underline{o}rtes de Sion *
        \\plus que toutes les deme\underline{u}res de Jacob.
         
${}^{3}Pour ta gloire on p\underline{a}rle de toi,
        v\underline{i}lle de Dieu ! *
${}^{4}« Je cite l’Ég\underline{y}pte et Babylone
        entre c\underline{e}lles qui me connaissent. »
         
        \\Voyez Tyr, la Philist\underline{i}e, l’Éthiopie :
        chacune est n\underline{é}e là-bas. *
${}^{5}Mais on appelle Si\underline{o}n : « Ma mère ! »
        car en elle, tout h\underline{o}mme est né.
         
        \\C’est lui, le Très-Ha\underline{u}t, qui la maintient. +
         
${}^{6}Au registre des peuples, le Seigne\underline{u}r écrit :
        « Chacun est n\underline{é} là-bas. » *
${}^{7}Tous ensemble ils d\underline{a}nsent, et ils chantent :
        « En toi, to\underline{u}tes nos sources ! »
      \bchapter{Psaume}
          
            \bchapter{Psaume}
            Dans cette nuit où je crie
${}^{1}Cantique. Psaume. Des fils de Coré. Du maître de chœur. Sur « mahalath ». Pour l’affliction. Poème. De Hémane l’Ézrahite.
         
${}^{2}Seigneur, mon Die\underline{u} et mon salut,
        \\dans cette nuit où je cr\underline{i}e en ta présence,
${}^{3}que ma prière parvi\underline{e}nne jusqu’à toi,
        \\ouvre l’or\underline{e}ille à ma plainte.
         
${}^{4}Car mon âme est rassasi\underline{é}e de malheur,
        \\ma vie est au b\underline{o}rd de l’abîme ;
${}^{5}on me voit déjà desc\underline{e}ndre à la fosse,*
        \\je suis comme un h\underline{o}mme fini.
         
${}^{6}Ma place est parm\underline{i} les morts,
        \\avec ceux que l’on a tu\underline{é}s, enterrés,
        \\ceux dont tu n’as pl\underline{u}s souvenir,
        \\qui sont exclus, et l\underline{o}in de ta main.
         
${}^{7}Tu m’as mis au plus prof\underline{o}nd de la fosse,
        \\en des lieux englout\underline{i}s, ténébreux ;
${}^{8}le poids de ta col\underline{è}re m’écrase,
        \\tu déverses tes fl\underline{o}ts contre moi.
         
${}^{9}Tu éloignes de m\underline{o}i mes amis,
        \\tu m’as rendu abomin\underline{a}ble pour eux ;
        \\enfermé, je n’ai p\underline{a}s d’issue :
${}^{10}à force de souffrir, mes ye\underline{u}x s’éteignent.
         
        *
         
        \\Je t’appelle, Seigne\underline{u}r, tout le jour,
        \\je tends les m\underline{a}ins vers toi :
${}^{11}fais-tu des mir\underline{a}cles pour les morts ?
        \\leur ombre se dresse-t-\underline{e}lle pour t’acclamer ?
         
${}^{12}Qui parlera de ton amo\underline{u}r dans la tombe,
        \\de ta fidélité au roya\underline{u}me de la mort ?
${}^{13}Connaît-on dans les tén\underline{è}bres tes miracles,
        \\et ta justice, au pa\underline{y}s de l’oubli ?
         
${}^{14}Moi, je crie vers t\underline{o}i, Seigneur ;
        \\dès le matin, ma pri\underline{è}re te cherche :
${}^{15}pourquoi me rejet\underline{e}r, Seigneur,
        \\pourquoi me cach\underline{e}r ta face ?
         
${}^{16}Malheureux, frappé à m\underline{o}rt depuis l’enfance,
        \\je n’en peux plus d’endur\underline{e}r tes fléaux ;
${}^{17}sur moi, ont déferl\underline{é} tes orages :
        \\tes effrois m’ont rédu\underline{i}t au silence.
         
${}^{18}Ils me cernent comme l’ea\underline{u} tout le jour,
        \\ensemble ils se ref\underline{e}rment sur moi.
${}^{19}Tu éloignes de moi am\underline{i}s et familiers ;
        \\ma compagne, c’\underline{e}st la ténèbre.
      \bchapter{Psaume}
          
            \bchapter{Psaume}
            Où donc est ton premier amour ?
${}^{1}Poème. D’Étane l’Ézrahite.
         
${}^{2}L’amour du Seigneur, sans f\underline{i}n je le chante ;
        \\ta fidélité, je l’ann\underline{o}nce d’âge en âge.
${}^{3}Je le dis : C’est un amour bât\underline{i} pour toujours ;
        \\ta fidélité est plus st\underline{a}ble que les cieux.
         
${}^{4}« Avec mon élu, j’ai f\underline{a}it une alliance,
        \\j’ai juré à Dav\underline{i}d, mon serviteur :
${}^{5}J’établirai ta dynast\underline{i}e pour toujours,
        \\je te bâtis un trône pour la su\underline{i}te des âges. »
         
        *
         
${}^{6}Que les cieux rendent grâce pour ta merv\underline{e}ille, Seigneur,
        \\et l’assemblée des s\underline{a}ints, pour ta fidélité.
${}^{7}Qui donc, là-haut, est compar\underline{a}ble au Seigneur ?
        \\Qui d’entre les dieux est sembl\underline{a}ble au Seigneur ?
         
${}^{8}Parmi tous les saints, Die\underline{u} est redoutable,
        \\plus terrible que tous ce\underline{u}x qui l’environnent.
${}^{9}Seigneur, Dieu de l’univers, qu\underline{i} est comme toi,
        \\Seigneur puissant que ta fidélit\underline{é} environne ?
         
${}^{10}C’est toi qui maîtrises l’orgu\underline{e}il de la mer ;
        \\quand ses flots se soulèvent, c’est t\underline{o}i qui les apaises.
${}^{11}C’est toi qui piétinas la dépo\underline{u}ille de Rahab ;
        \\par la force de ton bras, tu dispers\underline{a}s tes ennemis.
         
${}^{12}À toi, le ciel ! À toi auss\underline{i}, la terre !
        \\C’est toi qui fondas le m\underline{o}nde et sa richesse !
${}^{13}C’est toi qui créas le n\underline{o}rd et le midi :
        \\le Tabor et l’Hermon, à ton n\underline{o}m, crient de joie.
         
${}^{14}À toi, ce bras, et to\underline{u}te sa vaillance !
        \\Puissante est ta main, subl\underline{i}me est ta droite !
${}^{15}Justice et droit sont l’appu\underline{i} de ton trône.
        \\Amour et Vérité préc\underline{è}dent ta face.
         
${}^{16}Heureux le peuple qui conn\underline{a}ît l’ovation !
        \\Seigneur, il marche à la lumi\underline{è}re de ta face ;
${}^{17}tout le jour, à ton nom il d\underline{a}nse de joie,
        \\fier de ton j\underline{u}ste pouvoir.
         
${}^{18}Tu es sa f\underline{o}rce éclatante ;
        \\ta grâce accr\underline{o}ît notre vigueur.
${}^{19}Oui, notre r\underline{o}i est au Seigneur ;
        \\notre bouclier, au Dieu s\underline{a}int d’Israël.
         
        *
         
${}^{20}Autrefois, tu as parl\underline{é} à tes amis,
        \\dans une visi\underline{o}n tu leur as dit :
        \\« J’ai donné mon appui à un h\underline{o}mme d’élite,
        \\j’ai choisi dans ce pe\underline{u}ple un jeune homme.
         
${}^{21}« J’ai trouvé Dav\underline{i}d, mon serviteur,
        \\je l’ai sacré avec mon hu\underline{i}le sainte ;
${}^{22}et ma main sera pour toujo\underline{u}rs avec lui,
        \\mon bras fortifier\underline{a} son courage.
         
${}^{23}« L’ennemi ne pourr\underline{a} le surprendre,
        \\le traître ne pourr\underline{a} le renverser ;
${}^{24}j’écraserai devant lu\underline{i} ses adversaires
        \\et je frapper\underline{a}i ses agresseurs.
         
${}^{25}« Mon amour et ma fidélit\underline{é} sont avec lui,
        \\mon nom accr\underline{o}ît sa vigueur ;
${}^{26}j’étendrai son pouv\underline{o}ir sur la mer
        \\et sa dominati\underline{o}n jusqu’aux fleuves.
         
${}^{27}« Il me dira : “Tu \underline{e}s mon Père,
        \\mon Dieu, mon r\underline{o}c et mon salut !”
${}^{28}Et moi, j’en fer\underline{a}i mon fils aîné,
        \\le plus grand des r\underline{o}is de la terre !
         
${}^{29}« Sans fin je lui garder\underline{a}i mon amour,
        \\mon alliance avec lu\underline{i} sera fidèle ;
${}^{30}je fonderai sa dynast\underline{i}e pour toujours,
        \\son trône aussi dur\underline{a}ble que les cieux.
         
${}^{31}« Si ses fils aband\underline{o}nnent ma loi
        \\et ne suivent p\underline{a}s mes volontés,
${}^{32}s’ils osent viol\underline{e}r mes préceptes
        \\et ne gardent p\underline{a}s mes commandements,
         
${}^{33}« je punirai leur fa\underline{u}te en les frappant,
        \\et je châtier\underline{a}i leur révolte,
${}^{34}mais sans lui retir\underline{e}r mon amour,
        \\ni dément\underline{i}r ma fidélité.
         
${}^{35}« Jamais je ne violer\underline{a}i mon alliance,
        \\ne changerai un m\underline{o}t de mes paroles.
${}^{36}Je l’ai juré une f\underline{o}is sur ma sainteté ;
        \\non, je ne mentirai p\underline{a}s à David !
         
${}^{37}« Sa dynastie sans f\underline{i}n subsistera
        \\et son trône, comme le sol\underline{e}il en ma présence,
${}^{38}comme la lune établ\underline{i}e pour toujours,
        \\fidèle tém\underline{o}in là-haut ! »
         
        *
         
${}^{39}Pourtant tu l’as mépris\underline{é}, rejeté ;
        \\tu t’es emporté c\underline{o}ntre ton messie.
${}^{40}Tu as brisé l’alliance av\underline{e}c ton serviteur,
        \\jeté à terre et profan\underline{é} sa couronne.
         
${}^{41}Tu as percé to\underline{u}tes ses murailles,
        \\tu as démantel\underline{é} ses forteresses ;
${}^{42}tous les passants du chem\underline{i}n l’ont pillé :
        \\le voilà outrag\underline{é} par ses voisins.
         
${}^{43}Tu as accru le pouv\underline{o}ir de l’adversaire,
        \\tu as mis en joie to\underline{u}s ses ennemis ;
${}^{44}tu as émoussé le tranch\underline{a}nt de son épée,
        \\tu ne l’as pas épaul\underline{é} dans le combat.
         
${}^{45}Tu as mis f\underline{i}n à sa splendeur,
        \\jeté à t\underline{e}rre son trône ;
${}^{46}tu as écourté le t\underline{e}mps de sa jeunesse
        \\et déversé sur lu\underline{i} la honte.
         
${}^{47}Combien de temps, Seigneur, resteras-t\underline{u} caché,
        \\laisseras-tu flamber le fe\underline{u} de ta colère ?
         
${}^{48}Rappelle-toi le peu que d\underline{u}re ma vie,
        \\pour quel néant tu as cré\underline{é} chacun des hommes !
${}^{49}Qui donc peut vivre et ne pas v\underline{o}ir la mort ?
        \\Qui s’arracherait à l’empr\underline{i}se des enfers ?
         
${}^{50}Où donc, Seigneur, est ton premi\underline{e}r amour,
        \\celui que tu jurais à Dav\underline{i}d sur ta foi ?
         
${}^{51}Rappelle-toi, Seigneur, tes servite\underline{u}rs outragés,
        \\tous ces peuples dont j’ai reç\underline{u} la charge.
${}^{52}Oui, tes ennemis ont outrag\underline{é}, Seigneur,
        \\poursuivi de leurs outr\underline{a}ges ton messie.
         
        *
         
${}^{53}Béni soit le Seigne\underline{u}r pour toujours !
        Am\underline{e}n ! Amen !
