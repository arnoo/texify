  
  
    
    \bbook{JOËL}{JOËL}
      
         
      \bchapter{}
${}^{1}Parole du Seigneur adressée à Joël, fils de Petouël.
        
           
${}^{2}Écoutez ceci, les anciens,
        prêtez l’oreille, tous les habitants du pays !
        \\Cela s’est-il passé de votre temps,
        ou même du temps de vos pères ?
${}^{3}Cela, racontez-le à vos fils,
        et vos fils à leurs fils,
        et leurs fils à la génération qui suivra.
${}^{4}Ce que laisse la chenille,
        la sauterelle le dévore ;
        \\ce que laisse la sauterelle,
        le criquet le dévore ;
        \\ce que laisse le criquet,
        le grillon le dévore.
${}^{5}Réveillez-vous, ivrognes, et pleurez ;
        tous les buveurs, lamentez-vous sur le vin nouveau,
        \\car il est retiré de votre bouche.
${}^{6}Oui, une nation est montée contre mon pays,
        elle est puissante et innombrable.
        \\Ses dents sont les dents d’un lion,
        elle a les mâchoires d’une lionne.
${}^{7}Elle a fait de ma vigne un désert ;
        mon figuier, elle l’a réduit en pièces,
        \\l’a écorcé, abattu ;
        ses rameaux ont blanchi.
${}^{8}Soupire, comme une vierge vêtue de toile à sac,
        pleurant l’époux de sa jeunesse.
${}^{9}On a retiré offrandes et libations
        de la maison du Seigneur.
        \\Ils sont en deuil, les prêtres
        au service du Seigneur.
${}^{10}Les champs sont ravagés,
        la terre est en deuil.
        \\Le froment est ravagé,
        le vin nouveau fait défaut,
        l’huile fraîche est tarie.
${}^{11}Soyez consternés, laboureurs,
        vignerons, lamentez-vous,
        \\à cause du blé et de l’orge,
        car la moisson des champs est perdue.
${}^{12}La vigne a séché, le figuier est flétri ;
        le grenadier comme le dattier et le pommier,
        \\tous les arbres des champs ont séché.
        Et la joie a disparu de chez les hommes.
        ${}^{13}Prêtres, mettez un vêtement de deuil\\, et pleurez !
        Serviteurs de l’autel, faites entendre des lamentations !
        \\Venez, serviteurs de mon Dieu,
        passez la nuit vêtus de toile à sac !
        \\Car la maison de votre Dieu
        ne reçoit plus ni offrandes ni libations.
        ${}^{14}Prescrivez un jeûne sacré,
        annoncez une fête solennelle\\,
        \\réunissez les anciens et tous les habitants du pays
        dans la maison du Seigneur votre Dieu.
        \\Criez vers le Seigneur :
        ${}^{15}« Ah ! jour de malheur\\ ! »
        \\Le jour du Seigneur est proche,
        il vient du Puissant\\comme un fléau.
${}^{16}N’est-ce pas sous nos yeux
        que la nourriture est retirée,
        \\que la joie et la jubilation
        s’éloignent de la maison du Seigneur ?
${}^{17}Les graines se dessèchent dans la terre ;
        les silos sont en ruine, les greniers démolis :
        le froment a séché.
${}^{18}Comme il gémit, le bétail !
        \\Les troupeaux de bœufs sont inquiets,
        car ils n’ont plus de pâture.
        \\Même les troupeaux de brebis sont touchés.
${}^{19}Vers toi, Seigneur, je crie !
        \\Car un feu dévore les pâturages du désert,
        une flamme consume tous les arbres des champs.
${}^{20}Les bêtes sauvages soupirent aussi vers toi,
        \\car les cours d’eau sont à sec,
        un feu dévore les pâturages du désert.
      
         
      \bchapter{}
        ${}^{1}Sonnez du cor dans Sion,
        faites retentir la clameur sur ma montagne sainte !
        \\Qu’ils tremblent, tous les habitants du pays,
        car voici venir le jour du Seigneur,
        il est tout proche.
        ${}^{2}Jour de ténèbres et d’obscurité,
        jour de nuages et de sombres nuées.
        \\Comme la nuit qui envahit les montagnes,
        voici un peuple nombreux et fort ;
        \\il n’y en a jamais eu de pareil
        et il n’y en aura plus dans les générations à venir\\.
${}^{3}Devant lui, un feu dévore ;
        derrière lui, une flamme consume.
        \\Devant lui, le pays, comme un jardin d’Éden ;
        derrière lui, un désert désolé.
        \\Il n’y a rien qui lui échappe.
${}^{4}Son aspect est l’aspect des chevaux ;
        comme des coursiers, ils s’élancent.
${}^{5}C’est comme le bruit des chars
        bondissant aux sommets des montagnes ;
        \\comme le bruit des flammes,
        un feu qui dévore le chaume ;
        \\comme un peuple puissant
        rangé en bataille.
${}^{6}Devant lui frémissent des peuples,
        tous les visages ont perdu leur éclat.
${}^{7}Comme des braves ils courent ;
        tels des guerriers, ils escaladent le rempart.
        \\Chacun va son chemin,
        ils ne quittent pas leur voie.
${}^{8}Ils ne se bousculent pas l’un l’autre,
        chaque homme va sa route.
        \\Ils foncent à travers les projectiles,
        sans rompre les rangs.
${}^{9}Ils se ruent dans la ville,
        ils courent sur le rempart,
        \\ils escaladent les maisons,
        ils pénètrent par les fenêtres
        comme des voleurs.
${}^{10}Devant lui, la terre tremble,
        le ciel est ébranlé ;
        \\le soleil et la lune se sont obscurcis
        et les étoiles ont retiré leur clarté.
${}^{11}Le Seigneur a donné de la voix
        devant son armée :
        \\ils sont nombreux, ses bataillons ;
        il est puissant, l’exécuteur de sa parole ;
        \\il est grand, le jour du Seigneur, et très redoutable :
        qui peut l’affronter ?
        
           
        ${}^{12}Et maintenant – oracle du Seigneur –
        \\revenez à moi de tout votre cœur,
        dans le jeûne, les larmes et le deuil !
        ${}^{13}Déchirez vos cœurs et non pas vos vêtements,
        et revenez au Seigneur votre Dieu,
        \\car il est tendre et miséricordieux,
        lent à la colère et plein d’amour,
        renonçant au châtiment.
        ${}^{14}Qui sait ? Il pourrait revenir,
        il pourrait renoncer au châtiment,
        et laisser derrière lui sa bénédiction :
        \\alors, vous pourrez présenter\\offrandes et libations
        au Seigneur votre Dieu.
        ${}^{15}Sonnez du cor dans Sion\\ :
        prescrivez un jeûne sacré, annoncez une fête solennelle\\,
        ${}^{16}réunissez le peuple, tenez une assemblée sainte,
        rassemblez les anciens,
        réunissez petits enfants et nourrissons !
        \\Que le jeune époux sorte de sa maison,
        que la jeune mariée quitte sa chambre !
        ${}^{17}Entre le portail et l’autel,
        \\les prêtres, serviteurs\\du Seigneur,
        iront pleurer et diront :
        \\« Pitié, Seigneur, pour ton peuple,
        \\n’expose pas ceux qui t’appartiennent
        à l’insulte et aux moqueries des païens !
        \\Faudra-t-il qu’on dise :
        “Où donc est leur Dieu ?” »
        ${}^{18}Et le Seigneur s’est ému en faveur de son pays,
        il a eu pitié de son peuple.
         
        ${}^{19}Le Seigneur a répondu à son peuple, en disant :
        \\« Voici que je vous envoie
        le froment, le vin nouveau et l’huile fraîche.
        \\Vous en serez rassasiés,
        jamais plus je ne ferai de vous l’opprobre des nations.
${}^{20}Celui qui vient du nord, je l’éloignerai de chez vous,
        je le chasserai vers un pays aride et désolé :
        \\son avant-garde vers la mer orientale,
        son arrière-garde vers la mer occidentale ;
        \\sa puanteur s’élève,
        son infection s’élèvera. »
        \\Oui, il a fait de grandes choses !
         
        ${}^{21}Ô terre, ne crains plus !
        exulte et réjouis-toi !
        \\Car le Seigneur a fait de grandes choses.
        ${}^{22}Ne craignez plus, bêtes des champs :
        les pâturages du désert ont reverdi,
        \\les arbres ont produit leurs fruits,
        la vigne et le figuier ont donné leurs richesses.
        ${}^{23}Fils de Sion, exultez,
        réjouissez-vous dans le Seigneur votre Dieu !
        \\Car il vous a donné la pluie avec générosité,
        il a fait tomber pour vous les averses,
        \\celles de l’automne et celles du printemps,
        dès qu’il le fallait.
        ${}^{24}Les granges seront pleines de blé,
        les cuves déborderont de vin nouveau et d’huile fraîche.
${}^{25}Je vous rétribuerai pour les années
        dévorées par la sauterelle,
        \\par le criquet, le grillon, la chenille,
        ma grande armée envoyée contre vous.
        ${}^{26}Vous mangerez à votre faim, vous serez rassasiés,
        et vous célébrerez le nom du Seigneur votre Dieu
        car il a fait pour vous\\des merveilles.
        \\Mon peuple ne connaîtra plus jamais la honte.
        ${}^{27}Et vous saurez que moi, je suis au milieu d’Israël,
        \\que Je suis le Seigneur votre Dieu,
        il n’y en a pas d’autre.
        \\Mon peuple ne connaîtra plus jamais la honte.
      
         
      \bchapter{}
        ${}^{1}Alors, après cela, je répandrai mon esprit
        sur tout être de chair\\,
        \\vos fils et vos filles prophétiseront,
        vos anciens seront instruits par des songes,
        et vos jeunes gens par des visions.
        ${}^{2}Même sur les serviteurs et sur les servantes
        je répandrai mon esprit en ces jours-là.
        ${}^{3}Je ferai des prodiges au ciel et sur la terre :
        du sang, du feu, des nuages de fumée.
        ${}^{4}Le soleil sera changé en ténèbres,
        et la lune sera changée\\en sang,
        \\avant que vienne le jour du Seigneur,
        jour\\grand et redoutable.
        ${}^{5}Alors, quiconque invoquera le nom du Seigneur
        sera sauvé.
        \\Car sur la montagne de Sion et à Jérusalem
        il y aura des rescapés,
        \\selon la parole du Seigneur,
        les survivants que le Seigneur convoque.
        
           
      
         
      \bchapter{}
${}^{1}Oui, voici qu’en ces jours et en ce temps,
        où je ramènerai les captifs de Juda et de Jérusalem,
${}^{2}j’assemblerai toutes les nations
        et je les ferai descendre vers la Vallée de Josaphat
        (dont le nom signifie « Le Seigneur juge »).
        \\Là-bas, j’entrerai en jugement avec elles
        au sujet d’Israël, mon peuple et mon héritage,
        \\car elles l’ont dispersé parmi les nations,
        elles ont partagé ma terre.
${}^{3}Elles ont tiré au sort mon peuple,
        troqué le garçon contre la prostituée,
        vendu la fille pour du vin qu’elles ont bu.
        
           
         
${}^{4}Et vous aussi, qu’êtes-vous pour moi, Tyr et Sidon,
        et tous les districts de Philistie ?
        \\Useriez-vous de représailles envers moi ?
        \\Mais si vous usez de représailles envers moi,
        bien vite, je retournerai sur vos têtes vos représailles.
${}^{5}Mon argent et mon or,
        c’est vous qui les avez pris ;
        \\mes trésors les plus beaux,
        vous les avez emportés dans vos temples.
${}^{6}Les fils de Juda et ceux de Jérusalem,
        vous les avez vendus aux fils de Yavane (c’est-à-dire aux Grecs)
        pour les éloigner de leur territoire :
${}^{7}voici que moi je vais les éveiller
        du lieu où vous les avez vendus,
        et je retournerai sur vos têtes vos représailles.
${}^{8}Je vendrai vos fils et vos filles,
        par la main des fils de Juda ;
        \\ils les vendront aux Sabéens, nation lointaine.
        \\Oui, le Seigneur a parlé.
        
           
${}^{9}Criez ceci parmi les nations,
        sanctifiez-vous pour faire la guerre,
        \\éveillez les guerriers ;
        qu’ils s’avancent, qu’ils montent,
        tous les hommes de guerre !
${}^{10}De vos socs, forgez des épées,
        et de vos serpes, des javelots ;
        \\que le faible dise : « Je suis un brave ! »
${}^{11}Hâtez-vous et venez,
        toutes les nations d’alentour,
        \\assemblez-vous ici.
        \\Seigneur, fais descendre tes braves !
        ${}^{12}Que les nations se réveillent,
        qu’elles montent jusqu’à la vallée de Josaphat
        (dont le nom signifie « Le Seigneur juge »)\\,
        \\car c’est là que je vais siéger
        pour juger tous les peuples qui vous entourent.
        ${}^{13}Lancez la faucille :
        la moisson est mûre ;
        \\venez fouler la vendange\\ :
        le pressoir est rempli et les cuves débordent
        de tout le mal qu’ils ont fait !
        ${}^{14}Voici des multitudes et encore des multitudes
        dans la vallée du Jugement ;
        \\il est tout proche, le jour du Seigneur
        dans la vallée du Jugement !
        ${}^{15}Le soleil et la lune se sont obscurcis,
        les étoiles ont retiré leur clarté.
        ${}^{16}De Sion, le Seigneur fait entendre un rugissement,
        de Jérusalem, il donne de la voix\\.
        \\Le ciel et la terre sont ébranlés,
        \\mais le Seigneur est un refuge pour son peuple,
        une forteresse pour les fils d’Israël.
        ${}^{17}Vous saurez que je suis le Seigneur votre Dieu,
        qui demeure à Sion, sa\\montagne sainte.
        \\Jérusalem sera un lieu saint,
        les étrangers n’y passeront plus.
        ${}^{18}Ce jour-là, le vin nouveau ruissellera sur les montagnes,
        le lait coulera sur les collines.
        \\Tous les torrents de Juda seront pleins d’eau,
        \\une source jaillira de la maison du Seigneur
        et arrosera le ravin des Acacias\\.
        ${}^{19}L’Égypte sera vouée à la désolation,
        Édom sera un désert désolé\\,
        \\car ils ont multiplié les violences contre les fils de Juda,
        ils ont répandu leur sang innocent dans le pays.
        ${}^{20}Mais il y aura toujours des habitants en Juda,
        ainsi qu’à Jérusalem, de génération en génération.
        ${}^{21}Je vengerai\\leur sang,
        que je n’avais pas encore vengé\\.
         
        \\Et le Seigneur aura sa demeure à Sion.
