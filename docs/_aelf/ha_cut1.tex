  
  
    
    \bbook{HABACUC}{HABACUC}
      
         
      \bchapter{}
      \begin{verse}
${}^{1}La proclamation de ce que le prophète Habacuc a vu.
      
         
        ${}^{2}Combien de temps, Seigneur, vais-je appeler,
        sans que tu entendes ?
        \\Crier vers toi : « Violence ! »,
        sans que tu sauves\\ ?
         
        ${}^{3}Pourquoi me fais-tu voir le mal
        et regarder la misère ?
        \\Devant moi, pillage et violence ;
        dispute et discorde se déchaînent.
         
${}^{4}C’est pourquoi la loi est sans force
        et le droit n’apparaît plus jamais !
        \\Quand le méchant cerne le juste,
        alors le droit apparaît faussé.
${}^{5}Voyez chez les nations, et regardez !
        Soyez dans la stupeur et la stupéfaction !
        \\Car je ferai en votre temps une œuvre
        que vous ne croiriez pas, si on la racontait.
         
${}^{6}Oui, voici que je suscite les Chaldéens,
        la nation impétueuse et farouche,
        \\qui parcourt les étendues de la terre
        pour s’emparer des demeures d’autrui.
         
${}^{7}Elle est terrible et redoutable ;
        c’est elle qui se donne son droit et sa grandeur.
         
${}^{8}Ses chevaux sont plus rapides que des léopards,
        plus vifs que les loups du soir.
        \\Ses cavaliers bondissent,
        ils arrivent de loin, ses cavaliers,
        ils volent, comme un aigle qui fond sur sa proie.
${}^{9}Tous, ils arrivent pour la violence,
        leurs faces tendues vers l’avant, tous ensemble ;
        ils ramassent les captifs comme du sable.
${}^{10}Cette nation se moque des rois,
        les princes sont pour elle un jouet :
        \\elle se joue de toutes les forteresses,
        par un remblai de terre, elle les prend.
${}^{11}Puis le vent tourne, elle s’en va, la criminelle !
        Sa force est son dieu.
        ${}^{12}Seigneur, depuis les temps anciens,
        n’es-tu pas mon Dieu, mon Saint,
        \\toi qui es immortel\\ ?
        \\Seigneur, tu as établi les Chaldéens
        pour exécuter le jugement ;
        \\tu en as fait un roc
        pour exercer le châtiment.
        ${}^{13}Tes yeux sont trop purs pour voir le mal,
        tu ne peux supporter la vue de l’oppression\\.
        \\Alors, pourquoi regardes-tu ces perfides,
        pourquoi restes-tu silencieux
        quand le méchant engloutit l’homme juste\\ ?
        ${}^{14}Tu traites\\les hommes comme les poissons de la mer,
        et comme les reptiles que personne ne domine.
        ${}^{15}Le Chaldéen les pêche tous avec son hameçon,
        les prend avec son filet,
        \\et les recueille dans ses nasses,
        ce qui le comble de joie et d’allégresse !
        ${}^{16}Alors il offre des sacrifices à son filet,
        il fait fumer de l’encens devant ses nasses,
        \\car il leur doit une prise abondante
        et une nourriture copieuse.
        ${}^{17}N’arrêtera-t-il pas de vider son filet,
        de massacrer sans pitié des nations ?
      
         
      \bchapter{}
        ${}^{1}Je vais me tenir à mon poste de garde,
        rester debout sur mon rempart,
        \\guetter\\ce que Dieu me dira,
        et comment il répliquera\\à mes plaintes.
        
           
         
        ${}^{2}Alors le Seigneur me répondit :
        \\Tu vas mettre par écrit une vision,
        clairement, sur des tablettes,
        pour qu’on puisse la lire couramment.
        ${}^{3}Car c’est encore une vision pour le temps fixé ;
        elle tendra vers son accomplissement, et ne décevra pas.
        \\Si elle paraît tarder, attends-la :
        elle viendra certainement, sans retard\\.
        
           
         
        ${}^{4}Celui qui est insolent n’a pas l’âme droite,
        mais le juste vivra par sa fidélité\\.
        
           
         
${}^{5}Assurément, comme le vin est traître,
        l’homme fort est orgueilleux, sans repos ;
        \\il ouvre large sa gorge comme les enfers,
        il est comme la mort, jamais rassasié ;
        \\il entasse pour lui toutes les nations,
        il ramasse pour lui tous les peuples.
        
           
${}^{6}Tous ne vont-ils pas proférer sur lui une satire,
        des pamphlets et des énigmes contre lui ?
        \\Ils diront :
         
        \\Quel malheur pour celui qui s’enrichit du bien des autres
        – Combien de temps encore ? –
        et pour celui qui accumule des gages à son profit !
${}^{7}Ne vont-ils pas se dresser soudain, tes créanciers,
        et se réveiller, ceux qui te feront trembler ?
        \\Par eux, tu seras mis au pillage !
${}^{8}Comme tu as dépouillé de nombreuses nations,
        tout le reste des peuples te dépouillera
        \\à cause du sang de l’homme,
        à cause de la violence faite au pays,
        à la cité et à tous ses habitants.
         
${}^{9}Quel malheur pour celui qui réalise un profit malhonnête
        pour sa maison,
        \\afin d’établir son nid sur la hauteur,
        pour échapper à l’emprise du malheur !
${}^{10}C’est la honte de ta maison que tu as décidée ;
        en éliminant de nombreux peuples,
        c’est ta propre vie qui échoue.
${}^{11}Oui, du mur une pierre va crier,
        et de la charpente, une poutre lui répondra.
         
${}^{12}Quel malheur pour celui qui bâtit une ville dans le sang
        et fonde une cité sur le crime !
${}^{13}Ceci ne vient-il pas du Seigneur de l’univers
        que les peuples se fatiguent pour du feu,
        que les nations s’exténuent pour le néant ?
${}^{14}La connaissance de la gloire du Seigneur remplira la terre,
        comme les eaux recouvrent le fond de la mer !
         
${}^{15}Quel malheur pour qui fait boire son prochain,
        et lui verse du poison au point de l’enivrer,
        pour regarder sa nudité !
${}^{16}Tu t’es rassasié d’infamie plus que de gloire !
        À ton tour de boire et d’exhiber ton prépuce :
        \\sur toi se renversera la coupe de la droite du Seigneur,
        et sur ta gloire, l’ignominie !
         
${}^{17}Car la violence faite au Liban retombera sur toi
        et le pillage des troupeaux t’ effrayera,
        \\à cause du sang de l’homme,
        à cause de la violence faite au pays,
        à la cité et à tous ses habitants.
${}^{18}À quoi sert une image sculptée
        pour que la sculpte son auteur,
        \\une idole en métal qui enseigne le mensonge,
        pour qu’en elles se confie l’auteur qui les fabrique ?
        \\Les faux dieux qu’il fait sont muets.
         
${}^{19}Quel malheur pour celui qui dit au morceau de bois :
        « Réveille-toi ! »,
        à la pierre muette : « Lève-toi ! »,
        et qui dit : « Elle va enseigner ! »
        \\Tout cela est plaqué d’or et d’argent,
        sans aucun souffle à l’intérieur !
${}^{20}Mais le Seigneur est dans son temple saint :
        silence devant lui, terre entière !
      <p class="cantique" id="bib_ct-at_43"><span class="cantique_label">Cantique AT 43</span> = <span class="cantique_ref"><a class="unitex_link" href="#bib_ha_3_2">Ha 3, 2-4.13.15-19</a></span>
      
         
      \bchapter{}
${}^{1}Prière du prophète Habacuc sur le mode des complaintes.
        
           
         
        ${}^{2}Seigneur, j’ai entendu parler de toi\\ ;
        devant ton œuvre, Seigneur, j’ai craint !
        \\Dans le cours des années, fais-la revivre,
        dans le cours des années, fais-la connaître !
        
           
         
        \\Quand tu frémis de colère,
        tu te souviens d’avoir pitié.
        
           
         
        ${}^{3}Dieu vient de Témane
        et le saint, du mont de Parane\\ ;
        \\sa majesté couvre les cieux,
        sa louange emplit la terre.
        
           
         
        ${}^{4}Son éclat est pareil à la lumière ;
        deux rayons sortent de ses mains :
        là se tient cachée sa puissance.
        
           
         
${}^{5}\[Devant lui marche la peste,
        et la fièvre met ses pas dans les siens.
${}^{6}Il s’arrête, et la terre tremble,
        il regarde et fait sursauter les nations.
        
           
         
        \\Les montagnes de toujours se disloquent,
        les collines d’autrefois s’effondrent,
        qui furent autrefois des routes pour lui.
        
           
         
${}^{7}J’ai vu les tentes de Koushane dans la misère ;
        les abris du pays de Madiane chancellent !
        
           
         
${}^{8}Est-ce contre les fleuves, Seigneur, que flambe ta colère,
        contre les fleuves, contre la mer, ta fureur,
        \\pour que tu montes sur tes chevaux,
        sur tes chars de victoire ?
        
           
         
${}^{9}Tu sors ton arc, tu le tiens en éveil,
        tu le rassasies des traits de ta parole.
        \\Par des fleuves, tu ravines la terre.
${}^{10}Les montagnes t’ont vu : elles tremblent.
        \\Une trombe d’eau a passé,
        l’Abîme a donné de la voix.
        
           
         
        \\Le soleil, là-haut, a élevé ses mains,
${}^{11}la lune s’est arrêtée en sa demeure,
        \\à la lueur de tes flèches qui volent,
        à la clarté des éclairs de ta lance.
        
           
         
${}^{12}Dans ton indignation, tu parcours la terre ;
        dans ta colère, tu piétines des nations.\]
        ${}^{13}Tu es sorti pour sauver ton peuple,
        pour sauver ton messie.
        
           
         
        \\\[Tu as décapité la maison du méchant,
        tu l’as dénudée de fond en comble.
${}^{14}Tu as percé de ses traits le chef de ses guerriers ;
        ils se déchaînaient pour me disperser, joyeusement,
        comme pour dévorer dans leur repaire un malheureux.\]
        
           
         
        ${}^{15}Tu as foulé, de tes chevaux, la mer
        et le remous des eaux profondes.
        
           
         
        ${}^{16}J’ai entendu et mes entrailles ont frémi ;
        à cette voix, mes lèvres tremblent,
        la carie pénètre mes os.
        
           
         
        \\Et moi je frémis d’être là,
        d’attendre en silence le jour d’angoisse
        qui se lèvera sur le peuple dressé contre nous.
        
           
         
        ${}^{17}Le figuier n’a pas fleuri ;
        pas de récolte dans les vignes.
        \\Le fruit de l’olivier a déçu ;
        dans les champs, plus de nourriture.
        \\L’enclos s’est vidé de ses brebis,
        et l’étable, de son bétail.
        
           
         
        ${}^{18}Et moi, je bondis de joie dans le Seigneur,
        j’exulte en Dieu, mon Sauveur !
        ${}^{19}Le Seigneur mon Dieu est ma force ;
        il me donne l’agilité du chamois,
        il me fait marcher dans les hauteurs.
        
           
       
      Au maître de chant. Sur les instruments à cordes.
