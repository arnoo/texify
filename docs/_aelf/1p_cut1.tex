  
  
    
    \bbook{PREMIÈRE LETTRE DE SAINT PIERRE}{PREMIÈRE LETTRE DE SAINT PIERRE}
      
         
      \bchapter{}
${}^{1}Pierre, apôtre de Jésus Christ,
        \\à ceux qui sont choisis par Dieu,
        qui séjournent comme étrangers en diaspora
        \\dans les régions du Pont, de Galatie, de Cappadoce,
        dans la province d’Asie et en Bithynie,
${}^{2}qui sont désignés d’avance par Dieu le Père,
        et sanctifiés par l’Esprit,
        \\pour entrer dans l’obéissance
        et pour être purifiés par le sang de Jésus Christ.
        \\Que la grâce et la paix
        vous soient accordées en abondance.
        
           
${}^{3}Béni soit Dieu, le Père de notre Seigneur Jésus Christ : dans sa grande miséricorde, il nous a fait renaître pour une vivante espérance grâce à la résurrection de Jésus Christ d’entre les morts, 
${}^{4}pour un héritage qui ne connaîtra ni corruption, ni souillure, ni flétrissure. Cet héritage vous est réservé dans les cieux, 
${}^{5}à vous que la puissance de Dieu garde par la foi, pour un salut prêt à se révéler dans les derniers temps. 
${}^{6}Aussi vous exultez de joie, même s’il faut que vous soyez affligés, pour un peu de temps encore, par toutes sortes d’épreuves ; 
${}^{7}elles vérifieront la valeur de votre foi qui a bien plus de prix que l’or – cet or voué à disparaître et pourtant vérifié par le feu –, afin que votre foi reçoive louange, gloire et honneur quand se révélera Jésus Christ. 
${}^{8}Lui, vous l’aimez sans l’avoir vu ; en lui, sans le voir encore, vous mettez votre foi, vous exultez d’une joie inexprimable et remplie de gloire, 
${}^{9}car vous allez obtenir le salut des âmes qui est l’aboutissement de votre foi. 
${}^{10}Sur le salut, les prophètes ont fait porter leurs interrogations et leurs recherches, eux qui ont prophétisé pour annoncer la grâce qui vous est destinée. 
${}^{11}Ils cherchaient quel temps et quelles circonstances voulait indiquer l’Esprit du Christ, présent en eux, quand il attestait par avance les souffrances du Christ et la gloire qui s’ensuivrait. 
${}^{12}Il leur fut révélé que ce n’était pas pour eux-mêmes, mais pour vous, qu’ils étaient au service de ce message, annoncé maintenant par ceux qui vous ont évangélisés dans l’Esprit Saint envoyé du ciel ; même des anges désirent se pencher pour scruter ce message.
${}^{13}C’est pourquoi, après avoir disposé votre intelligence pour le service, restez sobres, mettez toute votre espérance dans la grâce que vous apporte la révélation de Jésus Christ. 
${}^{14}Comme des enfants qui obéissent, cessez de vous conformer aux convoitises d’autrefois, quand vous étiez dans l’ignorance, 
${}^{15}mais, à l’exemple du Dieu saint qui vous a appelés, devenez saints, vous aussi, dans toute votre conduite, 
${}^{16}puisqu’il est écrit : Vous serez saints, car moi, je suis saint. 
${}^{17}Si vous invoquez comme Père celui qui juge impartialement chacun selon son œuvre, vivez donc dans la crainte de Dieu, pendant le temps où vous résidez ici-bas en étrangers.
${}^{18}Vous le savez : ce n’est pas par des biens corruptibles, l’argent ou l’or, que vous avez été rachetés de la conduite superficielle héritée de vos pères ; 
${}^{19}mais c’est par un sang précieux, celui d’un agneau sans défaut et sans tache, le Christ.
${}^{20}Dès avant la fondation du monde, Dieu l’avait désigné d’avance et il l’a manifesté à la fin des temps à cause de vous. 
${}^{21}C’est bien par lui que vous croyez en Dieu, qui l’a ressuscité d’entre les morts et qui lui a donné la gloire ; ainsi vous mettez votre foi et votre espérance en Dieu.
${}^{22}En obéissant à la vérité, vous avez purifié vos âmes pour vous aimer sincèrement comme des frères ; aussi, d’un cœur pur, aimez-vous intensément les uns les autres, 
${}^{23}car Dieu vous a fait renaître, non pas d’une semence périssable, mais d’une semence impérissable : sa parole vivante qui demeure. 
${}^{24}C’est pourquoi il est écrit :
        \\Toute chair est comme l’herbe,
        \\toute sa gloire, comme l’herbe en fleur ;
        \\l’herbe se dessèche et la fleur tombe,
        ${}^{25}mais la parole du Seigneur demeure pour toujours.
      Or, cette parole est celle de la Bonne Nouvelle qui vous a été annoncée.
      <p class="cantique" id="bib_ct-nt_8"><span class="cantique_label">Cantique NT 8</span> = <span class="cantique_ref"><a class="unitex_link" href="#bib_1p_2_21">1 P 2, 21b-24</a></span>
      
         
      \bchapter{}
      \begin{verse}
${}^{1}Rejetez donc toute méchanceté, toute ruse, les hypocrisies, les jalousies et toutes les médisances ; 
${}^{2}comme des enfants nouveau-nés, soyez avides du lait non dénaturé de la Parole qui vous fera grandir pour arriver au salut, 
${}^{3}puisque vous avez goûté combien le Seigneur est bon.
      
         
${}^{4}Approchez-vous de lui : il est la pierre vivante rejetée par les hommes, mais choisie et précieuse devant Dieu. 
${}^{5}Vous aussi, comme pierres vivantes, entrez dans la construction de la demeure spirituelle, pour devenir le sacerdoce saint et présenter des sacrifices spirituels, agréables à Dieu, par Jésus Christ. 
${}^{6}En effet, il y a ceci dans l’Écriture :
        \\Je vais poser en Sion une pierre angulaire,
        \\une pierre choisie, précieuse ;
        \\celui qui met en elle sa foi
        \\ne saurait connaître la honte.
${}^{7}Ainsi donc, honneur à vous les croyants, mais, pour ceux qui refusent de croire, il est écrit : La pierre qu’ont rejetée les bâtisseurs est devenue la pierre d’angle, 
${}^{8}une pierre d’achoppement, un rocher sur lequel on trébuche. Ils achoppent, ceux qui refusent d’obéir à la Parole, et c’est bien ce qui devait leur arriver. 
${}^{9}Mais vous, vous êtes une descendance choisie, un sacerdoce royal, une nation sainte, un peuple destiné au salut, pour que vous annonciez les merveilles de celui qui vous a appelés des ténèbres à son admirable lumière. 
${}^{10}Autrefois vous n’étiez pas un peuple, mais maintenant vous êtes le peuple de Dieu ; vous n’aviez pas obtenu miséricorde, mais maintenant vous avez obtenu miséricorde.
${}^{11}Bien-aimés, puisque vous êtes comme des étrangers résidents ou de passage, je vous exhorte à vous abstenir des convoitises nées de la chair, qui combattent contre l’âme. 
${}^{12}Ayez une belle conduite parmi les gens des nations ; ainsi, sur le point même où ils disent du mal de vous en vous traitant de malfaiteurs, ils ouvriront les yeux devant vos belles actions et rendront gloire à Dieu, le jour de sa visite.
${}^{13}Soyez soumis à toute institution humaine à cause du Seigneur, soit à l’empereur, qui est le souverain, 
${}^{14}soit aux gouverneurs, qui sont ses délégués pour punir les malfaiteurs et reconnaître les mérites des gens de bien. 
${}^{15}Car la volonté de Dieu, c’est qu’en faisant le bien, vous fermiez la bouche aux insensés qui parlent sans savoir. 
${}^{16}Soyez des hommes libres, sans toutefois utiliser la liberté pour voiler votre méchanceté : mais soyez plutôt les esclaves de Dieu. 
${}^{17}Honorez tout le monde, aimez la communauté des frères, craignez Dieu, honorez l’empereur.
${}^{18}Vous les domestiques, soyez soumis en tout respect à vos maîtres, non seulement à ceux qui sont bons et bienveillants, mais aussi à ceux qui sont difficiles. 
${}^{19}En effet, c’est une grâce de supporter, par motif de conscience devant Dieu, des peines que l’on souffre injustement. 
${}^{20}En effet, si vous supportez des coups pour avoir commis une faute, quel honneur en attendre ? <a class="anchor verset_lettre" id="bib_1p_2_20_b"/>Mais si vous supportez la souffrance pour avoir fait le bien, c’est une grâce aux yeux de Dieu.
${}^{21}C’est bien à cela que vous avez été appelés, car
        \\C’est pour vous que le Christ,
        lui aussi, a souffert\\ ;
        \\il vous a laissé un modèle
        afin que vous suiviez ses traces.
         
        ${}^{22}Lui n’a pas commis de péché ;
        \\dans sa bouche\\,
        on n’a pas trouvé de mensonge.
         
        ${}^{23}Insulté, il ne rendait pas l’insulte,
        dans la souffrance, il ne menaçait pas,
        \\mais il s’abandonnait
        à Celui qui juge avec justice.
         
        ${}^{24}Lui-même a porté nos péchés,
        dans son corps, sur le bois,
        \\afin que, morts à nos péchés,
        nous vivions pour la justice.
         
        \\Par ses blessures, nous sommes\\guéris\\.
         
        ${}^{25}Car vous étiez errants
        comme des brebis ;
        \\mais à présent vous êtes retournés
        vers votre berger, le gardien de vos âmes.
      
         
      \bchapter{}
      \begin{verse}
${}^{1} Vous les femmes, soyez soumises à votre mari, pour que, même si certains refusent d’obéir à la parole de Dieu, ils soient gagnés par la conduite de leur femme et non par des paroles, 
${}^{2}en ouvrant les yeux devant votre attitude pure et pleine de respect. 
${}^{3}Que votre parure ne soit pas extérieure – coiffure élaborée, bijoux d’or, vêtements recherchés – 
${}^{4}mais qu’elle soit une qualité d’humanité au plus intime de votre cœur, parure impérissable d’un esprit doux et paisible : voilà ce qui a grande valeur devant Dieu. 
${}^{5}C’est cela qui faisait la parure des saintes femmes de jadis, elles qui espéraient en Dieu, soumises chacune à leur mari, 
${}^{6}comme Sara qui obéissait à Abraham, en l’appelant seigneur. Vous êtes devenues les filles de Sara en faisant le bien, sans vous laisser troubler par aucune crainte.
${}^{7}De même, vous les maris, sachez comprendre, dans la vie commune, que la femme est un être plus délicat ; accordez-lui l’honneur qui lui revient, puisqu’elle hérite, au même titre que vous, de la grâce de la vie. Ainsi, rien ne fera obstacle à vos prières.
${}^{8}Vous tous, enfin, vivez en parfait accord, dans la sympathie, l’amour fraternel, la compassion et l’esprit d’humilité. 
${}^{9}Ne rendez pas le mal pour le mal, ni l’insulte pour l’insulte ; au contraire, invoquez sur les autres la bénédiction, car c’est à cela que vous avez été appelés, afin de recevoir en héritage cette bénédiction.
${}^{10}En effet, comme il est écrit :
        \\Celui qui veut aimer la vie
        \\et connaître des jours heureux,
        \\qu’il garde sa langue du mal
        \\et ses lèvres des paroles perfides ;
${}^{11}qu’il se détourne du mal et qu’il fasse le bien,
        \\qu’il recherche la paix, et qu’il la poursuive.
${}^{12}Car le Seigneur regarde les justes,
        \\il écoute, attentif à leur demande.
        \\Mais le Seigneur affronte les méchants.
${}^{13}Qui donc vous fera du mal, si vous cherchez le bien avec ardeur ? 
${}^{14}Mais s’il vous arrivait de souffrir pour la justice, heureux seriez-vous ! Comme dit l’Écriture : N’ayez aucune crainte de ces gens-là, ne vous laissez pas troubler. 
${}^{15}Honorez dans vos cœurs la sainteté du Seigneur, le Christ. Soyez prêts à tout moment à présenter une défense devant quiconque vous demande de rendre raison de l’espérance qui est en vous ; 
${}^{16}mais faites-le avec douceur et respect. Ayez une conscience droite, afin que vos adversaires soient pris de honte sur le point même où ils disent du mal de vous pour la bonne conduite que vous avez dans le Christ. 
${}^{17}Car mieux vaudrait souffrir en faisant le bien, si c’était la volonté de Dieu, plutôt qu’en faisant le mal.
        ${}^{18}Car le Christ, lui aussi,
        a souffert pour les péchés, une seule fois,
        \\lui, le juste, pour les injustes,
        afin de vous introduire devant Dieu ;
        \\il a été mis à mort dans la chair,
        mais vivifié dans l’Esprit.
        ${}^{19}C’est en lui qu’il est parti proclamer son message
        aux esprits qui étaient en captivité.
${}^{20}Ceux-ci, jadis, avaient refusé d’obéir, au temps où se prolongeait la patience de Dieu, quand Noé construisit l’arche, dans laquelle un petit nombre, en tout huit personnes, furent sauvées à travers l’eau. 
${}^{21}C’était une figure du baptême qui vous sauve maintenant : le baptême ne purifie pas de souillures extérieures, mais il est l’engagement envers Dieu d’une conscience droite et il sauve par la résurrection de Jésus Christ, 
${}^{22}lui qui est à la droite de Dieu, après s’en être allé au ciel, lui à qui sont soumis les anges, ainsi que les Souverainetés et les Puissances.
      
         
      \bchapter{}
      \begin{verse}
${}^{1}Puisque le Christ a donc souffert dans la chair, vous aussi, armez-vous de la même pensée, à savoir : quiconque a souffert dans la chair en a fini avec le péché ; 
${}^{2}alors, vous vivrez le temps qui reste à passer dans la chair, non plus selon les convoitises humaines mais selon la volonté de Dieu. 
${}^{3}Il a assez duré, le temps passé à faire ce que veulent les gens des nations, quand vous vous laissiez aller aux débauches, aux convoitises, à l’ivrognerie, aux orgies, aux beuveries et aux cultes interdits des idoles. 
${}^{4}À ce propos, ils trouvent étrange que vous ne couriez plus avec eux vers les mêmes débordements d’inconduite, et ils vous couvrent d’injures. 
${}^{5}Ils auront des comptes à rendre à Celui qui se tient prêt à juger les vivants et les morts. 
${}^{6}C’est pour cela que l’Évangile a été annoncé aussi aux morts, afin que, jugés selon les hommes dans la chair, ils vivent selon Dieu dans l’Esprit.
      
         
${}^{7}La fin de toutes choses est proche. Soyez donc raisonnables et sobres en vue de la prière. 
${}^{8}Avant tout, ayez entre vous une charité intense, car la charité couvre une multitude de péchés. 
${}^{9}Pratiquez l’hospitalité les uns envers les autres sans récriminer. 
${}^{10}Ce que chacun de vous a reçu comme don de la grâce, mettez-le au service des autres, en bons gérants de la grâce de Dieu qui est si diverse : 
${}^{11}si quelqu’un parle, qu’il le fasse comme pour des paroles de Dieu ; celui qui assure le service, qu’il s’en acquitte comme avec la force procurée par Dieu. Ainsi, en tout, Dieu sera glorifié par Jésus Christ, à qui appartiennent la gloire et la souveraineté pour les siècles des siècles. Amen.
${}^{12}Bien-aimés, ne trouvez pas étrange le brasier allumé parmi vous pour vous mettre à l’épreuve ; ce qui vous arrive n’a rien d’étrange. 
${}^{13}Dans la mesure où vous communiez aux souffrances du Christ, réjouissez-vous, afin d’être dans la joie et l’allégresse quand sa gloire se révélera. 
${}^{14}Si l’on vous insulte pour le nom du Christ, heureux êtes-vous, parce que l’Esprit de gloire, l’Esprit de Dieu, repose sur vous. 
${}^{15}Que personne d’entre vous, en effet, n’ait à souffrir comme meurtrier, voleur, malfaiteur, ou comme agitateur. 
${}^{16}Mais si c’est comme chrétien, qu’il n’ait pas de honte, et qu’il rende gloire à Dieu pour ce nom-là. 
${}^{17}Car voici le temps du jugement : il commence par la famille de Dieu. Or, s’il vient d’abord sur nous, quelle sera la fin de ceux qui refusent d’obéir à l’Évangile de Dieu ? 
${}^{18}Et, si le juste est sauvé à grand-peine, l’impie, le pécheur, où va-t-il se montrer ? 
${}^{19}Ainsi, ceux qui souffrent en faisant la volonté de Dieu, qu’ils confient leurs âmes au Créateur fidèle, en faisant le bien.
      
         
      \bchapter{}
      \begin{verse}
${}^{1}Quant aux anciens en fonction parmi vous, je les exhorte, moi qui suis ancien comme eux et témoin des souffrances du Christ, communiant à la gloire qui va se révéler : 
${}^{2}soyez les pasteurs du troupeau de Dieu qui se trouve chez vous ; veillez sur lui, non par contrainte mais de plein gré, selon Dieu ; non par cupidité mais par dévouement ; 
${}^{3}non pas en commandant en maîtres à ceux qui vous sont confiés, mais en devenant les modèles du troupeau. 
${}^{4}Et, quand se manifestera le Chef des pasteurs, vous recevrez la couronne de gloire qui ne se flétrit pas.
      
         
${}^{5}De même, vous les jeunes gens, soyez soumis aux anciens.
      Et vous tous, les uns envers les autres, prenez l’humilité comme tenue de service. En effet, Dieu s’oppose aux orgueilleux, aux humbles il accorde sa grâce.
${}^{6}Abaissez-vous donc sous la main puissante de Dieu, pour qu’il vous élève en temps voulu. 
${}^{7}Déchargez-vous sur lui de tous vos soucis, puisqu’il prend soin de vous. 
${}^{8}Soyez sobres, veillez : votre adversaire, le diable, comme un lion rugissant, rôde, cherchant qui dévorer. 
${}^{9}Résistez-lui avec la force de la foi, car vous savez que tous vos frères, de par le monde, sont en butte aux mêmes souffrances. 
${}^{10}Après que vous aurez souffert un peu de temps, le Dieu de toute grâce, lui qui, dans le Christ Jésus, vous a appelés à sa gloire éternelle, vous rétablira lui-même, vous affermira, vous fortifiera, vous rendra inébranlables. 
${}^{11}À lui la souveraineté pour les siècles. Amen.
${}^{12}Par Silvain, que je considère comme un frère digne de confiance, je vous écris ces quelques mots pour vous exhorter, et pour attester que c’est vraiment dans la grâce de Dieu que vous tenez ferme. 
${}^{13}La communauté qui est à Babylone, choisie comme vous par Dieu, vous salue, ainsi que Marc, mon fils. 
${}^{14}Saluez-vous les uns les autres par un baiser fraternel.
      Paix à vous tous, qui êtes dans le Christ.
