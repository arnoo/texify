  
  
          
            \bchapter{Psaume}
            Reconnaître l’amour du Seigneur
        \\Alléluia !
         
${}^{1}Rendez grâce au Seigne\underline{u}r : Il est bon !
        \\Étern\underline{e}l est son amour !
         
${}^{2}Ils le diront, les rachet\underline{é}s du Seigneur,
        \\qu’il racheta de la m\underline{a}in de l’oppresseur,
${}^{3}qu’il rassembla de to\underline{u}s les pays,
        \\du nord et du midi, du lev\underline{a}nt et du couchant.
         
        *
         
${}^{4}Certains erraient dans le désert
        sur des chem\underline{i}ns perdus, *
        \\sans trouver de v\underline{i}lle où s’établir :
${}^{5}ils souffraient la f\underline{a}im et la soif,
        \\ils sentaient leur \underline{â}me défaillir.
         
${}^{6}Dans leur angoisse, ils ont cri\underline{é} vers le Seigneur,
        \\et lui les a tir\underline{é}s de la détresse :
${}^{7}il les conduit sur le b\underline{o}n chemin,
        \\les mène vers une v\underline{i}lle où s’établir.
         
${}^{8}Qu’ils rendent grâce au Seigne\underline{u}r de son amour,
        \\de ses merv\underline{e}illes pour les hommes :
${}^{9}car il ét\underline{a}nche leur soif,
        \\il comble de bi\underline{e}n les affamés !
         
        *
         
${}^{10}Certains gisaient dans les tén\underline{è}bres mortelles,
        \\captifs de la mis\underline{è}re et des fers :
${}^{11}ils avaient bravé les \underline{o}rdres de Dieu
        \\et méprisé les dess\underline{e}ins du Très-Haut ;
${}^{12}soumis par lui à des trava\underline{u}x accablants,
        \\ils succombaient, et n\underline{u}l ne les aidait.
         
${}^{13}Dans leur angoisse, ils ont cri\underline{é} vers le Seigneur,
        \\et lui les a tir\underline{é}s de la détresse :
${}^{14}il les délivre des tén\underline{è}bres mortelles,
        \\il fait tomb\underline{e}r leurs chaînes.
         
${}^{15}Qu’ils rendent grâce au Seigne\underline{u}r de son amour,
        \\de ses merv\underline{e}illes pour les hommes :
${}^{16}car il brise les p\underline{o}rtes de bronze,
        \\il casse les b\underline{a}rres de fer !
         
        *
         
${}^{17}Certains, égar\underline{é}s par leur péché,
        \\ployaient sous le p\underline{o}ids de leurs fautes :
${}^{18}ils avaient toute nourrit\underline{u}re en dégoût,
        \\ils touchaient aux p\underline{o}rtes de la mort.
         
${}^{19}Dans leur angoisse, ils ont cri\underline{é} vers le Seigneur,
        \\et lui les a tir\underline{é}s de la détresse :
${}^{20}il envoie sa par\underline{o}le, il les guérit,
        \\il arrache leur v\underline{i}e à la fosse.
         
${}^{21}Qu’ils rendent grâce au Seigne\underline{u}r de son amour,
        \\de ses merv\underline{e}illes pour les hommes ;
${}^{22}qu’ils offrent des sacrif\underline{i}ces d’action de grâce,
        \\à pleine voix qu’ils procl\underline{a}ment ses œuvres !
         
        *
         
${}^{23}Certains, embarqu\underline{é}s sur des navires,
        \\occupés à leur trav\underline{a}il en haute mer,
${}^{24}ont vu les œ\underline{u}vres du Seigneur
        \\et ses merveilles parm\underline{i} les océans.
         
${}^{25}Il parle, et prov\underline{o}que la tempête,
        \\un vent qui soul\underline{è}ve les vagues :
${}^{26}portés jusqu’au ciel, retomb\underline{a}nt aux abîmes,
        \\ils étaient mal\underline{a}des à rendre l’âme ;
${}^{27}ils tournoyaient, titub\underline{a}ient comme des ivrognes :
        \\leur sagesse ét\underline{a}it engloutie.
         
${}^{28}Dans leur angoisse, ils ont cri\underline{é} vers le Seigneur,
        \\et lui les a tir\underline{é}s de la détresse,
${}^{29}réduisant la temp\underline{ê}te au silence,
        \\faisant t\underline{a}ire les vagues.
${}^{30}Ils se réjouissent de les v\underline{o}ir s’apaiser,
        \\d’être conduits au p\underline{o}rt qu’ils désiraient.
         
${}^{31}Qu’ils rendent grâce au Seigne\underline{u}r de son amour,
        \\de ses merv\underline{e}illes pour les hommes ;
${}^{32}qu’ils l’exaltent à l’assembl\underline{é}e du peuple
        \\et le chantent parm\underline{i} les anciens !
         
        *
         
${}^{33}C’est lui qui change les fle\underline{u}ves en désert,
        \\les sources d’eau en pa\underline{y}s de la soif,
${}^{34}en salines une t\underline{e}rre généreuse
        \\quand ses habit\underline{a}nts se pervertissent.
         
${}^{35}C’est lui qui change le dés\underline{e}rt en étang,
        \\les terres ar\underline{i}des en source d’eau ;
${}^{36}là, il établ\underline{i}t les affamés
        \\pour y fonder une v\underline{i}lle où s’établir.
${}^{37}Ils ensemencent des champs et pl\underline{a}ntent des vignes :
        \\ils en réc\underline{o}ltent les fruits.
         
${}^{38}Dieu les bénit et leur n\underline{o}mbre s’accroît,
        \\il ne laisse pas diminu\underline{e}r leur bétail.
${}^{39}Puis, ils décl\underline{i}nent, ils dépérissent,
        \\écrasés de ma\underline{u}x et de peines.
         
${}^{40}Dieu livre au mépr\underline{i}s les puissants,
        \\il les égare dans un cha\underline{o}s sans chemin.
${}^{41}Mais il relève le pa\underline{u}vre de sa misère ;
        \\il rend prospères fam\underline{i}lles et troupeaux.
         
${}^{42}Les justes v\underline{o}ient, ils sont en fête ;
        \\et l’injustice f\underline{e}rme sa bouche.
${}^{43}Qui veut être sage retiendr\underline{a} ces choses :
        \\il y reconnaîtra l’amo\underline{u}r du Seigneur.
      \bchapter{Psaume}
          
            \bchapter{Psaume}
            Ton amour est plus grand que les cieux
${}^{1}Cantique. Psaume. De David.
         
${}^{2}Mon cœur est pr\underline{ê}t, mon Dieu, +
        \\je veux chant\underline{e}r, jouer des hymnes :
        \\\underline{ô} ma gloire !
         
${}^{3}Éveillez-vous, h\underline{a}rpe, cithare,
        \\que j’év\underline{e}ille l’aurore !
         
${}^{4}Je te rendrai grâce parmi les pe\underline{u}ples, Seigneur,
        \\et jouerai mes h\underline{y}mnes en tous pays.
${}^{5}Ton amour est plus gr\underline{a}nd que les cieux,
        \\ta vérité, plus ha\underline{u}te que les nues.
         
${}^{6}Dieu, lève-t\underline{o}i sur les cieux :
        \\que ta gloire dom\underline{i}ne la terre !
${}^{7}Que tes bien-aim\underline{é}s soient libérés,
        \\sauve-les par ta dr\underline{o}ite : réponds-nous !
         
        *
         
${}^{8}Dans le sanctuaire, Die\underline{u} a parlé : +
        \\« Je triomphe ! Je part\underline{a}ge Sichem,
        \\je divise la vall\underline{é}e de Souccoth.
         
${}^{9}« À moi Galaad, à m\underline{o}i Manassé ! +
        \\Éphraïm est le c\underline{a}sque de ma tête,
        \\Juda, mon bât\underline{o}n de commandement.
         
${}^{10}« Moab est le bass\underline{i}n où je me lave ; +
        \\sur Édom, je p\underline{o}se le talon,
        \\sur la Philist\underline{i}e, je crie victoire ! »
         
${}^{11}Qui me conduir\underline{a} dans la Ville-forte,
        \\qui me mèner\underline{a} jusqu’en Édom,
${}^{12}sinon toi, Die\underline{u} qui nous rejettes
        \\et ne sors plus av\underline{e}c nos armées ?
         
${}^{13}Porte-nous seco\underline{u}rs dans l’épreuve :
        \\néant, le sal\underline{u}t qui vient des hommes !
${}^{14}Avec Dieu nous fer\underline{o}ns des prouesses,
        \\et lui piétiner\underline{a} nos oppresseurs !
      \bchapter{Psaume}
          
            \bchapter{Psaume}
            Ils maudissent, toi, tu bénis
${}^{1}Du maître de chœur. De David. Psaume.
         
        Die\underline{u} de ma louange,
        s\underline{o}rs de ton silence !
         
${}^{2}La bouche de l’impie, la bouche du fourbe,
        s’o\underline{u}vrent contre moi : *
        \\ils parlent de moi pour d\underline{i}re des mensonges ;
${}^{3}ils me cernent de prop\underline{o}s haineux,
        \\ils m’att\underline{a}quent sans raison.
         
${}^{4}Pour prix de mon amiti\underline{é}, ils m’accusent,
        \\moi qui ne su\underline{i}s que prière.
${}^{5}Ils me rendent le m\underline{a}l pour le bien,
        \\ils paient mon amiti\underline{é} de leur haine.
         
        *
         
${}^{6}« Chargeons un imp\underline{i}e de l’attaquer :
        \\qu’un accusateur se ti\underline{e}nne à sa droite.
${}^{7}À son procès, qu’on le décl\underline{a}re impie,
        \\que sa prière soit compt\underline{é}e comme une faute.
         
${}^{8}« Que les jours de sa v\underline{i}e soient écourtés,
        \\qu’un autre pr\underline{e}nne sa charge.
${}^{9}Que ses fils devi\underline{e}nnent orphelins,
        \\que sa f\underline{e}mme soit veuve.
         
${}^{10}« Qu’ils soient errants, vagab\underline{o}nds, ses fils,
        \\qu’ils mendient, expuls\underline{é}s de leurs ruines.
${}^{11}Qu’un usurier sais\underline{i}sse tout son bien,
        \\que d’autres s’emparent du fru\underline{i}t de son travail.
         
${}^{12}« Que nul ne lui r\underline{e}ste fidèle,
        \\que nul n’ait piti\underline{é} de ses orphelins.
${}^{13}Que soit retranch\underline{é}e sa descendance,
        \\que son nom s’eff\underline{a}ce avec ses enfants.
         
${}^{14}« Qu’on rappelle au Seigneur les fa\underline{u}tes de ses pères,
        \\que les péchés de sa mère ne soient p\underline{a}s effacés.
${}^{15}Que le Seigneur garde cel\underline{a} devant ses yeux,
        \\et retranche de la t\underline{e}rre leur mémoire ! »
         
        *
         
${}^{16}Ainsi, celu\underline{i} qui m’accuse
        oubl\underline{i}e d’être fidèle : *
        \\il persécute un pa\underline{u}vre, un malheureux,
        un homme bless\underline{é} à mort.
         
${}^{17}Puisqu’il \underline{a}ime la malédiction,
        qu’elle \underline{e}ntre en lui ; *
        \\il ref\underline{u}se la bénédiction,
        qu’elle s’él\underline{o}igne de lui !
         
${}^{18}Il a revêtu comme un mantea\underline{u} la malédiction, *
        \\qu’elle entre en lui comme de l’eau,
        comme de l’hu\underline{i}le dans ses os !
${}^{19}Qu’elle soit l’ét\underline{o}ffe qui l’habille,
        \\la ceinture qui ne le qu\underline{i}tte plus !
         
${}^{20}C’est ainsi que le Seigneur paier\underline{a} mes accusateurs,
        \\ceux qui profèrent le m\underline{a}l contre moi.
         
        *
         
${}^{21}Mais toi, Seigneur Dieu,
        agis pour moi à ca\underline{u}se de ton nom. *
        \\Ton amour est fidèle : d\underline{é}livre-moi.
         
${}^{22}Vois, je suis pa\underline{u}vre et malheureux ;
        \\au fond de moi, mon cœur est blessé.
${}^{23}Je m’en vais comme le jo\underline{u}r qui décline,
        \\comme l’ins\underline{e}cte qu’on chasse.
         
${}^{24}J’ai tant jeûné que mes geno\underline{u}x se dérobent,
        \\je suis amaigr\underline{i}, décharné.
${}^{25}Et moi, on me to\underline{u}rne en dérision,
        \\ceux qui me voient h\underline{o}chent la tête.
         
${}^{26}Aide-moi, Seigne\underline{u}r mon Dieu :
        \\sauve-m\underline{o}i par ton amour !
${}^{27}Ils connaîtront que l\underline{à} est ta main,
        \\que toi, Seigne\underline{u}r, tu agis.
         
${}^{28}Ils maudissent, t\underline{o}i, tu bénis, *
        \\ils se sont dressés, ils sont humiliés :
        ton servite\underline{u}r est dans la joie.
${}^{29}Qu’ils soient couverts d’infam\underline{i}e, mes accusateurs,
        \\et revêtus du mantea\underline{u} de la honte !
         
${}^{30}À pleine voix, je rendrai gr\underline{â}ce au Seigneur,
        \\je le louerai parm\underline{i} la multitude,
${}^{31}car il se tient à la dr\underline{o}ite du pauvre
        \\pour le sauver de ce\underline{u}x qui le condamnent.
      \bchapter{Psaume}
          
            \bchapter{Psaume}
            « Siège à ma droite »
${}^{1}De David. Psaume.
         
        \\Oracle du Seigne\underline{u}r à mon seigneur :
        « Si\underline{è}ge à ma droite, *
        \\et je fer\underline{a}i de tes ennemis
        le marchepi\underline{e}d de ton trône. »
         
${}^{2}De Sion, le Seigne\underline{u}r te présente
        le sc\underline{e}ptre de ta force : *
        \\« Domine jusqu’au cœ\underline{u}r de l’ennemi. »
         
${}^{3}Le jour où par\underline{a}ît ta puissance,
        tu es prince, éblouiss\underline{a}nt de sainteté : *
        \\« Comme la rosée qui n\underline{a}ît de l’aurore,
        je t’\underline{a}i engendré. »
         
${}^{4}Le Seigne\underline{u}r l’a juré
        dans un serm\underline{e}nt irrévocable : *
        \\« Tu es pr\underline{ê}tre à jamais
        selon l’ordre du r\underline{o}i Melkisédek. »
         
${}^{5}À ta droite se ti\underline{e}nt le Seigneur : *
        \\il brise les rois au jo\underline{u}r de sa colère.
${}^{6}\[Il juge les nations : les cad\underline{a}vres s’entassent ; *
        \\il brise les chefs, l\underline{o}in sur la terre.\]
         
${}^{7}Au torrent il s’abre\underline{u}ve en chemin, *
        \\c’est pourquoi il redr\underline{e}sse la tête.
      \bchapter{Psaume}
          
            \bchapter{Psaume}
            Grandes sont les œuvres du Seigneur !
${}^{1}Alléluia !
         
        \\De tout cœur je rendrai gr\underline{â}ce au Seigneur
        \\dans l’assemblée, parm\underline{i} les justes.
${}^{2}Grandes sont les œ\underline{u}vres du Seigneur ;
        \\tous ceux qui les \underline{a}iment s’en instruisent.
${}^{3}Noblesse et beaut\underline{é} dans ses actions :
        \\à jamais se maintiend\underline{ra} sa justice.
         
${}^{4}De ses merveilles il a laiss\underline{é} un mémorial ;
        \\le Seigneur est tendr\underline{e}sse et pitié.
${}^{5}Il a donné des v\underline{i}vres à ses fidèles,
        \\gardant toujours mém\underline{o}ire de son alliance.
${}^{6}Il a montré sa f\underline{o}rce à son peuple,
        \\lui donnant le dom\underline{a}ine des nations.
         
${}^{7}Justesse et sûreté, les œ\underline{u}vres de ses mains,
        \\sécurité, to\underline{u}tes ses lois,
${}^{8}établies pour toujo\underline{u}rs et à jamais,
        \\accomplies avec droit\underline{u}re et sûreté !
         
${}^{9}Il apporte la délivr\underline{a}nce à son peuple ; +
        \\son alliance est promulgu\underline{é}e pour toujours :
        \\saint et redout\underline{a}ble est son nom.
         
${}^{10}La sagesse commence avec la cr\underline{a}inte du Seigneur. +
        \\Qui accomplit sa volonté en \underline{e}st éclairé.
        \\À jamais se maintiendr\underline{a} sa louange.
      \bchapter{Psaume}
          
            \bchapter{Psaume}
            Heureux qui craint le Seigneur
${}^{1}Alléluia !
         
        \\Heureux qui cr\underline{a}int le Seigneur,
        \\qui aime entièrem\underline{e}nt sa volonté !
${}^{2}Sa lignée sera puiss\underline{a}nte sur la terre ;
        \\la race des j\underline{u}stes est bénie.
         
${}^{3}Les richesses affl\underline{u}ent dans sa maison :
        \\à jamais se maintiendr\underline{a} sa justice.
${}^{4}Lumière des cœurs droits, il s’est lev\underline{é} dans les ténèbres,
        \\homme de justice, de tendr\underline{e}sse et de pitié.
         
${}^{5}L’homme de bien a piti\underline{é}, il partage ;
        \\il mène ses aff\underline{a}ires avec droiture.
${}^{6}Cet homme jam\underline{a}is ne tombera ;
        \\toujours on fera mém\underline{o}ire du juste.
         
${}^{7}Il ne craint pas l’ann\underline{o}nce d’un malheur :
        \\le cœur ferme, il s’appu\underline{i}e sur le Seigneur.
${}^{8}Son cœur est confi\underline{a}nt, il ne craint pas :
        \\il verra ce que val\underline{a}ient ses oppresseurs.
         
${}^{9}À pleines mains, il d\underline{o}nne au pauvre ; +
        \\à jamais se maintiendr\underline{a} sa justice,
        \\sa puissance grandir\underline{a}, et sa gloire !
         
${}^{10}L’impie le v\underline{o}it et s’irrite ; +
        \\il grince des d\underline{e}nts et se détruit.
        \\L’ambition des imp\underline{i}es se perdra.
      \bchapter{Psaume}
          
            \bchapter{Psaume}
            Béni soit le nom du Seigneur
${}^{1}Alléluia !
         
        \\Louez, servite\underline{u}rs du Seigneur,
        \\louez le n\underline{o}m du Seigneur !
${}^{2}Béni soit le n\underline{o}m du Seigneur,
        \\maintenant et pour les si\underline{è}cles des siècles !
${}^{3}Du levant au couch\underline{a}nt du soleil,
        \\loué soit le n\underline{o}m du Seigneur !
         
${}^{4}Le Seigneur dom\underline{i}ne tous les peuples,
        \\sa gloire dom\underline{i}ne les cieux.
${}^{5}Qui est semblable au Seigne\underline{u}r notre Dieu ?
        \\Lui, il si\underline{è}ge là-haut.
${}^{6}Mais il ab\underline{a}isse son regard
        \\vers le ci\underline{e}l et vers la terre.
         
${}^{7}De la poussière il rel\underline{è}ve le faible,
        \\il retire le pa\underline{u}vre de la cendre
${}^{8}pour qu’il si\underline{è}ge parmi les princes,
        \\parmi les pr\underline{i}nces de son peuple.
${}^{9}Il installe en sa maison la f\underline{e}mme stérile,
        \\heureuse mère au milie\underline{u} de ses fils.
      \bchapter{Psaume}
          
            \bchapter{Psaume}
            La mer voit et s’enfuit
        \\Alléluia !
         
${}^{1}Quand Israël sort\underline{i}t d’Égypte,
        \\et Jacob, de chez un pe\underline{u}ple étranger,
${}^{2}Juda fut pour Die\underline{u} un sanctuaire,
        \\Israël dev\underline{i}nt son domaine.
         
${}^{3}La mer v\underline{o}it et s’enfuit,
        \\le Jourdain reto\underline{u}rne en arrière.
${}^{4}Comme des béliers, bond\underline{i}ssent les montagnes,
        \\et les collines, c\underline{o}mme des agneaux.
         
${}^{5}Qu’as-tu, m\underline{e}r, à t’enfuir,
        \\Jourdain, à retourn\underline{e}r en arrière ?
${}^{6}Montagnes, pourquoi bond\underline{i}r comme des béliers,
        \\collines, c\underline{o}mme des agneaux ?
         
${}^{7}Tremble, t\underline{e}rre, devant le Maître,
        \\devant la face du Die\underline{u} de Jacob,
${}^{8}lui qui change le roch\underline{e}r en source
        \\et la pi\underline{e}rre en fontaine !
      \bchapter{Psaume}
          
            \bchapter{Psaume}
            À ton nom, donne la gloire !
${}^{1}Non pas à nous, Seigne\underline{u}r, non pas à nous, *
        \\mais à ton nom, donne la gloire,
        pour ton amo\underline{u}r et ta vérité.
         
${}^{2}Pourquoi les paï\underline{e}ns diraient-ils :
        \\« Où d\underline{o}nc est leur Dieu ? »
         
${}^{3}Notre Dieu, il \underline{e}st au ciel ;
        \\tout ce qu’il ve\underline{u}t, il le fait.
${}^{4}Leurs idoles : \underline{o}r et argent,
        \\ouvrages de m\underline{a}ins humaines.
         
${}^{5}Elles ont une bo\underline{u}che et ne parlent pas,
        \\des ye\underline{u}x et ne voient pas,
${}^{6}des oreilles et n’ent\underline{e}ndent pas,
        \\des narines et ne s\underline{e}ntent pas.
         
${}^{7}Leurs mains ne pe\underline{u}vent toucher, +
        \\leurs pieds ne pe\underline{u}vent marcher, *
        \\pas un son ne s\underline{o}rt de leur gosier !
         
${}^{8}Qu’ils deviennent comme elles, tous ce\underline{u}x qui les font, *
        \\ceux qui mettent leur f\underline{o}i en elles.
         
        *
         
${}^{9}Israël, mets ta f\underline{o}i dans le Seigneur :
        \\le secours, le boucli\underline{e}r, c’est lui !
${}^{10}Famille d’Aaron, mets ta f\underline{o}i dans le Seigneur :
        \\le secours, le boucli\underline{e}r, c’est lui !
${}^{11}Vous qui le craignez, ayez f\underline{o}i dans le Seigneur :
        \\le secours, le boucli\underline{e}r, c’est lui !
         
${}^{12}Le Seigneur se souvient de no\underline{u}s : il bénira ! *
        \\Il bénira la fam\underline{i}lle d’Israël,
        \\il bénira la fam\underline{i}lle d’Aaron ; *
${}^{13}il bénira tous ceux qui craignent le Seigneur,
        du plus gr\underline{a}nd au plus petit.
         
${}^{14}Que le Seigneur multipl\underline{i}e ses bienfaits
        \\pour vo\underline{u}s et vos enfants !
${}^{15}Soyez bén\underline{i}s par le Seigneur
        \\qui a fait le ci\underline{e}l et la terre !
${}^{16}Le ciel, c’est le ci\underline{e}l du Seigneur ;
        \\aux hommes, il a donn\underline{é} la terre.
         
${}^{17}Les morts ne louent p\underline{a}s le Seigneur,
        \\ni ceux qui desc\underline{e}ndent au silence.
${}^{18}Nous, les vivants, béniss\underline{o}ns le Seigneur,
        \\maintenant et pour les si\underline{è}cles des siècles !
      \bchapter{Psaume}
          
            \bchapter{Psaume}
            J’étais faible, il m’a sauvé
        \\Alléluia !
         
${}^{1}J’\underline{a}ime le Seigneur :
        \\il entend le cr\underline{i} de ma prière ;
${}^{2}il incline vers m\underline{o}i son oreille :
        \\toute ma v\underline{i}e, je l’invoquerai.
         
${}^{3}J’étais pris dans les filets de la mort,
        retenu dans les li\underline{e}ns de l’abîme, *
        \\j’éprouvais la trist\underline{e}sse et l’angoisse ;
${}^{4}j’ai invoqué le n\underline{o}m du Seigneur :
        \\« Seigneur, je t’en pr\underline{i}e, délivre-moi ! »
         
${}^{5}Le Seigneur est just\underline{i}ce et pitié,
        \\notre Die\underline{u} est tendresse.
${}^{6}Le Seigneur déf\underline{e}nd les petits :
        \\j’étais f\underline{a}ible, il m’a sauvé.
         
${}^{7}Retrouve ton rep\underline{o}s, mon âme,
        \\car le Seigneur t’a f\underline{a}it du bien.
${}^{8}Il a sauvé mon \underline{â}me de la mort, *
        \\gardé mes yeux des larmes
        et mes pi\underline{e}ds du faux pas.
         
${}^{9}Je marcherai en prés\underline{e}nce du Seigneur
        \\sur la t\underline{e}rre des vivants.
      \bchapter{Psaume}
          
            \bchapter{Psaume}
            J’élèverai la coupe du salut
${}^{10}Je crois, et je p\underline{a}rlerai,
        \\moi qui ai beauco\underline{u}p souffert,
${}^{11}moi qui ai d\underline{i}t dans mon trouble :
        \\« L’h\underline{o}mme n’est que mensonge. »
         
${}^{12}Comment rendr\underline{a}i-je au Seigneur
        \\tout le bi\underline{e}n qu’il m’a fait ?
${}^{13}J’élèverai la co\underline{u}pe du salut,
        \\j’invoquerai le n\underline{o}m du Seigneur.
${}^{14}Je tiendrai mes prom\underline{e}sses au Seigneur,
        \\oui, devant to\underline{u}t son peuple !
         
${}^{15}Il en co\underline{û}te au Seigneur
        \\de voir mour\underline{i}r les siens !
${}^{16}Ne suis-je pas, Seigneur, ton serviteur,
        ton serviteur, le f\underline{i}ls de ta servante, *
        \\moi, dont tu bris\underline{a}s les chaînes ?
         
${}^{17}Je t’offrirai le sacrif\underline{i}ce d’action de grâce,
        \\j’invoquerai le n\underline{o}m du Seigneur.
${}^{18}Je tiendrai mes prom\underline{e}sses au Seigneur,
        \\oui, devant to\underline{u}t son peuple,
${}^{19}à l’entrée de la mais\underline{o}n du Seigneur,
        \\au milie\underline{u} de Jérusalem !
      \bchapter{Psaume}
          
            \bchapter{Psaume}
            L’amour le plus fort
${}^{1}Louez le Seigne\underline{u}r, tous les peuples ;
        \\fêtez-l\underline{e}, tous les pays !
         
${}^{2}Son amour envers nous s’est montr\underline{é} le plus fort ;
        \\éternelle est la fidélit\underline{é} du Seigneur !
      \bchapter{Psaume}
          
            \bchapter{Psaume}
            Le jour que le Seigneur a fait
        \\Alléluia
         
${}^{1}Rendez grâce au Seigne\underline{u}r : Il est bon ! *
        Étern\underline{e}l est son amour !
         
${}^{2}Oui, que le d\underline{i}se Israël :
        Étern\underline{e}l est son amour !
${}^{3}Que le dise la mais\underline{o}n d’Aaron :
        Étern\underline{e}l est son amour !
${}^{4}Qu’ils le disent, ceux qui cr\underline{a}ignent le Seigneur :
        Étern\underline{e}l est son amour !
         
        *
         
${}^{5}Dans mon angoisse j’ai cri\underline{é} vers le Seigneur,
        \\et lui m’a exauc\underline{é}, mis au large.
${}^{6}Le Seigneur est pour m\underline{o}i, je ne crains pas ;
        \\que pourrait un h\underline{o}mme contre moi ?
${}^{7}Le Seigneur est avec m\underline{o}i pour me défendre,
        \\et moi, je braver\underline{a}i mes ennemis.
         
${}^{8}Mieux vaut s’appuy\underline{e}r sur le Seigneur
        que de compt\underline{e}r sur les hommes ; *
${}^{9}mieux vaut s’appuy\underline{e}r sur le Seigneur
        que de compt\underline{e}r sur les puissants !
         
${}^{10}Toutes les nati\underline{o}ns m’ont encerclé :
        \\au nom du Seigne\underline{u}r, je les détruis !
${}^{11}Elles m’ont cern\underline{é}, encerclé :
        \\au nom du Seigne\underline{u}r, je les détruis !
${}^{12}Elles m’ont cern\underline{é} comme des guêpes : +
        \\(– ce n’ét\underline{a}it qu’un feu de ronces –) *
        \\au nom du Seigne\underline{u}r, je les détruis !
         
${}^{13}On m’a poussé, bouscul\underline{é} pour m’abattre ;
        \\mais le Seigne\underline{u}r m’a défendu.
${}^{14}Ma force et mon ch\underline{a}nt, c’est le Seigneur ;
        \\il est pour m\underline{o}i le salut.
         
${}^{15}Clameurs de j\underline{o}ie et de victoire *
        \\sous les t\underline{e}ntes des justes :
        \\« Le bras du Seigneur est fort,
${}^{16}le bras du Seigne\underline{u}r se lève, *
        \\le bras du Seigne\underline{u}r est fort ! »
         
${}^{17}Non, je ne mourrai p\underline{a}s, je vivrai
        \\pour annoncer les acti\underline{o}ns du Seigneur :
${}^{18}il m’a frappé, le Seigne\underline{u}r, il m’a frappé,
        \\mais sans me livr\underline{e}r à la mort.
         
        *
         
${}^{19}Ouvrez-moi les p\underline{o}rtes de justice :
        \\j’entrerai, je rendrai gr\underline{â}ce au Seigneur.
${}^{20}« C’est ici la p\underline{o}rte du Seigneur :
        \\qu’ils \underline{e}ntrent, les justes ! »
${}^{21}Je te rends grâce car tu m’\underline{a}s exaucé :
        \\tu es pour m\underline{o}i le salut.
         
${}^{22}La pierre qu’ont rejet\underline{é}e les bâtisseurs
        \\est deven\underline{u}e la pierre d’angle :
${}^{23}c’est là l’œ\underline{u}vre du Seigneur,
        \\la merv\underline{e}ille devant nos yeux.
${}^{24}Voici le jour que f\underline{i}t le Seigneur,
        \\qu’il soit pour nous jour de f\underline{ê}te et de joie !
         
        *
         
${}^{25}Donne, Seigneur, d\underline{o}nne le salut !
        \\Donne, Seigneur, d\underline{o}nne la victoire !
         
${}^{26}Béni soit au n\underline{o}m du Seigneur
        celu\underline{i} qui vient ! *
        \\De la mais\underline{o}n du Seigneur,
        no\underline{u}s vous bénissons !
         
${}^{27}Dieu, le Seigne\underline{u}r, nous illumine. *
        \\Rameaux en main, formez vos cortèges
        jusqu’aupr\underline{è}s de l’autel.
         
${}^{28}Tu es mon Die\underline{u}, je te rends grâce, *
        \\mon Die\underline{u}, je t’exalte !
         
        *
         
${}^{29}Rendez grâce au Seigne\underline{u}r : Il est bon !
        Étern\underline{e}l est son amour !
       
      \bchapter{Psaume}
          
            \bchapter{Psaume}
            De quel amour j’aime ta loi !
\textbf{Aleph - Ils marchent dans tes voies}
${}^{1}Heureux les hommes int\underline{è}gres dans leurs voies
        qui marchent suivant la l\underline{o}i du Seigneur !
${}^{2}Heureux ceux qui g\underline{a}rdent ses exigences,
        ils le ch\underline{e}rchent de tout cœur !
${}^{3}Jamais ils ne comm\underline{e}ttent d’injustice,
        ils m\underline{a}rchent dans ses voies.
${}^{4}Toi, tu prom\underline{u}lgues des préceptes
        à observ\underline{e}r entièrement.
${}^{5}Puissent mes v\underline{o}ies s’affermir
        à observ\underline{e}r tes commandements !
${}^{6}Ainsi je ne serai p\underline{a}s humilié
        quand je cont\underline{e}mple tes volontés.
${}^{7}D’un cœur droit, je pourr\underline{a}i te rendre grâce,
        instruit de tes j\underline{u}stes décisions.
${}^{8}Tes commandem\underline{e}nts, je les observe :
        ne m’abandonne p\underline{a}s entièrement.
\textbf{Beth - Méditer sur tes préceptes}
${}^{9}Comment, jeune, garder p\underline{u}r son chemin ?
        En observ\underline{a}nt ta parole.
${}^{10}De tout mon cœ\underline{u}r, je te cherche ;
        garde-moi de fu\underline{i}r tes volontés.
${}^{11}Dans mon cœur, je cons\underline{e}rve tes promesses
        pour ne pas faill\underline{i}r envers toi.
${}^{12}Toi, Seigne\underline{u}r, tu es béni :
        apprends-m\underline{o}i tes commandements.
${}^{13}Je fais repass\underline{e}r sur mes lèvres
        chaque décisi\underline{o}n de ta bouche.
${}^{14}Je trouve dans la v\underline{o}ie de tes exigences
        plus de joie que dans to\underline{u}tes les richesses.
${}^{15}Je veux médit\underline{e}r sur tes préceptes
        et contempl\underline{e}r tes voies.
${}^{16}Je trouve en tes commandem\underline{e}nts mon plaisir,
        je n’oublie p\underline{a}s ta parole.
\textbf{Guimel - Brûlé de désir}
${}^{17}Sois bon pour ton servite\underline{u}r, et je vivrai,
        j’observer\underline{a}i ta parole.
${}^{18}O\underline{u}vre mes yeux,
        que je contemple les merv\underline{e}illes de ta loi.
${}^{19}Je suis un étrang\underline{e}r sur la terre ;
        ne me cache p\underline{a}s tes volontés.
${}^{20}Mon âme a brûl\underline{é} de désir
        en tout t\underline{e}mps pour tes décisions.
${}^{21}Tu menaces les orgueille\underline{u}x, les maudits,
        ceux qui fu\underline{i}ent tes volontés.
${}^{22}Épargne-moi l’ins\underline{u}lte et le mépris :
        je g\underline{a}rde tes exigences.
${}^{23}Lorsque des grands acc\underline{u}sent ton serviteur,
        je méd\underline{i}te sur tes ordres.
${}^{24}Je trouve mon plais\underline{i}r en tes exigences :
        ce sont \underline{e}lles qui me conseillent.
\textbf{Daleth - Collé à tes exigences}
${}^{25}Mon âme est coll\underline{é}e à la poussière ;
        fais-moi vivre sel\underline{o}n ta parole.
${}^{26}J’énumère mes v\underline{o}ies : tu me réponds ;
        apprends-m\underline{o}i tes commandements.
${}^{27}Montre-moi la v\underline{o}ie de tes préceptes,
        que je méd\underline{i}te sur tes merveilles.
${}^{28}La tristesse m’arr\underline{a}che des larmes :
        relève-m\underline{o}i selon ta parole.
${}^{29}Détourne-moi de la v\underline{o}ie du mensonge,
        fais-moi la gr\underline{â}ce de ta loi.
${}^{30}J’ai choisi la v\underline{o}ie de la fidélité,
        je m’aj\underline{u}ste à tes décisions.
${}^{31}Je me tiens coll\underline{é} à tes exigences ;
        Seigneur, garde-m\underline{o}i d’être humilié.
${}^{32}Je cours dans la v\underline{o}ie de tes volontés,
        car tu mets au l\underline{a}rge mon cœur.
\textbf{Hé - Incline mon cœur}
${}^{33}Enseigne-moi, Seigneur, le chem\underline{i}n de tes ordres ;
        à les garder, j’aur\underline{a}i ma récompense.
${}^{34}Montre-moi comment gard\underline{e}r ta loi,
        que je l’obs\underline{e}rve de tout cœur.
${}^{35}Guide-moi sur la v\underline{o}ie de tes volontés,
        l\underline{à}, je me plais.
${}^{36}Incline mon cœ\underline{u}r vers tes exigences,
        non p\underline{a}s vers le profit.
${}^{37}Détourne mes ye\underline{u}x des idoles :
        que tes chem\underline{i}ns me fassent vivre.
${}^{38}Pour ton serviteur accompl\underline{i}s ta promesse
        qui nous fer\underline{a} t’adorer.
${}^{39}Détourne l’ins\underline{u}lte qui m’effraie ;
        tes décisi\underline{o}ns sont bienfaisantes.
${}^{40}Vois, j’ai désir\underline{é} tes préceptes :
        par ta just\underline{i}ce fais-moi vivre.
\textbf{Waw - Tes volontés, je les aime}
${}^{41}Que vienne à moi, Seigne\underline{u}r, ton amour,
        et ton sal\underline{u}t, selon ta promesse.
${}^{42}J’aurai pour qui m’ins\underline{u}lte une réponse,
        car je m’appu\underline{i}e sur ta parole.
${}^{43}N’ôte pas de ma bouche la par\underline{o}le de vérité,
        car j’esp\underline{è}re tes décisions.
${}^{44}J’observerai sans rel\underline{â}che ta loi,
        toujo\underline{u}rs et à jamais.
${}^{45}Je marcher\underline{a}i librement,
        car je ch\underline{e}rche tes préceptes.
${}^{46}Devant les rois je parler\underline{a}i de tes exigences
        et ne serai p\underline{a}s humilié.
${}^{47}Je trouve mon plais\underline{i}r en tes volontés,
        oui, vraim\underline{e}nt, je les aime.
${}^{48}Je tends les mains vers tes volont\underline{é}s, je les aime,
        je méd\underline{i}te sur tes ordres.
\textbf{Zaïn - Je me rappelle ton nom}
${}^{49}Rappelle-toi ta par\underline{o}le à ton serviteur,
        celle dont tu f\underline{i}s mon espoir.
${}^{50}Elle est ma consolati\underline{o}n dans mon épreuve :
        ta prom\underline{e}sse me fait vivre.
${}^{51}Des orgueilleux m’ont accabl\underline{é} de railleries,
        je n’ai pas dévi\underline{é} de ta loi.
${}^{52}Je me rappelle tes décisi\underline{o}ns d’autrefois :
        voilà ma consolati\underline{o}n, Seigneur.
${}^{53}Face aux impies, la fure\underline{u}r me prend,
        car ils aband\underline{o}nnent ta loi.
${}^{54}J’ai fait de tes commandem\underline{e}nts mon cantique
        dans ma deme\underline{u}re d’étranger.
${}^{55}La nuit, je me rapp\underline{e}lle ton nom
        pour observ\underline{e}r ta loi.
${}^{56}Ce qui me revi\underline{e}nt, Seigneur,
        c’est de gard\underline{e}r tes préceptes.
\textbf{Heth - Je n’oublie pas ta loi}
${}^{57}Mon partage, Seigne\underline{u}r, je l’ai dit,
        c’est d’observ\underline{e}r tes paroles.
${}^{58}De tout mon cœur, je qu\underline{ê}te ton regard :
        pitié pour m\underline{o}i selon tes promesses.
${}^{59}J’examine la v\underline{o}ie que j’ai prise :
        mes pas me ram\underline{è}nent à tes exigences.
${}^{60}Je me hâte, et ne t\underline{a}rde pas,
        d’observ\underline{e}r tes volontés.
${}^{61}Les pièges de l’imp\underline{i}e m’environnent,
        je n’oublie p\underline{a}s ta loi.
${}^{62}Au milieu de la nuit, je me l\underline{è}ve et te rends grâce
        pour tes j\underline{u}stes décisions.
${}^{63}Je suis lié à tous ce\underline{u}x qui te craignent
        et qui obs\underline{e}rvent tes préceptes.
${}^{64}Ton amour, Seigne\underline{u}r, emplit la terre ;
        apprends-m\underline{o}i tes commandements.
\textbf{Teth - J’ai souffert pour mon bien}
${}^{65}Tu fais le bonhe\underline{u}r de ton serviteur,
        Seigne\underline{u}r, selon ta parole.
${}^{66}Apprends-moi à bien sais\underline{i}r, à bien juger :
        je me f\underline{i}e à tes volontés.
${}^{67}Avant d’avoir souff\underline{e}rt, je m’égarais ;
        maintenant, j’obs\underline{e}rve tes ordres.
${}^{68}Toi, tu es b\underline{o}n, tu fais du bien :
        apprends-m\underline{o}i tes commandements.
${}^{69}Des orgueilleux m’ont couv\underline{e}rt de calomnies :
        de tout cœur, je g\underline{a}rde tes préceptes.
${}^{70}Leur cœur, alourd\underline{i}, s’est fermé ;
        moi, je prends plais\underline{i}r à ta loi.
${}^{71}C’est pour mon bi\underline{e}n que j’ai souffert,
        ainsi ai-je appr\underline{i}s tes commandements.
${}^{72}Mon bonheur, c’est la l\underline{o}i de ta bouche,
        plus qu’un monceau d’\underline{o}r ou d’argent.
\textbf{Yod - Éclaire-moi}
${}^{73}Tes mains m’ont façonn\underline{é}, affermi ;
        éclaire-moi, que j’appr\underline{e}nne tes volontés.
${}^{74}À me voir, ceux qui te cr\underline{a}ignent se réjouissent,
        car j’esp\underline{è}re en ta parole.
${}^{75}Seigneur, je le sais, tes décisi\underline{o}ns sont justes ;
        tu es fid\underline{è}le quand tu m’éprouves.
${}^{76}Que j’aie pour consolati\underline{o}n ton amour
        selon tes prom\underline{e}sses à ton serviteur !
${}^{77}Que vienne à moi ta tendr\underline{e}sse, et je vivrai :
        ta l\underline{o}i fait mon plaisir.
${}^{78}Honte aux orgueilleux qui m’acc\underline{a}blent de mensonges ;
        moi, je méd\underline{i}te sur tes préceptes.
${}^{79}Qu’ils se tournent vers m\underline{o}i, ceux qui te craignent,
        ceux qui conn\underline{a}issent tes exigences.
${}^{80}Que j’aie par tes commandem\underline{e}nts le cœur intègre :
        alors je ne serai p\underline{a}s humilié.
\textbf{Kaph - Usé, j’espère encore}
${}^{81}Usé par l’att\underline{e}nte du salut,
        j’espère enc\underline{o}re ta parole.
${}^{82}L’œil usé d’att\underline{e}ndre tes promesses,
        j’ai dit : « Quand vas-t\underline{u} me consoler ? »
${}^{83}Devenu comme une outre durc\underline{i}e par la fumée,
        je n’oublie p\underline{a}s tes commandements.
${}^{84}Combien de jours ton servite\underline{u}r vivra-t-il ?
        quand jugeras-t\underline{u} mes persécuteurs ?
${}^{85}Des orgueilleux ont creusé pour m\underline{o}i une fosse
        au mépr\underline{i}s de ta loi.
${}^{86}Tous tes ordres ne s\underline{o}nt que fidélité ;
        mensonge, mes poursuiv\underline{a}nts : aide-moi !
${}^{87}Ils ont failli m’us\underline{e}r, me mettre à terre :
        je n’ai pas abandonn\underline{é} tes préceptes.
${}^{88}Fais-moi vivre sel\underline{o}n ton amour :
        j’observerai les décr\underline{e}ts de ta bouche.
\textbf{Lamed - Ta fidélité demeure}
${}^{89}Pour toujours, ta par\underline{o}le, Seigneur,
        se dr\underline{e}sse dans les cieux.
${}^{90}Ta fidélité deme\underline{u}re d’âge en âge,
        la terre que tu fix\underline{a}s tient bon.
${}^{91}Jusqu’à ce jour, le monde ti\underline{e}nt par tes décisions :
        toute ch\underline{o}se est ta servante.
${}^{92}Si je n’avais mon plais\underline{i}r dans ta loi,
        je périr\underline{a}is de misère.
${}^{93}Jamais je n’oublier\underline{a}i tes préceptes :
        par e\underline{u}x tu me fais vivre.
${}^{94}Je suis à t\underline{o}i : sauve-moi,
        car je ch\underline{e}rche tes préceptes.
${}^{95}Des impies esc\underline{o}mptent ma perte :
        moi, je réfléch\underline{i}s à tes exigences.
${}^{96}De toute perfection, j’ai v\underline{u} la limite ;
        tes volontés sont d’une ample\underline{u}r infinie.
\textbf{Mem - Une saveur dans ma bouche}
${}^{97}De quel amour j’\underline{a}ime ta loi :
        tout le jo\underline{u}r je la médite !
${}^{98}Je surpasse en habilet\underline{é} mes ennemis,
        car je fais miennes pour toujo\underline{u}rs tes volontés.
${}^{99}Je surpasse en sag\underline{e}sse tous mes maîtres,
        car je méd\underline{i}te tes exigences.
${}^{100}Je surpasse en intellig\underline{e}nce les anciens,
        car je g\underline{a}rde tes préceptes.
${}^{101}Des chemins du mal, je déto\underline{u}rne mes pas,
        afin d’observ\underline{e}r ta parole.
${}^{102}De tes décisions, je ne veux p\underline{a}s m’écarter,
        car c’est t\underline{o}i qui m’enseignes.
${}^{103}Qu’elle est douce à mon pal\underline{a}is ta promesse :
        le miel a moins de save\underline{u}r dans ma bouche !
${}^{104}Tes préceptes m’ont donn\underline{é} l’intelligence :
        je hais tout chem\underline{i}n de mensonge.
\textbf{Noun - La lampe de ma route}
${}^{105}Ta parole est la lumi\underline{è}re de mes pas,
        la l\underline{a}mpe de ma route.
${}^{106}Je l’ai juré, je tiendr\underline{a}i mon serment,
        j’observerai tes j\underline{u}stes décisions.
${}^{107}J’ai vraiment trop souff\underline{e}rt, Seigneur ;
        fais-moi v\underline{i}vre selon ta parole.
${}^{108}Accepte en offrande ma pri\underline{è}re, Seigneur :
        apprends-m\underline{o}i tes décisions.
${}^{109}À tout instant j’exp\underline{o}se ma vie :
        je n’oublie ri\underline{e}n de ta loi.
${}^{110}Des impies me t\underline{e}ndent un piège :
        je ne dévie p\underline{a}s de tes préceptes.
${}^{111}Tes exigences rester\underline{o}nt mon héritage,
        la j\underline{o}ie de mon cœur.
${}^{112}Mon cœur incline à pratiqu\underline{e}r tes commandements :
        c’est à jam\underline{a}is ma récompense.
\textbf{Samek - J’aime, j’espère}
${}^{113}Je hais les cœurs partagés ;
        j’\underline{a}ime ta loi.
${}^{114}Toi, mon abr\underline{i}, mon bouclier !
        J’esp\underline{è}re en ta parole.
${}^{115}Écartez-vous de m\underline{o}i, méchants :
        je garderai les volont\underline{é}s de mon Dieu.
${}^{116}Que ta promesse me souti\underline{e}nne, et je vivrai :
        ne déçois p\underline{a}s mon attente.
${}^{117}Sois mon appu\underline{i} : je serai sauvé ;
        j’ai toujours tes commandem\underline{e}nts devant les yeux.
${}^{118}Tu rejettes ceux qui fu\underline{i}ent tes commandements :
        leur r\underline{u}se les égare.
${}^{119}Tu mets au rebut tous les imp\underline{i}es de la terre ;
        c’est pourquoi j’\underline{a}ime tes exigences.
${}^{120}Ma chair tremble de pe\underline{u}r devant toi :
        tes décisions m’insp\underline{i}rent la crainte.
\textbf{Aïn - Il est temps que tu agisses}
${}^{121}J’ai agi selon le dr\underline{o}it et la justice :
        ne me livre p\underline{a}s à mes bourreaux.
${}^{122}Assure le bonhe\underline{u}r de ton serviteur :
        que les orgueille\underline{u}x ne me tourmentent plus !
${}^{123}Mes yeux se sont usés à guett\underline{e}r le salut
        et les prom\underline{e}sses de ta justice.
${}^{124}Agis pour ton serviteur sel\underline{o}n ton amour,
        apprends-m\underline{o}i tes commandements.
${}^{125}Je suis ton servite\underline{u}r, éclaire-moi :
        je connaîtr\underline{a}i tes exigences.
${}^{126}Seigneur, il est t\underline{e}mps que tu agisses :
        on a viol\underline{é} ta loi.
${}^{127}Aussi j’\underline{a}ime tes volontés,
        plus que l’\underline{o}r le plus précieux.
${}^{128}Je me règle sur chac\underline{u}n de tes préceptes,
        je hais tout chem\underline{i}n de mensonge.
\textbf{Pé - Ta parole illumine}
${}^{129}Quelle merv\underline{e}ille, tes exigences,
        aussi mon \underline{â}me les garde !
${}^{130}Déchiffrer ta par\underline{o}le illumine
        et les s\underline{i}mples comprennent.
${}^{131}La bouche grande ouv\underline{e}rte, j’aspire,
        assoiff\underline{é} de tes volontés.
${}^{132}Aie pitié de m\underline{o}i, regarde-moi :
        tu le fais pour qui \underline{a}ime ton nom.
${}^{133}Que ta promesse ass\underline{u}re mes pas :
        qu’aucun mal ne tri\underline{o}mphe de moi !
${}^{134}Rachète-moi de l’oppressi\underline{o}n des hommes,
        que j’obs\underline{e}rve tes préceptes.
${}^{135}Pour ton serviteur que ton vis\underline{a}ge s’illumine :
        apprends-m\underline{o}i tes commandements.
${}^{136}Mes yeux ruiss\underline{e}llent de larmes
        car on n’observe p\underline{a}s ta loi.
\textbf{Çadé - Vérité, ta loi}
${}^{137}Toi, tu es j\underline{u}ste, Seigneur,
        tu es dr\underline{o}it dans tes décisions.
${}^{138}Tu promulgues tes exig\underline{e}nces avec justice,
        avec enti\underline{è}re fidélité.
${}^{139}Quand mes oppresseurs oubl\underline{i}ent ta parole,
        une arde\underline{u}r me consume.
${}^{140}Ta promesse tout enti\underline{è}re est pure,
        elle est aim\underline{é}e de ton serviteur.
${}^{141}Moi, le chét\underline{i}f, le méprisé,
        je n’oublie p\underline{a}s tes préceptes.
${}^{142}Justice étern\underline{e}lle est ta justice,
        et vérit\underline{é}, ta loi.
${}^{143}La détresse et l’ang\underline{o}isse m’ont saisi ;
        je trouve en tes volont\underline{é}s mon plaisir.
${}^{144}Justice étern\underline{e}lle, tes exigences ;
        éclaire-m\underline{o}i, et je vivrai.
\textbf{Qoph - J’appelle : tu es proche}
${}^{145}J’appelle de tout mon cœ\underline{u}r : réponds-moi ;
        je garder\underline{a}i tes commandements.
${}^{146}Je t’appelle, Seigne\underline{u}r, sauve-moi ;
        j’observer\underline{a}i tes exigences.
${}^{147}Je devance l’aur\underline{o}re et j’implore :
        j’esp\underline{è}re en ta parole.
${}^{148}Mes yeux devancent la f\underline{i}n de la nuit
        pour médit\underline{e}r sur ta promesse.
${}^{149}Dans ton amour, Seigneur, éco\underline{u}te ma voix :
        selon tes décisi\underline{o}ns fais-moi vivre !
${}^{150}Ceux qui poursuivent le m\underline{a}l s’approchent,
        ils s’él\underline{o}ignent de ta loi.
${}^{151}Toi, Seigne\underline{u}r, tu es proche,
        tout dans tes \underline{o}rdres est vérité.
${}^{152}Depuis longt\underline{e}mps je le sais :
        tu as fondé pour toujo\underline{u}rs tes exigences.
\textbf{Resh - Fais-moi vivre}
${}^{153}Vois ma mis\underline{è}re : délivre-moi ;
        je n’oublie p\underline{a}s ta loi.
${}^{154}Soutiens ma ca\underline{u}se : défends-moi,
        en ta prom\underline{e}sse fais-moi vivre !
${}^{155}Le salut s’él\underline{o}igne des impies
        qui ne cherchent p\underline{a}s tes commandements.
${}^{156}Seigneur, ta tendr\underline{e}sse est sans mesure :
        selon ta décisi\underline{o}n fais-moi vivre !
${}^{157}Ils sont nombreux mes persécute\underline{u}rs, mes oppresseurs ;
        je ne dévie p\underline{a}s de tes exigences.
${}^{158}J’ai vu les renég\underline{a}ts : ils me répugnent,
        car ils ign\underline{o}rent ta promesse.
${}^{159}Vois combien j’aime tes préc\underline{e}ptes, Seigneur,
        fais-moi v\underline{i}vre selon ton amour !
${}^{160}Le fondement de ta par\underline{o}le est vérité ;
        éternelles sont tes j\underline{u}stes décisions.
\textbf{Shine - La paix de qui aime ta loi}
${}^{161}Des grands me perséc\underline{u}tent sans raison ;
        mon cœur ne cr\underline{a}int que ta parole.
${}^{162}Tel celui qui tro\underline{u}ve un grand butin,
        je me réjou\underline{i}s de tes promesses.
${}^{163}Je hais, je dét\underline{e}ste le mensonge ;
        ta l\underline{o}i, je l’aime.
${}^{164}Sept fois chaque jo\underline{u}r, je te loue
        pour tes j\underline{u}stes décisions.
${}^{165}Grande est la paix de qui \underline{a}ime ta loi ;
        jam\underline{a}is il ne trébuche.
${}^{166}Seigneur, j’attends de t\underline{o}i le salut :
        j’accompl\underline{i}s tes volontés.
${}^{167}Tes exigences, mon \underline{â}me les observe :
        oui, vraim\underline{e}nt, je les aime.
${}^{168}J’observe tes exig\underline{e}nces et tes préceptes :
        toutes mes v\underline{o}ies sont devant toi.
\textbf{Taw - Viens chercher ton serviteur}
${}^{169}Que mon cri parvi\underline{e}nne devant toi,
        éclaire-moi selon ta par\underline{o}le, Seigneur.
${}^{170}Que ma prière arr\underline{i}ve jusqu’à toi ;
        délivre-m\underline{o}i selon ta promesse.
${}^{171}Que chante sur mes l\underline{è}vres ta louange,
        car tu m’appr\underline{e}nds tes commandements.
${}^{172}Que ma langue red\underline{i}se tes promesses,
        car tout est just\underline{i}ce en tes volontés.
${}^{173}Que ta main vi\underline{e}nne à mon aide,
        car j’ai chois\underline{i} tes préceptes.
${}^{174}J’ai le désir de ton sal\underline{u}t, Seigneur :
        ta l\underline{o}i fait mon plaisir.
${}^{175}Que je vive et que mon \underline{â}me te loue !
        Tes décisi\underline{o}ns me soient en aide !
${}^{176}Je m’égare, breb\underline{i}s perdue : *
        \\viens chercher ton serviteur.
        Je n’oublie p\underline{a}s tes volontés.
