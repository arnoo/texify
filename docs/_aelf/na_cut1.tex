  
  
    
    \bbook{NAHOUM}{NAHOUM}
      
         
      \bchapter{}
      \begin{verse}
${}^{1}Proclamation sur Ninive. Livre de la vision de Nahoum, du village d’Elqosh.
      
         
${}^{2}Un Dieu jaloux et vengeur, tel est le Seigneur !
        Il se venge, le Seigneur, il est empli de fureur !
        \\Le Seigneur se venge de ses adversaires,
        lui, il garde rancune à ses ennemis.
${}^{3}Le Seigneur est lent à la colère,
        et sa puissance est grande,
        \\mais il ne laisse absolument rien d’impuni,
        lui, le Seigneur.
         
        \\Dans l’ouragan et la tempête, son chemin !
        La nuée est la poussière que soulèvent ses pas.
${}^{4}Il menace la mer et la dessèche,
        il fait tarir tous les fleuves.
        \\Le Bashane et le Carmel sont flétris,
        flétrie, la fleur du Liban !
${}^{5}Les montagnes tremblent devant lui,
        les collines chancellent,
        \\la terre se soulève devant sa face,
        le monde et tous ses habitants.
${}^{6}Devant son indignation, qui peut tenir ?
        Qui peut se dresser devant l’ardeur de sa colère ?
        \\Sa fureur se répand comme le feu,
        et les rochers se brisent devant lui.
         
${}^{7}Le Seigneur est bon,
        c’est une forteresse au jour de la détresse.
        \\Il protège ceux qui se réfugient en lui,
${}^{8}quand déborde le flot impétueux.
        \\Il réduit à néant ceux qui se dressent contre lui,
        il poursuit ses ennemis jusqu’aux ténèbres.
${}^{9}Quelle idée vous faites-vous du Seigneur ?
        C’est lui qui réduit à néant Ninive ;
        vous ne connaîtrez pas une nouvelle détresse.
${}^{10}Tels des fourrés d’épines enchevêtrées,
        tels des liserons entrelacés,
        \\tes ennemis seront dévorés
        comme de la paille bien sèche.
${}^{11}Le voici loin de toi
        celui qui trame le mal contre le Seigneur,
        l’homme aux projets de vaurien.
${}^{12}Ainsi parle le Seigneur :
        \\Si nombreux et si prospères soient-ils,
        ils seront fauchés et ils disparaîtront.
        \\Si je t’ai humiliée,
        désormais je ne t’humilierai plus.
${}^{13}Et maintenant, je vais briser le joug qui pèse sur toi,
        et rompre tes chaînes.
${}^{14}Voici ce que le Seigneur a décrété contre le roi de Ninive :
        Nulle descendance ne perpétuera ton nom.
        \\De la maison de tes dieux je supprimerai les idoles,
        qu’elles soient sculptées ou en métal fondu.
        \\Je te prépare un tombeau car tu es méprisable.
       
      
         
      \bchapter{}
        ${}^{1}Voici sur les montagnes
        les pas du messager qui annonce la paix.
        \\Célèbre tes fêtes, ô Juda,
        accomplis tes vœux,
        \\car le Mauvais\\ne recommencera plus à passer sur toi :
        il a été entièrement anéanti.
        
           
${}^{2}Et voici\\contre toi ceux qui veulent te détruire.
        \\Monte la garde au rempart,
        surveille la route,
        \\ceinture-toi les reins,
        rassemble toutes tes forces.
        ${}^{3}Le Seigneur revient.
        \\Avec lui, la splendeur de Jacob
        comme celle d’Israël,
        \\alors que les pillards les avaient pillés
        et avaient ravagé leurs vignobles.
         
${}^{4}Le bouclier de ses guerriers rougeoie,
        ses soldats sont vêtus d’écarlate.
        \\Les chars flamboient de tous leurs aciers
        quand ils montent en ligne,
        et les coursiers s’agitent.
${}^{5}Dans les rues, les chars foncent avec furie,
        ils se précipitent vers les places ;
        \\à les voir, on dirait des torches,
        comme des éclairs, ils zigzaguent.
${}^{6}On fait appel aux troupes d’élite ;
        dans leur course elles trébuchent,
        elles se hâtent vers le rempart.
        On met en place le bouclier de protection.
${}^{7}Les portes qui donnent sur le Fleuve s’ouvrent,
        le palais vacille et s’effondre.
${}^{8}La Princesse est déportée ;
        ses servantes sont emmenées,
        \\elles gémissent comme des colombes,
        elles se frappent la poitrine.
${}^{9}Ninive est comme un réservoir
        dont les eaux s’échappent.
        \\« Arrêtez, arrêtez ! »
        Mais nul ne se retourne.
${}^{10}Pillez l’argent ! Pillez l’or !
        C’est un trésor inépuisable,
        une richesse inimaginable d’objets précieux !
${}^{11}Pillage, saccage, ravage !
        Le cœur fond, les genoux flageolent.
        \\Tremblement des reins !
        Tous les visages changent de couleur.
         
${}^{12}Où est le repaire des lions, l’antre des lionceaux ?
        La lionne restait là quand partait le lion,
        et nul n’inquiétait les lionceaux.
${}^{13}Pour ses petits, le lion déchirait ;
        pour ses lionnes, il étranglait ;
        \\il remplissait de proies ses tanières,
        et de viande déchirée ses antres.
         
${}^{14}Maintenant je m’adresse à toi, Ninive
        – oracle du Seigneur de l’univers – :
        \\Je ferai flamber tes chars et les réduirai en fumée ;
        tes lionceaux, l’épée les dévorera.
        \\Je supprimerai de la terre tes rapines,
        et l’on n’entendra plus la voix de tes messagers.
      
         
      \bchapter{}
        ${}^{1}Malheur à la ville sanguinaire
        toute de mensonge, pleine de rapines,
        et qui ne lâche jamais sa proie.
        ${}^{2}Écoutez\\ ! Claquements des fouets,
        fracas des roues,
        \\galop des chevaux,
        roulement des chars !
        ${}^{3}Cavaliers qui chargent,
        épées qui flamboient,
        lances qui étincellent !
        \\Innombrables blessés,
        accumulation de morts,
        \\cadavres à perte de vue !
        On bute sur les cadavres !
        
           
         
${}^{4}Voilà pour les prostitutions sans nombre de la Prostituée,
        belle et pleine de charme, maîtresse en sortilèges,
        \\prenant des nations dans ses filets par ses prostitutions,
        et des peuples par ses sortilèges !
${}^{5}Maintenant je m’adresse à toi
        – oracle du Seigneur de l’univers – :
        \\je vais relever ta robe jusqu’à ton visage,
        j’exhiberai ta nudité devant les nations,
        devant les royaumes ton infamie.
        ${}^{6}Je vais jeter sur toi des choses horribles,
        te déshonorer, te donner en spectacle.
        ${}^{7}Tous ceux qui te verront s’enfuiront en disant :
        « Ninive est dévastée !
        Qui la plaindra\\ ? »
        \\Où donc te trouver des consolateurs ?
${}^{8}Vaudrais-tu mieux que la ville de No-Amone,
        située le long du Nil, entourée d’eau,
        \\avec pour avant-mur la mer,
        et la mer encore pour rempart ?
${}^{9}L’Éthiopie et l’Égypte étaient sa force,
        une force sans limite.
        Pouth et les Libyens lui portaient assistance.
${}^{10}Or elle a été condamnée à l’exil,
        elle est partie en captivité :
        \\ses petits enfants eux-mêmes
        ont été massacrés à tous les carrefours.
        \\Ses notables, on les a tirés au sort,
        et tous ses grands ont été chargés de chaînes.
        
           
         
${}^{11}Toi aussi, tu seras ivre,
        complètement hébétée.
        \\Toi aussi, tu chercheras un refuge
        loin de l’ennemi.
${}^{12}Tes places fortes sont toutes comme des figuiers
        chargés de fruits précoces :
        \\on les secoue, les figues tombent
        dans la bouche de qui les mange.
${}^{13}Regarde les troupes qui sont dans tes murs,
        ce sont de vraies femmelettes.
        \\Les portes de ton pays
        s’ouvrent toutes grandes à l’ennemi ;
        le feu a dévoré tes verrous.
${}^{14}Puise de l’eau en prévision du siège,
        consolide tes places fortes.
        \\Va dans la boue, foule l’argile,
        saisis le moule à briques.
${}^{15}Là, le feu te dévorera, l’épée te supprimera,
        comme dévorent les criquets.
        
           
         
        \\Pullule comme les criquets,
        pullule comme les sauterelles !
${}^{16}Tu as multiplié tes marchands
        plus que les étoiles du ciel,
        les criquets déploient leurs ailes, ils s’envolent.
${}^{17}Tes gardes sont comme des sauterelles,
        tes recruteurs comme un essaim d’insectes.
        \\Ils campent sur les murs par un jour de froid ;
        vienne le soleil, ils s’envolent,
        et nul ne sait où ils se trouvent.
        \\Mais où sont-ils ?
        
           
         
${}^{18}Tes bergers sommeillent, ô roi d’Assour,
        tes capitaines sont endormis,
        \\tes troupes sont dispersées sur les montagnes,
        et nul ne les rassemble.
${}^{19}À ta blessure, pas de remède,
        ta plaie est incurable !
        \\Tous ceux qui apprennent de tes nouvelles
        applaudissent des deux mains à ton sujet.
        \\Car ta cruauté, sur qui n’a-t-elle pas passé,
        et repassé ?
        
           
