  
  
    
    \bbook{LETTRE AUX HÉBREUX}{LETTRE AUX HÉBREUX}
      
         
      \bchapter{}
        ${}^{1}À bien des reprises
        et de bien des manières,
        \\Dieu, dans le passé,
        a parlé à nos pères par les prophètes ;
        ${}^{2}mais à la fin, en ces jours où nous sommes,
        il nous a parlé par son Fils
        \\qu’il a établi héritier de toutes choses
        et par qui il a créé les mondes.
        ${}^{3}Rayonnement de la gloire de Dieu,
        expression parfaite de son être,
        \\le Fils, qui porte l’univers
        par sa parole puissante,
        \\après avoir accompli la purification des péchés,
        \\s’est assis à la droite de la Majesté divine
        dans les hauteurs des cieux ;
        ${}^{4}et il est devenu bien supérieur aux anges,
        dans la mesure même où il a reçu en héritage
        un nom si différent du leur.
        
           
${}^{5}En effet, Dieu déclara-t-il jamais à un ange :
        \\Tu es mon Fils,
        \\moi, aujourd’hui, je t’ai engendré ?
      Ou bien encore :
        \\Moi, je serai pour lui un père,
        \\et lui sera pour moi un fils ?
${}^{6}À l’inverse, au moment d’introduire le Premier-né dans le monde à venir, il dit :
        \\Que se prosternent devant lui
        \\tous les anges de Dieu.
${}^{7}À l’adresse des anges, il dit :
        \\Il fait de ses anges des esprits,
        \\et de ses serviteurs des flammes ardentes .
${}^{8}Mais à l’adresse du Fils, il dit :
        \\Ton trône à toi, Dieu, est pour les siècles des siècles,
        \\le sceptre de la droiture est ton sceptre royal  ;
${}^{9}tu as aimé la justice, tu as réprouvé le mal,
        \\c’est pourquoi, toi, Dieu, ton Dieu t’a consacré
        \\d’une onction de joie, de préférence à tes compagnons ;
${}^{10}et encore :
        \\C’est toi, Seigneur,
        \\qui, au commencement, as fondé la terre,
        \\et le ciel est l’ouvrage de tes mains.
${}^{11}Ils passeront, mais toi, tu demeures ;
        \\ils s’useront comme un habit, l’un et l’autre ;
${}^{12}comme un manteau, tu les enrouleras,
        \\comme un habit, ils seront remplacés ;
        \\mais toi, tu es le même,
        \\tes années n’auront pas de fin.
${}^{13}Dieu a-t-il jamais dit à l’adresse d’un ange :
        \\Siège à ma droite,
        \\jusqu’à ce que je fasse de tes ennemis
        \\le marchepied de ton trône ?
${}^{14}Les anges ne sont-ils pas tous des esprits chargés d’une fonction, envoyés pour le service de ceux qui doivent avoir en héritage le salut ?
      
         
      \bchapter{}
      \begin{verse}
${}^{1}Il nous faut donc d’autant plus prêter attention à ce que nous avons entendu, afin de ne pas nous fourvoyer. 
${}^{2}En effet, si la parole annoncée par des anges s’est trouvée confirmée, et si toute transgression ou désobéissance a reçu une juste sanction, 
${}^{3}comment pourrons-nous sortir d’affaire si nous négligeons un pareil salut ? Celui-ci a été annoncé par le Seigneur au commencement ; ceux qui avaient entendu ont confirmé pour nous ce salut, 
${}^{4}et Dieu joignait son témoignage par des signes, des prodiges, toutes sortes de miracles, et le partage des dons de l’Esprit Saint, selon sa volonté.
      
         
${}^{5}Car ce n’est pas à des anges que Dieu a soumis le monde à venir, dont nous parlons. 
${}^{6}Un psaume l’atteste en disant :
        \\Qu’est-ce que l’homme pour que tu penses à lui,
        \\le fils d’un homme, que tu en prennes souci ?
        ${}^{7}Tu l’as abaissé un peu au-dessous des anges,
        \\tu l’as couronné de gloire et d’honneur  ;
        ${}^{8}tu as mis sous ses pieds toutes choses .
      Quand Dieu lui a tout soumis, il n’a rien exclu de cette soumission. Maintenant, nous ne voyons pas encore que tout lui soit soumis ; 
${}^{9}mais Jésus, qui a été abaissé un peu au-dessous des anges, nous le voyons couronné de gloire et d’honneur à cause de sa Passion et de sa mort. Si donc il a fait l’expérience de la mort, c’est, par grâce de Dieu, au profit de tous.
${}^{10}Celui pour qui et par qui tout existe voulait conduire une multitude de fils jusqu’à la gloire ; c’est pourquoi il convenait qu’il mène à sa perfection, par des souffrances, celui qui est à l’origine de leur salut. 
${}^{11}Car celui qui sanctifie, et ceux qui sont sanctifiés, doivent tous avoir même origine ; pour cette raison, Jésus n’a pas honte de les appeler ses frères, 
${}^{12}quand il dit :
        \\Je proclamerai ton nom devant mes frères,
        \\je te chanterai en pleine assemblée,
${}^{13}et encore :
        \\Moi, je mettrai ma confiance en lui,
      et encore :
        \\Me voici, moi et les enfants que Dieu m’a donnés.
${}^{14}Puisque les enfants des hommes ont en commun le sang et la chair, Jésus a partagé, lui aussi, pareille condition : ainsi, par sa mort, il a pu réduire à l’impuissance celui qui possédait le pouvoir de la mort, c’est-à-dire le diable, 
${}^{15}et il a rendu libres tous ceux qui, par crainte de la mort, passaient toute leur vie dans une situation d’esclaves. 
${}^{16}Car ceux qu’il prend en charge, ce ne sont pas les anges, c’est la descendance d’Abraham. 
${}^{17}Il lui fallait donc se rendre en tout semblable à ses frères, pour devenir un grand prêtre miséricordieux et digne de foi pour les relations avec Dieu, afin d’enlever les péchés du peuple. 
${}^{18}Et parce qu’il a souffert jusqu’au bout l’épreuve de sa Passion, il est capable de porter secours à ceux qui subissent une épreuve.
      <h2 class="intertitle" id="d85e395261">1. Grand prêtre digne de foi (3 – 4,14)</h2>
      
         
      \bchapter{}
      \begin{verse}
${}^{1}Ainsi donc, frères saints, vous qui avez en partage une vocation céleste, considérez Jésus, l’apôtre et le grand prêtre de notre confession de foi : 
${}^{2}pour celui qui l’a institué, il est, comme Moïse, digne de foi dans toute sa maison. 
${}^{3}Lui, il a même été jugé digne d’une plus grande gloire que Moïse, dans la mesure où le constructeur de la maison reçoit plus d’honneur que la maison elle-même. 
${}^{4}Car toute maison est construite par quelqu’un, et celui qui a tout construit, le Christ, est Dieu. 
${}^{5}Moïse, lui, a été digne de foi dans toute la maison de Dieu en qualité d’intendant, pour attester ce qui allait être dit. 
${}^{6}Mais le Christ, lui, est digne de foi en qualité de Fils à la tête de sa maison ; et nous sommes sa maison, si du moins nous maintenons l’assurance et la fierté de l’espérance.
      
         
${}^{7}C’est pourquoi, comme le dit l’Esprit Saint dans un psaume :
        \\Aujourd’hui, si vous entendez sa voix,
        ${}^{8}n’endurcissez pas votre cœur
        \\comme au temps du défi,
        \\comme au jour de l’épreuve dans le désert,
        ${}^{9}quand vos pères m’ont mis à l’épreuve et provoqué.
        <p class="verset_anchor" id="para_bib_he_3_10">Alors ils m’ont vu à l’œuvre 
${}^{10}pendant quarante ans ;
        \\oui, je me suis emporté contre cette génération,
        \\et j’ai dit : Toujours ils ont le cœur égaré,
        \\ils n’ont pas connu mes chemins.
        ${}^{11}Dans ma colère, j’en ai fait le serment :
        \\On verra bien s’ils entreront dans mon repos !
${}^{12}Frères, veillez à ce que personne d’entre vous n’ait un cœur mauvais que le manque de foi sépare du Dieu vivant. 
${}^{13}Au contraire, encouragez-vous les uns les autres jour après jour, aussi longtemps que retentit l’« aujourd’hui » de ce psaume, afin que personne parmi vous ne s’endurcisse en se laissant tromper par le péché. 
${}^{14}Car nous sommes devenus les compagnons du Christ, si du moins nous maintenons fermement, jusqu’à la fin, notre engagement premier. 
${}^{15}Il est dit en effet :
        \\Aujourd’hui, si vous entendez sa voix,
        \\n’endurcissez pas votre cœur
        \\comme au temps du défi.
${}^{16}Qui donc a défié Dieu après l’avoir entendu ? N’est-ce pas tous ceux que Moïse avait fait sortir d’Égypte ? 
${}^{17}Contre qui Dieu s’est-il emporté pendant quarante ans ? N’est-ce pas contre ceux qui avaient péché, et dont les cadavres sont tombés dans le désert ? 
${}^{18}À qui a-t-il fait le serment qu’ils n’entreraient pas dans son repos, sinon à ceux qui avaient refusé de croire ? 
${}^{19}Nous constatons qu’ils n’ont pas pu entrer à cause de leur manque de foi.
      
         
      \bchapter{}
      \begin{verse}
${}^{1}Craignons donc, tant que demeure la promesse d’entrer dans le repos de Dieu, craignons que l’un d’entre vous n’arrive, en quelque sorte, trop tard. 
${}^{2}Certes, nous avons reçu une Bonne Nouvelle, comme ces gens-là ; cependant, la parole entendue ne leur servit à rien, parce qu’elle ne fut pas accueillie avec foi par ses auditeurs. 
${}^{3}Mais nous qui sommes venus à la foi, nous entrons dans le repos dont il est dit :
        \\Dans ma colère, j’en ai fait le serment :
        \\On verra bien s’ils entreront dans mon repos !
      Le travail de Dieu, assurément, était accompli depuis la fondation du monde, 
${}^{4}comme l’Écriture le dit à propos du septième jour : Et Dieu se reposa le septième jour de tout son travail. 
${}^{5}Et dans le psaume, de nouveau : On verra bien s’ils entreront dans mon repos ! 
${}^{6}Puisque certains doivent encore y entrer, et que les premiers à avoir reçu une Bonne Nouvelle n’y sont pas entrés à cause de leur refus de croire, 
${}^{7}il fixe de nouveau un jour, un aujourd’hui, en disant bien longtemps après, dans le psaume de David déjà cité :
        \\Aujourd’hui, si vous entendez sa voix,
        \\n’endurcissez pas votre cœur.
${}^{8}Car si Josué leur avait donné le repos, David ne parlerait pas après cela d’un autre jour. 
${}^{9}Ainsi, un repos sabbatique doit encore advenir pour le peuple de Dieu. 
${}^{10}Car Celui qui est entré dans son repos s’est reposé lui aussi de son travail, comme Dieu s’est reposé du sien. 
${}^{11}Empressons-nous donc d’entrer dans ce repos-là, afin que plus personne ne tombe en suivant l’exemple de ceux qui ont refusé de croire.
${}^{12}Elle est vivante, la parole de Dieu, énergique et plus coupante qu’une épée à deux tranchants ; elle va jusqu’au point de partage de l’âme et de l’esprit, des jointures et des moelles ; elle juge des intentions et des pensées du cœur. 
${}^{13}Pas une créature n’échappe à ses yeux, tout est nu devant elle, soumis à son regard ; nous aurons à lui rendre des comptes.
${}^{14}En Jésus, le Fils de Dieu, nous avons le grand prêtre par excellence, celui qui a traversé les cieux ; tenons donc ferme l’affirmation de notre foi.
      <h2 class="intertitle hmbot" id="d85e395672">2. Grand prêtre compatissant (4,15 – 5,10)</h2>
${}^{15}En effet, nous n’avons pas un grand prêtre incapable de compatir à nos faiblesses, mais un grand prêtre éprouvé en toutes choses, à notre ressemblance, excepté le péché. 
${}^{16}Avançons-nous donc avec assurance vers le Trône de la grâce, pour obtenir miséricorde et recevoir, en temps voulu, la grâce de son secours.
      
         
      \bchapter{}
      \begin{verse}
${}^{1}Tout grand prêtre, en effet, est pris parmi les hommes ; il est établi pour intervenir en faveur des hommes dans leurs relations avec Dieu ; il doit offrir des dons et des sacrifices pour les péchés. 
${}^{2}Il est capable de compréhension envers ceux qui commettent des fautes par ignorance ou par égarement, car il est, lui aussi, rempli de faiblesse ; 
${}^{3}et, à cause de cette faiblesse, il doit offrir des sacrifices pour ses propres péchés comme pour ceux du peuple. 
${}^{4}On ne s’attribue pas cet honneur à soi-même, on est appelé par Dieu, comme Aaron.
${}^{5}Il en est bien ainsi pour le Christ : il ne s’est pas donné à lui-même la gloire de devenir grand prêtre ; il l’a reçue de Dieu, qui lui a dit :
        \\Tu es mon Fils,
        \\moi, aujourd’hui, je t’ai engendré ,
${}^{6}car il lui dit aussi dans un autre psaume :
        \\Tu es prêtre de l’ordre de Melkisédek
        \\pour l’éternité .
${}^{7}Pendant les jours de sa vie dans la chair, il offrit, avec un grand cri et dans les larmes, des prières et des supplications à Dieu qui pouvait le sauver de la mort, et il fut exaucé en raison de son grand respect. 
${}^{8}Bien qu’il soit le Fils, il apprit par ses souffrances l’obéissance 
${}^{9}et, conduit à sa perfection, il est devenu pour tous ceux qui lui obéissent la cause du salut éternel, 
${}^{10}car Dieu l’a proclamé grand prêtre de l’ordre de Melkisédek.
${}^{11}Sur ce sujet, nous avons bien des choses à dire, et elles sont difficiles à expliquer, puisque vous êtes devenus paresseux pour écouter. 
${}^{12}Depuis le temps, vous devriez être capables d’enseigner mais, de nouveau, vous avez besoin qu’on vous enseigne les tout premiers éléments des paroles de Dieu ; vous en êtes au point d’avoir besoin de lait, et non de nourriture solide. 
${}^{13}Celui qui est encore nourri de lait ne comprend rien à la parole de justice : ce n’est qu’un petit enfant. 
${}^{14}Aux adultes, la nourriture solide, eux qui, par la pratique, ont des sens exercés au discernement du bien et du mal.
      
         
      \bchapter{}
      \begin{verse}
${}^{1}Dès lors, laissons de côté l’enseignement élémentaire sur le Christ, élevons-nous à la perfection d’adultes, au lieu de poser une nouvelle fois les fondements, à savoir : conversion avec rejet des œuvres mortes et foi en Dieu, 
${}^{2}instruction sur les baptêmes et l’imposition des mains, la résurrection des morts et le jugement définitif. 
${}^{3}Nous élever à la perfection d’adultes, voilà donc ce que nous allons faire si Dieu le permet.
${}^{4}Une fois que l’on a reçu la lumière, goûté au don du ciel, que l’on a eu part à l’Esprit Saint, 
${}^{5}que l’on a goûté la parole excellente de Dieu, ainsi que les puissances du monde à venir, 
${}^{6}si l’on retombe, il est impossible d’être amené à une nouvelle conversion, alors que soi-même, on crucifie de nouveau le Fils de Dieu et on le tourne en dérision. 
${}^{7}En effet, si la terre a absorbé la pluie qui tombe fréquemment sur elle, et produit des plantes utiles à ceux pour qui elle est cultivée, elle reçoit de Dieu sa part de bénédiction. 
${}^{8}Mais si elle donne des épines et des chardons, elle est jugée sans valeur et bien près d’être maudite : elle finira par être brûlée.
${}^{9}En ce qui vous concerne, mes bien-aimés, et malgré ce que nous venons de dire, nous sommes convaincus que vous êtes dans la meilleure de ces situations, celle qui est liée au salut. 
${}^{10}Car Dieu n’est pas injuste : il n’oublie pas votre action ni l’amour que vous avez manifesté à son égard, en vous mettant au service des fidèles et en vous y tenant. 
${}^{11}Notre désir est que chacun d’entre vous manifeste le même empressement jusqu’à la fin, pour que votre espérance se réalise pleinement ; 
${}^{12}ne devenez pas paresseux, imitez plutôt ceux qui, par la foi et la persévérance, obtiennent l’héritage promis.
${}^{13}Quand Dieu fit la promesse à Abraham, comme il ne pouvait prêter serment par quelqu’un de plus grand que lui, il prêta serment par lui-même, 
${}^{14}et il dit :
        \\Je te comblerai de bénédictions
        \\et je multiplierai ta descendance .
${}^{15}Et ainsi, par sa persévérance, Abraham a obtenu ce que Dieu lui avait promis. 
${}^{16}Les hommes prêtent serment par un plus grand qu’eux, et le serment est entre eux une garantie qui met fin à toute discussion ; 
${}^{17}Dieu a donc pris le moyen du serment quand il a voulu montrer aux héritiers de la promesse, de manière encore plus claire, que sa décision était irrévocable. 
${}^{18}Dieu s’est ainsi engagé doublement de façon irrévocable, et il est impossible que Dieu ait menti. Cela nous encourage fortement, nous qui avons cherché refuge dans l’espérance qui nous était proposée et que nous avons saisie. 
${}^{19}Cette espérance, nous la tenons comme une ancre sûre et solide pour l’âme ; elle entre au-delà du rideau, dans le Sanctuaire 
${}^{20}où Jésus est entré pour nous en précurseur, lui qui est devenu grand prêtre de l’ordre de Melkisédek pour l’éternité.
      <h2 class="intertitle" id="d85e396035">1. Grand prêtre d’un genre différent (7)</h2>
      
         
      \bchapter{}
      \begin{verse}
${}^{1}Ce Melkisédek était roi de Salem, prêtre du Dieu très-haut ; il vint à la rencontre d’Abraham quand celui-ci rentrait de son expédition contre les rois ; il le bénit, 
${}^{2}et Abraham lui remit le dixième de tout ce qu’il avait pris. D’abord, Melkisédek porte un nom qui veut dire « roi de justice » ; ensuite, il est roi de Salem, c’est-à-dire roi « de paix », 
${}^{3}et à son sujet on ne parle ni de père ni de mère, ni d’ancêtres, ni d’un commencement d’existence ni d’une fin de vie ; cela le fait ressembler au Fils de Dieu : il demeure prêtre pour toujours.
${}^{4}Regardez comme il est grand, celui à qui Abraham, le patriarche, a donné la dîme de son meilleur butin. 
${}^{5}Or, selon la loi de Moïse, les fils de Lévi qui reçoivent le sacerdoce ont l’ordre de percevoir la dîme sur le peuple, c’est-à-dire sur leurs frères, qui pourtant sont issus d’Abraham, eux aussi. 
${}^{6}Melkisédek, lui qui n’était pas d’ascendance lévitique, a soumis Abraham à la dîme, et il a béni celui qui possédait les promesses. 
${}^{7}Or il est indiscutable que c’est toujours le supérieur qui bénit l’inférieur. 
${}^{8}D’ordinaire, ceux qui perçoivent la dîme sont des hommes qui meurent, et ici, on atteste que celui-là reste en vie. 
${}^{9}À travers Abraham, Lévi lui-même, qui normalement perçoit la dîme, a été, pour ainsi dire, soumis à la dîme, 
${}^{10}car il était en germe dans le corps de son ancêtre quand Melkisédek vint à la rencontre de celui-ci.
${}^{11}Si l’on atteignait la perfection par le moyen du sacerdoce lévitique, sur lequel repose la législation du peuple, pourquoi faudrait-il que se lève un autre prêtre de l’ordre de Melkisédek, et qu’il ne soit pas appelé prêtre de l’ordre d’Aaron ? 
${}^{12}Or s’il y a changement de sacerdoce, il y a nécessairement aussi changement de loi. 
${}^{13}Celui dont il s’agit ici appartient à une autre tribu, dont aucun membre n’a jamais été au service de l’autel. 
${}^{14}En effet, il est clair que notre Seigneur a surgi de la tribu de Juda, pour laquelle Moïse ne dit rien quand il parle des prêtres. 
${}^{15}Les choses sont encore beaucoup plus claires si cet autre prêtre se lève à la ressemblance de Melkisédek 
${}^{16}et devient prêtre, non pas selon une exigence légale de filiation humaine, mais par la puissance d’une vie indestructible. 
${}^{17}Car voici le témoignage de l’Écriture :
        \\Toi, tu es prêtre de l’ordre de Melkisédek
        \\pour l’éternité.
${}^{18}On a là, d’une part, l’abrogation du commandement précédent, à cause de sa faiblesse et de son inutilité – puisque la Loi n’a rien mené à la perfection – 
${}^{19}et, d’autre part, l’introduction d’une espérance meilleure qui nous fait approcher de Dieu.
${}^{20}Cela ne s’est pas fait sans qu’il y ait eu prestation de serment : en effet, tandis que les autres devenaient prêtres sans aucun serment, 
${}^{21}celui-là a fait l’objet d’un serment de la part de celui qui lui a dit :
        \\Le Seigneur l’a juré
        \\dans un serment irrévocable ;
        \\toi, tu es prêtre pour l’éternité.
${}^{22}Pour cette raison, Jésus est devenu le garant d’une alliance meilleure. 
${}^{23}Jusque-là, un grand nombre de prêtres se sont succédé parce que la mort les empêchait de rester en fonction. 
${}^{24}Jésus, lui, parce qu’il demeure pour l’éternité, possède un sacerdoce qui ne passe pas. 
${}^{25}C’est pourquoi il est capable de sauver d’une manière définitive ceux qui par lui s’avancent vers Dieu, car il est toujours vivant pour intercéder en leur faveur.
${}^{26}C’est bien le grand prêtre qu’il nous fallait : saint, innocent, immaculé ; séparé maintenant des pécheurs, il est désormais plus haut que les cieux. 
${}^{27}Il n’a pas besoin, comme les autres grands prêtres, d’offrir chaque jour des sacrifices, d’abord pour ses péchés personnels, puis pour ceux du peuple ; cela, il l’a fait une fois pour toutes en s’offrant lui-même. 
${}^{28}La loi de Moïse établit comme grands prêtres des hommes remplis de faiblesse ; mais la parole du serment divin, qui vient après la Loi, établit comme grand prêtre le Fils, conduit pour l’éternité à sa perfection.
      <h2 class="intertitle" id="d85e396316">2. Une offrande sacrificielle différente (8 – 9)</h2>
      
         
      \bchapter{}
      \begin{verse}
${}^{1}Et voici l’essentiel de ce que nous voulons dire : c’est bien ce grand prêtre-là que nous avons, lui qui s’est assis à la droite de la Majesté divine dans les cieux, 
${}^{2}après avoir accompli le service du véritable Sanctuaire et de la véritable Tente, celle qui a été dressée par le Seigneur et non par un homme. 
${}^{3}Tout grand prêtre est établi pour offrir des dons et des sacrifices ; il était donc nécessaire que notre grand prêtre ait, lui aussi, quelque chose à offrir. 
${}^{4}À vrai dire, s’il était sur la terre, il ne serait même pas prêtre, puisqu’il y a déjà les prêtres qui offrent les dons conformément à la Loi : 
${}^{5}ceux-ci rendent leur culte dans un sanctuaire qui est une image et une ébauche des réalités célestes, comme en témoigne l’oracle reçu par Moïse au moment où il allait construire la Tente : Regarde, dit le Seigneur, tu exécuteras tout selon le modèle qui t’a été montré sur la montagne. 
${}^{6}Quant au grand prêtre que nous avons, le service qui lui revient se distingue d’autant plus que lui est médiateur d’une alliance meilleure, reposant sur de meilleures promesses.
      
         
${}^{7}En effet, si la première Alliance avait été irréprochable, il n’y aurait pas eu lieu d’en chercher une deuxième. 
${}^{8}Or, c’est bien un reproche que Dieu fait à son peuple quand il dit :
        \\Voici venir des jours, dit le Seigneur,
        \\où je conclurai avec la maison d’Israël
        \\et avec la maison de Juda
        \\une alliance nouvelle.
        ${}^{9}Ce ne sera pas comme l’Alliance
        \\que j’ai faite avec leurs pères,
        \\le jour où je les ai pris par la main
        \\pour les faire sortir du pays d’Égypte :
        \\eux ne sont pas restés dans mon alliance ;
        \\alors moi, je les ai délaissés,
        \\dit le Seigneur.
        ${}^{10}Mais voici quelle sera l’Alliance
        \\que j’établirai avec la maison d’Israël
        \\quand ces jours-là seront passés,
        \\dit le Seigneur.
        \\Quand je leur donnerai mes lois,
        \\je les inscrirai dans leur pensée et sur leurs cœurs.
        \\Je serai leur Dieu,
        \\et ils seront mon peuple.
        ${}^{11}Ils n’auront plus à instruire
        \\chacun son concitoyen ni chacun son frère
        \\en disant : « Apprends à connaître le Seigneur ! »
        \\Car tous me connaîtront,
        \\des plus petits jusqu’aux plus grands.
        ${}^{12}Je serai indulgent pour leurs fautes,
        \\je ne me rappellerai plus leurs péchés.
${}^{13}En parlant d’Alliance nouvelle, Dieu a rendu ancienne la première ; or ce qui devient ancien et qui vieillit est près de disparaître.
      
         
      \bchapter{}
      \begin{verse}
${}^{1}La première Alliance avait donc ses préceptes pour le culte ainsi que son Lieu saint dans ce monde. 
${}^{2}Une tente y était disposée, la première, où se trouvaient le chandelier à sept branches et la table avec les pains de l’offrande ; c’est ce qu’on nomme le Saint. 
${}^{3}Derrière le second rideau, il y avait la tente appelée le Saint des Saints, 
${}^{4}contenant un brûle-parfum en or et l’arche d’Alliance entièrement recouverte d’or, dans laquelle se trouvaient un vase d’or contenant la manne, le bâton d’Aaron qui avait fleuri, et les tables de l’Alliance ; 
${}^{5}au-dessus de l’arche, les kéroubim de gloire couvraient de leur ombre la plaque d’or appelée propitiatoire. Mais il n’y a pas lieu maintenant d’entrer dans les détails. 
${}^{6}Les choses étant ainsi disposées, les prêtres entrent continuellement dans la première tente quand ils célèbrent le culte. 
${}^{7}Mais dans la deuxième tente, une fois par an, le grand prêtre entre seul, et il ne le fait pas sans offrir du sang pour lui-même et pour les fautes que le peuple a commises par ignorance. 
${}^{8}L’Esprit Saint montre ainsi que le chemin du sanctuaire n’a pas encore été manifesté tant que la première tente reste debout. 
${}^{9}C’est là une préfiguration pour le temps présent : les dons et les sacrifices qui sont offerts ne sont pas capables de mener à la perfection dans sa conscience celui qui célèbre le culte ; 
${}^{10}ces préceptes, liés à des observances pour les aliments, boissons et ablutions diverses, concernent seulement la chair et ne sont valables que jusqu’au temps du relèvement !
      
         
${}^{11}Le Christ est venu, grand prêtre des biens à venir. Par la tente plus grande et plus parfaite, celle qui n’est pas œuvre de mains humaines et n’appartient pas à cette création, 
${}^{12}il est entré une fois pour toutes dans le sanctuaire, en répandant, non pas le sang de boucs et de jeunes taureaux, mais son propre sang. De cette manière, il a obtenu une libération définitive. 
${}^{13}S’il est vrai qu’une simple aspersion avec le sang de boucs et de taureaux, et de la cendre de génisse, sanctifie ceux qui sont souillés, leur rendant la pureté de la chair, 
${}^{14}le sang du Christ fait bien davantage, car le Christ, poussé par l’Esprit éternel, s’est offert lui-même à Dieu comme une victime sans défaut ; son sang purifiera donc notre conscience des actes qui mènent à la mort, pour que nous puissions rendre un culte au Dieu vivant.
${}^{15}Voilà pourquoi il est le médiateur d’une alliance nouvelle, d’un testament nouveau : puisque sa mort a permis le rachat des transgressions commises sous le premier Testament, ceux qui sont appelés peuvent recevoir l’héritage éternel jadis promis. 
${}^{16}Or, quand il y a testament, il est nécessaire que soit constatée la mort de son auteur. 
${}^{17}Car un testament ne vaut qu’après la mort, il est sans effet tant que son auteur est en vie. 
${}^{18}C’est pourquoi le premier Testament lui-même n’a pas été inauguré sans que soit utilisé du sang : 
${}^{19}lorsque Moïse eut proclamé chaque commandement à tout le peuple conformément à la Loi, il prit le sang des veaux et des boucs avec de l’eau, de la laine écarlate et de l’hysope, et il en aspergea le livre lui-même et tout le peuple, 
${}^{20}en disant :
        \\Ceci est le sang de l’Alliance
        \\que Dieu a prescrite pour vous.
${}^{21}Puis il aspergea de même avec le sang la tente et tous les objets du service liturgique. 
${}^{22}D’après la Loi, on purifie presque tout avec du sang, et s’il n’y a pas de sang versé, il n’y a pas de pardon.
${}^{23}S’il est nécessaire que soient purifiées par ces rites les images de ce qui est dans les cieux, les réalités célestes elles-mêmes doivent l’être par des sacrifices bien meilleurs que ceux d’ici-bas.
${}^{24}Car le Christ n’est pas entré dans un sanctuaire fait de main d’homme, figure du sanctuaire véritable ; il est entré dans le ciel même, afin de se tenir maintenant pour nous devant la face de Dieu. 
${}^{25}Il n’a pas à s’offrir lui-même plusieurs fois, comme le grand prêtre qui, tous les ans, entrait dans le sanctuaire en offrant un sang qui n’était pas le sien ; 
${}^{26}car alors, le Christ aurait dû plusieurs fois souffrir la Passion depuis la fondation du monde. Mais en fait, c’est une fois pour toutes, à la fin des temps, qu’il s’est manifesté pour détruire le péché par son sacrifice. 
${}^{27}Et, comme le sort des hommes est de mourir une seule fois et puis d’être jugés, 
${}^{28}ainsi le Christ s’est-il offert une seule fois pour enlever les péchés de la multitude ; il apparaîtra une seconde fois, non plus à cause du péché, mais pour le salut de ceux qui l’attendent.
      <h2 class="intertitle" id="d85e396874">3. Une offrande parfaitement efficace (10,1-18)</h2>
      
         
      \bchapter{}
      \begin{verse}
${}^{1}La loi de Moïse ne présente que l’ébauche des biens à venir, et non pas l’expression même des réalités. Elle n’est donc jamais capable, par ses sacrifices qui sont toujours les mêmes, offerts indéfiniment chaque année, de mener à la perfection ceux qui viennent y prendre part. 
${}^{2}Si ce culte les avait purifiés une fois pour toutes, ils n’auraient plus aucun péché sur la conscience et, dans ce cas, n’aurait-on pas cessé d’offrir les sacrifices ? 
${}^{3}Mais ceux-ci, au contraire, comportent chaque année un rappel des péchés. 
${}^{4}Il est impossible, en effet, que du sang de taureaux et de boucs enlève les péchés. 
${}^{5}Aussi, en entrant dans le monde, le Christ dit :
        \\Tu n’as voulu ni sacrifice ni offrande,
        \\mais tu m’as formé un corps.
        ${}^{6}Tu n’as pas agréé les holocaustes
        \\ni les sacrifices pour le péché ;
        ${}^{7}alors, j’ai dit : Me voici,
        \\je suis venu, mon Dieu, pour faire ta volonté,
        \\ainsi qu’il est écrit de moi dans le Livre.
${}^{8}Le Christ commence donc par dire : Tu n’as pas voulu ni agréé les sacrifices et les offrandes, les holocaustes et les sacrifices pour le péché, ceux que la Loi prescrit d’offrir. 
${}^{9}Puis il déclare : Me voici, je suis venu pour faire ta volonté. Ainsi, il supprime le premier état de choses pour établir le second. 
${}^{10}Et c’est grâce à cette volonté que nous sommes sanctifiés, par l’offrande que Jésus Christ a faite de son corps, une fois pour toutes.
${}^{11}Tout prêtre, chaque jour, se tenait debout dans le Lieu saint pour le service liturgique, et il offrait à maintes reprises les mêmes sacrifices, qui ne peuvent jamais enlever les péchés. 
${}^{12}Jésus Christ, au contraire, après avoir offert pour les péchés un unique sacrifice, s’est assis pour toujours à la droite de Dieu. 
${}^{13}Il attend désormais que ses ennemis soient mis sous ses pieds. 
${}^{14}Par son unique offrande, il a mené pour toujours à leur perfection ceux qu’il sanctifie.
${}^{15}L’Esprit Saint, lui aussi, nous l’atteste dans l’Écriture, car, après avoir dit :
        ${}^{16}Voici quelle sera l’Alliance
        \\que j’établirai avec eux
        \\quand ces jours-là seront passés,
      <p class="retrait0">le Seigneur dit :
        \\Quand je leur donnerai mes lois,
        \\je les inscrirai sur leurs cœurs et dans leur pensée
        ${}^{17}et je ne me rappellerai plus leurs péchés ni leurs fautes.
${}^{18}Or, quand le pardon est accordé, on n’offre plus le sacrifice pour le péché.
      <h2 class="intertitle" id="d85e397153">4. Exhortation finale (10,19-39)</h2>
${}^{19}Frères, c’est avec assurance que nous pouvons entrer dans le véritable sanctuaire grâce au sang de Jésus : 
${}^{20}nous avons là un chemin nouveau et vivant qu’il a inauguré en franchissant le rideau du Sanctuaire ; or, ce rideau est sa chair. 
${}^{21}Et nous avons le prêtre par excellence, celui qui est établi sur la maison de Dieu. 
${}^{22}Avançons-nous donc vers Dieu avec un cœur sincère et dans la plénitude de la foi, le cœur purifié de ce qui souille notre conscience, le corps lavé par une eau pure. 
${}^{23}Continuons sans fléchir d’affirmer notre espérance, car il est fidèle, celui qui a promis. 
${}^{24}Soyons attentifs les uns aux autres pour nous stimuler à vivre dans l’amour et à bien agir. 
${}^{25}Ne délaissons pas nos assemblées, comme certains en ont pris l’habitude, mais encourageons-nous, d’autant plus que vous voyez s’approcher le jour du Seigneur.
${}^{26}Car si nous demeurons volontairement dans le péché après avoir reçu la pleine connaissance de la vérité, il ne reste plus pour les péchés aucun sacrifice, 
${}^{27}mais une attente redoutable du jugement et l’ardeur d’un feu qui va dévorer les rebelles. 
${}^{28}Si quelqu’un enfreint la loi de Moïse, c’est sans pitié qu’il est mis à mort sur la parole de deux ou trois témoins. 
${}^{29}Qu’en pensez-vous ? Ne sera-t-elle pas encore plus grave, la peine que méritera celui qui aura foulé aux pieds le Fils de Dieu, tenu pour profane le sang de l’Alliance par lequel il a été sanctifié, et outragé l’Esprit qui donne la grâce ? 
${}^{30}Car nous connaissons celui qui a dit :
        \\C’est à moi de faire justice,
        \\c’est moi qui rendrai à chacun ce qui lui revient ;
      <p class="retrait0">et encore :
        \\Le Seigneur jugera son peuple.
${}^{31}Il est redoutable de tomber entre les mains du Dieu vivant !
${}^{32}Souvenez-vous de ces premiers jours où vous veniez de recevoir la lumière du Christ : vous avez soutenu alors le dur combat des souffrances, 
${}^{33}tantôt donnés en spectacle sous les insultes et les brimades, tantôt solidaires de ceux qu’on traitait ainsi. 
${}^{34}En effet, vous avez montré de la compassion à ceux qui étaient en prison ; vous avez accepté avec joie qu’on vous arrache vos biens, car vous étiez sûrs de posséder un bien encore meilleur, et permanent. 
${}^{35}Ne perdez pas votre assurance ; grâce à elle, vous serez largement récompensés. 
${}^{36}Car l’endurance vous est nécessaire pour accomplir la volonté de Dieu et obtenir ainsi la réalisation des promesses.
        ${}^{37}En effet, encore un peu, très peu de temps,
        \\et celui qui doit venir arrivera,
        \\il ne tardera pas.
        ${}^{38}Celui qui est juste à mes yeux par la foi vivra ;
        \\mais s’il abandonne,
        \\je ne trouve plus mon bonheur en lui.
${}^{39}Or nous ne sommes pas, nous, de ceux qui abandonnent et vont à leur perte, mais de ceux qui ont la foi et sauvegardent leur âme.
      
         
      \bchapter{}
      \begin{verse}
${}^{1}La foi est une façon de posséder ce que l’on espère, un moyen de connaître des réalités qu’on ne voit pas. 
${}^{2}Et quand l’Écriture rend témoignage aux anciens, c’est à cause de leur foi. 
${}^{3}Grâce à la foi, nous comprenons que les mondes ont été formés par une parole de Dieu, et donc ce qui est visible n’a pas son origine dans ce qui apparaît au regard. 
${}^{4}Grâce à la foi, Abel offrit à Dieu un sacrifice plus grand que celui de Caïn ; à cause de sa foi, il fut déclaré juste : Dieu lui-même rendait témoignage à son offrande ; à cause de sa foi, bien qu’il soit mort, il parle encore. 
${}^{5}Grâce à la foi, Hénok fut retiré de ce monde, et il ne connut pas la mort ; personne ne le retrouva parce que Dieu l’avait retiré ; avant cet événement, il avait été agréable à Dieu, l’Écriture en témoigne. 
${}^{6}Or, sans la foi, il est impossible d’être agréable à Dieu ; car, pour s’avancer vers lui, il faut croire qu’il existe et qu’il récompense ceux qui le cherchent. 
${}^{7}Grâce à la foi, Noé, averti de choses encore invisibles, accueillit cet oracle avec respect et construisit une arche pour le salut de sa famille. Sa foi condamnait le monde, et il reçut en héritage la justice qui s’obtient par la foi.
${}^{8}Grâce à la foi, Abraham obéit à l’appel de Dieu : il partit vers un pays qu’il devait recevoir en héritage, et il partit sans savoir où il allait. 
${}^{9}Grâce à la foi, il vint séjourner en immigré dans la Terre promise, comme en terre étrangère ; il vivait sous la tente, ainsi qu’Isaac et Jacob, héritiers de la même promesse, 
${}^{10}car il attendait la ville qui aurait de vraies fondations, la ville dont Dieu lui-même est le bâtisseur et l’architecte. 
${}^{11}Grâce à la foi, Sara, elle aussi, malgré son âge, fut rendue capable d’être à l’origine d’une descendance parce qu’elle pensait que Dieu est fidèle à ses promesses. 
${}^{12}C’est pourquoi, d’un seul homme, déjà marqué par la mort, a pu naître une descendance aussi nombreuse que les étoiles du ciel et que le sable au bord de la mer, une multitude innombrable.
${}^{13}C’est dans la foi, sans avoir connu la réalisation des promesses, qu’ils sont tous morts ; mais ils l’avaient vue et saluée de loin, affirmant que, sur la terre, ils étaient des étrangers et des voyageurs. 
${}^{14}Or, parler ainsi, c’est montrer clairement qu’on est à la recherche d’une patrie. 
${}^{15}S’ils avaient songé à celle qu’ils avaient quittée, ils auraient eu la possibilité d’y revenir. 
${}^{16}En fait, ils aspiraient à une patrie meilleure, celle des cieux. Aussi Dieu n’a pas honte d’être appelé leur Dieu, puisqu’il leur a préparé une ville.
${}^{17}Grâce à la foi, quand il fut soumis à l’épreuve, Abraham offrit Isaac en sacrifice. Et il offrait le fils unique, alors qu’il avait reçu les promesses 
${}^{18}et entendu cette parole :
        \\C’est par Isaac qu’une descendance portera ton nom.
${}^{19}Il pensait en effet que Dieu est capable même de ressusciter les morts ; c’est pourquoi son fils lui fut rendu : il y a là une préfiguration. 
${}^{20}Grâce à la foi encore, Isaac bénit Jacob et Ésaü en vue de l’avenir. 
${}^{21}Grâce à la foi, Jacob mourant bénit l’un et l’autre des fils de Joseph, et il se prosterna, appuyé sur l’extrémité de son bâton. 
${}^{22}Grâce à la foi, Joseph, à la fin de sa vie, évoqua l’exode des fils d’Israël et donna des ordres au sujet de ses ossements.
${}^{23}Grâce à la foi, Moïse, après sa naissance, fut caché pendant trois mois par ses parents, car ils avaient vu que l’enfant était beau, et ils n’eurent pas peur du décret du roi. 
${}^{24}Grâce à la foi, Moïse, devenu grand, renonça au titre de fils de la fille du Pharaon. 
${}^{25}Choisissant d’être maltraité avec le peuple de Dieu plutôt que de connaître une éphémère jouissance du péché, 
${}^{26}il considéra l’injure subie par le Christ comme une richesse plus grande que les trésors de l’Égypte : en effet, il regardait plus loin, vers la récompense. 
${}^{27}Grâce à la foi, il quitta l’Égypte sans craindre la colère du roi ; il tint ferme, comme s’il voyait Celui qui est invisible. 
${}^{28}Grâce à la foi, il a fait célébrer la Pâque et appliquer du sang sur les portes, pour que l’Exterminateur des premiers-nés ne touche pas ceux d’Israël. 
${}^{29}Grâce à la foi, ils passèrent à travers la mer Rouge comme sur une terre sèche, alors que les Égyptiens, essayant d’en faire autant, furent engloutis.
${}^{30}Grâce à la foi, les remparts de Jéricho tombèrent après qu’on en eut fait le tour pendant sept jours. 
${}^{31}Grâce à la foi, Rahab la prostituée ne périt pas avec ceux qui avaient résisté, car elle avait accueilli pacifiquement les hommes envoyés en reconnaissance.
${}^{32}Que dire encore ? Le temps me manquerait pour rappeler l’histoire de Gédéon, Baraq, Samson, Jephté, David, Samuel et les prophètes. 
${}^{33}Par leur foi, ils ont conquis des royaumes, pratiqué la justice, obtenu la réalisation de certaines promesses. Ils ont fermé la gueule des lions, 
${}^{34}éteint la flamme des brasiers, échappé au tranchant de l’épée, retrouvé leurs forces après la maladie, montré du courage à la guerre, mis en fuite des armées étrangères. 
${}^{35}Des femmes dont les enfants étaient morts les ont retrouvés ressuscités. Mais certains autres ont été torturés et n’ont pas accepté la libération qui leur était proposée, car ils voulaient obtenir une meilleure résurrection. 
${}^{36}D’autres ont subi l’épreuve des moqueries et des coups de fouet, des chaînes et de la prison. 
${}^{37}Ils furent lapidés, sciés en deux, massacrés à coups d’épée. Ils allèrent çà et là, vêtus de peaux de moutons ou de toisons de chèvres, manquant de tout, harcelés et maltraités 
${}^{38}– mais en fait, c’est le monde qui n’était pas digne d’eux ! Ils menaient une vie errante dans les déserts et les montagnes, dans les grottes et les cavernes de la terre.
${}^{39}Et, bien que, par leur foi, ils aient tous reçu le témoignage de Dieu, ils n’ont pas obtenu la réalisation de la promesse. 
${}^{40}En effet, pour nous Dieu avait prévu mieux encore, et il ne voulait pas les mener sans nous à la perfection.
      
         
      \bchapter{}
      \begin{verse}
${}^{1}Ainsi donc, nous aussi, entourés de cette immense nuée de témoins, et débarrassés de tout ce qui nous alourdit – en particulier du péché qui nous entrave si bien –, courons avec endurance l’épreuve qui nous est proposée, 
${}^{2}les yeux fixés sur Jésus, qui est à l’origine et au terme de la foi. Renonçant à la joie qui lui était proposée, il a enduré la croix en méprisant la honte de ce supplice, et il siège à la droite du trône de Dieu. 
${}^{3}Méditez l’exemple de celui qui a enduré de la part des pécheurs une telle hostilité, et vous ne serez pas accablés par le découragement. 
${}^{4}Vous n’avez pas encore résisté jusqu’au sang dans votre lutte contre le péché, 
${}^{5}et vous avez oublié cette parole de réconfort, qui vous est adressée comme à des fils :
        \\Mon fils, ne néglige pas les leçons du Seigneur,
        \\ne te décourage pas quand il te fait des reproches.
        ${}^{6}Quand le Seigneur aime quelqu’un,
        \\il lui donne de bonnes leçons ;
        \\il corrige tous ceux qu’il accueille comme ses fils.
${}^{7}Ce que vous endurez est une leçon. Dieu se comporte envers vous comme envers des fils ; et quel est le fils auquel son père ne donne pas des leçons ? 
${}^{8}Si vous êtes privés des leçons que tous les autres reçoivent, c’est que vous êtes des bâtards et non des fils. 
${}^{9}D’ailleurs, nos parents selon la chair nous faisaient la leçon, et nous les respections. Ne devons-nous pas d’autant plus nous soumettre au Père de nos esprits pour avoir la vie ? 
${}^{10}Les leçons que nos parents nous donnaient en croyant bien faire n’avaient qu’un effet passager. Mais celles de Dieu sont vraiment pour notre bien : il veut nous faire partager sa sainteté. 
${}^{11}Quand on vient de recevoir une leçon, on n’éprouve pas de la joie mais plutôt de la tristesse. Mais plus tard, quand on s’est repris grâce à la leçon, celle-ci produit un fruit de paix et de justice.
${}^{12}C’est pourquoi, redressez les mains inertes et les genoux qui fléchissent, 
${}^{13}et rendez droits pour vos pieds les sentiers tortueux. Ainsi, celui qui boite ne se fera pas d’entorse ; bien plus, il sera guéri.
${}^{14}Recherchez activement la paix avec tous, et la sainteté sans laquelle personne ne verra le Seigneur. 
${}^{15}Soyez vigilants : que personne ne se dérobe à la grâce de Dieu, qu’il ne pousse chez vous aucune plante aux fruits amers, cela causerait du trouble, et beaucoup en seraient infectés ; 
${}^{16}qu’il n’y ait pas de débauché ni de profanateur, comme Ésaü qui vendit son droit d’aînesse en échange d’un seul plat. 
${}^{17}Vous savez que par la suite, quand il voulut recevoir en héritage la bénédiction, il fut rejeté, car il ne trouva aucune possibilité de changement à son égard, malgré ses réclamations et ses larmes.
${}^{18}Vous n’êtes pas venus vers une réalité palpable, embrasée par le feu, comme la montagne du Sinaï : pas d’obscurité, de ténèbres ni d’ouragan, 
${}^{19}pas de son de trompettes ni de paroles prononcées par cette voix que les fils d’Israël demandèrent à ne plus entendre. 
${}^{20}Car ils ne supportaient pas cette interdiction : Qui touchera la montagne, même si c’est un animal, sera lapidé. 
${}^{21}Le spectacle était si effrayant que Moïse dit : Je suis effrayé et tremblant. 
${}^{22}Mais vous êtes venus vers la montagne de Sion et vers la ville du Dieu vivant, la Jérusalem céleste, vers des myriades d’anges en fête 
${}^{23}et vers l’assemblée des premiers-nés dont les noms sont inscrits dans les cieux. Vous êtes venus vers Dieu, le juge de tous, et vers les esprits des justes amenés à la perfection. 
${}^{24}Vous êtes venus vers Jésus, le médiateur d’une alliance nouvelle, et vers le sang de l’aspersion, son sang qui parle plus fort que celui d’Abel.
${}^{25}Prenez garde, ne refusez pas d’entendre celui qui vous parle ; car si les fils d’Israël n’ont pas échappé au châtiment quand ils ont refusé d’entendre celui qui les avertissait par un oracle sur la terre, à plus forte raison nous n’y échapperons pas non plus, si nous nous détournons de celui qui nous parle depuis les cieux. 
${}^{26}Sa voix a jadis ébranlé la terre. Maintenant il fait cette annonce solennelle : une seule fois encore, moi, je ferai trembler, non seulement la terre, mais aussi le ciel. 
${}^{27}Ces mots une seule fois encore montrent clairement qu’il y aura une transformation de ce qui sera ébranlé parce que ce sont des choses créées, afin que subsiste ce qui ne sera pas ébranlé. 
${}^{28}C’est pourquoi, nous qui recevons une royauté inébranlable, soyons reconnaissants et rendons ainsi notre culte à Dieu d’une manière qui lui est agréable, avec grand respect et crainte. 
${}^{29}Car notre Dieu est un feu dévorant.
      
         
      \bchapter{}
      \begin{verse}
${}^{1}Que demeure l’amour fraternel ! 
${}^{2}N’oubliez pas l’hospitalité : elle a permis à certains, sans le savoir, de recevoir chez eux des anges. 
${}^{3}Souvenez-vous de ceux qui sont en prison, comme si vous étiez prisonniers avec eux. Souvenez-vous de ceux qui sont maltraités, car vous aussi, vous avez un corps.
${}^{4}Que le mariage soit honoré de tous, que l’union conjugale ne soit pas profanée, car les débauchés et les adultères seront jugés par Dieu. 
${}^{5}Que votre conduite ne soit pas inspirée par l’amour de l’argent : contentez-vous de ce que vous avez, car Dieu lui-même a dit :
        \\Jamais je ne te lâcherai,
        \\jamais je ne t’abandonnerai.
      \begin{verse}
${}^{6}C’est pourquoi nous pouvons dire en toute assurance :
        \\Le Seigneur est mon secours,
        \\je n’ai rien à craindre !
        \\Que pourrait me faire un homme ?
${}^{7}Souvenez-vous de ceux qui vous ont dirigés : ils vous ont annoncé la parole de Dieu. Méditez sur l’aboutissement de la vie qu’ils ont menée, et imitez leur foi. 
${}^{8}Jésus Christ, hier et aujourd’hui, est le même, il l’est pour l’éternité. 
${}^{9}Ne vous laissez pas égarer par toutes sortes de doctrines étrangères. Il est bon de fortifier nos cœurs par la grâce, et non par des observances alimentaires qui n’ont jamais profité à leurs adeptes. 
${}^{10}Nous avons un autel auquel n’ont pas le droit de se nourrir ceux qui rendent un culte selon l’ancienne Alliance. 
${}^{11}En effet, quand le grand prêtre portait dans le sanctuaire le sang des animaux en sacrifice pour le péché, c’est en dehors de l’enceinte que leurs corps étaient brûlés. 
${}^{12}C’est pourquoi Jésus, lui aussi, voulant sanctifier le peuple par son propre sang, a souffert sa Passion à l’extérieur des portes de la ville. 
${}^{13}Eh bien ! pour aller à sa rencontre, sortons en dehors de l’enceinte, en supportant l’injure qu’il a subie. 
${}^{14}Car la ville que nous avons ici-bas n’est pas définitive : nous recherchons la ville qui doit venir.
${}^{15}En toute circonstance, offrons à Dieu, par Jésus, un sacrifice de louange, c’est-à-dire les paroles de nos lèvres qui proclament son nom. 
${}^{16}N’oubliez pas d’être généreux et de partager. C’est par de tels sacrifices que l’on plaît à Dieu. 
${}^{17}Faites confiance à ceux qui vous dirigent et soyez-leur soumis ; en effet, ils sont là pour veiller sur vos âmes, ce dont ils auront à rendre compte. Ainsi, ils accompliront leur tâche avec joie, sans avoir à se plaindre, ce qui ne vous serait d’aucun profit.
${}^{18}Priez pour nous ; en effet, nous sommes convaincus d’avoir une conscience pure, puisque nous voulons en toute circonstance avoir une bonne conduite. 
${}^{19}Je vous demande instamment de le faire, pour que je vous sois rendu plus vite.
        ${}^{20}Que le Dieu de la paix,
        \\lui qui a fait remonter d’entre les morts,
        \\grâce au sang de l’Alliance éternelle,
        \\le berger des brebis, le Pasteur par excellence,
        \\notre Seigneur Jésus,
        ${}^{21}que ce Dieu vous forme en tout ce qui est bon
        \\pour accomplir sa volonté,
        \\qu’il réalise en nous ce qui est agréable à ses yeux,
        \\par Jésus Christ, à qui appartient la gloire
        \\pour les siècles des siècles. Amen.
${}^{22}Je vous invite, frères, à supporter cette parole d’exhortation. D’ailleurs, je ne vous envoie que quelques mots. 
${}^{23}Sachez que notre frère Timothée est libéré. J’irai vous voir avec lui s’il vient assez vite.
${}^{24}Saluez tous ceux qui vous dirigent et tous les fidèles. Ceux d’Italie vous saluent.
${}^{25}La grâce soit avec vous tous.
