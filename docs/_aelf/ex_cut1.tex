  
  
    
    \bbook{EXODE}{EXODE}
      
         
      \bchapter{}
      \begin{verse}
${}^{1}Voici les noms des fils d’Israël venus en Égypte avec Jacob, leur père. Chacun y vint avec sa famille. 
${}^{2}C’étaient Roubène, Siméon, Lévi et Juda, 
${}^{3}Issakar, Zabulon et Benjamin, 
${}^{4}Dane et Nephtali, Gad et Asher. 
${}^{5}Toutes les personnes issues de Jacob étaient au nombre de soixante-dix. Joseph, lui, était déjà en Égypte. 
${}^{6}Puis Joseph mourut, ainsi que tous ses frères et toute cette génération-là. 
${}^{7}Les fils d’Israël furent féconds, ils devinrent très nombreux, ils se multiplièrent et devinrent de plus en plus forts : tout le pays en était rempli.
      
         
${}^{8}Un nouveau roi vint au pouvoir\\en Égypte. Il n’avait pas connu Joseph. 
${}^{9} Il dit à son peuple : « Voici que le peuple des fils d’Israël est maintenant plus nombreux et plus puissant que nous. 
${}^{10} Prenons donc les dispositions voulues pour l’empêcher de se multiplier. Car, s’il y avait une guerre, il se joindrait à nos ennemis, combattrait contre nous, et ensuite il sortirait\\du pays. » 
${}^{11} On imposa donc aux fils d’Israël\\des chefs de corvée pour les accabler de travaux pénibles. Ils durent bâtir pour Pharaon les villes d’entrepôts\\de Pithome et de Ramsès. 
${}^{12} Mais, plus on les accablait, plus ils se multipliaient et proliféraient, ce qui les fit détester. 
${}^{13} Les Égyptiens soumirent les fils d’Israël à un dur esclavage 
${}^{14} et leur rendirent la vie intenable à force de corvées : préparation de l’argile et des briques et toutes sortes de travaux à la campagne ; tous ces travaux étaient pour eux un dur esclavage.
${}^{15}Alors le roi d’Égypte parla aux sages-femmes des Hébreux dont l’une s’appelait Shifra et l’autre Poua ; 
${}^{16}il leur dit : « Quand vous accoucherez les femmes des Hébreux, regardez bien le sexe de l’enfant : si c’est un garçon, faites-le mourir ; si c’est une fille, laissez-la vivre. » 
${}^{17}Mais les sages-femmes craignirent Dieu et n’obéirent pas à l’ordre du roi : elles laissèrent vivre les garçons. 
${}^{18}Alors le roi d’Égypte les appela et leur dit : « Pourquoi avez-vous agi de la sorte, pourquoi avez-vous laissé vivre les garçons ? » 
${}^{19}Les sages-femmes répondirent à Pharaon : « Les femmes des Hébreux ne sont pas comme les Égyptiennes, elles sont pleines de vitalité ; avant l’arrivée de la sage-femme, elles ont déjà accouché. » 
${}^{20}Dieu accorda ses bienfaits aux sages-femmes ; le peuple devint très nombreux et très fort. 
${}^{21}Comme les sages-femmes avaient craint Dieu, il leur avait accordé une descendance.
${}^{22}Pharaon donna cet ordre à tout son peuple : « Tous les fils qui naîtront aux Hébreux\\, jetez-les dans le Nil. Ne laissez vivre que les filles. »
      
         
      \bchapter{}
      \begin{verse}
${}^{1}Un homme de la tribu de Lévi avait épousé une femme de la même tribu. 
${}^{2} Elle devint enceinte, et elle enfanta un fils. Voyant qu’il était beau, elle le cacha durant trois mois. 
${}^{3} Lorsqu’il lui fut impossible de le tenir caché plus longtemps, elle prit une corbeille de jonc, qu’elle enduisit de bitume et de goudron\\. Elle y plaça l’enfant, et déposa la corbeille au bord du Nil, au milieu des roseaux. 
${}^{4} La sœur de l’enfant\\se tenait à distance pour voir ce qui allait arriver.
${}^{5}La fille de Pharaon descendit au fleuve pour s’y baigner, tandis que ses suivantes se promenaient sur la rive. Elle aperçut la corbeille parmi les roseaux et envoya sa servante pour la prendre. 
${}^{6} Elle l’ouvrit et elle vit l’enfant. C’était un petit garçon, il pleurait. Elle en eut pitié et dit : « C’est un enfant des Hébreux. » 
${}^{7} La sœur de l’enfant\\dit alors à la fille de Pharaon : « Veux-tu que j’aille te chercher, parmi les femmes des Hébreux, une nourrice qui, pour toi, nourrira l’enfant ? » 
${}^{8} La fille de Pharaon lui répondit : « Va. » La jeune fille alla donc chercher la mère de l’enfant. 
${}^{9} La fille de Pharaon dit à celle-ci : « Emmène cet enfant et nourris-le pour moi. C’est moi qui te donnerai ton salaire. » Alors la femme emporta l’enfant et le nourrit. 
${}^{10} Lorsque l’enfant eut grandi, elle le ramena à la fille de Pharaon qui le traita comme son propre fils ; elle lui donna le nom de Moïse, en disant : « Je l’ai tiré des eaux. »
${}^{11}Or vint le jour où Moïse, qui avait grandi, se rendit\\auprès de ses frères et les vit accablés de corvées. Il vit un Égyptien qui frappait un Hébreu, l’un de ses frères. 
${}^{12} Regardant autour de lui et ne voyant personne, il frappa à mort l’Égyptien et l’enfouit dans le sable. 
${}^{13} Le lendemain, il sortit de nouveau : voici que deux\\Hébreux se battaient. Il dit à l’agresseur : « Pourquoi frappes-tu ton compagnon ? » 
${}^{14} L’homme\\lui répliqua : « Qui t’a institué\\chef et juge sur nous ? Veux-tu me tuer comme tu as tué l’Égyptien ? » Moïse eut peur et se dit : « Pas de doute, la chose est connue. » 
${}^{15} Pharaon en fut informé et chercha à faire tuer Moïse. Celui-ci s’enfuit loin de Pharaon et habita au pays de Madiane.
      Il vint s’asseoir près du puits. 
${}^{16}Le prêtre de Madiane avait sept filles. Elles allèrent puiser de l’eau et remplir les auges pour abreuver le troupeau de leur père. 
${}^{17}Des bergers survinrent et voulurent les chasser. Alors Moïse se leva pour leur porter secours et il abreuva leur troupeau. 
${}^{18}Elles retournèrent chez Réouël, leur père, qui leur dit : « Pourquoi êtes-vous revenues si tôt, aujourd’hui ? » 
${}^{19}Elles répondirent : « Un Égyptien nous a délivrées de la main des bergers, il a même puisé l’eau pour nous et abreuvé le troupeau ! » 
${}^{20}Réouël demanda : « Mais où est-il ? Pourquoi l’avez-vous laissé là-bas ? Appelez-le ! Invitez-le à manger ! » 
${}^{21}Et Moïse accepta de s’établir chez cet homme qui lui donna comme épouse sa fille Cippora. 
${}^{22}Elle enfanta un fils à qui Moïse donna le nom de Guershom (ce qui signifie : Immigré en ce lieu) car, dit-il, « Je suis devenu un immigré en terre étrangère ».
${}^{23}Au cours de cette longue période, le roi d’Égypte mourut. Du fond de leur esclavage, les fils d’Israël gémirent et crièrent. Du fond de leur esclavage, leur appel monta vers Dieu. 
${}^{24}Dieu entendit leur plainte ; Dieu se souvint de son alliance avec Abraham, Isaac et Jacob. 
${}^{25}Dieu regarda les fils d’Israël, et Dieu les reconnut.
      
         
      \bchapter{}
      \begin{verse}
${}^{1}Moïse était berger du troupeau de son beau-père Jéthro\\, prêtre de Madiane. Il mena le troupeau au-delà du désert et parvint à la montagne de Dieu, à l’Horeb. 
${}^{2} L’ange du Seigneur lui apparut dans la flamme d’un buisson en feu\\. Moïse regarda\\ : le buisson brûlait sans se consumer. 
${}^{3} Moïse se dit alors : « Je vais faire un détour pour voir cette chose extraordinaire : pourquoi le buisson ne se consume-t-il pas\\ ? » 
${}^{4} Le Seigneur vit qu’il avait fait un détour pour voir, et Dieu l’appela du milieu du buisson : « Moïse ! Moïse ! » Il dit : « Me voici ! » 
${}^{5} Dieu dit alors\\ : « N’approche pas d’ici ! Retire les sandales de tes pieds, car le lieu où tu te tiens est une terre sainte ! » 
${}^{6} Et il déclara : « Je suis le Dieu de ton père, le Dieu d’Abraham, le Dieu d’Isaac, le Dieu de Jacob. » Moïse se voila le visage car il craignait de porter son regard sur Dieu.
${}^{7}Le Seigneur dit : « J’ai vu, oui, j’ai vu la misère de mon peuple qui est en Égypte, et j’ai entendu ses cris sous les coups des surveillants. Oui, je connais ses souffrances. 
${}^{8}Je suis descendu pour le délivrer de la main des Égyptiens et le faire monter de ce pays vers un beau et vaste pays, vers un pays ruisselant de lait et de miel, vers le lieu où vivent le Cananéen, le Hittite, l’Amorite, le Perizzite, le Hivvite et le Jébuséen. 
${}^{9}Maintenant, le cri des fils d’Israël est parvenu jusqu’à moi, et j’ai vu l’oppression que leur font subir les Égyptiens\\. 
${}^{10}Maintenant donc, va ! Je t’envoie chez Pharaon : tu feras sortir d’Égypte mon peuple, les fils d’Israël. »
${}^{11}Moïse dit à Dieu : « Qui suis-je pour aller trouver Pharaon, et pour faire sortir d’Égypte les fils d’Israël ? » 
${}^{12} Dieu lui répondit\\ : « Je suis avec toi. Et tel est le signe que c’est moi qui t’ai envoyé : quand tu auras fait sortir d’Égypte mon peuple, vous rendrez un culte\\à Dieu sur cette montagne. »
${}^{13}Moïse répondit à Dieu : « J’irai donc trouver les fils d’Israël, et je leur dirai : “Le Dieu de vos pères m’a envoyé vers vous.” Ils vont me demander quel est son nom ; que leur répondrai-je ? »
${}^{14}Dieu dit à Moïse : « Je suis qui je suis\\. Tu parleras ainsi aux fils d’Israël : “Celui qui m’a envoyé vers vous, c’est : Je-suis”. »
${}^{15}Dieu dit encore à Moïse : « Tu parleras ainsi aux fils d’Israël : “Celui qui m’a envoyé vers vous, c’est Le Seigneur\\, le Dieu de vos pères, le Dieu d’Abraham, le Dieu d’Isaac, le Dieu de Jacob”. C’est là mon nom pour toujours, c’est par lui que vous ferez mémoire de moi\\, d’âge en d’âge. 
${}^{16}Va, rassemble les anciens d’Israël. Tu leur diras : “Le Seigneur, le Dieu de vos pères, le Dieu d’Abraham, d’Isaac et de Jacob, m’est apparu. Il m’a dit : Je vous ai visités et ainsi j’ai vu\\comment on vous traite en Égypte. 
${}^{17}J’ai dit : Je vous ferai monter de la misère qui vous accable en Égypte vers le pays du Cananéen, du Hittite, de l’Amorite, du Perizzite, du Hivvite\\et du Jébuséen, le pays ruisselant de lait et de miel.” 
${}^{18}Ils écouteront ta voix ; alors tu iras, avec les anciens d’Israël, auprès du roi d’Égypte, et vous lui direz : “Le Seigneur, le Dieu des Hébreux, est venu nous trouver. Et maintenant, laisse-nous aller dans le désert, à trois jours de marche, pour y offrir un sacrifice au Seigneur notre Dieu.” 
${}^{19}Or, je sais, moi, que le roi d’Égypte ne vous laissera pas partir s’il n’y est pas forcé. 
${}^{20}Aussi j’étendrai la main, je frapperai l’Égypte par toutes sortes de prodiges que j’accomplirai au milieu d’elle. Après cela, il vous permettra de partir.
${}^{21}Je ferai que ce peuple trouve grâce aux yeux des Égyptiens. Aussi, quand vous partirez, vous n’aurez pas les mains vides. 
${}^{22}Chaque femme demandera à sa voisine et à l’étrangère qui réside en sa maison des objets d’argent, des objets d’or et des manteaux : vous les ferez porter par vos fils et vos filles. Ainsi vous dépouillerez les Égyptiens. »
      
         
      \bchapter{}
      \begin{verse}
${}^{1}Moïse reprit la parole et dit : « Mais voilà ! Ils ne me croiront pas ; ils n’écouteront pas ma voix. Ils diront : Le Seigneur ne t’est pas apparu ! » 
${}^{2}Le Seigneur dit : « Que tiens-tu en main ? » Moïse répondit : « Un bâton. » 
${}^{3}Le Seigneur dit : « Jette-le à terre. » Moïse le jeta à terre : le bâton devint un serpent, et Moïse s’enfuit devant lui. 
${}^{4}Le Seigneur dit à Moïse : « Étends la main et prends-le par la queue. » Il étendit la main et le saisit : dans sa main, le serpent redevint un bâton. 
${}^{5}Dieu reprit : « Ainsi croiront-ils que le Seigneur t’est apparu, le Dieu de leurs pères, Dieu d’Abraham, Dieu d’Isaac, Dieu de Jacob. »
${}^{6}Le Seigneur dit encore à Moïse : « Mets donc la main sur ta poitrine. » Il mit la main sur sa poitrine, puis la retira : et sa main était lépreuse, blanche comme neige. 
${}^{7}Le Seigneur dit : « Remets la main sur ta poitrine. » Il remit la main sur sa poitrine, puis la retira : elle était redevenue comme le reste de son corps. 
${}^{8}« Ainsi donc, s’ils ne te croient pas, s’ils restent sourds à la voix du premier signe, ils croiront à cause du second signe. 
${}^{9}Et s’ils ne croient pas encore à ces deux signes et restent sourds à ta voix, alors tu prendras de l’eau du Nil et tu la répandras sur la terre sèche. Et l’eau que tu auras puisée dans le Nil deviendra du sang sur la terre sèche. »
${}^{10}Moïse dit encore au Seigneur : « Pardon, mon Seigneur, mais moi, je n’ai jamais été doué pour la parole, ni d’hier ni d’avant-hier, ni même depuis que tu parles à ton serviteur ; j’ai la bouche lourde et la langue pesante, moi ! » 
${}^{11}Le Seigneur lui dit : « Qui donc a donné une bouche à l’homme ? Qui rend muet ou sourd, voyant ou aveugle ? N’est-ce pas moi, le Seigneur ? 
${}^{12}Et maintenant, va. Je suis avec ta bouche et je te ferai savoir ce que tu devras dire. »
${}^{13}Moïse répliqua : « Je t’en prie, mon Seigneur, envoie n’importe quel autre émissaire. » 
${}^{14}Alors la colère du Seigneur s’enflamma contre Moïse, et il dit : « Et ton frère Aaron, le lévite ? Je sais qu’il a la parole facile, lui ! Le voici justement qui sort à ta rencontre, et quand il te verra, son cœur se réjouira. 
${}^{15}Tu lui parleras et tu mettras mes paroles dans sa bouche. Et moi, je suis avec ta bouche et avec sa bouche, et je vous ferai savoir ce que vous aurez à faire. 
${}^{16}C’est lui qui parlera pour toi au peuple ; il sera ta bouche et tu seras son dieu. 
${}^{17}Quant à ce bâton, prends-le en main ! C’est par lui que tu accompliras les signes. »
${}^{18}Moïse s’en alla et retourna chez son beau-père Jéthro. Il lui dit : « Je dois m’en aller et retourner chez mes frères, en Égypte, pour voir s’ils vivent encore. » Jéthro lui dit : « Va en paix. » 
${}^{19}Au pays de Madiane, le Seigneur dit à Moïse : « Va, retourne en Égypte, car ils sont morts, tous ceux qui en voulaient à ta vie. » 
${}^{20}Moïse prit sa femme et ses fils, les installa sur l’âne et retourna au pays d’Égypte. Il avait pris en main le bâton de Dieu. 
${}^{21}Le Seigneur dit à Moïse : « Sur le chemin du retour vers l’Égypte, songe aux prodiges que j’ai mis en ta main. Tu les accompliras devant Pharaon. Mais moi, je ferai en sorte qu’il s’obstine, et il ne laissera pas le peuple s’en aller. 
${}^{22}Tu diras à Pharaon : “Ainsi parle le Seigneur : 
${}^{23}Mon fils premier-né, c’est Israël. Je te dis : Laisse partir mon fils pour qu’il me serve ; et tu refuses de le laisser partir ! Eh bien, moi, je vais faire périr ton fils premier-né !” »
${}^{24}Or, en cours de route, au campement de nuit, le Seigneur rencontra Moïse et chercha à le faire mourir. 
${}^{25}Cippora, sa femme, prit un silex, coupa le prépuce de son fils, en toucha le sexe de Moïse et dit : « Tu es pour moi un époux de sang. » 
${}^{26}Alors Dieu s’éloigna de Moïse. Cippora avait parlé d’« époux de sang » à cause des circoncisions.
${}^{27}Le Seigneur dit à Aaron : « Va sur la route du désert au-devant de Moïse. » Il y alla, le rencontra à la montagne de Dieu et l’embrassa. 
${}^{28}Moïse transmit à son frère toutes les paroles que le Seigneur l’avait envoyé dire et tous les signes qu’il avait ordonné de faire. 
${}^{29}Moïse et Aaron se mirent en route et réunirent tous les anciens des fils d’Israël. 
${}^{30}Aaron redit toutes les paroles que le Seigneur avait adressées à Moïse et il accomplit les signes sous les yeux du peuple. 
${}^{31}Et le peuple crut : il comprit que le Seigneur avait visité les fils d’Israël et qu’il avait vu leur misère. Alors ils s’inclinèrent et se prosternèrent.
      
         
      \bchapter{}
      \begin{verse}
${}^{1}Ensuite, Moïse et Aaron s’en vinrent déclarer à Pharaon : « Ainsi parle le Seigneur, Dieu d’Israël : Laisse partir mon peuple pour qu’il célèbre en mon honneur une fête au désert. » 
${}^{2}Pharaon dit : « Qui est le Seigneur pour que j’écoute sa voix et laisse partir Israël ? Je ne connais pas le Seigneur et je ne veux pas laisser partir Israël. » 
${}^{3}Ils dirent : « Le Dieu des Hébreux s’est présenté à nous : il nous faut aller à trois jours de marche dans le désert pour offrir un sacrifice au Seigneur notre Dieu. Sinon, il nous frappera de la peste ou de l’épée. » 
${}^{4}Le roi d’Égypte leur dit : « Moïse et Aaron, pourquoi voulez-vous détourner le peuple de ses travaux ? Retournez à vos corvées ! » 
${}^{5}Et Pharaon ajouta : « Maintenant que les gens du peuple sont nombreux, vous voudriez qu’ils se reposent de leurs corvées ! »
${}^{6}Ce jour-là, Pharaon ordonna aux surveillants du peuple et aux contremaîtres : 
${}^{7}« Vous ne fournirez plus au peuple, comme vous le faisiez auparavant, la paille pour fabriquer les briques. Ils iront eux-mêmes ramasser la paille. 
${}^{8}Quant au nombre de briques imposé jusqu’à présent, continuez à l’exiger. Ne réduisez en rien la cadence. Ce ne sont que des paresseux ! C’est pourquoi ils crient : “Allons offrir un sacrifice à notre Dieu !” 
${}^{9}Que la servitude pèse sur ces gens et qu’ils travaillent, sans rêvasser à des paroles mensongères ! »
${}^{10}Les surveillants du peuple et les contremaîtres sortirent et déclarèrent au peuple : « Ainsi parle Pharaon : Je ne vous donne plus de paille. 
${}^{11}Allez vous-mêmes en prendre où vous en trouverez ! Mais votre production ne sera réduite en rien. » 
${}^{12}Alors le peuple se dispersa dans tout le pays d’Égypte, pour ramasser du chaume et en faire de la paille à torchis. 
${}^{13}Les surveillants les harcelaient : « Achevez votre travail ! Chaque jour la quantité exigée, comme lorsqu’il y avait de la paille ! » 
${}^{14}On frappa les contremaîtres des fils d’Israël – ceux que leur avaient imposés les surveillants de Pharaon – en disant : « Pourquoi n’avez-vous pas exécuté la commande de briques comme auparavant ? Faites aujourd’hui comme hier ! »
${}^{15}Les contremaîtres des fils d’Israël vinrent alors crier vers Pharaon : « Pourquoi traiter ainsi tes serviteurs ? 
${}^{16}De la paille, on n’en donne plus à tes serviteurs, et on nous dit : Faites des briques ! Et voici qu’on frappe tes serviteurs. Tes gens ont tort ! » 
${}^{17}Pharaon répondit : « Vous êtes des paresseux, oui, des paresseux ! C’est pourquoi vous dites : Allons offrir un sacrifice au Seigneur. 
${}^{18}Et maintenant, au travail ! On ne vous fournira pas la paille, mais vous fournirez le même nombre de briques. »
${}^{19}Les contremaîtres des fils d’Israël se virent en mauvaise posture lorsqu’on leur dit : « Vous ne réduirez pas le nombre de briques : chaque jour la quantité exigée ! » 
${}^{20}En sortant de chez Pharaon, ils tombèrent sur Moïse et Aaron qui les attendaient. 
${}^{21}Ils leur dirent : « Que le Seigneur vous tienne à l’œil et qu’il juge : à cause de vous, Pharaon et ses serviteurs nous détestent ; vous leur avez mis en main l’épée pour nous tuer. »
${}^{22}Moïse retourna trouver le Seigneur et lui dit : « Mon Seigneur, pourquoi as-tu maltraité ce peuple ? Pourquoi donc m’as-tu envoyé ? 
${}^{23}Depuis que je suis allé chez Pharaon et lui ai parlé en ton nom, il a maltraité ce peuple, et tu ne fais absolument rien pour délivrer ton peuple. »
      
         
      \bchapter{}
      \begin{verse}
${}^{1}Le Seigneur dit à Moïse :
      « Maintenant, tu vas voir ce que je vais faire à Pharaon : contraint par une main forte, il les laissera partir ; contraint par une main forte, il les chassera de son pays. »
${}^{2}Dieu parla à Moïse. Il lui dit : « Je suis le Seigneur. 
${}^{3}Je suis apparu à Abraham, à Isaac et à Jacob comme le Dieu-Puissant ; mais mon nom “Le Seigneur”, je ne l’ai pas fait connaître. 
${}^{4}Ensuite, j’ai établi mon alliance avec eux, m’engageant à leur donner la terre de Canaan, la terre étrangère où ils étaient venus en immigrés. 
${}^{5}Puis enfin, j’ai entendu la plainte des fils d’Israël réduits en esclavage par les Égyptiens, et je me suis souvenu de mon alliance. 
${}^{6}C’est pourquoi, parle ainsi aux fils d’Israël : “Je suis le Seigneur. Je vous ferai sortir loin des corvées qui vous accablent en Égypte. Je vous délivrerai de la servitude. Je vous rachèterai d’un bras vigoureux et par de grands châtiments. 
${}^{7}Je vous prendrai pour peuple, et moi, je serai votre Dieu. Alors, vous saurez que je suis le Seigneur, votre Dieu, celui qui vous fait sortir loin des corvées qui vous accablent en Égypte. 
${}^{8}Puis, je vous ferai entrer dans la terre que, la main levée, je me suis engagé à donner à Abraham, à Isaac et à Jacob. Je vous la donnerai pour que vous la possédiez. Je suis le Seigneur.” »
${}^{9}Moïse rapporta ces paroles aux fils d’Israël, mais ils n’écoutèrent pas Moïse, car ils étaient à bout de souffle, tant leur esclavage était dur.
${}^{10}Le Seigneur dit à Moïse : 
${}^{11}« Va dire à Pharaon, le roi d’Égypte, qu’il laisse partir de son pays les fils d’Israël. » 
${}^{12}Mais Moïse prit la parole en présence du Seigneur et dit : « Les fils d’Israël ne m’ont pas écouté. Comment Pharaon m’écouterait-il, moi qui n’ai pas la parole facile ? »
${}^{13}Le Seigneur parla à Moïse et Aaron, et leur communiqua ses ordres pour les fils d’Israël et pour Pharaon, roi d’Égypte, en vue de faire sortir les fils d’Israël du pays d’Égypte.
${}^{14}Voici les chefs de leurs familles : fils de Roubène, premier-né d’Israël : Hénok, Pallou, Hesrone et Karmi ; tels sont les clans de Roubène.
${}^{15}Fils de Siméon : Yemouël, Yamine, Ohad, Yakine, Sohar et Shaoul, le fils de la Cananéenne ; tels sont les clans de Siméon.
${}^{16}Voici les noms des fils de Lévi avec leurs descendances : Guershone, Qehath et Merari. Lévi vécut cent trente-sept ans.
${}^{17}Fils de Guershone : Libni et Shiméï, par clans.
${}^{18}Fils de Qehath : Amram, Yisehar, Hébrone et Ouzziël. Qehath vécut cent trente-trois ans.
${}^{19}Fils de Merari : Mahli et Moushi. Tels sont les clans de Lévi avec leurs descendances.
${}^{20}Amram prit pour femme Yokèbed, sa tante, qui lui enfanta Aaron et Moïse. Amram vécut cent trente-sept ans.
${}^{21}Fils de Yisehar : Coré, Néfeg et Zikri ; 
${}^{22}fils d’Ouzziël : Mishaël, Elçafane et Sitri.
${}^{23}Aaron prit pour femme Élishèba, fille d’Amminadab, sœur de Nashone, et elle lui enfanta Nadab, Abihou, Éléazar et Itamar.
${}^{24}Fils de Coré : Assir, Elcana et Abiasaph ; tels sont les clans des Coréites.
${}^{25}Éléazar, fils d’Aaron, prit pour femme l’une des filles de Poutiel, qui lui enfanta Pinhas. Tels sont les chefs de famille des Lévites, par clans.
${}^{26}C’est à Aaron et Moïse que le Seigneur avait dit : « Faites sortir du pays d’Égypte les fils d’Israël rangés comme une armée. » 
${}^{27}Ce sont eux – Moïse et Aaron – qui parlèrent à Pharaon, roi d’Égypte, pour faire sortir d’Égypte les fils d’Israël.
${}^{28}C’était le jour où le Seigneur parla à Moïse au pays d’Égypte. 
${}^{29}Le Seigneur parla à Moïse. Il dit : « Je suis le Seigneur. Répète à Pharaon, le roi d’Égypte, tout ce que moi, je vais te dire. » 
${}^{30}Alors, Moïse dit en présence du Seigneur : « Je n’ai pas la parole facile. Comment Pharaon m’écouterait-il ? »
      
         
      \bchapter{}
      \begin{verse}
${}^{1}Le Seigneur dit à Moïse : « Vois, j’ai fait de toi un dieu pour Pharaon, et ton frère Aaron sera ton prophète. 
${}^{2}Toi, tu lui diras tout ce que je t’ordonnerai, et ton frère Aaron le répétera à Pharaon pour qu’il laisse partir de son pays les fils d’Israël. 
${}^{3}Mais moi, je rendrai le cœur de Pharaon inflexible ; je multiplierai mes signes et mes prodiges dans le pays d’Égypte, 
${}^{4}mais Pharaon ne vous écoutera pas. Alors je poserai la main sur l’Égypte et je ferai sortir du pays d’Égypte mes armées, mon peuple, les fils d’Israël, en exerçant de terribles jugements. 
${}^{5}Les Égyptiens reconnaîtront que Je suis le Seigneur, quand j’étendrai la main contre l’Égypte et que j’en ferai sortir les fils d’Israël. »
${}^{6}Moïse et Aaron s’exécutèrent : ce que le Seigneur leur avait ordonné, ils le firent. 
${}^{7}Moïse était âgé de quatre-vingts ans et Aaron de quatre-vingt-trois ans, lorsqu’ils parlèrent à Pharaon.
${}^{8}Le Seigneur dit à Moïse et Aaron : 
${}^{9}« Si Pharaon vous demande d’accomplir un prodige, tu diras alors à Aaron : Prends ton bâton, jette-le devant Pharaon, et qu’il devienne un serpent. » 
${}^{10}Moïse et Aaron allèrent trouver Pharaon et firent comme l’avait ordonné le Seigneur. Aaron jeta son bâton devant Pharaon et ses serviteurs, et le bâton devint un serpent. 
${}^{11}Pharaon, à son tour, convoqua les sages et les enchanteurs. Les magiciens d’Égypte en firent autant avec leurs sortilèges. 
${}^{12}Chacun jeta son bâton qui devint un serpent, mais le bâton d’Aaron engloutit leurs bâtons. 
${}^{13}Cependant, Pharaon s’obstina ; il n’écouta pas Moïse et Aaron, ainsi que l’avait annoncé le Seigneur.
${}^{14}Le Seigneur dit à Moïse : « Le cœur de Pharaon s’est appesanti ; il refuse de laisser partir le peuple. 
${}^{15}Va trouver Pharaon demain matin : il sortira pour se rendre près de l’eau ; tu te posteras au bord du Nil pour le rencontrer. Le bâton qui s’est changé en serpent, tu le prendras en main. 
${}^{16}Tu diras à Pharaon : “Le Seigneur, le Dieu des Hébreux, m’a envoyé vers toi pour te dire : Laisse partir mon peuple afin qu’il me serve dans le désert.” Jusqu’à présent tu n’as pas écouté. 
${}^{17}Ainsi parle le Seigneur : À ceci tu reconnaîtras que je suis le Seigneur. Voici que moi, je vais frapper les eaux du Nil avec le bâton que j’ai dans la main, et elles se changeront en sang. 
${}^{18}Les poissons du Nil crèveront, le Nil s’empuantira et les Égyptiens ne pourront plus boire l’eau du fleuve. »
${}^{19}Le Seigneur dit à Moïse : « Va dire à Aaron : Prends ton bâton, étends la main sur les eaux d’Égypte – sur ses rivières, ses canaux, ses étangs, sur toutes ses réserves d’eau – et qu’elles soient du sang ! Qu’il y ait du sang dans tout le pays d’Égypte, jusque dans les récipients de bois et de pierre. »
${}^{20}Moïse et Aaron firent comme le Seigneur l’avait ordonné. Aaron leva son bâton et frappa les eaux du Nil sous les yeux de Pharaon et de ses serviteurs, et toutes les eaux du Nil se changèrent en sang. 
${}^{21}Les poissons du Nil crevèrent et le Nil s’empuantit ; les Égyptiens ne pouvaient plus boire l’eau du fleuve ; il y avait du sang dans tout le pays d’Égypte. 
${}^{22}Mais les magiciens d’Égypte en firent autant avec leurs sortilèges, et Pharaon s’obstina ; il n’écouta pas Moïse et Aaron, ainsi que l’avait annoncé le Seigneur.
${}^{23}Pharaon s’en retourna. Il rentra chez lui sans prendre la chose à cœur. 
${}^{24}En quête d’eau, tous les Égyptiens se mirent à creuser aux alentours du Nil, car ils ne pouvaient plus boire les eaux du fleuve. 
${}^{25}Sept jours s’écoulèrent après que le Seigneur eut frappé le Nil.
${}^{26}Le Seigneur dit à Moïse : « Va trouver Pharaon et dis-lui : Ainsi parle le Seigneur : “Laisse partir mon peuple, afin qu’il me serve.” 
${}^{27}Si toi, tu refuses de le laisser partir, moi je vais infester de grenouilles tout ton territoire. 
${}^{28}Le Nil grouillera de grenouilles ; elles monteront, elles entreront dans ta maison, dans ta chambre à coucher, sur ton lit, dans les maisons de tes serviteurs et de ton peuple, dans tes fours et dans tes pétrins. 
${}^{29}Sur toi, sur les gens de ton peuple et sur tous tes serviteurs, grimperont les grenouilles. »
      
         
      \bchapter{}
      \begin{verse}
${}^{1}Le Seigneur dit à Moïse : « Va dire à Aaron : Étends la main avec ton bâton sur les rivières, les canaux, les étangs, et fais grimper les grenouilles sur le pays d’Égypte. » 
${}^{2}Aaron étendit la main sur les eaux d’Égypte ; les grenouilles grimpèrent et couvrirent le pays d’Égypte. 
${}^{3}Mais les magiciens en firent autant avec leurs sortilèges ; ils firent grimper, eux aussi, des grenouilles sur le pays d’Égypte.
${}^{4}Pharaon appela Moïse et Aaron, et leur dit : « Priez le Seigneur de nous débarrasser des grenouilles, moi et mon peuple, et j’accepterai de laisser partir le peuple des Hébreux pour qu’il offre un sacrifice au Seigneur. » 
${}^{5}Moïse dit à Pharaon : « Daigne me dire quand je devrai prier pour toi, pour tes serviteurs et pour ton peuple, afin que les grenouilles disparaissent de chez toi et de toutes les maisons, et qu’il n’en reste plus, sinon dans le Nil. » 
${}^{6}Pharaon répondit : « Demain. » Moïse reprit : « Il en sera donc selon ta parole, afin que tu reconnaisses que nul n’est comme le Seigneur notre Dieu. 
${}^{7}Les grenouilles s’éloigneront de toi, de tes maisons, de tes serviteurs et de ton peuple ; il n’en restera plus, sinon dans le Nil. »
${}^{8}Moïse et Aaron sortirent de chez Pharaon, et Moïse cria vers le Seigneur à propos des grenouilles dont il avait accablé Pharaon. 
${}^{9}Le Seigneur agit selon la parole de Moïse : les grenouilles crevèrent dans les maisons, dans les cours, dans les champs. 
${}^{10}On en fit des tas et des tas, et le pays s’empuantit. 
${}^{11}Pharaon vit qu’il y avait un répit ; il s’entêta ; il n’écouta pas Moïse et Aaron, ainsi que l’avait annoncé le Seigneur.
${}^{12}Le Seigneur dit à Moïse : « Va dire à Aaron : Étends ton bâton et frappe la poussière du sol ; elle se changera en moustiques dans tout le pays d’Égypte. » 
${}^{13}Ils firent ainsi. Aaron étendit la main, il frappa de son bâton la poussière du sol, et les moustiques s’abattirent sur les gens et sur les bêtes ; toute la poussière du sol se changea en moustiques dans tout le pays d’Égypte. 
${}^{14}Les magiciens firent le même geste avec leurs sortilèges pour éliminer les moustiques, mais ils n’y réussirent pas : les moustiques restaient sur les gens et sur les bêtes. 
${}^{15}Les magiciens dirent alors à Pharaon : « C’est le doigt de Dieu ! » Mais Pharaon s’obstina ; il n’écouta pas Moïse et Aaron, ainsi que l’avait annoncé le Seigneur.
${}^{16}Le Seigneur dit à Moïse : « Lève-toi de bon matin, et tu te posteras devant Pharaon quand il sortira pour se rendre près de l’eau. Tu lui diras : Ainsi parle le Seigneur : Laisse partir mon peuple afin qu’il me serve. 
${}^{17}Si toi, tu ne renvoies pas mon peuple, moi j’enverrai la vermine sur toi, sur tes serviteurs, sur ton peuple et dans tes maisons. Les maisons des Égyptiens seront pleines de vermine, et même le sol qu’ils foulent en sera couvert. 
${}^{18}Mais ce jour-là, je mettrai à part le pays de Goshèn où réside mon peuple : là il n’y aura pas de vermine, afin que tu reconnaisses que moi, le Seigneur Dieu, je suis au milieu du pays. 
${}^{19}J’établirai une distinction entre mon peuple et ton peuple ; c’est demain qu’aura lieu ce signe. » 
${}^{20}Et le Seigneur fit ainsi. La vermine envahit la maison de Pharaon, celles de ses serviteurs et tout le pays d’Égypte ; le pays en fut infesté.
${}^{21}Pharaon appela Moïse et Aaron, et leur dit : « Allez, offrez un sacrifice à votre Dieu, mais ici, dans le pays. » 
${}^{22}Moïse répondit : « Sûrement pas ! Car les Égyptiens ont en abomination les sacrifices que nous offrons au Seigneur notre Dieu. Pourrions-nous faire sous leurs yeux, sans qu’ils nous lapident, un sacrifice qu’ils ont en abomination ? 
${}^{23}Nous voulons aller à trois jours de marche dans le désert pour offrir un sacrifice au Seigneur notre Dieu, selon ce qu’il nous dira. » 
${}^{24}Pharaon dit : « Moi, je vous laisserai partir et, dans le désert, vous offrirez un sacrifice au Seigneur votre Dieu. Seulement, ne vous éloignez pas trop et priez pour moi ! » 
${}^{25}Moïse répondit : « Eh bien, je vais sortir de chez toi, et je prierai le Seigneur. Demain, Pharaon, ses serviteurs et son peuple seront débarrassés de la vermine. Mais, que Pharaon cesse de se moquer de nous, en refusant de laisser partir le peuple afin qu’il offre un sacrifice au Seigneur ! »
${}^{26}Moïse sortit de chez Pharaon et pria le Seigneur. 
${}^{27}Le Seigneur agit selon la parole de Moïse, et la vermine s’éloigna de Pharaon, de ses serviteurs et de son peuple ; il n’en resta plus. 
${}^{28}Une fois encore, Pharaon s’entêta et ne laissa pas le peuple partir.
      
         
      \bchapter{}
      \begin{verse}
${}^{1}Le Seigneur dit à Moïse : « Va trouver Pharaon et dis-lui : Ainsi parle le Seigneur, le Dieu des Hébreux : Laisse partir mon peuple afin qu’il me serve. 
${}^{2}Si tu refuses de les laisser partir, si tu les retiens plus longtemps, 
${}^{3}voici que la main du Seigneur s’abattra sur tes troupeaux qui sont dans les champs, sur les chevaux, les ânes, les chameaux, sur le gros et le petit bétail : ils seront frappés d’une terrible peste ! 
${}^{4}Le Seigneur fera la distinction entre les troupeaux d’Israël et les troupeaux des Égyptiens ; pas une seule bête appartenant aux fils d’Israël ne périra. » 
${}^{5}Le Seigneur fixa le moment en disant : « Demain, le Seigneur accomplira cela dans le pays. »
${}^{6}Le lendemain, le Seigneur mit sa parole à exécution : tous les troupeaux des Égyptiens périrent ; mais, dans les troupeaux des fils d’Israël, pas une bête ne périt. 
${}^{7}Pharaon s’informa, et voici : pas une seule bête des troupeaux d’Israël n’avait péri. Mais Pharaon s’entêta et il ne laissa pas le peuple partir.
${}^{8}Le Seigneur dit à Moïse et Aaron : « Prenez deux pleines poignées de suie de fourneau, et que Moïse la lance en l’air, sous les yeux de Pharaon. 
${}^{9}Elle deviendra une fine poussière qui retombera sur tout le pays d’Égypte et provoquera, chez les gens et les bêtes, des ulcères bourgeonnant en pustules. » 
${}^{10}Ils prirent de la suie de fourneau et vinrent se tenir devant Pharaon ; Moïse la lança en l’air : hommes et bêtes furent couverts d’ulcères bourgeonnant en pustules. 
${}^{11}Les magiciens ne purent se tenir devant Moïse à cause des ulcères : en effet, ils en étaient couverts comme tous les Égyptiens.
${}^{12}Le Seigneur fit en sorte que Pharaon s’obstine ; et celui-ci n’écouta pas Moïse et Aaron, ainsi que l’avait annoncé le Seigneur.
${}^{13}Le Seigneur dit à Moïse : « Lève-toi de bon matin, et tu te posteras devant Pharaon. Tu lui diras : Ainsi parle le Seigneur, le Dieu des Hébreux : “Laisse partir mon peuple, afin qu’il me serve.” 
${}^{14}Car cette fois-ci, je vais envoyer tous mes fléaux contre ta personne, contre tes serviteurs et contre ton peuple, afin que tu reconnaisses que, sur toute la terre, nul n’est comme moi. 
${}^{15}Si, dès l’abord, j’avais laissé aller ma main et t’avais frappé de la peste, toi et ton peuple, tu aurais été effacé de la terre. 
${}^{16}Cependant, je t’ai laissé subsister, et voici pourquoi : c’est afin que tu voies ma force et qu’on proclame mon nom par toute la terre.
${}^{17}Tu le prends de haut avec mon peuple en t’opposant à son départ ! 
${}^{18}Eh bien, moi, demain, à pareille heure, je ferai tomber une grêle d’une extrême violence, comme il n’y en a jamais eu en Égypte depuis le jour de sa fondation jusqu’à présent. 
${}^{19}Maintenant, envoie donc mettre à l’abri tes troupeaux, ainsi que tout ce qui t’appartient dans les champs. Tout homme et toute bête qui se trouveront dans les champs et n’auront pas regagné les maisons, tous, quand la grêle s’abattra, périront. » 
${}^{20}Parmi les serviteurs de Pharaon, celui qui craignit la parole du Seigneur mit à l’abri serviteurs et troupeaux. 
${}^{21}Mais celui qui ne prit pas à cœur la parole du Seigneur laissa dans les champs serviteurs et troupeaux.
${}^{22}Le Seigneur dit à Moïse : « Étends la main vers le ciel, et qu’il y ait de la grêle partout en Égypte, sur les hommes et sur les bêtes, et sur l’herbe des champs dans ce pays d’Égypte. » 
${}^{23}Moïse étendit son bâton vers le ciel, et le Seigneur déchaîna tonnerre et grêle. La foudre tomba sur terre, et le Seigneur fit pleuvoir la grêle sur le pays d’Égypte. 
${}^{24}Il y eut grêle et foudre mêlée à la grêle. Ce fut d’une violence extrême : jamais il n’y eut rien de tel dans le pays d’Égypte depuis qu’il est une nation. 
${}^{25}Partout en Égypte, la grêle frappa tout ce qui était dans les champs, depuis l’homme jusqu’au bétail ; la grêle frappa toute l’herbe des champs et brisa tout arbre dans les champs. 
${}^{26}Au seul pays de Goshèn, là où résidaient les fils d’Israël, il n’y eut pas de grêle.
${}^{27}Pharaon fit appeler Moïse et Aaron, et leur dit : « Cette fois-ci, je reconnais mon péché. C’est le Seigneur qui est le juste ; moi et mon peuple, nous sommes les coupables. 
${}^{28}Priez le Seigneur ! Assez de tonnerre et de grêle ! Je vais vous laisser partir : ne restez pas plus longtemps sur place. » 
${}^{29}Moïse lui dit : « Dès que je serai sorti de la ville, je tendrai les mains vers le Seigneur : le tonnerre cessera, la grêle ne tombera plus, afin que tu reconnaisses que le pays appartient au Seigneur. 
${}^{30}Et pourtant, toi et tes serviteurs, je le sais, vous ne craindrez pas encore le Seigneur Dieu. »
${}^{31}Le lin et l’orge avaient été saccagés, car l’orge était en épis, et le lin en fleurs. 
${}^{32}Le blé ainsi que l’épeautre avaient été épargnés, car ils sont tardifs.
${}^{33}Moïse quitta Pharaon et sortit de la ville ; il tendit les mains vers le Seigneur : le tonnerre et la grêle cessèrent, et la pluie s’arrêta de tomber sur la terre. 
${}^{34}Pharaon, voyant que la pluie, la grêle et le tonnerre avaient cessé, persévéra dans son péché ; lui et ses serviteurs s’entêtèrent.
${}^{35}Pharaon s’obstina : il ne laissa pas partir les fils d’Israël, ainsi que le Seigneur l’avait annoncé par l’intermédiaire de Moïse.
      
         
      \bchapter{}
      \begin{verse}
${}^{1}Le Seigneur dit à Moïse : « Rends-toi chez Pharaon, car c’est moi qui l’ai rendu entêté, lui et ses serviteurs, afin d’accomplir mes signes au milieu d’eux, 
${}^{2}et afin que tu puisses raconter à ton fils et au fils de ton fils comment je me suis joué des Égyptiens et quels signes j’ai accomplis parmi eux. Alors, vous saurez que je suis le Seigneur. » 
${}^{3}Moïse et Aaron allèrent trouver Pharaon et lui dirent : « Ainsi parle le Seigneur, le Dieu des Hébreux : Combien de temps refuseras-tu de t’humilier devant moi ? Laisse partir mon peuple afin qu’il me serve. 
${}^{4}Si toi, tu refuses toujours de laisser partir mon peuple, moi, dès demain, je ferai venir des sauterelles sur ton territoire. 
${}^{5}Elles recouvriront le pays, et l’on ne pourra plus en voir le sol. Elles dévoreront ce qui reste, ce qui a échappé à la grêle, ce que la grêle vous a laissé ; elles dévoreront tout arbre qui pousse dans vos champs. 
${}^{6}Elles envahiront tes maisons, les maisons de tous tes serviteurs, les maisons de tous les Égyptiens : ni tes pères, ni les pères de tes pères n’ont jamais vu cela, depuis le jour où ils sont venus au monde, jusqu’à ce jour. » Moïse tourna le dos et sortit de chez Pharaon. 
${}^{7}Les serviteurs de Pharaon lui dirent : « Combien de temps encore cet individu sera-t-il un piège pour nous ? Laisse partir les hommes, afin qu’ils servent le Seigneur leur Dieu. N’as-tu pas encore compris que l’Égypte va à sa ruine ? »
${}^{8}On fit revenir Moïse et Aaron auprès de Pharaon, qui leur dit : « Allez, servez le Seigneur votre Dieu. Mais qui donc va partir ? » 
${}^{9}Moïse répondit : « Nous partirons avec nos jeunes gens et nos vieillards, nous partirons avec nos fils et nos filles, notre petit et notre gros bétail, car c’est pour nous une fête en l’honneur du Seigneur. » 
${}^{10}Pharaon dit : « Que le Seigneur soit avec vous si je vous laisse partir, vous et vos enfants ! Voyez comme vos projets sont pervers ! 
${}^{11}Non et non ! Vous, les hommes, allez donc et servez le Seigneur, puisque c’est cela que vous cherchez. » Et on les chassa de chez Pharaon.
${}^{12}Le Seigneur dit à Moïse : « Étends la main sur le pays d’Égypte pour que viennent les sauterelles ; qu’elles montent sur le pays d’Égypte et qu’elles dévorent toute l’herbe du pays, tout ce qu’a laissé la grêle. » 
${}^{13}Moïse étendit son bâton sur le pays d’Égypte, et le Seigneur fit lever sur le pays un vent d’est qui souffla tout ce jour-là et toute la nuit. Au matin, le vent d’est avait amené les sauterelles. 
${}^{14}Des nuées de sauterelles montèrent sur tout le pays d’Égypte et se posèrent sur l’ensemble du territoire. Jamais auparavant et jamais depuis lors, il n’y eut une telle masse de sauterelles. 
${}^{15}Elles recouvrirent tout le pays, qui en fut obscurci. Elles dévorèrent toute l’herbe du pays et tous les fruits des arbres épargnés par la grêle ; il ne resta rien de vert ni sur les arbres ni dans les prairies, par tout le pays d’Égypte. 
${}^{16}Pharaon se hâta d’appeler Moïse et Aaron, et leur dit : « J’ai péché contre le Seigneur votre Dieu, et contre vous. 
${}^{17}Et maintenant, je t’en prie : une fois encore, enlève ma faute. Priez le Seigneur votre Dieu, pour qu’il écarte de moi cette mort. »
${}^{18}Moïse sortit de chez Pharaon et pria le Seigneur. 
${}^{19}Le Seigneur changea le vent d’est en un très fort vent d’ouest qui emporta les sauterelles et les précipita dans la mer des Roseaux. Il ne resta plus une seule sauterelle sur tout le territoire d’Égypte.
${}^{20}Mais le Seigneur fit en sorte que Pharaon s’obstine ; et celui-ci ne laissa pas partir les fils d’Israël.
${}^{21}Le Seigneur dit à Moïse : « Étends la main vers le ciel. Qu’il y ait des ténèbres sur le pays d’Égypte, des ténèbres où l’on tâtonne. » 
${}^{22}Moïse étendit la main vers le ciel et, pendant trois jours, il y eut d’épaisses ténèbres sur tout le pays d’Égypte. 
${}^{23}Les gens ne se voyaient plus l’un l’autre, et chacun resta sur place pendant trois jours. Mais il y avait de la lumière pour les fils d’Israël, là où ils habitaient. 
${}^{24}Pharaon appela Moïse et lui dit : « Allez-vous-en, servez le Seigneur ! Votre petit et votre gros bétail devra rester ici, mais vos enfants pourront vous accompagner. » 
${}^{25}Moïse dit : « C’est donc toi qui mettras dans nos mains de quoi offrir sacrifices et holocaustes au Seigneur notre Dieu ? 
${}^{26}Nos troupeaux partiront également avec nous. Pas une bête ne restera ; c’est parmi nos troupeaux que nous prendrons de quoi servir le Seigneur notre Dieu. Nous ne pouvons pas savoir, avant d’arriver là-bas, ce que nous devrons offrir au Seigneur pour le servir. »
${}^{27}Mais le Seigneur fit en sorte que Pharaon s’obstine ; et celui-ci ne voulut pas les laisser partir.
${}^{28}Pharaon dit alors à Moïse : « Hors d’ici ! Prends garde à toi ! Ne t’avise plus de paraître devant ma face ! Le jour où tu te présenteras devant ma face, tu mourras. » 
${}^{29}Et Moïse de répondre : « Tu l’as dit ! Je ne te reverrai plus ! »
      
         
      \bchapter{}
      \begin{verse}
${}^{1}Le Seigneur dit à Moïse : « Pour la dernière fois, je vais frapper Pharaon et l’Égypte. Après cela, non seulement il vous laissera partir, mais il vous renverra définitivement, il vous chassera d’ici. 
${}^{2}Parle donc au peuple : que chaque homme demande à son voisin, et chaque femme à sa voisine, des objets d’argent et des objets d’or. » 
${}^{3}Le Seigneur fit que son peuple trouve grâce aux yeux des Égyptiens. D’ailleurs, en Égypte, Moïse lui-même était un très grand personnage, aux yeux des serviteurs de Pharaon comme aux yeux du peuple.
${}^{4}Alors Moïse dit : « Ainsi parle le Seigneur : Au milieu de la nuit, en plein cœur de l’Égypte, je sortirai 
${}^{5}et, chez les Égyptiens, tous les premiers-nés mourront, aussi bien le premier-né de Pharaon qui siège sur le trône, que le premier-né de la servante qui est derrière la meule, et que tous les premiers-nés du bétail. 
${}^{6}Alors s’élèvera, dans tout le pays d’Égypte, une immense clameur, comme il n’y en eut jamais auparavant, et comme il n’y en aura plus jamais. 
${}^{7}Cependant, chez les fils d’Israël, pas un seul chien ne devra grogner contre qui que ce soit, homme ou bête ; ainsi, vous reconnaîtrez que le Seigneur fait la distinction entre l’Égypte et Israël. 
${}^{8}Alors tous tes serviteurs que voici viendront me trouver et se prosterneront devant moi, en disant : “Sors, toi et tout le peuple qui marche à ta suite !” Après cela, je sortirai. » Et Moïse, enflammé de colère, sortit de chez Pharaon.
${}^{9}Le Seigneur avait dit à Moïse : « Pharaon ne vous écoutera pas, tant et si bien que mes prodiges se multiplieront au pays d’Égypte. » 
${}^{10}Moïse et Aaron avaient accompli toutes sortes de prodiges devant Pharaon ; mais le Seigneur avait fait en sorte que Pharaon s’obstine ; et celui-ci ne laissa pas les fils d’Israël sortir de son pays.
      
         
      \bchapter{}
      \begin{verse}
${}^{1}Dans le pays d’Égypte, le Seigneur dit à Moïse et à son frère Aaron : 
${}^{2} « Ce mois-ci sera pour vous le premier des mois, il marquera pour vous le commencement de l’année. 
${}^{3} Parlez ainsi à toute la communauté d’Israël : le dix de ce mois, que l’on prenne un agneau\\par famille, un agneau par maison. 
${}^{4} Si la maisonnée est trop peu nombreuse pour un agneau, elle le prendra avec son voisin le plus proche, selon le nombre des personnes. Vous choisirez l’agneau d’après ce que chacun peut manger. 
${}^{5} Ce sera une bête sans défaut, un mâle, de l’année. Vous prendrez un agneau ou un chevreau. 
${}^{6} Vous le garderez jusqu’au quatorzième jour du mois. Dans toute l’assemblée de la communauté d’Israël, on l’immolera au coucher du soleil\\. 
${}^{7} On prendra du sang, que l’on mettra sur les deux montants et sur le linteau des maisons où on le mangera. 
${}^{8} On mangera sa chair cette nuit-là, on la mangera rôtie au feu, avec des pains sans levain et des herbes amères. 
${}^{9} Vous n’en mangerez aucun morceau qui soit à moitié cuit ou qui soit bouilli\\ ; tout\\sera rôti au feu, y compris la tête, les jarrets et les entrailles. 
${}^{10} Vous n’en garderez rien pour le lendemain ; ce qui resterait pour le lendemain, vous le détruirez en le brûlant\\. 
${}^{11} Vous mangerez ainsi : la ceinture aux reins, les sandales aux pieds, le bâton à la main. Vous mangerez en toute hâte : c’est la Pâque du Seigneur. 
${}^{12} Je traverserai le pays d’Égypte, cette nuit-là ; je frapperai tout premier-né au pays d’Égypte, depuis les hommes jusqu’au bétail. Contre tous les dieux de l’Égypte j’exercerai mes jugements : Je suis le Seigneur. 
${}^{13} Le sang sera pour vous un signe, sur les maisons où vous serez. Je verrai le sang, et je passerai : vous ne serez pas atteints par le fléau dont je frapperai le pays d’Égypte.
${}^{14}Ce jour-là sera pour vous un mémorial. Vous en ferez pour le Seigneur une fête de pèlerinage. C’est un décret perpétuel : d’âge en âge vous la fêterez.
${}^{15}Pendant sept jours, vous mangerez des pains sans levain. Dès le premier jour, vous ferez disparaître le levain de vos maisons. Et celui qui mangera du pain levé, entre le premier et le septième jour, celui-là sera retranché du peuple d’Israël.
${}^{16}Le premier jour, vous tiendrez une assemblée sainte ; vous ferez de même le septième jour. Ces jours-là, on ne fera aucun travail, sauf pour préparer le repas de chacun ; on ne fera rien d’autre. 
${}^{17}Vous observerez la fête des Pains sans levain car, en ce jour même, j’ai fait sortir vos armées du pays d’Égypte. D’âge en âge, vous observerez ce jour. C’est un décret perpétuel.
${}^{18}Le premier mois, du quatorzième jour au soir jusqu’au vingt et unième jour au soir, vous mangerez du pain sans levain. 
${}^{19}Pendant sept jours, on ne trouvera pas de levain dans vos maisons. Et celui qui mangera du pain levé – qu’il soit immigré ou israélite originaire du pays –, celui-là sera retranché de la communauté d’Israël. 
${}^{20}Vous ne mangerez aucun pain levé. Où que vous habitiez, vous mangerez des pains sans levain. »
${}^{21}Moïse convoqua tous les anciens d’Israël et leur dit : « Prenez un agneau\\par clan et immolez-le pour la Pâque. 
${}^{22}Puis vous prendrez un bouquet d’hysope, vous le tremperez dans le sang que vous aurez recueilli dans un récipient, et vous étendrez le sang\\sur le linteau et les deux montants de la porte. Que nul d’entre vous ne sorte de sa maison avant le matin. 
${}^{23}Ainsi, lorsque le Seigneur traversera l’Égypte pour la frapper, et qu’il verra le sang sur le linteau et les deux montants, il passera cette maison sans permettre à l’Exterminateur d’y entrer pour la frapper. 
${}^{24}Vous observerez cette parole comme un décret perpétuel pour vous et vos fils. 
${}^{25}Quand vous serez entrés dans le pays que le Seigneur vous donnera comme il l’a dit, vous conserverez ce rite. 
${}^{26}Et quand vos fils vous demanderont : “Que signifie pour vous ce rite ?” 
${}^{27}vous répondrez : “C’est le sacrifice de la Pâque en l’honneur du Seigneur : il a passé les maisons des fils d’Israël en Égypte ; lorsqu’il a frappé l’Égypte, il a épargné nos maisons !” »
      Alors, le peuple s’inclina et se prosterna.
${}^{28}Puis, les fils d’Israël s’en allèrent et firent comme le Seigneur l’avait ordonné à Moïse et Aaron.
${}^{29}Au milieu de la nuit, le Seigneur frappa tous les premiers-nés de l’Égypte, du premier-né de Pharaon qui siège sur le trône, jusqu’au premier-né du captif dans sa prison, et tous les premiers-nés du bétail. 
${}^{30}Cette nuit-là, Pharaon se leva, ainsi que tous ses serviteurs et tous les Égyptiens ; et une immense clameur s’éleva en Égypte, car il n’y avait pas une seule maison sans un mort. 
${}^{31}Pharaon convoqua Moïse et Aaron en pleine nuit, et leur dit : « Levez-vous ! Sortez du milieu de mon peuple, vous et les fils d’Israël. Allez ! Servez le Seigneur comme vous l’avez demandé. 
${}^{32}Même votre bétail, le petit et le gros, prenez-le comme vous l’avez demandé, et partez ! Appelez sur moi la bénédiction ! » 
${}^{33}Les Égyptiens pressèrent le peuple d’Israël de quitter le pays au plus vite, car ils se disaient : « Nous allons tous mourir ! » 
${}^{34}Le peuple emporta la pâte avant qu’elle n’ait levé : ils enveloppèrent les pétrins dans leurs manteaux et les mirent sur leurs épaules.
${}^{35}Les fils d’Israël avaient agi selon la parole de Moïse : ils avaient demandé aux Égyptiens des objets d’argent, des objets d’or et des manteaux. 
${}^{36}Le Seigneur fit que son peuple trouve grâce aux yeux des Égyptiens : ils cédèrent à leur demande. Ainsi les fils d’Israël dépouillèrent-ils les Égyptiens.
${}^{37}Les fils d’Israël partirent de la ville de Ramsès en direction de Souccoth, au nombre d’environ six cent mille\\sans compter les enfants. 
${}^{38} Une multitude disparate les accompagnait, ainsi qu’un immense troupeau de moutons et de bœufs. 
${}^{39} Ils firent cuire des galettes sans levain avec la pâte qu’ils avaient emportée d’Égypte et qui n’avait pas levé ; en effet, ils avaient été chassés d’Égypte sans avoir eu le temps de faire des provisions.
${}^{40}Le séjour des fils d’Israël en Égypte avait duré quatre cent trente ans. 
${}^{41} Et c’est au bout de quatre cent trente ans, c’est en ce jour même\\que toutes les armées du Seigneur sortirent du pays d’Égypte. 
${}^{42} Ce fut une nuit de veille pour le Seigneur, quand il fit sortir d’Égypte les fils d’Israël ; ce doit être pour eux, de génération en génération, une nuit de veille en l’honneur du Seigneur.
${}^{43}Le Seigneur dit à Moïse et Aaron : « Voici le rituel pour la Pâque : aucun étranger n’en mangera. 
${}^{44}Tout esclave acquis à prix d’argent, tu le circonciras, et alors il pourra en manger. 
${}^{45}Ni l’hôte, ni le salarié n’en mangeront. 
${}^{46}On la mangera dans une seule maison. Tu ne sortiras de cette maison aucun morceau de viande. Vous ne briserez aucun de ses os. 
${}^{47}Toute la communauté d’Israël observera ce rituel. 
${}^{48}Si un immigré qui réside chez toi veut célébrer la Pâque pour le Seigneur, tous les hommes de sa maison devront être circoncis. Alors il pourra s’approcher pour célébrer ; il sera considéré comme un Israélite originaire du pays. Mais celui qui n’aura pas été circoncis n’en mangera pas. 
${}^{49}La loi sera la même pour l’Israélite de souche et pour l’immigré qui réside chez vous. » 
${}^{50}Tous les fils d’Israël firent comme le Seigneur l’avait ordonné à Moïse et Aaron. Ils firent ainsi. 
${}^{51}C’est en ce jour même que le Seigneur fit sortir du pays d’Égypte les fils d’Israël rangés comme une armée.
      
         
      \bchapter{}
      \begin{verse}
${}^{1}Le Seigneur parla à Moïse. Il dit : 
${}^{2}« Consacre-moi tous les premiers-nés parmi les fils d’Israël, car les premiers-nés des hommes et les premiers-nés du bétail m’appartiennent. »
${}^{3}Moïse dit au peuple : « Souvenez-vous de ce jour, le jour de votre sortie du pays d’Égypte, la maison d’esclavage, car c’est par la force de sa main que le Seigneur vous en a fait sortir. On ne mangera pas de pain levé, ce jour-là. 
${}^{4}C’est aujourd’hui, au mois des Épis, que vous sortez. 
${}^{5}Le Seigneur te fera entrer dans le pays du Cananéen, du Hittite, de l’Amorite, du Hivvite et du Jébuséen, le pays qu’il a juré à tes pères de te donner, le pays ruisselant de lait et de miel. Alors, en ce mois des Épis, tu pratiqueras ce rite-ci. 
${}^{6}Pendant sept jours, tu mangeras des pains sans levain. Et, le septième jour, tu célébreras la fête en l’honneur du Seigneur. 
${}^{7}On mangera du pain sans levain pendant les sept jours. Sur ton territoire tout entier, on ne trouvera pas de pain levé, on ne trouvera même pas de levain. 
${}^{8}Ce jour-là, tu donneras à ton fils cette explication : “C’est en raison de ce que le Seigneur a fait pour moi lors de ma sortie d’Égypte.” 
${}^{9}Ce rite sera pour toi comme un signe sur ta main, comme un mémorial entre tes yeux, afin que la loi du Seigneur soit dans ta bouche ; car, par la force de sa main, le Seigneur t’a fait sortir d’Égypte. 
${}^{10}Tu observeras ce décret au moment prescrit, d’année en année.
${}^{11}Alors, quand le Seigneur t’aura fait entrer dans le pays de Canaan, cette terre qu’il a juré à toi et à tes pères de te donner, 
${}^{12}alors tu remettras au Seigneur tout premier-né : tout premier-né de sexe masculin et tout premier-né mâle du bétail appartiennent au Seigneur. 
${}^{13}Le premier-né des ânes, tu le rachèteras par un mouton. Si tu ne le rachètes pas, tu lui briseras la nuque. Mais chez les hommes, tout fils premier-né, tu le rachèteras.
${}^{14}Alors, demain, quand ton fils te demandera : « Que fais-tu là ? », tu lui répondras : « C’est par la force de sa main que le Seigneur nous a fait sortir d’Égypte, la maison d’esclavage. 
${}^{15}En effet, comme Pharaon multipliait les obstacles pour nous laisser partir, le Seigneur fit mourir tous les premiers-nés au pays d’Égypte, du premier-né des hommes au premier-né du bétail. C’est pourquoi j’offre en sacrifice au Seigneur tous les premiers-nés de sexe mâle ; mais le premier-né de mes fils, je le rachète. » 
${}^{16}Ce rite sera pour toi comme un signe à ton poignet, comme un bandeau sur ton front : c’est par la force de sa main que le Seigneur nous a fait sortir d’Égypte. »
${}^{17}Quand Pharaon laissa partir le peuple, Dieu ne leur fit pas prendre la route du pays des Philistins, bien qu’elle fût la plus directe. Dieu s’était dit : « Il ne faudrait pas qu’à la perspective des combats, le peuple revienne sur sa décision et retourne en Égypte. » 
${}^{18}Dieu fit donc faire au peuple un détour par le désert de la mer des Roseaux. C’est, rangés comme une armée, que les fils d’Israël étaient montés du pays d’Égypte.
${}^{19}Moïse prit avec lui les ossements de Joseph, car celui-ci avait exigé des fils d’Israël un serment solennel, en leur disant : « Dieu ne manquera pas de vous visiter : alors, quand vous remonterez d’Égypte, emportez mes ossements avec vous. »
${}^{20}Ils partirent de Souccoth et campèrent à Étam, en bordure du désert. 
${}^{21}Le Seigneur lui-même marchait à leur tête : le jour dans une colonne de nuée pour leur ouvrir la route, la nuit dans une colonne de feu pour les éclairer ; ainsi pouvaient-ils marcher jour et nuit. 
${}^{22}Le jour, la colonne de nuée ne quittait pas la tête du peuple ; ni, la nuit, la colonne de feu.
      
         
      \bchapter{}
      \begin{verse}
${}^{1}Le Seigneur parla à Moïse. Il dit : 
${}^{2}« Va dire aux fils d’Israël de revenir camper devant Pi-Hahiroth, entre Migdol et la mer, devant Baal-Sefone ; vous camperez juste en face, au bord de la mer. 
${}^{3}Alors Pharaon dira : “Voilà que les fils d’Israël, affolés, errent dans le pays ! Le désert s’est refermé sur eux !” 
${}^{4}Alors, je ferai en sorte que Pharaon s’obstine, et il les poursuivra. Mais je me glorifierai aux dépens de Pharaon et de toute son armée, et les Égyptiens reconnaîtront que je suis le Seigneur. » Les fils d’Israël firent ainsi.
${}^{5}On annonça au roi d’Égypte, que le peuple d’Israël s’était enfui. Alors Pharaon et ses serviteurs changèrent de sentiment envers ce peuple. Ils dirent : « Qu’avons-nous fait en laissant partir Israël : il ne sera plus à notre service ! » 
${}^{6}Pharaon fit atteler son char et rassembler ses troupes ; 
${}^{7}il prit six cents chars d’élite et tous les chars de l’Égypte, chacun avec son équipage. 
${}^{8}Le Seigneur fit en sorte que s’obstine Pharaon, roi d’Égypte, qui se lança à la poursuite des fils d’Israël, tandis que ceux-ci avançaient librement\\. 
${}^{9}Les Égyptiens, tous les chevaux\\, les chars de Pharaon, ses guerriers et son armée, les poursuivirent et les rejoignirent alors qu’ils campaient au bord de la mer, près de Pi-Hahiroth, en face de Baal-Sefone.
${}^{10}Comme Pharaon approchait, les fils d’Israël regardèrent et, voyant les Égyptiens lancés à leur poursuite\\, ils eurent très peur, et ils crièrent vers le Seigneur. 
${}^{11}Ils dirent à Moïse : « L’Égypte manquait-elle de tombeaux, pour que tu nous aies emmenés mourir dans le désert ? Quel mauvais service tu nous as rendu en nous faisant sortir d’Égypte ! 
${}^{12}C’est bien là ce que nous te disions en Égypte : “Ne t’occupe pas de nous, laisse-nous servir les Égyptiens. Il vaut mieux les servir que de mourir dans le désert !” » 
${}^{13}Moïse répondit au peuple : « N’ayez pas peur ! Tenez bon ! Vous allez voir aujourd’hui ce que le Seigneur va faire pour vous sauver ! Car, ces Égyptiens que vous voyez aujourd’hui, vous ne les verrez plus jamais. 
${}^{14}Le Seigneur combattra pour vous, et vous, vous n’aurez rien à faire. »
${}^{15}Le Seigneur dit à Moïse : « Pourquoi crier vers moi ? Ordonne aux fils d’Israël de se mettre en route ! 
${}^{16} Toi, lève ton bâton, étends le bras sur la mer, fends-la en deux, et que les fils d’Israël entrent au milieu de la mer à pied sec\\. 
${}^{17} Et moi, je ferai en sorte que les Égyptiens s’obstinent : ils y entreront derrière eux ; je me glorifierai aux dépens de Pharaon et de toute son armée, de ses chars et de ses guerriers. 
${}^{18} Les Égyptiens sauront que je suis le Seigneur, quand je me serai glorifié aux dépens de Pharaon, de ses chars et de ses guerriers. »
${}^{19}L’ange de Dieu, qui marchait en avant\\d’Israël, se déplaça et marcha à l’arrière. La colonne de nuée se déplaça depuis l’avant-garde et vint se tenir à l’arrière, 
${}^{20} entre le camp des Égyptiens et le camp d’Israël. Cette nuée était à la fois ténèbres et lumière dans la nuit, si bien que, de toute la nuit, ils ne purent se rencontrer.
${}^{21}Moïse étendit le bras sur la mer. Le Seigneur chassa la mer toute la nuit par un fort vent d’est ; il mit la mer à sec, et les eaux se fendirent. 
${}^{22} Les fils d’Israël entrèrent au milieu de la mer à pied sec, les eaux formant une muraille à leur droite et à leur gauche. 
${}^{23} Les Égyptiens les poursuivirent ; tous les chevaux de Pharaon, ses chars et ses guerriers entrèrent derrière eux jusqu’au milieu de la mer.
${}^{24}Aux dernières heures de la nuit\\, le Seigneur observa, depuis la colonne de feu et de nuée, l’armée\\des Égyptiens, et il la frappa de panique\\. 
${}^{25} Il faussa les roues de leurs chars, et ils eurent beaucoup de peine à les conduire. Les Égyptiens s’écrièrent : « Fuyons devant Israël, car c’est le Seigneur qui combat pour eux contre nous\\ ! » 
${}^{26} Le Seigneur dit à Moïse : « Étends le bras sur la mer : que les eaux reviennent sur les Égyptiens, leurs chars et leurs guerriers ! » 
${}^{27} Moïse étendit le bras sur la mer. Au point du jour, la mer reprit sa place ; dans leur fuite, les Égyptiens s’y heurtèrent, et le Seigneur les précipita au milieu de la mer. 
${}^{28} Les eaux refluèrent et recouvrirent les chars et les guerriers, toute l’armée de Pharaon qui était entrée dans la mer à la poursuite d’Israël\\. Il n’en resta pas un seul. 
${}^{29} Mais les fils d’Israël avaient marché à pied sec au milieu de la mer, les eaux formant une muraille à leur droite et à leur gauche.
${}^{30}Ce jour-là, le Seigneur sauva Israël de la main de l’Égypte, et Israël vit les Égyptiens morts sur le bord de la mer. 
${}^{31} Israël vit avec quelle main puissante le Seigneur avait agi contre l’Égypte. Le peuple craignit le Seigneur, il mit sa foi dans le Seigneur et dans son serviteur Moïse.
      <p class="cantique" id="bib_ct-at_1"><span class="cantique_label">Cantique AT 1</span> = <span class="cantique_ref"><a class="unitex_link" href="#bib_ex_15_1">Ex 15, 1-4a.8-13.17-18</a></span>
      
         
      \bchapter{}
      \begin{verse}
${}^{1}Alors Moïse et les fils d’Israël chantèrent ce cantique au Seigneur\\ :
      
         
       
        \\« Je chanterai pour le Seigneur ! Éclatante est sa gloire :
        \\il a jeté dans la mer cheval et cavalier !
         
        ${}^{2}Ma force et mon chant\\, c’est le Seigneur :
        il est pour moi le salut.
        \\Il est mon Dieu, je le célèbre ;
        j’exalte le Dieu de mon père.
         
        ${}^{3}Le Seigneur est le guerrier des combats ;
        \\son nom est « Le Seigneur ».
         
        ${}^{4}Les chars du Pharaon et ses armées, il les lance dans la mer.
        \\\[L’élite de leurs chefs a sombré dans la mer Rouge\\.
         
        ${}^{5}L’abîme les recouvre :
        \\ils descendent, comme la pierre, au fond des eaux.
         
        ${}^{6}Ta droite, Seigneur, magnifique en sa force,
        \\ta droite, Seigneur, écrase l’ennemi.
         
${}^{7}La grandeur de ta majesté brise tes adversaires :
        \\tu envoies ta colère qui les brûle comme un chaume.\]
         
        ${}^{8}Au souffle de tes narines, les eaux s’amoncellent :
        \\comme une digue, se dressent les flots ;
        \\les abîmes se figent au cœur de la mer.
         
        ${}^{9}L’ennemi disait : « Je poursuis, je domine,
        \\je partage le butin, je m’en repais ;
        \\je tire mon épée : je prends les dépouilles ! »
         
        ${}^{10}Tu souffles ton haleine : la mer les recouvre ;
        \\comme du plomb, ils s’abîment dans les eaux redoutables.
         
        ${}^{11}Qui est comme toi parmi les dieux, Seigneur ?
        \\Qui est comme toi, magnifique en sainteté,
        \\terrible en ses exploits, auteur de prodiges ?
         
        ${}^{12}Tu étends ta main droite : la terre les avale.
        ${}^{13}Par ta fidélité\\tu conduis\\ce peuple que tu as racheté ;
        \\tu les guides par ta force vers ta sainte demeure.
         
${}^{14}\[Les peuples ont entendu : ils tremblent ;
        \\les douleurs ont saisi les habitants de Philistie.
${}^{15}Les princes d’Édom sont pris d’effroi.
         
        \\Un tremblement a saisi les puissants de Moab ;
        \\tous les habitants de Canaan sont terrifiés,
${}^{16}la peur et la terreur tombent sur eux.
         
        \\Sous la vigueur de ton bras, ils se taisent, pétrifiés,
        \\pendant que ton peuple passe, Seigneur,
        \\que passe le peuple acquis par toi.\]
         
        ${}^{17}Tu les amènes, tu les plantes sur la montagne, ton héritage,
        \\le lieu que tu as fait, Seigneur, pour l’habiter,
        \\le sanctuaire, Seigneur\\, fondé par tes mains.
         
        ${}^{18}Le Seigneur régnera pour les siècles des siècles. »
       
${}^{19}Le cheval de Pharaon, ses chars et ses guerriers étaient entrés dans la mer, et le Seigneur avait fait revenir sur eux les eaux de la mer. Mais les fils d’Israël, eux, avaient marché à pied sec au milieu de la mer.
${}^{20}La prophétesse Miryam, sœur d’Aaron, saisit un tambourin, et toutes les femmes la suivirent, dansant et jouant du tambourin. 
${}^{21}Et Miryam leur entonna :
        \\« Chantez pour le Seigneur ! Éclatante est sa gloire :
        \\il a jeté dans la mer cheval et cavalier ! »
