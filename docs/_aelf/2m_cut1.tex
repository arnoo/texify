  
  
    
    \bbook{DEUXIÈME LIVRE DES MARTYRS D’ISRAËL}{DEUXIÈME LIVRE DES MARTYRS D’ISRAËL}
      
         
      \bchapter{}
      \begin{verse}
${}^{1}« Aux frères juifs qui sont en Égypte, salut ! Leurs frères juifs qui sont à Jérusalem et dans le pays de Judée leur souhaitent paix et prospérité. 
${}^{2}Que Dieu vous comble de bienfaits ; qu’il se souvienne de son alliance en faveur d’Abraham, d’Isaac et de Jacob, ses fidèles serviteurs ! 
${}^{3}Qu’il vous donne à tous un cœur pour l’adorer, pour accomplir ses volontés généreusement et de plein gré ! 
${}^{4}Qu’il ouvre votre cœur à sa Loi et à ses décrets ; qu’il établisse la paix ! 
${}^{5}Qu’il exauce vos demandes, se réconcilie avec vous et ne vous délaisse pas au temps du malheur ! 
${}^{6}Telle est la prière que nous formulons pour vous ici, en ce moment. 
${}^{7}Sous le règne de Démétrios, en l’année 169 de l’empire grec, nous, les Juifs, nous vous avons écrit ceci : “Au plus fort de la détresse qui s’est abattue sur nous durant ces années-là, alors que Jason et ses compagnons avaient trahi la Terre sainte et le royaume, 
${}^{8}mis le feu au portail du Temple et répandu le sang des innocents, nous avons imploré le Seigneur et il nous a exaucés. Nous avons offert des sacrifices et de la fleur de farine, allumé les lampes et présenté les pains.” 
${}^{9}Et maintenant, nous vous invitons à célébrer les jours de la fête des Tentes du mois de Kisléou. 
${}^{10}Écrit en l’année 188 de l’empire grec. »
      
         
      « Les habitants de Jérusalem et de la Judée, le Conseil des anciens et Judas, à Aristobule, le précepteur du roi Ptolémée, issu de la lignée des prêtres consacrés, ainsi qu’aux Juifs d’Égypte : salut et bonne santé ! 
${}^{11}Sauvés par Dieu de graves périls, nous le remercions grandement, nous qui combattons contre le roi, 
${}^{12}car Dieu lui-même a vaincu ceux qui voulaient attaquer la Ville sainte. 
${}^{13}En effet, leur chef, qui s’était rendu en Perse avec une armée apparemment invincible, fut mis en pièces dans le temple de la déesse Nanéa, victime d’un stratagème de ses prêtres. 
${}^{14}Sous le prétexte d’épouser la déesse, Antiocos s’était rendu dans ce lieu, avec ses amis, afin d’en prendre les très grandes richesses à titre de dot. 
${}^{15}Les prêtres du sanctuaire de Nanéa les avaient étalées devant lui. Antiocos lui-même pénétra avec quelques hommes dans l’enceinte du lieu de culte. Mais dès qu’il fut entré, les prêtres fermèrent le temple, 
${}^{16}ouvrirent la porte secrète du plafond et abattirent le chef et ses hommes en lançant des pierres. Puis ils dépecèrent les corps, coupèrent les têtes et les jetèrent à ceux qui étaient dehors. 
${}^{17}Béni soit notre Dieu en toutes choses, lui qui a livré les impies à la mort !
${}^{18}Comme nous allons bientôt célébrer la purification du Temple, le vingt-cinq du mois de Kisléou, nous avons estimé devoir vous en informer, afin que vous la célébriez, vous aussi, à la manière de la fête des Tentes, et en souvenir du feu qui se manifesta quand Néhémie, après avoir rebâti le Temple et l’autel, offrit des sacrifices. 
${}^{19}En effet, lorsque nos pères furent emmenés en Perse, les prêtres d’alors, remplis de piété, prirent du feu de l’autel et le cachèrent secrètement dans la cavité d’un puits qui se trouvait à sec. Ils l’y mirent en sécurité de manière à ce que l’endroit demeure ignoré de tous. 
${}^{20}Bien des années plus tard, au moment choisi par Dieu, Néhémie, envoyé par le roi de Perse, fit rechercher ce feu par les descendants des prêtres qui l’avaient caché. Ceux-ci informèrent Néhémie qu’ils n’avaient pas trouvé de feu, mais plutôt un liquide épais, et Néhémie leur ordonna d’en puiser et d’en rapporter. 
${}^{21}Quand on eut tout préparé pour les sacrifices, Néhémie ordonna aux prêtres de répandre ce liquide sur le bois et sur ce que l’on y avait déposé. 
${}^{22}Après cela, il se passa un peu de temps. Le soleil, d’abord caché par les nuages, se mit à briller. Alors, un grand brasier s’alluma, à la stupéfaction de tous. 
${}^{23}Pendant que le sacrifice se consumait, les prêtres prononcèrent une prière et, avec les prêtres, tous ceux qui étaient présents. Jonathan commençait, et les autres, de même que Néhémie, joignaient leurs voix à la sienne. 
${}^{24}Cette prière était ainsi formulée :
       
        \\Seigneur, Seigneur Dieu, créateur de toutes choses,
        \\redoutable et fort, juste et miséricordieux,
        \\toi, le seul roi, le seul bon,
${}^{25}le seul généreux, le seul juste,
        \\tout-puissant et éternel,
        \\tu sauves Israël de tout mal,
        \\tu as fait de nos pères tes élus
        \\et tu les as sanctifiés.
${}^{26}Accepte ce sacrifice offert pour tout ton peuple, Israël ;
        \\protège ta part d’héritage, sanctifie-la.
${}^{27}Rassemble nos frères dispersés,
        \\libère ceux qui sont esclaves parmi les nations,
        \\jette les yeux sur ceux que l’on méprise
        \\et dont on se détourne avec dégoût.
        \\Ainsi, les nations reconnaîtront que toi, tu es notre Dieu.
${}^{28}Tourmente ceux qui nous oppriment,
        \\qui nous écrasent de leur orgueil plein d’arrogance.
${}^{29}Enracine ton peuple dans ton Lieu saint,
        \\comme l’a dit Moïse.
       
${}^{30}Les prêtres, alors, chantèrent les hymnes au son des harpes. 
${}^{31}Quand ce qui était offert en sacrifice fut entièrement consumé, Néhémie ordonna de répandre encore le reste de l’eau sur de grandes pierres. 
${}^{32}Après cela, une flamme s’alluma, mais son éclat fut absorbé par la lumière qui rayonnait en face, en provenance de l’autel. 
${}^{33}L’événement fut bientôt connu. On rapporta au roi des Perses qu’à l’endroit où les prêtres exilés avaient caché le feu sacré, était apparue une eau avec laquelle Néhémie et ceux qui l’entouraient avaient sanctifié par le feu ce qui était offert en sacrifice. 
${}^{34}Après avoir vérifié le fait, le roi fit clôturer cet endroit et le rendit sacré. 
${}^{35}À ceux qui avaient sa faveur, le roi donnait une part des grands revenus qu’il en retirait. 
${}^{36}Néhémie et ceux qui l’entouraient appelèrent cette eau “nephtar”, ce qui signifie “purification”. Mais on l’appelle généralement “naphte”.
      
         
      \bchapter{}
      \begin{verse}
${}^{1}On peut lire dans les archives que le prophète Jérémie donna l’ordre aux déportés d’emporter du feu sacré, comme on vient de l’indiquer. 
${}^{2}Après avoir donné aux déportés un exemplaire de la Loi, le prophète leur recommanda aussi de ne pas oublier les décrets du Seigneur et de ne pas laisser leurs esprits s’égarer au spectacle des statues d’or et d’argent, revêtues de leur parure. 
${}^{3}Parmi d’autres conseils du même genre, il les encouragea à ne pas laisser la Loi s’écarter de leur cœur. 
${}^{4}Ce document racontait aussi comment le prophète, averti par un oracle, avait ordonné que la Tente et l’Arche l’accompagnent, lorsqu’il se rendit à la montagne que Moïse avait gravie pour contempler l’héritage promis par Dieu. 
${}^{5}Arrivé là, Jérémie trouva un site caverneux. Il y introduisit la Tente, l’Arche et l’autel de l’encens, puis il en obstrua l’accès. 
${}^{6}Quelques-uns de ceux qui l’avaient accompagné revinrent pour marquer de signes le chemin, mais ils ne purent le retrouver. 
${}^{7}Quand Jérémie l’apprit, il leur fit des reproches et leur dit : “Ce lieu restera inconnu, jusqu’à ce que Dieu ait accompli le rassemblement de son peuple et lui ait montré sa miséricorde. 
${}^{8}Alors, le Seigneur fera voir de nouveau ces objets ; alors, la gloire du Seigneur se manifestera, ainsi que la nuée, comme elle se montrait au temps de Moïse et lorsque Salomon adressa une supplication pour que le Lieu saint soit magnifiquement consacré.” 
${}^{9}Le document rapportait aussi comment Salomon, cet homme plein de sagesse, offrit un sacrifice pour la dédicace et l’achèvement du Temple. 
${}^{10}De même que Moïse avait prié le Seigneur et obtenu que le feu tombe du ciel pour dévorer ce qui était offert en sacrifice, de même Salomon pria, et le feu venu d’en haut consuma les holocaustes. 
${}^{11}Moïse avait dit : “C’est parce qu’il n’a pas été mangé que le sacrifice offert pour le péché a été consumé par le feu.” 
${}^{12}Comme Moïse l’avait fait, Salomon prolongea la fête pendant huit jours.
${}^{13}Les mêmes faits étaient relatés dans les archives et les mémoires de Néhémie, où l’on racontait en outre comment celui-ci constitua une bibliothèque, en rassemblant les livres concernant les rois et les prophètes, les œuvres de David et les lettres des rois concernant les dons. 
${}^{14}De la même façon, Judas, lui aussi, a rassemblé tous les livres dispersés à cause de la guerre qu’on nous a faite. Ils sont maintenant à notre disposition. 
${}^{15}Donc, si vous en avez besoin, envoyez des gens qui vous les rapporteront. 
${}^{16}Nous allons bientôt célébrer la purification du Temple, et c’est pour cela que nous vous écrivons : il serait bon que vous célébriez, vous aussi, les jours de cette fête. 
${}^{17}C’est Dieu qui a sauvé tout son peuple et qui a donné à tous l’héritage, la royauté, le sacerdoce et la sanctification, 
${}^{18}comme il l’avait promis par la Loi. Ce Dieu, nous l’espérons, aura très vite pitié de nous et nous rassemblera de toutes les régions qui sont sous le ciel vers le Lieu saint. Car il nous a déjà arrachés à de grands malheurs et a purifié le Lieu saint. »
      <h2 class="intertitle hmbot" id="d85e122507">1. Préface de l’auteur (2,19-32)</h2>
${}^{19}Je vais vous faire le récit des événements survenus au temps de Judas Maccabée et de ses frères : la purification du Temple magnifique et la dédicace de l’autel, 
${}^{20}les guerres soutenues contre Antiocos Épiphane et contre son fils Eupator, 
${}^{21}les manifestations célestes qui se produisirent en faveur de ceux qui rivalisèrent d’exploits pour le judaïsme. Ceux-ci, en effet, malgré leur petit nombre, pillèrent toute la région et pourchassèrent les hordes barbares, 
${}^{22}reconquirent le Temple célèbre dans le monde entier, délivrèrent la ville et rétablirent les lois menacées d’abolition, grâce à l’entière bienveillance du Seigneur qui leur était favorable. 
${}^{23}Tout cela, Jason de Cyrène l’a exposé en cinq livres, que nous allons tenter de résumer en un seul ouvrage. 
${}^{24}En effet, devant le flot des chiffres et la difficulté qu’éprouvent ceux qui veulent pénétrer dans le récit détaillé de cette histoire, à cause de l’abondance de la matière, 
${}^{25}nous avons eu le souci de faire une œuvre agréable pour qui aime lire, et commode pour qui préfère retenir les choses par cœur, une œuvre utile à tous les lecteurs, quels qu’ils soient. 
${}^{26}Mais pour nous qui avons entrepris ce pénible travail de résumé, la tâche, loin d’être aisée, nous coûta sueur et veilles ; 
${}^{27}elle fut comparable à la délicate mission de celui qui prépare un banquet et recherche le bien-être des autres. Pourtant, en raison de la reconnaissance d’un grand nombre, nous supporterons volontiers ce pénible travail. 
${}^{28}Laissant à l’auteur des cinq livres le soin d’entrer dans les détails de chaque événement, nous nous efforcerons de tracer les grandes lignes du résumé. 
${}^{29}L’architecte d’une maison neuve doit surveiller l’ensemble de la construction, tandis que le peintre et le décorateur doivent rechercher ce qui convient à l’ornementation ; il en va de même pour nous, me semble-t-il. 
${}^{30}Pénétrer dans le sujet, faire le tour des questions, examiner avec soin tous les détails, cela revient à celui qui, le premier, écrit l’histoire ; 
${}^{31}mais celui qui en fait le résumé doit rechercher la concision du récit et renoncer à traiter le sujet de façon trop minutieuse. 
${}^{32}Commençons donc ici notre récit, sans rien ajouter de plus à ce qui vient d’être dit, car il serait absurde d’être prolixe dans les préambules de l’histoire, mais concis dans l’histoire elle-t même.
      <h2 class="intertitle" id="d85e122581">2. Expédition d’Héliodore contre le temple de Jérusalem (3 – 4,6)</h2>
      
         
      \bchapter{}
      \begin{verse}
${}^{1}Les habitants de la Ville sainte jouissaient d’une paix totale ; on y observait au mieux les lois, grâce à la piété du grand prêtre Onias et à sa haine du mal. 
${}^{2}À cette époque, les rois eux-mêmes en vinrent à honorer le Lieu saint et à rehausser la gloire du Temple par les dons les plus magnifiques. 
${}^{3}Séleucos, roi d’Asie, couvrait lui-même de ses revenus personnels toutes les dépenses exigées par la liturgie des sacrifices. 
${}^{4}Or, un certain Simon, de la tribu de Bilga, qui avait été nommé administrateur du Temple, se trouva en désaccord avec le grand prêtre, au sujet de la surveillance des marchés de la ville. 
${}^{5}Comme il ne pouvait l’emporter sur Onias, il alla trouver Apollonios, fils de Thraséas, qui était à cette époque le gouverneur militaire de Cœlé-Syrie et de Phénicie. 
${}^{6}Il le mit au courant du fait que le trésor de Jérusalem regorgeait de richesses inouïes, au point qu’on ne pouvait en calculer la somme, et qu’elles étaient sans proportion avec le budget requis pour les sacrifices. Il ajouta qu’il était possible de les faire tomber en la possession du roi.
      
         
${}^{7}Lors d’une entrevue avec le roi, Apollonios l’informa des richesses dont on lui avait dénoncé l’existence. Le roi désigna Héliodore, qui était à la tête de ses affaires. Il l’envoya avec l’ordre de procéder à l’enlèvement des richesses indiquées. 
${}^{8}Aussitôt, Héliodore fit le voyage, officiellement pour visiter les villes de Cœlé-Syrie et de Phénicie, mais en fait pour exécuter le mandat du roi. 
${}^{9}Arrivé à Jérusalem, il fut reçu avec bienveillance par le grand prêtre et par la ville. Il communiqua ce dont on l’avait informé et il exposa la raison de sa présence. Il désirait savoir, en effet, si tout cela correspondait bien à la réalité. 
${}^{10}Le grand prêtre lui expliqua que le trésor contenait les dépôts de veuves et d’orphelins, 
${}^{11}ainsi qu’une somme appartenant à Hyrcan, fils de Tobie, personnage qui occupait une situation très élevée. Contrairement aux allégations fallacieuses de l’impie Simon, l’ensemble ne comprenait que quatre cents talents d’argent et deux cents talents d’or. 
${}^{12}D’ailleurs, ajouta-t-il, il était absolument inconcevable de léser ceux qui avaient mis leur confiance dans la sainteté de ce lieu, dans le caractère sacré et inviolable du Temple vénéré dans le monde entier.
${}^{13}Mais Héliodore, en raison des ordres qu’il avait reçus du roi, soutenait absolument que ces richesses devaient être confisquées au profit du trésor royal. 
${}^{14}Au jour fixé par lui, il entra pour dresser l’inventaire de ces richesses. Grande fut l’angoisse qui se répandit alors dans toute la ville. 
${}^{15}Les prêtres, prosternés devant l’autel en habits sacerdotaux, invoquaient le Ciel, lui qui avait institué la loi sur les dépôts, pour qu’il garde intacts les biens de ceux qui les avaient mis en dépôt. 
${}^{16}À voir l’aspect du grand prêtre, on ne pouvait manquer d’être profondément blessé, tant son apparence et l’altération de son teint trahissaient l’angoisse de son âme. 
${}^{17}La frayeur dont cet homme était envahi et le tremblement de son corps manifestaient clairement à ceux qui le regardaient la souffrance intime de son cœur. 
${}^{18}Devant la profanation qui menaçait le Lieu saint, les gens se précipitaient en foule hors des maisons où ils se trouvaient, pour s’unir dans une supplication commune. 
${}^{19}Les femmes, enveloppées d’une toile à sac serrée au-dessous de la poitrine, se répandaient dans les rues. Quant aux jeunes filles, habituellement retenues à l’intérieur, les unes couraient vers les portails, d’autres sur les murailles, d’autres encore se penchaient aux fenêtres. 
${}^{20}Toutes faisaient monter leur imploration, les mains tendues vers le Ciel. 
${}^{21}C’était pitié de voir la confusion de cette foule prostrée, et l’extrême angoisse dans laquelle attendait le grand prêtre. 
${}^{22}Tandis qu’on invoquait le Seigneur tout-puissant pour qu’il garde intacts, en toute sécurité, les dépôts de ceux qui les avaient confiés au Temple, 
${}^{23}Héliodore, lui, exécutait ce qui avait été décidé.
${}^{24}Mais à l’endroit précis où il se trouvait déjà, avec ses gardes, près de la salle du trésor, le Souverain des esprits célestes et de toute autorité se manifesta avec un tel éclat que tous ceux qui avaient eu l’audace d’entrer, frappés par la force de Dieu, défaillirent d’épouvante. 
${}^{25}Un cheval leur apparut, monté par un redoutable cavalier et orné d’un harnachement somptueux. S’élançant avec impétuosité, il projetait les sabots antérieurs vers Héliodore. L’homme qui le chevauchait paraissait avoir une armure d’or. 
${}^{26}En même temps lui apparurent deux autres jeunes gens, d’une force extraordinaire, éclatants de beauté et magnifiquement vêtus. Se plaçant de part et d’autre d’Héliodore, ils le flagellaient sans relâche, lui portant une grêle de coups. 
${}^{27}Subitement, Héliodore fut terrassé et environné d’épaisses ténèbres ; on le ramassa pour le mettre sur une civière. 
${}^{28}Cet homme, qui venait de pénétrer dans la salle du trésor avec une escorte nombreuse et toute sa garde, n’était plus d’aucun secours à lui-même ; on l’emporta, en reconnaissant ouvertement le pouvoir de Dieu. 
${}^{29}Lui, par l’action de la force divine, gisait sans voix, privé de tout espoir de salut ; 
${}^{30}les autres bénissaient le Seigneur qui avait grandement glorifié son Lieu saint. Le Temple qui, un instant auparavant, était rempli de frayeur et de trouble, débordait de joie et d’allégresse, grâce à la manifestation du Seigneur tout-puissant. 
${}^{31}Bien vite, quelques-uns des compagnons d’Héliodore supplièrent Onias d’invoquer le Très-Haut et d’obtenir la grâce de la vie pour cet homme qui gisait là et en était à son tout dernier souffle.
${}^{32}Le grand prêtre, dans la crainte que le roi ne soupçonne les Juifs d’avoir commis un mauvais coup contre Héliodore, offrit un sacrifice pour le salut de cet homme. 
${}^{33}Or, tandis qu’il accomplissait le sacrifice d’expiation, les mêmes jeunes gens, revêtus des mêmes habits, apparurent une seconde fois à Héliodore. Ils se tinrent près de lui et lui dirent : « Rends pleinement grâce à Onias, le grand prêtre, car c’est à cause de lui que le Seigneur t’a accordé la grâce de vivre. 
${}^{34}Toi qui as été flagellé par le Ciel, proclame devant tous le pouvoir grandiose de Dieu. » Après ces paroles, ils disparurent.
${}^{35}Héliodore présenta un sacrifice au Seigneur et adressa de ferventes prières à Celui qui lui avait conservé la vie. Puis il prit congé d’Onias et revint avec son armée auprès du roi. 
${}^{36}À tous, il rendait témoignage des œuvres du Dieu très grand, œuvres qu’il avait contemplées de ses yeux. 
${}^{37}Lorsque le roi lui demanda quel genre d’homme il conviendrait d’envoyer une fois encore à Jérusalem, Héliodore répondit : 
${}^{38}« Si tu as quelque ennemi ou conspirateur contre les affaires publiques, envoie-le là-bas, et il te reviendra roué de coups, si du moins il en réchappe, car il y a vraiment une force divine autour du Lieu saint. 
${}^{39}En effet, Celui qui a sa demeure dans le ciel veille lui-même sur ce lieu et le protège : ceux qui s’en approchent avec des intentions mauvaises, il les frappe et les fait périr. » 
${}^{40}C’est ainsi que se déroulèrent les événements concernant Héliodore et la préservation du trésor.
      
         
      \bchapter{}
      \begin{verse}
${}^{1}Simon, dont il a déjà été question, était devenu celui qui dénonçait les richesses de sa patrie. Il continuait à calomnier Onias, en prétendant que c’était lui qui avait assailli Héliodore et provoqué ce malheur. 
${}^{2}Il osait présenter le bienfaiteur de la ville, le protecteur des gens de sa nation, l’ardent défenseur des lois, comme un conspirateur contre les affaires publiques. 
${}^{3}Cette haine alla si loin que même des meurtres furent commis par l’un des partisans de Simon. 
${}^{4}Considérant l’importance de cette rivalité, et constatant qu’Apollonios, fils de Ménesthée, gouverneur militaire de Cœlé-Syrie et de Phénicie, ne faisait qu’accroître la malveillance de Simon, 
${}^{5}Onias se rendit chez le roi, non comme accusateur de ses concitoyens, mais pour veiller à l’intérêt général de tout le peuple et de chacun en particulier. 
${}^{6}Il voyait bien, en effet, que sans une intervention royale, il était impossible de rétablir la paix publique, et que Simon ne mettrait pas un terme à sa folie.
      
         
      <h2 class="intertitle" id="d85e122869">3. Les débuts de l’hellénisation sous Antiocos Épiphane (4,7 – 7)</h2>
${}^{7}Après la mort de Séleucos et l’accès au trône d’Antiocos surnommé Épiphane, Jason, le frère d’Onias, usurpa la charge de grand prêtre. 
${}^{8}Au cours d’une entrevue avec le roi, il lui promit trois cent soixante talents d’argent de l’impôt, ainsi que quatre-vingts talents prélevés sur quelque autre revenu. 
${}^{9}Il s’engageait en outre à payer cent cinquante autres talents, si on lui accordait le pouvoir de fonder un gymnase et un lieu d’éducation pour les jeunes gens, et de recenser les partisans de l’hellénisme à Jérusalem. 
${}^{10}Le roi consentit, et Jason s’empara du pouvoir. Aussitôt, il entreprit de faire adopter le mode de vie des Grecs par ses frères de race. 
${}^{11}Il supprima les mesures de bienveillance prises par les rois en faveur des Juifs. Ces mesures avaient été obtenues par l’entremise de Jean, le père de cet Eupolème qui conduisit l’ambassade pour obtenir amitié et alliance avec les Romains. Jason détruisit les institutions légitimes et inaugura des usages contraires à la Loi. 
${}^{12}Il se fit en effet un plaisir de fonder un gymnase au pied même de l’acropole et de mener l’élite des jeunes gens aux exercices du gymnase. 
${}^{13}En raison de l’extrême perversité de ce Jason, impie et pas grand prêtre du tout, l’hellénisme atteignit une telle vigueur, et la mode étrangère un tel degré, 
${}^{14}que les prêtres ne montraient plus aucun empressement pour le service liturgique de l’autel. Au contraire, méprisant le Temple et négligeant les sacrifices, ils se hâtaient, dès le signal du gong, de prendre part dans la palestre à l’organisation des exercices prohibés par la Loi. 
${}^{15}Ce faisant, ils comptaient pour rien les valeurs honorées par leurs pères, mais portaient la plus haute estime aux gloires de la culture hellénique. 
${}^{16}Pour ces raisons, ils se trouvèrent par la suite dans une situation difficile, et ceux-là mêmes dont ils enviaient les façons de vivre et qu’ils voulaient imiter en tout, devinrent pour eux des ennemis et des bourreaux. 
${}^{17}Car on ne viole pas à la légère les lois divines, comme le démontrera la période suivante.
${}^{18}Comme on célébrait tous les quatre ans à Tyr des jeux en présence du roi, 
${}^{19}l’infâme Jason envoya comme délégués de Jérusalem un groupe de partisans de l’hellénisme, apportant avec eux trois cents pièces d’argent, pour faire un sacrifice à Héraclès. Mais ceux-là mêmes qui les apportaient jugèrent qu’il ne convenait pas de les utiliser pour le sacrifice et décidèrent de les réserver pour une autre dépense. 
${}^{20}C’est ainsi que l’argent, destiné au sacrifice d’Héraclès par celui qui l’avait envoyé, fut affecté à la construction de navires, après l’intervention de ceux qui l’apportaient.
${}^{21}Comme Apollonios, fils de Ménesthée, avait été envoyé en Égypte pour l’intronisation du roi Philométor, Antiocos apprit que ce dernier était hostile à sa politique. Il se préoccupa dès lors d’assurer sa propre sécurité. C’est ce qui le conduisit à Joppé, d’où il se rendit à Jérusalem. 
${}^{22}Magnifiquement reçu par Jason et par la ville, il y fut introduit à la lumière des flambeaux et au milieu des acclamations. Après quoi, il partit installer en Phénicie le camp de son armée.
${}^{23}Jason était grand prêtre depuis trois ans lorsqu’il envoya Ménélas, le frère du Simon dont il a été question plus haut, porter l’argent au roi et mener à terme des affaires urgentes restées en suspens. 
${}^{24}Une fois admis en présence du roi, Ménélas l’aborda avec les manières d’un personnage important et se fit attribuer à lui-même la fonction de grand prêtre, en offrant trois cents talents de plus que n’avait offerts Jason. 
${}^{25}Puis il revint, muni des lettres royales d’investiture. Il n’était en rien digne de la fonction de grand prêtre ; au contraire, il n’avait en lui que les rages d’un tyran cruel et les fureurs d’une bête sauvage. 
${}^{26}Ainsi Jason, qui avait supplanté son propre frère, fut supplanté à son tour par un autre, et contraint de gagner en fugitif le pays des Ammonites. 
${}^{27}Quant à Ménélas, il détenait certes le pouvoir, mais ne s’acquittait en rien des sommes d’argent qu’il avait promises au roi ; 
${}^{28}et cela, malgré les réclamations de Sostrate, gouverneur de l’acropole, qui était chargé de percevoir les impôts. Pour cette raison, ils furent tous deux convoqués par le roi. 
${}^{29}Ménélas laissa pour le remplacer comme grand prêtre son propre frère Lysimaque, tandis que Sostrate laissait Kratès, le chef des mercenaires chypriotes.
${}^{30}Sur ces entrefaites, il arriva que les habitants de Tarse et de Mallos provoquèrent des émeutes, parce que leurs villes avaient été données en présent à Antiochis, la concubine du roi. 
${}^{31}En toute hâte, le roi alla donc régler cette affaire, laissant pour le remplacer Andronicos, l’un des grands dignitaires. 
${}^{32}Convaincu de tenir une occasion favorable, Ménélas déroba quelques objets d’or du Temple, et en fit cadeau à Andronicos ; il parvint également à en vendre d’autres à Tyr et aux villes voisines. 
${}^{33}L’ayant appris de bonne source, Onias, retiré dans le lieu d’asile à Daphné, près d’Antioche, lui en faisait reproche. 
${}^{34}Dès lors, Ménélas, prenant à part Andronicos, l’incitait à supprimer Onias. Andronicos se rendit alors auprès d’Onias : confiant dans sa propre ruse, il lui tendit la main droite en s’engageant sous serment, et malgré le soupçon qu’il inspirait, il le décida à sortir de son asile. Alors, aussitôt, il le mit à mort, au mépris de toute justice. 
${}^{35}Pour cette raison, non seulement les Juifs, mais aussi beaucoup de gens parmi les autres nations furent indignés et choqués du meurtre injuste de cet homme.
${}^{36}Lorsque le roi fut revenu des régions de Cilicie à Antioche, les Juifs de la ville, en compagnie des Grecs qui partageaient leur haine du mal, vinrent le trouver au sujet du meurtre injustifié d’Onias. 
${}^{37}Antiocos, affligé jusqu’au fond de l’âme et saisi de pitié, versa des larmes au souvenir de la sagesse et de la conduite exemplaire du défunt. 
${}^{38}Puis, enflammé de colère, il dépouilla aussitôt Andronicos de la pourpre, lui déchira les vêtements et le fit mener, à travers toute la ville, à l’endroit même où il avait porté une main sacrilège sur Onias. Là, il exécuta le meurtrier. Le Seigneur lui infligea ainsi le châtiment qu’il méritait.
${}^{39}Or, à Jérusalem, un grand nombre de vols sacrilèges avaient été commis par Lysimaque, avec l’accord de Ménélas, et le bruit s’en était répandu au-dehors. Le peuple s’attroupa contre Lysimaque, alors que beaucoup d’objets d’or avaient déjà été emportés de tous côtés. 
${}^{40}Comme la foule se soulevait, débordante de colère, Lysimaque arma près de trois mille hommes et donna le signal d’injustes violences, sous le commandement d’un certain Auranos, un homme avancé en âge et non moins en folie. 
${}^{41}Prenant conscience de la manœuvre de Lysimaque, les uns s’armaient de pierres, d’autres de gourdins, certains ramassaient à poignées la poussière du sol et assaillaient les gens de Lysimaque en une mêlée confuse. 
${}^{42}C’est ainsi qu’ils en blessèrent un grand nombre, en tuèrent même quelques-uns, et mirent tous les autres en fuite. Quant au voleur sacrilège lui-même, ils le supprimèrent près de la salle du trésor.
${}^{43}Un procès fut intenté à Ménélas à propos de ces faits. 
${}^{44}Lorsque le roi vint à Tyr, les trois hommes envoyés par le Conseil des anciens plaidèrent leur cause en sa présence. 
${}^{45}Se voyant déjà perdu, Ménélas promit des sommes importantes à Ptolémée, fils de Dorymène, pour emporter la conviction du roi. 
${}^{46}Ptolémée entraîna donc le roi à l’écart, sous un péristyle, comme pour prendre le frais, et le fit changer d’avis, 
${}^{47}si bien que Ménélas, le responsable de tout ce mal, fut renvoyé libre des accusations portées contre lui. En revanche, le roi condamna à mort les malheureux accusateurs de Ménélas qui, même s’ils avaient plaidé devant des barbares, auraient été renvoyés innocents. 
${}^{48}Eux, qui avaient pris la défense de la ville, du peuple et des objets sacrés, subirent sans tarder cette peine injuste. 
${}^{49}Aussi vit-on même des habitants de Tyr, horrifiés par un tel méfait, pourvoir magnifiquement à leur sépulture. 
${}^{50}Quant à Ménélas, il se maintint au pouvoir, grâce à la cupidité des puissants. Sa méchanceté ne fit que croître, et il devint le principal adversaire de ses concitoyens.
      
         
      \bchapter{}
      \begin{verse}
${}^{1}Vers cette époque, Antiocos se mit à préparer sa deuxième expédition contre l’Égypte. 
${}^{2}Or, il arriva que dans la ville de Jérusalem tout entière, pendant près de quarante jours, apparurent des cavaliers courant dans les airs, vêtus de robes brodées d’or, des troupes armées disposées en cohorte, 
${}^{3}des glaives dégainés, des escadrons de cavalerie en ordre de bataille, des attaques et des charges lancées de part et d’autre, des mouvements de boucliers, des forêts de piques, des projectiles volants, un éclat fulgurant d’armures d’or et des cuirasses en tout genre. 
${}^{4}Aussi, tous priaient pour que cette apparition soit de bon augure.
${}^{5}Comme une fausse rumeur avait répandu la nouvelle de la mort d’Antiocos, Jason prit avec lui pas moins d’un millier d’hommes et dirigea à l’improviste une attaque contre la ville. Ceux qui étaient sur le rempart furent repoussés, et la ville fut bientôt prise. Ménélas se réfugia alors dans l’acropole. 
${}^{6}Jason se livra sans pitié au massacre de ses propres concitoyens, oubliant qu’une victoire sur des frères de race est la plus grande des défaites. Il se comportait comme si ses trophées étaient remportés sur des ennemis, non sur des compatriotes. 
${}^{7}En fait, il ne réussit pas à s’emparer du pouvoir, il finit même par se couvrir de honte à cause de ses machinations, si bien qu’il gagna de nouveau en fugitif le pays des Ammonites. 
${}^{8}En définitive, il subit un misérable retournement des choses : enfermé d’abord chez Arétas, prince des Arabes, il s’enfuit de ville en ville et, pourchassé par tous, détesté parce qu’il rejetait les lois, abhorré comme le bourreau de sa patrie et de ses concitoyens, il alla échouer en Égypte. 
${}^{9}Lui, qui avait banni tant d’hommes de leur patrie, périt sur une terre étrangère, après s’être rendu auprès des Lacédémoniens, dans l’espoir d’y trouver un refuge en raison de leur parenté avec son peuple. 
${}^{10}Lui qui avait privé tant de gens de sépulture, nul ne le pleura ; il n’eut aucune espèce de funérailles ni aucune place dans le tombeau de ses pères.
${}^{11}Lorsque ces faits parvinrent à la connaissance du roi, celui-ci en conclut que la Judée s’était révoltée. Il quitta donc l’Égypte, furieux comme une bête sauvage, et s’empara de la ville à la pointe de la lance. 
${}^{12}Il ordonna aux soldats d’abattre sans pitié ceux qui leur tomberaient entre les mains et d’égorger ceux qui se réfugieraient dans les maisons. 
${}^{13}On extermina jeunes et vieux, on massacra femmes et enfants, on égorgea jeunes filles et tout-petits. 
${}^{14}Il y eut quatre-vingt mille victimes en ces trois jours : quarante mille tombèrent sous les coups, et autant furent vendus comme esclaves. 
${}^{15}Non content de cela, Antiocos eut l’audace de pénétrer dans le Temple le plus saint de toute la terre, sous la conduite de ce Ménélas qui en était venu à trahir et les lois et la patrie. 
${}^{16}De ses mains infâmes, il s’empara des objets sacrés. Et les dons que d’autres rois avaient déposés pour l’enrichissement, la gloire et la dignité du Lieu saint, il les déroba de ses mains profanes. 
${}^{17}Enflé d’orgueil, Antiocos ne voyait pas que le Maître suprême était un moment irrité à cause des péchés des habitants de la ville, et que c’est pour cela qu’il détachait ses regards du Lieu saint. 
${}^{18}Car si les habitants de la ville n’avaient pas été plongés dans une multitude de péchés, Antiocos lui aussi aurait été flagellé dès son arrivée et ainsi détourné de sa témérité, tout comme cet Héliodore qui avait été envoyé par le roi Séleucos pour inspecter la salle du trésor. 
${}^{19}Pourtant, ce n’est pas à cause du Lieu saint que le Seigneur a choisi le peuple, c’est à cause du peuple qu’il a choisi le Lieu saint. 
${}^{20}C’est pourquoi le Lieu saint lui-même, après avoir été associé aux malheurs du peuple, eut ensuite part avec lui à son rétablissement. Lui qui avait été délaissé au temps de la colère du Tout-Puissant, il fut à nouveau restauré dans toute sa gloire, lors de la réconciliation avec le Maître suprême.
${}^{21}Antiocos, après avoir enlevé au Temple dix-huit cents talents, s’empressa de regagner Antioche. Dans son arrogance, il s’imaginait rendre la terre navigable et la mer praticable à la marche, tellement son cœur était enflé d’orgueil. 
${}^{22}Mais il laissa des gouverneurs chargés de faire du mal à la nation : Philippe à Jérusalem, phrygien de race, mais de caractère encore plus barbare que celui qui l’avait établi, 
${}^{23}et Andronicos au Garizim. En plus de ceux-ci, il laissa Ménélas, qui surpassait tous les autres en méchanceté envers ses concitoyens. Nourrissant à l’égard des citoyens de Judée une profonde hostilité, 
${}^{24}le roi envoya Apollonios le Mysarque, à la tête d’une armée de vingt-deux mille hommes, avec l’ordre d’égorger tous ceux qui étaient dans la force de l’âge et de vendre les femmes et les jeunes enfants. 
${}^{25}Arrivé à Jérusalem et jouant le personnage pacifique, Apollonios attendit le saint jour du sabbat. Alors, profitant du repos des Juifs, il commanda à ses hommes d’organiser une parade militaire. 
${}^{26}Tous ceux qui étaient sortis pour voir le défilé, il les fit massacrer ; puis, parcourant la ville avec ses soldats en armes, il abattit une foule considérable de gens.
${}^{27}Or Judas, appelé aussi Maccabée, s’étant retiré dans les montagnes avec une dizaine de compagnons, vivait avec eux à la manière des bêtes sauvages ; ils ne mangeaient que des herbes pour ne pas contracter la moindre souillure.
      
         
      \bchapter{}
      \begin{verse}
${}^{1}Peu de temps après, le roi envoya Géronte l’Athénien pour contraindre les Juifs à se détourner des lois de leurs pères et à ne plus se gouverner selon les lois de Dieu. 
${}^{2}Ils devaient en outre souiller le temple de Jérusalem en le dédiant à Zeus Olympien, et le temple du Garizim en le dédiant à Zeus Hospitalier, comme le demandaient les habitants de ce lieu. 
${}^{3}Cette invasion du mal fut pénible et difficile à supporter, même pour la population. 
${}^{4}Débauches et parties de plaisir emplissaient le Temple : les païens s’y divertissaient avec des prostituées, avaient commerce avec des femmes sur les parvis sacrés, où ils introduisaient aussi des choses défendues. 
${}^{5}L’autel était recouvert d’offrandes non conformes aux lois et illicites. 
${}^{6}Il n’était possible ni de célébrer le sabbat, ni d’observer les fêtes de nos pères, ni simplement de se déclarer juif. 
${}^{7}Chaque mois, le jour anniversaire de la naissance du roi, on était contraint par une amère nécessité de prendre part à un repas sacrilège. Et lors des fêtes dionysiaques, on était forcé de suivre, couronné de lierre, le cortège en l’honneur de Dionysos. 
${}^{8}Un décret fut promulgué, à l’instigation de Ptolémée, pour que, dans les villes grecques du voisinage, on tienne la même conduite à l’égard des Juifs, on organise des repas sacrilèges, 
${}^{9}et que l’on égorge ceux qui ne choisiraient pas d’adopter les coutumes grecques. Tout ceci laissait entrevoir l’imminence de la détresse. 
${}^{10}Ainsi, deux femmes furent déférées en justice pour avoir fait circoncire leurs enfants. On suspendit leurs nourrissons à leurs seins et on les traîna publiquement à travers la ville, avant de les précipiter du haut des remparts. 
${}^{11}D’autres étaient accourus ensemble dans les cavernes voisines, pour y célébrer en cachette le septième jour. On les dénonça à Philippe, et ils furent tous brûlés, car ils s’étaient gardés de se défendre eux-mêmes, par respect pour la sainteté du jour.
${}^{12}Je recommande donc à ceux qui liront ce livre de ne pas se laisser décourager par ces événements, mais de penser que ces châtiments ont eu lieu non pour la ruine, mais pour l’éducation de notre race. 
${}^{13}Car c’est le signe d’une grande bonté que de ne pas tolérer longtemps ceux qui commettent l’impiété, mais de leur infliger sans retard des châtiments. 
${}^{14}En effet, à l’égard des autres nations, le Maître attend avec grande patience, pour les châtier, qu’elles aient atteint le comble de leurs péchés. Mais ce n’est pas ainsi qu’il a jugé bon d’agir avec nous, 
${}^{15}afin de ne pas avoir à nous punir plus tard, quand nos péchés seraient arrivés à leur pleine mesure. 
${}^{16}Il est donc vrai que jamais il ne nous retire sa miséricorde : tout en éduquant son peuple par des événements, il ne l’abandonne pas. 
${}^{17}Qu’il nous suffise d’avoir rappelé cela. Après ces quelques mots, il nous faut revenir à notre récit.
${}^{18}Éléazar était l’un des scribes les plus éminents. C’était un homme très âgé, et de très belle allure. On voulut l’obliger à manger du porc en lui ouvrant la bouche de force. 
${}^{19} Préférant avoir une mort prestigieuse plutôt qu’une vie abjecte, il marchait de son plein gré vers l’instrument du supplice, 
${}^{20} après avoir recraché cette viande\\, comme on doit le faire quand on a le courage de rejeter ce qu’il n’est pas permis de manger, même\\par amour de la vie. 
${}^{21} Ceux qui étaient chargés de ce repas sacrilège le connaissaient de longue date. Ils le prirent à part et lui conseillèrent de faire apporter des viandes dont l’usage était permis, et qu’il aurait préparées lui-même. Il n’aurait qu’à faire semblant de manger les chairs de la victime pour obéir au roi ; 
${}^{22} en agissant ainsi, il échapperait à la mort et serait traité avec humanité grâce à la vieille amitié qu’il avait pour eux. 
${}^{23} Mais il fit un beau raisonnement, bien digne de son âge, du rang que lui donnait sa vieillesse, du respect que lui valaient ses cheveux blancs, de sa conduite irréprochable depuis l’enfance, et surtout digne de la législation sainte établie par Dieu. Il s’exprima en conséquence, demandant qu’on l’envoyât sans tarder au séjour des morts : 
${}^{24} « Une telle comédie est indigne de mon âge. Car beaucoup de jeunes gens croiraient qu’Éléazar, à quatre-vingt-dix ans, adopte la manière de vivre des étrangers. 
${}^{25} À cause de cette comédie, par ma faute, ils se laisseraient égarer eux aussi ; et moi, pour un misérable reste de vie, j’attirerais sur ma vieillesse la honte et le déshonneur. 
${}^{26} Même si j’évite, pour le moment, le châtiment qui vient des hommes, je n’échapperai pas, vivant ou mort, aux mains du Tout-Puissant. 
${}^{27} C’est pourquoi, en quittant aujourd’hui la vie avec courage, je me montrerai digne de ma vieillesse 
${}^{28} et, en choisissant de mourir avec détermination et noblesse pour nos vénérables et saintes lois, j’aurai laissé aux jeunes gens le noble exemple d’une belle mort. »
      Sur ces mots, il alla tout droit au supplice. 
${}^{29} Pour ceux qui le conduisaient, ces propos étaient de la folie ; c’est pourquoi ils passèrent subitement de la bienveillance à l’hostilité. 
${}^{30} Quant à lui, au moment de mourir sous les coups, il dit en gémissant : « Le Seigneur, dans sa science sainte, le voit bien : alors que je pouvais échapper à la mort, j’endure sous le fouet des douleurs qui font souffrir mon corps ; mais dans mon âme je les supporte avec joie, parce que je crains Dieu. » 
${}^{31} Telle fut la mort de cet homme. Il laissa ainsi, non seulement à la jeunesse mais à l’ensemble de son peuple, un exemple de noblesse et un mémorial de vertu.
      
         
      \bchapter{}
      \begin{verse}
${}^{1}Sept frères avaient été arrêtés avec leur mère. À coups de fouet et de nerf de bœuf, le roi Antiocos\\voulut les contraindre à manger du porc, viande interdite. 
${}^{2}L’un d’eux se fit leur porte-parole et déclara : « Que cherches-tu à savoir de nous ? Nous sommes prêts à mourir plutôt que de transgresser les lois de nos pères. » 
${}^{3}Fou de rage, le roi ordonna que l’on chauffe des poêles et des chaudrons. 
${}^{4}Dès qu’ils furent brûlants, il ordonna de couper la langue de celui qui s’était fait leur porte-parole, de lui arracher la peau de la tête et de lui couper les mains et les pieds, sous les yeux de ses autres frères et de sa mère. 
${}^{5}Lorsqu’il fut complètement mutilé, le roi ordonna de l’amener près du brasier et de le faire passer à la poêle, alors qu’il respirait encore. Tandis que des vapeurs abondantes se répandaient autour de la poêle, les autres s’exhortaient mutuellement, avec leur mère, à mourir avec noblesse. Ils disaient : 
${}^{6}« Le Seigneur Dieu nous voit et, en vérité, nous apporte le réconfort, comme Moïse l’a clairement affirmé dans son cantique où il témoigne, à la face de tous, que Dieu réconfortera ses serviteurs. »
${}^{7}Lorsque le premier fut mort de cette manière, on amena le deuxième pour le torturer. On lui arracha la peau de la tête avec les cheveux, puis on lui demanda : « Mangeras-tu, plutôt que d’être châtié dans ton corps, membre par membre ? » 
${}^{8}Mais il répondit, dans la langue de ses pères : « Non ! » C’est pourquoi lui aussi subit aussitôt les mêmes sévices que le premier. 
${}^{9}Au moment de rendre le dernier soupir, il dit :\\« Tu es un scélérat, toi qui nous arraches à cette vie présente, mais puisque nous mourons par fidélité à ses lois, le Roi du monde nous ressuscitera pour une vie éternelle. » 
${}^{10}Après cela, le troisième fut mis à la torture. Il tendit la langue aussitôt qu’on le lui ordonna et il présenta les mains avec intrépidité, 
${}^{11}en déclarant avec noblesse : « C’est du Ciel que je tiens ces membres, mais à cause de ses lois je les méprise, et c’est par lui que j’espère les retrouver. » 
${}^{12}Le roi et sa suite furent frappés de la grandeur d’âme de ce jeune homme qui comptait pour rien les souffrances. 
${}^{13}Lorsque celui-ci fut mort, le quatrième frère fut soumis aux mêmes sévices. 
${}^{14}Sur le point d’expirer, il parla ainsi :\\« Mieux vaut mourir par la main des hommes, quand on attend la résurrection promise par Dieu, tandis que toi, tu ne connaîtras pas la résurrection pour la vie. » 
${}^{15}On amena aussitôt le cinquième pour le tourmenter. 
${}^{16}Fixant les yeux sur le roi, il dit : « Tout mortel que tu es, tu as autorité sur les hommes et tu fais ce que tu veux. Ne t’imagine pas pour autant que notre race soit abandonnée de Dieu. 
${}^{17}Mais toi, attends : tu verras combien sa puissance est grande et de quelle manière il sévira contre toi-même et ta descendance ! » 
${}^{18}Après celui-là, on amena le sixième, et lorsqu’il fut sur le point de mourir, il dit : « Ne te fais pas de vaine illusion : c’est à cause de nous-mêmes que nous endurons ces souffrances, pour avoir péché contre notre propre Dieu. De là viennent ces malheurs surprenants. 
${}^{19}Mais toi, ne va pas croire que tu resteras impuni, pour avoir entrepris de faire la guerre à Dieu. »
${}^{20}Leur mère fut particulièrement admirable et digne d’une illustre mémoire : voyant mourir ses sept fils dans l’espace d’un seul jour, elle le supporta vaillamment parce qu’elle avait mis son espérance dans le Seigneur. 
${}^{21} Elle exhortait chacun d’eux dans la langue de ses pères ; cette femme héroïque\\leur parlait avec un courage viril\\ : 
${}^{22} « Je suis incapable de dire comment vous vous êtes formés dans mes entrailles. Ce n’est pas moi qui vous ai donné l’esprit et la vie, qui ai organisé les éléments dont chacun de vous est composé. 
${}^{23} C’est le Créateur du monde qui façonne l’enfant à l’origine\\, qui préside à l’origine de toute chose. Et c’est lui qui, dans sa miséricorde, vous rendra l’esprit et la vie, parce que, pour l’amour de ses lois, vous méprisez maintenant votre propre existence. »
${}^{24}Antiocos s’imagina qu’on le méprisait, et soupçonna que ce discours contenait des insultes. Il se mit à exhorter le plus jeune, le dernier survivant. Bien plus, il lui promettait avec serment de le rendre à la fois riche et très heureux s’il abandonnait les usages de ses pères : il en ferait son ami et lui confierait des fonctions publiques. 
${}^{25} Comme le jeune homme n’écoutait pas, le roi appela la mère, et il l’exhortait à conseiller l’adolescent pour le sauver. 
${}^{26} Au bout de ces longues exhortations, elle consentit à persuader son fils. 
${}^{27} Elle se pencha vers lui, et lui parla dans la langue de ses pères, trompant ainsi le cruel tyran : « Mon fils, aie pitié de moi : je t’ai porté neuf mois dans mon sein, je t’ai allaité pendant trois ans, je t’ai nourri et élevé jusqu’à l’âge où tu es parvenu, j’ai pris soin de toi. 
${}^{28} Je t’en conjure, mon enfant, regarde le ciel et la terre avec tout ce qu’ils contiennent : sache que Dieu a fait tout cela de rien\\, et que la race des hommes est née de la même manière. 
${}^{29} Ne crains pas ce bourreau, montre-toi digne de tes frères et accepte la mort, afin que je te retrouve avec eux au jour de la miséricorde\\. »
${}^{30}Lorsqu’elle eut fini de parler, le jeune homme déclara : « Qu’attendez-vous ? Je n’obéis pas à l’ordre du roi, mais j’écoute l’ordre de la Loi donnée à nos pères par Moïse. 
${}^{31}Et toi qui as inventé toutes sortes de mauvais traitements contre les Hébreux, tu n’échapperas pas à la main de Dieu. 
${}^{32}Car nous, c’est à cause de nos propres péchés que nous souffrons. 
${}^{33}En effet, notre Seigneur qui est vivant s’est irrité un moment contre nous, en vue de nous réprimander et de nous éduquer, mais de nouveau il se réconciliera avec ses serviteurs. 
${}^{34}Et toi, impie, le plus infâme de tous les hommes, ne t’enfle pas d’orgueil sans raison en te berçant d’espoirs incertains, alors que tu portes la main sur les serviteurs du Ciel, 
${}^{35}car tu n’as pas encore échappé au jugement du Dieu tout-puissant qui voit tout ! 
${}^{36}Nos frères, maintenant, ont supporté une épreuve passagère, pour une vie intarissable : ils sont tombés à cause de l’alliance de Dieu. Mais toi, par le jugement de Dieu, tu recevras le juste châtiment de ton arrogance. 
${}^{37}Quant à moi, comme mes frères, je me livre corps et âme pour les lois de nos pères, en suppliant Dieu de se montrer bientôt favorable à la nation et de t’amener, par des épreuves et des fléaux, à confesser que lui seul est Dieu. 
${}^{38}Je prie aussi pour que la colère du Tout-Puissant, justement déchaînée sur l’ensemble de notre race, prenne fin avec ma mort et celle de mes frères. »
${}^{39}Fou de rage, exaspéré par la moquerie, le roi s’acharna contre ce dernier plus cruellement encore que contre les autres. 
${}^{40}Le jeune homme mourut donc, pur de toute souillure, mettant toute sa confiance dans le Seigneur. 
${}^{41}Enfin, après tous ses fils, la mère mourut la dernière. 
${}^{42}Nous en resterons là pour le récit des repas sacrilèges et des tourments sans mesure.
      <h2 class="intertitle" id="d85e123847">4. La révolte menée par Judas Maccabée (8 – 15,36)</h2>
      
         
      \bchapter{}
      \begin{verse}
${}^{1}Or Judas, appelé aussi Maccabée, et ses compagnons s’introduisaient furtivement dans les villages et faisaient appel à leurs frères de race. Réunissant ceux qui demeuraient fidèles au judaïsme, ils en rassemblèrent environ six mille. 
${}^{2}Ils suppliaient le Seigneur de poser son regard sur le peuple que tous opprimaient, d’avoir compassion pour le Temple profané par les impies, 
${}^{3}de prendre en pitié la ville que l’on dévastait, et qu’on allait réduire au ras du sol. Ils le suppliaient d’écouter le sang qui criait jusqu’à lui, 
${}^{4}de se souvenir du massacre criminel des petits enfants innocents, et de déchaîner sa haine du mal, en raison des blasphèmes proférés envers son nom. 
${}^{5}Dès que Judas Maccabée eut une troupe organisée, les païens furent incapables de lui résister, car la colère du Seigneur s’était changée en miséricorde. 
${}^{6}Survenant à l’improviste, il incendiait villes et villages, s’emparait des positions favorables, mettait en fuite un grand nombre d’ennemis, 
${}^{7}choisissant surtout la complicité des nuits pour de telles expéditions. La renommée de sa bravoure se répandait partout.
${}^{8}Voyant que cet homme progressait peu à peu et remportait des succès toujours plus fréquents, Philippe écrivit à Ptolémée, le gouverneur militaire de Cœlé-Syrie et de Phénicie, de venir au secours des affaires du roi. 
${}^{9}Sans retard, celui-ci désigna Nicanor, fils de Patrocle, du rang des premiers amis du roi, et l’envoya, à la tête d’au moins vingt mille hommes de diverses nations, pour exterminer la race juive tout entière. Il lui adjoignit Gorgias, un général qui avait l’expérience des choses de la guerre. 
${}^{10}Nicanor se proposait d’acquitter pour le roi le tribut de deux mille talents dû aux Romains, au moyen de la vente des Juifs que l’on ferait prisonniers. 
${}^{11}Il s’empressa donc d’envoyer aux villes de la côte l’invitation de venir acheter des esclaves juifs, promettant d’en livrer quatre-vingt-dix pour un talent ; il ne s’attendait pas au jugement qui devait s’ensuivre pour lui, de la part du Tout-Puissant.
${}^{12}La nouvelle de l’avance de Nicanor parvint à Judas. Quand celui-ci eut averti ses compagnons de l’approche de l’armée ennemie, 
${}^{13}les lâches et ceux qui manquaient de confiance dans le jugement de Dieu s’enfuirent de tous côtés et gagnèrent d’autres lieux. 
${}^{14}D’autres mettaient en vente tout ce qui leur restait, et priaient le Seigneur de les délivrer de l’impie Nicanor, qui les avait vendus avant même que la rencontre eût lieu. 
${}^{15}Ils le suppliaient d’intervenir, sinon à cause d’eux-mêmes, du moins en raison des alliances en faveur de leurs pères et parce que son nom vénérable et plein de majesté avait été invoqué sur eux. 
${}^{16}Judas Maccabée rassembla ceux qui l’entouraient : ils étaient au nombre de six mille. Il les exhortait à ne pas être frappés de crainte devant l’ennemi, à ne pas s’inquiéter du très grand nombre des païens qui marchaient contre eux injustement, mais à combattre noblement. 
${}^{17}Il les encourageait à garder devant les yeux l’outrage criminel commis par ces gens contre le Lieu saint, les tourments infligés à la ville ravagée, et même la ruine des institutions ancestrales. 
${}^{18}« Ceux-là, disait-il, mettent leur confiance à la fois dans leurs armes et dans leurs actions téméraires. Mais nous, nous mettons notre confiance en Dieu tout-puissant, qui peut d’un seul signe de tête renverser aussi bien ceux qui marchent contre nous, que le monde tout entier ». 
${}^{19}Il leur rappela en outre les cas de protection divine qui avaient eu lieu en faveur de leurs ancêtres, notamment sous le règne de Sennakérib, lorsque cent quatre-vingt-cinq mille hommes avaient péri, 
${}^{20}et la bataille livrée en Babylonie contre les Galates. Ce jour-là, les Juifs qui avaient pris part à l’action étaient, en tout, huit mille hommes, aux côtés de quatre mille Macédoniens. Or les Macédoniens se trouvant dans une situation critique, les huit mille avaient fait périr cent vingt mille hommes, grâce au secours qui leur était venu du Ciel. Ils en avaient tiré un avantage important.
${}^{21}Après avoir ainsi galvanisé ses compagnons, et les avoir rendus prêts à mourir pour les lois et la patrie, Judas divisa l’armée en quatre corps. 
${}^{22}À la tête de chaque unité, il plaça ses frères Simon, Joseph et Jonathan, chacun ayant quinze cents hommes sous ses ordres, 
${}^{23}et en outre Éléazar. Il fit la lecture du Livre saint, puis donna pour mot d’ordre : « Secours de Dieu ! » ; il prit personnellement la tête du premier détachement et engagea le combat avec Nicanor. 
${}^{24}Le Tout-Puissant s’étant fait leur allié, ils égorgèrent plus de neuf mille ennemis, blessèrent et mutilèrent la plus grande partie des soldats de Nicanor et les mirent tous en fuite. 
${}^{25}Ils prirent aussi l’argent de ceux qui étaient venus pour les acheter. Après les avoir poursuivis assez loin, ils revinrent sur leurs pas, pressés par l’heure, 
${}^{26}car c’était la veille du sabbat, motif pour lequel ils ne s’attardèrent pas à courir derrière eux. 
${}^{27}Quand ils eurent ramassé les armes et enlevé le butin des ennemis, ils célébrèrent le sabbat : ils multiplièrent les bénédictions et les louanges au Seigneur qui les avait sauvés, qui avait fixé à ce jour la première manifestation de sa miséricorde à leur égard. 
${}^{28}Après le sabbat, ils distribuèrent une part du butin aux victimes de la persécution, aux veuves et aux orphelins ; ils partagèrent le reste entre eux et leurs enfants. 
${}^{29}Après avoir accompli cela, ils adressèrent au Seigneur miséricordieux une supplication commune, pour le prier de se réconcilier définitivement avec ses serviteurs.
${}^{30}Se mesurant avec les soldats de Timothée et de Bacchidès, ils en tuèrent plus de vingt mille et se rendirent totalement maîtres de hautes forteresses. À nouveau, ils se partagèrent leurs importantes prises de guerre en deux parts égales, l’une pour eux-mêmes et l’autre pour les victimes de la persécution, les orphelins et les veuves, sans oublier les vieillards. 
${}^{31}Ayant ramassé les armes des ennemis, ils prirent soin de les rassembler toutes dans les endroits favorables ; le reste du butin, ils le portèrent à Jérusalem. 
${}^{32}Ils tuèrent le commandant des soldats de Timothée, un homme des plus pervers, qui avait plongé les Juifs dans une grande affliction. 
${}^{33}Lorsqu’ils célébrèrent, dans leur patrie, les fêtes de la victoire, ils brûlèrent ceux qui avaient incendié les portails sacrés, et qui s’étaient réfugiés avec Callisthène dans une même petite maison ; ces hommes reçurent ainsi le digne salaire de leur impiété.
${}^{34}Nicanor, ce triple scélérat qui avait fait venir les mille marchands pour la vente des Juifs, 
${}^{35}se trouva humilié, grâce à l’intervention du Seigneur, par ceux-là mêmes qu’il considérait comme ce qu’il y avait de plus bas. Il ôta son habit d’apparat et, s’isolant de tous les autres, il traversa l’intérieur des terres à la manière d’un fugitif. Par une chance extraordinaire, il arriva ainsi à Antioche, alors que son armée avait été détruite. 
${}^{36}Lui qui s’était engagé à régler le tribut que l’on devait aux Romains, avec le produit de la ventedes prisonniers de guerre de Jérusalem, il proclama que les Juifs avaient un défenseur, et qu’ils étaient invulnérables par le fait même qu’ils suivaient les lois édictées par ce défenseur.
      
         
      \bchapter{}
      \begin{verse}
${}^{1}À cette époque, Antiocos était piteusement revenu des régions de Perse. 
${}^{2}Il s’était en effet rendu dans la ville de Persépolis ; il y avait entrepris de piller le temple et d’opprimer la ville. Mais la foule s’était soulevée, en ayant recours aux armes, si bien qu’Antiocos fut mis en déroute par les habitants du pays, et contraint d’opérer une retraite honteuse. 
${}^{3}Comme il se trouvait près d’Ecbatane, il apprit ce qui était arrivé à Nicanor et aux gens de Timothée. 
${}^{4}Transporté de fureur, il résolut de faire payer aux Juifs l’injure infligée par ceux qui avaient causé sa fuite. Pour ce motif, il ordonna au conducteur de lancer le char en avant, et de continuer sans répit jusqu’à la fin du voyage. En réalité, la sentence du Ciel était déjà sur lui. Il avait dit, en effet, dans son arrogance : « Arrivé à Jérusalem, je ferai de cette ville la fosse commune des Juifs. » 
${}^{5}Mais le Seigneur qui voit tout, le Dieu d’Israël, le frappa d’un mal incurable et mystérieux. À peine avait-il prononcé ces mots qu’il fut saisi d’une implacable douleur aux entrailles et de terribles souffrances internes. 
${}^{6}Ce n’était que justice pour cet homme qui avait lui-même torturé les entrailles d’autres hommes par des supplices nombreux et inouïs. 
${}^{7}Pourtant, il ne se départit nullement de son attitude provocante. Toujours rempli d’arrogance, il exhalait contre les Juifs le feu de sa colère et commandait d’accélérer la marche. C’est alors que, soudain, il tomba du char qui roulait avec fracas : tous les membres de son corps, entraînés dans une chute malheureuse, furent désarticulés. 
${}^{8}Cet homme qui, l’instant d’avant, dans sa prétention surhumaine, croyait pouvoir donner des ordres aux flots de la mer, lui qui s’imaginait pouvoir peser dans une balance les hauteurs des montagnes, gisait à terre. Il fut transporté sur une civière, manifestant à tous la puissance de Dieu. 
${}^{9}C’était au point que des vers sortaient en grouillant des yeux de l’impie et qu’au milieu d’atroces douleurs, sa chair se décomposait, alors qu’il vivait encore. La puanteur de cette pourriture accablait toute l’armée. 
${}^{10}Celui qui peu auparavant croyait toucher aux astres du ciel, personne maintenant ne pouvait l’escorter à cause de son intolérable puanteur !
${}^{11}C’est alors que, broyé, il commença à se départir de cet excès d’arrogance ; sous le fléau divin, tiraillé à chaque instant par de vives douleurs, il prit conscience de sa situation. 
${}^{12}Ne pouvant supporter sa propre puanteur, il déclara : « Il est juste de se soumettre à Dieu et, quand on est mortel, de ne pas se croire l’égal de la divinité. » 
${}^{13}Mais les prières de cet homme infâme allaient vers un maître qui ne devait plus avoir pitié de lui. 
${}^{14}La Ville sainte, vers laquelle il s’était hâté pour la réduire au ras du sol et la transformer en fosse commune, il promettait maintenant de la rendre libre. 
${}^{15}Les Juifs, auxquels il avait résolu de ne pas même accorder une sépulture, et qu’il avait décidé de jeter avec leurs petits enfants aux bêtes sauvages et en pâture aux oiseaux de proie, il disait maintenant vouloir les rendre tous égaux aux Athéniens. 
${}^{16}Le Temple saint, qu’autrefois il avait pillé, il l’ornerait des présents les plus beaux ; il lui restituerait en bien plus grand nombre tous les objets sacrés et pourvoirait de ses propres revenus à toutes les dépenses nécessaires aux sacrifices. 
${}^{17}En plus de tout cela, il promettait de devenir lui-même juif et de parcourir tout lieu habité en proclamant la souveraineté de Dieu. 
${}^{18}Mais ses souffrances ne s’atténuaient en rien, car le juste jugement de Dieu était sur lui. Alors, désespérant de son état, il écrivit aux Juifs la lettre ci-dessous, véritable supplique, qui disait :
${}^{19}« Aux excellents Juifs, ses concitoyens, salut, santé et bonheur parfaits de la part d’Antiocos, roi et général en chef. 
${}^{20}Si vous vous portez bien, ainsi que vos enfants, et si vos affaires marchent à souhait, nous en rendons vivement grâce à Dieu, car nous plaçons notre espérance dans le Ciel. 
${}^{21}Quant à moi, je suis alité, sans force, mais je garde un affectueux souvenir de votre respect et de votre bienveillance. À mon retour des régions de Perse, je suis tombé gravement malade, et j’ai donc estimé nécessaire de veiller à la sécurité de tous. 
${}^{22}Je n’ai pas d’illusion sur mon état, et cependant j’ai le ferme espoir d’échapper à cette maladie. 
${}^{23}Mais j’observe que mon père aussi, à l’époque où il fit campagne dans le haut pays, désigna son successeur, 
${}^{24}afin que les habitants du pays sachent à qui les affaires avaient été confiées, et ne soient donc pas troublés en cas d’événement inattendu ou de mauvaise nouvelle. 
${}^{25}Par ailleurs, je me rends compte que les souverains proches de nous et les voisins du royaume épient les occasions et guettent la suite des événements. C’est pourquoi j’ai désigné comme roi mon fils Antiocos, que j’avais souvent confié et recommandé à la plupart d’entre vous, lorsque je me hâtais vers les provinces du haut pays. Je lui ai d’ailleurs écrit la lettre transcrite ci-dessous. 
${}^{26}Je vous prie donc et vous conjure, en souvenir des bienfaits que je vous ai accordés en général et en particulier, de conserver chacun la bienveillance que vous avez pour moi et pour mon fils. 
${}^{27}En effet, j’en ai la conviction, il poursuivra ma politique, et se conduira envers vous avec modération et humanité ».
${}^{28}Ainsi, ce meurtrier, ce blasphémateur connut le sort lamentable de terminer sa vie en terre étrangère, dans les montagnes, en proie aux pires souffrances, comme celles qu’il avait infligées aux autres. 
${}^{29}Philippe, son ami d’enfance, ramena son corps, mais ensuite il se retira en Égypte, auprès de Ptolémée Philométor, car il craignait le fils d’Antiocos.
      
         
      \bchapter{}
      \begin{verse}
${}^{1}Sous la conduite du Seigneur, Judas Maccabée et ses compagnons reconquirent le Temple et la ville de Jérusalem. 
${}^{2}Ils détruisirent les autels païens édifiés sur la place publique par les étrangers, ainsi que leurs lieux de culte. 
${}^{3}Après avoir purifié le Temple, ils bâtirent un nouvel autel. Après deux ans d’interruption, ils offrirent des sacrifices, en prenant le feu obtenu par le moyen de pierres à feu. Ils brûlèrent de l’encens, allumèrent les lampes et placèrent les pains de l’offrande. 
${}^{4}Cela fait, ils se jetèrent à plat ventre et supplièrent le Seigneur de ne plus les faire tomber dans de tels malheurs, mais de les corriger lui-même avec modération s’il leur arrivait encore de pécher, et de ne pas les livrer aux païens blasphémateurs et barbares. 
${}^{5}Or, c’est au jour anniversaire de la profanation du Temple par les étrangers, qu’eut lieu la purification du Temple, le même jour, à savoir le vingt-cinq du même mois, le mois de Kisléou. 
${}^{6}Ils célébrèrent cette fête dans l’allégresse, durant huit jours, à la manière de la fête des Tentes, se souvenant comment, peu de temps auparavant, lors de la fête des Tentes, ils campaient dans les montagnes et les cavernes à la manière des bêtes sauvages. 
${}^{7}C’est pourquoi, portant des thyrses, des rameaux verdoyants et des palmes, ils faisaient monter des hymnes vers Celui qui avait mené à bien la purification de son propre Lieu saint. 
${}^{8}Puis ils décrétèrent par un édit public et un vote que toute la nation des Juifs célébrerait ces jours-là chaque année.
      
         
${}^{9}Telles furent les circonstances de la mort d’Antiocos surnommé Épiphane. 
${}^{10}Nous allons maintenant exposer les événements survenus sous Antiocos Eupator, le fils de cet homme impie, en résumant les malheurs liés aux guerres. 
${}^{11}Dès son accès au trône, ce prince mit à la tête des affaires du royaume un certain Lysias, gouverneur militaire en chef de Cœlé-Syrie et de Phénicie. 
${}^{12}En effet, Ptolémée, appelé Macrone, avait pris l’initiative d’agir avec justice à l’égard des Juifs, à cause des torts qu’on leur avait infligés. Il s’efforçait d’administrer leurs affaires dans la paix, 
${}^{13}mais il fut accusé pour cela par les amis du roi, auprès d’Eupator. En toute occasion, il s’entendait appeler traître pour avoir abandonné Chypre que lui avait confiée Philométor, et s’être retiré auprès d’Antiocos Épiphane. Comme il ne pouvait faire honneur à la dignité de sa charge, il mit fin à ses jours en prenant du poison.
${}^{14}Gorgias devint gouverneur militaire de la région. Il entretenait des troupes mercenaires et alimentait en toute occasion la guerre contre les Juifs. 
${}^{15}En même temps, les Iduméens, qui s’étaient rendus maîtres de forteresses stratégiques, harcelaient les Juifs ; en attirant ceux qui étaient bannis de Jérusalem, ils s’employaient à alimenter la guerre. 
${}^{16}Judas Maccabée et ceux qui l’entouraient, après avoir imploré et prié Dieu de se montrer leur allié, s’élancèrent contre les forteresses des Iduméens. 
${}^{17}Ils les attaquèrent avec vigueur et se rendirent maîtres de ces positions ; ils repoussèrent tous ceux qui combattaient sur le rempart, égorgèrent ceux qui leur tombaient entre les mains ; ils n’en tuèrent pas moins de vingt mille. 
${}^{18}Mais neuf mille hommes au moins s’étaient réfugiés dans deux tours particulièrement fortes, contenant tout ce qu’il fallait pour soutenir un siège. 
${}^{19}Aussi, Judas Maccabée y laissa Simon et Joseph, ainsi que Zachée et ses compagnons, en nombre suffisant pour les assiéger ; lui-même partit pour des endroits où il y avait urgence. 
${}^{20}Mais les hommes de l’entourage de Simon, par amour de l’argent, se laissèrent acheter par quelques-uns de ceux qui étaient dans les tours et, pour soixante-dix mille pièces d’argent, permirent à un certain nombre de s’échapper. 
${}^{21}Informé de ce qui était arrivé, Judas Maccabée réunit les chefs du peuple et accusa les coupables d’avoir vendu leurs frères à prix d’argent, en relâchant contre eux leurs ennemis. 
${}^{22}Il fit donc mettre à mort ces hommes convaincus de trahison et, aussitôt après, il s’empara des deux tours. 
${}^{23}Menant tout à bien par la force des armes, il tua plus de vingt mille hommes dans ces deux forteresses.
${}^{24}Timothée, précédemment vaincu par les Juifs, réunit des troupes mercenaires très nombreuses et rassembla un nombre considérable de chevaux venus d’Asie ; puis il parut en Judée dans l’intention de s’en emparer à la pointe de la lance. 
${}^{25}À son approche, Judas Maccabée et ceux qui l’entouraient se répandirent en supplications devant Dieu, la tête couverte de poussière et les reins ceints de toile à sac. 
${}^{26}Prosternés contre le soubassement antérieur de l’autel, ils demandaient à Dieu de leur être favorable et de se montrer l’ennemi de leurs ennemis, l’adversaire de leurs adversaires, comme l’affirme clairement la Loi. 
${}^{27}Au sortir de cette prière, ils prirent les armes et s’avancèrent hors de la ville, à bonne distance. Quand ils furent près de l’ennemi, ils s’arrêtèrent. 
${}^{28}Aux premières lueurs de l’aurore, on engagea la lutte de part et d’autre. Les uns avaient pour gage de succès et de victoire, en plus de leur valeur, le recours au Seigneur ; les autres prenaient leur fureur pour guide des combats. 
${}^{29}Au plus fort de la bataille, les adversaires virent apparaître, sortant du ciel sur des chevaux harnachés d’or, cinq hommes magnifiques qui se mirent à la tête des Juifs. 
${}^{30}Plaçant Judas Maccabée au milieu d’eux et le protégeant de leur armement, ils le rendaient invulnérable. Mais sur les adversaires, ils lançaient des flèches et des éclairs, si bien que ceux-ci, bouleversés et aveuglés, se dispersaient en pleine panique. 
${}^{31}Vingt mille cinq cents furent égorgés, en plus de six cents cavaliers.
${}^{32}Timothée lui-même se réfugia dans une forteresse appelée Gazara, importante citadelle que commandait Chéréas. 
${}^{33}Pendant quatre jours, Judas Maccabée et ceux qui l’entouraient assiégèrent la citadelle avec beaucoup d’entrain. 
${}^{34}Confiants dans la solidité de la place, ceux qui se trouvaient à l’intérieur proféraient d’énormes blasphèmes et lançaient des paroles que la Loi interdit. 
${}^{35}Mais lorsque le cinquième jour se mit à poindre, une vingtaine de jeunes gens de l’entourage de Judas Maccabée, enflammés de fureur par ces blasphèmes, s’élancèrent avec bravoure sur le rempart et abattirent sauvagement ceux qui leur tombaient entre les mains. 
${}^{36}D’autres montèrent pareillement contre les assiégés, à la faveur de cette diversion. Ils incendièrent les tours, allumèrent des bûchers et brûlèrent vifs les blasphémateurs. Les premiers défoncèrent les portes, pour laisser entrer le reste de l’armée, et s’emparèrent de la ville. 
${}^{37}Timothée, qui s’était caché dans une citerne, ils l’égorgèrent, de même que son frère Chéréas et Apollophane. 
${}^{38}Après avoir accompli ces exploits, ils entonnèrent des hymnes et des louanges pour bénir le Seigneur, qui accordait à Israël de si grands bienfaits et lui donnait la victoire.
      
         
      \bchapter{}
      \begin{verse}
${}^{1}Très peu de temps après, Lysias, tuteur et parent du roi, l’administrateur du royaume, fort irrité par ce qui s’était passé, 
${}^{2}rassembla environ quatre-vingt mille hommes et toute sa cavalerie, et marcha contre les Juifs. Il comptait faire de Jérusalem une ville grecque, 
${}^{3}soumettre le Temple à l’impôt, tout comme les autres lieux de culte païens, et mettre en vente chaque année la dignité de grand prêtre. 
${}^{4}Grisé par ses myriades de fantassins, ses milliers de cavaliers et ses quatre-vingts éléphants, il ne tenait aucun compte de la souveraineté de Dieu. 
${}^{5}Il pénétra donc en Judée, s’approcha de Bethsour et en bloqua les issues. Or, cette place fortifiée était distante de Jérusalem d’environ cent cinquante stades. 
${}^{6}Dès que ceux qui entouraient Judas Maccabée apprirent que Lysias assiégeait les forteresses, ils se mirent à se lamenter et à pleurer avec la foule, suppliant le Seigneur d’envoyer un bon ange pour sauver Israël. 
${}^{7}Judas Maccabée lui-même, prenant les armes le premier, exhorta les autres à secourir leurs frères, en s’exposant avec lui au danger. Ensemble, ils s’élancèrent, avec détermination. 
${}^{8}Ils étaient encore tout proches de Jérusalem, lorsqu’apparut à leur tête un cavalier vêtu de blanc et brandissant des armes d’or. 
${}^{9}Alors, tous ensemble, ils bénirent le Dieu de miséricorde et se sentirent animés d’une grande vigueur : ils étaient prêts à pourfendre non seulement des hommes, mais les bêtes les plus féroces et jusqu’à des murailles de fer. 
${}^{10}Ils s’avancèrent donc en ordre de bataille, avec cet allié venu du ciel, car le Seigneur les avait pris en pitié. 
${}^{11}Comme des lions, ils foncèrent sur les ennemis ; ils en abattirent onze mille, en plus de seize cents cavaliers, et contraignirent tous les autres à fuir. 
${}^{12}La plupart d’entre eux se sauvèrent, blessés, privés de leurs armes. Lysias lui-même se sauva en prenant honteusement la fuite.
${}^{13}Mais Lysias ne manquait pas de bon sens : il réfléchit sur la défaite qu’il venait de subir et comprit que les Hébreux étaient invincibles, parce que le Dieu puissant combattait avec eux. Il envoya donc des émissaires 
${}^{14}pour les persuader de conclure un accord à des conditions tout à fait justes, promettant aussi de persuader le roi de la nécessité de devenir leur ami. 
${}^{15}Judas Maccabée consentit à tout ce que proposait Lysias, par souci de l’intérêt commun. Et de fait, tout ce que Judas Maccabée transmit par écrit à Lysias au sujet des Juifs, le roi l’accorda.
${}^{16}Voici donc en quels termes Lysias écrivit aux Juifs :
      « Lysias à l’ensemble des Juifs, salut ! 
${}^{17}Jean et Absalom, vos envoyés, ont remis l’acte transcrit ci-dessous et demandé une réponse au sujet de ce qui s’y trouvait indiqué. 
${}^{18}J’ai donc informé le roi de tout ce qu’il fallait lui soumettre, et ce qui était possible, je l’ai accordé. 
${}^{19}Si donc vous conservez votre bienveillance à l’égard de l’État, je m’efforcerai encore à l’avenir de contribuer à votre bien. 
${}^{20}Quant aux questions de détail, j’ai donné ordre à vos envoyés et aux miens d’en discuter avec vous. 
${}^{21}Portez-vous bien ! L’an 148, le 24 du mois de Dioscore. »
${}^{22}La lettre du roi était ainsi conçue :
      « Le roi Antiocos à son frère Lysias, salut ! 
${}^{23}Notre père est allé rejoindre les dieux, et nous-même, nous souhaitons que les sujets de notre royaume soient à l’abri des troubles et s’appliquent à leurs propres affaires. 
${}^{24}Or, nous avons appris que les Juifs ne consentent pas à s’adapter aux coutumes grecques, voulues par notre père, mais qu’ils préfèrent leur genre de vie à eux et demandent qu’on leur accorde d’observer leurs lois. 
${}^{25}Désirant donc que ce peuple soit lui aussi exempt de trouble, nous décidons que le Temple lui sera restitué et que les Juifs pourront vivre selon les usages de leurs ancêtres. 
${}^{26}Tu ferais donc bien de leur envoyer des messagers pour leur tendre la main droite en signe de paix afin que, ayant pris connaissance de nos intentions, ils soient rassurés et continuent avec bonheur à gérer leurs propres affaires. »
${}^{27}Voici la lettre que le roi adressa à la nation des Juifs :
      « Le roi Antiocos au Conseil des anciens et aux autres Juifs, salut ! 
${}^{28}Si vous vous portez bien, nous en sommes heureux. Nous aussi, nous sommes en bonne santé. 
${}^{29}Ménélas nous a fait part de votre désir de rentrer chez vous pour vaquer à vos propres affaires. 
${}^{30}Donc ceux qui retourneront chez eux d’ici le 30 du mois de Xanthique obtiendront l’assurance de l’impunité. 
${}^{31}Les Juifs pourront consommer leurs aliments spéciaux et suivre leurs lois, comme auparavant. Aucun d’entre eux ne sera inquiété, d’aucune façon, pour des fautes commises par ignorance. 
${}^{32}J’ai d’ailleurs envoyé Ménélas pour vous encourager. 
${}^{33}Portez-vous bien. L’an 148, le 15 du mois de Xanthique. »
${}^{34}Les Romains aussi adressèrent aux Juifs une lettre qui était ainsi conçue :
      « Quintus Memmius et Titus Manius, légats des Romains, au peuple des Juifs, salut ! 
${}^{35}Les choses que Lysias, parent du roi, vous a accordées, nous aussi, nous y consentons. 
${}^{36}Quant aux choses qu’il a jugé devoir soumettre au roi, examinez-les, puis envoyez-nous sans délai quelqu’un qui nous en informe. Ainsi, nous pourrons les exposer au roi d’une façon qui vous convienne, car nous nous rendons à Antioche. 
${}^{37}Hâtez-vous donc d’envoyer des messagers, afin que nous sachions, nous aussi, quel est votre avis. 
${}^{38}Soyez en bonne santé. L’an 148, le 15 du mois de Xanthique. »
      
         
      \bchapter{}
      \begin{verse}
${}^{1}Après la conclusion de ces traités, Lysias retourna auprès du roi, tandis que les Juifs se remettaient aux travaux des champs. 
${}^{2}Mais parmi les gouverneurs militaires de la région, Timothée et Apollonios, fils de Gennaios, ainsi que Jérôme et Démophon, et aussi Nicanor le Cypriarque, ne leur laissaient ni trêve, ni repos. 
${}^{3}Quant aux habitants de Joppé, ils commirent un acte particulièrement infâme. Ils invitèrent les Juifs qui habitaient parmi eux à monter, avec leurs femmes et leurs enfants, sur des embarcations qu’ils avaient préparées. Ils paraissaient n’avoir aucune intention malveillante à leur égard, 
${}^{4}mais agir suivant une décision publique votée par la cité. Les Juifs acceptèrent, pour marquer qu’ils voulaient la paix et qu’ils étaient sans défiance. Mais quand ils furent au large, on les précipita par le fond. Ils étaient au nombre d’au moins deux cents. 
${}^{5}Dès que Judas apprit l’acte de cruauté commis contre les gens de sa nation, il le fit savoir à ses hommes. 
${}^{6}Après avoir invoqué Dieu, le juge impartial, il marcha contre les meurtriers de ses frères. De nuit, il incendia le port, brûla les embarcations, et transperça ceux qui s’y étaient réfugiés. 
${}^{7}Comme la place forte avait été fermée, il s’éloigna, mais avec l’intention de revenir pour détruire de fond en comble la cité de Joppé tout entière. 
${}^{8}Ayant appris que les habitants de Jamnia voulaient, eux aussi, agir de la même façon que ceux de Joppé à l’égard des Juifs qui habitaient chez eux, 
${}^{9}il attaqua également les Jamnites pendant la nuit et mit le feu au port et à sa flotte. Les lueurs des flammes étaient visibles jusqu’à Jérusalem, pourtant distante de deux cent quarante stades.
${}^{10}Comme il s’était éloigné avec son armée à neuf stades de là, lors d’une marche contre Timothée, des Arabes tombèrent sur lui, au nombre d’au moins cinq mille, avec cinq cents cavaliers. 
${}^{11}Un violent combat s’engagea et, grâce à l’aide de Dieu, les troupes de Judas connurent un beau succès. Vaincus, les nomades supplièrent Judas de leur tendre la main droite en signe de paix, promettant de lui livrer du bétail et de lui rendre service en toute autre circonstance. 
${}^{12}Judas, estimant qu’ils pourraient réellement lui être utiles en bien des choses, consentit à faire la paix avec eux. Ils se donnèrent donc la main droite, puis ils se retirèrent dans leurs tentes.
${}^{13}Judas attaqua aussi une ville forte du nom de Kaspine, entourée de remparts et habitée par un mélange de peuples. 
${}^{14}Confiants dans la solidité de leurs murs et dans leurs réserves de vivres, les assiégés se montraient extrêmement grossiers à l’égard des troupes de Judas, joignant les blasphèmes aux insultes et proférant des paroles sacrilèges. 
${}^{15}Mais ceux qui entouraient Judas invoquèrent le grand Souverain du monde, lui qui sans béliers ni machines de guerre renversa Jéricho au temps de Josué ; puis ils s’élancèrent sauvagement à l’assaut des remparts. 
${}^{16}Devenus maîtres de la ville par la volonté de Dieu, ils firent un carnage indescriptible, au point que l’étang voisin, large de deux stades, paraissait rempli du sang qui avait coulé.
${}^{17}À sept cent cinquante stades de là, ils atteignirent le camp retranché qui se trouvait chez les Juifs appelés Toubiens. 
${}^{18}Quant à Timothée, ils ne le trouvèrent pas en ces lieux, qu’il avait alors quittés sans avoir rien fait ; il avait cependant laissé en un certain lieu une garnison très forte. 
${}^{19}Dosithée et Sosipatros, deux généraux de l’entourage de Judas Maccabée, allèrent attaquer ce poste fortifié et tuèrent les hommes que Timothée y avait laissés, au nombre de plus de dix mille. 
${}^{20}Judas Maccabée, de son côté, divisa son armée en cohortes et en confia le commandement à ces deux généraux, puis il s’élança en expédition contre Timothée, qui était entouré de cent vingt mille fantassins et de deux mille cinq cents cavaliers. 
${}^{21}Informé de l’approche de Judas, Timothée commença par envoyer les femmes, les enfants et tout l’équipement au lieu appelé Carnione, une place imprenable et difficile d’accès en raison de l’étroitesse de tous les passages. 
${}^{22}Quand la cohorte de Judas apparut la première, la frayeur fondit sur les ennemis, et même la crainte devant l’apparition de Celui qui voit tout. Ils prirent la fuite dans toutes les directions, au point d’être souvent gênés l’un par l’autre et blessés par les pointes de leurs épées. 
${}^{23}Judas les poursuivit avec une vigueur extrême, embrochant ces criminels dont il fit périr jusqu’à trente mille hommes. 
${}^{24}Timothée lui-même tomba aux mains des hommes de Dosithée et de Sosipatros, mais il les supplia, avec une grande fourberie, de le laisser aller sain et sauf, prétendant qu’il avait en son pouvoir des parents – et même des frères – de beaucoup d’entre eux, à qui il pourrait bien arriver malheur. 
${}^{25}Par de longs discours, il les persuada de sa détermination à libérer ces hommes sans leur faire de mal. Alors, ils le relâchèrent, pour sauver leurs frères.
${}^{26}Ensuite Judas marcha sur Carnione et le temple de la déesse Atargatis où il égorgea vingt-cinq mille personnes.
${}^{27}Après la déroute et l’extermination de ces gens-là, il conduisit son armée contre Éphrone, ville forte où habitait une population de diverses tribus. De robustes jeunes gens, ayant pris position devant les remparts, combattaient avec vigueur. En outre, il y avait là de grandes quantités de machines et de projectiles en réserve. 
${}^{28}Après avoir invoqué le Seigneur souverain qui a le pouvoir de briser les forces ennemies, les Juifs se rendirent maîtres de la ville et abattirent environ vingt-cinq mille personnes à l’intérieur. 
${}^{29}Partis de là, ils s’élancèrent vers Scythopolis, à six cents stades de Jérusalem. 
${}^{30}Mais les Juifs qui s’étaient établis en ce lieu attestèrent que les habitants de Scythopolis avaient eu pour eux de la bienveillance et leur avaient réservé un accueil aimable au temps du malheur. 
${}^{31}Alors, Judas et les siens les remercièrent et les encouragèrent à garder à l’avenir ces bonnes dispositions envers leur race. Puis ils rentrèrent à Jérusalem, parce que la fête des Semaines était proche.
${}^{32}Après la fête appelée Pentecôte, ils s’élancèrent contre Gorgias, gouverneur militaire de l’Idumée. 
${}^{33}Celui-ci sortit à la tête de trois mille fantassins et de quatre cents cavaliers. 
${}^{34}Dans la bataille qui s’engagea, un petit nombre de Juifs succombèrent. 
${}^{35}Mais un certain Dosithée, parmi les soldats de Bakénor, cavalier et homme robuste, parvint à s’emparer de Gorgias. Le tenant par le manteau, il l’entraînait avec vigueur, dans l’intention de capturer vivant ce maudit. Mais un cavalier thrace, fonçant sur Dosithée, lui trancha l’épaule. Gorgias s’échappa et s’enfuit à Marisa. 
${}^{36}Cependant, les hommes d’Esdrias combattaient depuis longtemps et tombaient d’épuisement. Alors Judas invoqua le Seigneur pour qu’il se montre leur allié et leur guide au combat. 
${}^{37}Il se mit à pousser le cri de guerre dans la langue de ses pères et entonna des hymnes, puis se jeta à l’improviste sur les hommes de Gorgias et les mit en déroute.
${}^{38}Ensuite, Judas regroupa son armée et gagna la ville d’Odollam. Comme c’était le septième jour de la semaine, ils se purifièrent, selon la coutume, et célébrèrent le sabbat en ce lieu. 
${}^{39}Le lendemain, alors qu’il était devenu grand temps de le faire, les hommes de Judas vinrent enlever les corps de ceux qui avaient succombé dans la bataille, afin de les déposer avec leurs proches dans les tombeaux de leurs pères. 
${}^{40}Or, ils trouvèrent sous la tunique de chacun des morts des objets consacrés aux idoles de Jamnia, ce que la Loi interdit aux Juifs. Il fut évident pour tous que c’est pour cette raison qu’ils avaient succombé. 
${}^{41}Tous bénirent donc la conduite du Seigneur, le juge impartial qui rend manifestes les choses cachées. 
${}^{42}Puis, ils se répandirent en supplications pour demander que le péché commis soit entièrement effacé. Le noble Judas exhorta la troupe à se garder de tout péché, ayant sous les yeux le malheur de ceux qui avaient succombé pour avoir commis cette faute. 
${}^{43}Il organisa une collecte auprès de chacun et envoya deux mille pièces d’argent\\à Jérusalem afin d’offrir un sacrifice pour le péché. C’était un fort beau geste, plein de délicatesse, inspiré par la pensée de la résurrection. 
${}^{44}Car, s’il n’avait pas espéré que ceux qui avaient succombé ressusciteraient, la prière pour les morts était superflue et absurde. 
${}^{45}Mais il jugeait qu’une très belle récompense est réservée à ceux qui meurent avec piété : 
${}^{46}c’était\\là une pensée religieuse et sainte\\. Voilà pourquoi il fit ce sacrifice d’expiation, afin que les morts soient délivrés de leurs péchés.
      
         
      \bchapter{}
      \begin{verse}
${}^{1}En l’année 149 de l’empire grec, la nouvelle parvint à Judas et aux siens qu’Antiocos Eupator marchait sur la Judée avec une armée considérable. 
${}^{2}Il était accompagné de Lysias, son tuteur, administrateur du royaume. Ils disposaient d’une armée grecque de cent dix mille fantassins, cinq mille trois cents cavaliers, vingt-deux éléphants et trois cents chars armés de faux. 
${}^{3}Le grand prêtre Ménélas aussi se joignit à eux. Avec beaucoup d’habileté, il cherchait à convaincre Antiocos, non pour le salut de sa patrie, mais avec l’espoir d’être rétabli dans sa dignité. 
${}^{4}Mais le Seigneur, le Roi des rois, éveilla contre ce criminel la colère d’Antiocos. En effet, Lysias ayant démontré à ce dernier que Ménélas était la cause de tous les malheurs, Antiocos ordonna de le conduire à Bérée et de l’y exécuter suivant la coutume du lieu. 
${}^{5}Or, il y a en ce lieu une tour de cinquante coudées, remplie de cendre, munie d’un dispositif tournant qui, de tout autour, précipite dans la cendre. 
${}^{6}C’est de là que l’on pousse, pour le supprimer, celui qui est coupable de vol sacrilège ou de quelque autre crime particulièrement grave. 
${}^{7}Tel fut le supplice par lequel mourut cet homme infidèle à la Loi, ce Ménélas, qui ne reçut même pas de sépulture. 
${}^{8}Et ce n’était que justice : en effet, lui qui avait commis de nombreux péchés contre l’autel, dont le feu et la cendre étaient sacrés, c’est dans la cendre qu’il trouva la mort.
${}^{9}Entre-temps, le roi avançait, ruminant les projets les plus barbares et décidé à infliger aux Juifs les pires traitements que son père leur avait fait subir. 
${}^{10}Apprenant cela, Judas prescrivit au peuple d’invoquer le Seigneur jour et nuit : il espérait que maintenant encore, comme en d’autres circonstances, le Seigneur viendrait en aide 
${}^{11}à ceux qui étaient menacés de perdre la Loi, la patrie et le Temple saint, et qu’il ne laisserait pas tomber aux mains des païens blasphémateurs ce peuple qui commençait seulement à reprendre haleine. 
${}^{12}Tous ensemble, ils firent donc cette prière, adressant au Seigneur miséricordieux des supplications accompagnées de larmes, jeûnant et se prosternant, pendant trois jours d’affilée. Puis Judas les encouragea et leur dit de se tenir prêts. 
${}^{13}Lui-même prit conseil des anciens et décida de ne pas attendre que l’armée du roi envahisse la Judée et s’empare de la ville, mais de se mettre en marche et de tenter une action décisive avec l’aide de Dieu. 
${}^{14}Ayant ainsi confié le sort des armes au Créateur du monde, il exhorta ses compagnons à lutter noblement jusqu’à la mort pour les lois, le Temple, la ville, la patrie et les institutions ; puis il établit son campement aux environs de Modine. 
${}^{15}Il donna aux siens pour mot d’ordre : « Victoire de Dieu ! » Il choisit les plus courageux parmi les jeunes gens, et partit de nuit attaquer les quartiers du roi. Dans le camp, il tua environ deux mille hommes et transperça le chef de file des éléphants avec son conducteur. 
${}^{16}Finalement, ils remplirent le camp de frayeur et de trouble, puis se retirèrent avec un plein succès, 
${}^{17}alors que déjà le jour commençait à poindre. Tout cela s’était accompli grâce à la protection dont le Seigneur couvrait Judas.
${}^{18}Ayant eu ainsi un avant-goût de la hardiesse des Juifs, le roi essaya de prendre les places fortes par des stratagèmes. 
${}^{19}Il s’approcha de Bethsour, puissante forteresse des Juifs, mais il fut repoussé, revint à la charge, fut vaincu. 
${}^{20}Judas fit passer aux assiégés les provisions nécessaires, 
${}^{21}mais Rhodocos, un membre de l’armée juive, mit les ennemis au courant des secrets ; il fut recherché, arrêté, puis éliminé. 
${}^{22}Pour la seconde fois, le roi parlementa avec les gens de Bethsour : il leur tendit la main droite en signe de paix, prit la leur, se retira. 
${}^{23}Il s’attaqua aux hommes de Judas et se fit battre. Il apprit que Philippe, laissé à la tête des affaires publiques à Antioche, s’était révolté. Il en fut bouleversé. Il se mit à traiter avec les Juifs, se soumit à leurs conditions et jura de respecter toutes les mesures justes. Il conclut un accord avec eux, offrit un sacrifice, honora le Temple et fit preuve de générosité envers le Lieu saint. 
${}^{24}Il fit bon accueil à Judas Maccabée et laissa Hégémonide comme gouverneur militaire depuis Ptolémaïs jusqu’au pays des Guerréniens. 
${}^{25}Puis il se rendit à Ptolémaïs. Les habitants de cette ville, mécontents du traité, s’en indignaient fortement et voulaient en rejeter les clauses. 
${}^{26}Alors, Lysias monta à la tribune, plaida le mieux possible, les persuada, les calma, les amena à la bienveillance. Puis il partit pour Antioche. Ainsi se déroulèrent l’offensive et la retraite du roi.
      
         
      \bchapter{}
      \begin{verse}
${}^{1}Trois ans plus tard, Judas et les siens apprirent que Démétrios, fils de Séleucos, avait abordé au port de Tripoli avec une puissante armée et une flotte 
${}^{2}et qu’il s’était rendu maître du pays, après avoir fait disparaître Antiocos et son tuteur Lysias. 
${}^{3}Or, un certain Alkime, qui avait été grand prêtre, mais qui s’était volontairement souillé au temps de la révolte contre les coutumes païennes, comprenant qu’en aucune façon il ne pourrait être sauvé, ni avoir encore accès à l’autel sacré, 
${}^{4}vint trouver le roi Démétrios. C’était aux alentours de l’an 151 de l’empire grec. Il lui offrit une couronne d’or et une palme, et en outre des rameaux d’olivier qui sont les redevances habituelles du Temple.
      Ce jour-là, il ne fit rien de plus, 
${}^{5}mais il saisit une occasion favorable à ses projets insensés lorsque Démétrios le convoqua à son conseil et l’interrogea sur les dispositions et les desseins des Juifs. Il répondit : 
${}^{6}« Parmi les Juifs, ceux que l’on appelle Assidéens, dirigés par Judas Maccabée, alimentent la guerre et les émeutes, empêchant le royaume de connaître la tranquillité. 
${}^{7}Aussi, ayant été dépouillé de ma dignité héréditaire – je veux dire celle de grand prêtre –, je suis venu ici, 
${}^{8}tout d’abord parce que j’ai un réel souci des intérêts du roi, ensuite parce que je me préoccupe également de mes concitoyens. En effet, la déraison de ceux que je viens de nommer plonge toute notre race dans la misère, et non des moindres ! 
${}^{9}Toi donc, ô roi, quand tu auras pris connaissance de chacun de ces points, veille sur notre pays et notre race menacée de toutes parts, et traite-nous aussi avec cette humanité et cette bienveillance dont tu fais preuve à l’égard de tous. 
${}^{10}Car tant que Judas reste en vie, la paix n’est pas possible. »
${}^{11}Dès qu’il eut parlé de la sorte, les autres amis du roi, hostiles à l’action de Judas, s’empressèrent d’enflammer plus encore la colère de Démétrios. 
${}^{12}Celui-ci désigna aussitôt Nicanor, qui avait commandé la troupe des éléphants. Il le nomma gouverneur militaire de Judée et l’envoya 
${}^{13}avec l’ordre de faire disparaître Judas lui-même, de disperser ceux qui étaient avec lui et d’introniser Alkime comme grand prêtre du plus grand des temples. 
${}^{14}Quant aux païens de Judée qui avaient fui devant Judas, ils se rassemblèrent en masse autour de Nicanor, pensant bien que les malheurs et les revers des Juifs seraient leur propre chance. 
${}^{15}Lorsqu’ils apprirent que Nicanor approchait et que les païens venaient les attaquer, les Juifs se couvrirent de poussière et implorèrent Celui qui avait constitué son propre peuple pour l’éternité et qui toujours secourait sa part d’héritage avec des signes manifestes. 
${}^{16}Sur l’ordre de leur chef, ils quittèrent aussitôt le lieu où ils se trouvaient, et rencontrèrent l’ennemi près du village de Dessaou. 
${}^{17}Simon, le frère de Judas, avait engagé le combat avec Nicanor mais, brusquement décontenancé devant les adversaires, il avait subi un léger échec. 
${}^{18}Toutefois, apprenant quelle était la bravoure des hommes de Judas, leur ardeur dans les combats livrés pour la patrie, Nicanor eut quelque crainte de s’en remettre au jugement par le sang. 
${}^{19}Aussi envoya-t-il Posidonios, Théodote et Mattathias pour tendre la main droite aux Juifs et recevoir la leur, en signe de paix. 
${}^{20}Les propositions de paix furent longuement examinées, le chef les communiqua aux troupes et, comme les avis semblaient unanimes, on approuva le traité. 
${}^{21}On fixa un jour où les chefs se rencontreraient en particulier. De part et d’autre, un char s’avança ; on disposa des sièges d’honneur. 
${}^{22}Judas avait placé aux endroits stratégiques des hommes en armes, prêts à combattre, au cas où les adversaires tenteraient brusquement un mauvais coup. Mais l’entretien aboutit à un accord. 
${}^{23}Nicanor séjourna à Jérusalem sans rien y faire de déplacé ; il renvoya même les foules de ses partisans qui s’étaient rassemblées en masse autour de lui. 
${}^{24}Il avait continuellement Judas en face de lui et il éprouvait pour cet homme une profonde sympathie. 
${}^{25}Il l’encouragea à se marier et à avoir des enfants. Judas se maria, jouit de la tranquillité et goûta la vie de tous les jours.
${}^{26}Mais Alkime, voyant la bonne entente de Nicanor et de Judas, se procura les clauses du traité, vint trouver Démétrios et lui dit que Nicanor avait des idées hostiles à l’État, puisqu’il avait désigné comme son successeur l’ennemi du royaume, Judas. 
${}^{27}Fou de rage, exaspéré par les calomnies d’Alkime, cet être malfaisant, le roi écrivit à Nicanor, lui déclarant qu’il était fort mécontent de ce traité de paix et lui ordonnant d’envoyer sans retard Judas Maccabée, enchaîné, à Antioche. 
${}^{28}Quand ces nouvelles parvinrent à Nicanor, il en fut bouleversé, car il lui en coûtait de violer ses engagements vis-à-vis d’un homme qui n’avait commis aucune injustice. 
${}^{29}Mais comme il n’était pas possible de s’opposer au roi, il guettait une occasion favorable pour exécuter cet ordre au moyen d’une manœuvre trompeuse. 
${}^{30}Judas Maccabée, quant à lui, s’aperçut que Nicanor se comportait avec plus de raideur à son égard et que son abord ordinaire se faisait plus rude. Pensant que ce raidissement ne présageait rien de très bon, il rassembla un grand nombre de ses partisans et se déroba à Nicanor. 
${}^{31}Celui-ci se rendit compte qu’il avait été bel et bien trompé par la manœuvre de Judas. Alors, il se rendit au Temple saint, le plus grand de tous, tandis que les prêtres offraient les sacrifices habituels. Il leur ordonna de lui livrer cet homme. 
${}^{32}Comme les prêtres déclarèrent sous serment ne pas savoir où se trouvait celui qu’on recherchait, 
${}^{33}Nicanor étendit la main droite vers le sanctuaire, et jura ceci : « Si vous ne me livrez pas Judas enchaîné, je raserai cette enceinte consacrée à Dieu, j’abattrai l’autel et j’élèverai en ce lieu un temple splendide pour Dionysos. » 
${}^{34}Sur ces paroles, il se retira. Mais les prêtres tendirent les mains vers le ciel, en invoquant Celui qui s’est toujours montré le défenseur de notre nation. Ils disaient : 
${}^{35}« Toi, Seigneur, tu n’as besoin de rien au monde, mais il t’a plu d’avoir parmi nous le Temple où tu demeures. 
${}^{36}Et maintenant, toi, le Saint, Seigneur de toute sanctification, préserve à jamais de la profanation cette maison qui vient d’être purifiée. »
${}^{37}Un certain Razis, parmi les anciens de Jérusalem, fut dénoncé à Nicanor. Cet homme aimait ses concitoyens et jouissait d’une excellente réputation. En raison de sa bienveillance, on l’appelait « père des Juifs ». 
${}^{38}Au début de la révolte contre les coutumes païennes, il avait été inculpé de judaïsme. Il s’était en effet dépensé corps et âme, de toute sa ténacité, à la défense du judaïsme. 
${}^{39}Nicanor, voulant donner la preuve de sa malveillance à l’égard des Juifs, envoya plus de cinq cents soldats pour arrêter Razis, 
${}^{40}car il pensait qu’en arrêtant cet homme, il causerait le malheur de tous. 
${}^{41}La troupe, sur le point de s’emparer de la tour où Razis s’était réfugié, enfonçait violemment le portail d’entrée, avec l’ordre de mettre le feu et de brûler les portes. Alors Razis, cerné de toutes parts, se jeta sur son épée, 
${}^{42}préférant mourir noblement plutôt que de tomber entre des mains criminelles et de subir des outrages indignes de sa noble naissance. 
${}^{43}Mais, dans l’urgence du combat, il manqua son coup. Comme la foule des soldats se ruait à l’intérieur des portes, il courut héroïquement sur la muraille et se précipita courageusement sur la foule. 
${}^{44}Tous reculèrent aussitôt, laissant un espace vide, au milieu duquel il tomba. 
${}^{45}Respirant encore et enflammé d’ardeur, il se releva tout ruisselant de sang et, malgré ses douloureuses blessures, il traversa la foule en courant. Enfin, debout sur une roche escarpée, 
${}^{46}ayant déjà perdu tout son sang, il s’arracha les entrailles, les prit à deux mains et les jeta sur la foule, invoquant le Maître de la vie et de l’esprit afin qu’il les lui rende un jour. C’est de cette manière qu’il mourut.
      
         
      \bchapter{}
      \begin{verse}
${}^{1}Nicanor apprit que Judas et les siens se trouvaient dans les parages de Samarie. Il fit le projet de les attaquer en toute sécurité le jour du repos hebdomadaire. 
${}^{2}Les Juifs qui le suivaient par contrainte lui dirent : « Ne va pas les faire périr d’une manière aussi cruelle et barbare ! Mais rends honneur au jour qui a été choisi de préférence et sanctifié par Celui qui voit tout. » 
${}^{3}Nicanor, ce triple scélérat, leur demanda s’il y avait au ciel un souverain qui prescrivait d’observer le jour du sabbat. 
${}^{4}On lui déclara : « C’est le Seigneur vivant lui-même, souverain dans le ciel, qui a ordonné de respecter le septième jour. » 
${}^{5}Nicanor reprit : « Et moi, je suis souverain sur terre : je commande qu’on prenne les armes pour accomplir le service du roi. » Toutefois, il ne fut pas maître d’accomplir son funeste projet.
      
         
       
${}^{6}Haussant la tête avec une extrême prétention, Nicanor était décidé à dresser un monument public de la victoire qu’il remporterait sur les hommes de Judas. 
${}^{7}Mais, de son côté, Judas Maccabée gardait une confiance inébranlable et le plein espoir d’obtenir la protection du Seigneur. 
${}^{8}Il exhortait ses compagnons à ne pas redouter l’approche des païens, mais à garder présents à l’esprit les secours déjà reçus du Ciel et à compter maintenant aussi sur la victoire qui leur viendrait du Tout-Puissant. 
${}^{9}En les encourageant par des mots de la Loi et des Prophètes, en leur rappelant aussi les combats qu’ils avaient déjà soutenus, il les remplit d’une détermination nouvelle. 
${}^{10}Ayant ainsi réveillé leur ardeur, il acheva de les stimuler en leur montrant combien les païens avaient manqué à leur parole et méprisé leurs serments. 
${}^{11}Il arma ainsi chacun d’eux, non pas de la sécurité que donnent les boucliers et les lances, mais du réconfort qu’on trouve dans les paroles de bon conseil. Il leur raconta en outre un songe digne de foi, une sorte de vision qui les réjouit tous. 
${}^{12}Voici ce qu’il avait vu : Onias, jadis grand prêtre, cet homme noble et bon, modeste dans son abord et doux de caractère, distingué dans son langage et adonné dès l’enfance à tout ce qui concerne la vertu, étendait les mains et priait pour toute la communauté des Juifs. 
${}^{13}Ensuite apparut de la même manière un homme remarquable par ses cheveux blancs et par sa dignité, dont le maintien était admirable et tout empreint de majesté. 
${}^{14}Prenant la parole, Onias disait : « Cet homme est l’ami de ses frères, celui qui prie beaucoup pour le peuple et pour la Ville sainte, Jérémie, le prophète de Dieu. » 
${}^{15}De la main droite, Jérémie tendit à Judas une épée d’or. En la lui donnant, il s’exprima ainsi : 
${}^{16}« Reçois cette épée sainte que Dieu te donne. Par elle, tu briseras tes adversaires. »
       
${}^{17}Ce très beau discours de Judas eut le pouvoir d’inciter à la vertu et de donner aux jeunes une âme virile. Rassurés, les Juifs résolurent de ne pas se retrancher dans un camp, mais de prendre héroïquement l’offensive et de décider de l’issue de la bataille en se jetant dans la mêlée avec toute leur bravoure. Car c’étaient la ville, les lieux saints et le Temple qui étaient menacés. 
${}^{18}En effet, la crainte qu’ils avaient pour leurs femmes et leurs enfants, ainsi que pour leurs frères et leurs proches, comptait moins que la plus grande et la première des craintes, celle qu’ils éprouvaient pour le Temple consacré. 
${}^{19}L’angoisse de ceux qui avaient été laissés en ville n’était pas moins grande : ils tremblaient pour l’attaque menée en rase campagne. 
${}^{20}Déjà, tous attendaient l’issue prochaine du combat, déjà les adversaires opéraient leur concentration, rangeaient leur armée en ordre de bataille, disposaient les éléphants en un point favorable et la cavalerie sur les ailes de l’armée. 
${}^{21}Judas Maccabée, observant les troupes en présence, la variété de leur armement et l’aspect féroce des éléphants, tendit les mains vers le ciel et invoqua le Seigneur, auteur de prodiges. Il savait bien, en effet, que la victoire ne résulte pas de la force des armes, mais que Dieu l’accorde, comme il le juge bon, à ceux qui en sont dignes. 
${}^{22}Il prononça l’invocation suivante : « Au temps d’Ézékias, roi de Judée, toi, Maître, tu as envoyé ton ange, et il a exterminé jusqu’à cent quatre-vingt-cinq mille hommes de l’armée de Sennakérib. 
${}^{23}Maintenant encore, ô Souverain des cieux, envoie un bon ange en avant de nous, pour semer terreur et effroi. 
${}^{24}Que, par la force de ton bras, soient frappés de crainte ceux qui, en blasphémant, marchent contre ton peuple saint ! » Et il acheva sur ces mots.
       
${}^{25}Alors que les hommes de Nicanor s’avançaient au son des trompettes et des chants de guerre, 
${}^{26}c’est avec des invocations et des prières que les hommes de Judas affrontèrent les ennemis. 
${}^{27}Combattant de leurs mains, priant Dieu de tout leur cœur, ils abattirent au moins trente-cinq mille hommes et se réjouirent grandement de cette manifestation de Dieu. 
${}^{28}L’action terminée, comme ils s’en retournaient dans la joie, ils reconnurent Nicanor, tombé avec toutes ses armes. 
${}^{29}Au milieu des cris et du tumulte, ils se mirent à bénir le Seigneur souverain dans la langue de leurs pères. 
${}^{30}Alors Judas, lui qui de tout son être, corps et âme, avait combattu au premier rang pour ses concitoyens, lui qui avait conservé pour les gens de sa nation l’affection de sa jeunesse, ordonna de trancher la tête de Nicanor et son bras jusqu’à l’épaule, et de les porter à Jérusalem. 
${}^{31}Arrivé là, il convoqua les gens de sa nation, plaça les prêtres devant l’autel et envoya chercher les occupants de la citadelle. 
${}^{32}Il exposa la tête de l’infâme Nicanor et la main que ce blasphémateur avait tendue avec insolence contre la sainte maison du Tout-Puissant. 
${}^{33}Puis, il coupa la langue de l’impie Nicanor avec l’ordre de la donner aux oiseaux, morceau par morceau, et de suspendre en face du Temple son bras, pour prix de sa folie. 
${}^{34}Alors, tournés vers le ciel, tous bénirent le Seigneur qui s’était manifesté. Ils disaient : « Béni soit celui qui a préservé de la profanation son Lieu saint ! »
       
${}^{35}Judas accrocha la tête de Nicanor à l’extérieur de la citadelle, comme un signe manifeste et visible pour tous de l’aide du Seigneur. 
${}^{36}Tous décrétèrent, par un vote public, de ne laisser nullement ce jour passer inaperçu, mais de le solenniser le treizième jour du douzième mois – le mois nommé Adar en araméen –, la veille du jour dit « de Mardochée ».
      <h2 class="intertitle hmbot" id="d85e125946">5. Épilogue (15,37-39)</h2>
${}^{37}Ainsi se déroulèrent donc les événements concernant Nicanor. Puisque la ville est demeurée depuis ce temps en possession des Hébreux, je finirai également en ce point mon récit. 
${}^{38}L’ouvrage est-il bien composé et réussi ? c’est ce que je voulais. Est-il au contraire faible et médiocre ? c’est tout ce que j’ai pu obtenir. 
${}^{39}De fait, il est nuisible de ne boire que du vin, tout comme il est nuisible de ne boire que de l’eau, tandis que le vin mêlé à l’eau est agréable et procure une délicieuse jouissance. De la même façon, c’est la construction du récit qui charme les oreilles de ceux qui lisent l’ouvrage. En voici donc la fin.
