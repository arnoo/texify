  
  
    
    \bbook{JÉRÉMIE}{JÉRÉMIE}
      
         
      \bchapter{}
      \begin{verse}
${}^{1}Paroles de Jérémie, fils de Helkias, l’un des prêtres qui étaient à Anatoth, au pays de Benjamin. 
${}^{2}La parole du Seigneur lui fut adressée au temps de Josias, fils d’Amone, roi de Juda, la treizième année de son règne ; 
${}^{3}puis au temps de Joakim, fils de Josias, roi de Juda, jusqu’à la fin de la onzième année de Sédécias, fils de Josias, roi de Juda, jusqu’à la déportation de Jérusalem, au cinquième mois.
      
         
${}^{4}La parole du Seigneur me fut adressée :
        ${}^{5}« Avant même de te façonner dans le sein de ta mère\\,
        je te connaissais ;
        \\avant que tu viennes au jour,
        je t’ai consacré ;
        je fais de toi un prophète pour les nations. »
${}^{6}Et je dis : « Ah ! Seigneur mon Dieu\\ ! Vois donc : je ne sais pas parler, je suis un enfant ! » 
${}^{7} Le Seigneur reprit :
        \\« Ne dis pas : “Je suis un enfant !”
        \\Tu iras vers tous ceux à qui je t’enverrai ;
        tout ce que je t’ordonnerai, tu le diras.
        ${}^{8}Ne les crains pas,
        car je suis avec toi pour te délivrer
        – oracle du Seigneur. »
        ${}^{9}Puis le Seigneur étendit la main et me toucha la bouche.
        Il me dit :
        \\« Voici, je mets dans ta bouche mes paroles !
        ${}^{10}Vois : aujourd’hui, je te donne autorité
        sur les nations et les royaumes,
        \\pour arracher et renverser,
        pour détruire et démolir,
        pour bâtir et planter. »
       
${}^{11}La parole du Seigneur me fut adressée : « Que vois-tu, Jérémie ? » Je dis : « C’est une branche d’amandier que je vois. » 
${}^{12}Le Seigneur me dit : « Tu as bien vu, car je veille sur ma parole pour l’accomplir. » 
${}^{13}Une deuxième fois, la parole du Seigneur me fut adressée : « Que vois-tu ? » Je dis : « C’est un chaudron bouillonnant que je vois ; il s’ouvre depuis le nord. » 
${}^{14}Le Seigneur me dit :
        \\« Du nord, va déferler le malheur
        sur tous les habitants du pays.
${}^{15}Voici, je convoque tous les clans des royaumes du nord
        – oracle du Seigneur.
        \\Ils arrivent, et chacun placera son trône
        à l’entrée des portes de Jérusalem,
        \\contre tous les remparts qui l’entourent
        et contre toutes les villes de Juda.
${}^{16}Je vais prononcer sur eux mes jugements
        à cause de toute leur méchanceté,
        \\car ils m’ont abandonné,
        ils ont brûlé de l’encens pour d’autres dieux
        et se sont prosternés devant l’œuvre de leurs mains.
        ${}^{17}Toi, mets ta ceinture autour des reins et lève-toi,
        tu diras contre eux tout ce que je t’ordonnerai.
        \\Ne tremble pas devant eux,
        sinon c’est moi qui te ferai trembler devant eux.
        ${}^{18}Moi, je fais de toi aujourd’hui une ville fortifiée,
        une colonne de fer, un rempart de bronze,
        \\pour faire face à tout le pays,
        aux rois de Juda et à ses princes,
        à ses prêtres et à tout le peuple du pays.
        ${}^{19}Ils te combattront,
        mais ils ne pourront rien contre toi,
        \\car je suis avec toi pour te délivrer
        – oracle du Seigneur. »
      
         
      \bchapter{}
      \begin{verse}
${}^{1}La parole du Seigneur me fut adressée : 
${}^{2} Va proclamer aux oreilles de Jérusalem :
        \\Ainsi parle le Seigneur :
        \\Je me souviens de la tendresse de tes jeunes années,
        ton amour de jeune mariée,
        \\lorsque tu me suivais au désert,
        dans une terre inculte\\.
        ${}^{3}Israël était consacré au Seigneur,
        première gerbe de sa récolte ;
        \\celui qui en mangeait était coupable :
        il lui arrivait malheur,
        – oracle du Seigneur.
        ${}^{4}Écoutez la parole du Seigneur, maison de Jacob
        et toutes les familles de la maison d’Israël.
        ${}^{5}Ainsi parle le Seigneur :
        \\En quoi vos pères m’ont-ils trouvé injuste,
        eux qui se sont éloignés de moi,
        \\qui ont suivi des dieux\\de néant
        pour devenir eux-mêmes néant ?
        ${}^{6}Ils n’ont pas dit :
        \\« Où est-il, le Seigneur,
        \\lui qui nous a fait monter de la terre d’Égypte
        et marcher dans le désert,
        \\terre aride et ravinée,
        terre sèche et sinistre,
        \\terre où personne n’est jamais passé,
        où aucun homme n’a jamais habité ? »
        ${}^{7}Je vous ai fait entrer dans une terre plantureuse
        pour vous nourrir de tous ses fruits.
        \\Mais à peine entrés, vous avez profané ma terre,
        changé mon héritage en abomination.
        ${}^{8}Les prêtres n’ont pas dit :
        « Où est-il, le Seigneur ? »
        \\Les dépositaires de la Loi ne m’ont pas connu,
        les pasteurs se sont révoltés contre moi ;
        \\les prophètes ont prophétisé au nom du dieu Baal,
        ils ont suivi des dieux\\qui ne servent à rien.
        ${}^{9}C’est pourquoi, de nouveau, je vais faire un procès contre vous,
        – oracle du Seigneur
        un procès contre les fils de vos fils.
        ${}^{10}Passez jusqu’aux rivages de l’Occident\\, et regardez ;
        envoyez faire des recherches en Orient\\,
        et regardez si pareille chose est arrivée !
        ${}^{11}Une nation a-t-elle jamais changé de dieux ?
        – Et ce ne sont même pas des dieux !
        \\Or mon peuple a échangé sa gloire
        contre ce qui ne sert à rien.
        ${}^{12}Cieux, soyez-en consternés, horrifiés, épouvantés !
        – oracle du Seigneur.
        ${}^{13}Oui, mon peuple a commis un double méfait :
        ils m’ont abandonné, moi, la source d’eau vive,
        \\et ils se sont creusé des citernes,
        des citernes fissurées qui ne retiennent pas l’eau !
         
${}^{14}Israël est-il un esclave acheté ?
        Est-il un esclave né dans la maison ?
        Pourquoi fait-il partie du butin ?
${}^{15}Contre lui rugissent de jeunes lions,
        ils ont donné de la voix
        \\et livré son pays à la dévastation :
        ses villes sont incendiées, vidées de leurs habitants.
         
${}^{16}Même les fils de Noph et de Tapanès
        rasent ta chevelure !
${}^{17}N’es-tu pas celle qui a tout provoqué,
        toi qui as abandonné le Seigneur ton Dieu,
        au temps même où il te guidait sur le chemin ?
${}^{18}Et maintenant, qu’as-tu à prendre le chemin de l’Égypte,
        pour boire l’eau du Nil ?
        \\Qu’as-tu à prendre le chemin d’Assour,
        pour boire l’eau du Fleuve ?
${}^{19}Que ta méchanceté te corrige,
        que tes infidélités te punissent !
        \\Comprends et vois comme il est mauvais et amer
        d’abandonner le Seigneur ton Dieu,
        \\de ne plus me craindre,
        – oracle du Seigneur, Dieu de l’univers.
         
${}^{20}Oui, depuis longtemps tu as brisé ton joug, rompu tes liens.
        Tu as dit : « Je ne servirai pas ! »
        \\Mais sur toute colline élevée, sous tout arbre vert,
        tu te couches, prostituée !
${}^{21}Moi pourtant, j’avais fait de toi une vigne de raisin vermeil,
        tout entière d’un cépage de qualité.
        \\Comment t’es-tu changée pour moi
        en vigne méconnaissable et sauvage ?
${}^{22}Tu aurais beau te lessiver à la soude,
        y rajouter quantité de potasse,
        \\devant moi, ta faute reste incrustée,
        – oracle du Seigneur Dieu.
${}^{23}Comment peux-tu dire : « Je ne me suis pas souillée,
        après les Baals je n’ai pas couru » ?
        \\Vois ton chemin dans la vallée.
        \\Comprends ce que tu as fait,
        chamelle volage dont les pistes s’embrouillent !
${}^{24}Ânesse habituée au désert,
        dans l’ardeur de son désir, elle aspire le vent :
        son rut, qui peut le refouler ?
        \\Ils n’ont pas à se fatiguer, tous ceux qui la cherchent :
        en son temps, ils la trouvent !
${}^{25}Prends garde, tu vas avoir les pieds nus,
        et le gosier assoiffé.
        \\Mais tu dis : « Rien à faire ! Non, j’aime les étrangers
        et je veux courir à leur suite ! »
         
${}^{26}Comme est confondu un voleur pris sur le fait,
        ainsi sont confondus les gens de la maison d’Israël,
        leurs rois, leurs princes, leurs prêtres et leurs prophètes,
${}^{27}eux qui disent au bois : « Tu es mon père »,
        et à la pierre : « Toi, tu m’as enfanté. »
        \\Ils tournent vers moi leur dos, non leur visage ;
        mais au temps de leur malheur, ils disent :
        « Lève-toi ! Sauve-nous ! »
${}^{28}Où sont-ils, les dieux que tu t’es fabriqués ?
        \\Qu’ils se lèvent, s’ils peuvent te sauver au temps de ton malheur !
        Oui, tes dieux sont aussi nombreux que tes villes, ô Juda !
${}^{29}Pourquoi donc me faire un procès ?
        \\Vous vous êtes tous révoltés contre moi,
        – oracle du Seigneur.
${}^{30}En vain, j’ai frappé vos fils :
        ils n’ont pas accepté la leçon.
        \\Votre épée a dévoré vos prophètes,
        comme un lion destructeur.
         
${}^{31}Ô vous, gens de cette génération, voyez ce que dit le Seigneur :
        \\Ai-je été un désert pour Israël ?
        Ai-je été un pays de ténèbres ?
        \\Pourquoi ceux de mon peuple disent-ils :
        \\« Nous errons çà et là,
        nous n’allons plus vers toi » ?
${}^{32}Une vierge oublie-t-elle ses ornements,
        une fiancée, sa parure ?
        \\Or mon peuple m’a oublié
        depuis des jours sans nombre.
${}^{33}Comme tu disposes bien ton chemin pour quêter l’amour !
        Avec de tels chemins tu enseignes même le mal.
${}^{34}Même sur les pans de ton vêtement,
        on trouve le sang des malheureux, des innocents,
        que tu n’as pas surpris en flagrant délit.
        <p class="verset_anchor" id="para_bib_jr_2_35">Et par-dessus tout cela, 
${}^{35}tu dis :
        \\« Je suis innocente :
        vraiment sa colère s’est détournée de moi ! »
        \\Voici que j’entre en jugement avec toi,
        puisque tu dis : « Je n’ai pas péché ! »
${}^{36}Que tu es versatile, toi qui changes de chemin !
        \\Même de l’Égypte tu auras honte,
        comme tu as eu honte d’Assour.
${}^{37}De là aussi tu sortiras, les mains sur la tête,
        car le Seigneur a rejeté ceux en qui tu mets ta confiance ;
        \\avec eux tu ne réussiras pas !
      
         
      \bchapter{}
${}^{1}Si un homme renvoie sa femme
        et qu’elle s’en aille de chez lui pour appartenir à un autre,
        peut-il encore revenir à elle ?
        \\N’en serait-elle pas profanée, cette terre-là ?
        \\Et toi, qui t’es prostituée à de nombreux amants,
        tu reviendrais à moi !
        – oracle du Seigneur.
${}^{2}Lève les yeux vers les hauteurs et vois :
        en quel endroit ne t’es-tu pas livrée ?
        \\Pour eux, tu étais assise sur les chemins,
        comme un Arabe dans le désert,
        \\et tu as profané le pays
        par tes prostitutions et ta malice.
${}^{3}Aussi les averses ont-elles été retenues
        et la pluie de printemps a-t-elle manqué.
        \\Mais tu avais un front de prostituée
        et tu refusais d’en rougir.
${}^{4}Encore maintenant, ne m’appelles-tu pas :
        « Mon père, toi le guide de ma jeunesse !
${}^{5}Gardera-t-il rancune à jamais ?
        Tiendra-t-il rigueur jusqu’à la fin ? »
        \\Voilà ce que tu dis,
        puis tu commets le mal, et tu y réussis !
        
           
       
${}^{6}Le Seigneur me dit, au temps du roi Josias : As-tu vu ce qu’a fait l’infidèle Israël ? Elle allait sur toute montagne élevée, sous tout arbre vert, pour s’y prostituer. 
${}^{7}Je me disais : « Après avoir fait tout cela, elle reviendra vers moi. » Mais elle n’est pas revenue ! Sa sœur, Juda la perfide, a vu. 
${}^{8}Et moi j’ai vu : quand j’avais renvoyé l’infidèle Israël, à cause de tous ses adultères, et lui avais donné sa lettre de répudiation, Juda, la sœur perfide, n’en a éprouvé aucune crainte ; elle aussi, elle est allée se prostituer. 
${}^{9}Par sa prostitution volage, elle a profané le pays. Avec la pierre et le bois, elle a commis l’adultère. 
${}^{10}Et même après tout cela, Juda, la sœur perfide, n’est pas revenue à moi de tout son cœur ! Ce n’était que mensonge ! – oracle du Seigneur.
${}^{11}Le Seigneur me dit : Au fond d’elle-même, l’infidèle Israël est juste, comparée à Juda la perfide ! 
${}^{12}Va proclamer au nord ces paroles. Tu diras :
        \\Reviens, infidèle Israël ! – oracle du Seigneur.
        Je ne ferai pas tomber sur vous ma colère,
        \\car je suis bon – oracle du Seigneur
        et je ne garde pas rancune à jamais.
${}^{13}Reconnais seulement ta faute :
        contre le Seigneur ton Dieu, tu t’es révoltée ;
        \\tu as couru en tous sens vers les étrangers,
        sous tout arbre vert.
        \\Et vous n’avez pas écouté ma voix,
        – oracle du Seigneur. »
        ${}^{14}Revenez, fils renégats\\ – oracle du Seigneur ;
        c’est moi qui suis votre maître\\.
        \\Je vais vous prendre, un par ville, deux par clan,
        et vous faire venir à Sion.
        ${}^{15}Je vous donnerai des pasteurs selon mon cœur :
        ils vous conduiront avec savoir et intelligence.
        ${}^{16}Quand vous vous serez multipliés,
        quand vous aurez fructifié dans le pays,
        \\en ces jours-là – oracle du Seigneur –,
        on ne dira plus « Arche de l’Alliance du Seigneur »,
        \\on ne gardera plus mémoire\\de l’Arche,
        on ne s’en souviendra plus,
        \\on ne s’en occupera plus,
        on n’en fera pas une autre.
         
        ${}^{17}En ce temps-là,
        on appellera Jérusalem « Trône du Seigneur ».
        \\Toutes les nations convergeront vers elle,
        vers le nom du Seigneur, à Jérusalem ;
        \\elles ne suivront plus les penchants mauvais
        de leur cœur endurci.
         
${}^{18}En ces jours-là, la maison de Juda ira vers la maison d’Israël ;
        ensemble, elles viendront du pays du nord
        vers le pays que j’ai donné en héritage à vos pères.
${}^{19}Or moi, je m’étais dit :
        \\« Comment te placer au rang des fils
        et te donner une terre désirable, un splendide héritage,
        toute la splendeur des nations ? »
        \\Je disais : « Tu m’appelleras “Mon père”,
        tu ne te détourneras plus de moi.
${}^{20}Mais comme une femme qui trahit son compagnon,
        ainsi m’avez-vous trahi, maison d’Israël,
        – oracle du Seigneur. »
         
${}^{21}Sur les hauteurs, une voix se fait entendre,
        pleurs et supplications des fils d’Israël ;
        \\car ils se sont dévoyés,
        ils ont oublié le Seigneur leur Dieu.
${}^{22}« Revenez, fils renégats !
        Je guérirai vos infidélités. »
        \\– « Nous voici, nous venons à toi,
        car tu es le Seigneur notre Dieu.
${}^{23}Oui, mensonge, ce qui vient des collines,
        tumulte, ce qui vient des montagnes !
        \\Oui, le salut d’Israël est dans le Seigneur notre Dieu !
${}^{24}La Honte a dévoré le travail de nos pères
        depuis notre jeunesse,
        \\leur petit et leur gros bétail,
        leurs fils et leurs filles.
${}^{25}Couchons-nous dans notre honte,
        que notre confusion nous couvre,
        \\car nous avons péché, nous et nos pères,
        contre le Seigneur notre Dieu,
        depuis notre jeunesse jusqu’à ce jour,
        \\et nous n’avons pas écouté la voix du Seigneur notre Dieu. »
       
      
         
      \bchapter{}
${}^{1}Si tu reviens Israël – oracle du Seigneur –,
        c’est à moi que tu reviendras.
        \\Si tu fais disparaître tes horreurs loin de ma face,
        tu n’auras plus à errer.
${}^{2}Alors tu jureras par le Seigneur vivant,
        dans la vérité, le droit et la justice ;
        \\et les nations se béniront en lui,
        en lui, elles se glorifieront.
        
           
         
${}^{3}Ainsi parle le Seigneur
        aux gens de Juda et à Jérusalem :
        \\Défrichez pour vous ce qui est en friche,
        ne semez pas dans les ronces !
${}^{4}Soyez circoncis pour le Seigneur,
        enlevez le prépuce de votre cœur,
        \\gens de Juda et habitants de Jérusalem,
        de peur que ma colère n’éclate comme un feu
        \\et ne brûle, sans personne pour l’éteindre,
        à cause de la malice de vos actes.
        
           
${}^{5}Annoncez-le en Juda,
        dans Jérusalem faites-le entendre,
        \\dites : « Sonnez du cor dans le pays ! »
        Criez à pleine voix,
        \\dites : « Rassemblez-vous !
        Entrons dans les villes fortifiées ! »
${}^{6}Vers Sion levez l’étendard,
        cherchez un refuge, ne vous arrêtez pas,
        \\car c’est le malheur que je fais venir du nord,
        et un grand désastre.
${}^{7}Le lion monte de son fourré,
        le destructeur des nations se met en route ;
        \\il sort de chez lui
        pour réduire ton pays en un lieu désolé :
        \\tes villes seront ruinées,
        vidées de leurs habitants.
${}^{8}À cause de cela, revêtez-vous de toile à sac,
        lamentez-vous et gémissez,
        \\car l’ardente colère du Seigneur
        ne s’est pas détournée de nous !
         
${}^{9}Il arrivera en ce jour-là – oracle du Seigneur
        que le cœur du roi et le cœur des princes défailliront.
        \\Les prêtres seront consternés,
        et les prophètes, stupéfaits.
${}^{10}Alors je dis : « Ah ! Seigneur mon Dieu,
        vraiment, tu as bien trompé ce peuple et Jérusalem,
        \\en disant : “Vous aurez la paix”,
        tandis que l’épée nous atteint à la gorge. »
${}^{11}En ce temps-là, on dira à ce peuple et à Jérusalem :
        \\Au désert, un vent brûlant des hauteurs
        est en route vers la fille de mon peuple,
        non pour vanner, non pour épurer ;
${}^{12}un vent plein de violence me vient de là-bas.
        \\Et moi, maintenant, je prononce contre eux des jugements.
${}^{13}Le voici qui monte comme les nuages ;
        ses chars sont pareils à l’ouragan,
        et ses chevaux, plus vifs que les aigles.
        \\Malheur à nous, car nous sommes dévastés !
         
${}^{14}Lave ton cœur de tout mal, Jérusalem,
        afin d’être sauvée !
        \\Combien de temps encore accueilleras-tu en toi
        des pensées malfaisantes ?
${}^{15}Oui, une voix l’annonce depuis Dane ;
        depuis la montagne d’Éphraïm, elle publie le malheur.
${}^{16}Répétez-le aux nations,
        publiez-le contre Jérusalem :
        \\des assaillants arrivent d’un pays lointain,
        ils élèvent la voix contre les villes de Juda.
${}^{17}Comme les gardiens d’un champ,
        ils sont là, tout autour de Jérusalem,
        \\car elle s’est révoltée contre moi
        – oracle du Seigneur.
${}^{18}Ta conduite et tes actes t’ont valu cela :
        Voilà ton malheur.
        \\Ah, quelle amertume ! Elle te frappe en plein cœur.
         
${}^{19}Oh ! Mes entrailles ! Mes entrailles !
        Au fond de moi, je me tords de douleur.
        \\Mon cœur gémit en moi,
        je ne peux pas me taire.
        \\Ô mon âme, tu as entendu
        l’appel du cor, le cri de guerre.
${}^{20}On proclame désastre sur désastre,
        car tout le pays est dévasté.
        \\Soudain, mes tentes sont dévastées,
        ainsi que mes abris, en un instant.
${}^{21}Combien de temps verrai-je l’étendard,
        entendrai-je l’appel du cor ?
         
${}^{22}Oui, mon peuple est fou :
        ils ne me connaissent pas.
        \\Ce sont des enfants stupides :
        ils n’ont pas de discernement.
        \\Ils sont sages pour faire le mal,
        mais ne savent pas faire le bien.
         
${}^{23}Je regarde la terre, et voici : c’est un chaos ;
        le ciel : il a perdu sa lumière.
${}^{24}Je regarde les montagnes, et voici : elles tremblent,
        toutes les collines sont secouées.
${}^{25}Je regarde, et voici qu’il n’y a plus d’hommes,
        tous les oiseaux du ciel ont fui.
${}^{26}Je regarde, et voici que le verger est un désert,
        toutes les villes sont détruites
        \\devant le Seigneur,
        devant l’ardeur de sa colère.
         
${}^{27}Ainsi parle le Seigneur :
        \\Toute la terre sera désolée,
        mais je n’en ferai pas l’extermination.
${}^{28}Aussi la terre sera-t-elle en deuil,
        et là-haut, le ciel s’obscurcira.
        \\Puisque je l’ai dit et décidé,
        je n’y renoncerai pas, je ne reviendrai pas en arrière.
         
${}^{29}À la clameur du cavalier et de l’archer,
        toute ville prend la fuite ;
        \\on s’enfonce dans les broussailles,
        on escalade les rochers ;
        \\toute ville est abandonnée,
        plus personne n’y habite.
         
${}^{30}Et toi, dévastée, que vas-tu faire ?
        \\Tu te revêts d’écarlate
        et te pares d’une parure d’or,
        \\tu te fardes les yeux pour les agrandir !
        C’est en vain que tu te fais belle !
        \\Ceux qui te convoitaient te méprisent,
        ils en veulent à ton âme.
${}^{31}J’entends une voix, comme celle d’une femme en travail,
        comme l’angoisse d’une jeune accouchée.
        \\C’est la voix de la fille de Sion ;
        elle suffoque, elle étend les mains :
        \\« Malheur à moi ! Mon âme défaille devant les tueurs. »
      
         
      \bchapter{}
${}^{1}Parcourez les rues de Jérusalem,
        regardez donc et renseignez-vous !
        \\Cherchez sur ses places :
        si vous trouvez un homme,
        \\un seul, qui pratique le droit
        et recherche la vérité,
        \\alors je pardonnerai à la ville.
${}^{2}Lorsqu’ils déclarent : « Le Seigneur est vivant ! »
        c’est pour faire un faux serment.
${}^{3}Ton regard, Seigneur, n’est-il pas tourné vers la vérité ?
        \\Tu les as frappés, et ils n’ont pas frémi ;
        tu les as exterminés, ils ont refusé la leçon.
        \\Ils ont rendu leur visage plus dur que le roc,
        ils ont refusé de se convertir.
${}^{4}Et moi, je me disais : « Ce sont des misérables,
        ils agissent comme des sots,
        \\parce qu’ils ne connaissent pas le chemin du Seigneur,
        ni le droit de leur Dieu.
${}^{5}Alors, j’irai chez les grands et je leur parlerai,
        \\car ceux-là connaissent le chemin du Seigneur
        et le droit de leur Dieu. »
        \\Mais eux aussi, ils ont brisé le joug,
        ils ont rompu les liens.
${}^{6}Voilà pourquoi le lion les attaque depuis la forêt,
        le loup des steppes les dévaste ;
        \\le léopard est aux aguets devant leurs villes :
        tous ceux qui en sortent sont mis en pièces ;
        \\car leurs révoltes se sont multipliées
        et leurs infidélités n’ont cessé de grandir.
${}^{7}Comment te pardonnerais-je ?
        \\Tes fils m’ont abandonné,
        ils ont juré par des dieux qui n’en sont pas.
        \\Je les avais comblés, ils ont commis l’adultère,
        ils se précipitent à la maison de la prostituée.
${}^{8}Ils sont comme des étalons en rut,
        chacun hennissant vers la femme de son prochain.
${}^{9}Et je pourrais ne pas sévir contre eux,
        – oracle du Seigneur –,
        \\ne pas me venger d’une telle nation ?
        
           
         
${}^{10}Escaladez ses terrasses et saccagez,
        mais n’en faites pas l’extermination.
        \\Enlevez ses sarments,
        car ils ne sont pas au Seigneur.
${}^{11}Oui, la maison d’Israël et la maison de Juda
        m’ont vraiment trahi – oracle du Seigneur.
${}^{12}Ils ont renié le Seigneur
        et ils ont dit : « Non, pas lui !
        \\Le malheur ne viendra pas sur nous ;
        nous ne verrons ni l’épée ni la famine.
${}^{13}Quant aux prophètes, ils ne sont que du vent ;
        la parole n’est pas en eux.
        \\Qu’il leur soit fait selon ce qu’ils disent ! »
${}^{14}C’est pourquoi, ainsi parle le Seigneur, Dieu de l’univers :
        \\Parce que vous avez dit cette parole,
        voici que je fais de mes paroles un feu dans ta bouche,
        et de ce peuple, du bois que le feu dévorera.
${}^{15}Contre vous, maison d’Israël,
        \\voici que je fais venir de loin une nation
        – oracle du Seigneur ;
        \\c’est une nation impétueuse, une nation très ancienne,
        une nation dont tu ne connais pas la langue,
        dont tu ne comprends pas ce qu’elle dit.
${}^{16}Leur carquois est un sépulcre béant,
        ils sont tous des héros !
${}^{17}Ils dévoreront ta moisson et ton pain,
        ils dévoreront tes fils et tes filles,
        \\ils dévoreront ton petit et ton gros bétail,
        ils dévoreront ta vigne et ton figuier ;
        \\par l’épée, ils réduiront à rien tes villes fortifiées
        dans lesquelles tu mets ta confiance.
        
           
         
${}^{18}Pourtant, même en ces jours-là – oracle du Seigneur –,
        je ne vous exterminerai pas.
${}^{19}Et quand vous demanderez :
        \\« Pourquoi le Seigneur notre Dieu nous a-t-il fait tout cela ? »,
        tu leur répondras :
        \\« De même que vous m’avez abandonné
        pour servir dans votre pays les dieux de l’étranger,
        \\de même vous servirez des étrangers
        dans un pays qui n’est pas le vôtre. »
        
           
${}^{20}Annoncez-le dans la maison de Jacob,
        faites-le entendre en Juda :
${}^{21}Écoutez donc ceci, peuple stupide et sans intelligence !
        \\– Ils ont des yeux et ne voient pas,
        des oreilles et n’entendent pas !
${}^{22}Et moi, ne me craindrez-vous pas ? – oracle du Seigneur.
        Ne tremblerez-vous pas devant moi ?
        \\J’ai posé le sable pour limite à la mer,
        pour frontière à jamais infranchissable ;
        \\ses vagues s’agitent, elles ne peuvent rien,
        elles mugissent, mais ne la franchiront pas.
${}^{23}Or le cœur de ce peuple est indocile et rebelle ;
        ils se sont détournés, ils s’en vont.
${}^{24}Ils n’ont pas dit en leur cœur :
        \\« Craignons le Seigneur notre Dieu,
        lui qui nous donne la pluie en sa saison,
        \\celle du printemps et celle de l’automne,
        lui qui assure les semaines prévues pour la moisson. »
${}^{25}Vos fautes ont dérangé tout cela,
        vos péchés vous ont privés de ces bienfaits.
${}^{26}Oui, dans mon peuple on trouve des méchants :
        ils guettent comme des oiseleurs à l’affût,
        \\ils dressent des pièges
        et ils attrapent des hommes.
${}^{27}Comme une cage remplie d’oiseaux,
        leur maison est remplie de rapines :
        c’est ainsi qu’ils ont grandi et se sont enrichis.
${}^{28}Gras et repus, ils ont même franchi les limites du mal ;
        ils n’ont pas jugé selon le droit la cause de l’orphelin,
        \\et ils en tirent profit.
        Ils n’ont pas rendu justice aux malheureux.
${}^{29}Et je pourrais ne pas sévir contre eux,
        – oracle du Seigneur –,
        ne pas me venger d’une telle nation ?
         
${}^{30}Des infamies, des choses monstrueuses
        se commettent dans le pays :
${}^{31}les prophètes prophétisent le mensonge,
        les prêtres se remplissent les mains
        – et mon peuple aime cela !
        \\Mais que ferez-vous quand viendra la fin ?
      
         
      \bchapter{}
${}^{1}Fils de Benjamin, cherchez refuge hors de Jérusalem !
        \\Sonnez du cor à Teqoa,
        sur Beth-ha-Kérem, dressez un signal,
        \\car un malheur survient depuis le nord,
        un grand désastre.
${}^{2}La toute belle, la délicate,
        la fille de Sion, je l’ai réduite au silence.
${}^{3}Vers elle se dirigent des bergers et leurs troupeaux ;
        contre elle, tout autour, ils ont planté leurs tentes,
        et chacun fait brouter sa part.
${}^{4}Sanctifiez-vous pour lui faire la guerre.
        Debout ! Montons à l’assaut en plein midi.
        \\Malheur à nous, le jour décline
        et les ombres du soir s’allongent !
${}^{5}Debout ! Montons à l’assaut en pleine nuit,
        détruisons ses citadelles.
        
           
         
${}^{6}Ainsi parle le Seigneur de l’univers :
        \\Coupez du bois, amoncelez un remblai contre Jérusalem,
        la ville qui sera punie ;
        tout en elle est oppression.
${}^{7}Comme une citerne garde son eau,
        elle garde sa malice.
        \\On entend chez elle violence et dévastation ;
        mes yeux ne voient que blessure et maladie.
${}^{8}Corrige-toi, Jérusalem,
        \\de peur que je me détache de toi,
        que je te réduise en lieu désolé, en terre inhabitée !
        
           
         
${}^{9}Ainsi parle le Seigneur de l’univers :
        \\On va grappiller à fond, comme la vigne,
        ce qui reste d’Israël.
        \\Que ta main repasse
        comme celle du vendangeur sur les sarments !
${}^{10}Devant qui parlerai-je,
        devant qui témoigner pour qu’ils entendent ?
        \\Voici que leur oreille est incirconcise :
        ils ne peuvent pas être attentifs.
        \\Voici que la parole du Seigneur
        est devenue pour eux une insulte :
        ils n’en veulent pas.
${}^{11}Je suis rempli de la colère du Seigneur
        et fatigué de la contenir !
        \\Déverse-la tout à la fois sur le petit enfant dans la rue
        et sur le groupe des jeunes gens.
        \\Oui, l’homme et la femme seront pris,
        le vieillard et celui qui est comblé de jours.
${}^{12}Leurs maisons passeront à des étrangers,
        ainsi que leurs champs et leurs femmes,
        \\car j’étendrai la main contre les habitants du pays,
        – oracle du Seigneur.
${}^{13}Du plus petit jusqu’au plus grand,
        ils sont tous assoiffés de profits ;
        \\du prophète jusqu’au prêtre,
        ils s’adonnent tous au mensonge.
${}^{14}Ils traitent à la légère la blessure de mon peuple,
        en disant : « Paix ! La paix ! »
        \\alors qu’il n’y a pas de paix.
${}^{15}Par leurs abominations ils se couvrent de honte,
        mais ils n’éprouvent pas la moindre honte,
        ils ne savent même plus rougir.
        \\Aussi tomberont-ils avec les autres,
        ils trébucheront au temps où je les visiterai,
        dit le Seigneur.
        
           
         
${}^{16}Ainsi parle le Seigneur :
        \\Arrêtez-vous en chemin et voyez,
        interrogez les sentiers de toujours.
        \\Où donc est le chemin du bien ? Suivez-le,
        et trouvez pour vous-mêmes le repos.
        \\Mais ils disent : « Nous ne le suivrons pas ! »
${}^{17}J’ai suscité pour vous des guetteurs :
        « Faites attention au son du cor ! »
        \\Mais ils disent : « Nous ne ferons pas attention. »
${}^{18}C’est pourquoi, nations, écoutez ;
        et toi, assemblée du peuple, apprends ce qui va leur arriver.
${}^{19}Terre, écoute !
        \\Voici que je fais venir sur ce peuple un malheur,
        le fruit de leurs projets,
        \\car ils n’ont pas fait attention à mes paroles,
        ils ont méprisé ma loi.
${}^{20}Que m’importe l’encens de Saba,
        ou les aromates d’une terre lointaine ?
        \\Vos holocaustes ne me plaisent pas,
        vos sacrifices ne me sont pas agréables.
${}^{21}C’est pourquoi, ainsi parle le Seigneur :
        \\Voici que je place devant ce peuple des embûches
        sur lesquelles pères et fils trébucheront ensemble,
        le voisin et son ami périront.
        
           
         
${}^{22}Ainsi parle le Seigneur :
        \\Voici qu’un peuple arrive du pays du nord ;
        aux confins de la terre, s’éveille une grande nation.
${}^{23}Ils empoignent arc et javelot,
        ils sont cruels, sans aucune compassion ;
        leur voix gronde comme la mer.
        \\Sur des chevaux ils sont montés,
        rangés comme un seul homme pour la bataille,
        contre toi, fille de Sion.
${}^{24}Nous en avons entendu la rumeur ;
        \\nos mains faiblissent, l’angoisse nous saisit,
        comme les douleurs d’une femme qui accouche.
${}^{25}Ne sortez pas dans les champs,
        n’allez pas sur la route :
        \\l’épée de l’ennemi est là,
        de tous côtés c’est l’épouvante.
${}^{26}Ô fille de mon peuple, revêts-toi de sac
        et roule-toi dans la cendre !
        \\Prends le deuil comme pour un fils unique :
        amertume et complainte ;
        \\car le dévastateur, soudain, arrive sur nous.
        
           
${}^{27}Je t’ai établi comme une forteresse,
        pour soumettre mon peuple à l’épreuve,
        \\afin que tu connaisses et éprouves sa conduite.
${}^{28}Ils sont tous des rebelles invétérés,
        des semeurs de calomnies, du bronze et du fer ;
        ce sont tous des destructeurs.
${}^{29}Le soufflet gronde,
        pour que le plomb soit éliminé par le feu.
        \\Mais c’est en vain que l’on s’efforce d’épurer,
        ce qui est mauvais ne se détache pas.
${}^{30}« Argent de rebut » : voilà comment on les appelle.
        Oui, le Seigneur les a mis au rebut.
      <p class="cantique" id="bib_ct-at_33"><span class="cantique_label">Cantique AT 33</span> = <span class="cantique_ref"><a class="unitex_link" href="#bib_jr_7_2">Jr 7, 2c-7</a></span>
      
         
      \bchapter{}
      \begin{verse}
${}^{1}Parole du Seigneur adressée à Jérémie\\ : 
${}^{2} Tiens-toi à la porte de la maison du Seigneur, et là, tu proclameras cette parole, tu diras :
      
         
       
        \\Écoutez la parole du Seigneur,
        vous tous de Juda,
        \\vous qui entrez par ces portes
        pour vous prosterner devant le Seigneur.
         
        ${}^{3}Ainsi parle le Seigneur de l’univers,
        le Dieu d’Israël :
        \\Rendez meilleurs vos chemins et vos actes :
        je vous ferai demeurer dans ce lieu.
         
        ${}^{4}Ne faites pas confiance à des paroles de mensonge,
        \\en disant : « Temple du Seigneur ! Temple du Seigneur !
        \\C’est ici le temple du Seigneur ! »
         
        ${}^{5}Si vraiment vous rendez meilleurs
        vos chemins et vos actes,
        \\si vraiment vous maintenez le droit
        entre un homme et son prochain,
         
        ${}^{6}si vous n’opprimez pas l’immigré,
        l’orphelin ou la veuve,
        \\si vous ne versez pas, dans ce lieu,
        le sang de l’innocent,
        \\si vous ne suivez pas, pour votre malheur,
        d’autres dieux,
         
        ${}^{7}alors, je vous ferai demeurer dans ce lieu,
        \\dans le pays que j’ai donné à vos pères,
        \\depuis toujours et pour toujours.
       
${}^{8}Mais voici, vous faites confiance à des paroles de mensonge qui ne servent à rien. 
${}^{9} Quoi ! Vous pouvez voler, tuer, commettre l’adultère, faire des faux serments, brûler de l’encens pour le dieu Baal, suivre d’autres dieux que vous ne connaissez pas ; 
${}^{10} et ensuite, dans cette Maison sur laquelle mon nom est invoqué, vous pouvez vous présenter devant moi, en disant : « Nous sommes sauvés » ; et vous faites\\toutes ces abominations ! 
${}^{11} Est-elle à vos yeux une caverne de bandits, cette Maison sur laquelle mon nom est invoqué ? Pour moi, c’est ainsi que je la vois – oracle du Seigneur.
${}^{12}Allez donc à Silo, ce lieu qui était le mien, où j’avais fait autrefois demeurer mon nom, et voyez ce que j’en ai fait à cause de la méchanceté de mon peuple Israël ! 
${}^{13}Or maintenant – oracle du Seigneur –, puisque vous avez commis tous ces actes – inlassablement je vous ai parlé sans que vous écoutiez, et je vous ai appelés sans que vous répondiez –, 
${}^{14}ce que j’ai fait de Silo, je le ferai de cette Maison sur laquelle mon nom est invoqué et dans laquelle vous mettez votre confiance, ce lieu que je vous ai donné, à vous et à vos pères. 
${}^{15}Et je vous rejetterai loin de ma face, comme j’ai rejeté tous vos frères, toute la race d’Éphraïm.
${}^{16}Toi, n’intercède pas en faveur de ce peuple, n’élève pour eux ni supplication, ni prière, n’insiste pas auprès de moi : je ne t’écouterai pas ! 
${}^{17}Ne vois-tu pas ce qu’ils font dans les villes de Juda et dans les rues de Jérusalem ? 
${}^{18}Les fils ramassent le bois, les pères allument le feu, et les femmes pétrissent la pâte : ils font des gâteaux pour la Reine du ciel, ils versent des libations à d’autres dieux ; c’est ainsi qu’ils m’offensent. 
${}^{19}Mais est-ce bien moi qu’ils offensent ? – oracle du Seigneur. N’est-ce pas plutôt eux-mêmes, pour leur propre honte ? 
${}^{20}C’est pourquoi, ainsi parle le Seigneur mon Dieu : Voici que mon ardente colère se déverse sur ce lieu, sur l’homme et le bétail, sur l’arbre des champs et le fruit du sol. Elle brûle et ne s’éteindra pas.
       
${}^{21}Ainsi parle le Seigneur de l’univers, le Dieu d’Israël : Ajoutez vos holocaustes à vos sacrifices et mangez-en la viande, 
${}^{22}car je n’ai rien dit à vos pères, ni rien ordonné, à propos des holocaustes et des sacrifices, le jour où je les fis sortir du pays d’Égypte. 
${}^{23}Mais voici l’ordre que je leur ai donné : « Écoutez ma voix : je serai votre Dieu, et vous, vous serez mon peuple ; vous suivrez tous les chemins que je vous prescris, afin que vous soyez heureux. » 
${}^{24}Mais ils n’ont pas écouté, ils n’ont pas prêté l’oreille, ils ont suivi les mauvais penchants de leur cœur endurci ; ils ont tourné leur dos et non leur visage. 
${}^{25}Depuis le jour où vos pères sont sortis du pays d’Égypte jusqu’à ce jour, j’ai envoyé vers vous, inlassablement, tous mes serviteurs les prophètes. 
${}^{26}Mais ils ne m’ont pas écouté, ils n’ont pas prêté l’oreille, ils ont raidi leur nuque, ils ont été pires que leurs pères. 
${}^{27}Tu leur diras toutes ces paroles, et ils ne t’écouteront pas. Tu les appelleras, et ils ne te répondront pas. 
${}^{28}Alors, tu leur diras : Voilà bien la nation qui n’a pas écouté la voix du Seigneur son Dieu, et n’a pas accepté de leçon ! La vérité s’est perdue, elle a disparu de leur bouche.
${}^{29}Rase ta chevelure de consacrée, Jérusalem, et jette-la !
        \\Sur les hauteurs, entonne une lamentation,
        \\car le Seigneur a rejeté et délaissé
        \\la génération qui provoque sa fureur.
${}^{30}Oui – oracle du Seigneur –, les fils de Juda ont fait ce qui est mal à mes yeux. Dans la Maison sur laquelle mon nom est invoqué, ils ont installé leurs horreurs pour la souiller. 
${}^{31}Ils ont édifié les lieux sacrés du Tofeth au Val-de-la-Géhenne pour consumer par le feu leurs fils et leurs filles ; cela, je ne l’avais pas ordonné, cela n’était pas venu à mon esprit ! 
${}^{32}C’est pourquoi, voici venir des jours – oracle du Seigneur –, où l’on ne dira plus : « Le Tofeth » ni « Val-de-la-Géhenne », mais « Val-du-Massacre », et où l’on enterrera, faute de place, même au Tofeth. 
${}^{33}Les cadavres de ce peuple serviront de pâture aux oiseaux du ciel et aux bêtes de la terre, sans que personne les dérange. 
${}^{34}Je ferai cesser dans les villes de Juda et dans les rues de Jérusalem les chants d’allégresse et les chants de joie, le chant de l’époux et le chant de l’épousée, car le pays ne sera plus qu’une ruine.
       
      
         
      \bchapter{}
      \begin{verse}
${}^{1}En ce temps-là – oracle du Seigneur –, on fera sortir de leurs tombeaux les ossements des rois de Juda et les ossements de ses princes, les ossements des prêtres et les ossements des prophètes, les ossements des habitants de Jérusalem. 
${}^{2}On les étalera devant le soleil, la lune et toute l’armée du ciel, qu’ils ont aimés, servis, suivis, consultés, et devant qui ils se sont prosternés. Ces ossements ne seront ni recueillis, ni enterrés : ils deviendront du fumier à la surface du sol. 
${}^{3}Et la mort sera préférable à la vie pour les derniers survivants de cette engeance mauvaise, dans les derniers lieux où je les aurai chassés – oracle du Seigneur de l’univers.
      
         
${}^{4}Tu leur diras : Ainsi parle le Seigneur :
        \\Ceux qui tombent ne se relèvent-ils pas ?
        Celui qui se détourne ne retourne-t-il pas ?
${}^{5}Pourquoi Jérusalem, cette ville infidèle,
        s’est-elle détournée pour toujours ?
        \\Ils se cramponnent à ce qui est trompeur
        et refusent de retourner.
${}^{6}J’ai fait attention et j’ai entendu :
        ce qu’ils disent ne tient pas.
        \\Pas un qui regretterait sa méchanceté
        en disant : « Qu’ai-je fait ? »
        \\Tous, ils retournent à leur course,
        comme un cheval qui se rue à la guerre.
${}^{7}Même la cigogne, dans le ciel,
        connaît la saison de ses migrations ;
        \\la tourterelle, l’hirondelle et la grue
        respectent le temps de leur venue ;
        \\mais mon peuple ne connaît pas
        l’ordre fixé par le Seigneur.
${}^{8}Comment pouvez-vous dire : « Nous sommes sages,
        et la loi du Seigneur est avec nous »,
        \\alors que le stylet mensonger des scribes l’a falsifiée ?
${}^{9}Les sages sont couverts de honte,
        consternés, pris au piège,
        \\pour avoir méprisé la parole du Seigneur ;
        quelle est donc leur sagesse ?
${}^{10}C’est pourquoi je donnerai leurs femmes à des étrangers
        et leurs champs à de nouveaux maîtres.
        \\Du plus petit jusqu’au plus grand,
        ils sont tous assoiffés de profits ;
        \\du prophète jusqu’au prêtre,
        ils s’adonnent tous au mensonge.
${}^{11}Ils traitent à la légère la blessure de la fille de mon peuple,
        en disant : « Paix ! La paix ! »
        \\alors qu’il n’y a pas de paix.
${}^{12}Par leurs abominations ils se couvrent de honte,
        mais ils n’éprouvent pas la moindre honte,
        ils ne savent plus rougir.
        \\Aussi tomberont-ils avec les autres,
        ils trébucheront au temps où ils seront visités,
        dit le Seigneur.
${}^{13}Avec eux, je vais en finir – oracle du Seigneur – :
        \\pas de raisins dans la vigne,
        pas de figues sur le figuier,
        le feuillage est flétri.
        \\Eux, je les donnerai aux passants.
         
${}^{14}Pourquoi restons-nous assis ?
        Rassemblez-vous ! Entrons dans les villes fortifiées !
        \\Et là, faisons silence
        puisque le Seigneur notre Dieu nous réduit au silence ;
        \\il nous abreuve d’eaux empoisonnées,
        car nous avons péché contre lui.
${}^{15}Nous attendions la paix, et rien de bon !
        le temps du remède, et voici l’épouvante !
${}^{16}Depuis Dane on entend le souffle des chevaux ;
        à leur hennissement, au bruit des puissants coursiers,
        toute la terre tremble.
        \\Ils arrivent,
        ils dévorent le pays et ce qu’il contient,
        la ville et ses habitants.
${}^{17}Oui, voici que je lâche parmi vous des serpents venimeux,
        contre lesquels il n’est pas de charmeur,
        \\et ils vous mordront – oracle du Seigneur.
${}^{18}Pas de remède pour mon chagrin,
        mon cœur gémit sur moi !
${}^{19}J’entends la plainte de la fille de mon peuple
        depuis un pays lointain :
        \\« Le Seigneur n’est-il plus en Sion ?
        Son roi n’est-il plus en elle ? »
        \\Pourquoi m’ont-ils offensé avec leurs idoles,
        ces vanités étrangères ?
${}^{20} « La moisson est passée, l’été est fini,
        et nous, nous ne sommes pas sauvés ! »
${}^{21}Par la blessure de la fille de mon peuple, je suis blessé,
        je suis assombri, et la stupeur me saisit.
${}^{22}N’y a-t-il pas de baume en Galaad ?
        N’y a-t-il pas de médecin là-bas ?
        \\Pourquoi donc n’est-elle pas guérie,
        la fille de mon peuple ?
         
${}^{23}Qui changera ma tête en source d’eau
        et mes yeux en fontaine de larmes,
        \\pour que je pleure, jour et nuit,
        les victimes de la fille de mon peuple ?
      
         
      \bchapter{}
${}^{1}Qui me donnera un gîte au désert ?
        \\Je veux abandonner mon peuple
        et m’en aller loin d’eux,
        car ils sont tous adultères,
        une bande de traîtres.
${}^{2}Avec le mensonge, ils arment leur langue comme un arc ;
        par la déloyauté, ils sont devenus forts dans le pays,
        \\car ils vont de méfait en méfait ;
        mais moi, ils ne me connaissent pas
        – oracle du Seigneur.
        
           
         
${}^{3}Gardez-vous chacun de votre compagnon,
        défiez-vous de tout frère,
        \\car tout frère ne pense qu’à supplanter,
        et tout compagnon sème la calomnie.
${}^{4}Ils se jouent chacun de son compagnon,
        ils ne disent pas la vérité ;
        \\ils exercent leur langue à mentir et à pécher,
        <p class="verset_anchor" id="para_bib_jr_9_5">ils n’ont plus la force 
${}^{5}de revenir ;
        \\violence sur violence,
        tromperie sur tromperie,
        \\ils refusent de me connaître
        – oracle du Seigneur.
${}^{6}C’est pourquoi, ainsi parle le Seigneur de l’univers :
        \\Voici que je vais les épurer, les éprouver.
        Que faire d’autre avec la fille de mon peuple ?
${}^{7}Leur langue est une flèche meurtrière :
        elle parle pour tromper ;
        \\des lèvres, chacun parle de paix à son compagnon,
        mais en lui-même il prépare une embuscade.
${}^{8}Et je pourrais ne pas sévir contre eux
        – oracle du Seigneur –,
        \\ne pas me venger d’une telle nation ?
        
           
         
${}^{9}Sur les montagnes, je fais retentir des pleurs et des plaintes,
        et dans les enclos du désert, une lamentation,
        \\car ils sont incendiés, plus personne n’y passe,
        on n’y entend plus la rumeur des troupeaux ;
        \\des oiseaux du ciel jusqu’au bétail,
        tous ont fui et disparu.
${}^{10}Je ferai de Jérusalem un tas de pierres,
        un repaire de chacals ;
        \\et des villes de Juda, un lieu désolé :
        plus personne n’y habitera.
${}^{11}Quel est l’homme assez sage pour comprendre cela,
        l’homme à qui le Seigneur a parlé pour qu’il l’annonce ?
        \\Pourquoi le pays est-il perdu,
        brûlé comme un désert où ne passe personne ?
        
           
${}^{12}Le Seigneur dit : Ils ont abandonné ma Loi, celle que j’avais mise sous leurs yeux ; ils n’ont pas écouté ma voix, ils n’ont pas marché selon ma Loi ; 
${}^{13}ils ont marché suivant les penchants de leur cœur endurci, à la suite des Baals que leur ont fait connaître leurs pères. 
${}^{14}C’est pourquoi, ainsi parle le Seigneur de l’univers, le Dieu d’Israël : Je vais nourrir ce peuple d’absinthe et l’abreuver d’eau empoisonnée ; 
${}^{15}je les disperserai parmi des nations qu’ils n’ont pas connues, ni eux ni leurs pères, et je les poursuivrai de l’épée jusqu’à en finir avec eux !
       
${}^{16}Ainsi parle le Seigneur de l’univers :
        \\Réfléchissez.
        \\Appelez les pleureuses : qu’elles viennent ;
        allez chercher les plus habiles : qu’elles viennent.
${}^{17}Qu’elles se hâtent et entonnent sur nous une plainte !
        \\Que nos yeux ruissellent de larmes,
        que nos paupières soient noyées de pleurs !
         
${}^{18}Oui, on entend depuis Sion les accents d’une plainte :
        \\« Comment ! Nous sommes dévastés, accablés de honte,
        car nous avons abandonné le pays,
        \\et ils ont abattu nos demeures ! »
${}^{19}Femmes, écoutez donc la parole du Seigneur ;
        que vos oreilles saisissent la parole de sa bouche.
        \\Apprenez à vos filles cette plainte,
        et cette lamentation, chacune à sa compagne,
${}^{20}car elle monte par nos fenêtres, la Mort,
        elle pénètre dans nos citadelles,
        \\elle fauche l’enfant dans la rue
        et les jeunes gens sur les places.
${}^{21}Parle ! Voici l’oracle du Seigneur :
        \\Des cadavres d’hommes tombent
        comme du fumier à la surface des champs,
        comme des gerbes derrière un moissonneur,
        \\et personne qui les ramasse.
         
${}^{22}Ainsi parle le Seigneur :
        \\Que le sage ne se vante pas de sa sagesse,
        que le fort ne se vante pas de sa force,
        que le riche ne se vante pas de sa richesse.
${}^{23}Mais celui qui se vante, qu’il se vante plutôt de ceci :
        avoir de l’intelligence pour me connaître,
        \\moi, le Seigneur qui exerce sur la terre
        la fidélité, le droit et la justice.
        \\Oui, en cela je me plais – oracle du Seigneur.
         
${}^{24}Voici venir des jours – oracle du Seigneur –
        où je châtierai tout homme circoncis dans sa chair,
${}^{25}en Égypte et en Juda, en Édom et chez les fils d’Ammone,
        en Moab et chez les hommes aux tempes rasées
        qui habitent le désert,
        \\car toutes ces nations sont des incirconcis
        comme toute la maison d’Israël : des incirconcis de cœur.
      
         
      \bchapter{}
      \begin{verse}
${}^{1}Écoutez, maison d’Israël, la parole que le Seigneur a prononcée sur vous. 
${}^{2}Ainsi parle le Seigneur :
        \\N’apprenez pas la conduite des nations ;
        \\ne soyez pas terrifiés par les signes du ciel,
        du fait que les nations en sont terrifiées :
${}^{3}les décrets des peuples ne sont que vanité.
        \\Voici du bois coupé dans la forêt,
        travaillé au ciseau par la main de l’artisan,
${}^{4}enjolivé d’argent et d’or ;
        \\on les fixe avec des clous et des marteaux
        pour qu’ils ne vacillent pas.
${}^{5}Ils sont comme un épouvantail dans un champ,
        ils ne parlent pas.
        \\On doit les porter, car ils ne marchent pas.
        \\N’en ayez pas peur : ils ne peuvent faire du mal,
        et du bien, pas davantage !
${}^{6}Nul n’est comme toi, Seigneur ;
        tu es grand, ton nom est grand et puissant.
${}^{7}Qui ne te craindrait, roi des nations,
        comme cela te revient ?
        \\Oui, parmi tous les sages des nations
        et dans tous leurs royaumes,
        nul n’est comme toi.
${}^{8}Tous, sans exception, ils sont stupides et fous :
        le bois montre leur vanité.
${}^{9}L’argent qui le recouvre est importé de Tarsis,
        et l’or, d’Oufaz.
        \\C’est l’œuvre d’un artisan,
        façonnée par un orfèvre,
        \\revêtue de pourpre violette ou écarlate,
        tout entière œuvre d’artistes.
${}^{10}Mais le Seigneur est le Dieu véritable,
        c’est lui le Dieu vivant, le roi pour toujours.
        \\Dès qu’il s’emporte, la terre tremble,
        et les nations ne peuvent soutenir son courroux.
         
${}^{11}– Vous leur parlerez ainsi : les dieux qui n’ont pas fait le ciel et la terre, qu’ils disparaissent de la terre et de dessous le ciel !
         
${}^{12}Il fait la terre par sa puissance,
        il établit le monde par sa sagesse,
        et par son intelligence il a déployé les cieux.
${}^{13}Quand il donne de la voix,
        et que les eaux grondent dans les cieux,
        \\des extrémités de la terre il fait monter les nuages,
        il lance les éclairs pour la pluie,
        il libère le vent qu’il tenait en réserve.
         
${}^{14}Tout homme est stupide, faute de connaissance,
        tout orfèvre, méprisable à cause de son idole ;
        \\ce qu’il a coulé est un mensonge :
        pas de souffle en elle !
${}^{15}Ce n’est que vanité, œuvre dérisoire ;
        au temps du châtiment, tout disparaîtra.
${}^{16}Il n’est pas ainsi, Celui qui est la part de Jacob,
        car il a façonné toute chose ;
        \\Israël est la tribu de son héritage,
        son nom est « Le Seigneur de l’univers ».
${}^{17}Ramasse à terre ton bagage,
        toi, la ville en état de siège ;
${}^{18}car ainsi parle le Seigneur :
        \\Cette fois-ci, je vais poursuivre à la fronde les habitants du pays,
        je les regrouperai pour qu’on les atteigne.
         
${}^{19}Malheur à moi ! Quelle blessure ! Quelle plaie profonde !
        \\Et moi, j’ai dit : Si telle est ma souffrance,
        je la supporterai.
${}^{20}Ma tente est dévastée, toutes mes cordes sont rompues ;
        mes fils m’ont quittée, ils ne sont plus :
        \\personne pour dresser encore ma tente,
        pour monter mes abris.
${}^{21}En effet, les pasteurs sont stupides,
        ils n’ont pas consulté le Seigneur.
        \\C’est pourquoi ils manquent d’intelligence,
        et tout leur troupeau s’est dispersé.
         
${}^{22}On entend une rumeur, la voici : elle approche !
        \\C’est un énorme ébranlement qui vient du pays du nord,
        \\pour réduire les villes de Juda en lieu désolé,
        en repaire de chacals.
${}^{23}Seigneur, je le sais :
        \\l’homme n’est pas maître de son chemin ;
        à celui qui marche il n’est pas donné d’assurer son pas.
${}^{24}Corrige-moi, Seigneur, avec justice,
        mais sans colère, sinon tu me réduirais à rien.
${}^{25}Déverse ta fureur sur les nations
        qui ne te connaissent pas,
        \\sur les tribus qui n’invoquent pas ton nom,
        \\car elles ont dévoré Jacob, dévoré et anéanti,
        elles ont ravagé son territoire.
      
         
      \bchapter{}
      \begin{verse}
${}^{1}Parole du Seigneur adressée à Jérémie : 
${}^{2}Écoutez les paroles de cette alliance ; dites-les aux gens de Juda et aux habitants de Jérusalem. 
${}^{3}Tu leur diras : « Ainsi parle le Seigneur, le Dieu d’Israël : Maudit soit l’homme qui n’écoute pas les paroles de cette alliance, 
${}^{4}paroles que j’ai prescrites à vos pères, le jour où je les fis sortir de la terre d’Égypte, cette fournaise à fondre le fer. Je leur ai dit : “Écoutez ma voix et mettez en pratique tout ce que je vous prescris : vous serez mon peuple, et moi, je serai votre Dieu.” 
${}^{5}Et ainsi je tiendrai le serment juré à vos pères de leur donner un pays ruisselant de lait et de miel, comme vous le voyez aujourd’hui. » Et je répondis : « Amen, Seigneur » !
${}^{6}Alors, le Seigneur me dit : « Proclame toutes ces paroles dans les villes de Juda et dans les rues de Jérusalem ! Tu leur diras : “Écoutez les paroles de cette alliance et mettez-les en pratique.” 
${}^{7}Car j’ai instamment averti vos pères, au jour où je les fis monter du pays d’Égypte, et jusqu’à ce jour. Sans cesse, je les ai avertis en disant : “Écoutez ma voix !” 
${}^{8}Mais ils n’ont pas écouté, ni prêté l’oreille ; chacun a suivi les penchants de son cœur endurci. Alors, je leur ai appliqué toutes les paroles de cette alliance, que je leur avais prescrit de mettre en pratique, et qu’ils n’ont pas mises en pratique. »
${}^{9}Et le Seigneur me dit : « Un complot a été découvert chez les gens de Juda et les habitants de Jérusalem. 
${}^{10}Ils sont retournés aux fautes de leurs pères d’autrefois qui refusaient d’écouter mes paroles, et ils ont suivi d’autres dieux pour les servir. La maison d’Israël et la maison de Juda ont rompu l’alliance que j’avais conclue avec leurs pères. 
${}^{11}C’est pourquoi, ainsi parle le Seigneur : Voici que je fais venir sur eux un malheur auquel ils ne pourront échapper. Ils crieront vers moi, et je ne les écouterai pas. 
${}^{12}Alors les villes de Juda et les habitants de Jérusalem iront crier vers les dieux pour lesquels ils ont brûlé de l’encens, mais ceux-là ne les sauveront pas au temps de leur malheur.
${}^{13}Oui, Juda, tes dieux sont aussi nombreux que tes villes, et les autels, aussi nombreux que les rues de Jérusalem, ces autels que vous avez édifiés pour la Honte, où l’on brûle de l’encens au dieu Baal.
${}^{14}Et toi, n’intercède pas en faveur de ce peuple, n’élève pour lui ni supplication ni prière, car je n’écouterai pas quand ils m’invoqueront à cause de leur malheur. »
       
${}^{15}Pourquoi vient-elle dans ma maison, ma bien-aimée,
        elle qui mène tant d’intrigues ?
        \\Arrête de sacrifier de la viande ;
        alors, quand le malheur t’atteindra, tu pourras bien jubiler !
${}^{16}« Olivier toujours vert, orné de fruits superbes »,
        ainsi t’avait nommée le Seigneur ;
        \\au bruit d’un grand fracas,
        \\il met le feu à ton feuillage,
        et tes rameaux sont mis à mal.
${}^{17}Et le Seigneur de l’univers, celui qui t’a plantée,
        a proféré contre toi le malheur,
        \\à cause du mal que la maison d’Israël et la maison de Juda
        se sont fait à elles-mêmes,
        \\en brûlant, pour m’offenser, de l’encens au dieu Baal.
        ${}^{18}Seigneur, tu m’as fait savoir,
        et maintenant je sais\\,
        \\tu m’as fait voir leurs manœuvres.
        ${}^{19}Moi, j’étais comme un agneau docile
        qu’on emmène à l’abattoir,
        \\et je ne savais pas qu’ils montaient un complot contre moi.
        \\Ils disaient : « Coupons l’arbre à la racine\\,
        retranchons-le de la terre des vivants,
        afin qu’on oublie jusqu’à son nom. »
        ${}^{20}Seigneur de l’univers, toi qui juges avec justice,
        qui scrutes les reins et les cœurs,
        \\fais-moi voir la revanche que tu leur infligeras,
        car c’est à toi que j’ai remis ma cause\\.
         
${}^{21}Oui, ainsi parle le Seigneur
        contre les gens d’Anatoth qui en veulent à ta vie
        \\et qui disent : « Ne prophétise pas au nom du Seigneur,
        sinon tu mourras de notre main ! »
${}^{22}Oui, ainsi parle le Seigneur de l’univers :
        \\Je vais les châtier,
        \\leurs jeunes gens mourront par l’épée,
        leurs fils et leurs filles mourront par la famine.
${}^{23}Il ne leur restera plus personne,
        car je ferai venir le malheur sur les gens d’Anatoth,
        l’année de leur châtiment.
       
      
         
      \bchapter{}
${}^{1}Tu es trop juste, Seigneur, pour que je te fasse un procès ;
        pourtant, je parlerai contre toi de jugement :
        \\Pourquoi le chemin des méchants est-il prospère ?
        Pourquoi sont-ils paisibles, tous les traîtres ?
${}^{2}Tu les plantes, et ils s’enracinent ;
        ils vont bien, et ils portent du fruit.
        \\Tu es près de leur bouche et loin de leur cœur !
${}^{3}Mais toi, Seigneur, tu me connais, tu me vois,
        tu scrutes mon cœur : il est avec toi.
        \\Traîne-les à l’abattoir comme des moutons,
        réserve-les pour le jour du massacre !
${}^{4}Combien de temps encore la terre sera-t-elle en deuil,
        et toute l’herbe des champs, desséchée ?
        \\Les bêtes et les oiseaux ont disparu
        à cause de la malice de ses habitants qui disaient :
        « Dieu ne voit pas notre avenir. »
${}^{5}Si la course avec des coureurs te fatigue,
        comment rivaliseras-tu avec des chevaux ?
        \\S’il te faut un pays en paix pour être confiant,
        comment feras-tu dans les maquis du Jourdain ?
${}^{6}Oui, même tes frères et la maison de ton père,
        même eux te trahissent,
        même eux parlent sans retenue derrière toi.
        \\Ne les crois pas quand ils te diront de bonnes paroles.
        
           
${}^{7}J’ai abandonné ma maison, délaissé mon héritage,
        livré ma bien-aimée à la poigne de ses ennemis.
${}^{8}Mon héritage a été pour moi comme un lion dans la forêt :
        contre moi il a donné de la voix, aussi l’ai-je détesté.
${}^{9}Mon héritage est pour moi un étrange rapace,
        et contre lui font cercle les rapaces.
        \\Allez, rassemblez toutes les bêtes sauvages :
        qu’elles viennent le dévorer !
${}^{10}De nombreux pasteurs ont saccagé ma vigne,
        piétiné la part qui me revient ;
        \\ils ont changé ma part délicieuse
        en solitude désolée.
${}^{11}Ils l’ont réduite en lieu désolé :
        la voici devant moi en deuil et désolée ;
        \\tout le pays est désolé,
        et personne ne prend cela à cœur.
${}^{12}Sur toutes les hauteurs du désert,
        s’avancent les dévastateurs :
        \\c’est l’épée du Seigneur qui dévore,
        d’une extrémité à l’autre de la terre.
        \\Plus de paix pour aucun être de chair !
${}^{13}Ils ont semé du blé, ils moissonnent des ronces ;
        ils se sont fatigués, ils n’en profitent pas.
        \\Ils sont confus de leurs récoltes,
        à cause de l’ardente colère du Seigneur.
${}^{14}Ainsi parle le Seigneur : Tous mes mauvais voisins,
        \\qui ont touché à l’héritage
        dont j’ai fait hériter mon peuple Israël,
        \\je vais les arracher de leur sol,
        j’arracherai du milieu d’eux la maison de Juda.
${}^{15}Mais après les avoir arrachés,
        je reviendrai à ma tendresse pour eux
        \\et je les ferai revenir, chacun dans son héritage,
        chacun dans son pays.
${}^{16}Et s’ils apprennent bien les chemins de mon peuple,
        jurant par mon nom : « Le Seigneur est vivant ! »,
        \\comme ils ont appris à mon peuple à jurer par le dieu Baal,
        alors, ils seront établis au milieu de mon peuple.
${}^{17}Mais s’ils n’écoutent pas,
        j’arracherai cette nation-là pour la faire périr,
        – oracle du Seigneur.
      
         
      \bchapter{}
      \begin{verse}
${}^{1}Ainsi m’a parlé le Seigneur : « Va, tu achèteras une ceinture de lin et tu la mettras sur tes reins. Évite de la tremper dans l’eau. » 
${}^{2} Selon la parole du Seigneur, j’ai acheté une ceinture et je l’ai mise sur mes reins. 
${}^{3} De nouveau, la parole du Seigneur me fut adressée : 
${}^{4} « Avec la ceinture que tu as achetée et que tu portes sur les reins, lève-toi, va jusqu’à l’Euphrate, et là-bas cache-la dans la fente d’un rocher. » 
${}^{5} Je suis donc allé la cacher près de l’Euphrate, comme le Seigneur me l’avait ordonné. 
${}^{6} Longtemps après, le Seigneur m’a dit : « Lève-toi, va jusqu’à l’Euphrate, et reprends la ceinture que je t’ai ordonné de cacher là-bas. » 
${}^{7} Je suis donc allé jusqu’à l’Euphrate, j’ai creusé, et j’ai repris la ceinture de l’endroit où je l’avais cachée. Et voici : la ceinture était pourrie, hors d’usage !
${}^{8}Alors la parole du Seigneur me fut adressée : 
${}^{9} « Ainsi parle le Seigneur : Voilà comment je ferai pourrir l’immense orgueil de Juda et de Jérusalem. 
${}^{10} Ce peuple mauvais, qui suit les penchants de son cœur endurci\\et qui marche à la suite d’autres dieux, pour les servir et se prosterner devant eux, il deviendra pareil à cette ceinture qui est hors d’usage. 
${}^{11} En effet, de même qu’un homme s’attache une ceinture autour des reins, de même je m’étais attaché toute la maison d’Israël et toute la maison de Juda – oracle du Seigneur, pour qu’elles soient mon peuple, mon renom, ma louange et ma parure. Mais elles n’ont pas écouté ! »
${}^{12}Tu leur diras cette parole : « Ainsi parle le Seigneur, le Dieu d’Israël : Toute cruche peut être remplie de vin. » Ils te diront : « Ne savons-nous pas que toute cruche peut être remplie de vin ? » 
${}^{13}Alors tu leur répondras : « Ainsi parle le Seigneur : Tous les habitants de ce pays, les rois qui siègent sur le trône de David, les prêtres, les prophètes et les habitants de Jérusalem, je vais les remplir d’ivresse. 
${}^{14}Puis, je les briserai chacun contre son frère, les pères et les fils tous ensemble – oracle du Seigneur. Je serai sans pitié, sans merci, sans compassion pour les détruire. »
${}^{15}Écoutez ! Tendez l’oreille ! Ne vous pavanez pas,
        car le Seigneur parle.
${}^{16}Rendez gloire au Seigneur votre Dieu,
        avant qu’il ne fasse venir les ténèbres,
        \\avant que vos pieds ne s’entrechoquent
        dans les montagnes du crépuscule.
        \\Vous attendrez la lumière :
        il la changera en ombre de mort,
        la transformera en sombre nuée.
${}^{17}Si vous n’écoutez pas,
        mon âme pleurera en secret sur votre orgueil,
        \\mes yeux tout en pleurs ruisselleront de larmes,
        car le troupeau du Seigneur est emmené captif.
${}^{18}Dis au roi et à la reine mère :
        \\Abaissez-vous jusqu’à terre,
        \\car elle est tombée de votre tête,
        la couronne qui faisait votre splendeur !
${}^{19}Les villes du Néguev sont fermées,
        et personne qui les ouvre.
        \\Tout Juda a été déporté,
        entièrement déporté.
         
${}^{20}Lève les yeux et vois
        ceux qui viennent du nord !
        \\Où est-il le troupeau qui te fut confié,
        les brebis de ta splendeur ?
${}^{21}Que diras-tu quand on t’imposera
        ceux que tu avais instruits,
        des disciples qui deviendront tes chefs ?
        \\Ne seras-tu pas saisie de douleurs,
        comme une femme qui accouche ?
${}^{22}Alors, si tu dis en ton cœur :
        « Pourquoi tout cela m’arrive-t-il ? »
        \\– C’est à cause de l’immensité de ta faute ;
        les pans de ta robe ont été retroussés,
        et tu as subi violence.
${}^{23}Un Éthiopien peut-il changer de peau,
        une panthère, changer de pelage ?
        \\Et vous pourriez faire le bien,
        vous, les habitués du mal ?
${}^{24}Je vous disperserai
        comme la paille qui s’en va au souffle du désert.
${}^{25}Tel sera ton sort, la part que je mesurerai pour toi,
        – oracle du Seigneur –,
        \\toi qui m’as oublié,
        pour faire confiance au mensonge.
${}^{26}Moi aussi, je relèverai
        les pans de ta robe sur ton visage,
        et ta honte sera vue.
${}^{27}Tes adultères, tes hennissements de plaisir,
        tes ignobles prostitutions,
        \\sur les collines, dans la campagne,
        tes horreurs, je les ai vues.
        \\Malheur à toi, Jérusalem, tu es impure !
        Et pour combien de temps encore ?
      <p class="cantique" id="bib_ct-at_34"><span class="cantique_label">Cantique AT 34</span> = <span class="cantique_ref"><a class="unitex_link" href="#bib_jr_14_17">Jr 14, 17-21</a></span>
      
         
      \bchapter{}
      \begin{verse}
${}^{1}Parole du Seigneur qui fut adressée à Jérémie à propos de la grande sécheresse :
${}^{2}Juda est en deuil ;
        \\aux portes de ses villes,
        on dépérit, assombri et atterré,
        \\tandis que monte la clameur de Jérusalem.
${}^{3}Les puissants envoient les petites gens chercher de l’eau :
        arrivés aux réservoirs, ils n’y trouvent pas d’eau !
        \\Alors, ils reviennent avec leurs récipients vides ;
        honteux et confus, ils se voilent la tête.
${}^{4}À cause du sol crevassé,
        faute de pluie sur la terre,
        les laboureurs, honteux, se voilent la tête.
${}^{5}Oui, dans la campagne,
        même la biche qui met bas
        abandonne son petit, faute de verdure.
${}^{6}Les ânes sauvages se tiennent sur les hauteurs,
        \\ils aspirent le vent comme des chacals,
        et leurs yeux s’épuisent à chercher une herbe absente.
         
${}^{7}Si nos fautes parlent contre nous,
        agis, Seigneur, à cause de ton nom.
        \\Oui, nos infidélités se sont multipliées :
        contre toi, nous avons péché.
${}^{8}Espoir d’Israël,
        toi qui le sauves au temps de l’angoisse,
        \\pourquoi serais-tu comme un immigré dans le pays,
        comme un voyageur qui fait un détour pour la nuit ?
${}^{9}Pourquoi serais-tu comme un homme abasourdi,
        comme un guerrier qui ne peut sauver ?
        \\Toi, Seigneur, tu es au milieu de nous,
        ton nom est invoqué sur nous ;
        \\ne nous délaisse pas !
       
${}^{10}Ainsi parle le Seigneur au sujet de ce peuple : « Assurément, ils aiment s’agiter, sans ménager leurs pas. » Mais le Seigneur ne se complaît pas en eux ; à présent il se souvient de leur faute et va châtier leurs péchés.
${}^{11}Le Seigneur me dit : « N’intercède pas pour le bien de ce peuple. 
${}^{12}S’ils jeûnent, je n’écouterai pas leur supplication ; s’ils font monter vers moi holocaustes et offrandes, je ne me complairai pas en eux ; oui, je vais les exterminer par l’épée, la famine et la peste. » 
${}^{13}Et je dis : « Ah ! Seigneur mon Dieu, voici pourtant que les prophètes leur disent : “Non, vous ne verrez pas l’épée, et pour vous il n’y aura pas de famine ! Mais je vous donnerai en ce lieu une paix véritable.” » 
${}^{14}Alors le Seigneur me dit : « C’est le mensonge que ces prophètes prophétisent en mon nom ! Je ne les ai pas envoyés, je ne leur ai pas donné d’ordre, je ne leur ai pas parlé. Vision mensongère, divination, ineptie et tromperie de leur cœur, voilà ce qu’ils vous prophétisent ! » 
${}^{15}C’est pourquoi, ainsi parle le Seigneur : « Les prophètes qui prophétisent en mon nom, alors que moi, je ne les ai pas envoyés, eux qui disent : “Il n’y aura ni épée ni famine dans ce pays”, ces prophètes-là périront par l’épée et la famine ! 
${}^{16}Quant aux gens du peuple à qui ils prophétisent, ils seront jetés dans les rues de Jérusalem, victimes de la famine et de l’épée, et ils n’auront personne pour les enterrer, eux et leurs femmes, leurs fils et leurs filles. Je déverserai sur eux leur méchanceté. »
       
${}^{17}Tu leur diras cette parole :
       
        \\Que mes yeux ruissellent de larmes
        nuit et jour, sans s’arrêter !
         
        \\Elle est blessée d’une grande blessure,
        la vierge, la fille de mon peuple,
        \\meurtrie d’une plaie profonde.
         
        ${}^{18}Si je sors dans la campagne,
        voici les victimes de l’épée ;
        \\si j’entre dans la ville,
        voici les souffrants de la faim.
         
        \\Même le prophète, même le prêtre
        \\parcourent le pays sans comprendre.
         
        ${}^{19}As-tu donc rejeté Juda ?
        \\Es-tu pris de dégoût pour Sion ?
        \\Pourquoi nous frapper sans remède ?
         
        \\Nous attendions la paix, et rien de bon !
        \\le temps du remède, et voici l’épouvante !
         
        ${}^{20}Seigneur, nous connaissons notre révolte,
        la faute de nos pères :
        \\oui, nous avons péché contre toi !
         
        ${}^{21} À cause de ton nom, ne méprise pas,
        \\n’humilie pas le trône de ta gloire !
         
        \\Rappelle-toi :
        \\ne romps pas ton alliance avec nous !
         
        ${}^{22}Parmi les idoles des nations,
        en est-il qui fassent pleuvoir ?
        \\Est-ce le ciel qui nous donnera les pluies ?
        N’est-ce pas toi, Seigneur notre Dieu ?
        \\Nous espérons en toi,
        car c’est toi qui as fait tout cela.
       
      
         
      \bchapter{}
      \begin{verse}
${}^{1}Le Seigneur me dit : Même si Moïse et Samuel se tenaient devant moi, je n’aurais pas d’égard pour ce peuple. Renvoie-les loin de moi : qu’ils s’en aillent ! 
${}^{2}Et quand ils te diront : « Où irons-nous ? », tu leur répondras : Ainsi parle le Seigneur :
      Qui est pour la mort, qu’il aille à la mort !
        Qui est pour l’épée, qu’il aille à l’épée !
        \\Qui est pour la famine, à la famine !
        Qui est pour la captivité, à la captivité !
${}^{3}Je leur imposerai quatre fléaux – oracle du Seigneur : l’épée pour tuer, les chiens pour déchirer, les oiseaux du ciel et les bêtes de la terre pour dévorer et pour détruire. 
${}^{4}Et je ferai d’eux un objet de stupeur pour tous les royaumes de la terre, à cause de Manassé, fils d’Ézékias, roi de Juda, et de ce qu’il a fait à Jérusalem.
       
${}^{5}Qui donc aura compassion de toi, Jérusalem ?
        Qui aura pour toi un geste de pitié ?
        \\Qui fera un détour pour demander de tes nouvelles ?
${}^{6}Toi, tu m’as délaissé – oracle du Seigneur.
        Tu t’en vas en me tournant le dos.
        \\Alors, j’ai étendu la main contre toi pour te détruire,
        je suis las de t’épargner !
${}^{7}Je les ai vannés avec un van aux portes du pays,
        j’ai fait périr mon peuple, je l’ai privé d’enfants ;
        \\mais ils ne sont pas revenus de leurs chemins.
${}^{8}J’ai fait que leurs veuves soient nombreuses,
        plus que le sable au bord des mers.
        \\J’ai fait venir le dévastateur,
        en plein midi, sur la mère du jeune guerrier.
        \\J’ai fait tomber sur elle, à l’improviste,
        la panique et la peur.
${}^{9}Elle dépérit, celle qui enfanta sept fois,
        elle est à bout de souffle ;
        \\son soleil s’est couché avant la fin du jour,
        elle pâlit de honte.
        \\Et ceux qui restent, je vais les livrer à l’épée
        devant leurs ennemis – oracle du Seigneur.
        ${}^{10}C’est pour mon malheur, ô ma mère,
        que tu m’as enfanté,
        homme de querelle et de dispute pour tout le pays.
        \\Je ne suis le créancier ni le débiteur de personne,
        et pourtant tout le monde me maudit !
         
${}^{11}Le Seigneur dit :
        \\Ne t’ai-je pas libéré pour ton bien ?
        \\N’ai-je pas fait que l’ennemi te supplie
        au temps du malheur, au temps de l’angoisse ?
${}^{12}Le fer brisera-t-il le fer qui vient du nord,
        et le bronze… ?
${}^{13}Ta richesse, tes trésors, je les livrerai en butin,
        sans contrepartie, à cause de tous les péchés
        que tu commets sur l’ensemble de ton territoire.
${}^{14}Je t’asservirai à tes ennemis
        dans un pays que tu ne connais pas,
        \\car le feu de ma colère s’est allumé,
        il brûlera contre vous.
         
${}^{15}Seigneur, toi qui sais,
        souviens-toi de moi et visite-moi !
        \\Venge-moi de mes persécuteurs,
        ne me rends pas victime de la lenteur de ta colère !
        \\Tu le sais : à cause de toi je supporte l’insulte.
        ${}^{16}Quand je rencontrais tes paroles, je les dévorais ;
        elles faisaient ma joie, les délices de mon cœur,
        \\parce que ton nom était invoqué sur moi,
        Seigneur, Dieu de l’univers.
        ${}^{17}Jamais je ne me suis assis dans le cercle des moqueurs
        pour m’y divertir ;
        \\sous le poids de ta main, je me suis assis à l’écart,
        parce que tu m’as rempli d’indignation.
        ${}^{18}Pourquoi ma souffrance est-elle sans fin,
        ma blessure, incurable, refusant la guérison ?
        Serais-tu pour moi un mirage,
        comme une eau incertaine ?
         
        ${}^{19}Voilà pourquoi, ainsi parle le Seigneur :
        \\Si tu reviens, si je te fais revenir,
        tu reprendras ton service devant moi\\.
        \\Si tu sépares ce qui est précieux de ce qui est méprisable,
        tu seras comme ma propre bouche.
        \\C’est eux qui reviendront vers toi,
        et non pas toi qui reviendras vers eux.
        ${}^{20}Je fais de toi pour ce peuple
        un rempart de bronze infranchissable ;
        \\ils te combattront,
        mais ils ne pourront rien contre toi,
        \\car je suis avec toi pour te sauver et te délivrer
        – oracle du Seigneur.
        ${}^{21}Je te délivrerai de la main des méchants,
        je t’affranchirai de la poigne des puissants.
      
         
      \bchapter{}
      \begin{verse}
${}^{1}La parole du Seigneur me fut adressée : 
${}^{2}Tu ne prendras pas de femme et tu n’auras ni fils ni fille en ce lieu. 
${}^{3}Car ainsi parle le Seigneur au sujet des fils et des filles qui vont naître en ce lieu, des mères qui leur donneront naissance et des pères qui les engendreront dans ce pays : 
${}^{4}ils mourront de maladies mortelles ; ils ne seront ni pleurés ni enterrés ; ils deviendront du fumier à la surface du sol ; par l’épée, par la famine, ils seront exterminés et leurs cadavres serviront de pâture aux oiseaux du ciel et aux bêtes de la terre.
${}^{5}Oui, ainsi parle le Seigneur : N’entre pas à la maison du deuil ; ne va pas les pleurer, n’aie pour eux aucun geste de pitié, car j’ai retiré de ce peuple ma paix, et la fidélité et la tendresse – oracle du Seigneur. 
${}^{6}Grands et petits mourront dans ce pays sans être enterrés. On ne les pleurera pas, pour eux on ne se fera pas d’incision, on ne se tondra pas. 
${}^{7}Avec celui qui est dans le deuil, on ne rompra pas le pain afin de le consoler du mort ; on ne lui fera pas boire la coupe de consolation pour son père ou sa mère.
${}^{8}Tu n’entreras pas non plus à la maison du festin pour t’asseoir avec les convives, pour manger et boire. 
${}^{9}Car ainsi parle le Seigneur de l’univers, le Dieu d’Israël : Voici qu’en ce lieu, de vos jours et sous vos yeux, je ferai cesser chants d’allégresse et chants de joie, le chant de l’époux et le chant de l’épousée.
${}^{10}Lorsque tu annonceras au peuple toutes ces choses, ils te demanderont : « Pourquoi donc le Seigneur a-t-il proféré contre nous ce grand malheur ? Quelle est notre faute et quels péchés avons-nous commis contre le Seigneur notre Dieu ? » 
${}^{11}Alors tu leur répondras : « C’est que vos pères m’ont abandonné – oracle du Seigneur – pour suivre d’autres dieux, les servir et se prosterner devant eux. Ils m’ont abandonné, ils n’ont pas gardé ma Loi. 
${}^{12}Et vous, vous avez agi plus mal encore que vos pères. Voici que chacun de vous, pour ne pas m’écouter, suit les penchants mauvais de son cœur endurci. 
${}^{13}Je vous jetterai hors de ce pays, dans un pays inconnu de vous et de vos pères, et vous y servirez jour et nuit d’autres dieux, car je ne vous ferai plus grâce. »
${}^{14}C’est pourquoi, voici venir des jours – oracle du Seigneur – où, pour prêter serment, on ne dira plus : « Par le Seigneur vivant, qui a fait monter du pays d’Égypte les fils d’Israël », 
${}^{15}mais : « Par le Seigneur vivant, qui a fait monter les fils d’Israël du pays du nord et de tous les pays où il les avait chassés. » Car je les ferai revenir sur leur sol, celui que j’ai donné à leurs pères.
${}^{16}Voici que j’envoie en grand nombre – oracle du Seigneur – des pêcheurs qui les pêcheront. Après cela, j’enverrai en grand nombre des chasseurs qui les chasseront de toute montagne et de toute colline, jusque dans les fentes des rochers. 
${}^{17}Oui, j’ai les yeux sur tous leurs chemins, pour moi ils ne sont pas cachés ; et leurs fautes ne peuvent se dérober à mes yeux. 
${}^{18}Avant tout, je leur revaudrai le double de leurs fautes et de leurs péchés, car ils ont profané mon pays : comme des cadavres, leurs horreurs et leurs abominations ont rempli mon héritage.
${}^{19}Seigneur, ma force et mon abri,
        mon refuge au jour d’angoisse,
        \\vers toi viendront les nations
        depuis les lointains de la terre ;
        \\elles diront : « Nos pères ont eu pour seul héritage
        mensonge et vanité qui ne servent à rien. »
${}^{20}Un être humain se fabrique-t-il des dieux ?
        – Et ce ne sont même pas des dieux !
${}^{21}C’est pourquoi, je vais donner aux nations la connaissance ;
        \\cette fois-ci, je leur ferai connaître
        ma main et ma puissance :
        \\elles reconnaîtront que mon nom est « Le Seigneur ».
      <p class="cantique" id="bib_ct-at_35"><span class="cantique_label">Cantique AT 35</span> = <span class="cantique_ref"><a class="unitex_link" href="#bib_jr_17_7">Jr 17, 7-8</a></span>
      
         
      \bchapter{}
${}^{1}Le péché de Juda est inscrit avec un burin de fer,
        avec une pointe de diamant ;
        \\il est gravé sur la tablette de leur cœur
        et aux cornes de leurs autels.
${}^{2}Ainsi leurs fils en font-ils mémoire
        sur leurs autels et leurs poteaux sacrés,
        près des arbres verts, sur les collines élevées.
${}^{3}Ô ma montagne au milieu des champs,
        je livrerai en butin ta richesse, tous tes trésors,
        \\à cause du péché de tes lieux sacrés,
        sur l’ensemble de ton territoire.
${}^{4}Tu abandonneras toi-même l’héritage que je t’ai donné,
        \\et je t’asservirai à tes ennemis,
        dans un pays que tu ne connais pas,
        \\car vous avez allumé le feu de ma colère
        qui brûlera pour toujours.
        
           
        ${}^{5}Ainsi parle le Seigneur :
        \\Maudit soit l’homme
        qui met sa foi dans un mortel,
        \\qui s’appuie sur un être de chair,
        tandis que son cœur se détourne du Seigneur.
         
        ${}^{6}Il sera comme un buisson sur une terre désolée,
        il ne verra pas venir le bonheur.
        \\Il aura pour demeure les lieux arides du désert,
        une terre salée, inhabitable.
         
        ${}^{7}Béni soit l’homme
        qui met sa foi dans le Seigneur,
        dont le Seigneur est la confiance.
         
        ${}^{8}Il sera comme un arbre, planté près des eaux,
        qui pousse, vers le courant, ses racines.
         
        \\Il ne craint pas quand vient la chaleur :
        son feuillage reste vert.
         
        \\L’année de la sécheresse, il est sans inquiétude :
        il ne manque pas de porter du fruit.
         
        ${}^{9}Rien n’est plus faux que le cœur de l’homme\\,
        il est incurable.
        Qui peut le connaître ?
        ${}^{10}Moi, le Seigneur, qui pénètre les cœurs
        et qui scrute les reins,
        \\afin de rendre à chacun selon sa conduite,
        selon le fruit de ses actes\\.
         
${}^{11}La perdrix couve des œufs qu’elle n’a pas pondus,
        tel est celui qui s’enrichit injustement ;
        \\au milieu de ses jours, la richesse l’abandonne,
        en fin de compte, il n’est qu’un sot.
${}^{12}Trône de la gloire, élevé dès l’origine,
        tel est notre lieu saint !
${}^{13}Seigneur, espoir d’Israël,
        tous ceux qui t’abandonnent seront couverts de honte ;
        \\ils seront inscrits dans la terre,
        ceux qui se détournent de toi,
        \\car ils ont abandonné le Seigneur,
        la source d’eau vive.
${}^{14}Guéris-moi, Seigneur, et je serai guéri,
        sauve-moi, et je serai sauvé,
        \\car tu es ma louange.
         
${}^{15}Voici qu’ils me disent :
        « Où donc est la parole du Seigneur ? Qu’elle vienne ! »
${}^{16}Moi, pourtant, je ne me suis pas hâté derrière toi
        pour annoncer le malheur ;
        \\je n’ai pas désiré le jour fatal, tu le sais bien :
        ce qui sort de mes lèvres est à découvert devant toi.
${}^{17}Ne deviens pas pour moi une cause d’effroi,
        toi, mon refuge au jour du malheur.
${}^{18}Qu’ils aient honte, mes persécuteurs,
        et que moi, je n’aie pas honte !
        \\Qu’ils soient effrayés,
        et non pas moi !
        \\Fais venir sur eux le jour du malheur,
        et brise-les d’une double brisure !
${}^{19}Ainsi m’a parlé le Seigneur : Va, et tiens-toi à la porte des Fils du Peuple, par où entrent et sortent les rois de Juda, et à toutes les portes de Jérusalem. 
${}^{20}Tu leur diras : Écoutez la parole du Seigneur, rois de Juda, tout Juda et tous les habitants de Jérusalem, vous qui entrez par ces portes ! 
${}^{21}Ainsi parle le Seigneur : Prenez garde à vous-mêmes et ne transportez aucun fardeau le jour du sabbat, n’en faites pas entrer par les portes de Jérusalem. 
${}^{22}Le jour du sabbat, ne faites sortir aucun fardeau de vos maisons et ne faites aucun travail. Sanctifiez le jour du sabbat, comme je l’ai ordonné à vos pères. 
${}^{23}Mais ils n’ont pas écouté ni prêté l’oreille, ils ont raidi leur nuque pour ne pas accepter ni recevoir de leçon. 
${}^{24}Si vous m’écoutez bien – oracle du Seigneur – en ne faisant entrer aucun fardeau par les portes de cette ville le jour du sabbat, en sanctifiant le jour du sabbat sans faire aucun travail, alors voici : 
${}^{25}par les portes de cette ville entreront des rois et des princes siégeant sur le trône de David ; ils entreront sur un char attelé de plusieurs chevaux, eux et leurs princes, puis les gens de Juda et les habitants de Jérusalem ; et cette ville sera habitée pour toujours. 
${}^{26}Ils viendront des villes de Juda et des alentours de Jérusalem, du pays de Benjamin, du Bas-Pays, de la Montagne et du Néguev, pour présenter holocaustes et sacrifices, offrandes et encens, pour présenter l’action de grâce dans la maison du Seigneur. 
${}^{27}Mais si vous ne m’écoutez pas, si vous refusez de sanctifier le jour du sabbat, en portant des fardeaux et en franchissant les portes de Jérusalem le jour du sabbat, alors je mettrai le feu à ses portes ; il dévorera les fortifications de Jérusalem et ne s’éteindra pas.
      
         
      \bchapter{}
      \begin{verse}
${}^{1}Parole du Seigneur adressée à Jérémie : 
${}^{2} « Lève-toi, descends à la maison du potier ; là, je te ferai entendre mes paroles. » 
${}^{3} Je descendis donc à la maison du potier. Il était en train de travailler sur son tour. 
${}^{4} Le vase qu’il façonnait de sa main avec l’argile fut manqué. Alors il recommença, et il fit un autre vase, selon ce qu’il est bon de faire, aux yeux d’un potier.
${}^{5}Alors la parole du Seigneur me fut adressée : 
${}^{6}« Maison d’Israël, est-ce que je ne pourrais pas vous traiter comme fait ce potier ? – oracle du Seigneur. Oui, comme l’argile est dans la main du potier, ainsi êtes-vous dans ma main, maison d’Israël ! 
${}^{7}Parfois, je parle d’arracher, de renverser et de détruire une nation ou un royaume. 
${}^{8}Mais cette nation contre laquelle j’ai parlé se détourne du mal, alors je renonce au mal que j’avais projeté de lui faire. 
${}^{9}Parfois, je parle de bâtir et de planter une nation ou un royaume. 
${}^{10}Mais ils font ce qui est mal à mes yeux et ils n’écoutent pas ma voix ; alors je renonce au bien que j’avais décidé de leur faire. 
${}^{11}Maintenant, parle donc aux gens de Juda et aux habitants de Jérusalem : Ainsi parle le Seigneur : Voici que moi, comme un potier, je façonne contre vous un malheur, je médite contre vous un projet. Revenez chacun de votre mauvais chemin ; rendez meilleurs vos chemins et vos actes ! 
${}^{12}Mais ils disent : Rien à faire ! Nous suivrons nos propres projets ; nous agirons chacun selon les penchants mauvais de son cœur endurci. »
${}^{13}C’est pourquoi, ainsi parle le Seigneur :
        \\Interrogez donc les nations :
        Qui a entendu rien de pareil ?
        \\Elle a commis tant de choses monstrueuses,
        la vierge d’Israël !
${}^{14}Disparaît-elle du rocher, vers la campagne,
        la neige du Liban ?
        \\Peuvent-elles se tarir, les eaux toujours fraîches
        qui jaillissent des montagnes ?
${}^{15}Or mon peuple m’a oublié,
        aux vaines idoles ils brûlent de l’encens ;
        \\on les fait trébucher sur leurs chemins,
        sur leurs voies de toujours,
        \\pour les faire marcher sur des sentiers,
        sur des chemins non frayés,
${}^{16}et réduire leur pays en lieu désolé,
        en dérision pour toujours.
        \\Quiconque passera par là
        se désolera et hochera la tête.
${}^{17}Comme le vent d’est, je les disperserai face à l’ennemi ;
        je les verrai de dos et non de face,
        au jour de leur débâcle.
${}^{18}Mes ennemis\\ont dit : « Allons, montons un complot contre Jérémie. La loi ne va pas disparaître par manque de prêtre, ni le conseil, par manque de sage, ni la parole, par manque de prophète. Allons, attaquons-le par notre langue, ne faisons pas attention à toutes ses paroles. »
        ${}^{19}Mais toi, Seigneur, fais attention à moi,
        écoute ce que disent mes adversaires.
        ${}^{20}Comment peut-on rendre le mal pour le bien ?
        Ils ont creusé une fosse pour me perdre\\.
        \\Souviens-toi que je me suis tenu en ta présence
        pour te parler en leur faveur,
        pour détourner d’eux ta colère.
${}^{21}C’est pourquoi, livre leurs fils à la famine ;
        passe-les au fil de l’épée !
        \\Que leurs femmes soient privées d’enfants et de maris ;
        que leurs hommes soient emportés par la mort,
        et leurs jeunes gens, frappés de l’épée à la guerre !
${}^{22}On entendra crier dans leurs maisons,
        quand soudain tu feras venir contre eux des bandits,
        \\car ils ont creusé une fosse pour me prendre,
        et dissimulé des pièges sous mes pas.
${}^{23}Mais toi, Seigneur, tu connais
        tout leur dessein de mort contre moi.
        \\Ne recouvre pas leur faute,
        et n’efface pas leur péché devant toi !
        \\Fais-les trébucher en ta présence ;
        agis contre eux, au temps de ta colère !
      
         
      \bchapter{}
      \begin{verse}
${}^{1}Ainsi parle le Seigneur : Va, et achète une cruche en terre cuite. Prends quelques anciens parmi le peuple et les prêtres, 
${}^{2}et sors vers le Val-de-la-Géhenne, à l’entrée de la porte des Tessons. Là, tu proclameras les paroles que je te dirai. 
${}^{3}Tu parleras ainsi : Écoutez la parole du Seigneur, rois de Juda et habitants de Jérusalem. Ainsi parle le Seigneur de l’univers, le Dieu d’Israël : Voici que je fais venir en ce lieu un malheur ; à qui l’apprendra, les oreilles vont tinter. 
${}^{4}Car ils m’ont abandonné ; ils ont rendu ce lieu méconnaissable ; ils y ont brûlé de l’encens pour d’autres dieux que ni eux, ni leurs pères, ni les rois de Juda n’avaient connus ; ils l’ont rempli du sang des innocents. 
${}^{5}Ils ont édifié les lieux sacrés du dieu Baal, pour consumer par le feu leurs fils en holocauste à Baal : cela, je ne l’avais pas ordonné, je ne l’avais pas dit, ce n’était pas venu à mon esprit ! 
${}^{6}C’est pourquoi, voici venir des jours – oracle du Seigneur – où l’on n’appellera plus ce lieu « Le Tofeth » ni « Val-de-la-Géhenne », mais « Val-du-Massacre ». 
${}^{7}En ce lieu, je briserai les conseillers de Juda et de Jérusalem. Devant leurs ennemis, je les abattrai par l’épée, par la main de ceux qui en veulent à leur vie, et je donnerai leurs cadavres en pâture aux oiseaux du ciel et aux bêtes de la terre. 
${}^{8}Je réduirai cette ville en lieu désolé, en objet de dérision ; quiconque passera près d’elle se désolera et fera entendre un sifflement de stupeur à la vue de toutes ses plaies. 
${}^{9}Je leur ferai manger la chair de leurs fils et la chair de leurs filles ; chacun mangera la chair de son prochain pendant le siège, dans la détresse où les auront plongés leurs ennemis et ceux qui en veulent à leur vie.
${}^{10}Tu briseras la cruche sous les yeux des hommes qui t’auront accompagné, 
${}^{11}et tu leur diras : Ainsi parle le Seigneur de l’univers : Je briserai ce peuple et cette ville, comme on brise une poterie qui ne peut plus être réparée. Faute de place pour enterrer, on enterrera même au Tofeth. 
${}^{12}Ainsi ferai-je de ce lieu – oracle du Seigneur – et de ses habitants : Je rendrai cette ville pareille au Tofeth. 
${}^{13}Les maisons de Jérusalem et celles des rois de Juda seront pareilles à ce lieu du Tofeth, impures. Ainsi ferai-je de toutes ces maisons, où l’on brûle de l’encens sur les terrasses à toute l’armée du ciel, où l’on verse des libations à d’autres dieux.
${}^{14}Jérémie revint du Tofeth où le Seigneur l’avait envoyé prophétiser, et il se tenait dans la cour de la maison du Seigneur. Alors il dit à tout le peuple : 
${}^{15}Ainsi parle le Seigneur de l’univers, le Dieu d’Israël : Voici que je fais venir sur cette ville et toutes celles qui en dépendent, tout le malheur que j’ai proféré contre elles, car ils ont raidi leur nuque pour ne pas écouter mes paroles.
      
         
      \bchapter{}
      \begin{verse}
${}^{1}Le prêtre Pashehour, fils d’Immer, responsable de l’ordre dans la maison du Seigneur, entendit ce que prophétisait Jérémie. 
${}^{2}Alors Pashehour frappa le prophète Jérémie et le fit attacher au pilori qui est à la porte Haute de Benjamin, celle de la maison du Seigneur. 
${}^{3}Le lendemain, comme Pashehour le faisait détacher du pilori, Jérémie lui dit : Le Seigneur ne t’appelle plus « Pashehour », mais « Épouvante-de-tous-côtés » 
${}^{4}car, ainsi parle le Seigneur : Voici que je vais faire de toi un épouvantail, pour toi-même et tous tes amis. Ils tomberont sous l’épée de leurs ennemis : tu le verras de tes yeux. Je vais livrer tous les gens de Juda aux mains du roi de Babylone. Il les déportera à Babylone ; il les frappera de l’épée. 
${}^{5}Je livrerai toutes les réserves de cette ville, tout le fruit de son labeur et tout ce qu’elle a de précieux. Je livrerai tous les trésors des rois de Juda aux mains de leurs ennemis qui les pilleront, les prendront et les emporteront à Babylone. 
${}^{6}Toi, Pashehour, et tous les habitants de ta maison, vous partirez en captivité. Tu iras à Babylone ; là, tu mourras ; là, tu seras enterré, toi et tous tes amis auxquels tu as prophétisé le mensonge.
      
         
        ${}^{7}Seigneur, tu m’as séduit, et j’ai été séduit ;
        tu m’as saisi, et tu as réussi.
        \\À longueur de journée je suis exposé à la raillerie,
        tout le monde se moque de moi.
        ${}^{8}Chaque fois que j’ai à dire la parole,
        je dois crier, je dois proclamer :
        « Violence et dévastation\\ ! »
         
        \\À longueur de journée, la parole du Seigneur
        attire sur moi l’insulte et la moquerie.
        ${}^{9}Je me disais : « Je ne penserai plus à lui,
        je ne parlerai plus en son nom. »
        \\Mais elle était comme un feu brûlant dans mon cœur,
        elle était enfermée dans mes os.
        \\Je m’épuisais à la maîtriser,
        sans y réussir.
       
        ${}^{10}J’entends les calomnies de la foule :
        \\« Dénoncez-le ! Allons le dénoncer,
        celui-là, l’Épouvante-de-tous-côtés\\. »
        \\Tous mes amis guettent mes faux pas, ils disent\\ :
        \\« Peut-être se laissera-t-il séduire…
        Nous réussirons,
        et nous prendrons sur lui notre revanche ! »
        ${}^{11}Mais le Seigneur est avec moi, tel un guerrier redoutable :
        mes persécuteurs trébucheront, ils ne réussiront pas.
        \\Leur défaite les couvrira de honte,
        d’une confusion éternelle, inoubliable.
        ${}^{12}Seigneur de l’univers, toi qui scrutes l’homme juste,
        toi qui vois les reins et les cœurs,
        \\fais-moi voir la revanche que tu leur infligeras\\,
        car c’est à toi que j’ai remis ma cause.
         
        ${}^{13}Chantez le Seigneur, louez le Seigneur :
        il a délivré le malheureux de la main des méchants.
         
${}^{14}Maudit soit le jour où je suis né !
        Le jour où ma mère m’a enfanté, qu’il ne soit pas béni !
${}^{15}Maudit soit l’homme qui annonça à mon père
        cette nouvelle qui le combla de joie :
        « Il t’est né un fils, un garçon ! »
${}^{16}Cet homme deviendra pareil aux villes
        que le Seigneur a renversées sans pitié.
        \\Il entendra la clameur au matin,
        et le cri de guerre en plein midi.
${}^{17}Maudit soit le jour qui ne m’a pas fait mourir dès le ventre :
        \\ma mère serait devenue mon tombeau,
        et son ventre me porterait toujours.
${}^{18}Pourquoi donc suis-je sorti du ventre ?
        \\Pour voir peine et tourments,
        et mes jours s’achever dans la honte ?
      
         
      \bchapter{}
      \begin{verse}
${}^{1}Parole du Seigneur adressée à Jérémie, lorsque le roi Sédécias envoya Pashehour, fils de Malkiya, et le prêtre Sophonie, fils de Maaséya, pour lui dire : 
${}^{2}« Consulte donc pour nous le Seigneur, car Nabucodonosor, roi de Babylone, nous fait la guerre. Peut-être le Seigneur va-t-il faire en notre faveur un de ses miracles, pour que s’en aille celui qui nous assiège ? » 
${}^{3}Jérémie leur dit : « Vous parlerez ainsi à Sédécias : 
${}^{4}Ainsi parle le Seigneur, le Dieu d’Israël : Les armes qui sont entre vos mains, avec lesquelles vous faites la guerre au roi de Babylone et aux Chaldéens, vos assiégeants, je vais les retourner de l’extérieur des remparts et les diriger vers l’intérieur de cette ville. 
${}^{5}Et moi-même, je vous ferai la guerre, par ma main puissante et la force de mon bras, avec colère, avec fureur, avec une grande irritation. 
${}^{6}Je frapperai les habitants de cette ville, l’homme et le bétail ; ils mourront d’une grande peste. 
${}^{7}Après cela – oracle du Seigneur – je livrerai Sédécias, roi de Juda, ses serviteurs, le peuple, ceux qui dans cette ville auront échappé à la peste, à l’épée et à la famine, je les livrerai aux mains de Nabucodonosor, roi de Babylone, aux mains de leurs ennemis, aux mains de ceux qui en veulent à leur vie. On les passera au tranchant de l’épée sans merci ni pitié, sans compassion.
${}^{8}Et tu diras à ce peuple : Ainsi parle le Seigneur : Voici que je mets devant vous le chemin de la vie et le chemin de la mort. 
${}^{9}Qui restera dans cette ville mourra par l’épée, la famine ou la peste. Mais qui en sortira pour se rendre aux Chaldéens, vos assiégeants, celui-là vivra : il aura la vie sauve, comme part de butin. 
${}^{10}Oui, je tournerai mon visage contre cette ville, pour son malheur et non pour son bonheur – oracle du Seigneur. Elle sera livrée aux mains du roi de Babylone qui l’incendiera. »
${}^{11}Sur la maison royale de Juda.
         
        \\Écoutez la parole du Seigneur,
${}^{12}Maison de David !
        \\Ainsi parle le Seigneur :
        \\Dès le matin, jugez selon le droit,
        délivrez l’exploité des mains de l’oppresseur,
        \\de peur que ma colère n’éclate comme un feu
        à cause de la malice de vos actes,
        et ne brûle, sans personne pour l’éteindre.
${}^{13}Me voici contre toi qui sièges dans la vallée,
        toi, Rocher-dans-la-plaine – oracle du Seigneur.
        \\Ils disent : « Qui nous attaquera ?
        Qui entrera dans nos demeures ? »
${}^{14}Je sévirai contre vous selon le fruit de vos actes
        – oracle du Seigneur.
        \\À la maison de la Forêt je mettrai le feu :
        il dévorera tout ce qui l’entoure.
       
      
         
      \bchapter{}
      \begin{verse}
${}^{1}Ainsi parle le Seigneur : Descends à la maison du roi de Juda, et là, tu prononceras cette parole. 
${}^{2}Tu diras : Écoute la parole du Seigneur, roi de Juda qui sièges sur le trône de David, toi, tes serviteurs et ton peuple, vous tous qui entrez par ces portes. 
${}^{3}Ainsi parle le Seigneur : Pratiquez le droit et la justice, délivrez l’exploité des mains de l’oppresseur, ne maltraitez pas l’immigré, l’orphelin et la veuve, ne leur faites pas violence ; et ne versez pas en ce lieu le sang de l’innocent. 
${}^{4}Si vous accomplissez cette parole, alors des rois siégeant sur le trône de David entreront par les portes de cette maison, montés sur un char attelé de plusieurs chevaux, chacun avec ses serviteurs et son peuple. 
${}^{5}Mais si vous n’écoutez pas ces paroles, j’en fais serment par moi-même – oracle du Seigneur –, cette maison ne sera plus qu’une ruine.
      
         
       
${}^{6}Oui, ainsi parle le Seigneur au sujet de la maison royale de Juda :
        \\Pour moi, tu es Galaad, tu es le sommet du Liban !
        Pourtant, je te changerai en désert, en villes inhabitées.
${}^{7}Je voue des hommes à ta destruction,
        chacun d’eux avec ses armes ;
        \\ils abattront tes cèdres de choix
        et les feront tomber dans le feu.
       
${}^{8}Alors, des gens de nombreuses nations passeront près de cette ville ; ils se diront l’un à l’autre : « Pourquoi le Seigneur a-t-il ainsi traité cette grande ville ? » 
${}^{9}Et l’on dira : « C’est qu’ils ont abandonné l’alliance du Seigneur leur Dieu, pour se prosterner devant d’autres dieux et les servir. »
${}^{10}Ne pleurez pas sur un mort,
        n’ayez pour lui aucun geste de deuil.
        \\Pleurez plutôt celui qui s’en va,
        car il ne reviendra plus,
        il ne verra plus son pays natal.
${}^{11}Oui, ainsi parle le Seigneur, au sujet de Shalloum, fils de Josias, roi de Juda, qui régna à la place de Josias son père, et qui est sorti de ce lieu : il ne reviendra plus ici, 
${}^{12}car il mourra dans le lieu où on l’a déporté ; et ce pays, il ne le verra plus.
${}^{13}Malheur à qui bâtit sa maison au mépris de la justice,
        et ses chambres hautes au mépris du droit,
        \\qui fait travailler gratuitement son prochain
        et ne lui verse pas de salaire,
${}^{14}lui qui se dit :
        \\« Je vais me bâtir une maison bien vaste
        et de spacieuses chambres hautes. »
        \\Il perce des ouvertures, il recouvre de cèdre,
        il enduit de vermillon.
${}^{15}Règnes-tu pour ta passion du cèdre ?
        Ton père n’a-t-il pas mangé et bu ?
        \\Il pratiquait aussi le droit et la justice ;
        alors, pour lui, tout allait bien.
${}^{16}Il défendait la cause du pauvre et du malheureux ;
        alors, tout allait bien.
        \\N’est-ce pas cela, me connaître ?
        – oracle du Seigneur.
${}^{17}Mais toi, tu n’as des yeux et un cœur que pour ton profit,
        pour verser le sang de l’innocent,
        et agir dans l’oppression et la brutalité.
${}^{18}C’est pourquoi, ainsi parle le Seigneur au sujet de Joakim,
        fils de Josias, roi de Juda :
        \\On ne fera pas sur lui cette lamentation :
        « Mon frère, quel malheur ! Ma sœur, quel malheur ! »
        \\On ne fera pas sur lui cette lamentation :
        « Quel malheur, monseigneur ! Quel malheur, majesté ! »
${}^{19}On l’enterrera comme on enterre un âne :
        traîné et jeté loin des portes de Jérusalem.
${}^{20}Monte au Liban et crie,
        dans le Bashane donne de la voix,
        \\crie du haut des Abarim,
        car tous tes amants sont brisés.
${}^{21}Je t’ai parlé quand tu étais tranquille ;
        tu as dit : « Je n’écouterai pas ! »
        \\Telle fut ta conduite depuis ta jeunesse :
        tu n’as pas écouté ma voix.
${}^{22}Tous tes pasteurs, le vent les mènera paître,
        et tes amants s’en iront captifs.
        \\Alors, tu seras honteuse et confuse
        à cause de toute ta malice.
${}^{23}Toi qui sièges au Liban,
        qui es nichée dans les cèdres,
        \\combien gémiras-tu,
        quand douleurs et frissons viendront sur toi,
        comme sur la femme qui accouche !
       
${}^{24}Par ma vie – oracle du Seigneur –, Konias, fils de Joakim, roi de Juda, si tu étais un anneau à ma main droite, je t’en arracherais. 
${}^{25}Je vais te livrer aux mains de ceux qui en veulent à ta vie, aux mains de ceux qui t’épouvantent, aux mains de Nabucodonosor, roi de Babylone, et aux mains des Chaldéens. 
${}^{26}Et je te jetterai, toi et ta mère qui t’a enfanté, sur une autre terre où vous n’êtes pas nés ; et là, vous mourrez. 
${}^{27}– Sur la terre où ils désirent ardemment revenir, ils ne reviendront pas.
       
${}^{28}Est-il un ustensile méprisable et cassé,
        cet homme, ce Konias,
        est-il un objet dont personne ne veut plus ?
        \\Pourquoi, lui et sa descendance, ont-ils été enlevés,
        et jetés sur une terre qu’ils ne connaissaient pas ?
${}^{29}Terre, terre, terre,
        écoute la parole du Seigneur !
${}^{30}Ainsi parle le Seigneur : Au sujet de cet homme, inscrivez :
        « Sans postérité ; individu qui n’a pas réussi dans sa vie »
        \\car aucun de ses descendants ne réussira
        à siéger sur le trône de David
        et à dominer encore en Juda.
      
         
      \bchapter{}
      \begin{verse}
${}^{1}Quel malheur pour vous, pasteurs ! Vous laissez périr et vous dispersez les brebis de mon pâturage – oracle du Seigneur ! 
${}^{2} C’est pourquoi, ainsi parle le Seigneur, le Dieu d’Israël, contre les pasteurs qui conduisent mon peuple : Vous avez dispersé mes brebis, vous les avez chassées, et vous ne vous êtes pas occupés d’elles. Eh bien ! Je vais m’occuper de vous, à cause de la malice de vos actes – oracle du Seigneur. 
${}^{3} Puis, je rassemblerai moi-même le reste de mes brebis de tous les pays où je les ai chassées. Je les ramènerai dans leur enclos, elles seront fécondes et se multiplieront. 
${}^{4} Je susciterai pour elles des pasteurs qui les conduiront ; elles ne seront plus apeurées ni effrayées, et aucune ne sera perdue – oracle du Seigneur.
        ${}^{5}Voici venir des jours – oracle du Seigneur\\–,
        où je susciterai pour David un Germe juste :
        \\il régnera en vrai roi, il agira avec intelligence,
        il exercera dans le pays le droit et la justice.
        ${}^{6}En ces jours-là\\, Juda sera sauvé,
        et Israël habitera en sécurité.
        \\Voici le nom qu’on lui donnera :
        « Le-Seigneur-est-notre-justice\\. »
${}^{7}C’est pourquoi, voici venir des jours – oracle du Seigneur\\ – où, pour prêter serment\\, on ne dira plus : « Par le Seigneur vivant, qui a fait monter du pays d’Égypte les fils d’Israël », 
${}^{8} mais : « Par le Seigneur vivant, qui a fait monter du pays du nord les gens\\de la maison d’Israël, qui les a ramenés de tous les pays où il les avait chassés\\. » Car ils demeureront sur leur sol.
${}^{9}Sur les prophètes.
         
        \\Mon cœur en moi s’est brisé,
        tous mes os frémissent.
        \\Je suis comme un ivrogne,
        comme un homme pris de vin,
        \\à cause du Seigneur,
        à cause de ses paroles de sainteté.
${}^{10}Car le pays est rempli d’adultères,
        à cause de la malédiction le pays est en deuil,
        \\les pâturages du désert sont desséchés.
        \\La course de ces gens est tendue vers le mal,
        et leur force, vers ce qui n’est pas juste.
${}^{11}Oui, même le prophète, même le prêtre sont corrompus,
        et jusqu’en ma maison, j’ai découvert leur malice,
        – oracle du Seigneur.
${}^{12}C’est pourquoi leur chemin sera pour eux
        comme un sentier glissant en pleine obscurité :
        \\ils y seront poussés, ils tomberont,
        \\car je ferai venir sur eux le malheur,
        l’année de leur châtiment,
        – oracle du Seigneur.
         
${}^{13}Chez les prophètes de Samarie,
        j’ai vu des extravagances :
        \\ils prophétisaient par le dieu Baal
        et ils égaraient mon peuple Israël.
${}^{14}Mais chez les prophètes de Jérusalem,
        j’ai vu des choses monstrueuses :
        \\ils sont adultères et familiers du mensonge,
        \\ils prêtent main-forte aux malfaiteurs,
        et nul ne revient de sa malice.
        \\Tous, ils sont devenus pour moi pareils à Sodome,
        et les habitants de Jérusalem, pareils à Gomorrhe.
${}^{15}C’est pourquoi, ainsi parle le Seigneur de l’univers contre les prophètes :
        \\Je vais les nourrir d’absinthe
        et les abreuver d’eau empoisonnée,
        \\car, à partir des prophètes de Jérusalem,
        la corruption s’est répandue dans tout le pays.
         
${}^{16}Ainsi parle le Seigneur de l’univers :
        \\N’écoutez pas les paroles de ces prophètes
        qui prophétisent pour vous
        et font de vous des êtres de rien.
        \\Ils disent les visions de leur cœur
        et non ce qui sort de la bouche du Seigneur.
${}^{17}Ils ne cessent de dire à ceux qui me méprisent :
        « Le Seigneur a parlé, vous aurez la paix ! »
        \\Et tous disent, en suivant les penchants de leur cœur endurci :
        « Le malheur ne viendra pas sur vous ! »
${}^{18}Mais qui donc s’est tenu au conseil du Seigneur ?
        \\Qui a vu et entendu sa parole ?
        Qui a fait attention à sa parole et l’a entendue ?
${}^{19}Voici la tempête du Seigneur :
        sa fureur éclate, la tempête tourbillonne,
        elle tournoie sur la tête des méchants.
${}^{20}La colère du Seigneur ne se détournera pas
        avant d’avoir agi et réalisé les desseins de son cœur.
        \\Dans les derniers jours, vous le comprendrez vraiment.
${}^{21}Je n’ai pas envoyé ces prophètes,
        et pourtant ils courent !
        \\Je ne leur ai pas parlé,
        et pourtant ils prophétisent !
${}^{22}S’ils s’étaient tenus à mon conseil,
        ils auraient fait entendre mes paroles à mon peuple,
        \\ils l’auraient fait revenir de sa conduite mauvaise
        et de la malice de ses actes.
         
${}^{23}Suis-je Dieu de près
        – oracle du Seigneur –,
        \\et non Dieu de loin ?
${}^{24}Si un homme se dissimule dans des lieux cachés,
        ne le verrais-je pas ?
        – oracle du Seigneur.
        \\Ne suis-je pas celui qui remplit ciel et terre ?
        – oracle du Seigneur.
         
${}^{25}J’ai entendu ce que disent les prophètes,
        \\ceux qui prophétisent en mon nom le mensonge,
        et qui disent : « J’ai eu un songe ! J’ai eu un songe ! »
${}^{26}Combien de temps y aura-t-il encore parmi les prophètes
        \\des gens qui prophétisent le mensonge,
        qui prophétisent la tromperie de leur cœur ?
${}^{27}Avec les songes qu’ils se racontent l’un à l’autre,
        pensent-ils faire oublier mon nom à mon peuple
        \\comme leurs pères avaient oublié mon nom
        avec le dieu Baal ?
${}^{28}Le prophète qui a un songe,
        qu’il raconte le songe ;
        \\celui qui tient de moi une parole,
        qu’il dise ma parole en vérité.
        \\Qu’y a-t-il de commun entre la paille et le grain ?
        – oracle du Seigneur.
${}^{29}Ma parole n’est-elle pas comme un feu
        – oracle du Seigneur –,
        comme un marteau qui fracasse le roc ?
${}^{30}C’est pourquoi, me voici contre les prophètes
        – oracle du Seigneur –,
        ceux qui se volent l’un à l’autre mes paroles.
${}^{31}Me voici contre les prophètes
        – oracle du Seigneur –,
        ceux qui agitent leur langue pour proférer un oracle.
${}^{32}Me voici contre ceux qui prophétisent des rêves mensongers
        – oracle du Seigneur.
        \\En les racontant, ils égarent mon peuple
        par leurs mensonges et par leurs vantardises,
        \\alors que moi je ne les ai pas envoyés
        et ne leur ai pas donné d’ordre ;
        \\ils ne sont d’aucune utilité à ce peuple
        – oracle du Seigneur.
       
${}^{33}Quand ce peuple, le prophète ou un prêtre te demandera : « De quelle proclamation le Seigneur t’a chargé ? », tu leur répondras : « Pour ce qui est d’une charge, je vais me débarrasser de vous ! – oracle du Seigneur. »
${}^{34}Et le prophète, le prêtre ou celui du peuple qui dira : « Proclamation du Seigneur ! », cet homme-là, je le châtierai, lui et sa maison. 
${}^{35}Vous allez parler ainsi, chacun à son prochain, chacun à son frère : « Que répond le Seigneur ? », « Que déclare le Seigneur ? » 
${}^{36}Vous ne prononcerez plus : « Proclamation du Seigneur », car, pour chacun, la proclamation n’est que sa propre parole. Vous avez perverti les paroles du Dieu vivant, le Seigneur de l’univers, notre Dieu. 
${}^{37}Va plutôt demander au prophète : « Que te répond le Seigneur ? », « Que déclare le Seigneur ? » 
${}^{38}Si, au contraire, vous dites : « Proclamation du Seigneur », alors ainsi parle le Seigneur : Puisque vous dites les mots « Proclamation du Seigneur », alors que je vous ai fait avertir de ne plus le dire, 
${}^{39}à cause de cela, je vais vous soulever comme une charge et me débarrasser de vous et de la ville que je vous ai donnée, à vous et à vos pères. 
${}^{40}Et je mettrai sur vous un éternel reproche, une confusion éternelle, qui ne s’oublieront pas.
      
         
      \bchapter{}
      \begin{verse}
${}^{1}Voici ce que le Seigneur me fit voir : Deux corbeilles de figues étaient disposées devant le temple du Seigneur. C’était après que Nabucodonosor, roi de Babylone, eut déporté loin de Jérusalem Jékonias, fils de Joakim, roi de Juda, ainsi que les princes de Juda, les artisans et forgerons, et qu’il les eut emmenés à Babylone. 
${}^{2}L’une des corbeilles contenait de très bonnes figues, comme le sont les figues précoces. L’autre, de très mauvaises figues, si mauvaises qu’elles étaient immangeables.
${}^{3}Le Seigneur me dit : « Que vois-tu, Jérémie ? » Je répondis : « Des figues ! Les bonnes figues sont très bonnes, et les mauvaises, si mauvaises qu’elles sont immangeables. » 
${}^{4}Alors la parole du Seigneur me fut adressée : 
${}^{5}Ainsi parle le Seigneur, le Dieu d’Israël : Comme on apprécie ces bonnes figues, j’apprécierai les déportés de Juda que j’ai expulsés de ce lieu au pays des Chaldéens. 
${}^{6}Pour leur bien, je poserai sur eux mon regard et les ramènerai sur cette terre. Je les bâtirai, je ne démolirai pas ; je les planterai, je n’arracherai pas. 
${}^{7}Je leur donnerai un cœur qui me connaisse, car je suis le Seigneur ; ils seront mon peuple, et moi, je serai leur Dieu, car ils reviendront à moi de tout leur cœur. 
${}^{8}Comme on traite les mauvaises figues, si mauvaises qu’elles sont immangeables – ainsi parle le Seigneur – je traiterai Sédécias roi de Juda, ses princes et le reste de Jérusalem, ceux qui sont restés dans ce pays et ceux qui habitent au pays d’Égypte : 
${}^{9}je ferai d’eux un objet de stupeur, une calamité pour tous les royaumes de la terre, une insulte et une fable, un objet de raillerie et de malédiction en tous lieux où je les chasserai. 
${}^{10}Et j’enverrai contre eux l’épée, la famine et la peste, jusqu’à ce qu’ils disparaissent de la terre que je leur ai donnée, à eux comme à leurs pères.
      
         
      \bchapter{}
      \begin{verse}
${}^{1}Parole adressée à Jérémie pour tout le peuple de Juda, la quatrième année du règne de Joakim, fils de Josias, roi de Juda ; c’était la première année de Nabucodonosor, roi de Babylone. 
${}^{2}Le prophète Jérémie la prononça pour tout le peuple de Juda et pour tous les habitants de Jérusalem ; il dit :
${}^{3}Depuis la treizième année de Josias, fils d’Amone, roi de Juda, jusqu’à ce jour, cela fait vingt-trois ans que la parole du Seigneur m’est adressée, et qu’inlassablement je vous parle sans que vous n’écoutiez. 
${}^{4}Inlassablement, le Seigneur vous a envoyé tous ses serviteurs les prophètes ; mais vous n’avez pas écouté ni prêté l’oreille pour entendre, 
${}^{5}lorsqu’ils disaient : Que chacun de vous revienne de son mauvais chemin et de la malice de ses actes ! Alors, vous habiterez sur la terre que le Seigneur vous a donnée, à vous et à vos pères, depuis toujours et pour toujours. 
${}^{6}N’allez pas suivre d’autres dieux pour les servir et vous prosterner devant eux ! Cessez de m’offenser par les œuvres de vos mains, et je ne vous ferai aucun mal. 
${}^{7}Mais vous ne m’avez pas écouté – oracle du Seigneur –, de sorte que vous m’avez offensé par les œuvres de vos mains, pour votre malheur.
${}^{8}C’est pourquoi, ainsi parle le Seigneur de l’univers : Puisque vous n’avez pas écouté mes paroles, 
${}^{9}voici que j’envoie chercher tous les peuples du nord – oracle du Seigneur –, et je les amènerai à mon serviteur Nabucodonosor, roi de Babylone, contre ce pays, contre ses habitants, et contre toutes les nations d’alentour. Je les vouerai à l’anathème, je les livrerai à la désolation, à la dérision et à la ruine pour toujours. 
${}^{10}Je ferai disparaître de chez eux chants d’allégresse et chants de joie, le chant de l’époux et le chant de l’épousée, le chant des deux meules et la clarté de la lampe. 
${}^{11}Tout ce pays ne sera que ruines et désolation, et ces nations serviront le roi de Babylone pendant soixante-dix ans. 
${}^{12}Mais lorsque les soixante-dix ans seront accomplis, je sévirai contre le roi de Babylone et contre cette nation – oracle du Seigneur – à cause de leur faute, et contre le pays des Chaldéens ; je le livrerai à la désolation pour toujours. 
${}^{13}Je ferai venir sur ce pays tout ce que mes paroles ont annoncé contre lui, tout ce qui est écrit dans ce livre, ce que Jérémie a prophétisé contre toutes les nations. 
${}^{14}Des nations nombreuses et des grands rois les asserviront, eux aussi ; je les rétribuerai selon leur conduite et les œuvres de leurs mains.
${}^{15}Oui, ainsi m’a parlé le Seigneur, le Dieu d’Israël : « Prends de ma main cette coupe d’un vin de colère et fais-la boire à toutes les nations auxquelles je t’envoie. 
${}^{16}Elles boiront, tituberont, s’affoleront à cause de l’épée que j’envoie au milieu d’elles. » 
${}^{17}Je pris la coupe de la main du Seigneur et je fis boire toutes les nations auxquelles le Seigneur m’avait envoyé : 
${}^{18}Jérusalem et les villes de Juda, ses rois et ses princes, pour en faire une ruine, un lieu désolé, un objet de dérision et de malédiction, comme on le voit aujourd’hui ; 
${}^{19}Pharaon, roi d’Égypte, ses serviteurs, ses princes et tout son peuple ; 
${}^{20}tout un mélange de peuples et tous les rois du pays de Ouç ; tous les rois du pays des Philistins, Ascalon, Gaza, Éqrone et ce qui reste d’Ashdod ; 
${}^{21}Édom, Moab et les fils d’Ammone ; 
${}^{22}tous les rois de Tyr et tous les rois de Sidon, tous les rois de l’Île qui est au-delà de la Mer ; 
${}^{23}Dedane, Téma, Bouz, et tous les hommes aux tempes rasées ; 
${}^{24}tous les rois d’Arabie et tous les rois d’un mélange de peuples qui demeurent au désert ; 
${}^{25}tous les rois de Zimri, tous les rois d’Élam et tous les rois de Médie ; 
${}^{26}tous les rois du nord, proches ou lointains, les uns après les autres, et tous les royaumes qui sont à la surface de la terre. Et après eux, boira le roi de Babylone.
${}^{27}Tu leur diras : « Ainsi parle le Seigneur de l’univers, le Dieu d’Israël : Buvez, enivrez-vous et vomissez ! Tombez et ne vous relevez pas, à cause de l’épée que j’envoie au milieu de vous. » 
${}^{28}Et s’ils refusent de prendre la coupe de ta main et de boire, tu leur diras : « Ainsi parle le Seigneur de l’univers : Vous devez boire ! 
${}^{29}Oui, c’est dans la ville sur laquelle mon nom est invoqué que j’inaugure le malheur ; alors, comment seriez-vous quittes, vous ? Non, vous ne serez pas quittes, car j’appelle l’épée contre tous les habitants de la terre – oracle du Seigneur de l’univers. »
       
${}^{30}Toi, tu proclameras contre eux toutes ces prophéties ; tu leur diras :
        \\D’en haut, le Seigneur rugit,
        de sa demeure sainte, il donne de la voix,
        \\il pousse des rugissements contre son domaine,
        \\il entonne le cri des fouleurs de raisins
        contre tous les habitants de la terre.
${}^{31}Un tumulte parvient jusqu’aux extrémités de la terre,
        \\car le Seigneur entre en procès avec les nations,
        en jugement avec tout être de chair.
        \\Les méchants, il les livre à l’épée,
        – oracle du Seigneur.
${}^{32}Ainsi parle le Seigneur de l’univers :
        \\Le malheur va se propager de nation en nation,
        une grande tempête se lève aux confins de la terre.
${}^{33}Il arrivera, en ce jour-là, que les victimes du Seigneur,
        d’une extrémité à l’autre de la terre,
        \\ne seront ni pleurées, ni recueillies, ni enterrées ;
        elles deviendront du fumier à la surface du sol.
${}^{34}Gémissez, pasteurs, et criez !
        Roulez-vous par terre, maîtres du troupeau,
        \\car ils sont accomplis, vos jours, pour l’abattage ;
        vous serez dispersés
        et comme un vase précieux vous tomberez.
${}^{35}Pour les pasteurs, plus de refuge,
        et plus d’issue pour les maîtres du troupeau.
${}^{36}Clameur et cri des pasteurs,
        gémissement des maîtres du troupeau !
        \\Le Seigneur a dévasté leurs pâturages.
${}^{37}Les enclos prospères sont réduits au silence
        sous l’effet de l’ardente colère du Seigneur.
${}^{38}Comme un jeune lion, il a quitté sa tanière ;
        leur pays est devenu un lieu désolé,
        \\sous l’effet de son ardeur impitoyable,
        sous l’effet de son ardente colère.
