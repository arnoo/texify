  
  
    
    \bbook{APOCALYPSE}{APOCALYPSE}
      
         
      \bchapter{}
      \begin{verse}
${}^{1}Révélation de Jésus Christ, que Dieu lui a confiée pour montrer à ses serviteurs ce qui doit bientôt advenir ; cette révélation, il l’a fait connaître à son serviteur Jean par l’envoi de son ange. 
${}^{2}Jean atteste comme parole de Dieu et témoignage de Jésus Christ tout ce qu’il a vu. 
${}^{3}Heureux celui qui lit, heureux ceux qui écoutent les paroles de la prophétie et gardent ce qui est écrit en elle, car le temps est proche.
      
         
${}^{4}Jean, aux sept Églises qui sont en Asie Mineure : à vous, la grâce et la paix, de la part de Celui qui est, qui était et qui vient, de la part des sept esprits qui sont devant son trône, 
${}^{5}de la part de Jésus Christ, le témoin fidèle, le premier-né des morts, le prince des rois de la terre. À lui qui nous aime, qui nous a délivrés de nos péchés par son sang, 
${}^{6}qui a fait de nous un royaume et des prêtres pour son Dieu et Père, à lui, la gloire et la souveraineté pour les siècles des siècles. Amen.
        ${}^{7}Voici qu’il vient avec les nuées, tout œil le verra,
        \\ils le verront, ceux qui l’ont transpercé ;
        \\et sur lui se lamenteront toutes les tribus de la terre.
        \\Oui ! Amen !
${}^{8}Moi, je suis l’Alpha et l’Oméga, dit le Seigneur Dieu, Celui qui est, qui était et qui vient, le Souverain de l’univers.
${}^{9}Moi, Jean, votre frère, partageant avec vous la détresse, la royauté et la persévérance en Jésus, je me trouvai dans l’île de Patmos à cause de la parole de Dieu et du témoignage de Jésus. 
${}^{10}Je fus saisi en esprit, le jour du Seigneur, et j’entendis derrière moi une voix forte, pareille au son d’une trompette. 
${}^{11}Elle disait : « Ce que tu vois, écris-le dans un livre et envoie-le aux sept Églises : à Éphèse, Smyrne, Pergame, Thyatire, Sardes, Philadelphie et Laodicée. »
${}^{12}Je me retournai pour regarder quelle était cette voix qui me parlait. M’étant retourné, j’ai vu sept chandeliers d’or, 
${}^{13}et au milieu des chandeliers un être qui semblait un Fils d’homme, revêtu d’une longue tunique, une ceinture d’or à hauteur de poitrine ; 
${}^{14}sa tête et ses cheveux étaient blancs comme la laine blanche, comme la neige, et ses yeux comme une flamme ardente ; 
${}^{15}ses pieds semblaient d’un bronze précieux affiné au creuset, et sa voix était comme la voix des grandes eaux ; 
${}^{16}il avait dans la main droite sept étoiles ; de sa bouche sortait un glaive acéré à deux tranchants. Son visage brillait comme brille le soleil dans sa puissance.
${}^{17}Quand je le vis, je tombai à ses pieds comme mort, mais il posa sur moi sa main droite, en disant : « Ne crains pas. Moi, je suis le Premier et le Dernier, 
${}^{18}le Vivant : j’étais mort, et me voilà vivant pour les siècles des siècles ; je détiens les clés de la mort et du séjour des morts. 
${}^{19}Écris donc ce que tu as vu, ce qui est, ce qui va ensuite advenir. 
${}^{20}Quant au mystère des sept étoiles que tu as vues sur ma main droite, et celui des sept chandeliers d’or : les sept étoiles sont les anges des sept Églises, et les sept chandeliers sont les sept Églises. »
      
         
      \bchapter{}
      \begin{verse}
${}^{1}À l’ange de l’Église qui est à Éphèse, écris :
      \begin{verse}Ainsi parle celui qui tient les sept étoiles dans sa main droite, qui marche au milieu des sept chandeliers d’or : 
${}^{2}Je connais tes actions, ta peine, ta persévérance, je sais que tu ne peux supporter les malfaisants ; tu as mis à l’épreuve ceux qui se disent apôtres et ne le sont pas ; tu as découvert qu’ils étaient menteurs. 
${}^{3}Tu ne manques pas de persévérance, et tu as tant supporté pour mon nom, sans ménager ta peine. 
${}^{4}Mais j’ai contre toi que ton premier amour, tu l’as abandonné. 
${}^{5}Eh bien, rappelle-toi d’où tu es tombé, convertis-toi, reviens à tes premières actions. Sinon je vais venir à toi et je délogerai ton chandelier de sa place, si tu ne t’es pas converti. 
${}^{6}Pourtant, tu as cela pour toi que tu détestes les agissements des Nicolaïtes – et je les déteste, moi aussi.
${}^{7}Celui qui a des oreilles, qu’il entende ce que l’Esprit dit aux Églises. Au vainqueur, je donnerai de goûter à l’arbre de la vie qui est dans le paradis de Dieu.
${}^{8}À l’ange de l’Église qui est à Smyrne, écris :
      Ainsi parle celui qui est le Premier et le Dernier, celui qui était mort et qui est entré dans la vie : 
${}^{9}Je sais ta détresse et ta pauvreté ; pourtant tu es riche ! Je connais les propos blasphématoires de ceux qui se disent Juifs et ne le sont pas : ils sont une synagogue de Satan. 
${}^{10}Sois sans aucune crainte pour ce que tu vas souffrir. Voici que le diable va jeter en prison certains des vôtres pour vous mettre à l’épreuve, et vous serez dans la détresse pendant dix jours. Sois fidèle jusqu’à la mort, et je te donnerai la couronne de la vie.
${}^{11}Celui qui a des oreilles, qu’il entende ce que l’Esprit dit aux Églises. Le vainqueur ne pourra être atteint par la seconde mort.
${}^{12}À l’ange de l’Église qui est à Pergame, écris :
      Ainsi parle celui qui a le glaive acéré à deux tranchants : 
${}^{13}Je sais où tu habites : c’est là que Satan a son trône ; mais tu tiens ferme à mon nom, et tu n’as pas renié ma foi, même dans les jours où Antipas, mon témoin fidèle, a été mis à mort chez vous, là où Satan habite. 
${}^{14}Mais j’ai quelque chose contre toi : tu as là des gens qui tiennent ferme à la doctrine de Balaam ; celui-ci enseignait à Balaq comment faire trébucher les fils d’Israël, pour qu’ils mangent des viandes offertes aux idoles et qu’ils se prostituent. 
${}^{15}De même, tu as, toi aussi, des gens qui tiennent ferme à la doctrine des Nicolaïtes. 
${}^{16}Eh bien, convertis-toi : sinon je vais venir à toi sans tarder ; avec le glaive de ma bouche je les combattrai.
${}^{17}Celui qui a des oreilles, qu’il entende ce que l’Esprit dit aux Églises. Au vainqueur je donnerai de la manne cachée, je lui donnerai un caillou blanc, et, inscrit sur ce caillou, un nom nouveau que nul ne sait, sauf celui qui le reçoit.
${}^{18}À l’ange de l’Église qui est à Thyatire, écris :
      Ainsi parle le Fils de Dieu, celui qui a les yeux comme une flamme ardente et des pieds qui semblent de bronze précieux : 
${}^{19}Je connais tes actions, je sais ton amour, ta foi, ton engagement, ta persévérance, et tes dernières actions surpassent les premières. 
${}^{20}Mais j’ai contre toi que tu laisses faire Jézabel, cette femme qui se dit prophétesse, et qui égare mes serviteurs en leur enseignant à se prostituer et à manger des viandes offertes aux idoles. 
${}^{21}Je lui ai donné du temps pour se convertir, mais elle ne veut pas se convertir de sa prostitution. 
${}^{22}Voici que je vais la jeter sur un lit de grande détresse, elle et ses compagnons d’adultère, à moins que, renonçant aux agissements de cette femme, ils ne se convertissent ; 
${}^{23}et ses enfants, je vais les frapper de mort. Toutes les Églises reconnaîtront que moi, je suis celui qui scrute les reins et les cœurs, et je donnerai à chacun de vous selon ses œuvres. 
${}^{24}Mais vous, les autres de Thyatire, qui ne partagez pas cette doctrine et n’avez pas connu les « profondeurs de Satan » – comme ils disent –, je vous déclare que je ne vous impose pas d’autre fardeau ; 
${}^{25}tenez fermement, du moins, ce que vous avez, jusqu’à ce que je vienne.
${}^{26}Le vainqueur, celui qui reste fidèle jusqu'à la fin à ma façon d’agir, je lui donnerai autorité sur les nations, 
${}^{27}et il les conduira avec un sceptre de fer, comme des vases de potier que l’on brise. 
${}^{28}Il sera comme moi qui ai reçu autorité de mon Père, et je lui donnerai l’étoile du matin. 
${}^{29}Celui qui a des oreilles, qu’il entende ce que l’Esprit dit aux Églises.
      
         
      \bchapter{}
${}^{1}À l’ange de l’Église qui est à Sardes, écris :
      \begin{verse}Ainsi parle celui qui a les sept esprits de Dieu et les sept étoiles : Je connais ta conduite, je sais que ton nom est celui d’un vivant, mais tu es mort. 
${}^{2}Sois vigilant, raffermis ce qui te reste et qui allait mourir, car je n’ai pas trouvé que tes actes soient parfaits devant mon Dieu. 
${}^{3}Eh bien, rappelle-toi ce que tu as reçu et entendu, garde-le et convertis-toi. Si tu ne veilles pas, je viendrai comme un voleur et tu ne pourras savoir à quelle heure je viendrai te surprendre. 
${}^{4}À Sardes, pourtant, tu en as qui n’ont pas sali leurs vêtements ; habillés de blanc, ils marcheront avec moi, car ils en sont dignes.
${}^{5}Ainsi, le vainqueur portera des vêtements blancs ; jamais je n’effacerai son nom du livre de la vie ; son nom, je le proclamerai devant mon Père et devant ses anges. 
${}^{6}Celui qui a des oreilles, qu’il entende ce que l’Esprit dit aux Églises.
${}^{7}À l’ange de l’Église qui est à Philadelphie, écris :
      Ainsi parle le Saint, le Vrai, celui qui détient la clé de David, celui qui ouvre – et nul ne fermera –, celui qui ferme – et nul ne peut ouvrir. 
${}^{8}Je connais ta conduite ; voici que j’ai mis devant toi une porte ouverte que nul ne peut fermer, car, sans avoir beaucoup de puissance, tu as gardé ma parole et tu n’as pas renié mon nom. 
${}^{9}Voici que je vais te donner des gens de la synagogue de Satan, qui se disent Juifs et ne le sont pas : ils mentent. Voici ce que je leur ferai : ils viendront, ils se prosterneront à tes pieds ; alors ils connaîtront que moi, je t’ai aimé. 
${}^{10}Puisque tu as gardé mon appel à persévérer, moi aussi je te garderai de l’heure de l’épreuve qui va venir sur le monde entier pour éprouver les habitants de la terre. 
${}^{11}Je viens sans tarder : tiens fermement ce que tu as, pour que personne ne prenne ta couronne.
${}^{12}Du vainqueur, je ferai une colonne au sanctuaire de mon Dieu ; il n’aura plus jamais à en sortir, et je graverai sur lui le nom de mon Dieu et le nom de la ville de mon Dieu, la Jérusalem nouvelle qui descend du ciel d’auprès de mon Dieu, ainsi que mon nom nouveau. 
${}^{13}Celui qui a des oreilles, qu’il entende ce que l’Esprit dit aux Églises.
${}^{14}À l’ange de l’Église qui est à Laodicée, écris :
      Ainsi parle celui qui est l’Amen, le témoin fidèle et vrai, le principe de la création de Dieu : 
${}^{15}Je connais tes actions, je sais que tu n’es ni froid ni brûlant – mieux vaudrait que tu sois ou froid ou brûlant. 
${}^{16}Aussi, puisque tu es tiède – ni brûlant ni froid – je vais te vomir de ma bouche. 
${}^{17}Tu dis : « Je suis riche, je me suis enrichi, je ne manque de rien », et tu ne sais pas que tu es malheureux, pitoyable, pauvre, aveugle et nu ! 
${}^{18}Alors, je te le conseille : achète chez moi, pour t’enrichir, de l’or purifié au feu, des vêtements blancs pour te couvrir et ne pas laisser paraître la honte de ta nudité, un remède pour l’appliquer sur tes yeux afin que tu voies. 
${}^{19}Moi, tous ceux que j’aime, je leur montre leurs fautes, et je les corrige. Eh bien, sois fervent et convertis-toi. 
${}^{20}Voici que je me tiens à la porte, et je frappe. Si quelqu’un entend ma voix et ouvre la porte, j’entrerai chez lui ; je prendrai mon repas avec lui, et lui avec moi.
${}^{21}Le vainqueur, je lui donnerai de siéger avec moi sur mon Trône, comme moi-même, après ma victoire, j’ai siégé avec mon Père sur son Trône. 
${}^{22}Celui qui a des oreilles, qu’il entende ce que l’Esprit dit aux Églises.
      <h2 class="intertitle" id="d85e404627">1. Liturgie dans le ciel et ouverture des sept sceaux (4 – 8,5)</h2>
      <p class="cantique"><span class="cantique_label"><a href="#bib_ct-nt_9">Cantique NT 9</a></span> = <span class="cantique_ref"><a class="unitex_link" href="#bib_ap_4_11">Ap 4,11</a> ; <a class="unitex_link" href="#bib_ap_5_9">5,9.10.12</a></span>
      
         
      \bchapter{}
      \begin{verse}
${}^{1}Après cela, j’ai vu : et voici qu’il y avait une porte ouverte dans le ciel. Et la voix que j’avais entendue, pareille au son d’une trompette, me parlait en disant : « Monte jusqu’ici, et je te ferai voir ce qui doit ensuite advenir. »
${}^{2}Aussitôt je fus saisi en esprit. Voici qu’un trône était là dans le ciel, et sur le Trône siégeait quelqu’un. 
${}^{3}Celui qui siège a l’aspect d’une pierre de jaspe ou de cornaline ; il y a, tout autour du Trône, un halo de lumière, avec des reflets d’émeraude. 
${}^{4}Tout autour de ce Trône, vingt-quatre trônes, où siègent vingt-quatre Anciens portant des vêtements blancs et, sur leurs têtes, des couronnes d’or. 
${}^{5}Et du Trône sortent des éclairs, des fracas, des coups de tonnerre, et sept torches enflammées brûlent devant le Trône : ce sont les sept esprits de Dieu. 
${}^{6}Devant le Trône, il y a comme une mer, aussi transparente que du cristal. Au milieu, autour du Trône, quatre Vivants, ayant des yeux innombrables en avant et en arrière. 
${}^{7}Le premier Vivant ressemble à un lion, le deuxième Vivant ressemble à un jeune taureau, le troisième Vivant a comme un visage d’homme, le quatrième Vivant ressemble à un aigle en plein vol. 
${}^{8}Les quatre Vivants ont chacun six ailes, avec des yeux innombrables tout autour et au-dedans. Jour et nuit, ils ne cessent de dire :
        « Saint ! Saint ! Saint, le Seigneur Dieu,
        le Souverain de l’univers,
        Celui qui était, qui est et qui vient. »
${}^{9}Lorsque les Vivants rendent gloire, honneur et action de grâce à celui qui siège sur le Trône, lui qui vit pour les siècles des siècles, 
${}^{10}les vingt-quatre Anciens se jettent devant celui qui siège sur le Trône, ils se prosternent face à celui qui vit pour les siècles des siècles ; ils lancent leur couronne devant le Trône en disant :
        ${}^{11}« Tu es digne, Seigneur notre Dieu,
        \\de recevoir
        la gloire, l’honneur et la puissance.
        \\C’est toi qui créas l’univers ;
        \\tu as voulu qu’il soit :
        il fut créé. »
      
         
      \bchapter{}
      \begin{verse}
${}^{1}J’ai vu, dans la main droite de celui qui siège sur le Trône, un livre en forme de rouleau, écrit au-dedans et à l’extérieur, scellé de sept sceaux. 
${}^{2}Puis j’ai vu un ange plein de force, qui proclamait d’une voix puissante : « Qui donc est digne d’ouvrir le Livre et d’en briser les sceaux ? » 
${}^{3}Mais personne, au ciel, sur terre ou sous la terre, ne pouvait ouvrir le Livre et regarder. 
${}^{4}Je pleurais beaucoup, parce que personne n’avait été trouvé digne d’ouvrir le Livre et de regarder. 
${}^{5}Mais l’un des Anciens me dit : « Ne pleure pas. Voilà qu’il a remporté la victoire, le lion de la tribu de Juda, le rejeton de David : il ouvrira le Livre aux sept sceaux. »
${}^{6}Et j’ai vu, entre le Trône, les quatre Vivants et les Anciens, un Agneau debout, comme égorgé ; ses cornes étaient au nombre de sept, ainsi que ses yeux, qui sont les sept esprits de Dieu envoyés sur toute la terre. 
${}^{7}Il s’avança et prit le Livre dans la main droite de celui qui siégeait sur le Trône.
${}^{8}Quand l’Agneau eut pris le Livre, les quatre Vivants et les vingt-quatre Anciens se jetèrent à ses pieds. Ils tenaient chacun une cithare et des coupes d’or pleines de parfums qui sont les prières des saints. 
${}^{9}Ils chantaient ce cantique nouveau :
        \\« Tu es digne, de prendre le Livre
        \\et d’en ouvrir les sceaux,
        \\car tu fus immolé\\,
        \\rachetant\\pour Dieu, par ton sang,
        \\des gens de toute tribu,
        \\langue, peuple et nation.
        ${}^{10}Pour notre Dieu, tu en as fait
        \\un royaume et des prêtres :
        \\ils régneront sur la terre. »
${}^{11}Alors j’ai vu : et j’entendis la voix d’une multitude d’anges qui entouraient le Trône, les Vivants et les Anciens ; ils étaient des myriades de myriades, par milliers de milliers. 
${}^{12}Ils disaient d’une voix forte :
        \\« Il est digne, l’Agneau immolé\\,
        \\de recevoir puissance et richesse,
        \\sagesse et force,
        \\honneur, gloire et louange. »
      <div class="box_other filet_bleu">
          <h3 class="intertitle cantique_chap" id="bib_ct-nt_9">Cantique NT 9</h3><a class="cantique_chap" href="#bib_ap_4">4</a>
            <a class="cantique_verset" href="#bib_ap_4_11"><span class="cantique_verset_in">11</span></a>« Tu es digne, Seigneur notre Dieu,
            \\de recevoir
            la gloire, l’honneur et la puissance.
             
            \\C’est toi qui créas l’univers ;
            \\tu as voulu qu’il soit :
            il fut créé.
           <a class="cantique_chap" href="#bib_ap_5">5</a>
            <a class="cantique_verset" href="#bib_ap_5_11"><span class="cantique_verset_in">9</span></a>Tu es digne, Christ et Seigneur\\,
            \\de prendre le Livre
            \\et d’en ouvrir les sceaux,
             
            \\car tu fus immolé\\,
            \\rachetant\\pour Dieu, par ton sang,
            \\des gens de toute tribu,
            \\langue, peuple et nation.
             
            <a class="cantique_verset" href="#bib_ap_5_10"><span class="cantique_verset_in">10</span></a>Pour notre Dieu, tu en as fait
            \\un royaume et des prêtres :
            \\ils régneront sur la terre.
             
            <a class="cantique_verset" href="#bib_ap_5_12"><span class="cantique_verset_in">12</span></a>Il est digne, l’Agneau immolé\\,
            \\de recevoir puissance et richesse,
            \\sagesse et force,
            \\honneur, gloire et louange. »
${}^{13}Toute créature dans le ciel et sur la terre, sous la terre et sur la mer, et tous les êtres qui s’y trouvent, je les entendis proclamer :
        \\« À celui qui siège sur le Trône, et à l’Agneau,
        \\la louange et l’honneur,
        \\la gloire et la souveraineté
        \\pour les siècles des siècles. »
${}^{14}Et les quatre Vivants disaient : « Amen ! » ; et les Anciens, se jetant devant le Trône, se prosternèrent.
      
         
      \bchapter{}
      \begin{verse}
${}^{1}Alors j’ai vu : quand l’Agneau ouvrit l’un des sept sceaux, j’entendis l’un des quatre Vivants dire d’une voix de tonnerre : « Viens ! » 
${}^{2}Alors j’ai vu : et voici un cheval blanc ; celui qui le montait tenait un arc, une couronne lui fut donnée, et il sortit vainqueur, pour vaincre à nouveau.
${}^{3}Et quand il ouvrit le deuxième sceau, j’entendis le deuxième Vivant qui disait : « Viens ! » 
${}^{4}Alors sortit un autre cheval, rouge feu ; à celui qui le montait il fut donné d’enlever la paix à la terre, pour que les gens s’entretuent, et une grande épée lui fut donnée.
${}^{5}Et quand il ouvrit le troisième sceau, j’entendis le troisième Vivant qui disait : « Viens ! » Alors j’ai vu : et voici un cheval noir ; celui qui le montait tenait à la main une balance. 
${}^{6}Et j’entendis comme une voix au milieu des quatre Vivants ; elle disait : « Un denier, la mesure de blé ! Un denier, les trois mesures d’orge ! Ne fraude pas sur l’huile et sur le vin ! »
${}^{7}Et quand il ouvrit le quatrième sceau, j’entendis la voix du quatrième Vivant qui disait : « Viens ! » 
${}^{8}Alors j’ai vu : et voici un cheval verdâtre ; celui qui le montait se nomme la Mort, et le séjour des morts l’accompagnait. Et il leur fut donné pouvoir sur un quart de la terre pour tuer par le glaive, par la famine et par la peste, et par les fauves de la terre.
${}^{9}Et quand il ouvrit le cinquième sceau, je vis sous l’autel les âmes de ceux qui furent égorgés à cause de la parole de Dieu et du témoignage qu’ils avaient porté. 
${}^{10}Ils crièrent d’une voix forte : « Jusques à quand, Maître saint et vrai, resteras-tu sans juger, sans venger notre sang sur les habitants de la terre ? » 
${}^{11}Et il fut donné à chacun une robe blanche, et il leur fut dit de patienter encore quelque temps, jusqu’à ce que soient au complet leurs compagnons de service, leurs frères, qui allaient être tués comme eux.
${}^{12}Alors j’ai vu : quand il ouvrit le sixième sceau, il y eut un grand tremblement de terre, le soleil devint noir comme une étoffe de crin, et la lune entière, comme du sang, 
${}^{13}et les étoiles du ciel tombèrent sur la terre comme lorsqu’un figuier secoué par grand vent jette ses fruits. 
${}^{14}Le ciel se retira comme un livre qu’on referme ; toutes les montagnes et les îles furent déplacées. 
${}^{15}Les rois de la terre et les grands, les chefs d’armée, les riches et les puissants, tous les esclaves et les hommes libres allèrent se cacher dans les cavernes et les rochers des montagnes. 
${}^{16}Et ils disaient aux montagnes et aux rochers : « Tombez sur nous, et cachez-nous du regard de celui qui siège sur le Trône et aussi de la colère de l’Agneau. 
${}^{17}Car il est venu, le grand jour de leur colère, et qui pourrait tenir ? »
      
         
      \bchapter{}
      \begin{verse}
${}^{1}Après cela, j’ai vu quatre anges debout aux quatre coins de la terre, maîtrisant les quatre vents de la terre, pour empêcher le vent de souffler sur la terre, sur la mer et sur tous les arbres. 
${}^{2}Puis j’ai vu un autre ange qui montait du côté où le soleil se lève, avec le sceau qui imprime la marque du Dieu vivant ; d’une voix forte, il cria aux quatre anges qui avaient reçu le pouvoir de faire du mal à la terre et à la mer : 
${}^{3}« Ne faites pas de mal à la terre, ni à la mer, ni aux arbres, avant que nous ayons marqué du sceau le front des serviteurs de notre Dieu. »
${}^{4}Et j’entendis le nombre de ceux qui étaient marqués du sceau : ils étaient cent quarante-quatre mille, de toutes les tribus des fils d’Israël.
${}^{5}De la tribu de Juda, douze mille marqués du sceau ;
        \\de la tribu de Roubène, douze mille ;
        \\de la tribu de Gad, douze mille ;
${}^{6}de la tribu d’Aser, douze mille ;
        \\de la tribu de Nephtali, douze mille ;
        \\de la tribu de Manassé, douze mille ;
${}^{7}de la tribu de Siméon, douze mille ;
        \\de la tribu de Lévi, douze mille ;
        \\de la tribu d’Issakar, douze mille ;
${}^{8}de la tribu de Zabulon, douze mille ;
        \\de la tribu de Joseph, douze mille ;
        \\de la tribu de Benjamin, douze mille marqués du sceau.
${}^{9}Après cela, j’ai vu : et voici une foule immense, que nul ne pouvait dénombrer, une foule de toutes nations, tribus, peuples et langues. Ils se tenaient debout devant le Trône et devant l’Agneau, vêtus de robes blanches, avec des palmes à la main. 
${}^{10}Et ils s’écriaient d’une voix forte :
        \\« Le salut appartient à notre Dieu
        \\qui siège sur le Trône
        \\et à l’Agneau ! »
${}^{11}Tous les anges se tenaient debout autour du Trône, autour des Anciens et des quatre Vivants ; se jetant devant le Trône, face contre terre, ils se prosternèrent devant Dieu. 
${}^{12}Et ils disaient :
        \\« Amen !
        \\Louange, gloire, sagesse et action de grâce,
        \\honneur, puissance et force
        \\à notre Dieu, pour les siècles des siècles ! Amen ! »
${}^{13}L’un des Anciens prit alors la parole et me dit : « Ces gens vêtus de robes blanches, qui sont-ils, et d’où viennent-ils ? » 
${}^{14}Je lui répondis : « Mon seigneur, toi, tu le sais. » Il me dit :
        \\« Ceux-là viennent de la grande épreuve ;
        \\ils ont lavé leurs robes,
        \\ils les ont blanchies par le sang de l’Agneau.
        ${}^{15}C’est pourquoi ils sont devant le trône de Dieu,
        \\et le servent, jour et nuit, dans son sanctuaire.
        \\Celui qui siège sur le Trône
        \\établira sa demeure chez eux.
        ${}^{16}Ils n’auront plus faim, ils n’auront plus soif,
        \\ni le soleil ni la chaleur ne les accablera,
        ${}^{17}puisque l’Agneau qui se tient au milieu du Trône
        \\sera leur pasteur
        \\pour les conduire aux sources des eaux de la vie.
        \\Et Dieu essuiera toute larme de leurs yeux. »
      
         
      \bchapter{}
      \begin{verse}
${}^{1}Quand il ouvrit le septième sceau, il y eut dans le ciel un silence d’environ une demi-heure. 
${}^{2}Et j’ai vu les sept anges qui se tiennent devant Dieu : il leur fut donné sept trompettes. 
${}^{3}Un autre ange vint se placer près de l’autel ; il portait un encensoir d’or ; il lui fut donné quantité de parfums pour les offrir, avec les prières de tous les saints, sur l’autel d’or qui est devant le Trône. 
${}^{4}Et par la main de l’ange monta devant Dieu la fumée des parfums, avec les prières des saints. 
${}^{5}Puis l’ange prit l’encensoir et le remplit du feu de l’autel ; il le jeta sur la terre : il y eut des coups de tonnerre, des fracas, des éclairs et un tremblement de terre.
      
         
      <h2 class="intertitle" id="d85e405568">2. Sonneries des sept trompettes (8,6 – 11)</h2>
${}^{6}Puis les sept anges qui avaient les sept trompettes se préparèrent à en sonner.
${}^{7}Le premier sonna de la trompette : il y eut de la grêle et du feu mêlés de sang, qui furent jetés sur la terre, et le tiers de la terre brûla, le tiers des arbres brûlèrent, toute l’herbe verte brûla.
${}^{8}Le deuxième ange sonna de la trompette : dans la mer fut jetée comme une grande montagne embrasée, et le tiers de la mer fut changé en sang ; 
${}^{9}dans la mer, le tiers des créatures vivantes mourut, et le tiers des bateaux fut détruit.
${}^{10}Le troisième ange sonna de la trompette : du ciel tomba une grande étoile qui flambait comme une torche ; elle tomba sur le tiers des fleuves et sur les sources des eaux. 
${}^{11}L’étoile se nomme « Absinthe », et le tiers des eaux devint de l’absinthe : beaucoup de gens moururent à cause des eaux devenues amères.
${}^{12}Le quatrième ange sonna de la trompette : le tiers du soleil fut frappé, et le tiers de la lune et le tiers des étoiles ; ainsi chacun d’entre eux fut obscurci d’un tiers, le jour perdit le tiers de sa clarté et, de même, la nuit.
${}^{13}Alors j’ai vu : et j’entendis un aigle qui volait en plein ciel, disant d’une voix forte : « Malheur ! Malheur ! Malheur pour ceux qui habitent la terre, car la trompette, encore, doit retentir quand les trois anges sonneront ! »
      
         
      \bchapter{}
      \begin{verse}
${}^{1}Le cinquième ange sonna de la trompette, et j’ai vu une étoile qui était tombée du ciel sur la terre : c’est à elle que fut donnée la clé du puits de l’abîme. 
${}^{2}Elle ouvrit le puits de l’abîme, et du puits monta une fumée comme celle d’une grande fournaise ; le soleil et l’air furent obscurcis par la fumée du puits. 
${}^{3}Et de la fumée sortirent vers la terre des sauterelles ; un pouvoir leur fut donné, pareil au pouvoir des scorpions de la terre. 
${}^{4}Il leur fut dit de ne pas faire de mal à l’herbe de la terre, ni à la verdure, ni à aucun arbre, mais seulement aux hommes, ceux qui n’ont pas sur le front la marque du sceau de Dieu. 
${}^{5}Il leur fut donné, non pas de les tuer, mais de les tourmenter pendant cinq mois d’un tourment comme celui qu’inflige le scorpion quand il pique un homme. 
${}^{6}En ces jours-là, les hommes chercheront la mort et ne la trouveront pas ; ils désireront mourir et la mort les fuira.
${}^{7}Ces sortes de sauterelles ressemblent à des chevaux équipés pour la guerre ; elles ont comme des couronnes d’or sur la tête, et un visage comme un visage humain. 
${}^{8}Elles ont des cheveux comme des cheveux de femmes, leurs dents sont comme celles des lions. 
${}^{9}Elles ont des poitrails comme des cuirasses de fer, et le bruit de leurs ailes est comme celui de chars à plusieurs chevaux courant au combat. 
${}^{10}Elles ont des queues comme des scorpions, et des dards venimeux. Dans leur queue se trouve le pouvoir de faire du mal aux hommes pendant cinq mois. 
${}^{11}Elles ont comme roi l’ange de l’abîme ; il se nomme en hébreu Abaddôn et en grec Apollyôn (c’est-à-dire : Destructeur).
${}^{12}Le premier « Malheur ! » est passé ; voici que deux « Malheur ! » vont encore arriver.
${}^{13}Le sixième ange sonna de la trompette, et j’entendis une voix venant des quatre cornes de l’autel d’or qui est devant Dieu ; 
${}^{14}elle disait au sixième ange qui avait la trompette : « Libère les quatre anges qui sont enchaînés au bord de l’Euphrate, le grand fleuve. » 
${}^{15}Alors furent libérés les quatre anges qui étaient prêts pour cette heure, ce jour, ce mois, cette année, afin de tuer le tiers de l’humanité. 
${}^{16}Les troupes de cavalerie comptaient deux myriades de myriades : j’ai entendu ce nombre.
${}^{17}Ainsi, dans ma vision, j’ai vu les chevaux et ceux qui les montaient : ils ont des cuirasses couleur de feu, d’hyacinthe et de soufre ; les têtes des chevaux sont comme des têtes de lion ; de leurs bouches sortent du feu, de la fumée et du soufre. 
${}^{18}Le tiers de l’humanité fut tué par ces trois fléaux, le feu, la fumée et le soufre qui sortaient de leurs bouches. 
${}^{19}Car le pouvoir des chevaux se trouve dans leurs bouches, et aussi dans leurs queues. En effet, celles-ci sont semblables à des serpents, et elles ont des têtes qui font du mal.
${}^{20}Et le reste des hommes, ceux qui n’avaient pas été tués par ces fléaux, ne se sont pas convertis, ne renonçant pas aux œuvres de leurs mains ; ils n’ont pas cessé de se prosterner devant les démons, les idoles d’or, d’argent, de bronze, de pierre et de bois, qui ne peuvent pas voir, ni entendre, ni marcher. 
${}^{21}Ils ne se sont pas convertis, ne renonçant ni à leurs meurtres, ni à leurs sortilèges, ni à leur débauche, ni à leurs vols.
      
         
      \bchapter{}
      \begin{verse}
${}^{1}Et j’ai vu un autre ange, plein de force, descendre du ciel, ayant une nuée pour manteau, et sur la tête un halo de lumière ; son visage était comme le soleil, et ses jambes comme des colonnes de feu. 
${}^{2}Il tenait à la main un petit livre ouvert. Il posa le pied droit sur la mer, et le gauche sur la terre ; 
${}^{3}il cria d’une voix forte, comme un lion qui rugit. Et quand il cria, les sept tonnerres parlèrent, faisant résonner leur voix. 
${}^{4}Et quand les sept tonnerres eurent parlé, j’allais me mettre à écrire ; mais j’entendis une voix venant du ciel qui disait : « Ce que viennent de dire les sept tonnerres, garde-le scellé, ne l’écris pas ! »
${}^{5}Et l’ange que j’avais vu debout sur la mer et sur la terre leva la main droite vers le ciel ; 
${}^{6}il fit un serment par celui qui est vivant pour les siècles des siècles, celui qui a créé le ciel et tout ce qu’il contient, la terre et tout ce qu’elle contient, la mer et tout ce qu’elle contient. Il déclara : « Du temps, il n’y en aura plus ! 
${}^{7}Dans les jours où retentira la voix du septième ange, quand il sonnera de la trompette, alors se trouvera accompli le mystère de Dieu, selon la bonne nouvelle qu’il a annoncée à ses serviteurs les prophètes. »
${}^{8}Et la voix que j’avais entendue, venant du ciel, me parla de nouveau et me dit : « Va prendre le livre ouvert dans la main de l’ange qui se tient debout sur la mer et sur la terre. » 
${}^{9}Je m’avançai vers l’ange pour lui demander de me donner le petit livre. Il me dit : « Prends, et dévore-le ; il remplira tes entrailles d’amertume, mais dans ta bouche il sera doux comme le miel. »
${}^{10}Je pris le petit livre de la main de l’ange, et je le dévorai. Dans ma bouche il était doux comme le miel, mais, quand je l’eus mangé, il remplit mes entrailles d’amertume. 
${}^{11}Alors on me dit : « Il te faut de nouveau prophétiser sur un grand nombre de peuples, de nations, de langues et de rois. »
      <p class="cantique"><span class="cantique_label"><a href="#bib_ct-nt_10">Cantique NT 10</a></span> = <span class="cantique_ref"><a class="unitex_link" href="#bib_ap_11_17">Ap 11,17-18</a> ; <a class="unitex_link" href="#bib_ap_12_10">12,10b-12b</a></span>
      
         
      \bchapter{}
      \begin{verse}
${}^{1}Puis il me fut donné un roseau, une sorte de règle, avec cette parole : « Lève-toi, mesure le sanctuaire de Dieu, l’autel et ceux qui s’y prosternent. 
${}^{2}Mais la cour au-dehors du Sanctuaire, tiens-la en dehors, ne la mesure pas, car elle a été donnée aux nations : elles fouleront aux pieds la Ville sainte pendant quarante-deux mois.
${}^{3}Et je donnerai à mes deux témoins de prophétiser, vêtus de toile à sac, pendant mille deux cent soixante jours. » 
${}^{4}Ce sont eux les deux oliviers, les deux chandeliers, qui se tiennent devant le Seigneur de la terre. 
${}^{5}Si quelqu’un veut leur faire du mal, un feu jaillit de leur bouche et dévore leurs ennemis ; oui, celui qui voudra leur faire du mal, c’est ainsi qu’il doit mourir. 
${}^{6}Ces deux témoins ont le pouvoir de fermer le ciel, pour que la pluie ne tombe pas pendant les jours de leur prophétie. Ils ont aussi le pouvoir de changer l’eau en sang et de frapper la terre de toutes sortes de fléaux, aussi souvent qu’ils le voudront.
${}^{7}Mais, quand ils auront achevé leur témoignage, la Bête qui monte de l’abîme leur fera la guerre, les vaincra et les fera mourir. 
${}^{8}Leurs cadavres restent sur la place de la grande ville, qu’on appelle, au sens figuré, Sodome et l’Égypte, là où leur Seigneur aussi a été crucifié. 
${}^{9}De tous les peuples, tribus, langues et nations, on vient regarder leurs cadavres pendant trois jours et demi, sans qu’il soit permis de les mettre au tombeau. 
${}^{10}Les habitants de la terre s’en réjouissent, ils sont dans la joie, ils échangent des présents ; ces deux prophètes, en effet, avaient causé bien du tourment aux habitants de la terre.
${}^{11}Mais, après ces trois jours et demi, un souffle de vie venu de Dieu entra en eux : ils se dressèrent sur leurs pieds, et une grande crainte tomba sur ceux qui les regardaient. 
${}^{12}Alors les deux témoins entendirent une voix forte venant du ciel, qui leur disait : « Montez jusqu’ici ! » Et ils montèrent au ciel dans la nuée, sous le regard de leurs ennemis. 
${}^{13}Et à cette heure-là, il y eut un grand tremblement de terre : le dixième de la ville s’écroula, et dans le tremblement de terre sept mille personnes furent tuées. Les survivants furent saisis de crainte et rendirent gloire au Dieu du ciel.
${}^{14}Le deuxième « Malheur ! » est passé ; voici que le troisième « Malheur ! » vient sans tarder.
${}^{15}Le septième ange sonna de la trompette. Il y eut dans le ciel des voix fortes qui disaient :
        \\« Il est advenu sur le monde,
        \\le règne de notre Seigneur et de son Christ.
        \\C’est un règne pour les siècles des siècles. »
${}^{16}Et les vingt-quatre Anciens qui siègent sur leurs trônes en présence de Dieu, se jetant face contre terre, se prosternèrent devant Dieu 
${}^{17}en disant :
        \\« À toi, nous rendons grâce,
        \\Seigneur Dieu, Souverain de l’univers,
        \\toi qui es, toi qui étais !
        \\Tu as saisi ta grande puissance
        \\et pris possession de ton règne.
        ${}^{18}Les nations s’étaient mises en colère ;
        \\alors, ta colère est venue
        \\et le temps du jugement pour les morts,
        \\le temps de récompenser tes serviteurs,
        \\les prophètes et les saints,
        \\ceux qui craignent ton nom,
        \\les petits et les grands,
        \\le temps de détruire
        ceux qui détruisent la terre. »
${}^{19}Le sanctuaire de Dieu, qui est dans le ciel, s’ouvrit, et l’arche de son Alliance apparut dans le Sanctuaire ; et il y eut des éclairs, des fracas, des coups de tonnerre, un tremblement de terre et une forte grêle.
      <h2 class="intertitle" id="d85e406091">1. Combats dans le ciel et sur la terre (12 – 14,5)</h2>
      
         
      \bchapter{}
      \begin{verse}
${}^{1}Un grand signe apparut dans le ciel : une Femme, ayant le soleil pour manteau, la lune sous les pieds, et sur la tête une couronne de douze étoiles. 
${}^{2}Elle est enceinte, elle crie, dans les douleurs et la torture d’un enfantement. 
${}^{3}Un autre signe apparut dans le ciel : un grand dragon, rouge feu, avec sept têtes et dix cornes, et, sur chacune des sept têtes, un diadème. 
${}^{4}Sa queue, entraînant le tiers des étoiles du ciel, les précipita sur la terre. Le Dragon vint se poster devant la femme qui allait enfanter, afin de dévorer l’enfant dès sa naissance. 
${}^{5}Or, elle mit au monde un fils, un enfant mâle, celui qui sera le berger de toutes les nations, les conduisant avec un sceptre de fer. L’enfant fut enlevé jusqu’auprès de Dieu et de son Trône, 
${}^{6}et la Femme s’enfuit au désert, où Dieu lui a préparé une place, pour qu’elle y soit nourrie pendant mille deux cent soixante jours.
${}^{7}Il y eut alors un combat dans le ciel : Michel, avec ses anges, dut combattre le Dragon. Le Dragon, lui aussi, combattait avec ses anges, 
${}^{8}mais il ne fut pas le plus fort ; pour eux désormais, nulle place dans le ciel. 
${}^{9}Oui, il fut rejeté, le grand Dragon, le Serpent des origines, celui qu’on nomme Diable et Satan, le séducteur du monde entier. Il fut jeté sur la terre, et ses anges furent jetés avec lui. 
${}^{10}Alors j’entendis dans le ciel une voix forte, qui proclamait :
        \\« Maintenant voici le salut,
        \\la puissance et le règne de notre Dieu,
        \\voici le pouvoir de son Christ !
        \\Car il est rejeté, l’accusateur de nos frères,
        \\lui qui les accusait, jour et nuit,
        \\devant notre Dieu.
        ${}^{11}Eux-mêmes l’ont vaincu par le sang de l’Agneau,
        \\par la parole dont ils furent les témoins ;
        \\détachés de leur propre vie,
        \\ils sont allés jusqu’à mourir.
        ${}^{12}Cieux, soyez donc dans la joie,
        \\et vous qui avez aux cieux votre demeure !
        \\Malheur à la terre et à la mer :
        \\le diable est descendu vers vous,
        \\plein d’une grande fureur ;
        \\il sait qu’il lui reste peu de temps. »
      <div class="box_other filet_bleu">
          <h3 class="intertitle cantique_chap" id="bib_ct-nt_10">Cantique NT 10</h3><a class="cantique_chap" href="#bib_ap_11">11</a>
            <a class="cantique_verset" href="#bib_ap_12_17"><span class="cantique_verset_in">17</span></a>« À toi, nous rendons grâce,
            \\Seigneur Dieu, Souverain de l’univers,
            \\toi qui es, toi qui étais !
             
            \\Tu as saisi ta grande puissance
            \\et pris possession de ton règne.
             
            <a class="cantique_verset" href="#bib_ap_11_18"><span class="cantique_verset_in">18</span></a>Les nations s’étaient mises en colère ;
            \\alors, ta colère est venue
            \\et le temps du jugement pour les morts,
             
            \\le temps de récompenser tes serviteurs,
            \\les prophètes et les saints,
            \\ceux qui craignent ton nom,
            \\les petits et les grands.
           <a class="cantique_chap" href="#bib_ap_12">12</a>
            <a class="cantique_verset" href="#bib_ap_12_10"><span class="cantique_verset_in">10 b</span></a>Maintenant voici le salut,
            \\la puissance et le règne de notre Dieu,
            \\voici le pouvoir de son Christ !
             
            \\Car il est rejeté, l’accusateur de nos frères,
            \\lui qui les accusait, jour et nuit,
            \\devant notre Dieu.
             
            <a class="cantique_verset" href="#bib_ap_12_11"><span class="cantique_verset_in">11</span></a>Eux-mêmes l’ont vaincu par le sang de l’Agneau,
            \\par la parole dont ils furent les témoins ;
            \\détachés de leur propre vie,
            \\ils sont allés jusqu’à mourir.
             
            <a class="cantique_verset" href="#bib_ap_12_12"><span class="cantique_verset_in">12</span></a>Cieux, soyez donc dans la joie,
            \\et vous qui avez aux cieux votre demeure !
${}^{13}Et quand le Dragon vit qu’il était jeté sur la terre, il se mit à poursuivre la Femme qui avait mis au monde l’enfant mâle. 
${}^{14}Alors furent données à la Femme les deux ailes du grand aigle pour qu’elle s’envole au désert, à la place où elle doit être nourrie pour un temps, deux temps et la moitié d’un temps, loin de la présence du Serpent. 
${}^{15}Puis, de sa gueule, le Serpent projeta derrière la Femme de l’eau comme un fleuve, pour qu’elle soit emportée par ce fleuve. 
${}^{16}Mais la terre vint au secours de la Femme : la terre ouvrit la bouche et engloutit le fleuve projeté par la gueule du Dragon. 
${}^{17}Alors le Dragon se mit en colère contre la Femme, il partit faire la guerre au reste de sa descendance, ceux qui observent les commandements de Dieu et gardent le témoignage de Jésus.
${}^{18}Et il se posta sur le sable au bord de la mer.
      
         
      \bchapter{}
      \begin{verse}
${}^{1}Alors, j’ai vu monter de la mer une Bête ayant dix cornes et sept têtes, avec un diadème sur chacune des dix cornes et, sur les têtes, des noms blasphématoires. 
${}^{2}Et la Bête que j’ai vue ressemblait à une panthère ; ses pattes étaient comme celles d’un ours, et sa gueule, comme celle d’un lion. Le Dragon lui donna sa puissance et son trône, et un grand pouvoir. 
${}^{3}L’une de ses têtes était comme blessée à mort, mais sa plaie mortelle fut guérie.
      Émerveillée, la terre entière suivit la Bête, 
${}^{4}et l’on se prosterna devant le Dragon parce qu’il avait donné le pouvoir à la Bête. Et, devant elle, on se prosterna aussi, en disant : « Qui est comparable à la Bête, et qui peut lui faire la guerre ? »
${}^{5}Il lui fut donné une bouche qui disait des énormités, des blasphèmes, et il lui fut donné pouvoir d’agir pendant quarante-deux mois. 
${}^{6}Elle ouvrit la bouche pour proférer des blasphèmes contre Dieu, pour blasphémer contre son nom et sa demeure, contre ceux qui demeurent au ciel. 
${}^{7}Il lui fut donné de faire la guerre aux saints et de les vaincre, il lui fut donné pouvoir sur toute tribu, peuple, langue et nation. 
${}^{8}Ils se prosterneront devant elle, tous ceux qui habitent sur la terre, et dont le nom n’est pas inscrit dans le livre de vie de l’Agneau immolé, depuis la fondation du monde.
${}^{9}Si quelqu’un a des oreilles, qu’il entende.
${}^{10}Si quelqu’un doit aller en captivité,
        \\il ira en captivité ;
        \\si quelqu’un doit être tué par l’épée,
        \\il sera tué par l’épée.
      C’est ici qu’on reconnaît la persévérance et la foi des saints.
${}^{11}Puis, j’ai vu monter de la terre une autre Bête ; elle avait deux cornes comme un agneau, et elle parlait comme un dragon. 
${}^{12}Elle exerce tout le pouvoir de la première Bête en sa présence, amenant la terre et tous ceux qui l’habitent à se prosterner devant la première Bête, dont la plaie mortelle a été guérie. 
${}^{13}Elle produit de grands signes, jusqu’à faire descendre le feu du ciel sur la terre aux yeux des hommes : 
${}^{14}elle égare les habitants de la terre par les signes qu’il lui a été donné de produire en présence de la Bête ; elle dit aux habitants de la terre de dresser une image en l’honneur de la première Bête qui porte une plaie faite par l’épée mais qui a repris vie.
${}^{15}Il lui a été donné d’animer l’image de la Bête, au point que cette image se mette à parler, et fasse tuer tous ceux qui ne se prosternent pas devant elle. 
${}^{16}À tous, petits et grands, riches et pauvres, hommes libres et esclaves, elle fait mettre une marque sur la main droite ou sur le front, 
${}^{17}afin que personne ne puisse acheter ou vendre, s’il ne porte cette marque-là : le nom de la Bête ou le chiffre de son nom.
${}^{18}C’est ici qu’on reconnaît la sagesse. Celui qui a l’intelligence, qu’il se mette à calculer le chiffre de la Bête, car c’est un chiffre d’homme, et ce chiffre est six cent soixante-six.
      
         
      \bchapter{}
      \begin{verse}
${}^{1}Alors j’ai vu : et voici que l’Agneau se tenait debout sur la montagne de Sion, et avec lui les cent quarante-quatre mille qui portent, inscrits sur leur front, le nom de l’Agneau et celui de son Père. 
${}^{2}Et j’ai entendu une voix venant du ciel comme la voix des grandes eaux ou celle d’un fort coup de tonnerre ; mais cette voix que j’entendais était aussi comme celle des joueurs de cithare qui chantent et s’accompagnent sur leur cithare. 
${}^{3}Ils chantent un cantique nouveau devant le Trône, et devant les quatre Vivants et les Anciens. Personne ne pouvait apprendre ce cantique sinon les cent quarante-quatre mille, ceux qui ont été rachetés et retirés de la terre.
        ${}^{4}Ceux-là ne se sont pas souillés avec des femmes ;
        \\ils sont vierges, en effet.
        \\Ceux-là suivent l’Agneau partout où il va ;
        \\ils ont été pris d’entre les hommes,
        \\achetés comme prémices pour Dieu et pour l’Agneau.
        ${}^{5}Dans leur bouche, on n’a pas trouvé de mensonge ;
        \\ils sont sans tache.
      <h2 class="intertitle" id="d85e406751">2. Les sept coupes de la colère de Dieu (14,6 – 16)</h2>
${}^{6}Puis j’ai vu un autre ange volant en plein ciel ; il avait un évangile éternel à proclamer, bonne nouvelle pour ceux qui résident sur la terre, pour toute nation, tribu, langue et peuple. 
${}^{7}Il disait d’une voix forte :
        \\« Craignez Dieu et rendez-lui gloire,
        \\car elle est venue, l’heure où il doit juger ;
        \\prosternez-vous devant celui qui a fait le ciel,
        \\la terre, la mer, et les sources des eaux. »
       
${}^{8}Un autre ange, le deuxième, vint à sa suite. Il disait :
        \\« Elle est tombée, elle est tombée, Babylone la Grande,
        \\elle qui abreuvait toutes les nations
        \\du vin de la fureur de sa prostitution. »
       
${}^{9}Un autre ange, le troisième, vint à leur suite. Il disait d’une voix forte :
        \\« Si quelqu’un se prosterne devant la Bête et son image,
        \\s’il en reçoit la marque sur le front ou sur la main,
${}^{10}lui aussi boira du vin de la fureur de Dieu,
        \\versé sans mélange dans la coupe de sa colère ;
        \\il sera torturé par le feu et le soufre
        \\devant les anges saints et devant l’Agneau.
${}^{11}Et la fumée de ces tortures
        \\monte pour les siècles des siècles.
        \\Ils n’ont de repos ni le jour ni la nuit,
        \\ceux qui se prosternent devant la Bête et son image,
        \\et quiconque reçoit la marque de son nom. »
${}^{12}C’est ici qu’on reconnaît la persévérance des saints, ceux-là qui gardent les commandements de Dieu et la foi de Jésus.
${}^{13}Alors j’ai entendu une voix qui venait du ciel. Elle disait :
        \\« Écris :
        \\Heureux, dès à présent,
        \\les morts qui meurent dans le Seigneur.
        \\Oui, dit l’Esprit,
        \\qu’ils se reposent de leurs peines,
        \\car leurs actes les suivent ! »
${}^{14}Alors j’ai vu : et voici une nuée blanche, et sur cette nuée, quelqu’un siégeait, qui semblait un Fils d’homme. Il avait sur la tête une couronne d’or et, à la main, une faucille aiguisée. 
${}^{15}Un autre ange sortit du Sanctuaire. Il cria d’une voix forte à celui qui siégeait sur la nuée :
        \\« Lance ta faucille et moissonne :
        \\elle est venue, l’heure de la moisson,
        \\car la moisson de la terre se dessèche. »
${}^{16}Alors, celui qui siégeait sur la nuée jeta la faucille sur la terre, et la terre fut moissonnée.
${}^{17}Puis un autre ange sortit du Sanctuaire qui est dans le ciel ; il avait, lui aussi, une faucille aiguisée. 
${}^{18}Un autre ange encore sortit, venant de l’autel ; il avait pouvoir sur le feu. Il interpella d’une voix forte celui qui avait la faucille aiguisée : « Lance ta faucille aiguisée, et vendange les grappes de la vigne sur la terre, car les raisins sont mûrs. » 
${}^{19}L’ange, alors, jeta la faucille sur la terre, il vendangea la vigne de la terre et jeta la vendange dans la cuve immense de la fureur de Dieu. 
${}^{20}On se mit à fouler hors de la ville, et de la cuve sortit du sang, jusqu’à hauteur du mors des chevaux, sur une distance de mille six cents stades.
      <p class="cantique" id="bib_ct-nt_11"><span class="cantique_label">Cantique NT 11</span> = <span class="cantique_ref"><a class="unitex_link" href="#bib_ap_15_3">Ap 15, 3-4</a></span>
      
         
      \bchapter{}
      \begin{verse}
${}^{1}Alors j’ai vu dans le ciel un autre signe, grand et merveilleux : sept anges qui détiennent sept fléaux ; ce sont les derniers, puisque s’achève avec eux la fureur de Dieu.
${}^{2}J’ai vu comme une mer de cristal, mêlée de feu, et ceux qui sont victorieux de la Bête, de son image, et du chiffre qui correspond à son nom : ils se tiennent debout sur cette mer de cristal, ils ont en main les cithares de Dieu. 
${}^{3}Ils chantent le cantique de Moïse, serviteur de Dieu, et le cantique de l’Agneau. Ils disent :
       
        ${}^{(3)}« Grandes, merveilleuses, tes œuvres\\,
        \\Seigneur Dieu, Souverain de l’univers\\ !
         
        \\Ils sont justes, ils sont vrais, tes chemins  ,
        \\Roi des nations  .
         
        ${}^{4}Qui ne te craindrait, Seigneur ?
        \\À ton nom, qui ne rendrait gloire   ?
         
        \\Oui, toi seul es saint !
        \\Oui, toutes les nations viendront
        et se prosterneront devant toi\\ ;
        \\oui, ils sont manifestés, tes jugements. »
${}^{5}Et après cela, j’ai vu : le Sanctuaire où se trouve la Demeure du Témoignage s’ouvrit dans le ciel, 
${}^{6}et les sept anges aux sept fléaux sortirent du Sanctuaire, habillés de lin pur et resplendissant ; ils portaient des ceintures d’or autour de la poitrine. 
${}^{7}L’un des quatre Vivants donna aux sept anges sept coupes d’or, remplies de la fureur de Dieu, lui qui est vivant pour les siècles des siècles. 
${}^{8}Et le Sanctuaire fut rempli de fumée par la gloire de Dieu et sa puissance, et personne ne pouvait entrer dans le Sanctuaire jusqu’à ce que s’achèvent les sept fléaux des sept anges.
      
         
      \bchapter{}
      \begin{verse}
${}^{1}Alors j’entendis une voix forte venant du Sanctuaire, qui disait aux sept anges : « Allez répandre sur la terre les sept coupes de la fureur de Dieu. »
${}^{2}Le premier partit et répandit sa coupe sur la terre : il y eut un ulcère malin et pernicieux sur les hommes qui portaient la marque de la Bête, et sur ceux qui se prosternaient devant son image.
${}^{3}Le deuxième répandit sa coupe sur la mer : il y eut du sang comme d’un mort, et toute vie dans la mer mourut.
${}^{4}Le troisième répandit sa coupe sur les fleuves et les sources des eaux : et il y eut du sang. 
${}^{5}Alors j’entendis l’ange des eaux qui disait :
        \\« Tu es juste, toi qui es, et qui étais, toi le Saint,
        \\parce que tu as rendu ce jugement.
${}^{6}Ils ont répandu le sang des saints et des prophètes ;
        \\tu leur as donné du sang à boire :
        \\c’est ce qu’ils méritent ! »
${}^{7}Puis j’ai entendu l’autel qui disait :
        \\« Oui, Seigneur Dieu, Souverain de l’univers,
        \\ils sont vrais, ils sont justes, tes jugements. »
${}^{8}Le quatrième ange répandit sa coupe sur le soleil : il lui fut donné de brûler les hommes de son feu. 
${}^{9}Les hommes furent brûlés d’une grande brûlure ; ils blasphémèrent le nom du Dieu qui a de tels fléaux en son pouvoir, au lieu de se convertir en lui rendant gloire.
${}^{10}Le cinquième répandit sa coupe sur le trône de la Bête : il y eut de l’obscurité sur son royaume. Les gens se mordaient la langue de douleur 
${}^{11}et ils blasphémèrent le Dieu du ciel sous le coup de leurs douleurs et de leurs ulcères, au lieu de se repentir de leurs agissements.
${}^{12}Le sixième répandit sa coupe sur le grand fleuve, l’Euphrate : et l’eau en fut tarie pour préparer la route des rois venant du côté où le soleil se lève.
${}^{13}Puis j’ai vu sortir de la gueule du Dragon, de celle de la Bête et de celle du faux prophète, trois esprits impurs, pareils à des grenouilles. 
${}^{14}Ce sont, en effet, des esprits démoniaques qui produisent des signes, et s’en vont vers les rois du monde entier afin de les rassembler pour la bataille du grand jour de Dieu, le Souverain de l’univers. 
${}^{15}– Voici que je viens comme un voleur. Heureux celui qui veille et garde sur lui ses vêtements pour ne pas aller nu en laissant voir sa honte. 
${}^{16}Et ils les rassemblèrent en un lieu appelé en hébreu Harmaguédone.
${}^{17}Le septième ange répandit sa coupe dans les airs : une voix forte venant du trône sortit du Sanctuaire ; elle disait : « C’en est fait ! » 
${}^{18}Il y eut des éclairs, des fracas, des coups de tonnerre ; il y eut un grand tremblement de terre : depuis que sur la terre il y a des hommes, il n'y eut jamais de tremblement de terre aussi grand. 
${}^{19}Et la grande ville se disloqua en trois parties, et les villes des nations tombèrent. Et Dieu se souvint de Babylone la Grande, pour lui donner à boire le vin de sa fureur, la coupe de sa colère. 
${}^{20}Toutes les îles s’enfuirent, et les montagnes disparurent. 
${}^{21}Des grêlons d’une masse énorme tombèrent du ciel sur les hommes, qui blasphémèrent Dieu à cause du fléau de la grêle, car c’était un terrible fléau.
      <h2 class="intertitle" id="d85e407432">3. Le jugement de la grande ville (17 – 19,10)</h2>
      
         
      \bchapter{}
      \begin{verse}
${}^{1}L’un des sept anges aux sept coupes vint me parler : « Viens, dit-il, je vais te montrer ce que sera la condamnation de la grande prostituée assise au bord des grandes eaux. 
${}^{2}Les rois de la terre se sont prostitués avec elle, et ceux qui habitent la terre se sont enivrés du vin de sa prostitution. » 
${}^{3}Il me transporta en esprit au désert.
      Et j’ai vu une femme assise sur une bête écarlate qui était couverte de noms blasphématoires et qui avait sept têtes et dix cornes. 
${}^{4}Cette femme était vêtue de pourpre et d’écarlate, toute parée d’or, de pierres précieuses et de perles ; elle avait dans la main une coupe d’or remplie d’abominations, avec les impuretés de sa prostitution. 
${}^{5}Il y avait sur son front un nom écrit, un mystère : « Babylone la Grande, la mère des prostitutions et des abominations de la terre. » 
${}^{6}Et j’ai vu la femme ivre du sang des saints et du sang des témoins de Jésus. En la voyant, je fus saisi d’un grand étonnement.
${}^{7}Et l’ange me dit : « Pourquoi es-tu étonné ? Moi, je te dirai le mystère de la femme et de la Bête qui la porte, celle qui a les sept têtes et les dix cornes.
${}^{8}La Bête que tu as vue, elle était, mais elle n’est plus ; elle va monter de l’abîme pour aller à sa perdition. Quant aux habitants de la terre dont le nom n’est pas inscrit dans le livre de la vie depuis la fondation du monde, ils seront étonnés au spectacle de la Bête qui était, qui n’est plus et qui va reparaître. 
${}^{9}Ici, il faut l’intelligence mais avec la sagesse. Les sept têtes sont sept collines sur lesquelles réside la femme ; elles sont aussi sept rois : 
${}^{10}cinq sont tombés, un est là maintenant, et l’autre n’est pas encore venu, mais quand il viendra, il ne devra rester que peu de temps. 
${}^{11}Et la Bête qui était et qui n’est plus, est elle-même un huitième roi, mais elle fait partie des sept ; elle va à sa perdition. 
${}^{12}Les dix cornes que tu as vues sont dix rois qui n’ont pas encore reçu la royauté, mais reçoivent le pouvoir royal avec la Bête pour une heure. 
${}^{13}Ceux-ci ont un même projet : donner leur puissance et leur pouvoir à la Bête. 
${}^{14}Ils feront la guerre à l’Agneau, et l’Agneau les vaincra car il est Seigneur des seigneurs et Roi des rois ; et les siens, les appelés, les élus, les fidèles, vaincront avec lui. »
${}^{15}Puis il me dit : « Les eaux que tu as vues, là où la prostituée est assise, ce sont des peuples et des foules, des nations et des langues. 
${}^{16}Quant aux dix cornes que tu as vues, ainsi que la Bête, elles se prendront de haine pour la prostituée, elles la laisseront dépouillée et nue, elles mangeront ses chairs et la brûleront au feu. 
${}^{17}Car Dieu leur a mis au cœur de réaliser son projet, de réaliser ensemble un même projet : donner à la Bête leur royauté jusqu’à ce que s’accomplissent les paroles de Dieu. 
${}^{18}La femme que tu as vue, c’est la grande ville qui exerce la royauté sur les rois de la terre. »
      
         
      \bchapter{}
      \begin{verse}
${}^{1}Après cela, j’ai vu descendre du ciel un autre ange, ayant un grand pouvoir, et la terre fut illuminée de sa gloire. 
${}^{2}Il s’écria d’une voix puissante :
        \\« Elle est tombée, elle est tombée,
        \\Babylone la Grande !
        \\La voilà devenue tanière de démons,
        \\repaire de tous les esprits impurs,
        \\repaire de tous les oiseaux impurs,
        \\repaire de toutes les bêtes impures et répugnantes !
${}^{3}Car toutes les nations ont bu du vin de sa fureur,
        \\de sa prostitution ;
        \\les rois de la terre se sont prostitués avec elle,
        \\et les marchands de la terre se sont enrichis
        \\de son luxe insolent. »
       
${}^{4}Et j’entendis une autre voix venant du ciel qui disait :
        \\« Sortez de la ville, vous mon peuple,
        \\pour ne pas prendre part à ses péchés
        \\et ne rien subir des fléaux qui l’affligent.
${}^{5}Car ses péchés se sont amoncelés jusqu’au ciel,
        \\et, de ses injustices, Dieu s’est souvenu.
${}^{6}Traitez-la comme elle vous a traités,
        \\rendez-lui au double selon ses actes ;
        \\dans la coupe qu’elle a préparée,
        \\préparez-lui le double.
${}^{7}À la mesure de la gloire et du luxe qu’elle a étalés,
        \\donnez-lui torture et deuil.
        \\Car elle dit dans son cœur :
        \\“Je trône, je suis reine,
        \\je ne suis pas veuve,
        \\je ne verrai jamais le deuil.”
${}^{8}C’est pourquoi
        \\des fléaux, en un seul jour, viendront sur elle :
        \\mort, deuil, famine,
        \\et elle sera brûlée au feu,
        \\car il est fort, le Seigneur Dieu qui l’a jugée. »
${}^{9}Alors, ils pleureront et se lamenteront sur elle, les rois de la terre qui se sont prostitués avec elle et qui ont partagé son luxe, quand ils verront la fumée de son incendie. 
${}^{10}Ils se tiendront à distance par peur de ses tortures, et ils diront :
        \\« Malheur ! Malheur !
        \\la grande ville,
        \\Babylone, ville puissante :
        \\en une heure, ton jugement est arrivé ! »
${}^{11}Et les marchands de la terre pleurent et prennent le deuil à cause d’elle, puisque personne n’achète plus leur cargaison : 
${}^{12}cargaison d’or, d’argent, de pierres précieuses et de perles, de lin fin, de pourpre, de soie et d’écarlate ; toutes sortes de bois odorants, d’objets en ivoire, en bois très précieux, en bronze, en fer et en marbre ; 
${}^{13}cannelle, épices, parfums, baume et encens, vin, huile, fleur de farine et blé, bestiaux et moutons, chevaux et chariots, esclaves et marchandise humaine.
${}^{14} « Les fruits mûrs de tes convoitises
        \\sont partis loin de toi,
        \\tout ce qui était brillance et splendeur est perdu pour toi,
        \\et cela plus jamais ne se retrouvera. »
${}^{15} Les marchands qu’elle avait ainsi enrichis se tiendront à distance par peur de ses tortures, dans les pleurs et le deuil. 
${}^{16}Ils diront :
        \\« Malheur ! Malheur ! La grande ville,
        \\vêtue de lin fin, de pourpre et d’écarlate,
        \\toute parée d’or, de pierres précieuses et de perles,
${}^{17} car, en une heure, tant de richesses furent dévastées ! »
      Tous les capitaines de navires et ceux qui font le cabotage, les marins et tous les travailleurs de la mer se tenaient à distance, 
${}^{18}et ils criaient en voyant la fumée de son incendie. Ils disaient : « Quelle ville fut comparable à la grande ville ? » 
${}^{19}Et jetant de la poussière sur leur tête, ils criaient dans les pleurs et le deuil. Ils disaient :
        \\« Malheur ! Malheur ! La grande ville,
        \\dont l’opulence enrichissait
        \\tous ceux qui avaient des bateaux sur la mer :
        \\en une heure, elle a été dévastée ! »
${}^{20} Ciel, sois dans la joie à cause d’elle,
        \\ainsi que vous, les saints, les apôtres et les prophètes,
        \\car Dieu, en la jugeant, vous a rendu justice.
${}^{21} Alors un ange plein de force prit une pierre pareille à une grande meule, et la précipita dans la mer, en disant :
        \\« Ainsi, d’un coup, sera précipitée
        \\Babylone, la grande ville,
        \\on ne la retrouvera jamais plus.
        ${}^{22} La voix des joueurs de cithares et des musiciens,
        \\des joueurs de flûte et de trompette,
        \\chez toi ne s’entendra jamais plus.
        \\Aucun artisan d’aucun métier
        \\chez toi ne se trouvera jamais plus,
        \\et la voix de la meule
        \\chez toi ne s’entendra jamais plus.
        ${}^{23} La lumière de la lampe
        \\chez toi ne brillera jamais plus.
        \\La voix du jeune époux et de son épouse
        \\chez toi ne s’entendra jamais plus.
        \\Pourtant, tes marchands étaient les magnats de la terre,
        \\et tes sortilèges égaraient toutes les nations !
${}^{24} Mais c’est chez toi qu’on a trouvé le sang
        \\des prophètes et des saints,
        \\et de tous ceux qui furent immolés sur la terre. »
      <p class="cantique"><span class="cantique_label"><a href="#bib_ct-nt_12">Cantique NT 12</a></span> = <span class="cantique_ref"><a class="unitex_link" href="#bib_ap_19_1">Ap 19, 1-2.5-7</a></span>
      
         
      \bchapter{}
      \begin{verse}
${}^{1} Après cela, j’entendis comme la voix forte d’une foule immense dans le ciel, qui proclamait :
        \\« Alléluia !
        \\Le salut, la gloire,
        \\la puissance à notre Dieu.
        ${}^{2}Ils sont vrais, ils sont justes,
        \\ses jugements\\.
        \\Il a jugé la grande prostituée
        \\qui corrompait la terre par sa prostitution ;
        \\il a réclamé justice du sang de ses serviteurs,
        \\qu’elle a versé de sa main. »
${}^{3} Et la foule reprit :
        \\« Alléluia !
        \\La fumée de l’incendie s’élève pour les siècles des siècles. »
${}^{4} Les vingt-quatre Anciens et les quatre Vivants se prosternèrent et adorèrent Dieu qui siège sur le trône ; ils proclamaient :
        \\« Amen ! Alléluia ! »
${}^{5} Et du Trône sortit une voix qui disait :
        \\« Louez notre Dieu,
        vous tous qui le servez,
        \\vous tous\\qui le craignez,
        les petits et les grands. »
${}^{6} Alors j’entendis comme la voix d’une foule immense, comme la voix des grandes eaux, ou celle de violents coups de tonnerre. Elle proclamait :
        \\« Alléluia !
        \\Il règne, le Seigneur notre Dieu,
        \\le Souverain de l’univers.
        ${}^{7}Soyons dans la joie, exultons,
        \\et rendons gloire à Dieu !
        \\Car elles sont venues,
        \\les Noces de l’Agneau,
        \\et pour lui son épouse
        \\a revêtu sa parure\\.
${}^{8} Un vêtement de lin fin lui a été donné,
        \\splendide et pur. »
      Car le lin, ce sont les actions justes des saints.
      <div class="box_other filet_bleu">
          <h3 class="intertitle cantique_chap" id="bib_ct-nt_12">Cantique NT 12</h3>
            \\Alléluia !
            \\Le salut, la gloire,
            \\la puissance à notre Dieu.
            \\(Alléluia !)
            <a class="cantique_verset" href="#bib_ap_19_2"><span class="cantique_verset_in">2</span></a>Ils sont vrais, ils sont justes,
            \\ses jugements\\.
            \\(Alléluia !)
             
            <a class="cantique_verset" href="#bib_ap_19_5"><span class="cantique_verset_in">5</span></a>Louez notre Dieu,
            \\vous tous qui le servez,
            \\(Alléluia !)
            \\vous tous\\qui le craignez,
            \\les petits et les grands.
            \\(Alléluia !)
             
            <a class="cantique_verset" href="#bib_ap_19_6"><span class="cantique_verset_in">6</span></a>Il règne, le Seigneur notre Dieu,
            \\le Souverain de l’univers.
            \\(Alléluia !)
             
            <a class="cantique_verset" href="#bib_ap_19_7"><span class="cantique_verset_in">7</span></a>Soyons dans la joie, exultons,
            \\et rendons gloire à Dieu !
            \\(Alléluia !)
             
            \\Car elles sont venues,
            \\les Noces de l’Agneau.
            \\(Alléluia !)
            \\Et pour lui son épouse
            \\a revêtu sa parure\\.
            \\(Alléluia !)
${}^{9} Puis l’ange me dit : « Écris : Heureux les invités au repas des noces de l’Agneau ! » Il ajouta : « Ce sont les paroles véritables de Dieu. » 
${}^{10}Je me jetai à ses pieds pour me prosterner devant lui. Il me dit : « Non, ne fais pas cela ! Je suis un serviteur comme toi, comme tes frères qui portent le témoignage de Jésus. Prosterne-toi devant Dieu ! Car c’est le témoignage de Jésus qui inspire la prophétie. »
      <h2 class="intertitle" id="d85e408488">4. Défaite des autres ennemis de Dieu (19,11 – 20)</h2>
${}^{11} Puis j’ai vu le ciel ouvert, et voici un cheval blanc : celui qui le monte s’appelle Fidèle et Vrai, il juge et fait la guerre avec justice. 
${}^{12}Ses yeux sont comme une flamme ardente, il a sur la tête plusieurs diadèmes, il porte un nom écrit que nul ne connaît, sauf lui-même. 
${}^{13}Le vêtement qui l’enveloppe est trempé de sang, et on lui donne ce nom : « le Verbe de Dieu ».
${}^{14} Les armées du ciel le suivaient sur des chevaux blancs, elles étaient vêtues de lin fin, d’un blanc pur. 
${}^{15}De sa bouche sort un glaive acéré, pour en frapper les nations ; lui-même les conduira avec un sceptre de fer, lui-même foulera la cuve du vin de la fureur, la colère de Dieu, Souverain de l’univers ; 
${}^{16}sur son vêtement et sur sa cuisse, il porte un nom écrit : « Roi des rois et Seigneur des seigneurs ».
${}^{17} Puis j’ai vu un ange debout dans le soleil ; il cria d’une voix forte à tous les oiseaux qui volent en plein ciel : « Venez, rassemblez-vous pour le grand repas de Dieu, 
${}^{18}pour manger la chair des rois, celle des chefs d’armée, celle des puissants, celle des chevaux et de ceux qui les montent, celle de tous les hommes, libres ou esclaves, des petits et des grands. »
${}^{19} Et j’ai vu la Bête, les rois de la terre, et leurs armées, rassemblés pour faire la guerre au cavalier et à son armée. 
${}^{20}La Bête fut capturée, et avec elle le faux prophète, lui qui, en produisant des signes devant elle, avait égaré ceux qui portent la marque de la Bête et se prosternent devant son image. Ils furent jetés vivants, tous les deux, dans l’étang de feu embrasé de soufre. 
${}^{21}Les autres furent tués par le glaive du cavalier, le glaive qui sort de sa bouche, et tous les oiseaux se rassasièrent de leurs chairs.
      
         
      \bchapter{}
      \begin{verse}
${}^{1} Alors j’ai vu un ange qui descendait du ciel ; il tenait à la main la clé de l’abîme et une énorme chaîne. 
${}^{2}Il s’empara du Dragon, le serpent des origines, qui est le Diable, le Satan, et il l’enchaîna pour une durée de mille ans. 
${}^{3}Il le précipita dans l’abîme, qu’il referma sur lui ; puis il mit les scellés pour que le Dragon n’égare plus les nations, jusqu’à ce que les mille ans arrivent à leur terme. Après cela, il faut qu’il soit relâché pour un peu de temps.
${}^{4} Puis j’ai vu des trônes : à ceux qui vinrent y siéger fut donné le pouvoir de juger. Et j’ai vu les âmes de ceux qui ont été décapités à cause du témoignage pour Jésus, et à cause de la parole de Dieu, eux qui ne se sont pas prosternés devant la Bête et son image, et qui n’ont pas reçu sa marque sur le front ou sur la main. Ils revinrent à la vie, et ils régnèrent avec le Christ pendant mille ans. 
${}^{5}Le reste des morts ne revint pas à la vie tant que les mille ans ne furent pas arrivés à leur terme.
      Telle est la première résurrection. 
${}^{6} Heureux et saints, ceux qui ont part à la première résurrection ! Sur eux, la seconde mort n’a pas de pouvoir : ils seront prêtres de Dieu et du Christ, et régneront avec lui pendant les mille ans.
${}^{7} Et quand les mille ans seront arrivés à leur terme, Satan sera relâché de sa prison, 
${}^{8}il sortira pour égarer les gens des nations qui sont aux quatre coins de la terre, Gog et Magog, afin de les rassembler pour la guerre ; ils sont aussi nombreux que le sable de la mer. 
${}^{9}Ils montèrent, couvrant l’étendue de la terre, ils encerclèrent le camp des saints et la Ville bien-aimée, mais un feu descendit du ciel et les dévora. 
${}^{10}Et le diable qui les égarait fut jeté dans l’étang de feu et de soufre, où sont aussi la Bête et le faux prophète ; ils y seront torturés jour et nuit pour les siècles des siècles.
${}^{11} Puis j’ai vu un grand trône blanc et celui qui siégeait sur ce trône. Devant sa face, le ciel et la terre s’enfuirent : nulle place pour eux ! 
${}^{12}J’ai vu aussi les morts, les grands et les petits, debout devant le Trône. On ouvrit des livres, puis un autre encore : le livre de la vie. D’après ce qui était écrit dans les livres, les morts furent jugés selon leurs actes. 
${}^{13}La mer rendit les morts qu’elle retenait ; la Mort et le séjour des morts rendirent aussi ceux qu’ils retenaient, et ils furent jugés, chacun selon ses actes. 
${}^{14}Puis la Mort et le séjour des morts furent précipités dans l’étang de feu – l’étang de feu, c’est la seconde mort. 
${}^{15}Et si quelqu’un ne se trouvait pas inscrit dans le livre de la vie, il était précipité dans l’étang de feu.
      <h2 class="intertitle" id="d85e408693">5. L’avènement du monde nouveau (21 – 22,5)</h2>
      
         
      \bchapter{}
      \begin{verse}
${}^{1} Alors j’ai vu un ciel nouveau et une terre nouvelle, car le premier ciel et la première terre s’en étaient allés et, de mer, il n’y en a plus. 
${}^{2}Et la Ville sainte, la Jérusalem nouvelle, je l’ai vue qui descendait du ciel, d’auprès de Dieu, prête pour les noces, comme une épouse parée pour son mari. 
${}^{3}Et j’entendis une voix forte qui venait du Trône. Elle disait :
        \\« Voici la demeure de Dieu avec les hommes ;
        \\il demeurera avec eux,
        \\et ils seront ses peuples,
        \\et lui-même, Dieu avec eux, sera leur Dieu.
        ${}^{4}Il essuiera toute larme de leurs yeux,
        \\et la mort ne sera plus,
        \\et il n’y aura plus ni deuil, ni cri, ni douleur :
        \\ce qui était en premier s’en est allé. »
${}^{5} Alors celui qui siégeait sur le Trône déclara : « Voici que je fais toutes choses nouvelles. » Et il dit : « Écris, car ces paroles sont dignes de foi et vraies. » 
${}^{6}Puis il me dit : « C’est fait. Moi, je suis l’alpha et l’oméga, le commencement et la fin. À celui qui a soif, moi, je donnerai l’eau de la source de vie, gratuitement. 
${}^{7}Tel sera l’héritage du vainqueur ; je serai son Dieu, et lui sera mon fils. 
${}^{8}Quant aux lâches, perfides, êtres abominables, meurtriers, débauchés, sorciers, idolâtres et tous les menteurs, la part qui leur revient, c’est l’étang embrasé de feu et de soufre, qui est la seconde mort. »
${}^{9} Alors arriva l’un des sept anges aux sept coupes remplies des sept derniers fléaux, et il me parla ainsi : « Viens, je te montrerai la Femme, l’Épouse de l’Agneau. » 
${}^{10}En esprit, il m’emporta sur une grande et haute montagne ; il me montra la Ville sainte, Jérusalem, qui descendait du ciel, d’auprès de Dieu : 
${}^{11}elle avait en elle la gloire de Dieu ; son éclat était celui d’une pierre très précieuse, comme le jaspe cristallin. 
${}^{12}Elle avait une grande et haute muraille, avec douze portes et, sur ces portes, douze anges ; des noms y étaient inscrits : ceux des douze tribus des fils d’Israël. 
${}^{13}Il y avait trois portes à l’orient, trois au nord, trois au midi, et trois à l’occident. 
${}^{14}La muraille de la ville reposait sur douze fondations portant les douze noms des douze Apôtres de l’Agneau.
${}^{15} Celui qui me parlait tenait un roseau d’or comme mesure, pour mesurer la ville, ses portes, et sa muraille. 
${}^{16}La ville a la forme d’un carré : sa longueur est égale à sa largeur. Il mesura la ville avec le roseau : douze mille stades ; sa longueur, sa largeur et sa hauteur sont égales. 
${}^{17}Puis il mesura sa muraille : cent quarante-quatre coudées, mesure d’homme et mesure d’ange. 
${}^{18}Le matériau de la muraille est de jaspe, et la ville est d’or pur, d’une pureté transparente. 
${}^{19}Les fondations de la muraille de la ville sont ornées de toutes sortes de pierres précieuses. La première fondation est de jaspe, la deuxième de saphir, la troisième de calcédoine, la quatrième d’émeraude, 
${}^{20}la cinquième de sardoine, la sixième de cornaline, la septième de chrysolithe, la huitième de béryl, la neuvième de topaze, la dixième de chrysoprase, la onzième d’hyacinthe, la douzième d’améthyste. 
${}^{21}Les douze portes sont douze perles, chaque porte faite d’une seule perle ; la place de la ville est d’or pur d’une parfaite transparence.
${}^{22} Dans la ville, je n’ai pas vu de sanctuaire, car son sanctuaire, c’est le Seigneur Dieu, Souverain de l’univers, et l’Agneau. 
${}^{23}La ville n’a pas besoin du soleil ni de la lune pour l’éclairer, car la gloire de Dieu l’illumine : son luminaire, c’est l’Agneau. 
${}^{24}Les nations marcheront à sa lumière, et les rois de la terre y porteront leur gloire. 
${}^{25}Jour après jour, jamais les portes ne seront fermées, car il n’y aura plus de nuit. 
${}^{26}On apportera dans la ville la gloire et le faste des nations. 
${}^{27}Rien de souillé n’y entrera jamais, ni personne qui pratique abomination ou mensonge, mais seulement ceux qui sont inscrits dans le livre de vie de l’Agneau.
      
         
      \bchapter{}
      \begin{verse}
${}^{1} Puis l’ange me montra l’eau de la vie : un fleuve resplendissant comme du cristal, qui jaillit du trône de Dieu et de l’Agneau. 
${}^{2}Au milieu de la place de la ville, entre les deux bras du fleuve, il y a un arbre de vie qui donne des fruits douze fois : chaque mois il produit son fruit ; et les feuilles de cet arbre sont un remède pour les nations. 
${}^{3}Toute malédiction aura disparu. Le trône de Dieu et de l’Agneau sera dans la ville, et les serviteurs de Dieu lui rendront un culte ; 
${}^{4}ils verront sa face, et son nom sera sur leur front. 
${}^{5}La nuit aura disparu, ils n’auront plus besoin de la lumière d’une lampe ni de la lumière du soleil, parce que le Seigneur Dieu les illuminera ; ils régneront pour les siècles des siècles.
      
         
${}^{6} Puis l’ange me dit : « Ces paroles sont dignes de foi et vraies : le Seigneur, le Dieu qui inspire les prophètes, a envoyé son ange pour montrer à ses serviteurs ce qui doit bientôt advenir. 
${}^{7}Voici que je viens sans tarder. Heureux celui qui garde les paroles de ce livre de prophétie. »
${}^{8} C’est moi, Jean, qui entendais et voyais ces choses. Et après avoir entendu et vu, je me jetai aux pieds de l’ange qui me montrait cela, pour me prosterner devant lui. 
${}^{9}Il me dit : « Non, ne fais pas cela ! Je suis un serviteur comme toi, comme tes frères les prophètes et ceux qui gardent les paroles de ce livre. Prosterne-toi devant Dieu ! »
${}^{10} Puis il me dit : « Ne mets pas les scellés sur les paroles de ce livre de prophétie. Le temps est proche, en effet. 
${}^{11}Que celui qui fait le mal fasse encore le mal, et que l’homme sali se salisse encore ; que le juste pratique encore la justice, et que le saint se sanctifie encore.
${}^{12} Voici que je viens sans tarder, et j’apporte avec moi le salaire que je vais donner à chacun selon ce qu’il a fait. 
${}^{13}Moi, je suis l’alpha et l’oméga, le premier et le dernier, le commencement et la fin.
${}^{14} Heureux ceux qui lavent leurs vêtements : ils auront droit d’accès à l’arbre de la vie et, par les portes, ils entreront dans la ville. 
${}^{15}Dehors les chiens, les sorciers, les débauchés, les meurtriers, les idolâtres, et tous ceux qui aiment et pratiquent le mensonge !
${}^{16} Moi, Jésus, j’ai envoyé mon ange vous apporter ce témoignage au sujet des Églises. Moi, je suis le rejeton, le descendant de David, l’étoile resplendissante du matin. »
        ${}^{17} L’Esprit et l’Épouse disent :
        « Viens ! »
        \\Celui qui entend, qu’il dise :
        « Viens ! »
        \\Celui qui a soif,
        qu’il vienne.
        \\Celui qui le désire,
        qu’il reçoive l’eau de la vie,
        gratuitement.
${}^{18} Et moi, devant tout homme qui écoute les paroles de ce livre de prophétie, je l’atteste : si quelqu’un y fait des surcharges, Dieu le chargera des fléaux qui sont décrits dans ce livre ; 
${}^{19}et si quelqu’un enlève des paroles à ce livre de prophétie, Dieu lui enlèvera sa part : il n’aura plus accès à l’arbre de la vie ni à la Ville sainte, qui sont décrits dans ce livre.
        ${}^{20} Et celui qui donne ce témoignage déclare :
        « Oui, je viens sans tarder. »
        \\– Amen ! Viens, Seigneur Jésus !
       
${}^{21} Que la grâce du Seigneur Jésus soit avec tous !
