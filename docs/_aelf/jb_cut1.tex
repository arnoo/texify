  
  
    
    \bbook{JOB}{JOB}
      
         
      \bchapter{}
      \begin{verse}
${}^{1}Il était une fois, au pays de Ouç, un homme appelé Job. Cet homme, intègre et droit, craignait Dieu et s’écartait du mal. 
${}^{2}Sept fils et trois filles lui étaient nés. 
${}^{3}Il avait un troupeau de sept mille brebis, trois mille chameaux, cinq cents paires de bœufs, cinq cents ânesses, et il possédait un grand nombre de serviteurs. Cet homme était le plus riche de tous les fils de l’Orient.
${}^{4}Or ses fils avaient coutume d’aller festoyer les uns chez les autres à tour de rôle, et ils faisaient inviter leurs trois sœurs à manger et à boire avec eux. 
${}^{5}Une fois terminé le cycle des festins, Job les faisait venir pour les purifier. Levé de bon matin, il offrait un holocauste pour chacun d’eux. Car Job se disait : « Peut-être mes fils ont-ils péché et maudit Dieu dans leur cœur. » C’est ainsi que Job agissait, chaque fois.
${}^{6}Le jour où les fils de Dieu se rendaient à l’audience du Seigneur, le Satan, l’Adversaire\\, lui aussi, vint parmi eux. 
${}^{7} Le Seigneur lui dit : « D’où viens-tu ? » L’Adversaire répondit\\ : « De parcourir la terre et d’y rôder. » 
${}^{8} Le Seigneur reprit : « As-tu remarqué mon serviteur Job ? Il n’a pas son pareil sur la terre : c’est un homme intègre et droit, qui craint Dieu et s’écarte du mal. » 
${}^{9} L’Adversaire riposta : « Est-ce pour rien que Job craint Dieu ? 
${}^{10} N’as-tu pas élevé une clôture pour le protéger, lui, sa maison et tout ce qu’il possède ? Tu as béni son travail, et ses troupeaux se multiplient dans le pays. 
${}^{11} Mais étends seulement la main, et touche à tout ce qu’il possède : je parie qu’il te maudira\\en face ! » 
${}^{12} Le Seigneur dit à l’Adversaire : « Soit ! Tu as pouvoir sur tout ce qu’il possède, mais tu ne porteras pas la main sur lui. » Et l’Adversaire se retira.
${}^{13}Le jour où les fils et les filles de Job étaient en train de festoyer\\et de boire du vin dans la maison de leur frère aîné, 
${}^{14} un messager arriva auprès de Job et lui dit : « Les bœufs étaient en train de labourer et les ânesses étaient au pâturage non loin de là. 
${}^{15} Les Bédouins se sont jetés sur eux et les ont enlevés, et ils ont passé les serviteurs au fil de l’épée. Moi seul, j’ai pu m’échapper pour te l’annoncer. » 
${}^{16} Il parlait encore quand un autre survint et lui dit : « Le feu du ciel est tombé\\, il a brûlé troupeaux et serviteurs, et les a dévorés. Moi seul, j’ai pu m’échapper pour te l’annoncer. » 
${}^{17} Il parlait encore quand un troisième survint et lui dit : « Trois bandes de Chaldéens se sont emparées des chameaux, ils les ont enlevés et ils ont passé les serviteurs au fil de l’épée. Moi seul, j’ai pu m’échapper pour te l’annoncer. » 
${}^{18} Il parlait encore quand un quatrième survint et lui dit : « Tes fils et tes filles étaient en train de festoyer et de boire du vin dans la maison de leur frère aîné, 
${}^{19} lorsqu’un ouragan s’est levé du fond du désert et s’est rué contre la maison. Ébranlée aux quatre coins, elle s’est écroulée sur les jeunes gens, et ils sont morts. Moi seul, j’ai pu m’échapper pour te l’annoncer. »
${}^{20}Alors Job se leva, il déchira son manteau et se rasa la tête, il se jeta à terre et se prosterna. 
${}^{21} Puis il dit : « Nu je suis sorti du ventre de ma mère, nu j’y retournerai\\. Le Seigneur a donné, le Seigneur a repris : Que le nom du Seigneur soit béni ! »
${}^{22}En tout cela, Job ne commit pas de péché. Il n’adressa à Dieu aucune parole déplacée.
       
      
         
      \bchapter{}
      \begin{verse}
${}^{1}Le jour où les fils de Dieu se rendaient à l’audience du Seigneur, l’Adversaire, lui aussi, vint parmi eux à l’audience. 
${}^{2}Le Seigneur lui dit : « D’où viens-tu ? » L’Adversaire répondit : « De parcourir la terre et d’y rôder. » 
${}^{3}Le Seigneur reprit : « As-tu remarqué mon serviteur Job ? Il n’a pas son pareil sur la terre : c’est un homme intègre et droit, qui craint Dieu et s’écarte du mal ; il persiste encore dans son intégrité, et c’est pour rien que tu m’as incité à le détruire. » 
${}^{4}Mais l’Adversaire répliqua au Seigneur : « Peau pour peau ! L’homme donne tout ce qu’il a pour sauver sa vie. 
${}^{5}Mais étends la main, touche à ses os et à sa chair, je parie qu’il te maudira en face ! » 
${}^{6}Le Seigneur dit à l’Adversaire : « Soit ! le voici en ton pouvoir, mais préserve sa vie. »
${}^{7}Et l’Adversaire, quittant la présence du Seigneur, frappa Job d’un ulcère malin depuis la plante des pieds jusqu’au sommet de la tête. 
${}^{8}Job prit un tesson pour se gratter, assis parmi les cendres. 
${}^{9}Sa femme lui dit : « Tu persistes encore dans ton intégrité ! Maudis Dieu et meurs ! » 
${}^{10}Il lui répondit : « Tu parles comme une insensée. Si nous accueillons le bonheur comme venant de Dieu, comment ne pas accueillir de même le malheur ? » En tout cela, Job ne commit pas de péché par ses lèvres.
       
${}^{11}Trois amis de Job apprirent tout ce malheur qui lui était advenu. Ils arrivèrent chacun de son pays, Élifaz de Témane, Bildad de Shouah et Sofar de Naama, et ils se concertèrent pour venir le plaindre et le consoler. 
${}^{12}De loin, levant les yeux sur lui, ils ne le reconnurent pas. Alors, ils éclatèrent en sanglots. Ils déchirèrent chacun son manteau et projetèrent de la poussière sur leur tête. 
${}^{13}Sept jours et sept nuits, ils restèrent assis par terre auprès de lui et, à la vue d’une si grande douleur, personne ne lui disait mot.
      <h2 class="intertitle hmbot" id="d85e126618">1. Premier discours de Job (3)</h2>
      
         
      \bchapter{}
      \begin{verse}
${}^{1}Après cela, Job ouvrit la bouche et maudit le jour de sa naissance\\.
${}^{2}Il prit la parole et dit :
       
        ${}^{3}« Périssent le jour qui m’a vu naître
        et la nuit qui a déclaré : “Un homme vient d’être conçu !”
${}^{4}Ce jour-là, qu’il soit ténèbres ;
        que Dieu, de là-haut, ne le convoque pas,
        \\que nulle clarté sur lui ne resplendisse !
${}^{5}Que le revendiquent ténèbres et ombre de mort,
        qu’une nuée sur lui repose,
        \\que les éclipses l’épouvantent !
${}^{6}Cette nuit-là, que l’obscurité s’en empare,
        qu’elle ne s’ajoute pas aux jours de l’année,
        qu’elle n’entre pas dans le compte des mois !
${}^{7}Oui, que cette nuit soit stérile,
        que nul cri d’allégresse n’y résonne !
${}^{8}Qu’elle soit malédiction pour ceux qui maudissent le jour,
        ceux qui sont prêts à réveiller Léviathan !
${}^{9}Que s’éteignent les étoiles de son aube,
        que cette nuit attende en vain la lumière,
        et n’entrevoie pas les paupières de l’aurore !
${}^{10}Car elle n’a pas scellé pour moi les portes de la matrice
        ni voilé à ma vue la misère.
         
        ${}^{11}Pourquoi ne suis-je pas mort dès le sein de ma mère,
        n’ai-je pas expiré\\au sortir de son ventre ?
        ${}^{12}Pourquoi s’est-il trouvé deux genoux pour me recevoir,
        deux seins pour m’allaiter ?
        ${}^{13}Maintenant je serais étendu, au calme,
        je dormirais d’un sommeil reposant,
        ${}^{14}avec les rois et les conseillers de la terre
        qui se bâtissent des mausolées,
        ${}^{15}ou avec les princes qui ont de l’or
        et remplissent d’argent leurs demeures.
        ${}^{16}Ou bien, comme l’avorton que l’on dissimule,
        \\je n’aurais pas connu l’existence,
        comme les petits qui n’ont pas vu le jour.
        ${}^{17}Là, au séjour des morts\\,
        prend fin l’agitation des méchants,
        là reposent ceux qui sont exténués.
${}^{18}De même, les prisonniers sont en paix,
        ils n’entendent plus les cris du gardien.
${}^{19}Petits et grands, là, sont égaux,
        et l’esclave est affranchi de son maître.
        ${}^{20}Pourquoi Dieu\\donne-t-il la lumière à un malheureux,
        la vie à ceux qui sont pleins d’amertume,
        ${}^{21}qui aspirent à la mort sans qu’elle vienne,
        qui la recherchent plus avidement qu’un trésor ?
        ${}^{22}Ils se réjouiraient, ils seraient dans l’allégresse,
        ils exulteraient s’ils trouvaient le tombeau.
        ${}^{23}Pourquoi Dieu donne-t-il la vie
        à un homme dont la route est sans issue,
        et qu’il enferme de toutes parts ?
${}^{24}En guise de pain, je n’ai que mes sanglots ;
        comme les eaux, mes rugissements déferlent.
${}^{25}La terreur qui me terrifie se réalise,
        et ce que je redoute m’arrive.
${}^{26}Ni calme pour moi, ni tranquillité,
        ni repos, rien que tourment ! »
      <h2 class="intertitle hmbot" id="d85e126988">2. Premier discours d’Élifaz (4 – 5)</h2>
      
         
      \bchapter{}
      \begin{verse}
${}^{1}Élifaz de Témane prit la parole et dit :
      
         
       
${}^{2}« Allons-nous t’adresser une parole ? Tu n’en peux plus !
        Mais qui pourrait garder le silence ?
${}^{3}Tu faisais la leçon à beaucoup,
        tu soutenais les mains défaillantes ;
${}^{4}tes propos redressaient celui qui perdait pied,
        tu fortifiais les genoux chancelants.
${}^{5}Et maintenant que cela t’arrive, tu te décourages ;
        te voici atteint, et tu es bouleversé.
${}^{6}Ta piété n’est-elle pas ton appui,
        ta vie intègre n’est-elle pas ton espérance ?
${}^{7}Souviens-toi : quel innocent a jamais péri ?
        En quel lieu des hommes droits ont-ils disparu ?
${}^{8}Je l’ai bien vu, moi : les laboureurs d’iniquité
        et les semeurs de misère eux-mêmes la moissonnent.
${}^{9}Sous l’haleine de Dieu ils périssent,
        au souffle de sa colère ils sont anéantis.
${}^{10}Le lion a beau rugir, le fauve gronder :
        les crocs des lionceaux seront brisés.
${}^{11}Le lion adulte périt faute de proie,
        les petits de la lionne se dispersent.
${}^{12}Une parole furtive m’est venue,
        mon oreille en a perçu le murmure.
${}^{13}Dans les cauchemars, les visions de la nuit,
        quand tombe une torpeur sur les humains,
${}^{14}un effroi m’a saisi, un frisson
        a fait trembler tous mes os :
${}^{15}un souffle a glissé sur ma face,
        il a hérissé les poils de ma chair.
${}^{16}Quelqu’un se tenait là, inconnu de moi,
        une forme devant mes yeux.
        \\Un silence… puis une voix s’est fait entendre :
         
${}^{17}“Le mortel aurait-il raison contre Dieu,
        l’homme serait-il pur devant son Auteur ?
${}^{18}Si Dieu ne fait pas même confiance à ses serviteurs,
        et qu’il persuade ses anges d’égarement,
${}^{19}que dire alors des habitants de ces maisons d’argile,
        fondées elles-mêmes sur la poussière !
        On les écrase comme une teigne ;
${}^{20}en l’espace d’un jour, ils sont pulvérisés ;
        sans qu’on y prenne garde, à jamais ils périssent.
${}^{21}Leurs attaches ne sont-elles pas rompues ?
        Ils meurent, faute de sagesse.”
       
      
         
      \bchapter{}
${}^{1}Fais donc appel ! Y a-t-il quelqu’un pour te répondre ?
        Parmi les saints, auquel pourras-tu t’adresser ?
${}^{2}En vérité, la hargne tue l’insensé,
        la jalousie fait mourir le sot.
${}^{3}J’ai vu, moi, l’insensé prendre racine,
        mais aussitôt j’ai maudit sa demeure :
        
           
         
${}^{4}“Que ses fils soient écartés du salut,
        accablés au tribunal
        sans que personne les délivre !
${}^{5}Ce qu’il a récolté, que l’affamé le dévore,
        que malgré les épines on s’en empare,
        et sa fortune, que les assoiffés l’engloutissent !”
        
           
         
${}^{6}L’iniquité ne sourd pas de la terre,
        la misère ne germe pas du sol ;
${}^{7}l’homme, lui, est né pour la misère,
        comme les aigles sont faits pour s’envoler.
${}^{8}Quant à moi, j’aurai recours à Dieu ;
        à Dieu, j’exposerai ma cause.
${}^{9}Il est l’auteur de grandes œuvres, insondables,
        d’innombrables merveilles.
${}^{10}Il répand la pluie à la surface de la terre,
        il arrose les campagnes ;
${}^{11}il élève les humbles,
        les affligés parviennent au salut ;
${}^{12}il déjoue les astuces des fourbes,
        empêchés de mener à bien leurs intrigues ;
${}^{13}il attrape les sages à leur astuce,
        il prend de vitesse le conseil des retors.
${}^{14}Ceux-là, en plein jour, se heurtent aux ténèbres,
        à midi ils tâtonnent comme en pleine nuit.
${}^{15}Le Seigneur sauve le pauvre du glaive,
        de leur bouche et de leur main puissante.
${}^{16}Alors le faible renaît à l’espoir
        et l’injustice se trouve muselée.
${}^{17}Oui, heureux l’homme que Dieu corrige !
        Ne va pas dédaigner la leçon du Puissant !
${}^{18}Car c’est lui qui blesse et panse la plaie,
        lui qui meurtrit et dont les mains guérissent.
${}^{19}De six angoisses il te préservera ;
        à la septième, le mal ne t’atteindra pas.
${}^{20}Dans la famine, il t’affranchira de la mort,
        dans le combat, des atteintes du glaive.
${}^{21}Du fouet de la langue tu seras à l’abri ;
        rien à craindre à l’approche du pillage.
${}^{22}Désastre, famine, tu t’en riras ;
        des bêtes de la terre, n’aie pas peur !
${}^{23}Tu concluras une alliance avec les pierres des champs,
        et la bête sauvage sera en paix avec toi.
${}^{24}La tente où tu habites, tu la trouveras en paix ;
        quand tu visiteras ta demeure, rien n’y manquera.
${}^{25}Ta postérité, tu la verras nombreuse,
        tes rejetons, comme la verdure de la terre.
${}^{26}Tu entreras dans la tombe
        mûr comme la gerbe mise en meule en son temps.
${}^{27}Voilà ce que nous avons observé : c’est ainsi.
        Écoute, et fais-en ton profit. »
        
           
      <h2 class="intertitle hmbot" id="d85e127486">3. Deuxième discours de Job (6 – 7)</h2>
      
         
      \bchapter{}
      \begin{verse}
${}^{1}Job prit la parole et dit :
      
         
       
${}^{2}« Ah ! Si l’on pouvait peser mon affliction,
        et sur la balance mettre aussi ma détresse !
${}^{3}Mais elles sont plus pesantes que le sable des mers.
        C’est pourquoi mes paroles s’étranglent.
${}^{4}Les flèches du Puissant me transpercent,
        c’est leur venin que boit mon esprit.
        Les terreurs de Dieu se rangent contre moi.
${}^{5}L’âne sauvage va-t-il braire devant l’herbe tendre,
        le bœuf meugler auprès de son fourrage ?
${}^{6}Un mets fade se mange-t-il sans sel,
        le blanc de l’œuf a-t-il quelque saveur ?
${}^{7}Je me refuse à y toucher ;
        ce n’est que nourriture écœurante.
${}^{8}Ah, si seulement se réalisait ma requête,
        si Dieu répondait à mon attente,
${}^{9}si Dieu consentait à me broyer,
        s’il étendait sa main et me retranchait !
${}^{10}J’aurais du moins la consolation
        – sursaut de joie dans une torture insoutenable – 
        de n’avoir pas renié les décrets du Dieu Saint.
${}^{11}Quelle est ma force pour que j’espère ?
        Qu’y a-t-il au terme pour que je prolonge ma vie ?
${}^{12}Ma force est-elle celle du roc,
        ma chair est-elle de bronze ?
${}^{13}Ne suis-je pas sans appui,
        et toute ressource ne m’a-t-elle pas quitté ?
${}^{14}À l’homme découragé devrait aller la pitié de son prochain,
        même s’il rejette la crainte du Puissant.
${}^{15}Mes frères, eux, ont trahi comme un torrent,
        comme le lit des torrents passagers :
${}^{16}la glace les assombrit,
        sur eux s’amoncelle la neige ;
${}^{17}mais à la saison brûlante, ils tarissent,
        sous l’ardeur du soleil, sur place, ils s’évaporent.
${}^{18}À leur recherche, les caravanes quittent la piste,
        s’enfoncent dans le désert et périssent.
${}^{19}Les caravanes de Téma les cherchent du regard,
        en eux espèrent les convois de Saba.
${}^{20}Mais ils sont déçus dans leur confiance ;
        arrivés sur les lieux, ils restent confondus.
${}^{21}Ainsi êtes-vous pour moi à présent :
        à ma vue, saisis d’effroi, vous êtes pris de panique.
${}^{22}Vous ai-je dit : “Faites-moi un cadeau,
        et sur votre fortune offrez-moi un présent ;
${}^{23}de la main de l’ennemi arrachez-moi,
        libérez-moi du pouvoir des tyrans” ?
${}^{24}Instruisez-moi, alors je me tairai ;
        montrez-moi en quoi j’ai failli !
${}^{25}En quoi peuvent blesser des paroles de droiture ?
        Que trouvez-vous à critiquer ?
${}^{26}Prétendez-vous censurer des mots ?
        Les paroles d’un désespéré, le vent les emporte.
${}^{27}Vous iriez jusqu’à tirer au sort l’orphelin,
        jusqu’à mettre aux enchères votre ami !
${}^{28}Et maintenant, décidez-vous, tournez-vous vers moi !
        Vous mentirais-je en face ?
${}^{29}Revenez donc ! Pas de perfidie !
        Encore une fois, revenez : il y va de ma justice !
${}^{30}Y a-t-il de la perfidie sur ma langue ?
        Mon palais ne sait-il pas discerner l’infortune ?
       
      
         
      \bchapter{}
        ${}^{1}Vraiment, la vie de l’homme sur la terre est une corvée,
        il fait des journées de manœuvre.
        ${}^{2}Comme l’esclave qui désire un peu d’ombre,
        comme le manœuvre qui attend sa paye,
        ${}^{3}depuis des mois je n’ai en partage que le néant,
        je ne compte que des nuits de souffrance.
        ${}^{4}À peine couché, je me dis :
        “Quand pourrai-je me lever ?”
        \\Le soir n’en finit pas :
        je suis envahi de cauchemars jusqu’à l’aube.
${}^{5}Ma chair s’est revêtue de vermine et de croûtes terreuses,
        ma peau se crevasse et suppure.
        ${}^{6}Mes jours sont plus rapides que la navette du tisserand,
        ils s’achèvent faute de fil.
        ${}^{7}Souviens-toi, Seigneur : ma vie n’est qu’un souffle,
        mes yeux ne verront plus le bonheur.
        ${}^{8}Je serai invisible aux yeux qui me voyaient ;
        tes yeux seront sur moi, mais je ne serai plus.
        ${}^{9}Comme la nuée se dissipe et s’évanouit,
        celui qui descend au séjour des morts n’en remonte pas ;
        ${}^{10}il ne retourne pas dans sa maison,
        sa demeure ne le connaît plus.
        ${}^{11}C’est pourquoi je ne peux retenir ma langue,
        dans mon angoisse\\je parlerai,
        dans mon amertume\\je me plaindrai.
        ${}^{12}Et moi, suis-je la Mer, ou le Dragon,
        pour que tu postes une garde contre moi ?
        ${}^{13}Je me dis : “Le sommeil me consolera,
        la nuit apaisera mes plaintes.”
        ${}^{14}Mais alors tu m’effraies par des songes,
        tu m’épouvantes par des cauchemars.
        ${}^{15}J’en arrive à souhaiter qu’on m’étrangle :
        la mort plutôt que mes douleurs !
        ${}^{16}Je suis à bout de patience, je ne vivrai pas toujours ;
        laisse-moi donc : mes jours ne sont qu’un souffle !
        ${}^{17}Qu’est-ce que l’homme,
        pour que tu en fasses tant de cas ?
        \\Tu fixes sur lui ton attention,
        ${}^{18}tu l’inspectes chaque matin,
        tu le scrutes à tout instant.
        ${}^{19}Ne peux-tu cesser de me regarder,
        le temps que j’avale ma salive ?
        ${}^{20}Si j’ai péché, en quoi t’ai-je offensé,
        “toi, le gardien de l’homme” ?
        \\Pourquoi me prendre pour cible,
        pourquoi te serais-je un fardeau ?
        ${}^{21}Ne peux-tu tolérer mes péchés,
        passer sur mes fautes ?
        \\Me voici bientôt étendu dans la poussière ;
        tu me chercheras, mais je ne serai plus. »
        
           
      <h2 class="intertitle hmbot" id="d85e128139">4. Premier discours de Bildad (8)</h2>
      
         
      \bchapter{}
      \begin{verse}
${}^{1}Bildad de Shouah prit la parole et dit :
      
         
       
${}^{2}« Vas-tu longtemps encore tenir de tels propos,
        et donner libre cours à ce vent de paroles ?
${}^{3}Dieu peut-il fausser le droit,
        le Puissant, fausser la justice ?
${}^{4}Si tes fils pèchent contre lui,
        il les livre au pouvoir de leur crime.
${}^{5}Quant à toi, si tu cherches Dieu,
        si tu supplies le Puissant,
${}^{6}si tu es honnête et droit,
        alors, il veillera sur toi
        et rétablira ta demeure dans la justice.
${}^{7}Ta condition ancienne sera peu de chose
        au regard de la nouvelle.
${}^{8}Interroge la génération passée,
        médite sur l’expérience de ses pères.
${}^{9}Puisque, nés d’hier, nous ne savons rien
        et que nos jours passent comme une ombre sur terre,
${}^{10}ne vont-ils pas, eux, t’instruire et t’enseigner,
        et de leur cœur tirer des sentences ?
         
${}^{11}“Le jonc pousse-t-il hors des marais ?
        Privé d’eau, le roseau peut-il croître ?
${}^{12}Encore en sa fleur et sans qu’on l’ait cueilli,
        avant toute herbe il se dessèche.”
         
${}^{13}Tel est le sort de ceux qui oublient Dieu,
        ainsi périt l’espoir de l’impie.
${}^{14}Son assurance n’est qu’un fil,
        sa confiance, une toile d’araignée.
${}^{15}S’appuie-t-il sur sa maison, elle ne tient pas,
        s’y cramponne-t-il, elle ne résiste pas.
${}^{16}Plein de sève au soleil,
        il étend ses jeunes pousses par-dessus son jardin,
${}^{17}ses racines se nouent dans un amas de pierres,
        il explore les creux des rochers.
${}^{18}Mais si on l’arrache de son lieu,
        celui-ci le renie : “Je ne t’ai jamais vu !”
${}^{19}Et voilà que son destin se corrompt,
        tandis que du sol quelqu’un d’autre va germer.
${}^{20}Vois : Dieu ne rejette pas l’homme intègre,
        ni ne prête main-forte aux malfaiteurs.
${}^{21}De rire encore il emplira ta bouche,
        et tes lèvres d’ovations.
${}^{22}Tes ennemis seront couverts de honte,
        et les tentes des méchants disparaîtront. »
      <h2 class="intertitle hmbot" id="d85e128352">5. Troisième discours de Job (9 – 10)</h2>
      
         
      \bchapter{}
      \begin{verse}
${}^{1}Job prit la parole et dit :
      
         
       
        ${}^{2}« En vérité, je sais bien qu’il en est ainsi :
        Comment l’homme pourrait-il avoir raison contre Dieu ?
        ${}^{3}Si l’on s’avise de discuter avec lui,
        on ne trouvera pas à lui répondre une fois sur mille.
        ${}^{4}Il est plein de sagesse et d’une force invincible,
        on ne lui tient pas tête impunément.
        ${}^{5}C’est lui qui déplace les montagnes à leur insu,
        qui les renverse dans sa colère ;
        ${}^{6}il secoue la terre sur sa base,
        et fait vaciller ses colonnes.
        ${}^{7}Il donne un ordre, et le soleil ne se lève pas,
        et sur les étoiles il appose un sceau.
        ${}^{8}À lui seul il déploie les cieux,
        il marche sur la crête des vagues.
        ${}^{9}Il fabrique la Grande Ourse, Orion,
        les Pléiades et les constellations\\du Sud.
        ${}^{10}Il est l’auteur de grandes œuvres, insondables,
        d’innombrables merveilles\\.
        ${}^{11}S’il passe à côté de moi, je ne le vois pas ;
        s’il me frôle, je ne m’en aperçois pas.
        ${}^{12}S’il s’empare d’une proie, qui donc lui fera lâcher prise,
        qui donc osera lui demander : “Que fais-tu là ?”
${}^{13}Dieu ne retient pas sa colère :
        sous ses pieds se prosternent les auxiliaires de Rahab.
        ${}^{14}Et moi, je prétendrais lui répliquer !
        je chercherais des arguments contre lui !
        ${}^{15}Même si j’ai raison, à quoi bon me défendre\\ ?
        Je ne puis que demander grâce à mon juge.
        ${}^{16}Même s’il répond quand je fais appel,
        je ne suis pas sûr qu’il écoute ma voix !
${}^{17}Lui qui dans la tempête m’écrase
        et multiplie sans raison mes blessures,
${}^{18}il ne me laisse même pas reprendre haleine,
        tant il m’abreuve d’amertume.
${}^{19}Recourir à la force ? Il est la puissance même.
        Faire appel au droit ? Mais qui l’assignera ?
${}^{20}Même si je suis juste, ma bouche me condamne ;
        innocent, elle me déclare pervers.
${}^{21}Suis-je un homme intègre ? Je ne sais plus moi-même.
        Vivre me répugne.
${}^{22}C’est tout un, je l’ai bien dit :
        il extermine pareillement l’homme intègre et le criminel.
${}^{23}Si un fléau répand soudain la mort,
        lui se moque de la détresse des innocents.
${}^{24}Un pays est-il livré aux mains du criminel,
        il met un voile sur la face de ses juges.
        Si ce n’est lui, qui est-ce donc ?
${}^{25}Mes jours, plus rapides qu’un coureur,
        ont fui sans voir le bonheur.
${}^{26}Ils ont glissé comme barques de jonc,
        comme l’aigle qui fond sur sa proie.
${}^{27}Si je me dis : “Oublie ta plainte,
        déride ton visage, montre ta joie”,
${}^{28}je redoute tous mes tourments,
        je sais que tu ne m’acquitteras pas.
${}^{29}Si je suis coupable,
        à quoi bon me fatiguer en vain ?
${}^{30}Si je me lave avec de la neige,
        si je purifie mes mains à la soude,
${}^{31}tu me plonges dans la fange,
        et mes vêtements ont horreur de moi.
${}^{32}Car lui n’est pas comme moi un humain
        pour que je lui réplique
        \\et qu’ensemble nous allions en justice.
${}^{33}Pas d’arbitre entre nous
        pour poser la main sur nous deux,
${}^{34}pour écarter de moi son bâton,
        \\et pour que sa terreur ne m’épouvante plus.
${}^{35}Alors je parlerais sans avoir peur de lui.
        Mais il n’en est rien : je suis face à moi-même.
       
      
         
      \bchapter{}
      \begin{verse}
${}^{1}La vie me dégoûte,
        je veux donner libre cours à ma plainte,
        d’un cœur amer, je parlerai.
${}^{2}Je dirai à Dieu : Ne me condamne pas,
        fais-moi connaître tes griefs contre moi.
${}^{3}Est-ce un bien pour toi d’opprimer,
        de renier l’œuvre de tes mains
        et de favoriser les intrigues des méchants ?
${}^{4}As-tu des yeux de chair,
        et ton regard est-il celui des humains ?
${}^{5}tes jours sont-ils comme les jours d’un mortel,
        et tes années, comme celles d’un homme,
${}^{6}pour que tu recherches mon crime
        et que tu enquêtes sur mon péché,
${}^{7}bien que tu me saches non coupable
        et que nul ne puisse délivrer de ta main ?
${}^{8}Tes mains m’ont façonné, créé, de toutes pièces,
        et tu voudrais me détruire !
${}^{9}Souviens-toi : tu m’as pétri comme l’argile,
        et tu me ramènerais à la poussière !
${}^{10}Ne m’as-tu pas versé comme le lait,
        et fait prendre comme le fromage ?
${}^{11}De peau et de chair tu m’as vêtu,
        d’os et de nerfs tu m’as tissé.
${}^{12}Tu m’as donné vie et amour,
        veillant sur mon souffle avec sollicitude.
${}^{13}Mais tu as gardé une arrière-pensée,
        je sais ce que tu trames :
${}^{14}si je commets un péché, tu me prends sur le fait
        et ne me tiens pas quitte de ma faute.
${}^{15}Si je suis coupable, malheur à moi !
        Si j’ai raison, je n’ose lever la tête,
        gorgé de honte, abreuvé d’affliction.
${}^{16}Si je me relève, tel un lion tu me pourchasses,
        tu redoubles contre moi tes exploits,
${}^{17}tu m’opposes de nouveaux témoins,
        tu augmentes ta colère envers moi,
        des troupes contre moi se relayent.
${}^{18}Pourquoi donc m’as-tu fait sortir du sein maternel ?
        J’aurais expiré, nul œil ne m’aurait vu ;
${}^{19}je serais comme n’ayant pas été,
        on m’aurait porté du ventre à la tombe.
${}^{20}N’est-ce pas peu de chose que la durée de mes jours ?
        Retire-toi de moi pour que j’éprouve un peu de joie,
${}^{21}avant que je m’en aille sans retour
        au pays des ténèbres et de l’ombre de mort,
${}^{22}pays où le crépuscule est obscurité,
        ombre de mort et désordre,
        où la clarté même est obscure. »
      <h2 class="intertitle hmbot" id="d85e129009">6. Premier discours de Sofar (11)</h2>
      
         
      \bchapter{}
      \begin{verse}
${}^{1}Sofar de Naama prit la parole et dit :
      
         
       
${}^{2}« Un tel flot de paroles restera-t-il sans réponse ?
        Suffit-il d’être verbeux pour avoir raison ?
${}^{3}Tes bavardages feront-ils taire les gens,
        te moqueras-tu sans que nul te confonde ?
${}^{4}Tu as dit : “Mon savoir est irréprochable,
        je suis pur à tes yeux !”
${}^{5}Mais si seulement Dieu voulait parler,
        si pour toi il desserrait les lèvres,
${}^{6}s’il te dévoilait les secrets de la sagesse
        tellement subtils à entendre,
        \\alors tu saurais que Dieu
        oublie une part de tes fautes.
${}^{7}Prétends-tu sonder la profondeur de Dieu,
        atteindre la perfection du Puissant ?
${}^{8}Elle est haute comme les cieux : que feras-tu ?
        plus abyssale que le séjour des morts : qu’en sauras-tu ?
${}^{9}Plus longue que la terre est son étendue,
        et plus vaste que la mer !
${}^{10}S’il vient à passer, s’il emprisonne,
        s’il convoque en justice, qui l’en détournera ?
${}^{11}Car lui connaît les hommes de rien,
        sans peine il discerne le mal.
${}^{12}Un écervelé peut accéder à la raison,
        un ânon sauvage devenir un homme !
${}^{13}Et toi, si tu affermis ton cœur
        et tends les paumes vers Dieu,
${}^{14}si tu écartes le mal dont tu es responsable
        et n’héberges pas l’injustice sous ta tente,
${}^{15}alors tu lèveras un visage sans reproche,
        tu seras ferme et sans crainte.
${}^{16}Ta peine, tu l’oublieras,
        tu t’en souviendras comme d’une eau déjà écoulée.
${}^{17}Plus radieuse que midi ta vie se lèvera,
        le crépuscule brillera comme le matin.
${}^{18}Tu seras confiant car il y aura de l’espoir,
        et, protégé, tu dormiras tranquille.
${}^{19}Ton repos, nul ne le troublera,
        et beaucoup rechercheront tes faveurs.
${}^{20}Quant aux méchants, leurs yeux se consument,
        tout refuge leur fait défaut.
        Leur espoir, c’est de rendre l’âme. »
      <h2 class="intertitle hmbot" id="d85e129246">7. Quatrième discours de Job (12 – 14)</h2>
      
         
      \bchapter{}
      \begin{verse}
${}^{1}Job prit la parole et dit :
      
         
       
${}^{2}« Vraiment, c’est vous qui êtes la voix du peuple,
        et avec vous mourra la sagesse !
${}^{3}Moi aussi, comme vous, je sais réfléchir,
        je ne vous suis nullement inférieur.
        Et qui ne disposerait d’arguments semblables ?
${}^{4}Je suis la risée du prochain,
        moi qui appelle vers Dieu pour qu’il réponde.
        Objet de risée, le juste parfait !
${}^{5}Au malchanceux, le mépris ! pense l’homme heureux.
        Un coup de plus à ceux dont le pied chancelle !
${}^{6}Elles sont en paix, les tentes des pillards ;
        ils sont tranquilles, ceux qui provoquent Dieu
        et celui qui veut mettre Dieu en son pouvoir.
${}^{7}Mais interroge donc le bétail, il t’instruira,
        l’oiseau du ciel, il te renseignera ;
${}^{8}parle avec la terre, elle t’apprendra ;
        ils te raconteront, les poissons de la mer.
${}^{9}Qui ne sait, parmi tous ces êtres,
        que la main du Seigneur a fait cela,
${}^{10}lui qui tient dans sa main l’âme de tout vivant
        et le souffle de toute créature humaine ?
${}^{11}N’est-ce pas l’oreille qui apprécie les mots,
        le palais qui goûte les mets ?
${}^{12}N’est-ce pas chez les vieillards que se trouve la sagesse,
        dans le grand âge le discernement ?
         
${}^{13}En Dieu sagesse et puissance,
        à lui conseil et intelligence.
${}^{14}S’il détruit, nul ne peut rebâtir ;
        s’il enferme un homme, nul n’ouvrira.
${}^{15}S’il retient les eaux, c’est la sécheresse ;
        s’il les relâche, elles bouleversent la terre.
${}^{16}En lui force et prudence,
        à lui l’homme égaré et celui qui égare.
${}^{17}Il fait marcher nu-pieds les conseillers
        et frappe les juges de folie.
${}^{18}Il enlève le baudrier des rois
        et leur passe une corde aux reins.
${}^{19}Il fait marcher nu-pieds les prêtres
        et renverse les puissants.
${}^{20}Il ôte le langage aux hommes les plus sûrs,
        retire aux vieillards la sagacité.
${}^{21}Il déverse le mépris sur les notables,
        dénoue le ceinturon des forts.
${}^{22}Il met à découvert les profondeurs des ténèbres,
        fait sortir à la lumière l’ombre de mort.
${}^{23}Il agrandit les nations, et les fait périr,
        laisse les peuples s’étendre, et les déporte.
${}^{24}Il ôte l’intelligence aux chefs de la populace,
        les égare dans un chaos sans chemin ;
${}^{25}là ils tâtonnent dans les ténèbres, sans lumière,
        égarés comme des ivrognes.
       
      
         
      \bchapter{}
${}^{1}Oui, tout cela, mon œil l’a vu,
        mon oreille l’a entendu et compris.
${}^{2}Ce que vous savez, je le sais, moi aussi,
        je ne vous suis nullement inférieur.
${}^{3}Mais moi, c’est au Puissant que je veux parler,
        c’est contre Dieu que je veux récriminer.
${}^{4}Vous, vous n’êtes que badigeonneurs de mensonge,
        guérisseurs de néant !
${}^{5}Ah ! Si seulement vous gardiez une bonne fois le silence,
        il vous tiendrait lieu de sagesse !
${}^{6}Écoutez donc ma récrimination,
        au plaidoyer de mes lèvres, prêtez l’oreille.
${}^{7}Est-ce pour Dieu que vous dites des paroles injustes,
        pour lui que vous débitez des faussetés ?
${}^{8}Prenez-vous donc son parti ?
        Est-ce pour Dieu que vous plaidez ?
${}^{9}Serait-il bon qu’il enquête sur vous ?
        Se moque-t-on de lui comme on se joue d’un homme ?
${}^{10}Il vous reprocherait sûrement
        d’avoir pris parti en secret !
${}^{11}Sa grandeur ne vous effraie donc pas ?
        Est-ce que sa terreur ne fond pas sur vous ?
${}^{12}Vos références : des maximes de cendre !
        Vos défenses : des défenses d’argile !
${}^{13}Taisez-vous devant moi ! C’est moi qui vais parler,
        et m’advienne que pourra !
${}^{14}J’emporte ma chair entre mes dents,
        je mets ma vie en jeu.
${}^{15}S’il doit me tuer, je n’ai plus d’espoir :
        je veux seulement, face à lui, justifier ma conduite.
${}^{16}Et cela même sera mon salut,
        car nul impie ne viendrait en sa présence.
${}^{17}Écoutez, écoutez ma parole,
        prêtez l’oreille à mon explication.
${}^{18}Voici : j’ai intenté un procès ;
        c’est moi qui ai raison, je le sais.
${}^{19}Y aurait-il quelqu’un pour plaider contre moi ?
        À l’instant, je n’aurais qu’à me taire et à expirer !
${}^{20}Épargne-moi seulement deux choses,
        alors je ne me cacherai pas devant ta face :
${}^{21}éloigne ta main qui pèse sur moi,
        et que ta terreur ne m’épouvante plus.
${}^{22}Puis appelle, et moi je répondrai ;
        ou bien je parlerai, et tu me répliqueras.
${}^{23}Combien ai-je commis de fautes et de péchés ?
        Ma transgression et mon péché, fais-les moi connaître.
${}^{24}Pourquoi caches-tu ta face
        et me considères-tu comme un ennemi ?
${}^{25}Veux-tu faire trembler une feuille qui s’envole,
        et poursuivre une paille sèche,
${}^{26}pour que tu rédiges contre moi d’amères sentences,
        que tu m’imputes des fautes de jeunesse,
${}^{27}que tu fixes mes pieds dans des blocs de bois,
        que tu observes toutes mes démarches
        et relèves l’empreinte de mes pas ?
${}^{28}Et tout cela contre un être qui se désagrège comme bois vermoulu,
        comme vêtement dévoré par la teigne !
        
           
       
      
         
      \bchapter{}
${}^{1}L’homme, né de la femme,
        vit peu de jours, rassasié de tourments\\ ;
${}^{2}comme fleur, il germe et se fane ;
        tel une ombre, il fuit sans s’arrêter.
${}^{3}Et toi, Dieu\\, c’est sur lui que tu fixes ton regard,
        c’est moi que tu obliges à comparaître avec toi !
${}^{4}Qui tirera le pur de l’impur ?
        Personne !
${}^{5}Puisque ses jours sont décrétés,
        que tu as décidé du nombre de ses mois,
        et fixé sa limite, infranchissable,
${}^{6}détourne de lui ton regard, et laisse-le,
        jusqu’à ce que, tel un salarié, il s’acquitte de sa journée !
${}^{7}Car il y a pour l’arbre un espoir :
        une fois coupé, il peut verdir encore
        et les jeunes pousses ne lui feront pas défaut.
${}^{8}Quand bien même sa racine aurait vieilli en terre,
        et que la souche serait morte dans le sol,
${}^{9}dès qu’il flaire l’eau, il bourgeonne
        et se fait une ramure comme un jeune plant.
${}^{10}L’homme qui meurt reste inerte ;
        quand un humain expire, où donc est-il ?
${}^{11}Les eaux pourront quitter la mer,
        les fleuves, tarir et se dessécher :
${}^{12}mais l’homme, une fois couché, ne se relèvera plus.
        \\Les cieux disparaîtront avant qu’il ne s’éveille,
        qu’il ne sorte de son sommeil.
${}^{13}Ah ! Si seulement tu me cachais au séjour des morts
        et me dissimulais jusqu’à ce que reflue ta colère !
        Tu me fixerais un terme où tu te souviendrais de moi.
${}^{14}– Mais l’homme qui meurt va-t-il revivre ?
        Tous les jours de mon service, j’attendrais,
        jusqu’à ce que vienne ma relève.
${}^{15}Tu m’appellerais et je te répondrais,
        tu languirais après l’œuvre de tes mains.
${}^{16}Alors que maintenant tu dénombres mes pas,
        tu n’épierais plus mon péché ;
${}^{17}scellée dans un coffret serait ma transgression,
        et tu blanchirais ma faute.
${}^{18}Mais la montagne tombe et s’écroule,
        le rocher bouge de sa place,
${}^{19}l’eau creuse les pierres,
        l’averse emporte la poussière du sol :
        ainsi, l’espoir de l’homme, tu l’anéantis.
${}^{20}Tu terrasses l’homme pour toujours et il s’en va ;
        tu le défigures, puis tu le renvoies.
${}^{21}Ses fils sont-ils honorés, il n’en sait rien ;
        sont-ils méprisés, il l’ignore.
${}^{22}Sa chair ne ressent que ses propres souffrances,
        son âme ne gémit que sur lui-même. »
        
           
      <h2 class="intertitle hmbot" id="d85e130132">8. Deuxième discours d’Élifaz (15)</h2>
      
         
      \bchapter{}
      \begin{verse}
${}^{1}Élifaz de Témane prit la parole et dit :
      
         
       
${}^{2}« Le sage répond-il par des raisons en l’air,
        gonfle-t-il ses poumons de vent,
${}^{3}pour argumenter avec des discours sans valeur,
        des mots inutiles ?
${}^{4}Tu en viens à saper la piété,
        tu discrédites la méditation devant Dieu.
${}^{5}C’est ta faute qui inspire ta bouche,
        et tu adoptes le langage des fourbes.
${}^{6}Ce qui te condamne, c’est ta bouche, ce n’est pas moi,
        tes lèvres mêmes témoignent contre toi.
${}^{7}Es-tu né le premier des hommes,
        as-tu été enfanté avant les collines ?
${}^{8}Aurais-tu écouté au conseil de Dieu,
        aurais-tu accaparé la sagesse ?
${}^{9}Que sais-tu que nous ne sachions ?
        Qu’as-tu compris qui ne nous soit familier ?
${}^{10}Parmi nous aussi il y a des cheveux blancs et des vieillards,
        plus chargés de jours que ton père.
${}^{11}Est-ce trop peu pour toi que le réconfort de Dieu
        et la parole modérée qui t’est adressée ?
${}^{12}Pourquoi te laisser emporter par ton cœur
        et pourquoi cligner des yeux
${}^{13}quand tu tournes contre Dieu ta colère
        et que ta bouche profère des discours ?
${}^{14}Qu’est-ce que l’homme pour se dire intègre,
        l’enfant de la femme, pour se prétendre juste ?
${}^{15}Dieu, même à ses saints, ne fait pas confiance
        et le ciel n’est pas pur à ses yeux.
${}^{16}Encore moins le répugnant, le corrompu,
        l’homme qui boit la perfidie comme de l’eau !
${}^{17}Je vais t’instruire, écoute-moi :
        ce que j’ai vu, je vais le raconter,
${}^{18}ce que les sages, sans rien dissimuler,
        relatent d’après leurs pères,
${}^{19}eux à qui seul le pays fut donné
        sans qu’aucun étranger se soit infiltré parmi eux.
${}^{20}Tous les jours de sa vie, le méchant se tourmente,
        les années du tyran sont strictement comptées.
${}^{21}Des voix effrayantes hurlent à ses tympans ;
        en pleine paix, le dévastateur vient l’attaquer.
${}^{22}Il ne croit plus pouvoir échapper aux ténèbres
        et se voit promis au glaive.
${}^{23}Il erre çà et là, mais où trouver du pain ?
        Il le sait : le sort qui l’attend, c’est un jour de ténèbres.
${}^{24}La détresse et l’angoisse l’envahissent,
        elles le terrassent comme un roi qui se prépare à l’assaut.
${}^{25}Car il a levé la main contre Dieu,
        il a bravé le Puissant.
${}^{26}Il fonçait sur lui tête baissée,
        sous le dos épais de ses boucliers.
${}^{27}Oui, son visage s’est couvert de graisse,
        ses reins ont pris de l’embonpoint.
${}^{28}Il a occupé des villes détruites,
        des maisons inhabitables, qui menaçaient ruine.
${}^{29}Il ne s’enrichira pas, sa fortune ne tiendra pas,
        il n’étendra pas ses possessions dans le pays.
${}^{30}Il n’échappera pas aux ténèbres,
        une flamme desséchera ses jeunes pousses
        et il s’enfuira au souffle de la bouche de Dieu.
${}^{31}Qu’il ne mise pas sur la fraude, il ferait fausse route,
        car la fraude serait son salaire.
${}^{32}Cela s’accomplira avant le temps
        et sa ramure ne reverdira plus.
${}^{33}Comme la vigne, il laissera choir ses fruits encore verts,
        il perdra, comme l’olivier, sa floraison.
${}^{34}Car stérile est l’engeance de l’impie,
        et le feu dévore les tentes de la corruption.
${}^{35}Qui conçoit le méfait enfante le malheur
        et c’est la perfidie qui mûrit dans son sein. »
      <h2 class="intertitle hmbot" id="d85e130474">9. Cinquième discours de Job (16 – 17)</h2>
      
         
      \bchapter{}
      \begin{verse}
${}^{1}Job prit la parole et dit :
      
         
       
${}^{2}« Que de fois ai-je entendu de tels propos !
        Vous êtes tous de piètres consolateurs !
${}^{3}À ces paroles de vent, y aura-t-il une fin ?
        Et qu’est-ce qui t’incite, toi, à répliquer ?
${}^{4}Moi aussi, je parlerais comme vous,
        si vous étiez à ma place ;
        je vous accablerais de discours
        et je hocherais la tête à votre sujet.
${}^{5}Je vous réconforterais par mes paroles,
        et le mouvement de mes lèvres vous serait un soulagement.
${}^{6}À présent, si je parle, ma douleur n’est pas soulagée ;
        si je m’abstiens, va-t-elle pour autant s’en aller ?
${}^{7}Or maintenant on m’a poussé à bout.
        – Oui, tu as ravagé tout mon entourage,
${}^{8}tu m’as couvert de rides.
        Ma maigreur se fait témoin ;
        elle se dresse devant moi et m’accuse en face.
${}^{9}Mon adversaire, dans sa colère, me déchire, me traque,
        il grince des dents contre moi,
        il aiguise sur moi ses regards.
${}^{10}Les gens ouvrent leur bouche contre moi ;
        ils me giflent en m’insultant ;
        ensemble, contre moi ils s’ameutent.
${}^{11}Dieu me livre à des garnements,
        aux mains des méchants il me jette.
${}^{12}J’étais paisible, il m’a brisé,
        il m’a saisi par la nuque et mis en pièces ;
        il m’a dressé comme cible,
${}^{13}ses flèches me cernent,
        il transperce mes reins sans pitié,
        il répand ma bile sur le sol.
${}^{14}Il ouvre en moi brèche sur brèche,
        fonce sur moi, tel un guerrier.
${}^{15}J’ai cousu le sac de deuil sur ma peau,
        traîné mon front dans la poussière.
${}^{16}Mon visage est rougi par les pleurs,
        sur mes paupières s’étend l’ombre de mort.
${}^{17}Pourtant, nulle violence en mes mains
        et ma prière est pure !
${}^{18}Terre, ne couvre pas mon sang,
        et que rien n’arrête mes cris !
${}^{19}Même maintenant, j’ai dans le ciel mon témoin,
        dans les hauteurs mon répondant.
${}^{20}Mes amis se moquent de moi,
        c’est vers Dieu que pleurent mes yeux.
${}^{21}Ah, si Dieu pouvait être arbitre entre l’homme et lui-même,
        comme entre un fils d’homme et son semblable !
${}^{22}Car elles sont comptées, les années à venir,
        et je vais prendre un chemin sans retour.
       
      
         
      \bchapter{}
${}^{1}Mon souffle s’épuise, mes jours s’éteignent ;
        pour moi le cimetière !
${}^{2}Ne suis-je pas objet de raillerie,
        l’œil tenu éveillé par leurs provocations ?
${}^{3}Dépose donc ma caution près de toi :
        qui d’autre accepterait un gage de ma main ?
${}^{4}Puisque tu as fermé leur cœur à la raison,
        tu ne vas pas les faire triompher !
${}^{5}“Tel homme invite ses amis à un partage,
        alors que se consument les yeux de ses enfants.”
${}^{6}Voilà le proverbe que les gens m’appliquent,
        je suis celui à qui l’on crache au visage.
${}^{7}De chagrin mon œil s’éteint,
        tous mes membres sont comme l’ombre.
${}^{8}Les hommes droits en sont stupéfaits
        et l’innocent contre l’impie s’indigne.
${}^{9}Cependant le juste tient ferme son chemin,
        et celui qui a les mains pures redouble d’efforts.
${}^{10}Quant à vous, revenez tous, venez donc !
        Je ne trouverai aucun sage parmi vous.
${}^{11}Mes jours ont passé,
        brisés sont mes plans, les désirs de mon cœur.
${}^{12}On veut faire de la nuit le jour ;
        face aux ténèbres, on prétend que la lumière est proche.
${}^{13}Si je dois espérer le séjour des morts comme demeure,
        étendre dans les ténèbres ma couche,
${}^{14}appeler la fosse “mon père”,
        la vermine “ma mère et ma sœur”,
${}^{15}où donc est mon espoir ?
        mon espérance, qui l’entrevoit ?
${}^{16}Elle descendra jusqu’au fond du séjour des morts,
        quand ensemble nous enfoncerons dans la poussière. »
        
           
      <h2 class="intertitle hmbot" id="d85e130882">10. Deuxième discours de Bildad (18)</h2>
      
         
      \bchapter{}
      \begin{verse}
${}^{1}Bildad de Shouah prit la parole et dit :
      
         
       
${}^{2}« Quand donc mettrez-vous un terme aux discours ?
        Réfléchissez, ensuite nous parlerons.
${}^{3}Pourquoi passerions-nous pour des bêtes
        et serions-nous stupides à vos yeux ?
${}^{4}Ô toi, qui te déchires dans ta colère,
        à cause de toi la terre sera-t-elle désolée,
        le rocher, changé de place ?
${}^{5}Oui, la lumière du méchant s’éteint,
        la flamme de son feu ne brille plus.
${}^{6}La lumière s’obscurcit dans sa tente,
        sa lampe au-dessus de lui s’éteint.
${}^{7}Ses pas vigoureux se raccourcissent,
        ses intrigues le font trébucher,
${}^{8}car il se prend les pieds dans le filet,
        il marche sur un piège.
${}^{9}Un lacet le saisit au talon ;
        un collet se referme sur lui.
${}^{10}Cachée dans la terre, une corde l’attend,
        et sur le sentier, une trappe.
${}^{11}Autour de lui des terreurs l’épouvantent
        et s’accrochent à ses pas.
${}^{12}En pleine vigueur il souffre la faim,
        et le malheur se tient à ses côtés.
${}^{13}Il dévore sa peau, le démon de la mort,
        il la dévore par lambeaux.
${}^{14}On l’arrache à la sécurité de sa tente,
        on le conduit vers le roi des terreurs.
${}^{15}Un autre habite sous sa tente
        qui ne lui appartient plus ;
        sur sa demeure on répand du soufre.
${}^{16}En bas, ses racines se dessèchent ;
        en haut, ses rameaux se flétrissent.
${}^{17}Sa mémoire est effacée de la terre,
        il n’a plus de nom dans la contrée.
${}^{18}De la lumière on le pousse dans les ténèbres,
        et du monde on le chasse.
${}^{19}Pas de lignée pour lui, ni de postérité dans son peuple,
        et point de survivant en ses lieux de séjour.
${}^{20}Son destin stupéfie ceux de l’Occident,
        ceux de l’Orient sont saisis d’effroi.
${}^{21}Oui, telles sont les demeures du criminel,
        le lieu de qui ne connaît pas Dieu. »
      <h2 class="intertitle hmbot" id="d85e131120">11. Sixième discours de Job (19)</h2>
      
         
      \bchapter{}
      \begin{verse}
${}^{1}Job prit la parole et dit :
      
         
       
${}^{2}« Allez-vous longtemps encore affliger mon âme
        et m’écraser avec des mots ?
${}^{3}Voilà dix fois que vous m’outragez,
        que sans vergogne vous me rudoyez.
${}^{4}S’il est vrai que j’ai fait un faux pas,
        mon faux pas ne regarde que moi.
${}^{5}Si vraiment avec moi vous le prenez de haut
        et me reprochez mon déshonneur,
${}^{6}sachez alors que c’est Dieu qui a violé mon droit
        et qui m’a pris dans son filet.
${}^{7}Si je crie à la violence, pas de réponse ;
        j’ai beau appeler, pas de jugement !
${}^{8}Il a barré ma route pour que je ne passe pas,
        et sur mes sentiers il a mis des ténèbres.
${}^{9}De ma gloire il m’a dépouillé,
        il a enlevé la couronne de ma tête.
${}^{10}Il me ruine de toutes parts, et je m’en vais ;
        il déracine, comme un arbre, mon espérance.
${}^{11}Enflammé de colère contre moi,
        il me traite comme ses ennemis.
${}^{12}Ensemble arrivent ses troupes,
        elles remblayent leur route jusqu’à moi,
        elles campent autour de ma tente.
${}^{13}Mes frères, il les a éloignés de moi ;
        ceux qui me connaissent prennent soin de m’éviter.
${}^{14}Mes proches ont disparu,
        mes intimes m’ont oublié.
${}^{15}Les hôtes de ma maison et mes servantes
        me considèrent comme un inconnu ;
        à leurs yeux, je suis devenu un étranger.
${}^{16}Si j’appelle mon serviteur, il ne répond pas,
        je dois le supplier de ma bouche.
${}^{17}Mon haleine répugne à ma femme,
        mon souffle à mes propres enfants.
${}^{18}Même les garnements ont pour moi du mépris ;
        si je me lève, ils parlent contre moi.
${}^{19}Tous mes confidents m’ont en horreur,
        ceux que j’aimais se sont tournés contre moi.
${}^{20}Mes os collent à ma peau et à ma chair,
        et je n’ai pu sauver que ma peau et mes dents !
        ${}^{21}Ayez pitié de moi, ayez pitié de moi,
        vous du moins, mes amis,
        car la main de Dieu\\m’a frappé.
        ${}^{22}Pourquoi me poursuivre comme Dieu lui-même ?
        Ne serez-vous jamais rassasiés de ma chair ?
        ${}^{23}Ah, si seulement on écrivait mes paroles,
        si on les gravait sur une stèle
        ${}^{24}avec un ciseau de fer et du plomb\\,
        si on les sculptait dans le roc pour toujours !
        ${}^{25}Mais je sais, moi, que mon rédempteur est vivant\\,
        que, le dernier, il se lèvera sur la poussière ;
        ${}^{26}et quand bien même on m’arracherait la peau\\,
        de ma chair je verrai Dieu\\.
        ${}^{27}Je le verrai, moi en personne,
        et si mes yeux le regardent, il ne sera plus un étranger.
        Mon cœur en défaille au-dedans de moi.
${}^{28}Lorsque vous dites : “Comment le poursuivre
        et trouver en lui prétexte à procès ?”,
${}^{29}craignez pour vous-mêmes le glaive,
        car la colère mérite châtiment par le glaive.
        Ainsi vous saurez qu’il y a une justice. »
      <h2 class="intertitle hmbot" id="d85e131496">12. Deuxième discours de Sofar (20)</h2>
      
         
      \bchapter{}
      \begin{verse}
${}^{1}Sofar de Naama prit la parole et dit :
      
         
       
${}^{2}« Eh bien ! Mon trouble m’incite à répliquer
        à cause de l’émotion que je ressens.
${}^{3}J’entends une leçon qui m’outrage :
        ma raison me souffle la réponse.
${}^{4}Ne le sais-tu pas ? Depuis toujours,
        depuis que l’homme a été mis sur la terre,
${}^{5}la jubilation des méchants tourne court
        et la joie de l’impie ne dure qu’un instant.
${}^{6}Quand sa taille s’élèverait jusqu’au ciel
        et que sa tête toucherait aux nuages,
${}^{7}comme son ordure, il disparaît à jamais ;
        ceux qui le voyaient disent : “Où est-il ?”
${}^{8}Comme un songe il s’envole, on ne le trouve plus ;
        il est chassé comme une vision nocturne.
${}^{9}L’œil qui le regardait le perd de vue
        et la place où il était ne l’aperçoit plus.
${}^{10}Ses fils doivent mendier auprès des pauvres,
        et ses propres mains restituer sa fortune.
${}^{11}Ses os étaient pleins de jeunesse :
        les voilà étendus avec lui sur la poussière.
${}^{12}Même si dans sa bouche le mal est doux,
        s’il le cache sous sa langue,
${}^{13}le conserve, ne l’abandonne pas,
        et le retient au fond de son palais,
${}^{14}dans ses entrailles sa nourriture s’altère,
        dans son intestin c’est un venin d’aspic.
${}^{15}Les richesses qu’il a englouties, il les vomit ;
        de son ventre, Dieu les expulse.
${}^{16}Il suçait du venin d’aspic :
        la langue de la vipère le tue.
${}^{17}Il ne verra plus les ruisseaux,
        les fleuves, les torrents de miel et de crème.
${}^{18}Il rendra son gain, sans pouvoir l’engloutir ;
        il ne jouira pas non plus du fruit de son commerce.
${}^{19}Parce qu’il a maltraité, abandonné les pauvres,
        s’est emparé d’une maison au lieu de la bâtir,
${}^{20}parce qu’il n’a pas su modérer son appétit,
        il ne sauvera aucun de ses trésors.
${}^{21}Nul ne pouvait se soustraire à sa voracité,
        voilà pourquoi son bonheur ne dure pas.
${}^{22}Au comble de l’abondance, il connaît la gêne ;
        tous les malheureux portent la main sur lui.
${}^{23}Quand il est sur le point de se remplir le ventre,
        Dieu lui envoie l’ardeur de sa colère
        et la fait pleuvoir sur lui en guise de nourriture.
${}^{24}S’il fuit devant l’arme de fer,
        l’arc de bronze le transperce.
${}^{25}Quand on retire la flèche,
        qu’elle sort de son dos,
        que la pointe étincelante sort de son foie,
        sur lui passent les terreurs.
${}^{26}Toutes les ténèbres menacent ses trésors,
        un feu le dévore que nul homme n’attise,
        il ravage ce qui reste dans sa tente.
${}^{27}Les cieux révèlent son crime,
        et la terre se dresse contre lui.
${}^{28}Les biens de sa maison sont dispersés :
        grandes eaux, au jour de la colère !
${}^{29}Telle est la part que Dieu réserve à l’homme méchant,
        l’héritage que Dieu lui promet. »
      <h2 class="intertitle hmbot" id="d85e131799">13. Septième discours de Job (21)</h2>
      
         
      \bchapter{}
      \begin{verse}
${}^{1}Job prit la parole et dit :
      
         
       
${}^{2}« Écoutez, écoutez mes paroles,
        et que s’arrêtent là vos consolations.
${}^{3}Supportez que je parle à mon tour,
        et quand j’aurai parlé, tu pourras te moquer.
${}^{4}Est-ce d’un homme que je me plains ?
        Pourquoi, dès lors, ne perdrais-je point patience ?
${}^{5}Tournez-vous vers moi, soyez stupéfaits ;
        mettez la main sur la bouche.
${}^{6}Quand j’y repense, je suis effrayé
        et ma chair est saisie d’un frisson.
${}^{7}Pourquoi les méchants demeurent-ils en vie,
        et même, en vieillissant, accroissent-ils leur fortune ?
${}^{8}Ils voient leur postérité s’affermir auprès d’eux,
        et leurs rejetons sous leurs yeux.
${}^{9}Leurs maisons en paix ignorent la peur,
        la férule de Dieu les épargne.
${}^{10}Leur taureau féconde à coup sûr,
        leur vache met bas sans avorter.
${}^{11}Ils laissent courir leurs gamins comme des brebis,
        et danser leurs enfants.
${}^{12}Ils saisissent le tambourin et la cithare,
        ils se réjouissent au son de la flûte.
${}^{13}Ils achèvent leurs jours dans le bonheur,
        et descendent en paix au séjour des morts.
${}^{14}Pourtant, ils disent à Dieu : “Écarte-toi de nous ;
        nous ne désirons pas connaître tes chemins !
${}^{15}Qu’est-ce que le Puissant pour que nous le servions ?
        Quel profit avons-nous à le supplier ?”
         
${}^{16}– En fait, leur bonheur n’est pas dans leur main :
        je rejette ces pensées des méchants !
         
${}^{17}Voit-on souvent la lampe des méchants s’éteindre,
        le malheur fondre sur eux,
        et Dieu, dans sa colère, leur donner en partage des souffrances ?
${}^{18}Sont-ils comme paille au vent,
        comme la bale qu’enlève le tourbillon ?
${}^{19}Dieu réserverait-il pour leurs fils le châtiment ?
        Qu’il punisse le coupable lui-même, pour qu’il sache !
${}^{20}Que ses propres yeux voient son infortune,
        et qu’il s’abreuve à la colère du Puissant.
${}^{21}En effet, que lui importe, après lui, sa maison,
        une fois qu’est tranché le nombre de ses mois !
${}^{22}Est-ce à Dieu qu’on enseigne la science,
        alors qu’il juge les êtres célestes !
${}^{23}Tel meurt en pleine force,
        tout tranquille et paisible,
${}^{24}les flancs chargés de graisse,
        la moelle de ses os encore fraîche.
${}^{25}Tel autre meurt, l’amertume dans l’âme,
        sans avoir goûté au bonheur.
${}^{26}L’un comme l’autre, dans la poussière ils se couchent,
        et la vermine les recouvre.
${}^{27}Certes, je connais vos pensées,
        les plans que vous forgez contre moi.
${}^{28}Quand vous dites : “Où est la maison du notable,
        où est la tente qu’habitent les méchants ?”,
${}^{29}n’avez-vous pas questionné les voyageurs,
        ignorez-vous leurs témoignages ?
${}^{30}Au jour du désastre, le méchant est épargné ;
        au jour de la fureur, il en réchappe.
${}^{31}Qui lui reproche en face sa conduite,
        et ce qu’il a commis, qui le lui fait payer ?
${}^{32}Lui, on l’escorte au cimetière
        et on veille sur son tertre.
${}^{33}Douces lui sont les mottes de la vallée,
        derrière lui tout un peuple défile,
        devant lui une foule innombrable.
${}^{34}Comment pouvez-vous m’offrir d’aussi vaines consolations ?
        De vos réponses il ne reste que tromperie. »
      <h2 class="intertitle hmbot" id="d85e132174">14. Troisième discours d’Élifaz (22)</h2>
      
         
      \bchapter{}
      \begin{verse}
${}^{1}Élifaz de Témane prit la parole et dit :
      
         
       
${}^{2}« Est-ce à Dieu qu’un homme est utile ?
        Non, l’homme avisé n’est utile qu’à lui-même.
${}^{3}Qu’importe au Puissant que tu sois juste,
        que gagne-t-il si tu améliores ta conduite ?
${}^{4}Est-ce à cause de ta piété qu’il te reprend,
        qu’il vient en jugement avec toi ?
${}^{5}Ta malice n’est-elle pas considérable,
        et tes fautes sans limite ?
${}^{6}Car tu prenais indûment des gages à tes frères,
        tu dépouillais de leurs vêtements ceux qui étaient démunis.
${}^{7}Tu n’abreuvais pas d’eau l’homme altéré ;
        à l’affamé tu refusais le pain.
${}^{8}L’homme de poigne s’emparait de la terre,
        et son protégé s’y installait.
${}^{9}Tu renvoyais les veuves les mains vides,
        et tu broyais les bras des orphelins.
${}^{10}Voilà pourquoi des pièges t’environnent,
        et une terreur soudaine t’épouvante.
${}^{11}Ou bien c’est l’obscurité, tu n’y vois plus,
        et une masse d’eau te recouvre.
${}^{12}Dieu n’est-il pas là-haut dans le ciel ?
        Regarde la cime des étoiles : comme elles sont élevées !
${}^{13}Et tu disais : “Que peut savoir Dieu ?
        Peut-il juger derrière la nuée sombre ?
${}^{14}Les nuages lui forment un voile et lui cachent la vue,
        il se déplace sur le pourtour des cieux.”
${}^{15}Veux-tu donc suivre la route de jadis
        que foulèrent les hommes d’iniquité ?
${}^{16}Ils furent emportés avant le temps,
        quand un fleuve submergea leurs fondations,
${}^{17}eux qui disaient à Dieu : “Écarte-toi de nous !”
        Or, que faisait pour eux le Puissant ?
${}^{18}Il avait rempli leurs maisons de bonheur !
         
        – Je rejette, moi aussi, les pensées des méchants !
         
${}^{19}Que les justes voient et se réjouissent,
        et que l’innocent se moque d’eux :
${}^{20}“Voilà nos adversaires anéantis !
        Un feu a consumé leurs biens !”
${}^{21}Allons ! Accorde-toi avec Dieu et fais la paix ;
        ainsi te reviendra le bonheur.
${}^{22}Accueille de sa bouche l’enseignement,
        et mets ses paroles dans ton cœur.
${}^{23}Si tu reviens au Puissant,
        si tu éloignes de ta tente l’iniquité, tu seras rétabli.
${}^{24}Jette à la poussière ton or
        et aux cailloux du torrent, le métal d’Ophir.
${}^{25}Le Puissant sera ton or
        et, pour toi, des monceaux d’argent.
${}^{26}Ainsi, tu trouveras tes délices dans le Puissant,
        et vers Dieu tu élèveras ta face.
${}^{27}Tu le supplieras, il t’écoutera,
        et tu accompliras tes offrandes votives.
${}^{28}Si tu prends une décision, elle te réussira,
        et sur tes sentiers brillera la lumière.
${}^{29}Quand Dieu humilie quelqu’un,
        tu peux dire : “C’est pour son orgueil !”,
        car celui qui baisse les yeux, il le sauve.
${}^{30}Il délivrera même l’homme qui n’est pas innocent ;
        celui-ci sera délivré par la pureté de tes mains. »
      <h2 class="intertitle hmbot" id="d85e132492">15. Huitième discours de Job (23 – 24)</h2>
      
         
      \bchapter{}
      \begin{verse}
${}^{1}Job prit la parole et dit :
      
         
       
${}^{2}« Aujourd’hui encore ma plainte se révolte,
        quand de la main je retiens mon gémissement.
${}^{3}Ah ! Qui me donnera de savoir où le trouver,
        de parvenir jusqu’à sa demeure !
${}^{4}J’organiserais devant lui un procès,
        et ma bouche serait remplie d’arguments.
${}^{5}Je saurais en quels termes il me répondrait
        et je comprendrais ce qu’il me dirait.
${}^{6}Lui faudrait-il une grande force pour débattre avec moi ?
        Non, il n’aurait qu’à me prêter attention.
${}^{7}Là, un homme droit argumenterait avec lui ;
        pour toujours je serais quitte envers mon juge.
${}^{8}Mais si je vais à l’orient, il n’y est pas ;
        à l’occident, je ne l’aperçois pas ;
${}^{9}agit-il au nord ? je ne l’atteins pas ;
        se cache-t-il au midi ? je ne le vois pas.
${}^{10}Lui connaît mon chemin.
        Qu’il me passe au creuset : j’en sortirai comme l’or.
${}^{11}Mon pied s’est attaché à son pas ;
        j’ai suivi son chemin sans dévier.
${}^{12}Le précepte de ses lèvres, je ne m’en suis pas écarté ;
        au-delà de mon devoir j’ai gardé les paroles de sa bouche.
${}^{13}Lui est immuable : qui le fera changer ?
        Ce qu’il désire, il l’exécute.
${}^{14}Il accomplira son décret sur moi ;
        et de tels projets, il en a d’innombrables.
${}^{15}Voilà pourquoi, devant lui, je suis effrayé ;
        plus je réfléchis, plus j’ai peur de lui.
${}^{16}Dieu a découragé mon cœur,
        le Puissant m’a effrayé :
${}^{17}certes, je n’ai pas été anéanti face aux ténèbres,
        mais pour autant il n’a pas épargné à mon visage l’obscurité.
       
      
         
      \bchapter{}
${}^{1}Puisque les occasions favorables ne sont pas cachées au Puissant,
        pourquoi ses fidèles ne le voient-ils pas intervenir ?
${}^{2}Les méchants repoussent les bornes,
        ils conduisent au pâturage des troupeaux volés,
${}^{3}ils emmènent l’âne des orphelins,
        ils prennent en gage le bœuf de la veuve,
${}^{4}ils écartent du chemin les nécessiteux.
        Les malheureux du pays doivent se terrer ensemble.
${}^{5}Tels les ânes sauvages du désert, ils sortent pour leur ouvrage
        en quête de nourriture ;
        le pain pour leurs petits, c’est la steppe.
${}^{6}Dans les champs, ils coupent du fourrage,
        et ils grappillent la vigne du méchant.
${}^{7}La nuit, ils la passent nus, faute de vêtements,
        sans couverture dans le froid.
${}^{8}Trempés par la pluie des montagnes,
        privés d’abri, ils se blottissent contre le rocher.
${}^{9}On arrache l’orphelin du sein de sa mère
        et on réclame des gages au pauvre.
${}^{10}Ils s’en vont nus, faute de vêtements ;
        affamés, ils doivent porter des gerbes ;
${}^{11}dans les enclos des autres, ils extraient de l’huile ;
        ils foulent aux pressoirs, alors qu’ils sont assoiffés.
${}^{12}Dans la ville, les gens se lamentent ;
        les blessés, dans un souffle, appellent à l’aide ;
        mais Dieu ne prête pas attention à la prière !
${}^{13}Quant aux méchants, ils se rebellent contre la lumière,
        ils n’en reconnaissent pas les chemins
        et n’en fréquentent pas les sentiers.
${}^{14}Le meurtrier se lève au point du jour,
        il assassine le pauvre et l’indigent,
        et, la nuit, il se fait voleur.
${}^{15}L’œil de l’adultère guette le crépuscule ;
        « Personne ne me verra », dit-il,
        et il se met un masque sur le visage.
${}^{16}Un autre, dans l’obscurité, force les maisons.
        Le jour, ils se tiennent claquemurés,
        ils ne connaissent pas la lumière.
${}^{17}Car pour eux tous, l’ombre de mort est clair matin,
        accoutumés qu’ils sont aux terreurs de cette ombre.
        
           
         
${}^{18}Ils sont emportés à la surface des eaux,
        leur part est maudite dans le pays,
        ils ne prennent plus le chemin des vignes.
${}^{19}Comme la chaleur et l’aridité absorbent l’eau des neiges,
        le séjour des morts engloutit les pécheurs.
${}^{20}Le sein maternel les oublie,
        la vermine fait d’eux ses délices,
        personne ne garde leur souvenir.
        La perfidie est brisée comme un arbre.
${}^{21}Ils maltraitent la femme stérile,
        parce qu’elle ne donne pas d’enfant ;
        ils ne veillent pas au bien-être de la veuve.
${}^{22}Dieu, par sa force, fait durer les puissants,
        mais quand il se dresse pour juger,
        l’homme n’est plus sûr de vivre.
${}^{23}S’il leur accorde la confiance pour appui,
        il garde pourtant les yeux sur leur conduite :
${}^{24}élevés pour un temps, ils ne sont plus ;
        rabaissés, ils sont moissonnés comme tous les hommes
        et se fanent comme la tête d’un épi.
${}^{25}N’en est-il pas ainsi ? Qui me démentira ?
        Qui réduira mes paroles à néant ? »
        
           
      <h2 class="intertitle hmbot" id="d85e133007">16. Troisième discours de Bildad (25)</h2>
      
         
      \bchapter{}
      \begin{verse}
${}^{1}Bildad de Shouah prit la parole et dit :
      
         
       
${}^{2}« À lui l’empire et la terreur,
        lui qui établit la paix dans ses hauteurs.
${}^{3}Peut-on dénombrer ses légions,
        et sur qui sa lumière ne se lève-t-elle pas ?
${}^{4}Comment le mortel pourrait-il avoir raison contre Dieu,
        comment serait-il pur, l’enfant de la femme ?
${}^{5}Si même la lune perd son éclat,
        si les étoiles ne sont pas pures à ses yeux,
${}^{6}que dire du mortel, ce ver,
        du fils d’homme, ce vermisseau ! »
      <h2 class="intertitle hmbot" id="d85e133065">17. Neuvième discours de Job (26)</h2>
      
         
      \bchapter{}
      \begin{verse}
${}^{1}Job prit la parole et dit :
      
         
       
${}^{2}« Comme tu assistes celui qui est sans force
        et secours le bras sans vigueur !
${}^{3}Comme tu conseilles qui n’a pas de sagesse
        et dispenses largement le savoir-faire !
${}^{4}À qui adresses-tu des paroles,
        et qui t’inspire ce qui sort de toi ?
         
${}^{5}Les ombres tremblent
        au-dessous des eaux et de leurs habitants.
${}^{6}Le séjour des morts est à nu devant lui,
        et l’abîme est sans voile.
${}^{7}Il étend les espaces du nord au-dessus du chaos,
        suspend la terre sur le vide.
${}^{8}Il enserre les eaux dans ses nuages,
        sans que la nuée crève sous leur poids.
${}^{9}Il dérobe la vue de son trône
        en déployant sur lui sa nuée.
${}^{10}Il a tracé un cercle sur la face des eaux,
        à la limite de la lumière et des ténèbres.
${}^{11}Les colonnes du ciel vacillent,
        épouvantées, à sa menace.
${}^{12}Par sa force il a dompté la mer
        et, par son intelligence, écrasé le Monstre marin.
${}^{13}Par son souffle il a rendu le ciel serein,
        sa main a transpercé le Serpent fuyard.
${}^{14}Tels sont les contours de ses œuvres :
        nous n’en percevons qu’un simple murmure,
        mais le tonnerre de sa puissance, qui le comprendra ? »
      <h2 class="intertitle hmbot" id="d85e133211">18. Dixième discours de Job (27 – 28)</h2>
      
         
      \bchapter{}
      \begin{verse}
${}^{1}Job reprit le fil de son propos et dit :
      
         
       
${}^{2}« Par la vie de Dieu qui a récusé mon droit,
        par le Puissant qui m’a rempli d’amertume,
${}^{3}tant que la respiration sera en moi,
        et le souffle de Dieu dans mes narines,
${}^{4}mes lèvres ne vont pas dire de paroles injustes,
        ni ma langue murmurer la fausseté.
${}^{5}Loin de moi la pensée de vous donner raison !
        Tant que je vivrai, je ne renoncerai pas à mon intégrité.
${}^{6}Je tiens à ma justice, et n’en démordrai pas ;
        mon cœur ne condamne aucun de mes jours.
${}^{7}Que mon ennemi ait le sort du méchant,
        et mon adversaire celui de l’injuste !
${}^{8}Car quel sera l’espoir de l’impie quand Dieu le retranchera,
        quand il ravira son âme ?
${}^{9}Dieu entendra-t-il son cri,
        quand fondra sur lui la détresse ?
${}^{10}Dans le Puissant trouvait-il ses délices,
        invoquait-il Dieu en tout temps ?
${}^{11}Je vous enseignerai la manière divine,
        je ne vous cacherai pas la pensée du Puissant.
${}^{12}Si tous, vous avez vu ce qu’il en est,
        pourquoi tenir vainement de si vains discours ?
         
${}^{13}Voici la part que le méchant trouve auprès de Dieu,
        l’héritage que les tyrans reçoivent du Puissant.
${}^{14}Si ses enfants se multiplient, le glaive les attend ;
        ses descendants ne pourront se rassasier de pain.
${}^{15}Ses survivants n’auront que la Mort pour les ensevelir,
        sans que ses veuves puissent pleurer.
${}^{16}S’il amasse l’argent comme poussière,
        s’il empile des vêtements comme du limon,
${}^{17}qu’il empile ! c’est le juste qui s’en vêtira,
        et l’argent, c’est l’innocent qui l’aura en partage.
${}^{18}La maison qu’il construit est comme celle de la mite,
        comme la hutte bâtie par le gardien.
${}^{19}Riche il se couche, mais c’est la fin ;
        il ouvre les yeux : il n’est plus.
${}^{20}Les terreurs l’assaillent comme les flots ;
        la nuit, l’ouragan l’emporte.
${}^{21}Soulevé par le vent d’est, il s’en va,
        un tourbillon le chasse loin de sa demeure.
${}^{22}Sans pitié, on tire sur lui,
        il cherche à fuir la main qui le frappe.
${}^{23}On applaudit à ce qui lui arrive,
        il quitte sa demeure sous les sifflets.
       
      
         
      \bchapter{}
${}^{1}Certes, il y a une mine pour l’argent,
        un lieu pour l’or que l’on affine.
${}^{2}Le fer est tiré du sol,
        et le cuivre s’obtient d’une pierre fondue.
${}^{3}On met fin aux ténèbres,
        jusqu’au tréfonds on fouille la pierre obscure et sombre.
${}^{4}On creuse une galerie à l’écart des habitants.
        Ignorés des passants, les mineurs sont suspendus ;
        loin de tout être humain, ils oscillent.
${}^{5}La terre d’où sort le pain
        est bouleversée en ses entrailles comme par un feu.
${}^{6}Ses pierres recèlent des saphirs
        et l’on y voit des poussières d’or.
${}^{7}Sentier qu’ignore le rapace,
        que l’œil du vautour n’a pas aperçu.
${}^{8}Les fauves orgueilleux ne l’ont pas foulé,
        le lion n’y est jamais passé.
${}^{9}Sur le silex le mineur a porté la main,
        il a bouleversé les montagnes par la racine.
${}^{10}Dans les rochers il a percé des galeries,
        et tout ce qui est précieux, son œil l’a vu.
${}^{11}Il a colmaté les suintements des fleuves,
        et amené au jour ce qui était caché.
        
           
         
${}^{12}Mais la Sagesse, où la trouver ?
        L’Intelligence, quel est son lieu ?
${}^{13}L’homme n’en connaît pas la valeur,
        elle ne se trouve pas sur la terre des vivants.
${}^{14}L’Abîme a dit : “Elle n’est pas en moi.”
        Et la Mer a déclaré : “Elle n’est pas chez moi.”
${}^{15}On ne peut l’échanger contre de l’or massif,
        ni peser l’argent pour son prix.
${}^{16}L’or d’Ophir ne saurait la payer,
        ni la cornaline précieuse, ni le saphir.
${}^{17}Même l’or et le verre ne peuvent l’égaler ;
        on ne l’obtiendrait pas contre un vase d’or fin.
${}^{18}Corail et cristal, n’en parlons pas !
        Mieux vaut recueillir la Sagesse que les perles !
${}^{19}La topaze de Nubie ne l’égale pas,
        et l’or pur ne saurait la payer.
        
           
         
${}^{20}Mais la Sagesse, où la trouver ?
        L’Intelligence, quel est son lieu ?
${}^{21}Elle a été cachée aux yeux de tout vivant,
        et dissimulée à l’oiseau du ciel.
${}^{22}L’Abîme et la Mort ont dit :
        “Nos oreilles ont perçu sa renommée.”
${}^{23}Dieu en a discerné le chemin ;
        il a su, lui, où elle était.
${}^{24}Lorsque du regard il atteignait les confins de la terre
        et voyait partout sous les cieux,
${}^{25}pour régler le poids du vent
        et fixer la mesure des eaux,
${}^{26}lorsqu’à la pluie il assignait sa limite,
        et son chemin au nuage qui tonne,
${}^{27}c’est alors qu’il la vit et l’évalua,
        qu’il l’établit et même l’explora.
${}^{28}Puis il dit à l’homme :
        “La crainte du Seigneur, voilà la Sagesse,
        s’éloigner du mal, voilà l’Intelligence.” »
        
           
      <h2 class="intertitle hmbot" id="d85e133698">19. Onzième discours de Job (29 – 31)</h2>
      
         
      \bchapter{}
      \begin{verse}
${}^{1}Job reprit le fil de son propos et dit :
      
         
       
${}^{2}« Ah, qui me rendra tel que j’étais au temps jadis,
        aux jours où Dieu me tenait en sa garde,
${}^{3}lorsqu’il faisait briller sa lampe sur ma tête
        et que dans la ténèbre je marchais à sa lumière,
${}^{4}tel que j’étais à l’automne de mes jours,
        quand Dieu était le familier de ma demeure,
${}^{5}quand le Puissant était encore avec moi,
        et que mes garçons m’entouraient,
${}^{6}quand je lavais mes pieds dans le lait
        et que le rocher près de moi ruisselait d’huile à flots !
${}^{7}Lorsque je sortais aux portes de la cité
        et que sur la place j’installais mon siège,
${}^{8}à ma vue les jeunes gens s’esquivaient,
        les vieillards se levaient et restaient debout.
${}^{9}Les notables retenaient leurs paroles
        et mettaient la main sur leur bouche.
${}^{10}La voix des chefs s’atténuait,
        la langue leur collait au palais.
${}^{11}L’oreille qui m’entendait me proclamait heureux,
        et l’œil qui me voyait me rendait témoignage.
${}^{12}Car je délivrais le pauvre qui appelait,
        l’orphelin et l’homme sans recours.
${}^{13}La bénédiction du mourant venait sur moi
        et je faisais crier de joie le cœur de la veuve.
${}^{14}Je revêtais la justice, c’était mon vêtement ;
        mon droit me servait de manteau et de turban.
${}^{15}J’étais les yeux de l’aveugle
        et les pieds du boiteux.
${}^{16}Pour les indigents, j’étais un père ;
        la cause d’un inconnu, je l’étudiais à fond.
${}^{17}Je brisais les crocs de l’injuste,
        de ses dents j’arrachais la proie.
${}^{18}Et je disais : “Je mourrai dans mon nid,
        comme le phénix je multiplierai mes jours.
${}^{19}Vers les eaux mes racines s’étirent,
        la rosée se dépose la nuit sur mes rameaux.
${}^{20}Ma gloire sera en moi toujours neuve,
        mon arc dans ma main se retendra sans cesse.”
         
${}^{21}Les gens m’écoutaient, ils attendaient,
        ils accueillaient en silence mes avis.
${}^{22}Quand j’avais parlé, nul ne répliquait ;
        sur eux, goutte à goutte, tombait ma parole.
${}^{23}Ils m’attendaient comme la pluie,
        ils ouvraient leur bouche à l’ondée de printemps.
${}^{24}Si je leur souriais, ils n’osaient y croire,
        et la lumière de mon visage, ils n’en laissaient rien perdre.
${}^{25}Je choisissais leur route et siégeais à leur tête,
        je m’installais tel un roi dans la troupe
        quand il console les affligés.
       
      
         
      \bchapter{}
${}^{1}Et maintenant, je suis la risée de plus jeunes que moi,
        dont je méprisais trop les pères
        pour les mettre avec les chiens de mon troupeau.
${}^{2}Même la force de leurs mains, à quoi m’eût-elle servi ?
        Toute énergie en eux avait péri.
${}^{3}Épuisés par la disette et la famine,
        ils rongeaient la steppe,
        crépuscule de malheur et de désolation.
${}^{4}Ils cueillaient sur les buissons une herbe au goût de sel ;
        racine de genêts : c’était là tout leur pain !
${}^{5}On les bannissait de la société,
        on criait sur eux comme sur un voleur.
${}^{6}Ils faisaient leur logis dans le creux des ravins,
        les cavités du sol et les rochers.
${}^{7}Ils vociféraient au milieu des buissons ;
        sous les chardons, ils s’entassaient.
${}^{8}Fils d’insensé, pire : fils d’un homme sans nom,
        ils étaient expulsés du pays !
        
           
         
${}^{9}Je suis maintenant leur chanson,
        et ils parlent sur moi.
${}^{10}Ils m’ont en horreur et prennent leur distance,
        à mon visage ils n’épargnent pas le crachat.
${}^{11}Parce que Dieu a relâché la corde de mon arc et m’a humilié,
        eux, devant moi, perdent toute retenue.
${}^{12}À ma droite surgit la canaille ;
        ils me font lâcher pied ;
        ils élèvent contre moi leurs rampes de malheur.
${}^{13}Ils détruisent mon sentier
        et s’affairent à ma ruine,
        sans avoir besoin d’aide.
${}^{14}Ils arrivent comme par une large brèche,
        sous les décombres ils se bousculent.
${}^{15}Les terreurs se tournent contre moi.
        Ma dignité est emportée par le vent,
        mon salut est passé comme nuage !
        
           
         
${}^{16}Et maintenant mon âme en moi s’épanche ;
        des jours d’affliction m’ont saisi.
${}^{17}La nuit transperce mes os,
        et ce qui me ronge n’a pas de répit.
${}^{18}Avec une grande violence Dieu saisit mon vêtement,
        il me serre au col de ma tunique.
${}^{19}Il m’a jeté dans la fange :
        me voici pareil à la poussière et à la cendre.
${}^{20}Vers toi je crie, et tu ne réponds pas ;
        je me tiens devant toi, et tu me fixes du regard !
${}^{21}Tu es devenu cruel pour moi,
        de ta poigne vigoureuse tu t’acharnes sur moi.
${}^{22}Tu m’emportes sur le vent, tu m’y fais chevaucher,
        tu me dissous dans l’orage.
${}^{23}Oui, je le sais, tu me ramènes à la mort,
        au rendez-vous de tout vivant.
${}^{24}Pourtant on ne porte pas la main sur celui qui s’effondre,
        si, dans son malheur, il crie.
${}^{25}N’ai-je pas pleuré sur l’homme à la vie dure ?
        Mon âme ne s’est-elle pas émue sur l’indigent ?
${}^{26}J’espérais le bonheur, et le malheur survient ;
        j’attendais la lumière, et vient l’obscurité !
${}^{27}Mes entrailles bouillonnent sans repos ;
        des jours d’affliction viennent à ma rencontre.
${}^{28}Je marche, assombri, sans soleil ;
        je me lève dans l’assemblée et je crie.
${}^{29}Me voici devenu le frère des chacals,
        le compagnon des autruches.
${}^{30}Ma peau a noirci sur moi,
        mes os brûlent de fièvre.
${}^{31}Ma cithare sert à la plainte,
        et ma flûte à la voix des pleureurs.
        
           
       
      
         
      \bchapter{}
${}^{1}J’avais conclu un pacte avec mes yeux :
        comment, alors, aurais-je fixé du regard une jeune fille vierge ?
${}^{2}Quel est donc le sort que de là-haut Dieu assigne ?
        Quelle part le Puissant réserve-t-il depuis les hauteurs célestes ?
${}^{3}N’est-ce pas le malheur pour l’injuste,
        et l’infortune pour les artisans du mal ?
${}^{4}Ne voit-il pas mes chemins,
        de toutes mes démarches ne fait-il point le compte ?
${}^{5}Si j’ai fait route avec le mensonge,
        si j’ai hâté le pas vers la fausseté,
${}^{6}qu’il me pèse sur une juste balance !
        Dieu reconnaîtra mon intégrité.
${}^{7}Si mon pas a dévié du chemin,
        si mon cœur a suivi mes yeux
        et si une tache me colle aux mains,
${}^{8}qu’un autre mange ce que je sème,
        et que soient déracinées mes jeunes pousses !
${}^{9}Si mon cœur a été séduit par une femme
        et si j’ai guetté à la porte du voisin,
${}^{10}que ma femme tourne la meule pour autrui
        et que d’autres la possèdent !
${}^{11}Car c’est une infamie,
        une faute relevant des juges ;
${}^{12}oui, c’est un feu qui dévore jusqu’à l’abîme,
        capable de détruire à la racine toute ma récolte.
${}^{13}Si j’ai méprisé le droit de mon serviteur ou de ma servante
        en litige avec moi,
${}^{14}que ferai-je quand Dieu se lèvera ?
        quand il enquêtera, quelle sera ma réponse ?
${}^{15}Ne les a-t-il pas formés dans le ventre tout comme moi ?
        N’est-ce pas le même qui nous a façonnés
        dans le sein maternel ?
${}^{16}Ai-je repoussé les désirs des pauvres,
        ai-je laissé s’éteindre le regard de la veuve ?
${}^{17}Ai-je mangé seul mon morceau de pain,
        sans que l’orphelin en mange aussi ?
${}^{18}Au contraire, dès ma jeunesse, il a grandi avec moi
        comme avec un père ;
        dès mon enfance, j’étais le guide de la veuve.
${}^{19}Si je voyais un miséreux sans vêtements,
        un indigent sans rien sur le dos,
${}^{20}est-ce que ses reins ne me bénissaient pas,
        réchauffés par la toison de mes agneaux ?
${}^{21}Si contre l’orphelin j’ai brandi la main,
        parce que je me voyais soutenu par les notables,
${}^{22}que mon épaule tombe de ma nuque,
        que mon bras se brise au coude !
${}^{23}Car le châtiment de Dieu serait ma terreur ;
        devant sa majesté je ne pourrais tenir.
${}^{24}Ai-je fait de l’or mon appui,
        ai-je dit à l’or pur : “C’est toi ma confiance” ?
${}^{25}Me suis-je réjoui de posséder de grandes richesses,
        d’avoir mis la main sur une fortune ?
${}^{26}À la vue de la lumière dans son éclat,
        de la lune splendide en sa marche,
${}^{27}mon cœur, secrètement, a-t-il été séduit ?
        la main à la bouche, leur ai-je fait un baiser ?
${}^{28}Cela aussi serait une faute relevant du juge,
        car j’aurais renié le Dieu d’en haut.
${}^{29}Me suis-je réjoui de la ruine de mon ennemi ?
        Ai-je bondi de joie quand le malheur le frappait ?
${}^{30}Jamais je n’ai laissé ma langue pécher
        en réclamant sa vie par une imprécation !
${}^{31}Ceux qui logeaient sous ma tente le disaient bien :
        “Qui n’a-t-il pas rassasié de viande ?”
${}^{32}Jamais un étranger ne passait la nuit dehors,
        ma porte restait ouverte au voyageur.
${}^{33}Comme tout un chacun ai-je dissimulé mes transgressions,
        en cachant ma faute dans un repli de ma tunique,
${}^{34}parce que je craignais la rumeur de la foule
        et que me terrifiait le mépris des familles,
        au point de rester figé au seuil de ma maison ?
        
           
         
${}^{35}Si j’avais seulement quelqu’un pour m’écouter !
        Voilà mon dernier mot. Que le Puissant me réponde !
        Que la partie adverse rédige son mémoire !
${}^{36}Je le porterai sur l’épaule,
        comme un diadème je le ceindrai.
${}^{37}Je rendrai compte au Puissant du nombre de mes pas ;
        tel un prince, je m’avancerai vers lui.
${}^{38}Si ma terre crie contre moi,
        si ensemble pleurent ses sillons,
${}^{39}si je mange ses fruits sans donner d’argent,
        si à ses métayers je fais rendre l’âme,
${}^{40}qu’au lieu de blé pousse la ronce,
        et à la place de l’orge, l’herbe fétide ! »
        
           
         
        \\Fin des paroles de Job.
        
           
      
         
      \bchapter{}
      \begin{verse}
${}^{1}Ces trois hommes cessèrent de répondre à Job, puisqu’il était juste à ses propres yeux. 
${}^{2}Alors s’enflamma la colère d’Élihou, fils de Barakéel, le Bouzite, du clan de Ram. Envers Job s’enflamma sa colère, parce qu’il prétendait avoir raison contre Dieu. 
${}^{3}Sa colère s’enflamma envers ses trois amis parce que ceux-ci n’avaient pas trouvé de réponse pour donner tort à Job. 
${}^{4}Élihou avait attendu pour s’adresser à Job, parce qu’ils étaient plus âgés que lui. 
${}^{5}Mais quand Élihou vit que ces trois hommes n’avaient plus de réponse à la bouche, sa colère s’enflamma.
      
         
      <h2 class="intertitle hmbot" id="d85e134779">1. Premier discours d’Élihou (32,6 – 33)</h2>
${}^{6}Élihou, fils de Barakéel, le Bouzite, prit la parole et dit :
       
        \\« Je suis jeune, moi,
        et vous êtes des anciens.
        \\C’est pourquoi, intimidé,
        je craignais de vous manifester mon savoir.
${}^{7}Je me disais : “Il faut que l’âge parle
        et que le nombre des années fasse connaître la sagesse !”
${}^{8}En réalité, c’est l’esprit dans l’homme,
        le souffle du Puissant, qui le rend intelligent.
${}^{9}Les plus âgés ne sont pas les plus sages,
        ce ne sont pas les vieillards qui discernent le droit.
${}^{10}C’est pourquoi je dis : “Écoute-moi,
        je veux, moi aussi, manifester mon savoir.”
${}^{11}Voici : je comptais sur vos paroles,
        je prêtais l’oreille à vos raisonnements,
        tandis que vous cherchiez des mots.
${}^{12}Sur vous je fixais mon attention,
        et voici que nul n’a réfuté Job,
        aucun de vous n’a répondu à ses déclarations.
${}^{13}N’allez pas dire : “Nous avons trouvé la sagesse :
        Dieu seul le confondra, non un homme.”
${}^{14}Ce n’est pas contre moi qu’il alignait les mots
        et ce n’est pas avec vos paroles que je lui répliquerai.
         
${}^{15}Stupéfaits, ils n’ont plus répondu,
        les mots leur ont manqué !
${}^{16}Vais-je attendre, puisqu’ils ne parlent pas,
        se sont arrêtés et ne répondent plus ?
${}^{17}Je répondrai, pour ma part, moi aussi ;
        je manifesterai, moi aussi, mon savoir.
${}^{18}Car je suis rempli de paroles,
        un souffle intérieur me contraint.
${}^{19}C’est en moi comme un vin sous pression,
        comme dans des outres neuves qui vont éclater.
${}^{20}Parler me soulagera,
        j’ouvrirai les lèvres et je répondrai !
${}^{21}Je ne prendrai le parti d’aucun,
        et je ne flatterai personne.
${}^{22}Je ne sais pas flatter :
        en un rien de temps, mon Créateur m’emporterait.
       
      
         
      \bchapter{}
${}^{1}Je t’en prie, Job, écoute donc mes discours,
        à toutes mes paroles prête l’oreille.
${}^{2}Voici que j’ouvre la bouche,
        ma langue forme des mots dans mon palais.
${}^{3}C’est la droiture de mon cœur que j’exprime,
        et mes lèvres disent clairement ce que je sais.
${}^{4}L’esprit de Dieu m’a créé,
        le souffle du Puissant me fait vivre.
${}^{5}Si tu le peux, réplique-moi !
        Argumente devant moi, prends position !
${}^{6}Vois, pour Dieu je suis ton égal ;
        d’argile j’ai été façonné, moi aussi.
${}^{7}Ainsi, tu n’auras de moi ni terreur ni épouvante,
        et ma main ne pèsera pas sur toi.
        
           
         
${}^{8}Mais tu as dit à mes oreilles
        – et j’entends le son de tes paroles :
${}^{9}“Je suis pur, sans péché,
        je suis net, et en moi pas de faute.
${}^{10}Or Dieu invente des griefs contre moi ;
        il me tient pour son ennemi.
${}^{11}Il fixe mes pieds dans des blocs de bois,
        il observe toutes mes démarches !”
        
           
         
${}^{12}Eh bien ! te répondrai-je, en cela tu n’as pas raison,
        car Dieu est plus grand que l’homme.
${}^{13}Pourquoi lui cherches-tu querelle
        sous prétexte qu’il ne rend compte d’aucun de ses actes ?
${}^{14}C’est que Dieu parle une fois, deux fois,
        sans que l’on y prenne garde.
${}^{15}Dans un songe, une vision nocturne,
        quand tombe une torpeur sur les hommes
        et qu’ils sont assoupis sur leur lit,
${}^{16}alors, il leur ouvre l’oreille
        et leur adresse des sommations,
${}^{17}pour détourner l’être humain de ses œuvres,
        et pour prémunir le héros de l’orgueil.
${}^{18}Ainsi il préserve son âme de la fosse,
        sa vie, du passage au chenal de la mort.
        
           
         
${}^{19}Sur son lit, l’homme est corrigé par la douleur,
        quand ses os ne cessent de s’entrechoquer.
${}^{20}Sa vie lui donne le dégoût du pain,
        il perd l’appétit pour les mets délicats.
${}^{21}Sa chair dépérit à vue d’œil
        et ses os qu’on ne voyait pas deviennent saillants.
${}^{22}Son âme approche de la fosse,
        et sa vie, des exterminateurs.
${}^{23}S’il y a près de lui un ange,
        un interprète, un seul entre mille,
        pour signifier à l’homme son devoir,
${}^{24}s’il le prend en grâce et demande à Dieu :
        “Exempte-le de descendre dans la fosse,
        j’ai trouvé une rançon pour sa vie”,
${}^{25}alors sa chair retrouve la fraîcheur de la jeunesse,
        il revient aux jours de son adolescence ;
${}^{26}il implore Dieu et Dieu se plaît en lui,
        avec allégresse l’homme voit la face
        de celui qui le restaure en sa justice.
${}^{27}Il chante devant les hommes en disant :
        “J’avais péché, j’avais perverti le droit,
        et je n’ai pas eu ce que je méritais !
${}^{28}Il a épargné à mon âme de passer par la fosse,
        et mon être contemple la lumière !”
        
           
         
${}^{29}Voilà tout ce que fait Dieu,
        deux fois, trois fois, à l’égard de l’homme,
${}^{30}pour ramener son âme de la fosse,
        pour l’illuminer de la lumière des vivants.
${}^{31}Sois attentif, Job, écoute-moi,
        tais-toi, c’est moi qui parlerai.
${}^{32}Si tu trouves des mots, réplique-moi,
        parle, car je voudrais te donner raison.
${}^{33}Sinon, toi, écoute-moi ;
        fais silence, que je t’enseigne la sagesse ! »
        
           
      <h2 class="intertitle hmbot" id="d85e135356">2. Deuxième discours d’Élihou (34)</h2>
      
         
      \bchapter{}
      \begin{verse}
${}^{1}Élihou prit la parole et dit :
      
         
       
${}^{2}« Sages, écoutez mes paroles ;
        savants, prêtez-moi l’oreille.
${}^{3}Car l’oreille apprécie les discours
        comme le palais goûte la nourriture.
${}^{4}Mettons-nous en quête du droit,
        reconnaissons entre nous ce qui est bon.
${}^{5}Job a déclaré : “J’ai raison,
        mais Dieu a écarté mon droit.
${}^{6}Malgré mon bon droit je passe pour menteur ;
        une flèche est en moi, blessure incurable,
        sans que j’aie péché.”
${}^{7}Y a-t-il un homme comme Job ?
        Il boit le sarcasme comme de l’eau.
${}^{8}Il chemine en compagnie des malfaiteurs
        et fait route avec les hommes de méchanceté.
${}^{9}Car il a dit : “L’homme ne gagne rien
        à mettre sa joie en Dieu.”
${}^{10}Aussi, écoutez-moi, hommes sensés :
        loin de Dieu la méchanceté,
        loin du Puissant l’injustice !
${}^{11}Il rend à l’homme selon ses actes
        et traite chacun d’après sa conduite.
${}^{12}Non, certes, Dieu ne fait pas le mal,
        le Puissant ne fausse pas le droit.
         
${}^{13}Qui donc lui a confié la terre ?
        Qui l’a chargé du monde entier ?
${}^{14}S’il ne pensait qu’à lui-même,
        s’il concentrait en lui son esprit et son souffle,
${}^{15}toute chair expirerait à la fois
        et l’homme retournerait à la poussière.
${}^{16}Si tu as de l’intelligence, écoute ceci,
        prête l’oreille au son de mes paroles !
${}^{17}S’il détestait le droit, pourrait-il gouverner ?
        Condamneras-tu le Juste, le Puissant ?
${}^{18}Dit-on au roi : “Vaurien !”
        et aux notables : “Criminels !” ?
${}^{19}Dieu, lui, ne prend pas le parti des princes,
        ne reconnaît pas plus le nanti que le faible,
        car tous sont l’œuvre de ses mains.
${}^{20}En un instant, les princes meurent, même au milieu de la nuit,
        le peuple s’agite et ils disparaissent,
        on écarte un tyran sans effort.
${}^{21}Car Dieu a les yeux sur les chemins de l’homme,
        il voit tous ses pas.
${}^{22}Ni ténèbres ni ombre de mort
        où puissent se cacher les malfaiteurs.
${}^{23}Il n’a pas besoin d’observer longtemps quelqu’un
        pour le faire venir devant lui en jugement.
${}^{24}Sans enquête, il brise les puissants
        et met d’autres hommes à leur place.
${}^{25}C’est qu’il démasque leurs manœuvres ;
        il les renverse dans la nuit, ils sont écrasés.
${}^{26}Tels des criminels, il les gifle
        dans un lieu bien en vue,
${}^{27}parce qu’ils se sont détournés de lui,
        qu’ils ont méconnu tous ses chemins,
${}^{28}au point de faire monter vers lui le cri du faible,
        le cri des pauvres qu’il entend.
         
${}^{29}Même s’il reste inactif, qui le condamnera,
        s’il cache sa face, qui l’apercevra ?
        Il veille pourtant sur les nations et sur l’individu,
${}^{30}pour que ne règne aucun impie,
        qu’aucun piège ne soit tendu au peuple.
         
${}^{31}Supposons que l’on dise à Dieu :
        “J’ai expié, je ne ferai plus le mal ;
${}^{32}ce qui échappe à ma vue, toi, montre-le-moi !
        si j’ai commis l’injustice, je ne continuerai pas.”
         
${}^{33}Dieu devrait-il alors punir selon tes convictions ?
        Puisque tu as critiqué,
        puisque c’est toi qui décides et non moi,
        ce que tu sais, dis-le !
${}^{34}Les gens sensés me diront,
        ainsi que tout homme sage qui m’écoute :
${}^{35}“Job parle sans savoir,
        et ses propos ne sont pas raisonnables.”
         
${}^{36}Ah ! si seulement Job était examiné jusqu’au bout
        pour ses réponses dignes d’un mécréant !
${}^{37}Car à sa faute il ajoute la révolte,
        il s’applaudit lui-même au milieu de nous,
        il multiplie ses discours contre Dieu. »
      <h2 class="intertitle hmbot" id="d85e135781">3. Troisième discours d’Élihou (35)</h2>
      
         
      \bchapter{}
      \begin{verse}
${}^{1}Élihou prit la parole et dit :
      
         
       
${}^{2}« Penses-tu être dans ton droit
        quand tu déclares : “J’ai raison contre Dieu”,
${}^{3}quand tu lui dis : “Toi, que t’importe,
        et pour moi quel profit, si je pèche ou non” ?
${}^{4}Moi, je te répliquerai,
        en même temps qu’à tes amis.
${}^{5}Considère les cieux, et vois ;
        regarde les nuées : elles sont plus hautes que toi.
${}^{6}Si tu pèches, en quoi vas-tu l’atteindre ?
        Et si tu multiplies les offenses, que lui fais-tu ?
${}^{7}Si tu es juste, que lui donnes-tu,
        ou que reçoit-il de ta main ?
${}^{8}Ta méchanceté ne touche que tes semblables,
        et ta justice, des fils d’homme.
         
${}^{9}Sous le poids de l’oppression ils gémissent,
        sous la violence des grands ils crient.
${}^{10}Mais on ne dit pas : “Où est Dieu qui m’a fait,
        qui inspire des hymnes dans la nuit,
${}^{11}qui nous instruit plus que les bêtes de la terre
        et nous rend plus sages que les oiseaux du ciel ?”
${}^{12}Dès lors, on crie, et lui ne répond pas,
        à cause de l’orgueil des malfaisants.
${}^{13}Assurément, ce qui est illusoire, Dieu ne l’écoute pas,
        le Puissant n’y prête pas attention.
${}^{14}Encore moins, quand tu dis : “Je ne l’aperçois pas,
        mon procès est ouvert devant lui et je l’attends.”
${}^{15}Et maintenant tu dis : “Sa colère ne châtie pas,
        il ne tient guère compte de l’arrogance.”
         
${}^{16}C’est en vain que Job ouvre la bouche,
        par ignorance il accumule des mots. »
      <h2 class="intertitle hmbot" id="d85e135963">4. Quatrième discours d’Élihou (36 – 37)</h2>
      
         
      \bchapter{}
      \begin{verse}
${}^{1}Élihou continua et dit :
      
         
       
${}^{2}« Patiente un peu avec moi, je vais t’instruire,
        car il y a d’autres choses à dire en faveur de Dieu.
${}^{3}J’irai chercher ma science au loin
        pour donner raison à celui qui m’a fait.
${}^{4}Car, en vérité, mes paroles ne sont pas mensonge,
        c’est un parfait connaisseur que tu as devant toi.
         
${}^{5}Vois : Dieu est puissant, il ne méprise pas,
        il est puissant et d’un cœur magnanime.
${}^{6}Il ne laisse pas vivre le méchant
        mais rend justice aux pauvres.
${}^{7}Il ne détourne pas ses yeux des justes ;
        à l’instar des rois sur le trône,
        il les fait siéger pour toujours,
        mais ils s’enorgueillissent.
${}^{8}Et s’ils se retrouvent prisonniers des chaînes,
        pris dans les liens de la misère,
${}^{9}Dieu leur montre leurs œuvres
        et leurs péchés commis par orgueil.
${}^{10}Il leur ouvre l’oreille pour les avertir
        et leur ordonne de se détourner du mal.
${}^{11}S’ils écoutent et se mettent à son service,
        leurs jours s’achèveront dans le bonheur,
        et leurs années dans les délices.
${}^{12}Mais s’ils n’écoutent pas, ils passent par le chenal de la mort
        et ils périssent faute d’intelligence.
${}^{13}Quant aux impies qui dans leur cœur se mettent en colère,
        ils ne crient pas vers Dieu lorsqu’il les enchaîne ;
${}^{14}leur âme meurt en pleine jeunesse,
        et leur vie s’achève dans la prostitution.
${}^{15}Dieu sauve le malheureux par son malheur ;
        par la détresse il lui ouvre l’oreille.
${}^{16}Toi aussi, il te fait passer de l’étreinte de l’angoisse
        à un espace où rien ne gêne,
        et la table disposée pour toi débordera de mets succulents.
         
${}^{17}Si tu dois mener à bien le jugement du méchant,
        que le jugement et le droit soient ton appui !
${}^{18}Prends garde que l’abondance ne te séduise
        et que de riches présents ne te fassent dévier.
${}^{19}Est-ce en criant que tu mettras à égalité l’homme démuni
        et tous les détenteurs de pouvoir ?
${}^{20}Ne soupire pas après la nuit
        où des peuples monteront pour prendre la place.
${}^{21}Garde-toi de te tourner vers le mal
        car c’est à cause de cela que tu as été éprouvé par le malheur.
         
${}^{22}Vois : Dieu est sublime en sa force.
        Qui enseigne comme lui ?
${}^{23}Qui lui a jamais dicté sa conduite ?
        Qui peut lui dire : “Tu as commis l’injustice ?”
${}^{24}Souviens-toi de magnifier son œuvre
        que les hommes célèbrent par des chants.
${}^{25}Tout homme la contemple,
        de loin le mortel la regarde.
         
${}^{26}Vois : Dieu est grand, au-delà de notre savoir,
        le nombre de ses années est sans mesure.
${}^{27}Il attire les gouttes d’eau,
        distille la pluie en un grand flot.
${}^{28}Les nuages en ruissellent
        et le répandent sur la foule des hommes.
${}^{29}Qui comprendra aussi les déploiements du nuage,
        les craquements de la hutte céleste ?
${}^{30}Voici qu’il s’enveloppe de sa lumière
        et couvre les racines de la mer.
${}^{31}Par les éléments il juge les peuples
        et donne la nourriture à profusion.
${}^{32}Il couvre ses deux paumes de l’éclair
        et lui désigne une cible.
${}^{33}Son tonnerre proclame sa présence,
        et la tempête, la passion de sa colère.
      
         
      \bchapter{}
${}^{1}C’est aussi pour cela que tremble mon cœur
        et qu’il bondit hors de sa place.
${}^{2}Écoutez, écoutez la vibration de sa voix,
        et le grondement qui sort de sa bouche !
${}^{3}Il le prolonge sous tous les cieux,
        et son éclair atteint les extrémités de la terre.
${}^{4}Derrière lui rugit une voix ;
        il tonne de sa voix majestueuse
        et ne retient pas les éclairs
        quand sa voix se fait entendre.
${}^{5}Dieu tonne à pleine voix : Merveilles !
        Il opère de grandes choses que nous ignorons.
        
           
         
${}^{6}Quand il dit à la neige : “Descends sur la terre”,
        à la pluie d’averse, à l’averse torrentielle : “Tombez dru”,
${}^{7}il paralyse l’activité de chaque homme
        pour que tous les humains qu’il a créés le reconnaissent.
${}^{8}La bête sauvage se retire dans son antre
        et se tapit dans ses tanières.
${}^{9}Du sud arrive l’ouragan,
        et des vents du nord, la froidure.
${}^{10}Au souffle de Dieu se forme la glace,
        et l’étendue des eaux se fige.
${}^{11}Il charge d’humidité le nuage,
        il disperse ses nuées de lumière ;
${}^{12}elles tournoient en cercles selon ses desseins,
        pour œuvrer à tout ce qu’il leur ordonne
        sur la face du monde terrestre.
${}^{13}Il en fait soit un fléau pour sa propre terre,
        soit une marque de bonté.
${}^{14}Prête l’oreille à ceci, Job,
        arrête-toi et considère les merveilles de Dieu !
${}^{15}Sais-tu comment Dieu leur commande
        et fait briller la lumière de sa nuée,
${}^{16}sais-tu comment il suspend le nuage ?
        Prodiges d’un parfait connaisseur !
${}^{17}Toi, dont les habits sont trop chauds,
        quand repose la terre au vent du midi,
${}^{18}as-tu avec lui tassé les nuées,
        durcies comme un miroir de métal fondu ?
${}^{19}Fais-nous savoir ce que nous devons lui dire :
        dans les ténèbres où nous sommes, nous manquons d’arguments.
${}^{20}Lui est-il rendu compte quand je parle ?
        Faut-il qu’un homme le dise pour qu’il soit informé ?
${}^{21}Et maintenant, on ne voit plus la lumière,
        obscurcie qu’elle est par les nuages ;
        \\mais qu’un vent passe et les dissipe,
${}^{22}du nord survient une lumière dorée.
        \\Sur Dieu, quelle redoutable splendeur !
${}^{23}Le Puissant, nous ne pouvons l’atteindre,
        il est sublime en force ;
        il ne viole pas le droit et la pleine justice.
${}^{24}C’est pourquoi les hommes le craignent ;
        il n’a pas de regard pour les prétendus sages.
        
           
      <h2 class="intertitle hmbot" id="d85e136733">1. Premier discours du Seigneur (38 – 40,2)</h2>
      
         
      \bchapter{}
      \begin{verse}
${}^{1}Le Seigneur s’adressa à Job du milieu de la tempête et dit :
      
         
       
${}^{2}« Quel est celui-là qui obscurcit mes plans
        par des propos dénués de sens ?
${}^{3}Ceins donc tes reins comme un homme.
        Je vais t’interroger, et tu m’instruiras.
${}^{4}Où étais-tu quand j’ai fondé la terre ?
        Indique-le, si tu possèdes la science !
${}^{5}Qui en a fixé les mesures ? Le sais-tu ?
        Qui sur elle a tendu le cordeau ?
${}^{6}Sur quoi ses bases furent-elles appuyées,
        et qui posa sa pierre angulaire
${}^{7}tandis que chantaient ensemble les étoiles du matin
        et que tous les fils de Dieu criaient d’allégresse ?
        ${}^{8}Qui donc a retenu la mer avec des portes,
        quand elle jaillit du sein primordial\\ ;
        ${}^{9}quand je lui mis pour vêtement la nuée,
        en guise de langes le nuage sombre ;
        ${}^{10}quand je lui imposai ma limite,
        et que je disposai verrou et portes ?
        ${}^{11}Et je dis : “Tu viendras jusqu’ici !
        tu n’iras pas plus loin,
        ici s’arrêtera l’orgueil de tes flots !”
        ${}^{12}As-tu, une seule fois dans ta vie,
        donné des ordres au matin,
        assigné son poste à l’aurore,
        ${}^{13}pour qu’elle saisisse la terre aux quatre coins
        et en secoue les méchants ?
        ${}^{14}La terre alors\\prend forme comme argile sous le sceau
        et se déploie tel un vêtement ;
        ${}^{15}aux méchants est enlevée la lumière,
        et le bras qui se levait est brisé.
         
        ${}^{16}Es-tu parvenu jusqu’aux sources de la mer,
        as-tu circulé au fond de l’abîme ?
        ${}^{17}Les portes de la mort se sont-elles montrées à toi,
        les as-tu vues, les portes de l’ombre de mort ?
        ${}^{18}As-tu réfléchi à l’immensité de la terre ?
        Raconte, si tu sais tout cela !
        ${}^{19}Quel chemin mène à la demeure de la lumière,
        et l’obscurité, quel est son lieu,
        ${}^{20}pour que tu conduises chacune à son domaine
        et discernes les sentiers de sa maison ?
        ${}^{21}Si tu le sais, alors tu étais né,
        et le nombre de tes jours est bien grand !
${}^{22}Es-tu parvenu aux réserves de neige,
        as-tu vu les réserves de grêle
${}^{23}que j’ai ménagées pour le temps de détresse,
        pour le jour de combat et de guerre ?
${}^{24}Par quel chemin se diffuse la lumière ?
        Par où le vent d’est se répand-il sur terre ?
${}^{25}Qui donc a creusé à l’ondée une rigole,
        une route à la nuée qui gronde,
${}^{26}pour faire pleuvoir sur une terre sans homme,
        sur un désert sans nul être humain,
${}^{27}pour abreuver les solitudes désolées
        et faire germer l’herbe de la steppe ?
${}^{28}La pluie a-t-elle un père ?
        Qui donc a engendré les gouttelettes de rosée ?
${}^{29}De quel ventre est sortie la glace,
        et le givre des cieux, qui l’enfanta ?
${}^{30}Les eaux se condensent comme la pierre
        et la surface de l’abîme se fige.
${}^{31}Peux-tu nouer les liens des Pléiades
        ou desserrer les cordes d’Orion,
${}^{32}faire paraître en leur temps les constellations,
        conduire la Grande Ourse avec ses petits ?
${}^{33}Connais-tu les décrets des cieux ?
        Appliques-tu leur charte sur la terre ?
${}^{34}Te suffit-il d’élever la voix vers un nuage
        pour qu’une masse d’eau te couvre ?
${}^{35}Est-ce toi qui lances les éclairs pour qu’ils partent,
        en te disant : “Nous voici” ?
${}^{36}Qui a mis dans l’ibis la sagesse,
        qui a donné au coq l’intelligence ?
${}^{37}Qui peut avec sagesse dénombrer les nuées ?
        \\Qui incline les outres des cieux
${}^{38}tandis que s’agglomère la poussière
        et que les mottes s’agglutinent ?
${}^{39}Chasses-tu pour la lionne une proie ?
        \\Peux-tu assouvir la voracité des lionceaux
${}^{40}lorsqu’ils se tapissent dans les tanières
        et se tiennent aux aguets dans le fourré ?
${}^{41}Qui prépare au corbeau sa provende,
        quand ses petits crient vers Dieu
        et titubent faute de nourriture ?
       
      
         
      \bchapter{}
${}^{1}Sais-tu quand mettent bas les chamois du rocher ?
        Peux-tu observer les biches en travail ?
${}^{2}Compteras-tu les mois de leur gestation,
        et sais-tu l’instant de leur délivrance ?
${}^{3}Elles s’accroupissent, expulsent leurs petits,
        et se libèrent de leurs douleurs.
${}^{4}Leurs petits prennent des forces,
        grandissent en pleine nature ;
        ils partent et ne reviennent plus vers elles.
        
           
         
${}^{5}Qui a lâché l’onagre en liberté ?
        \\Qui a desserré les liens de l’âne sauvage
${}^{6}auquel j’ai assigné la steppe pour maison,
        la terre salée pour demeure ?
${}^{7}Il se rit du vacarme de la cité,
        il n’entend pas les vociférations de son maître.
${}^{8}Il explore les montagnes, son pâturage,
        en quête de la moindre verdure.
        
           
         
${}^{9}Le buffle voudra-t-il te servir,
        passera-t-il la nuit à ta mangeoire ?
${}^{10}L’attacheras-tu au sillon par la bride ?
        Pourra-t-il herser derrière toi les vallons ?
${}^{11}Lui feras-tu confiance parce que sa force est grande,
        lui laisseras-tu ta besogne ?
${}^{12}Compteras-tu sur lui pour rentrer ton grain
        et l’entasser sur ton aire ?
        
           
         
${}^{13}L’aile de l’autruche bat allègrement.
        Que n’a-t-elle pennage et plumage de la fidèle cigogne !
${}^{14}Quand elle abandonne ses œufs à la terre
        et les laisse chauffer sur la poussière,
${}^{15}elle oublie qu’un pied peut les écraser,
        une bête sauvage, les fouler.
${}^{16}Dure pour ses petits, comme s’ils n’étaient pas les siens,
        elle n’a cure du mal qu’elle s’est donné en vain,
${}^{17}car Dieu lui a refusé la sagesse
        et ne lui a pas départi l’intelligence.
${}^{18}Mais sitôt qu’elle se dresse pour s’élancer,
        elle se rit du cheval et de son cavalier.
        
           
         
${}^{19}Est-ce toi qui donnes au cheval la bravoure,
        qui revêts son cou d’une crinière ?
${}^{20}Le fais-tu bondir comme la sauterelle ?
        Altier, son hennissement répand l’effroi.
${}^{21}Il piaffe dans la vallée, tout joyeux de sa force,
        il se jette au-devant de la mêlée.
${}^{22}Il se rit de la peur et ne s’effraie pas,
        il ne recule pas devant l’épée.
${}^{23}Sur lui résonnent le carquois,
        la lance étincelante et le javelot.
${}^{24}Frémissant d’impatience, il dévore l’espace,
        il ne se tient plus dès que sonne le cor.
${}^{25}Quand retentit le cor, il crie : “Héah !”
        \\De loin, il flaire la bataille,
        tonnerre des chefs, clameur de guerre.
        
           
         
${}^{26}Est-ce par ton intelligence que l’épervier prend son vol,
        qu’il déploie ses ailes vers le sud ?
${}^{27}Est-ce sur ton ordre que l’aigle s’élève
        et va nicher dans les hauteurs ?
${}^{28}Il habite un rocher et passe la nuit
        sur une dent de roc, sa forteresse.
${}^{29}De là, il guette sa proie,
        ses yeux fixent les lointains.
${}^{30}Ses petits se gorgent de sang,
        là où sont les cadavres, là il est. »
        
           
      
         
      \bchapter{}
      \begin{verse}
${}^{1}Le Seigneur s’adressa à Job et dit :
      
         
       
${}^{2}« Celui qui dispute avec le Puissant va-t-il le censurer ?
        Celui qui critique Dieu répondra-t-il à cela ? »
      <h2 class="intertitle hmbot" id="d85e137561">2. Première réponse de Job (40,3-5)</h2>
${}^{3}Job s’adressa au Seigneur et dit :
       
        ${}^{4}« Moi qui suis si peu de chose, que pourrais-je te répliquer ?
        Je mets la main sur ma bouche.
        ${}^{5}J’ai parlé une fois, je ne répondrai plus ;
        deux fois, je n’ajouterai plus rien. »
      <h2 class="intertitle hmbot" id="d85e137602">3. Deuxième discours du Seigneur (40,6 – 41)</h2>
${}^{6}Le Seigneur s’adressa à Job du milieu de la tempête et dit :
       
${}^{7}« Ceins donc tes reins comme un homme.
        Je vais t’interroger, et tu m’instruiras.
${}^{8}Veux-tu me débouter de mon droit,
        me condamner pour avoir raison ?
${}^{9}As-tu un bras comme celui de Dieu,
        et ta voix peut-elle tonner comme la sienne ?
${}^{10}Pare-toi donc de fierté, de grandeur,
        revêts-toi de splendeur et de majesté,
${}^{11}répands les débordements de ta colère ;
        regarde tous les arrogants, abaisse-les ;
${}^{12}oui, regarde tous les arrogants, terrasse-les,
        écrase sur place les méchants !
${}^{13}Cache-les ensemble dans la poussière,
        emprisonne-les tous dans le cachot,
${}^{14}et moi-même, je te louerai,
        car alors ta droite t’aura sauvé !
         
${}^{15}Vois donc Behémoth ; je l’ai fait tout comme toi.
        Comme le bœuf, il mange de l’herbe.
${}^{16}Vois donc : sa force est dans ses reins,
        et sa vigueur dans les muscles de son ventre.
${}^{17}Il se raidit comme un cèdre,
        les nerfs de ses cuisses s’entrelacent !
${}^{18}Ses os sont des tubes de bronze,
        ses membres, comme des barres de fer.
${}^{19}C’est lui la première des œuvres de Dieu ;
        son Créateur lui fournit un glaive.
${}^{20}Les montagnes lui paient leur tribut,
        ainsi que toutes les bêtes sauvages qui s’y ébattent.
${}^{21}Sous les lotus il est couché,
        dans le secret des roseaux et des marais.
${}^{22}Les lotus le protègent de leur ombre,
        les saules de la rivière l’entourent.
${}^{23}Voici que le fleuve grossit ; lui ne bronche pas.
        Le Jourdain jaillirait-il vers sa gueule, il resterait calme.
${}^{24}C’est par les yeux qu’on va le prendre,
        avec des crocs, lui percer le naseau.
         
${}^{25}Et Léviathan, vas-tu le pêcher à l’hameçon,
        et lui serrer la langue avec une corde ?
${}^{26}Lui passeras-tu un jonc dans le naseau,
        d’un crochet lui perceras-tu la mâchoire ?
${}^{27}Va-t-il redoubler envers toi les supplications
        et te dire des mots tendres ?
${}^{28}Fera-t-il alliance avec toi ?
        Le prendras-tu pour serviteur à vie ?
${}^{29}Joueras-tu avec lui comme avec un oiseau,
        l’attacheras-tu pour tes petites filles ?
${}^{30}Sera-t-il mis en vente par des associés,
        et débité entre marchands ?
${}^{31}Cribleras-tu de dards sa peau,
        et sa tête, de harpons ?
${}^{32}Pose seulement la main sur lui :
        imagine la lutte, tu ne continueras pas !
      
         
      \bchapter{}
${}^{1}Vois, la témérité est illusoire :
        rien qu’à son aspect, n’est-on pas terrassé ?
${}^{2}N’est-il pas cruel dès qu’on le réveille ?
        Qui donc oserait me tenir tête, à moi ?
${}^{3}Qui m’a donné d’avance, que je doive le payer de retour ?
        Tout ce qui est sous les cieux est à moi.
        
           
         
${}^{4}Je ne passerai pas ses membres sous silence,
        ni le détail de ses prouesses, ni l’élégance de ses proportions.
${}^{5}Qui a jamais soulevé le devant de sa cuirasse ?
        Qui pénétrera dans sa double denture ?
${}^{6}Qui a jamais ouvert les battants de sa gueule ?
        Autour de ses dents, c’est l’effroi !
${}^{7}Son dos : des rangées de boucliers
        étroitement rivés par un sceau,
${}^{8}si rapprochés l’un de l’autre
        que l’air ne passe pas entre eux.
${}^{9}Ils adhèrent l’un à l’autre,
        pris ensemble, sans fissure.
${}^{10}Ses éternuements font jaillir la lumière ;
        ses yeux sont les paupières de l’aurore.
${}^{11}De sa gueule partent des éclairs,
        des étincelles de feu s’en échappent.
${}^{12}De ses naseaux sort une fumée,
        comme d’une marmite chauffée et bouillante.
${}^{13}Son haleine embrase les braises,
        et de sa gueule sort une flamme.
${}^{14}En son cou réside la force,
        devant lui bondit l’épouvante.
${}^{15}Les fanons de sa chair tiennent ferme,
        durs sur lui et compacts.
${}^{16}Son cœur est dur comme pierre,
        dur comme la meule de dessous.
${}^{17}Quand il se dresse, les vaillants prennent peur
        et se dérobent par crainte des coups.
${}^{18}L’épée l’atteint sans pouvoir s’enfoncer,
        pas plus que lance, trait ou javeline.
${}^{19}Il regarde le fer comme paille,
        le bronze, comme bois vermoulu.
${}^{20}Le tir de l’arc ne le fait pas fuir ;
        pour lui, les pierres de fronde
        se changent en fétu de paille.
${}^{21}La massue lui semble un fétu,
        il se rit du sifflement du javelot.
${}^{22}Son ventre est garni de tessons pointus,
        herse qu’il traîne sur la vase.
${}^{23}Il fait bouillonner le gouffre comme un chaudron,
        transforme la mer en brûle-parfums.
${}^{24}Il laisse derrière lui un sillage de lumière ;
        on dirait que l’abîme a pris des cheveux blancs.
${}^{25}Sur terre il n’a pas son pareil,
        lui qui fut créé intrépide.
${}^{26}Tout ce qui est altier, il le toise,
        lui, le roi de tous les fauves. »
        
           
      <h2 class="intertitle hmbot" id="d85e138145">4. Dernière réponse de Job (42,1-6)</h2>
      
         
      \bchapter{}
      \begin{verse}
${}^{1}Job s’adressa au Seigneur et dit :
      
         
       
        ${}^{2}« Je sais que tu peux tout
        et que nul projet pour toi n’est impossible.
        ${}^{3}Quel est celui qui déforme tes plans
        sans rien y connaître ?
        \\De fait, j’ai parlé, sans les comprendre,
        de merveilles hors de ma portée, dont je ne savais rien.
${}^{4}Daigne écouter, et moi, je parlerai ;
        je vais t’interroger, et tu m’instruiras.
        ${}^{5}C’est par ouï-dire que je te connaissais,
        mais maintenant mes yeux t’ont vu.
        ${}^{6}C’est pourquoi je me rétracte et me repens
        sur la poussière et sur la cendre. »
${}^{7}Or, après avoir adressé ces discours à Job, le Seigneur dit à Élifaz de Témane : « Ma colère s’est enflammée contre toi et contre tes deux amis, parce que vous n’avez pas parlé de moi avec justesse comme l’a fait mon serviteur Job. 
${}^{8}Maintenant, prenez sept taureaux et sept béliers, allez trouver mon serviteur Job. Offrez un holocauste en votre faveur, et Job mon serviteur intercédera pour vous. Uniquement par égard pour lui, je ne vous infligerai pas l’infamie méritée pour n’avoir pas parlé de moi avec justesse, comme l’a fait mon serviteur Job. » 
${}^{9}Élifaz de Témane, Bildad de Shouah et Sofar de Naama s’en allèrent et firent comme le Seigneur leur avait dit, et le Seigneur eut égard à l’intervention de Job.
${}^{10}Le Seigneur rétablit la condition de Job tandis qu’il intercédait pour son prochain, et le Seigneur porta au double tous les biens de Job. 
${}^{11}Tous ses frères, toutes ses sœurs et toutes ses connaissances d’autrefois vinrent à lui. Ils mangèrent le pain avec lui dans sa maison. Ils le plaignirent et le consolèrent de tout le malheur que le Seigneur avait fait venir sur lui. Ils lui donnèrent chacun une pièce d’argent et chacun un anneau d’or.
${}^{12}Le Seigneur bénit la nouvelle situation de Job plus encore que l’ancienne. Job posséda quatorze mille moutons et six mille chameaux, mille paires de bœufs et mille ânesses. 
${}^{13} Il eut encore sept fils et trois filles. 
${}^{14} Il nomma la première Colombe, la deuxième Fleur-de-Laurier, et la troisième Ombre-du-Regard. 
${}^{15} On ne trouvait pas dans tout le pays de femmes aussi belles que les filles de Job. Leur père leur donna une part d’héritage avec leurs frères. 
${}^{16} Après cela, Job vécut encore cent quarante ans, et il vit ses fils et les fils de ses fils : quatre générations. 
${}^{17} Et Job mourut âgé, rassasié de jours.
