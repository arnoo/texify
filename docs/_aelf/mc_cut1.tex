  
  
    
    \bbook{ÉVANGILE SELON SAINT MARC}{ÉVANGILE SELON SAINT MARC}
      
         
      \bchapter{}
      \begin{verse}
${}^{1}Commencement de l’Évangile de Jésus, Christ, Fils de Dieu.
      
         
${}^{2}Il est écrit dans Isaïe, le prophète :
        \\Voici que j’envoie mon messager en avant de toi,
        \\pour ouvrir ton chemin.
        ${}^{3}Voix de celui qui crie dans le désert :
        \\Préparez le chemin du Seigneur,
        \\rendez droits ses sentiers.
${}^{4}Alors Jean, celui qui baptisait, parut dans le désert. Il proclamait un baptême de conversion pour le pardon des péchés. 
${}^{5}Toute la Judée, tous les habitants de Jérusalem se rendaient auprès de lui, et ils étaient baptisés par lui dans le Jourdain, en reconnaissant publiquement leurs péchés. 
${}^{6}Jean était vêtu de poil de chameau, avec une ceinture de cuir autour des reins ; il se nourrissait de sauterelles et de miel sauvage.
${}^{7}Il proclamait : « Voici venir derrière moi celui qui est plus fort que moi ; je ne suis pas digne de m’abaisser pour défaire la courroie de ses sandales. 
${}^{8}Moi, je vous ai baptisés avec de l’eau ; lui vous baptisera dans l’Esprit Saint. »
${}^{9}En ces jours-là, Jésus vint de Nazareth, ville de Galilée, et il fut baptisé par Jean dans le Jourdain. 
${}^{10}Et aussitôt, en remontant de l’eau, il vit les cieux se déchirer et l’Esprit descendre sur lui comme une colombe. 
${}^{11}Il y eut une voix venant des cieux : « Tu es mon Fils bien-aimé ; en toi, je trouve ma joie. »
${}^{12}Aussitôt l’Esprit pousse Jésus au désert 
${}^{13}et, dans le désert, il resta quarante jours, tenté par Satan. Il vivait parmi les bêtes sauvages, et les anges le servaient.
      <h2 class="intertitle" id="d85e344114">1. La venue du règne de Dieu (1,14 – 3,6)</h2>
${}^{14}Après l’arrestation de Jean, Jésus partit pour la Galilée proclamer l’Évangile de Dieu ; 
${}^{15}il disait : « Les temps sont accomplis : le règne de Dieu est tout proche. Convertissez-vous et croyez à l’Évangile. »
${}^{16}Passant le long de la mer de Galilée, Jésus vit Simon et André, le frère de Simon, en train de jeter les filets dans la mer, car c’étaient des pêcheurs. 
${}^{17}Il leur dit : « Venez à ma suite. Je vous ferai devenir pêcheurs d’hommes. » 
${}^{18}Aussitôt, laissant leurs filets, ils le suivirent. 
${}^{19}Jésus avança un peu et il vit Jacques, fils de Zébédée, et son frère Jean, qui étaient dans la barque et réparaient les filets. 
${}^{20}Aussitôt, Jésus les appela. Alors, laissant dans la barque leur père Zébédée avec ses ouvriers, ils partirent à sa suite.
${}^{21}Ils entrèrent à Capharnaüm. Aussitôt, le jour du sabbat, Jésus se rendit à la synagogue, et là, il enseignait. 
${}^{22}On était frappé par son enseignement, car il enseignait en homme qui a autorité, et non pas comme les scribes.
${}^{23}Or, il y avait dans leur synagogue un homme tourmenté par un esprit impur, qui se mit à crier : 
${}^{24}« Que nous veux-tu, Jésus de Nazareth ? Es-tu venu pour nous perdre ? Je sais qui tu es : tu es le Saint de Dieu. » 
${}^{25}Jésus l’interpella vivement : « Tais-toi ! Sors de cet homme. » 
${}^{26}L’esprit impur le fit entrer en convulsions, puis, poussant un grand cri, sortit de lui. 
${}^{27}Ils furent tous frappés de stupeur et se demandaient entre eux : « Qu’est-ce que cela veut dire ? Voilà un enseignement nouveau, donné avec autorité ! Il commande même aux esprits impurs, et ils lui obéissent. » 
${}^{28}Sa renommée se répandit aussitôt partout, dans toute la région de la Galilée.
${}^{29}Aussitôt sortis de la synagogue, ils allèrent, avec Jacques et Jean, dans la maison de Simon et d’André. 
${}^{30}Or, la belle-mère de Simon était au lit, elle avait de la fièvre. Aussitôt, on parla à Jésus de la malade. 
${}^{31}Jésus s’approcha, la saisit par la main et la fit lever. La fièvre la quitta, et elle les servait.
${}^{32}Le soir venu, après le coucher du soleil, on lui amenait tous ceux qui étaient atteints d’un mal ou possédés par des démons. 
${}^{33}La ville entière se pressait à la porte. 
${}^{34}Il guérit beaucoup de gens atteints de toutes sortes de maladies, et il expulsa beaucoup de démons ; il empêchait les démons de parler, parce qu’ils savaient, eux, qui il était.
${}^{35}Le lendemain, Jésus se leva, bien avant l’aube. Il sortit et se rendit dans un endroit désert, et là il priait. 
${}^{36}Simon et ceux qui étaient avec lui partirent à sa recherche. 
${}^{37}Ils le trouvent et lui disent : « Tout le monde te cherche. » 
${}^{38}Jésus leur dit : « Allons ailleurs, dans les villages voisins, afin que là aussi je proclame l’Évangile ; car c’est pour cela que je suis sorti. »
${}^{39}Et il parcourut toute la Galilée, proclamant l’Évangile dans leurs synagogues, et expulsant les démons.
${}^{40}Un lépreux vient auprès de lui ; il le supplie et, tombant à ses genoux, lui dit : « Si tu le veux, tu peux me purifier. » 
${}^{41}Saisi de compassion, Jésus étendit la main, le toucha et lui dit : « Je le veux, sois purifié. » 
${}^{42}À l’instant même, la lèpre le quitta et il fut purifié. 
${}^{43}Avec fermeté, Jésus le renvoya aussitôt 
${}^{44}en lui disant : « Attention, ne dis rien à personne, mais va te montrer au prêtre, et donne pour ta purification ce que Moïse a prescrit dans la Loi : cela sera pour les gens un témoignage. » 
${}^{45}Une fois parti, cet homme se mit à proclamer et à répandre la nouvelle, de sorte que Jésus ne pouvait plus entrer ouvertement dans une ville, mais restait à l’écart, dans des endroits déserts. De partout cependant on venait à lui.
      
         
      \bchapter{}
      \begin{verse}
${}^{1}Quelques jours plus tard, Jésus revint à Capharnaüm, et l’on apprit qu’il était à la maison. 
${}^{2}Tant de monde s’y rassembla qu’il n’y avait plus de place, pas même devant la porte, et il leur annonçait la Parole. 
${}^{3}Arrivent des gens qui lui amènent un paralysé, porté par quatre hommes. 
${}^{4}Comme ils ne peuvent l’approcher à cause de la foule, ils découvrent le toit au-dessus de lui, ils font une ouverture, et descendent le brancard sur lequel était couché le paralysé. 
${}^{5}Voyant leur foi, Jésus dit au paralysé : « Mon enfant, tes péchés sont pardonnés. »
${}^{6}Or, il y avait quelques scribes, assis là, qui raisonnaient en eux-mêmes : 
${}^{7}« Pourquoi celui-là parle-t-il ainsi ? Il blasphème. Qui donc peut pardonner les péchés, sinon Dieu seul ? » 
${}^{8}Percevant aussitôt dans son esprit les raisonnements qu’ils se faisaient, Jésus leur dit : « Pourquoi tenez-vous de tels raisonnements ? 
${}^{9}Qu’est-ce qui est le plus facile ? Dire à ce paralysé : “Tes péchés sont pardonnés”, ou bien lui dire : “Lève-toi, prends ton brancard et marche” ? 
${}^{10}Eh bien ! Pour que vous sachiez que le Fils de l’homme a autorité pour pardonner les péchés sur la terre… – Jésus s’adressa au paralysé – 
${}^{11}je te le dis, lève-toi, prends ton brancard, et rentre dans ta maison. » 
${}^{12}Il se leva, prit aussitôt son brancard, et sortit devant tout le monde. Tous étaient frappés de stupeur et rendaient gloire à Dieu, en disant : « Nous n’avons jamais rien vu de pareil. »
${}^{13}Jésus sortit de nouveau le long de la mer ; toute la foule venait à lui, et il les enseignait. 
${}^{14}En passant, il aperçut Lévi, fils d’Alphée, assis au bureau des impôts. Il lui dit : « Suis-moi. » L’homme se leva et le suivit.
${}^{15}Comme Jésus était à table dans la maison de Lévi, beaucoup de publicains (c’est-à-dire des collecteurs d’impôts) et beaucoup de pécheurs vinrent prendre place avec Jésus et ses disciples, car ils étaient nombreux à le suivre. 
${}^{16}Les scribes du groupe des pharisiens, voyant qu’il mangeait avec les pécheurs et les publicains, disaient à ses disciples : « Comment ! Il mange avec les publicains et les pécheurs ! » 
${}^{17}Jésus, qui avait entendu, leur déclara : « Ce ne sont pas les gens bien portants qui ont besoin du médecin, mais les malades. Je ne suis pas venu appeler des justes, mais des pécheurs. »
${}^{18}Comme les disciples de Jean le Baptiste et les pharisiens jeûnaient, on vient demander à Jésus : « Pourquoi, alors que les disciples de Jean et les disciples des pharisiens jeûnent, tes disciples ne jeûnent-ils pas ? » 
${}^{19}Jésus leur dit : « Les invités de la noce pourraient-ils jeûner, pendant que l’Époux est avec eux ? Tant qu’ils ont l’Époux avec eux, ils ne peuvent pas jeûner. 
${}^{20}Mais des jours viendront où l’Époux leur sera enlevé ; alors, ce jour-là, ils jeûneront.
${}^{21}Personne ne raccommode un vieux vêtement avec une pièce d’étoffe neuve ; autrement le morceau neuf ajouté tire sur le vieux tissu et la déchirure s’agrandit. 
${}^{22}Ou encore, personne ne met du vin nouveau dans de vieilles outres ; car alors, le vin fera éclater les outres, et l’on perd à la fois le vin et les outres. À vin nouveau, outres neuves. »
${}^{23}Un jour de sabbat, Jésus marchait à travers les champs de blé ; et ses disciples, chemin faisant, se mirent à arracher des épis. 
${}^{24}Les pharisiens lui disaient : « Regarde ce qu’ils font le jour du sabbat ! Cela n’est pas permis. » 
${}^{25}Et Jésus leur dit : « N’avez-vous jamais lu ce que fit David, lorsqu’il fut dans le besoin et qu’il eut faim, lui-même et ceux qui l’accompagnaient ? 
${}^{26}Au temps du grand prêtre Abiatar, il entra dans la maison de Dieu et mangea les pains de l’offrande que nul n’a le droit de manger, sinon les prêtres, et il en donna aussi à ceux qui l’accompagnaient. »
${}^{27}Il leur disait encore : « Le sabbat a été fait pour l’homme, et non pas l’homme pour le sabbat. 
${}^{28}Voilà pourquoi le Fils de l’homme est maître, même du sabbat. »
      
         
      \bchapter{}
      \begin{verse}
${}^{1}Jésus entra de nouveau dans la synagogue ; il y avait là un homme dont la main était atrophiée. 
${}^{2}On observait Jésus pour voir s’il le guérirait le jour du sabbat. C’était afin de pouvoir l’accuser. 
${}^{3}Il dit à l’homme qui avait la main atrophiée : « Lève-toi, viens au milieu. » 
${}^{4}Et s’adressant aux autres : « Est-il permis, le jour du sabbat, de faire le bien ou de faire le mal ? de sauver une vie ou de tuer ? » Mais eux se taisaient. 
${}^{5}Alors, promenant sur eux un regard de colère, navré de l’endurcissement de leurs cœurs, il dit à l’homme : « Étends la main. » Il l’étendit, et sa main redevint normale. 
${}^{6}Une fois sortis, les pharisiens se réunirent en conseil avec les partisans d’Hérode contre Jésus, pour voir comment le faire périr.
      
         
      <h2 class="intertitle" id="d85e344577">2. Jésus et ses proches (3,7 – 6,6a)</h2>
${}^{7}Jésus se retira avec ses disciples près de la mer, et une grande multitude de gens, venus de la Galilée, le suivirent. 
${}^{8}De Judée, de Jérusalem, d’Idumée, de Transjordanie, et de la région de Tyr et de Sidon vinrent aussi à lui une multitude de gens qui avaient entendu parler de ce qu’il faisait. 
${}^{9}Il dit à ses disciples de tenir une barque à sa disposition pour que la foule ne l’écrase pas. 
${}^{10}Car il avait fait beaucoup de guérisons, si bien que tous ceux qui souffraient de quelque mal se précipitaient sur lui pour le toucher. 
${}^{11}Et lorsque les esprits impurs le voyaient, ils se jetaient à ses pieds et criaient : « Toi, tu es le Fils de Dieu ! »
${}^{12}Mais il leur défendait vivement de le faire connaître.
${}^{13}Puis, il gravit la montagne, et il appela ceux qu’il voulait. Ils vinrent auprès de lui, 
${}^{14}et il en institua douze pour qu’ils soient avec lui et pour les envoyer proclamer la Bonne Nouvelle 
${}^{15}avec le pouvoir d’expulser les démons. 
${}^{16}Donc, il établit les Douze : Pierre – c’est le nom qu’il donna à Simon –, 
${}^{17}Jacques, fils de Zébédée, et Jean, le frère de Jacques – il leur donna le nom de « Boanerguès », c’est-à-dire : « Fils du tonnerre » –, 
${}^{18}André, Philippe, Barthélemy, Matthieu, Thomas, Jacques, fils d’Alphée, Thaddée, Simon le Zélote, 
${}^{19}et Judas Iscariote, celui-là même qui le livra.
${}^{20}Alors Jésus revient à la maison, où de nouveau la foule se rassemble, si bien qu’il n’était même pas possible de manger. 
${}^{21}Les gens de chez lui, l’apprenant, vinrent pour se saisir de lui, car ils affirmaient : « Il a perdu la tête. »
${}^{22}Les scribes, qui étaient descendus de Jérusalem, disaient : « Il est possédé par Béelzéboul ; c’est par le chef des démons qu’il expulse les démons. »
${}^{23}Les appelant près de lui, Jésus leur dit en parabole : « Comment Satan peut-il expulser Satan ? 
${}^{24}Si un royaume est divisé contre lui-même, ce royaume ne peut pas tenir. 
${}^{25}Si les gens d’une même maison se divisent entre eux, ces gens ne pourront pas tenir. 
${}^{26}Si Satan s’est dressé contre lui-même, s’il est divisé, il ne peut pas tenir ; c’en est fini de lui. 
${}^{27}Mais personne ne peut entrer dans la maison d’un homme fort et piller ses biens, s’il ne l’a d’abord ligoté. Alors seulement il pillera sa maison.
${}^{28}Amen, je vous le dis : Tout sera pardonné aux enfants des hommes : leurs péchés et les blasphèmes qu’ils auront proférés. 
${}^{29}Mais si quelqu’un blasphème contre l’Esprit Saint, il n’aura jamais de pardon. Il est coupable d’un péché pour toujours. » 
${}^{30}Jésus parla ainsi parce qu’ils avaient dit : « Il est possédé par un esprit impur. »
${}^{31}Alors arrivent sa mère et ses frères. Restant au-dehors, ils le font appeler. 
${}^{32}Une foule était assise autour de lui ; et on lui dit : « Voici que ta mère et tes frères sont là dehors : ils te cherchent. » 
${}^{33}Mais il leur répond : « Qui est ma mère ? qui sont mes frères ? » 
${}^{34}Et parcourant du regard ceux qui étaient assis en cercle autour de lui, il dit : « Voici ma mère et mes frères. 
${}^{35}Celui qui fait la volonté de Dieu, celui-là est pour moi un frère, une sœur, une mère. »
      
         
      \bchapter{}
      \begin{verse}
${}^{1}Jésus se mit de nouveau à enseigner au bord de la mer de Galilée. Une foule très nombreuse se rassembla auprès de lui, si bien qu’il monta dans une barque où il s’assit. Il était sur la mer, et toute la foule était près de la mer, sur le rivage. 
${}^{2}Il leur enseignait beaucoup de choses en paraboles, et dans son enseignement il leur disait :
${}^{3}« Écoutez ! Voici que le semeur sortit pour semer. 
${}^{4}Comme il semait, du grain est tombé au bord du chemin ; les oiseaux sont venus et ils ont tout mangé. 
${}^{5}Du grain est tombé aussi sur du sol pierreux, où il n’avait pas beaucoup de terre ; il a levé aussitôt, parce que la terre était peu profonde ; 
${}^{6}et lorsque le soleil s’est levé, ce grain a brûlé et, faute de racines, il a séché. 
${}^{7}Du grain est tombé aussi dans les ronces, les ronces ont poussé, l’ont étouffé, et il n’a pas donné de fruit. 
${}^{8}Mais d’autres grains sont tombés dans la bonne terre ; ils ont donné du fruit en poussant et en se développant, et ils ont produit trente, soixante, cent, pour un. » 
${}^{9}Et Jésus disait : « Celui qui a des oreilles pour entendre, qu’il entende ! »
${}^{10}Quand il resta seul, ceux qui étaient autour de lui avec les Douze l’interrogeaient sur les paraboles. 
${}^{11}Il leur disait : « C’est à vous qu’est donné le mystère du royaume de Dieu ; mais à ceux qui sont dehors, tout se présente sous forme de paraboles. 
${}^{12}Et ainsi, comme dit le prophète :
        \\Ils auront beau regarder de tous leurs yeux,
        \\ils ne verront pas ;
        \\ils auront beau écouter de toutes leurs oreilles,
        \\ils ne comprendront pas ;
        \\sinon ils se convertiraient
        \\et recevraient le pardon. »
${}^{13}Il leur dit encore : « Vous ne saisissez pas cette parabole ? Alors, comment comprendrez-vous toutes les paraboles ? 
${}^{14}Le semeur sème la Parole. 
${}^{15}Il y a ceux qui sont au bord du chemin où la Parole est semée : quand ils l’entendent, Satan vient aussitôt et enlève la Parole semée en eux. 
${}^{16}Et de même, il y a ceux qui ont reçu la semence dans les endroits pierreux : ceux-là, quand ils entendent la Parole, ils la reçoivent aussitôt avec joie ; 
${}^{17}mais ils n’ont pas en eux de racine, ce sont les gens d’un moment ; que vienne la détresse ou la persécution à cause de la Parole, ils trébuchent aussitôt. 
${}^{18}Et il y en a d’autres qui ont reçu la semence dans les ronces : ceux-ci entendent la Parole, 
${}^{19}mais les soucis du monde, la séduction de la richesse et toutes les autres convoitises les envahissent et étouffent la Parole, qui ne donne pas de fruit. 
${}^{20}Et il y a ceux qui ont reçu la semence dans la bonne terre : ceux-là entendent la Parole, ils l’accueillent, et ils portent du fruit : trente, soixante, cent, pour un. »
${}^{21}Il leur disait encore : « Est-ce que la lampe est apportée pour être mise sous le boisseau ou sous le lit ? N’est-ce pas pour être mise sur le lampadaire ? 
${}^{22}Car rien n’est caché, sinon pour être manifesté ; rien n’a été gardé secret, sinon pour venir à la clarté. 
${}^{23}Si quelqu’un a des oreilles pour entendre, qu’il entende ! »
${}^{24}Il leur disait encore : « Faites attention à ce que vous entendez ! La mesure que vous utilisez sera utilisée aussi pour vous, et il vous sera donné encore plus. 
${}^{25}Car celui qui a, on lui donnera ; celui qui n’a pas, on lui enlèvera même ce qu’il a. »
${}^{26}Il disait : « Il en est du règne de Dieu comme d’un homme qui jette en terre la semence : 
${}^{27}nuit et jour, qu’il dorme ou qu’il se lève, la semence germe et grandit, il ne sait comment. 
${}^{28}D’elle-même, la terre produit d’abord l’herbe, puis l’épi, enfin du blé plein l’épi. 
${}^{29}Et dès que le blé est mûr, il y met la faucille, puisque le temps de la moisson est arrivé. »
${}^{30}Il disait encore : « À quoi allons-nous comparer le règne de Dieu ? Par quelle parabole pouvons-nous le représenter ? 
${}^{31}Il est comme une graine de moutarde : quand on la sème en terre, elle est la plus petite de toutes les semences. 
${}^{32}Mais quand on l’a semée, elle grandit et dépasse toutes les plantes potagères ; et elle étend de longues branches, si bien que les oiseaux du ciel peuvent faire leur nid à son ombre. »
${}^{33}Par de nombreuses paraboles semblables, Jésus leur annonçait la Parole, dans la mesure où ils étaient capables de l’entendre. 
${}^{34}Il ne leur disait rien sans parabole, mais il expliquait tout à ses disciples en particulier.
${}^{35}Ce jour-là, le soir venu, il dit à ses disciples : « Passons sur l’autre rive. » 
${}^{36}Quittant la foule, ils emmenèrent Jésus, comme il était, dans la barque, et d’autres barques l’accompagnaient. 
${}^{37}Survient une violente tempête. Les vagues se jetaient sur la barque, si bien que déjà elle se remplissait. 
${}^{38}Lui dormait sur le coussin à l’arrière. Les disciples le réveillent et lui disent : « Maître, nous sommes perdus ; cela ne te fait rien ? » 
${}^{39}Réveillé, il menaça le vent et dit à la mer : « Silence, tais-toi ! » Le vent tomba, et il se fit un grand calme. 
${}^{40}Jésus leur dit : « Pourquoi êtes-vous si craintifs ? N’avez-vous pas encore la foi ? » 
${}^{41}Saisis d’une grande crainte, ils se disaient entre eux : « Qui est-il donc, celui-ci, pour que même le vent et la mer lui obéissent ? »
      
         
      \bchapter{}
      \begin{verse}
${}^{1}Ils arrivèrent sur l’autre rive, de l’autre côté de la mer de Galilée, dans le pays des Géraséniens. 
${}^{2}Comme Jésus sortait de la barque, aussitôt un homme possédé d’un esprit impur s’avança depuis les tombes à sa rencontre ; 
${}^{3}il habitait dans les tombeaux et personne ne pouvait plus l’attacher, même avec une chaîne ; 
${}^{4}en effet on l’avait souvent attaché avec des fers aux pieds et des chaînes, mais il avait rompu les chaînes, brisé les fers, et personne ne pouvait le maîtriser. 
${}^{5}Sans arrêt, nuit et jour, il était parmi les tombeaux et sur les collines, à crier, et à se blesser avec des pierres.
${}^{6}Voyant Jésus de loin, il accourut, se prosterna devant lui 
${}^{7}et cria d’une voix forte : « Que me veux-tu, Jésus, fils du Dieu Très-Haut ? Je t’adjure par Dieu, ne me tourmente pas ! » 
${}^{8}Jésus lui disait en effet : « Esprit impur, sors de cet homme ! » 
${}^{9}Et il lui demandait : « Quel est ton nom ? » L’homme lui dit : « Mon nom est Légion, car nous sommes beaucoup. »
${}^{10}Et ils suppliaient Jésus avec insistance de ne pas les chasser en dehors du pays. 
${}^{11}Or, il y avait là, du côté de la colline, un grand troupeau de porcs qui cherchait sa nourriture. 
${}^{12}Alors, les esprits impurs supplièrent Jésus : « Envoie-nous vers ces porcs, et nous entrerons en eux. » 
${}^{13}Il le leur permit. Ils sortirent alors de l’homme et entrèrent dans les porcs. Du haut de la falaise, le troupeau se précipita dans la mer : il y avait environ deux mille porcs, et ils se noyaient dans la mer.
${}^{14}Ceux qui les gardaient prirent la fuite, ils annoncèrent la nouvelle dans la ville et dans la campagne, et les gens vinrent voir ce qui s’était passé. 
${}^{15}Ils arrivent auprès de Jésus, ils voient le possédé assis, habillé, et revenu à la raison, lui qui avait eu la légion de démons, et ils furent saisis de crainte. 
${}^{16}Ceux qui avaient vu tout cela leur racontèrent l’histoire du possédé et ce qui était arrivé aux porcs. 
${}^{17}Alors ils se mirent à supplier Jésus de quitter leur territoire.
${}^{18}Comme Jésus remontait dans la barque, le possédé le suppliait de pouvoir être avec lui. 
${}^{19}Il n’y consentit pas, mais il lui dit : « Rentre à la maison, auprès des tiens, annonce-leur tout ce que le Seigneur a fait pour toi dans sa miséricorde. » 
${}^{20}Alors l’homme s’en alla, il se mit à proclamer dans la région de la Décapole ce que Jésus avait fait pour lui, et tout le monde était dans l’admiration.
${}^{21}Jésus regagna en barque l’autre rive, et une grande foule s’assembla autour de lui. Il était au bord de la mer. 
${}^{22}Arrive un des chefs de synagogue, nommé Jaïre. Voyant Jésus, il tombe à ses pieds 
${}^{23}et le supplie instamment : « Ma fille, encore si jeune, est à la dernière extrémité. Viens lui imposer les mains pour qu’elle soit sauvée et qu’elle vive. » 
${}^{24}Jésus partit avec lui, et la foule qui le suivait était si nombreuse qu’elle l’écrasait.
${}^{25}Or, une femme, qui avait des pertes de sang depuis douze ans… – 
${}^{26}elle avait beaucoup souffert du traitement de nombreux médecins, et elle avait dépensé tous ses biens sans avoir la moindre amélioration ; au contraire, son état avait plutôt empiré – … 
${}^{27}cette femme donc, ayant appris ce qu’on disait de Jésus, vint par-derrière dans la foule et toucha son vêtement. 
${}^{28}Elle se disait en effet : « Si je parviens à toucher seulement son vêtement, je serai sauvée. » 
${}^{29}À l’instant, l’hémorragie s’arrêta, et elle ressentit dans son corps qu’elle était guérie de son mal.
${}^{30}Aussitôt Jésus se rendit compte qu’une force était sortie de lui. Il se retourna dans la foule, et il demandait : « Qui a touché mes vêtements ? » 
${}^{31}Ses disciples lui répondirent : « Tu vois bien la foule qui t’écrase, et tu demandes : “Qui m’a touché ?” » 
${}^{32}Mais lui regardait tout autour pour voir celle qui avait fait cela. 
${}^{33}Alors la femme, saisie de crainte et toute tremblante, sachant ce qui lui était arrivé, vint se jeter à ses pieds et lui dit toute la vérité. 
${}^{34}Jésus lui dit alors : « Ma fille, ta foi t’a sauvée. Va en paix et sois guérie de ton mal. »
${}^{35}Comme il parlait encore, des gens arrivent de la maison de Jaïre, le chef de synagogue, pour dire à celui-ci : « Ta fille vient de mourir. À quoi bon déranger encore le Maître ? » 
${}^{36}Jésus, surprenant ces mots, dit au chef de synagogue : « Ne crains pas, crois seulement. »
${}^{37}Il ne laissa personne l’accompagner, sauf Pierre, Jacques, et Jean, le frère de Jacques. 
${}^{38}Ils arrivent à la maison du chef de synagogue. Jésus voit l’agitation, et des gens qui pleurent et poussent de grands cris. 
${}^{39}Il entre et leur dit : « Pourquoi cette agitation et ces pleurs ? L’enfant n’est pas morte : elle dort. » 
${}^{40}Mais on se moquait de lui. Alors il met tout le monde dehors, prend avec lui le père et la mère de l’enfant, et ceux qui étaient avec lui ; puis il pénètre là où reposait l’enfant. 
${}^{41}Il saisit la main de l’enfant, et lui dit : « Talitha koum », ce qui signifie : « Jeune fille, je te le dis, lève-toi ! » 
${}^{42}Aussitôt la jeune fille se leva et se mit à marcher – elle avait en effet douze ans. Ils furent frappés d’une grande stupeur. 
${}^{43}Et Jésus leur ordonna fermement de ne le faire savoir à personne ; puis il leur dit de la faire manger.
      
         
      \bchapter{}
      \begin{verse}
${}^{1}Sorti de là, Jésus se rendit dans son lieu d’origine, et ses disciples le suivirent. 
${}^{2}Le jour du sabbat, il se mit à enseigner dans la synagogue. De nombreux auditeurs, frappés d’étonnement, disaient : « D’où cela lui vient-il ? Quelle est cette sagesse qui lui a été donnée, et ces grands miracles qui se réalisent par ses mains ? 
${}^{3}N’est-il pas le charpentier, le fils de Marie, et le frère de Jacques, de José, de Jude et de Simon ? Ses sœurs ne sont-elles pas ici chez nous ? » Et ils étaient profondément choqués à son sujet. 
${}^{4}Jésus leur disait : « Un prophète n’est méprisé que dans son pays, sa parenté et sa maison. » 
${}^{5}Et là il ne pouvait accomplir aucun miracle ; il guérit seulement quelques malades en leur imposant les mains. 
${}^{6}Et il s’étonna de leur manque de foi.
      
         
      <h2 class="intertitle" id="d85e345455">3. L’identité de Jésus peu à peu découverte (6,6b – 8,30)</h2>
      Jésus parcourait les villages d’alentour en enseignant. 
${}^{7}Il appela les Douze ; alors il commença à les envoyer en mission deux par deux. Il leur donnait autorité sur les esprits impurs, 
${}^{8}et il leur prescrivit de ne rien prendre pour la route, mais seulement un bâton ; pas de pain, pas de sac, pas de pièces de monnaie dans leur ceinture. 
${}^{9}« Mettez des sandales, ne prenez pas de tunique de rechange. » 
${}^{10}Il leur disait encore : « Quand vous avez trouvé l’hospitalité dans une maison, restez-y jusqu’à votre départ. 
${}^{11}Si, dans une localité, on refuse de vous accueillir et de vous écouter, partez et secouez la poussière de vos pieds : ce sera pour eux un témoignage. » 
${}^{12}Ils partirent, et proclamèrent qu’il fallait se convertir. 
${}^{13}Ils expulsaient beaucoup de démons, faisaient des onctions d’huile à de nombreux malades, et les guérissaient.
${}^{14}Le roi Hérode apprit cela ; en effet, le nom de Jésus devenait célèbre. On disait : « C’est Jean, celui qui baptisait : il est ressuscité d’entre les morts, et voilà pourquoi des miracles se réalisent par lui. » 
${}^{15}Certains disaient : « C’est le prophète Élie. » D’autres disaient encore : « C’est un prophète comme ceux de jadis. » 
${}^{16}Hérode entendait ces propos et disait : « Celui que j’ai fait décapiter, Jean, le voilà ressuscité ! »
${}^{17}Car c’était lui, Hérode, qui avait donné l’ordre d’arrêter Jean et de l’enchaîner dans la prison, à cause d’Hérodiade, la femme de son frère Philippe, que lui-même avait prise pour épouse. 
${}^{18}En effet, Jean lui disait : « Tu n’as pas le droit de prendre la femme de ton frère. » 
${}^{19}Hérodiade en voulait donc à Jean, et elle cherchait à le faire mourir. Mais elle n’y arrivait pas 
${}^{20}parce que Hérode avait peur de Jean : il savait que c’était un homme juste et saint, et il le protégeait ; quand il l’avait entendu, il était très embarrassé ; cependant il l’écoutait avec plaisir.
${}^{21}Or, une occasion favorable se présenta quand, le jour de son anniversaire, Hérode fit un dîner pour ses dignitaires, pour les chefs de l’armée et pour les notables de la Galilée. 
${}^{22}La fille d’Hérodiade fit son entrée et dansa. Elle plut à Hérode et à ses convives. Le roi dit à la jeune fille : « Demande-moi ce que tu veux, et je te le donnerai. » 
${}^{23}Et il lui fit ce serment : « Tout ce que tu me demanderas, je te le donnerai, même si c’est la moitié de mon royaume. » 
${}^{24}Elle sortit alors pour dire à sa mère : « Qu’est-ce que je vais demander ? » Hérodiade répondit : « La tête de Jean, celui qui baptise. » 
${}^{25}Aussitôt la jeune fille s’empressa de retourner auprès du roi, et lui fit cette demande : « Je veux que, tout de suite, tu me donnes sur un plat la tête de Jean le Baptiste. » 
${}^{26}Le roi fut vivement contrarié ; mais à cause du serment et des convives, il ne voulut pas lui opposer un refus. 
${}^{27}Aussitôt il envoya un garde avec l’ordre d’apporter la tête de Jean. Le garde s’en alla décapiter Jean dans la prison. 
${}^{28}Il apporta la tête sur un plat, la donna à la jeune fille, et la jeune fille la donna à sa mère. 
${}^{29}Ayant appris cela, les disciples de Jean vinrent prendre son corps et le déposèrent dans un tombeau.
${}^{30}Les Apôtres se réunirent auprès de Jésus, et lui annoncèrent tout ce qu’ils avaient fait et enseigné. 
${}^{31}Il leur dit : « Venez à l’écart dans un endroit désert, et reposez-vous un peu. »
      De fait, ceux qui arrivaient et ceux qui partaient étaient nombreux, et l’on n’avait même pas le temps de manger. 
${}^{32}Alors, ils partirent en barque pour un endroit désert, à l’écart. 
${}^{33}Les gens les virent s’éloigner, et beaucoup comprirent leur intention. Alors, à pied, de toutes les villes, ils coururent là-bas et arrivèrent avant eux.
${}^{34}En débarquant, Jésus vit une grande foule. Il fut saisi de compassion envers eux, parce qu’ils étaient comme des brebis sans berger. Alors, il se mit à les enseigner longuement.
${}^{35}Déjà l’heure était avancée ; s’étant approchés de lui, ses disciples disaient : « L’endroit est désert et déjà l’heure est tardive. 
${}^{36}Renvoie-les : qu’ils aillent dans les campagnes et les villages des environs s’acheter de quoi manger. » 
${}^{37}Il leur répondit : « Donnez-leur vous-mêmes à manger. » Ils répliquent : « Irons-nous dépenser le salaire de deux cents journées pour acheter des pains et leur donner à manger ? » 
${}^{38}Jésus leur demande : « Combien de pains avez-vous ? Allez voir. » S’étant informés, ils lui disent : « Cinq, et deux poissons. »
${}^{39}Il leur ordonna de les faire tous asseoir par groupes sur l’herbe verte. 
${}^{40}Ils se disposèrent par carrés de cent et de cinquante. 
${}^{41}Jésus prit les cinq pains et les deux poissons, et, levant les yeux au ciel, il prononça la bénédiction et rompit les pains ; il les donnait aux disciples pour qu’ils les distribuent à la foule. Il partagea aussi les deux poissons entre eux tous. 
${}^{42}Ils mangèrent tous et ils furent rassasiés. 
${}^{43}Et l’on ramassa les morceaux de pain qui restaient, de quoi remplir douze paniers, ainsi que les restes des poissons. 
${}^{44}Ceux qui avaient mangé les pains étaient au nombre de cinq mille hommes.
${}^{45}Aussitôt après, Jésus obligea ses disciples à monter dans la barque et à le précéder sur l’autre rive, vers Bethsaïde, pendant que lui-même renvoyait la foule. 
${}^{46}Quand il les eut congédiés, il s’en alla sur la montagne pour prier. 
${}^{47}Le soir venu, la barque était au milieu de la mer et lui, tout seul, à terre. 
${}^{48}Voyant qu’ils peinaient à ramer, car le vent leur était contraire, il vient à eux vers la fin de la nuit en marchant sur la mer, et il voulait les dépasser. 
${}^{49}En le voyant marcher sur la mer, les disciples pensèrent que c’était un fantôme et ils se mirent à pousser des cris. 
${}^{50}Tous, en effet, l’avaient vu et ils étaient bouleversés. Mais aussitôt Jésus parla avec eux et leur dit : « Confiance ! c’est moi ; n’ayez pas peur ! » 
${}^{51}Il monta ensuite avec eux dans la barque et le vent tomba ; et en eux-mêmes ils étaient au comble de la stupeur, 
${}^{52}car ils n’avaient rien compris au sujet des pains : leur cœur était endurci.
${}^{53}Après la traversée, abordant à Génésareth, ils accostèrent. 
${}^{54}Ils sortirent de la barque, et aussitôt les gens reconnurent Jésus : 
${}^{55}ils parcoururent toute la région, et se mirent à apporter les malades sur des brancards là où l’on apprenait que Jésus se trouvait. 
${}^{56}Et dans tous les endroits où il se rendait, dans les villages, les villes ou les campagnes, on déposait les infirmes sur les places. Ils le suppliaient de leur laisser toucher ne serait-ce que la frange de son manteau. Et tous ceux qui la touchèrent étaient sauvés.
      
         
      \bchapter{}
      \begin{verse}
${}^{1}Les pharisiens et quelques scribes, venus de Jérusalem, se réunissent auprès de Jésus, 
${}^{2}et voient quelques-uns de ses disciples prendre leur repas avec des mains impures, c’est-à-dire non lavées. 
${}^{3}– Les pharisiens en effet, comme tous les Juifs, se lavent toujours soigneusement les mains avant de manger, par attachement à la tradition des anciens ; 
${}^{4}et au retour du marché, ils ne mangent pas avant de s’être aspergés d’eau, et ils sont attachés encore par tradition à beaucoup d’autres pratiques : lavage de coupes, de carafes et de plats. 
${}^{5}Alors les pharisiens et les scribes demandèrent à Jésus : « Pourquoi tes disciples ne suivent-ils pas la tradition des anciens ? Ils prennent leurs repas avec des mains impures. »
${}^{6}Jésus leur répondit : « Isaïe a bien prophétisé à votre sujet, hypocrites, ainsi qu’il est écrit :
        \\Ce peuple m’honore des lèvres,
        \\mais son cœur est loin de moi.
        ${}^{7}C’est en vain qu’ils me rendent un culte ;
        \\les doctrines qu’ils enseignent ne sont que des préceptes humains.
${}^{8}Vous aussi, vous laissez de côté le commandement de Dieu, pour vous attacher à la tradition des hommes. »
${}^{9}Il leur disait encore : « Vous rejetez bel et bien le commandement de Dieu pour établir votre tradition. 
${}^{10}En effet, Moïse a dit : Honore ton père et ta mère. Et encore : Celui qui maudit son père ou sa mère sera mis à mort. 
${}^{11}Mais vous, vous dites : Supposons qu’un homme déclare à son père ou à sa mère : “Les ressources qui m’auraient permis de t’aider sont korbane, c’est-à-dire don réservé à Dieu”, 
${}^{12}alors vous ne l’autorisez plus à faire quoi que ce soit pour son père ou sa mère ; 
${}^{13}vous annulez ainsi la parole de Dieu par la tradition que vous transmettez. Et vous faites beaucoup de choses du même genre. »
${}^{14}Appelant de nouveau la foule, il lui disait : « Écoutez-moi tous, et comprenez bien. 
${}^{15}Rien de ce qui est extérieur à l’homme et qui entre en lui ne peut le rendre impur. Mais ce qui sort de l’homme, voilà ce qui rend l’homme impur. »
${}^{17}Quand il eut quitté la foule pour rentrer à la maison, ses disciples l’interrogeaient sur cette parabole. 
${}^{18}Alors il leur dit : « Êtes-vous donc sans intelligence, vous aussi ? Ne comprenez-vous pas que tout ce qui entre dans l’homme, en venant du dehors, ne peut pas le rendre impur, 
${}^{19}parce que cela n’entre pas dans son cœur, mais dans son ventre, pour être éliminé ? » C’est ainsi que Jésus déclarait purs tous les aliments.
${}^{20}Il leur dit encore : « Ce qui sort de l’homme, c’est cela qui le rend impur. 
${}^{21}Car c’est du dedans, du cœur de l’homme, que sortent les pensées perverses : inconduites, vols, meurtres, 
${}^{22}adultères, cupidités, méchancetés, fraude, débauche, envie, diffamation, orgueil et démesure. 
${}^{23}Tout ce mal vient du dedans, et rend l’homme impur. »
${}^{24}En partant de là, Jésus se rendit dans le territoire de Tyr. Il était entré dans une maison, et il ne voulait pas qu’on le sache. Mais il ne put rester inaperçu : 
${}^{25}une femme entendit aussitôt parler de lui ; elle avait une petite fille possédée par un esprit impur ; elle vint se jeter à ses pieds. 
${}^{26}Cette femme était païenne, syro-phénicienne de naissance, et elle lui demandait d’expulser le démon hors de sa fille. 
${}^{27}Il lui disait : « Laisse d’abord les enfants se rassasier, car il n’est pas bien de prendre le pain des enfants et de le jeter aux petits chiens. » 
${}^{28}Mais elle lui répliqua : « Seigneur, les petits chiens, sous la table, mangent bien les miettes des petits enfants ! » Alors il lui dit : 
${}^{29}« À cause de cette parole, va : le démon est sorti de ta fille. » 
${}^{30}Elle rentra à la maison, et elle trouva l’enfant étendue sur le lit : le démon était sorti d’elle.
${}^{31}Jésus quitta le territoire de Tyr ; passant par Sidon, il prit la direction de la mer de Galilée et alla en plein territoire de la Décapole. 
${}^{32}Des gens lui amènent un sourd qui avait aussi de la difficulté à parler et supplient Jésus de poser la main sur lui. 
${}^{33}Jésus l’emmena à l’écart, loin de la foule, lui mit les doigts dans les oreilles, et, avec sa salive, lui toucha la langue. 
${}^{34}Puis, les yeux levés au ciel, il soupira et lui dit : « Effata ! », c’est-à-dire : « Ouvre-toi ! » 
${}^{35}Ses oreilles s’ouvrirent ; sa langue se délia, et il parlait correctement. 
${}^{36}Alors Jésus leur ordonna de n’en rien dire à personne ; mais plus il leur donnait cet ordre, plus ceux-ci le proclamaient. 
${}^{37}Extrêmement frappés, ils disaient : « Il a bien fait toutes choses : il fait entendre les sourds et parler les muets. »
      
         
      \bchapter{}
      \begin{verse}
${}^{1}En ces jours-là, comme il y avait de nouveau une grande foule, et que les gens n’avaient rien à manger, Jésus appelle à lui ses disciples et leur dit : 
${}^{2}« J’ai de la compassion pour cette foule, car depuis trois jours déjà ils restent auprès de moi, et n’ont rien à manger. 
${}^{3}Si je les renvoie chez eux à jeun, ils vont défaillir en chemin, et certains d’entre eux sont venus de loin. » 
${}^{4}Ses disciples lui répondirent : « Où donc pourra-t-on trouver du pain pour les rassasier ici, dans le désert ? » 
${}^{5}Il leur demanda : « Combien de pains avez-vous ? » Ils lui dirent : « Sept. » 
${}^{6}Alors il ordonna à la foule de s’asseoir par terre. Puis, prenant les sept pains et rendant grâce, il les rompit, et il les donnait à ses disciples pour que ceux-ci les distribuent ; et ils les distribuèrent à la foule. 
${}^{7}Ils avaient aussi quelques petits poissons, que Jésus bénit et fit aussi distribuer. 
${}^{8}Les gens mangèrent et furent rassasiés. On ramassa les morceaux qui restaient : cela faisait sept corbeilles. 
${}^{9}Or, ils étaient environ quatre mille. Puis Jésus les renvoya. 
${}^{10}Aussitôt, montant dans la barque avec ses disciples, il alla dans la région de Dalmanoutha.
      
         
${}^{11}Les pharisiens survinrent et se mirent à discuter avec Jésus ; pour le mettre à l’épreuve, ils cherchaient à obtenir de lui un signe venant du ciel. 
${}^{12}Jésus soupira au plus profond de lui-même et dit : « Pourquoi cette génération cherche-t-elle un signe ? Amen, je vous le déclare : aucun signe ne sera donné à cette génération. » 
${}^{13}Puis il les quitta, remonta en barque, et il partit vers l’autre rive.
${}^{14}Les disciples avaient oublié d’emporter des pains ; ils n’avaient qu’un seul pain avec eux dans la barque. 
${}^{15}Or Jésus leur faisait cette recommandation : « Attention ! Prenez garde au levain des pharisiens et au levain d’Hérode ! » 
${}^{16}Mais ils discutaient entre eux sur ce manque de pains. 
${}^{17}Jésus s’en rend compte et leur dit : « Pourquoi discutez-vous sur ce manque de pains ? Vous ne saisissez pas ? Vous ne comprenez pas encore ? Vous avez le cœur endurci ? 
${}^{18}Vous avez des yeux et vous ne voyez pas, vous avez des oreilles et vous n’entendez pas ! Vous ne vous rappelez pas ? 
${}^{19}Quand j’ai rompu les cinq pains pour cinq mille personnes, combien avez-vous ramassé de paniers pleins de morceaux ? » Ils lui répondirent : « Douze. 
${}^{20}– Et quand j’en ai rompu sept pour quatre mille, combien avez-vous rempli de corbeilles en ramassant les morceaux ? » Ils lui répondirent : « Sept. » 
${}^{21}Il leur disait : « Vous ne comprenez pas encore ? »
${}^{22}Jésus et ses disciples arrivent à Bethsaïde. Des gens lui amènent un aveugle et le supplient de le toucher. 
${}^{23}Jésus prit l’aveugle par la main et le conduisit hors du village. Il lui mit de la salive sur les yeux et lui imposa les mains. Il lui demandait : « Aperçois-tu quelque chose ? » 
${}^{24}Levant les yeux, l’homme disait : « J’aperçois les gens : ils ressemblent à des arbres que je vois marcher. » 
${}^{25}Puis Jésus, de nouveau, imposa les mains sur les yeux de l’homme ; celui-ci se mit à voir normalement, il se trouva guéri, et il distinguait tout avec netteté. 
${}^{26}Jésus le renvoya dans sa maison en disant : « Ne rentre même pas dans le village. »
${}^{27}Jésus s’en alla, ainsi que ses disciples, vers les villages situés aux environs de Césarée-de-Philippe. Chemin faisant, il interrogeait ses disciples : « Au dire des gens, qui suis-je ? » 
${}^{28}Ils lui répondirent : « Jean le Baptiste ; pour d’autres, Élie ; pour d’autres, un des prophètes. » 
${}^{29}Et lui les interrogeait : « Et vous, que dites-vous ? Pour vous, qui suis-je ? » Pierre, prenant la parole, lui dit : « Tu es le Christ. » 
${}^{30}Alors, il leur défendit vivement de parler de lui à personne.
      <h2 class="intertitle" id="d85e346385">1. Vers la Passion : marcher à la suite de Jésus (8,31 – 10)</h2>
${}^{31}Il commença à leur enseigner qu’il fallait que le Fils de l’homme souffre beaucoup, qu’il soit rejeté par les anciens, les grands prêtres et les scribes, qu’il soit tué, et que, trois jours après, il ressuscite. 
${}^{32}Jésus disait cette parole ouvertement. Pierre, le prenant à part, se mit à lui faire de vifs reproches. 
${}^{33}Mais Jésus se retourna et, voyant ses disciples, il interpella vivement Pierre : « Passe derrière moi, Satan ! Tes pensées ne sont pas celles de Dieu, mais celles des hommes. »
${}^{34}Appelant la foule avec ses disciples, il leur dit : « Si quelqu’un veut marcher à ma suite, qu’il renonce à lui-même, qu’il prenne sa croix et qu’il me suive. 
${}^{35}Car celui qui veut sauver sa vie la perdra ; mais celui qui perdra sa vie à cause de moi et de l’Évangile la sauvera. 
${}^{36}Quel avantage, en effet, un homme a-t-il à gagner le monde entier si c’est au prix de sa vie ? 
${}^{37}Que pourrait-il donner en échange de sa vie ? 
${}^{38}Celui qui a honte de moi et de mes paroles dans cette génération adultère et pécheresse, le Fils de l’homme aussi aura honte de lui, quand il viendra dans la gloire de son Père avec les saints anges. »
      
         
      \bchapter{}
      \begin{verse}
${}^{1}Et il leur disait : « Amen, je vous le dis : parmi ceux qui sont ici, certains ne connaîtront pas la mort avant d’avoir vu le règne de Dieu venu avec puissance. »
      
         
${}^{2}Six jours après, Jésus prend avec lui Pierre, Jacques et Jean, et les emmène, eux seuls, à l’écart sur une haute montagne. Et il fut transfiguré devant eux. 
${}^{3}Ses vêtements devinrent resplendissants, d’une blancheur telle que personne sur terre ne peut obtenir une blancheur pareille. 
${}^{4}Élie leur apparut avec Moïse, et tous deux s’entretenaient avec Jésus. 
${}^{5}Pierre alors prend la parole et dit à Jésus : « Rabbi, il est bon que nous soyons ici ! Dressons donc trois tentes : une pour toi, une pour Moïse, et une pour Élie. » 
${}^{6}De fait, Pierre ne savait que dire, tant leur frayeur était grande. 
${}^{7}Survint une nuée qui les couvrit de son ombre, et de la nuée une voix se fit entendre : « Celui-ci est mon Fils bien-aimé : écoutez-le ! » 
${}^{8}Soudain, regardant tout autour, ils ne virent plus que Jésus seul avec eux.
${}^{9}Ils descendirent de la montagne, et Jésus leur ordonna de ne raconter à personne ce qu’ils avaient vu, avant que le Fils de l’homme soit ressuscité d’entre les morts. 
${}^{10}Et ils restèrent fermement attachés à cette parole, tout en se demandant entre eux ce que voulait dire : « ressusciter d’entre les morts ».
${}^{11}Ils l’interrogeaient : « Pourquoi les scribes disent-ils que le prophète Élie doit venir d’abord ? » 
${}^{12}Jésus leur dit : « Certes, Élie vient d’abord pour remettre toute chose à sa place. Mais alors, pourquoi l’Écriture dit-elle, au sujet du Fils de l’homme, qu’il souffrira beaucoup et sera méprisé ? 
${}^{13}Eh bien ! je vous le déclare : Élie est déjà venu, et ils lui ont fait tout ce qu’ils ont voulu, comme l’Écriture le dit à son sujet. »
${}^{14}En rejoignant les autres disciples, ils virent une grande foule qui les entourait, et des scribes qui discutaient avec eux. 
${}^{15}Aussitôt qu’elle vit Jésus, toute la foule fut stupéfaite, et les gens accouraient pour le saluer. 
${}^{16}Il leur demanda : « De quoi discutez-vous avec eux ? » 
${}^{17}Quelqu’un dans la foule lui répondit : « Maître, je t’ai amené mon fils, il est possédé par un esprit qui le rend muet ; 
${}^{18}cet esprit s’empare de lui n’importe où, il le jette par terre, l’enfant écume, grince des dents et devient tout raide. J’ai demandé à tes disciples d’expulser cet esprit, mais ils n’en ont pas été capables. »
${}^{19}Prenant la parole, Jésus leur dit : « Génération incroyante, combien de temps resterai-je auprès de vous ? Combien de temps devrai-je vous supporter ? Amenez-le-moi. » 
${}^{20}On le lui amena. Dès qu’il vit Jésus, l’esprit fit entrer l’enfant en convulsions ; l’enfant tomba et se roulait par terre en écumant. 
${}^{21}Jésus interrogea le père : « Depuis combien de temps cela lui arrive-t-il ? » Il répondit : « Depuis sa petite enfance. 
${}^{22}Et souvent il l’a même jeté dans le feu ou dans l’eau pour le faire périr. Mais si tu peux quelque chose, viens à notre secours, par compassion envers nous ! » 
${}^{23}Jésus lui déclara : « Pourquoi dire : “Si tu peux”… ? Tout est possible pour celui qui croit. » 
${}^{24}Aussitôt le père de l’enfant s’écria : « Je crois ! Viens au secours de mon manque de foi ! »
${}^{25}Jésus vit que la foule s’attroupait ; il menaça l’esprit impur, en lui disant : « Esprit qui rends muet et sourd, je te l’ordonne, sors de cet enfant et n’y rentre plus jamais ! » 
${}^{26}Ayant poussé des cris et provoqué des convulsions, l’esprit sortit. L’enfant devint comme un cadavre, de sorte que tout le monde disait : « Il est mort. » 
${}^{27}Mais Jésus, lui saisissant la main, le releva, et il se mit debout.
${}^{28}Quand Jésus fut rentré à la maison, ses disciples l’interrogèrent en particulier : « Pourquoi est-ce que nous, nous n’avons pas réussi à l’expulser ? » 
${}^{29}Jésus leur répondit : « Cette espèce-là, rien ne peut la faire sortir, sauf la prière. »
${}^{30}Partis de là, ils traversaient la Galilée, et Jésus ne voulait pas qu’on le sache, 
${}^{31}car il enseignait ses disciples en leur disant : « Le Fils de l’homme est livré aux mains des hommes ; ils le tueront et, trois jours après sa mort, il ressuscitera. » 
${}^{32}Mais les disciples ne comprenaient pas ces paroles et ils avaient peur de l’interroger.
${}^{33}Ils arrivèrent à Capharnaüm, et, une fois à la maison, Jésus leur demanda : « De quoi discutiez-vous en chemin ? » 
${}^{34}Ils se taisaient, car, en chemin, ils avaient discuté entre eux pour savoir qui était le plus grand. 
${}^{35}S’étant assis, Jésus appela les Douze et leur dit : « Si quelqu’un veut être le premier, qu’il soit le dernier de tous et le serviteur de tous. » 
${}^{36}Prenant alors un enfant, il le plaça au milieu d’eux, l’embrassa, et leur dit : 
${}^{37}« Quiconque accueille en mon nom un enfant comme celui-ci, c’est moi qu’il accueille. Et celui qui m’accueille, ce n’est pas moi qu’il accueille, mais Celui qui m’a envoyé. »
${}^{38}Jean, l’un des Douze, disait à Jésus : « Maître, nous avons vu quelqu’un expulser les démons en ton nom ; nous l’en avons empêché, car il n’est pas de ceux qui nous suivent. » 
${}^{39}Jésus répondit : « Ne l’en empêchez pas, car celui qui fait un miracle en mon nom ne peut pas, aussitôt après, mal parler de moi ; 
${}^{40}celui qui n’est pas contre nous est pour nous. 
${}^{41}Et celui qui vous donnera un verre d’eau au nom de votre appartenance au Christ, amen, je vous le dis, il ne restera pas sans récompense.
${}^{42}« Celui qui est un scandale, une occasion de chute, pour un seul de ces petits qui croient en moi, mieux vaudrait pour lui qu’on lui attache au cou une de ces meules que tournent les ânes, et qu’on le jette à la mer. 
${}^{43}Et si ta main est pour toi une occasion de chute, coupe-la. Mieux vaut pour toi entrer manchot dans la vie éternelle que de t’en aller dans la géhenne avec tes deux mains, là où le feu ne s’éteint pas. 
${}^{45}Si ton pied est pour toi une occasion de chute, coupe-le. Mieux vaut pour toi entrer estropié dans la vie éternelle que de t’en aller dans la géhenne avec tes deux pieds. 
${}^{47}Si ton œil est pour toi une occasion de chute, arrache-le. Mieux vaut pour toi entrer borgne dans le royaume de Dieu que de t’en aller dans la géhenne avec tes deux yeux, 
${}^{48}là où le ver ne meurt pas et où le feu ne s’éteint pas.
${}^{49}Chacun sera salé au feu. 
${}^{50}C’est une bonne chose que le sel ; mais s’il cesse d’être du sel, avec quoi allez-vous lui rendre sa saveur ? Ayez du sel en vous-mêmes, et vivez en paix entre vous. »
      
         
      \bchapter{}
      \begin{verse}
${}^{1}Partant de là, Jésus arrive dans le territoire de la Judée, au-delà du Jourdain. De nouveau, des foules s’assemblent près de lui, et de nouveau, comme d’habitude, il les enseignait.
${}^{2}Des pharisiens l’abordèrent et, pour le mettre à l’épreuve, ils lui demandaient : « Est-il permis à un mari de renvoyer sa femme ? » 
${}^{3}Jésus leur répondit : « Que vous a prescrit Moïse ? » 
${}^{4}Ils lui dirent : « Moïse a permis de renvoyer sa femme à condition d’établir un acte de répudiation. » 
${}^{5}Jésus répliqua : « C’est en raison de la dureté de vos cœurs qu’il a formulé pour vous cette règle. 
${}^{6}Mais, au commencement de la création, Dieu les fit homme et femme. 
${}^{7}À cause de cela, l’homme quittera son père et sa mère, 
${}^{8}il s’attachera à sa femme, et tous deux deviendront une seule chair. Ainsi, ils ne sont plus deux, mais une seule chair. 
${}^{9}Donc, ce que Dieu a uni, que l’homme ne le sépare pas ! »
${}^{10}De retour à la maison, les disciples l’interrogeaient de nouveau sur cette question. 
${}^{11}Il leur déclara : « Celui qui renvoie sa femme et en épouse une autre devient adultère envers elle. 
${}^{12}Si une femme qui a renvoyé son mari en épouse un autre, elle devient adultère. »
${}^{13}Des gens présentaient à Jésus des enfants pour qu’il pose la main sur eux ; mais les disciples les écartèrent vivement. 
${}^{14}Voyant cela, Jésus se fâcha et leur dit : « Laissez les enfants venir à moi, ne les empêchez pas, car le royaume de Dieu est à ceux qui leur ressemblent. 
${}^{15}Amen, je vous le dis : celui qui n’accueille pas le royaume de Dieu à la manière d’un enfant n’y entrera pas. » 
${}^{16}Il les embrassait et les bénissait en leur imposant les mains.
${}^{17}Jésus se mettait en route quand un homme accourut et, tombant à ses genoux, lui demanda : « Bon Maître, que dois-je faire pour avoir la vie éternelle en héritage ? » 
${}^{18}Jésus lui dit : « Pourquoi dire que je suis bon ? Personne n’est bon, sinon Dieu seul. 
${}^{19}Tu connais les commandements : Ne commets pas de meurtre, ne commets pas d’adultère, ne commets pas de vol, ne porte pas de faux témoignage, ne fais de tort à personne, honore ton père et ta mère. » 
${}^{20}L’homme répondit : « Maître, tout cela, je l’ai observé depuis ma jeunesse. » 
${}^{21}Jésus posa son regard sur lui, et il l’aima. Il lui dit : « Une seule chose te manque : va, vends ce que tu as et donne-le aux pauvres ; alors tu auras un trésor au ciel. Puis viens, suis-moi. » 
${}^{22}Mais lui, à ces mots, devint sombre et s’en alla tout triste, car il avait de grands biens.
${}^{23}Alors Jésus regarda autour de lui et dit à ses disciples : « Comme il sera difficile à ceux qui possèdent des richesses d’entrer dans le royaume de Dieu ! » 
${}^{24}Les disciples étaient stupéfaits de ces paroles. Jésus reprenant la parole leur dit : « Mes enfants, comme il est difficile d’entrer dans le royaume de Dieu ! 
${}^{25}Il est plus facile à un chameau de passer par le trou d’une aiguille qu’à un riche d’entrer dans le royaume de Dieu. » 
${}^{26}De plus en plus déconcertés, les disciples se demandaient entre eux : « Mais alors, qui peut être sauvé ? » 
${}^{27}Jésus les regarde et dit : « Pour les hommes, c’est impossible, mais pas pour Dieu ; car tout est possible à Dieu. »
${}^{28}Pierre se mit à dire à Jésus : « Voici que nous avons tout quitté pour te suivre. » 
${}^{29}Jésus déclara : « Amen, je vous le dis : nul n’aura quitté, à cause de moi et de l’Évangile, une maison, des frères, des sœurs, une mère, un père, des enfants ou une terre 
${}^{30}sans qu’il reçoive, en ce temps déjà, le centuple : maisons, frères, sœurs, mères, enfants et terres, avec des persécutions, et, dans le monde à venir, la vie éternelle.
${}^{31}Beaucoup de premiers seront derniers, et les derniers seront les premiers. »
${}^{32}Les disciples étaient en route pour monter à Jérusalem ; Jésus marchait devant eux ; ils étaient saisis de frayeur, et ceux qui suivaient étaient aussi dans la crainte. Prenant de nouveau les Douze auprès de lui, il se mit à leur dire ce qui allait lui arriver : 
${}^{33}« Voici que nous montons à Jérusalem. Le Fils de l’homme sera livré aux grands prêtres et aux scribes ; ils le condamneront à mort, ils le livreront aux nations païennes, 
${}^{34}qui se moqueront de lui, cracheront sur lui, le flagelleront et le tueront, et trois jours après, il ressuscitera. »
${}^{35}Alors, Jacques et Jean, les fils de Zébédée, s’approchent de Jésus et lui disent : « Maître, ce que nous allons te demander, nous voudrions que tu le fasses pour nous. » 
${}^{36}Il leur dit : « Que voulez-vous que je fasse pour vous ? » 
${}^{37}Ils lui répondirent : « Donne-nous de siéger, l’un à ta droite et l’autre à ta gauche, dans ta gloire. » 
${}^{38}Jésus leur dit : « Vous ne savez pas ce que vous demandez. Pouvez-vous boire la coupe que je vais boire, être baptisé du baptême dans lequel je vais être plongé ? » 
${}^{39}Ils lui dirent : « Nous le pouvons. » Jésus leur dit : « La coupe que je vais boire, vous la boirez ; et vous serez baptisés du baptême dans lequel je vais être plongé. 
${}^{40}Quant à siéger à ma droite ou à ma gauche, ce n’est pas à moi de l’accorder ; il y a ceux pour qui cela est préparé. »
${}^{41}Les dix autres, qui avaient entendu, se mirent à s’indigner contre Jacques et Jean. 
${}^{42}Jésus les appela et leur dit : « Vous le savez : ceux que l’on regarde comme chefs des nations les commandent en maîtres ; les grands leur font sentir leur pouvoir. 
${}^{43}Parmi vous, il ne doit pas en être ainsi. Celui qui veut devenir grand parmi vous sera votre serviteur. 
${}^{44}Celui qui veut être parmi vous le premier sera l’esclave de tous : 
${}^{45}car le Fils de l’homme n’est pas venu pour être servi, mais pour servir, et donner sa vie en rançon pour la multitude. »
${}^{46}Jésus et ses disciples arrivent à Jéricho. Et tandis que Jésus sortait de Jéricho avec ses disciples et une foule nombreuse, le fils de Timée, Bartimée, un aveugle qui mendiait, était assis au bord du chemin. 
${}^{47}Quand il entendit que c’était Jésus de Nazareth, il se mit à crier : « Fils de David, Jésus, prends pitié de moi ! » 
${}^{48}Beaucoup de gens le rabrouaient pour le faire taire, mais il criait de plus belle : « Fils de David, prends pitié de moi ! » 
${}^{49}Jésus s’arrête et dit : « Appelez-le. » On appelle donc l’aveugle, et on lui dit : « Confiance, lève-toi ; il t’appelle. » 
${}^{50}L’aveugle jeta son manteau, bondit et courut vers Jésus. 
${}^{51}Prenant la parole, Jésus lui dit : « Que veux-tu que je fasse pour toi ? » L’aveugle lui dit : « Rabbouni, que je retrouve la vue ! » 
${}^{52}Et Jésus lui dit : « Va, ta foi t’a sauvé. » Aussitôt l’homme retrouva la vue, et il suivait Jésus sur le chemin.
      <h2 class="intertitle" id="d85e347290">2. Jésus et Jérusalem (11 – 13)</h2>
      
         
      \bchapter{}
      \begin{verse}
${}^{1}Lorsqu’ils approchent de Jérusalem, vers Bethphagé et Béthanie, près du mont des Oliviers, Jésus envoie deux de ses disciples 
${}^{2}et leur dit : « Allez au village qui est en face de vous. Dès que vous y entrerez, vous trouverez un petit âne attaché, sur lequel personne ne s’est encore assis. Détachez-le et amenez-le. 
${}^{3}Si l’on vous dit : “Que faites-vous là ?”, répondez : “Le Seigneur en a besoin, mais il vous le renverra aussitôt.” » 
${}^{4}Ils partirent, trouvèrent un petit âne attaché près d’une porte, dehors, dans la rue, et ils le détachèrent. 
${}^{5}Des gens qui se trouvaient là leur demandaient : « Qu’avez-vous à détacher cet ânon ? » 
${}^{6}Ils répondirent ce que Jésus leur avait dit, et on les laissa faire.
${}^{7}Ils amenèrent le petit âne à Jésus, le couvrirent de leurs manteaux, et Jésus s’assit dessus. 
${}^{8}Alors, beaucoup de gens étendirent leurs manteaux sur le chemin, d’autres, des feuillages coupés dans les champs. 
${}^{9}Ceux qui marchaient devant et ceux qui suivaient criaient :
        \\« Hosanna !
        \\Béni soit celui qui vient au nom du Seigneur !
        ${}^{10}Béni soit le Règne qui vient,
        \\celui de David, notre père.
        \\Hosanna au plus haut des cieux ! »
${}^{11}Jésus entra à Jérusalem, dans le Temple. Il parcourut du regard toutes choses et, comme c’était déjà le soir, il sortit pour aller à Béthanie avec les Douze.
${}^{12}Le lendemain, quand ils quittèrent Béthanie, il eut faim. 
${}^{13}Voyant de loin un figuier qui avait des feuilles, il alla voir s’il y trouverait quelque chose ; mais, en s’approchant, il ne trouva que des feuilles, car ce n’était pas la saison des figues. 
${}^{14}Alors il dit au figuier : « Que jamais plus personne ne mange de tes fruits ! » Et ses disciples avaient bien entendu.
${}^{15}Ils arrivèrent à Jérusalem. Entré dans le Temple, Jésus se mit à expulser ceux qui vendaient et ceux qui achetaient dans le Temple. Il renversa les comptoirs des changeurs et les sièges des marchands de colombes, 
${}^{16}et il ne laissait personne transporter quoi que ce soit à travers le Temple. 
${}^{17}Il enseignait, et il déclarait aux gens : « L’Écriture ne dit-elle pas : Ma maison sera appelée maison de prière pour toutes les nations ? Or vous, vous en avez fait une caverne de bandits. »
${}^{18}Apprenant cela, les grands prêtres et les scribes cherchaient comment le faire périr. En effet, ils avaient peur de lui, car toute la foule était frappée par son enseignement.
${}^{19}Et quand le soir tomba, Jésus et ses disciples s’en allèrent hors de la ville.
${}^{20}Le lendemain matin, en passant, ils virent le figuier qui était desséché jusqu’aux racines. 
${}^{21}Pierre, se rappelant ce qui s’était passé, dit à Jésus : « Rabbi, regarde : le figuier que tu as maudit est desséché. » 
${}^{22}Alors Jésus, prenant la parole, leur dit : « Ayez foi en Dieu. 
${}^{23}Amen, je vous le dis : quiconque dira à cette montagne : “Enlève-toi de là, et va te jeter dans la mer”, s’il ne doute pas dans son cœur, mais s’il croit que ce qu’il dit arrivera, cela lui sera accordé ! 
${}^{24}C’est pourquoi, je vous le dis : tout ce que vous demandez dans la prière, croyez que vous l’avez obtenu, et cela vous sera accordé. 
${}^{25}Et quand vous vous tenez en prière, si vous avez quelque chose contre quelqu’un, pardonnez, afin que votre Père qui est aux cieux vous pardonne aussi vos fautes. »
${}^{27}Jésus et ses disciples reviennent à Jérusalem. Et comme Jésus allait et venait dans le Temple, les grands prêtres, les scribes et les anciens vinrent le trouver. 
${}^{28}Ils lui demandaient : « Par quelle autorité fais-tu cela ? Ou alors qui t’a donné cette autorité pour le faire ? » 
${}^{29}Jésus leur dit : « Je vais vous poser une seule question. Répondez-moi, et je vous dirai par quelle autorité je fais cela. 
${}^{30}Le baptême de Jean venait-il du ciel ou des hommes ? Répondez-moi. » 
${}^{31}Ils se faisaient entre eux ce raisonnement : « Si nous disons : “Du ciel”, il va dire : “Pourquoi donc n’avez-vous pas cru à sa parole ?” 
${}^{32}Mais allons-nous dire : “Des hommes” ? » Ils avaient peur de la foule, car tout le monde estimait que Jean était réellement un prophète. 
${}^{33}Ils répondent donc à Jésus : « Nous ne savons pas ! » Alors Jésus leur dit : « Moi, je ne vous dis pas non plus par quelle autorité je fais cela. »
      
         
      \bchapter{}
      \begin{verse}
${}^{1}Jésus se mit à leur parler en paraboles : « Un homme planta une vigne, il l’entoura d’une clôture, y creusa un pressoir et y bâtit une tour de garde. Puis il loua cette vigne à des vignerons, et partit en voyage. 
${}^{2}Le moment venu, il envoya un serviteur auprès des vignerons pour se faire remettre par eux ce qui lui revenait des fruits de la vigne. 
${}^{3}Mais les vignerons se saisirent du serviteur, le frappèrent, et le renvoyèrent les mains vides. 
${}^{4}De nouveau, il leur envoya un autre serviteur ; et celui-là, ils l’assommèrent et l’humilièrent. 
${}^{5}Il en envoya encore un autre, et celui-là, ils le tuèrent ; puis beaucoup d’autres serviteurs : ils frappèrent les uns et tuèrent les autres. 
${}^{6}Il lui restait encore quelqu’un : son fils bien-aimé. Il l’envoya vers eux en dernier, en se disant : “Ils respecteront mon fils.” 
${}^{7}Mais ces vignerons-là se dirent entre eux : “Voici l’héritier : allons-y ! tuons-le, et l’héritage va être à nous !” 
${}^{8}Ils se saisirent de lui, le tuèrent, et le jetèrent hors de la vigne.
${}^{9}Que fera le maître de la vigne ? Il viendra, fera périr les vignerons, et donnera la vigne à d’autres. 
${}^{10}N’avez-vous pas lu ce passage de l’Écriture ?
        \\La pierre qu’ont rejetée les bâtisseurs
        \\est devenue la pierre d’angle :
        ${}^{11}c’est là l’œuvre du Seigneur,
        \\la merveille devant nos yeux ! »
${}^{12}Les chefs du peuple cherchaient à arrêter Jésus, mais ils eurent peur de la foule. – Ils avaient bien compris en effet qu’il avait dit la parabole à leur intention. Ils le laissèrent donc et s’en allèrent.
${}^{13}On envoya à Jésus des pharisiens et des partisans d’Hérode pour lui tendre un piège en le faisant parler, 
${}^{14}et ceux-ci vinrent lui dire : « Maître, nous le savons : tu es toujours vrai ; tu ne te laisses influencer par personne, car ce n’est pas selon l’apparence que tu considères les gens, mais tu enseignes le chemin de Dieu selon la vérité. Est-il permis, oui ou non, de payer l’impôt à César, l’empereur ? Devons-nous payer, oui ou non ? » 
${}^{15}Mais lui, sachant leur hypocrisie, leur dit : « Pourquoi voulez-vous me mettre à l’épreuve ? Faites-moi voir une pièce d’argent. » 
${}^{16}Ils en apportèrent une, et Jésus leur dit : « Cette effigie et cette inscription, de qui sont-elles ? – De César », répondent-ils. 
${}^{17}Jésus leur dit :
        \\« Ce qui est à César, rendez-le à César,
        \\et à Dieu ce qui est à Dieu. »
      Et ils étaient remplis d’étonnement à son sujet.
${}^{18}Des sadducéens – ceux qui affirment qu’il n’y a pas de résurrection – viennent trouver Jésus. Ils l’interrogeaient : 
${}^{19}« Maître, Moïse nous a prescrit : Si un homme a un frère qui meurt en laissant une femme, mais aucun enfant, il doit épouser la veuve pour susciter une descendance à son frère. 
${}^{20}Il y avait sept frères ; le premier se maria, et mourut sans laisser de descendance. 
${}^{21}Le deuxième épousa la veuve, et mourut sans laisser de descendance. Le troisième pareillement. 
${}^{22}Et aucun des sept ne laissa de descendance. Et en dernier, après eux tous, la femme mourut aussi. 
${}^{23}À la résurrection, quand ils ressusciteront, duquel d’entre eux sera-t-elle l’épouse, puisque les sept l’ont eue pour épouse ? »
${}^{24}Jésus leur dit : « N’êtes-vous pas en train de vous égarer, en méconnaissant les Écritures et la puissance de Dieu ? 
${}^{25}Lorsqu’on ressuscite d’entre les morts, on ne prend ni femme ni mari, mais on est comme les anges dans les cieux. 
${}^{26}Et sur le fait que les morts ressuscitent, n’avez-vous pas lu dans le livre de Moïse, au récit du buisson ardent, comment Dieu lui a dit : Moi, je suis le Dieu d’Abraham, le Dieu d’Isaac, le Dieu de Jacob ? 
${}^{27}Il n’est pas le Dieu des morts, mais des vivants. Vous vous égarez complètement. »
${}^{28}Un scribe qui avait entendu la discussion, et remarqué que Jésus avait bien répondu, s’avança pour lui demander : « <a class="anchor verset_lettre" id="bib_mc_12_28_b"/>Quel est le premier de tous les commandements ? » 
${}^{29}Jésus lui fit cette réponse : « Voici le premier : Écoute, Israël : le Seigneur notre Dieu est l’unique Seigneur. 
${}^{30}Tu aimeras le Seigneur ton Dieu de tout ton cœur, de toute ton âme, de tout ton esprit et de toute ta force. 
${}^{31}Et voici le second : Tu aimeras ton prochain comme toi-même. Il n’y a pas de commandement plus grand que ceux-là. » 
${}^{32}Le scribe reprit : « Fort bien, Maître, tu as dit vrai : Dieu est l’Unique et il n’y en a pas d’autre que lui. 
${}^{33}L’aimer de tout son cœur, de toute son intelligence, de toute sa force, et aimer son prochain comme soi-même, vaut mieux que toute offrande d’holocaustes et de sacrifices. » 
${}^{34}Jésus, voyant qu’il avait fait une remarque judicieuse, lui dit : « Tu n’es pas loin du royaume de Dieu. » Et personne n’osait plus l’interroger.
${}^{35}Alors qu’il enseignait dans le Temple, Jésus, prenant la parole, déclarait : « Comment les scribes peuvent-ils dire que le Messie est le fils de David ? 
${}^{36}David lui-même a dit, inspiré par l’Esprit Saint :
        \\Le Seigneur a dit à mon Seigneur :
        “Siège à ma droite
        \\jusqu’à ce que j’aie placé tes ennemis
        sous tes pieds !”
${}^{37}David lui-même le nomme Seigneur. D’où vient alors qu’il est son fils ? » Et la foule nombreuse l’écoutait avec plaisir.
${}^{38}Dans son enseignement, il disait : « Méfiez-vous des scribes, qui tiennent à se promener en vêtements d’apparat et qui aiment les salutations sur les places publiques, 
${}^{39}les sièges d’honneur dans les synagogues, et les places d’honneur dans les dîners. 
${}^{40}Ils dévorent les biens des veuves et, pour l’apparence, ils font de longues prières : ils seront d’autant plus sévèrement jugés. »
${}^{41}Jésus s’était assis dans le Temple en face de la salle du trésor, et regardait comment la foule y mettait de l’argent. Beaucoup de riches y mettaient de grosses sommes. 
${}^{42}Une pauvre veuve s’avança et mit deux petites pièces de monnaie. 
${}^{43}Jésus appela ses disciples et leur déclara : « Amen, je vous le dis : cette pauvre veuve a mis dans le Trésor plus que tous les autres. 
${}^{44}Car tous, ils ont pris sur leur superflu, mais elle, elle a pris sur son indigence : elle a mis tout ce qu’elle possédait, tout ce qu’elle avait pour vivre. »
      
         
      \bchapter{}
      \begin{verse}
${}^{1}Comme Jésus sortait du Temple, un de ses disciples lui dit : « Maître, regarde : quelles belles pierres ! quelles constructions ! » 
${}^{2}Mais Jésus lui dit : « Tu vois ces grandes constructions ? Il ne restera pas ici pierre sur pierre ; tout sera détruit. »
${}^{3}Et comme il s’était assis au mont des Oliviers, en face du Temple, Pierre, Jacques, Jean et André l’interrogeaient à l’écart : 
${}^{4}« Dis-nous quand cela arrivera et quel sera le signe donné lorsque tout cela va se terminer. »
${}^{5}Alors Jésus se mit à leur dire : « Prenez garde que personne ne vous égare. 
${}^{6}Beaucoup viendront sous mon nom, et diront : “C’est moi”, et ils égareront bien des gens. 
${}^{7}Quand vous entendrez parler de guerres et de rumeurs de guerre, ne vous laissez pas effrayer ; il faut que cela arrive, mais ce ne sera pas encore la fin. 
${}^{8}Car on se dressera nation contre nation, royaume contre royaume, il y aura des tremblements de terre en divers lieux, il y aura des famines ; c’est le commencement des douleurs de l’enfantement.
${}^{9}Vous, soyez sur vos gardes ; on vous livrera aux tribunaux et aux synagogues ; on vous frappera, on vous traduira devant des gouverneurs et des rois à cause de moi ; ce sera pour eux un témoignage. 
${}^{10}Mais il faut d’abord que l’Évangile soit proclamé à toutes les nations. 
${}^{11}Et lorsqu’on vous emmènera pour vous livrer, ne vous inquiétez pas d’avance pour savoir ce que vous direz, mais dites ce qui vous sera donné à cette heure-là. Car ce n’est pas vous qui parlerez, mais l’Esprit Saint. 
${}^{12}Le frère livrera son frère à la mort, et le père, son enfant ; les enfants se dresseront contre leurs parents et les feront mettre à mort. 
${}^{13}Vous serez détestés de tous à cause de mon nom. Mais celui qui aura persévéré jusqu’à la fin, celui-là sera sauvé.
${}^{14}Lorsque vous verrez l’Abomination de la désolation installée là où elle ne doit pas être – que le lecteur comprenne ! – alors, ceux qui seront en Judée, qu’ils s’enfuient dans les montagnes ; 
${}^{15}celui qui sera sur sa terrasse, qu’il n’en descende pas et n’entre pas pour emporter quelque chose de sa maison ; 
${}^{16}celui qui sera dans son champ, qu’il ne retourne pas en arrière pour emporter son manteau. 
${}^{17}Malheureuses les femmes qui seront enceintes et celles qui allaiteront en ces jours-là ! 
${}^{18}Priez pour que cela n’arrive pas en hiver, 
${}^{19}car en ces jours-là il y aura une détresse telle qu’il n’y en a jamais eu depuis le commencement de la création, quand Dieu créa le monde, jusqu’à maintenant, et telle qu’il n’y en aura jamais plus. 
${}^{20}Et si le Seigneur n’abrégeait pas le nombre des jours, personne n’aurait la vie sauve ; mais à cause des élus, de ceux qu’il a choisis, il a abrégé ces jours-là.
${}^{21}Alors si quelqu’un vous dit : “Voilà le Messie ! Il est ici ! Voilà ! Il est là-bas !”, ne le croyez pas. 
${}^{22}Il surgira des faux messies et des faux prophètes qui feront des signes et des prodiges afin d’égarer, si c’était possible, les élus. 
${}^{23}Quant à vous, prenez garde : je vous ai tout dit à l’avance.
        ${}^{24}En ces jours-là, après une pareille détresse,
        \\le soleil s’obscurcira
        \\et la lune ne donnera plus sa clarté ;
        ${}^{25}les étoiles tomberont du ciel,
        \\et les puissances célestes seront ébranlées.
${}^{26}Alors on verra le Fils de l’homme venir dans les nuées avec grande puissance et avec gloire. 
${}^{27}Il enverra les anges pour rassembler les élus des quatre coins du monde, depuis l’extrémité de la terre jusqu’à l’extrémité du ciel.
${}^{28}Laissez-vous instruire par la comparaison du figuier : dès que ses branches deviennent tendres et que sortent les feuilles, vous savez que l’été est proche. 
${}^{29}De même, vous aussi, lorsque vous verrez arriver cela, sachez que le Fils de l’homme est proche, à votre porte. 
${}^{30}Amen, je vous le dis : cette génération ne passera pas avant que tout cela n’arrive. 
${}^{31}Le ciel et la terre passeront, mes paroles ne passeront pas.
${}^{32}Quant à ce jour et à cette heure-là, nul ne les connaît, pas même les anges dans le ciel, pas même le Fils, mais seulement le Père.
${}^{33}Prenez garde, restez éveillés : car vous ne savez pas quand ce sera le moment. 
${}^{34}C’est comme un homme parti en voyage : en quittant sa maison, il a donné tout pouvoir à ses serviteurs, fixé à chacun son travail, et demandé au portier de veiller. 
${}^{35}Veillez donc, car vous ne savez pas quand vient le maître de la maison, le soir ou à minuit, au chant du coq ou le matin ; 
${}^{36}s’il arrive à l’improviste, il ne faudrait pas qu’il vous trouve endormis. 
${}^{37}Ce que je vous dis là, je le dis à tous : Veillez ! »
      <h2 class="intertitle" id="d85e348220">3. Passion et mort de Jésus (14 – 15)</h2>
      
         
      \bchapter{}
      \begin{verse}
${}^{1}La fête de la Pâque et des pains sans levain allait avoir lieu deux jours après. Les grands prêtres et les scribes cherchaient comment arrêter Jésus par ruse, pour le faire mourir. 
${}^{2}Car ils se disaient : « Pas en pleine fête, pour éviter des troubles dans le peuple. »
${}^{3}Jésus se trouvait à Béthanie, dans la maison de Simon le lépreux. Pendant qu’il était à table, une femme entra, avec un flacon d’albâtre contenant un parfum très pur et de grande valeur. Brisant le flacon, elle lui versa le parfum sur la tête. 
${}^{4}Or, de leur côté, quelques-uns s’indignaient : « À quoi bon gaspiller ce parfum ? 
${}^{5}On aurait pu, en effet, le vendre pour plus de trois cents pièces d’argent, que l’on aurait données aux pauvres. » Et ils la rudoyaient. 
${}^{6}Mais Jésus leur dit : « Laissez-la ! Pourquoi la tourmenter ? Il est beau, le geste qu’elle a fait envers moi. 
${}^{7}Des pauvres, vous en aurez toujours avec vous, et, quand vous le voulez, vous pouvez leur faire du bien ; mais moi, vous ne m’aurez pas toujours. 
${}^{8}Ce qu’elle pouvait faire, elle l’a fait. D’avance elle a parfumé mon corps pour mon ensevelissement. 
${}^{9}Amen, je vous le dis : partout où l’Évangile sera proclamé – dans le monde entier –, on racontera, en souvenir d’elle, ce qu’elle vient de faire. »
${}^{10}Judas Iscariote, l’un des Douze, alla trouver les grands prêtres pour leur livrer Jésus. 
${}^{11}À cette nouvelle, ils se réjouirent et promirent de lui donner de l’argent. Et Judas cherchait comment le livrer au moment favorable.
${}^{12}Le premier jour de la fête des pains sans levain, où l’on immolait l’agneau pascal, les disciples de Jésus lui disent : « Où veux-tu que nous allions faire les préparatifs pour que tu manges la Pâque ? » 
${}^{13}Il envoie deux de ses disciples en leur disant : « Allez à la ville ; un homme portant une cruche d’eau viendra à votre rencontre. Suivez-le, 
${}^{14}et là où il entrera, dites au propriétaire : “Le Maître te fait dire : Où est la salle où je pourrai manger la Pâque avec mes disciples ?” 
${}^{15}Il vous indiquera, à l’étage, une grande pièce aménagée et prête pour un repas. Faites-y pour nous les préparatifs. » 
${}^{16}Les disciples partirent, allèrent à la ville ; ils trouvèrent tout comme Jésus leur avait dit, et ils préparèrent la Pâque.
${}^{17}Le soir venu, Jésus arrive avec les Douze. 
${}^{18}Pendant qu’ils étaient à table et mangeaient, Jésus déclara : « Amen, je vous le dis : l’un de vous, qui mange avec moi, va me livrer. » 
${}^{19}Ils devinrent tout tristes et, l’un après l’autre, ils lui demandaient : « Serait-ce moi ? » 
${}^{20}Il leur dit : « C’est l’un des Douze, celui qui est en train de se servir avec moi dans le plat. 
${}^{21}Le Fils de l’homme s’en va, comme il est écrit à son sujet ; mais malheureux celui par qui le Fils de l’homme est livré ! Il vaudrait mieux pour lui qu’il ne soit pas né, cet homme-là ! »
${}^{22}Pendant le repas, Jésus, ayant pris du pain et prononcé la bénédiction, le rompit, le leur donna, et dit : « Prenez, ceci est mon corps. » 
${}^{23}Puis, ayant pris une coupe et ayant rendu grâce, il la leur donna, et ils en burent tous. 
${}^{24}Et il leur dit : « Ceci est mon sang, le sang de l’Alliance, versé pour la multitude. 
${}^{25}Amen, je vous le dis : je ne boirai plus du fruit de la vigne, jusqu’au jour où je le boirai, nouveau, dans le royaume de Dieu. »
${}^{26}Après avoir chanté les psaumes, ils partirent pour le mont des Oliviers. 
${}^{27}Jésus leur dit : « Vous allez tous être exposés à tomber, car il est écrit :
        \\Je frapperai le berger,
        \\et les brebis seront dispersées.
${}^{28}Mais, une fois ressuscité, je vous précéderai en Galilée. »
${}^{29}Pierre lui dit alors : « Même si tous viennent à tomber, moi, je ne tomberai pas. » 
${}^{30}Jésus lui répond : « Amen, je te le dis : toi, aujourd’hui, cette nuit même, avant que le coq chante deux fois, tu m’auras renié trois fois. » 
${}^{31}Mais lui reprenait de plus belle : « Même si je dois mourir avec toi, je ne te renierai pas. » Et tous en disaient autant.
${}^{32}Ils parviennent à un domaine appelé Gethsémani. Jésus dit à ses disciples : « Asseyez-vous ici, pendant que je vais prier. » 
${}^{33}Puis il emmène avec lui Pierre, Jacques et Jean, et commence à ressentir frayeur et angoisse. 
${}^{34}Il leur dit : « Mon âme est triste à mourir. Restez ici et veillez. » 
${}^{35}Allant un peu plus loin, il tombait à terre et priait pour que, s’il était possible, cette heure s’éloigne de lui. 
${}^{36}Il disait : « Abba… Père, tout est possible pour toi. Éloigne de moi cette coupe. Cependant, non pas ce que moi, je veux, mais ce que toi, tu veux ! » 
${}^{37}Puis il revient et trouve les disciples endormis. Il dit à Pierre : « Simon, tu dors ! Tu n’as pas eu la force de veiller seulement une heure ? 
${}^{38}Veillez et priez, pour ne pas entrer en tentation ; l’esprit est ardent, mais la chair est faible. » 
${}^{39}De nouveau, il s’éloigna et pria, en répétant les mêmes paroles. 
${}^{40}Et de nouveau, il vint près des disciples qu’il trouva endormis, car leurs yeux étaient alourdis de sommeil. Et eux ne savaient que lui répondre. 
${}^{41}Une troisième fois, il revient et leur dit : « Désormais, vous pouvez dormir et vous reposer. C’est fait ; l’heure est venue : voici que le Fils de l’homme est livré aux mains des pécheurs. 
${}^{42}Levez-vous ! Allons ! Voici qu’il est proche, celui qui me livre. »
${}^{43}Jésus parlait encore quand Judas, l’un des Douze, arriva et avec lui une foule armée d’épées et de bâtons, envoyée par les grands prêtres, les scribes et les anciens. 
${}^{44}Or, celui qui le livrait leur avait donné un signe convenu : « Celui que j’embrasserai, c’est lui : arrêtez-le, et emmenez-le sous bonne garde. » 
${}^{45}À peine arrivé, Judas, s’approchant de Jésus, lui dit : « Rabbi ! » Et il l’embrassa. 
${}^{46}Les autres mirent la main sur lui et l’arrêtèrent. 
${}^{47}Or un de ceux qui étaient là tira son épée, frappa le serviteur du grand prêtre et lui trancha l’oreille.
${}^{48}Alors Jésus leur déclara : « Suis-je donc un bandit, pour que vous soyez venus vous saisir de moi, avec des épées et des bâtons ? 
${}^{49}Chaque jour, j’étais auprès de vous dans le Temple en train d’enseigner, et vous ne m’avez pas arrêté. Mais c’est pour que les Écritures s’accomplissent. » 
${}^{50}Les disciples l’abandonnèrent et s’enfuirent tous.
${}^{51}Or, un jeune homme suivait Jésus ; il n’avait pour tout vêtement qu’un drap. On essaya de l’arrêter. 
${}^{52}Mais lui, lâchant le drap, s’enfuit tout nu.
${}^{53}Ils emmenèrent Jésus chez le grand prêtre. Ils se rassemblèrent tous, les grands prêtres, les anciens et les scribes. 
${}^{54}Pierre avait suivi Jésus à distance, jusqu’à l’intérieur du palais du grand prêtre, et là, assis avec les gardes, il se chauffait près du feu.
${}^{55}Les grands prêtres et tout le Conseil suprême cherchaient un témoignage contre Jésus pour le faire mettre à mort, et ils n’en trouvaient pas. 
${}^{56}De fait, beaucoup portaient de faux témoignages contre Jésus, et ces témoignages ne concordaient pas. 
${}^{57}Quelques-uns se levèrent pour porter contre lui ce faux témoignage : 
${}^{58}« Nous l’avons entendu dire : “Je détruirai ce sanctuaire fait de main d’homme, et en trois jours j’en rebâtirai un autre qui ne sera pas fait de main d’homme.” » 
${}^{59}Et même sur ce point, leurs témoignages n’étaient pas concordants. 
${}^{60}Alors s’étant levé, le grand prêtre, devant tous, interrogea Jésus : « Tu ne réponds rien ? Que dis-tu des témoignages qu’ils portent contre toi ? » 
${}^{61}Mais lui gardait le silence et ne répondait rien. Le grand prêtre l’interrogea de nouveau : « Es-tu le Christ, le Fils du Dieu béni ? » 
${}^{62}Jésus lui dit :
        \\« Je le suis.
        \\Et vous verrez le Fils de l’homme
        \\siéger à la droite du Tout-Puissant,
        \\et venir parmi les nuées du ciel. »
${}^{63}Alors, le grand prêtre déchire ses vêtements et dit : « Pourquoi nous faut-il encore des témoins ? 
${}^{64}Vous avez entendu le blasphème. Qu’en pensez-vous ? » Tous prononcèrent qu’il méritait la mort.
${}^{65}Quelques-uns se mirent à cracher sur lui, couvrirent son visage d’un voile, et le giflèrent, en disant : « Fais le prophète ! » Et les gardes lui donnèrent des coups.
${}^{66}Comme Pierre était en bas, dans la cour, arrive une des jeunes servantes du grand prêtre. 
${}^{67}Elle voit Pierre qui se chauffe, le dévisage et lui dit : « Toi aussi, tu étais avec Jésus de Nazareth ! » 
${}^{68}Pierre le nia : « Je ne sais pas, je ne comprends pas de quoi tu parles. » Puis il sortit dans le vestibule, au dehors. Alors un coq chanta.
${}^{69}La servante, ayant vu Pierre, se mit de nouveau à dire à ceux qui se trouvaient là : « Celui-ci est l’un d’entre eux ! » 
${}^{70}De nouveau, Pierre le niait. Peu après, ceux qui se trouvaient là lui disaient à leur tour : « Sûrement tu es l’un d’entre eux ! D’ailleurs, tu es Galiléen. » 
${}^{71}Alors il se mit à protester violemment et à jurer : « Je ne connais pas cet homme dont vous parlez. » 
${}^{72}Et aussitôt, pour la seconde fois, un coq chanta. Alors Pierre se rappela cette parole que Jésus lui avait dite : « Avant que le coq chante deux fois, tu m’auras renié trois fois. » Et il fondit en larmes.
      
         
      \bchapter{}
      \begin{verse}
${}^{1}Dès le matin, les grands prêtres convoquèrent les anciens et les scribes, et tout le Conseil suprême. Puis, après avoir ligoté Jésus, ils l’emmenèrent et le livrèrent à Pilate.
${}^{2}Celui-ci l’interrogea : « Es-tu le roi des Juifs ? » Jésus répondit : « C’est toi-même qui le dis. » 
${}^{3}Les grands prêtres multipliaient contre lui les accusations. 
${}^{4}Pilate lui demanda à nouveau : « Tu ne réponds rien ? Vois toutes les accusations qu’ils portent contre toi. » 
${}^{5}Mais Jésus ne répondit plus rien, si bien que Pilate fut étonné.
${}^{6}À chaque fête, il leur relâchait un prisonnier, celui qu’ils demandaient. 
${}^{7}Or, il y avait en prison un dénommé Barabbas, arrêté avec des émeutiers pour un meurtre qu’ils avaient commis lors de l’émeute. 
${}^{8}La foule monta donc chez Pilate, et se mit à demander ce qu’il leur accordait d’habitude. 
${}^{9}Pilate leur répondit : « Voulez-vous que je vous relâche le roi des Juifs ? » 
${}^{10}Il se rendait bien compte que c’était par jalousie que les grands prêtres l’avaient livré. 
${}^{11}Ces derniers soulevèrent la foule pour qu’il leur relâche plutôt Barabbas. 
${}^{12}Et comme Pilate reprenait : « Que voulez-vous donc que je fasse de celui que vous appelez le roi des Juifs ? », 
${}^{13}de nouveau ils crièrent : « Crucifie-le ! » 
${}^{14}Pilate leur disait : « Qu’a-t-il donc fait de mal ? » Mais ils crièrent encore plus fort : « Crucifie-le ! »
${}^{15}Pilate, voulant contenter la foule, relâcha Barabbas et, après avoir fait flageller Jésus, il le livra pour qu’il soit crucifié.
${}^{16}Les soldats l’emmenèrent à l’intérieur du palais, c’est-à-dire dans le Prétoire. Alors ils rassemblent toute la garde, 
${}^{17}ils le revêtent de pourpre, et lui posent sur la tête une couronne d’épines qu’ils ont tressée. 
${}^{18}Puis ils se mirent à lui faire des salutations, en disant : « Salut, roi des Juifs ! » 
${}^{19}Ils lui frappaient la tête avec un roseau, crachaient sur lui, et s’agenouillaient pour lui rendre hommage. 
${}^{20}Quand ils se furent bien moqués de lui, ils lui enlevèrent le manteau de pourpre, et lui remirent ses vêtements.
      Puis, de là, ils l’emmènent pour le crucifier, 
${}^{21}et ils réquisitionnent, pour porter sa croix, un passant, Simon de Cyrène, le père d’Alexandre et de Rufus, qui revenait des champs. 
${}^{22}Et ils amènent Jésus au lieu dit Golgotha, ce qui se traduit : Lieu-du-Crâne (ou Calvaire). 
${}^{23}Ils lui donnaient du vin aromatisé de myrrhe ; mais il n’en prit pas.
${}^{24}Alors ils le crucifient, puis se partagent ses vêtements, en tirant au sort pour savoir la part de chacun. 
${}^{25}C’était la troisième heure (c’est-à-dire : neuf heures du matin) lorsqu’on le crucifia. 
${}^{26}L’inscription indiquant le motif de sa condamnation portait ces mots :
      « Le roi des Juifs ».
${}^{27}Avec lui ils crucifient deux bandits, l’un à sa droite, l’autre à sa gauche.
${}^{29}Les passants l’injuriaient en hochant la tête : ils disaient : « Hé ! toi qui détruis le Sanctuaire et le rebâtis en trois jours, 
${}^{30}sauve-toi toi-même, descends de la croix ! » 
${}^{31}De même, les grands prêtres se moquaient de lui avec les scribes, en disant entre eux : « Il en a sauvé d’autres, et il ne peut pas se sauver lui-même ! 
${}^{32}Qu’il descende maintenant de la croix, le Christ, le roi d’Israël ; alors nous verrons et nous croirons. » Même ceux qui étaient crucifiés avec lui l’insultaient.
${}^{33}Quand arriva la sixième heure (c’est-à-dire : midi), l’obscurité se fit sur toute la terre jusqu’à la neuvième heure. 
${}^{34}Et à la neuvième heure, Jésus cria d’une voix forte :
      « Éloï, Éloï, lema sabactani ? »,
      ce qui se traduit :
      « Mon Dieu, mon Dieu,
      pourquoi m’as-tu abandonné ? »
${}^{35}L’ayant entendu, quelques-uns de ceux qui étaient là disaient : « Voilà qu’il appelle le prophète Élie ! » 
${}^{36}L’un d’eux courut tremper une éponge dans une boisson vinaigrée, il la mit au bout d’un roseau, et il lui donnait à boire, en disant : « Attendez ! Nous verrons bien si Élie vient le descendre de là ! » 
${}^{37}Mais Jésus, poussant un grand cri, expira.
${}^{38}Le rideau du Sanctuaire se déchira en deux, depuis le haut jusqu’en bas. 
${}^{39}Le centurion qui était là en face de Jésus, voyant comment il avait expiré, déclara : « Vraiment, cet homme était Fils de Dieu ! »
${}^{40}Il y avait aussi des femmes, qui observaient de loin, et parmi elles, Marie Madeleine, Marie, mère de Jacques le Petit et de José, et Salomé, 
${}^{41}qui suivaient Jésus et le servaient quand il était en Galilée, et encore beaucoup d’autres, qui étaient montées avec lui à Jérusalem.
${}^{42}Déjà il se faisait tard ; or, comme c’était le jour de la Préparation, qui précède le sabbat, 
${}^{43}Joseph d’Arimathie intervint. C’était un homme influent, membre du Conseil, et il attendait lui aussi le règne de Dieu. Il eut l’audace d’aller chez Pilate pour demander le corps de Jésus. 
${}^{44}Pilate s’étonna qu’il soit déjà mort ; il fit appeler le centurion, et l’interrogea pour savoir si Jésus était mort depuis longtemps. 
${}^{45}Sur le rapport du centurion, il permit à Joseph de prendre le corps. 
${}^{46}Alors Joseph acheta un linceul, il descendit Jésus de la croix, l’enveloppa dans le linceul et le déposa dans un tombeau qui était creusé dans le roc. Puis il roula une pierre contre l’entrée du tombeau.
${}^{47}Or, Marie Madeleine et Marie, mère de José, observaient l’endroit où on l’avait mis.
      
         
      \bchapter{}
      \begin{verse}
${}^{1}Le sabbat terminé, Marie Madeleine, Marie, mère de Jacques, et Salomé achetèrent des parfums pour aller embaumer le corps de Jésus. 
${}^{2}De grand matin, le premier jour de la semaine, elles se rendent au tombeau dès le lever du soleil. 
${}^{3}Elles se disaient entre elles : « Qui nous roulera la pierre pour dégager l’entrée du tombeau ? »
${}^{4}Levant les yeux, elles s’aperçoivent qu’on a roulé la pierre, qui était pourtant très grande. 
${}^{5}En entrant dans le tombeau, elles virent, assis à droite, un jeune homme vêtu de blanc. Elles furent saisies de frayeur. 
${}^{6}Mais il leur dit : « Ne soyez pas effrayées ! Vous cherchez Jésus de Nazareth, le Crucifié ? Il est ressuscité : il n’est pas ici. Voici l’endroit où on l’avait déposé. 
${}^{7}Et maintenant, allez dire à ses disciples et à Pierre : “Il vous précède en Galilée. Là vous le verrez, comme il vous l’a dit.” »
${}^{8}Elles sortirent et s’enfuirent du tombeau, parce qu’elles étaient toutes tremblantes et hors d’elles-mêmes. Elles ne dirent rien à personne, car elles avaient peur.
${}^{9}Ressuscité le matin, le premier jour de la semaine, Jésus apparut d’abord à Marie Madeleine, de laquelle il avait expulsé sept démons. 
${}^{10}Celle-ci partit annoncer la nouvelle à ceux qui, ayant vécu avec lui, s’affligeaient et pleuraient. 
${}^{11}Quand ils entendirent que Jésus était vivant et qu’elle l’avait vu, ils refusèrent de croire.
${}^{12}Après cela, il se manifesta sous un autre aspect à deux d’entre eux qui étaient en chemin pour aller à la campagne. 
${}^{13}Ceux-ci revinrent l’annoncer aux autres, qui ne les crurent pas non plus.
${}^{14}Enfin, il se manifesta aux Onze eux-mêmes pendant qu’ils étaient à table : il leur reprocha leur manque de foi et la dureté de leurs cœurs parce qu’ils n’avaient pas cru ceux qui l’avaient contemplé ressuscité. 
${}^{15}Puis il leur dit : « Allez dans le monde entier. Proclamez l’Évangile à toute la création. 
${}^{16}Celui qui croira et sera baptisé sera sauvé ; celui qui refusera de croire sera condamné. 
${}^{17}Voici les signes qui accompagneront ceux qui deviendront croyants : en mon nom, ils expulseront les démons ; ils parleront en langues nouvelles ; 
${}^{18}ils prendront des serpents dans leurs mains et, s’ils boivent un poison mortel, il ne leur fera pas de mal ; ils imposeront les mains aux malades, et les malades s’en trouveront bien. »
${}^{19}Le Seigneur Jésus, après leur avoir parlé, fut enlevé au ciel et s’assit à la droite de Dieu. 
${}^{20}Quant à eux, ils s’en allèrent proclamer partout l’Évangile. Le Seigneur travaillait avec eux et confirmait la Parole par les signes qui l’accompagnaient.
