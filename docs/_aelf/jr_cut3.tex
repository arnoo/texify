  
  
      
         
      \bchapter{}
      \begin{verse}
${}^{1}La parole du Seigneur fut adressée au prophète Jérémie contre les nations.
      
         
${}^{2}Sur l’Égypte. Contre l’armée du pharaon Néko, roi d’Égypte, qui se trouvait près de l’Euphrate, à Karkémish, et qui fut battue par Nabucodonosor, roi de Babylone, la quatrième année du règne de Joakim, fils de Josias, roi de Juda.
       
${}^{3}Préparez boucliers et rondaches,
        en avant pour la bataille !
${}^{4}Harnachez les chevaux,
        montez sur les chars !
        \\Alignez-vous, avec vos casques,
        fourbissez les lances,
        revêtez les cuirasses !
${}^{5}Pourquoi ce spectacle ?
        \\Ils sont pris de panique,
        ils battent en retraite ;
        \\leurs guerriers sont écrasés,
        ils s’enfuient éperdument,
        ne se retournent pas :
        \\de tous côtés, c’est l’épouvante !
        – oracle du Seigneur.
${}^{6}Que le plus agile ne s’enfuie pas !
        Que le guerrier ne s’échappe !
        \\Au nord, sur la rive de l’Euphrate,
        ils trébuchent, ils tombent !
${}^{7}Qui est-ce qui monte comme le Nil,
        et dont les eaux bouillonnent
        comme les eaux des fleuves ?
${}^{8}C’est l’Égypte qui monte comme le Nil,
        et ses eaux bouillonnent
        comme les eaux des fleuves.
        \\Elle disait :
        \\« Je monterai, je submergerai la terre,
        je ferai périr toute ville avec ses habitants. »
${}^{9}Les chevaux, chargez !
        Les chars, foncez !
        \\Et que sortent les guerriers,
        ceux de Koush et de Pouth qui manient le bouclier,
        ceux de Loud qui manient et tendent l’arc.
${}^{10}Pour le Seigneur, Dieu de l’univers,
        \\ce jour est un jour de revanche
        où il prendra sa revanche sur ses adversaires :
        \\l’épée dévore et se rassasie,
        elle s’abreuve de leur sang.
        \\Oui, quel sacrifice
        pour le Seigneur, Dieu de l’univers,
        dans le pays du nord, près de l’Euphrate !
${}^{11}Vierge, fille d’Égypte,
        monte en Galaad et prends du baume !
        \\En vain tu multiplies les remèdes :
        rien ne peut cicatriser ta plaie.
${}^{12}Les nations apprennent ta honte,
        et la terre est remplie de ta clameur,
        \\car le guerrier a trébuché sur le guerrier,
        et tous deux sont tombés ensemble.
       
${}^{13}Parole que le Seigneur adressa au prophète Jérémie, quand Nabucodonosor, roi de Babylone, arriva pour frapper le pays d’Égypte.
       
${}^{14}Annoncez-le en Égypte,
        \\faites-le entendre à Migdol,
        faites-le entendre à Noph et à Tapanès,
        \\dites : Dresse-toi, tiens-toi prêt,
        car autour de toi l’épée dévore !
${}^{15}Pourquoi donc Apis, ton dieu, a-t-il fui ?
        Ce taureau, pourquoi n’a-t-il pas résisté ?
        \\C’est que le Seigneur l’a renversé !
${}^{16}Il en fait trébucher un grand nombre,
        chacun tombe sur son compagnon.
        \\Ils se disent :
        \\« Debout ! Retournons vers notre peuple,
        dans notre pays natal, loin de l’épée impitoyable. »
${}^{17}À Pharaon, roi d’Égypte, donnez ce nom :
        « Vacarme et Rendez-vous manqué » !
${}^{18}Je suis vivant,
        – oracle du Roi qui a pour nom « Le Seigneur de l’univers ».
        \\Oui, comme le Thabor domine les monts,
        et comme le Carmel domine la mer,
        il vient !
${}^{19}Prépare ton bagage de déportée,
        habitante et fille d’Égypte,
        \\car Noph sera livrée à la désolation,
        elle sera brûlée, vidée de ses habitants.
${}^{20}L’Égypte était une génisse de belle allure,
        mais du nord un frelon est venu sur elle.
${}^{21}Même ses mercenaires au milieu d’elle
        comme des veaux à l’engrais,
        \\même eux se sont détournés ;
        ensemble ils ont fui, ils n’ont pas résisté,
        \\car le jour de leur débâcle est venu,
        le temps de leur châtiment.
${}^{22}Son bruit est celui d’un serpent qui s’en va,
        car on marche avec une armée,
        \\on vient sur elle avec des haches,
        comme ceux qui abattent les arbres.
${}^{23}Ils coupent sa forêt – oracle du Seigneur –,
        car elle est impénétrable ;
        \\oui, ils sont plus nombreux que des sauterelles,
        et personne ne peut les compter.
${}^{24}La fille d’Égypte est couverte de honte,
        livrée aux mains du peuple du nord.
       
${}^{25}Le Seigneur de l’univers, le Dieu d’Israël, parle : « Je vais châtier Amon, dieu de Thèbes, Pharaon, l’Égypte, ses dieux et ses rois, Pharaon et ceux qui mettent leur confiance en lui. 
${}^{26}Je les livrerai aux mains de ceux qui en veulent à leur vie, aux mains de Nabucodonosor, roi de Babylone, et aux mains de ses serviteurs. Et après cela, l’Égypte sera habitée comme aux jours d’autrefois » – oracle du Seigneur.
${}^{27}Mais toi, Jacob mon serviteur, ne crains pas,
        ne tremble pas, Israël,
        \\car je vais te sauver des terres lointaines,
        sauver ta descendance de la terre où elle est captive.
        \\Jacob reviendra, il sera en sécurité,
        tranquille, sans personne qui l’inquiète.
${}^{28}Toi, Jacob mon serviteur, ne crains pas
        – oracle du Seigneur –,
        car je suis avec toi.
        \\Oui, j’exterminerai toutes les nations
        parmi lesquelles je t’ai chassé,
        \\et toi, je ne t’exterminerai pas ;
        \\mais je te corrigerai selon le droit
        et ne te laisserai pas impuni.
      
         
      \bchapter{}
      \begin{verse}
${}^{1}La parole du Seigneur fut adressée au prophète Jérémie au sujet des Philistins, avant que Pharaon n’ait frappé Gaza.
${}^{2}Ainsi parle le Seigneur :
        \\Voici que les eaux montent du nord,
        elles deviennent un torrent débordant,
        \\elles débordent sur le pays et ce qu’il contient,
        sur la ville et ses habitants.
        \\Les hommes crient,
        tous les habitants du pays gémissent.
${}^{3}Au bruit des coursiers,
        au martèlement de leurs sabots,
        \\au tumulte des chars qui s’ébranlent
        dans un grondement de roues,
        \\les pères ne se soucient plus de leurs fils,
        tant leurs mains sont défaillantes.
${}^{4}Puisque le jour est venu
        de la dévastation de tous les Philistins
        et de la suppression du moindre secours à Tyr et à Sidon,
        \\le Seigneur dévaste les Philistins,
        ce qui reste de l’île de Crète.
${}^{5}On impose à Gaza d’être tondue,
        Ascalon est réduite au silence.
        \\Vous, les rescapés de leur vallée,
        combien de temps encore
        vous ferez-vous des incisions ?
${}^{6}Hélas ! Épée du Seigneur,
        jusqu’où iras-tu sans te reposer ?
        \\Retire-toi dans ton fourreau,
        fais relâche et reste calme !
${}^{7}Comment pourrais-tu te reposer ?
        \\– Le Seigneur lui a donné des ordres.
        Près d’Ascalon, sur le rivage de la mer,
        il lui a fixé un rendez-vous.
      
         
      \bchapter{}
         
${}^{1}Sur Moab.
        
           
         
        \\Ainsi parle le Seigneur de l’univers, le Dieu d’Israël :
        \\Malheur à Nébo : elle est dévastée !
        \\Dans la honte, Qiryataïm a été prise ;
        dans la honte, la citadelle s’est effondrée.
${}^{2}Plus de louange pour Moab.
        \\À Heshbone, on a tramé contre elle un malheur :
        « Allons, retranchons-la des nations ! »
        \\Et toi aussi, Madmène, tu seras réduite au silence,
        l’épée te suivra de près.
${}^{3}Clameur et cri, depuis Horonaïm :
        « Dévastation et grand désastre ! »
${}^{4}C’est le désastre en Moab :
        vers Sohar faites entendre un cri.
${}^{5}Oui, pleurs sur pleurs,
        en montant la montée de Louhith ;
        \\oui, à la descente de Horonaïm,
        on entend les adversaires crier au désastre.
${}^{6}Fuyez donc, sauvez votre vie,
        devenez comme Aroër dans le désert !
${}^{7}Puisque tu as mis ta confiance
        dans tes ouvrages et tes réserves,
        \\toi aussi, tu seras pris ;
        \\Camosh, ton dieu, partira en exil,
        avec ses prêtres et ses princes, tous ensemble.
${}^{8}Un dévastateur vient sur chaque ville,
        aucune ville ne lui échappe.
        \\La Vallée disparaîtra, le Plateau sera saccagé,
        ainsi que le Seigneur l’a dit.
${}^{9}Donnez donc des ailes à Moab :
        qu’il vole, qu’il s’envole !
        \\Ses villes seront réduites à la désolation,
        sans aucun habitant.
${}^{10}Maudit soit celui qui fait l’œuvre du Seigneur avec indolence,
        et maudit, celui qui retient son épée de verser le sang !
        
           
         
${}^{11}Moab était tranquille depuis sa jeunesse,
        il reposait comme un vin sur sa lie ;
        \\n’étant pas versé d’un vase dans un autre
        – n’étant pas allé en exil –,
        \\tout son goût lui restait
        et son parfum n’était pas altéré.
${}^{12}C’est pourquoi, voici venir des jours
        – oracle du Seigneur –,
        où je lui enverrai des transvaseurs qui le transvaseront ;
        \\ils renverseront ses vases
        et casseront ses cruches.
${}^{13}Alors Moab rougira de Camosh,
        comme la maison d’Israël a rougi de Béthel,
        en qui elle avait mis sa confiance.
${}^{14}Comment pouvez-vous dire : « Nous sommes des héros,
        des soldats faits pour la guerre » ?
${}^{15}Moab est dévasté, on monte à l’assaut de ses villes,
        et la fleur de sa jeunesse descend à l’abattoir
        – oracle du Roi qui a pour nom « Le Seigneur de l’univers ».
${}^{16}Elle est proche, la débâcle de Moab,
        et son malheur arrive en grande hâte.
${}^{17}Faites pour lui les gestes du deuil, vous tous qui l’entourez,
        vous tous qui connaissez son nom.
        \\Dites : Comment ! Il est brisé, le sceptre fort,
        le bâton magnifique !
        
           
         
${}^{18}Habitante et fille de Dibone,
        descends de la gloire,
        habite au pays de la soif,
        \\car le dévastateur de Moab monte à l’assaut contre toi,
        il détruit tes forteresses.
${}^{19}Habitante d’Aroër,
        tiens-toi sur le chemin et observe ;
        \\interroge le fuyard et la fugitive,
        et demande : « Que s’est-il passé ? »
${}^{20}Moab est dans la honte,
        il s’est effondré.
        \\Hurlez, poussez des cris !
        \\Au ravin de l’Arnon proclamez :
        « Moab est dévasté ! »
${}^{21}Le jugement arrive pour le pays du Plateau,
        pour Holone et pour Yahasa ;
        \\il arrive sur Méfaath,
${}^{22}sur Dibone, Nébo, Beth-Diblatayim,
${}^{23}sur Qiryataïm, Beth-Gamoul, Beth-Méone,
${}^{24}sur Qeriyoth et sur Bosra,
        \\sur toutes les villes du pays de Moab,
        proches ou lointaines.
${}^{25}Elle est abattue, la puissance de Moab,
        elle est brisée, la force de son bras,
        – oracle du Seigneur.
        
           
         
${}^{26}Enivrez-le,
        puisqu’il s’est fait plus grand que le Seigneur !
        \\Que Moab se débatte dans ses vomissures,
        qu’il devienne, lui aussi, un objet de risée !
${}^{27}Pour toi, Moab, Israël n’a-t-il pas été un objet de risée ?
        A-t-il été trouvé au milieu des voleurs,
        \\pour que tu hoches la tête
        chaque fois que tu parles de lui ?
${}^{28}Quittez les villes,
        demeurez dans les rochers,
        habitants de Moab !
        \\Soyez comme la colombe
        qui niche dans les parois du gouffre.
        
           
         
${}^{29}Nous avons appris l’orgueil de Moab,
        son immense orgueil,
        \\sa morgue, son orgueil arrogant,
        son cœur hautain.
${}^{30}Moi, je connais sa présomption
        – oracle du Seigneur ;
        \\ses bavardages ne sont rien,
        ses actes ne sont rien.
${}^{31}C’est pourquoi sur Moab je gémis,
        sur Moab tout entier j’élève mon cri ;
        pour les hommes de Qir-Hérès on soupire.
${}^{32}Plus que l’on pleure sur Yazèr,
        je pleure sur toi, vigne de Sibma ;
        \\tes sarments passaient la mer,
        ils atteignaient la mer de Yazèr ;
        \\mais le dévastateur s’est jeté
        sur ta récolte et sur ta vendange.
${}^{33}Allégresse et jubilation se sont retirées
        des vergers et de la terre de Moab.
        \\J’ai tari le vin des cuves ;
        \\on ne foule plus en criant,
        le cri du fouleur a disparu.
        
           
         
${}^{34}La clameur de Heshbone parvient jusqu’à Élalé ;
        jusqu’à Yahaç, on donne de la voix,
        \\depuis Soar jusqu’à Horonaïm, Églath-Shelishiya,
        car même les eaux de Nimrim ne sont plus que désolation.
${}^{35}Je ferai disparaître de Moab
        – oracle du Seigneur –
        \\celui qui offre des holocaustes sur les lieux sacrés
        et celui qui brûle de l’encens pour ses dieux.
${}^{36}Voilà pourquoi sur Moab
        mon cœur se plaint comme une flûte ;
        \\sur les hommes de Qir-Hérès
        mon cœur se plaint comme une flûte ;
        \\voilà qu’ils ont perdu ce qu’ils avaient gagné.
${}^{37}Oui, tous les crânes sont rasés,
        toutes les barbes sont coupées ;
        \\sur toutes les mains, des incisions,
        et sur les reins, de la toile à sac.
${}^{38}Sur toutes les terrasses de Moab et sur ses places,
        ce n’est plus que lamentation,
        \\car j’ai brisé Moab comme un objet dont personne ne veut
        – oracle du Seigneur.
${}^{39}Comment ! Il s’est effondré.
        Poussez des gémissements.
        \\Comment ! Moab tourne le dos, plein de honte.
        Moab est devenu objet de risée et d’effroi
        pour tous ceux qui l’entourent.
        
           
         
${}^{40}Oui, ainsi parle le Seigneur :
        \\Voici qu’il plane comme un aigle, le dévastateur,
        il étend ses ailes sur Moab.
${}^{41}Les cités sont prises,
        les places fortes, occupées.
        \\En ce jour-là, le cœur des guerriers de Moab
        sera comme celui d’une femme en travail.
${}^{42}Moab, saccagé, cessera d’être un peuple,
        lui qui s’est fait plus grand que le Seigneur.
${}^{43}La frayeur, la fosse et le filet
        sont pour toi, habitant de Moab,
        – oracle du Seigneur.
${}^{44}Celui qui fuit devant la frayeur
        tombe dans la fosse ;
        \\celui qui remonte de la fosse
        est pris dans le filet !
        \\Oui, je fais venir sur lui, sur Moab,
        l’année de son châtiment
        – oracle du Seigneur.
${}^{45}À l’ombre de Heshbone
        se sont arrêtés les fuyards à bout de forces,
        \\mais un feu a jailli de Heshbone
        et une flamme du milieu de Séhone,
        \\pour dévorer la frontière de Moab
        et le centre des fauteurs de tumulte.
        
           
         
${}^{46}Malheur à toi, Moab !
        Il est perdu, le peuple du dieu Camosh :
        \\tes fils sont emmenés captifs,
        et tes filles, captives.
${}^{47}Mais dans les derniers jours
        je ramènerai les captifs de Moab
        – oracle du Seigneur.
        
           
         
        \\Là, s’arrête le jugement de Moab.
        
           
      
         
      \bchapter{}
         
${}^{1}Sur les fils d’Ammone.
        
           
         
        Ainsi parle le Seigneur :
        \\Israël n’a-t-il pas de fils ?
        N’a-t-il pas d’héritier ?
        \\Pourquoi le dieu Milcom a-t-il pris possession de Gad,
        et son peuple en occupe-t-il les villes ?
${}^{2}Eh bien, voici venir des jours
        – oracle du Seigneur – 
        \\où je ferai retentir le cri de guerre
        contre Rabba, la ville des fils d’Ammone :
        \\elle deviendra un tas de ruines, une désolation ;
        ses filles, les autres villes, seront incendiées,
        \\et Israël reprendra possession de son héritage,
        dit le Seigneur.
${}^{3}Tu peux gémir, Heshbone, sur la dévastation de Aï.
        Filles de Rabba, poussez des cris,
        \\revêtez-vous de toile à sac, lamentez-vous,
        errez sur les murailles,
        \\car Milcom s’en va en exil,
        avec ses prêtres et ses princes, tous ensemble.
${}^{4}Pourquoi te glorifier dans les vallées ?
        Ta vallée est fertile, fille rebelle,
        \\toi qui te fies à tes réserves, et dis :
        « Qui viendra vers moi ? »
${}^{5}Voici que je fais venir contre toi la Terreur
        – oracle du Seigneur, Dieu de l’univers –,
        \\elle viendra de tous ceux d’alentour.
        \\Vous serez chassés, chacun devant soi,
        et personne ne rassemblera les fuyards.
${}^{6}Mais après cela,
        je ramènerai les captifs des fils d’Ammone
        – oracle du Seigneur.
        
           
${}^{7}Sur Édom.
         
        \\Ainsi parle le Seigneur de l’univers :
        \\N’y a-t-il plus de sagesse à Témane ?
        L’art du conseil fait-il défaut aux intelligents ?
        Leur sagesse s’est-elle évanouie ?
${}^{8}Fuyez ! Détournez-vous !
        \\Creusez des trous pour y habiter,
        habitants de Dédane,
        \\car je fais venir la débacle sur Ésaü
        le temps où je le châtierai.
${}^{9}Si des vendangeurs viennent chez toi,
        ils ne laisseront pas de quoi grappiller ;
        \\si des voleurs viennent de nuit,
        ils saccageront à leur guise.
${}^{10}C’est moi qui ai dénudé Ésaü,
        qui ai découvert ses cachettes ;
        il ne peut plus se dissimuler.
        \\Sa descendance est anéantie,
        ainsi que ses frères et ses voisins ;
        il n’existe plus.
${}^{11}Abandonne tes orphelins : c’est moi qui fais vivre ;
        et tes veuves, elles peuvent compter sur moi.
${}^{12}Oui, ainsi parle le Seigneur :
        \\Alors qu’ils ont dû boire la coupe,
        ceux qui n’étaient pas condamnés à la boire,
        \\toi, tu en serais quitte ?
        \\Non ! Tu n’en seras pas quitte,
        et tu devras boire.
${}^{13}Oui, j’en fais serment par moi-même
        – oracle du Seigneur –,
        \\Bosra deviendra un lieu désolé, une ruine,
        un objet d’insulte et de malédiction,
        et toutes ses villes deviendront des ruines pour toujours.
${}^{14}J’ai entendu, de la part du Seigneur, une nouvelle ;
        \\un messager a été envoyé parmi les nations :
        « Rassemblez-vous, marchez contre elle !
        Debout pour le combat ! »
${}^{15}Voici : je t’abaisse parmi les nations,
        je te livre au mépris parmi les hommes.
${}^{16}L’effroi que tu inspires t’a trompé,
        ainsi que l’arrogance de ton cœur,
        \\toi qui demeures dans les creux du rocher,
        toi qui occupes le sommet de la colline.
        \\Comme l’aigle, tu fais ton nid dans les hauteurs,
        mais je t’en ferai descendre !
        – oracle du Seigneur.
${}^{17}Édom deviendra un lieu désolé ;
        \\quiconque passera près de lui se désolera
        et fera entendre un sifflement de stupeur
        à la vue de toutes ses plaies.
${}^{18}Comme il en fut lors de la catastrophe de Sodome et Gomorrhe
        et des villes voisines – dit le Seigneur –,
        \\plus personne n’y habitera,
        aucun être humain n’y séjournera.
         
${}^{19}Le dévastateur, voici qu’il monte comme un lion
        des maquis du Jourdain vers les enclos prospères.
        \\Oui, en un clin d’œil,
        \\de là je ferai déguerpir Édom,
        et sur lui j’établirai celui qui est choisi.
        \\En effet, qui est semblable à moi ?
        Qui m’assignera en justice ?
        Et quel pasteur tiendra devant moi ?
${}^{20}Écoutez donc le plan
        que le Seigneur a conçu pour Édom,
        \\et les projets qu’il a formés
        pour les habitants de Témane :
        \\vraiment, les petits du troupeau les entraîneront,
        vraiment, à cause d’eux, leurs enclos seront désolés.
${}^{21}Au bruit de leur chute, la terre tremble ;
        \\c’est un cri dont l’écho se fait entendre
        jusqu’à la mer des Roseaux.
${}^{22}Le dévastateur, voici qu’il monte comme un aigle,
        il plane, il étend ses ailes sur Bosra.
        \\En ce jour-là, le cœur des guerriers d’Édom
        sera comme celui d’une femme en travail.
${}^{23}Sur Damas.
         
        \\Hamath et Arpad sont couvertes de honte,
        \\car elles ont appris une mauvaise nouvelle, elles sont agitées :
        tourmente sur la mer que rien ne peut calmer.
${}^{24}Damas, découragée, se prépare à fuir.
        Un frisson l’étreint ;
        \\elle est saisie d’angoisse et de douleurs
        telle une femme qui accouche.
${}^{25}Comment ! Ne serait-elle pas abandonnée,
        la ville célèbre, la cité joyeuse ?
${}^{26}De fait, ses jeunes gens tomberont sur ses places,
        et tous les hommes de guerre seront réduits au silence,
        en ce jour-là – oracle du Seigneur de l’univers.
${}^{27}Alors, je mettrai le feu aux remparts de Damas,
        il dévorera les palais de Ben-Hadad.
${}^{28}Sur Qédar et les royaumes de Haçor
        qui furent battus par Nabucodonosor, roi de Babylone.
        \\Ainsi parle le Seigneur :
        \\Debout ! Montez à l’assaut de Qédar
        et dévastez les fils de l’Orient !
${}^{29}On s’empare de leurs tentes et de leurs troupeaux,
        de leurs abris et de toutes leurs affaires ;
        \\on enlève leurs chameaux,
        \\et on leur crie :
        « Épouvante de tous côtés ! »
${}^{30}Fuyez ! Vite, sauvez-vous !
        \\Creusez des trous pour y habiter,
        habitants de Haçor
        – oracle du Seigneur –,
        \\car Nabucodonosor, roi de Babylone,
        a conçu un plan contre vous,
        il a formé contre vous un projet.
${}^{31}Debout ! Montez à l’assaut de la nation paisible
        qui habite en sécurité – oracle du Seigneur.
        \\Ils n’ont ni portes ni verrous,
        ils demeurent à l’écart.
${}^{32}Leurs chameaux deviendront une proie,
        et leurs nombreux troupeaux, du butin.
        \\Je disperserai à tout vent les hommes aux tempes rasées ;
        de tous côtés, je leur apporterai la débacle,
        – oracle du Seigneur.
${}^{33}Haçor deviendra un repaire de chacals,
        un lieu à jamais désolé.
        \\Plus personne n’y habitera,
        aucun être humain n’y séjournera.
${}^{34}Parole du Seigneur
        \\qui fut adressée au prophète Jérémie, sur Élam,
        au début du règne de Sédécias, roi de Juda :
         
${}^{35}Ainsi parle le Seigneur de l’univers :
        \\Je vais briser l’arc d’Élam, d’où lui vient sa puissance.
${}^{36}Je ferai venir sur Élam
        quatre vents des quatre extrémités du ciel,
        \\je le disperserai à tous ces vents :
        \\il n’y aura pas une nation
        où ne viendront des fugitifs d’Élam.
${}^{37}Je ferai trembler Élam devant ses ennemis,
        devant ceux qui en veulent à sa vie ;
        \\je ferai venir sur Élam le malheur : mon ardente colère !
        – oracle du Seigneur.
        \\J’enverrai l’épée à leur poursuite
        jusqu’à en finir avec eux !
${}^{38}J’établirai mon trône sur Élam
        et j’en ferai disparaître roi et princes
        – oracle du Seigneur.
${}^{39}Mais dans les derniers jours
        je ramènerai les captifs d’Élam
        – oracle du Seigneur.
      
         
      \bchapter{}
      \begin{verse}
${}^{1}Parole que le Seigneur a prononcée sur Babylone, le pays des Chaldéens, par l’intermédiaire du prophète Jérémie :
${}^{2}Annoncez-le parmi les nations,
        \\faites-le entendre et levez l’étendard,
        \\faites-le entendre, ne le cachez pas,
        \\dites-le : Babylone est prise,
        le dieu Bel est couvert de honte,
        Mardouk, terrorisé ;
        \\ses faux dieux sont couverts de honte,
        ses idoles immondes, terrorisées ;
${}^{3}car depuis le nord une nation monte contre elle,
        pour livrer son pays à la dévastation ;
        \\nul n’y habitera plus :
        \\depuis l’homme jusqu’au bétail,
        ils ont fui, ils sont partis.
         
        ${}^{4}En ces jours-là, en ce temps-là,
        – oracle du Seigneur –
        \\les fils d’Israël et les fils de Juda
        viendront tous ensemble ;
        \\ils marcheront en pleurant,
        ils chercheront le Seigneur leur Dieu.
        ${}^{5}Ils demanderont le chemin de Sion
        et tourneront vers elle leur visage :
        \\« Venez, attachons-nous au Seigneur
        par une alliance éternelle, inoubliable. »
        ${}^{6}Les gens de mon peuple étaient des brebis perdues :
        ceux qui étaient\\leurs bergers les avaient égarées,
        \\ils les avaient laissées errer dans les montagnes ;
        elles allaient de montagne en colline,
        elles avaient oublié leur bercail.
        ${}^{7}Tous ceux qui les rencontraient les dévoraient,
        et leurs adversaires disaient :
        \\« Nous ne sommes pas en faute,
        car ces gens\\ont péché
        contre le Seigneur, la demeure de justice,
        contre le Seigneur, l’espoir\\de leurs pères. »
         
${}^{8}Fuyez du milieu de Babylone,
        du pays des Chaldéens, sortez,
        \\et soyez comme des béliers
        à la tête des brebis !
${}^{9}Car voici que je suscite
        pour monter à l’assaut de Babylone
        \\un rassemblement de grandes nations
        venant des pays du nord ;
        \\elles prendront position contre elle
        et s’en empareront.
        \\Leurs flèches seront comme celles d’un habile guerrier
        qui ne revient pas les mains vides.
${}^{10}La Chaldée sera mise au pillage :
        tous ceux qui la pilleront seront rassasiés
        – oracle du Seigneur.
${}^{11}Oui, vous vous réjouissez, oui, vous bondissez de joie,
        vous qui dépouillez mon héritage ;
        \\oui, vous gambadez comme génisses dans les prés,
        vous hennissez comme des étalons ;
${}^{12}mais votre mère est accablée de honte,
        celle qui vous a enfantés est couverte de confusion.
        \\La voici à la traîne des nations :
        un désert, une terre sèche et aride.
${}^{13}À cause de l’irritation du Seigneur,
        \\elle ne sera plus habitée
        mais sera tout entière un lieu désolé ;
        \\quiconque passera près de Babylone se désolera
        et fera entendre un sifflement de stupeur
        à la vue de toutes ses plaies.
${}^{14}Prenez position autour de Babylone,
        vous tous qui tendez l’arc ;
        \\tirez sur elle, n’épargnez pas les flèches,
        car elle a péché contre le Seigneur.
${}^{15}Faites retentir autour d’elle le cri de guerre,
        elle se rend.
        \\Ses défenses tombent,
        ses remparts sont démolis.
        \\Oui, c’est la revanche du Seigneur !
        \\Vengez-vous d’elle,
        faites-lui ce qu’elle a fait.
${}^{16}De Babylone retranchez celui qui sème
        et celui qui manie la faucille au temps de la moisson.
        \\Devant l’épée impitoyable,
        \\que chacun se tourne vers son peuple,
        que chacun s’enfuie dans son pays !
${}^{17}Israël était une brebis égarée,
        pourchassée par des lions :
        \\le premier qui la dévora fut le roi d’Assour ;
        \\le dernier, qui l’acheva jusqu’aux os,
        ce fut Nabucodonosor, roi de Babylone.
${}^{18}C’est pourquoi, ainsi parle le Seigneur de l’univers,
        le Dieu d’Israël :
        \\Voici que je vais châtier
        le roi de Babylone et son pays,
        comme j’ai châtié le roi d’Assour.
         
        ${}^{19}Je ramènerai Israël à son enclos
        pour le faire paître sur le Carmel et le Bashane ;
        \\sur les montagnes d’Éphraïm et de Galaad,
        il sera rassasié.
        ${}^{20}En ces jours-là, en ce temps-là,
        – oracle du Seigneur –
        \\on cherchera la faute d’Israël,
        mais il n’y en aura plus ;
        \\on cherchera les péchés de Juda,
        mais on n’en trouvera plus,
        car je pardonnerai à ce reste que j’aurai préservé.
         
${}^{21}Monte, peuple du nord, au pays de « Double révolte »,
        monte contre lui
        et contre les habitants de « Châtiment » !
        \\Massacre, et voue à l’anathème les derniers d’entre eux
        – oracle du Seigneur – 
        et agis en tout comme je te l’ai ordonné.
${}^{22}Bruit de guerre dans le pays,
        et grand désastre !
${}^{23}Comment est-il abattu et brisé,
        le Marteau de toute la terre ?
        \\Comment est-elle devenue un lieu désolé,
        Babylone, parmi les nations ?
${}^{24}Je t’ai tendu un piège,
        et tu as été attrapée, Babylone,
        \\mais toi, tu ne l’as pas su ;
        \\tu as été trouvée, et tu as été empoignée,
        pour avoir provoqué le Seigneur.
         
${}^{25}Le Seigneur a ouvert ses réserves,
        il en a tiré les armes de son indignation.
        \\Oui, telle est l’œuvre du Seigneur, Dieu de l’univers,
        sur la terre des Chaldéens.
${}^{26}De partout, venez vers elle,
        ouvrez ses granges !
        \\Faites-en des tas comme des tas de gerbes
        et vouez-la à l’anathème :
        qu’il n’en reste rien !
${}^{27}Massacrez tous ses taurillons,
        qu’ils descendent à l’abattoir !
        \\Malheur à eux, car leur jour est arrivé,
        le temps de leur châtiment !
         
${}^{28}Voix des fuyards et des rescapés
        sortis du pays de Babylone
        \\pour annoncer dans Sion
        \\la revanche du Seigneur notre Dieu,
        la revanche de son temple !
         
${}^{29}Convoquez à Babylone les archers,
        tous ceux qui tendent l’arc.
        \\Dressez le camp contre elle, tout autour ;
        que personne n’en réchappe !
        \\Rendez-lui selon sa conduite,
        faites-lui tout ce qu’elle a fait,
        \\elle qui s’est dressée avec arrogance
        contre le Seigneur, contre le Saint d’Israël.
${}^{30}Voilà pourquoi ses jeunes gens tomberont sur ses places,
        et tous ses hommes de guerre seront réduits au silence
        en ce jour-là – oracle du Seigneur.
${}^{31}Me voici contre toi, Arrogance
        – oracle du Seigneur, Dieu de l’univers –,
        \\car ton jour est arrivé,
        le temps où je vais te châtier.
${}^{32}Alors, Arrogance trébuchera, elle tombera :
        et personne pour la relever.
        \\Je mettrai le feu à ses villes :
        il dévorera tout ce qui l’entoure.
         
${}^{33}Ainsi parle le Seigneur de l’univers :
        \\Les fils d’Israël sont opprimés,
        ainsi que les fils de Juda.
        \\Tous ceux qui les ont déportés les retiennent de force
        et ne veulent pas les relâcher.
${}^{34}Mais il est fort, leur rédempteur
        qui a pour nom « Le Seigneur de l’univers ».
        \\Il les défendra, il défendra leur cause,
        pour donner au pays le repos
        et faire trembler les habitants de Babylone.
${}^{35}Guerre aux Chaldéens
        – oracle du Seigneur –
        \\et aux habitants de Babylone,
        à ses princes et à ses sages !
${}^{36}Guerre aux devins : qu’ils perdent la tête !
        Guerre à ses soldats : qu’ils soient terrorisés !
${}^{37}Guerre à ses chevaux et à ses chars,
        à tout ce mélange de peuples qu’elle abrite :
        qu’ils soient des femmelettes !
        \\Guerre à tous ses trésors :
        qu’ils soient pillés !
${}^{38}Sécheresse dans ses cours d’eau :
        qu’ils tarissent,
        \\car c’est une terre à idoles,
        où l’on délire devant des horreurs !
${}^{39}C’est pourquoi les chats sauvages
        s’y installeront avec les hyènes,
        les autruches s’y installeront ;
        \\elle ne sera plus jamais habitée :
        on n’y demeurera plus,
        de génération en génération.
         
${}^{40}Comme il en fut lorsque Dieu provoqua la catastrophe de Sodome et Gomorrhe
        et des villes voisines – oracle du Seigneur –,
        \\plus personne n’y habitera,
        aucun être humain n’y séjournera.
         
${}^{41}Voici qu’un peuple arrive du nord ;
        aux confins de la terre, s’éveille une grande nation
        avec des rois nombreux.
${}^{42}Ils empoignent arc et javelot,
        ils sont cruels, sans aucune compassion ;
        \\leur voix gronde comme la mer.
        \\Sur des chevaux ils sont montés,
        rangés comme un seul homme pour la bataille,
        contre toi, fille de Babylone.
${}^{43}Le roi de Babylone a entendu la rumeur ;
        ses mains faiblissent, l’angoisse le saisit,
        comme les douleurs d’une femme qui accouche.
${}^{44}Le peuple du nord, voici qu’il monte comme un lion,
        des maquis du Jourdain vers les enclos prospères.
        \\Oui, en un clin d’œil,
        de là je ferai déguerpir les Chaldéens,
        et sur eux j’établirai celui qui est choisi.
        \\En effet, qui est semblable à moi ?
        Qui m’assignera en justice ?
        Et quel pasteur tiendra devant moi ?
${}^{45}Écoutez donc le plan
        que le Seigneur a conçu pour Babylone,
        \\et les projets qu’il a formés
        pour le pays des Chaldéens :
        \\vraiment, les petits du troupeau les entraîneront,
        vraiment, à cause d’eux, les enclos seront désolés.
${}^{46}Au bruit de la prise de Babylone, la terre est ébranlée :
        on entend son cri chez les nations.
       
      
         
      \bchapter{}
${}^{1}Ainsi parle le Seigneur :
        \\Je vais éveiller contre Babylone
        et contre les habitants de la Chaldée
        un vent destructeur.
${}^{2}J’enverrai à Babylone des étrangers qui la vanneront
        \\et feront le vide sur sa terre
        \\quand ils l’auront cernée au jour du malheur.
        
           
         
${}^{3}Que les archers tirent sur les archers adverses
        et sur ceux qui se pavanent dans leurs cuirasses !
        \\N’épargnez pas ses jeunes gens,
        vouez à l’anathème toute son armée.
${}^{4}Au pays des Chaldéens, les victimes tomberont,
        les blessés graves, dans les rues de Babylone.
${}^{5}Non, Israël, pas plus que Juda, n’est veuf de son Dieu,
        le Seigneur de l’univers,
        \\alors même que leur pays est rempli d’offenses
        envers le Saint d’Israël.
        
           
         
${}^{6}Fuyez du milieu de Babylone,
        sauvez chacun votre vie !
        \\Ne soyez pas anéantis par sa faute :
        pour le Seigneur, c’est le temps de la revanche,
        il va lui rendre son dû.
${}^{7}Babylone était une coupe d’or dans la main du Seigneur,
        elle enivrait toute la terre ;
        \\les nations ont bu de son vin,
        c’est pourquoi elles sont devenues folles.
${}^{8}Soudain, Babylone est tombée,
        elle s’est brisée.
        \\Lamentez-vous sur elle,
        prenez du baume pour sa douleur :
        peut-être guérira-t-elle ?
${}^{9}Nous voulions guérir Babylone,
        mais elle n’est pas guérie.
        \\Abandonnez-la,
        et repartons chacun dans notre pays ;
        \\sa condamnation atteint les cieux
        et s’élève jusqu’aux nues.
${}^{10}Le Seigneur a manifesté
        sa justice envers nous.
        \\Venez ! Allons raconter dans Sion
        l’œuvre du Seigneur notre Dieu.
        
           
         
${}^{11}Aiguisez les flèches,
        emplissez les carquois.
        \\Le Seigneur éveille l’esprit des rois des Mèdes,
        car il forme contre Babylone le projet de la détruire.
        \\Oui, c’est la revanche du Seigneur,
        la revanche de son temple.
${}^{12}Contre les remparts de Babylone, levez l’étendard !
        Renforcez la garde,
        \\postez des gardiens
        et dressez les embuscades !
        \\Oui, selon son projet, le Seigneur accomplira
        ce qu’il a dit contre les habitants de Babylone.
${}^{13}Toi qui demeures auprès des grandes eaux,
        toi qui regorges de trésors,
        \\ta fin est arrivée,
        le terme de tes profits.
${}^{14}Le Seigneur de l’univers l’a juré par lui-même :
        \\« Je te remplirai d’hommes
        nombreux comme les sauterelles ;
        \\ils pousseront contre toi
        des cris de fouleurs. »
        
           
         
${}^{15}Il fait la terre par sa puissance,
        il établit le monde par sa sagesse,
        et par son intelligence il a déployé les cieux.
${}^{16}Quand il donne de la voix,
        et que les eaux grondent dans les cieux,
        \\des extrémités de la terre il fait monter les nuages,
        il lance les éclairs pour la pluie,
        il libère le vent qu’il tenait en réserve.
        
           
         
${}^{17}Tout homme est stupide, faute de connaissance,
        tout orfèvre, méprisable à cause de son idole ;
        \\ce qu’il a coulé est un mensonge :
        pas de souffle en elle !
${}^{18}Ce n’est que vanité, œuvre dérisoire ;
        au temps du châtiment, tout disparaîtra.
${}^{19}Il n’est pas ainsi, Celui qui est la part de Jacob,
        car il a façonné toute chose ;
        \\Israël est la tribu de son héritage,
        son nom est « Le Seigneur de l’univers ».
        
           
         
${}^{20}Tu étais pour moi une massue qui écrase,
        une arme de guerre :
        \\par toi, j’ai écrasé des nations ;
        par toi, j’ai détruit des royaumes ;
${}^{21}par toi, j’ai écrasé le cheval et celui qui le monte ;
        par toi, j’ai écrasé le char et celui qui le monte ;
${}^{22}par toi, j’ai écrasé l’homme et la femme ;
        par toi, j’ai écrasé le vieillard et l’enfant ;
        \\par toi, j’ai écrasé le jeune homme et la jeune fille ;
${}^{23}par toi, j’ai écrasé le berger et son troupeau ;
        \\par toi, j’ai écrasé le laboureur et son attelage ;
        par toi, j’ai écrasé gouverneurs et préfets.
${}^{24}Mais je ferai payer à Babylone
        et à tous les habitants de la Chaldée
        \\tout le mal qu’ils ont fait à Sion, sous vos yeux
        – oracle du Seigneur.
${}^{25}Me voici contre toi, Montagne-qui-Détruit,
        – oracle du Seigneur –
        toi qui détruisais toute la terre.
        \\J’étendrai la main contre toi,
        je te ferai rouler du haut des rochers
        et je te changerai en Montagne-qui-Brûle.
${}^{26}On n’extraira plus de chez toi
        ni pierre d’angle ni pierre de fondation,
        \\car tu seras un lieu à jamais désolé
        – oracle du Seigneur.
        
           
         
${}^{27}Levez l’étendard dans le pays,
        sonnez du cor parmi les nations !
        \\Contre Babylone mobilisez des nations,
        alertez contre elle des royaumes :
        Ararat, Minni, Ashkenaz.
        \\Nommez contre elle un chef de guerre,
        faites charger la cavalerie
        comme des sauterelles cuirassées !
${}^{28}Mobilisez contre elle des nations,
        les rois des Mèdes,
        \\leurs gouverneurs et tous leurs préfets,
        tout leur empire.
${}^{29}La terre tremble et frémit,
        quand se réalisent les projets du Seigneur contre Babylone :
        \\réduire le pays de Babylone en un lieu désolé,
        vidé de ses habitants.
${}^{30}Les braves de Babylone ont cessé le combat,
        ils sont assis dans les places fortes ;
        \\leur bravoure est épuisée,
        ce sont des femmelettes.
        \\On a mis le feu à ses demeures,
        les barres de ses portes sont brisées.
${}^{31}Le courrier court à la rencontre du courrier,
        et le messager, à la rencontre du messager,
        \\pour porter au roi de Babylone ce message :
        sa ville tout entière est prise.
${}^{32}Les passages sont occupés,
        les redoutes, incendiées,
        les hommes de guerre, pris de panique.
${}^{33}Oui, ainsi parle le Seigneur de l’univers, le Dieu d’Israël :
        \\La fille de Babylone est comme une aire à grain
        au temps où elle est aplanie ;
        \\encore un peu,
        et pour elle viendra le temps de la moisson.
        
           
         
${}^{34}Jérusalem dit :
        \\Il m’a dévorée, avalée,
        Nabucodonosor, le roi de Babylone ;
        il m’a laissée telle un plat vide.
        \\Comme un dragon, il m’a engloutie,
        il a rempli son ventre de mes délices ;
        il m’a rejetée.
${}^{35}« Que retombe sur Babylone,
        la violence faite à ma chair ! »,
        dit l’habitante de Sion.
        \\« Et mon sang,
        sur les habitants de la Chaldée ! »,
        dit Jérusalem.
${}^{36}Voilà pourquoi, ainsi parle le Seigneur :
        \\Je vais défendre ta cause,
        assurer ta revanche.
        \\Je dessécherai la mer de Babylone
        et tarirai sa source.
${}^{37}Babylone deviendra un tas de pierres,
        un repaire de chacals,
        \\une désolation, une dérision,
        sans aucun habitant.
        
           
         
${}^{38}Ensemble, ils rugissent comme des lionceaux,
        ils grondent comme des petits de lionne.
${}^{39}Quand ils auront chaud, je leur donnerai de quoi boire,
        je les enivrerai pour les mettre en joie :
        \\ils s’endormiront d’un sommeil éternel
        et ne se réveilleront pas,
        – oracle du Seigneur.
${}^{40}Je les ferai descendre à l’abattoir comme des agneaux,
        comme des béliers et des boucs.
        
           
         
${}^{41}Comment ! Elle est prise, Babylone,
        elle est conquise, la fierté de toute la terre.
        \\Comment ! Elle est devenue un lieu désolé,
        Babylone, parmi les nations.
${}^{42}La mer submerge Babylone,
        ses vagues tumultueuses la recouvrent.
${}^{43}Ses villes sont changées en lieux désolés,
        en terre sèche et aride,
        \\une terre où plus personne n’habite,
        où ne passe plus aucun être humain.
${}^{44}Dans Babylone je sévirai contre le dieu Bel,
        je ferai sortir de sa bouche ce qu’il avait englouti,
        \\et les nations cesseront d’affluer vers lui ;
        même le rempart de Babylone tombera.
${}^{45}Sortez du milieu d’elle,
        vous, mon peuple ;
        \\sauvez chacun votre vie,
        loin de l’ardente colère du Seigneur !
${}^{46}Que votre cœur ne faiblisse pas,
        ne craignez pas la rumeur qui circule dans le pays :
        \\une année, celle-ci ; l’autre année, celle-là ;
        violence dans le pays, tyran contre tyran !
${}^{47}C’est pourquoi, voici venir des jours
        où je sévirai contre les idoles de Babylone ;
        \\tout son pays rougira de honte
        et toutes ses victimes tomberont au milieu d’elle.
${}^{48}Alors le ciel et la terre et tout ce qu’ils contiennent
        pousseront des cris de joie au sujet de Babylone,
        \\car, depuis le nord, viendront sur elle les dévastateurs
        – oracle du Seigneur.
        
           
         
${}^{49}Ô victimes d’Israël,
        Babylone, elle aussi, doit tomber,
        \\comme Babylone a fait tomber
        des victimes sur toute la terre.
${}^{50}Vous qui avez échappé à l’épée,
        allez, ne vous arrêtez pas.
        \\Là-bas, souvenez-vous du Seigneur,
        et que Jérusalem vous revienne en mémoire !
        
           
         
${}^{51}À entendre l’insulte, nous avons rougi de honte,
        et la confusion a couvert notre visage,
        \\quand des étrangers ont envahi
        les lieux saints de la maison du Seigneur.
${}^{52}C’est pourquoi, voici venir des jours
        – oracle du Seigneur –
        \\où je sévirai contre les idoles de Babylone,
        et des victimes gémiront dans tout son pays.
${}^{53}Même si Babylone montait jusqu’au ciel
        et rendait inaccessible sa haute forteresse,
        \\je ferais venir sur elle des dévastateurs
        – oracle du Seigneur.
        
           
         
${}^{54}Clameur et cri depuis Babylone,
        grand désastre au pays des Chaldéens !
${}^{55}Oui, le Seigneur dévaste Babylone,
        il en fait taire la grande clameur.
        \\Les vagues de dévastateurs mugissent comme les grandes eaux
        quand s’élève le vacarme de leurs clameurs.
${}^{56}Oui, le dévastateur arrive sur elle, sur Babylone :
        les guerriers sont capturés,
        leurs arcs sont brisés,
        \\car le Seigneur est un Dieu qui rétribue,
        qui rend à chacun son dû.
${}^{57}J’enivrerai ses princes et ses sages,
        ses gouverneurs, ses préfets et ses guerriers :
        \\ils s’endormiront d’un sommeil éternel
        et ne se réveilleront pas
        – oracle du Roi qui a pour nom « Le Seigneur de l’univers ».
        
           
         
${}^{58}Ainsi parle le Seigneur de l’univers :
        \\Les remparts de Babylone, la spacieuse,
        seront complètement rasés,
        \\et ses portes monumentales,
        consumées par le feu.
        
           
         
        \\C’est ainsi que les peuples peinent pour le néant,
        et les nations, pour le feu :
        \\ils s’épuisent !
        
           
       
${}^{59}Voici l’ordre que le prophète Jérémie donna à Seraya, fils de Nériya, fils de Mahséya, quand il partit pour Babylone avec Sédécias roi de Juda, alors que ce dernier était dans la quatrième année de son règne. Seraya était grand chambellan. 
${}^{60}Jérémie avait mis par écrit dans un seul livre tout le malheur qui allait venir sur Babylone : toutes les paroles contre Babylone écrites ci-dessus. 
${}^{61}Jérémie dit à Seraya : « Dès que tu arriveras à Babylone, tu auras soin de lire toutes ces paroles. 
${}^{62}Et tu diras : “C’est toi, Seigneur, qui as dit de ce lieu qu’il serait supprimé, de sorte qu’il n’y ait plus en lui d’habitants, ni homme ni bétail, et qu’il soit un lieu à jamais désolé.” 
${}^{63}Dès que tu auras achevé la lecture de ce livre, tu lui attacheras une pierre et tu le jetteras au milieu de l’Euphrate. 
${}^{64}Et tu diras : “Ainsi sombrera Babylone. Elle ne se relèvera pas du malheur que je fais venir sur elle.” » Ils s’épuisent !
       
      Jusqu’ici les paroles de Jérémie.
