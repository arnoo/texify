  
  
          
            \bchapter{Psaume}
            Vous bâillonnez la justice
${}^{1}Du maître de chœur. « Ne détruis pas ». De David. À mi-voix.
         
${}^{2}Vraiment, vous bâillonnez la justice, vo\underline{u}s qui jugez !
        \\Est-ce le droit que vous suiv\underline{e}z, fils des hommes ?
${}^{3}Mais non, dans vos cœurs vous commett\underline{e}z le crime ;
        \\sur la terre vos mains font régn\underline{e}r la violence.
         
${}^{4}Les méchants sont dévoyés dès le s\underline{e}in maternel,
        \\menteurs, égarés depu\underline{i}s leur naissance ;
${}^{5}ils ont du venin, un ven\underline{i}n de vipère,
        \\ils se bouchent les oreilles, c\underline{o}mme des serpents
${}^{6}qui refusent d’écouter la v\underline{o}ix de l’enchanteur,
        \\du charmeur le plus hab\underline{i}le aux charmes.
         
${}^{7}Dieu, brise leurs d\underline{e}nts et leur mâchoire,
        \\Seigneur, casse les cr\underline{o}cs de ces lions :
${}^{8}Qu’ils s’en aillent comme les ea\underline{u}x qui se perdent !
        \\Que Dieu les transperce, \underline{e}t qu’ils en périssent,
${}^{9}comme la limace qui gl\underline{i}sse en fondant,
        \\ou l’avorton qui ne voit p\underline{a}s le soleil !
         
${}^{10}Plus vite qu’un feu de ronces ne l\underline{è}che la marmite,
        \\que le feu de ta col\underline{è}re les emporte !
${}^{11}Joie pour le juste de v\underline{o}ir la vengeance,
        \\de laver ses pieds dans le s\underline{a}ng de l’impie !
${}^{12}Et l’homme dira : « Oui, le juste p\underline{o}rte du fruit ;
        \\oui, il existe un Dieu pour jug\underline{e}r sur la terre. »
      \bchapter{Psaume}
          
            \bchapter{Psaume}
            Des meurtriers, sauve-moi
${}^{1}Du maître de chœur. « Ne détruis pas ». De David. À mi-voix. Lorsque Saül envoya garder sa maison pour le faire mourir.
         
${}^{2}Délivre-moi de mes ennem\underline{i}s, mon Dieu ;
        \\de mes agresse\underline{u}rs, protège-moi.
${}^{3}Délivre-moi des h\underline{o}mmes criminels ;
        \\des meurtri\underline{e}rs, sauve-moi.
         
${}^{4}Voici qu’on me prép\underline{a}re une embuscade :
        \\des puissants se j\underline{e}ttent sur moi.
${}^{5}Je n’ai commis ni faute, ni péché, ni le m\underline{a}l, Seigneur,
        \\pourtant ils acco\underline{u}rent et s’installent.
         
        \\Réveille-toi ! Viens à m\underline{o}i, regarde,
${}^{6}Seigneur, Dieu de l’univ\underline{e}rs, Dieu d’Israël :
        \\\[lève-toi et pun\underline{i}s tous ces païens,
        \\sans pitié pour tous ces tr\underline{a}îtres de malheur !
         
${}^{7}Le s\underline{o}ir, ils reviennent : *
        comme des chiens, ils grondent,
        ils c\underline{e}rnent la ville.
         
${}^{8}Les voici, l’éc\underline{u}me à la bouche,
        \\l’épée aux lèvres : « Qui d\underline{o}nc entendrait ? »
${}^{9}Mais toi, Seigne\underline{u}r, tu t’en amuses,
        \\tu te ris de to\underline{u}s ces païens.\]
         
${}^{10}Auprès de toi, ma forter\underline{e}sse, je veille ;
        \\oui, mon remp\underline{a}rt, c’est Dieu !
${}^{11}Le Dieu de mon amo\underline{u}r vient à moi :
        \\avec lui je déf\underline{i}e mes adversaires.
         
        *
         
${}^{12}\[Ne les supprime p\underline{a}s, Seigneur,
        \\de peur que mon pe\underline{u}ple n’oublie !
        \\Que ta puissance les terr\underline{a}sse et les disperse,
        \\Seigneur, n\underline{o}tre bouclier !
         
${}^{13}Ils pèchent dès qu’ils o\underline{u}vrent la bouche ; +
        \\qu’ils soient pr\underline{i}s à leur orgueil
        \\puisqu’ils m\underline{e}ntent et qu’ils maudissent !
         
${}^{14}Dans ta col\underline{è}re, détruis-les ;
        \\détruis-l\underline{e}s, qu’ils disparaissent !
        \\Alors on saura que Dieu r\underline{è}gne en Jacob
        \\et sur l’étend\underline{u}e de la terre.
         
${}^{15}Le s\underline{o}ir, ils reviennent : *
        comme des chiens, ils grondent,
        ils c\underline{e}rnent la ville.
${}^{16}Ils vont en qu\underline{ê}te d’une proie, *
        \\affamés, hurl\underline{a}nt dans la nuit.\]
         
${}^{17}Et moi, je chanter\underline{a}i ta force,
        \\au matin j’acclamer\underline{a}i ton amour.
        \\Tu as été pour m\underline{o}i un rempart,
        \\un refuge au t\underline{e}mps de ma détresse.
         
${}^{18}Je te fêterai, t\underline{o}i, ma forteresse :
        \\oui, mon remp\underline{a}rt, c’est Dieu,
        le Die\underline{u} de mon amour.
      \bchapter{Psaume}
          
            \bchapter{Psaume}
            Porte-nous secours dans l’épreuve
${}^{1}Du maître de chœur. Sur l’air de « Le lis de la charte ».
        \\À mi-voix. De David. Pour apprendre.
         
${}^{2}Lorsqu’il combattait les Araméens de Mésopotamie et les Araméens de Soba, et que Joab revint pour battre Édom dans la vallée du Sel : douze mille hommes.
         
${}^{3}Dieu, tu nous as rejet\underline{é}s, brisés ;
        \\tu étais en col\underline{è}re, reviens-nous !
${}^{4}Tu as secoué, disloqu\underline{é} le pays ;
        \\répare ses br\underline{è}ches : il s’effondre.
         
${}^{5}Tu mets à dure épre\underline{u}ve ton peuple,
        \\tu nous fais boire un v\underline{i}n de vertige.
${}^{6}Tu as donné un étend\underline{a}rd à tes fidèles,
        \\était-ce pour qu’ils fu\underline{i}ent devant l’arc ?
         
${}^{7}Que tes bien-aim\underline{é}s soient libérés ;
        \\sauve-les par ta dr\underline{o}ite, réponds-nous !
         
${}^{8}Dans le sanctuaire, Die\underline{u} a parlé : +
        \\« Je triomphe ! Je part\underline{a}ge Sichem,
        \\je divise la vall\underline{é}e de Souccoth.
         
${}^{9}« À moi Galaad, à m\underline{o}i Manassé ! +
        \\Éphraïm est le c\underline{a}sque de ma tête,
        \\Juda, mon bât\underline{o}n de commandement.
         
${}^{10}« Moab est le bass\underline{i}n où je me lave ; +
        \\sur Édom, je p\underline{o}se le talon.
        \\Crieras-tu victoire sur m\underline{o}i, Philistie ? »
         
${}^{11}Qui me conduir\underline{a} dans la Ville-forte,
        \\qui me mèner\underline{a} jusqu’en Édom,
${}^{12}sinon toi, Die\underline{u}, qui nous rejettes
        \\et ne sors plus av\underline{e}c nos armées ?
         
${}^{13}Porte-nous seco\underline{u}rs dans l’épreuve :
        \\néant, le sal\underline{u}t qui vient des hommes !
${}^{14}Avec Dieu nous fer\underline{o}ns des prouesses,
        \\et lui piétiner\underline{a} nos oppresseurs !
      \bchapter{Psaume}
          
            \bchapter{Psaume}
            Tu es pour moi un refuge
${}^{1}Du maître de chœur. Sur les instruments à corde. De David.
         
${}^{2}Dieu, entends ma plainte,
        exa\underline{u}ce ma prière ; *
${}^{3}des terres lointaines je t’appelle
        quand le cœ\underline{u}r me manque.
         
        \\Jusqu’au rocher trop loin de moi
        t\underline{u} me conduiras, *
${}^{4}car tu es pour moi un refuge,
        un bastion, f\underline{a}ce à l’ennemi.
         
${}^{5}Je veux être chez t\underline{o}i pour toujours,
        \\me réfugier à l’abr\underline{i} de tes ailes.
         
        *
         
${}^{6}Oui, mon Dieu, tu exa\underline{u}ces mon vœu,
        \\tu fais largesse à ceux qui cr\underline{a}ignent ton nom.
         
${}^{7}Accorde au roi des jo\underline{u}rs et des jours :
        \\que ses années devi\underline{e}nnent des siècles !
         
${}^{8}Qu’il trône à jamais devant la f\underline{a}ce de Dieu !
        \\Assigne à sa garde Amo\underline{u}r et Vérité.
         
${}^{9}Alors, je chanterai sans c\underline{e}sse ton nom,
        \\j’accomplirai mon vœu jo\underline{u}r après jour.
      \bchapter{Psaume}
          
            \bchapter{Psaume}
            Je n’ai de repos qu’en Dieu
${}^{1}Du maître de chœur. D’après Yedoutoune. Psaume. De David.
         
${}^{2}Je n’ai de rep\underline{o}s qu’en Dieu seul,
        \\mon sal\underline{u}t vient de lui.
         
${}^{3}Lui seul est mon roch\underline{e}r, mon salut,
        \\ma citadelle : je su\underline{i}s inébranlable.
         
${}^{4}Combien de temps tomberez-vous sur un homme
        pour l’ab\underline{a}ttre, vous tous, *
        \\comme un mur qui penche,
        une clôt\underline{u}re qui croule ?
         
${}^{5}Détruire mon honneur est leur seule pensée : +
        ils se pl\underline{a}isent à mentir. *
        \\Des lèvres, ils bénissent ;
        au fond d’eux-m\underline{ê}mes, ils maudissent.
         
${}^{6}Je n’ai mon rep\underline{o}s qu’en Dieu seul ;
        \\oui, mon esp\underline{o}ir vient de lui.
         
${}^{7}Lui seul est mon roch\underline{e}r, mon salut,
        \\ma citadelle : je r\underline{e}ste inébranlable.
         
${}^{8}Mon salut et ma gloire
        se tro\underline{u}vent près de Dieu. *
        \\Chez Dieu, mon refuge,
        mon roch\underline{e}r imprenable !
         
        *
         
${}^{9}Comptez sur lui en tous temps,
        vo\underline{u}s, le peuple. *
        \\Devant lui épanchez votre cœur :
        Dieu est pour no\underline{u}s un refuge.
         
${}^{10}L’homme n’est qu’un souffle,
        les fils des h\underline{o}mmes, un mensonge : *
        \\sur un plateau de balance, tous ensemble,
        ils ser\underline{a}ient moins qu’un souffle.
         
${}^{11}N’allez pas compter sur la fraude
        et n’aspirez p\underline{a}s au profit ; *
        \\si vous amassez des richesses,
        n’y mettez p\underline{a}s votre cœur.
         
${}^{12}Dieu a dit une chose,
        deux choses que j’\underline{a}i entendues. +
        \\Ceci : que la force est à Dieu ;
${}^{13}à toi, Seigne\underline{u}r, la grâce ! *
        \\Et ceci : tu rends à chaque homme
        sel\underline{o}n ce qu’il fait.
      \bchapter{Psaume}
          
            \bchapter{Psaume}
            Je te cherche dès l’aube
${}^{1}Psaume. De David. Lorsqu’il était dans le désert de Juda.
         
${}^{2}Dieu, tu es mon Dieu,
        je te ch\underline{e}rche dès l’aube : *
        \\mon âme a s\underline{o}if de toi ;
        \\après toi langu\underline{i}t ma chair,
        \\terre aride, altér\underline{é}e, sans eau.
         
${}^{3}Je t’ai contempl\underline{é} au sanctuaire,
        \\j’ai vu ta f\underline{o}rce et ta gloire.
${}^{4}Ton amour vaut mie\underline{u}x que la vie :
        \\tu seras la lou\underline{a}nge de mes lèvres !
         
${}^{5}Toute ma vie je v\underline{a}is te bénir,
        \\lever les mains en invoqu\underline{a}nt ton nom.
${}^{6}Comme par un festin je ser\underline{a}i rassasié ;
        \\la joie sur les lèvres, je dir\underline{a}i ta louange.
         
${}^{7}Dans la nuit, je me souvi\underline{e}ns de toi
        \\et je reste des he\underline{u}res à te parler.
${}^{8}Oui, tu es ven\underline{u} à mon secours :
        \\je crie de joie à l’\underline{o}mbre de tes ailes.
${}^{9}Mon âme s’att\underline{a}che à toi,
        \\ta main dr\underline{o}ite me soutient.
         
${}^{10}\[Mais ceux qui pourch\underline{a}ssent mon âme,
        \\qu’ils descendent aux profonde\underline{u}rs de la terre,
${}^{11}qu’on les passe au f\underline{i}l de l’épée,
        \\qu’ils deviennent la pât\underline{u}re des loups !
         
${}^{12}Et le roi se réjouir\underline{a} de son Dieu.
        \\Qui jure par lui en ser\underline{a} glorifié,
        \\tandis que l’h\underline{o}mme de mensonge
        \\aura la bo\underline{u}che close !\]
      \bchapter{Psaume}
          
            \bchapter{Psaume}
            Protège ma vie
${}^{1}Du maître de chœur. Psaume. De David.
         
${}^{2}Écoute, ô mon Dieu, le cr\underline{i} de ma plainte ;
        \\face à l’ennemi redoutable, prot\underline{è}ge ma vie.
${}^{3}Garde-moi du compl\underline{o}t des méchants,
        \\à l’abri de cette me\underline{u}te criminelle.
         
        *
         
${}^{4}Ils affûtent leur l\underline{a}ngue comme une épée,
        \\ils ajustent leur flèche, par\underline{o}le empoisonnée,
${}^{5}pour tirer en cach\underline{e}tte sur l’innocent ;
        \\ils tirent soud\underline{a}in, sans rien craindre.
         
${}^{6}Ils se forgent des form\underline{u}les maléfiques, +
        \\ils dissimulent avec s\underline{o}in leurs pièges ;
        \\ils disent : « Qu\underline{i} les verra ? »
         
${}^{7}Ils mach\underline{i}nent leur crime : +
        \\Notre machinati\underline{o}n est parfaite ;
        \\le cœur de chacun deme\underline{u}re impénétrable !
         
        *
         
${}^{8}Mais c’est Dieu qui leur t\underline{i}re une flèche, +
        \\soudain, ils en ress\underline{e}ntent la blessure,
${}^{9}ils sont les vict\underline{i}mes de leur langue.
         
        \\Tous ceux qui les voient h\underline{o}chent la tête ;
${}^{10}tout homme est sais\underline{i} de crainte :
        \\il proclame ce que Die\underline{u} a fait,
        \\il compr\underline{e}nd ses actions.
         
${}^{11}Le juste trouvera dans le Seigneur
        j\underline{o}ie et refuge, *
        \\et tous les hommes au cœur droit,
        le\underline{u}r louange.
      \bchapter{Psaume}
          
            \bchapter{Psaume}
            Tu visites la terre
${}^{1}Du maître de chœur. Psaume. De David. Cantique.
         
${}^{2}Il est bea\underline{u} de te louer,
        Die\underline{u}, dans Sion, *
        \\de tenir ses prom\underline{e}sses envers toi
${}^{3}qui éco\underline{u}tes la prière.
         
        \\Jusqu’à toi vi\underline{e}nt toute chair
${}^{4}avec son p\underline{o}ids de péché ; *
        \\nos fautes ont domin\underline{é} sur nous :
        t\underline{o}i, tu les pardonnes.
         
${}^{5}Heureux ton invit\underline{é}, ton élu :
        il hab\underline{i}te ta demeure ! *
        \\Les biens de ta mais\underline{o}n nous rassasient,
        les dons sacr\underline{é}s de ton temple !
         
        *
         
${}^{6}Ta justice nous rép\underline{o}nd par des prodiges,
        Die\underline{u} notre sauveur, *
        \\espoir des horiz\underline{o}ns de la terre
        et des r\underline{i}ves lointaines.
         
${}^{7}Sa force enrac\underline{i}ne les montagnes,
        il s’ento\underline{u}re de puissance ; *
${}^{8}il apaise le vac\underline{a}rme des mers,
        le vacarme de leurs flots
        et la rume\underline{u}r des peuples.
         
${}^{9}Les habitants des bouts du m\underline{o}nde sont pris d’effroi
        à la v\underline{u}e de tes signes ; *
        \\aux portes du lev\underline{a}nt et du couchant
        tu fais jaill\underline{i}r des cris de joie.
         
        *
         
${}^{10}Tu visites la t\underline{e}rre et tu l’abreuves,
        tu la c\underline{o}mbles de richesses ; *
        \\les ruisseaux de Die\underline{u} regorgent d’eau :
        tu prép\underline{a}res les moissons.
         
        \\Ainsi, tu prép\underline{a}res la terre,
${}^{11}tu arr\underline{o}ses les sillons ; *
        \\tu aplanis le sol, tu le détr\underline{e}mpes sous les pluies,
        tu bén\underline{i}s les semailles.
         
${}^{12}Tu couronnes une ann\underline{é}e de bienfaits ; *
        \\sur ton passage, ruiss\underline{e}lle l’abondance.
${}^{13}Au désert, les pâtur\underline{a}ges ruissellent, *
        \\les collines déb\underline{o}rdent d’allégresse.
         
${}^{14}Les herbages se p\underline{a}rent de troupeaux +
        \\et les plaines se co\underline{u}vrent de blé. *
        Tout ex\underline{u}lte et chante !
      \bchapter{Psaume}
          
            \bchapter{Psaume}
            Voyez les hauts faits de Dieu
${}^{1}Du maître de chœur. Cantique. Psaume.
         
        \\Acclamez Dieu, to\underline{u}te la terre ; +
${}^{2}fêtez la gl\underline{o}ire de son nom,
        \\glorifiez-le en célébr\underline{a}nt sa louange.
         
${}^{3}Dites à Dieu : « Que tes acti\underline{o}ns sont redoutables !
        \\En présence de ta force, tes ennem\underline{i}s s’inclinent.
${}^{4}Toute la terre se prost\underline{e}rne devant toi,
        \\elle chante pour toi, elle ch\underline{a}nte pour ton nom. »
         
${}^{5}Venez et voyez les hauts f\underline{a}its de Dieu,
        \\ses exploits redoutables pour les f\underline{i}ls des hommes.
${}^{6}Il changea la m\underline{e}r en terre ferme :
        \\ils passèrent le fle\underline{u}ve à pied sec.
         
        \\De là, cette j\underline{o}ie qu’il nous donne.
${}^{7}Il règne à jam\underline{a}is par sa puissance.
        \\Ses yeux obs\underline{e}rvent les nations :
        \\que les rebelles co\underline{u}rbent la tête !
         
${}^{8}Peuples, béniss\underline{e}z notre Dieu !
        \\Faites retent\underline{i}r sa louange,
${}^{9}car il rend la v\underline{i}e à notre âme,
        \\il a gardé nos pi\underline{e}ds de la chute.
         
${}^{10}C’est toi, Dieu, qui nous \underline{a}s éprouvés,
        \\affinés comme on aff\underline{i}ne un métal ;
${}^{11}tu nous as condu\underline{i}ts dans un piège,
        \\tu as serré un éta\underline{u} sur nos reins.
         
${}^{12}Tu as mis des mort\underline{e}ls à notre tête ; +
        \\nous sommes entrés dans l’ea\underline{u} et le feu,
        \\tu nous as fait sort\underline{i}r vers l’abondance.
         
        *
         
${}^{13}Je viens dans ta maison av\underline{e}c des holocaustes,
        \\je tiendrai mes prom\underline{e}sses envers toi,
${}^{14}les promesses qui m’ouvr\underline{i}rent les lèvres,
        \\que ma bouche a prononc\underline{é}es dans ma détresse.
         
${}^{15}Je t’offrirai de bea\underline{u}x holocaustes +
        \\avec le fum\underline{e}t des béliers ;
        \\je prépare des bœufs et des chevreaux.
         
${}^{16}Venez, écoutez, vous to\underline{u}s qui craignez Dieu :
        \\je vous dirai ce qu’il a f\underline{a}it pour mon âme ;
${}^{17}quand je poussai vers lu\underline{i} mon cri,
        \\ma bouche faisait déj\underline{à} son éloge.
         
${}^{18}Si mon cœur avait regard\underline{é} vers le mal,
        \\le Seigneur n’aurait p\underline{a}s écouté.
${}^{19}Et pourtant, Die\underline{u} a écouté,
        \\il entend le cr\underline{i} de ma prière.
         
${}^{20}Bén\underline{i} soit Dieu +
        \\qui n’a pas écart\underline{é} ma prière,
        \\ni détourné de m\underline{o}i son amour !
      \bchapter{Psaume}
          
            \bchapter{Psaume}
            Que les peuples te rendent grâce
${}^{1}Du maître de chœur. Avec instruments à corde. Psaume. Cantique.
         
${}^{2}Que Dieu nous prenne en gr\underline{â}ce et nous bénisse,
        \\que son visage s’illum\underline{i}ne pour nous ;
${}^{3}et ton chemin sera conn\underline{u} sur la terre,
        \\ton salut, parmi to\underline{u}tes les nations.
         
${}^{4}Que les peuples, Die\underline{u}, te rendent grâce ;
        \\qu’ils te rendent gr\underline{â}ce tous ensemble !
         
${}^{5}Que les nations ch\underline{a}ntent leur joie,
        \\car tu gouvernes le m\underline{o}nde avec justice ;
        \\tu gouvernes les pe\underline{u}ples avec droiture,
        \\sur la terre, tu condu\underline{i}s les nations.
         
${}^{6}Que les peuples, Die\underline{u}, te rendent grâce ;
        \\qu’ils te rendent gr\underline{â}ce tous ensemble !
         
${}^{7}La terre a donn\underline{é} son fruit ;
        \\Dieu, notre Die\underline{u}, nous bénit.
${}^{8}Que Die\underline{u} nous bénisse,
        \\et que la terre tout enti\underline{è}re l’adore !
      \bchapter{Psaume}
          
            \bchapter{Psaume}
            Le Dieu des victoires
${}^{1}Du maître de chœur. De David. Psaume. Cantique.
         
${}^{2}Dieu se lève et ses ennem\underline{i}s se dispersent,
        \\ses adversaires fu\underline{i}ent devant sa face.
${}^{3}Comme on dissipe une fum\underline{é}e, tu les dissipes ; +
        \\comme on voit fondre la cire en f\underline{a}ce du feu,
        \\les impies disparaissent devant la f\underline{a}ce de Dieu.
         
${}^{4}Mais les justes sont en f\underline{ê}te, ils exultent ;
        \\devant la face de Dieu ils d\underline{a}nsent de joie.
${}^{5}Chantez pour Dieu, jou\underline{e}z pour son nom, +
        \\frayez la route à celui qui cheva\underline{u}che les nuées.
        \\Son nom est Le Seigneur ; dans\underline{e}z devant sa face.
         
${}^{6}Père des orphelins, défense\underline{u}r des veuves,
        \\tel est Dieu dans sa s\underline{a}inte demeure.
${}^{7}À l’isolé, Dieu acc\underline{o}rde une maison ; +
        \\aux captifs, il r\underline{e}nd la liberté ;
        \\mais les rebelles vont habit\underline{e}r les lieux arides.
         
        *
         
${}^{8}Dieu, quand tu sortis en av\underline{a}nt de ton peuple,
        \\quand tu marchas dans le désert, la t\underline{e}rre trembla ;
${}^{9}les cieux m\underline{ê}mes fondirent +
        \\devant la face de Dieu, le Die\underline{u} du Sinaï,
        \\devant la face de Dieu, le Die\underline{u} d’Israël.
         
${}^{10}Tu répandais sur ton héritage une plu\underline{i}e généreuse,
        \\et quand il défaillait, t\underline{o}i, tu le soutenais.
${}^{11}Sur les lieux où camp\underline{a}it ton troupeau,
        \\tu le soutenais, Dieu qui es b\underline{o}n pour le pauvre.
         
${}^{12}Le Seigneur pron\underline{o}nce un oracle,
        \\une armée de messag\underline{è}res le répand :
${}^{13}« Rois en déroute, arm\underline{é}es en déroute !
        \\On reçoit en partage les trés\underline{o}rs du pays.
         
${}^{14}« Resterez-vous au repos derri\underline{è}re vos murs +
        \\quand les ailes de la Colombe se co\underline{u}vrent d’argent,
        \\et son plumage, de fl\underline{a}mmes d’or,
${}^{15}quand le Puissant, là-bas, pulvér\underline{i}se des rois
        \\et qu’il n\underline{e}ige au Mont-Sombre ? »
         
${}^{16}Mont de Bashane, div\underline{i}ne montagne,
        \\mont de Bashane, fi\underline{è}re montagne !
${}^{17}Pourquoi jalouser, fi\underline{è}re montagne, +
        \\la montagne que Dieu s’est chois\underline{i}e pour demeure ?
        \\Là, le Seigneur habiter\underline{a} jusqu’à la fin.
         
${}^{18}Les chars de Dieu sont des milli\underline{e}rs de myriades ;
        \\au milieu, le Seigneur ; au sanctu\underline{a}ire, le Sinaï.
${}^{19}Tu es monté sur la hauteur, captur\underline{a}nt des captifs, +
        \\recevant un tribut, m\underline{ê}me de rebelles,
        \\pour avoir une demeure, Seigne\underline{u}r notre Dieu.
         
        *
         
${}^{20}Que le Seigne\underline{u}r soit béni !
        \\Jour après jour, ce Dieu nous acc\underline{o}rde la victoire.
         
${}^{21}Le Dieu qui est le nôtre est le Die\underline{u} des victoires,
        \\et les portes de la mort sont à Die\underline{u}, le Seigneur.
${}^{22}À qui le hait, Dieu frac\underline{a}sse la tête ;
        \\à qui vit dans le crime, il déf\underline{o}nce le crâne.
         
${}^{23}Le Seigneur a dit : « Je les ram\underline{è}ne de Bashane,
        \\je les ramène des ab\underline{î}mes de la mer,
${}^{24}afin que tu enfonces ton pi\underline{e}d dans leur sang,
        \\que la langue de tes chiens ait sa pât\underline{u}re d’ennemis. »
         
${}^{25}Dieu, on a v\underline{u} ton cortège,
        \\le cortège de mon Dieu, de mon r\underline{o}i dans le Temple :
${}^{26}en tête les chantres, les musici\underline{e}ns derrière,
        \\parmi les jeunes filles frapp\underline{a}nt le tambourin.
         
${}^{27}Rassemblez-vous, b\underline{é}nissez Dieu ;
        \\aux sources d’Israël, il y \underline{a} le Seigneur !
${}^{28}Voici Benjamin, le plus je\underline{u}ne, ouvrant la marche, +
        \\les princes de Jud\underline{a} et leur suite,
        \\les princes de Zabulon, les pr\underline{i}nces de Nephtali.
         
        *
         
${}^{29}Ton Dieu l’a command\underline{é} : « Sois fort ! »
        \\Montre ta force, Dieu, quand tu ag\underline{i}s pour nous !
${}^{30}De ton palais, qui dom\underline{i}ne Jérusalem,
        \\on voit des rois t’apport\underline{e}r leurs présents.
         
${}^{31}Menace la B\underline{ê}te des marais,
        \\la bande de fauves, la me\underline{u}te des peuples :
        \\qu’ils se prosternent avec leurs pi\underline{è}ces d’argent ;
        \\désunis les peuples qui \underline{a}iment la guerre.
         
${}^{32}De l’Égypte arriveront des ét\underline{o}ffes somptueuses ;
        \\l’Éthiopie viendra vers Die\underline{u} les mains pleines.
${}^{33}Royaumes de la terre, chantez pour Dieu,
        jou\underline{e}z pour le Seigneur,*
${}^{34}celui qui chevauche au plus haut des cieux,
        les cie\underline{u}x antiques.
         
        \\Voici qu’il élève la voix, une v\underline{o}ix puissante ;
${}^{35}rendez la puiss\underline{a}nce à Dieu.
        \\Sur Isra\underline{ë}l, sa splendeur !
        \\Dans la nu\underline{é}e, sa puissance !
         
${}^{36}Redoutable est Dieu dans son temple saint,
        le Die\underline{u} d’Israël ; *
        \\c’est lui qui donne à son peuple
        force et puissance.
        Bén\underline{i} soit Dieu !
      \bchapter{Psaume}
          
            \bchapter{Psaume}
            Dans l’abîme des eaux
${}^{1}Du maître de chœur. Sur l’air de « Des lis… ». De David.
         
${}^{2}Sauve-m\underline{o}i, mon Dieu :
        \\les eaux m\underline{o}ntent jusqu’à ma gorge !
         
${}^{3}J’enfonce dans la v\underline{a}se du gouffre,
        ri\underline{e}n qui me retienne ; *
        \\je descends dans l’ab\underline{î}me des eaux,
        le fl\underline{o}t m’engloutit.
         
${}^{4}Je m’épu\underline{i}se à crier,
        ma g\underline{o}rge brûle.*
        \\Mes ye\underline{u}x se sont usés
        d’att\underline{e}ndre mon Dieu.
         
${}^{5}Plus abondants que les cheve\underline{u}x de ma tête,
        ceux qui m’en ve\underline{u}lent sans raison ; *
        \\ils sont nombre\underline{u}x, mes détracteurs,
        à me ha\underline{ï}r injustement.
         
        \\Moi qui n’ai ri\underline{e}n volé,
        que devr\underline{a}i-je rendre ? *
${}^{6}Dieu, tu conn\underline{a}is ma folie,
        mes fautes sont à n\underline{u} devant toi.
         
${}^{7}Qu’ils n’aient pas honte pour m\underline{o}i, ceux qui t’espèrent,
        Seigneur, Die\underline{u} de l’univers ; *
        \\qu’ils ne rougissent pas de m\underline{o}i, ceux qui te cherchent,
        Die\underline{u} d’Israël !
         
${}^{8}C’est pour toi que j’end\underline{u}re l’insulte,
        \\que la honte me co\underline{u}vre le visage :
${}^{9}je suis un étrang\underline{e}r pour mes frères,
        \\un inconnu pour les f\underline{i}ls de ma mère.
${}^{10}L’amour de ta mais\underline{o}n m’a perdu ;
        \\on t’insulte, et l’insulte ret\underline{o}mbe sur moi.
         
${}^{11}Si je pleure et m’imp\underline{o}se un jeûne,
        \\je reç\underline{o}is des insultes ;
${}^{12}si je revêts un hab\underline{i}t de pénitence,
        \\je deviens la f\underline{a}ble des gens :
${}^{13}on parle de m\underline{o}i sur les places,
        \\les buveurs de v\underline{i}n me chansonnent.
         
        *
         
${}^{14}Et moi, je te pr\underline{i}e, Seigneur :
        c’est l’he\underline{u}re de ta grâce ; *
        \\dans ton grand amour, Die\underline{u}, réponds-moi,
        par ta vérit\underline{é} sauve-moi.
         
${}^{15}Tire-m\underline{o}i de la boue,
        sin\underline{o}n je m’enfonce : *
        \\que j’échappe à ce\underline{u}x qui me haïssent,
        à l’ab\underline{î}me des eaux.
         
${}^{16}Que les flots ne me subm\underline{e}rgent pas,
        que le go\underline{u}ffre ne m’avale, *
        \\que la gue\underline{u}le du puits
        ne se ferme p\underline{a}s sur moi.
         
${}^{17}Réponds-m\underline{o}i, Seigneur,
        car il est b\underline{o}n, ton amour ; *
        \\dans ta gr\underline{a}nde tendresse,
        r\underline{e}garde-moi.
         
${}^{18}Ne cache pas ton vis\underline{a}ge à ton serviteur ;
        je suffoque : v\underline{i}te, réponds-moi. *
${}^{19}Sois proche de m\underline{o}i, rachète-moi,
        paie ma ranç\underline{o}n à l’ennemi.
         
${}^{20}Toi, tu le s\underline{a}is, on m’insulte :
        je suis bafou\underline{é}, déshonoré ; *
        \\to\underline{u}s mes oppresseurs
        sont l\underline{à}, devant toi.
         
${}^{21}L’insulte m’a broy\underline{é} le cœur,
        le m\underline{a}l est incurable ; *
        \\j’espérais un seco\underline{u}rs, mais en vain,
        des consolateurs, je n’en ai p\underline{a}s trouvé.
         
${}^{22}À mon pain, ils ont mêl\underline{é} du poison ;
        \\quand j’avais soif, ils m’ont donn\underline{é} du vinaigre.
${}^{23}\[Que leur table devi\underline{e}nne un piège,
        \\un guet-ap\underline{e}ns pour leurs convives !
${}^{24}Que leurs yeux aveugl\underline{é}s ne voient plus,
        \\qu’à tout instant les r\underline{e}ins leur manquent !
         
${}^{25}Déverse sur e\underline{u}x ta fureur,
        \\que le feu de ta col\underline{è}re les saisisse,
${}^{26}que leur camp devi\underline{e}nne un désert,
        \\que nul n’hab\underline{i}te sous leurs tentes !
         
${}^{27}Celui que tu frapp\underline{a}is, ils le pourchassent
        \\en comptant les co\underline{u}ps qu’il reçoit.
${}^{28}Charge-les, fa\underline{u}te sur faute ;
        \\qu’ils n’aient pas d’acc\underline{è}s à ta justice.
${}^{29}Qu’ils soient rayés du l\underline{i}vre de vie,
        \\retranchés du n\underline{o}mbre des justes.\]
         
        *
         
${}^{30}Et moi, humili\underline{é}, meurtri,
        \\que ton salut, Die\underline{u}, me redresse.
${}^{31}Et je louerai le nom de Die\underline{u} par un cantique,
        \\je vais le magnifi\underline{e}r, lui rendre grâce.
${}^{32}Cela plaît au Seigne\underline{u}r plus qu’un taureau,
        \\plus qu’une bête ayant c\underline{o}rnes et sabots.
         
${}^{33}Les pauvres l’ont v\underline{u}, ils sont en fête :
        \\« Vie et joie, à vo\underline{u}s qui cherchez Dieu ! »
${}^{34}Car le Seigneur éco\underline{u}te les humbles,
        \\il n’oublie pas les si\underline{e}ns emprisonnés.
${}^{35}Que le ciel et la t\underline{e}rre le célèbrent,
        \\les mers et to\underline{u}t leur peuplement !
         
${}^{36}Car Dieu viendra sauv\underline{e}r Sion
        \\et rebâtir les v\underline{i}lles de Juda.
        \\Il en fera une habitati\underline{o}n, un héritage : *
${}^{37}patrimoine pour les descendants de ses serviteurs,
        demeure pour ceux qui \underline{a}iment son nom.
      \bchapter{Psaume}
          
            \bchapter{Psaume}
            Viens vite à mon secours
${}^{1}Du maître de chœur. De David. En mémorial.
         
${}^{2}Mon Dieu, vi\underline{e}ns me délivrer ;
        Seigneur, viens v\underline{i}te à mon secours ! *
${}^{3}Qu’ils soient humili\underline{é}s, déshonorés,
        ceux qui s’en pr\underline{e}nnent à ma vie !
         
        \\Qu’ils reculent, couv\underline{e}rts de honte,
        ceux qui ch\underline{e}rchent mon malheur ; *
${}^{4}que l’humiliati\underline{o}n les écrase,
        ceux qui me d\underline{i}sent : « C’est bien fait ! »
         
${}^{5}Mais tu seras l’allégr\underline{e}sse et la joie
        de tous ce\underline{u}x qui te cherchent ; *
        \\toujours ils redir\underline{o}nt : « Dieu est grand ! »
        ceux qui \underline{a}iment ton salut.
         
${}^{6}Je suis pa\underline{u}vre et malheureux,
        mon Die\underline{u}, viens vite ! *
        \\Tu es mon seco\underline{u}rs, mon libérateur :
        Seigne\underline{u}r, ne tarde pas !
      \bchapter{Psaume}
          
            \bchapter{Psaume}
            Tu me feras vivre à nouveau
${}^{1}En toi, Seigne\underline{u}r, j’ai mon refuge :
        \\garde-moi d’être humili\underline{é} pour toujours.
${}^{2}Dans ta justice, défends-m\underline{o}i, libère-moi,
        \\tends l’oreille vers m\underline{o}i, et sauve-moi.
         
${}^{3}Sois le rocher qui m’accueille,
        toujo\underline{u}rs accessible ; *
        \\tu as résolu de me sauver :
        ma forteresse et mon r\underline{o}c, c’est toi !
         
${}^{4}Mon Dieu, libère-moi des m\underline{a}ins de l’impie,
        \\des prises du fo\underline{u}rbe et du violent.
${}^{5}Seigneur mon Dieu, tu \underline{e}s mon espérance,
        \\mon appu\underline{i} dès ma jeunesse.
         
${}^{6}Toi, mon soutien dès av\underline{a}nt ma naissance, +
        \\tu m’as choisi dès le v\underline{e}ntre de ma mère ;
        \\tu seras ma lou\underline{a}nge toujours !
         
${}^{7}Pour beaucoup, je f\underline{u}s comme un prodige ;
        \\tu as été mon seco\underline{u}rs et ma force.
${}^{8}Je n’avais que ta lou\underline{a}nge à la bouche,
        \\tout le jo\underline{u}r, ta splendeur.
         
        *
         
${}^{9}Ne me rejette pas
        mainten\underline{a}nt que j’ai vieilli ; *
        \\alors que décline ma vigueur,
        ne m’aband\underline{o}nne pas.
         
${}^{10}Mes ennemis p\underline{a}rlent contre moi,
        \\ils me surv\underline{e}illent et se concertent.
${}^{11}Ils disent : « Die\underline{u} l’abandonne !
        \\Traquez-le, empoignez-le, il n’a p\underline{a}s de défenseur ! »
         
${}^{12}Dieu, ne sois p\underline{a}s loin de moi ;
        \\mon Dieu, viens v\underline{i}te à mon secours !
${}^{13}Qu’ils soient humiliés, anéantis,
        ceux qui se dr\underline{e}ssent contre moi ; *
        \\qu’ils soient couverts de honte et d’infamie,
        ceux qui ve\underline{u}lent mon malheur !
         
        *
         
${}^{14}Et moi qui ne c\underline{e}sse d’espérer,
        \\j’ajoute enc\underline{o}re à ta louange.
${}^{15}Ma bouche ann\underline{o}nce tout le jour +
        \\tes actes de just\underline{i}ce et de salut ;
        \\(je n’en connais p\underline{a}s le nombre).
         
${}^{16}Je revivrai les expl\underline{o}its du Seigneur
        \\en rappelant que ta just\underline{i}ce est la seule.
${}^{17}Mon Dieu, tu m’as instru\underline{i}t dès ma jeunesse,
        \\jusqu’à présent, j’ai proclam\underline{é} tes merveilles.
         
${}^{18}Aux jours de la vieill\underline{e}sse et des cheveux blancs,
        \\ne m’abandonne p\underline{a}s, ô mon Dieu ;
        \\et je dirai aux hommes de ce t\underline{e}mps ta puissance,
        \\à tous ceux qui viendr\underline{o}nt, tes exploits.
         
${}^{19}Si haute est ta just\underline{i}ce, mon Dieu, +
        \\toi qui as f\underline{a}it de grandes choses :
        \\Dieu, qui d\underline{o}nc est comme toi ?
         
${}^{20}Toi qui m’as fait voir tant de ma\underline{u}x et de détresses,
        \\tu me feras v\underline{i}vre à nouveau,
        \\à nouveau tu me tireras des ab\underline{î}mes de la terre, *
${}^{21}tu m’élèveras et me grandiras,
        tu reviendr\underline{a}s me consoler.
         
${}^{22}Et moi, je te rendrai grâce sur la harpe
        pour ta vérit\underline{é}, ô mon Dieu ! *
        \\Je jouerai pour toi de ma cithare,
        S\underline{a}int d’Israël !
         
${}^{23}Joie sur mes lèvres qui ch\underline{a}ntent pour toi,
        \\et dans mon âme que tu \underline{a}s rachetée !
${}^{24}Alors, tout au long du jour,
        ma langue redir\underline{a} ta justice ; *
        \\c’est la honte, c’est l’infamie
        pour ceux qui ve\underline{u}lent mon malheur.
      \bchapter{Psaume}
          
            \bchapter{Psaume}
            Les rois se prosterneront devant lui
${}^{1}De Salomon.
         
        \\Dieu, donne au r\underline{o}i tes pouvoirs,
        \\à ce fils de r\underline{o}i ta justice.
${}^{2}Qu’il gouverne ton pe\underline{u}ple avec justice,
        \\qu’il fasse dr\underline{o}it aux malheureux !
         
${}^{3}Montagnes, portez au pe\underline{u}ple la paix,
        \\collines, portez-lu\underline{i} la justice !
${}^{4}Qu’il fasse droit aux malheure\underline{u}x de son peuple,
        \\qu’il sauve les pauvres gens, qu’il écr\underline{a}se l’oppresseur !
         
${}^{5}Qu’il dure sous le sol\underline{e}il et la lune
        \\de générati\underline{o}n en génération !
${}^{6}Qu’il descende comme la plu\underline{i}e sur les regains,
        \\une pluie qui pén\underline{è}tre la terre.
         
${}^{7}En ces jours-là, fleurir\underline{a} la justice,
        \\grande paix jusqu’à la f\underline{i}n des lunes !
${}^{8}Qu’il domine de la m\underline{e}r à la mer,
        \\et du Fleuve jusqu’au bo\underline{u}t de la terre !
         
${}^{9}Des peuplades s’incliner\underline{o}nt devant lui,
        \\ses ennemis lècher\underline{o}nt la poussière.
${}^{10}Les rois de Tars\underline{i}s et des Îles
        \\apporter\underline{o}nt des présents.
         
        \\Les rois de Sab\underline{a} et de Seba
        \\fer\underline{o}nt leur offrande.
${}^{11}Tous les rois se prosterner\underline{o}nt devant lui,
        \\tous les pa\underline{y}s le serviront.
         
${}^{12}Il délivrera le pa\underline{u}vre qui appelle
        \\et le malheure\underline{u}x sans recours.
${}^{13}Il aura souci du f\underline{a}ible et du pauvre,
        \\du pauvre dont il sa\underline{u}ve la vie.
         
${}^{14}Il les rachète à l’oppressi\underline{o}n, à la violence ;
        \\leur sang est d’un grand pr\underline{i}x à ses yeux.
${}^{15}Qu’il vive ! On lui donnera l’\underline{o}r de Saba. *
        \\On priera sans relâche pour lui ;
        tous les jo\underline{u}rs, on le bénira.
         
${}^{16}Que la terre jusqu’au sommet des montagnes soit un ch\underline{a}mp de blé :
        \\et ses épis onduleront comme la for\underline{ê}t du Liban !
        \\Que la ville devi\underline{e}nne florissante
        \\comme l’h\underline{e}rbe sur la terre !
         
${}^{17}Que son nom d\underline{u}re toujours ;
        \\sous le soleil, que subs\underline{i}ste son nom !
        \\En lui, que soient bénies toutes les fam\underline{i}lles de la terre ;
        \\que tous les pays le d\underline{i}sent bienheureux !
         
        *
         
${}^{18}Béni soit le Seigneur, le Die\underline{u} d’Israël,
        lui seul f\underline{a}it des merveilles !
${}^{19}Béni soit à jamais son n\underline{o}m glorieux,
        toute la terre soit remplie de sa gloire !
        Am\underline{e}n ! Amen !
         
${}^{20}Fin des prières de David, fils de Jessé.
