  
  
    
    \bbook{PREMIÈRE LETTRE À TIMOTHÉE}{PREMIÈRE LETTRE À TIMOTHÉE}
      
         
      \bchapter{}
        ${}^{1}Paul, apôtre du Christ Jésus
        par ordre de Dieu notre Sauveur
        et du Christ Jésus notre espérance,
        ${}^{2}à Timothée,
        mon véritable enfant dans la foi.
        \\À toi, la grâce, la miséricorde et la paix
        \\de la part de Dieu le Père
        et du Christ Jésus notre Seigneur.
        
           
${}^{3}Comme je te l’ai recommandé en partant pour la Macédoine, reste à Éphèse pour interdire à certains de donner un enseignement différent 
${}^{4}ou de s’attacher à des récits mythologiques et à des généalogies interminables : cela ne porte qu’à de vaines recherches, plutôt qu’au dessein de Dieu qu’on accueille dans la foi. 
${}^{5}Le but de cette interdiction, c’est l’amour, la charité, qui vient d’un cœur pur, d’une conscience droite et d’une foi sans détours. 
${}^{6}Pour s’être écartés de ce chemin, certains se sont tournés vers des discours inconsistants ; 
${}^{7}ils veulent passer pour des spécialistes de la Loi, alors qu’ils ne comprennent ni ce qu’ils disent, ni ce dont ils se portent garants. 
${}^{8}Or nous savons que la Loi est bonne, à condition d’en faire un usage légitime, 
${}^{9}car, on le sait bien, une loi ne vise pas l’homme juste, mais les sans-loi et les insoumis, les impies et les pécheurs, les sacrilèges et les profanateurs, les parricides et matricides, et autres meurtriers, 
${}^{10}débauchés, sodomites, trafiquants d’êtres humains, menteurs, parjures, et tout ce qui s’oppose à l’enseignement de la saine doctrine. 
${}^{11}Voilà ce qui est conforme à l’Évangile qui m’a été confié, celui de la gloire du Dieu bienheureux.
${}^{12}Je suis plein de gratitude envers celui qui me donne la force, le Christ Jésus notre Seigneur, car il m’a estimé digne de confiance lorsqu’il m’a chargé du ministère, 
${}^{13}moi qui étais autrefois blasphémateur, persécuteur, violent. Mais il m’a été fait miséricorde, car j’avais agi par ignorance, n’ayant pas encore la foi ; 
${}^{14}la grâce de notre Seigneur a été encore plus abondante, avec elle la foi, et avec l’amour qui est dans le Christ Jésus. 
${}^{15}Voici une parole digne de foi, et qui mérite d’être accueillie sans réserve : le Christ Jésus est venu dans le monde pour sauver les pécheurs ; et moi, je suis le premier des pécheurs. 
${}^{16}Mais s’il m’a été fait miséricorde, c’est afin qu’en moi le premier, le Christ Jésus montre toute sa patience, pour donner un exemple à ceux qui devaient croire en lui, en vue de la vie éternelle. 
${}^{17}Au roi des siècles, Dieu immortel, invisible et unique, honneur et gloire pour les siècles des siècles ! Amen.
${}^{18}Voici la consigne que je te transmets, Timothée mon enfant, conformément aux paroles prophétiques jadis prononcées sur toi : livre ainsi la bonne bataille, 
${}^{19}en gardant la foi et une conscience droite ; pour avoir abandonné cette droiture, certains ont connu le naufrage de leur foi. 
${}^{20}Tels étaient Hyménaios et Alexandre, que j’ai livrés à Satan pour leur apprendre à ne plus blasphémer.
      
         
      \bchapter{}
      \begin{verse}
${}^{1}J’encourage, avant tout, à faire des demandes, des prières, des intercessions et des actions de grâce pour tous les hommes, 
${}^{2}pour les chefs d’État et tous ceux qui exercent l’autorité, afin que nous puissions mener notre vie dans la tranquillité et le calme, en toute piété et dignité. 
${}^{3}Cette prière est bonne et agréable à Dieu notre Sauveur, 
${}^{4}car il veut que tous les hommes soient sauvés et parviennent à la pleine connaissance de la vérité.
        ${}^{5}En effet, il n’y a qu’un seul Dieu ;
        \\il n’y a aussi qu’un seul médiateur entre Dieu et les hommes :
        \\un homme, le Christ Jésus,
        ${}^{6}qui s’est donné lui-même en rançon pour tous.
      Aux temps fixés, il a rendu ce témoignage, 
${}^{7}pour lequel j’ai reçu la charge de messager et d’apôtre – je dis vrai, je ne mens pas – moi qui enseigne aux nations la foi et la vérité.
${}^{8}Je voudrais donc qu’en tout lieu les hommes prient en élevant les mains, saintement, sans colère ni dispute. 
${}^{9}De même les femmes : qu’elles portent une tenue décente, avec pudeur et modestie, plutôt que de se parer de tresses, d’or ou de perles, ou de vêtements précieux ; 
${}^{10}ce qui convient à des femmes qui veulent exprimer leur piété envers Dieu, c’est de faire le bien. 
${}^{11}Que la femme reçoive l’instruction dans le calme, en toute soumission. 
${}^{12}Je ne permets pas à une femme d’enseigner, ni de dominer son mari ; mais qu’elle reste dans le calme. 
${}^{13}En effet, Adam a été modelé le premier, et Ève ensuite. 
${}^{14}Et ce n’est pas Adam qui a été trompé par le serpent, c’est la femme qui s’est laissé tromper, et qui est tombée dans la transgression. 
${}^{15}Mais la femme sera sauvée en devenant mère, à condition de rester avec modestie dans la foi, la charité et la recherche de la sainteté.
      <p class="cantique" id="bib_ct-nt_7"><span class="cantique_label">Cantique NT 7</span> = <span class="cantique_ref"><a class="unitex_link" href="#bib_1tm_3_16">1 Tm 3, 16</a></span>
      
         
      \bchapter{}
      \begin{verse}
${}^{1}Voici une parole digne de foi : si quelqu’un aspire à la responsabilité d’une communauté, c’est une belle tâche qu’il désire. 
${}^{2}Le responsable doit être irréprochable, époux d’une seule femme, un homme sobre, raisonnable, équilibré, accueillant, capable d’enseigner, 
${}^{3}ni buveur ni brutal mais bienveillant, ni querelleur ni cupide. 
${}^{4}Il faut qu’il dirige bien les gens de sa propre maison, qu’il obtienne de ses enfants l’obéissance et se fasse respecter. 
${}^{5}Car si quelqu’un ne sait pas diriger sa propre maison, comment pourrait-il prendre en charge une Église de Dieu ? 
${}^{6}Il ne doit pas être un nouveau converti ; sinon, aveuglé par l’orgueil, il pourrait tomber sous la même condamnation que le diable. 
${}^{7}Il faut aussi que les gens du dehors portent sur lui un bon témoignage, pour qu’il échappe au mépris des hommes et au piège du diable.
${}^{8}Les diacres, eux aussi, doivent être dignes de respect, n’avoir qu’une parole, ne pas s’adonner à la boisson, refuser les profits malhonnêtes, 
${}^{9}garder le mystère de la foi dans une conscience pure. 
${}^{10}On les mettra d’abord à l’épreuve ; ensuite, s’il n’y a rien à leur reprocher, ils serviront comme diacres. 
${}^{11}Les femmes, elles aussi, doivent être dignes de respect, ne pas être médisantes, mais sobres et fidèles en tout. 
${}^{12}Que le diacre soit l’époux d’une seule femme, qu’il mène bien ses enfants et sa propre famille. 
${}^{13}Les diacres qui remplissent bien leur ministère obtiennent ainsi une position estimable et beaucoup d’assurance grâce à leur foi au Christ Jésus.
${}^{14}Je t’écris avec l’espoir d’aller te voir bientôt. 
${}^{15}Mais au cas où je tarderais, je veux que tu saches comment il faut se comporter dans la maison de Dieu, c’est-à-dire la communauté, l’Église du Dieu vivant, elle qui est le pilier et le soutien de la vérité.
        ${}^{16}Assurément, il est grand, le mystère de notre religion :
       
        c’est le Christ
         
        manifesté dans la chair,
        justifié dans l’Esprit,
         
        apparu aux anges,
        proclamé dans les nations,
         
        cru dans le monde,
        enlevé dans la gloire !
      
         
      \bchapter{}
      \begin{verse}
${}^{1}L’Esprit dit clairement qu’aux derniers temps certains abandonneront la foi, pour s’attacher à des esprits trompeurs, à des doctrines démoniaques ; 
${}^{2}ils seront égarés par le double jeu des menteurs dont la conscience est marquée au fer rouge ; 
${}^{3}ces derniers empêchent les gens de se marier, ils disent de s’abstenir d’aliments, créés pourtant par Dieu pour être consommés dans l’action de grâce par ceux qui sont croyants et connaissent pleinement la vérité. 
${}^{4}Or tout ce que Dieu a créé est bon, et rien n’est à rejeter si on le prend dans l’action de grâce, 
${}^{5}car alors, cela est sanctifié par la parole de Dieu et la prière.
${}^{6}En exposant ces choses aux frères, tu seras un bon serviteur du Christ Jésus, nourri des paroles de la foi et de la bonne doctrine que tu as toujours suivie. 
${}^{7}Quant aux récits mythologiques, ces racontars de vieilles femmes, écarte-les. Exerce-toi, au contraire, à la piété. 
${}^{8}En effet, l’exercice physique n’a qu’une utilité partielle, mais la religion concerne tout, car elle est promesse de vie, de vie présente et de vie future. 
${}^{9}Voilà une parole digne de foi, et qui mérite d’être accueillie sans réserve : 
${}^{10}si nous nous donnons de la peine et si nous combattons, c’est parce que nous avons mis notre espérance dans le Dieu vivant, qui est le Sauveur de tous les hommes et, au plus haut point, des croyants.
${}^{11}Voilà ce que tu dois prescrire et enseigner. 
${}^{12}Que personne n’ait lieu de te mépriser parce que tu es jeune ; au contraire, sois pour les croyants un modèle par ta parole et ta conduite, par ta charité, ta foi et ta pureté. 
${}^{13}En attendant que je vienne, applique-toi à lire l’Écriture aux fidèles, à les encourager et à les instruire. 
${}^{14}Ne néglige pas le don de la grâce en toi, qui t’a été donné au moyen d’une parole prophétique, quand le collège des Anciens a imposé les mains sur toi. 
${}^{15}Prends à cœur tout cela, applique-toi, afin que tous voient tes progrès. 
${}^{16}Veille sur toi-même et sur ton enseignement. Maintiens-toi dans ces dispositions. En agissant ainsi, tu obtiendras le salut, et pour toi-même et pour ceux qui t’écoutent.
      
         
      \bchapter{}
      \begin{verse}
${}^{1}Avec un homme âgé, ne sois pas brutal, mais encourage-le comme un père, les jeunes gens comme des frères, 
${}^{2}les femmes âgées comme des mères, et les plus jeunes comme des sœurs, en toute pureté.
      
         
${}^{3}Honore et assiste les veuves qui sont vraiment seules. 
${}^{4}Si une veuve a des enfants ou des petits-enfants, ils doivent apprendre que c’est à eux d’abord de s’acquitter de leurs devoirs filiaux et de rendre à leurs parents ce qu’ils ont reçu d’eux. Voilà une conduite agréable à Dieu. 
${}^{5}Mais la véritable veuve, celle qui reste seule, a mis son espérance en Dieu : elle ne cesse de faire des demandes et des prières nuit et jour. 
${}^{6}Quant à celle qui se livre aux plaisirs, elle a beau vivre, elle est morte. 
${}^{7}Insiste sur tout cela, pour qu’elles soient irréprochables. 
${}^{8}Si quelqu’un ne s’occupe pas des siens, surtout des plus proches, il a renié la foi, il est pire qu’un incroyant.
${}^{9}Pour être inscrite comme veuve, une femme doit avoir au moins soixante ans, n’avoir eu qu’un seul mari, 
${}^{10}être connue pour le bien qu’elle a fait, avoir élevé des enfants, donné l’hospitalité aux voyageurs, rendu aux fidèles les plus humbles services, secouru les malheureux. Bref, il faut que, dans tous les domaines, elle se soit dévouée. 
${}^{11}Mais les veuves plus jeunes, ne les admets pas. En effet, quand la passion les détourne du Christ, elles veulent se remarier, 
${}^{12}et se condamnent ainsi en rejetant leur premier engagement. 
${}^{13}En même temps, n’ayant rien à faire, elles s’habituent à courir toutes les maisons, non seulement sans rien faire, mais bavardant, se mêlant de tout, parlant à tort et à travers. 
${}^{14}Je veux donc que les plus jeunes se remarient, qu’elles aient des enfants, qu’elles tiennent leur maison, sans donner aucune prise aux insultes de l’adversaire. 
${}^{15}Car déjà quelques-unes se sont détournées pour suivre Satan. 
${}^{16}Si une croyante a des veuves dans sa famille, qu’elle les assiste : ainsi l’Église ne sera pas surchargée et pourra assister les veuves qui sont vraiment seules.
${}^{17}Les Anciens qui exercent bien la présidence méritent une double rémunération, surtout ceux qui se donnent de la peine pour la Parole et l’enseignement. 
${}^{18}Car l’Écriture dit : Le bœuf qui foule le grain, tu ne lui mettras pas de muselière, et encore : L’ouvrier mérite son salaire. 
${}^{19}Contre un Ancien n’accepte pas d’accusation, sauf s’il y a deux ou trois témoins. 
${}^{20}Ceux qui commettent des péchés, reprends-les devant tout le monde, afin que les autres aussi éprouvent de la crainte. 
${}^{21}Devant Dieu et le Christ Jésus et devant les anges que Dieu a choisis, je te le demande solennellement : garde ces règles sans parti pris, et ne fais rien par favoritisme. 
${}^{22}Ne décide pas trop vite d’imposer les mains à quelqu’un, ne te rends pas complice des péchés d’autrui, garde-toi pur.
${}^{23}Cesse de ne boire que de l’eau, mais prends un peu de vin, à cause de ton estomac et de tes fréquents malaises. 
${}^{24}Il y a des gens dont les péchés sont manifestes avant même tout jugement ; chez d’autres, ils n’apparaissent que plus tard. 
${}^{25}De même, ce que l’on fait de bien est souvent manifeste, et s’il en va autrement, cela ne peut rester longtemps caché.
      
         
      \bchapter{}
      \begin{verse}
${}^{1}Tous ceux qui sont sous le joug de l’esclavage doivent considérer leurs maîtres comme tout à fait dignes d’honneur, pour que le nom de Dieu et l’enseignement de la foi ne soient pas blasphémés. 
${}^{2}Et s’ils ont des maîtres croyants, qu’ils ne les respectent pas moins sous prétexte que ce sont des frères ; mais qu’ils les servent d’autant mieux que ceux qui bénéficient de leur activité sont des croyants bien-aimés.
      
         
      <a class="anchor verset_lettre" id="bib_1tm_6_2_c"/>Voilà ce que tu dois enseigner et recommander. 
${}^{3}Si quelqu’un donne un enseignement différent, et n’en vient pas aux paroles solides, celles de notre Seigneur Jésus Christ, et à l’enseignement qui est en accord avec la piété, 
${}^{4}un tel homme est aveuglé par l’orgueil, il ne sait rien, c’est un malade de la discussion et des querelles de mots. De tout cela, il ne sort que jalousie, rivalité, blasphèmes, soupçons malveillants, 
${}^{5}disputes interminables de gens à l’intelligence corrompue, qui sont coupés de la vérité et ne voient dans la religion qu’une source de profit.
${}^{6}Certes, il y a un grand profit dans la religion si l’on se contente de ce que l’on a. 
${}^{7}De même que nous n’avons rien apporté dans ce monde, nous n’en pourrons rien emporter. 
${}^{8}Si nous avons de quoi manger et nous habiller, sachons nous en contenter.
${}^{9}Ceux qui veulent s’enrichir tombent dans le piège de la tentation, dans une foule de convoitises absurdes et dangereuses, qui plongent les gens dans la ruine et la perdition. 
${}^{10}Car la racine de tous les maux, c’est l’amour de l’argent. Pour s’y être attachés, certains se sont égarés loin de la foi et se sont infligé à eux-mêmes des tourments sans nombre.
${}^{11}Mais toi, homme de Dieu, fuis tout cela ; recherche la justice, la piété, la foi, la charité, la persévérance et la douceur. 
${}^{12}Mène le bon combat, celui de la foi, empare-toi de la vie éternelle ! C’est à elle que tu as été appelé, c’est pour elle que tu as prononcé ta belle profession de foi devant de nombreux témoins.
${}^{13}Et maintenant, en présence de Dieu qui donne vie à tous les êtres, et en présence du Christ Jésus qui a témoigné devant Ponce Pilate par une belle affirmation, voici ce que je t’ordonne : 
${}^{14}garde le commandement du Seigneur, en demeurant sans tache, irréprochable jusqu’à la Manifestation de notre Seigneur Jésus Christ. 
${}^{15}Celui qui le fera paraître aux temps fixés, c’est Dieu,
        \\Souverain unique et bienheureux,
        \\Roi des rois et Seigneur des seigneurs ;
        ${}^{16}lui seul possède l’immortalité,
        \\habite une lumière inaccessible ;
        \\aucun homme ne l’a jamais vu,
        \\et nul ne peut le voir.
        \\À lui, honneur et puissance éternelle. Amen.
${}^{17}Quant aux riches de ce monde, ordonne-leur de ne pas céder à l’orgueil. Qu’ils mettent leur espérance non pas dans des richesses incertaines, mais en Dieu qui nous procure tout en abondance pour que nous en profitions. 
${}^{18}Qu’ils fassent du bien et deviennent riches du bien qu’ils font ; qu’ils donnent de bon cœur et sachent partager. 
${}^{19}De cette manière, ils amasseront un trésor pour bien construire leur avenir et obtenir la vraie vie.
${}^{20}Timothée, garde le dépôt de la foi. Tourne le dos aux bavardages impies et aux objections de la pseudo-connaissance : 
${}^{21}en s’y engageant, certains se sont écartés de la foi.
      Que la grâce soit avec vous.
