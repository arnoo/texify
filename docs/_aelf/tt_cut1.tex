  
  
    
    \bbook{LETTRE À TITE}{LETTRE À TITE}
      
         
      \bchapter{}
        ${}^{1}Paul, serviteur de Dieu,
        \\apôtre de Jésus Christ
        au service de la foi de ceux que Dieu a choisis
        et de la pleine connaissance de la vérité
        qui est en accord avec la piété.
        ${}^{2}Nous avons l’espérance de la vie éternelle,
        promise depuis toujours par Dieu qui ne ment pas.
        ${}^{3}Aux temps fixés, il a manifesté sa parole
        dans la proclamation de l’Évangile
        qui m’a été confiée par ordre de Dieu notre Sauveur.
        ${}^{4}Je m’adresse à toi, Tite, mon véritable enfant
        selon la foi qui nous est commune :
        \\à toi, la grâce et la paix
        \\de la part de Dieu le Père
        et du Christ Jésus notre Sauveur.
        
           
${}^{5}Si je t’ai laissé en Crète, c’est pour que tu finisses de tout organiser et que, dans chaque ville, tu établisses des Anciens comme je te l’ai commandé moi-même. 
${}^{6}L’Ancien doit être quelqu’un qui soit sans reproche, époux d’une seule femme, ayant des enfants qui soient croyants et ne soient pas accusés d’inconduite ou indisciplinés. 
${}^{7}Il faut en effet que le responsable de communauté soit sans reproche, puisqu’il est l’intendant de Dieu ; il ne doit être ni arrogant, ni coléreux, ni buveur, ni brutal, ni avide de profits malhonnêtes ; 
${}^{8}mais il doit être accueillant, ami du bien, raisonnable, juste, saint, maître de lui. 
${}^{9}Il doit être attaché à la parole digne de foi, celle qui est conforme à la doctrine, pour être capable d’exhorter en donnant un enseignement solide, et aussi de réfuter les opposants.
${}^{10}Car il y a beaucoup de réfractaires, des gens au discours inconsistant, des marchands d’illusion, surtout parmi ceux qui viennent du judaïsme. 
${}^{11}Il faut fermer la bouche à ces gens qui, pour faire des profits malhonnêtes, bouleversent des maisons entières, en enseignant ce qu’il ne faut pas. 
${}^{12}Car l’un d’entre eux, un de leurs prophètes, l’a bien dit : Crétois toujours menteurs, mauvaises bêtes, gloutons fainéants !
${}^{13}Ce témoignage est vrai. Pour cette raison, réfute-les vigoureusement, afin qu’ils retrouvent la santé de la foi, 
${}^{14}au lieu de s’attacher à des récits légendaires du judaïsme et à des préceptes de gens qui se détournent de la vérité. 
${}^{15}Tout est pur pour les purs ; mais pour ceux qui sont souillés et qui refusent de croire, rien n’est pur : leur intelligence, aussi bien que leur conscience, est souillée. 
${}^{16}Ils proclament qu’ils connaissent Dieu, mais, par leurs actes, ils le rejettent, abominables qu’ils sont, révoltés, totalement inaptes à faire le bien.
      
         
      \bchapter{}
      \begin{verse}
${}^{1}Quant à toi, dis ce qui est conforme à l’enseignement de la saine doctrine. 
${}^{2}Que les hommes âgés soient sobres, dignes de respect, pondérés, et solides dans la foi, la charité et la persévérance. 
${}^{3}De même, que les femmes âgées mènent une vie sainte, ne soient pas médisantes ni esclaves de la boisson, et qu’elles soient de bon conseil, 
${}^{4}pour apprendre aux jeunes femmes à aimer leur mari et leurs enfants, 
${}^{5}à être raisonnables et pures, bonnes maîtresses de maison, aimables, soumises à leur mari, afin que la parole de Dieu ne soit pas exposée au blasphème.
${}^{6}Les jeunes aussi, exhorte-les à être raisonnables 
${}^{7}en toutes choses. Toi-même, sois un modèle par ta façon de bien agir, par un enseignement sans défaut et digne de respect, 
${}^{8}par la solidité inattaquable de ta parole, pour la plus grande confusion de l’adversaire, qui ne trouvera aucune critique à faire sur nous.
${}^{9}Que les esclaves soient soumis à leur maître en toutes choses, qu’ils se rendent agréables, qu’ils ne soient pas contestataires, 
${}^{10}qu’ils ne dérobent rien, mais qu’ils montrent une parfaite fidélité, pour faire honneur en tout à l’enseignement de Dieu notre Sauveur.
${}^{11}Car la grâce de Dieu s’est manifestée pour le salut de tous les hommes. 
${}^{12}Elle nous apprend à renoncer à l’impiété et aux convoitises de ce monde, et à vivre dans le temps présent de manière raisonnable, avec justice et piété, 
${}^{13}attendant que se réalise la bienheureuse espérance : la manifestation de la gloire de notre grand Dieu et Sauveur, Jésus Christ. 
${}^{14}Car il s’est donné pour nous afin de nous racheter de toutes nos fautes, et de nous purifier pour faire de nous son peuple, un peuple ardent à faire le bien. 
${}^{15}Voilà comment tu dois parler, exhorter et réfuter, en toute autorité. Que personne n’ait lieu de te mépriser.
      
         
      \bchapter{}
      \begin{verse}
${}^{1}Rappelle à tous qu’ils doivent être soumis aux gouvernants et aux autorités, qu’ils doivent leur obéir et être prêts à faire tout ce qui est bien ; 
${}^{2}qu’ils n’insultent personne, ne soient pas violents, mais bienveillants, montrant une douceur constante à l’égard de tous les hommes.
${}^{3}Car nous aussi, autrefois, nous étions insensés, révoltés, égarés, esclaves de toutes sortes de convoitises et de plaisirs ; nous vivions dans la méchanceté et la jalousie, nous étions odieux et remplis de haine les uns pour les autres.
        ${}^{4}Mais lorsque Dieu, notre Sauveur,
        \\a manifesté sa bonté
        et son amour pour les hommes,
        ${}^{5}il nous a sauvés,
        \\non pas à cause de la justice de nos propres actes,
        mais par sa miséricorde.
        \\Par le bain du baptême,
        il nous a fait renaître
        \\et nous a renouvelés
        dans l’Esprit Saint.
        ${}^{6}Cet Esprit, Dieu l’a répandu
        sur nous en abondance,
        \\par Jésus Christ notre Sauveur,
        ${}^{7}afin que, rendus justes par sa grâce,
        \\nous devenions en espérance
        héritiers de la vie éternelle.,
${}^{8}Voilà une parole digne de foi, et je veux que tu t’en portes garant, afin que ceux qui ont mis leur foi en Dieu aient à cœur d’être les premiers pour faire le bien : c’est cela qui est bon et utile pour les hommes. 
${}^{9}Mais les recherches folles, les généalogies, les disputes et les polémiques sur la Loi, évite-les, car elles sont inutiles et vaines. 
${}^{10}Quant à l’hérétique, après un premier et un second avertissement, écarte-le, 
${}^{11}sachant qu’un tel homme est perverti et pécheur : il se condamne lui-même.
${}^{12}Lorsque je t’aurai envoyé Artémas ou Tychique, efforce-toi de me rejoindre à Nicopolis : c’est là que j’ai décidé de passer l’hiver. 
${}^{13}Prends soin de fournir au juriste Zénas ainsi qu’à Apollos ce qu’il faut pour leur voyage, afin qu’ils ne manquent de rien. 
${}^{14}Que ceux de chez nous apprennent aussi à être les premiers pour faire le bien et répondre aux nécessités urgentes : ainsi ils ne manqueront pas de produire du fruit.
${}^{15}Ceux qui sont avec moi te saluent tous. Salue nos amis dans la foi.
      Que la grâce soit avec vous tous.
