  
  
    
    \bbook{MALACHIE}{MALACHIE}
      
         
      \bchapter{}
      \begin{verse}
${}^{1}Proclamation. Parole du Seigneur à Israël par l’intermédiaire de Malachie.
      
         
${}^{2}Je vous ai aimés, dit le Seigneur,
        et vous dites : « En quoi nous as-tu aimés ? »
        \\Ésaü n’était-il pas frère de Jacob ?
        – oracle du Seigneur.
        \\J’ai eu de l’amour pour Jacob
${}^{3}mais je n’ai pas aimé Ésaü.
        \\J’ai livré ses montagnes à la désolation,
        son héritage aux chacals du désert.
${}^{4}Si Édom déclare : « Nous avons été détruits,
        mais nous recommencerons, nous relèverons les ruines »,
        \\ainsi parle le Seigneur de l’univers :
        \\« Qu’ils relèvent, eux ! Moi, je démolirai !
        On les appellera “Territoire-de-méchanceté”,
        “Peuple-qui-met-en-colère-le-Seigneur-pour-toujours”.
${}^{5}Vos yeux le verront et vous direz :
        “Le Seigneur est grand par-delà le territoire d’Israël !” »
${}^{6}Un fils honore son père,
        et un serviteur, son maître.
        \\Si donc je suis père,
        où est l’honneur qui m’est dû ?
        \\Et si je suis maître,
        où est le respect qui m’est dû ?
        \\– déclare le Seigneur de l’univers
        à vous, les prêtres qui méprisez mon nom.
        \\Et vous dites : « En quoi avons-nous méprisé ton nom ? »
${}^{7}– En présentant sur mon autel un aliment impur.
        \\Mais vous dites : « En quoi t’avons-nous rendu impur ? »
        – En affirmant : « La table du Seigneur est méprisable ! »
${}^{8}Et quand vous présentez au sacrifice une bête aveugle,
        n’est-ce pas faire le mal ?
        \\Et quand vous présentez une bête boiteuse ou malade,
        n’est-ce pas faire le mal ?
        \\Offre-la donc à ton gouverneur !
        Sera-t-il content de toi ? Te sera-t-il favorable ?
        \\– Le Seigneur de l’univers a parlé.
         
${}^{9}Et maintenant, apaisez donc le visage de Dieu,
        pour qu’il nous fasse grâce !
        \\Cela est venu de vos mains.
        Vous sera-t-il favorable ?
        \\– Le Seigneur de l’univers a parlé.
         
${}^{10}Qui donc d’entre vous fermera les portes du sanctuaire,
        pour que vous n’allumiez plus en vain le feu sur mon autel ?
        \\Je ne prends aucun plaisir en vous,
        – dit le Seigneur de l’univers –,
        \\je ne désire plus l’offrande de vos mains.
${}^{11}Car du levant au couchant du soleil,
        mon nom est grand parmi les nations.
        \\En tout lieu, on brûle de l’encens pour mon nom
        et on présente une offrande pure,
        \\car mon nom est grand parmi les nations,
        – dit le Seigneur de l’univers.
         
${}^{12}Vous, cependant, vous le profanez en disant :
        \\« La table du Seigneur est impure ;
        méprisable, la nourriture qu’on en retire. »
${}^{13}Vous dites : « Quel ennui ! », et vous la dédaignez,
        – dit le Seigneur de l’univers.
        \\Vous apportez ce qui est volé, boiteux ou malade,
        et vous l’apportez en offrande !
        \\Puis-je l’agréer de vos mains ?
        – dit le Seigneur.
        ${}^{14}Maudit soit le tricheur qui possède un mâle dans son troupeau,
        qui fait un vœu et qui sacrifie au Seigneur une bête mutilée !
        \\Je suis un grand roi – dit le Seigneur de l’univers –,
        et mon nom inspire la crainte parmi les nations.
       
      
         
      \bchapter{}
        ${}^{1}Maintenant, prêtres, à vous cet avertissement\\ :
        ${}^{2}Si vous n’écoutez pas,
        si vous ne prenez pas à cœur de glorifier mon nom
        – dit le Seigneur de l’univers –,
        \\j’enverrai sur vous la malédiction,
        je maudirai les bénédictions que vous prononcerez.
        \\Oui, je les maudis,
        car aucun de vous ne prend rien à cœur.
${}^{3}Voici : je vais menacer votre descendance.
        \\Je vous jetterai du fumier à la figure,
        le fumier qui provient de vos fêtes ;
        on vous enlèvera avec lui.
${}^{4}Vous saurez alors que je vous ai adressé cet avertissement,
        pour que subsiste mon alliance avec mon serviteur Lévi,
        – dit le Seigneur de l’univers.
${}^{5}Mon alliance avec lui était vie et paix,
        \\je les lui accordais, ainsi que la crainte,
        et il me craignait.
        \\Devant mon nom, il restait saisi.
${}^{6}La loi de vérité était dans sa bouche,
        et rien de mal ne se trouvait sur ses lèvres.
        \\Dans la paix et la droiture, il marchait avec moi ;
        nombreux furent ceux qu’il ramena de la faute.
${}^{7}En effet, les lèvres du prêtre gardent la connaissance de la Loi,
        et l’on recherche l’instruction de sa bouche,
        \\car il est le messager du Seigneur de l’univers.
        ${}^{8}Mais vous vous êtes écartés de la route,
        vous avez fait de la Loi une occasion de chute
        pour la multitude,
        \\vous avez détruit mon alliance avec mon serviteur\\Lévi\\,
        – dit le Seigneur de l’univers.
        ${}^{9}À mon tour je vous ai méprisés,
        abaissés devant tout le peuple,
        \\puisque vous n’avez pas gardé mes chemins,
        mais agi avec partialité
        dans l’application de la Loi.
        
           
        ${}^{10}Et nous\\, n’avons-nous pas tous un seul Père\\ ?
        N’est-ce pas un seul Dieu qui nous a créés ?
        \\Pourquoi nous trahir les uns les autres,
        profanant ainsi l’Alliance de nos pères ?
${}^{11}Juda a trahi.
        Une abomination a été commise en Israël et à Jérusalem.
        \\Oui, Juda a profané ce qui est saint,
        ce qui est aimé du Seigneur :
        \\il a épousé la fille d’un dieu étranger.
${}^{12}Le Seigneur retranchera l’homme qui agit ainsi,
        celui qui en est le témoin,
        \\celui qui s’en fait l’écho depuis les tentes de Jacob
        et qui présente une offrande au Seigneur de l’univers.
${}^{13}Et encore, une deuxième chose que vous faites :
        couvrir de larmes, de pleurs et de gémissements
        l’autel du Seigneur,
        \\car il ne fait plus attention à l’offrande
        et ne désire plus recevoir ce qui vient de vos mains.
${}^{14}Et vous dites : « Pourquoi cela ? »
        \\– C’est que le Seigneur a été témoin
        entre toi et la femme de ta jeunesse :
        \\tu l’as trahie, elle, ta compagne,
        la femme de ton alliance.
${}^{15}Un seul n’a-t-il pas fait la chair,
        et le souffle de vie qui est en elle ?
        \\Et que recherche-t-il ? Une descendance divine.
        \\Vous prendrez garde à votre souffle de vie :
        que nul ne trahisse la femme de sa jeunesse.
${}^{16}Car je hais la répudiation,
        – dit le Seigneur, Dieu d’Israël –,
        \\et celui qui se couvre d’un vêtement de violence,
        – dit le Seigneur de l’univers.
        \\Vous prendrez garde à votre souffle de vie
        et vous ne trahirez pas.
         
${}^{17}Vous fatiguez le Seigneur par vos discours,
        et vous dites : « En quoi l’avons-nous fatigué ? »
        \\– C’est lorsque vous dites :
        \\« Quiconque fait le mal est bon aux yeux du Seigneur,
        en de tels hommes il prend plaisir »,
        \\– et encore : « Où est le Dieu de justice ? »
      
         
      \bchapter{}
        ${}^{1}Voici que j’envoie mon messager
        pour qu’il prépare le chemin devant moi ;
        \\et soudain viendra dans son Temple
        le Seigneur que vous cherchez.
        \\Le messager de l’Alliance que vous désirez,
        le voici qui vient, – dit le Seigneur de l’univers.
        ${}^{2}Qui pourra soutenir le jour de sa venue ?
        Qui pourra rester debout lorsqu’il se montrera ?
        \\Car il est pareil au feu du fondeur,
        pareil à la lessive des blanchisseurs.
        ${}^{3}Il s’installera pour fondre et purifier :
        il purifiera les fils de Lévi,
        \\il les affinera comme l’or et l’argent ;
        ainsi pourront-ils, aux yeux du Seigneur,
        présenter l’offrande en toute justice.
        ${}^{4}Alors, l’offrande de Juda et de Jérusalem
        sera bien accueillie du Seigneur,
        \\comme il en fut aux jours anciens,
        dans les années d’autrefois.
${}^{5}Je m’approcherai de vous pour le jugement ;
        sans attendre, je témoignerai
        \\contre les magiciens, contre les adultères,
        contre ceux qui font de faux serments,
        \\contre ceux qui oppriment le salarié, la veuve et l’orphelin,
        qui excluent l’immigré et qui ne me craignent pas,
        – dit le Seigneur de l’univers.
        
           
         
${}^{6}Moi, le Seigneur, je n’ai pas changé,
        mais vous, fils de Jacob, vous n’en finissez pas de changer :
${}^{7}depuis les jours de vos pères,
        vous vous écartez de mes décrets et ne les gardez pas.
        \\Revenez à moi, et je reviendrai à vous,
        – dit le Seigneur de l’univers.
        \\Vous demandez : « En quoi devrons-nous revenir ? »
${}^{8}– Un homme peut-il tromper Dieu ?
        Et vous me trompez !
        \\Vous dites : « En quoi t’avons-nous trompé ? »
        – Pour la dîme et les redevances.
${}^{9}Vous êtes maudits de malédiction,
        vous me trompez, vous, la nation entière !
${}^{10}Apportez toute la dîme à la maison du trésor,
        pour qu’il y ait de la nourriture dans ma Maison.
        \\Soumettez-moi donc ainsi à l’épreuve,
        – dit le Seigneur de l’univers –,
        \\et vous verrez si je n’ouvre pas pour vous les écluses du ciel
        si je ne répands pas sur vous la bénédiction en abondance !
${}^{11}Pour vous, je menacerai l’insecte vorace :
        qu’il ne détruise plus les fruits de votre sol,
        \\et que la vigne de vos campagnes ne soit plus stérile,
        – dit le Seigneur de l’univers.
${}^{12}Toutes les nations vous diront bienheureux,
        car vous serez alors une terre de délices,
        – dit le Seigneur de l’univers.
        
           
         
        ${}^{13}Vous avez contre moi des paroles dures,
        – dit le Seigneur.
        \\Et vous osez demander :
        « Qu’avons-nous dit entre nous contre toi ? »
        ${}^{14}Voici ce que vous avez dit :
        « Servir Dieu n’a pas de sens.
        À quoi bon garder ses observances,
        mener une vie sans joie
        en présence du Seigneur de l’univers ?
        ${}^{15}Nous en venons à dire bienheureux les arrogants ;
        même ceux qui font le mal sont prospères\\,
        même s’ils mettent Dieu à l’épreuve,
        ils en réchappent ! »
        
           
         
        ${}^{16}Alors ceux qui craignent le Seigneur
        s’exhortèrent mutuellement.
        \\Le Seigneur fut attentif et les écouta ;
        un livre fut écrit devant lui pour en garder mémoire,
        \\en faveur de ceux qui le craignent
        et qui ont le souci de son nom.
        ${}^{17}Le Seigneur de l’univers déclara :
        \\Ils seront mon domaine particulier
        pour le jour que je prépare.
        \\Je serai indulgent envers eux,
        comme un homme est indulgent
        envers le fils qui le sert fidèlement\\.
        ${}^{18}Vous verrez de nouveau qu’il y a une différence
        entre le juste et le méchant,
        entre celui qui sert Dieu et celui qui refuse de le servir.
        ${}^{19}Voici que vient le jour du Seigneur,
        brûlant comme la fournaise.
        \\Tous les arrogants, tous ceux qui commettent l’impiété,
        seront de la paille.
        \\Le jour qui vient les consumera,
        – dit le Seigneur de l’univers –,
        \\il ne leur laissera ni racine ni branche.
        ${}^{20}Mais pour vous qui craignez mon nom,
        le Soleil de justice se lèvera\\ :
        \\il apportera la guérison dans son rayonnement.
        \\Vous sortirez en bondissant
        comme de jeunes veaux à la pâture.
${}^{21}Vous foulerez les méchants,
        car ils seront de la cendre sous la plante de vos pieds,
        \\au jour que je prépare,
        – dit le Seigneur de l’univers.
${}^{22}Souvenez-vous de la loi de Moïse, mon serviteur,
        à qui j’ai prescrit, sur l’Horeb,
        décrets et ordonnances pour tout Israël.
        
           
         
        ${}^{23}Voici que je vais vous envoyer Élie le prophète,
        avant que vienne le jour du Seigneur,
        jour grand et redoutable\\.
        ${}^{24}Il ramènera le cœur des pères vers leurs fils,
        et le cœur des fils vers leurs pères,
        \\pour que je ne vienne pas frapper d’anathème le pays !
        <p class="verset_no_anchor"/>
        
           
