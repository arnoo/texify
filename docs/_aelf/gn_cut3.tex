  
  
      
         
      \bchapter{}
      \begin{verse}
${}^{1}De là, Abraham leva le camp pour le pays du Néguev, il habita entre Cadès et Shour, puis séjourna à Guérar. 
${}^{2}Comme Abraham disait de sa femme Sara : « C’est ma sœur », Abimélek, roi de Guérar, envoya prendre Sara. 
${}^{3}Mais, pendant la nuit, Dieu vint en songe auprès d’Abimélek et lui dit : « Voici que tu vas mourir à cause de la femme que tu as prise, car elle est mariée. » 
${}^{4}Abimélek, qui ne s’était pas approché d’elle, répondit : « Seigneur, est-ce que tu vas tuer des gens, même s’ils sont justes ? 
${}^{5}N’est-ce pas lui qui m’avait dit : “C’est ma sœur” et elle, elle aussi, ne disait-elle pas : “C’est mon frère” ? J’ai fait cela, le cœur intègre et les mains innocentes. » 
${}^{6}Toujours en songe, Dieu lui répondit : « Oui, je sais bien que tu as fait cela, le cœur intègre ; aussi, moi-même je t’ai retenu de pécher contre moi. C’est pourquoi je ne t’ai pas laissé la toucher. 
${}^{7}Maintenant, rends sa femme à cet homme, car c’est un prophète. Il intercédera en ta faveur et tu resteras en vie. Mais si tu ne rends pas la femme, sache qu’il te faudra mourir, toi et tous les tiens. »
${}^{8}Abimélek se leva de bon matin, convoqua tous ses serviteurs et leur rapporta toute l’affaire. Les hommes eurent très peur. 
${}^{9}Ensuite, Abimélek convoqua Abraham et lui dit : « Que nous as-tu fait là ! En quoi ai-je péché contre toi pour que tu nous aies exposés, moi et mon royaume, à un si grave péché ? Tu as fait à mon égard une chose qui ne se fait pas ! » 
${}^{10}Abimélek dit encore à Abraham : « Qu’avais-tu en vue pour agir ainsi ? » 
${}^{11}Abraham répondit : « Je m’étais dit : pour sûr, en cet endroit, il n’y a aucune crainte de Dieu ; ils me tueront à cause de ma femme. 
${}^{12}De plus, c’est vrai qu’elle est ma sœur, la fille de mon père mais non celle de ma mère. Et elle est devenue ma femme. 
${}^{13}Lorsque Dieu me fit errer loin de la maison de mon père, j’ai dit à Sara : “Voici la faveur que tu me feras : partout où nous irons, dis de moi : C’est mon frère.” »
${}^{14}Alors, Abimélek prit du petit et du gros bétail, des serviteurs et des servantes ; il les donna à Abraham et lui rendit Sara, sa femme. 
${}^{15}Puis Abimélek dit : « Voici, devant toi, mon pays. Habite où bon te semblera ! » 
${}^{16}Et il dit à Sara : « Voici que je donne mille pièces d’argent à ton frère ; ce sera pour toi comme un voile sur les yeux de tous ceux qui t’entourent et, vis-à-vis de tous, tu seras réhabilitée. »
${}^{17}Abraham intercéda auprès de Dieu, et Dieu guérit Abimélek, sa femme et ses servantes qui purent avoir des enfants. 
${}^{18}En effet, Dieu avait rendu stériles toutes les femmes de la maison d’Abimélek à cause de Sara, la femme d’Abraham.
      
         
      \bchapter{}
      \begin{verse}
${}^{1}Le Seigneur visita Sara comme il l’avait annoncé ; il agit pour elle comme il l’avait dit. 
${}^{2}Elle devint enceinte, et elle enfanta un fils pour Abraham dans sa vieillesse, à la date que Dieu avait fixée. 
${}^{3}Et Abraham donna un nom au fils que Sara lui avait enfanté : il l’appela Isaac (c’est-à-dire : Il rit). 
${}^{4}Quand Isaac eut huit jours, Abraham le circoncit, comme Dieu le lui avait ordonné. 
${}^{5}Abraham avait cent ans quand naquit son fils Isaac.
${}^{6}Sara dit :
        \\« Dieu m’a donné l’occasion de rire :
        \\quiconque l’apprendra rira à mon sujet. »
${}^{7}Puis elle ajouta :
      « Qui aurait dit à Abraham que Sara allaiterait des fils ? Et pourtant j’ai donné un fils à sa vieillesse ! »
${}^{8}L’enfant grandit, et il fut sevré. Abraham donna un grand festin le jour où Isaac fut sevré. 
${}^{9} Or, Sara regardait s’amuser Ismaël, ce fils qu’Abraham avait eu d’Agar l’Égyptienne. 
${}^{10} Elle dit à Abraham : « Chasse cette servante et son fils ; car le fils de cette servante ne doit pas partager l’héritage de mon fils Isaac. » 
${}^{11} Cette parole attrista beaucoup Abraham, à cause de son fils Ismaël\\, 
${}^{12} mais Dieu lui dit : « Ne sois pas triste à cause du garçon et de ta servante ; écoute tout ce que Sara te dira\\, car c’est par Isaac qu’une descendance portera ton nom ; 
${}^{13} mais je ferai aussi une nation du fils de la servante, car lui aussi est de ta descendance. »
${}^{14}Abraham se leva de bon matin, il prit du pain et une outre d’eau, il les posa sur l’épaule d’Agar, il lui remit l’enfant, puis il la renvoya. Elle partit et alla errer dans le désert de Bershéba. 
${}^{15} Quand l’eau de l’outre fut épuisée, elle laissa l’enfant sous un buisson, 
${}^{16} et alla s’asseoir non loin de là, à la distance d’une portée de flèche. Elle se disait : « Je ne veux pas voir mourir l’enfant ! » Elle s’assit non loin de là. Elle éleva la voix et pleura.
${}^{17}Dieu entendit la voix du petit garçon ; et du ciel, l’ange de Dieu appela Agar : « Qu’as-tu, Agar ? Sois sans crainte, car Dieu a entendu la voix du petit garçon, sous le buisson où il était. 
${}^{18} Debout ! Prends le garçon et tiens-le par la main, car je ferai de lui une grande nation. » 
${}^{19} Alors, Dieu ouvrit les yeux d’Agar, et elle aperçut un puits. Elle alla remplir l’outre et fit boire le garçon.
${}^{20}Dieu fut avec lui, il grandit et habita au désert, et il devint un tireur à l’arc. 
${}^{21}Il habita au désert de Parane, et sa mère lui choisit une femme du pays d’Égypte.
${}^{22}En ce temps-là, Abimélek accompagné de Pikol, le chef de son armée, vint dire à Abraham : « Dieu est avec toi en tout ce que tu fais. 
${}^{23}Maintenant, ici même, jure-moi par Dieu de ne pas me trahir, ni moi, ni ma lignée, ni ma postérité : tu montreras envers moi et envers ce pays où tu séjournes la même faveur que celle que j’ai montrée envers toi. » 
${}^{24}Abraham répondit : « Moi, je le jure. »
${}^{25}Abraham fit des reproches à Abimélek au sujet d’un puits d’eau que les serviteurs de ce dernier avaient accaparé. 
${}^{26}Abimélek dit : « Je ne sais qui a fait cela ! Jamais tu ne m’en as informé, et moi, je n’ai rien entendu à ce sujet avant ce jour ! »
${}^{27}Abraham prit du petit et du gros bétail, et les donna à Abimélek : tous deux conclurent une alliance. 
${}^{28}Abraham mit à part sept jeunes brebis de son petit bétail. 
${}^{29}Abimélek demanda à Abraham : « Que font là ces sept jeunes brebis que tu as mises à part ? » 
${}^{30}Abraham répondit : « C’est pour que tu les reçoives de ma propre main et qu’elles soient un témoignage de ce que j’ai moi-même creusé ce puits. »
${}^{31}C’est pourquoi on appela ce lieu « Bershéba » (c’est-à-dire : le Puits-du-Serment), car tous deux y avaient prêté serment. 
${}^{32}Ils conclurent donc une alliance à Bershéba. Abimélek se leva et, accompagné de Pikol, le chef de son armée, il retourna au pays des Philistins.
${}^{33}Abraham planta un tamaris à Bershéba et y invoqua le nom du Seigneur, Dieu éternel. 
${}^{34}Abraham séjourna longtemps au pays des Philistins.
      
         
      \bchapter{}
      \begin{verse}
${}^{1}Après ces événements, Dieu mit Abraham à l’épreuve. Il lui dit : « Abraham ! » Celui-ci répondit : « Me voici ! » 
${}^{2}Dieu dit : « Prends ton fils, ton unique, celui que tu aimes, Isaac, va au pays de Moriah, et là tu l’offriras en holocauste sur la montagne que je t’indiquerai. »
${}^{3}Abraham se leva de bon matin, sella son âne, et prit avec lui deux de ses serviteurs et son fils Isaac. Il fendit le bois pour l’holocauste, et se mit en route vers l’endroit que Dieu lui avait indiqué. 
${}^{4} Le troisième jour, Abraham, levant les yeux, vit l’endroit de loin. 
${}^{5} Abraham dit à ses serviteurs : « Restez ici avec l’âne. Moi et le garçon nous irons jusque là-bas pour adorer\\, puis nous reviendrons vers vous. »
${}^{6}Abraham prit le bois pour l’holocauste et le chargea sur son fils Isaac ; il prit\\le feu et le couteau, et tous deux s’en allèrent ensemble. 
${}^{7} Isaac dit à son père Abraham\\ : « Mon père\\ ! – Eh bien, mon fils ? » Isaac reprit : « Voilà le feu et le bois, mais où est l’agneau pour l’holocauste ? » 
${}^{8} Abraham répondit : « Dieu saura bien trouver\\l’agneau pour l’holocauste, mon fils. » Et ils s’en allaient tous les deux ensemble.
${}^{9}Ils arrivèrent à l’endroit que Dieu avait indiqué. Abraham y bâtit l’autel et disposa le bois ; puis il lia son fils Isaac et le mit sur l’autel, par-dessus le bois. 
${}^{10} Abraham étendit la main et saisit le couteau pour immoler son fils.
${}^{11}Mais l’ange du Seigneur l’appela du haut du ciel et dit : « Abraham ! Abraham ! » Il répondit : « Me voici ! » 
${}^{12}L’ange lui dit : « Ne porte pas la main sur le garçon ! Ne lui fais aucun mal ! Je sais maintenant que tu crains Dieu : tu ne m’as pas refusé ton fils, ton unique. » 
${}^{13}Abraham leva les yeux et vit un bélier retenu par les cornes dans un buisson. Il alla prendre le bélier et l’offrit en holocauste à la place de son fils. 
${}^{14}Abraham donna à ce lieu le nom de « Le-Seigneur-voit ». On l’appelle aujourd’hui : « Sur-le-mont-le-Seigneur-est-vu. »
${}^{15}Du ciel, l’ange du Seigneur appela une seconde fois Abraham. 
${}^{16} Il déclara : « Je le jure par moi-même, oracle du Seigneur : parce que tu as fait cela, parce que tu ne m’as pas refusé ton fils, ton unique, 
${}^{17} je te comblerai de bénédictions, je rendrai ta descendance aussi nombreuse que les étoiles du ciel et que le sable au bord de la mer, et ta descendance occupera\\les places fortes de ses ennemis. 
${}^{18} Puisque tu as écouté ma voix, toutes les nations de la terre s’adresseront l’une à l’autre la bénédiction par le nom de\\ta descendance. »
${}^{19}Alors Abraham retourna auprès de ses serviteurs et ensemble ils se mirent en route pour Bershéba ; et Abraham y habita.
${}^{20}Après ces événements, on annonça à Abraham la nouvelle suivante : « Voici que Milka, elle aussi, a donné des fils à ton frère Nahor : 
${}^{21}Ouç, son premier-né, Bouz, le frère de celui-ci, Quemouël, père d’Aram ; 
${}^{22}Késed, Hazo, Pildash, Yidlaf et Betouël. 
${}^{23}– Betouël est celui qui engendra Rébecca. » Ce sont les huit enfants que Milka donna à Nahor, le frère d’Abraham. 
${}^{24}Sa concubine, nommée Réouma, eut aussi des enfants : Tébah, Gaham, Tahash et Maaka.
      
         
      \bchapter{}
      \begin{verse}
${}^{1}Sara vécut cent vingt-sept ans\\. 
${}^{2}Elle mourut à Kiriath-Arba, c’est-à-dire à Hébron, dans le pays de Canaan. Abraham s’y rendit pour le deuil\\et les lamentations. 
${}^{3}Puis il laissa le corps pour aller parler aux Hittites qui habitaient le pays\\ : 
${}^{4}« Je ne suis qu’un immigré, un hôte, parmi vous ; accordez-moi d’acquérir chez vous une propriété funéraire où je pourrai enterrer cette morte\\. » 
${}^{5}Les Hittites répondirent à Abraham : 
${}^{6}« Écoute, mon seigneur. Tu es, au milieu de nous, un prince de Dieu. Ensevelis ta morte dans le meilleur de nos tombeaux. Aucun d’entre nous ne te refusera son tombeau pour y ensevelir ta morte. »
${}^{7}Abraham se leva et se prosterna devant le peuple de ce pays, les Hittites. 
${}^{8}Puis il leur parla en ces termes : « Si vous acceptez que j’ensevelisse cette morte, alors écoutez-moi. Intervenez pour moi auprès d’Éphrone, fils de Sohar, 
${}^{9}pour qu’il me cède la caverne de Macpéla qui lui appartient et qui se trouve au bout de son champ. Qu’il me la cède contre sa valeur en argent, comme une propriété funéraire au milieu de vous. » 
${}^{10}Éphrone était assis parmi les Hittites. Éphrone le Hittite répondit à Abraham de façon à être entendu des Hittites et de tous ceux qui entraient par la porte de la ville. Il dit : 
${}^{11}« Non, mon seigneur ! Écoute-moi ! Le champ, je te le donne ; et la caverne qui s’y trouve, je te la donne ; aux yeux des fils de mon peuple, je te la donne : ensevelis ta morte ! »
${}^{12}Alors, Abraham se prosterna devant le peuple du pays. 
${}^{13}Il parla à Éphrone de façon à être entendu par le peuple du pays. Il dit : « Si seulement tu voulais m’écouter ! Je te donne l’argent pour le champ. Accepte-le de moi. Et là j’ensevelirai ma morte. » 
${}^{14}Éphrone répondit à Abraham : 
${}^{15}« Écoute-moi, mon seigneur ! Un terrain de quatre cents pièces d’argent, qu’est-ce donc entre toi et moi ? Ensevelis donc ta morte ! » 
${}^{16}Abraham écouta Éphrone et pesa pour lui l’argent dont il avait parlé de façon à être entendu des Hittites : quatre cents pièces d’argent au taux du marché.
${}^{17}Ainsi, le champ d’Éphrone qui se trouve à Macpéla, en face de Mambré, le champ et la caverne, avec tous les arbres qui y poussent, sur toute sa superficie, tout devint 
${}^{18}possession d’Abraham, aux yeux des Hittites et de tous ceux qui entraient par la porte de la ville. 
${}^{19}Après quoi, Abraham ensevelit sa femme Sara dans la caverne du champ de Macpéla, qui est en face de Mambré c’est-à-dire à Hébron, dans le pays de Canaan. 
${}^{20}Ainsi donc les Hittites garantirent à Abraham la propriété funéraire du champ et de la caverne qui s’y trouvait.
      
         
      \bchapter{}
      \begin{verse}
${}^{1}Abraham était vieux, avancé en âge, et le Seigneur l’avait béni en toute chose. 
${}^{2}Abraham dit au plus ancien serviteur de sa maison, l’intendant de tous ses biens\\ : 
${}^{3}« Je te fais prêter serment par le Seigneur, Dieu du ciel et Dieu de la terre : tu ne prendras pas pour mon fils une épouse parmi les filles des Cananéens au milieu desquels j’habite. 
${}^{4}Mais tu iras dans mon pays, dans ma parenté, chercher une épouse pour mon fils Isaac. » 
${}^{5}Le serviteur lui demanda : « Et si cette femme ne consent pas à me suivre pour venir ici ? Devrai-je alors ramener ton fils dans le pays d’où tu es sorti ? » 
${}^{6}Abraham lui répondit : « Garde-toi d’y ramener mon fils ! 
${}^{7}Le Seigneur, le Dieu du ciel, lui qui m’a pris de la maison de mon père et du pays de ma parenté, m’a déclaré avec serment : “À ta descendance je donnerai le pays que voici\\.” C’est lui qui enverra son ange devant toi, et tu prendras là-bas une épouse pour mon fils. 
${}^{8}Si cette femme ne consent pas à te suivre, tu seras dégagé du serment que je t’impose. Mais, en tout cas, tu n’y ramèneras pas mon fils. » 
${}^{9}Le serviteur prêta à son maître Abraham un serment solennel concernant cette affaire.
${}^{10}Parmi les chameaux de son maître, le serviteur en prit dix et il s’en alla, emportant tout ce que son maître avait de meilleur. Il se leva et s’en alla vers l’Aram-des-deux-Fleuves, à la ville de Nahor. 
${}^{11}Il fit agenouiller les chameaux en dehors de la ville, près d’un puits d’eau, à l’heure du soir, l’heure où les femmes sortent pour y puiser. 
${}^{12}Il dit : « Seigneur, Dieu de mon maître Abraham, permets-moi de faire aujourd’hui une heureuse rencontre et montre ta faveur à l’égard de mon maître Abraham. 
${}^{13}Me voici debout près de la source, et les filles des gens de la ville sortent pour puiser de l’eau. 
${}^{14}La jeune fille à qui je dirai : “Incline ta cruche pour que je boive”, et qui répondra : “Bois et je vais aussi abreuver tes chameaux”, que cette jeune fille soit celle que tu destines à ton serviteur Isaac ; je saurai ainsi que tu as montré ta faveur à l’égard de mon maître. »
${}^{15}Il n’avait pas fini de parler que sortit Rébecca, la fille de Betouël, fils de Milka, elle-même femme de Nahor, le frère d’Abraham ; elle portait sa cruche sur l’épaule. 
${}^{16}La jeune fille avait très belle apparence, elle était vierge, aucun homme ne s’était uni à elle. Elle descendit à la source, emplit sa cruche et remonta. 
${}^{17}Le serviteur courut à sa rencontre et dit : « De grâce, donne-moi à boire une gorgée d’eau de ta cruche ! » 
${}^{18}Elle répondit : « Bois, mon seigneur. » Et, de la main, elle s’empressa d’abaisser la cruche pour lui donner à boire. 
${}^{19}Quand elle eut fini de lui donner à boire, elle dit : « Pour tes chameaux aussi, j’irai puiser jusqu’à ce qu’ils aient bu à satiété. » 
${}^{20}Elle s’empressa de vider la cruche dans l’abreuvoir et courut de nouveau chercher de l’eau au puits. Elle puisa ainsi pour tous les chameaux. 
${}^{21}L’homme la regardait, silencieux, se demandant si, oui ou non, le Seigneur avait fait réussir son voyage.
${}^{22}Dès que les chameaux eurent fini de boire, l’homme prit un anneau d’or pesant un demi-sicle et deux bracelets d’or pesant dix sicles pour ses poignets. 
${}^{23}Il lui demanda : « De qui es-tu la fille ? Dis-le moi, je t’en prie. Y a-t-il dans la maison de ton père un endroit où passer la nuit ? » 
${}^{24}Elle répondit : « Je suis la fille de Betouël, le fils que Milka a donné à Nahor. » 
${}^{25}Et elle ajouta : « Il y a beaucoup de paille et de fourrage chez nous, et aussi de la place où passer la nuit. » 
${}^{26}L’homme s’inclina et se prosterna devant le Seigneur, 
${}^{27}en disant : « Béni soit le Seigneur, Dieu de mon maître Abraham ! Il n’a pas cessé de manifester sa faveur et sa fidélité à l’égard de mon maître. Il m’a conduit sur la route jusqu’à la maison des frères de mon maître. »
${}^{28}La jeune fille courut à la maison de sa mère raconter ce qui venait d’arriver. 
${}^{29}Rébecca avait un frère qui s’appelait Laban. Laban sortit et courut vers la source, à la rencontre de l’homme. 
${}^{30}Après avoir vu l’anneau et les bracelets aux poignets de sa sœur, et entendu Rébecca lui dire : « Voilà ce que m’a dit cet homme », il partit à la rencontre de l’homme ; celui-ci se tenait debout près de la source avec les chameaux. 
${}^{31}Il dit : « Viens, béni du Seigneur ! Pourquoi rester dehors ? J’ai fait place dans la maison pour les chameaux. » 
${}^{32}L’homme entra donc dans la maison et déchargea les chameaux. On leur donna de la paille et du fourrage et, pour lui et ses compagnons, de l’eau pour se laver les pieds. 
${}^{33}On lui présenta de quoi manger, mais il déclara : « Non, je ne mangerai pas avant d’avoir dit ce que j’ai à dire. » On lui répondit : « Parle donc. »
${}^{34}Alors, il dit : « Je suis le serviteur d’Abraham. 
${}^{35}Le Seigneur a comblé mon maître de bénédictions et il est devenu riche. Le Seigneur lui a donné petit et gros bétail, argent et or, serviteurs et servantes, chameaux et ânes. 
${}^{36}Sara, la femme de mon maître, alors qu’elle était âgée, lui a donné un fils. Et mon maître a transmis à celui-ci tous ses biens. 
${}^{37}Mon maître, alors, me fit prêter serment. Il m’a dit : “Tu ne prendras pas pour mon fils une épouse parmi les filles des Cananéens dont j’habite le pays. 
${}^{38}Mais jure-moi d’aller à la maison de mon père, dans ma famille, chercher une épouse pour mon fils.” 
${}^{39}Je dis alors à mon maître : “Cette femme ne me suivra peut-être pas !” 
${}^{40}Il me dit : “Le Seigneur devant qui je marche enverra son ange avec toi et fera réussir ton voyage : tu prendras pour mon fils une épouse de ma famille, de la maison de mon père. 
${}^{41}Tu seras dégagé de ton serment, quand tu auras été dans ma famille ; même si on ne te donne pas de femme, tu échapperas à la malédiction.” 
${}^{42}Aujourd’hui, en arrivant près de la source, j’ai dit : “Seigneur, Dieu de mon maître Abraham, daigne faire réussir le voyage que j’ai entrepris. 
${}^{43}Me voici debout près de la source : la jeune fille qui viendra puiser et à qui je dirai : De grâce, donne-moi à boire une gorgée d’eau de ta cruche ! 
${}^{44}et qui me répondra : Bois toi-même, et je puiserai aussi pour tes chameaux !, que cette jeune fille soit celle que le Seigneur destine au fils de mon maître !” 
${}^{45}Je n’avais pas encore fini de parler en moi-même que Rébecca survient, sa cruche sur l’épaule. Elle descend à la source pour y puiser. Je lui dis : “De grâce, donne-moi à boire !” 
${}^{46}Elle s’empresse de descendre la cruche de son épaule et me dit : “Bois ! Je vais aussi abreuver tes chameaux.” J’ai bu, et elle abreuva aussi les chameaux. 
${}^{47}Alors je lui ai demandé : “De qui es-tu la fille ?” et elle m’a répondu : “Je suis la fille de Betouël, le fils que Milka a donné à Nahor.” Alors j’ai mis l’anneau à son nez et les bracelets à ses poignets. 
${}^{48}Puis, je me suis incliné et prosterné devant le Seigneur ; j’ai béni le Seigneur, le Dieu de mon maître Abraham, lui qui m’a conduit par le bon chemin, afin de prendre la fille de son frère\\, pour la donner à son fils Isaac\\. 
${}^{49}Et maintenant, si vous voulez montrer à mon maître faveur et fidélité, dites-le franchement ; si vous refusez, dites-le moi aussi, pour que je sache quelle direction prendre\\. »
${}^{50}Laban prit la parole. Lui et Betouël déclarèrent : « Le Seigneur s’est prononcé, ce n’est pas à nous de décider\\. 
${}^{51}Voici Rébecca devant toi : emmène-la, et qu’elle devienne l’épouse d’Isaac\\le fils de ton maître, comme l’a dit le Seigneur. » 
${}^{52}Quand le serviteur d’Abraham entendit leurs paroles, il se prosterna jusqu’à terre devant le Seigneur. 
${}^{53}Puis il sortit des objets d’argent, des objets d’or, des vêtements et les donna à Rébecca. Il offrit aussi de riches présents à son frère et à sa mère. 
${}^{54}Ils mangèrent et burent, lui et les hommes qui l’accompagnaient, ils passèrent la nuit et, le matin, ils se levèrent.
      Le serviteur dit alors : « Laissez-moi retourner chez mon maître. » 
${}^{55}Le frère et la mère de la jeune fille répondirent : « Qu’elle reste encore avec nous une dizaine de jours ; ensuite, elle partira. » 
${}^{56}Mais le serviteur leur dit : « Ne me retardez pas. Le Seigneur a fait réussir mon voyage. Laissez-moi retourner et j’irai chez mon maître. » 
${}^{57}Ils reprirent : « Appelons la jeune fille et demandons-lui son avis. »
${}^{58}Ils appelèrent Rébecca et lui dirent : « Veux-tu bien partir avec cet homme ? » Elle répondit : « Oui, je partirai. » 
${}^{59} Alors ils laissèrent leur sœur\\Rébecca et sa nourrice s’en aller avec le serviteur d’Abraham et ses hommes.
${}^{60}Ils bénirent Rébecca en lui disant :
        \\« Ô toi, notre sœur, puisses-tu devenir
        \\une multitude sans nombre\\ !
        \\Que ta descendance occupe
        \\les places fortes de ses ennemis ! »
${}^{61}Rébecca et ses servantes se levèrent, montèrent sur les chameaux, et suivirent le serviteur. Celui-ci emmena donc Rébecca.
${}^{62}Isaac s’en revenait du puits de Lahaï-Roï. Il habitait alors le Néguev. 
${}^{63} Il était sorti à la tombée du jour, pour se promener dans la campagne, lorsque, levant les yeux, il vit arriver des chameaux. 
${}^{64} Rébecca, levant les yeux elle aussi, vit Isaac. Elle sauta à bas de son chameau 
${}^{65} et dit au serviteur : « Quel est cet homme qui vient dans la campagne à notre rencontre ? » Le serviteur répondit : « C’est mon maître. » Alors elle prit son voile et s’en couvrit.
${}^{66}Le serviteur raconta à Isaac tout ce qu’il avait fait. 
${}^{67} Isaac introduisit Rébecca dans la tente de sa mère Sara ; il l’épousa, elle devint sa femme, et il l’aima. Et Isaac se consola de la mort de sa mère.
      
         
      \bchapter{}
      \begin{verse}
${}^{1}Abraham prit encore une femme ; elle s’appelait Qetoura. 
${}^{2}Elle lui donna Zimrane, Yoqshane, Medane, Madiane, Yshbaq et Shouah. 
${}^{3}Yoqshane engendra Saba et Dedane. Dedane eut pour fils les Ashourites, les Letoushites et les Léoummites. 
${}^{4}Madiane eut pour fils Eifa, Éfer, Hanok, Abida et Eldaa. Ce sont là tous les fils de Qetoura.
${}^{5}Abraham donna tous ses biens à Isaac. 
${}^{6}Il fit des donations aux fils de ses concubines et, tandis qu’il était encore en vie, il les éloigna d’Isaac, son fils, en les envoyant à l’est, dans un pays d’Orient.
${}^{7}Cent soixante-quinze ans : telle fut la durée de la vie d’Abraham, 
${}^{8}puis il expira. Abraham mourut au terme d’une heureuse vieillesse, très âgé, rassasié de jours ; et il fut réuni aux siens. 
${}^{9}Ses fils, Isaac et Ismaël, l’ensevelirent dans la caverne de Macpéla, dans le champ d’Éphrone, fils de Sohar le Hittite, ce champ qui est en face de Mambré 
${}^{10}et qu’Abraham avait acheté aux Hittites. C’est là qu’Abraham fut enseveli auprès de Sara, sa femme. 
${}^{11}Après la mort d’Abraham, Dieu bénit son fils Isaac qui habitait près du puits de Lahaï-Roï.
${}^{12}Voici la descendance d’Ismaël, le fils d’Abraham que lui donna Agar l’Égyptienne, la servante de Sara. 
${}^{13}Voici les noms des fils d’Ismaël, selon l’ordre de leurs naissances : Nebayoth, le premier-né, puis Qédar, Adbéel, Mibsam, 
${}^{14}Mishma, Douma, Massa, 
${}^{15}Hadad, Téma, Yetour, Nafish et Qédma. 
${}^{16}Voilà les fils d’Ismaël et leurs noms selon leurs villages et leurs campements : douze chefs pour leurs clans. 
${}^{17}Ismaël vécut cent trente-sept ans, puis il expira. Il mourut et fut réuni aux siens. 
${}^{18}Les Ismaélites s’étaient établis de Havila jusqu’à Shour, aux confins de l’Égypte, en direction d’Ashour. Face à tous ses frères, Ismaël se coucha dans la mort.
      <h2 class="intertitle" id="d85e10926">2. Histoire d’Isaac et de ses fils Ésaü et Jacob (25,19 – 36)</h2>
${}^{19}Voici l’histoire d’Isaac, fils d’Abraham. Abraham avait engendré Isaac. 
${}^{20}Isaac avait quarante ans quand il prit pour femme Rébecca, fille de Betouël l’Araméen, originaire de Paddane-Aram, et sœur de l’Araméen Laban. 
${}^{21}Isaac implora le Seigneur en faveur de sa femme, car elle était stérile. Et le Seigneur l’exauça : sa femme Rébecca devint enceinte. 
${}^{22}Comme ses fils se heurtaient dans son sein, elle dit : « Pourquoi faut-il que cela se passe ainsi pour moi ? » et elle alla consulter le Seigneur.
${}^{23}Le Seigneur lui dit :
        \\« Deux nations sont dans ton ventre.
        \\Deux peuples différents sortiront de tes entrailles :
        \\l’un sera plus fort que l’autre,
        \\et l’aîné servira le cadet. »
       
${}^{24}Quand arriva le jour où elle devait enfanter, voici qu’il y avait des jumeaux dans son ventre ! 
${}^{25}Le premier qui sortit était roux, tout couvert de poils comme d’une fourrure. On lui donna le nom d’Ésaü. 
${}^{26}Après quoi sortit son frère, la main agrippée au talon d’Ésaü. On lui donna le nom de Jacob (c’est-à-dire : Il talonne). À leur naissance, Isaac avait soixante ans. 
${}^{27}Les garçons grandirent. Ésaü devint un chasseur habile, un homme des champs ; Jacob était un homme délicat demeurant sous les tentes. 
${}^{28}Isaac préférait Ésaü, car il appréciait le gibier, mais Rébecca préférait Jacob.
${}^{29}Un jour, Jacob préparait un plat, quand Ésaü revint des champs, épuisé. 
${}^{30}Ésaü dit à Jacob : « Laisse-moi donc avaler cette sauce, le roux qui est là, car je suis épuisé ! » C’est pour cela qu’on a donné à Ésaü le nom d’Édom (c’est-à-dire : le Roux). 
${}^{31}Jacob lui dit : « Vends-moi maintenant ton droit d’aînesse ! » 
${}^{32}Ésaü répondit : « Je suis en train de mourir ! À quoi bon mon droit d’aînesse ? » 
${}^{33}Jacob reprit : « Jure-le moi, maintenant ! » Et Ésaü le jura, il vendit son droit d’aînesse à Jacob. 
${}^{34}Alors Jacob donna à Ésaü du pain et un plat de lentilles. Celui-ci mangea et but, puis il se leva et s’en alla. C’est ainsi qu’Ésaü montra du mépris pour le droit d’aînesse.
      
         
      \bchapter{}
      \begin{verse}
${}^{1}Il y eut une famine dans le pays – en plus de la première, celle qui avait eu lieu au temps d’Abraham – et Isaac partit pour Guérar chez Abimélek, roi des Philistins. 
${}^{2}Le Seigneur lui apparut et dit : « Ne descends pas en Égypte, mais demeure dans le pays que je t’indiquerai ; 
${}^{3}séjourne dans ce pays ; je serai avec toi et je te bénirai, car, à toi et à ta descendance, je donnerai tous ces pays. Je tiendrai le serment que j’ai prêté à Abraham, ton père. 
${}^{4}Je rendrai ta descendance aussi nombreuse que les étoiles du ciel et je lui donnerai tous ces pays ; c’est par ta descendance que se béniront toutes les nations de la terre, 
${}^{5}puisque Abraham a écouté ma voix et qu’il a gardé mes observances, mes commandements, mes décrets et mes lois. »
${}^{6}Isaac demeura donc à Guérar. 
${}^{7}Les hommes de cet endroit l’interrogèrent sur sa femme, et il répondit : « C’est ma sœur », car il avait peur de répondre : « C’est ma femme ». Il se disait : « Les hommes de cet endroit pourraient me tuer à cause de Rébecca, elle est si belle à regarder. » 
${}^{8}Il était là depuis longtemps déjà, et voici qu’Abimélek, roi des Philistins, regardant par la fenêtre, le vit caresser Rébecca, sa femme. 
${}^{9}Abimélek convoqua Isaac et lui dit : « À coup sûr, c’est ta femme ! Comment as-tu pu dire : “C’est ma sœur” ? » Isaac lui répondit : « Je l’ai dit car j’avais peur de mourir à cause d’elle. » 
${}^{10}Abimélek reprit : « Que nous as-tu fait là ! Un peu plus, et un homme de ce peuple aurait couché avec ta femme, et à cause de toi nous serions devenus coupables. » 
${}^{11}Abimélek donna cet ordre à tout le peuple : « Quiconque touchera à cet homme ou à sa femme sera mis à mort. »
${}^{12}Isaac fit des semailles sur cette terre et récolta, cette année-là, le centuple. Le Seigneur le bénit, 
${}^{13}et Isaac devint un personnage important, de plus en plus important, jusqu’à devenir vraiment très important. 
${}^{14}Il avait un troupeau de petit bétail, un troupeau de gros bétail et de nombreux serviteurs. Aussi les Philistins en furent-ils jaloux. 
${}^{15}Tous les puits qu’avaient creusés les serviteurs de son père Abraham, au temps de celui-ci, les Philistins les bouchèrent en les remplissant de terre. 
${}^{16}Abimélek dit à Isaac : « Va-t’en de chez nous, car tu es devenu beaucoup plus puissant que nous. » 
${}^{17}Isaac s’en alla donc. Il campa dans la vallée de Guérar et y demeura. 
${}^{18}Isaac se mit à creuser de nouveau les puits d’eau qu’on avait creusés au temps d’Abraham, son père, et que les Philistins avaient bouchés après la mort d’Abraham. Et il leur donna les noms que leur avait donnés son père.
${}^{19}Puis les serviteurs d’Isaac creusèrent dans la vallée et y trouvèrent un puits d’eaux vives. 
${}^{20}Les bergers de Guérar cherchèrent querelle aux bergers d’Isaac, en disant : « Cette eau est à nous ! » Isaac donna donc à ce puits le nom d’Esseq (c’est-à-dire : la Dispute), car les bergers de Guérar s’étaient disputés avec lui. 
${}^{21}Ils creusèrent un autre puits et se querellèrent encore à son sujet. Isaac lui donna donc le nom de Sitna (c’est-à-dire : l’Accusation). 
${}^{22}Il partit de là et creusa un autre puits. Ils ne cherchèrent plus querelle à son sujet. Isaac donna donc à ce puits le nom de Rehoboth (c’est-à-dire : les Largesses) et dit : « Maintenant le Seigneur nous a mis au large. Nous allons prospérer dans le pays. »
${}^{23}De là, il monta à Bershéba. 
${}^{24}Le Seigneur lui apparut, cette nuit-là, et dit :
        \\« Je suis le Dieu d’Abraham, ton père :
        \\ne crains pas, car je suis avec toi ;
        \\je te bénirai et je multiplierai ta descendance
        \\à cause d’Abraham, mon serviteur. »
${}^{25}À cet endroit, Isaac bâtit un autel et invoqua le nom du Seigneur ; là, il planta sa tente et ses serviteurs forèrent un puits. 
${}^{26}Abimélek, accompagné d’Ahouzzath, un de ses proches, et de Pikol, le chef de son armée, sortit de Guérar pour aller rencontrer Isaac. 
${}^{27}Celui-ci leur dit : « Pourquoi êtes-vous venus vers moi, alors que vous me détestez et m’avez renvoyé de chez vous ? » 
${}^{28}Ils répondirent : « Nous avons bien dû constater que le Seigneur est avec toi, et nous avons dit : Qu’un même serment nous unisse, nous et toi, et nous conclurons ensemble une alliance : 
${}^{29}tu ne nous feras pas de mal, de même que nous ne t’avons pas frappé, et que nous t’avons uniquement fait du bien et renvoyé en paix. À présent, tu es le béni du Seigneur. »
${}^{30}Alors il leur fit un festin, ils mangèrent et ils burent. 
${}^{31}Ils se levèrent de bon matin et se prêtèrent serment l’un à l’autre. Puis Isaac les congédia et ils le quittèrent en paix. 
${}^{32}Ce jour-là, les serviteurs d’Isaac vinrent l’informer au sujet du puits qu’ils creusaient. Ils lui dirent : « Nous avons trouvé de l’eau ! » 
${}^{33}Il donna au puits le nom de Shibéa (c’est-à-dire : le Serment), et c’est pourquoi aujourd’hui encore on appelle cette ville Bershéba.
${}^{34}Ésaü avait quarante ans quand il prit pour femmes Judith, fille de Beéri le Hittite, et Basmate, fille d’Élone le Hittite. 
${}^{35}Elles furent un sujet d’amertume pour Isaac et Rébecca.
      
         
      \bchapter{}
      \begin{verse}
${}^{1}Isaac était devenu vieux, ses yeux avaient faibli et il n’y voyait plus. Il appela Ésaü son fils aîné\\ : « Mon fils ! » Celui-ci répondit : « Me voici. » 
${}^{2} Isaac reprit : « Tu vois : je suis devenu vieux, mais je ne sais pas le jour de ma mort. 
${}^{3} Prends donc maintenant tes armes, ton carquois et ton arc, sors dans la campagne et tue-moi du gibier. 
${}^{4} Prépare-moi un bon plat\\comme je les aime et apporte-le-moi pour que je mange, et que je te bénisse avant de mourir. »
${}^{5}Pendant qu’Isaac parlait ainsi à son fils Ésaü, Rébecca écoutait. Ésaü alla donc dans la campagne chasser du gibier pour son père.
${}^{6}Alors Rébecca dit à son fils Jacob : « Voici que j’ai entendu ton père parler à ton frère Ésaü. Il lui disait : 
${}^{7}“Apporte-moi du gibier et prépare-moi un bon plat pour que je mange et que je te bénisse devant le Seigneur avant de mourir.” 
${}^{8}Maintenant, mon fils, écoute bien ce que je t’ordonne. 
${}^{9}Va dans le troupeau de petit bétail et ramène-moi deux beaux chevreaux. Je préparerai pour ton père un bon plat, comme il les aime, 
${}^{10}et tu le lui apporteras à manger ; alors il pourra te bénir avant de mourir. »
${}^{11}Jacob répondit à sa mère Rébecca : « Mais mon frère Ésaü est un homme velu, tandis que ma peau est lisse ! 
${}^{12}Si jamais mon père me palpe, il croira que je me suis moqué de lui et j’attirerai sur moi la malédiction au lieu de la bénédiction. »
${}^{13}Mais sa mère lui répliqua : « Qu’elle vienne sur moi, ta malédiction, mon fils ! Écoute seulement ce que je te dis et va me chercher les chevreaux. » 
${}^{14}Il alla donc les chercher et les apporta à sa mère. Et celle-ci prépara un bon plat comme son père les aimait.
${}^{15}Rébecca prit les meilleurs habits d’Ésaü, son fils aîné, ceux qu’elle gardait à la maison ; elle en revêtit Jacob, son fils cadet. 
${}^{16} Puis, avec des peaux de chevreau, elle lui couvrit les mains\\et la partie lisse du cou. 
${}^{17} Elle remit ensuite entre ses mains\\le plat et le pain qu’elle avait préparés.
${}^{18}Jacob entra chez son père et dit : « Mon père ! » Celui-ci répondit : « Me voici. Qui es-tu, mon fils ? » 
${}^{19} Jacob dit à son père : « Je suis Ésaü, ton premier-né ; j’ai fait ce que tu m’as dit. Viens donc t’asseoir, mange de mon gibier ; alors, tu pourras me bénir. » 
${}^{20} Isaac lui dit\\ : « Comme tu as trouvé vite, mon fils ! » Jacob répondit : « C’est que le Seigneur, ton Dieu, a favorisé ma chasse\\. » 
${}^{21} Isaac lui dit : « Approche donc, mon fils, que je te palpe, pour savoir si tu es bien mon fils Ésaü ! » 
${}^{22} Jacob s’approcha de son père Isaac. Celui-ci le palpa et dit : « La voix est la voix de Jacob, mais les mains sont les mains d’Ésaü. » 
${}^{23} Il ne reconnut pas Jacob car ses mains étaient velues comme celles de son frère Ésaü, et il le bénit. 
${}^{24} Il dit encore : « C’est bien toi mon fils Ésaü ? » Jacob répondit : « C’est bien moi. » 
${}^{25} Isaac reprit : « Apporte-moi le gibier, mon fils, j’en mangerai, et alors je pourrai te bénir. » Jacob le servit, et il mangea. Jacob lui présenta du vin, et il but. 
${}^{26} Son père Isaac dit alors : « Approche-toi et embrasse-moi, mon fils. »
${}^{27}Comme Jacob s’approchait et l’embrassait, Isaac respira l’odeur de ses vêtements, et il le bénit en disant :
        \\« Voici que l’odeur de mon fils
        \\est comme l’odeur d’un champ que le Seigneur a béni.
        ${}^{28}Que Dieu te donne la rosée du ciel et une terre fertile,
        \\froment et vin nouveau en abondance !
        ${}^{29}Que des peuples te servent,
        \\que des nations\\se prosternent devant toi.
        \\Sois un chef pour tes frères,
        \\que les fils de ta mère se prosternent devant toi.
        \\Maudit soit celui qui te maudira,
        \\béni soit celui qui te bénira\\ ! »
${}^{30}À peine Isaac avait-il fini de bénir Jacob, et Jacob avait-il quitté son père, que son frère Ésaü revint de la chasse. 
${}^{31}Lui aussi prépara un bon plat et l’apporta à son père en lui disant : « Que mon père se lève et mange du gibier de son fils ; alors tu pourras me bénir. » 
${}^{32}Isaac lui demanda : « Qui es-tu ? » Il répondit : « Je suis Ésaü, ton fils premier-né. » 
${}^{33}Isaac se mit alors à trembler violemment et dit : « Qui donc est celui qui a été à la chasse et m’a rapporté du gibier ? J’ai mangé de tout avant que tu n’arrives. Celui-là, je l’ai béni ; béni il restera. »
${}^{34}Dès qu’Ésaü entendit les paroles de son père, il poussa un très grand cri, plein d’amertume. Il dit à son père : « Ô mon père, bénis-moi aussi ! » 
${}^{35}Isaac répondit : « Ton frère est venu par ruse et il a volé ta bénédiction ! » 
${}^{36}Ésaü reprit : « Est-ce parce qu’on lui a donné le nom de Jacob (c’est-à-dire : le Trompeur) que, par deux fois, celui-ci m’a trompé ? Il a volé mon droit d’aînesse et voici que, maintenant, il a volé ma bénédiction. Ne m’as-tu pas réservé une bénédiction ? » 
${}^{37}Isaac répondit à Ésaü : « Voici que j’ai fait de lui ton chef, je lui ai donné tous ses frères pour serviteurs, je l’ai pourvu de froment et de vin nouveau : que puis-je encore faire pour toi, mon fils ! » 
${}^{38}Ésaü répondit à son père : « N’as-tu donc qu’une seule bénédiction, mon père ? Ô mon père, bénis-moi aussi ! » Puis Ésaü éleva la voix et pleura.
${}^{39}Alors Isaac reprit la parole et dit :
        \\« Loin des terres fertiles sera ta demeure,
        \\loin de la rosée qui tombe du ciel.
${}^{40}Tu vivras grâce à ton épée
        \\et tu serviras ton frère ;
        \\mais à force de vagabonder,
        \\tu ôteras son joug de ton cou ! »
${}^{41}Ésaü se mit à considérer Jacob comme son ennemi à cause de la bénédiction qu’il avait reçue de son père. Il se disait en lui-même : « Le moment du deuil de mon père approche. Alors je tuerai mon frère Jacob. » 
${}^{42}On rapporta à Rébecca les paroles d’Ésaü, son fils aîné, et elle fit appeler Jacob, son fils cadet. Elle lui dit : « Voici que ton frère Ésaü veut se venger de toi en te tuant. 
${}^{43}Maintenant, mon fils, écoute-moi bien : lève-toi et fuis à Harane chez mon frère Laban. 
${}^{44}Tu habiteras avec lui quelque temps, jusqu’à ce que la fureur de ton frère se détourne de toi, 
${}^{45}oui, que sa colère se détourne de toi et qu’il oublie ce que tu lui as fait ; alors j’enverrai quelqu’un te chercher là-bas. Pourquoi serais-je privée de mes deux enfants le même jour ? »
${}^{46}Rébecca dit à Isaac : « Je suis dégoûtée de la vie à cause des filles de Hittites, les femmes d’Ésaü. Si jamais Jacob devait épouser une fille comme celles-là, une fille de ce pays, à quoi bon vivre encore ! »
      
         
      \bchapter{}
      \begin{verse}
${}^{1}Isaac appela Jacob, le bénit et lui donna cet ordre : « Tu n’épouseras pas une fille de Canaan. 
${}^{2}Lève-toi, va dans la région dePaddane-Aram, à la maison de Betouël, le père de ta mère, et là tu prendras pour femme l’une des filles de Laban, le frère de ta mère. 
${}^{3}Que le Dieu-Puissant te bénisse, qu’il te fasse fructifier et te multiplier, et que tu deviennes ainsi une assemblée de peuples, 
${}^{4}qu’il te donne la bénédiction d’Abraham, à toi et à ta descendance, pour que tu possèdes la terre où tu es venu en immigré, la terre que Dieu a donnée à Abraham ! »
${}^{5}Ainsi, Isaac envoya Jacob et celui-ci partit pour la région dePaddane-Aram, chez Laban, fils de Betouël l’Araméen, frère de Rébecca, la mère de Jacob et d’Ésaü.
${}^{6}Ésaü vit qu’Isaac avait béni Jacob, et l’avait envoyé en Paddane-Aram pour y prendre femme ; en le bénissant, il lui avait donné cet ordre : « Tu n’épouseras pas une fille de Canaan ». 
${}^{7}Jacob avait obéi à son père et à sa mère, il était parti en Paddane-Aram. 
${}^{8}Ésaü comprit alors que les filles de Canaan ne valaient rien aux yeux de son père Isaac ; 
${}^{9}et il alla trouver Ismaël : en plus de ses femmes, il épousa Mahalath, fille d’Ismaël, fils d’Abraham ; Mahalath était la sœur de Nebayoth.
${}^{10}Jacob partit de Bershéba et se dirigea vers Harane. 
${}^{11} Il atteignit le lieu où il allait passer la nuit car le soleil s’était couché. Il y prit une pierre pour la mettre sous sa tête, et dormit\\en ce lieu. 
${}^{12} Il eut un songe : voici qu’une échelle était dressée sur la terre, son sommet touchait le ciel, et\\des anges de Dieu montaient et descendaient.
${}^{13}Le\\Seigneur se tenait près de lui. Il dit : « Je suis le Seigneur, le Dieu d’Abraham ton père, le Dieu d’Isaac. La terre sur laquelle tu es couché, je te la donne, à toi et à tes descendants. 
${}^{14} Tes descendants seront nombreux\\comme la poussière du sol, vous vous répandrez\\à l’orient et à l’occident, au nord et au midi ; en toi et en ta descendance seront bénies toutes les familles\\de la terre. 
${}^{15} Voici que je suis avec toi ; je te garderai partout où tu iras, et je te ramènerai sur cette terre ; car je ne t’abandonnerai pas avant d’avoir accompli ce que je t’ai dit. »
${}^{16}Jacob sortit de son sommeil et déclara : « En vérité, le Seigneur est en ce lieu ! Et moi, je ne le savais pas. » 
${}^{17} Il fut saisi de crainte et il dit : « Que ce lieu est redoutable ! C’est vraiment la maison de Dieu, la porte du ciel ! » 
${}^{18} Jacob se leva de bon matin, il prit la pierre qu’il avait mise sous sa tête, il la dressa pour en faire une stèle, et sur le sommet il versa de l’huile. 
${}^{19} Jacob donna le nom de Béthel (c’est-à-dire : Maison de Dieu)\\à ce lieu qui auparavant s’appelait Louz.
${}^{20}Alors Jacob prononça ce vœu : « Si Dieu est avec moi, s’il me garde sur le chemin où je marche, s’il me donne du pain pour manger et des vêtements pour me couvrir, 
${}^{21}et si je reviens sain et sauf à la maison de mon père, le Seigneur sera mon Dieu. 
${}^{22}Cette pierre dont j’ai fait une stèle sera la maison de Dieu. De tout ce que tu me donneras, je prélèverai la dîme pour toi. »
      
         
      \bchapter{}
      \begin{verse}
${}^{1}Jacob se remit en marche et partit pour le pays des fils de l’Orient. 
${}^{2}Tout à coup, il aperçut un puits dans la campagne et, près de ce puits, trois troupeaux de petit bétail ; les bêtes étaient couchées car c’est à ce puits qu’on abreuvait les troupeaux. Sur l’orifice du puits était posée une grande pierre. 
${}^{3}C’était là que se rassemblaient tous les troupeaux : on roulait la pierre posée sur l’orifice du puits, on abreuvait le petit bétail, puis on remettait la pierre à sa place sur l’orifice du puits.
${}^{4}Jacob dit aux gens : « Mes frères, d’où êtes-vous ? » Ils répondirent : « Nous sommes de Harane. » 
${}^{5}Il leur dit : « Connaissez-vous Laban, le fils de Nahor ? » Ils répondirent : « Nous le connaissons. » 
${}^{6}Il leur demanda : « Va-t-il bien ? » Et ils répondirent : « Il va très bien. Et voici sa fille Rachel qui arrive avec le petit bétail ! » 
${}^{7}Jacob reprit : « Mais il fait encore grand jour. Ce n’est pas le moment de rassembler le bétail : abreuvez donc les bêtes et allez les faire paître ! » 
${}^{8}Ils répliquèrent : « Nous ne pouvons le faire tant que tous les troupeaux ne sont pas rassemblés : alors on roule la pierre posée sur l’orifice du puits et on abreuve le petit bétail. »
${}^{9}Jacob parlait encore avec eux quand Rachel arriva avec le petit bétail qui appartenait à son père ; en effet, elle était bergère.
${}^{10}Dès que Jacob vit Rachel, fille de Laban, le frère de sa mère, et le petit bétail de Laban, il s’avança, roula la pierre posée sur l’orifice du puits et abreuva le petit bétail de Laban. 
${}^{11}Alors Jacob embrassa Rachel, et il éclata en sanglots. 
${}^{12}Jacob apprit à Rachel qu’il était un parent de son père et le fils de Rébecca. Elle courut en informer son père. 
${}^{13}Dès que Laban entendit parler de Jacob, le fils de sa sœur, il courut à sa rencontre, l’étreignit, l’embrassa et l’amena chez lui. Jacob raconta toute l’affaire à Laban 
${}^{14}et celui-ci lui dit : « Tu es vraiment mes os et ma chair ! » Jacob habita chez lui pendant tout un mois.
${}^{15}Laban dit à Jacob : « Devrais-tu me servir gratuitement parce que nous sommes parents ? Indique-moi donc ton salaire. » 
${}^{16}Or Laban avait deux filles : l’aînée s’appelait Léa et la cadette, Rachel. 
${}^{17}Les yeux de Léa étaient délicats, tandis que Rachel avait belle allure et beau visage. 
${}^{18}Et Jacob se mit à aimer Rachel. Il dit : « Je te servirai sept ans pour Rachel, ta fille cadette. » 
${}^{19}Laban répondit : « Je préfère te la donner à toi plutôt qu’à un autre ; reste donc chez moi. »
${}^{20}Jacob travailla sept ans pour Rachel – sept ans qui lui semblèrent quelques jours, tellement il l’aimait. 
${}^{21}Jacob dit alors à Laban : « Donne-moi ma femme car les jours que je te devais sont accomplis et je veux m’unir à elle. »
${}^{22}Laban rassembla tous les gens de l’endroit et fit un festin. 
${}^{23}Le soir venu, il prit sa fille Léa, l’amena à Jacob et Jacob s’unit à elle. 
${}^{24}Laban mit au service de sa fille Léa une de ses servantes, nommée Zilpa. 
${}^{25}Au matin, voilà que c’était Léa et non Rachel ! Et Jacob dit à Laban : « Que m’as-tu fait là ? N’est-ce pas pour Rachel que je t’ai servi ? Pourquoi m’as-tu trompé ? » 
${}^{26}Laban répondit : « Cela ne se fait pas chez nous de marier la cadette avant l’aînée ! 
${}^{27}Achève la semaine de noces de celle-ci et nous te donnerons aussi celle-là pour le service que tu feras encore chez nous pendant sept autres années. » 
${}^{28}Jacob agit ainsi : la semaine achevée, Laban lui donna sa fille Rachel pour qu’elle devienne sa femme. 
${}^{29}Laban mit au service de sa fille Rachel une de ses servantes nommée Bilha. 
${}^{30}Jacob s’unit aussi à Rachel et il aimait Rachel plus que Léa. Il servit donc Laban pendant sept autres années encore.
