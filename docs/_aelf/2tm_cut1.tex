  
  
    
    \bbook{DEUXIÈME LETTRE À TIMOTHÉE}{DEUXIÈME LETTRE À TIMOTHÉE}
      
         
      \bchapter{}
        ${}^{1}Paul, apôtre du Christ Jésus
        par la volonté de Dieu,
        selon la promesse de la vie
        que nous avons dans le Christ Jésus,
        ${}^{2}à Timothée,
        mon enfant bien-aimé.
        \\À toi, la grâce, la miséricorde et la paix
        \\de la part de Dieu le Père
        et du Christ Jésus notre Seigneur.
        
           
${}^{3}Je suis plein de gratitude envers Dieu, à qui je rends un culte avec une conscience pure, à la suite de mes ancêtres, je lui rends grâce en me souvenant continuellement de toi dans mes prières, nuit et jour. 
${}^{4}Me rappelant tes larmes, j’ai un très vif désir de te revoir pour être rempli de joie. 
${}^{5}J’ai souvenir de la foi sincère qui est en toi : c’était celle qui habitait d’abord Loïs, ta grand-mère, et celle d’Eunice, ta mère, et j’ai la conviction que c’est aussi la tienne.
${}^{6}Voilà pourquoi, je te le rappelle, ravive le don gratuit de Dieu, ce don qui est en toi depuis que je t’ai imposé les mains. 
${}^{7}Car ce n’est pas un esprit de peur que Dieu nous a donné, mais un esprit de force, d’amour et de pondération. 
${}^{8}N’aie donc pas honte de rendre témoignage à notre Seigneur, et n’aie pas honte de moi, qui suis son prisonnier ; mais, avec la force de Dieu, prends ta part des souffrances liées à l’annonce de l’Évangile. 
${}^{9}Car Dieu nous a sauvés, il nous a appelés à une vocation sainte, non pas à cause de nos propres actes, mais à cause de son projet à lui et de sa grâce. Cette grâce nous avait été donnée dans le Christ Jésus avant tous les siècles, 
${}^{10}et maintenant elle est devenue visible, car notre Sauveur, le Christ Jésus, s’est manifesté : il a détruit la mort, et il a fait resplendir la vie et l’immortalité par l’annonce de l’Évangile, 
${}^{11}pour lequel j’ai reçu la charge de messager, d’apôtre et d’enseignant. 
${}^{12}Et c’est pour cette raison que je souffre ainsi ; mais je n’en ai pas honte, car je sais en qui j’ai cru, et j’ai la conviction qu’il est assez puissant pour sauvegarder, jusqu’au jour de sa venue, le dépôt de la foi qu’il m’a confié. 
${}^{13}Tiens-toi au modèle donné par les paroles solides que tu m’as entendu prononcer dans la foi et dans l’amour qui est dans le Christ Jésus. 
${}^{14}Garde le dépôt de la foi dans toute sa beauté, avec l’aide de l’Esprit Saint qui habite en nous.
${}^{15}Tu sais bien que tous ceux de la province d’Asie se sont détournés de moi, et entre autres Phygèlos et Hermogène. 
${}^{16}Que le Seigneur fasse miséricorde à la famille d’Onésiphore qui m’a plusieurs fois rendu courage et qui n’a pas eu honte de mes chaînes de prisonnier. 
${}^{17}Arrivé à Rome, il s’est empressé de me chercher, et il m’a trouvé. 
${}^{18}Que le Seigneur lui donne de trouver miséricorde auprès de Dieu au jour de sa venue ! Et tous les services qu’il a rendus à Éphèse, tu les connais mieux que personne.
      
         
      \bchapter{}
      \begin{verse}
${}^{1}Toi donc, mon enfant, trouve ta force dans la grâce qui est en Jésus Christ. 
${}^{2}Ce que tu m’as entendu dire en présence de nombreux témoins, confie-le à des hommes dignes de foi qui seront capables de l’enseigner aux autres, à leur tour. 
${}^{3}Prends ta part de souffrance comme un bon soldat du Christ Jésus. 
${}^{4}Celui qui est dans l’armée ne s’embarrasse pas des affaires de la vie ordinaire, il cherche à satisfaire celui qui l’a enrôlé. 
${}^{5}De même, dans une compétition sportive, on ne reçoit la couronne de laurier que si l’on a observé les règles de la compétition. 
${}^{6}Le cultivateur qui se donne de la peine doit être le premier à recevoir une part de la récolte. 
${}^{7}Réfléchis à ce que je dis, car le Seigneur te donnera de tout comprendre.
${}^{8}Souviens-toi de Jésus Christ, ressuscité d’entre les morts, le descendant de David : voilà mon évangile. 
${}^{9}C’est pour lui que j’endure la souffrance, jusqu’à être enchaîné comme un malfaiteur. Mais on n’enchaîne pas la parole de Dieu ! 
${}^{10}C’est pourquoi je supporte tout pour ceux que Dieu a choisis, afin qu’ils obtiennent, eux aussi, le salut qui est dans le Christ Jésus, avec la gloire éternelle. 
${}^{11}Voici une parole digne de foi :
        \\Si nous sommes morts avec lui,
        \\avec lui nous vivrons.
        ${}^{12}Si nous supportons l’épreuve,
        \\avec lui nous régnerons.
        \\Si nous le rejetons,
        \\lui aussi nous rejettera.
        ${}^{13}Si nous manquons de foi,
        \\lui reste fidèle à sa parole,
        \\car il ne peut se rejeter lui-même.
${}^{14}Voilà ce que tu dois rappeler, en déclarant solennellement devant Dieu qu’il faut bannir les querelles de mots : elles ne servent à rien, sinon à perturber ceux qui les écoutent.
${}^{15}Toi-même, efforce-toi de te présenter devant Dieu comme quelqu’un qui a fait ses preuves, un ouvrier qui n’a pas à rougir de ce qu’il a fait et qui trace tout droit le chemin de la parole de vérité. 
${}^{16}Quant aux bavardages impies, évite-les ; leurs auteurs progressent sans cesse en impiété 
${}^{17}et leur parole se propage comme la gangrène. Tels sont Hyménaios et Philètos, 
${}^{18}qui se sont écartés de la vérité en prétendant que la résurrection est déjà arrivée, et ils bouleversent la foi de quelques-uns. 
${}^{19}Cependant le fondement solide posé par Dieu tient bon ; il est marqué du sceau de ces paroles : Le Seigneur connaît les siens, et aussi : Qu’il se détourne de l’iniquité, celui qui prononce le nom du Seigneur. 
${}^{20}Dans une grande maison, il n’y a pas seulement des instruments d’or et d’argent, mais il y en a aussi en bois et en terre cuite, les premiers pour ce qui est honorable, et les autres pour ce qui ne l’est pas. 
${}^{21}Si donc quelqu’un se purifie des travers dont j’ai parlé, il sera un instrument pour ce qui est honorable, sanctifié, utile au Maître, prêt à faire tout ce qui est bien.
${}^{22}Fuis les passions de la jeunesse. Cherche à vivre dans la justice, la foi, la charité et la paix, avec ceux qui invoquent le Seigneur d’un cœur pur. 
${}^{23}Évite les discussions folles et simplistes : tu sais qu’elles provoquent des querelles. 
${}^{24}Or un serviteur du Seigneur ne doit pas être querelleur ; il doit être attentionné envers tous, capable d’enseigner et de supporter la malveillance ; 
${}^{25}il doit reprendre avec douceur les opposants, car Dieu leur donnera peut-être de se convertir, de connaître pleinement la vérité : 
${}^{26}ils retrouveront alors leur bon sens et se dégageront des pièges du diable qui les retient captifs, soumis à sa volonté.
      
         
      \bchapter{}
      \begin{verse}
${}^{1}Sache-le bien : dans les derniers jours surviendront des moments difficiles. 
${}^{2}En effet, les gens seront égoïstes, cupides, fanfarons, orgueilleux, blasphémateurs, révoltés contre leurs parents, ingrats, sacrilèges, 
${}^{3}sans cœur, implacables, médisants, incapables de se maîtriser, intraitables, ennemis du bien, 
${}^{4}traîtres, emportés, aveuglés par l’orgueil, amis du plaisir plutôt que de Dieu ; 
${}^{5}ils auront des apparences de piété, mais rejetteront ce qui fait sa force. Détourne-toi aussi de ces gens-là ! 
${}^{6}Parmi eux, il y en a qui s’introduisent dans les maisons et captivent des bonnes femmes chargées de péchés, entraînées par toutes sortes de convoitises, 
${}^{7}toujours en train d’apprendre et jamais capables de parvenir à la pleine connaissance de la vérité. 
${}^{8}De la même façon que Jannès et Jambrès se sont opposés à Moïse, ceux-là aussi s’opposent à la vérité ; ces gens ont un esprit corrompu et une foi sans valeur. 
${}^{9}Cependant ils n’iront pas bien loin, car leur stupidité sera évidente pour tous, comme le fut celle des deux autres.
      
         
${}^{10}Mais toi, tu m’as suivi pas à pas dans l’enseignement, la manière de diriger et les projets, dans la foi, la patience, la charité et la persévérance, 
${}^{11}dans les persécutions et les souffrances, celles qui me sont arrivées à Antioche, à Iconium et à Lystres, toutes les persécutions que j’ai subies. Et de tout cela le Seigneur m’a délivré. 
${}^{12}D’ailleurs, tous ceux qui veulent vivre en hommes religieux dans le Christ Jésus subiront la persécution. 
${}^{13}Quant aux hommes mauvais et aux charlatans, ils iront toujours plus loin dans le mal, ils seront à la fois trompeurs et trompés. 
${}^{14}Mais toi, demeure ferme dans ce que tu as appris : de cela tu as acquis la certitude, sachant bien de qui tu l’as appris. 
${}^{15}Depuis ton plus jeune âge, tu connais les Saintes Écritures : elles ont le pouvoir de te communiquer la sagesse, en vue du salut par la foi que nous avons en Jésus Christ. 
${}^{16}Toute l’Écriture est inspirée par Dieu ; elle est utile pour enseigner, dénoncer le mal, redresser, éduquer dans la justice ; 
${}^{17}grâce à elle, l’homme de Dieu sera accompli, équipé pour faire toute sorte de bien.
      
         
      \bchapter{}
      \begin{verse}
${}^{1}Devant Dieu, et devant le Christ Jésus qui va juger les vivants et les morts, je t’en conjure, au nom de sa Manifestation et de son Règne : 
${}^{2}proclame la Parole, interviens à temps et à contretemps, dénonce le mal, fais des reproches, encourage, toujours avec patience et souci d’instruire. 
${}^{3}Un temps viendra où les gens ne supporteront plus l’enseignement de la saine doctrine ; mais, au gré de leurs caprices, ils iront se chercher une foule de maîtres pour calmer leur démangeaison d’entendre du nouveau. 
${}^{4}Ils refuseront d’entendre la vérité pour se tourner vers des récits mythologiques. 
${}^{5}Mais toi, en toute chose garde la mesure, supporte la souffrance, fais ton travail d’évangélisateur, accomplis jusqu’au bout ton ministère.
${}^{6}Moi, en effet, je suis déjà offert en sacrifice, le moment de mon départ est venu. 
${}^{7}J’ai mené le bon combat, j’ai achevé ma course, j’ai gardé la foi. 
${}^{8}Je n’ai plus qu’à recevoir la couronne de la justice : le Seigneur, le juste juge, me la remettra en ce jour-là, et non seulement à moi, mais aussi à tous ceux qui auront désiré avec amour sa Manifestation glorieuse.
${}^{9}Efforce-toi de me rejoindre au plus vite, 
${}^{10}car Démas m’a abandonné par amour de ce monde, et il est parti pour Thessalonique. Crescent est parti pour la Galatie, et Tite pour la Dalmatie. 
${}^{11}Luc est seul avec moi. Amène Marc avec toi, il m’est très utile pour le ministère. 
${}^{12}J’ai envoyé Tychique à Éphèse. 
${}^{13}En venant, rapporte-moi le manteau que j’ai laissé à Troas chez Carpos. Apporte-moi aussi mes livres, surtout les parchemins. 
${}^{14}Alexandre, le forgeron, m’a fait beaucoup de mal. Le Seigneur lui rendra selon ses œuvres. 
${}^{15}Toi aussi, prends garde à cet individu, car il s’est violemment opposé à nos paroles.
${}^{16}La première fois que j’ai présenté ma défense, personne ne m’a soutenu : tous m’ont abandonné. Que cela ne soit pas retenu contre eux. 
${}^{17}Le Seigneur, lui, m’a assisté. Il m’a rempli de force pour que, par moi, la proclamation de l’Évangile s’accomplisse jusqu’au bout et que toutes les nations l’entendent. J’ai été arraché à la gueule du lion ; 
${}^{18}le Seigneur m’arrachera encore à tout ce qu’on fait pour me nuire. Il me sauvera et me fera entrer dans son Royaume céleste. À lui la gloire pour les siècles des siècles. Amen.
${}^{19}Salue Prisca et Aquilas, ainsi que ceux de la maison d’Onésiphore. 
${}^{20}Éraste est à Corinthe. J’ai laissé Trophime à Milet ; il était malade. 
${}^{21}Efforce-toi de venir avant l’hiver. Eubule et Pudens te saluent, ainsi que Lin, Claudia et tous les frères.
${}^{22}Que le Seigneur soit avec ton esprit. Que la grâce soit avec vous.
