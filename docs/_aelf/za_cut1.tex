  
  
    
    \bbook{ZACHARIE}{ZACHARIE}
      
         
      \bchapter{}
      \begin{verse}
${}^{1}La deuxième année du règne de Darius, au huitième mois, la parole du Seigneur fut adressée au prophète Zacharie, fils de Bèrèkya, fils de Iddo, pour qu’il dise : 
${}^{2}Le Seigneur s’est profondément irrité contre vos pères. 
${}^{3}Mais tu leur diras : Ainsi parle le Seigneur de l’univers : Revenez à moi – oracle du Seigneur de l’univers –, et je reviendrai à vous, dit le Seigneur de l’univers. 
${}^{4}Ne soyez pas comme vos pères que les prophètes de jadis ont interpellés en disant : Ainsi parle le Seigneur de l’univers : Revenez donc de vos mauvais chemins, de vos actions mauvaises. Mais ils ne m’ont pas écouté, ils n’ont pas prêté attention – oracle du Seigneur. 
${}^{5}Vos pères, où sont-ils ? Et les prophètes, vivent-ils pour toujours ? 
${}^{6}Cependant, mes paroles et mes décrets, que j’avais confiés à mes serviteurs les prophètes, n’ont-ils pas atteint vos pères ? Alors ils sont revenus et ils ont dit : « De la manière dont le Seigneur de l’univers avait résolu de nous traiter selon nos chemins et nos actions, ainsi nous a-t-il traités. »
      
         
${}^{7}La deuxième année du règne de Darius, le vingt-quatrième jour du onzième mois, le mois de Shebath, la parole du Seigneur fut adressée au prophète Zacharie, fils de Bèrèkya, fils de Iddo, pour qu’il dise :
${}^{8}J’ai eu, pendant la nuit, une vision : voici qu’un homme monté sur un cheval roux se tenait entre les myrtes de l’abîme et, derrière lui, il y avait des chevaux roux, bruns et blancs. 
${}^{9}Je dis : « Ceux-ci, que sont-ils, mon Seigneur ? » L’ange qui me parlait répondit : « Moi, je te ferai voir ce qu’ils sont. » 
${}^{10}L’homme qui se tenait entre les myrtes intervint et dit : « Ceux-là, le Seigneur les a envoyés parcourir la terre. » 
${}^{11}Ils s’adressèrent à l’ange du Seigneur qui se tenait entre les myrtes, et ils dirent : « Nous venons de parcourir la terre, et voici que toute la terre est tranquille. »
${}^{12}L’ange du Seigneur reprit alors la parole et dit : « Seigneur de l’univers, combien de temps refuseras-tu d’avoir compassion de Jérusalem et des villes de Juda auxquelles tu fais sentir ta colère depuis soixante-dix ans ? » 
${}^{13}À l’ange qui me parlait, le Seigneur répondit des paroles de bonté, des paroles de consolation.
${}^{14}L’ange qui me parlait me dit alors : Fais cette annonce : Ainsi parle le Seigneur de l’univers :
       
        \\Pour Jérusalem j’éprouve un amour jaloux,
        \\pour Sion, une grande jalousie,
${}^{15}et je suis très profondément irrité
        \\contre les nations qui vivent bien tranquilles :
        \\alors que moi, j’étais faiblement irrité contre Jérusalem,
        \\elles, elles ont ajouté à son malheur.
${}^{16}C’est pourquoi, ainsi parle le Seigneur :
        \\De nouveau je me tourne vers Jérusalem avec compassion ;
        \\ma Maison y sera rebâtie
        \\– oracle du Seigneur de l’univers –,
        \\et le cordeau à mesurer sera tendu sur Jérusalem.
         
${}^{17}Annonce encore : Ainsi parle le Seigneur de l’univers :
        \\Mes villes regorgeront encore de biens.
        \\Le Seigneur va encore consoler Sion,
        \\encore choisir Jérusalem.
      
         
      \bchapter{}
      \begin{verse}
${}^{1}Je levai les yeux et voici ce que j’ai vu : c’était quatre cornes. 
${}^{2}Je dis à l’ange qui me parlait : « Ces cornes, que sont-elles ? » Il me répondit : « Ce sont les cornes qui ont dispersé Juda, Israël et Jérusalem. »
      
         
       
${}^{3}Puis le Seigneur me fit voir quatre forgerons. 
${}^{4}Et je dis : « Ceux-ci, que viennent-ils faire ? » Il me répondit : « Les cornes ont dispersé Juda, au point que personne n’osait plus relever la tête ; mais ceux-ci sont venus pour effrayer et faire trembler les cornes des nations qui se sont levées contre le pays de Juda afin de le disperser. »
${}^{5}Je levai les yeux et voici ce que j’ai vu : un homme qui tenait à la main une chaîne d’arpenteur. 
${}^{6} Je lui demandai : « Où vas-tu ? » Il me répondit : « Je vais mesurer Jérusalem, pour voir quelle est sa largeur et quelle est sa longueur. » 
${}^{7} L’ange qui me parlait était en train de sortir, lorsqu’un autre ange sortit le rejoindre 
${}^{8} et lui dit :
        \\Cours, et dis à ce jeune homme :
        \\Jérusalem doit rester une ville ouverte\\,
        \\à cause de la quantité d’hommes et de bétail
        \\qui la peupleront
        ${}^{9}Quant à moi, je serai pour elle
        \\– oracle du Seigneur –
        \\une muraille de feu qui l’entoure,
        \\et je serai sa gloire au milieu d’elle.
         
${}^{10}Allez ! Allez ! Quittez en hâte le pays du nord
        \\– oracle du Seigneur !
        \\Aux quatre vents des cieux je vous avais dispersés
        \\– oracle du Seigneur !
${}^{11}Allez ! Sion, sauve-toi,
        \\toi qui es installée à Babylone.
${}^{12}Car ainsi parle le Seigneur de l’univers,
        \\lui dont la Gloire m’a envoyé aux nations
        \\qui vous dépouillèrent :
        \\Celui qui vous touche, touche à la prunelle de mon œil.
${}^{13}Oui, me voici, sur elles je vais lever la main :
        \\elles deviendront le butin de leurs esclaves.
        \\Alors vous saurez que le Seigneur de l’univers m’a envoyé !
         
        ${}^{14}Chante et réjouis-toi, fille de Sion ;
        \\voici que je viens, j’habiterai au milieu de toi
        \\– oracle du Seigneur\\.
        ${}^{15}Ce jour-là, des nations nombreuses
        \\s’attacheront au Seigneur ;
        \\elles seront pour moi un peuple,
        \\et j’habiterai au milieu de toi.
        \\Alors tu sauras que le Seigneur de l’univers
        \\m’a envoyé vers toi.
         
        ${}^{16}Le Seigneur prendra possession de Juda,
        \\son domaine sur la terre sainte ;
        \\il choisira de nouveau Jérusalem.
        ${}^{17}Que tout être de chair fasse silence devant le Seigneur,
        \\car il se réveille et sort de sa Demeure sainte.
      
         
      \bchapter{}
      \begin{verse}
${}^{1}Le Seigneur me fit voir Josué, le grand prêtre, qui se tenait devant l’ange du Seigneur, tandis que le Satan était debout à sa droite pour l’accuser. 
${}^{2}Le Seigneur dit au Satan : « Que le Seigneur te réprime, Satan ; que le Seigneur te réprime, lui qui a fait choix de Jérusalem. Josué n’est-il pas un tison tiré du feu ? » 
${}^{3}Or Josué, debout devant l’ange, était vêtu d’habits sordides. 
${}^{4}Le Seigneur reprit et dit à ceux qui se tenaient devant lui : « Enlevez-lui ses habits sordides. » Puis il dit à Josué : « Vois, je passe sur ta faute et je te revêts de parures. » 
${}^{5}Il reprit : « Mettez sur sa tête un turban immaculé. » Ils mirent sur sa tête un turban immaculé, ils le revêtirent d’habits, et l’ange du Seigneur se tenait là.
${}^{6}L’ange du Seigneur donna cet avertissement à Josué :
${}^{7}Ainsi parle le Seigneur de l’univers :
        \\Si tu marches dans mes voies,
        \\si tu gardes mes observances,
        \\tu gouverneras ma maison,
        \\tu garderas mes parvis
        \\et je te ferai accéder
        \\au rang de ceux qui se tiennent là.
${}^{8}Écoute donc, Josué, grand prêtre,
        \\toi et tes compagnons qui siègent devant toi,
        \\écoute, car ces hommes sont un signe :
        \\voici que je fais venir mon serviteur le « Germe » ;
${}^{9}voici la pierre que je dépose devant Josué :
        \\sur cette seule pierre il y a des yeux au nombre de sept ;
        \\voici que je grave moi-même son inscription
        \\– oracle du Seigneur de l’univers – 
        \\et j’ôterai la faute de ce pays, en un seul jour.
${}^{10}Ce jour-là – oracle du Seigneur de l’univers –,
        \\vous vous inviterez l’un l’autre
        \\sous la vigne et sous le figuier.
      
         
      \bchapter{}
      \begin{verse}
${}^{1}L’ange qui me parlait revint et me réveilla comme on réveille un homme de son sommeil. 
${}^{2}Il me dit : « Que vois-tu ? » Je répondis : « Je vois un chandelier tout en or, avec un vase à son sommet, surmonté de sept lampes et de sept canaux pour ces lampes ; 
${}^{3}sur lui, il y a deux oliviers, l’un à la droite du vase et l’autre à sa gauche. »
${}^{4}Prenant la parole, je dis à l’ange qui me parlait : « Qu’est-ce que cela, mon Seigneur ? » 
${}^{5}L’ange qui me parlait me répondit : « Ne le sais-tu pas ? » Je dis : « Non, mon Seigneur. »
       
${}^{6}Alors il reprit et me dit :
        \\Voici la parole que le Seigneur adresse à Zorobabel :
        \\« Ni par la bravoure ni par la force,
        \\mais par mon Esprit seulement ! »
        \\– déclare le Seigneur de l’univers.
${}^{7}Qui es-tu, grande montagne ?
        \\Devant Zorobabel, te voici une plaine !
        \\Il en extrait la première pierre,
        \\parmi les acclamations : La grâce, la grâce sur elle !
         
${}^{8}La parole du Seigneur me fut adressée :
${}^{9}Les mains de Zorobabel ont fondé cette Maison ;
        \\ses mains l’achèveront.
        \\Alors vous saurez que le Seigneur de l’univers
        \\m’a envoyé vers vous !
${}^{10}Qui donc méprisait le jour des modestes commencements ? Qu’on se réjouisse plutôt en voyant le fil à plomb dans la main de Zorobabel ! Quant aux sept lampes, ce sont les yeux du Seigneur, eux qui parcourent toute la terre.
       
${}^{11}Je pris encore la parole et je lui dis : « Que sont ces deux oliviers, sur la droite du chandelier et sur sa gauche ? » Une seconde fois je lui demandai : 
${}^{12}« Que sont donc ces deux branches d’olivier qui, par deux conduits en or, font couler de l’or ? » 
${}^{13}Il me répondit : « Ne le sais-tu pas ? » Je dis : « Non, mon Seigneur. » 
${}^{14}Alors il me dit : « Ce sont les deux hommes qui ont reçu l’onction et qui se tiennent devant le Maître de toute la terre. »
      
         
      \bchapter{}
      \begin{verse}
${}^{1}De nouveau, je levai les yeux et voici ce que j’ai vu : c’était un livre qui volait. 
${}^{2}L’ange qui me parlait me dit : « Que vois-tu ? » Je répondis : « Je vois un livre qui vole, long de vingt coudées et large de dix. » 
${}^{3}Alors il me dit : « Ceci est la Malédiction qui s’élance sur tout le pays, car tout voleur reste impuni, et impunis, tous les parjures. »
${}^{4}Je la lancerai – oracle du Seigneur de l’univers –, elle entrera dans la maison du voleur et dans la maison de celui qui jure faussement par mon nom ; elle viendra loger au cœur de leurs maisons et les consumera, poutres et pierres.
${}^{5}L’ange qui me parlait s’avança et me dit : « Lève les yeux et regarde ce qui s’avance. » 
${}^{6}Et je dis : « Qu’est-ce que c’est ? » Il me répondit : « C’est un boisseau qui s’avance. » Il ajouta : « C’est leur injustice dans tout le pays. »
       
${}^{7}Or voici qu’un couvercle de plomb se souleva : une femme était assise à l’intérieur du boisseau. 
${}^{8}Il dit alors : « C’est la méchanceté. » Il la repoussa à l’intérieur du boisseau et jeta une masse de plomb sur l’orifice.
       
${}^{9}Je levai les yeux et voici ce que j’ai vu : deux femmes qui s’avançaient. Le vent soufflait dans leurs ailes, des ailes pareilles à celles d’une cigogne ; elles soulevèrent le boisseau entre terre et ciel. 
${}^{10}Alors je dis à l’ange qui me parlait : « Où emportent-elles le boisseau ? » 
${}^{11}Il me répondit : « Au pays de Babylone, pour lui construire une maison. Ensuite elles lui prépareront un socle et le fixeront là. »
      
         
      \bchapter{}
      \begin{verse}
${}^{1}De nouveau, je levai les yeux et voici ce que j’ai vu : quatre chars qui s’élançaient d’entre les deux montagnes ; et ces montagnes étaient de bronze. 
${}^{2}Le premier char avait des chevaux roux ; le deuxième char, des chevaux noirs ; 
${}^{3}le troisième char, des chevaux blancs, et le quatrième char, des chevaux tachetés, vigoureux.
${}^{4}Je demandai à l’ange qui me parlait : « Ceux-ci, que sont-ils, mon Seigneur ? » 
${}^{5}L’ange me répondit : « Ce sont les quatre vents du ciel qui s’élancent après s’être tenus devant le Maître de toute la terre. » 
${}^{6}Les chevaux noirs s’élançaient vers la terre du nord ; les blancs s’élançaient à leur suite, et les chevaux tachetés s’élançaient vers la terre du midi. 
${}^{7}Vigoureux, ils s’élançaient, impatients de parcourir la terre. Alors le Seigneur leur ordonna : « Allez parcourir la terre. » Et ils parcoururent la terre. 
${}^{8}Il m’appela et me dit : « Vois, ceux qui s’élancent vers la terre du nord font descendre mon Souffle sur la terre du nord. »
${}^{9}La parole du Seigneur me fut adressée : 
${}^{10}Fais une collecte auprès des déportés, auprès de Heldaï, auprès de Tobie et auprès de Yedaya. Toi-même, tu viendras, ce jour-là, tu viendras à la maison de Josias, fils de Sophonie, où ils sont arrivés de Babylone. 
${}^{11}Tu prendras l’argent et l’or pour en faire des couronnes ; tu les mettras sur la tête de Josué, fils de Josédeq, le grand prêtre. 
${}^{12}Puis tu lui parleras pour dire ceci :
        \\Ainsi parle le Seigneur de l’univers :
        \\Voici un homme dont le nom est « Germe » ;
        \\de lui, quelque chose va germer,
        \\car il reconstruira le temple du Seigneur.
${}^{13}C’est lui qui reconstruira le temple du Seigneur,
        \\c’est lui qui sera vêtu de majesté.
        \\Il siégera sur son trône, il aura le pouvoir ;
        \\un prêtre aussi siégera sur son trône :
        \\il y aura une volonté de paix entre eux deux.
       
${}^{14}Quant aux couronnes, elles deviendront, pour Hélem, Tobie, Yedaya, et pour Hèn, fils de Sophonie, un mémorial dans le temple du Seigneur.
${}^{15}Alors ceux qui sont au loin viendront reconstruire le temple du Seigneur, et vous saurez que le Seigneur de l’univers m’a envoyé vers vous. Cela arrivera si vous écoutez vraiment la voix du Seigneur votre Dieu.
      
         
      \bchapter{}
      \begin{verse}
${}^{1}La quatrième année du roi Darius, le quatrième jour du neuvième mois, le mois de Kisléou, la parole du Seigneur fut adressée à Zacharie. 
${}^{2}Béthel envoya Sarècer et Règuem-Mélek avec ses gens pour apaiser la face du Seigneur 
${}^{3}et demander aux prêtres attachés à la maison du Seigneur de l’univers, ainsi qu’aux prophètes : « Dois-je pleurer au cinquième mois, en m’imposant des privations comme je le fais depuis tant d’années ? »
${}^{4}Alors, la parole du Seigneur de l’univers me fut adressée :
${}^{5}Dis à tous les gens du pays et aux prêtres :
        \\Quand vous avez jeûné en vous lamentant
        \\aux cinquième et septième mois,
        \\et cela depuis soixante-dix ans,
        \\était-ce bien mon jeûne que vous pratiquiez ?
${}^{6}Et quand vous mangez, quand vous buvez,
        \\n’est-ce pas vous qui mangez, vous qui buvez ?
${}^{7}Ces paroles ne sont-elles pas ce que le Seigneur avait proclamé
        \\par les prophètes de jadis,
        \\quand Jérusalem était habitée et tranquille,
        \\avec ses villes d’alentour,
        \\quand le Néguev était habité, ainsi que le Bas-Pays ?
       
${}^{8}La parole du Seigneur fut adressée à Zacharie :
${}^{9}Ainsi parlait le Seigneur de l’univers. Il disait : Rendez une justice vraie ; que chacun ait pour son frère amour et tendresse. 
${}^{10}La veuve et l’orphelin, l’étranger et le pauvre, ne les opprimez pas ; que personne de vous ne médite en son cœur du mal contre son frère.
${}^{11}Mais ils ont refusé de prêter attention : ils ont présenté des nuques rebelles ; ils se sont bouché les oreilles pour ne pas entendre ; 
${}^{12}ils ont rendu leur cœur aussi dur qu’un diamant, pour ne pas écouter l’instruction et les paroles que le Seigneur de l’univers leur avait adressées en son Esprit, par les prophètes de jadis. Grande fut alors l’irritation du Seigneur de l’univers.
${}^{13}Tout comme j’ai lancé des appels qu’ils n’ont pas écoutés, ainsi lanceront-ils des appels que je n’écouterai pas, dit le Seigneur de l’univers. 
${}^{14}Je les ai dispersés parmi toutes sortes de nations qu’ils ne connaissaient pas. Le pays fut dévasté derrière eux : plus personne pour aller et venir. Ainsi, d’une terre de délices ont-ils fait une désolation !
      
         
      \bchapter{}
        ${}^{1}Parole du Seigneur de l’univers :
        ${}^{2}Ainsi parle le Seigneur de l’univers :
        \\J’éprouve pour Sion un amour jaloux,
        \\j’ai pour elle une ardeur passionnée.
        ${}^{3}Ainsi parle le Seigneur :
        \\Je suis revenu vers Sion,
        \\et je fixerai ma demeure au milieu de Jérusalem.
        \\Jérusalem s’appellera : « Ville de la loyauté\\ »,
        \\et la montagne du Seigneur de l’univers :
        \\« Montagne sainte ».
        
           
         
        ${}^{4}Ainsi parle le Seigneur de l’univers :
        \\Les vieux et les vieilles
        \\reviendront s’asseoir sur les places de Jérusalem,
        \\le bâton à la main, à cause de leur grand âge ;
        ${}^{5}les places de la ville
        \\seront pleines de petits garçons et de petites filles
        \\qui viendront y jouer.
        
           
         
        ${}^{6}Ainsi parle le Seigneur de l’univers :
        \\Si tout cela paraît une merveille
        \\aux yeux des survivants de ce temps-là,
        \\ce sera aussi une merveille à mes yeux
        \\– oracle du Seigneur de l’univers.
        
           
         
        ${}^{7}Ainsi parle le Seigneur de l’univers :
        \\Voici que je sauve mon peuple,
        \\en le ramenant du pays de l’orient et du pays de l’occident.
        ${}^{8}Je les ferai venir
        \\pour qu’ils demeurent au milieu de Jérusalem.
        \\Ils seront mon peuple,
        \\et moi, je serai leur Dieu,
        \\dans la loyauté et dans la justice.
        
           
         
${}^{9}Ainsi parle le Seigneur de l’univers :
        \\Que vos mains se fortifient,
        \\vous qui entendez en ces jours
        \\ces paroles sorties de la bouche des prophètes
        \\le jour où furent posées les fondations de la maison du Seigneur
        \\pour rebâtir son Temple.
${}^{10}Car, avant ces jours-ci,
        \\il n’y avait pas de profit pour les hommes,
        \\pas de profit pour les bêtes.
        \\On ne pouvait aller et venir en paix,
        \\à cause de l’adversaire :
        \\j’avais dressé tous les hommes les uns contre les autres.
${}^{11}Mais maintenant, moi, je n’agis plus comme aux jours passés
        \\pour le reste de ce peuple
        \\– oracle du Seigneur de l’univers.
${}^{12}Oui, il y aura une semence de paix :
        \\la vigne donnera son fruit,
        \\la terre donnera son produit,
        \\le ciel donnera sa rosée.
        \\J’accorderai tout cela en partage
        \\au reste de ce peuple.
${}^{13}Comme vous étiez une malédiction parmi les nations,
        \\maison de Juda et maison d’Israël,
        \\ainsi je vous sauverai et vous serez une bénédiction.
        \\Ne craignez pas, et que vos mains se fortifient !
        
           
         
${}^{14}Car, ainsi parle le Seigneur de l’univers :
        \\Comme j’avais résolu de vous faire du mal
        \\parce que vos pères m’avaient irrité,
        \\et je n’ai pas fléchi
        \\– déclare le Seigneur de l’univers –,
${}^{15}ainsi, en ces jours, je me suis ravisé
        \\et j’ai résolu de faire du bien
        \\à Jérusalem et à la maison de Juda.
        \\Ne craignez pas !
${}^{16}Voici les paroles que vous mettrez en pratique :
        \\chacun dira la vérité à son prochain ;
        \\au tribunal vous rendrez des jugements de paix dans la vérité.
${}^{17}Ne méditez pas en votre cœur du mal contre votre prochain,
        \\n’aimez pas le faux serment,
        \\car tout cela, je le hais
        \\– oracle du Seigneur.
        
           
         
${}^{18}La parole du Seigneur de l’univers me fut adressée :
${}^{19}Ainsi parle le Seigneur de l’univers :
        \\Le jeûne du quatrième mois,
        \\le jeûne du cinquième, du septième et du dixième mois
        \\deviendront pour la maison de Juda
        \\allégresse, réjouissance et belles fêtes.
        \\Aimez la vérité et la paix !
        
           
         
        ${}^{20}Ainsi parle le Seigneur de l’univers :
        \\Voici que, de nouveau, des peuples afflueront,
        \\des habitants de nombreuses villes.
        ${}^{21}Les habitants d’une ville iront dans une autre ville
        \\et diront :
        \\« Allons apaiser la face du Seigneur,
        \\allons chercher le Seigneur de l’univers !
        \\Quant à moi, j’y vais. »
        ${}^{22}Des peuples nombreux et des nations puissantes
        \\viendront à Jérusalem
        \\chercher le Seigneur de l’univers et apaiser sa face\\.
        
           
         
        ${}^{23}Ainsi parle le Seigneur de l’univers :
        \\En ces jours-là, dix hommes de toute langue
        \\et de toute nation
        \\saisiront un Juif par son vêtement et lui diront :
        \\« Nous voulons aller avec vous,
        \\car nous avons appris que Dieu est avec vous\\. »
        
           
      
         
      \bchapter{}
${}^{1}Proclamation.
        \\Parole du Seigneur au pays de Hadrak
        \\et de Damas, son repos,
        \\car le Seigneur a les yeux sur les hommes
        \\et sur toutes les tribus d’Israël,
${}^{2}il en sera de même pour Hamath qui en fera partie,
        \\ainsi que pour Tyr et Sidon.
        \\Parce qu’elle est très habile,
${}^{3}Tyr s’est construit une forteresse,
        \\amoncelant l’argent comme de la poussière,
        \\et l’or comme la boue des rues.
${}^{4}Voici que le Seigneur en prendra possession,
        \\il précipitera ses remparts dans la mer ;
        \\elle-même sera dévorée par le feu.
${}^{5}Ascalon le verra et sera épouvantée,
        \\et Gaza, qui se tordra de douleur,
        \\Éqrone aussi, car son appui s’est couvert de honte.
        \\Le roi disparaîtra de Gaza,
        \\Ascalon n’aura plus d’habitants,
${}^{6}et un bâtard habitera dans Ashdod.
        \\Je supprimerai l’orgueil du Philistin,
${}^{7}j’ôterai de sa bouche le sang qu’il boit,
        \\et de ses dents, les horreurs qu’il mange.
        \\Il sera lui aussi un reste pour notre Dieu,
        \\il sera comme un familier en Juda ;
        \\Éqrone sera pareil au Jébuséen.
${}^{8}Auprès de ma maison je camperai comme une garde
        \\contre ceux qui vont et viennent :
        \\plus personne pour venir l’opprimer,
        \\maintenant que j’ai vu de mes yeux !
        
           
        ${}^{9}Exulte de toutes tes forces, fille de Sion !
        \\Pousse des cris de joie, fille de Jérusalem\\ !
        \\Voici ton roi qui vient à toi :
        \\il est juste et victorieux\\,
        \\pauvre et monté sur un âne,
        \\un ânon, le petit d’une ânesse.
         
        ${}^{10}Ce roi\\fera disparaître d’Éphraïm les chars de guerre,
        \\et de Jérusalem les chevaux de combat ;
        \\il brisera l’arc de guerre,
        \\et il proclamera la paix aux nations.
        \\Sa domination s’étendra d’une mer à l’autre,
        \\et de l’Euphrate à l’autre bout du pays.
${}^{11}Quant à toi, par le sang de ton alliance,
        \\je fais sortir tes captifs de la citerne sans eau.
${}^{12}Revenez à la place forte,
        \\captifs pleins d’espérance.
        \\Aujourd’hui même, je l’affirme,
        \\je te rendrai au double.
${}^{13}Car j’ai tendu mon arc – c’est Juda.
        \\Je le garnis d’une flèche – c’est Éphraïm.
        \\Je vais exciter tes fils, ô Sion, contre les fils des Grecs ;
        \\je ferai de toi une épée de héros.
${}^{14}Alors le Seigneur apparaîtra au-dessus d’eux,
        \\et sa flèche jaillira comme l’éclair ;
        \\le Seigneur Dieu sonnera du cor,
        \\il s’avancera dans les ouragans du midi.
${}^{15}Le Seigneur de l’univers les protégera,
        \\ils mangeront et piétineront les pierres de fronde,
        \\ils boiront, en faisant du tapage comme pris de vin,
        \\et ils seront remplis comme la coupe d’aspersion,
        \\comme les cornes de l’autel.
${}^{16}Et le Seigneur leur Dieu les sauvera, ce jour-là,
        \\eux, les brebis de son peuple.
        \\Oui, des pierres de diadème scintilleront sur sa terre ;
${}^{17}oui, quelle prospérité, quelle beauté que la leur !
        \\Le froment épanouira les jeunes gens,
        \\et le vin nouveau, les jeunes filles.
      
         
      \bchapter{}
${}^{1}Demandez au Seigneur la pluie, la pluie de printemps ;
        \\c’est le Seigneur qui provoque les orages.
        \\Il leur donnera une pluie abondante,
        \\et à chacun, de l’herbe dans son champ.
        
           
${}^{2}Puisque les terafim ont fait de fausses prédictions,
        \\que les devins ont eu des visions mensongères,
        \\puisqu’ils ont débité des songes trompeurs
        \\et donné de vaines consolations,
        \\voilà pourquoi le peuple est parti
        \\comme un troupeau malheureux faute de berger.
${}^{3}Contre les bergers s’est enflammée ma colère,
        \\contre les boucs je vais sévir.
        \\Car le Seigneur de l’univers visitera son troupeau
        \\– la maison de Juda –,
        \\dans le combat il en fera sa monture d’honneur.
${}^{4}De Juda sortira une pierre d’angle,
        \\de lui sortira un piquet solide,
        \\de lui sortira un arc de guerre,
        \\de lui sortiront tous ceux qui s’imposent.
        <p class="verset_anchor" id="para_bib_za_10_5">Ensemble, 
${}^{5}ils seront comme des héros
        \\piétinant, dans le combat, la boue des chemins.
        \\Ils combattront, car le Seigneur est avec eux,
        \\et ceux qui montent des chevaux seront couverts de honte.
${}^{6}De la maison de Juda je ferai des héros,
        \\et je sauverai la maison de Joseph.
        \\Je les rétablirai car j’en aurai compassion ;
        \\ils seront comme si je ne les avais pas rejetés,
        \\car moi, le Seigneur, je suis leur Dieu,
        \\et je les exaucerai.
${}^{7}Ceux d’Éphraïm seront comme des héros,
        \\comme sous l’effet du vin leur cœur se réjouira ;
        \\en les voyant, leurs fils se réjouiront,
        \\leur cœur exultera dans le Seigneur.
${}^{8}Je vais siffler pour les rassembler,
        \\car je les ai rachetés,
        \\et ils seront aussi nombreux qu’autrefois.
${}^{9}Je les ai disséminés parmi les nations,
        \\mais au loin, ils se souviendront de moi,
        \\ils vivront avec leurs fils et ils reviendront.
${}^{10}Je les ramènerai de la terre d’Égypte,
        \\et d’Assour je les rassemblerai ;
        \\je les ferai entrer au pays de Galaad et au Liban,
        \\et cela ne sera pas suffisant pour eux.
${}^{11}– Il passera par la mer, une mer d’angoisse,
        \\il frappera les flots dans la mer,
        \\toutes les profondeurs du Nil seront asséchées.
        \\L’orgueil d’Assour sera abattu
        \\et le sceptre de l’Égypte disparaîtra.
${}^{12}J’en ferai des héros dans la puissance du Seigneur ;
        \\c’est en son nom qu’ils marcheront
        \\– oracle du Seigneur.
      
         
      \bchapter{}
${}^{1}Ouvre tes portes, Liban,
        \\et que le feu dévore tes cèdres !
${}^{2}Hurle, cyprès, car le cèdre est tombé,
        \\les plus majestueux sont anéantis.
        \\Hurlez, chênes de Bashane,
        \\car elle est abattue, la forêt imprenable.
${}^{3}Voix des bergers qui hurlent,
        \\car leur splendeur est anéantie.
        \\Voix des lionceaux qui rugissent,
        \\car l’orgueil du Jourdain est anéanti.
        
           
${}^{4}Ainsi parle le Seigneur mon Dieu : « Fais paître des brebis destinées à l’abattoir, 
${}^{5}celles que leurs acheteurs abattent impunément, et dont leurs vendeurs disent : “Béni soit le Seigneur, me voilà riche !” ; leurs bergers ne les ont pas épargnées. 
${}^{6}En effet, je n’épargnerai plus les habitants du pays – oracle du Seigneur ! Désormais, moi aussi, je vais livrer les hommes, chacun aux mains de son prochain et aux mains de son roi ; ceux-ci écraseront le pays, mais je ne délivrerai pas les gens de leurs mains. » 
${}^{7}Je me mis donc à faire paître ces brebis, destinées à l’abattoir, pour les marchands de brebis. Je pris pour moi deux bâtons, j’appelai le premier « Faveur », et le second « Entente », et je fis paître les brebis. 
${}^{8}Je fis disparaître les trois pasteurs en un seul mois. Mais je perdis patience envers les brebis, et elles, de leur côté, me prirent en dégoût. 
${}^{9}Alors je dis : « Je ne veux plus vous faire paître. Celle qui doit mourir, qu’elle meure ; celle qui doit disparaître, qu’elle disparaisse, et celles qui survivent, qu’elles se dévorent entre elles. » 
${}^{10}Puis je pris mon bâton « Faveur » et je le brisai pour rompre mon alliance, celle que j’avais conclue avec tous les peuples. 
${}^{11}Ce jour-là, elle fut rompue, et les marchands de brebis qui m’observaient surent que c’était là une parole du Seigneur.
${}^{12}Je leur dis alors : « Si cela vous semble bon, donnez-moi mon salaire, sinon n’en faites rien. » Ils pesèrent mon salaire : trente pièces d’argent. 
${}^{13}Le Seigneur me dit : « Jette-le au fondeur, ce joli prix auquel ils m’ont apprécié ! » Alors je ramassai les trente pièces d’argent et je les jetai au fondeur dans la maison du Seigneur. 
${}^{14}Puis je brisai mon deuxième bâton « Entente » pour rompre la fraternité entre Juda et Israël. 
${}^{15}Le Seigneur me dit alors : « Prends encore l’équipement d’un berger insensé, 
${}^{16}car voici que moi je vais susciter un berger dans le pays ; la brebis disparue, il n’en aura pas souci ; celle qui est égarée, il ne la cherchera pas ; celle qui est blessée, il ne la soignera pas ; celle qui est épuisée, il ne la soutiendra pas, mais il dévorera la chair des bêtes grasses, il arrachera même leurs sabots.
${}^{17}Malheur à ce vaurien de berger qui délaisse les brebis !
        \\Que l’épée s’en prenne à son bras et à son œil droit !
        \\Que son bras se dessèche, oui, se dessèche !
        \\Que son œil droit s’éteigne, oui, qu’il s’éteigne ! »
      
         
      \bchapter{}
      \begin{verse}
${}^{1}Proclamation.
      \begin{verse}Parole du Seigneur adressée à Israël.
      Oracle du Seigneur qui a déployé les cieux et fondé la terre, qui a formé un souffle au-dedans de l’homme : 
${}^{2}Voici que moi je vais faire de Jérusalem une coupe de vertige pour tous les peuples d’alentour. De même en sera-t-il de Juda lors du siège de Jérusalem.
${}^{3}Il arrivera, en ce jour-là, que je ferai de Jérusalem, face à tous les peuples, une pierre impossible à soulever. Quiconque voudra la soulever se blessera grièvement. Contre elle vont se coaliser toutes les nations de la terre.
${}^{4}Ce jour-là – oracle du Seigneur –, je frapperai d’affolement tous les chevaux, et de folie leurs cavaliers. J’ouvrirai les yeux sur la maison de Juda. Je rendrai aveugles tous les chevaux des peuples. 
${}^{5}Alors les chefs de Juda diront en leur cœur : « Pour les habitants de Jérusalem, la force est dans le Seigneur de l’univers, leur Dieu. »
${}^{6}Ce jour-là, je ferai des chefs de Juda un brasier allumé sous du bois, une torche allumée dans des gerbes. Ils dévoreront à droite et à gauche tous les peuples d’alentour. Et Jérusalem demeurera habitée au même lieu, à Jérusalem. 
${}^{7}Le Seigneur sauvera d’abord les tentes de Juda, pour que la fierté de la maison de David et la fierté de l’habitant de Jérusalem ne s’exaltent pas aux dépens de Juda.
${}^{8}Ce jour-là, le Seigneur protégera les habitants de Jérusalem ; le plus chancelant d’entre eux, ce jour-là, sera comme David, et la maison de David sera comme Dieu, comme l’ange du Seigneur qui se tient devant eux.
${}^{9}Il arrivera, en ce jour-là, que je m’appliquerai à détruire toutes les nations qui viendront contre Jérusalem.
${}^{10}Je répandrai sur la maison de David et sur les habitants de Jérusalem un esprit de grâce et de supplication. Ils regarderont vers moi\\. Celui qu’ils ont transpercé\\, ils feront une lamentation sur lui, comme on se lamente sur un fils unique ; ils pleureront sur lui amèrement, comme on pleure sur un premier-né.
${}^{11}Ce jour-là, il y aura grande lamentation dans Jérusalem, comme il y a une lamentation à Hadad-Rimmone, dans la plaine de Meguiddo.
${}^{12}Et tout le pays se lamentera, clan par clan :
        \\le clan de la maison de David à part,
        \\et leurs femmes à part ;
        \\le clan de la maison de Natane à part,
        \\et leurs femmes à part ;
${}^{13}le clan de la maison de Lévi à part,
        \\et leurs femmes à part ;
        \\le clan de la maison de Shiméï à part,
        \\et leurs femmes à part ;
${}^{14}et tous les clans, ceux qui restent, clan par clan à part,
        \\et leurs femmes à part.
      
         
      \bchapter{}
      \begin{verse}
${}^{1}Ce jour-là, il y aura une source qui jaillira pour la maison de David et pour les habitants de Jérusalem : elle les lavera de\\leur péché et de leur souillure\\.
${}^{2}Il arrivera, en ce jour-là – oracle du Seigneur de l’univers –, que je retrancherai du pays les noms des idoles : on n’en fera plus mémoire. Je chasserai aussi du pays les prophètes et leur esprit d’impureté.
${}^{3}Si quelqu’un veut encore prophétiser, son père et sa mère, qui l’ont engendré, lui diront : « Tu ne vivras pas, car ce sont des mensonges que tu prononces au nom du Seigneur. » Alors son propre père et sa propre mère, qui l’ont engendré, le transperceront pendant qu’il prophétisera.
${}^{4}Il arrivera, en ce jour-là, que les prophètes rougiront de leur vision quand ils prophétiseront. Ils ne revêtiront plus le manteau de prophètes pour tromper.
${}^{5}Mais ils diront : « Moi, je ne suis pas prophète ; moi, je travaille la terre : un homme m’a acheté depuis ma jeunesse. » 
${}^{6}Et si on lui demande : « Que sont donc ces blessures sur ta poitrine ? », il répondra : « Je les ai reçues dans la maison de ceux qui m’aiment. »
${}^{7}Épée, réveille-toi contre mon berger,
        \\contre l’homme qui m’est proche
        \\– oracle du Seigneur de l’univers.
        \\Frappe le berger, et que les brebis soient dispersées,
        \\contre les petits je tournerai ma main.
${}^{8}Il arrivera dans tout le pays – oracle du Seigneur –
        \\que deux tiers en seront retranchés, périront,
        \\et que l’autre tiers y restera.
${}^{9}Je ferai passer ce tiers par le feu ;
        \\je l’épurerai comme on épure l’argent,
        \\je l’éprouverai comme on éprouve l’or.
        \\Lui, il invoquera mon nom,
        \\et moi, je lui répondrai.
        \\Je dirai : « C’est mon peuple ! »,
        \\et lui, il dira : « Le Seigneur est mon Dieu ! »
      
         
      \bchapter{}
      \begin{verse}
${}^{1}Voici qu’un jour vient pour le Seigneur, où l’on partagera tes dépouilles au milieu de toi, Jérusalem.
${}^{2}Je rassemblerai toutes les nations devant Jérusalem pour le combat ; la ville sera prise, les maisons pillées, les femmes violées ; la moitié de la ville partira en exil, mais le reste du peuple ne sera pas retranché de la ville.
${}^{3}Alors le Seigneur sortira pour combattre avec les nations, comme lorsqu’il combat au jour de la bataille. 
${}^{4}Ses pieds se poseront, ce jour-là, sur le mont des Oliviers qui est en face de Jérusalem, à l’orient. Et le mont des Oliviers se fendra par le milieu, d’est en ouest ; il deviendra une immense vallée. Une moitié de la montagne reculera vers le nord, et l’autre vers le sud. 
${}^{5}Vous fuirez la vallée de mes montagnes, car elle atteindra Yasol. Vous fuirez comme vous avez fui devant le tremblement de terre, au temps d’Ozias, roi de Juda. Alors le Seigneur mon Dieu viendra, et tous les saints avec lui. 
${}^{6}Ce jour-là, il n’y aura pas de lumière, mais du froid et du gel.
${}^{7}Ce sera un jour unique – le Seigneur le connaît –, il n’y aura pas le jour puis la nuit, mais à l’heure du soir, la lumière. 
${}^{8}Ce jour-là, des eaux vives sortiront de Jérusalem, moitié vers la mer orientale, moitié vers la mer occidentale : il en sera ainsi en été, comme en hiver. 
${}^{9}Alors le Seigneur deviendra roi sur toute la terre ; ce jour-là, le Seigneur sera unique, et unique, son nom.
${}^{10}Tout le pays sera changé en plaine, depuis Guéba jusqu’à Rimmone, au sud de Jérusalem. Jérusalem sera surélevée, demeurant habitée au même lieu, depuis la porte de Benjamin jusqu’à l’emplacement de l’ancienne porte, jusqu’à la porte des Angles, et depuis la tour de Hananéel jusqu’aux pressoirs du roi. 
${}^{11}On y habitera, il n’y aura plus d’anathème : Jérusalem sera habitée en sécurité.
${}^{12}Et voici le fléau dont le Seigneur frappera tous les peuples qui auront combattu contre Jérusalem : il fera pourrir la chair de chacun quand il sera encore debout sur ses pieds ; ses yeux pourriront dans leurs orbites, sa langue pourrira dans sa bouche. 
${}^{13}Il arrivera, en ce jour-là, que le Seigneur provoquera une grande panique parmi eux. Chacun empoignera son compagnon et ils lutteront corps à corps. 
${}^{14}Juda aussi combattra avec Jérusalem. Les richesses de toutes les nations d’alentour seront rassemblées, or, argent, vêtements, en énorme quantité.
${}^{15}Un fléau semblable atteindra les chevaux, les mulets, les chameaux, les ânes et toutes les bêtes qui se trouveront dans leurs camps : ce sera le même fléau.
${}^{16}Alors tous les survivants des nations qui auront marché contre Jérusalem monteront année après année se prosterner devant le Roi Seigneur de l’univers, et célébrer la fête des Tentes. 
${}^{17}Mais pour les familles de la terre qui ne monteront pas se prosterner à Jérusalem devant le Roi Seigneur de l’univers, la pluie ne tombera pas. 
${}^{18}Et si la famille d’Égypte ne veut pas monter, si elle ne vient pas, alors fondra sur elle le fléau dont le Seigneur frappera les nations qui ne monteront pas célébrer la fête des Tentes. 
${}^{19}Tel sera le châtiment de l’Égypte et le châtiment de toutes les nations qui ne monteront pas célébrer la fête des Tentes.
${}^{20}Ce jour-là, les grelots des chevaux porteront l’inscription « Consacré au Seigneur » ; les marmites, dans la maison du Seigneur, seront comme des coupes d’aspersion devant l’autel. 
${}^{21}Toute marmite, à Jérusalem et en Juda, sera consacrée au Seigneur de l’univers ; tous ceux qui offrent un sacrifice viendront les prendre pour cuire ce qu’ils présentent. Il n’y aura plus de marchand dans la maison du Seigneur de l’univers, en ce jour-là.
