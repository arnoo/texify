  
  
    
    \bbook{LETTRE DE SAINT JACQUES}{LETTRE DE SAINT JACQUES}
      
         
      \bchapter{}
        ${}^{1}Jacques, serviteur de Dieu
        et du Seigneur Jésus Christ,
        \\aux douze tribus de la Diaspora,
        salut !
        
           
${}^{2}Considérez comme une joie extrême, mes frères, de buter sur toute sorte d’épreuves. 
${}^{3}Vous le savez, une telle vérification de votre foi produit l’endurance, 
${}^{4}et l’endurance doit s’accompagner d’une action parfaite, pour que vous soyez parfaits et intègres, sans que rien ne vous manque.
${}^{5}Mais si l’un de vous manque de sagesse, qu’il la demande à Dieu, lui qui donne à tous sans réserve et sans faire de reproches : elle lui sera donnée. 
${}^{6}Mais qu’il demande avec foi, sans la moindre hésitation, car celui qui hésite ressemble aux vagues de la mer que le vent agite et soulève. 
${}^{7}Qu’il ne s’imagine pas, cet homme-là, qu’il recevra du Seigneur quoi que ce soit, 
${}^{8}s’il est partagé, instable dans toute sa conduite.
${}^{9}Que le frère d’humble condition tire sa fierté d’être élevé, 
${}^{10}et le riche, d’être humilié, car il passera comme l’herbe en fleur. 
${}^{11}En effet, le soleil s’est levé, ainsi que le vent brûlant, il a desséché l’herbe, sa fleur est tombée, la beauté de son aspect a disparu ; de même, le riche se flétrira dans toutes ses entreprises.
${}^{12}Heureux l’homme qui supporte l’épreuve avec persévérance, car, sa valeur une fois vérifiée, il recevra la couronne de la vie promise à ceux qui aiment Dieu. 
${}^{13}Dans l’épreuve de la tentation, que personne ne dise : « Ma tentation vient de Dieu. » Dieu, en effet, ne peut être tenté de faire le mal, et lui-même ne tente personne. 
${}^{14}Chacun est tenté par sa propre convoitise qui l’entraîne et le séduit. 
${}^{15}Puis la convoitise conçoit et enfante le péché, et le péché, arrivé à son terme, engendre la mort.
${}^{16}Ne vous y trompez pas, mes frères bien-aimés, 
${}^{17}les présents les meilleurs, les dons parfaits, proviennent tous d’en haut, ils descendent d’auprès du Père des lumières, lui qui n’est pas, comme les astres, sujet au mouvement périodique ni aux éclipses. 
${}^{18}Il a voulu nous engendrer par sa parole de vérité, pour faire de nous comme les prémices de toutes ses créatures.
${}^{19}Sachez-le, mes frères bien-aimés : chacun doit être prompt à écouter, lent à parler, lent à la colère, 
${}^{20}car la colère de l’homme ne réalise pas ce qui est juste selon Dieu. 
${}^{21}C’est pourquoi, ayant rejeté tout ce qui est sordide et tout débordement de méchanceté, accueillez dans la douceur la Parole semée en vous ; c’est elle qui peut sauver vos âmes.
${}^{22}Mettez la Parole en pratique, ne vous contentez pas de l’écouter : ce serait vous faire illusion. 
${}^{23}Car si quelqu’un écoute la Parole sans la mettre en pratique, il est comparable à un homme qui observe dans un miroir son visage tel qu’il est, 
${}^{24}et qui, aussitôt après, s’en va en oubliant comment il était. 
${}^{25}Au contraire, celui qui se penche sur la loi parfaite, celle de la liberté, et qui s’y tient, lui qui l’écoute non pour l’oublier, mais pour la mettre en pratique dans ses actes, celui-là sera heureux d’agir ainsi.
${}^{26}Si l’on pense être quelqu’un de religieux sans mettre un frein à sa langue, on se trompe soi-même, une telle religion est sans valeur. 
${}^{27}Devant Dieu notre Père, un comportement religieux pur et sans souillure, c’est de visiter les orphelins et les veuves dans leur détresse, et de se garder sans tache au milieu du monde.
      
         
      \bchapter{}
      \begin{verse}
${}^{1}Mes frères, dans votre foi en Jésus Christ, notre Seigneur de gloire, n’ayez aucune partialité envers les personnes. 
${}^{2}Imaginons que, dans votre assemblée, arrivent en même temps un homme au vêtement rutilant, portant une bague en or, et un pauvre au vêtement sale. 
${}^{3}Vous tournez vos regards vers celui qui porte le vêtement rutilant et vous lui dites : « Assieds-toi ici, en bonne place » ; et vous dites au pauvre : « Toi, reste là debout », ou bien : « Assieds-toi au bas de mon marchepied ». 
${}^{4}Cela, n’est-ce pas faire des différences entre vous, et juger selon de faux critères ? 
${}^{5}Écoutez donc, mes frères bien-aimés ! Dieu, lui, n’a-t-il pas choisi ceux qui sont pauvres aux yeux du monde pour en faire des riches dans la foi, et des héritiers du Royaume promis par lui à ceux qui l’auront aimé ? 
${}^{6}Mais vous, vous avez privé le pauvre de sa dignité. Or n’est-ce pas les riches qui vous oppriment, et vous traînent devant les tribunaux ? 
${}^{7}Ce sont eux qui blasphèment le beau nom du Seigneur qui a été invoqué sur vous.
${}^{8}Certes, si vous accomplissez la loi du Royaume selon l’Écriture : Tu aimeras ton prochain comme toi-même, vous faites bien. 
${}^{9}Mais si vous montrez de la partialité envers les personnes, vous commettez un péché, et cette loi vous convainc de transgression. 
${}^{10}En effet, si quelqu’un observe intégralement la loi, sauf en un seul point sur lequel il trébuche, le voilà coupable par rapport à l’ensemble. 
${}^{11}En effet, si Dieu a dit : Tu ne commettras pas d’adultère, il a dit aussi : Tu ne commettras pas de meurtre. Donc, si tu ne commets pas d’adultère mais si tu commets un meurtre, te voilà transgresseur de la loi.
${}^{12}Parlez et agissez comme des gens qui vont être jugés par une loi de liberté. 
${}^{13}Car le jugement est sans miséricorde pour celui qui n’a pas fait miséricorde, mais la miséricorde l’emporte sur le jugement.
${}^{14}Mes frères, si quelqu’un prétend avoir la foi, sans la mettre en œuvre, à quoi cela sert-il ? Sa foi peut-elle le sauver ? 
${}^{15}Supposons qu’un frère ou une sœur n’ait pas de quoi s’habiller, ni de quoi manger tous les jours ; 
${}^{16}si l’un de vous leur dit : « Allez en paix ! Mettez-vous au chaud, et mangez à votre faim ! » sans leur donner le nécessaire pour vivre, à quoi cela sert-il ? 
${}^{17}Ainsi donc, la foi, si elle n’est pas mise en œuvre, est bel et bien morte.
${}^{18}En revanche, on va dire : « Toi, tu as la foi ; moi, j’ai les œuvres. Montre-moi donc ta foi sans les œuvres ; moi, c’est par mes œuvres que je te montrerai la foi. 
${}^{19}Toi, tu crois qu’il y a un seul Dieu. Fort bien ! Mais les démons, eux aussi, le croient et ils tremblent.
${}^{20}Homme superficiel, veux-tu reconnaître que la foi sans les œuvres ne sert à rien ? 
${}^{21}N’est-ce pas par ses œuvres qu’Abraham notre père est devenu juste, lorsqu’il a présenté son fils Isaac sur l’autel du sacrifice ? 
${}^{22}Tu vois bien que la foi agissait avec ses œuvres et, par les œuvres, la foi devint parfaite. 
${}^{23}Ainsi fut accomplie la parole de l’Écriture : Abraham eut foi en Dieu ; aussi, il lui fut accordé d’être juste, et il reçut le nom d’ami de Dieu. » 
${}^{24}Vous voyez bien : l’homme devient juste par les œuvres, et non seulement par la foi. 
${}^{25}Il en fut de même pour Rahab, la prostituée : n’est-elle pas, elle aussi, devenue juste par ses œuvres, en accueillant les envoyés de Josué et en les faisant repartir par un autre chemin ? 
${}^{26}Ainsi, comme le corps privé de souffle est mort, de même la foi sans les œuvres est morte.
      
         
      \bchapter{}
      \begin{verse}
${}^{1}Mes frères, ne soyez pas nombreux à devenir des maîtres : comme vous le savez, nous qui enseignons, nous serons jugés plus sévèrement. 
${}^{2}Tous, en effet, nous commettons des écarts, et souvent. Si quelqu’un ne commet pas d’écart quand il parle, c’est un homme parfait, capable de maîtriser son corps tout entier.
${}^{3}En mettant un frein dans la bouche des chevaux pour qu’ils nous obéissent, nous dirigeons leur corps tout entier. 
${}^{4}Voyez aussi les navires : quelles que soient leur taille et la force des vents qui les poussent, ils sont dirigés par un tout petit gouvernail au gré de l’impulsion donnée par le pilote. 
${}^{5}De même, notre langue est une petite partie de notre corps et elle peut se vanter de faire de grandes choses. Voyez encore : un tout petit feu peut embraser une très grande forêt. 
${}^{6}La langue aussi est un feu ; monde d’injustice, cette langue tient sa place parmi nos membres ; c’est elle qui contamine le corps tout entier, elle enflamme le cours de notre existence, étant elle même enflammée par la géhenne. 
${}^{7}Toute espèce de bêtes sauvages et d’oiseaux, de reptiles et d’animaux marins peut être domptée et, de fait, toutes furent domptées par l’espèce humaine ; 
${}^{8}mais la langue, personne ne peut la dompter : elle est un fléau, toujours en mouvement, remplie d’un venin mortel. 
${}^{9}Elle nous sert à bénir le Seigneur notre Père, elle nous sert aussi à maudire les hommes, qui sont créés à l’image de Dieu. 
${}^{10}De la même bouche sortent bénédiction et malédiction. Mes frères, il ne faut pas qu’il en soit ainsi. 
${}^{11}Une source fait-elle jaillir par le même orifice de l’eau douce et de l’eau amère ? 
${}^{12}Mes frères, un figuier peut-il donner des olives ? Une vigne peut-elle donner des figues ? Une source d’eau salée ne peut pas davantage donner de l’eau douce.
${}^{13}Quelqu’un, parmi vous, a-t-il la sagesse et le savoir ? Qu’il montre par sa vie exemplaire que la douceur de la sagesse inspire ses actes. 
${}^{14}Mais si vous avez dans le cœur la jalousie amère et l’esprit de rivalité, ne vous en vantez pas, ne mentez pas, n’allez pas contre la vérité. 
${}^{15}Cette prétendue sagesse ne vient pas d’en haut ; au contraire, elle est terrestre, purement humaine, démoniaque. 
${}^{16}Car la jalousie et les rivalités mènent au désordre et à toutes sortes d’actions malfaisantes. 
${}^{17}Au contraire, la sagesse qui vient d’en haut est d’abord pure, puis pacifique, bienveillante, conciliante, pleine de miséricorde et féconde en bons fruits, sans parti pris, sans hypocrisie. 
${}^{18}C’est dans la paix qu’est semée la justice, qui donne son fruit aux artisans de la paix.
      
         
      \bchapter{}
      \begin{verse}
${}^{1}D’où viennent les guerres, d’où viennent les conflits entre vous ? N’est-ce pas justement de tous ces désirs qui mènent leur combat en vous-mêmes ? 
${}^{2}Vous êtes pleins de convoitises et vous n’obtenez rien, alors vous tuez ; vous êtes jaloux et vous n’arrivez pas à vos fins, alors vous entrez en conflit et vous faites la guerre. Vous n’obtenez rien parce que vous ne demandez pas ; 
${}^{3}vous demandez, mais vous ne recevez rien ; en effet, vos demandes sont mauvaises, puisque c’est pour tout dépenser en plaisirs. 
${}^{4}Adultères que vous êtes ! Ne savez-vous pas que l’amour pour le monde rend ennemi de Dieu ? Donc celui qui veut être ami du monde se pose en ennemi de Dieu. 
${}^{5}Ou bien pensez-vous que l’Écriture parle pour rien quand elle dit : Dieu veille jalousement sur l’Esprit qu’il a fait habiter en nous ? 
${}^{6}Dieu ne nous donne-t-il pas une grâce plus grande encore ? C’est ce que dit l’Écriture :
        \\Dieu s’oppose aux orgueilleux,
        \\aux humbles il accorde sa grâce.
${}^{7}Soumettez-vous donc à Dieu, et résistez au diable : il s’enfuira loin de vous. 
${}^{8}Approchez-vous de Dieu, et lui s’approchera de vous. Pécheurs, enlevez la souillure de vos mains ; esprits doubles, purifiez vos cœurs. 
${}^{9}Reconnaissez votre misère, prenez le deuil et pleurez ; que votre rire se change en deuil et votre joie en accablement. 
${}^{10}Abaissez-vous devant le Seigneur, et il vous élèvera.
${}^{11}Frères, cessez de dire du mal les uns des autres ; dire du mal de son frère ou juger son frère, c’est dire du mal de la Loi et juger la Loi. Or, si tu juges la Loi, tu ne la pratiques pas, mais tu en es le juge. 
${}^{12}Un seul est à la fois législateur et juge, celui qui a le pouvoir de sauver et de perdre. Pour qui te prends-tu donc, toi qui juges ton prochain ?
${}^{13}Vous autres, maintenant, vous dites : « Aujourd’hui ou demain nous irons dans telle ou telle ville, nous y passerons l’année, nous ferons du commerce et nous gagnerons de l’argent », 
${}^{14}alors que vous ne savez même pas ce que sera votre vie demain ! Vous n’êtes qu’un peu de brume, qui paraît un instant puis disparaît. 
${}^{15}Vous devriez dire au contraire : « Si le Seigneur le veut bien, nous serons en vie et nous ferons ceci ou cela. » 
${}^{16}Et voilà que vous mettez votre fierté dans vos vantardises. Toute fierté de ce genre est mauvaise ! 
${}^{17}Être en mesure de faire le bien et ne pas le faire, c’est un péché.
      
         
      \bchapter{}
      \begin{verse}
${}^{1}Et vous autres, maintenant, les riches ! Pleurez, lamentez-vous sur les malheurs qui vous attendent. 
${}^{2}Vos richesses sont pourries, vos vêtements sont mangés des mites, 
${}^{3}votre or et votre argent sont rouillés. Cette rouille sera un témoignage contre vous, elle dévorera votre chair comme un feu. Vous avez amassé des richesses, alors que nous sommes dans les derniers jours ! 
${}^{4}Le salaire dont vous avez frustré les ouvriers qui ont moissonné vos champs, le voici qui crie, et les clameurs des moissonneurs sont parvenues aux oreilles du Seigneur de l’univers. 
${}^{5}Vous avez mené sur terre une vie de luxe et de délices, et vous vous êtes rassasiés au jour du massacre. 
${}^{6}Vous avez condamné le juste et vous l’avez tué, sans qu’il vous oppose de résistance.
      
         
${}^{7}Frères, en attendant la venue du Seigneur, prenez patience. Voyez le cultivateur : il attend les fruits précieux de la terre avec patience, jusqu’à ce qu’il ait fait la récolte précoce et la récolte tardive. 
${}^{8}Prenez patience, vous aussi, et tenez ferme car la venue du Seigneur est proche. 
${}^{9}Frères, ne gémissez pas les uns contre les autres, ainsi vous ne serez pas jugés. Voyez : le Juge est à notre porte. 
${}^{10}Frères, prenez pour modèles d’endurance et de patience les prophètes qui ont parlé au nom du Seigneur. 
${}^{11}Voyez : nous proclamons heureux ceux qui tiennent bon. Vous avez entendu dire comment Job a tenu bon, et vous avez vu ce qu’à la fin le Seigneur a fait pour lui, car le Seigneur est tendre et miséricordieux.
${}^{12}Et avant tout, mes frères, ne faites pas de serment : ne jurez ni par le ciel ni par la terre, ni d’aucune autre manière ; que votre « oui » soit un « oui », que votre « non » soit un « non » ; ainsi vous ne tomberez pas sous le jugement.
${}^{13}L’un de vous se porte mal ? Qu’il prie. Un autre va bien ? Qu’il chante le Seigneur. 
${}^{14}L’un de vous est malade ? Qu’il appelle les Anciens en fonction dans l’Église : ils prieront sur lui après lui avoir fait une onction d’huile au nom du Seigneur. 
${}^{15}Cette prière inspirée par la foi sauvera le malade : le Seigneur le relèvera et, s’il a commis des péchés, il recevra le pardon. 
${}^{16}Confessez donc vos péchés les uns aux autres, et priez les uns pour les autres afin d’être guéris.
      La supplication du juste agit avec beaucoup de force. 
${}^{17}Le prophète Élie n’était qu’un homme pareil à nous ; pourtant, lorsqu’il a prié avec insistance pour qu’il ne pleuve pas, il n’est pas tombé de pluie sur la terre pendant trois ans et demi ; 
${}^{18}puis il a prié de nouveau, et le ciel a donné la pluie, et la terre a fait germer son fruit.
${}^{19}Mes frères, si l’un de vous s’égare loin de la vérité et qu’un autre l’y ramène, 
${}^{20}alors, sachez-le : celui qui ramène un pécheur du chemin où il s’égarait sauvera son âme de la mort et couvrira une multitude de péchés.
