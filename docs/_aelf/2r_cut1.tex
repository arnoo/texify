  
  
    
    \bbook{DEUXIÈME LIVRE DES ROIS}{DEUXIÈME LIVRE DES ROIS}
      
         
      \bchapter{}
      \begin{verse}
${}^{1}A<span class="smallcaps">PRÈS LA MORT D’<span class="no_smallcaps">A</span>CAB</span>, le pays de Moab se révolta contre Israël. 
${}^{2}À Samarie, Ocozias tomba du balcon de sa chambre haute. Il se fit très mal. Il envoya des messagers et il leur dit : « Allez consulter Baal-Zéboub, dieu d’Éqrone, pour savoir si je guérirai de ce mal. » 
${}^{3}Mais l’ange du Seigneur dit à Élie de Tishbé : « Lève-toi ! Monte à la rencontre des messagers du roi de Samarie et tu leur diras : “N’y a-t-il donc pas de Dieu en Israël, que vous alliez consulter Baal-Zéboub, dieu d’Éqrone ? 
${}^{4}C’est pourquoi ainsi parle le Seigneur : Le lit sur lequel tu es monté, tu n’en descendras pas, car, à coup sûr, tu mourras.” » Élie s’en alla.
${}^{5}Les messagers revinrent auprès du roi, qui leur dit : « Pourquoi donc êtes-vous revenus ? » 
${}^{6}Ils répondirent : « Un homme est monté à notre rencontre et nous a dit : “Allez ! Retournez auprès du roi qui vous a envoyés et dites-lui : Ainsi parle le Seigneur : N’y a-t-il donc pas de Dieu en Israël, que tu envoies consulter Baal-Zéboub, dieu d’Éqrone ? C’est pourquoi le lit sur lequel tu es monté, tu n’en descendras pas, car, à coup sûr, tu mourras.” » 
${}^{7}Il leur dit : « Comment était habillé l’homme qui est venu à votre rencontre et qui vous a dit ces paroles ? » 
${}^{8}Ils répondirent : « C’était un homme portant un vêtement de poils et une ceinture de cuir autour des reins. » Il déclara : « C’est Élie de Tishbé. »
${}^{9}Ocozias envoya vers Élie un officier avec ses cinquante hommes. Celui-ci monta vers Élie, le trouva assis au sommet de la montagne et lui dit : « Homme de Dieu, par ordre du roi : Descends ! » 
${}^{10}Élie répondit à l’officier : « Si je suis un homme de Dieu, qu’un feu du ciel descende et te dévore, toi et tes cinquante hommes ! » Et un feu du ciel descendit et le dévora, lui et ses cinquante hommes. 
${}^{11}Le roi envoya encore vers Élie un autre officier avec ses cinquante hommes. Celui-ci prit la parole : « Homme de Dieu, ainsi parle le roi : Descends vite ! » 
${}^{12}Élie leur répondit : « Si je suis un homme de Dieu, qu’un feu du ciel descende et te dévore, toi et tes cinquante hommes ! » Et le feu de Dieu descendit du ciel et le dévora, lui et ses cinquante hommes. 
${}^{13}Le roi envoya encore un troisième officier avec ses cinquante hommes. Le troisième officier monta. En arrivant, il fléchit les genoux devant Élie et le supplia par ces mots : « Homme de Dieu, je t’en prie, que ma vie et la vie de tes serviteurs, ces cinquante hommes, aient du prix à tes yeux ! 
${}^{14}Voilà qu’un feu du ciel est descendu et a dévoré les deux premiers officiers avec leurs cinquante hommes. Mais maintenant, que ma vie ait du prix à tes yeux ! » 
${}^{15}L’ange du Seigneur dit à Élie : « Descends avec lui. Ne crains rien de sa part ! » Élie se leva et descendit avec lui vers le roi. 
${}^{16}Il dit au roi : « Ainsi parle le Seigneur : Tu as envoyé des messagers consulter Baal-Zéboub, dieu d’Éqrone. N’y avait-il donc pas de Dieu en Israël pour consulter sa parole ? Eh bien ! le lit sur lequel tu es monté, tu n’en descendras pas, car, à coup sûr, tu mourras ! » 
${}^{17}Conformément à la parole du Seigneur dite par Élie, Ocozias mourut et, comme il n’avait pas eu de fils, Joram, son frère, régna à sa place. C’était la deuxième année de Joram, fils de Josaphat, roi de Juda.
${}^{18}Le reste des actions d’Ocozias, ce qu’il a fait,
        \\cela n’est-il pas écrit dans le livre des Annales des rois d’Israël ?
      
         
      \bchapter{}
      \begin{verse}
${}^{1}Voici comment le Seigneur enleva Élie au ciel dans un ouragan. Ce jour-là, Élie et Élisée étaient partis de Guilgal. 
${}^{2}Élie dit à Élisée : « Arrête-toi ici ; et moi, le Seigneur m’envoie à Béthel. » Élisée répliqua : « Par le Seigneur qui est vivant, et par ta vie, je ne te quitterai pas. » Ils allèrent tous deux à Béthel. 
${}^{3}Les frères-prophètes de Béthel sortirent à la rencontre d’Élisée et lui dirent : « Sais-tu qu’aujourd’hui le Seigneur va enlever ton maître au-dessus de ta tête ? » Élisée répondit : « Oui, je le sais. Taisez-vous ! » 
${}^{4}Élie lui dit de nouveau : « Arrête-toi ici ; et moi, le Seigneur m’envoie à Jéricho. » Élisée répliqua : « Par le Seigneur qui est vivant, et par ta vie, je ne te quitterai pas. » Ils allèrent tous deux à Jéricho. 
${}^{5}Les frères-prophètes de Jéricho s’approchèrent d’Élisée et lui dirent : « Sais-tu bien qu’aujourd’hui le Seigneur va enlever ton maître au-dessus de ta tête ? » Élisée répondit : « Oui, je le sais. Taisez-vous ! » 
${}^{6}Une troisième fois, Élie dit à Élisée : « Arrête-toi ici ; et moi, le Seigneur m’envoie au Jourdain. » Mais Élisée répliqua : « Par le Seigneur qui est vivant, et par ta vie, je ne te quitterai pas. » Ils continuèrent donc tous les deux.
${}^{7}Cinquante frères-prophètes, qui les avaient suivis, s’arrêtèrent à distance, pendant que tous deux se tenaient au bord du Jourdain. 
${}^{8} Élie prit son manteau, le roula et en frappa les eaux, qui s’écartèrent de part et d’autre. Ils traversèrent tous deux à pied sec. 
${}^{9} Pendant qu’ils passaient, Élie dit à Élisée : « Dis-moi ce que tu veux que je fasse pour toi avant d’être enlevé loin de toi. » Élisée répondit : « Que je reçoive une double part de l’esprit que tu as reçu ! » 
${}^{10} Élie reprit : « Tu demandes quelque chose de difficile : tu l’obtiendras si tu me vois lorsque je serai enlevé loin de toi. Sinon, tu ne l’obtiendras pas. » 
${}^{11} Ils étaient en train de marcher tout en parlant lorsqu’un char de feu, avec des chevaux de feu, les sépara. Alors, Élie monta au ciel dans un ouragan. 
${}^{12} Élisée le vit et se mit à crier : « Mon père !... Mon père !... Char d’Israël et ses cavaliers\\ ! » Puis il cessa de le voir. Il saisit ses vêtements et les déchira en deux. 
${}^{13} Il ramassa le manteau qu’Élie avait laissé tomber, il revint et s’arrêta sur la rive du Jourdain. 
${}^{14} Avec le manteau d’Élie\\, il frappa les eaux, mais elles ne s’écartèrent pas\\. Élisée dit alors : « Où est donc le Seigneur, le Dieu d’Élie ? » Il frappa encore une fois, les eaux s’écartèrent, et il traversa.
${}^{15}Depuis l’autre rive, les frères-prophètes, ceux de Jéricho, l’aperçurent et dirent : « L’esprit d’Élie repose sur Élisée ». Ils vinrent donc à sa rencontre et se prosternèrent jusqu’à terre devant lui. 
${}^{16}Ils lui dirent : « Voici justement qu’il y a, parmi tes serviteurs, cinquante hommes valeureux. Permets qu’ils aillent à la recherche de ton maître. Peut-être l’Esprit du Seigneur l’a-t-il enlevé et déposé sur quelque montagne ou dans quelque vallée ! » Il répondit : « N’envoyez personne ! » 
${}^{17}Mais ils insistèrent tellement qu’il leur dit : « Envoyez-les donc ! » Et ils envoyèrent les cinquante hommes qui cherchèrent Élie pendant trois jours sans le trouver. 
${}^{18}Ils revinrent vers Élisée qui était resté à Jéricho. Il leur dit : « Ne vous avais-je pas dit : N’y allez pas ? »
${}^{19}Des gens de la ville dirent à Élisée : « Comme mon seigneur peut le constater, l’emplacement de la ville est bon. Toutefois les eaux sont mauvaises et le pays stérile ! » 
${}^{20}Il dit : « Apportez-moi une écuelle neuve et mettez-y du sel. » Ils la lui apportèrent. 
${}^{21}Il sortit vers la source des eaux, y jeta du sel et déclara : « Ainsi parle le Seigneur : J’ai assaini ces eaux ; il n’y aura plus en elles ni mort ni stérilité. » 
${}^{22}Et les eaux furent assainies jusqu’à ce jour, selon la parole qu’Élisée avait dite. 
${}^{23}De là il se rendit à Béthel. Comme il montait par le chemin, des gamins sortirent de la ville et se moquèrent de lui, en lui disant : « Vas-y, le chauve ! Vas-y, le chauve ! » 
${}^{24}Élisée se retourna, les regarda et les maudit au nom du Seigneur. Alors deux ourses sortirent du bois et déchiquetèrent quarante-deux des enfants. 
${}^{25}De là il se rendit au mont Carmel, puis revint à Samarie.
      
         
      \bchapter{}
      \begin{verse}
${}^{1}Joram, fils d’Acab, devint roi sur Israël, à Samarie, la dix-huitième année du règne de Josaphat, roi de Juda. Il régna douze ans. 
${}^{2}Il fit ce qui est mal aux yeux du Seigneur, mais pas autant que son père et sa mère, car il supprima la stèle de Baal que son père avait dressée. 
${}^{3}Toutefois il resta attaché aux péchés que Jéroboam, fils de Nebath, avait fait commettre à Israël ; il ne s’en écarta pas.
      
         
${}^{4}Mésha, roi de Moab, était éleveur de troupeaux et payait en tribut au roi d’Israël cent mille agneaux et cent mille béliers avec leur laine. 
${}^{5}Mais, à la mort d’Acab, le roi de Moab se révolta contre le roi d’Israël. 
${}^{6}Ce jour-là, le roi Joram sortit de Samarie et passa en revue tout Israël. 
${}^{7}Puis il partit et envoya dire à Josaphat, roi de Juda : « Le roi de Moab s’est révolté contre moi. Viendras-tu avec moi pour combattre Moab ? » Josaphat répondit : « Je monterai. Ce sera pour moi comme pour toi, pour mon peuple comme pour ton peuple, pour mes chevaux comme pour tes chevaux. » 
${}^{8}Il ajouta : « Par quel chemin monterons-nous ? » Joram reprit : « Par le chemin du désert d’Édom. »
${}^{9}Ainsi partirent le roi d’Israël, le roi de Juda et le roi d’Édom. Après sept jours de chemin, l’eau vint à manquer pour la troupe et pour les bêtes de somme qui suivaient. 
${}^{10}Le roi d’Israël dit alors : « Malheur ! Le Seigneur a donc convoqué les trois rois pour les livrer aux mains de Moab. » 
${}^{11}Josaphat demanda : « N’y a-t-il pas ici un prophète du Seigneur, par qui nous puissions consulter le Seigneur ? » Un des serviteurs du roi d’Israël répondit : « Il y a ici Élisée, fils de Shafath, qui versait l’eau sur les mains d’Élie. » 
${}^{12}Josaphat dit alors : « La parole du Seigneur est avec lui ! » Le roi d’Israël ainsi que Josaphat et le roi d’Édom descendirent vers lui. 
${}^{13}Élisée dit au roi d’Israël : « Que me veux-tu ? Va trouver les prophètes de ton père et les prophètes de ta mère. » Le roi d’Israël lui répondit : « Non ! Car le Seigneur a convoqué ces trois rois pour les livrer aux mains de Moab. » 
${}^{14}Élisée reprit : « Par la vie du Seigneur de l’univers devant qui je me tiens, si je n’avais égard à Josaphat, roi de Juda, je ne te prêterais aucune attention, je ne te regarderais pas ! 
${}^{15}Maintenant, amenez-moi un musicien. » Dès que le musicien jouait, la main du Seigneur était sur Élisée. 
${}^{16}Celui-ci déclara : « Ainsi parle le Seigneur : Creusez dans ce ravin des fosses et des fosses. 
${}^{17}Car ainsi parle le Seigneur : Le vent, vous ne le verrez pas ; la pluie, vous ne la verrez pas, et pourtant l’eau emplira ce ravin ; et vous boirez, vous, vos troupeaux et vos bêtes de somme. 
${}^{18}Encore est-ce trop peu aux yeux du Seigneur : il va livrer Moab entre vos mains. 
${}^{19}Vous abattrez toutes les villes fortifiées, toutes les villes importantes ; tous les bons arbres, vous les couperez ; toutes les sources, vous les comblerez ; tous les champs fertiles, vous les dévasterez en y jetant des pierres. » 
${}^{20}Or, au matin, à l’heure de l’offrande, voici que l’eau arriva par le chemin d’Édom, et la terre en fut inondée.
${}^{21}Tous les gens de Moab avaient appris que les rois étaient montés pour les combattre. On avait convoqué tous ceux qui avaient l’âge de porter les armes, et même ceux qui l’avaient passé. Ils avaient pris position sur la frontière. 
${}^{22}Au matin, quand ils se levèrent, le soleil brillait sur les eaux ; les gens de Moab virent devant eux les eaux rouges comme le sang. 
${}^{23}Ils dirent : « C’est du sang ! Sûrement, les rois se sont entre-tués à coups d’épée, ils se sont frappés l’un l’autre. Maintenant, au pillage, Moab ! »
${}^{24}Ils s’approchèrent du camp d’Israël. Ceux d’Israël se dressèrent, ils frappèrent ceux de Moab qui s’enfuirent devant eux, et ils les pourchassèrent. 
${}^{25}Ils démolirent les villes, jetèrent chacun sa pierre dans les champs fertiles et les en recouvrirent ; toutes les sources, ils les comblèrent ; tous les bons arbres, ils les coupèrent. Finalement il ne resta debout que les murs de Qir-Harèsheth. Les porteurs de fronde encerclèrent la ville et la frappèrent. 
${}^{26}Le roi de Moab vit que le combat était trop fort pour lui ; il prit avec lui sept cents hommes portant l’épée, pour tenter une percée vers le roi d’Édom, mais en vain. 
${}^{27}Il prit alors son fils aîné, qui devait régner après lui, et l’offrit en holocauste sur le rempart. Une grande colère vint sur Israël, qui leva le camp et retourna dans son pays.
      
         
      \bchapter{}
      \begin{verse}
${}^{1}La femme d’un des frères-prophètes implora Élisée en disant : « Ton serviteur, mon mari, est mort. Tu sais que ton serviteur craignait le Seigneur. Or le créancier est venu prendre pour lui mes deux enfants comme esclaves. » 
${}^{2}Élisée lui demanda : « Que puis-je faire pour toi ? Dis-moi ce que tu as dans ta maison. » Elle répondit : « Ta servante n’a rien du tout dans sa maison, juste un peu d’huile comme parfum. » 
${}^{3}Il reprit : « Va, emprunte au-dehors des vases à tous tes voisins, des vases vides. Et pas en petit nombre ! 
${}^{4}Puis, rentre chez toi, ferme la porte sur toi et sur tes fils, verse de l’huile dans tous ces vases. Une fois qu’ils seront pleins, mets-les de côté. » 
${}^{5}Elle le quitta, ferma la porte sur elle et sur ses fils. Ceux-ci lui apportaient les vases, et elle y versait de l’huile. 
${}^{6}Lorsque les vases furent remplis, elle dit à son fils : « Apporte-moi encore un vase ! » Il lui répondit : « Il n’y a plus de vase ! » Alors l’huile cessa de couler. 
${}^{7}Elle vint informer l’homme de Dieu, qui lui dit : « Va vendre l’huile et acquitte ta dette ; tu vivras du reste, toi et tes fils ! »
      
         
${}^{8}Un jour, Élisée passait à Sunam ; une femme riche de ce pays insista pour qu’il vienne manger chez elle. Depuis, chaque fois qu’il passait par là, il allait manger chez elle. 
${}^{9} Elle dit à son mari : « Écoute, je sais que celui qui s’arrête toujours chez nous est un saint homme de Dieu. 
${}^{10} Faisons-lui une petite chambre sur la terrasse ; nous y mettrons un lit, une table, un siège et une lampe, et quand il viendra chez nous, il pourra s’y retirer. »
${}^{11}Le jour où il revint, il se retira dans cette chambre pour y coucher. 
${}^{12}Élisée dit à Guéhazi, son serviteur : « Appelle notre Sunamite ! » Guéhazi l’appela, et elle se tint devant lui. 
${}^{13}Élisée reprit : « Dis-lui donc : Voici que tu t’es donné beaucoup de peine pour nous. Que peut-on faire pour toi ? Faut-il parler pour toi au roi ou au chef de l’armée ? » Mais elle répondit : « Je vis tranquille au milieu des miens. » 
${}^{14}Puis il dit à son serviteur\\ : « Que peut-on faire pour cette femme\\ ? » Le serviteur\\répondit : « Hélas, elle n’a pas de fils, et son mari est âgé. » 
${}^{15}Élisée lui dit : « Appelle-la. » Le serviteur l’appela et elle se présenta à la porte. 
${}^{16}Élisée lui dit : « À cette même époque, au temps fixé pour la naissance\\, tu tiendras un fils dans tes bras. » Mais elle dit : « Non, mon seigneur, homme de Dieu, ne dis pas de mensonge à ta servante. » 
${}^{17}Or, la femme devint enceinte et, l’année suivante, à la même époque, elle enfanta un fils, au moment prédit par Élisée.
${}^{18}L’enfant grandit. Un jour que l’enfant était allé trouver son père auprès des moissonneurs, 
${}^{19}il se mit à crier : « Oh ! ma tête ! ma tête ! » Le père dit à un serviteur : « Porte-le à sa mère. » 
${}^{20}Le serviteur emporta l’enfant et le remit à sa mère. Celle-ci garda l’enfant sur ses genoux jusqu’à midi, puis il mourut. 
${}^{21}Alors elle monta l’étendre sur le lit de l’homme de Dieu, ferma la porte et sortit. 
${}^{22}Elle appela son mari et dit : « Envoie-moi, je te prie, un des serviteurs et une des ânesses ; je cours jusque chez l’homme de Dieu et je reviens. » 
${}^{23}Il dit : « Pourquoi vas-tu chez lui aujourd’hui ? Ce n’est pas une nouvelle lune ni un sabbat. » Elle répondit : « Ne t’inquiète pas ! » 
${}^{24}Elle fit donc seller l’ânesse et dit à son serviteur : « Conduis-moi, vas-y ! N’arrête ma course que lorsque je te le dirai ! » 
${}^{25}Elle partit et se rendit auprès de l’homme de Dieu, au mont Carmel. Quand l’homme de Dieu la vit venir, il dit à Guéhazi, son serviteur : « Voici notre Sunamite ! 
${}^{26}Maintenant, cours à sa rencontre et dis-lui : “Comment vas-tu ? Comment va ton mari ? Comment va ton enfant ?” » Elle répondit : « Tout va bien ! » 
${}^{27}Arrivée auprès de l’homme de Dieu sur la montagne, elle saisit ses pieds. Guéhazi s’avança pour la repousser, mais l’homme de Dieu dit : « Laisse-la, car son âme est dans l’amertume. Le Seigneur me l’a caché, il ne m’a rien annoncé. » 
${}^{28}Elle dit : « Avais-je demandé un fils à mon seigneur ? N’avais-je pas dit : Ne me donne pas de faux espoir ? » 
${}^{29}Il dit à Guéhazi : « Boucle ta ceinture, prends mon bâton dans ta main et va ! Si tu rencontres un homme, ne le salue pas ! Si un homme te salue, ne lui réponds pas ! Tu mettras mon bâton sur le visage du garçon. » 
${}^{30}Mais la mère du garçon reprit : « Par le Seigneur qui est vivant, et par ta vie, je ne te quitterai pas. » Alors il se leva et la suivit. 
${}^{31}Guéhazi les avait précédés. Il avait mis le bâton sur le visage de l’enfant, mais pas le moindre son, aucun signe de vie ! Il revint au-devant d’Élisée et lui annonça : « Le garçon ne s’est pas réveillé ! »
${}^{32}Quand Élisée arriva dans la maison, il trouva l’enfant mort, étendu sur le lit. 
${}^{33} Il entra, ferma la porte pour être seul avec lui, et il se mit à prier le Seigneur. 
${}^{34} Il monta sur le lit, se coucha sur l’enfant, mit sa bouche sur sa bouche, ses yeux sur ses yeux et ses mains sur ses mains. Il resta étendu sur lui, et le corps de l’enfant se réchauffa. 
${}^{35} Le prophète redescendit et marcha de long en large dans la maison. Puis il remonta s’étendre sur l’enfant. Celui-ci éternua sept fois, et ouvrit les yeux. 
${}^{36} Élisée appela son serviteur et lui dit : « Fais venir sa mère\\. » Le serviteur la fit venir. Lorsqu’elle arriva auprès de lui, Élisée lui dit : « Reprends ton fils. » 
${}^{37} Elle entra, tomba à ses pieds et se prosterna jusqu’à terre. Elle reprit son fils et sortit.
${}^{38}Élisée revint à Guilgal. La famine était dans le pays. Comme les frères-prophètes étaient assis devant lui, il dit à son serviteur : « Prépare la grande marmite et fais cuire une soupe pour les frères-prophètes. » 
${}^{39}L’un de ceux-ci sortit dans la campagne pour ramasser des herbes. Il trouva une vigne sauvage, y cueillit des coloquintes sauvages, plein son vêtement, puis il revint et les coupa en morceaux dans la marmite de soupe, car on ne savait pas ce que c’était. 
${}^{40}On servit à manger aux hommes, mais dès qu’ils eurent mangé de la soupe, ils ne poussèrent qu’un cri : « La mort est dans la marmite, homme de Dieu ! » Et ils ne purent manger. 
${}^{41}Élisée dit : « Apportez de la farine. » Il la jeta dans la marmite et dit : « Sers les gens, et qu’ils mangent ! » Il n’y avait plus rien de mauvais dans la marmite.
${}^{42}Un homme\\vint de Baal-Shalisha et, prenant sur la récolte nouvelle\\, il apporta à Élisée\\, l’homme de Dieu, vingt pains d’orge et du grain frais dans un sac. Élisée dit alors : « Donne-le à tous ces gens pour qu’ils mangent. » 
${}^{43} Son serviteur répondit : « Comment donner cela à cent personnes\\ ? » Élisée reprit : « Donne-le à tous ces gens pour qu’ils mangent, car ainsi parle\\le Seigneur : On mangera, et il en restera. » 
${}^{44} Alors, il le leur donna, ils mangèrent, et il en resta, selon la parole du Seigneur.
      
         
      \bchapter{}
      \begin{verse}
${}^{1}Naaman, général de l’armée du roi d’Aram, était un homme de grande valeur et hautement estimé par son maître, car c’est par lui que le Seigneur avait donné la victoire au royaume\\d’Aram. Or, ce\\vaillant guerrier était lépreux. 
${}^{2} Des Araméens, au cours d’une expédition en terre d’Israël, avaient fait prisonnière une fillette qui fut mise au service de la femme de Naaman. 
${}^{3} Elle dit à sa maîtresse : « Ah ! si mon maître s’adressait au prophète qui est à Samarie, celui-ci le délivrerait de sa lèpre. » 
${}^{4} Naaman alla auprès du roi et lui dit : « Voilà ce que la jeune fille d’Israël a déclaré. » 
${}^{5} Le roi d’Aram lui répondit : « Va, mets-toi en route. J’envoie une lettre au roi d’Israël. » Naaman partit donc ; il emportait dix lingots d’argent, six mille pièces d’or\\et dix vêtements de fête. 
${}^{6} Il remit la lettre au roi d’Israël. Celle-ci portait : « En même temps que te parvient cette lettre, je t’envoie Naaman mon serviteur, pour que tu le délivres de sa lèpre. » 
${}^{7} Quand le roi d’Israël lut ce message, il déchira ses vêtements et s’écria : « Est-ce que je suis Dieu, maître de la vie et de la mort ? Ce roi m’envoie un homme pour que je le délivre de sa lèpre ! Vous le voyez bien : c’est une provocation ! »
${}^{8}Quand Élisée, l’homme de Dieu, apprit que le roi d’Israël avait déchiré ses vêtements, il lui fit dire : « Pourquoi as-tu déchiré tes vêtements ? Que cet homme vienne à moi, et il saura qu’il y a un prophète en Israël. » 
${}^{9} Naaman arriva avec ses chevaux et son char, et s’arrêta à la porte de la maison d’Élisée. 
${}^{10} Élisée envoya un messager lui dire : « Va te baigner sept fois dans le Jourdain, et ta chair redeviendra nette, tu seras purifié. » 
${}^{11} Naaman se mit en colère et s’éloigna en disant : « Je m’étais dit : Sûrement il va sortir, et se tenir debout pour invoquer le nom du Seigneur son Dieu ; puis il agitera sa main au-dessus de l’endroit malade et guérira ma lèpre. 
${}^{12} Est-ce que les fleuves de Damas, l’Abana et le Parpar, ne valent pas mieux que toutes les eaux d’Israël ? Si je m’y baignais, est-ce que je ne serais pas purifié ? » Il tourna bride et partit en colère. 
${}^{13} Mais ses serviteurs s’approchèrent pour lui dire : « Père ! Si le prophète t’avait ordonné quelque chose de difficile, tu l’aurais fait, n’est-ce pas ? Combien plus, lorsqu’il te dit : “Baigne-toi, et tu seras purifié.” » 
${}^{14} Il descendit jusqu’au Jourdain et s’y plongea sept fois, pour obéir à la parole de l’homme de Dieu ; alors sa chair redevint semblable à celle d’un petit enfant : il était purifié !
${}^{15}Il retourna chez l’homme de Dieu avec toute son escorte ; il entra, se présenta devant lui et déclara : « Désormais, je le sais : il n’y a pas d’autre Dieu, sur toute la terre, que celui d’Israël ! Je t’en prie, accepte un présent de ton serviteur. » 
${}^{16}Mais Élisée répondit : « Par la vie du Seigneur que je sers, je n’accepterai rien. » Naaman le pressa d’accepter, mais il refusa. 
${}^{17}Naaman dit alors : « Puisque c’est ainsi, permets que ton serviteur emporte de la terre de ce pays\\autant que deux mulets peuvent en transporter, car je ne veux plus offrir ni holocauste ni sacrifice à d’autres dieux qu’au Seigneur Dieu d’Israël\\. 
${}^{18}Mais que le Seigneur pardonne à ton serviteur le geste que voici : lorsque mon maître entre dans le temple de Rimmone pour s’y prosterner, et qu’il s’appuie sur ma main, je me prosterne aussi dans le temple de Rimmone. Daigne le Seigneur pardonner ce geste à ton serviteur. » 
${}^{19}Élisée lui dit : « Va en paix. » Et Naaman s’éloigna.
      Il était à une certaine distance 
${}^{20}quand Guéhazi, serviteur d’Élisée, l’homme de Dieu, se dit : « Voici que mon maître a ménagé Naaman, cet Araméen, en n’acceptant pas de sa main ce qu’il avait apporté. Par le Seigneur qui est vivant, je vais courir derrière lui, et j’obtiendrai de lui quelque chose. » 
${}^{21}Guéhazi s’élança derrière Naaman. Celui-ci le vit courir derrière lui et descendit de son char pour aller à sa rencontre. Il lui dit : « Tout va bien ? » 
${}^{22}L’autre répondit : « Tout va bien ! Mon maître m’a envoyé te dire ceci : “À l’instant, voici qu’arrivent auprès de moi deux jeunes gens venant de la montagne d’Éphraïm, de chez les frères-prophètes. Donne pour eux, je te prie, un lingot d’argent et deux vêtements de fête.” » 
${}^{23}Naaman répondit : « Accepte de prendre deux lingots. » Et il insista auprès de lui. Naaman prit deux lingots d’argent qu’il enferma avec deux habits dans deux sacoches. Il les remit à deux de ses serviteurs qui les portèrent en précédant Guéhazi. 
${}^{24}Arrivé à l’Ophel de Samarie, celui-ci prit le tout de leurs mains et le déposa chez lui ; il renvoya les hommes, qui s’en allèrent.
${}^{25}Puis il vint se présenter devant son maître. Élisée lui demanda : « D’où viens-tu, Guéhazi ? » Il répondit : « Ton serviteur n’est allé nulle part. » 
${}^{26}Élisée lui dit : « Mon cœur n’était-il pas avec toi, lorsqu’un homme s’est précipité de son char pour aller à ta rencontre ? Est-ce le moment de prendre de l’argent, de prendre des vêtements, des oliviers et des vignes, du petit et du gros bétail, des serviteurs et des servantes, 
${}^{27}alors que la lèpre de Naaman va s’attacher à toi et à ta descendance pour toujours ? » Et Guéhazi se retira, lépreux, couleur de neige.
      
         
      \bchapter{}
      \begin{verse}
${}^{1}Les frères-prophètes dirent à Élisée : « Vois, le lieu où nous nous tenons assis devant toi est trop étroit pour nous ! 
${}^{2}Allons donc jusqu’au Jourdain : chacun de nous y prendra une poutre et faisons-nous là-bas un lieu pour nous y tenir assis. » Il répondit : « Allez ! » 
${}^{3}L’un d’eux ajouta : « Accepte, je te prie, de venir avec tes serviteurs. » Il répondit : « Je viens ! » 
${}^{4}Puis il partit avec eux. Arrivés au Jourdain, ils coupèrent des arbres. 
${}^{5}Comme l’un d’eux abattait une poutre, le fer de l’outil tomba dans l’eau. Il s’écria : « Ah ! mon seigneur, on me l’avait prêté ! » 
${}^{6}L’homme de Dieu dit : « Où est-il tombé ? » L’autre lui montra l’endroit. Élisée cassa un morceau de bois, l’y jeta, et le fer surnagea. 
${}^{7}Il dit : « Retire-le ! » L’homme étendit la main et le prit.
      
         
${}^{8}Le roi d’Aram était en guerre avec Israël. Il tint conseil avec ses serviteurs et dit : « À tel endroit sera mon campement ! » 
${}^{9}L’homme de Dieu envoya dire au roi d’Israël : « Garde-toi de passer par cet endroit-là, car les Araméens y descendent ! » 
${}^{10}Le roi d’Israël envoya des gens à l’endroit dont l’homme de Dieu lui avait parlé. Ainsi averti, il se tint sur ses gardes, et cela, plutôt deux fois qu’une !
${}^{11}Le cœur du roi d’Aram fut troublé par cette affaire. Il appela ses serviteurs et leur dit : « Ne pouvez-vous pas me faire savoir qui d’entre nous est pour le roi d’Israël ? » 
${}^{12}Un des serviteurs lui répondit : « Personne, mon seigneur le roi, mais c’est Élisée, le prophète en Israël, qui fait savoir au roi d’Israël toutes les paroles que tu dis dans ta chambre à coucher. » 
${}^{13}Le roi répondit : « Allez voir où il se trouve, et j’enverrai le prendre. » On lui fit savoir : « Le voici à Dotane ! » 
${}^{14}Le roi envoya là-bas des chevaux, des chars et une troupe importante. Ils arrivèrent de nuit et encerclèrent la ville. 
${}^{15}Le serviteur de l’homme de Dieu se leva de bon matin et sortit. Et voici qu’une troupe nombreuse entourait la ville, avec des chevaux et des chars. Le serviteur dit à Élisée : « Ah ! Mon seigneur, comment allons-nous faire ? » 
${}^{16}Élisée répondit : « N’aie pas peur ! Car ceux qui sont avec nous sont plus nombreux que ceux qui sont avec eux ! » 
${}^{17}Et il pria en disant : « Seigneur, daigne lui ouvrir les yeux, et qu’il voie ! » Le Seigneur ouvrit les yeux du serviteur, et celui-ci vit la montagne couverte de chevaux et de chars de feu tout autour d’Élisée.
${}^{18}Comme les Araméens descendaient vers lui, Élisée pria le Seigneur en ces termes : « Daigne frapper d’aveuglement cette nation ! » Et le Seigneur les frappa d’aveuglement, selon la parole d’Élisée. 
${}^{19}Puis Élisée leur dit : « Ce n’est pas le chemin, et ce n’est pas la ville. Suivez-moi, je vous conduirai vers l’homme que vous cherchez. » Mais c’est à Samarie qu’il les conduisit. 
${}^{20}Comme ils entraient à Samarie, Élisée dit : « Seigneur, ouvre les yeux de ces gens, et qu’ils voient ! » Le Seigneur leur ouvrit les yeux, et ils virent : voici qu’ils étaient au milieu de Samarie ! 
${}^{21}Lorsqu’il les vit, le roi d’Israël dit à Élisée : « Mon père, alors, faut-il les tuer ? » 
${}^{22}Il répondit : « Tu ne les tueras pas. As-tu coutume de tuer ceux que tu as faits prisonniers par ton épée et par ton arc ? Fais-leur servir du pain et de l’eau. Qu’ils mangent, qu’ils boivent, et qu’ils s’en aillent chez leur maître ! » 
${}^{23}Le roi leur fit servir un grand repas : ils mangèrent et ils burent. Puis il les renvoya, et ils s’en allèrent chez leur maître. Et les bandes araméennes ne revinrent plus sur la terre d’Israël.
${}^{24}À quelque temps de là, Ben-Hadad, roi d’Aram, rassembla toute son armée et monta assiéger Samarie. 
${}^{25}Il y eut à Samarie une grande famine : le siège fut si rude qu’une tête d’âne coûtait quatre-vingts pièces d’argent, et un quart de mesure de fiente de pigeon, cinq pièces d’argent.
${}^{26}Or, comme le roi d’Israël passait sur le rempart, une femme lui cria : « Au secours, mon seigneur le roi ! » 
${}^{27}Il dit : « Non ! Que le Seigneur te secoure ! Avec quoi pourrais-je, moi, te secourir ? Avec les produits de l’aire à grain ou du pressoir ? » 
${}^{28}Le roi lui dit encore : « Que veux-tu ? » Elle répondit : « Cette femme-là m’a dit : “Donne ton fils, pour que nous le mangions aujourd’hui, et demain c’est le mien que nous mangerons.” 
${}^{29}Alors nous avons fait cuire mon fils et nous l’avons mangé. Je lui ai dit le jour suivant : “Donne ton fils, que nous le mangions.” Mais elle l’avait caché ! » 
${}^{30}Quand le roi entendit les paroles de cette femme, il déchira ses vêtements, et comme il passait sur le rempart, le peuple vit qu’il portait en dessous, à même la peau, une toile à sac. 
${}^{31}Le roi dit : « Que Dieu amène le malheur sur moi, et pire encore, si la tête d’Élisée, fils de Shafath, reste aujourd’hui sur ses épaules ! »
${}^{32}Élisée était assis dans sa maison, et les anciens étaient assis avec lui. Le roi envoya un de ses hommes, mais, avant que le messager n’arrive jusqu’à lui, Élisée dit aux anciens : « Vous l’avez vu ? Ce fils d’assassin a envoyé quelqu’un pour me couper la tête ! Attention ! Dès que le messager arrivera, fermez la porte, repoussez-le avec la porte. N’est-ce pas, derrière lui, le bruit des pas de son maître ? » 
${}^{33}Il parlait encore que déjà le messager descendait vers lui. Alors Élisée dit : « Voici le malheur qui vient du Seigneur ! Que puis-je encore espérer du Seigneur ? »
      
         
      \bchapter{}
      \begin{verse}
${}^{1}Élisée dit : « Écoutez la parole du Seigneur. Ainsi parle le Seigneur : Demain, à la porte de Samarie, exactement à la même heure, on aura pour une pièce d’argent une mesure de fleur de farine ou deux mesures d’orge. » 
${}^{2}L’écuyer du roi, celui sur la main duquel il s’appuie, répondit à l’homme de Dieu : « Même si le Seigneur ouvrait des fenêtres dans les cieux, cette parole s’accomplirait-elle ? » Élisée dit : « Eh bien ! Tu le verras de tes yeux, mais tu n’en mangeras pas ! »
      
         
${}^{3}Il y avait devant la porte de Samarie quatre hommes qui étaient lépreux. Ils se dirent l’un à l’autre : « Pourquoi restons-nous ici à attendre la mort ? 
${}^{4}Si nous décidons d’entrer dans la ville, la famine étant dans la ville, nous y mourrons. Si nous restons ici, nous mourrons également. Allez ! Passons au camp des Araméens. S’ils nous laissent en vie, nous vivrons ; s’ils nous mettent à mort, eh bien, nous mourrons. » 
${}^{5}Au crépuscule, ils se mirent en route, pour se rendre au camp des Araméens. Ils allèrent jusqu’à l’extrémité du camp, et voilà qu’il n’y avait plus personne ! 
${}^{6}Le Seigneur avait fait entendre dans le camp des Araméens un bruit de chars, un bruit de chevaux, le bruit d’une grande troupe, et ils s’étaient dit l’un à l’autre : « Voici que le roi d’Israël a pris à sa solde les rois des Hittites et les rois d’Égypte, pour marcher contre nous. » 
${}^{7}Au crépuscule, les Araméens s’étaient mis en route et avaient pris la fuite, abandonnant leurs tentes, leurs chevaux et leurs ânes, en un mot, le camp tel qu’il était ; ils s’étaient enfuis pour sauver leur vie. 
${}^{8}Les lépreux allèrent jusqu’à l’extrémité du camp et entrèrent dans une tente. Après avoir mangé et bu, ils emportèrent de là argent, or et vêtements, qu’ils allèrent cacher. Puis ils revinrent, entrèrent dans une autre tente et en emportèrent du butin qu’ils allèrent cacher.
${}^{9}Alors ils se dirent l’un à l’autre : « Ce n’est pas bien, ce que nous faisons là ! Ce jour est un jour de bonne nouvelle. Si nous nous taisons et si nous attendons jusqu’à la lumière du jour, une faute pèsera sur nous. Allez ! Rentrons pour informer la Maison du roi ! » 
${}^{10}Ils revinrent et appelèrent les gardiens de la porte de la ville ; ils les informèrent en disant : « Nous sommes entrés dans le camp des Araméens, et voici qu’il n’y avait personne, aucune voix humaine : seulement les chevaux et les ânes attachés, ainsi que des tentes laissées telles quelles. » 
${}^{11}Les gardiens de la porte crièrent, et on informa la Maison du roi, à l’intérieur.
${}^{12}Le roi se leva de nuit et dit à ses serviteurs : « Il faut que je vous explique ce que les Araméens nous ont fait. Ils savent en effet que nous sommes affamés. Ils sont donc sortis du camp pour se cacher dans la campagne. Ils se sont dit : “Les gens de Samarie sortiront de la ville, nous les prendrons vivants et nous entrerons dans la ville.” » 
${}^{13}Un des serviteurs répondit : « Que l’on prenne donc cinq chevaux sur les derniers qui restent dans la ville ! Ils sont comme toute la multitude d’Israël qui reste dans la ville, comme toute la multitude d’Israël qui est proche de sa fin. Envoyons-les et nous verrons. » 
${}^{14}On prit donc deux chars avec leurs chevaux, et le roi les envoya sur les traces de l’armée des Araméens, en disant : « Allez et voyez ! » 
${}^{15}Ils partirent sur leurs traces jusqu’au Jourdain ; or tout le chemin était jonché de vêtements et d’ustensiles, jetés par les Araméens dans leur fuite précipitée. Les messagers revinrent en informer le roi.
${}^{16}Alors le peuple sortit et pilla le camp des Araméens. On eut, pour une pièce d’argent, une mesure de fleur de farine ou deux mesures d’orge, selon la parole du Seigneur. 
${}^{17}À la porte, le roi avait posté l’écuyer, celui sur la main duquel il s’appuie. Les gens le piétinèrent à la porte, et il mourut, comme l’avait dit l’homme de Dieu, lorsque le roi était descendu vers lui. 
${}^{18}Tout se passa selon la parole que l’homme de Dieu avait dite au roi : « On aura pour une pièce d’argent deux mesures d’orge ou une mesure de fleur de farine, demain, exactement à la même heure, à la porte de Samarie. » 
${}^{19}L’écuyer avait répondu à l’homme de Dieu : « Même si le Seigneur ouvrait des fenêtres dans les cieux, cette parole-là s’accomplirait-elle ? » Élisée lui avait dit : « Eh bien ! Tu le verras de tes yeux, mais tu n’en mangeras pas ! » 
${}^{20}C’est ce qui lui arriva. Les gens le piétinèrent à la porte, et il mourut.
      
         
      \bchapter{}
      \begin{verse}
${}^{1}Élisée avait dit à la femme dont il avait fait revivre le fils : « Lève-toi, pars, toi et ta famille, séjourne où tu pourras, car le Seigneur a appelé la famine, et même, elle vient sur le pays pour sept ans. » 
${}^{2}La femme se leva et agit selon la parole de l’homme de Dieu ; elle partit, elle et sa famille, et séjourna au pays des Philistins pendant sept ans.
${}^{3}Sept années passèrent, et la femme revint du pays des Philistins. Elle alla implorer le roi au sujet de sa maison et de son champ. 
${}^{4}Or le roi était en conversation avec Guéhazi, le serviteur de l’homme de Dieu. Il lui avait dit : « Raconte-moi donc toutes les grandes actions accomplies par Élisée. » 
${}^{5}Guéhazi était en train de raconter au roi comment Élisée avait fait revivre le mort, quand, justement, la femme dont il avait fait revivre le fils vint implorer le roi au sujet de sa maison et de son champ. Guéhazi dit alors : « Mon seigneur le roi, voici la femme et son fils qu’Élisée a fait revivre. » 
${}^{6}Le roi interrogea la femme, qui lui en fit le récit. Il mit à sa disposition un de ses dignitaires et dit à celui-ci : « Fais-lui restituer tout ce qui lui appartient, avec tous les revenus de ce champ, depuis le jour où elle a quitté le pays jusqu’à maintenant. »
${}^{7}Élisée se rendit à Damas. Ben-Hadad, roi d’Aram, était malade. On l’informa en disant : « L’homme de Dieu est venu jusqu’ici. » 
${}^{8}Le roi dit à Hazaël : « Prends avec toi un cadeau et va trouver l’homme de Dieu. Par lui, tu consulteras le Seigneur, en disant : “Sortirai-je vivant de cette maladie ?” » 
${}^{9}Hazaël alla trouver Élisée avec, en cadeau, tout ce qu’il y avait de meilleur à Damas, de quoi charger quarante chameaux. Il entra, se tint devant Élisée et dit : « Ton fils Ben-Hadad, roi d’Aram, m’envoie te demander : “Sortirai-je vivant de cette maladie ?” » 
${}^{10}Élisée lui répondit : « Va lui dire : “C’est sûr, tu vivras”. Mais le Seigneur m’a fait voir qu’en réalité, il mourra. » 
${}^{11}Le visage de l’homme de Dieu devint immobile, complètement figé, et il pleura. 
${}^{12}Hazaël demanda : « Pourquoi mon seigneur pleure-t-il ? » Il dit : « Parce que je sais le mal que tu feras aux fils d’Israël : leurs villes fortes, tu les livreras au feu ; leurs jeunes gens, tu les tueras par l’épée ; leurs petits enfants, tu les écraseras ; leurs femmes enceintes, tu les éventreras. » 
${}^{13}Hazaël reprit : « Mais qu’est donc ton serviteur, ce chien, pour faire ces atrocités ? » Élisée répondit : « Dans une vision du Seigneur, je t’ai vu roi d’Aram. » 
${}^{14}Hazaël quitta Élisée et retourna chez son maître. Le roi lui demanda : « Que t’a dit Élisée ? » Il répondit : « Il m’a dit : “C’est sûr, tu vivras”. » 
${}^{15}Mais le lendemain Hazaël prit une couverture, la plongea dans l’eau et l’appliqua sur le visage du roi, qui mourut. Hazaël régna à sa place.
${}^{16}La cinquième année du règne de Joram, fils d’Acab, roi d’Israël – Josaphat étant roi de Juda –, Joram, fils de Josaphat, devint roi. 
${}^{17}Il avait trente-deux ans lorsqu’il devint roi, et il régna huit ans à Jérusalem. 
${}^{18}Il marcha dans le chemin des rois d’Israël, comme avait fait la maison d’Acab, car il avait pour femme une fille d’Acab. Et il fit ce qui est mal aux yeux du Seigneur. 
${}^{19}Mais le Seigneur ne voulut pas détruire Juda, à cause de la promesse faite à David son serviteur, de lui donner, à lui et à ses fils, une lampe pour toujours.
${}^{20}Du temps de Joram, le pays d’Édom se révolta contre la domination de Juda et se donna un roi. 
${}^{21}Joram passa à l’endroit appelé Saïr, et tous ses chars avec lui. S’étant levé de nuit, il battit les Édomites qui l’encerclaient, lui et les commandants de chars ; et le peuple s’enfuit vers ses tentes. 
${}^{22}Édom se révolta contre la domination de Juda, et cela jusqu’à ce jour. En ce temps-là, la ville de Libna se révolta.
${}^{23}Le reste des actions de Joram et tout ce qu’il a fait,
        \\cela n’est-il pas écrit dans le livre des Annales des rois de Juda ?
${}^{24}Joram reposa avec ses pères,
        \\il fut enseveli avec eux dans la Cité de David.
        \\Son fils Ocozias régna à sa place.
${}^{25}La douzième année du règne de Joram, fils d’Acab, roi d’Israël, Ocozias, fils de Joram, roi de Juda, devint roi. 
${}^{26}Ocozias avait vingt-deux ans lorsqu’il devint roi, et il régna un an à Jérusalem. Le nom de sa mère était Athalie, fille d’Omri, roi d’Israël. 
${}^{27}Il marcha dans le chemin de la maison d’Acab, et il fit ce qui est mal aux yeux du Seigneur, comme la maison d’Acab, car il était apparenté à la maison d’Acab.
${}^{28}Il partit avec Joram, fils d’Acab, pour combattre à Ramoth-de-Galaad Hazaël, roi d’Aram. Mais les Araméens blessèrent Joram. 
${}^{29}Le roi Joram revint à Yizréel se faire soigner des blessures que les Araméens lui avaient faites dans le combat contre Hazaël, roi d’Aram. Ocozias, fils de Joram, roi de Juda, descendit à Yizréel pour voir Joram, fils d’Acab, parce qu’il était blessé.
      
         
      \bchapter{}
      \begin{verse}
${}^{1}Le prophète Élisée appela un des frères-prophètes et lui dit : « Boucle ta ceinture, prends cette fiole d’huile dans ta main et va à Ramoth-de-Galaad. 
${}^{2}Arrivé là, tu chercheras à voir Jéhu, fils de Josaphat, fils de Namsi. Tu entreras, tu le feras se lever du milieu de ses frères, et tu le conduiras dans une chambre retirée. 
${}^{3}Tu prendras la fiole d’huile, tu la verseras sur sa tête et tu diras : “Ainsi parle le Seigneur : Je t’ai donné l’onction pour te faire roi sur Israël.” Puis tu ouvriras la porte et tu t’enfuiras sans attendre. » 
${}^{4}Le jeune homme, le jeune prophète, partit pour Ramoth-de-Galaad. 
${}^{5}Lorsqu’il arriva, les chefs de l’armée étaient assis. Il dit : « J’ai un mot à te dire, chef ! » Jéhu demanda : « Auquel d’entre nous ? » Il répondit : « À toi, chef ! » 
${}^{6}Jéhu se leva et entra dans la maison. Le jeune prophète versa l’huile sur sa tête et lui dit : « Ainsi parle le Seigneur, Dieu d’Israël : Je te donne l’onction pour te faire roi sur le peuple du Seigneur, sur Israël. 
${}^{7}Tu frapperas la maison d’Acab, ton maître ; je vengerai ainsi le sang de mes serviteurs les prophètes et le sang de tous les serviteurs du Seigneur, répandu par la main de Jézabel. 
${}^{8}Toute la maison d’Acab périra ; j’exterminerai les mâles de chez Acab, esclaves ou hommes libres en Israël. 
${}^{9}Je ferai à la maison d’Acab ce que j’ai fait à celle de Jéroboam, fils de Nebath, et à celle de Baasa, fils d’Ahias. 
${}^{10}Quant à Jézabel, les chiens la dévoreront dans le champ de Yizréel, et personne ne l’ensevelira. » Puis il ouvrit la porte et s’enfuit.
${}^{11}Jéhu sortit rejoindre les serviteurs de son maître, et on lui dit : « Tout va bien ? Pourquoi cet exalté est-il venu te trouver ? » Il répondit : « Vous connaissez bien l’homme. C’est toujours la même rengaine ! » 
${}^{12}Ils lui dirent : « Mensonge ! Explique-nous donc ! » Il reprit : « On a parlé de choses et d’autres. Puis il m’a dit : “Ainsi parle le Seigneur : Je t’ai donné l’onction pour te faire roi sur Israël.” » 
${}^{13}Ils se hâtèrent de prendre chacun son vêtement et les étendirent sous ses pieds en haut des marches. Puis ils sonnèrent du cor et dirent : « Jéhu est roi ! »
${}^{14}Jéhu, fils de Josaphat, fils de Namsi, conspira contre Joram. Celui-ci, avec tout Israël, défendait alors Ramoth-de-Galaad contre Hazaël, roi d’Aram. 
${}^{15}Mais le roi Joram était retourné se faire soigner à Yizréel des blessures que les Araméens lui avaient faites, dans le combat contre Hazaël, roi d’Aram. Jéhu dit : « Si vous êtes bien d’accord, que personne ne s’échappe de la ville pour aller rapporter l’information à Yizréel ! » 
${}^{16}Jéhu monta sur son char et partit pour Yizréel, puisque Joram s’y trouvait alité. Ocozias, roi de Juda, était descendu voir Joram. 
${}^{17}Le guetteur posté sur la tour de Yizréel vit venir la troupe de Jéhu. Il dit : « Je vois une troupe ». Joram dit : « Prends un cavalier. Envoie-le à leur rencontre pour qu’il demande si tout va bien. » 
${}^{18}Le cavalier partit à leur rencontre et dit : « Ainsi parle le roi : Tout va bien ? » Jéhu répondit : « Que t’importe si tout va bien ? Passe derrière moi. » Le guetteur annonça : « Le messager est arrivé jusqu’à eux, mais il ne revient pas. » 
${}^{19}Le roi envoya un second cavalier qui les rejoignit et dit : « Ainsi parle le roi : Tout va bien ? » Jéhu répondit : « Que t’importe si tout va bien ? Passe derrière moi. » 
${}^{20}Le guetteur annonça : « Il est arrivé jusqu’à eux, mais il ne revient pas. La façon de conduire le char est celle de Jéhu, fils de Namsi, car il conduit comme un fou ! » 
${}^{21}Joram dit alors : « Attelez mon char ! » Et on l’attela. Joram, roi d’Israël, et Ocozias, roi de Juda, sortirent sur leurs chars à la rencontre de Jéhu et le trouvèrent dans le champ de Naboth de Yizréel. 
${}^{22}Lorsque Joram vit Jéhu, il demanda : « Tout va bien, Jéhu ? » Il répondit : « Est-il possible que tout aille bien, aussi longtemps que durent les débauches de ta mère Jézabel et ses nombreuses sorcelleries ? » 
${}^{23}Joram tourna bride et s’enfuit. Il dit à Ocozias : « Trahison, Ocozias ! » 
${}^{24}Jéhu saisit son arc et atteignit Joram entre les épaules. La flèche lui transperça le cœur, et il s’écroula dans son char. 
${}^{25}Jéhu dit à Bidqar, son écuyer : « Enlève-le et jette-le dans le champ de Naboth de Yizréel. Car souviens-toi : nous étions, moi et toi, ensemble sur un char, à la suite d’Acab son père, quand le Seigneur prononça contre lui la sentence que voici : 
${}^{26}“N’ai-je pas vu hier le sang de Naboth et le sang de ses fils, oracle du Seigneur ? Je te le ferai payer dans ce champ – oracle du Seigneur !” Maintenant donc, enlève Joram et jette-le dans ce champ, selon la parole du Seigneur. »
${}^{27}Voyant cela, Ocozias, roi de Juda, s’enfuit par le chemin de Beth-Gane. Jéhu le poursuivit et dit : « Frappez-le, lui aussi ! » On le frappa sur son char, à la montée de Gour, près d’Ibléam. Il s’enfuit à Meguiddo, où il mourut. 
${}^{28}Ses serviteurs le transportèrent sur un char à Jérusalem, et on l’ensevelit dans son tombeau avec ses pères, dans la Cité de David.
${}^{29}La onzième année du règne de Joram, fils d’Acab,
        \\Ocozias était devenu roi sur Juda.
${}^{30}Jéhu entra dans la ville de Yizréel. Jézabel, l’ayant appris, se farda les yeux, apprêta son visage et se pencha par la fenêtre. 
${}^{31}Comme Jéhu franchissait la porte de la ville, elle dit : « Tout va-t-il bien, Zimri, l’assassin de son maître ? » 
${}^{32}Il leva les yeux vers la fenêtre et dit : « Qui est avec moi ? Qui ? » Deux ou trois dignitaires se penchèrent vers lui. 
${}^{33}Il dit : « Jetez-la en bas ! » Et ils la jetèrent. Son sang éclaboussa le mur et les chevaux, et Jéhu la piétina. 
${}^{34}Il entra, mangea et but. Il dit ensuite : « Occupez-vous donc de cette maudite ; ensevelissez-la, car elle est fille de roi ! » 
${}^{35}Ils allèrent pour l’ensevelir, mais on ne retrouva d’elle que le crâne, les pieds et les mains. 
${}^{36}On revint en informer Jéhu. Il dit alors : « C’est bien ce que le Seigneur avait annoncé par l’intermédiaire d’Élie de Tishbé : “Dans le champ de Yizréel, les chiens dévoreront la chair de Jézabel. 
${}^{37}Le cadavre de Jézabel servira de fumier, à la surface du champ, dans le domaine de Yizréel, et nul ne pourra dire : C’est Jézabel !” »
      
         
      \bchapter{}
      \begin{verse}
${}^{1}Acab avait à Samarie soixante-dix fils. Jéhu écrivit des lettres et les envoya à Samarie aux chefs de la ville, aux anciens et aux précepteurs des fils d’Acab, pour leur dire : 
${}^{2}« Au moment où cette lettre vous parvient, vous avez avec vous des fils de votre maître ; vous avez aussi des chars et des chevaux, une ville fortifiée et des armes. 
${}^{3}Voyez qui, parmi les fils de votre maître, est bon et loyal ; placez-le sur le trône de son père et combattez pour la maison de votre maître. » 
${}^{4}Ils eurent très peur et se dirent : « Si les deux rois n’ont pas tenu en face de lui, comment nous-mêmes pourrions-nous tenir ? » 
${}^{5}Le maître du palais et le gouverneur de la ville, les anciens et les précepteurs envoyèrent dire à Jéhu : « Nous sommes tes serviteurs et nous ferons tout ce que tu nous diras, mais nous ne choisirons personne comme roi ! Ce qui est bon à tes yeux, fais-le ! »
      
         
       
${}^{6}Jéhu leur écrivit une seconde lettre pour leur dire : « Si vous êtes pour moi et si vous écoutez ma voix, prenez les têtes des hommes, les fils de votre maître, et venez au-devant de moi à Yizréel, demain, à la même heure. » Or, les soixante-dix fils du roi étaient chez les grands de la ville, qui s’occupaient de leur éducation. 
${}^{7}Dès que la lettre leur parvint, ceux-ci prirent les fils du roi, égorgèrent ces soixante-dix hommes, mirent leurs têtes dans des corbeilles et les envoyèrent à Jéhu, qui était à Yizréel. 
${}^{8}Un messager vint l’informer en disant : « On a apporté les têtes des fils du roi ! » Jéhu dit alors : « Mettez-les en deux tas à l’entrée de la ville, jusqu’au matin. » 
${}^{9}Au matin, il sortit et, debout, dit à tout le peuple : « Vous êtes des hommes justes ! Moi, j’ai conspiré contre mon maître et je l’ai tué, mais tous ceux-ci, qui les a frappés ? 
${}^{10}Sachez-le donc : aucune des paroles prononcées par le Seigneur contre la maison d’Acab ne restera sans effet ; le Seigneur a accompli ce qu’il avait dit par l’intermédiaire de son serviteur Élie. » 
${}^{11}Et Jéhu frappa tous ceux de la maison d’Acab qui restaient à Yizréel, et tous ses grands, ses familiers et ses prêtres, sans laisser un seul survivant. 
${}^{12}Puis il se leva et partit pour se rendre à Samarie.
      Comme il était en chemin, vers Beth-Éqed-Aroïm, 
${}^{13}il rencontra les frères d’Ocozias, roi de Juda. Jéhu demanda : « Qui êtes-vous ? » Ils répondirent : « Nous sommes les frères d’Ocozias, nous descendons saluer les fils du roi et les fils de la reine mère. » 
${}^{14}Il dit : « Saisissez-les vivants ! » On les saisit vivants et on les égorgea à la citerne de Beth-Éqed. Ils étaient quarante-deux ; pas un n’en réchappa.
${}^{15}Jéhu partit de là et trouva Jonadab, fils de Récab, qui venait au-devant de lui. Il le salua et lui dit : « Ton cœur est-il loyal envers le mien, comme mon cœur envers le tien ? » Jonadab répondit : « Il l’est. » Jéhu reprit : « S’il l’est, donne-moi ta main. » Jonadab lui donna sa main, et Jéhu le fit monter près de lui sur son char. 
${}^{16}Il lui dit : « Viens avec moi et vois mon ardeur jalouse pour le Seigneur. » Et il l’emmena sur son char. 
${}^{17}Arrivé à Samarie, Jéhu frappa tous ceux de la famille d’Acab qui restaient à Samarie, jusqu’à l’extermination, selon ce que le Seigneur avait annoncé à Élie.
${}^{18}Jéhu rassembla ensuite tout le peuple ; il leur dit : « Acab a peu servi Baal, Jéhu le servira beaucoup. 
${}^{19}Maintenant donc, convoquez auprès de moi tous les prophètes de Baal, tous ceux qui le servent et tous ses prêtres. Qu’il n’en manque pas un, car moi, je vais offrir un grand sacrifice à Baal ! Qui manquera perdra la vie. » Mais Jéhu agissait par ruse, pour faire disparaître ceux qui servaient Baal. 
${}^{20}Il déclara : « Qu’il y ait une assemblée sainte pour Baal ! » Et on la convoqua. 
${}^{21}Jéhu envoya des messagers dans tout Israël. Ceux qui servaient Baal vinrent, tous sans exception. Ils entrèrent dans le temple de Baal, qui fut rempli d’un mur à l’autre. 
${}^{22}Jéhu dit au préposé du vestiaire : « Sors un vêtement pour tous ceux qui servent Baal. » Et il sortit pour eux les vêtements. 
${}^{23}Jéhu arriva avec Jonadab, fils de Récab, au temple de Baal. Il dit à ceux qui servaient Baal : « Cherchez bien et veillez à ce qu’il n’y ait ici avec vous aucun des serviteurs du Seigneur, mais uniquement ceux qui servent Baal. » 
${}^{24}Ils entrèrent alors pour offrir des sacrifices et des holocaustes. Jéhu avait placé au-dehors quatre-vingts de ses hommes en leur disant : « Quiconque laissera échapper un seul des hommes que je mets, moi, entre vos mains, sa vie répondra pour sa vie. » 
${}^{25}On acheva d’offrir l’holocauste. Puis Jéhu dit aux gardes et aux écuyers : « Entrez et frappez-les : que personne ne sorte ! » Les gardes et les écuyers les frappèrent du tranchant de l’épée. Après les avoir jetés dehors, les gardes et les écuyers revinrent dans la ville, au temple de Baal. 
${}^{26}Ils sortirent la stèle du temple de Baal et la brûlèrent. 
${}^{27}Après avoir détruit la stèle de Baal, ils démolirent son temple et en firent un tas d’immondices, comme il en est jusqu’à ce jour.
${}^{28}Jéhu extirpa Baal du milieu d’Israël. 
${}^{29}Toutefois il ne s’écarta pas des péchés que Jéroboam, fils de Nebath, avait fait commettre à Israël, à savoir les veaux d’or, celui de Béthel et celui de Dane. 
${}^{30}Le Seigneur dit à Jéhu : « Parce que tu as bien agi en faisant ce qui est droit à mes yeux, parce que tu as traité la maison d’Acab selon tout ce que j’avais dans le cœur, tes fils, jusqu’à la quatrième génération, s’assiéront sur le trône d’Israël. » 
${}^{31}Mais Jéhu ne s’appliqua pas à marcher de tout son cœur dans la loi du Seigneur, Dieu d’Israël. Il ne s’écarta pas des péchés que Jéroboam avait fait commettre à Israël. 
${}^{32}En ces jours-là, le Seigneur commença à amputer Israël. Hazaël frappa les habitants dans tout le territoire, 
${}^{33}depuis le Jourdain, au soleil levant, tout le pays de Galaad et de Gad, de Roubène et de Manassé, depuis Aroër qui est au-dessus du torrent de l’Arnon ; et aussi le Galaad et le Bashane.
${}^{34}Le reste des actions de Jéhu et tout ce qu’il a fait,
        \\et toute sa bravoure,
        \\cela n’est-il pas écrit dans le livre des Annales des rois d’Israël ?
${}^{35}Jéhu reposa avec ses pères,
        \\et on l’ensevelit à Samarie.
        \\Son fils Joakaz régna à sa place.
${}^{36}Le temps que Jéhu régna sur Israël, à Samarie, fut de vingt-huit ans.
      
         
      \bchapter{}
      \begin{verse}
${}^{1}Lorsque Athalie, mère d’Ocozias, apprit que son fils était mort, elle entreprit de faire périr toute la descendance royale. 
${}^{2} Mais Josabeth, fille du roi Joram et sœur d’Ocozias, prit Joas, un des fils du roi Ocozias, pour le soustraire au massacre\\. Elle le cacha, lui et sa nourrice, dans une chambre de la maison du Seigneur\\, pour le dissimuler aux regards d’Athalie ; c’est ainsi qu’il évita la mort. 
${}^{3} Il demeura avec Josabeth pendant six ans, caché dans la maison du Seigneur, tandis qu’Athalie régnait sur le pays.
${}^{4}Au bout de sept ans, le prêtre\\Joad envoya chercher les officiers des mercenaires\\et des gardes, et les fit venir près de lui dans la maison du Seigneur. Il conclut une alliance avec eux, leur fit prêter serment dans la maison du Seigneur, et leur montra le fils du roi. 
${}^{5}Il leur donna cet ordre : « Voilà ce que vous allez faire : un tiers d’entre vous, ceux qui entrent en service le jour du sabbat, gardera la maison du roi ; 
${}^{6}un tiers se tiendra à la porte de Sour, et un tiers à la porte située derrière les gardes. Vous monterez à tour de rôle la garde de la maison. 
${}^{7}Alors les deux sections qui sortent de service le jour du sabbat prendront la garde à la maison du Seigneur, auprès du roi. 
${}^{8}Vous ferez cercle autour du roi, chacun les armes à la main. Celui qui forcera les rangs sera mis à mort. Et vous accompagnerez le roi dans ses allées et venues. »
${}^{9}Les officiers exécutèrent tous les ordres du prêtre Joad. Chacun prit ses hommes, ceux qui entraient en service le jour du sabbat, et ceux qui en sortaient ce jour-là, et tous\\rejoignirent le prêtre Joad. 
${}^{10} Celui-ci leur remit les lances et les carquois du roi David, qui étaient conservés dans la maison du Seigneur. 
${}^{11} Les gardes se postèrent, les armes à la main, devant l’autel, du côté sud\\et du côté nord\\de la Maison, afin d’entourer le futur\\roi. 
${}^{12} Alors Joad fit avancer le fils du roi, lui remit le diadème et la charte de l’Alliance\\, et on le fit roi. On lui donna l’onction, on l’acclama en battant des mains et en criant : « Vive le roi ! »
${}^{13}Athalie entendit cette clameur des gardes et du peuple, et elle accourut vers le peuple à la maison du Seigneur. 
${}^{14} Et voilà ce qu’elle vit : le roi debout sur l’estrade, selon le rituel ; auprès de lui les officiers et les trompettes, et tout le peuple du pays criant sa joie tandis que les trompettes sonnaient. Alors, elle déchira ses vêtements et s’écria : « Trahison ! Trahison ! » 
${}^{15} Le prêtre Joad donna cet ordre aux officiers\\ : « Faites-la sortir de la Maison, à travers vos rangs. Si quelqu’un veut la suivre, frappez-le par l’épée. » En effet, le prêtre Joad avait interdit de la mettre à mort dans la maison du Seigneur. 
${}^{16} On mit la main sur elle, et elle arriva au palais\\par la porte des Chevaux. C’est là qu’elle fut mise à mort.
${}^{17}Joad conclut une alliance entre le Seigneur, le roi et le peuple, pour que le peuple\\soit le peuple du Seigneur ; il conclut l’alliance\\entre le roi et le peuple. 
${}^{18}Alors, tous les gens du pays entrèrent dans le temple\\de Baal et le démolirent. Ils mirent en pièces ses autels et ses statues et, devant les autels, ils tuèrent Matane, prêtre de Baal. Le prêtre Joad\\posta ensuite des gardes devant la maison du Seigneur. 
${}^{19}Il prit ensuite les officiers, les mercenaires, les gardes et tous les gens du pays. Ils firent descendre le roi de la maison du Seigneur et ils entrèrent dans celle du roi par la porte des gardes. Alors, Joas prit place sur le trône royal. 
${}^{20}Tous les gens du pays étaient dans la joie, et la ville retrouva le calme. Quant à Athalie, on l’avait mise à mort par l’épée dans la maison du roi.
      
         
      \bchapter{}
      \begin{verse}
${}^{1}Joas avait sept ans lorsqu’il devint roi. 
${}^{2}C’est la septième année du règne de Jéhu que Joas devint roi, et il régna quarante ans à Jérusalem. Le nom de sa mère était Cibya ; elle était de Bershéba. 
${}^{3}Joas fit ce qui est droit aux yeux du Seigneur, pendant tout le temps que l’instruisit le prêtre Joad. 
${}^{4}Cependant les lieux sacrés ne disparurent pas : le peuple y offrait encore des sacrifices et y brûlait de l’encens.
${}^{5}Joas dit aux prêtres : « Tout l’argent consacré que l’on apporte dans la maison du Seigneur, l’argent liquide de chacun, l’argent de la taxe personnelle, et tout l’argent que chacun voudra bien apporter à la maison du Seigneur, 
${}^{6}les prêtres le recevront, chacun de la part des gens de sa connaissance. Ce sont eux qui feront réparer les dégradations de la Maison, partout où ce sera nécessaire. » 
${}^{7}Mais la vingt-troisième année du règne de Joas, les prêtres n’avaient pas encore réparé les dégradations de la maison du Seigneur. 
${}^{8}Le roi Joas convoqua le prêtre Joad et les autres prêtres. Il leur dit : « Pourquoi ne réparez-vous pas les dégradations de la Maison ? Désormais, ne prenez plus pour vous l’argent de ceux que vous connaissez, mais donnez-le pour les dégradations de la Maison. » 
${}^{9}Les prêtres consentirent à ne plus recevoir d’argent de la part du peuple et à ne plus se charger de réparer les dégradations de la Maison.
${}^{10}Le prêtre Joad prit un coffre, perça une ouverture dans le couvercle et le plaça à côté de l’autel, à droite quand on entre dans la maison du Seigneur. Les prêtres, gardiens du seuil, y déposaient tout l’argent que l’on apportait à la maison du Seigneur. 
${}^{11}Quand ils voyaient qu’il y avait beaucoup d’argent dans le coffre, le secrétaire du roi montait, avec le grand prêtre ; on ramassait et on comptait l’argent qui se trouvait dans la maison du Seigneur. 
${}^{12}On remettait l’argent contrôlé entre les mains des maîtres d’œuvre, préposés à la maison du Seigneur. Cet argent, ils le dépensaient pour les charpentiers et les ouvriers du bâtiment qui travaillaient à la maison du Seigneur, 
${}^{13}et aussi pour les maçons et les tailleurs de pierre, ainsi que pour acheter le bois et les pierres de taille en vue de réparer les dégradations de la maison du Seigneur, en un mot, tout ce qui devait être dépensé pour réparer la Maison. 
${}^{14}Toutefois, on ne fabriqua pas, pour la maison du Seigneur, des récipients en argent, pas davantage des ciseaux, ni des coupes à aspersion, ni des trompettes, ni aucun ustensile d’or, aucun ustensile d’argent, avec l’argent que l’on apportait à la maison du Seigneur. 
${}^{15}Car on le donnait aux maîtres d’œuvre, qui l’utilisaient pour réparer la maison du Seigneur. 
${}^{16}On ne demandait pas de comptes aux hommes entre les mains desquels on remettait l’argent pour le distribuer aux maîtres d’œuvre, car ils agissaient avec honnêteté. 
${}^{17}L’argent des sacrifices d’expiation et l’argent des sacrifices pour les péchés, on ne l’apportait pas à la maison du Seigneur : il était pour les prêtres. 
${}^{18}Alors Hazaël, roi d’Aram, monta contre Gath, il l’attaqua et s’en empara. Puis Hazaël se disposa à monter contre Jérusalem. 
${}^{19}Joas, roi de Juda, prit tous les objets consacrés ; ceux qu’avaient consacrés Josaphat, Joram et Ocozias, ses pères, les rois de Juda, et ceux qu’il avait lui-même consacrés, ainsi que tout l’or qui se trouvait dans les trésors de la maison du Seigneur et de la maison du roi. Il envoya le tout à Hazaël, roi d’Aram, et celui-ci renonça à monter contre Jérusalem.
${}^{20}Le reste des actions de Joas et tout ce qu’il a fait,
        \\cela n’est-il pas écrit dans le livre des Annales des rois de Juda ?
${}^{21}Ses serviteurs se soulevèrent ; ils tramèrent un complot et frappèrent Joas à Beth-Millo, alors qu’il descendait vers Silla. 
${}^{22}Jozacar, fils de Shiméath, et Jéhozabad, fils de Shomer, ses serviteurs, le frappèrent et il mourut.
        \\On l’ensevelit avec ses pères dans la Cité de David.
        \\Son fils Amasias régna à sa place.
      
         
      \bchapter{}
      \begin{verse}
${}^{1}La vingt-troisième année du règne de Joas, fils d’Ocozias, roi de Juda, Joakaz, fils de Jéhu, devint roi sur Israël à Samarie. Il régna dix-sept ans. 
${}^{2}Il fit ce qui est mal aux yeux du Seigneur : il imita les péchés que Jéroboam, fils de Nebath, avait fait commettre à Israël. Il ne s’en écarta pas. 
${}^{3}La colère du Seigneur s’enflamma contre Israël ; il les livra aux mains d’Hazaël, roi d’Aram, et aux mains de Ben-Hadad, fils d’Hazaël, tout le temps de son règne. 
${}^{4}Joakaz apaisa le visage du Seigneur, et le Seigneur l’entendit, car il avait vu l’oppression d’Israël ; en effet, le roi d’Aram les opprimait.
${}^{5}Alors le Seigneur donna à Israël un sauveur : soustraits à la main d’Aram, les fils d’Israël habitèrent sous leurs tentes comme auparavant. 
${}^{6}Toutefois ils ne s’écartèrent pas des péchés que la maison de Jéroboam avait fait commettre à Israël ; ils y persistèrent, et même le Poteau sacré resta debout à Samarie. 
${}^{7}C’est pourquoi ne furent laissés comme troupes à Joakaz que cinquante cavaliers, dix chars et dix mille hommes de pied. Car le roi d’Aram avait fait périr les autres et les avait réduits en poussière qu’on foule aux pieds.
${}^{8}Le reste des actions de Joakaz et tout ce qu’il a fait,
        \\ainsi que sa bravoure,
        \\cela n’est-il pas écrit dans le livre des Annales des rois d’Israël ?
${}^{9}Joakaz reposa avec ses pères,
        \\et on l’ensevelit à Samarie.
        \\Son fils Joas régna à sa place.
${}^{10}La trente-septième année du règne de Joas, roi de Juda, Joas, fils de Joakaz, devint roi sur Israël, à Samarie, pour seize ans.
${}^{11}Il fit ce qui est mal aux yeux du Seigneur ; il ne s’écarta d’aucun des péchés que Jéroboam, fils de Nebath, avait fait commettre à Israël ; il persista dans cette conduite.
${}^{12}Le reste des actions de Joas et tout ce qu’il a fait,
        \\ainsi que la bravoure
        \\avec laquelle il combattit Amasias, roi de Juda,
        \\cela n’est-il pas écrit dans le livre des Annales des rois d’Israël ?
${}^{13}Joas reposa avec ses pères,
        \\et Jéroboam s’assit sur son trône.
        \\Joas fut enseveli à Samarie avec les rois d’Israël.
${}^{14}Élisée tomba malade de la maladie dont il devait mourir. Joas, roi d’Israël, descendit chez lui et pleura sur son épaule. Il disait : « Mon père !... Mon père !... Char d’Israël et ses cavaliers ! » 
${}^{15}Élisée lui dit : « Prends un arc et des flèches ! » Et il prit un arc et des flèches. 
${}^{16}Élisée dit au roi d’Israël : « Tends l’arc ! » Et il le tendit. Élisée mit alors ses mains sur les mains du roi. 
${}^{17}Puis il dit : « Ouvre la fenêtre qui donne sur l’orient ! » Et il l’ouvrit. Élisée dit alors : « Tire ! » Et il tira. Élisée prononça : « Flèche de victoire pour le Seigneur ! Flèche de victoire contre Aram ! Tu frapperas Aram à Afeq jusqu’à extermination. » 
${}^{18}Puis il dit : « Prends des flèches ! » Et il les prit. Il dit alors au roi d’Israël : « Frappe contre terre ! » Et il frappa trois fois, puis s’arrêta. 
${}^{19}L’homme de Dieu s’irrita contre lui. Il lui dit : « Il fallait frapper cinq à six fois, et tu aurais frappé Aram jusqu’à extermination. Mais maintenant, tu ne le frapperas que trois fois. »
${}^{20}Élisée mourut, et on l’ensevelit. Or, chaque année, des bandes venant de Moab pénétraient dans la région. 
${}^{21}Il advint que des gens qui portaient un homme en terre aperçurent une de ces bandes ; ils jetèrent l’homme dans la tombe d’Élisée et partirent. L’homme toucha les ossements d’Élisée, il reprit vie et se dressa sur ses pieds.
${}^{22}Hazaël, roi d’Aram, avait opprimé les Israélites tout le temps de Joakaz. 
${}^{23}Mais le Seigneur fit grâce aux fils d’Israël et les prit en pitié. Il se tourna vers eux à cause de son Alliance avec Abraham, Isaac et Jacob. Il ne voulut pas les détruire, – jusque-là, il ne les avait pas rejetés loin de sa face. 
${}^{24}Hazaël, roi d’Aram, mourut, et son fils Ben-Hadad régna à sa place. 
${}^{25}Alors Joas, fils de Joakaz, reprit des mains de Ben-Hadad, fils de Hazaël, les villes que Hazaël avait enlevées par les armes des mains de Joakaz, son père. Joas frappa Ben-Hadad trois fois et recouvra les villes d’Israël.
      
         
      \bchapter{}
      \begin{verse}
${}^{1}La deuxième année du règne de Joas, fils de Joakaz, roi d’Israël, Amasias, fils de Joas, roi de Juda, devint roi. 
${}^{2}Il avait vingt-cinq ans lorsqu’il devint roi, et il régna vingt-neuf ans à Jérusalem. Sa mère s’appelait Yoadane ; elle était de Jérusalem. 
${}^{3}Il fit ce qui est droit aux yeux du Seigneur, mais non pas comme David son ancêtre. En tout il fit ce qu’avait fait Joas son père. 
${}^{4}Cependant les lieux sacrés ne disparurent pas ; le peuple offrait encore des sacrifices et brûlait de l’encens dans les lieux sacrés. 
${}^{5}Lorsque la royauté fut affermie entre ses mains, Amasias frappa ceux de ses serviteurs qui avaient frappé le roi son père. 
${}^{6}Mais il ne mit pas à mort les fils des meurtriers, selon ce qui est écrit dans le livre de la loi de Moïse, où le Seigneur a donné cet ordre : « Les pères ne seront pas mis à mort à la place des fils, les fils ne seront pas mis à mort à la place des pères ; mais chacun sera mis à mort pour son propre péché. » 
${}^{7}C’est lui, Amasias, qui frappa les Édomites dans la vallée du Sel, au nombre de dix mille, et qui prit d’assaut La Roche. Il lui donna le nom de Yoqtéel, nom qu’elle porte encore aujourd’hui.
${}^{8}Alors Amasias envoya des messagers à Joas, fils de Joakaz, fils de Jéhu, roi d’Israël, pour lui dire : « Viens, et affrontons-nous ! » 
${}^{9}Joas, roi d’Israël, envoya à son tour des messagers à Amasias, roi de Juda, pour lui dire : « Le chardon du Liban envoya dire au cèdre du Liban : “Donne ta fille pour femme à mon fils”. Mais une bête sauvage du Liban est passée, elle a piétiné le chardon. 
${}^{10}Parce que tu as vaincu Édom, ton cœur s’enorgueillit. Glorifie-toi, mais reste chez toi ! Pourquoi provoquer le malheur et tomber, toi et Juda avec toi ? » 
${}^{11}Mais Amasias n’écouta pas. Alors Joas, roi d’Israël, se mit en marche, et affronta Amasias, roi de Juda, à Beth-Shèmesh, qui appartient à Juda. 
${}^{12}Ceux de Juda furent battus par Israël, et ils s’enfuirent chacun à sa tente. 
${}^{13}À Beth-Shèmesh, Joas, roi d’Israël, fit prisonnier Amasias, roi de Juda, fils de Joas, fils d’Ocozias. Puis il se rendit à Jérusalem. Il fit au rempart de Jérusalem une brèche de quatre cents coudées, depuis la porte d’Éphraïm jusqu’à la porte de l’Angle. 
${}^{14}Il prit tout l’or et tout l’argent, tous les objets qui se trouvaient dans la maison du Seigneur et dans les trésors de la maison du roi, ainsi que des otages. Puis il retourna à Samarie.
${}^{15}Le reste des actions de Joas, ce qu’il a fait,
        \\ainsi que sa bravoure,
        \\et comment il combattit Amasias, roi de Juda,
        \\cela n’est-il pas écrit dans le livre des Annales des rois d’Israël ?
${}^{16}Joas reposa avec ses pères,
        \\et il fut enseveli à Samarie avec les rois d’Israël.
        \\Son fils Jéroboam régna à sa place.
${}^{17}Amasias, fils de Joas, roi de Juda, vécut encore quinze ans après la mort de Joas, fils de Joakaz, roi d’Israël.
${}^{18}Le reste des actions d’Amasias,
        \\cela n’est-il pas écrit dans le livre des Annales des rois de Juda ?
${}^{19}À Jérusalem certains tramèrent un complot contre lui. Il s’enfuit dans la ville de Lakish, mais on envoya des gens à sa poursuite. C’est à Lakish qu’il fut mis à mort. 
${}^{20}Puis, on le transporta sur des chevaux, et il fut enseveli à Jérusalem avec ses pères dans la Cité de David. 
${}^{21}Tout le peuple de Juda prit Azarias, âgé de seize ans, et le fit roi, à la place de son père Amasias. 
${}^{22}C’est lui qui rebâtit Eilath et la réintégra en Juda, après que le roi Amasias eut reposé avec ses pères.
${}^{23}La quinzième année du règne d’Amasias, fils de Joas, roi de Juda, Jéroboam, fils de Joas, roi d’Israël, devint roi à Samarie pour quarante et un ans. 
${}^{24}Il fit ce qui est mal aux yeux du Seigneur ; il ne s’écarta d’aucun des péchés que Jéroboam, fils de Nebath, avait fait commettre à Israël.
${}^{25}C’est lui qui rétablit les frontières d’Israël, depuis l’Entrée-de-Hamath jusqu’à la mer de la Araba, conformément à la parole que le Seigneur Dieu d’Israël avait dite par l’intermédiaire de son serviteur le prophète Jonas, fils d’Amittaï, qui était de Gath-Héfer. 
${}^{26}Car le Seigneur avait vu la misère très amère d’Israël : plus d’esclave, plus d’homme libre, plus personne pour secourir Israël. 
${}^{27}Mais le Seigneur n’avait pas dit qu’il effacerait le nom d’Israël de dessous les cieux et il le sauva par la main de Jéroboam, fils de Joas.
${}^{28}Le reste des actions de Jéroboam et tout ce qu’il a fait,
        \\ainsi que sa bravoure,
        \\comment il avait combattu
        \\pour annexer Damas et Hamath au royaume d’Israël,
        \\cela n’est-il pas écrit dans le livre des Annales des rois d’Israël ?
${}^{29}Jéroboam reposa avec ses pères,
        \\avec les rois d’Israël.
        \\Son fils Zacharie régna à sa place.
      
         
      \bchapter{}
      \begin{verse}
${}^{1}La vingt-septième année du règne de Jéroboam, roi d’Israël, Azarias, fils d’Amasias, roi de Juda, devint roi. 
${}^{2}Il avait seize ans lorsqu’il devint roi, et il régna cinquante-deux ans à Jérusalem. Sa mère s’appelait Jékolie ; elle était de Jérusalem. 
${}^{3}Il fit ce qui est droit aux yeux du Seigneur, tout comme avait fait Amasias, son père. 
${}^{4}Cependant les lieux sacrés ne disparurent pas ; le peuple offrait encore des sacrifices et brûlait de l’encens dans les lieux sacrés. 
${}^{5}Le Seigneur frappa le roi qui fut lépreux jusqu’au jour de sa mort ; il habitait dans une maison à l’écart. Et Yotam, fils du roi, maître du palais, gouvernait les gens du pays.
${}^{6}Le reste des actions d’Azarias et tout ce qu’il a fait,
        \\cela n’est-il pas écrit dans le livre des Annales des rois de Juda ?
${}^{7}Azarias reposa avec ses pères,
        \\on l’ensevelit avec eux dans la Cité de David.
        \\Son fils Yotam régna à sa place.
${}^{8}La trente-huitième année du règne d’Azarias, roi de Juda, Zacharie, fils de Jéroboam, devint roi sur Israël, à Samarie, pour six mois. 
${}^{9}Il fit ce qui est mal aux yeux du Seigneur, comme avaient fait ses pères ; il ne s’écarta pas des péchés que Jéroboam, fils de Nebath, avait fait commettre à Israël. 
${}^{10}Shalloum, fils de Yabesh, complota contre lui, le frappa en présence du peuple, le mit à mort, et il régna à sa place.
${}^{11}Le reste des actions de Zacharie,
        \\cela est écrit dans le livre des Annales des rois d’Israël.
${}^{12}C’est la parole que le Seigneur avait dite à Jéhu :
        \\« Tes fils, jusqu’à la quatrième génération,
        \\s’assiéront sur le trône d’Israël. »
        \\Et il en fut ainsi.
${}^{13}Shalloum, fils de Yabesh, devint roi la trente-neuvième année du règne d’Ozias, roi de Juda, et ne régna qu’un mois à Samarie. 
${}^{14}Menahem, fils de Gadi, monta de Tirsa et vint à Samarie. À Samarie il frappa Shalloum, fils de Yabesh, le mit à mort et régna à sa place.
${}^{15}Le reste des actions de Shalloum
        \\et le complot qu’il fomenta,
        \\cela est écrit dans le livre des Annales des rois d’Israël.
${}^{16}Alors Menahem frappa la ville de Tifsah et tous ses habitants, ainsi que son territoire à partir de Tirsa. Parce qu’elle ne lui avait pas ouvert ses portes, il la frappa et il éventra toutes les femmes enceintes.
${}^{17}La trente-neuvième année du règne d’Azarias, roi de Juda, Menahem, fils de Gadi, devint roi sur Israël, à Samarie, pour dix ans. 
${}^{18}Il fit ce qui est mal aux yeux du Seigneur ; tout le temps qu’il régna, il ne s’écarta pas des péchés que Jéroboam, fils de Nebath, avait fait commettre à Israël.
${}^{19}Poul, roi d’Assour, envahit le pays, et Menahem lui donna mille talents d’argent, pour qu’il lui prête main-forte et affermisse la royauté entre ses mains. 
${}^{20}Menahem extorqua cet argent à Israël, en levant un impôt sur tous ceux qui avaient de grandes richesses, pour le donner au roi d’Assour : cinquante pièces d’argent par personne. Au lieu de rester dans le pays, le roi d’Assour s’en retourna.
${}^{21}Le reste des actions de Menahem et tout ce qu’il a fait,
        \\cela n’est-il pas écrit dans le livre des Annales des rois d’Israël ?
${}^{22}Menahem reposa avec ses pères.
        \\Son fils Peqahya régna à sa place.
${}^{23}La cinquantième année du règne d’Azarias, roi de Juda, Peqahya, fils de Menahem, devint roi sur Israël, à Samarie, pour deux ans. 
${}^{24}Il fit ce qui est mal aux yeux du Seigneur ; il ne s’écarta pas des péchés que Jéroboam, fils de Nebath, avait fait commettre à Israël. 
${}^{25}Pèqah, son écuyer, fils de Romélias, complota contre lui et le frappa à Samarie, dans le donjon de la maison du roi, tout comme il frappa Argob et Aryé. Il avait avec lui cinquante hommes pris parmi les Galaadites. Il mit à mort Peqahya et régna à sa place.
${}^{26}Le reste des actions de Peqahya, tout ce qu’il a fait,
        \\cela est écrit dans le livre des Annales des rois d’Israël.
${}^{27}La cinquante-deuxième année du règne d’Azarias, roi de Juda, Pèqah, fils de Romélias, devint roi sur Israël, à Samarie, pour vingt ans. 
${}^{28}Il fit ce qui est mal aux yeux du Seigneur ; il ne s’écarta pas des péchés que Jéroboam, fils de Nebath, avait fait commettre à Israël.
${}^{29}Au temps de Pèqah, roi d’Israël, Téglath-Phalasar, roi d’Assour, vint et prit Iyyone, Abel-Beth-Maaka, Yanoah, Qèdesh, Haçor, le Galaad et la Galilée, tout le pays de Nephtali, et il déporta les habitants au pays d’Assour. 
${}^{30}Osée, fils d’Éla, trama un complot contre Pèqah, fils de Romélias, le frappa, le mit à mort et régna à sa place, la vingtième année de Yotam, fils d’Ozias, roi de Juda.
${}^{31}Le reste des actions de Pèqah, tout ce qu’il a fait,
        \\cela est écrit dans le livre des Annales des rois d’Israël.
${}^{32}La deuxième année du règne de Pèqah, fils de Romélias, roi d’Israël, Yotam, fils d’Ozias, roi de Juda, devint roi. 
${}^{33}Il avait vingt-cinq ans lorsqu’il devint roi, et il régna seize ans à Jérusalem. Sa mère s’appelait Yerousha, fille de Sadoc. 
${}^{34}Il fit ce qui est droit aux yeux du Seigneur, tout comme avait fait Ozias, son père. 
${}^{35}Cependant les lieux sacrés ne disparurent pas ; le peuple offrait encore des sacrifices et brûlait de l’encens dans les lieux sacrés. C’est Yotam qui bâtit la porte Haute de la maison du Seigneur.
${}^{36}Le reste des actions de Yotam, ce qu’il a fait,
        \\cela n’est-il pas écrit dans le livre des Annales des rois de Juda ?
${}^{37}En ces jours-là, le Seigneur commença à lancer Recine, roi d’Aram, et Pèqah, fils de Romélias, contre Juda.
${}^{38}Yotam reposa avec ses pères,
        \\et il fut enseveli avec eux dans la Cité de David, son ancêtre.
        \\Son fils Acaz régna à sa place.
      
         
      \bchapter{}
      \begin{verse}
${}^{1}La dix-septième année du règne de Pèqah, fils de Romélias, Acaz, fils de Yotam, roi de Juda, devint roi. 
${}^{2}Il avait vingt ans lorsqu’il devint roi, et il régna seize ans à Jérusalem. Il ne fit pas ce qui est droit aux yeux du Seigneur, son Dieu, comme avait fait David, son ancêtre. 
${}^{3}Il marcha dans le chemin des rois d’Israël, et même, il fit passer son fils par le feu, selon les coutumes abominables des nations que le Seigneur avait dépossédées devant les fils d’Israël. 
${}^{4}Il offrit des sacrifices et brûla de l’encens dans les lieux sacrés, sur les collines et sous tout arbre verdoyant.
${}^{5}Alors Recine, roi d’Aram, et Pèqah, fils de Romélias, roi d’Israël, montèrent contre Jérusalem pour l’attaquer. Ils assiégèrent Acaz, mais ils ne purent le vaincre. 
${}^{6}C’est à cette époque que Recine, roi d’Aram, annexa Eilath à Aram. Il avait chassé d’Eilath les Judéens, et les Édomites y entrèrent. Ils y sont restés jusqu’à ce jour. 
${}^{7}Acaz envoya des messagers à Téglath-Phalasar, roi d’Assour, pour lui dire : « Je suis ton serviteur et ton fils ; monte et sauve-moi des mains du roi d’Aram et des mains du roi d’Israël, qui se dressent contre moi. » 
${}^{8}Acaz prit l’argent et l’or qui étaient dans la maison du Seigneur et dans les trésors de la maison du roi ; il les envoya au roi d’Assour en cadeau. 
${}^{9}Le roi d’Assour l’écouta et monta lui-même contre Damas, dont il s’empara ; il déporta les habitants à Qir et mit à mort Recine. 
${}^{10}Le roi Acaz se rendit à Damas pour y rencontrer Téglath-Phalasar, roi d’Assour, et il vit l’autel qui était à Damas. Le roi Acaz envoya au prêtre Ourias un dessin et un plan de l’autel, avec tous les détails de sa structure. 
${}^{11}Le prêtre Ourias bâtit l’autel ; c’est en se conformant à toutes les indications envoyées de Damas par le roi Acaz que le prêtre Ourias le construisit, et cela avant que le roi Acaz arrive de Damas. 
${}^{12}Quand le roi arriva de Damas, il vit l’autel. Le roi s’approcha de l’autel et y monta. 
${}^{13}Il fit fumer son holocauste et son offrande, répandit sa libation et aspergea l’autel avec le sang de ses sacrifices de paix. 
${}^{14}Quant à l’autel de bronze qui était devant le Seigneur – et se trouvait alors entre le nouvel autel et la maison du Seigneur –, le roi l’enleva de devant la Maison, et le mit sur le côté du nouvel autel, au nord. 
${}^{15}Puis le roi Acaz donna cet ordre au prêtre Ourias : « L’holocauste du matin et l’offrande du soir, l’holocauste du roi et son offrande, l’holocauste, l’offrande avec la libation de tous les gens du pays, tu les feras fumer sur le grand autel, que tu aspergeras avec tout le sang des holocaustes et tout le sang des sacrifices. Quant à l’autel de bronze, je m’en occuperai moi-même. » 
${}^{16}Et le prêtre Ourias fit tout ce qu’avait ordonné le roi Acaz.
${}^{17}Le roi Acaz arracha les panneaux et les bases, enleva les cuves de leurs supports ; puis, il ôta la Mer qui se trouvait sur des bœufs de bronze et la fixa sur un pavement de pierres. 
${}^{18}À cause du roi d’Assour, il fit déplacer hors de la maison du Seigneur le portique du Sabbat qui avait été édifié dans la Maison, et l’entrée du roi, située à l’extérieur.
${}^{19}Le reste des actions d’Acaz, ce qu’il a fait,
        \\cela n’est-il pas écrit dans le livre des Annales des rois de Juda ?
${}^{20}Acaz reposa avec ses pères,
        \\et il fut enseveli avec eux dans la Cité de David.
        \\Son fils Ézékias régna à sa place.
      
         
      \bchapter{}
      \begin{verse}
${}^{1}La douzième année du règne d’Acaz, roi de Juda, Osée, fils d’Éla, devint roi sur Israël, à Samarie, pour neuf ans. 
${}^{2}Il fit ce qui est mal aux yeux du Seigneur, mais pas autant que les rois d’Israël qui l’avaient précédé. 
${}^{3}Salmanasar, roi d’Assour, monta contre lui. Osée lui fut asservi et lui paya tribut. 
${}^{4}Mais le roi d’Assour découvrit qu’Osée complotait : celui-ci avait envoyé des messagers à Sô, roi d’Égypte, et n’avait pas payé comme chaque année le tribut au roi d’Assour. Le roi d’Assour le fit arrêter, charger de chaînes et mettre en prison.
${}^{5}Le roi d’Assour lança des attaques à travers tout le pays d’Israël, et monta contre Samarie, qu’il assiégea pendant trois ans. 
${}^{6}La neuvième année du règne\\d’Osée, il s’empara de Samarie et déporta les gens\\d’Israël au pays d’Assour. Il les fit habiter à Halah, sur le Habor, fleuve de Gozane, et dans les villes de Médie.
${}^{7}Cela arriva parce que les fils d’Israël avaient péché contre le Seigneur leur Dieu, lui qui les avait fait monter du pays d’Égypte et les avait arrachés au pouvoir de Pharaon, roi d’Égypte. Ils avaient adoré\\d’autres dieux 
${}^{8}et suivi les coutumes des nations que le Seigneur avait dépossédées devant eux\\. Voilà ce qu’avaient fait les rois d’Israël. 
${}^{9}Les fils d’Israël offensèrent le Seigneur leur Dieu par des actes répréhensibles : ils se construisirent des lieux sacrés dans toutes leurs villes, aussi bien dans les postes de garde que dans les places fortes. 
${}^{10}Ils dressèrent à leur usage des stèles et des poteaux sacrés sur toute colline élevée et sous tout arbre verdoyant. 
${}^{11}Là, dans tous les lieux sacrés, ils brûlèrent de l’encens comme les nations que le Seigneur avait bannies devant eux, et ils commirent des actions mauvaises pour provoquer l’indignation du Seigneur. 
${}^{12}Ils servirent les idoles immondes, alors que le Seigneur leur avait dit : « Vous ne ferez pas cela. »
${}^{13}En effet, le Seigneur avait donné cet avertissement à Israël et à Juda, par l’intermédiaire\\de tous les prophètes et de tous les voyants : « Détournez-vous de votre conduite mauvaise. Observez mes commandements et mes décrets, selon toute la Loi que j’ai prescrite à vos pères et que je leur ai fait parvenir par l’intermédiaire de mes serviteurs les prophètes. » 
${}^{14}Mais ils n’ont pas obéi et ils ont raidi leur nuque comme l’avaient fait leurs pères, qui n’avaient pas fait confiance au Seigneur leur Dieu. 
${}^{15}Ils ont méprisé ses lois, ainsi que l’Alliance qu’il avait conclue avec leurs pères et les avertissements qu’il leur avait donnés. Ils ont couru après le néant et eux-mêmes sont devenus néant. Ils ont suivi les nations qui les entouraient, alors que le Seigneur leur avait prescrit de ne pas faire comme elles. 
${}^{16}Ils ont abandonné tous les commandements du Seigneur leur Dieu et se sont fait des idoles en métal, deux statues de veaux ; ils ont fait des poteaux sacrés, se sont prosternés devant toute l’armée des cieux et ont servi Baal. 
${}^{17}Ils ont fait passer leurs fils et leurs filles par le feu ; ils ont scruté les présages et fait des incantations ; ils se sont déshonorés en faisant ce qui est mal aux yeux du Seigneur, pour provoquer son indignation.
${}^{18}Alors le Seigneur s’est mis dans une grande colère contre les tribus\\d’Israël et les a écartées loin de sa face. Il n’est resté que la seule tribu de Juda. 
${}^{19}Or Juda non plus n’a pas observé les commandements du Seigneur son Dieu, mais il a suivi les décrets qu’Israël avait établis. 
${}^{20}Alors le Seigneur a repoussé toute la race d’Israël ; il les a humiliés, il les a livrés aux mains de pillards, jusqu’à les rejeter loin de sa face. 
${}^{21}Il avait en effet arraché Israël à la maison de David, et on avait établi roi Jéroboam, fils de Nebath ; Jéroboam avait fourvoyé Israël loin du Seigneur et lui avait fait commettre un grand péché. 
${}^{22}Les fils d’Israël imitèrent tous les péchés que Jéroboam avait commis, ils ne s’en écartèrent pas. 
${}^{23}À tel point que le Seigneur a écarté Israël loin de sa face, conformément à ce qu’il avait dit par l’intermédiaire de tous ses serviteurs les prophètes, et qu’il a déporté Israël loin de sa terre, au pays d’Assour, jusqu’à ce jour.
${}^{24}Le roi d’Assour fit venir des gens de Babylone, de Kouta, d’Awwa, de Hamath et de Sefarwaïm, et les établit dans les villes de Samarie à la place des fils d’Israël. Ils prirent possession de la Samarie et s’établirent dans ses villes. 
${}^{25}Or, au début de leur installation en cet endroit, comme ils ne craignaient pas le Seigneur, le Seigneur envoya contre eux des lions qui les massacrèrent. 
${}^{26}Ils dirent au roi d’Assour : « Les nations que tu as déportées et établies dans les villes de Samarie ne connaissent pas le rituel du dieu du pays. Ce dieu a envoyé contre elles des lions qui les ont fait mourir, parce qu’elles ne connaissent pas le rituel du dieu du pays. » 
${}^{27}Le roi d’Assour donna cet ordre : « Faites partir là-bas un des prêtres que vous avez déportés ; qu’il aille habiter là-bas et qu’il leur enseigne le rite du dieu du pays. » 
${}^{28}L’un des prêtres que l’on avait déportés de Samarie vint s’établir à Béthel : il leur enseignait comment on doit craindre le Seigneur.
${}^{29}Mais chaque nation se faisait son dieu et le plaçait dans l’édifice que, sur les lieux sacrés, les habitants de Samarie avaient construit ; chaque nation agit ainsi dans la ville où elle habitait. 
${}^{30}Les gens de Babylone firent un Souccoth-Benoth ; ceux de Kouta, un Nergal ; ceux de Hamath, un Ashima ; 
${}^{31}ceux de Hawa, un Nibaz et un Tartaq ; ceux de Sefarwaïm passaient leurs fils par le feu en l’honneur d’Adramélek et d’Anammélek, dieux de Sefarwaïm. 
${}^{32}Ils craignaient le Seigneur, mais ils prirent parmi eux des prêtres pour les lieux sacrés, qui exerçaient pour eux dans les temples des lieux sacrés. 
${}^{33}Tout en craignant le Seigneur, ils continuaient à servir leurs propres dieux selon le rite des nations d’où on les avait déportés. 
${}^{34}Jusqu’à ce jour, ils agissent selon les rites anciens.
      Ils ne craignaient pas le Seigneur, ils n’agissaient pas selon les décrets et le rite, selon la Loi et le commandement que le Seigneur a prescrits aux fils de Jacob, à qui il a donné le nom d’Israël. 
${}^{35}Le Seigneur avait conclu avec eux une Alliance et leur avait donné cet ordre : « Vous ne craindrez pas d’autres dieux, vous ne vous prosternerez pas devant eux, vous ne les servirez pas, vous ne leur offrirez pas de sacrifices. 
${}^{36}Mais c’est le Seigneur, lui qui vous a fait monter du pays d’Égypte par sa grande force et son bras étendu, c’est lui que vous devez craindre ; c’est devant lui que vous devez vous prosterner, c’est à lui que vous devez offrir des sacrifices. 
${}^{37}Les décrets et les ordonnances, la Loi et le commandement qu’il a écrits pour vous, vous veillerez à les mettre en pratique tous les jours ; vous ne craindrez pas d’autres dieux. 
${}^{38}L’Alliance que j’ai conclue avec vous, vous ne l’oublierez pas, et vous ne craindrez pas d’autres dieux. 
${}^{39}Mais c’est le Seigneur votre Dieu que vous devez craindre, c’est lui qui vous délivrera des mains de tous vos ennemis. » 
${}^{40}Mais ils n’ont pas écouté ; ils ont au contraire continué d’agir conformément à leur ancien droit. 
${}^{41}Ainsi donc, ces nations craignaient le Seigneur, tout en continuant à servir leurs idoles. Tout comme leurs pères avaient agi, leurs fils et les fils de leurs fils agissent de même jusqu’à ce jour.
      
         
      \bchapter{}
      \begin{verse}
${}^{1}La troisième année du règne d’Osée, fils d’Éla, roi d’Israël, Ézékias, fils d’Acaz, roi de Juda, devint roi. 
${}^{2}Il avait vingt-cinq ans lorsqu’il devint roi, et il régna vingt-neuf ans à Jérusalem. Le nom de sa mère était Abi, fille de Zacharie. 
${}^{3}Il fit ce qui est droit aux yeux du Seigneur, tout comme avait fait David, son ancêtre. 
${}^{4}C’est lui qui supprima les lieux sacrés, brisa les stèles, coupa le Poteau sacré et mit en pièces le serpent de bronze que Moïse avait fabriqué ; car jusqu’à ces jours-là les fils d’Israël brûlaient de l’encens devant lui ; on l’appelait Nehoushtane.
${}^{5}C’est dans le Seigneur, le Dieu d’Israël, qu’Ézékias mit sa confiance, et aucun des rois de Juda ne lui fut comparable ni avant ni après lui. 
${}^{6}Il resta attaché au Seigneur, sans jamais s’écarter de lui. Il garda les commandements que le Seigneur avait prescrits à Moïse. 
${}^{7}Le Seigneur fut avec lui : il réussit dans toutes ses entreprises. Il se révolta contre le roi d’Assour et ne lui fut plus soumis. 
${}^{8}C’est lui qui battit les Philistins jusqu’à Gaza et son territoire, aussi bien les postes de garde que les places fortes.
${}^{9}La quatrième année du roi Ézékias, septième année du règne d’Osée, fils d’Éla, roi d’Israël, Salmanasar, roi d’Assour, monta contre Samarie et l’assiégea. 
${}^{10}Il prit la ville au bout de trois ans. La sixième année d’Ézékias, neuvième année d’Osée, roi d’Israël, Samarie fut prise. 
${}^{11}Le roi d’Assour déporta les gens d’Israël au pays d’Assour ; il les conduisit à Halah, sur le Habor, fleuve de Gozane, et dans les villes de Médie. 
${}^{12}Cela arriva parce qu’ils n’avaient pas écouté la voix du Seigneur leur Dieu, parce qu’ils avaient transgressé son Alliance, tout ce que Moïse, serviteur du Seigneur, avait prescrit. Ils n’avaient pas écouté et ils n’avaient rien mis en pratique.
${}^{13}La quatorzième année du roi Ézékias, Sennakérib, roi d’Assour, monta contre toutes les villes fortifiées de Juda et s’en empara. 
${}^{14}Ézékias, roi de Juda, envoya ce message au roi d’Assour, à Lakish : « J’ai commis une faute. Détourne-toi de moi ; l’impôt que tu me fixeras, je l’apporterai. » Le roi d’Assour exigea d’Ézékias, roi de Juda, trois cents talents d’argent et trente talents d’or. 
${}^{15}Ézékias remit tout l’argent qui se trouvait dans la maison du Seigneur et dans les trésors de la maison du roi. 
${}^{16}À ce moment-là, Ézékias arracha des portes du temple du Seigneur et de leurs montants le métal dont lui-même, roi de Juda, les avait recouverts ; il le remit au roi d’Assour.
${}^{17}De Lakish, le roi d’Assour envoya au roi Ézékias, à Jérusalem, le commandant en chef, le grand dignitaire et le grand échanson, avec une armée considérable. Ils montèrent à Jérusalem, y arrivèrent et prirent position près du canal du réservoir supérieur, sur la route du Champ du Foulon. 
${}^{18}Ils appelèrent le roi. Le maître du palais, Éliakim, fils d’Helcias, le secrétaire Shebna, et l’archiviste Joah, fils d’Assaf, sortirent à leur rencontre. 
${}^{19}Le grand échanson leur dit : « Je vous en prie, dites à Ézékias : Ainsi parle le grand roi, le roi d’Assour : Quelle est cette confiance en laquelle tu te reposes ? 
${}^{20}Tu te dis : “Parole des lèvres vaut conseil et vaillance pour la guerre !” En qui donc as-tu mis ta confiance pour te révolter contre moi ? 
${}^{21}Voici que tu as mis ta confiance dans le soutien d’un roseau brisé, l’Égypte, qui perce et pénètre la main de quiconque s’appuie sur lui : tel est Pharaon, roi d’Égypte, pour tous ceux qui mettent leur confiance en lui ! 
${}^{22}Vous me direz peut-être : “C’est dans le Seigneur notre Dieu que nous mettons notre confiance…” Mais n’est-ce pas ce Dieu dont Ézékias a supprimé les lieux sacrés et les autels, en disant aux gens de Juda et de Jérusalem : “C’est devant cet autel, à Jérusalem, que vous vous prosternerez” ? 
${}^{23}Eh bien ! lance donc un défi à mon seigneur le roi d’Assour, et je te donnerai deux mille chevaux si tu peux te procurer des cavaliers pour les monter ! 
${}^{24}Comment ferais-tu reculer un seul gouverneur, le moindre des serviteurs de mon seigneur ? Et tu mets ta confiance dans l’Égypte pour avoir chars et cavaliers ! 
${}^{25}Et puis, est-ce sans l’accord du Seigneur que je suis monté contre ce lieu pour le détruire ? C’est le Seigneur qui m’a dit : Monte contre ce pays et détruis-le ! »
${}^{26}Éliakim, fils d’Helcias, Shebna et Joah dirent au grand échanson : « Je t’en prie, parle en araméen à tes serviteurs, car nous le comprenons. Mais ne nous parle pas en judéen, près des oreilles du peuple qui est sur le rempart. » 
${}^{27}Le grand échanson leur répondit : « Est-ce à ton maître et à toi que mon seigneur m’a envoyé dire ces paroles ? N’est-ce pas aux hommes qui se tiennent sur le rempart, réduits, comme vous, à manger leurs excréments et à boire leur urine ? » 
${}^{28}Le grand échanson se tint debout et cria d’une voix forte en judéen. Il prononça ces mots : « Écoutez la parole du grand roi, le roi d’Assour. 
${}^{29}Ainsi parle le roi : Qu’Ézékias ne vous trompe pas car il ne pourra pas vous délivrer de ma main. 
${}^{30}Et qu’il ne vous persuade pas de mettre votre confiance dans le Seigneur, en disant : “Sûrement le Seigneur nous délivrera ; cette ville ne sera pas livrée aux mains du roi d’Assour.” 
${}^{31}N’écoutez pas Ézékias, car ainsi parle le roi d’Assour : “Faites la paix avec moi, et rendez-vous à moi. Que chacun de vous mange les fruits de sa vigne et de son figuier, et qu’il boive l’eau de sa citerne, 
${}^{32}jusqu’à ce que je vienne vous prendre pour vous emmener dans un pays comme le vôtre, un pays de froment et de vin nouveau, un pays de pain et de vignobles, un pays d’oliviers, d’huile fraîche et de miel. Ainsi, vous vivrez et ne mourrez pas.” N’écoutez pas Ézékias car il vous abuse lorsqu’il dit : “Le Seigneur nous délivrera.” 
${}^{33}Les dieux des nations ont-ils vraiment délivré chacun son pays de la main du roi d’Assour ? 
${}^{34}Où sont les dieux de Hamath et d’Arpad ? Où sont les dieux de Sefarwaïm, de Héna et de Iwwa ? Ont-ils délivré Samarie de ma main ? 
${}^{35}Parmi tous les dieux de ces pays, lesquels ont délivré leur pays de ma main, pour que le Seigneur délivre de ma main Jérusalem ? »
${}^{36}Le peuple garda le silence et ne lui répondit pas un mot, car tel était l’ordre du roi : « Vous ne lui répondrez pas ». 
${}^{37}Éliakim, fils d’Helcias, maître du palais, le secrétaire Shebna et l’archiviste Joah, fils d’Asaf, remontèrent vers Ézékias, les vêtements déchirés, et lui rapportèrent les paroles du grand échanson.
      
         
      \bchapter{}
      \begin{verse}
${}^{1}Quand le roi Ézékias entendit cela, il déchira ses vêtements, se couvrit d’une toile à sac et se rendit à la maison du Seigneur. 
${}^{2}Et il envoya le maître du palais Éliakim, le secrétaire Shebna et les plus anciens des prêtres, couverts de toile à sac, vers le prophète Isaïe, fils d’Amots. 
${}^{3}Ils lui dirent : « Ainsi parle Ézékias : Jour d’angoisse, de châtiment et de honte, que ce jour-ci ! Car des fils arrivent à terme, et la force manque pour enfanter. 
${}^{4}Peut-être le Seigneur ton Dieu va-t-il entendre toutes les paroles du grand échanson, lui que le roi d’Assour, son seigneur, a envoyé pour insulter le Dieu vivant. Peut-être va-t-il le châtier pour les paroles que le Seigneur ton Dieu aura entendues. Fais monter une prière en faveur du reste qui subsiste. »
${}^{5}Les serviteurs du roi Ézékias se rendirent auprès d’Isaïe, 
${}^{6}qui leur dit : « Vous parlerez ainsi à votre maître : Ainsi parle le Seigneur : Ne crains pas les paroles que tu as entendues, les outrages proférés contre moi par les valets du roi d’Assour. 
${}^{7}Voici ! Par une nouvelle qu’il apprendra, je vais lui inspirer de retourner dans son pays. C’est là que je le ferai tomber par l’épée. »
${}^{8}Le grand échanson s’en retourna. Ayant appris que le roi d’Assour avait quitté la ville de Lakish, il le trouva qui attaquait la ville de Libna. 
${}^{9}Le roi d’Assour avait appris la nouvelle que voici, au sujet de Tirhaqa, roi d’Éthiopie : « Voici qu’il s’est mis en campagne pour passer à l’attaque contre toi. »
      <a class="anchor verset_lettre" id="bib_2r_19_9_b"/>De nouveau, Sennakérib, roi d’Assour\\, envoya des messagers dire à Ézékias : 
${}^{10}« Vous parlerez à Ézékias, roi de Juda, en ces termes : Ne te laisse pas tromper par ton Dieu, en qui tu mets ta confiance, et ne dis pas : “Jérusalem ne sera pas livrée aux mains du roi d’Assour !” 
${}^{11}Tu sais bien ce que\\les rois d’Assour ont fait à tous les pays : ils les ont voués à l’anathème. Et toi seul, tu serais délivré ? 
${}^{12}Les dieux des nations les ont-ils délivrées, elles que mes pères ont détruites : Gozane, Harrane, Récef, et les gens d’Éden qui sont à Telassar ? 
${}^{13}Où sont le roi de Hamath, le roi d’Arpad, le roi de Lahir, Sefarwaïm, Héna et Iwwa ? »
${}^{14}Ézékias prit la lettre\\de la main des messagers ; il la lut. Puis il monta à la maison du Seigneur, déplia la lettre devant le Seigneur, 
${}^{15} et, devant lui, pria en disant : « Seigneur, Dieu d’Israël, toi qui sièges sur les Kéroubim, tu es le seul Dieu de tous les royaumes de la terre, c’est toi qui as fait le ciel et la terre. 
${}^{16} Prête l’oreille, Seigneur, et entends, ouvre les yeux, Seigneur, et vois ! Écoute le message\\envoyé par Sennakérib pour insulter le Dieu vivant. 
${}^{17} Il est vrai, Seigneur, que les rois d’Assour ont ravagé les nations et leur territoire, 
${}^{18} et brûlé leurs dieux : en réalité, ce n’étaient pas des dieux, mais un ouvrage de mains d’hommes, fait avec du bois et de la pierre ; c’est pourquoi ils ont pu les faire disparaître\\. 
${}^{19} Maintenant, je t’en supplie, Seigneur notre Dieu, sauve-nous de la main de Sennakérib, et tous les royaumes de la terre sauront que tu es, Seigneur, le seul Dieu ! »
${}^{20}Alors le prophète\\Isaïe, fils d’Amots, envoya dire à Ézékias : « Ainsi parle le Seigneur, Dieu d’Israël : J’ai entendu la prière que tu m’as adressée au sujet de Sennakérib, roi d’Assour. 
${}^{21} Voici la parole que le Seigneur a prononcée contre lui :
        \\Elle te méprise, elle te nargue,
        la vierge, la fille de Sion.
        \\Elle hoche la tête pour se moquer de toi,
        la fille de Jérusalem\\.
         
${}^{22}Qui as-tu insulté, outragé,
        contre qui as-tu élevé la voix ?
        \\Sur qui, avec hauteur, as-tu porté les yeux ?
        Sur le Saint d’Israël !
         
${}^{23}Par tes messagers tu as insulté mon Seigneur.
        \\Tu as dit : « Avec mes nombreux chars,
        \\moi, j’ai gravi le sommet des montagnes,
        les cimes du Liban ;
        \\j’ai abattu ses cèdres les plus fiers,
        ses cyprès les plus beaux ;
        \\j’ai atteint sa plus lointaine retraite,
        son parc forestier.
${}^{24}Moi, j’ai creusé, et j’ai bu
        des eaux étrangères,
        \\j’ai asséché sous mes pas
        tous les canaux de l’Égypte. »
         
${}^{25}N’entends-tu pas ? Depuis longtemps
        j’avais fait ce projet,
        \\depuis les temps anciens je l’ai formé ;
        maintenant je le réalise.
         
        \\Toi, tu étais destiné à réduire en tas de ruines
        les villes fortifiées.
${}^{26}Leurs habitants ont la main trop courte,
        ils sont effrayés, confondus ;
        \\ils ressemblent à l’herbe des champs,
        à la verdure des prés,
        \\à l’herbe des toits
        et au blé qui se consume avant d’avoir levé.
         
${}^{27}Mais je sais quand tu t’assieds,
        quand tu pars en campagne et en reviens,
        quand tu t’emportes contre moi.
${}^{28}Parce que tu t’es emporté contre moi,
        que tes insolences sont montées à mes oreilles,
        \\je passerai un crochet à ton nez,
        un mors à ta bouche ;
        \\je te ferai retourner par le chemin
        par lequel tu es venu.
         
${}^{29}Voici pour toi un signe, Ézékias :
        \\Cette année on mangera le grain tombé,
        l’an prochain, ce qui aura poussé tout seul ;
        \\mais la troisième année, semez et moissonnez,
        plantez des vignes et mangez-en le fruit.
${}^{30}Le reste, survivant de la maison de Juda,
        fera de nouveau des racines par en bas,
        et par en haut donnera des fruits.
        ${}^{31}Oui, un reste sortira de Jérusalem,
        et des survivants, de la montagne de Sion.
        \\Il fera cela, l’amour jaloux du Seigneur !
         
        ${}^{32}Et voici ce que dit le Seigneur au sujet du roi d’Assour :
        \\Il n’entrera pas dans cette ville,
        il ne lui lancera pas une seule flèche,
        \\il ne lui opposera pas un seul bouclier,
        il n’élèvera pas un seul remblai :
        ${}^{33}il retournera par le chemin
        par lequel il est venu.
        \\Non, il n’entrera pas dans cette ville,
        – oracle du Seigneur.
        ${}^{34}Je protégerai cette ville, je la sauverai
        \\à cause de moi-même
        et à cause de David mon serviteur\\. »
${}^{35}La nuit même, l’ange du Seigneur sortit et frappa cent quatre-vingt-cinq mille hommes dans le camp assyrien. Le matin, quand on se leva, ce n’était que des cadavres\\. 
${}^{36}Sennakérib, roi d’Assour, plia bagage et s’en alla. Il revint à Ninive et y demeura. 
${}^{37}Or, comme il se prosternait dans la maison de Nisrok, son dieu, ses fils Adrammélek et Sarècer le frappèrent de l’épée et s’enfuirent au pays d’Ararat. Son fils Asarhaddone régna à sa place.
      
         
      \bchapter{}
      \begin{verse}
${}^{1}En ces jours-là, le roi\\Ézékias souffrait d’une maladie mortelle. Le prophète Isaïe, fils d’Amots, vint lui dire : « Ainsi parle le Seigneur : Prends des dispositions pour ta maison, car tu vas mourir, tu ne guériras pas\\. » 
${}^{2} Ézékias se tourna vers le mur et fit cette prière au Seigneur : 
${}^{3} « Ah ! Seigneur, souviens-toi ! J’ai marché en ta présence, dans la loyauté et d’un cœur sans partage, et j’ai fait ce qui est bien à tes yeux. » Puis le roi Ézékias fondit en larmes. 
${}^{4} Isaïe allait sortir de la cour intérieure du palais quand la parole du Seigneur lui fut adressée : 
${}^{5} « Retourne dire à Ézékias, le chef de mon peuple : Ainsi parle le Seigneur, Dieu de David ton ancêtre : J’ai entendu ta prière, j’ai vu tes larmes. Eh bien ! je vais te guérir : dans trois jours tu monteras à la maison du Seigneur, 
${}^{6} et j’ajouterai quinze années à ta vie\\. Je te délivrerai, toi et cette ville, de la main du roi d’Assour. Si je protège cette ville, c’est à cause de moi-même et à cause de David mon serviteur\\. »
${}^{7}Puis Isaïe ajouta : « Prenez un gâteau de figues. » On en prit un, on le mit sur l’ulcère, et le roi s’en trouva mieux.
${}^{8}Ézékias dit à Isaïe : « À quel signe reconnaîtrai-je que le Seigneur me guérira et que, dans trois jours, je pourrai monter à la maison du Seigneur ? » 
${}^{9}Isaïe lui répondit : « Voici pour toi, de la part du Seigneur, le signe que le Seigneur accomplira la parole qu’il a prononcée : L’ombre avancera-t-elle de dix degrés ou reviendra-t-elle de dix degrés ? » 
${}^{10}Ézékias dit : « C’est peu de chose pour l’ombre de s’étendre de dix degrés ! Non, que l’ombre revienne de dix degrés en arrière ! » 
${}^{11}Alors le prophète Isaïe invoqua le Seigneur, qui fit revenir l’ombre de dix degrés en arrière, sur les degrés qu’elle avait descendus, les degrés d’Acaz.
${}^{12}En ce temps-là, Mérodak-Baladane, fils de Baladane, roi de Babylone, fit parvenir des lettres et un présent à Ézékias, car il avait appris qu’Ézékias avait été malade. 
${}^{13}Ézékias s’en réjouit et montra aux envoyés tous ses entrepôts, l’argent et l’or, les aromates et l’huile parfumée, ainsi que son arsenal et tout ce qui se trouvait dans ses trésors. Il n’y eut rien qu’Ézékias ne leur ait montré dans sa maison et dans tout son domaine.
${}^{14}Le prophète Isaïe vint alors trouver le roi Ézékias et lui demanda : « Ces gens-là, qu’ont-ils dit, et d’où venaient-ils ? » Ézékias répondit : « Ils venaient d’un pays lointain, de Babylone. » 
${}^{15}Il demanda : « Et qu’ont-ils vu dans ta maison ? » Ézékias dit : « Tout ce qui se trouve dans ma maison, ils l’ont vu. Il n’y a rien, dans mes trésors, que je ne leur aie montré. » 
${}^{16}Alors Isaïe dit à Ézékias : « Écoute la parole du Seigneur : 
${}^{17}Voici venir des jours, où tout ce qui est dans ta maison, ce que tes pères ont amassé jusqu’à aujourd’hui, sera emporté à Babylone ; il n’en restera rien, dit le Seigneur. 
${}^{18}On prendra plusieurs de tes fils, issus de toi, engendrés par toi ; ils seront des eunuques dans le palais du roi de Babylone. » 
${}^{19}Ézékias dit à Isaïe : « C’est une bonne chose, ce que tu me dis de la part du Seigneur. » Il se disait en effet : « Pourquoi pas ? S’il y a la paix et la stabilité pendant ma vie ! »
${}^{20}Le reste des actions d’Ézékias, et toute sa bravoure,
        \\comment il fit creuser le réservoir et le canal,
        \\et fit venir ainsi l’eau dans la ville,
        \\cela n’est-il pas écrit dans le livre des Annales des rois de Juda ?
${}^{21}Ézékias reposa avec ses pères.
        \\Son fils Manassé régna à sa place.
      
         
      \bchapter{}
      \begin{verse}
${}^{1}Manassé avait douze ans lorsqu’il devint roi, et il régna cinquante-cinq ans à Jérusalem. Sa mère s’appelait Hefsi-Bah. 
${}^{2}Il fit ce qui est mal aux yeux du Seigneur, selon les coutumes abominables des nations que le Seigneur avait dépossédées devant les fils d’Israël. 
${}^{3}Il rebâtit les lieux sacrés qu’avait fait disparaître Ézékias, son père, et il fit élever des autels à Baal. Il fabriqua un poteau sacré, comme l’avait fait Acab, roi d’Israël. Il se prosterna devant toute l’armée des cieux et s’en fit le serviteur. 
${}^{4}Il bâtit des autels dans la maison du Seigneur, alors que le Seigneur avait dit : « Dans Jérusalem je mettrai mon nom. » 
${}^{5}Manassé bâtit aussi des autels à toute l’armée des cieux dans les deux cours de la maison du Seigneur. 
${}^{6}Il fit passer son fils par le feu ; il pratiqua la divination et l’incantation, il interrogea les spectres et les esprits. Il fit de maintes façons ce qui est mal aux yeux du Seigneur, pour provoquer son indignation. 
${}^{7}Il plaça l’idole d’Ashéra, qu’il avait faite, dans la Maison dont le Seigneur avait dit à David, et à Salomon son fils : « Dans cette Maison, et dans Jérusalem que j’ai choisie d’entre toutes les tribus d’Israël, je mettrai mon nom à jamais. 
${}^{8}Et je ne ferai plus errer les pas d’Israël loin de la terre que j’ai donnée à leurs pères, pourvu qu’ils veillent à agir en toutes choses comme je leur ai ordonné, et selon toute la Loi que leur a prescrite mon serviteur Moïse. » 
${}^{9}Mais ils n’écoutèrent pas, et Manassé les égara, de sorte qu’ils firent le mal, plus encore que les nations que le Seigneur avait anéanties devant les fils d’Israël.
${}^{10}Alors, par l’intermédiaire de ses serviteurs les prophètes, le Seigneur parla en ces termes : 
${}^{11}« Parce que Manassé, roi de Juda, a commis ces abominations, qu’il a fait pire que tout ce qu’avaient fait avant lui les Amorites, et qu’il a, de plus, fait pécher Juda par ses idoles immondes, 
${}^{12}eh bien ! ainsi parle le Seigneur, Dieu d’Israël : Voici que je vais amener sur Jérusalem et sur Juda un malheur tel à faire tinter les deux oreilles de quiconque l’apprendra. 
${}^{13}Je tendrai sur Jérusalem le cordeau de Samarie et le fil à plomb de la maison d’Acab. Je nettoierai Jérusalem comme on nettoie un plat : on le nettoie et on le retourne à l’envers. 
${}^{14}Je délaisserai le reste de mon héritage, je les livrerai aux mains de leurs ennemis, ils deviendront une proie et un butin pour tous leurs ennemis, 
${}^{15}parce qu’ils ont fait ce qui est mal à mes yeux et qu’ils n’ont cessé de provoquer mon indignation, depuis le jour où leurs pères sont sortis d’Égypte et jusqu’à aujourd’hui. »
${}^{16}Manassé répandit aussi le sang innocent, en si grande quantité qu’il remplit Jérusalem d’un bout à l’autre, sans parler du péché qu’il fit commettre à Juda, en faisant ce qui est mal aux yeux du Seigneur.
${}^{17}Le reste des actions de Manassé, tout ce qu’il a fait,
        \\le péché qu’il a commis,
        \\cela n’est-il pas écrit dans le livre des Annales des rois de Juda ?
${}^{18}Manassé reposa avec ses pères
        \\et il fut enseveli dans le jardin de sa maison,
        \\dans le jardin d’Ouzza.
        \\Son fils Amone régna à sa place.
${}^{19}Amone avait vingt-deux ans lorsqu’il devint roi, et il régna deux ans à Jérusalem. Le nom de sa mère était Méshoullémeth, fille de Harouç, originaire de Yotba. 
${}^{20}Il fit ce qui est mal aux yeux du Seigneur, comme avait fait Manassé, son père. 
${}^{21}En tout il marcha sur le chemin où son père avait marché, il servit les idoles immondes qu’avait servies son père et se prosterna devant elles. 
${}^{22}Il abandonna le Seigneur, Dieu de ses pères, et ne marcha pas sur le chemin du Seigneur.
${}^{23}Les serviteurs d’Amone complotèrent contre lui et le mirent à mort dans sa maison. 
${}^{24}Mais les gens du peuple frappèrent tous ceux qui avaient comploté contre le roi Amone, et ce sont eux qui firent roi son fils Josias à sa place.
${}^{25}Le reste des actions d’Amone, ce qu’il a fait,
        \\cela n’est-il pas écrit dans le livre des Annales des rois de Juda ?
${}^{26}On l’ensevelit dans son tombeau,
        \\dans le jardin d’Ouzza.
        \\Son fils Josias régna à sa place.
      
         
      \bchapter{}
      \begin{verse}
${}^{1}Josias avait huit ans lorsqu’il devint roi, et il régna trente et un ans à Jérusalem. Le nom de sa mère était Yedida, fille d’Adaya, originaire de Bosqath. 
${}^{2}Il fit ce qui est droit aux yeux du Seigneur, en tout il marcha sur le chemin de David, son ancêtre ; il ne s’en écarta ni à droite ni à gauche.
${}^{3}Or, la dix-huitième année du règne de Josias, le roi envoya le secrétaire Shafane, fils d’Açalyahou, fils de Meshoullam, à la maison du Seigneur, en disant : 
${}^{4}« Monte chez Helcias, le grand-prêtre, et qu’il compte l’argent apporté à la maison du Seigneur, celui que les gardiens du seuil ont recueilli de la part du peuple. 
${}^{5}Que cet argent soit remis entre les mains des maîtres d’œuvre, préposés à la maison du Seigneur. Qu’il soit remis à ces derniers, qui sont dans la maison du Seigneur, afin d’en réparer les dégradations, 
${}^{6}ainsi qu’aux charpentiers, aux ouvriers du bâtiment et aux maçons, pour acheter le bois et les pierres de taille afin de réparer la Maison. 
${}^{7}Toutefois, ne leur demandez pas compte de l’argent remis entre leurs mains, car ils agissent avec honnêteté. »
${}^{8}Le grand prêtre Helcias dit au secrétaire Shafane : « J’ai trouvé le livre de la Loi dans la maison du Seigneur. » Et Helcias donna le livre à Shafane. Celui-ci le lut. 
${}^{9} Puis, le secrétaire Shafane alla chez le roi et lui rendit compte de ce qui s’était passé. Il déclara : « L’argent trouvé dans la Maison, tes serviteurs l’ont versé et remis entre les mains des maîtres d’œuvre, préposés à la maison du Seigneur. » 
${}^{10} Alors Shafane, le secrétaire, annonça au roi : « Le prêtre Helcias m’a donné un livre. » Et Shafane fit au roi la lecture de ce livre.
${}^{11}Après avoir entendu les paroles du livre de la Loi, le roi déchira ses vêtements. 
${}^{12}Il donna cet ordre au prêtre Helcias, à Ahiqam, fils de Shafane, à Akbor, fils de Mikaya, au secrétaire Shafane, ainsi qu’à Asaya, serviteur du roi : 
${}^{13}« Allez consulter le Seigneur pour moi, pour le peuple et pour tout Juda au sujet des paroles de ce livre qu’on vient de retrouver. La fureur du Seigneur est grande : elle s’est enflammée contre nous parce que nos pères n’ont pas obéi aux paroles de ce livre et n’ont pas pratiqué tout ce qui s’y trouve\\. »
${}^{14}Alors le prêtre Helcias et Ahiqam, Akbor, Shafane et Assaya allèrent chez la prophétesse Houlda, femme du gardien des vêtements Shalloum, fils de Tiqwa, fils de Harhas. Elle habitait à Jérusalem dans la ville nouvelle. Quand ils lui eurent parlé, 
${}^{15}elle leur dit : « Ainsi parle le Seigneur, Dieu d’Israël : Dites à l’homme qui vous a envoyés vers moi : 
${}^{16}“Ainsi parle le Seigneur : Moi, je vais faire venir un malheur en ce lieu et sur ses habitants, accomplissant ainsi toutes les paroles du livre que le roi de Juda a lu. 
${}^{17}Parce qu’ils m’ont abandonné et qu’ils ont brûlé de l’encens pour d’autres dieux, afin de provoquer mon indignation par toutes les œuvres de leurs mains, ma fureur s’est enflammée contre ce lieu et ne s’éteindra plus !” 
${}^{18}Mais au roi de Juda qui vous a envoyés consulter le Seigneur, vous direz : “Ainsi parle le Seigneur, Dieu d’Israël : Ces paroles, tu les as entendues : 
${}^{19}puisque ton cœur s’est attendri et que tu t’es humilié devant le Seigneur, quand tu as entendu ce que j’ai dit contre ce lieu et ses habitants pour qu’ils deviennent dévastation et malédiction, puisque tu as déchiré tes vêtements et pleuré devant moi, eh bien ! moi aussi, j’ai entendu – oracle du Seigneur. 
${}^{20}À cause de cela, moi, je te réunirai à tes pères ; tu seras ramené en paix dans leurs tombeaux ; tes yeux ne verront rien de tout le malheur que je fais venir sur ce lieu.” » Helcias et ses compagnons rapportèrent la réponse au roi.
      
         
      \bchapter{}
      \begin{verse}
${}^{1}Le roi fit convoquer auprès de lui tous les anciens de Juda et de Jérusalem. 
${}^{2} Il monta à la maison du Seigneur avec tous les gens de Juda, tous les habitants de Jérusalem, les prêtres et les prophètes, et tout le peuple, du plus petit au plus grand. Il lut devant eux toutes les paroles du livre de l’Alliance retrouvé dans la maison du Seigneur\\. 
${}^{3} Debout sur l’estrade, le roi conclut l’Alliance en présence du Seigneur. Il s’engageait à suivre le Seigneur en observant ses commandements, ses édits et ses décrets, de tout son cœur et de toute son âme, accomplissant ainsi les paroles de l’Alliance\\inscrites dans ce livre. Et tout le peuple s’engagea dans l’Alliance.
${}^{4}Alors le roi donna l’ordre à Helcias, le grand-prêtre, aux prêtres en second et aux gardiens du seuil de faire sortir du temple du Seigneur tous les objets qui avaient été faits pour Baal, pour Ashéra et pour toute l’armée des cieux ; il les fit brûler en dehors de Jérusalem, dans les champs du Cédron, et on porta leur cendre à Béthel. 
${}^{5}Il supprima les prêtres indignes que les rois de Juda avaient établis pour brûler de l’encens sur les lieux sacrés des villes de Juda et aux environs de Jérusalem. Il supprima également ceux qui brûlaient de l’encens en l’honneur de Baal, du Soleil, de la Lune, des Constellations et de toute l’armée des cieux. 
${}^{6}Le Poteau sacré, on le transporta de la maison du Seigneur, hors de Jérusalem, au ravin du Cédron, et on le brûla dans le ravin du Cédron. On le réduisit en poussière et on jeta la poussière dans la fosse commune. 
${}^{7}Dans la maison du Seigneur, il démolit les lieux où se pratiquait la prostitution sacrée, là où les femmes tissaient pour habiller Ashéra. 
${}^{8}Il fit venir des villes de Juda tous les prêtres, et il rendit impurs les lieux sacrés où ces prêtres avaient brûlé de l’encens, depuis Guéba jusqu’à Bersabée. Il abattit les lieux sacrés des portes, qui étaient à l’entrée de la porte de Josué, gouverneur de la ville, à gauche quand on entre dans la ville. 
${}^{9}Cependant, les prêtres des lieux sacrés ne montaient pas à l’autel du Seigneur, à Jérusalem ; mais ils mangeaient des pains sans levain au milieu de leurs frères.
${}^{10}Josias rendit impur le Tofèth, dans la Vallée de Ben-Hinnome, pour que personne ne fasse plus passer par le feu son fils ou sa fille en l’honneur de Moloch. 
${}^{11}Il supprima les chevaux que les rois de Juda avaient dédiés au Soleil, à l’entrée de la maison du Seigneur, près de la chambre du dignitaire Natane-Mélek, dans les dépendances ; et il détruisit par le feu les chars du Soleil. 
${}^{12}Le roi démolit les autels qui étaient sur la terrasse de la chambre haute d’Acaz, et que les rois de Juda avaient construits. Il démolit aussi les autels faits par Manassé dans les deux cours de la maison du Seigneur ; il les brisa sur place et en jeta la poussière dans le ravin du Cédron. 
${}^{13}Le roi rendit impurs les lieux sacrés qui étaient en face de Jérusalem, au sud du mont de la Destruction, et que Salomon, roi d’Israël, avait bâtis pour Astarté, l’horreur des Sidoniens, pour Camosh, l’horreur de Moab, et pour Milcom, l’abomination des fils d’Ammone. 
${}^{14}Il brisa aussi les stèles, coupa les poteaux sacrés et combla leur emplacement d’ossements humains.
       
${}^{15}De même, l’autel qui était à Béthel et le lieu sacré qu’avait fait Jéroboam, fils de Nebath, entraînant Israël dans le péché, cet autel et ce lieu sacré, le roi les détruisit. Il brûla le lieu sacré qu’il réduisit en poussière, et il brûla le Poteau sacré. 
${}^{16}Josias se retourna ; il aperçut les tombeaux qui étaient dans la montagne. Il envoya retirer les ossements de ces tombeaux et les brûla sur l’autel qu’il rendit impur. Il agit ainsi conformément à la parole du Seigneur proférée par l’homme de Dieu, alors que Jéroboam se tenait près de l’autel pendant la fête. S’étant retourné, Josias leva les yeux sur le tombeau de l’homme de Dieu qui avait proféré ces paroles. 
${}^{17}Il demanda : « Quel est ce monument que je vois là-bas ? » Les gens de la ville lui répondirent : « C’est le tombeau de l’homme de Dieu qui est venu de Juda et qui a proféré contre l’autel de Béthel ces paroles que tu as accomplies. 
${}^{18}Le roi dit : « Laissez-le. Que personne ne dérange ses ossements. » On épargna donc ses ossements, ainsi que les ossements du prophète qui était venu de Samarie.
${}^{19}De plus, dans les villes de Samarie, Josias supprima, sur les lieux sacrés, tous les édifices que les rois d’Israël avaient bâtis, provoquant ainsi l’indignation du Seigneur. Il agit pour ces édifices comme il avait agi à Béthel. 
${}^{20}Il immola sur les autels tous les prêtres des lieux sacrés qui étaient là et il y brûla des ossements humains. Puis il revint à Jérusalem.
${}^{21}Le roi donna cet ordre à tout le peuple : « Célébrez une Pâque en l’honneur du Seigneur votre Dieu, selon ce qui est écrit dans ce livre de l’Alliance. » 
${}^{22}Car on n’avait pas célébré de Pâque comme celle-là depuis le temps des Juges qui avaient jugé Israël, et pendant tout le temps des rois d’Israël et des rois de Juda. 
${}^{23}C’est seulement la dix-huitième année du roi Josias que l’on célébra cette Pâque en l’honneur du Seigneur, à Jérusalem.
${}^{24}De plus, les nécromanciens et les devins, les divinités domestiques, les idoles immondes et toutes les horreurs que l’on voyait dans le pays de Juda et dans Jérusalem, Josias les balaya, afin de réaliser les paroles de la Loi écrites dans le livre qu’avait trouvé le prêtre Helcias dans la maison du Seigneur. 
${}^{25}Avant lui, il ne s’était pas trouvé de roi comme lui, qui soit revenu au Seigneur de tout son cœur, de toute son âme et de toute sa force, selon toute la loi de Moïse. Après lui, il ne s’en leva aucun comme lui.
${}^{26}Toutefois le Seigneur ne revint pas de l’ardeur de sa grande colère, qui s’était enflammée contre Juda, à cause des offenses par lesquelles Manassé avait provoqué son indignation. 
${}^{27}Le Seigneur dit alors : « Même Juda, je l’écarterai loin de ma face, comme j’ai écarté Israël. Je rejetterai Jérusalem, cette Ville que j’avais choisie, et la Maison dont j’avais dit : “Là sera mon nom !” »
${}^{28}Le reste des actions de Josias, tout ce qu’il a fait,
        \\cela n’est-il pas écrit dans le livre des Annales des rois de Juda ?
${}^{29}De son temps, le pharaon Néko, roi d’Égypte, monta vers le roi d’Assour, près du fleuve Euphrate. Le roi Josias marcha à sa rencontre, mais Néko, dès qu’il le vit, le mit à mort à Meguiddo. 
${}^{30}Ses serviteurs le transportèrent mort sur un char, l’emmenèrent de Meguiddo à Jérusalem et l’ensevelirent dans son tombeau. Les gens du pays prirent alors Joakaz, fils de Josias ; ils lui donnèrent l’onction et le firent roi à la place de son père.
${}^{31}Joakaz avait vingt-trois ans lorsqu’il devint roi, et il régna trois mois à Jérusalem. Sa mère s’appelait Hamoutal, fille de Jérémie ; elle était de Libna. 
${}^{32}Il fit ce qui est mal aux yeux du Seigneur tout comme avaient fait ses pères. 
${}^{33}Le pharaon Néko le mit aux fers à Ribla, au pays de Hamath, pour qu’il ne règne plus à Jérusalem. Et il imposa au pays un tribut de cent talents d’argent et d’un talent d’or. 
${}^{34}Le pharaon Néko fit roi Élyakim, fils de Josias, à la place de Josias son père et il changea son nom en celui de Joakim. Quant à Joakaz, il le prit et l’emmena en Égypte, où celui-ci mourut.
${}^{35}L’argent et l’or, Joakim les livra à Pharaon. Pour fournir la somme exigée par Pharaon, il taxa le pays. Il exigea des gens du pays un impôt d’argent et d’or, chacun selon sa fortune, pour le donner au pharaon Néko.
${}^{36}Joakim avait vingt-cinq ans lorsqu’il devint roi, et il régna onze ans à Jérusalem. Sa mère s’appelait Zébida, fille de Pédaya ; elle était de Rouma. 
${}^{37}Il fit ce qui est mal aux yeux du Seigneur, tout comme avaient fait ses pères.
      
         
      \bchapter{}
      \begin{verse}
${}^{1}Au temps de Joakim, Nabucodonosor, roi de Babylone, se mit en campagne. Joakim lui fut assujetti pendant trois ans. Puis, changeant d’attitude, il se révolta contre lui. 
${}^{2}Le Seigneur envoya contre Joakim des bandes de Chaldéens, et des bandes venant d’Aram, de Moab et d’Ammone. Il les envoya contre Juda pour l’anéantir, conformément à la parole que le Seigneur avait prononcée par l’intermédiaire de ses serviteurs les prophètes. 
${}^{3}Cela se produisit en Juda, uniquement par ordre du Seigneur qui voulait l’écarter loin de sa face, à cause des péchés de Manassé, en tout ce qu’il avait fait. 
${}^{4}De même, le Seigneur ne voulut pas lui pardonner pour le sang innocent qu’il avait répandu, celui dont il avait rempli Jérusalem.
${}^{5}Le reste des actions de Joakim, tout ce qu’il a fait,
        \\cela n’est-il pas écrit dans le livre des Annales des rois de Juda ?
${}^{6}Joakim reposa avec ses pères.
        \\Son fils Jékonias régna à sa place.
${}^{7}Désormais le roi d’Égypte ne sortit plus de son pays, car le roi de Babylone occupait tout ce qui appartenait au roi d’Égypte, depuis le Torrent d’Égypte jusqu’à l’Euphrate.
${}^{8}Jékonias avait dix-huit ans lorsqu’il devint roi, et il régna trois mois à Jérusalem. Sa mère s’appelait Nehoushta, fille d’Elnatane ; elle était de Jérusalem. 
${}^{9} Il fit ce qui est mal aux yeux du Seigneur, tout comme avait fait son père.
${}^{10}En ce temps-là, les troupes\\de Nabucodonosor, roi de Babylone, montèrent contre Jérusalem, et la ville fut assiégée. 
${}^{11} Le roi de Babylone vint en personne attaquer la ville que son armée assiégeait. 
${}^{12} Alors, Jékonias, roi de Juda, avec sa mère, ses serviteurs, ses officiers et ses dignitaires, se rendit au roi de Babylone, qui les fit prisonniers\\. C’était en la huitième année du règne de Nabucodonosor. 
${}^{13} Celui-ci emporta\\tous les trésors de la maison du Seigneur avec ceux de la maison du roi. Il brisa tous les objets en or que Salomon, roi d’Israël, avait fait faire pour le Temple\\. Tout cela, le Seigneur l’avait annoncé\\. 
${}^{14} Nabucodonosor déporta tout Jérusalem, tous les officiers et tous les vaillants guerriers, soit dix mille hommes, sans compter tous les artisans et forgerons : on ne laissa sur place que la population la plus pauvre. 
${}^{15} Le roi Jékonias fut déporté à Babylone avec la reine mère, les épouses royales, les dignitaires, l’élite du pays : tous partirent en exil de Jérusalem à Babylone. 
${}^{16} Tous les soldats, au nombre de sept mille, les artisans et les forgerons au nombre de mille, tous ceux qui pouvaient combattre, furent déportés à Babylone par le roi Nabucodonosor\\.
${}^{17}Celui-ci fit roi, à la place de Jékonias, son oncle Mattanya, dont il changea le nom en celui de Sédécias.
${}^{18}Sédécias avait vingt et un ans lorsqu’il devint roi, et il régna onze ans à Jérusalem. Sa mère s’appelait Hamoutal, fille de Jérémie ; elle était de Libna. 
${}^{19}Il fit ce qui est mal aux yeux du Seigneur, tout comme avait fait Joakim. 
${}^{20}C’est à cause de la colère du Seigneur qu’il en fut ainsi à Jérusalem et en Juda, jusqu’à ce qu’il les rejette loin de sa face. Mais Sédécias se révolta contre le roi de Babylone.
      
         
      \bchapter{}
      \begin{verse}
${}^{1}La neuvième année du règne de Sédécias, le dixième jour du dixième mois, Nabucodonosor, roi de Babylone, vint attaquer Jérusalem avec toute son armée ; il établit son camp devant la ville qu’il entoura d’un ouvrage fortifié. 
${}^{2} La ville fut assiégée jusqu’à la onzième année du règne\\de Sédécias.
${}^{3}Le neuvième jour du quatrième\\mois, comme la famine était devenue terrible dans la ville et que les gens du pays n’avaient plus de pain, 
${}^{4} une brèche fut ouverte dans le rempart\\de la ville. Mais toute l’armée s’échappa dans la nuit, par la porte du double rempart, près du jardin du roi, dans la direction de la plaine du Jourdain\\, pendant que les Chaldéens cernaient la ville. 
${}^{5} Les troupes chaldéennes poursuivirent le roi et le rattrapèrent dans la plaine de Jéricho ; toute son armée en déroute l’avait abandonné. 
${}^{6} Les Chaldéens s’emparèrent du roi, ils le menèrent à Ribla, auprès du roi de Babylone, et l’on prononça la sentence. 
${}^{7} Les fils de Sédécias furent égorgés sous ses yeux, puis on lui creva les yeux, il fut attaché avec une double chaîne de bronze et emmené à Babylone.
${}^{8}Le septième jour du cinquième mois, la dix-neuvième année du règne de Nabucodonosor, roi de Babylone, Nabouzardane, commandant de la garde, au service\\du roi de Babylone, fit son entrée à Jérusalem. 
${}^{9} Il incendia la maison du Seigneur et la maison du roi ; il incendia toutes les maisons de Jérusalem, – toutes les maisons des notables. 
${}^{10} Toutes les troupes chaldéennes qui étaient avec lui\\abattirent les remparts de Jérusalem.
${}^{11}Nabouzardane\\déporta tout le peuple resté dans la ville, les déserteurs qui s’étaient ralliés au roi de Babylone, bref, toute la population. 
${}^{12} Il laissa seulement une partie du petit peuple de la campagne, pour avoir des vignerons et des laboureurs.
${}^{13}Les colonnes de bronze qui se trouvaient dans la maison du Seigneur, les bases et la Mer de bronze qui se trouvaient dans la maison du Seigneur, les Chaldéens les brisèrent et en emportèrent le bronze à Babylone. 
${}^{14}Ils prirent également les vases, les pelles, les ciseaux, les gobelets et tous les objets de bronze qui servaient au culte. 
${}^{15}Le commandant de la garde prit les brûle-parfums et les bols à aspersion, tout ce qui était en or et tout ce qui était en argent. 
${}^{16}Les deux colonnes, la Mer – qui était unique – et les bases, que Salomon avait faites pour la maison du Seigneur, tous ces objets étaient d’un poids de bronze qu’on ne pouvait évaluer. 
${}^{17}La hauteur de la première colonne était de dix-huit coudées ; elle était surmontée d’un chapiteau de bronze, et la hauteur d’un chapiteau était de trois coudées. Il y avait un filet et des grenades tout autour du chapiteau. Le tout était en bronze. La deuxième colonne, avec son filet, était semblable à la première.
${}^{18}Le commandant de la garde prit Seraya, chef des prêtres, Sophonie, prêtre en second, et les trois gardiens du seuil. 
${}^{19}Dans la ville, il prit un dignitaire, celui qui était préposé aux gens de guerre, cinq hommes parmi les familiers du roi qui furent trouvés dans la ville, puis le secrétaire du chef de l’armée, chargé d’enrôler les gens du pays, et soixante hommes des gens du pays, qui se trouvaient dans la ville. 
${}^{20}Nabouzardane, commandant de la garde, les ayant pris, les amena au roi de Babylone, à Ribla. 
${}^{21}Le roi de Babylone les frappa et les mit à mort, à Ribla, au pays de Hamath. Et Juda fut déporté loin de sa terre.
${}^{22}Quant aux gens restés dans le pays de Juda, et qu’avait laissés Nabucodonosor, roi de Babylone, celui-ci leur donna comme gouverneur Godolias, fils d’Ahiqam, fils de Shafane. 
${}^{23}Tous les officiers des troupes et leurs hommes apprirent que le roi de Babylone leur avait donné Godolias comme gouverneur. Ils vinrent, eux et leurs hommes, auprès de Godolias, à Mispa : c’étaient Ismaël, fils de Netanya, Yohanane, fils de Qaréah, Seraya, fils de Tanhoumeth, qui était de Netofa, et Jézanyahou, fils du Maakatite. 
${}^{24}Godolias leur fit un serment, à eux et à leurs hommes. Il leur dit : « N’ayez pas peur des officiers des Chaldéens, demeurez dans le pays, servez le roi de Babylone, et tout ira bien pour vous. »
${}^{25}Mais, le septième mois, Ismaël, fils de Netanya, fils d’Élishama, qui était de sang royal, vint avec dix hommes, et ils frappèrent à mort Godolias, les Judéens et les Chaldéens qui étaient avec lui à Mispa. 
${}^{26}Alors tous les gens, du plus petit jusqu’au plus grand, et les officiers des troupes se levèrent et se rendirent en Égypte, car ils avaient peur des Chaldéens.
${}^{27}La trente-septième année de la déportation de Jékonias, roi de Juda, le douzième mois, le vingt-sept du mois, Évil-Mérodak, roi de Babylone, l’année même où il devint roi, fit grâce à Jékonias, roi de Juda, et le fit sortir de prison. 
${}^{28}Il lui parla avec bonté et lui accorda un rang plus élevé que celui des rois qui étaient avec lui à Babylone. 
${}^{29}Il lui fit quitter ses vêtements de prisonnier, et Jékonias prit désormais ses repas en présence du roi, tous les jours de sa vie. 
${}^{30}Sa subsistance fut assurée en permanence par le roi, jour après jour, tous les jours de sa vie.
