  
  
    
    \bbook{MICHÉE}{MICHÉE}
      
         
      \bchapter{}
      \begin{verse}
${}^{1}Parole du Seigneur qui fut adressée à Michée de Morèsheth, au temps de Yotam, d’Acaz et d’Ézékias, rois de Juda – ce qu’il a vu au sujet de Samarie et de Jérusalem.
      
         
${}^{2}Vous, tous les peuples, écoutez !
        Sois attentive, terre et toute ta richesse !
        \\Dieu, le Seigneur, va témoigner contre vous,
        le Seigneur, du haut de son temple saint.
${}^{3}Voici que le Seigneur sort du lieu où il demeure ;
        il descend, il foule les sommets de la terre.
${}^{4}Les montagnes fondent sous ses pas,
        les vallées se fendent,
        \\comme la cire en présence du feu,
        comme l’eau qui coule sur une pente.
${}^{5}Tout cela, à cause de la révolte de Jacob,
        à cause des péchés de la maison d’Israël.
        \\Qui donc est la révolte de Jacob ?
        N’est-ce pas Samarie ?
        \\Qui donc est le lieu sacré de Juda ?
        N’est-ce pas Jérusalem ?
${}^{6}« Je ferai de Samarie un champ de décombres,
        une terre où planter des vignes.
        \\Je ferai rouler ses pierres au fond du ravin ;
        ses fondations, je les mettrai à nu.
${}^{7}Toutes ses statues seront brisées,
        tous les cadeaux qu’elle a reçus seront brûlés.
        \\Toutes ses idoles, je les réduirai à rien :
        elles avaient été amassées avec des gains de prostituée,
        gains de prostituée elles redeviendront. »
${}^{8}C’est pourquoi je vais me lamenter, moi, Michée, et hurler,
        je vais marcher déchaussé et nu.
        \\Je ferai une lamentation, comme les chacals,
        je pousserai des cris de deuil, comme les autruches.
         
${}^{9}Car le coup porté par le Seigneur est sans remède,
        il atteint jusqu’à Juda,
        \\il frappe jusqu’à la porte de mon peuple,
        jusqu’à Jérusalem !
${}^{10}Dans Gath, ne le publiez pas,
        ne faites pas entendre vos pleurs !
        \\Mais à Beth-Léafra,
        roulez-vous dans la poussière !
${}^{11}Va-t’en, honteuse et nue,
        habitante de Shafir !
        \\Elle ne sortira plus,
        l’habitante de Saanane !
        \\Lamente-toi, Beth-ha-Ésel :
        tout appui t’est retiré.
${}^{12}L’habitante de Maroth est privée de bonheur.
        Car le Seigneur fait descendre le malheur
        jusqu’aux portes de Jérusalem.
         
${}^{13}Attelle au char le coursier,
        habitante de Lakish,
        \\– ce fut bien là l’origine du péché pour la fille de Sion, comme ce fut l’occasion des révoltes d’Israël.
${}^{14}On donnera donc des cadeaux d’adieu
        pour Morèsheth-Gath,
        \\et les ateliers de Beth-Akzib
        ne seront que déception pour les rois d’Israël.
${}^{15}Je ferai de nouveau venir contre toi le conquérant,
        habitante de Marésha !
        \\Jusqu’à Adoullam s’en ira
        la gloire d’Israël.
         
${}^{16}Arrache tes cheveux,
        rase-toi le crâne, fille de Sion,
        à cause des fils qui faisaient ta joie !
        \\Que ta tête devienne chauve comme celle du vautour,
        car ils sont exilés loin de toi !
      
         
      \bchapter{}
        ${}^{1}Malheur à ceux qui préparent leur mauvais coup
        et, du fond de leur lit, élaborent le mal !
        \\Au point du jour, ils l’exécutent
        car c’est en leur pouvoir.
        ${}^{2}S’ils convoitent des champs, ils s’en emparent ;
        des maisons, ils les prennent ;
        \\ils saisissent le maître et sa maison,
        l’homme et son héritage.
        ${}^{3}C’est pourquoi, ainsi parle le Seigneur :
        \\Moi, je prépare contre cette engeance un malheur
        où ils enfonceront jusqu’au cou\\ ;
        \\vous ne marcherez plus la tête haute,
        car ce sera un temps de malheur.
        ${}^{4}Ce jour-là, on proférera sur vous une satire,
        et l’on entonnera une lamentation\\ ; on dira :
        \\« Nous sommes entièrement dévastés !
        \\On livre à d’autres la part de mon peuple !
        Hélas ! Elle m’échappe\\ !
        \\Nos champs sont partagés
        entre des infidèles ! »
        ${}^{5}Plus personne, en effet, ne t’assurera une part
        dans l’assemblée du Seigneur.
        
           
${}^{6}Mes ennemis déblatèrent contre moi.
        Ils disent : « Ne déblatérez pas.
        \\Cessez de déblatérer en répétant :
        “Jamais le déshonneur ne s’éloignera.”
${}^{7}Peut-on dire cela, maison de Jacob ?
        La patience du Seigneur est-elle à bout ?
        \\Est-ce là sa manière d’agir ?
        Ses paroles ne sont-elles pas bienveillantes ?
        N’accompagnent-elles pas celui qui marche droit ? »
         
${}^{8}Hier, mon peuple faisait face à l’ennemi ;
        mais vous, vous arrachez leur manteau
        à ceux qui avancent avec confiance, au retour du combat.
${}^{9}Les femmes de mon peuple, vous les chassez
        des maisons qu’elles aimaient ;
        \\à leurs enfants, vous enlevez pour toujours
        la gloire de m’appartenir.
         
${}^{10}Levez-vous, allez ! Ce n’est plus le temps du repos !
        À cause de votre impureté, vous serez détruits
        et la destruction sera cruelle.
${}^{11}Qu’un homme coure après le vent
        et que par son mensonge il vous dupe en disant :
        \\« En échange de vin et de boisson forte,
        je vais déblatérer à ton profit »,
        \\celui-là sera un homme
        qui déblatère pour ce peuple !
         
${}^{12}Mais moi, je veux te rassembler tout entier, Jacob,
        je veux réunir le reste d’Israël !
        \\Je les mettrai ensemble comme des brebis de Bosra,
        comme un troupeau au milieu de son pâturage ;
        et de cette foule humaine s’élèvera une rumeur.
${}^{13}Celui qui ouvre les brèches est monté ;
        devant eux il a ouvert la brèche.
        \\Ils ont passé la porte,
        ils sont sortis par elle ;
        \\leur roi, devant eux, est passé :
        le Seigneur est à leur tête.
      
         
      \bchapter{}
${}^{1}Et je dis :
        \\Écoutez donc, chefs de Jacob
        et dirigeants de la maison d’Israël !
        N’est-ce pas à vous de connaître le droit ?
${}^{2}Vous qui haïssez le bien et aimez le mal,
        vous qui leur arrachez la peau,
        et la chair de dessus leurs os.
${}^{3}Ceux qui ont dévoré la chair de mon peuple,
        qui lui ont arraché la peau et brisé les os,
        \\qui l’ont découpé comme viande en la marmite,
        comme chair à l’intérieur du chaudron,
${}^{4}ceux-là pourront bien crier vers le Seigneur,
        il ne leur répondra pas.
        \\Il leur cachera sa face en ce temps-là,
        à cause du mal qu’ils ont commis.
        
           
         
${}^{5}Ainsi parle le Seigneur
        contre les prophètes qui égarent mon peuple :
        \\Ont-ils quelque chose à se mettre sous la dent ?
        Ils annoncent la paix.
        \\À qui ne leur met rien dans la bouche,
        ils déclarent la guerre.
${}^{6}C’est pourquoi ce sera pour vous la nuit,
        et pas de vision ;
        \\ce sera pour vous les ténèbres,
        et pas de divination.
        \\Le soleil se couchera pour les prophètes,
        pour eux le jour s’obscurcira.
${}^{7}Les voyants seront alors couverts de honte
        et les devins, remplis de confusion ;
        \\tous, ils mettront la main sur leurs lèvres,
        car il n’y aura pas de réponse de Dieu.
${}^{8}Moi, au contraire, je suis rempli de force
        – celle du souffle du Seigneur –,
        je suis rempli de jugement et de courage,
        \\pour dénoncer à Jacob sa révolte,
        à Israël son péché.
        
           
         
${}^{9}Écoutez donc ceci, chefs de la maison de Jacob,
        dirigeants de la maison d’Israël,
        \\vous qui avez la justice en abomination,
        qui tordez tout ce qui est droit,
${}^{10}bâtissant Sion dans le sang
        et Jérusalem dans la perfidie !
${}^{11}Ses chefs jugent pour un cadeau,
        ses prêtres enseignent pour un salaire,
        ses prophètes pratiquent la divination pour de l’argent.
        \\Et ils s’appuient sur le Seigneur en disant :
        « Le Seigneur n’est-il pas au milieu de nous ?
        Aucun malheur ne peut nous atteindre ! »
${}^{12}C’est pourquoi, à cause de vous,
        \\Sion sera un champ qu’on laboure,
        Jérusalem, un monceau de décombres,
        \\et la montagne du Temple, des lieux sacrés envahis par la forêt.
        
           
      
         
      \bchapter{}
        ${}^{1}Il arrivera dans les derniers jours
        \\que la montagne de la maison du Seigneur
        se tiendra plus haut que les monts,
        \\elle s’élèvera au-dessus des collines.
        \\Vers elle afflueront des peuples
        ${}^{2}et viendront des nations nombreuses.
        
           
         
        \\Elles diront : « Venez !
        montons à la montagne du Seigneur,
        à la maison du Dieu de Jacob !
        \\Qu’il nous enseigne ses chemins,
        et nous irons par ses sentiers. »
        \\Oui, la loi sortira de Sion,
        et de Jérusalem, la parole du Seigneur.
        
           
         
        ${}^{3}Il sera le juge entre des peuples nombreux
        et, jusqu’aux lointains, l’arbitre de nations puissantes.
        \\De leurs épées, ils forgeront des socs,
        et de leurs lances, des faucilles.
        \\Jamais nation contre nation
        ne lèvera l’épée ;
        \\ils n’apprendront plus la guerre.
        
           
         
        ${}^{4}Chacun pourra s’asseoir sous sa vigne et son figuier,
        et personne pour l’inquiéter.
        \\La bouche du Seigneur de l’univers a parlé !
        
           
         
${}^{5}Oui, tous les peuples marchent,
        chacun au nom de son dieu ;
        \\mais nous, nous marchons
        au nom du Seigneur, notre Dieu,
        pour toujours et à jamais.
        
           
${}^{6}Ce jour-là – oracle du Seigneur –,
        je rassemblerai les brebis boiteuses,
        \\je réunirai les égarées,
        et celles que j’avais maltraitées.
${}^{7}Des boiteuses je ferai le reste d’Israël,
        de celles qui sont mises à l’écart, une nation puissante.
        \\Le Seigneur régnera sur eux
        à la montagne de Sion,
        dès maintenant et à jamais.
         
${}^{8}Et toi, Tour du Troupeau,
        Ophel de la fille de Sion,
        \\à toi va revenir la souveraineté d’antan,
        la royauté de la fille de Jérusalem.
${}^{9}Maintenant pourquoi pousses-tu de tels cris ?
        N’y a-t-il donc pas de Roi chez toi ?
        \\A-t-il péri, ton Conseiller,
        pour que les douleurs t’aient saisie
        comme la femme qui enfante ?
${}^{10}Oui, tords-toi de douleur, fille de Sion,
        et mets-toi en travail comme celle qui enfante,
        \\car maintenant il te faut sortir de la cité,
        camper dans les champs.
        \\Puis, tu iras jusqu’à Babel,
        c’est là que tu seras délivrée ;
        \\c’est là que le Seigneur te rachètera
        de la main de tes ennemis.
${}^{11}Maintenant se sont rassemblées contre toi
        des nations nombreuses, et elles disent :
        \\« Qu’on profane Sion !
        Que nos yeux se repaissent d’un tel spectacle ! »
${}^{12}Mais elles ne connaissent pas les pensées du Seigneur ;
        elles ne comprennent pas son projet :
        il les a rassemblées comme les gerbes sur l’aire à grain.
${}^{13}Lève-toi, fille de Sion, piétine-les !
        Je te ferai des cornes de fer,
        je te ferai des sabots de bronze.
        \\Tu vas broyer des peuples nombreux.
        \\Le fruit de leurs rapines, tu le voueras par anathème au Seigneur,
        et leurs richesses, au Maître de toute la terre.
${}^{14}Maintenant, rassemble tes troupes.
        Contre nous on a mis le siège.
        \\À coups de bâton, on frappe à la joue
        le juge d’Israël.
      
         
      \bchapter{}
        ${}^{1}Et toi, Bethléem Éphrata,
        le plus petit des clans de Juda,
        \\c’est de toi que sortira pour moi
        celui qui doit gouverner Israël\\.
        \\Ses origines remontent aux temps anciens,
        aux jours d’autrefois.
        ${}^{2}Mais Dieu livrera son peuple
        jusqu’au jour où enfantera...
        celle qui doit enfanter,
        \\et ceux de ses frères qui resteront
        rejoindront les fils d’Israël.
        ${}^{3}Il se dressera et il sera leur berger
        par la puissance du Seigneur,
        par la majesté du nom du Seigneur, son Dieu.
        \\Ils habiteront en sécurité\\, car désormais
        il sera grand jusqu’aux lointains\\de la terre,
        ${}^{4}et lui-même, il sera la paix !
        \\Alors, si Assour envahissait notre pays,
        s’il foulait aux pieds nos palais,
        \\nous susciterions contre lui sept pasteurs,
        et huit meneurs d’hommes.
${}^{5}Ils seraient les bergers de la terre d’Assour avec l’épée,
        de la terre de Nimrod avec le glaive.
        \\Car lui nous délivrerait d’Assour,
        si Assour venait à entrer sur notre terre,
        à fouler notre territoire.
        
           
         
${}^{6}Alors, le reste de Jacob sera,
        au milieu des peuples nombreux,
        \\comme une rosée venant du Seigneur,
        comme une ondée sur l’herbe
        \\qui n’espère rien de l’homme
        et n’attend rien des fils d’homme.
${}^{7}Alors, le reste de Jacob sera,
        au milieu des peuples nombreux,
        \\comme un lion parmi les bêtes de la forêt,
        comme un lionceau parmi les troupeaux de moutons :
        \\chaque fois qu’il passe, il piétine,
        il déchire, et personne qui délivre !
        
           
${}^{8}Seigneur, que ta main se lève sur tes adversaires,
        et que tous tes ennemis soient supprimés !
         
${}^{9}Voici ce qui arrivera ce jour-là – oracle du Seigneur –,
        je supprimerai de chez toi les chevaux
        et je ferai disparaître tes chars.
${}^{10}Je supprimerai les villes de ton pays
        et je démolirai toutes tes forteresses.
${}^{11}Je supprimerai de ta main tes sorcelleries ;
        il n’y aura plus chez toi de magiciens.
${}^{12}Je supprimerai de chez toi tes statues et tes stèles,
        et tu ne te prosterneras plus devant l’œuvre de tes mains.
${}^{13}Je supprimerai de chez toi tes poteaux sacrés
        et j’exterminerai tes villes.
${}^{14}Avec colère, avec fureur, j’exercerai ma vengeance
        sur les nations qui n’ont pas écouté.
      
         
      \bchapter{}
        ${}^{1}Écoutez donc ce que dit le Seigneur :
        Lève-toi ! Engage un procès avec les montagnes,
        et que les collines entendent ta voix.
        ${}^{2}Montagnes, écoutez le procès du Seigneur,
        vous aussi, fondements inébranlables de la terre.
        \\Car le Seigneur est en procès avec son peuple,
        il plaide contre Israël :
        ${}^{3}Mon peuple, que t’ai-je fait ?
        En quoi t’ai-je fatigué ? Réponds-moi.
        ${}^{4}Est-ce parce que je t’ai fait monter du pays d’Égypte,
        que je t’ai racheté de la maison d’esclavage,
        \\et que je t’ai donné comme guides
        Moïse, Aaron et Miryam ?
        ${}^{5}Ô mon peuple, souviens-toi, je te prie,
        du projet de Balac, roi de Moab,
        et de ce que lui répondit Balaam, fils de Béor.
        \\Souviens-toi du passage de Shittim jusqu’à Guilgal
        pour que tu reconnaisses les justes actions du Seigneur.
        
           
         
        ${}^{6}« Comment dois-je me présenter devant le Seigneur ?
        demande le peuple\\.
        Comment m’incliner devant le Très-Haut ?
        \\Dois-je me présenter avec de jeunes taureaux
        pour les offrir en holocaustes ?
        ${}^{7}Prendra-t-il plaisir à recevoir\\des milliers de béliers,
        à voir des flots d’huile répandus sur l’autel\\ ?
        \\Donnerai-je mon fils aîné\\pour prix de ma révolte,
        le fruit de mes entrailles pour mon propre péché ?
        
           
         
        ${}^{8}– Homme, répond le prophète\\,
        on t’a fait connaître ce qui est bien,
        ce que le Seigneur réclame de toi :
        \\rien d’autre que respecter le droit,
        aimer la fidélité,
        et t’appliquer à marcher\\avec ton Dieu. »
        
           
         
        ${}^{9}La voix du Seigneur appelle la cité :
        « Écoutez...
        ${}^{10}Puis-je supporter une mesure fausse,
        des biens acquis par fraude
        et un boisseau honteusement réduit ?
        ${}^{11}Puis-je tenir pour innocents
        ceux qui utilisent des balances fausses,
        et des sacoches de poids truqués ?
        ${}^{12}Les riches sont pleins de violence.
        Les habitants profèrent le mensonge,
        leur langage n’est que tromperie.
        ${}^{13}Moi-même j’ai commencé à te frapper,
        à te dévaster à cause de tes péchés.
        ${}^{14}Toi, tu mangeras, mais tu ne seras pas rassasiée,
        chez toi viendra la famine.
        \\Tu mettras de côté, mais tu ne pourras rien garder
        et ce que tu aurais gardé, je le livrerai à l’épée.
        ${}^{15}Toi, tu sèmeras, mais tu ne moissonneras pas ;
        tu presseras l’olive, mais tu ne t’enduiras pas d’huile,
        tu presseras le raisin, mais tu ne boiras pas de vin.
        
           
         
${}^{16}Les prescriptions d’Omri ont été observées,
        ainsi que toutes les pratiques de la maison d’Acab.
        Vous avez marché selon leurs conseils.
        \\C’est pour cela que je te livrerai à la dévastation
        et tes habitants à la raillerie.
        Le mépris des peuples pèsera sur vous. »
        
           
      
         
      \bchapter{}
${}^{1}Hélas pour moi !
        \\Je suis comme au temps des récoltes d’été,
        comme au grappillage des vendanges :
        \\plus une grappe à manger,
        plus de ces figues précoces que j’aime tant !
        ${}^{2}Les fidèles ont disparu du pays :
        plus un seul homme juste !
        \\Tous cherchent l’occasion de verser le sang,
        chacun tend un piège à son frère.
        ${}^{3}Leurs mains sont habiles pour faire le mal :
        le chef se fait payer, le juge également,
        \\et le grand fait savoir ce qu’il désire ;
        ensemble ils intriguent\\.
        ${}^{4}Le meilleur d’entre eux est pareil à un buisson de ronces,
        le plus juste est pire qu’une haie d’épines.
        \\Au jour annoncé par les guetteurs, le châtiment arrive ;
        c’est alors qu’ils seront confondus.
        ${}^{5}Ne mets pas ta foi\\dans ton ami,
        ne te confie pas à ton familier\\ ;
        \\devant celle qui repose entre tes bras,
        garde les portes\\de ta bouche.
        ${}^{6}Car le fils insulte son père,
        la fille se dresse contre sa mère,
        \\la belle-fille contre sa belle-mère,
        chacun a pour ennemis les gens de sa maison\\.
        
           
        ${}^{7}« Moi, Jérusalem\\, je veux guetter le Seigneur,
        attendre Dieu mon Sauveur ;
        lui, mon Dieu, m’entendra.
        ${}^{8}Ne te réjouis pas de mon malheur\\, ô mon ennemie ;
        oui, je suis tombée, mais je me relève ;
        \\j’habite dans les ténèbres,
        mais le Seigneur est ma lumière.
        ${}^{9}Puisque j’ai péché contre le Seigneur,
        je dois endurer sa colère
        \\jusqu’à ce qu’il prenne ma cause en main
        et rétablisse mon droit.
        \\Il me fera sortir à la lumière,
        et je contemplerai\\sa justice.
${}^{10}Mon ennemie verra tout cela,
        elle sera couverte de honte,
        \\elle qui me disait :
        “Où est-il, le Seigneur, ton Dieu ?”
        \\Mes yeux la contempleront
        tandis qu’elle sera piétinée
        comme la boue des rues. »
         
${}^{11}Au jour où l’on rebâtira ton enclos,
        ce jour-là on repoussera tes frontières,
${}^{12}ce jour-là on viendra jusqu’à toi,
        \\depuis Assour jusqu’à l’Égypte,
        depuis l’Égypte jusqu’au Fleuve,
        et de la mer à la mer, de la montagne à la montagne.
${}^{13}Le reste de la terre deviendra un lieu désolé
        à cause de ses habitants,
        tel sera le fruit de leur conduite.
         
        ${}^{14}Seigneur\\, avec ta houlette,
        sois le pasteur de ton peuple,
        du troupeau qui t’appartient\\,
        \\qui demeure isolé dans le maquis,
        entouré de vergers.
        \\Qu’il retrouve son pâturage à Bashane et Galaad,
        comme aux jours d’autrefois !
        ${}^{15}Comme aux jours où tu sortis d’Égypte\\,
        tu lui feras voir des merveilles\\ !
${}^{16}Les nations verront, et elles seront confondues
        en dépit de toute leur puissance.
        \\Elles mettront la main sur leur bouche,
        leurs oreilles deviendront sourdes.
${}^{17}Elles lécheront la poussière comme le serpent,
        comme les bêtes qui rampent sur la terre.
        \\Elles trembleront en sortant de leurs forteresses,
        elles viendront vers le Seigneur, notre Dieu,
        elles seront terrifiées, elles auront peur de toi.
         
        ${}^{18}Qui est Dieu comme toi, pour enlever le crime,
        pour passer sur la révolte
        comme tu le fais à l’égard du reste, ton héritage :
        \\un Dieu\\qui ne s’obstine pas pour toujours dans sa colère
        mais se plaît à manifester sa faveur ?
        ${}^{19}De nouveau, tu nous montreras ta miséricorde\\,
        tu fouleras aux pieds nos crimes,
        tu jetteras au fond de la mer tous nos péchés\\ !
        ${}^{20}Ainsi tu accordes à Jacob ta fidélité,
        à Abraham ta faveur,
        \\comme tu l’as juré à nos pères
        depuis les jours d’autrefois\\.
