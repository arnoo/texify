  
  
      
         
      \bchapter{}
        ${}^{1}Ainsi parle le Seigneur :
        \\Observez le droit,
        pratiquez la justice,
        \\car mon salut approche, il vient,
        et ma justice va se révéler.
        ${}^{2}Heureux l’homme qui agit ainsi,
        le fils d’homme qui s’y tient fermement ;
        \\il observe le sabbat sans le profaner
        et se garde\\de toute mauvaise action.
        ${}^{3}L’étranger\\qui s’est attaché au Seigneur,
        qu’il n’aille pas dire :
        « Le Seigneur va sûrement m’exclure de son peuple. »
        \\Et que l’eunuque ne dise pas :
        « Me voici comme un arbre sec ! »
${}^{4}Car ainsi parle le Seigneur :
        \\Aux eunuques qui observent mes sabbats,
        qui choisissent ce qui me plaît
        et qui tiennent ferme à mon alliance,
${}^{5}je placerai dans ma maison, dans mes remparts,
        une stèle à leur nom,
        préférable à des fils et à des filles ;
        \\je rendrai leur nom éternel, impérissable.
        ${}^{6}Les étrangers\\qui se sont attachés au Seigneur
        pour l’honorer, pour aimer son nom,
        pour devenir ses serviteurs,
        \\tous ceux qui observent le sabbat sans le profaner
        et tiennent ferme à mon alliance,
        ${}^{7}je les conduirai à ma montagne sainte,
        je les comblerai de joie dans ma maison de prière,
        \\leurs holocaustes et leurs sacrifices
        seront agréés sur mon autel,
        \\car ma maison s’appellera
        « Maison de prière pour tous les peuples ».
        ${}^{8}Oracle du Seigneur Dieu,
        qui rassemble les exilés d’Israël :
        \\J’en ai déjà rassemblé,
        j’en rassemblerai d’autres encore.
        
           
${}^{9}Vous, toutes les bêtes sauvages,
        vous, toutes les bêtes de la forêt,
        venez vous repaître :
${}^{10}les guetteurs d’Israël sont tous des aveugles,
        ils ne connaissent rien ;
        \\ce sont tous des chiens muets, incapables d’aboyer ;
        à bout de souffle, allongés,
        ils aiment somnoler.
${}^{11}Ce sont des chiens voraces, insatiables,
        des bergers incapables de comprendre !
        \\Ils suivent tous leur propre chemin,
        tous, sans exception, ne pensant qu’à leur intérêt,
${}^{12}chacun disant : « Venez, je vais chercher du vin,
        enivrons-nous de boisson forte ;
        \\demain sera comme aujourd’hui :
        il y a de quoi boire, plus qu’il n’en faut ! »
      
         
      \bchapter{}
${}^{1}Le juste périt,
        et nul n’y prête attention ;
        \\les hommes fidèles sont enlevés,
        et personne n’y prend garde.
        \\En fait, c’est pour être soustrait au mal
        que le juste est enlevé ;
${}^{2}il entrera dans la paix,
        \\et ceux qui suivent le droit chemin,
        quand ils se couchent, trouveront le repos.
        
           
${}^{3}Quant à vous, approchez, fils de la sorcière,
        race adultère qui t’es prostituée !
${}^{4}De qui vous moquez-vous ?
        \\Contre qui ouvrez-vous grande la bouche
        et tirez-vous la langue ?
        \\N’êtes-vous pas des enfants de révolte,
        une race de mensonge ?
${}^{5}Vous vous excitez près des térébinthes,
        sous tout arbre vert !
        \\Vous immolez des enfants dans les ravins,
        sous les saillies des rochers !
${}^{6}Les pierres polies du torrent, voilà ta part,
        tel est le sort qui te revient !
        \\Pour elles tu répands des libations,
        tu présentes des offrandes !
        \\Et moi, puis-je m’y résigner ?
${}^{7}Sur une grande et haute montagne
        tu as installé ta couche,
        \\et c’est là que tu montes
        pour offrir un sacrifice !
${}^{8}Derrière la porte et les montants
        tu as fixé ton emblème.
        \\Oui, loin de moi tu ouvres ton lit,
        tu y montes, tu y fais de la place
        \\et tu conviens d’un prix
        pour ceux avec qui tu aimes partager ta couche.
        Le membre, tu l’as contemplé !
${}^{9}Tu cours vers Mélek avec de l’huile,
        tu prodigues tes parfums,
        \\tu envoies tes messagers au loin,
        tu les as fait descendre jusqu’au séjour des morts.
${}^{10}À faire tant de chemin, tu t’es fatiguée,
        mais tu n’as pas dit : « C’est sans espoir ! »
        \\Tu as retrouvé ta vigueur,
        c’est pourquoi tu n’as pas défailli.
${}^{11}Par qui es-tu troublée et qui crains-tu
        pour mentir et ne plus te souvenir de moi,
        n’avoir plus souci de moi ?
        \\N’est-il pas vrai que tu ne me crains pas
        parce que, longtemps, j’ai gardé le silence ?
${}^{12}Moi, je vais parler de ta justice et de tes œuvres,
        qui ne te servent à rien.
${}^{13}Qu’elles te délivrent, lorsque tu crieras,
        tes collections d’idoles !
        \\Le vent les emportera toutes,
        un souffle les enlèvera.
        \\Mais qui s’abrite en moi héritera le pays
        et possédera ma sainte montagne !
${}^{14}Et l’on dira :
        \\« Frayez, frayez la route, préparez le chemin !
        Enlevez tout obstacle du chemin de mon peuple !
        ${}^{15}Car ainsi parle Celui qui est plus haut que tout\\,
        lui dont la demeure est éternelle
        et dont le nom est saint :
        \\J’habite une haute et sainte demeure,
        mais je suis avec qui est broyé, humilié dans son esprit,
        \\pour ranimer l’esprit des humiliés,
        pour ranimer le cœur de ceux qu’on a broyés.
        ${}^{16}Car je ne serai pas, pour toujours, en procès\\,
        ni sans cesse irrité ;
        \\sinon devant moi l’esprit défaillirait
        ainsi que les êtres à qui j’ai donné le souffle.
        ${}^{17}À cause de ses profits coupables,
        je me suis irrité contre mon peuple\\ ;
        \\je l’ai frappé en me détournant,
        j’étais irrité :
        \\il suivait, en renégat, le chemin de son cœur.
        ${}^{18}Ses chemins, je les ai vus,
        \\mais je le guérirai, je le conduirai,
        je le comblerai de consolations,
        \\lui et les siens qui sont en deuil ;
        ${}^{19}et, sur leurs lèvres, je vais créer la louange\\.
        \\Paix ! La paix à celui qui est loin,
        et à celui qui est proche !
        – dit le Seigneur.
        \\Oui, ce peuple\\, je le guérirai.
${}^{20}Mais les méchants sont comme une mer agitée
        qui ne peut se calmer
        et dont les eaux agitent la boue et la vase.
${}^{21}Pas de paix pour les méchants,
        – dit mon Dieu.
      
         
      \bchapter{}
        ${}^{1}Crie à pleine gorge ! Ne te retiens pas !
        Que s’élève ta voix comme le cor !
        \\Dénonce à mon peuple sa révolte,
        à la maison de Jacob ses péchés.
        ${}^{2}Ils viennent me consulter jour après jour,
        ils veulent connaître mes chemins.
        \\Comme une nation qui pratiquerait la justice
        et n’abandonnerait pas le droit de son Dieu,
        \\ils me demandent des ordonnances justes,
        ils voudraient que Dieu soit proche :
        ${}^{3}« Quand nous jeûnons,
        pourquoi ne le vois-tu pas ?
        \\Quand nous faisons pénitence\\,
        pourquoi ne le sais-tu pas ? »
        \\Oui, mais le jour où vous jeûnez,
        vous savez bien faire vos affaires,
        et vous traitez durement ceux qui peinent pour vous.
        ${}^{4}Votre jeûne se passe en disputes et querelles,
        en coups de poing sauvages.
        \\Ce n’est pas en jeûnant comme vous le faites aujourd’hui
        que vous ferez entendre là-haut votre voix.
        ${}^{5}Est-ce là le jeûne qui me plaît,
        un jour où l’homme se rabaisse ?
        \\S’agit-il de courber la tête comme un roseau,
        de coucher sur le sac et la cendre ?
        \\Appelles-tu cela un jeûne,
        un jour agréable au Seigneur ?
        
           
         
        ${}^{6}Le jeûne qui me plaît, n’est-ce pas ceci :
        \\faire tomber les chaînes injustes,
        délier les attaches du joug,
        \\rendre la liberté aux opprimés,
        briser tous les jougs ?
        ${}^{7}N’est-ce pas partager ton pain avec celui qui a faim,
        accueillir chez toi les pauvres sans abri,
        \\couvrir celui que tu verras sans vêtement,
        ne pas te dérober à ton semblable\\ ?
        ${}^{8}Alors ta lumière jaillira comme l’aurore,
        et tes forces reviendront vite.
        \\Devant toi marchera ta justice,
        et la gloire du Seigneur fermera la marche.
        ${}^{9}Alors, si tu appelles, le Seigneur répondra ;
        si tu cries, il dira : « Me voici. »
        \\Si tu fais disparaître de chez toi
        le joug, le geste accusateur\\, la parole malfaisante,
        ${}^{10}si tu donnes à celui qui a faim ce que toi, tu désires,
        et si tu combles les désirs du malheureux,
        \\ta lumière se lèvera dans les ténèbres
        et ton obscurité sera lumière de midi.
        ${}^{11}Le Seigneur sera toujours ton guide.
        En plein désert, il comblera tes désirs
        et te rendra vigueur\\.
        \\Tu seras comme un jardin bien irrigué,
        comme une source où les eaux ne manquent jamais\\.
        ${}^{12}Tu rebâtiras\\les ruines anciennes,
        tu restaureras les fondations séculaires\\.
        \\On t’appellera : « Celui qui répare les brèches »,
        « Celui qui remet en service les chemins\\ ».
        
           
        ${}^{13}Si tu t’abstiens de voyager le jour du sabbat,
        de traiter tes affaires pendant mon jour saint,
        \\si tu nommes « délices » le sabbat
        et déclares\\« glorieux » le jour saint du Seigneur,
        \\si tu le glorifies, en évitant
        démarches, affaires et pourparlers,
        ${}^{14}alors tu trouveras tes délices dans le Seigneur ;
        je te ferai chevaucher sur les hauteurs du pays,
        \\je te donnerai pour vivre\\l’héritage de Jacob ton père.
        Oui, la bouche du Seigneur a parlé.
      
         
      \bchapter{}
        ${}^{1}Non, le bras du Seigneur n’est pas trop court pour sauver,
        ni son oreille, trop dure pour entendre.
        ${}^{2}Mais ce sont vos crimes qui font la séparation
        entre vous et votre Dieu :
        \\vos péchés vous cachent son visage
        et l’empêchent de vous entendre.
        ${}^{3}Car vos mains sont souillées par le sang,
        vos doigts, par le crime ;
        \\vos lèvres ont proféré le mensonge,
        votre langue murmure la perfidie.
        ${}^{4}Nul ne porte plainte à juste titre,
        nul ne plaide de bonne foi.
        \\On s’appuie sur le néant, on dit des paroles creuses,
        on conçoit la peine, on enfante le méfait\\.
${}^{5}Ce sont des œufs de serpent qu’ils font éclore,
        et des toiles d’araignée qu’ils tissent.
        \\Qui mange de leurs œufs mourra ;
        que l’on en brise un, une vipère en sort !
${}^{6}De leurs toiles ne sera fait aucun vêtement :
        rien qui permette de s’en couvrir.
        \\Leurs œuvres sont des œuvres malfaisantes,
        leurs mains ne font que violence.
${}^{7}Ils courent au mal d’un pied rapide,
        ils ont hâte de verser le sang innocent.
        \\Leurs pensées sont des pensées malfaisantes ;
        sur leurs parcours, ravage et ruine !
${}^{8}Ils ne connaissent pas le chemin de la paix,
        sur leur passage ne se trouve pas le droit,
        \\ils rendent leurs sentiers tortueux.
        Qui prend ce chemin ne connaît pas la paix.
        
           
         
        ${}^{9}Voilà pourquoi le droit reste loin de nous,
        et pourquoi la justice n’arrive pas jusqu’à nous.
        \\Nous attendons la lumière, et voici les ténèbres ;
        la clarté, et nous marchons dans l’obscurité.
        ${}^{10}Nous tâtonnons comme des aveugles le long d’un mur,
        nous tâtonnons comme des gens qui ont perdu la vue.
        \\En plein midi nous trébuchons comme au crépuscule ;
        en pleine santé, nous voilà comme des morts.
        ${}^{11}Nous grognons tous comme des ours,
        nous gémissons sans trêve comme des colombes.
        \\Nous attendons le droit : il n’y en a pas ;
        le salut : il reste loin de nous !
        ${}^{12}Car nos révoltes se multiplient devant toi,
        nos péchés ont témoigné contre nous.
        \\Oui, notre révolte nous tient,
        et nos crimes, nous les connaissons :
        ${}^{13}se révolter, renier le Seigneur,
        abandonner notre Dieu,
        \\prêcher l’oppression, la rébellion,
        concevoir le mensonge et le ruminer dans son cœur.
        ${}^{14}Le jugement a été repoussé,
        la justice se tient à l’écart,
        \\car la vérité a trébuché sur la place,
        et la droiture ne peut y accéder.
        ${}^{15}La vérité est portée disparue ;
        qui se détourne du mal se fait dépouiller.
        
           
         
        \\Le Seigneur l’a vu, et c’est mal à ses yeux :
        il n’y a plus de droit.
${}^{16}Il a vu qu’il n’y avait personne,
        il s’est désolé que personne n’intervienne.
        \\Alors c’est son bras qui l’a sauvé,
        sa justice elle-même fut son appui.
${}^{17}Il a revêtu la justice comme cuirasse,
        et mis, sur sa tête, le casque du salut ;
        \\il a revêtu les vêtements de la vengeance,
        il s’est drapé de son ardeur jalouse comme d’un manteau.
${}^{18}Il rendra à chacun selon ses œuvres :
        fureur pour ses adversaires,
        représailles pour ses ennemis ;
        \\il rendra aux îles lointaines ce qui leur est dû.
${}^{19}Et l’on craindra depuis l’occident le nom du Seigneur,
        et depuis l’orient, sa gloire,
        \\car il viendra comme un cours d’eau encaissé,
        que précipite le souffle du Seigneur.
${}^{20}Alors il viendra en rédempteur pour Sion,
        pour ceux de Jacob revenus de leur révolte,
        – oracle du Seigneur.
        
           
         
${}^{21}Quant à moi, dit le Seigneur,
        voici mon alliance avec eux :
        \\Mon esprit qui repose sur toi
        et mes paroles que j’ai mises dans ta bouche
        ne quitteront plus ta bouche,
        \\ni celle de tes descendants,
        ni celle des descendants de tes descendants,
        – dit le Seigneur –
        \\dès maintenant et pour toujours.
        
           
      
         
      \bchapter{}
        ${}^{1}Debout, Jérusalem\\, resplendis !
        Elle est venue, ta lumière,
        et la gloire du Seigneur s’est levée sur toi.
        ${}^{2}Voici que les ténèbres couvrent la terre,
        et la nuée obscure couvre les peuples.
        \\Mais sur toi se lève le Seigneur,
        sur toi sa gloire apparaît.
        ${}^{3}Les nations marcheront vers ta lumière,
        et les rois, vers la clarté de ton aurore.
        ${}^{4}Lève les yeux alentour, et regarde :
        tous, ils se rassemblent, ils viennent vers toi ;
        \\tes fils reviennent de loin,
        et tes filles sont portées sur la hanche.
        ${}^{5}Alors tu verras, tu seras radieuse,
        ton cœur frémira et se dilatera.
        \\Les trésors d’au-delà des mers afflueront vers toi,
        vers toi viendront les richesses des nations.
        ${}^{6}En grand nombre, des chameaux t’envahiront,
        de jeunes chameaux de Madiane et d’Épha.
        \\Tous les gens de Saba viendront,
        apportant l’or et l’encens ;
        ils annonceront les exploits du Seigneur.
${}^{7}Tous les troupeaux de Qédar s’assembleront chez toi,
        avec les béliers de Nebayoth pour ton service :
        \\sur mon autel, ils seront présentés en sacrifice agréable,
        et je donnerai son éclat à la maison de ma splendeur.
${}^{8}Qui sont ceux-là qui volent comme un nuage,
        comme des colombes vers leur colombier ?
${}^{9}Oui, les îles mettent leur espoir en moi :
        les vaisseaux de Tarsis viennent en tête
        \\pour ramener tes fils du lointain,
        portant leur argent et leur or,
        \\en hommage au nom du Seigneur ton Dieu,
        en hommage au Saint d’Israël,
        car il t’a donné sa splendeur.
${}^{10}Des étrangers rebâtiront tes remparts,
        et leurs rois seront à ton service.
        \\Oui, dans ma colère je t’avais frappée,
        mais dans ma bienveillance je t’ai fait miséricorde.
${}^{11}On tiendra toujours tes portes ouvertes,
        elles ne seront jamais fermées, ni de jour ni de nuit,
        \\afin qu’on fasse entrer chez toi les richesses des nations
        et les rois avec leur suite.
${}^{12}Car nation ou royaume qui ne te servirait pas périra ;
        ces nations-là seront entièrement dévastées.
${}^{13}La gloire du Liban viendra chez toi :
        cyprès, orme et mélèze ensemble,
        \\pour faire resplendir le lieu de mon sanctuaire ;
        et ce lieu où je pose mes pieds, je le glorifierai.
${}^{14}Les fils de ceux qui t’humiliaient
        viendront se courber devant toi ;
        \\tous ceux qui te méprisaient
        se prosterneront à tes pieds.
        \\Ils t’appelleront « Ville du Seigneur »,
        « Sion du Saint d’Israël ».
${}^{15}Alors que tu étais délaissée, haïe,
        sans personne qui passe,
        \\je ferai de toi la fierté des siècles,
        une joie de génération en génération.
${}^{16}Tu suceras le lait des nations,
        tu te gorgeras de la richesse des rois,
        \\et tu sauras que moi, le Seigneur, je suis ton Sauveur,
        ton rédempteur, Force de Jacob.
${}^{17}Au lieu de bronze, je ferai venir de l’or,
        au lieu de fer, je ferai venir de l’argent,
        \\au lieu de bois, du bronze,
        au lieu de pierres, du fer.
        \\Je te donnerai, comme surveillants, la paix,
        comme gouvernants, la justice.
${}^{18}On n’entendra plus parler de violence dans ton pays,
        de ravages ni de ruines dans tes frontières.
        \\Tu appelleras tes remparts « Salut »,
        et tes portes « Louange ».
${}^{19}Le jour, tu n’auras plus le soleil comme lumière,
        et la clarté de la lune ne t’illuminera plus :
        \\le Seigneur sera pour toi lumière éternelle,
        ton Dieu sera ta splendeur.
${}^{20}Ton soleil ne se couchera plus,
        et la lune pour toi ne disparaîtra plus ;
        \\car le Seigneur sera pour toi lumière éternelle,
        et les jours de ton deuil seront accomplis.
${}^{21}Ton peuple ne comptera que des justes ;
        ils posséderont le pays pour toujours,
        \\eux, ce rejeton que j’ai planté,
        ouvrage de mes mains qui manifeste ma splendeur.
${}^{22}Le plus petit deviendra un millier,
        le plus chétif, une nation puissante.
        \\Moi, le Seigneur, je hâterai cela au temps voulu.
        
           
      <p class="cantique" id="bib_ct-at_29"><span class="cantique_label">Cantique AT 29</span> = <span class="cantique_ref"><a class="unitex_link" href="#bib_is_61_6">Is 61, 6-9</a></span>
      <p class="cantique" id="bib_ct-at_30"><span class="cantique_label">Cantique AT 30</span> = <span class="cantique_ref"><a class="unitex_link" href="#bib_is_61_10">Is 61,10-11</a> ; <a class="unitex_link" href="#bib_is_62_1">62,1-5</a></span>
      
         
      \bchapter{}
        ${}^{1}L’esprit du Seigneur Dieu est sur moi
        parce que le Seigneur m’a consacré par l’onction.
        \\Il m’a envoyé annoncer la bonne nouvelle aux humbles\\,
        guérir ceux qui ont le cœur brisé,
        \\proclamer aux captifs leur délivrance,
        aux prisonniers leur libération,
        ${}^{2}proclamer une année de bienfaits accordée par le Seigneur,
        et un jour de vengeance pour notre Dieu,
        \\consoler tous ceux qui sont en deuil,
        ${}^{3}ceux\\qui sont en deuil dans Sion,
        \\mettre le diadème sur leur tête au lieu de la cendre,
        l’huile de joie au lieu du deuil,
        un habit de fête au lieu d’un esprit abattu.
        
           
         
        \\Ils seront appelés « Térébinthes de justice »,
        « Plantation du Seigneur qui manifeste sa splendeur ».
${}^{4}Ils rebâtiront les ruines antiques,
        ils relèveront les demeures dévastées des ancêtres,
        \\ils restaureront les villes en ruine,
        dévastées depuis des générations.
        
           
         
${}^{5}Des gens venus d’ailleurs se présenteront
        pour paître vos troupeaux,
        \\des étrangers seront vos laboureurs
        et vos vignerons.
        
           
       
        ${}^{6}Vous serez appelés « Prêtres du Seigneur » ;
        on vous dira « Servants de notre Dieu. »
        \\Vous vivrez de la ressource des nations
        et leur gloire sera votre parure.
         
        ${}^{7}Au lieu de votre honte : double part !
        Au lieu de vos opprobres : cris de joie\\ !
        \\Ils recevront dans leur pays double héritage,
        ils auront l’allégresse éternelle.
         
        ${}^{8}Parce que moi, le Seigneur, j’aime le bon droit,
        parce que je hais le vol et l’injustice\\,
        \\loyalement, je leur donnerai la récompense,
        je conclurai avec eux une alliance éternelle.
         
        ${}^{9}Leurs descendants seront connus parmi les nations,
        et leur postérité, au milieu des peuples.
        \\Qui les verra pourra reconnaître
        la descendance bénie du Seigneur.
       
        ${}^{10}Je tressaille de joie\\dans le Seigneur,
        mon âme exulte en mon Dieu.
        \\Car il m’a vêtue des vêtements du salut,
        il m’a couverte du manteau de la justice,
        \\comme le jeune marié orné du diadème,
        la jeune mariée que parent ses joyaux.
         
        ${}^{11}Comme la terre fait éclore son germe,
        et le jardin, germer ses semences,
        \\le Seigneur Dieu fera germer la justice et la louange
        devant toutes les nations.
      
         
      \bchapter{}
        ${}^{1}Pour la cause de Sion, je ne me tairai pas,
        et pour Jérusalem\\, je n’aurai de cesse
        \\que sa justice ne paraisse dans la clarté,
        et son salut comme une torche qui brûle.
        
           
         
        ${}^{2}Et les nations verront ta justice ;
        tous les rois verront ta gloire.
        \\On te nommera d’un nom nouveau
        que la bouche du Seigneur dictera.
        
           
         
        ${}^{3}Tu seras une couronne brillante
        dans la main du Seigneur,
        \\un diadème royal
        entre les doigts\\de ton Dieu.
        
           
         
        ${}^{4}On ne te dira plus : « Délaissée ! »
        À ton pays, nul ne dira : « Désolation ! »
        \\Toi, tu seras appelée « Ma Préférence »,
        cette terre se nommera « L’Épousée ».
        \\Car le Seigneur t’a préférée,
        et cette terre deviendra « L’Épousée ».
        
           
         
        ${}^{5}Comme un jeune homme épouse une vierge,
        ton Bâtisseur\\t’épousera.
        \\Comme la jeune mariée fait la joie de son mari,
        tu seras la joie de ton Dieu.
        
           
         
${}^{6}Sur tes remparts, Jérusalem, j’ai placé des veilleurs ;
        ni de jour ni de nuit, jamais ils ne doivent se taire.
        \\Vous qui tenez en éveil la mémoire du Seigneur,
        ne prenez aucun repos !
        
           
         
${}^{7}Ne lui laissez aucun repos
        qu’il n’ait rendu Jérusalem inébranlable,
        qu’il ne l’ait faite louange pour la terre !
        
           
         
${}^{8}Le Seigneur l’a juré par sa droite
        et par son bras puissant :
        \\« Jamais plus je ne laisserai tes ennemis
        manger ton blé,
        \\jamais plus les étrangers ne boiront ton vin nouveau,
        fruit de ton labeur.
        
           
         
${}^{9}Ce sont les moissonneurs qui mangeront le blé :
        ils loueront le Seigneur ;
        \\ce sont les vendangeurs qui boiront le vin
        dans les cours de mon sanctuaire. »
        
           
         
${}^{10}Passez, passez les portes,
        préparez le chemin du peuple.
        \\Frayez, frayez la route, ôtez-en les pierres.
        Pour les peuples, dressez un étendard.
        
           
         
        ${}^{11}Voici que le Seigneur se fait entendre
        jusqu’aux extrémités de la terre :
        \\Dites à la fille de Sion :
        \\Voici ton Sauveur\\qui vient ;
        \\avec lui, le fruit de son travail\\,
        et devant lui, son ouvrage.
        
           
         
        ${}^{12}Eux seront appelés « Peuple-saint »,
        « Rachetés-par-le-Seigneur »,
        \\et toi, on t’appellera « La-Désirée »,
        « La-Ville-qui-n’est-plus-délaissée ».
        
           
      <p class="cantique" id="bib_ct-at_31"><span class="cantique_label">Cantique AT 31</span> = <span class="cantique_ref"><a class="unitex_link" href="#bib_is_63_1">Is 63, 1-5</a></span>
      
         
      \bchapter{}
        ${}^{1}Quel est celui-là qui arrive d’Édom,
        qui vient de Bosra, vêtu de rouge,
        \\celui-là, superbe en son habit,
        qui s’avance plein de force ?
        \\« Moi, je proclame la justice
        et j’ai le pouvoir de sauver. »
        ${}^{2}Mais pourquoi ces habits écarlates\\,
        ce vêtement de fouleur au pressoir ?
        ${}^{3}« À la cuve, j’étais seul à fouler :
        personne de mon peuple\\avec moi !
        \\Et je les ai foulés dans ma colère,
        je les ai piétinés dans ma fureur.
        \\Leur sang\\a giclé sur mes vêtements,
        j’ai taché tous mes habits.
        ${}^{4}Ce jour de vengeance, mon cœur y pensait :
        l’année des rédemptions était venue\\.
        ${}^{5}J’ai regardé : personne pour m’aider ;
        stupéfait, je restais sans appui.
        \\Alors mon bras m’a sauvé,
        ma fureur fut mon appui\\.
${}^{6}J’ai écrasé des peuples dans ma colère,
        je les ai brisés dans ma fureur,
        et j’ai répandu à terre leur sang. »
        
           
        ${}^{7}Je veux rappeler les bienfaits du Seigneur,
        les exploits du Seigneur\\,
        \\à la mesure de\\ce qu’il fit pour nous :
        sa grande bonté pour la maison d’Israël,
        \\ce qu’il fit pour eux dans sa tendresse,
        l’abondance de ses bienfaits.
         
        ${}^{8}Il avait dit : « Vraiment, ils sont mon peuple,
        des fils qui ne trahiront pas ! »
        \\Il fut donc pour eux un sauveur
        ${}^{9}dans toutes leurs détresses.
         
        \\Ce n’était ni un messager\\ni un ange,
        mais sa face qui les sauva.
        \\Dans son amour et sa compassion,
        lui-même les racheta ;
        \\il s’est chargé d’eux et les a portés
        tous ces jours d’autrefois.
         
${}^{10}Eux se sont rebellés,
        ils ont attristé son esprit saint.
        \\Alors il se retourna contre eux en ennemi,
        lui-même leur fit la guerre.
${}^{11}Et l’on se souvint des jours d’autrefois,
        de Moïse et de son peuple.
         
        \\Où est-il, Celui qui les fit remonter de la mer,
        avec le pasteur de son troupeau ?
        \\Où est Celui qui mit en lui
        son esprit saint ?
         
${}^{12}Où est Celui qui fit avancer, à la droite de Moïse,
        son bras resplendissant,
        \\qui fendit les eaux devant eux
        pour se faire un nom éternel,
${}^{13}qui les fit avancer dans les abîmes
        comme chevaux à travers le désert,
        sans qu’ils trébuchent ?
         
${}^{14}Comme on fait descendre le bétail dans la vallée,
        l’esprit du Seigneur les menait au repos.
        \\C’est ainsi que tu conduisais ton peuple
        pour donner splendeur à ton nom.
         
${}^{15}Du haut des cieux, regarde et vois,
        du haut de ta demeure sainte et resplendissante !
        \\Où sont ta jalousie et ta vaillance,
        le frémissement de tes entrailles ?
        Ta tendresse envers moi, l’aurais-tu contenue ?
         
        ${}^{16}Pourtant, c’est toi notre père !
        \\Abraham ne nous connaît pas,
        Israël ne nous reconnaît pas.
        \\C’est toi, Seigneur, notre père ;
        « Notre-rédempteur-depuis-toujours », tel est ton nom\\.
         
        ${}^{17}Pourquoi, Seigneur, nous laisses-tu errer
        hors de tes chemins ?
        \\Pourquoi laisser nos cœurs s’endurcir
        et ne plus te craindre ?
        \\Reviens, à cause de tes serviteurs,
        des tribus de ton héritage.
         
${}^{18}Ton peuple saint n’a pas joui longtemps de ses possessions :
        nos ennemis ont piétiné ton sanctuaire !
        ${}^{19}Nous sommes comme des gens
        que tu n’aurais jamais gouvernés,
        sur lesquels ton nom n’est pas invoqué.
         
        \\Ah ! Si tu déchirais les cieux, si tu descendais,
        les montagnes seraient ébranlées\\devant ta face,
      
         
      \bchapter{}
${}^{1}comme un feu qui enflamme les broussailles,
        un feu qui fait bouillonner les eaux !
        
           
         
        \\Ainsi tu manifesterais ton nom à tes ennemis,
        les nations trembleraient devant toi,
        ${}^{2}quand tu ferais des prodiges terrifiants
        que nous n’espérons plus.
        
           
         
        \\Voici que tu es descendu :
        les montagnes furent ébranlées devant ta face.
        
           
         
        ${}^{3}Jamais on n’a entendu,
        jamais on n’a ouï dire,
        \\nul œil n’a jamais vu un autre dieu que toi
        agir ainsi pour celui qui l’attend.
        
           
         
        ${}^{4}Tu viens rencontrer
        celui qui pratique avec joie la justice,
        \\qui se souvient de toi
        en suivant tes chemins.
        
           
         
        \\Tu étais irrité, mais nous avons encore péché,
        et nous nous sommes égarés\\.
        ${}^{5}Tous, nous étions comme des gens impurs,
        et tous nos actes justes n’étaient que linges souillés.
        \\Tous, nous étions desséchés comme des feuilles,
        et nos fautes, comme le vent, nous emportaient.
        
           
         
        ${}^{6}Personne n’invoque plus ton nom,
        nul ne se réveille pour prendre appui sur toi.
        \\Car tu nous as caché ton visage,
        tu nous as livrés\\au pouvoir de nos fautes.
        
           
         
        ${}^{7}Mais maintenant, Seigneur, c’est toi notre père\\.
        \\Nous sommes l’argile, c’est toi qui nous façonnes :
        nous sommes tous l’ouvrage de ta main.
${}^{8}Seigneur, ne t’irrite pas à l’excès,
        ne te rappelle pas la faute à jamais.
        \\Ah, de grâce, regarde :
        tous, nous sommes ton peuple !
        
           
         
${}^{9}Elles sont devenues un désert,
        tes villes saintes ;
        \\Sion est devenue un désert,
        Jérusalem, une désolation.
        
           
         
${}^{10}Notre Maison sainte et resplendissante,
        où nos pères te louaient,
        \\est devenue la proie du feu ;
        tout ce qui nous était cher est en ruine.
        
           
         
${}^{11}Peux-tu rester insensible à cela, Seigneur,
        te taire et nous humilier à l’excès ?
        
           
      
         
      \bchapter{}
${}^{1}Je me suis laissé approcher
        par qui ne me demandait rien,
        \\je me suis laissé trouver
        par ceux qui ne me cherchaient pas.
        \\J’ai dit : « Me voici ! Me voici ! »
        à une nation qui n’invoquait pas mon nom.
${}^{2}J’ai tendu les mains, tout le jour,
        vers un peuple rebelle,
        \\vers ceux qui suivent le mauvais chemin,
        entraînés par leurs pensées.
${}^{3}Ce peuple m’offense,
        ouvertement, sans cesse :
        \\ils sacrifient dans les jardins,
        brûlent de l’encens sur des briques,
${}^{4}ils habitent dans les tombeaux,
        passent la nuit dans des cachettes,
        \\ils mangent de la viande de porc,
        avec des sauces impures dans leurs plats ;
${}^{5}ils disent : « Retire-toi ! Ne m’approche pas,
        je suis trop saint pour toi ! »
        \\Cela fait monter en moi une fumée de colère,
        un feu qui brûle à longueur de jour.
${}^{6}Voilà, c’est écrit devant moi :
        je ne me tairai pas sans avoir réglé leur compte,
        tout leur compte,
${}^{7}le prix de leurs fautes et des fautes de leurs pères,
        toutes ensemble – dit le Seigneur ;
        \\ils ont fait brûler l’encens sur les montagnes,
        ils m’ont outragé sur les collines.
        \\Je mesurerai leur salaire, tout leur salaire,
        à leurs actions passées.
        
           
${}^{8}Ainsi parle le Seigneur :
        \\Quand on trouve du jus dans une grappe,
        on dit : « Ne la détruisez pas,
        car elle contient une bénédiction. »
        \\Ainsi ferai-je à cause de mes serviteurs,
        afin de ne pas tout détruire.
${}^{9}Je ferai sortir de Jacob une descendance,
        et de Juda quelqu’un
        qui possédera mes montagnes ;
        \\mes élus les posséderont,
        mes serviteurs y demeureront.
${}^{10}Le Sarone deviendra pacage de brebis,
        et le Val d’Akor un parc pour les bœufs,
        en faveur de mon peuple qui m’aura cherché.
${}^{11}Mais vous, qui abandonnez le Seigneur,
        qui oubliez ma montagne sainte,
        \\qui dressez une table pour le dieu Gad
        et remplissez une coupe de libation pour Meni,
${}^{12}je vous destine à l’épée ;
        tous, vous plierez le genou pour être abattus !
        \\Car j’ai appelé, et vous n’avez pas répondu,
        j’ai parlé, et vous n’avez pas écouté ;
        \\vous avez fait ce qui est mal à mes yeux,
        et vous avez choisi ce qui me déplaît.
${}^{13}C’est pourquoi, ainsi parle le Seigneur Dieu :
        \\Voici : mes serviteurs mangeront ;
        vous, vous aurez faim.
        \\mes serviteurs boiront ;
        vous, vous aurez soif.
        \\Voici : mes serviteurs seront pleins d’allégresse ;
        vous, vous serez pleins de honte.
${}^{14}mes serviteurs crieront de joie,
        le cœur en fête ;
        \\vous, vous pousserez des cris dans la douleur de votre cœur,
        vous hurlerez, l’esprit brisé !
${}^{15}Pour mes élus, votre nom servira de malédiction :
        « Qu’il te fasse mourir comme un tel, le Seigneur Dieu ! »
        \\– Mais à ses serviteurs Dieu donnera un autre nom.
${}^{16}Quiconque voudra se bénir lui-même dans le pays
        se bénira par le Dieu fidèle,
        \\et quiconque, dans le pays, fera un serment
        le fera par le Dieu fidèle.
        \\Car la détresse passée sera oubliée,
        elle aura disparu à mes yeux.
        ${}^{17}Oui, voici : je vais créer
        un ciel nouveau et une terre nouvelle,
        \\on ne se souviendra plus du passé,
        il ne reviendra plus à l’esprit.
        ${}^{18}Soyez plutôt dans la joie, exultez sans fin
        pour ce que je crée.
        \\Car je vais recréer Jérusalem,
        pour qu’elle soit exultation,
        et que son peuple devienne joie.
        ${}^{19}J’exulterai en Jérusalem,
        je trouverai ma joie dans mon peuple.
        \\On n’y entendra plus de pleurs
        ni de cris.
        ${}^{20}Là, plus de nourrisson emporté en quelques jours,
        ni d’homme qui ne parvienne au bout de sa vieillesse ;
        \\le plus jeune mourra centenaire,
        ne pas atteindre cent ans sera malédiction.
        ${}^{21}On bâtira des maisons, on y habitera ;
        on plantera des vignes, on mangera leurs fruits.
${}^{22}On ne bâtira pas pour qu’un autre habite,
        on ne plantera pas pour qu’un autre mange ;
        \\car les jours de mon peuple seront comme les jours d’un arbre,
        et mes élus jouiront des ouvrages de leurs mains.
${}^{23}Ils ne se fatigueront pas pour rien,
        ils n’enfanteront plus pour l’épouvante,
        \\car ils sont la descendance des bénis du Seigneur,
        eux et leur postérité.
${}^{24}Alors, avant qu’ils n’appellent,
        moi, je répondrai ;
        \\ils parleront encore
        que moi, je les aurai entendus.
${}^{25}Le loup et l’agneau auront même pâture,
        le lion, comme le bœuf, mangera du fourrage ;
        le serpent, lui, se nourrira de poussière.
        \\Il n’y aura plus de mal ni de corruption
        sur toute ma montagne sainte,
        – dit le Seigneur.
      <p class="cantique" id="bib_ct-at_32"><span class="cantique_label">Cantique AT 32</span> = <span class="cantique_ref"><a class="unitex_link" href="#bib_is_66_10">Is 66, 10-14b</a></span>
      
         
      \bchapter{}
${}^{1}Ainsi parle le Seigneur :
        \\Le ciel est mon trône,
        et la terre, l’escabeau de mes pieds.
        \\Où donc me bâtiriez-vous une maison ?
        Où serait le lieu de mon repos ?
${}^{2}Tout cela, c’est ma main qui l’a fait,
        et tout cela est à moi – oracle du Seigneur.
        \\Celui que je regarde, c’est le pauvre,
        celui qui a l’esprit abattu et tremble à ma parole.
        
           
         
${}^{3}On immole le bœuf,
        mais on abat aussi bien un homme ;
        \\on sacrifie le mouton,
        mais on brise la nuque d’un chien ;
        \\on présente une offrande,
        mais aussi bien du sang de porc ;
        \\on brûle de l’encens en mémorial,
        mais on adresse une bénédiction aux idoles !
        \\Ainsi, ces gens-là ont choisi leurs propres chemins,
        ils se complaisent dans leurs horreurs ;
${}^{4}eh bien moi, je choisirai leurs tourments
        et je ferai venir sur eux ce qu’ils redoutent,
        \\puisque j’ai appelé, et que personne n’a répondu,
        j’ai parlé, et personne n’a écouté.
        \\Ils ont fait ce qui est mal à mes yeux,
        ils ont choisi ce qui me déplaît.
        
           
${}^{5}Écoutez ce que dit le Seigneur,
        vous qui tremblez à sa parole.
        \\Vos frères, qui vous haïssent
        et vous rejettent à cause de mon nom,
        se sont moqués en disant :
        \\« Que le Seigneur manifeste sa gloire,
        et nous verrons votre joie ! »
        \\Eh bien, ce sont eux qui connaîtront la honte !
${}^{6}Une voix, un tumulte vient de la ville,
        une voix sort du Temple :
        \\c’est la voix du Seigneur,
        qui rend à ses ennemis ce qui leur est dû.
         
${}^{7}Avant d’être en travail,
        Sion a enfanté ;
        \\avant que lui viennent les douleurs,
        elle a accouché d’un garçon.
${}^{8}Qui a jamais entendu rien de tel ?
        Qui a jamais vu chose pareille ?
        \\Peut-on mettre au monde un pays en un jour ?
        Une nation est-elle enfantée en une fois ?
        \\Pourtant, Sion, à peine en travail, a enfanté ses fils !
${}^{9}Est-ce que moi, j’ouvrirais un passage à la vie,
        et je ne ferais pas enfanter ?
        – dit le Seigneur.
        \\Moi qui fais enfanter,
        je fermerais le passage de la vie ?
        – dit ton Dieu.
       
        ${}^{10}Réjouissez-vous avec Jérusalem !
        Exultez en elle, vous tous qui l’aimez !
        \\Avec elle, soyez pleins d’allégresse,
        vous tous qui la pleuriez !
         
        ${}^{11}Alors, vous serez nourris de son lait,
        rassasiés\\de ses consolations ;
        \\alors, vous goûterez\\avec délices
        à l’abondance\\de sa gloire.
         
        ${}^{12}Car le Seigneur le déclare :
        \\« Voici que je dirige vers elle
        la paix comme un fleuve
        \\et, comme un torrent qui déborde,
        la gloire des nations. »
         
        \\Vous serez nourris\\, portés sur la hanche ;
        vous serez choyés sur ses genoux.
        ${}^{13}Comme un enfant\\que sa mère console,
        ainsi, je vous consolerai.
        \\Oui, dans Jérusalem, vous serez consolés.
         
        ${}^{14}Vous verrez, votre cœur sera dans l’allégresse ;
        et vos os revivront comme l’herbe reverdit.
         
        \\Le Seigneur fera connaître sa puissance à ses serviteurs,
        il sera indigné par ses ennemis.
         
${}^{15}Car voici que le Seigneur arrive dans le feu,
        avec ses chars pareils à un ouragan,
        \\pour assouvir l’ardeur de sa colère,
        exécuter ses menaces par les flammes du feu.
         
${}^{16}Car le Seigneur vient juger par le feu,
        juger par son épée tout être de chair :
        nombreuses sont les victimes du Seigneur.
         
${}^{17}Ceux qui se sanctifient et se purifient
        pour entrer dans les jardins,
        derrière une idole placée au centre,
        \\ceux qui mangent de la viande de porc,
        des bêtes répugnantes et des souris,
        \\succomberont ensemble
        – oracle du Seigneur –
        ${}^{18}avec leurs actions et leurs pensées.
        \\Moi, je viens rassembler toutes les nations,
        de toute langue.
        \\Elles viendront et verront ma gloire :
        ${}^{19}je mettrai chez elles un signe !
        \\Et, du milieu d’elles, j’enverrai des rescapés
        vers les nations,
        \\vers Tarsis, Pouth et Loud, Mèshek, Rosh,
        Toubal et Yavane,
        \\vers les îles lointaines
        qui n’ont rien entendu de ma renommée,
        qui n’ont pas vu ma gloire ;
        \\ma gloire, ces rescapés\\l’annonceront
        parmi les nations.
        ${}^{20}Et, de toutes les nations, ils ramèneront tous vos frères,
        en offrande au Seigneur,
        \\sur des chevaux et des chariots, en litière,
        à dos de mulets et de dromadaires\\,
        \\jusqu’à ma montagne sainte, à Jérusalem,
        – dit le Seigneur.
        \\On les portera comme l’offrande qu’apportent les fils d’Israël,
        dans des vases purs, à la maison du Seigneur.
        ${}^{21}Je prendrai même des prêtres et des Lévites parmi eux,
        – dit le Seigneur.
         
${}^{22}Oui, comme le ciel nouveau et la terre nouvelle que je fais
        subsistent devant moi – oracle du Seigneur –,
        ainsi subsisteront votre descendance et votre nom !
${}^{23}Alors, de nouvelle lune en nouvelle lune,
        et de sabbat en sabbat,
        \\tout être de chair viendra se prosterner devant moi,
        – dit le Seigneur.
${}^{24}Et au-dehors, on verra les dépouilles des hommes
        qui se sont révoltés contre moi :
        \\leur vermine ne mourra pas,
        leur feu ne s’éteindra pas :
        \\ils n’inspireront que répulsion à tout être de chair.
