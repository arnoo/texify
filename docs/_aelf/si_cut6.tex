  
  
        ${}^{15}Je vais rappeler les œuvres du Seigneur.
        \\Ce que j’ai vu, je vais le raconter :
        c’est par sa parole que le Seigneur a réalisé ses œuvres,
        tel fut son décret par sa bénédiction.
        ${}^{16}Comme le soleil, dans son éclat, regarde chaque chose,
        ainsi la gloire du Seigneur rayonne dans toute son œuvre.
        ${}^{17}Il est impossible aux anges\\, les saints du Seigneur,
        de décrire toutes les merveilles
        \\que le Seigneur souverain de l’univers fit inébranlables
        pour que l’univers soit affermi dans sa gloire.
        ${}^{18}Le Seigneur a scruté les abîmes et les cœurs,
        il a discerné leurs subtilités.
        \\Car le Très-Haut possède toute connaissance,
        il a observé les signes des temps,
        ${}^{19}faisant connaître le passé et l’avenir,
        et dévoilant les traces des choses cachées.
        ${}^{20}Aucune pensée ne lui a échappé,
        pas une parole ne lui a été cachée.
        ${}^{21}Il a organisé les chefs-d’œuvre de sa sagesse,
        lui qui existe depuis toujours et pour toujours ;
        \\rien n’y fut ajouté ni retranché :
        il n’a eu besoin d’aucun conseiller.
        ${}^{22}Comme toutes ses œuvres sont attirantes,
        jusqu’à\\la plus petite étincelle qu’on peut apercevoir !
        ${}^{23}Tout cela vit et demeure à jamais,
        remplit son office et lui obéit\\.
        ${}^{24}Tout va par deux, l’un correspond à l’autre,
        il n’a rien fait de défectueux,
        ${}^{25}il a confirmé l’excellence d’une chose par l’autre ;
        qui se rassasierait de contempler sa gloire ?
      
         
      \bchapter{}
${}^{1}L’orgueil des hauteurs, c’est le firmament pur,
        le ciel qui se dévoile dans une vision de gloire.
${}^{2}Le soleil qui paraît proclame à son lever :
        « C’est chose admirable que l’œuvre du Très-Haut ! »
${}^{3}À son midi, il dessèche la terre.
        Devant son ardeur, qui pourrait tenir ?
${}^{4}On a beau activer la fournaise à la fonderie,
        le soleil embrase trois fois plus les montagnes.
        \\Il projette des vapeurs brûlantes
        et, quand il fait briller ses rayons, il éblouit les yeux.
${}^{5}Grand est le Seigneur qui l’a créé :
        sur son ordre, aussitôt il prend sa course.
        
           
         
${}^{6}La lune aussi, toujours fidèle à son rendez-vous,
        indique les époques et marque les temps.
${}^{7}C’est elle qui marque les fêtes,
        quand cet astre décroît après son plein.
${}^{8}Le mois lunaire lui doit son nom ;
        sa croissance, durant son cycle, est merveilleuse ;
        \\lampe au campement des armées célestes,
        elle resplendit au firmament du ciel.
        
           
         
${}^{9}L’éclat des astres fait la beauté du ciel,
        parure de lumière dans les hauteurs du Seigneur.
${}^{10}Ils se tiennent, selon son décret, aux ordres du Dieu Saint,
        ils ne se relâchent pas dans leurs veilles.
${}^{11}Regarde l’arc-en-ciel et bénis son créateur,
        tant il est beau dans son resplendissement :
${}^{12}il trace dans le ciel une courbe de gloire,
        les mains du Très-Haut l’ont tendu.
        
           
${}^{13}Le Très-Haut n’a qu’un ordre à donner pour que tombe la neige,
        il précipite les éclairs qui exécutent son décret.
${}^{14}Alors s’ouvrent les réservoirs célestes,
        et les nuages prennent leur envol comme des oiseaux.
${}^{15}Dans sa puissance, il durcit les nuages
        qui se pulvérisent en grêlons.
${}^{16}À sa vue, les montagnes chancellent ;
        quand il le veut, souffle le vent du sud,
${}^{17}la voix de son tonnerre invective la terre
        avec l’ouragan du nord et les cyclones.
        \\Il répand la neige comme un vol d’oiseaux qui se pose,
        elle descend comme les sauterelles s’abattent.
${}^{18}L’œil s’émerveille de l’éclat de sa blancheur,
        et l’esprit reste ébahi de la voir tomber.
${}^{19}Comme de sel il couvre la terre de givre,
        et le gel le transforme en pointes d’épines.
${}^{20}Le vent froid souffle du nord,
        et la glace se forme sur l’eau ;
        \\elle recouvre toute l’étendue des eaux,
        qu’elle revêt comme une cuirasse ;
${}^{21}et le vent dévore les montagnes, brûle le désert,
        il consume la verdure comme un feu.
${}^{22}À tout cela une brume soudaine porte remède ;
        la rosée se dépose après le vent brûlant et ramène la joie.
${}^{23}Selon son dessein, le Très-Haut a dompté l’abîme,
        il y a planté des îles.
${}^{24}Ceux qui naviguent sur la mer en décrivent les périls,
        nous n’en croyons pas nos oreilles.
${}^{25}Et là, ce sont choses étranges, merveilleuses,
        animaux de toute sorte et monstres marins de la création.
${}^{26}Celui que Dieu envoie y trouve son chemin :
        tout s’ordonne selon sa parole.
${}^{27}Nous parlerions longtemps sans en dire assez,
        un mot suffit : Il est le Tout.
${}^{28}Où trouver la force pour le glorifier ?
        Car il est le Grand, qui dépasse toutes ses œuvres.
${}^{29}Le Seigneur est redoutable et souverainement grand,
        sa puissance est merveilleuse.
${}^{30}Quand vous le glorifiez, exaltez le Seigneur autant que vous pouvez :
        il sera encore au-delà.
        \\Exaltez-le de toute votre force
        et ne vous découragez pas : vous ne sauriez en dire assez.
${}^{31}Qui l’a vu et pourrait le décrire ?
        Qui le magnifiera à la mesure de ce qu’il est ?
${}^{32}Il reste beaucoup de mystères plus grands que ceux-là,
        car nous ne voyons qu’une faible partie de ses œuvres.
${}^{33}Le Seigneur a créé toutes choses ;
        il donne la sagesse à qui est religieux.
      
         
      \bchapter{}
        ${}^{1}Faisons l’éloge de ces hommes glorieux
        qui sont nos ancêtres.
${}^{2}Le Seigneur a créé la gloire à profusion ;
        il manifeste sa grandeur depuis toujours.
${}^{3}C’étaient des souverains de royaumes,
        des hommes renommés pour leur puissance,
        \\des conseillers clairvoyants,
        des messagers de prophéties,
${}^{4}des guides du peuple par leurs conseils,
        leur compétence à l’instruire
        et les sages paroles de leur enseignement.
${}^{5}Ils inventaient des chants mélodieux
        et mettaient par écrit des récits poétiques.
${}^{6}C’étaient des hommes riches et influents,
        qui vivaient, paisibles, dans leurs domaines.
${}^{7}Tous ceux-là ont connu la gloire en leur temps
        et, de leur vivant, ils ont été à l’honneur.
${}^{8}Il y en a, parmi eux, qui ont laissé un nom ;
        ainsi peut-on faire leur éloge.
        ${}^{9}Il y en a d’autres dont le souvenir s’est perdu ;
        ils sont morts, et c’est comme s’ils n’avaient jamais existé,
        \\c’est comme s’ils n’étaient jamais nés,
        et de même leurs enfants après eux.
        ${}^{10}Il n’en est pas ainsi des hommes de miséricorde,
        leurs œuvres de justice n’ont pas été oubliées.
        ${}^{11}Avec leur postérité se maintiendra
        le bel héritage que sont leurs descendants.
        ${}^{12}Leur postérité a persévéré dans les lois de l’Alliance,
        leurs enfants y sont restés fidèles grâce à eux.
        ${}^{13}Leur descendance subsistera toujours,
        jamais leur gloire ne sera effacée.
${}^{14}Leurs corps ont été ensevelis dans la paix,
        et leur nom reste vivant pour toutes les générations.
${}^{15}Les peuples raconteront leur sagesse,
        l’assemblée proclamera leurs louanges.
        
           
${}^{16}Hénok fut agréable au Seigneur
        qui l’a retiré de ce monde ;
        \\c’est un exemple
        pour que se convertissent toutes les générations.
         
${}^{17}Noé fut trouvé juste, parfait ;
        au temps de la colère, il a été l’instrument de la réconciliation.
        \\Grâce à lui, un reste fut épargné sur la terre,
        lorsque le déluge arriva.
${}^{18}Le Seigneur fit une alliance avec lui pour toujours :
        les vivants ne seraient plus anéantis par un déluge.
${}^{19}Abraham fut grand,
        d’une multitude de peuples il est le père,
        personne n’a jamais égalé sa gloire.
${}^{20}Il observa la loi du Très-Haut
        et entra dans son alliance.
        \\Il inscrivit cette alliance dans sa chair
        et, jusque dans l’épreuve, il a été trouvé fidèle.
${}^{21}Aussi Dieu lui a-t-il assuré par serment
        que les nations seraient bénies en sa descendance,
        \\qu’il le multiplierait autant que la poussière sur la terre,
        qu’il exalterait ses descendants comme les étoiles ;
        \\il leur donnerait un héritage
        allant de la mer à la mer,
        et du Fleuve jusqu’à l’extrémité de la terre.
         
${}^{22}À Isaac il donna la même assurance
        à cause de son père Abraham.
         
        \\La bénédiction pour tous les hommes et l’Alliance,
${}^{23}Dieu les a fait reposer sur la tête de Jacob ;
        \\il l’a confirmé dans ses bénédictions
        et lui a donné un héritage,
        \\qu’il a divisé en lots
        et partagé entre les douze tribus.
        \\De Jacob, Dieu fit sortir un homme de miséricorde,
        qui trouva grâce aux yeux de tous
      
         
      \bchapter{}
${}^{1}et fut aimé de Dieu et des hommes :
        Moïse, dont la mémoire est en bénédiction.
${}^{2}Dieu lui a donné une gloire pareille à celle des anges,
        il l’a rendu grand, redouté de ses ennemis.
${}^{3}Par la parole de Moïse, il fit s’abattre des fléaux ;
        Dieu l’a glorifié à la face des rois,
        \\il lui a donné des commandements pour son peuple
        et lui a montré quelque chose de sa gloire.
${}^{4}Il l’a consacré pour sa fidélité et sa douceur
        et l’a choisi entre tous les vivants.
${}^{5}Il lui a fait entendre sa voix
        et l’a introduit dans la nuée obscure ;
        \\face à face, il lui a donné ses commandements,
        loi de vie et de savoir,
        \\pour enseigner l’Alliance à Jacob
        et ses décrets à Israël.
        
           
${}^{6}Aaron, Dieu l’a exalté,
        il en fit un saint comme Moïse,
        son frère, de la tribu de Lévi.
${}^{7}Il l’établit dans une alliance perpétuelle,
        faisant de lui le prêtre de son peuple ;
        \\il l’honora de splendides ornements
        et l’enveloppa d’une robe de gloire.
${}^{8}Il le revêtit d’une parure superbe
        et lui remit les insignes du pouvoir :
        longue tunique, éphod et caleçon de lin.
${}^{9}Il fixa au bord de son manteau
        grenades et clochettes d’or en grand nombre,
        \\qui devaient tinter quand il marchait
        et résonner dans le Temple,
        servant de mémorial pour les fils de son peuple.
${}^{10}Il l’entoura d’un vêtement sacré
        d’or, de pourpre violette et de pourpre rouge,
        œuvre d’un artisan brocheur ;
        \\il le revêtit du pectoral du jugement, oracle de vérité,
${}^{11}tissé de fils cramoisis, ouvrage d’un artiste,
        \\avec des pierres précieuses ciselées à la manière d’un sceau,
        serties dans une monture d’or, œuvre de lapidaire,
        \\et une inscription gravée pour servir de mémorial,
        selon le nombre des tribus d’Israël.
${}^{12}Il lui mit, par-dessus le turban, une couronne d’or
        gravée d’un sceau, marque de sa consécration ;
        \\insigne d’honneur, vrai chef-d’œuvre,
        cette parure attirait tous les regards.
${}^{13}Avant lui on n’avait rien vu d’aussi beau,
        et jamais un étranger ne pourrait revêtir ces ornements :
        \\ils sont réservés à ses fils et à ses descendants
        pour toujours.
${}^{14}Ses sacrifices se consumeront entièrement,
        deux fois par jour, à perpétuité.
${}^{15}Moïse lui a conféré l’investiture
        et donné l’onction avec l’huile sainte :
        \\c’est une alliance éternelle pour Aaron et pour sa descendance,
        tous les jours que durera le ciel,
        \\de servir Dieu, d’exercer le sacerdoce
        et de bénir le peuple par son nom.
${}^{16}Il a été choisi parmi tous les vivants
        pour présenter l’offrande au Seigneur,
        l’encens et le parfum en mémorial,
        et pour obtenir le pardon en faveur du peuple.
${}^{17}Le Seigneur lui a donné, dans ses commandements,
        autorité sur ses prescriptions et ses préceptes,
        \\pour enseigner à Jacob ses exigences
        et, par sa Loi, éclairer Israël.
${}^{18}Des étrangers à sa famille se soulevèrent contre Aaron
        et furent jaloux de lui au désert :
        \\les hommes de Datane et d’Abiram,
        et la bande de Coré, dans une colère furieuse.
${}^{19}Le Seigneur vit, et cela lui déplut ;
        il les extermina dans la fureur de sa colère ;
        \\il fit contre eux des prodiges,
        les dévorant par les flammes de son feu.
${}^{20}Il ajouta encore à la gloire d’Aaron
        et lui donna un héritage :
        \\il lui attribua comme part les prémices des récoltes
        et lui assura, en premier, le pain à satiété.
${}^{21}Les fils d’Aaron, en effet, ont pour nourriture les sacrifices offerts :
        le Seigneur les lui a donnés, ainsi qu’à sa descendance.
${}^{22}Par contre, il n’a pas d’héritage sur la terre de son peuple :
        dans son peuple, il n’y a aucune part pour lui,
        \\car le Seigneur lui a dit :
        « C’est moi, ta part et ton héritage. »
${}^{23}Pinhas, fils d’Éléazar, est le troisième en gloire
        pour son zèle dans la crainte du Seigneur
        \\et parce qu’il a tenu bon, lors de la révolte du peuple,
        avec un noble courage ;
        c’est ainsi qu’il obtint le pardon pour Israël.
${}^{24}C’est pourquoi le Seigneur conclut avec lui une alliance de paix :
        il l’établit chef du sanctuaire et du peuple
        \\pour qu’à lui et à sa descendance
        appartienne à jamais la dignité de grand prêtre.
${}^{25}Certes, il y eut aussi une alliance avec David,
        fils de Jessé, de la tribu de Juda.
        \\Mais l’héritage d’un roi passe à un seul de ses fils,
        tandis que l’héritage d’Aaron est pour toute sa descendance.
         
${}^{26}Que le Seigneur mette la sagesse en votre cœur, fils d’Aaron,
        pour juger son peuple avec justice,
        \\afin que les vertus de vos ancêtres ne disparaissent pas
        et que leur gloire se maintienne d’âge en âge.
      
         
      \bchapter{}
${}^{1}Josué, fils de Noun, fut un vaillant guerrier
        et succéda, comme prophète, à Moïse.
        \\Justifiant le nom qu’il portait,
        il se montra grand sauveur des élus du Seigneur :
        \\châtiant les ennemis dressés contre lui,
        il fit entrer Israël dans son héritage.
${}^{2}Qu’il était glorieux quand il levait les bras
        pour brandir l’épée contre les villes !
${}^{3}Qui donc, avant lui, avait eu cette fermeté ?
        Il mena lui-même les combats du Seigneur.
${}^{4}N’est-ce pas sa main qui arrêta le soleil
        et fit qu’un seul jour en devint deux ?
${}^{5}Il invoqua le Très-Haut, le Puissant,
        quand les ennemis le pressaient de toute part,
        \\et le souverain Seigneur l’exauça
        en lançant des grêlons d’une force terrible.
${}^{6}Il se précipita sur la nation ennemie
        et fit périr les adversaires dans la descente de Beth-Horone,
        \\pour faire connaître aux nations la force de ses armes
        et qu’il se battait au nom du Seigneur,
        car il marchait à la suite du Puissant.
${}^{7}Aux jours de Moïse déjà, il avait montré sa fidélité,
        – avec Caleb, fils de Yefounnè –,
        \\en tenant tête à l’assemblée,
        en détournant le peuple du péché
        et en apaisant les récriminations malveillantes.
${}^{8}Aussi, furent-ils épargnés tous les deux,
        seuls sur six cent mille hommes,
        \\pour faire entrer Israël dans son héritage,
        dans un pays ruisselant de lait et de miel.
${}^{9}Et le Seigneur donna à Caleb
        une vigueur qui lui resta jusque dans sa vieillesse,
        \\pour lui faire gravir les hauteurs d’un pays
        dont sa descendance conserva l’héritage.
${}^{10}C’est ainsi que tous les fils d’Israël virent
        combien il est bon de marcher à la suite du Seigneur.
        
           
${}^{11}Les Juges aussi ont laissé chacun leur nom :
        aucun n’a eu le cœur idolâtre,
        aucun ne s’est détourné du Seigneur.
        \\Que leur souvenir soit en bénédiction !
${}^{12}Du lieu où ils reposent, que leurs ossements refleurissent
        et que leur nom se renouvelle
        dans les fils de ces gens illustres !
${}^{13}Samuel fut aimé de son Seigneur.
        \\Comme prophète du Seigneur, il établit la royauté
        et donna l’onction aux chefs de son peuple.
${}^{14}Il fut juge dans l’assemblée selon la loi du Très-Haut,
        et le Seigneur a visité Jacob.
${}^{15}Par sa fidélité il s’est montré vrai prophète
        et, par ses oracles, il fut reconnu digne de foi dans ses visions.
${}^{16}Il invoqua le Seigneur, le Puissant,
        quand les ennemis le pressaient de toute part
        et il offrit un agneau de lait.
${}^{17}Le Seigneur tonna du haut du ciel
        et, dans un grand fracas, fit entendre sa voix ;
${}^{18}il extermina les chefs ennemis
        et tous les princes des Philistins.
${}^{19}Avant le moment du repos éternel,
        Samuel prit à témoin le Seigneur
        et celui à qui il avait donné l’onction ;
        \\il déclara :
        « Je n’ai jamais pris le bien de qui que ce soit,
        pas même une paire de sandales »,
        et personne ne l’accusa.
${}^{20}Même après s’être endormi, il prophétisa encore
        pour annoncer au roi sa fin prochaine ;
        \\du sein de la terre, il éleva une voix prophétique
        afin que soit effacée la faute du peuple.
       
      
         
      \bchapter{}
${}^{1}Après lui se leva Nathan,
        pour prophétiser aux jours de David.
        
           
        ${}^{2}Dans le sacrifice de communion,
        on met à part la graisse des animaux offerts à Dieu\\ ;
        ainsi David a été mis à part entre les fils d’Israël.
        ${}^{3}Il a joué avec les lions\\comme si c’étaient des chevreaux,
        et avec les ours comme avec des agneaux\\.
        ${}^{4}N’était-il pas tout jeune quand il a tué le géant
        et supprimé la honte de son peuple,
        \\lorsqu’il lança la pierre de sa fronde
        et abattit l’arrogance de Goliath\\ ?
        ${}^{5}Il invoqua le Seigneur Très-Haut
        qui a mis dans sa main la vigueur
        \\pour supprimer le puissant guerrier
        et pour exalter\\la force de son peuple.
        ${}^{6}C’est pourquoi on lui a fait gloire
        des dizaines de milliers qu’il a tués\\ :
        \\on l’a célébré en bénissant le Seigneur
        quand on lui a donné la glorieuse couronne royale.
        ${}^{7}En effet, il a détruit les ennemis alentour,
        il a anéanti ses adversaires philistins,
        \\il a détruit leur force comme on le voit encore aujourd’hui.
         
        ${}^{8}Dans tout ce qu’il a fait,
        il a célébré la louange du Saint, du Très-Haut,
        en proclamant sa gloire.
        \\De tout son cœur, il a chanté les psaumes,
        il a aimé son Créateur.
        ${}^{9}Devant l’autel, il a placé des chantres,
        et leur voix rendit les mélodies plus douces ;
        chaque jour ils loueront Dieu par leurs chants.
        ${}^{10}Il a donné de l’éclat aux fêtes,
        il a donné une parfaite splendeur aux solennités,
        \\pour que le saint nom du Seigneur soit célébré,
        et que les chants retentissent dans le sanctuaire dès le matin.
        ${}^{11}Le Seigneur a enlevé\\les péchés de David,
        il a pour toujours exalté sa force,
        \\il a fondé sur lui l’Alliance avec sa dynastie,
        le trône de gloire d’Israël\\.
${}^{12}Après David se leva un fils plein de savoir ;
        grâce à son père, il vécut en toute tranquillité.
${}^{13}Salomon connut un règne paisible,
        et Dieu lui accorda la sécurité alentour,
        \\pour qu’il élève une maison consacrée à son nom
        et qu’il établisse un sanctuaire à jamais.
${}^{14}Comme tu étais sage dans ta jeunesse !
        Tel un fleuve, tu débordais d’intelligence ;
${}^{15}ta pensée s’est répandue par toute la terre,
        que tu as remplie d’énigmes et de paraboles.
${}^{16}Ta renommée est parvenue jusqu’aux îles lointaines.
        Tu as été aimé parce que tu étais pacifique.
${}^{17}Tes chants, tes proverbes,
        tes paraboles et tes interprétations
        ont fait l’admiration du monde entier.
${}^{18}Au nom du Seigneur Dieu,
        celui qui est appelé le Dieu d’Israël,
        \\tu as amassé l’or, comme de l’étain
        et, comme du plomb, tu as accumulé l’argent.
${}^{19}Mais tu t’es vautré dans le plaisir avec les femmes
        et tu as été asservi dans ton corps.
${}^{20}Tu as terni ta gloire,
        tu as profané ta race,
        \\au point d’amener la colère sur tes enfants
        et provoquer des regrets par ta folie.
${}^{21}Ainsi la souveraineté fut partagée en deux
        et d’Éphraïm sortit un royaume rebelle.
${}^{22}Mais le Seigneur ne renonce pas à sa miséricorde,
        il ne détruit aucune de ses œuvres,
        \\il ne fait pas disparaître les descendants de son élu,
        il ne supprime pas la postérité de qui l’a aimé.
        \\À Jacob il a donné un reste,
        à David, un rejeton issu de lui.
${}^{23}Salomon reposa avec ses pères,
        laissant après lui, parmi sa descendance,
        \\le plus fou du peuple, un homme dénué d’intelligence,
        Roboam, qui, par sa décision, poussa le peuple à la révolte.
        \\Quant à Jéroboam, fils de Nebath, il fit pécher Israël ;
        c’est lui qui fit prendre à Éphraïm le chemin du péché.
${}^{24}Et tant se multiplièrent leurs péchés
        qu’ils furent déportés de leur pays.
${}^{25}Ils s’adonnèrent à tout ce qui est mal
        jusqu’à ce que vienne sur eux le châtiment.
      
         
      \bchapter{}
        ${}^{1}Le prophète Élie surgit comme un feu,
        sa parole brûlait comme une torche.
        ${}^{2}Il fit venir la famine sur Israël\\,
        et, dans son ardeur, les réduisit à un petit nombre\\.
        ${}^{3}Par la parole du Seigneur, il retint les eaux\\du ciel,
        et à trois reprises il en fit descendre le feu\\.
        ${}^{4}Comme tu étais redoutable\\, Élie, dans tes prodiges !
        Qui pourrait se glorifier d’être ton égal ?
        ${}^{5}Toi qui as réveillé un mort
        et, par la parole du Très-Haut, l’as fait revenir\\du séjour des morts ;
        ${}^{6}toi qui as précipité des rois vers leur perte,
        et jeté à bas de leur lit\\de glorieux personnages\\ ;
        ${}^{7}toi qui as entendu au Sinaï des reproches,
        au mont Horeb des décrets de châtiment\\ ;
        ${}^{8}toi qui as donné l’onction à des rois pour exercer la vengeance,
        et à des prophètes pour prendre ta succession\\ ;
        ${}^{9}toi qui fus enlevé dans un tourbillon de feu
        par un char aux coursiers de feu\\ ;
        ${}^{10}toi qui fus préparé\\pour la fin des temps,
        ainsi qu’il est écrit\\,
        \\afin d’apaiser la colère avant qu’elle n’éclate,
        afin de ramener le cœur des pères vers les fils
        et de rétablir les tribus de Jacob…
        ${}^{11}heureux ceux qui te verront\\,
        heureux ceux qui, dans l’amour, se seront endormis\\ ;
        nous aussi, nous posséderons la vraie vie\\.
        
           
         
        ${}^{12}Quand Élie fut enveloppé dans le tourbillon,
        \\Élisée fut rempli de son esprit\\,
        et pendant toute sa vie aucun prince ne l’a intimidé,
        personne n’a pu le faire fléchir.
        ${}^{13}Rien ne lui résista,
        et, jusque dans la tombe,
        \\son corps manifesta son pouvoir de prophète\\.
        ${}^{14}Pendant sa vie, il a fait des prodiges ;
        après sa mort, des œuvres merveilleuses.
        
           
         
${}^{15}Malgré tout cela, le peuple ne se repentit pas
        et ne renonça pas à ses péchés,
        \\jusqu’à ce qu’il soit emmené captif hors de son pays
        et dispersé par toute la terre.
        \\Il ne resta qu’un peuple très peu nombreux,
        avec un prince de la maison de David.
${}^{16}Quelques-uns d’entre eux firent ce qui plaît au Seigneur,
        d’autres multiplièrent les péchés.
        
           
${}^{17}Ézékias fortifia sa capitale,
        il amena l’eau à l’intérieur de la ville ;
        \\il fit creuser à coups de pic un tunnel dans le roc
        et construire des réservoirs pour les eaux.
${}^{18}De son temps, Sennakérib monta l’attaquer
        et envoya contre lui Rabsakès,
        \\qui leva la main contre Sion
        et se montra d’une grande arrogance.
${}^{19}Alors le cœur et les mains des assiégés tremblèrent,
        ils souffraient les douleurs d’une femme en travail.
${}^{20}Ils invoquèrent le Seigneur, le Miséricordieux,
        et tendirent les mains vers lui ;
        \\et le Saint, du haut des cieux, les exauça aussitôt,
        il les délivra selon la parole d’Isaïe.
${}^{21}Il frappa le camp des Assyriens,
        son ange les extermina.
${}^{22}Car Ézékias avait fait ce qui plaît au Seigneur,
        il était resté ferme dans les voies de son père David,
        \\comme l’avait ordonné le grand prophète Isaïe,
        digne de foi dans ses visions.
${}^{23}En ses jours, le soleil recula
        pour prolonger la vie du roi.
${}^{24}Par la puissance de l’esprit, Isaïe vit les derniers temps
        et consola les affligés de Sion.
${}^{25}Il révéla ce qui arriverait jusqu’à la fin des temps
        et les choses cachées avant qu’elles n’adviennent.
      
         
      \bchapter{}
${}^{1}Le souvenir de Josias est comme un mélange aromatique,
        préparé par les soins du parfumeur.
        \\Il est doux comme le miel dans la bouche,
        il est une musique dans un banquet bien arrosé.
${}^{2}C’est lui qui réussit à convertir le peuple,
        il supprima le culte abominable des idoles.
${}^{3}Il tourna son cœur vers le Seigneur
        et, dans ces temps d’abandon de la Loi, il raffermit la religion.
        
           
         
${}^{4}Hormis David, Ézékias et Josias,
        tous ne firent que faute sur faute.
        \\Ayant abandonné la loi du Très-Haut,
        les rois de Juda furent eux-mêmes abandonnés.
${}^{5}Ils durent céder leur pouvoir à d’autres,
        et leur gloire à une nation étrangère.
${}^{6}La ville élue, la ville du sanctuaire, fut incendiée,
        ses rues furent désertées,
${}^{7}selon la parole de Jérémie, qu’ils avaient maltraité,
        lui, le prophète consacré dès le sein de sa mère
        \\pour arracher, démolir et détruire,
        comme aussi pour bâtir et planter.
        
           
         
${}^{8}Ézékiel eut une vision de la Gloire :
        elle lui fut montrée sur le char des Kéroubim.
${}^{9}Il a prophétisé l’anéantissement des ennemis par des pluies torrentielles ;
        il a encouragé ceux qui suivent le droit chemin.
        
           
         
${}^{10}Quant aux douze prophètes,
        que refleurissent leurs ossements, depuis le lieu où ils reposent !
        \\Ils ont consolé Jacob
        et l’ont racheté par leur fidélité à l’espérance.
        
           
${}^{11}Comment dire la grandeur de Zorobabel,
        lui qui fut comme l’anneau à cacheter,
        le sceau que l’on porte à la main droite ?
${}^{12}Ou encore Josué, fils de Josédeq ?
        \\En leur temps, ils rebâtirent la maison de Dieu,
        ils relevèrent le Temple consacré au Seigneur
        et destiné à une gloire éternelle.
${}^{13}De même, Néhémie a laissé un souvenir éclatant,
        lui qui redressa nos murailles en ruines,
        rétablit portes et verrous
        et reconstruisit nos habitations.
${}^{14}Nul sur terre n’a été créé l’égal d’Hénok,
        lui qui fut enlevé de ce monde.
${}^{15}Et l’on n’a pas vu naître d’homme semblable à Joseph,
        chef de ses frères, soutien de son peuple ;
        ses ossements furent entourés d’honneurs.
${}^{16}Sem et Seth furent glorieux parmi les hommes,
        mais, dans la création, Adam surpasse tout être vivant.
      
         
      \bchapter{}
${}^{1}C’est le grand prêtre Simon, fils d’Onias,
        qui, pendant sa vie, répara la Maison de Dieu
        et, de son temps, consolida le sanctuaire.
${}^{2}C’est lui qui posa les fondations du double mur
        pour servir de soubassement à l’enceinte du Temple.
${}^{3}De son temps fut creusé le réservoir des eaux,
        un bassin large comme une mer.
${}^{4}Soucieux de préserver son peuple de la destruction,
        il fortifia la ville en prévision d’un siège.
${}^{5}Qu’il était glorieux, entouré de son peuple,
        quand il sortait de derrière le voile du Sanctuaire !
${}^{6}Il était comme l’étoile du matin au milieu des nuages,
        comme la lune au moment de son plein,
${}^{7}comme le soleil resplendissant sur le temple du Très-Haut,
        comme l’arc-en-ciel qui illumine de gloire les nuages,
${}^{8}comme la rose en fleur aux jours du printemps,
        comme le lis près des sources d’eaux,
        \\comme les jeunes pousses du Liban aux jours de l’été,
${}^{9}comme l’encens qui brûle sur l’encensoir,
        \\comme un vase d’or massif
        tout orné de pierres précieuses,
${}^{10}comme l’olivier qui se couvre de fruits,
        comme le cyprès qui s’élève jusqu’aux nuages.
${}^{11}Quand il prenait sa robe de gloire
        et revêtait sa superbe parure,
        \\quand il montait au saint autel,
        il remplissait de gloire l’enceinte du sanctuaire.
${}^{12}Et quand, debout près du foyer de l’autel,
        il recevait, de la main des prêtres, les viandes immolées,
        \\ses frères autour de lui formaient une couronne,
        pareils à la ramure des cèdres du Liban ;
        ils l’entouraient comme des tiges de palmiers.
${}^{13}Tous les fils d’Aaron, dans leur gloire,
        tenaient en mains l’offrande du Seigneur
        devant toute l’assemblée d’Israël.
${}^{14}Le grand prêtre accomplissait à l’autel les fonctions liturgiques
        et disposait l’offrande du Très-Haut, souverain de l’univers.
${}^{15}Il étendait la main sur la coupe
        et versait la libation du sang de la grappe,
        \\il la répandait à la base de l’autel,
        comme un parfum d’agréable odeur pour le Très-Haut, le Roi suprême.
${}^{16}Alors les fils d’Aaron poussaient des acclamations,
        ils sonnaient de leurs trompettes d’argent,
        \\ils faisaient entendre un son puissant
        en mémorial devant le Très-Haut.
${}^{17}Alors tout le peuple, d’un même élan,
        se jetait aussitôt face contre terre
        \\pour se prosterner devant son Seigneur,
        devant le Souverain de l’univers, le Dieu Très-Haut.
${}^{18}Et les chantres le louaient de toute leur voix,
        harmonieuse mélodie dans la clameur immense.
${}^{19}Le peuple suppliait le Seigneur Très-Haut,
        il se tenait en prière devant le Miséricordieux,
        \\jusqu’à ce que soit terminé le service du Seigneur
        et que s’achève la liturgie.
${}^{20}Alors le grand prêtre redescendait,
        il levait les mains sur toute l’assemblée des fils d’Israël
        \\pour donner lui-même la bénédiction du Seigneur
        et avoir l’honneur de prononcer son nom.
${}^{21}Une seconde fois, le peuple se prosternait
        pour recevoir la bénédiction du Très-Haut.
        
           
        ${}^{22}Et maintenant, bénissez le Dieu de l’univers :
        partout il fait de grandes choses,
        \\il nous fait croître\\dès le sein maternel,
        il agit envers nous selon sa miséricorde.
        ${}^{23}Qu’il nous accorde la joie du cœur,
        que la paix règne en Israël,
        aujourd’hui et pour toujours\\.
        ${}^{24}Que sa miséricorde nous soit fidèle
        et qu’il nous délivre tout au long de notre\\vie.
${}^{25}Il y a deux nations que mon âme déteste,
        et la troisième n’est pas une nation :
${}^{26}les habitants de la montagne de Séïr, les Philistins,
        et le peuple fou qui habite à Sichem.
${}^{27}Cet enseignement, plein d’intelligence et de savoir,
        a été gravé dans ce livre par Jésus,
        fils de Sira, fils d’Éléazar, de Jérusalem,
        qui a répandu comme une ondée la sagesse de son cœur.
${}^{28}Heureux qui reviendra toujours à ces paroles !
        Celui qui les place en son cœur deviendra sage.
${}^{29}S’il les met en pratique, il sera fort en toute circonstance,
        car la lumière du Seigneur sera son sentier :
        Dieu donne la sagesse à qui est religieux.
        \\Béni soit le Seigneur pour toujours ! Amen ! Amen !
