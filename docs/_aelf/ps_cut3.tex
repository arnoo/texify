  
  
          
            \bchapter{Psaume}
            Tu es avec moi
${}^{1}Psaume. De David.
         
        \\Le Seigne\underline{u}r est mon berger :
        je ne m\underline{a}nque de rien. *
${}^{2}Sur des pr\underline{é}s d’herbe fraîche,
        il me f\underline{a}it reposer.
         
        \\Il me mène vers les ea\underline{u}x tranquilles
${}^{3}et me f\underline{a}it revivre ; *
        \\il me conduit par le j\underline{u}ste chemin
        pour l’honne\underline{u}r de son nom.
         
${}^{4}Si je traverse les rav\underline{i}ns de la mort,
        je ne cr\underline{a}ins aucun mal, *
        \\car tu \underline{e}s avec moi :
        ton bâton me gu\underline{i}de et me rassure.
         
${}^{5}Tu prépares la t\underline{a}ble pour moi
        dev\underline{a}nt mes ennemis ; *
        \\tu répands le parf\underline{u}m sur ma tête,
        ma co\underline{u}pe est débordante.
         
${}^{6}Grâce et bonhe\underline{u}r m’accompagnent
        tous les jo\underline{u}rs de ma vie ; *
        \\j’habiterai la mais\underline{o}n du Seigneur
        pour la dur\underline{é}e de mes jours.
      \bchapter{Psaume}
          
            \bchapter{Psaume}
            C’est lui, le roi de gloire
${}^{1}Psaume. De David.
         
        \\Au Seigneur, le m\underline{o}nde et sa richesse,
        \\la terre et to\underline{u}s ses habitants !
${}^{2}C’est lui qui l’a fond\underline{é}e sur les mers
        \\et la garde inébranl\underline{a}ble sur les flots.
         
${}^{3}Qui peut gravir la mont\underline{a}gne du Seigneur
        \\et se ten\underline{i}r dans le lieu saint ?
${}^{4}L’homme au cœur pur, aux m\underline{a}ins innocentes,
        \\qui ne livre pas son \underline{â}me aux idoles
        (et ne dit p\underline{a}s de faux serments).
         
${}^{5}Il obtient, du Seigne\underline{u}r, la bénédiction,
        \\et de Dieu son Sauve\underline{u}r, la justice.
${}^{6}Voici le peuple de ce\underline{u}x qui le cherchent !
        \\Voici Jacob qui rech\underline{e}rche ta face !
         
        *
         
${}^{7}Portes, lev\underline{e}z vos frontons, +
        \\élevez-vous, p\underline{o}rtes éternelles :
        \\qu’il entre, le r\underline{o}i de gloire !
         
${}^{8}Qui est ce r\underline{o}i de gloire ? +
        \\C’est le Seigneur, le f\underline{o}rt, le vaillant,
        \\le Seigneur, le vaill\underline{a}nt des combats.
         
${}^{9}Portes, lev\underline{e}z vos frontons, +
        \\levez-les, p\underline{o}rtes éternelles :
        \\qu’il entre, le r\underline{o}i de gloire !
         
${}^{10}Qui donc est ce r\underline{o}i de gloire ? +
        \\C’est le Seigneur, Die\underline{u} de l’univers ;
        \\c’est lui, le r\underline{o}i de gloire.
      \bchapter{Psaume}
          
            \bchapter{Psaume}
            Dans ton amour, ne m’oublie pas
${}^{1}De David.
         
        \\Vers toi, Seigneur, j’él\underline{è}ve mon âme, *
${}^{2}vers t\underline{o}i, mon Dieu.
         
        *
         
        \\Je m’appuie sur toi : ép\underline{a}rgne-moi la honte ;
        \\ne laisse pas triomph\underline{e}r mon ennemi.
${}^{3}Pour qui espère en t\underline{o}i, pas de honte,
        \\mais honte et décepti\underline{o}n pour qui trahit.
         
${}^{4}Seigneur, enseigne-m\underline{o}i tes voies,
        \\fais-moi conn\underline{a}ître ta route.
${}^{5}Dirige-moi par ta vérit\underline{é}, enseigne-moi,
        \\car tu es le Die\underline{u} qui me sauve.
         
        \\C’est toi que j’esp\underline{è}re tout le jour
        \\en raison de ta bont\underline{é}, Seigneur.
${}^{6}Rappelle-toi, Seigne\underline{u}r, ta tendresse,
        \\ton amour qui \underline{e}st de toujours.
${}^{7}Oublie les révoltes, les péch\underline{é}s de ma jeunesse ;
        \\dans ton amo\underline{u}r, ne m’oublie pas.
         
        *
         
${}^{8}Il est droit, il est b\underline{o}n, le Seigneur,
        \\lui qui montre aux péche\underline{u}rs le chemin.
${}^{9}Sa justice dir\underline{i}ge les humbles,
        \\il enseigne aux h\underline{u}mbles son chemin.
         
${}^{10}Les voies du Seigneur sont amo\underline{u}r et vérité
        \\pour qui veille à son alli\underline{a}nce et à ses lois.
${}^{11}À cause de ton n\underline{o}m, Seigneur,
        \\pardonne ma fa\underline{u}te : elle est grande.
         
${}^{12}Est-il un homme qui cr\underline{a}igne le Seigneur ?
        \\Dieu lui montre le chem\underline{i}n qu’il doit prendre.
${}^{13}Son âme habiter\underline{a} le bonheur,
        \\ses descendants posséder\underline{o}nt la terre.
${}^{14}Le secret du Seigneur est pour ce\underline{u}x qui le craignent ;
        \\à ceux-là, il fait conn\underline{a}ître son alliance.
         
        *
         
${}^{15}J’ai les yeux tourn\underline{é}s vers le Seigneur :
        \\il tirera mes pi\underline{e}ds du filet.
${}^{16}Regarde, et pr\underline{e}nds pitié de moi,
        \\de moi qui suis se\underline{u}l et misérable.
         
${}^{17}L’angoisse grand\underline{i}t dans mon cœur :
        \\tire-m\underline{o}i de ma détresse.
${}^{18}Vois ma mis\underline{è}re et ma peine,
        \\enlève to\underline{u}s mes péchés.
         
${}^{19}Vois mes ennem\underline{i}s si nombreux,
        \\la haine viol\underline{e}nte qu’ils me portent.
${}^{20}Garde mon \underline{â}me, délivre-moi ;
        \\je m’abrite en toi : ép\underline{a}rgne-moi la honte.
${}^{21}Droiture et perfection v\underline{e}illent sur moi,
        \\sur m\underline{o}i qui t’espère !
         
${}^{22}Libère Isra\underline{ë}l, ô mon Dieu,
        \\de to\underline{u}tes ses angoisses !
      \bchapter{Psaume}
          
            \bchapter{Psaume}
            J’aime la maison que tu habites
${}^{1}De David.
         
        \\Seigneur, rends-moi justice :
        j’ai march\underline{é} sans faillir. *
        \\Je m’appuie sur le Seigneur,
        et ne f\underline{a}iblirai pas.
${}^{2}Éprouve-moi, Seigne\underline{u}r, scrute-moi, *
        \\passe au feu mes r\underline{e}ins et mon cœur.
         
${}^{3}J’ai devant les ye\underline{u}x ton amour,
        \\je marche sel\underline{o}n ta vérité.
${}^{4}Je ne m’assieds p\underline{a}s chez l’imposteur,
        \\je n’entre p\underline{a}s chez l’hypocrite.
${}^{5}L’assemblée des méch\underline{a}nts, je la hais,
        \\je ne m’assieds p\underline{a}s chez les impies.
         
${}^{6}Je lave mes mains en s\underline{i}gne d’innocence
        \\pour approcher de ton aut\underline{e}l, Seigneur,
${}^{7}pour dire à pleine v\underline{o}ix l’action de grâce
        \\et rappeler to\underline{u}tes tes merveilles.
${}^{8}Seigneur, j’aime la mais\underline{o}n que tu habites,
        \\le lieu où deme\underline{u}re ta gloire.
         
${}^{9}Ne m’inflige pas le s\underline{o}rt des pécheurs,
        \\le destin de ceux qui v\underline{e}rsent le sang :
${}^{10}ils ont dans les m\underline{a}ins la corruption ;
        \\leur droite est pl\underline{e}ine de profits.
         
${}^{11}Oui, j’ai march\underline{é} sans faillir :
        \\libère-moi ! pr\underline{e}nds pitié de moi !
${}^{12}Sous mes pieds le terr\underline{a}in est sûr ;
        \\dans l’assemblée je bénir\underline{a}i le Seigneur.
      \bchapter{Psaume}
          
            \bchapter{Psaume}
            Ma lumière et mon salut
${}^{1}De David.
         
        \\Le Seigneur est ma lumi\underline{è}re et mon salut ;
        de qu\underline{i} aurais-je crainte ? *
        \\Le Seigneur est le remp\underline{a}rt de ma vie ;
        devant qu\underline{i} tremblerais-je ?
         
${}^{2}Si des méchants s’av\underline{a}ncent contre moi
        po\underline{u}r me déchirer, +
        \\ce sont eux, mes ennem\underline{i}s, mes adversaires, *
        qui perdent pi\underline{e}d et succombent.
         
${}^{3}Qu’une armée se dépl\underline{o}ie devant moi,
        mon cœ\underline{u}r est sans crainte ; *
        \\que la bataille s’eng\underline{a}ge contre moi,
        je g\underline{a}rde confiance.
         
${}^{4}J’ai demandé une ch\underline{o}se au Seigneur,
        la se\underline{u}le que je cherche : +
        \\habiter la mais\underline{o}n du Seigneur
        tous les jo\underline{u}rs de ma vie, *
        \\pour admirer le Seigne\underline{u}r dans sa beauté
        et m’attach\underline{e}r à son temple.
         
${}^{5}Oui, il me rés\underline{e}rve un lieu sûr
        au jo\underline{u}r du malheur ; +
        \\il me cache au plus secr\underline{e}t de sa tente,
        il m’él\underline{è}ve sur le roc. *
${}^{6}Maintenant je rel\underline{è}ve la tête
        dev\underline{a}nt mes ennemis.
         
        \\J’irai célébr\underline{e}r dans sa tente
        le sacrif\underline{i}ce d’ovation ; *
        \\je chanterai, je fêter\underline{a}i le Seigneur.
         
        *
         
${}^{7}Écoute, Seigne\underline{u}r, je t’appelle ! *
        Piti\underline{é} ! Réponds-moi !
${}^{8}Mon cœur m’a red\underline{i}t ta parole :
        « Cherch\underline{e}z ma face. » *
        \\C’est ta face, Seigne\underline{u}r, que je cherche :
${}^{9}ne me cache p\underline{a}s ta face.
         
        \\N’écarte pas ton servite\underline{u}r avec colère : *
        tu r\underline{e}stes mon secours.
        \\Ne me laisse pas, ne m’aband\underline{o}nne pas,
        Die\underline{u}, mon salut !*
${}^{10}Mon père et ma m\underline{è}re m’abandonnent ;
        le Seigne\underline{u}r me reçoit.
         
${}^{11}Enseigne-moi ton chem\underline{i}n, Seigneur, *
        \\conduis-moi par des ro\underline{u}tes sûres,
        malgré ce\underline{u}x qui me guettent.
${}^{12}Ne me livre pas à la merc\underline{i} de l’adversaire : *
        \\contre moi se sont lev\underline{é}s de faux témoins
        qui so\underline{u}fflent la violence.
         
${}^{13}Mais j’en suis sûr, je verrai les bont\underline{é}s du Seigneur
        sur la t\underline{e}rre des vivants. *
${}^{14}« Espère le Seigneur, sois f\underline{o}rt et prends courage ;
        esp\underline{è}re le Seigneur. »
      \bchapter{Psaume}
          
            \bchapter{Psaume}
            Ne reste pas sans me répondre
${}^{1}De David.
         
        \\Seigneur, mon rocher, c’est t\underline{o}i que j’appelle : +
        ne reste p\underline{a}s sans me répondre, *
        \\car si tu gard\underline{a}is le silence,
        je m’en irais, moi auss\underline{i}, vers la tombe.
         
${}^{2}Entends la v\underline{o}ix de ma prière
        quand je cr\underline{i}e vers toi, *
        \\quand j’él\underline{è}ve les mains
        vers le S\underline{a}int des Saints !
         
${}^{3}Ne me traîne p\underline{a}s chez les impies,
        chez les h\underline{o}mmes criminels ; *
        \\à leurs voisins ils p\underline{a}rlent de paix
        quand le m\underline{a}l est dans leur cœur.
         
${}^{4}\[Traite-l\underline{e}s d’après leurs actes
        et sel\underline{o}n leurs méfaits ; *
        \\traite-l\underline{e}s d’après leurs œuvres,
        rends-le\underline{u}r ce qu’ils méritent.
         
${}^{5}Ils n’ont compris ni l’acti\underline{o}n du Seigneur
        ni l’œ\underline{u}vre de ses mains ; *
        \\que Die\underline{u} les renverse
        et jam\underline{a}is ne les relève !\]
         
        *
         
${}^{6}Bén\underline{i} soit le Seigneur *
        \\qui entend la v\underline{o}ix de ma prière !
         
${}^{7}Le Seigneur est ma f\underline{o}rce et mon rempart ;
        \\à lui, mon cœ\underline{u}r fait confiance :
        \\il m’a guéri, ma ch\underline{a}ir a refleuri,
        \\mes chants lui r\underline{e}ndent grâce.
         
${}^{8}Le Seigneur est la f\underline{o}rce de son peuple,
        \\le refuge et le sal\underline{u}t de son messie.
${}^{9}Sauve ton peuple, bén\underline{i}s ton héritage,
        \\veille sur lui, porte-l\underline{e} toujours.
      \bchapter{Psaume}
          
            \bchapter{Psaume}
            Voix du Seigneur
${}^{1}Psaume. De David.
         
        \\Rendez au Seigneur, vo\underline{u}s, les dieux,
        \\rendez au Seigneur gl\underline{o}ire et puissance.
         
${}^{2}Rendez au Seigneur la gl\underline{o}ire de son nom,
        \\adorez le Seigneur, éblouiss\underline{a}nt de sainteté.
         
${}^{3}La voix du Seigneur dom\underline{i}ne les eaux, +
        \\le Dieu de la gloire déch\underline{a}îne le tonnerre,
        \\le Seigneur domine la m\underline{a}sse des eaux.
         
${}^{4}Voix du Seigne\underline{u}r dans sa force, +
        \\voix du Seigne\underline{u}r qui éblouit,
${}^{5}voix du Seigneur : elle c\underline{a}sse les cèdres.
         
        \\Le Seigneur fracasse les c\underline{è}dres du Liban ; +
${}^{6}il fait bondir comme un poul\underline{a}in le Liban,
        \\le Sirion, comme un je\underline{u}ne taureau.
         
${}^{7}Voix du Seigneur : elle taille des l\underline{a}mes de feu ; +
${}^{8}voix du Seigneur : elle épouv\underline{a}nte le désert ;
        \\le Seigneur épouvante le dés\underline{e}rt de Cadès.
         
${}^{9}Voix du Seigneur qui affole les biches en travail,
        qui rav\underline{a}ge les forêts. *
        \\Et tous dans son temple s’écr\underline{i}ent : « Gloire ! »
         
${}^{10}Au déluge le Seigne\underline{u}r a siégé ;
        \\il siège, le Seigneur, il est r\underline{o}i pour toujours !
         
${}^{11}Le Seigneur accorde à son pe\underline{u}ple la puissance,
        \\le Seigneur bénit son peuple en lui donn\underline{a}nt la paix.
      \bchapter{Psaume}
          
            \bchapter{Psaume}
            Tu m’as guéri
${}^{1}Psaume. Cantique pour la dédicace de la Maison. De David.
         
${}^{2}Je t’exalte, Seigne\underline{u}r : tu m’as relevé,
        \\tu m’épargnes les r\underline{i}res de l’ennemi.
         
${}^{3}Quand j’ai crié vers t\underline{o}i, Seigneur,
        mon Die\underline{u}, tu m’as guéri ; *
${}^{4}Seigneur, tu m’as fait remont\underline{e}r de l’abîme
        et revivre quand je descend\underline{a}is à la fosse.
         
${}^{5}Fêtez le Seigneur, vo\underline{u}s, ses fidèles,
        \\rendez grâce en rappel\underline{a}nt son nom très saint.
         
${}^{6}Sa colère ne d\underline{u}re qu’un instant,
        sa bont\underline{é}, toute la vie ; *
        \\avec le soir, vi\underline{e}nnent les larmes,
        mais au mat\underline{i}n, les cris de joie.
         
        *
         
${}^{7}Dans mon bonhe\underline{u}r, je disais :
        \\Rien, jam\underline{a}is, ne m’ébranlera !
         
${}^{8}Dans ta bonté, Seigneur, tu m’av\underline{a}is fortifié
        sur ma puiss\underline{a}nte montagne ; *
        \\pourtant, tu m’as cach\underline{é} ta face
        et je f\underline{u}s épouvanté.
         
${}^{9}Et j’ai crié vers t\underline{o}i, Seigneur,
        \\j’ai suppli\underline{é} mon Dieu :
         
${}^{10}« À quoi te servir\underline{a}it mon sang
        si je descend\underline{a}is dans la tombe ? *
        \\La poussière peut-\underline{e}lle te rendre grâce
        et proclam\underline{e}r ta fidélité ?
         
${}^{11}Écoute, Seigne\underline{u}r, pitié pour moi !
        \\Seigneur, vi\underline{e}ns à mon aide ! »
         
        *
         
${}^{12}Tu as changé mon de\underline{u}il en une danse,
        \\mes habits funèbres en par\underline{u}re de joie.
         
${}^{13}Que mon cœur ne se t\underline{a}ise pas,
        qu’il soit en f\underline{ê}te pour toi, *
        \\et que sans fin, Seigne\underline{u}r, mon Dieu,
        je te r\underline{e}nde grâce !
      \bchapter{Psaume}
          
            \bchapter{Psaume}
            En tes mains je remets mon esprit
${}^{1}Du maître de chœur. Psaume. De David.
         
${}^{2}En toi, Seigne\underline{u}r, j’ai mon refuge ;
        \\garde-moi d’être humili\underline{é} pour toujours.
         
        *
         
        \\Dans ta justice, l\underline{i}bère-moi ;
${}^{3}écoute, et vi\underline{e}ns me délivrer.
        \\Sois le roch\underline{e}r qui m’abrite,
        \\la maison fortifi\underline{é}e qui me sauve.
         
${}^{4}Ma forteresse et mon r\underline{o}c, c’est toi :
        \\pour l’honneur de ton nom, tu me gu\underline{i}des et me conduis.
${}^{5}Tu m’arraches au fil\underline{e}t qu’ils m’ont tendu ;
        \\oui, c’est t\underline{o}i mon abri.
         
${}^{6}En tes mains je rem\underline{e}ts mon esprit ;
        \\tu me rachètes, Seigneur, Die\underline{u} de vérité.
${}^{7}Je hais les adorate\underline{u}rs de faux dieux,
        \\et moi, je suis s\underline{û}r du Seigneur.
         
${}^{8}Ton amour me fait dans\underline{e}r de joie :
        \\tu vois ma misère et tu s\underline{a}is ma détresse.
${}^{9}Tu ne m’as pas livré aux m\underline{a}ins de l’ennemi ;
        \\devant moi, tu as ouv\underline{e}rt un passage.
         
        *
         
${}^{10}Prends pitié de m\underline{o}i, Seigneur,
        je su\underline{i}s en détresse. *
        \\La douleur me r\underline{o}nge les yeux,
        la g\underline{o}rge et les entrailles.
         
${}^{11}Ma vie s’ach\underline{è}ve dans les larmes,
        et mes ann\underline{é}es, dans les souffrances. *
        \\Le péché m’a fait p\underline{e}rdre mes forces,
        il me r\underline{o}nge les os.
         
${}^{12}Je suis la risée de mes adversaires et m\underline{ê}me de mes voisins, +
        je fais pe\underline{u}r à mes amis *
        \\(s’ils me voient dans la r\underline{u}e, ils me fuient).
${}^{13}On m’ignore comme un m\underline{o}rt oublié, *
        comme une ch\underline{o}se qu’on jette.
         
${}^{14}J’entends les calomn\underline{i}es de la foule :
        de tous côt\underline{é}s c’est l’épouvante. *
        \\Ils ont tenu cons\underline{e}il contre moi,
        ils s’accordent pour m’ôt\underline{e}r la vie.
         
        *
         
${}^{15}Moi, je suis sûr de t\underline{o}i, Seigneur, +
        je dis : « Tu \underline{e}s mon Dieu ! » *
${}^{16}Mes jours sont dans ta m\underline{a}in : délivre-moi
        des mains host\underline{i}les qui s’acharnent.
         
${}^{17}Sur ton serviteur, que s’illum\underline{i}ne ta face ; +
        sauve-m\underline{o}i par ton amour. *
${}^{18}Seigneur, garde-m\underline{o}i d’être humilié,
        m\underline{o}i qui t’appelle.
         
        \\\[Mais qu’ils soient humili\underline{é}s, les impies ; *
        qu’ils entrent dans le sil\underline{e}nce des morts !
${}^{19}Qu’ils deviennent mu\underline{e}ts, ces menteurs, *
        car ils parlent contre le juste
        avec orgueil, insol\underline{e}nce et mépris.\]
         
        *
         
${}^{20}Qu’ils sont gr\underline{a}nds, tes bienfaits ! +
        Tu les réserves à ce\underline{u}x qui te craignent. *
        \\Tu combles, à la f\underline{a}ce du monde,
        ceux qui ont en t\underline{o}i leur refuge.
         
${}^{21}Tu les caches au plus secr\underline{e}t de ta face,
        loin des intr\underline{i}gues des hommes. *
        \\Tu leur rés\underline{e}rves un lieu sûr,
        loin des l\underline{a}ngues méchantes.
         
${}^{22}Bén\underline{i} soit le Seigneur : *
        \\son amour a fait pour m\underline{o}i des merveilles
        dans la v\underline{i}lle retranchée !
         
${}^{23}Et moi, dans mon tro\underline{u}ble, je disais :
        « Je ne suis pl\underline{u}s devant tes yeux. » *
        \\Pourtant, tu écout\underline{a}is ma prière
        quand je cri\underline{a}is vers toi.
         
${}^{24}Aimez le Seigneur, vo\underline{u}s, ses fidèles : +
        le Seigneur v\underline{e}ille sur les siens ; *
        \\mais il rétrib\underline{u}e avec rigueur
        qui se m\underline{o}ntre arrogant.
         
${}^{25}Soyez f\underline{o}rts, prenez courage, *
        \\vous tous qui espér\underline{e}z le Seigneur !
      \bchapter{Psaume}
          
            \bchapter{Psaume}
            Tu as enlevé ma faute
${}^{1}De David. Poème.
         
        \\Heureux l’homme dont la fa\underline{u}te est enlevée, *
        \\et le péch\underline{é} remis !
${}^{2}Heureux l’homme dont le Seigneur ne retient p\underline{a}s l’offense, *
        \\dont l’espr\underline{i}t est sans fraude !
         
        *
         
${}^{3}Je me taisais et mes f\underline{o}rces s’épuisaient
        à gém\underline{i}r tout le jour : +
${}^{4}ta main, le jo\underline{u}r et la nuit,
        pes\underline{a}it sur moi ; *
        \\ma vigue\underline{u}r se desséchait
        comme l’h\underline{e}rbe en été.
         
${}^{5}Je t’ai fait conn\underline{a}ître ma faute,
        je n’ai pas cach\underline{é} mes torts. +
        \\J’ai dit : « Je rendrai gr\underline{â}ce au Seigneur
        en confess\underline{a}nt mes péchés. » *
        \\Et t\underline{o}i, tu as enlevé
        l’off\underline{e}nse de ma faute.
         
${}^{6}Ainsi chacun des ti\underline{e}ns te priera
        aux he\underline{u}res décisives ; *
        \\même les ea\underline{u}x qui débordent
        ne pe\underline{u}vent l’atteindre.
         
${}^{7}Tu es un ref\underline{u}ge pour moi,
        mon abr\underline{i} dans la détresse ; *
        \\de ch\underline{a}nts de délivrance,
        tu m’\underline{a}s entouré.
         
        *
         
${}^{8}« Je vais t’instruire, te montr\underline{e}r la route à suivre, *
        te conseill\underline{e}r, veiller sur toi.
         
${}^{9}N’imite pas les m\underline{u}les et les chevaux
        qui ne compr\underline{e}nnent pas, +
        \\qu’il faut mater par la br\underline{i}de et le mors, *
        et ri\underline{e}n ne t’arrivera. »
         
${}^{10}Pour le méchant, doule\underline{u}rs sans nombre ; *
        \\mais l’amour du Seigne\underline{u}r entourera
        ceux qui c\underline{o}mptent sur lui.
         
${}^{11}Que le Seigne\underline{u}r soit votre joie !
        Exult\underline{e}z, hommes justes ! *
        \\Hommes droits, chant\underline{e}z votre allégresse !
      \bchapter{Psaume}
          
            \bchapter{Psaume}
            Heureux le peuple dont le Seigneur est le Dieu
${}^{1}Criez de joie pour le Seigne\underline{u}r, hommes justes !
        \\Hommes droits, à vo\underline{u}s la louange !
         
${}^{2}Rendez grâce au Seigne\underline{u}r sur la cithare,
        \\jouez pour lui sur la h\underline{a}rpe à dix cordes.
${}^{3}Chantez-lui le cant\underline{i}que nouveau,
        \\de tout votre art souten\underline{e}z l’ovation.
         
        *
         
${}^{4}Oui, elle est droite, la par\underline{o}le du Seigneur ;
        \\il est fidèle en to\underline{u}t ce qu’il fait.
${}^{5}Il aime le bon dr\underline{o}it et la justice ;
        \\la terre est rempl\underline{i}e de son amour.
         
${}^{6}Le Seigneur a fait les cie\underline{u}x par sa parole,
        \\l’univers, par le so\underline{u}ffle de sa bouche.
${}^{7}Il amasse, il reti\underline{e}nt l’eau des mers ;
        \\les océans, il les g\underline{a}rde en réserve.
         
${}^{8}Que la crainte du Seigneur sais\underline{i}sse la terre,
        \\que tremblent devant lui les habit\underline{a}nts du monde !
${}^{9}Il parla, et ce qu’il d\underline{i}t exista ;
        \\il commanda, et ce qu’il d\underline{i}t survint.
         
${}^{10}Le Seigneur a déjoué les pl\underline{a}ns des nations,
        \\anéanti les proj\underline{e}ts des peuples.
${}^{11}Le plan du Seigneur deme\underline{u}re pour toujours,
        \\les projets de son cœur subs\underline{i}stent d’âge en âge.
         
        *
         
${}^{12}Heureux le peuple dont le Seigne\underline{u}r est le Dieu,
        \\heureuse la nation qu’il s’est chois\underline{i}e pour domaine !
${}^{13}Du haut des cieux, le Seigne\underline{u}r regarde :
        \\il voit la r\underline{a}ce des hommes.
         
${}^{14}Du lieu qu’il hab\underline{i}te, il observe
        \\tous les habit\underline{a}nts de la terre,
${}^{15}lui qui forme le cœ\underline{u}r de chacun,
        \\qui pénètre to\underline{u}tes leurs actions.
         
${}^{16}Le salut d’un roi n’est p\underline{a}s dans son armée,
        \\ni la victoire d’un guerri\underline{e}r, dans sa force.
${}^{17}Illusion que des cheva\underline{u}x pour la victoire :
        \\une armée ne donne p\underline{a}s le salut.
         
${}^{18}Dieu veille sur ce\underline{u}x qui le craignent,
        \\qui mettent leur esp\underline{o}ir en son amour,
${}^{19}pour les délivr\underline{e}r de la mort,
        \\les garder en vie aux jo\underline{u}rs de famine.
         
        *
         
${}^{20}Nous attendons notre v\underline{i}e du Seigneur :
        \\il est pour nous un appu\underline{i}, un bouclier.
${}^{21}La joie de notre cœur vient de lui,
        \\notre confiance est dans son n\underline{o}m très saint.
         
${}^{22}Que ton amour, Seigne\underline{u}r, soit sur nous
        \\comme notre esp\underline{o}ir est en toi !
      \bchapter{Psaume}
          
            \bchapter{Psaume}
            Voyez, le Seigneur est bon
${}^{1}De David. Quand, simulant la folie devant Abimélek, il fut chassé par lui et s’en alla.
         
${}^{2}Je bénirai le Seigne\underline{u}r en tout temps,
        \\sa louange sans c\underline{e}sse à mes lèvres.
${}^{3}Je me glorifier\underline{a}i dans le Seigneur :
        \\que les pauvres m’ent\underline{e}ndent et soient en fête !
         
${}^{4}Magnifiez avec m\underline{o}i le Seigneur,
        \\exaltons tous ens\underline{e}mble son nom.
${}^{5}Je cherche le Seigne\underline{u}r, il me répond :
        \\de toutes mes fraye\underline{u}rs, il me délivre.
         
${}^{6}Qui regarde vers lu\underline{i} resplendira,
        \\sans ombre ni tro\underline{u}ble au visage.
${}^{7}Un pauvre crie ; le Seigne\underline{u}r entend :
        \\il le sauve de to\underline{u}tes ses angoisses.
         
${}^{8}L’ange du Seigneur c\underline{a}mpe alentour
        \\pour libér\underline{e}r ceux qui le craignent.
${}^{9}Goûtez et voyez : le Seigne\underline{u}r est bon !
        \\Heureux qui trouve en lu\underline{i} son refuge !
         
${}^{10}Saints du Seigne\underline{u}r, adorez-le :
        \\rien ne manque à ce\underline{u}x qui le craignent.
${}^{11}Des riches ont tout perd\underline{u}, ils ont faim ;
        \\qui cherche le Seigneur ne manquer\underline{a} d’aucun bien.
         
        *
         
${}^{12}Venez, mes f\underline{i}ls, écoutez-moi,
        \\que je vous enseigne la cr\underline{a}inte du Seigneur.
${}^{13}Qui donc \underline{a}ime la vie
        \\et désire les jours où il verr\underline{a} le bonheur ?
         
${}^{14}Garde ta l\underline{a}ngue du mal
        \\et tes lèvres des par\underline{o}les perfides.
${}^{15}Évite le mal, f\underline{a}is ce qui est bien,
        \\poursuis la p\underline{a}ix, recherche-la.
         
${}^{16}Le Seigneur reg\underline{a}rde les justes,
        \\il écoute, attent\underline{i}f à leurs cris.
${}^{17}Le Seigneur affr\underline{o}nte les méchants
        \\pour effacer de la t\underline{e}rre leur mémoire.
         
${}^{18}Le Seigneur ent\underline{e}nd ceux qui l’appellent :
        \\de toutes leurs ang\underline{o}isses, il les délivre.
${}^{19}Il est proche du cœur brisé,
        \\il sauve l’espr\underline{i}t abattu.
         
${}^{20}Malheur sur malhe\underline{u}r pour le juste,
        \\mais le Seigneur chaque f\underline{o}is le délivre.
${}^{21}Il veille sur chac\underline{u}n de ses os :
        \\pas un ne ser\underline{a} brisé.
         
${}^{22}Le mal tuer\underline{a} les méchants ;
        \\ils seront châtiés d’avoir ha\underline{ï} le juste.
${}^{23}Le Seigneur rachèter\underline{a} ses serviteurs :
        \\pas de châtiment pour qui trouve en lu\underline{i} son refuge.
      \bchapter{Psaume}
          
            \bchapter{Psaume}
            Lève-toi pour me défendre
${}^{1}De David.
         
        \\Accuse, Seigne\underline{u}r, ceux qui m’accusent,
        attaque ce\underline{u}x qui m’attaquent. *
${}^{2}Prends une arm\underline{u}re, un bouclier,
        lève-t\underline{o}i pour me défendre.
         
${}^{3}\[Brandis la l\underline{a}nce et l’épée
        contre ce\underline{u}x qui me poursuivent. *\]
        \\P\underline{a}rle et dis-moi :
        « Je su\underline{i}s ton salut. »
         
${}^{4}\[Qu’ils soient humili\underline{é}s, déshonorés,
        ceux qui s’en pr\underline{e}nnent à ma vie ! *
        \\Qu’ils reculent, couv\underline{e}rts de honte,
        ceux qui ve\underline{u}lent mon malheur !
         
${}^{5}Qu’ils soient comme la p\underline{a}ille dans le vent
        lorsque l’ange du Seigne\underline{u}r les balaiera ! *
${}^{6}Que leur chemin soit obsc\underline{u}r et glissant
        lorsque l’ange du Seigne\underline{u}r les chassera !
         
${}^{7}Sans raison ils ont tend\underline{u} leur filet, *
        \\et sans raison creusé un tro\underline{u} pour me perdre.
${}^{8}Qu’un désastre imprév\underline{u} les surprenne, *
        \\qu’ils soient pris dans le filet qu’ils ont caché,
        et dans ce dés\underline{a}stre, qu’ils succombent !\]
         
${}^{9}Pour moi, le Seigne\underline{u}r sera ma joie, *
        \\et son sal\underline{u}t, mon allégresse !
         
${}^{10}De tout mon être, je dirai :
        « Qui est comme t\underline{o}i, Seigneur, *
        \\pour arracher un pauvre à plus f\underline{o}rt que lui,
        un pauvre, un malheureux, à qu\underline{i} le dépouille. »
         
        *
         
${}^{11}Des témoins inj\underline{u}stes se lèvent,
        des inconn\underline{u}s m’interrogent. *
${}^{12}On me rend le m\underline{a}l pour le bien :
        \\je suis un h\underline{o}mme isolé.
         
${}^{13}Quand ils ét\underline{a}ient malades,
        je m’habill\underline{a}is d’un sac, +
        \\je m’épuis\underline{a}is à jeûner ; *
        sans cesse, reven\underline{a}it ma prière.
         
${}^{14}Comme pour un fr\underline{è}re, un ami,
        j’all\underline{a}is et venais ; *
        \\comme en de\underline{u}il de ma mère,
        j’étais s\underline{o}mbre et prostré.
         
${}^{15}Si je faiblis, on r\underline{i}t, on s’attroupe, +
        des misérables s’attro\underline{u}pent contre moi : *
        \\des g\underline{e}ns inconnus
        qui déch\underline{i}rent à grands cris.
         
${}^{16}Ils blasphèment, ils me co\underline{u}vrent de sarcasmes, *
        \\grinçant des d\underline{e}nts contre moi.
         
        *
         
${}^{17}Comment peux-tu voir cel\underline{a}, Seigneur ? *
        \\Tire ma vie de ce désastre, délivre-m\underline{o}i de ces fauves.
         
${}^{18}Je te rendrai grâce dans la gr\underline{a}nde assemblée, *
        \\avec un peuple nombre\underline{u}x, je te louerai.
         
${}^{19}Qu’ils n’aient plus à r\underline{i}re de moi,
        ceux qui me ha\underline{ï}ssent injustement ! *
        \\Et ceux qui me dét\underline{e}stent sans raison,
        qu’ils c\underline{e}ssent leurs clins d’œil !
         
${}^{20}\[Ils n’ont jamais une par\underline{o}le de paix,
        \\ils calomnient les gens tranqu\underline{i}lles du pays.
${}^{21}La bouche large ouv\underline{e}rte contre moi,
        \\ils disent : « Voil\underline{à}, nos yeux l’ont vu ! »\]
         
${}^{22}Tu as vu, Seigneur, s\underline{o}rs de ton silence !
        \\Seigneur, ne sois p\underline{a}s loin de moi !
${}^{23}Réveille-toi, lève-t\underline{o}i, Seigneur mon Dieu,
        \\pour défendre et jug\underline{e}r ma cause !
         
${}^{24}\[Juge-moi, Seigneur mon Dieu, sel\underline{o}n ta justice :
        \\qu’ils n’aient plus à r\underline{i}re de moi !
${}^{25}Qu’ils ne pensent pas : « Voil\underline{à}, c’en est fait ! »
        \\Qu’ils ne disent pas : « Nous l’av\underline{o}ns englouti ! »
         
${}^{26}Qu’ils soient tous humili\underline{é}s, confondus,
        ceux qui ri\underline{a}ient de mon malheur ! *
        \\Qu’ils soient déshonor\underline{é}s, couverts de honte,
        tous ce\underline{u}x qui triomphaient !\]
         
${}^{27}À ceux qui voulaient pour m\underline{o}i la justice,
        r\underline{i}res et cris de joie ! *
        \\Ils diront sans fin : « Le Seigne\underline{u}r triomphe,
        lui qui veut le bi\underline{e}n de son serviteur. »
         
${}^{28}Moi, je redir\underline{a}i ta justice *
        et chaque jo\underline{u}r ta louange.
      \bchapter{Psaume}
          
            \bchapter{Psaume}
            Par ta lumière, nous voyons la lumière
${}^{1}Du maître de chœur. Du serviteur du Seigneur. De David.
         
${}^{2}C’est le péch\underline{é} qui parle
        au cœur de l’impie ; *
        \\ses ye\underline{u}x ne voient pas
        que Die\underline{u} est terrible.
         
${}^{3}Il se voit d’un œil trop flatteur
        pour trouver et ha\underline{ï}r sa faute ; *
${}^{4}il n’a que ruse et fra\underline{u}de à la bouche,
        il a perd\underline{u} le sens du bien.
         
${}^{5}Il prépare en secr\underline{e}t ses mauvais coups. +
        \\La route qu’il suit n’est pas c\underline{e}lle du bien ; *
        il ne renonce p\underline{a}s au mal.
         
${}^{6}Dans les cieux, Seigne\underline{u}r, ton amour ;
        jusqu’aux n\underline{u}es, ta vérité ! *
${}^{7}Ta justice, une ha\underline{u}te montagne ;
        tes jugem\underline{e}nts, le grand abîme !
         
        \\Tu sauves, Seigneur, l’h\underline{o}mme et les bêtes :
${}^{8}qu’il est précieux ton amo\underline{u}r, ô mon Dieu !
         
        \\À l’ombre de tes ailes, tu abr\underline{i}tes les hommes : +
${}^{9}ils savourent les fest\underline{i}ns de ta maison ; *
        \\aux torrents du parad\underline{i}s, tu les abreuves.
         
${}^{10}En toi est la so\underline{u}rce de vie ;
        \\par ta lumière nous voy\underline{o}ns la lumière.
         
${}^{11}Garde ton amour à ce\underline{u}x qui t’ont connu,
        \\ta justice à to\underline{u}s les hommes droits.
         
${}^{12}Que l’orgueilleux n’entre p\underline{a}s chez moi,
        \\que l’impie ne me jette p\underline{a}s dehors !
         
${}^{13}Voyez : ils sont tomb\underline{é}s, les malfaisants ;
        \\abattus, ils ne pourr\underline{o}nt se relever.
      \bchapter{Psaume}
          
            \bchapter{Psaume}
            Les doux posséderont la terre
${}^{1}De David.
         
        \\Ne t’indigne pas à la v\underline{u}e des méchants,
        \\n’envie pas les g\underline{e}ns malhonnêtes ;
${}^{2}aussi vite que l’h\underline{e}rbe, ils se fanent ;
        \\comme la verd\underline{u}re, ils se flétrissent.
         
${}^{3}Fais confiance au Seigne\underline{u}r, agis bien,
        \\habite la terre et r\underline{e}ste fidèle ;
${}^{4}mets ta j\underline{o}ie dans le Seigneur :
        \\il comblera les dés\underline{i}rs de ton cœur.
         
${}^{5}Dirige ton chem\underline{i}n vers le Seigneur,
        \\fais-lui confiance, et lu\underline{i}, il agira.
${}^{6}Il fera lever comme le jo\underline{u}r ta justice,
        \\et ton droit comme le pl\underline{e}in midi.
         
${}^{7}Repose-t\underline{o}i sur le Seigneur
        \\et c\underline{o}mpte sur lui.
        \\Ne t’indigne pas devant celu\underline{i} qui réussit,
        \\devant l’homme qui \underline{u}se d’intrigues.
         
${}^{8}Laisse ta colère, c\underline{a}lme ta fièvre,
        \\ne t’indigne pas : il n’en viendr\underline{a}it que du mal ;
${}^{9}les méchants ser\underline{o}nt déracinés,
        \\mais qui espère le Seigneur posséder\underline{a} la terre.
         
${}^{10}Encore un peu de t\underline{e}mps : plus d’impie ;
        \\tu pénètres chez lu\underline{i} : il n’y est plus.
${}^{11}Les doux posséder\underline{o}nt la terre
        \\et jouiront d’une abond\underline{a}nte paix.
         
${}^{12}L’impie peut intrigu\underline{e}r contre le juste
        \\et grincer des d\underline{e}nts contre lui,
${}^{13}le Seigneur se m\underline{o}que du méchant
        \\car il voit son jo\underline{u}r qui arrive.
         
${}^{14}L’impie a tiré son épée, il a tend\underline{u} son arc
        \\pour abattre le pauvre et le faible,
        pour tu\underline{e}r l’homme droit.
${}^{15}Mais l’épée lui entrer\underline{a} dans le cœur,
        \\et son \underline{a}rc se brisera.
         
${}^{16}Pour le juste, avoir pe\underline{u} de biens
        \\vaut mieux que la fort\underline{u}ne des impies.
${}^{17}Car le bras de l’imp\underline{i}e sera brisé,
        \\mais le Seigneur souti\underline{e}nt les justes.
         
${}^{18}Il connaît les jo\underline{u}rs de l’homme intègre
        \\qui recevra un hérit\underline{a}ge impérissable.
${}^{19}Pas de honte pour lu\underline{i} aux mauvais jours ;
        \\aux temps de famine, il ser\underline{a} rassasié.
         
${}^{20}Mais oui, les imp\underline{i}es disparaîtront
        \\comme la par\underline{u}re des prés ;
        \\c’en est fini des ennem\underline{i}s du Seigneur :
        \\ils s’en v\underline{o}nt en fumée.
         
${}^{21}L’impie empr\underline{u}nte et ne rend pas ;
        \\le juste a piti\underline{é} : il donne.
${}^{22}Ceux qu’il bénit posséder\underline{o}nt la terre,
        \\ceux qu’il maudit ser\underline{o}nt déracinés.
         
${}^{23}Quand le Seigneur condu\underline{i}t les pas de l’homme,
        \\ils sont fermes et sa m\underline{a}rche lui plaît.
${}^{24}S’il trébuche, il ne t\underline{o}mbe pas
        \\car le Seigneur le souti\underline{e}nt de sa main.
         
${}^{25}Jamais, de ma jeun\underline{e}sse à mes vieux jours,
        \\je n’ai vu le juste abandonné
        ni ses enfants mendi\underline{e}r leur pain.
${}^{26}Chaque jour il a piti\underline{é}, il prête ;
        \\ses descend\underline{a}nts seront bénis.
         
${}^{27}Évite le mal, f\underline{a}is ce qui est bien,
        \\et tu auras une habitati\underline{o}n pour toujours,
${}^{28}car le Seigneur \underline{a}ime le bon droit,
        \\il n’abandonne p\underline{a}s ses amis.
         
        \\Ceux-là seront préserv\underline{é}s à jamais,
        \\les descendants de l’impie ser\underline{o}nt déracinés.
${}^{29}Les justes posséder\underline{o}nt la terre
        \\et toujo\underline{u}rs l’habiteront.
         
${}^{30}Les lèvres du juste red\underline{i}sent la sagesse
        \\et sa bouche én\underline{o}nce le droit.
${}^{31}La loi de son Die\underline{u} est dans son cœur ;
        \\il va, sans cr\underline{a}indre les faux pas.
         
${}^{32}Les impies gu\underline{e}ttent le juste,
        \\ils cherchent à le f\underline{a}ire mourir.
${}^{33}Mais le Seigneur ne saur\underline{a}it l’abandonner
        \\ni le laisser condamn\underline{e}r par ses juges.
         
${}^{34}Esp\underline{è}re le Seigneur,
        \\et g\underline{a}rde son chemin :
        \\il t’élèvera jusqu’à posséd\underline{e}r la terre ;
        \\tu verras les imp\underline{i}es déracinés.
         
${}^{35}J’ai vu l’imp\underline{i}e dans sa puissance
        \\se déployer comme un c\underline{è}dre vigoureux.
${}^{36}Il a passé, voic\underline{i} qu’il n’est plus ;
        \\je l’ai cherché, il \underline{e}st introuvable.
         
${}^{37}Considère l’homme droit, v\underline{o}is l’homme intègre :
        \\un avenir est prom\underline{i}s aux pacifiques.
${}^{38}Les pécheurs seront to\underline{u}s déracinés,
        \\et l’avenir des imp\underline{i}es, anéanti.
         
${}^{39}Le Seigneur est le sal\underline{u}t pour les justes,
        \\leur abri au t\underline{e}mps de la détresse.
${}^{40}Le Seigneur les \underline{a}ide et les délivre,
        \\il les délivre de l’impie, il les sauve,
        car ils cherchent en lu\underline{i} leur refuge.
      \bchapter{Psaume}
          
            \bchapter{Psaume}
            Ne m’abandonne jamais
${}^{1}Psaume. De David. En mémorial.
         
${}^{2}Seigneur, corrige-m\underline{o}i sans colère
        \\et reprends-m\underline{o}i sans violence.
         
${}^{3}Tes fl\underline{è}ches m’ont frappé,
        \\ta main s’est abatt\underline{u}e sur moi.
${}^{4}Rien n’est sain dans ma ch\underline{a}ir sous ta fureur,
        \\rien d’intact en mes \underline{o}s depuis ma faute.
         
${}^{5}Oui, mes péch\underline{é}s me submergent,
        \\leur poids trop pes\underline{a}nt m’écrase.
${}^{6}Mes plaies sont puante\underline{u}r et pourriture :
        \\c’est là le pr\underline{i}x de ma folie.
         
${}^{7}Accablé, prostr\underline{é}, à bout de forces,
        \\tout le jour j’av\underline{a}nce dans le noir.
${}^{8}La fièvre m’envah\underline{i}t jusqu’aux moelles,
        \\plus rien n’est s\underline{a}in dans ma chair.
         
${}^{9}Brisé, écras\underline{é}, à bout de forces,
        \\mon cœur gr\underline{o}nde et rugit.
${}^{10}Seigneur, tout mon dés\underline{i}r est devant toi,
        \\et rien de ma pl\underline{a}inte ne t’échappe.
         
${}^{11}Le cœur me bat, ma f\underline{o}rce m’abandonne,
        \\et même la lumi\underline{è}re de mes yeux.
${}^{12}Amis et compagnons se ti\underline{e}nnent à distance,
        \\et mes proches, à l’éc\underline{a}rt de mon mal.
         
${}^{13}Ceux qui veulent ma p\underline{e}rte me talonnent,
        \\ces gens qui ch\underline{e}rchent mon malheur ;
        \\ils prononcent des par\underline{o}les maléfiques,
        \\tout le jour ils rum\underline{i}nent leur traîtrise.
         
${}^{14}Moi, comme un so\underline{u}rd, je n’entends rien,
        \\comme un muet, je n’ouvre p\underline{a}s la bouche,
${}^{15}pareil à celu\underline{i} qui n’entend pas,
        \\qui n’a pas de répl\underline{i}que à la bouche.
         
${}^{16}C’est toi que j’esp\underline{è}re, Seigneur :
        \\Seigneur mon Dieu, t\underline{o}i, tu répondras.
${}^{17}J’ai dit : « Qu’ils ne tri\underline{o}mphent pas,
        \\ceux qui rient de m\underline{o}i quand je trébuche ! »
         
${}^{18}Et maintenant, je suis pr\underline{è}s de tomber,
        \\ma douleur est toujo\underline{u}rs devant moi.
${}^{19}Oui, j’avo\underline{u}e mon péché,
        \\je m’effr\underline{a}ie de ma faute.
         
${}^{20}Mes ennemis sont f\underline{o}rts et vigoureux,
        \\ils sont nombreux à m’en voul\underline{o}ir injustement.
${}^{21}Ils me rendent le m\underline{a}l pour le bien ;
        \\quand je cherche le bi\underline{e}n, ils m’accusent.
         
${}^{22}Ne m’abandonne jam\underline{a}is, Seigneur,
        \\mon Dieu, ne sois p\underline{a}s loin de moi.
${}^{23}Viens v\underline{i}te à mon aide,
        \\Seigne\underline{u}r, mon salut !
      \bchapter{Psaume}
          
            \bchapter{Psaume}
            L’homme n’est qu’un souffle
${}^{1}Du maître de chœur. De Yedoutoune. Psaume. De David.
         
${}^{2}J’ai dit : « Je garder\underline{a}i mon chemin
        \\sans laisser ma l\underline{a}ngue s’égarer ;
        \\je garderai un bâill\underline{o}n sur ma bouche,
        \\tant que l’impie se tiendr\underline{a} devant moi. »
         
${}^{3}Je suis resté muet, silencieux ;
        je me tais\underline{a}is, mais sans profit. *
        \\Mon tourment s’exaspérait,
${}^{4}mon cœur brûl\underline{a}it en moi.
        \\Quand j’y pens\underline{a}is, je m’enflammais,
        \\et j’ai laissé parl\underline{e}r ma langue.
         
${}^{5}Seigneur, fais-moi connaître ma fin,
        quel est le n\underline{o}mbre de mes jours :
        \\je connaîtrai combi\underline{e}n je suis fragile.
${}^{6}Vois le peu de jo\underline{u}rs que tu m’accordes :
        \\ma durée n’est ri\underline{e}n devant toi.
         
        \\L’homme ici-b\underline{a}s n’est qu’un souffle ;
${}^{7}il va, il vient, il n’\underline{e}st qu’une image.
        \\Rien qu’un souffle, to\underline{u}s ses tracas ;
        \\il amasse, mais qu\underline{i} recueillera ?
         
${}^{8}Maintenant, que puis-je att\underline{e}ndre, Seigneur ?
        \\Elle est en t\underline{o}i, mon espérance.
${}^{9}Délivre-moi de to\underline{u}s mes péchés,
        \\épargne-moi les inj\underline{u}res des fous.
         
${}^{10}Je me suis tu, je n’ouvre p\underline{a}s la bouche,
        \\car c’est t\underline{o}i qui es à l’œuvre.
${}^{11}Éloigne de m\underline{o}i tes coups :
        \\je succombe sous ta m\underline{a}in qui me frappe.
         
${}^{12}Tu redresses l’homme en corrige\underline{a}nt sa faute, +
        \\tu ronges comme un v\underline{e}r son désir ; *
        \\l’h\underline{o}mme n’est qu’un souffle.
         
${}^{13}Entends ma prière, Seigneur, éco\underline{u}te mon cri ;
        \\ne reste pas so\underline{u}rd à mes pleurs.
        \\Je ne suis qu’un h\underline{ô}te chez toi,
        \\un passant, comme to\underline{u}s mes pères.
         
${}^{14}Détourne de moi tes ye\underline{u}x, que je respire
        \\avant que je m’en \underline{a}ille et ne sois plus.
      \bchapter{Psaume}
          
            \bchapter{Psaume}
            Voici, je viens
${}^{1}Du maître de chœur. De David. Psaume.
         
${}^{2}D’un gr\underline{a}nd espoir
        j’espér\underline{a}is le Seigneur : *
        \\il s’est pench\underline{é} vers moi
        pour ent\underline{e}ndre mon cri.
         
${}^{3}Il m’a tiré de l’horre\underline{u}r du gouffre,
        de la v\underline{a}se et de la boue ; *
        \\il m’a fait reprendre pi\underline{e}d sur le roc,
        il a rafferm\underline{i} mes pas.
         
${}^{4}Dans ma bouche il a m\underline{i}s un chant nouveau,
        une lou\underline{a}nge à notre Dieu. *
        \\Beaucoup d’hommes verr\underline{o}nt, ils craindront,
        ils auront f\underline{o}i dans le Seigneur.
         
${}^{5}Heure\underline{u}x est l’homme
        qui met sa f\underline{o}i dans le Seigneur *
        \\et ne va pas du côt\underline{é} des violents,
        dans le part\underline{i} des traîtres.
         
${}^{6}Tu as fait pour no\underline{u}s tant de choses,
        toi, Seigne\underline{u}r mon Dieu ! *
        \\Tant de proj\underline{e}ts et de merveilles :
        non, tu n’as p\underline{o}int d’égal !
         
        \\Je les dis, je les red\underline{i}s encore ; *
        mais leur n\underline{o}mbre est trop grand !
         
        *
         
${}^{7}Tu ne voulais ni offr\underline{a}nde ni sacrifice,
        tu as ouv\underline{e}rt mes oreilles ; *
        \\tu ne demandais ni holoca\underline{u}ste ni victime,
${}^{8}alors j’ai dit : « Voic\underline{i}, je viens.
         
        \\Dans le livre, est écr\underline{i}t pour moi
${}^{9}ce que tu ve\underline{u}x que je fasse. *
        \\Mon Dieu, voil\underline{à} ce que j’aime :
        ta loi me ti\underline{e}nt aux entrailles. »
         
${}^{10}J’ann\underline{o}nce la justice
        dans la gr\underline{a}nde assemblée ; *
        \\vois, je ne retiens p\underline{a}s mes lèvres,
        Seigne\underline{u}r, tu le sais.
         
${}^{11}Je n’ai pas enfoui ta justice au f\underline{o}nd de mon cœur, +
        je n’ai pas caché ta fidélit\underline{é}, ton salut ; *
        \\j’ai dit ton amo\underline{u}r et ta vérité
        à la gr\underline{a}nde assemblée.
         
${}^{12}T\underline{o}i, Seigneur,
        ne retiens pas loin de m\underline{o}i ta tendresse ; *
        \\que ton amo\underline{u}r et ta vérité
        sans c\underline{e}sse me gardent !
         
        *
         
${}^{13}Les malhe\underline{u}rs m’ont assailli : *
        leur n\underline{o}mbre m’échappe !
         
        \\Mes péch\underline{é}s m’ont accablé :
        ils m’enl\underline{è}vent la vue ! *
        \\Plus nombreux que les cheve\underline{u}x de ma tête,
        ils me f\underline{o}nt perdre cœur.
         
${}^{14}Daigne, Seigne\underline{u}r, me délivrer ;
        Seigneur, viens v\underline{i}te à mon secours ! *
${}^{15}\[Qu’ils soient tous humili\underline{é}s, déshonorés,
        ceux qui s’en pr\underline{e}nnent à ma vie !
         
        \\Qu’ils reculent, couv\underline{e}rts de honte,
        ceux qui ch\underline{e}rchent mon malheur ; *
${}^{16}que l’humiliati\underline{o}n les écrase,
        ceux qui me d\underline{i}sent : « C’est bien fait ! »\]
         
${}^{17}Mais tu seras l’allégr\underline{e}sse et la joie
        de tous ce\underline{u}x qui te cherchent ; *
        \\toujours ils rediront : « Le Seigne\underline{u}r est grand ! »
        ceux qui \underline{a}iment ton salut.
         
${}^{18}Je suis pa\underline{u}vre et malheureux,
        mais le Seigne\underline{u}r pense à moi. *
        \\Tu es mon seco\underline{u}rs, mon libérateur :
        mon Die\underline{u}, ne tarde pas !
      \bchapter{Psaume}
          
            \bchapter{Psaume}
            Je saurai que tu m’aimes
${}^{1}Du maître de chœur. Psaume. De David.
         
${}^{2}Heureux qui pense au pa\underline{u}vre et au faible :
        \\le Seigneur le sauve au jo\underline{u}r du malheur !
${}^{3}Il le protège et le garde en vie, heure\underline{u}x sur la terre.
        \\Seigneur, ne le livre pas à la merc\underline{i} de l’ennemi !
${}^{4}Le Seigneur le soutient sur son l\underline{i}t de souffrance :
        \\si malade qu’il s\underline{o}it, tu le relèves.
         
${}^{5}J’avais dit : « Pitié pour m\underline{o}i, Seigneur,
        \\guéris-moi, car j’ai péch\underline{é} contre toi ! »
${}^{6}Mes ennemis me cond\underline{a}mnent déjà :
        \\« Quand sera-t-il mort ? son n\underline{o}m, effacé ? »
${}^{7}Si quelqu’un vient me voir, ses prop\underline{o}s sont vides ;
        \\il emplit son cœur de pensées méchantes,
        il sort, et dans la r\underline{u}e il parle.
         
${}^{8}Unis contre moi, mes ennem\underline{i}s murmurent,
        \\à mon sujet, ils prés\underline{a}gent le pire :
${}^{9}« C’est un mal pernicie\underline{u}x qui le ronge ;
        \\le voilà couché, il ne pourra pl\underline{u}s se lever. »
${}^{10}Même l’ami, qui av\underline{a}it ma confiance
        \\et partageait mon pain, m’a frapp\underline{é} du talon.
         
${}^{11}Mais toi, Seigneur, prends piti\underline{é} de moi ;
        \\relève-moi, je leur rendr\underline{a}i ce qu’ils méritent.
${}^{12}Oui, je saur\underline{a}i que tu m’aimes
        \\si mes ennemis ne chantent p\underline{a}s victoire.
${}^{13}Dans mon innocence tu m’\underline{a}s soutenu
        \\et rétabli pour toujo\underline{u}rs devant ta face.
         
        *
         
${}^{14}Béni soit le Seigneur,
        Die\underline{u} d’Israël, *
        depuis toujours et pour toujours !
        Am\underline{e}n ! Amen !
