  
  
    
    \bbook{TOBIE}{TOBIE}
      
         
      \bchapter{}
      \begin{verse}
${}^{1}Voici l’histoire de Tobith, fils de Tobiël, fils d’Ananiël, fils d’Adouël, fils de Gabaël, fils de Raphaël, fils de Ragouël, de la descendance d’Asiël, de la tribu de Nephtali.
${}^{2}À l’époque de Salmanasar, roi des Assyriens, Tobith fut déporté de Thisbé ; cette ville se trouve au sud de Cadès de Nephtali, en Haute-Galilée, à l’est de Haçor, au-delà de la route qui conduit à l’ouest, au nord de Phogor.
${}^{3}Moi, Tobith, j’ai marché dans les voies de la vérité et j’ai fait ce qui est juste tous les jours de ma vie ; j’ai fait beaucoup d’aumônes à mes frères et aux gens de ma nation qui avaient été emmenés captifs avec moi au pays des Assyriens, à Ninive.
${}^{4}Quand j’étais jeune, je vivais dans mon pays, la terre d’Israël. Toute la tribu de mon ancêtre Nephtali s’était séparée de la maison de David et de Jérusalem. Et pourtant, cette ville avait été choisie parmi tout Israël comme lieu de sacrifice pour toutes les tribus d’Israël : c’est là qu’avait été consacré le temple où Dieu réside, temple bâti pour toutes les générations à venir. 
${}^{5}Tous mes frères, ainsi que la maison de mon ancêtre Nephtali, offraient des sacrifices, sur tous les monts de Galilée, en l’honneur du veau que le roi d’Israël Jéroboam avait érigé à Dane. 
${}^{6}Quant à moi, j’étais le seul à me rendre souvent à Jérusalem pour les fêtes, selon ce qui est écrit pour tout Israël dans une ordonnance perpétuelle. J’accourais à Jérusalem, apportant les prémices, les premiers-nés, les dîmes des troupeaux et les premières tontes des brebis ; 
${}^{7}je les donnais aux prêtres, fils d’Aaron, pour le service de l’autel, et je donnais la dîme du froment, du vin, de l’huile, des grenades, des figues et des autres fruits aux fils de Lévi qui officient à Jérusalem. J’acquittais la deuxième dîme en argent, sauf les années sabbatiques, et j’allais dépenser cette somme à Jérusalem chaque année. 
${}^{8}La troisième dîme, je la donnais aux orphelins, aux veuves et aux immigrés qui résidaient chez les fils d’Israël, je la leur apportais tous les trois ans et nous la mangions. Je suivais en cela l’ordonnance faite à ce sujet dans la loi de Moïse et les commandements qu’avait donnés Débora, la mère de mon père, car ce dernier était mort, me laissant orphelin. 
${}^{9}Quand je fus arrivé à l’âge adulte, je pris femme dans la lignée de nos pères et, d’elle, j’ai eu un fils, que j’appelai Tobie.
${}^{10}Déporté chez les Assyriens, j’arrivai à Ninive. Tous mes frères et les gens de ma race mangeaient la même nourriture que les païens, 
${}^{11}mais moi, je me gardais de manger une telle nourriture. 
${}^{12}Puisque je me souvenais de mon Dieu de toute mon âme, 
${}^{13}le Très-Haut m’accorda grâce et beauté aux yeux de Salmanasar, et j’achetais pour le roi tout ce dont il avait besoin ; 
${}^{14}je me rendais en Médie où, jusqu’à sa mort, je fis des achats pour lui. Et dans ce pays de Médie, je confiai à Gabaël, frère de Gabri, des bourses qui contenaient dix talents d’argent. 
${}^{15}Quand Salmanasar fut mort et que Sennakérib, son fils, lui succéda, les routes de Médie furent bloquées et je ne pus continuer à me rendre en Médie.
${}^{16}À l’époque de Salmanasar, je faisais beaucoup d’aumônes à mes frères de race ; 
${}^{17}je donnais mon pain à ceux qui avaient faim et des vêtements à ceux qui étaient nus ; si je voyais le cadavre de quelqu’un de ma nation, jeté derrière le rempart de Ninive, je l’enterrais. 
${}^{18}De même, j’enterrais tous ceux que tua Sennakérib. Il faut savoir que, lorsqu’il revint en fuite de Judée, aux jours du jugement que lui infligea le Roi du ciel pour les blasphèmes qu’il avait proférés, Sennakérib tua, dans sa fureur, de nombreux fils d’Israël ; je dérobais leurs corps et je les enterrais ; Sennakérib les chercha et ne les trouva pas. 
${}^{19}Mais un des habitants de Ninive alla dire au roi que c’était moi qui les enterrais, et je me cachai. Quand j’appris que le roi était renseigné à mon sujet et que j’étais recherché pour être mis à mort, je pris peur et m’enfuis furtivement. 
${}^{20}Tous mes biens furent saisis, et il ne me resta rien qui ne fût confisqué au profit du trésor royal, sauf ma femme Anna et mon fils Tobie.
${}^{21}Moins de quarante jours plus tard, Sennakérib fut assassiné par deux de ses fils, qui s’enfuirent au mont Ararat, et son fils Asarhaddone lui succéda. Il plaça Ahikar, fils de mon frère Anaël, à la tête de toutes les finances de son royaume, et celui-ci eut donc la haute main sur toute l’administration. 
${}^{22}Alors Ahikar intervint en ma faveur, et je revins à Ninive. Ahikar, en effet, avait été grand échanson, garde du sceau, chef de l’administration et des finances sous Sennakérib, roi des Assyriens, et Asarhaddone l’avait reconduit dans ses fonctions. Or il était de ma famille : c’était mon neveu.
      
         
      \bchapter{}
      \begin{verse}
${}^{1}C’est ainsi que, sous le règne d’Asarhaddone, je revins chez moi, et ma femme Anna me fut rendue, ainsi que mon fils Tobie. Lors de notre fête de la Pentecôte, qui est la sainte fête des Semaines, on me prépara un bon repas et je m’étendis pour le prendre. 
${}^{2}On plaça devant moi une table et on me servit quantité de petits plats. Alors je dis à mon fils Tobie : « Va, mon enfant, essaie de trouver parmi nos frères déportés à Ninive un pauvre qui se souvienne de Dieu\\de tout son cœur ; amène-le pour qu’il partage mon repas. Moi, mon enfant, j’attendrai que tu sois de retour. » 
${}^{3}Tobie partit chercher un pauvre parmi nos frères. À son retour, il dit : « Père\\ ! – Qu’y a-t-il\\, mon enfant\\ ? – Père\\, quelqu’un de notre nation a été assassiné ; il a été jeté sur la place publique, il vient d’y être étranglé. » 
${}^{4}Laissant là mon repas avant même d’y avoir touché, je me précipitai, j’enlevai de la place le cadavre\\que je déposai dans une dépendance en attendant le coucher du soleil pour l’enterrer. 
${}^{5}À mon retour, je pris un bain\\et je mangeai mon pain dans le deuil, 
${}^{6}en me rappelant la parole\\que le prophète Amos avait dite sur Béthel :
        \\« Vos fêtes se changeront en deuil,
        \\et tous vos chants\\en lamentation. »
${}^{7}Et je me mis à pleurer. Puis, quand le soleil fut couché, je partis creuser une tombe pour enterrer le mort. 
${}^{8} Mes voisins se moquaient de moi : « N’a-t-il donc plus peur ?, disaient-ils. On l’a déjà recherché pour le tuer à cause de cette manière d’agir, et il a dû s’enfuir. Et voilà qu’il recommence à enterrer les morts ! »
${}^{9}Cette nuit-là, je pris un bain, puis j’entrai dans la cour de ma maison\\et je m’étendis contre le mur de la cour, le visage découvert à cause de la chaleur. 
${}^{10} Je ne m’aperçus pas qu’il y avait des moineaux dans le mur, au-dessus de moi, et leur fiente me tomba toute chaude dans les yeux et provoqua des leucomes. Je me rendis chez les médecins pour être soigné, mais plus ils m’appliquaient leurs baumes, plus ce voile blanchâtre\\m’empêchait de voir, et je finis par devenir complètement aveugle : je restai privé de la vue durant quatre ans. Tous mes frères\\s’apitoyaient sur mon sort, et Ahikar pourvut à mes besoins pendant deux ans jusqu’à son départ pour l’Élymaïde\\.
${}^{11}Pendant ce temps-là, ma femme Anna, pour gagner sa vie, exécutait des travaux d’ouvrière\\, 
${}^{12} qu’elle livrait à ses patrons\\, et ceux-ci lui réglaient son salaire. Or, le sept du mois de Dystros\\, elle acheva une pièce de tissu\\et l’envoya à ses patrons ; ils lui réglèrent tout ce qu’ils lui devaient et, pour un repas de fête, ils lui offrirent un chevreau pris à sa mère\\. 
${}^{13} Arrivé chez moi, le chevreau se mit à bêler. J’appelai ma femme et lui dis : « D’où vient ce chevreau ? N’aurait-il pas été volé ? Rends-le à ses propriétaires. Car nous ne sommes pas autorisés à manger quoi que ce soit de volé ! » 
${}^{14} Elle me dit : « Mais c’est un cadeau qu’on m’a donné en plus de mon salaire ! » Je refusai de la croire, je lui dis de rendre l’animal à ses propriétaires, et je me fâchai contre ma femme\\à cause de cela. Alors elle me répliqua : « Qu’en est-il donc de tes aumônes ? Qu’en est-il de tes bonnes œuvres ? On voit bien maintenant ce qu’elles signifient ! »
      
         
      \bchapter{}
      \begin{verse}
${}^{1}La mort dans l’âme, je gémissais et je pleurais ; puis, au milieu de mes gémissements, je commençai à prier :
      
         
       
        ${}^{2}« Tu es juste, Seigneur,
        \\toutes tes œuvres sont justes,
        \\tous tes chemins, miséricorde et vérité ;
        \\c’est toi qui juges le monde.
         
        ${}^{3}Et maintenant, Seigneur,
        \\souviens-toi de moi et regarde :
        \\ne me punis pas pour mes péchés, mes égarements,
        \\ni pour ceux de mes pères, qui ont péché\\devant toi
        ${}^{4}et refusé d’entendre\\tes commandements.
         
        \\Tu nous as livrés au pillage,
        \\à la déportation et à la mort,
        \\pour être la fable, la risée, le sarcasme
        \\de toutes les nations où tu nous as disséminés.
         
        ${}^{5}Et maintenant encore, ils sont vrais
        \\les nombreux jugements que tu portes contre moi,
        \\pour mes péchés et ceux de mes pères\\,
        \\car nous n’avons pas pratiqué tes commandements
        \\ni marché dans la vérité devant toi.
         
        ${}^{6}Et maintenant, agis avec moi comme il te plaira,
        \\ordonne que mon souffle me soit repris,
        \\pour que je disparaisse de la face de la terre
        \\et devienne, moi-même, terre.
         
        \\Pour moi, mieux vaut mourir que vivre,
        \\car j’ai entendu des insultes mensongères,
        \\et je suis accablé de tristesse.
         
        \\Seigneur, ordonne
        \\que je sois délivré de cette adversité\\,
        \\laisse-moi partir au séjour éternel,
        \\et ne détourne pas de moi ta face, Seigneur.
         
        \\Car, pour moi, mieux vaut mourir
        \\que connaître tant d’adversité à longueur de vie.
        \\Ainsi, je n’aurai plus à entendre
        \\de telles insultes. »
${}^{7}Or ce jour-là, Sarra, la fille de Ragouël d’Ecbatane\\en Médie, se fit, elle aussi, insulter par une jeune servante de son père : 
${}^{8} elle avait été mariée sept fois, et Asmodée, le pire des démons\\, tuait les maris avant qu’ils ne se soient approchés d’elle\\. Donc, la servante dit à Sarra : « C’est toi qui as tué tes maris ! En voilà déjà sept à qui tu as été donnée en mariage, et d’aucun d’entre eux tu n’as porté le nom. 
${}^{9} Pourquoi nous fouetter, sous prétexte que tes maris sont morts ? Va les rejoindre : puissions-nous ne jamais voir de toi un fils ni une fille ! » 
${}^{10} Ce jour-là, Sarra, la mort dans l’âme, se mit à pleurer. Et elle monta dans la chambre haute\\de la maison\\de son père avec l’intention de se pendre. Mais, à la réflexion, elle se dit : « Eh bien, non ! On irait insulter mon père et lui dire : “Tu n’avais qu’une fille, une fille très aimée, et elle s’est pendue à cause de ses malheurs !” Je ferais ainsi descendre mon vieux père plein de tristesse au séjour des morts. Mieux vaut pour moi ne pas me pendre, mais supplier le Seigneur de me faire mourir, pour que je n’aie plus à entendre de telles insultes à longueur de vie. »
${}^{11}À l’instant même, elle étendit les mains vers la fenêtre\\et fit cette prière :
       
        \\« Béni sois-tu, Dieu de miséricorde ;
        \\béni soit ton nom pour les siècles ;
        \\que toutes tes œuvres te bénissent à jamais !
         
${}^{12}Et maintenant, j’élève vers toi
        \\mon visage et mes yeux.
${}^{13}Parle : que je disparaisse de la terre
        \\et n’aie plus à entendre d’insultes.
         
${}^{14}Tu sais, toi, Maître, que je suis indemne
        \\de toute impureté d’homme.
${}^{15}Je n’ai pas déshonoré mon nom
        \\ni le nom de mon père sur ma terre d’exil.
         
        \\Je suis la fille unique de mon père,
        \\il n’a pas d’autre enfant pour hériter de lui,
        \\ni de parent proche ou lointain
        \\pour qui je devrais me garder comme épouse.
         
        \\J’ai déjà perdu sept maris :
        \\à quoi bon vivre encore ?
        \\Et s’il ne te semble pas bon de me tuer, Seigneur,
        \\entends au moins l’insulte qui m’est faite. »
${}^{16}À cet instant précis, la prière de l’un et de l’autre fut portée en présence de la gloire de Dieu où elle fut entendue\\. 
${}^{17} Et Raphaël fut envoyé pour les guérir tous deux : à Tobith pour enlever le voile blanchâtre qui couvrait\\ses yeux afin que, de ses yeux, il voie la lumière de Dieu, et à Sarra, fille de\\Ragouël, pour la donner en mariage à Tobie, fils de Tobith, et expulser d’elle Asmodée, le pire des démons ; en effet c’est à Tobie que revenait le droit de l’épouser plutôt qu’à tous ses prétendants.
      Juste à ce moment, Tobith rentrait de la cour dans sa maison tandis que Sarra, fille de Ragouël, descendait de la chambre haute.
      
         
      \bchapter{}
      \begin{verse}
${}^{1}Ce jour-là, Tobith se souvint de l’argent qu’il avait mis en dépôt chez Gabaël, à Raguès de Médie. 
${}^{2}Il se dit en lui-même : « Voici que j’ai réclamé la mort. Ne devrais-je pas appeler mon fils Tobie et lui parler de cet argent avant de mourir ? » 
${}^{3}Il appela son fils Tobie, qui vint à lui. Tobith lui dit : « Mon enfant, quand je mourrai, enterre-moi dignement. Honore ta mère et ne l’abandonne pas aussi longtemps qu’elle vivra. Fais ce qui lui est agréable et ne l’attriste en rien. 
${}^{4}Souviens-toi, mon enfant, de tous les risques qu’elle a courus pour toi quand tu étais dans son sein. Quand elle mourra, enterre-la auprès de moi, dans le même tombeau. 
${}^{5}Chaque jour, mon enfant, souviens-toi du Seigneur. Garde-toi de pécher et de transgresser ses commandements. Fais ce qui est juste tous les jours de ta vie et ne marche pas dans les voies de l’injustice. 
${}^{6}Car ceux qui agissent selon la vérité réussiront dans leurs entreprises. À tous ceux qui pratiquent la justice, 
${}^{7}fais l’aumône avec les biens qui t’appartiennent. Ne détourne ton visage d’aucun pauvre, et le visage de Dieu ne se détournera pas de toi.
${}^{8}Mon fils, agis suivant ce que tu as : si tu es dans l’abondance, donne davantage ; mais si tu as peu, donne selon le peu que tu as. Quand tu fais l’aumône, mon fils, n’aie aucun doute : 
${}^{9}tu te constitues un beau trésor pour les jours de détresse, 
${}^{10}car l’aumône délivre de la mort et empêche d’aller dans les ténèbres. 
${}^{11}Pour tous ceux qui la pratiquent, elle est une bonne offrande devant le Dieu Très-Haut. 
${}^{12}Mon fils, garde-toi de toute union illégale. Et tout d’abord, prends femme dans la descendance de tes pères. Ne prends pas une étrangère, car nous sommes les fils des prophètes : de Noé, qui fut le premier prophète, d’Abraham, Isaac et Jacob, nos pères des origines. Souviens-toi, mon fils : ils ont tous pris femme dans le clan de leurs frères ; ils ont été bénis dans leurs fils, et leur descendance aura un héritage. 
${}^{13}Quant à toi, mon fils, aime tes frères, et ne jette pas un regard orgueilleux sur les filles de tes frères. Car, dans l’orgueil, il y a ruine et grand désordre et, dans une conduite indigne, abaissement et indigence extrême : c’est le début de la misère.
${}^{14}Donne son salaire à quiconque aura travaillé pour toi ; paie-le aussitôt, et ne garde chez toi le salaire de personne. Ta récompense ne tardera pas si tu sers Dieu en vérité. Sois vigilant, mon fils, et fais preuve de sagesse dans toutes tes actions et toutes tes paroles. 
${}^{15}Ne fais à personne ce que tu détestes, et que cela n’entre dans ton cœur aucun jour de ta vie. 
${}^{16}Donne de ton pain aux affamés et de tes vêtements à ceux qui sont nus. En outre, fais l’aumône de tout ton superflu. 
${}^{17}Mon fils, répands ton pain et ton vin sur la tombe des justes, et ne donne rien aux pécheurs. 
${}^{18}Prends conseil auprès d’un homme sage, et ne méprise aucun conseil utile.
${}^{19}En toute occasion, bénis ton Dieu, demande-lui de rendre droits tes chemins et de bien orienter tes pensées, car les nations païennes ne pensent rien de bon. C’est le Seigneur qui donne le bon conseil. Le Seigneur abaisse qui il veut jusqu’au fond du séjour des morts. Ainsi donc, mon enfant, souviens-toi de ces commandements, et qu’ils ne s’effacent pas de ton cœur.
${}^{20}À présent, mon enfant, je te signale que j’ai mis en dépôt dix talents d’argent chez Gabaël, le frère de Gabri, à Raguès de Médie. 
${}^{21}Ne t’effraye pas, mon enfant, si nous sommes devenus pauvres : tu as de grands biens si tu crains Dieu, si tu fuis tout péché et si tu fais le bien devant le Seigneur ton Dieu. »
      
         
      \bchapter{}
      \begin{verse}
${}^{1}Tobie répondit à son père Tobith : « Père, je ferai tout ce que tu m’as commandé. 
${}^{2}Mais comment pourrais-je reprendre cet argent chez cet homme, alors que lui ne me connaît pas et que moi, je ne le connais pas ? Quel signe lui donner pour qu’il sache qui je suis, qu’il ait confiance en moi et me remette l’argent ? De plus, je ne connais pas les routes à prendre pour aller en Médie. » 
${}^{3}Tobith répondit à son fils Tobie : « Il a signé un reçu et je l’ai contresigné. Puis je l’ai partagé en deux pour que nous en ayons chacun une moitié, et j’en ai laissé une avec l’argent. Voilà déjà vingt ans que j’ai mis cet argent en dépôt. Et maintenant, mon enfant, cherche-toi un homme de confiance pour t’accompagner, et nous lui donnerons un salaire à ton retour. Va reprendre cet argent chez Gabaël. »
      
         
${}^{4}Tobie sortit chercher un homme qui connaisse la route, pour l’accompagner jusqu’en Médie. À peine sorti, il trouva l’ange Raphaël debout devant lui, mais il ne savait pas que c’était un ange de Dieu. 
${}^{5}Il lui dit : « D’où es-tu, mon ami ? » L’autre répondit : « Je suis un fils d’Israël, un de tes frères, et je suis venu ici pour trouver du travail. » Tobie lui dit : « Connais-tu la route pour aller en Médie ? 
${}^{6}– Oui, répondit Raphaël, j’ai été souvent là-bas et je connais tous les chemins par cœur. Durant mes nombreux séjours, je passais la nuit chez Gabaël, notre frère, qui habite à Raguès de Médie. Il faut deux bonnes journées de marche pour aller d’Ecbatane à Raguès, car Raguès se trouve dans la montagne, et Ecbatane, au milieu de la plaine. » 
${}^{7}Tobie lui dit : « Attends-moi, mon ami, le temps que j’aille avertir mon père. J’ai besoin que tu viennes avec moi, et je te donnerai un salaire. 
${}^{8}– Soit, je reste là, répondit Raphaël. Seulement, ne tarde pas. »
${}^{9}Tobie rentra chez lui avertir son père Tobith : « Voici, j’ai trouvé quelqu’un parmi nos frères, les fils d’Israël. » Et Tobith lui dit : « Fais venir cet homme. Je désire savoir quel est son clan et sa tribu, et si on peut se fier à lui pour t’accompagner, mon enfant. » 
${}^{10}Alors Tobie sortit et l’appela : « Mon ami, lui dit-il, mon père te demande. » Il entra donc chez Tobith, qui le salua le premier, et il lui répondit : « Grande joie pour toi ! » Tobith lui répliqua : « Quelle joie pourrais-je encore avoir ? Moi, qui suis privé de l’usage de mes yeux, je ne vois même plus la lumière du ciel, mais je suis plongé dans les ténèbres, comme les morts qui ne contemplent plus la lumière. Bien que vivant, me voici parmi les morts ; j’entends la voix des gens, mais eux, je ne les aperçois pas. » Raphaël lui dit : « Courage ! Dieu ne tardera pas à te guérir. Courage ! » Tobith lui dit alors : « Mon fils Tobie veut aller en Médie. Pourrais-tu l’accompagner et lui servir de guide ? Je te donnerai un salaire, mon frère. – Oui, répondit Raphaël, je suis en mesure de l’accompagner : je connais toutes les routes, car je suis allé souvent en Médie, j’en ai traversé toutes les plaines et toutes les montagnes ; toutes les routes me sont familières. » 
${}^{11}Tobith lui demanda : « Frère, de quelle famille es-tu et de quelle tribu ? Dis-le moi, frère. 
${}^{12}– Que t’importe ma tribu ? », répartit Raphaël. Tobith lui dit alors : « C’est que je veux savoir en vérité de qui tu es le fils, et quel est ton nom, frère. » 
${}^{13}Raphaël déclara : « Je suis Azarias, fils d’Ananias le Grand, l’un de tes frères. » 
${}^{14}Alors Tobith lui dit : « Sois le bienvenu, frère, salut à toi ! Ne te froisse pas, frère, si j’ai voulu connaître la vérité sur ta famille. Il se trouve que tu es mon frère, et que tu es de bonne famille. J’ai connu Ananias et Nathan, les deux fils de Sémélias le Grand ; ils m’accompagnaient autrefois à Jérusalem pour y faire leurs adorations avec moi, et ils ne sont pas tombés dans l’erreur. Tes frères sont des hommes de bien. Tu es de bonne souche. Entre et sois dans la joie ! » 
${}^{15}Et il ajouta : « Je te donne comme salaire une drachme par jour, et, pour tes frais de voyage, la même chose qu’à mon fils. 
${}^{16}Accompagne-le ; j’ajouterai un supplément à ton salaire. » 
${}^{17}Raphaël répondit : « J’irai avec lui. N’aie aucune crainte : c’est en bonne santé que nous partons et que nous te reviendrons, car la route est sûre. » Tobith lui dit alors : « Sois béni, frère ! » Il appela son fils et lui dit : « Mon enfant, prépare ce qu’il faut pour le voyage, et pars avec ton frère. Que Dieu, lui qui est au ciel, vous protège là-bas et vous ramène à moi en bonne santé ! Et que son ange vous accompagne pour vous protéger, mon enfant ! »
      Tobie sortit pour se mettre en route, il embrassa son père et sa mère, et Tobith lui dit : « Va, porte-toi bien ! » 
${}^{18}Mais sa mère fondit en larmes et dit à Tobith : « Pourquoi as-tu fait partir mon enfant ? N’est-il pas le bâton de nos mains, tant qu’il demeure avec nous ? 
${}^{19}Pourquoi vouloir de l’argent, et encore de l’argent ? Cela ne vaut rien en comparaison de notre fils ! 
${}^{20}Ce que le Seigneur nous avait donné pour vivre nous suffisait bien ! » 
${}^{21}Tobith lui répondit : « Ne te fais pas de souci : c’est en bonne santé que notre enfant va partir et qu’il nous reviendra. Tes yeux verront le jour où il te sera rendu bien portant. Ne t’inquiète pas, n’aie pas de crainte pour eux, ma sœur : 
${}^{22}un bon ange l’accompagnera, son voyage réussira, et il reviendra sain et sauf. » 
${}^{23}Mais elle continua de pleurer en silence.
      
         
      \bchapter{}
      \begin{verse}
${}^{1}Le garçon partit, et l’ange avec lui ; le chien partit aussi avec lui et il les accompagnait. Ils firent donc route ensemble. Survint la première nuit, et ils campèrent au bord d’un fleuve, le Tigre. 
${}^{2}Comme le garçon descendait se laver les pieds dans le Tigre, un grand poisson bondit hors de l’eau et voulut avaler son pied. Le garçon cria. 
${}^{3}Mais l’ange lui dit : « Attrape le poisson, et maîtrise-le. » Le garçon saisit le poisson et le hissa sur la berge. 
${}^{4}L’ange lui dit : « Éventre le poisson, enlève-lui le fiel, le cœur et le foie, mets-les à part pour les emporter, et jette les entrailles. Car le fiel, le cœur et le foie sont des remèdes efficaces. » 
${}^{5}Le garçon éventra le poisson, recueillit le fiel, le cœur et le foie, puis il grilla une partie du poisson et la mangea, et il garda l’autre partie après l’avoir salée. 
${}^{6}Ils poursuivirent tous deux la route, jusqu’aux abords de la Médie. 
${}^{7}Le garçon interrogea alors l’ange : « Azarias, mon frère, le cœur, le foie et le fiel du poisson, en quoi sont-ils un remède ? » 
${}^{8}L’ange lui répondit : « Si tu fais brûler le cœur et le foie du poisson devant un homme ou une femme attaqués par un démon ou un esprit mauvais, l’agresseur s’enfuit au loin, et ses victimes en seront délivrées pour toujours. 
${}^{9}Quant au fiel, si tu l’appliques sur les yeux d’un homme atteint de leucomes et si tu souffles dessus, les yeux seront guéris. »
      
         
${}^{10}Quand il fut entré en Médie et que déjà il approchait d’Ecbatane, 
${}^{11}Raphaël dit au garçon : « Tobie, mon frère », et celui-ci répondit : « Qu’y a-t-il\\ ? » Raphaël reprit : « Nous devons loger cette nuit chez Ragouël. Cet homme est ton parent, et il a une fille qui s’appelle Sarra. 
${}^{12}À part elle, il n’a ni fils ni fille. Tu es le plus proche parent de Sarra : c’est à toi qu’elle revient en priorité et tu as aussi le droit d’hériter de la fortune de son père. D’ailleurs, c’est une jeune fille intelligente, courageuse et très belle, et son père est un homme de bien. » 
${}^{13}Il ajouta : « C’est ton droit de l’épouser. Écoute-moi bien, mon frère. Cette nuit, je parlerai au père de la jeune fille pour qu’il t’accorde sa main, et, à notre retour de Raguès, nous célébrerons les noces. Je sais que Ragouël ne peut te la refuser ni la fiancer à un autre. Sinon, il encourrait la mort selon le décret du Livre de Moïse, car il sait que sa fille te revient de préférence à tout autre. Ainsi donc, écoute-moi bien, mon frère : dès cette nuit, nous aurons un entretien au sujet de cette jeune fille et nous conviendrons du mariage. Quand nous quitterons Raguès, nous la prendrons avec nous et nous l’emmènerons chez toi. »
${}^{14}Tobie répondit à Raphaël : « Azarias, mon frère, j’ai entendu dire qu’elle a déjà eu sept maris et qu’ils sont morts dans leur chambre nuptiale : ils ont succombé la nuit même où ils voulaient s’approcher d’elle. J’ai même entendu dire qu’un démon les tuait. 
${}^{15}Voilà pourquoi j’ai peur, car ce n’est pas elle que le démon attaque, mais il tue quiconque veut s’approcher d’elle. Or je suis le fils unique de mon père, et, si je venais à mourir, je causerais à mon père et ma mère un chagrin qui les conduirait dans la tombe, et ils n’ont pas d’autre fils que moi pour les enterrer ! » 
${}^{16}Raphaël lui répondit : « As-tu oublié les instructions de ton père, qui t’a commandé de prendre femme dans son clan ? Et maintenant, écoute-moi bien, mon frère : ne t’inquiète pas au sujet de ce démon et prends Sarra comme épouse. Car je sais que cette nuit même elle te sera accordée. 
${}^{17}Mais, quand tu entreras dans la chambre nuptiale, prends le cœur du poisson et un peu de son foie, dépose-les sur le brûle-parfums, et l’odeur s’en répandra. Dès que le démon l’aura sentie, il prendra la fuite et il ne reparaîtra plus jamais auprès d’elle. 
${}^{18}Quand tu seras sur le point de t’unir à elle, levez-vous d’abord tous les deux, priez et demandez au Seigneur du ciel de faire venir sur vous sa miséricorde et son salut. N’aie pas peur, car c’est à toi qu’elle a été destinée depuis toujours, et c’est toi qui la sauveras. Elle te suivra, et j’ai bien l’idée que tu auras d’elle des enfants, qui seront pour toi comme des frères. Ne t’inquiète pas. »
${}^{19}En apprenant de Raphaël qu’il avait une parente dans son clan, il s’éprit d’elle passionnément et il lui fut attaché de tout son cœur.
      
         
      \bchapter{}
      \begin{verse}
${}^{1}Entré à Ecbatane, Tobie dit à Raphaël : « Azarias, mon frère, conduis-moi tout droit chez notre frère Ragouël. » Raphaël le conduisit donc chez Ragouël. Ils le trouvèrent assis à l’entrée de la cour et le saluèrent les premiers. Il leur répondit : « Grande joie à vous, frères, soyez les bienvenus ! », et il les fit entrer dans sa maison. 
${}^{2}Il dit à sa femme Edna : « Comme ce jeune homme ressemble à mon frère Tobith ! » 
${}^{3}Edna les interrogea : « D’où êtes-vous, frères ? » Ils lui dirent : « Nous appartenons à la tribu des fils de Nephtali déportés à Ninive. 
${}^{4}– Connaissez-vous notre frère Tobith ?, demanda-t-elle. – Oui, nous le connaissons, dirent-ils. 
${}^{5}– Va-t-il bien ? – Il est vivant et en bonne santé. » Et Tobie ajouta : « C’est mon père. » 
${}^{6}Ragouël se précipita pour l’embrasser, se mit à pleurer et lui dit : « Béni sois-tu, mon enfant : tu es le fils d’un homme de bien\\ ! Quel grand malheur que soit devenu aveugle cet homme juste et généreux\\ ! » Il se jeta au cou de Tobie, son frère, et se remit\\à pleurer. 
${}^{7}Et sa femme Edna pleura sur Tobith, et Sarra, leur fille, pleura elle aussi. 
${}^{8}Ragouël tua un bélier de son troupeau pour recevoir ses hôtes chaleureusement.
      
         
${}^{9}Tobie et Raphaël prirent un bain, ils se lavèrent, avant de prendre place\\pour le repas. Puis, Tobie dit à Raphaël : « Azarias, mon frère, demande à Ragouël de me donner en mariage\\Sarra ma parente\\. » 
${}^{10} Ragouël entendit ces mots et dit au jeune Tobie : « Cette nuit, mange, bois, prends du bon temps : toi seul as le droit d’épouser\\ma fille Sarra, et moi-même je n’ai pas le pouvoir de la donner à un autre homme, puisque tu es mon plus proche parent. Pourtant, je dois te dire la vérité, mon enfant : 
${}^{11} je l’ai donnée en mariage à sept de nos frères, et ils sont morts la nuit même, au moment où ils allaient s’approcher d’elle. Mais à présent, mon enfant, mange et bois : le Seigneur interviendra en votre faveur. » 
${}^{12} Tobie répliqua : « Je ne mangerai ni ne boirai rien, tant que tu n’auras pas pris de décision à mon sujet. » Ragouël lui dit : « Soit ! elle t’est donnée en mariage selon le décret du Livre de Moïse ; c’est un jugement du ciel qui te l’a accordée. Emmène donc ta sœur. Car, dès à présent, tu es son frère et elle est ta sœur\\. À partir d’aujourd’hui elle t’est donnée pour toujours. Que le Seigneur du ciel veille sur vous cette nuit, mon enfant, et vous comble de sa miséricorde et de sa paix ! »
${}^{13}Ragouël appela Sarra\\, qui vint vers lui. Il prit la main de sa fille et la confia à Tobie, en disant : « Emmène-la : conformément à la Loi et au décret consigné dans le Livre de Moïse, elle t’est donnée pour femme. Prends-la et conduis-la en bonne santé chez ton père. Et que le Dieu du ciel vous guide dans la paix\\ ! » 
${}^{14} Puis il appela sa femme\\et lui dit d’apporter une feuille sur laquelle il écrivit l’acte de mariage\\, selon lequel il donnait Sarra\\à Tobie conformément au décret de la loi de Moïse\\. Après quoi, on commença à manger et à boire.
${}^{15}Ragouël s’adressa à sa femme Edna : « Va préparer la seconde chambre, ma sœur, et tu y conduiras notre fille. » 
${}^{16}Elle s’en alla préparer le lit dans la chambre, comme Ragouël l’avait demandé, y conduisit sa fille et pleura sur elle. Puis, elle essuya ses larmes et lui dit : 
${}^{17}« Confiance, ma fille ! Que le Seigneur du ciel change ta douleur en joie ! Confiance, ma fille ! » Puis elle se retira.
      
         
      \bchapter{}
      \begin{verse}
${}^{1}Quand on eut fini de manger et de boire, on décida d’aller se coucher. On conduisit le jeune homme jusqu’à la chambre, où on le fit entrer. 
${}^{2}Tobie se souvint alors des paroles de Raphaël ; il sortit de sa besace le foie et le cœur du poisson et les déposa sur le brûle-parfums. 
${}^{3}L’odeur du poisson repoussa le démon, qui s’enfuit par les airs jusqu’en Égypte. Raphaël s’y rendit, et aussitôt entrava et ligota le démon.
${}^{4}Or les parents de Sarra\\avaient quitté la chambre et fermé la porte. <a class="anchor verset_lettre" id="bib_tb_8_4_b"/> Tobie sortit du lit et dit à Sarra : « Lève-toi, ma sœur. Prions, et demandons à notre Seigneur de nous combler de sa miséricorde et de son salut. » 
${}^{5} Elle se leva, et ils se mirent à prier et à demander que leur soit accordé le salut. Tobie commença ainsi :
       
        \\« Béni sois-tu, Dieu de nos pères ;
        \\béni soit ton nom
        \\dans toutes les générations, à jamais.
        \\Que les cieux te bénissent
        \\et toute ta création, dans tous les siècles.
         
        ${}^{6}C’est toi qui as fait Adam ;
        \\tu lui as fait une aide et un appui :
        \\Ève, sa femme.
        \\Et de tous deux est né le genre humain.
         
        \\C’est toi qui as dit :
        \\“Il n’est pas bon\\que l’homme soit seul.
        \\Je vais lui faire une aide
        \\qui lui soit semblable.”
         
        ${}^{7}Ce n’est donc pas pour une union illégitime
        \\que je prends ma sœur que voici,
        \\mais dans la vérité de la Loi\\.
        \\Daigne me faire miséricorde, ainsi qu’à elle,
        \\et nous mener ensemble à un âge avancé. »
         
        ${}^{8}Puis ils dirent d’une seule voix\\ :
        \\« Amen ! Amen ! »
${}^{9}Et ils se couchèrent pour la nuit. Quant à Ragouël, il se leva, appela ses serviteurs, et ils s’en allèrent creuser une tombe. 
${}^{10}« Car, se disait-il, si jamais Tobie était mort, nous serions objet de risée et de blâme. » 
${}^{11}Quand ils eurent fini de creuser la tombe, Ragouël rentra chez lui, appela sa femme 
${}^{12}et lui dit : « Envoie une jeune servante : qu’elle aille voir si Tobie est encore vivant. Et s’il est mort, nous l’enterrerons, sans que personne ne le sache. » 
${}^{13}Ils envoyèrent donc la servante, allumèrent pour elle une lampe et lui ouvrirent la porte. Elle entra et les trouva couchés qui dormaient ensemble. 
${}^{14}La servante sortit pour annoncer que Tobie était vivant et que rien de mal n’était arrivé. 
${}^{15}Alors Ragouël bénit le Dieu du ciel en s’écriant :
       
        \\« Béni sois-tu, ô Dieu,
        \\par toute bénédiction pure !
        \\Béni sois-tu dans tous les siècles !
         
${}^{16}Béni sois-tu de m’avoir rempli de joie :
        \\ce que je redoutais ne s’est pas réalisé,
        \\mais tu as agi envers nous
        \\selon ta grande miséricorde.
         
${}^{17}Béni sois-tu d’avoir pris en pitié deux enfants uniques !
        \\Accorde-leur, ô Maître, miséricorde et salut ;
        \\fais qu’ils arrivent ensemble au terme de leur vie
        \\dans l’allégresse et la miséricorde. »
       
${}^{18}Puis Ragouël donna aux serviteurs l’ordre de combler la tombe, avant le point du jour.
${}^{19}Il dit à sa femme de cuire des pains en quantité ; lui, il alla choisir deux bœufs et quatre béliers de son troupeau, qu’il fit apprêter. Et on commença les préparatifs. 
${}^{20}Alors il appela Tobie et lui dit : « J’en fais le serment : Pendant quatorze jours, tu ne bougeras pas d’ici, mais tu resteras chez moi à manger et à boire, et tu rendras la joie à ma fille, qui a tant souffert. 
${}^{21}Reçois dès aujourd’hui la moitié de mes biens, puis tu retourneras en bonne santé chez ton père. Quant au reste de ma fortune, elle vous reviendra après ma mort et celle de ma femme. Confiance, mon enfant ! Je suis ton père et Edna est ta mère. Nous sommes auprès de toi et de ta sœur, nous le sommes dès maintenant et pour toujours. Confiance, mon enfant ! »
      
         
      \bchapter{}
      \begin{verse}
${}^{1}Tobie appela Raphaël et lui dit : 
${}^{2}« Azarias, mon frère, prends avec toi quatre serviteurs et deux chameaux. Va trouver Gabaël à Raguès, donne-lui le reçu, prends l’argent et amène Gabaël avec toi pour mes noces. <a class="anchor bib_verset anch-nopad" id="bib_tb_9_3">3</a><sup>-</sup>
${}^{4}Tu sais que mon père compte les jours et, si j’ai un seul jour de retard, je lui causerai beaucoup de peine. Tu connais aussi le serment qu’a fait Ragouël ; je ne puis aller à l’encontre de son serment. » 
${}^{5}Raphaël partit donc à Raguès de Médie avec quatre serviteurs et deux chameaux, et ils s’arrêtèrent pour la nuit chez Gabaël. Raphaël lui remit le reçu et lui apprit que Tobie, fils de Tobith, avait pris femme et l’invitait à son mariage. Gabaël alla chercher les sacoches munies de leurs sceaux, les compta devant Raphaël, et ils les chargèrent sur les chameaux. 
${}^{6}Ils partirent ensemble de bon matin pour aller aux noces. Arrivés chez Ragouël, ils trouvèrent Tobie à table. Celui-ci se leva d’un bond et salua Gabaël, qui se mit à pleurer et qui le bénit en disant : « Fils d’un homme de bien, juste et généreux, tu es toi-même un homme de bien ! Que le Seigneur du ciel te bénisse, toi et ta femme, ainsi que ton père et la mère de ta femme ! Que Dieu soit béni pour m’avoir donné de voir Tobith, mon cousin germain, dans un autre lui-même. »
      
         
      
         
      \bchapter{}
      \begin{verse}
${}^{1}De son côté, Tobith faisait quotidiennement le décompte des jours nécessaires à l’aller et au retour. Quand les jours furent écoulés, son fils n’était pas encore là. 
${}^{2}Il se dit : « Peut-être a-t-il été retenu là-bas ? À moins que Gabaël soit mort et qu’il ne trouve personne pour lui rendre l’argent ? » 
${}^{3}Tobith commençait à s’attrister. 
${}^{4}Quant à sa femme Anna, elle répétait : « Mon enfant a péri ; il n’est plus au nombre des vivants. » Elle se mettait à pleurer et à se lamenter sur son fils : 
${}^{5}« Hélas, mon enfant, je t’ai laissé partir, toi, la lumière de mes yeux ! » 
${}^{6}Et Tobith lui disait : « Tais-toi donc, ma sœur ! Ne t’inquiète pas ! Notre fils va bien. Ils ont dû avoir un contretemps là-bas. D’ailleurs, celui qui l’accompagne est un homme de confiance ; c’est un de nos frères. Ne te tourmente pas au sujet de Tobie, ma sœur : il sera bientôt là. » 
${}^{7}Mais Anna répondait : « Tais-toi ! N’essaie pas de me tromper. Mon enfant a péri. » Et, chaque jour, elle se précipitait pour surveiller elle-même la route par laquelle son fils était parti, car elle ne se fiait plus à personne. Après le coucher du soleil, elle rentrait pour se lamenter et pleurer toute la nuit, sans trouver le sommeil.
${}^{8}Au bout des quatorze jours de noces que Ragouël avait juré de faire pour sa fille, Tobie alla le trouver et lui dit : « Laisse-moi partir. Car je sais que mon père et ma mère ne croient plus me revoir. Je t’en prie, père, laisse-moi partir et je rentrerai chez mon père. Je t’ai déjà décrit dans quel état je l’ai laissé. » 
${}^{9}Ragouël lui répondit : « Reste, mon enfant, reste avec moi. J’enverrai à ton père Tobith des messagers qui lui apporteront de tes nouvelles. » Mais Tobie répliqua : « Pas du tout ! Je t’en prie, laisse-moi partir pour retrouver mon père. »
${}^{10}Aussitôt Ragouël confia à Tobie Sarra, sa femme, et lui remit la moitié de tous les biens qu’il possédait : serviteurs et servantes, bœufs et brebis, ânes et chameaux, vêtements, argent et mobilier. 
${}^{11}Il les laissa partir et fit ses adieux à Tobie en lui disant : « Porte-toi bien, mon enfant ! Va, et porte-toi bien ! Que le Seigneur du ciel vous guide, toi et ta femme Sarra, et qu’il m’accorde de voir vos enfants avant de mourir ! » 
${}^{12}Et il dit à sa fille Sarra : « Va chez ton beau-père et ta belle-mère. Désormais ils sont tes parents, tout comme ceux qui t’ont donné la vie. Va en paix, mon enfant. Puissé-je n’entendre dire que du bien de toi, tant que je vivrai ! » Puis il les embrassa et les laissa partir.
${}^{13}Quant à Edna, elle dit à Tobie : « Mon fils et frère bien-aimé, que le Seigneur te ramène et qu’il me donne de vivre assez longtemps pour voir, avant de mourir, tes enfants et ceux de ma fille Sarra ! Devant le Seigneur, je mets ma fille sous ta protection. Ne lui cause de peine aucun jour de ta vie. Va en paix, mon enfant. Désormais je suis ta mère, comme Sarra est ta sœur. Puissions-nous tous connaître un égal bonheur chaque jour de notre vie ! » Elle les embrassa tendrement tous les deux et les laissa partir.
${}^{14}Tobie quitta Ragouël, en bonne santé et tout joyeux. Il bénit le Seigneur du ciel et de la terre, le roi de l’univers, pour l’heureuse issue de son voyage. Puis, il bénit Ragouël et Edna en leur disant : « Que le Seigneur me permette de vous honorer tous les jours de ma vie. »
      
         
      \bchapter{}
      \begin{verse}
${}^{1}Comme ils approchaient de Kaserîn, qui se trouve en face de Ninive, 
${}^{2}Raphaël dit à Tobie : « Tu sais dans quel état nous avons laissé ton père. 
${}^{3}Prenons de l’avance sur ta femme et allons préparer la maison, avant que les autres n’arrivent. » 
${}^{4}Ils partirent donc tous deux ensemble. Raphaël dit : « Prends avec toi le fiel du poisson. » Et le chien les suivait.
${}^{5}Or, Anna était assise à l’entrée de la cour\\et surveillait la route par laquelle son fils était parti\\. 
${}^{6}Elle le reconnut qui arrivait et cria à Tobith\\ : « Voici ton fils qui revient, et aussi son compagnon de voyage. »
${}^{7}Raphaël dit à Tobie, avant que celui-ci ne s’approche de son père : « J’ai la certitude que ses yeux vont s’ouvrir. 
${}^{8} Étale sur eux le fiel du poisson ; le remède provoquera la contraction des yeux et en détachera le voile blanchâtre\\. Ton père retrouvera la vue et verra la lumière\\. »
${}^{9}Anna courut se jeter au cou de son fils et lui dit : « Je te revois, mon enfant. À présent, je peux mourir ! » Et elle se mit à pleurer. 
${}^{10} Quant à Tobith, il se leva et franchit l’entrée de la cour en trébuchant. 
${}^{11} Tobie alla vers lui, le fiel du poisson à la main. Il lui souffla dans les yeux, le saisit et lui dit : « Confiance, père ! » Puis il lui appliqua le remède et en rajouta. 
${}^{12} Ensuite, de ses deux mains, il lui retira les pellicules\\en partant du coin des yeux. 
${}^{13} Tobith se jeta alors au cou de son fils et lui dit en pleurant : « Je te revois, mon enfant, toi, la lumière de mes yeux ! » 
${}^{14} Et il ajouta :
       
        \\« Béni soit Dieu !
        \\Béni soit son grand nom !
        \\Bénis soient tous ses saints anges !
         
        \\Que son grand nom soit sur nous !
        \\Bénis soient tous les anges
        \\pour tous les siècles !
         
        \\Car Dieu m’avait frappé,
        \\mais voici que je revois
        \\mon fils Tobie ! »
       
${}^{15}Tobie entra dans la maison, tout joyeux et bénissant Dieu à pleine voix\\. Il raconta à son père qu’il avait fait bon voyage, qu’il rapportait l’argent et comment il avait épousé Sarra, la fille de Ragouël : « La voilà qui arrive, ajouta-t-il ; elle est aux portes de Ninive. »
${}^{16}Tobith partit à la rencontre de sa belle-fille, aux portes de Ninive ; il était tout joyeux et bénissait Dieu. En le voyant marcher d’un pas ferme et traverser la ville sans que personne le conduise par la main, les habitants furent émerveillés, et Tobith proclamait\\que Dieu l’avait pris en pitié et lui avait rouvert les yeux. 
${}^{17}Quand il arriva près de Sarra, la femme de son fils Tobie, il la bénit en disant : « Sois la bienvenue, ma fille\\ ! Béni soit ton Dieu de t’avoir menée vers nous ! Béni soit ton père ! Béni soit mon fils Tobie et bénie sois-tu, ma fille ! Sois la bienvenue dans ta maison, sois comblée de bénédiction et de joie. Entre, ma fille ! » 
${}^{18}Ce jour-là fut un jour de joie pour tous les Juifs qui habitaient Ninive. 
${}^{19}Alors arrivèrent, tout joyeux, chez Tobith ses neveux Ahikar et Nabad.
      
         
      \bchapter{}
      \begin{verse}
${}^{1}Quand les noces furent achevées, Tobith appela son fils Tobie et lui dit : « Mon enfant, pense à donner son salaire à ton compagnon de voyage, et ajoute un supplément\\. » 
${}^{2}Tobie lui répondit : « Père, quelle somme vais-je lui donner comme salaire ? Même si je lui donnais la moitié des biens qu’il a rapportés avec moi, je n’y perdrais pas : 
${}^{3}il m’a ramené ici en bonne santé, il a guéri ma femme, il a rapporté l’argent avec moi, et il t’a guéri. Quelle somme vais-je donc lui donner comme salaire ? 
${}^{4}– Mon enfant, reprit Tobith, il est juste qu’il reçoive la moitié de tout ce qu’il a rapporté. » 
${}^{5}Tobith appela Raphaël et lui dit : « Accepte comme salaire la moitié de tout ce que tu as rapporté, et va, porte-toi bien ! »
${}^{6}Alors l’ange les prit tous deux à part et leur dit : « Bénissez Dieu et célébrez-le devant tous les vivants pour le bien qu’il vous a fait. Bénissez-le et chantez son nom. Annoncez à tous les hommes les actions\\de Dieu comme elles le méritent, et n’hésitez pas à le célébrer. 
${}^{7} S’il est bon de tenir cachés les secrets d’un roi, il faut révéler les œuvres de Dieu et les célébrer comme elles le méritent.
      Faites le bien, et le mal ne vous atteindra pas. 
${}^{8} Mieux vaut prier avec vérité et faire l’aumône avec justice, qu’être riche avec injustice. Mieux vaut faire l’aumône qu’amasser de l’or. 
${}^{9} L’aumône délivre de la mort et purifie de tout péché. Ceux qui font l’aumône seront rassasiés de vie, 
${}^{10} tandis que le pécheur et l’homme injuste sont leurs propres ennemis. 
${}^{11} Je veux vous révéler toute la vérité, sans rien vous cacher. Je viens de vous dire que, s’il est bon de tenir cachés les secrets d’un roi, il faut révéler les œuvres de Dieu comme elles le méritent. 
${}^{12} Eh bien ! Quand tu priais en même temps que Sarra\\, c’était moi qui présentais votre prière devant la gloire de Dieu, pour qu’il la garde en mémoire\\, et je faisais de même lorsque tu enterrais les morts\\. 
${}^{13} Quand tu n’as pas hésité\\à te lever, à laisser ton repas et à partir enterrer un mort, c’est alors que j’ai été envoyé vers toi pour te mettre à l’épreuve, 
${}^{14} mais Dieu m’a aussi envoyé pour te guérir, ainsi que Sarra, ta belle-fille. 
${}^{15} Moi, je suis Raphaël, l’un des sept anges qui se tiennent ou se présentent devant la gloire du Seigneur. »
${}^{16}Les deux hommes furent alors bouleversés et ils tombèrent face contre terre, saisis de crainte. 
${}^{17}Mais Raphaël leur dit : « Ne craignez pas ! La paix soit avec vous ! Bénissez Dieu à jamais ! 
${}^{18}Tant que je me suis trouvé avec vous, je n’y étais point par un effet de ma bienveillance, mais par la volonté de Dieu. Bénissez-le donc chaque jour, chantez-lui des hymnes ! 
${}^{19}Vous avez cru me voir manger, mais ce que vous avez vu n’était qu’une apparence. 
${}^{20}Et maintenant, bénissez le Seigneur sur la terre ! Célébrez Dieu ! Voici que je remonte auprès de celui qui m’a envoyé. Mettez par écrit tout ce qui vous est arrivé. » Alors l’ange remonta au ciel. 
${}^{21}Ils se relevèrent, mais ils ne pouvaient plus le voir. 
${}^{22}Ils bénirent Dieu, chantèrent pour lui et le célébrèrent pour la grandeur de ses œuvres : un ange de Dieu leur était apparu !
      <p class="cantique" id="bib_ct-at_5"><span class="cantique_label">Cantique AT 5</span> = <span class="cantique_ref"><a class="unitex_link" href="#bib_tb_13_1">Tb 13, 1b-9</a></span>
      <p class="cantique" id="bib_ct-at_6"><span class="cantique_label">Cantique AT 6</span> = <span class="cantique_ref"><a class="unitex_link" href="#bib_tb_13_10">Tb 13, 10-13.15-18</a></span>
      
         
      \bchapter{}
      \begin{verse}
${}^{1}Tobith dit alors :
      
         
       
        \\<a class="anchor verset_lettre" id="bib_tb_13_1_b"/>Béni soit Dieu, le Vivant, à jamais !
        \\Béni soit son règne !
         
        ${}^{2}C’est lui qui châtie et prend pitié,
        \\qui fait descendre aux profondeurs des enfers
        \\et retire de la grande perdition :
        \\nul n’échappe à sa main.
         
        ${}^{3}Rendez-lui grâce, fils d’Israël, à la face des nations
        \\où lui-même vous a dispersés ;
        ${}^{4}là, il vous a montré\\sa grandeur :
        \\exaltez-le à la face des vivants\\.
         
        \\Car il est notre Seigneur,
        \\lui, notre Dieu, notre Père,
        \\il est Dieu, pour les siècles des siècles !
         
        ${}^{5}Il vous frappera pour vos péchés\\,
        \\mais il vous prendra tous en pitié,
        \\il vous rassemblera\\de toutes les nations
        \\où vous avez été disséminés.
         
        ${}^{6}Si vous revenez vers lui de cœur et d’âme
        pour vivre, dans la vérité, devant lui\\,
        \\alors il reviendra vers vous
        et jamais plus ne cachera sa face.
         
        ${}^{7}Regardez ce qu’il a fait\\pour vous,
        \\rendez-lui grâce à pleine voix !
        \\Bénissez le Seigneur de justice,
        \\exaltez le Roi des siècles !
         
        ${}^{8}Et moi, en terre d’exil, je lui rends grâce\\ ;
        \\je montre sa grandeur et sa force
        au peuple des pécheurs.
         
        \\« Revenez, pécheurs,
        \\et vivez devant lui dans la justice.
        \\Qui sait s’il ne vous rendra pas
        son amour et sa grâce\\ ! »
         
        ${}^{9}J’exalterai mon Dieu, le Roi du ciel ;
        \\mon âme se réjouit de sa grandeur.
        \\Que tous lui rendent grâce à Jérusalem
        et qu’ils disent\\ :
       
        ${}^{10}Jérusalem, ville sainte,
        \\Dieu va te frapper pour les œuvres de tes fils,
        \\mais de nouveau il aura pitié des fils des justes\\.
        ${}^{11}Rends toute grâce au Seigneur
        \\et bénis le Roi des siècles\\ !
         
        \\Qu’il relève en toi le sanctuaire\\,
        ${}^{12}Qu’il réjouisse en toi\\les exilés,
        \\qu’il aime en toi\\les malheureux,
        pour les siècles sans fin\\.
         
        ${}^{13}Une lumière brillante brillera
        \\jusqu’aux limites de la terre.
        \\De loin, viendront des peuples nombreux
        vers ton nom qui est saint,
        \\les mains chargées de leurs offrandes
        pour le Roi du ciel.
         
        \\Les générations des générations t’empliront d’allégresse,
        \\et le nom de l’Élue\\restera pour les siècles.
         
${}^{14}\[Maudits soient tous ceux qui te menaceront,
        \\maudits soient tous ceux qui te détruiront
        ou raseront tes remparts,
        \\tous ceux qui abattront tes tours
        et mettront le feu à tes maisons.
        \\Mais bénis soient tous ceux qui te respecteront,
        à jamais !\]
         
        ${}^{15}Réjouis-toi\\, exulte, à cause des fils des justes :
        \\tous rassemblés, ils béniront le Seigneur éternel.
        \\Heureux ceux qui t’aiment :
        \\ils se réjouiront\\de ta paix.
         
        ${}^{16}Heureux tous ceux qui s’affligeront sur toi
        à cause de toutes tes épreuves :
        \\en toi ils se réjouiront,
        ils prendront part à ta joie pour toujours.
         
        \\Mon âme, bénis le Seigneur, le Grand Roi :
        ${}^{17}il bâtira, dans Jérusalem, sa maison\\pour les siècles !
        \\heureux serai-je, s’il reste quelqu’un de ma descendance
        pour contempler ta gloire et célébrer le Roi du ciel.
         
        \\Les portes de Jérusalem seront d’émeraude et de saphir ;
        \\ses murs\\, de pierre précieuse ;
        \\les tours de Jérusalem seront en or,
        et leurs créneaux, en or pur ;
        \\ses rues, pavées de rubis et de pierres d’Ophir.
         
        ${}^{18}Ses portes retentiront de chants de joie,
        \\et ses demeures diront : « Alléluia !
        \\Béni soit le Dieu d’Israël ! »
         
        \\Que les bénis de Dieu\\bénissent le Nom très saint,
        \\pour les siècles et à jamais !
       
      
         
      \bchapter{}
      \begin{verse}
${}^{1}C’est ainsi que Tobith acheva son cantique d’action de grâce.
      
         
${}^{2}Tobith mourut dans la paix à l’âge de cent douze ans et il fut enterré dignement à Ninive. Il avait soixante-deux ans quand il perdit l’usage de ses yeux. Après avoir retrouvé la vue, il vécut dans l’abondance et fit des aumônes. Il continua de bénir Dieu et de célébrer la grandeur divine.
${}^{3}Au moment de mourir, il appela son fils Tobie et lui fit ces recommandations : « Mon enfant, emmène tes enfants, 
${}^{4}pars vite en Médie, car je crois à cette parole de Dieu que le prophète Nahoum a proférée contre Ninive : tout doit arriver, tout se produira contre Assour et Ninive, comme l’ont annoncé les prophètes d’Israël que Dieu a envoyés ; tout se produira, et rien ne sera retranché de toutes leurs paroles ; toutes choses arriveront en leur temps. Il y aura alors plus de sécurité en Médie qu’en Assour et en Babylonie. Car je sais bien, moi, et je crois que s’accomplira tout ce que Dieu a dit ; tout doit arriver, aucune de ses paroles ne sera effacée.
      Nos frères qui habitent sur la terre d’Israël seront tous disséminés et emmenés en exil loin de ce bon pays. Tout le pays d’Israël deviendra un désert, Samarie et Jérusalem seront un désert, et la maison de Dieu sera livrée à la désolation et à l’incendie, jusqu’au temps fixé.
${}^{5}Mais Dieu les prendra de nouveau en pitié, il les ramènera au pays d’Israël, et ils rebâtiront sa Maison, non pas dans l’état de jadis, mais en attendant que s’accomplissent les temps favorables. Ils reviendront alors tous de leur captivité, ils reconstruiront Jérusalem magnifiquement, et la maison de Dieu y sera rebâtie, comme l’ont annoncé les prophètes d’Israël. 
${}^{6}Toutes les nations de la terre entière, toutes se convertiront et craindront Dieu en vérité. Tous abandonneront les idoles qui les trompaient et les menaient à l’erreur, et, dans la justice, ils béniront le Dieu des siècles. 
${}^{7}Tous les fils d’Israël seront sauvés en ces jours-là, pour s’être souvenus de Dieu en vérité ; ils se rassembleront, ils viendront à Jérusalem et habiteront pour toujours en sécurité dans la terre d’Abraham, qui leur sera donnée. Ceux qui aiment Dieu en vérité se réjouiront ; ceux qui commettent le péché et l’injustice disparaîtront de toute la terre.
      <a class="anchor bib_verset anch-nopad" id="bib_tb_14_8">8</a><sup>-</sup>
${}^{9}Et maintenant, mes enfants, je vous donne ce commandement : servez Dieu en vérité, faites ce qui lui plaît. Ordonnez à vos enfants de pratiquer la justice et l’aumône, afin qu’ils se souviennent de Dieu et qu’en tout temps ils bénissent son nom en vérité et de toute leur force. Quant à toi, mon enfant, quitte Ninive, ne reste pas ici. Dès que tu auras enterré ta mère auprès de moi, ne reste pas un jour de plus dans cette contrée. Car, je le vois, il y a ici beaucoup d’injustices, et les impostures se multiplient, sans que personne en ait honte. 
${}^{10}Mon enfant, considère comment Nadab a traité Ahikar qui l’avait élevé : n’a-t-il pas voulu le précipiter en terre tout vivant ? Mais Dieu lui a jeté son infamie au visage : Ahikar est ressorti à la lumière, tandis que Nadab s’enfonçait dans les ténèbres éternelles, pour avoir tenté de tuer Ahikar. À cause de ses aumônes, Ahikar a échappé au piège mortel que Nadab lui avait tendu. Nadab y tomba lui-même et y trouva la mort. 
${}^{11}Ainsi donc, mes enfants, voyez ce que produit l’aumône ; voyez aussi ce que produit l’injustice : elle conduit à la mort… Mais voici que le souffle m’abandonne. »
      On déposa alors Tobith sur un lit, il mourut et on l’enterra dignement.
${}^{12}À la mort de sa mère, Tobie l’enterra près de son père. Puis il partit pour la Médie avec sa femme et s’établit à Ecbatane, chez son beau-père Ragouël. 
${}^{13}Il prit grand soin de ses beaux-parents dans leur vieillesse et il les enterra à Ecbatane de Médie. Il hérita alors du patrimoine de son beau-père, comme de celui de Tobith, son père. 
${}^{14}Il mourut, respecté de tous, à l’âge de cent dix-sept ans.
${}^{15}Avant de mourir, il apprit la ruine de Ninive et il vit ses habitants arriver en Médie, déportés par Cyaxare, roi de Médie. Tobie bénit Dieu pour tout ce qu’il avait fait aux gens de Ninive et d’Assour. Avant sa mort, il se réjouit du sort de Ninive et il bénit le Seigneur Dieu pour les siècles des siècles.
