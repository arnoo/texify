  
  
    
    \bbook{PREMIÈRE LETTRE DE SAINT JEAN}{PREMIÈRE LETTRE DE SAINT JEAN}
      
         
      \bchapter{}
${}^{1}Ce qui était depuis le commencement,
        ce que nous avons entendu,
        \\ce que nous avons vu de nos yeux,
        ce que nous avons contemplé
        \\et que nos mains ont touché
        du Verbe de vie,
        \\nous vous l’annonçons.
${}^{2}Oui, la vie s’est manifestée,
        nous l’avons vue,
        et nous rendons témoignage :
        \\nous vous annonçons
        la vie éternelle qui était auprès du Père
        et qui s’est manifestée à nous.
${}^{3}Ce que nous avons vu et entendu,
        \\nous vous l’annonçons à vous aussi,
        pour que, vous aussi, vous soyez en communion avec nous.
        \\Or nous sommes, nous aussi, en communion avec le Père
        et avec son Fils, Jésus Christ.
${}^{4}Et nous écrivons cela,
        afin que notre joie soit parfaite.
        
           
${}^{5}Tel est le message que nous avons entendu de Jésus Christ et que nous vous annonçons : Dieu est lumière ; en lui, il n’y a pas de ténèbres. 
${}^{6}Si nous disons que nous sommes en communion avec lui, alors que nous marchons dans les ténèbres, nous sommes des menteurs, nous ne faisons pas la vérité. 
${}^{7}Mais si nous marchons dans la lumière, comme il est lui-même dans la lumière, nous sommes en communion les uns avec les autres, et le sang de Jésus, son Fils, nous purifie de tout péché. 
${}^{8}Si nous disons que nous n’avons pas de péché, nous nous égarons nous-mêmes, et la vérité n’est pas en nous. 
${}^{9}Si nous reconnaissons nos péchés, lui qui est fidèle et juste va jusqu’à pardonner nos péchés et nous purifier de toute injustice. 
${}^{10}Si nous disons que nous sommes sans péché, nous faisons de lui un menteur, et sa parole n’est pas en nous.
      
         
      \bchapter{}
      \begin{verse}
${}^{1}Mes petits enfants, je vous écris cela pour que vous évitiez le péché. Mais si l’un de nous vient à pécher, nous avons un défenseur devant le Père : Jésus Christ, le Juste. 
${}^{2}C’est lui qui, par son sacrifice, obtient le pardon de nos péchés, non seulement les nôtres, mais encore ceux du monde entier.
      
         
      <h3 class="intertitle">Deuxième condition :<br/>
      garder les commandements, surtout celui de l’amour</h3>
${}^{3}Voici comment nous savons que nous le connaissons : si nous gardons ses commandements. 
${}^{4}Celui qui dit : « Je le connais », et qui ne garde pas ses commandements, est un menteur : la vérité n’est pas en lui. 
${}^{5}Mais en celui qui garde sa parole, l’amour de Dieu atteint vraiment la perfection : voilà comment nous savons que nous sommes en lui. 
${}^{6}Celui qui déclare demeurer en lui doit, lui aussi, marcher comme Jésus lui-même a marché.
${}^{7}Bien-aimés, ce n’est pas un commandement nouveau que je vous écris, mais un commandement ancien que vous aviez depuis le commencement. La parole que vous avez entendue, c’est le commandement ancien. 
${}^{8}Et pourtant, c’est un commandement nouveau que je vous écris ; ce qui est vrai en cette parole l’est aussi en vous ; en effet, les ténèbres passent et déjà brille la vraie lumière. 
${}^{9}Celui qui déclare être dans la lumière et qui a de la haine contre son frère est dans les ténèbres jusqu’à maintenant. 
${}^{10}Celui qui aime son frère demeure dans la lumière, et il n’y a en lui aucune occasion de chute. 
${}^{11}Mais celui qui a de la haine contre son frère est dans les ténèbres : il marche dans les ténèbres sans savoir où il va, parce que les ténèbres ont aveuglé ses yeux.
        ${}^{12}Je vous l’écris, petits enfants :
        Vos péchés vous sont remis à cause du nom de Jésus.
        ${}^{13}Je vous l’écris, parents :
        Vous connaissez celui qui existe depuis le commencement.
        \\Je vous l’écris, jeunes gens :
        Vous avez vaincu le Mauvais.
        ${}^{14}Je vous l’ai écrit, enfants :
        Vous connaissez le Père.
        \\Je vous l’ai écrit, parents :
        Vous connaissez celui qui existe depuis le commencement.
        \\Je vous l’ai écrit, jeunes gens :
        Vous êtes forts,
        la parole de Dieu demeure en vous,
        vous avez vaincu le Mauvais.
${}^{15}N’aimez pas le monde, ni ce qui est dans le monde. Si quelqu’un aime le monde, l’amour du Père n’est pas en lui. 
${}^{16}Tout ce qu’il y a dans le monde  – la convoitise de la chair, la convoitise des yeux, l’arrogance de la richesse –, tout cela ne vient pas du Père, mais du monde. 
${}^{17}Or, le monde passe, et sa convoitise avec lui. Mais celui qui fait la volonté de Dieu demeure pour toujours.
       
${}^{18}Mes enfants, c’est la dernière heure et, comme vous l’avez appris, un anti-Christ, un adversaire du Christ, doit venir ; or, il y a dès maintenant beaucoup d’anti-Christs ; nous savons ainsi que c’est la dernière heure. 
${}^{19}Ils sont sortis de chez nous mais ils n’étaient pas des nôtres ; s’ils avaient été des nôtres, ils seraient demeurés avec nous. Mais pas un d’entre eux n’est des nôtres, et cela devait être manifesté. 
${}^{20}Quant à vous, c’est de celui qui est saint que vous tenez l’onction, et vous avez tous la connaissance. 
${}^{21}Je ne vous ai pas écrit que vous ignorez la vérité, mais que vous la connaissez, et que de la vérité ne vient aucun mensonge.
${}^{22}Le menteur n’est-il pas celui qui refuse que Jésus soit le Christ ? Celui-là est l’anti-Christ : il refuse à la fois le Père et le Fils ; 
${}^{23}quiconque refuse le Fils n’a pas non plus le Père ; celui qui reconnaît le Fils a aussi le Père. 
${}^{24}Quant à vous, que demeure en vous ce que vous avez entendu depuis le commencement. Si ce que vous avez entendu depuis le commencement demeure en vous, vous aussi, vous demeurerez dans le Fils et dans le Père. 
${}^{25}Et telle est la promesse que lui-même nous a faite : la vie éternelle.
${}^{26}Je vous ai écrit cela à propos de ceux qui vous égarent. 
${}^{27}Quant à vous, l’onction que vous avez reçue de lui demeure en vous, et vous n’avez pas besoin d’enseignement. Cette onction vous enseigne toutes choses, elle qui est vérité et non pas mensonge ; et, selon ce qu’elle vous a enseigné, vous demeurez en lui.
${}^{28}Et maintenant, petits enfants, demeurez en lui ; ainsi, quand il se manifestera, nous aurons de l’assurance, et non pas la honte d’être loin de lui à son avènement.
${}^{29}Puisque vous savez que lui, Jésus, est juste, reconnaissez que celui qui pratique la justice est, lui aussi, né de Dieu.
      
         
      \bchapter{}
      \begin{verse}
${}^{1}Voyez quel grand amour nous a donné le Père pour que nous soyons appelés enfants de Dieu  – et nous le sommes. Voici pourquoi le monde ne nous connaît pas : c’est qu’il n’a pas connu Dieu. 
${}^{2}Bien-aimés, dès maintenant, nous sommes enfants de Dieu, mais ce que nous serons n’a pas encore été manifesté. Nous le savons : quand cela sera manifesté, nous lui serons semblables car nous le verrons tel qu’il est. 
${}^{3}Et quiconque met en lui une telle espérance se rend pur comme lui-même est pur.
${}^{4}Qui commet le péché transgresse la loi ; car le péché, c’est la transgression. 
${}^{5}Or, vous savez que lui, Jésus, s’est manifesté pour enlever les péchés, et qu’il n’y a pas de péché en lui. 
${}^{6}Quiconque demeure en lui ne pèche pas ; quiconque pèche ne l’a pas vu et ne le connaît pas.
${}^{7}Petits enfants, que nul ne vous égare : celui qui pratique la justice est juste comme lui, Jésus, est juste ; 
${}^{8}celui qui commet le péché est du diable, car, depuis le commencement, le diable est pécheur. C’est pour détruire les œuvres du diable que le Fils de Dieu s’est manifesté. 
${}^{9}Quiconque est né de Dieu ne commet pas de péché, car ce qui a été semé par Dieu demeure en lui : il ne peut donc pas pécher, puisqu’il est né de Dieu. 
${}^{10}Voici comment se manifestent les enfants de Dieu et les enfants du diable : quiconque ne pratique pas la justice n’est pas de Dieu, et pas davantage celui qui n’aime pas son frère.
${}^{11}Tel est le message que vous avez entendu depuis le commencement : aimons-nous les uns les autres. 
${}^{12}Ne soyons pas comme Caïn : il appartenait au Mauvais et il égorgea son frère. Et pourquoi l’a-t-il égorgé ? Parce que ses œuvres étaient mauvaises : au contraire, celles de son frère étaient justes.
${}^{13}Ne soyez pas étonnés, frères, si le monde a de la haine contre vous. 
${}^{14}Nous, nous savons que nous sommes passés de la mort à la vie, parce que nous aimons nos frères. Celui qui n’aime pas demeure dans la mort. 
${}^{15}Quiconque a de la haine contre son frère est un meurtrier, et vous savez que pas un meurtrier n’a la vie éternelle demeurant en lui. 
${}^{16}Voici comment nous avons reconnu l’amour : lui, Jésus, a donné sa vie pour nous. Nous aussi, nous devons donner notre vie pour nos frères. 
${}^{17}Celui qui a de quoi vivre en ce monde, s’il voit son frère dans le besoin sans faire preuve de compassion, comment l’amour de Dieu pourrait-il demeurer en lui ? 
${}^{18}Petits enfants, n’aimons pas en paroles ni par des discours, mais par des actes et en vérité.
${}^{19}Voilà comment nous reconnaîtrons que nous appartenons à la vérité, et devant Dieu nous apaiserons notre cœur ; 
${}^{20}car si notre cœur nous accuse, Dieu est plus grand que notre cœur, et il connaît toutes choses.
       
${}^{21}Bien-aimés, si notre cœur ne nous accuse pas, nous avons de l’assurance devant Dieu. 
${}^{22}Quoi que nous demandions à Dieu, nous le recevons de lui, parce que nous gardons ses commandements, et que nous faisons ce qui est agréable à ses yeux.
${}^{23}Or, voici son commandement : mettre notre foi dans le nom de son Fils Jésus Christ, et nous aimer les uns les autres comme il nous l’a commandé. 
${}^{24}Celui qui garde ses commandements demeure en Dieu, et Dieu en lui ; et voilà comment nous reconnaissons qu’il demeure en nous, puisqu’il nous a donné part à son Esprit.
      
         
      \bchapter{}
      \begin{verse}
${}^{1}Bien-aimés, ne vous fiez pas à n’importe quelle inspiration, mais examinez les esprits pour voir s’ils sont de Dieu, car beaucoup de faux prophètes se sont répandus dans le monde. 
${}^{2}Voici comment vous reconnaîtrez l’Esprit de Dieu : tout esprit qui proclame que Jésus Christ est venu dans la chair, celui-là est de Dieu. 
${}^{3}Tout esprit qui refuse de proclamer Jésus, celui-là n’est pas de Dieu : c’est l’esprit de l’anti-Christ, dont on vous a annoncé la venue et qui, dès maintenant, est déjà dans le monde.
${}^{4}Vous, petits enfants, vous êtes de Dieu, et vous avez vaincu ces gens-là ; car Celui qui est en vous est plus grand que celui qui est dans le monde. 
${}^{5}Eux, ils sont du monde ; voilà pourquoi ils parlent le langage du monde, et le monde les écoute. 
${}^{6}Nous, nous sommes de Dieu ; celui qui connaît Dieu nous écoute ; celui qui n’est pas de Dieu ne nous écoute pas. C’est ainsi que nous reconnaissons l’esprit de la vérité et l’esprit de l’erreur.
        ${}^{7}Bien-aimés,
        \\aimons-nous les uns les autres,
        puisque l’amour vient de Dieu.
        \\Celui qui aime
        \\est né de Dieu
        et connaît Dieu.
        ${}^{8}Celui qui n’aime pas n’a pas connu Dieu,
        car Dieu est amour.
         
        ${}^{9}Voici comment l’amour de Dieu s’est manifesté parmi nous :
        \\Dieu a envoyé son Fils unique dans le monde
        pour que nous vivions par lui.
        ${}^{10}Voici en quoi consiste l’amour :
        \\ce n’est pas nous qui avons aimé Dieu,
        mais c’est lui qui nous a aimés,
        \\et il a envoyé son Fils
        en sacrifice de pardon pour nos péchés,
${}^{11}Bien-aimés, puisque Dieu nous a tellement aimés, nous devons, nous aussi, nous aimer les uns les autres. 
${}^{12}Dieu, personne ne l’a jamais vu. Mais si nous nous aimons les uns les autres, Dieu demeure en nous, et, en nous, son amour atteint la perfection. 
${}^{13}Voici comment nous reconnaissons que nous demeurons en lui et lui en nous : il nous a donné part à son Esprit. 
${}^{14}Quant à nous, nous avons vu et nous attestons que le Père a envoyé son Fils comme Sauveur du monde. 
${}^{15}Celui qui proclame que Jésus est le Fils de Dieu, Dieu demeure en lui, et lui en Dieu. 
${}^{16}Et nous, nous avons reconnu l’amour que Dieu a pour nous, et nous y avons cru.
      Dieu est amour : qui demeure dans l’amour demeure en Dieu, et Dieu demeure en lui. 
${}^{17}Voici comment l’amour atteint, chez nous, sa perfection : avoir de l’assurance au jour du jugement ; comme Jésus, en effet, nous ne manquons pas d’assurance en ce monde. 
${}^{18}Il n’y a pas de crainte dans l’amour, l’amour parfait bannit la crainte ; car la crainte implique un châtiment, et celui qui reste dans la crainte n’a pas atteint la perfection de l’amour. 
${}^{19}Quant à nous, nous aimons parce que Dieu lui-même nous a aimés le premier. 
${}^{20}Si quelqu’un dit : « J’aime Dieu », alors qu’il a de la haine contre son frère, c’est un menteur. En effet, celui qui n’aime pas son frère, qu’il voit, est incapable d’aimer Dieu, qu’il ne voit pas. 
${}^{21}Et voici le commandement que nous tenons de lui : celui qui aime Dieu, qu’il aime aussi son frère.
      
         
      \bchapter{}
      \begin{verse}
${}^{1}Celui qui croit que Jésus est le Christ, celui-là est né de Dieu ; celui qui aime le Père qui a engendré aime aussi le Fils qui est né de lui. 
${}^{2}Voici comment nous reconnaissons que nous aimons les enfants de Dieu : lorsque nous aimons Dieu et que nous accomplissons ses commandements. 
${}^{3}Car tel est l’amour de Dieu : garder ses commandements ; et ses commandements ne sont pas un fardeau, 
${}^{4}puisque tout être qui est né de Dieu est vainqueur du monde. Or la victoire remportée sur le monde, c’est notre foi.
${}^{5}Qui donc est vainqueur du monde ? N’est-ce pas celui qui croit que Jésus est le Fils de Dieu ? 
${}^{6}C’est lui, Jésus Christ, qui est venu par l’eau et par le sang : non pas seulement avec l’eau, mais avec l’eau et avec le sang. Et celui qui rend témoignage, c’est l’Esprit, car l’Esprit est la vérité. 
${}^{7}En effet, ils sont trois qui rendent témoignage, 
${}^{8}l’Esprit, l’eau et le sang, et les trois n’en font qu’un. 
${}^{9}Nous acceptons bien le témoignage des hommes ; or, le témoignage de Dieu a plus de valeur, puisque le témoignage de Dieu, c’est celui qu’il rend à son Fils. 
${}^{10}Celui qui met sa foi dans le Fils de Dieu possède en lui-même ce témoignage. Celui qui ne croit pas Dieu, celui-là fait de Dieu un menteur, puisqu’il n’a pas mis sa foi dans le témoignage que Dieu rend à son Fils. 
${}^{11}Et ce témoignage, le voici : Dieu nous a donné la vie éternelle, et cette vie est dans son Fils. 
${}^{12}Celui qui a le Fils possède la vie ; celui qui n’a pas le Fils de Dieu ne possède pas la vie.
       
${}^{13}Je vous ai écrit cela pour que vous sachiez que vous avez la vie éternelle, vous qui mettez votre foi dans le nom du Fils de Dieu.
${}^{14}Voici l’assurance que nous avons auprès de Dieu : si nous faisons une demande selon sa volonté, il nous écoute. 
${}^{15}Et puisque nous savons qu’il nous écoute en toutes nos demandes, nous savons aussi que nous obtenons ce que nous lui avons demandé.
${}^{16}Si quelqu’un voit son frère commettre un péché qui n’entraîne pas la mort, il demandera, et Dieu lui donnera la vie,  – cela vaut pour ceux dont le péché n’entraîne pas la mort. Il y a un péché qui entraîne la mort, ce n’est pas pour celui-là que je dis de prier. 
${}^{17}Toute conduite injuste est péché, mais tout péché n’entraîne pas la mort.
${}^{18}Nous le savons : ceux qui sont nés de Dieu ne commettent pas de péché ; le Fils engendré par Dieu les protège et le Mauvais ne peut pas les atteindre. 
${}^{19}Nous savons que nous sommes de Dieu, alors que le monde entier est au pouvoir du Mauvais. 
${}^{20}Nous savons aussi que le Fils de Dieu est venu nous donner l’intelligence pour que nous connaissions Celui qui est vrai ; et nous sommes en Celui qui est vrai, en son Fils Jésus Christ. C’est lui qui est le Dieu vrai, et la vie éternelle.
${}^{21}Petits enfants, gardez-vous des idoles.
