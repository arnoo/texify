  
  
${}^{14}Lorsque Jésus, dans la puissance de l’Esprit, revint en Galilée, sa renommée se répandit dans toute la région. 
${}^{15}Il enseignait dans les synagogues, et tout le monde faisait son éloge.
${}^{16}Il vint à Nazareth, où il avait été élevé. Selon son habitude, il entra dans la synagogue le jour du sabbat, et il se leva pour faire la lecture. 
${}^{17}On lui remit le livre du prophète Isaïe. Il ouvrit le livre et trouva le passage où il est écrit :
        ${}^{18}L’Esprit du Seigneur est sur moi
        \\parce que le Seigneur m’a consacré par l’onction.
        \\Il m’a envoyé porter la Bonne Nouvelle aux pauvres,
        \\annoncer aux captifs leur libération,
        \\et aux aveugles qu’ils retrouveront la vue,
        \\remettre en liberté les opprimés,
        ${}^{19}annoncer une année favorable accordée par le Seigneur.
${}^{20}Jésus referma le livre, le rendit au servant et s’assit. Tous, dans la synagogue, avaient les yeux fixés sur lui. 
${}^{21}Alors il se mit à leur dire : « Aujourd’hui s’accomplit ce passage de l’Écriture que vous venez d’entendre. »
${}^{22}Tous lui rendaient témoignage et s’étonnaient des paroles de grâce qui sortaient de sa bouche. Ils se disaient : « N’est-ce pas là le fils de Joseph ? » 
${}^{23}Mais il leur dit : « Sûrement vous allez me citer le dicton : “Médecin, guéris-toi toi-même”, et me dire : “Nous avons appris tout ce qui s’est passé à Capharnaüm ; fais donc de même ici dans ton lieu d’origine !” » 
${}^{24}Puis il ajouta : « Amen, je vous le dis : aucun prophète ne trouve un accueil favorable dans son pays. 
${}^{25}En vérité, je vous le dis : Au temps du prophète Élie, lorsque pendant trois ans et demi le ciel retint la pluie, et qu’une grande famine se produisit sur toute la terre, il y avait beaucoup de veuves en Israël ; 
${}^{26}pourtant Élie ne fut envoyé vers aucune d’entre elles, mais bien dans la ville de Sarepta, au pays de Sidon, chez une veuve étrangère. 
${}^{27}Au temps du prophète Élisée, il y avait beaucoup de lépreux en Israël ; et aucun d’eux n’a été purifié, mais bien Naaman le Syrien. » 
${}^{28}À ces mots, dans la synagogue, tous devinrent furieux. 
${}^{29}Ils se levèrent, poussèrent Jésus hors de la ville, et le menèrent jusqu’à un escarpement de la colline où leur ville est construite, pour le précipiter en bas. 
${}^{30}Mais lui, passant au milieu d’eux, allait son chemin.
${}^{31}Jésus descendit à Capharnaüm, ville de Galilée, et il y enseignait, le jour du sabbat. 
${}^{32}On était frappé par son enseignement car sa parole était pleine d’autorité.
${}^{33}Or, il y avait dans la synagogue un homme possédé par l’esprit d’un démon impur, qui se mit à crier d’une voix forte : 
${}^{34}« Ah ! que nous veux-tu, Jésus de Nazareth ? Es-tu venu pour nous perdre ? Je sais qui tu es : tu es le Saint de Dieu. » 
${}^{35}Jésus le menaça : « Silence ! Sors de cet homme. » Alors le démon projeta l’homme en plein milieu et sortit de lui sans lui faire aucun mal. 
${}^{36}Tous furent saisis d’effroi et ils se disaient entre eux : « Quelle est cette parole ? Il commande avec autorité et puissance aux esprits impurs, et ils sortent ! » 
${}^{37}Et la réputation de Jésus se propageait dans toute la région.
${}^{38}Jésus quitta la synagogue et entra dans la maison de Simon. Or, la belle-mère de Simon était oppressée par une forte fièvre, et on demanda à Jésus de faire quelque chose pour elle. 
${}^{39}Il se pencha sur elle, menaça la fièvre, et la fièvre la quitta. À l’instant même, la femme se leva et elle les servait.
${}^{40}Au coucher du soleil, tous ceux qui avaient des malades atteints de diverses infirmités les lui amenèrent. Et Jésus, imposant les mains à chacun d’eux, les guérissait. 
${}^{41}Et même des démons sortaient de beaucoup d’entre eux en criant : « C’est toi le Fils de Dieu ! » Mais Jésus les menaçait et leur interdisait de parler, parce qu’ils savaient, eux, que le Christ, c’était lui.
${}^{42}Quand il fit jour, Jésus sortit et s’en alla dans un endroit désert. Les foules le cherchaient ; elles arrivèrent jusqu’à lui, et elles le retenaient pour l’empêcher de les quitter. 
${}^{43}Mais il leur dit : « Aux autres villes aussi, il faut que j’annonce la Bonne Nouvelle du règne de Dieu, car c’est pour cela que j’ai été envoyé. » 
${}^{44}Et il proclamait l’Évangile dans les synagogues du pays des Juifs.
      
         
      \bchapter{}
      \begin{verse}
${}^{1}Or, la foule se pressait autour de Jésus pour écouter la parole de Dieu, tandis qu’il se tenait au bord du lac de Génésareth. 
${}^{2}Il vit deux barques qui se trouvaient au bord du lac ; les pêcheurs en étaient descendus et lavaient leurs filets. 
${}^{3}Jésus monta dans une des barques qui appartenait à Simon, et lui demanda de s’écarter un peu du rivage. Puis il s’assit et, de la barque, il enseignait les foules. 
${}^{4}Quand il eut fini de parler, il dit à Simon : « Avance au large, et jetez vos filets pour la pêche. » 
${}^{5}Simon lui répondit : « Maître, nous avons peiné toute la nuit sans rien prendre ; mais, sur ta parole, je vais jeter les filets. » 
${}^{6}Et l’ayant fait, ils capturèrent une telle quantité de poissons que leurs filets allaient se déchirer. 
${}^{7}Ils firent signe à leurs compagnons de l’autre barque de venir les aider. Ceux-ci vinrent, et ils remplirent les deux barques, à tel point qu’elles enfonçaient. 
${}^{8}À cette vue, Simon-Pierre tomba aux genoux de Jésus, en disant : « Éloigne-toi de moi, Seigneur, car je suis un homme pécheur. » 
${}^{9}En effet, un grand effroi l’avait saisi, lui et tous ceux qui étaient avec lui, devant la quantité de poissons qu’ils avaient pêchés ; 
${}^{10}et de même Jacques et Jean, fils de Zébédée, les associés de Simon. Jésus dit à Simon : « Sois sans crainte, désormais ce sont des hommes que tu prendras. » 
${}^{11}Alors ils ramenèrent les barques au rivage et, laissant tout, ils le suivirent.
      
         
${}^{12}Jésus était dans une ville quand survint un homme couvert de lèpre ; voyant Jésus, il tomba face contre terre et le supplia : « Seigneur, si tu le veux, tu peux me purifier. » 
${}^{13}Jésus étendit la main et le toucha en disant : « Je le veux, sois purifié. » À l’instant même, la lèpre le quitta. 
${}^{14}Alors Jésus lui ordonna de ne le dire à personne : « Va plutôt te montrer au prêtre et donne pour ta purification ce que Moïse a prescrit ; ce sera pour tous un témoignage. » 
${}^{15}De plus en plus, on parlait de Jésus. De grandes foules accouraient pour l’entendre et se faire guérir de leurs maladies. 
${}^{16}Mais lui se retirait dans les endroits déserts, et il priait.
${}^{17}Un jour que Jésus enseignait, il y avait dans l’assistance des pharisiens et des docteurs de la Loi, venus de tous les villages de Galilée et de Judée, ainsi que de Jérusalem ; et la puissance du Seigneur était à l’œuvre pour lui faire opérer des guérisons. 
${}^{18}Arrivent des gens, portant sur une civière un homme qui était paralysé ; ils cherchaient à le faire entrer pour le placer devant Jésus. 
${}^{19}Mais, ne voyant pas comment faire à cause de la foule, ils montèrent sur le toit et, en écartant les tuiles, ils le firent descendre avec sa civière en plein milieu devant Jésus. 
${}^{20}Voyant leur foi, il dit : « Homme, tes péchés te sont pardonnés. » 
${}^{21}Les scribes et les pharisiens se mirent à raisonner : « Qui est-il celui-là ? Il dit des blasphèmes ! Qui donc peut pardonner les péchés, sinon Dieu seul ? » 
${}^{22}Mais Jésus, saisissant leurs pensées, leur répondit : « Pourquoi ces pensées dans vos cœurs ? 
${}^{23}Qu’est-ce qui est le plus facile ? Dire : “Tes péchés te sont pardonnés”, ou dire : “Lève-toi et marche” ? 
${}^{24}Eh bien ! Afin que vous sachiez que le Fils de l’homme a autorité sur la terre pour pardonner les péchés, – Jésus s’adressa à celui qui était paralysé – je te le dis, lève-toi, prends ta civière et retourne dans ta maison. » 
${}^{25}À l’instant même, celui-ci se releva devant eux, il prit ce qui lui servait de lit et s’en alla dans sa maison en rendant gloire à Dieu. 
${}^{26}Tous furent saisis de stupeur et ils rendaient gloire à Dieu. Remplis de crainte, ils disaient : « Nous avons vu des choses extraordinaires aujourd’hui ! »
${}^{27}Après cela, Jésus sortit et remarqua un publicain (c’est-à-dire un collecteur d’impôts) du nom de Lévi assis au bureau des impôts. Il lui dit : « Suis-moi. » 
${}^{28}Abandonnant tout, l’homme se leva ; et il le suivait. 
${}^{29}Lévi donna pour Jésus une grande réception dans sa maison ; il y avait là une foule nombreuse de publicains et d’autres gens attablés avec eux. 
${}^{30}Les pharisiens et les scribes de leur parti récriminaient en disant à ses disciples : « Pourquoi mangez-vous et buvez-vous avec les publicains et les pécheurs ? » 
${}^{31}Jésus leur répondit : « Ce ne sont pas les gens en bonne santé qui ont besoin du médecin, mais les malades. 
${}^{32}Je ne suis pas venu appeler des justes mais des pécheurs, pour qu’ils se convertissent. »
${}^{33}Ils lui dirent alors : « Les disciples de Jean le Baptiste jeûnent souvent et font des prières ; de même ceux des pharisiens. Au contraire, les tiens mangent et boivent ! » 
${}^{34}Jésus leur dit : « Pouvez-vous faire jeûner les invités de la noce, pendant que l’Époux est avec eux ? 
${}^{35}Mais des jours viendront où l’Époux leur sera enlevé ; alors, en ces jours-là, ils jeûneront. »
${}^{36}Il leur dit aussi en parabole : « Personne ne déchire un morceau à un vêtement neuf pour le coudre sur un vieux vêtement. Autrement, on aura déchiré le neuf, et le morceau qui vient du neuf ne s’accordera pas avec le vieux. 
${}^{37}Et personne ne met du vin nouveau dans de vieilles outres ; autrement, le vin nouveau fera éclater les outres, il se répandra et les outres seront perdues. 
${}^{38}Mais on doit mettre le vin nouveau dans des outres neuves. 
${}^{39}Jamais celui qui a bu du vin vieux ne désire du nouveau. Car il dit : “C’est le vieux qui est bon.” »
      
         
      \bchapter{}
      \begin{verse}
${}^{1}Un jour de sabbat, Jésus traversait des champs ; ses disciples arrachaient des épis et les mangeaient, après les avoir froissés dans leurs mains. 
${}^{2}Quelques pharisiens dirent alors : « Pourquoi faites-vous ce qui n’est pas permis le jour du sabbat ? » 
${}^{3}Jésus leur répondit : « N’avez-vous pas lu ce que fit David un jour qu’il eut faim, lui-même et ceux qui l’accompagnaient ? 
${}^{4}Il entra dans la maison de Dieu, prit les pains de l’offrande, en mangea et en donna à ceux qui l’accompagnaient, alors que les prêtres seulement ont le droit d’en manger. » 
${}^{5}Il leur disait encore : « Le Fils de l’homme est maître du sabbat. »
${}^{6}Un autre jour de sabbat, Jésus était entré dans la synagogue et enseignait. Il y avait là un homme dont la main droite était desséchée. 
${}^{7}Les scribes et les pharisiens observaient Jésus pour voir s’il ferait une guérison le jour du sabbat ; ils auraient ainsi un motif pour l’accuser. 
${}^{8}Mais lui connaissait leurs raisonnements, et il dit à l’homme qui avait la main desséchée : « Lève-toi, et tiens-toi debout, là au milieu. » L’homme se dressa et se tint debout. 
${}^{9}Jésus leur dit : « Je vous le demande : Est-il permis, le jour du sabbat, de faire le bien ou de faire le mal ? de sauver une vie ou de la perdre ? » 
${}^{10}Alors, promenant son regard sur eux tous, il dit à l’homme : « Étends la main. » Il le fit, et sa main redevint normale. 
${}^{11}Quant à eux, ils furent remplis de fureur et ils discutaient entre eux sur ce qu’ils feraient à Jésus.
${}^{12}En ces jours-là, Jésus s’en alla dans la montagne pour prier, et il passa toute la nuit à prier Dieu. 
${}^{13}Le jour venu, il appela ses disciples et en choisit douze auxquels il donna le nom d’Apôtres : 
${}^{14}Simon, auquel il donna le nom de Pierre, André son frère, Jacques, Jean, Philippe, Barthélemy, 
${}^{15}Matthieu, Thomas, Jacques fils d’Alphée, Simon appelé le Zélote, 
${}^{16}Jude fils de Jacques, et Judas Iscariote, qui devint un traître.
${}^{17}Jésus descendit de la montagne avec eux et s’arrêta sur un terrain plat. Il y avait là un grand nombre de ses disciples et une grande multitude de gens venus de toute la Judée, de Jérusalem, et du littoral de Tyr et de Sidon. 
${}^{18}Ils étaient venus l’entendre et se faire guérir de leurs maladies ; ceux qui étaient tourmentés par des esprits impurs retrouvaient la santé. 
${}^{19}Et toute la foule cherchait à le toucher, parce qu’une force sortait de lui et les guérissait tous.
        ${}^{20}Et Jésus, levant les yeux sur ses disciples, déclara :
        \\« Heureux, vous les pauvres,
        \\car le royaume de Dieu est à vous.
        ${}^{21}Heureux, vous qui avez faim maintenant,
        \\car vous serez rassasiés.
        \\Heureux, vous qui pleurez maintenant,
        \\car vous rirez.
        ${}^{22}Heureux êtes-vous quand les hommes vous haïssent
        et vous excluent,
        \\quand ils insultent
        et rejettent votre nom comme méprisable,
        à cause du Fils de l’homme.
${}^{23}Ce jour-là, réjouissez-vous, tressaillez de joie,
      car alors votre récompense est grande dans le ciel ;
      c’est ainsi, en effet, que leurs pères traitaient les prophètes.
        ${}^{24}Mais quel malheur pour vous, les riches,
        \\car vous avez votre consolation !
        ${}^{25}Quel malheur pour vous qui êtes repus maintenant,
        \\car vous aurez faim !
        \\Quel malheur pour vous qui riez maintenant,
        \\car vous serez dans le deuil et vous pleurerez !
        ${}^{26}Quel malheur pour vous
        \\lorsque tous les hommes disent du bien de vous !
        \\C’est ainsi, en effet, que leurs pères traitaient les faux prophètes.
${}^{27}Mais je vous le dis, à vous qui m’écoutez : Aimez vos ennemis, faites du bien à ceux qui vous haïssent. 
${}^{28}Souhaitez du bien à ceux qui vous maudissent, priez pour ceux qui vous calomnient. 
${}^{29}À celui qui te frappe sur une joue, présente l’autre joue. À celui qui te prend ton manteau, ne refuse pas ta tunique. 
${}^{30}Donne à quiconque te demande, et à qui prend ton bien, ne le réclame pas.
${}^{31}Ce que vous voulez que les autres fassent pour vous, faites-le aussi pour eux. 
${}^{32}Si vous aimez ceux qui vous aiment, quelle reconnaissance méritez-vous ? Même les pécheurs aiment ceux qui les aiment. 
${}^{33}Si vous faites du bien à ceux qui vous en font, quelle reconnaissance méritez-vous ? Même les pécheurs en font autant. 
${}^{34}Si vous prêtez à ceux dont vous espérez recevoir en retour, quelle reconnaissance méritez-vous ? Même les pécheurs prêtent aux pécheurs pour qu’on leur rende l’équivalent. 
${}^{35}Au contraire, aimez vos ennemis, faites du bien et prêtez sans rien espérer en retour. Alors votre récompense sera grande, et vous serez les fils du Très-Haut, car lui, il est bon pour les ingrats et les méchants.
${}^{36}Soyez miséricordieux comme votre Père est miséricordieux. 
${}^{37}Ne jugez pas, et vous ne serez pas jugés ; ne condamnez pas, et vous ne serez pas condamnés. Pardonnez, et vous serez pardonnés. 
${}^{38}Donnez, et l’on vous donnera : c’est une mesure bien pleine, tassée, secouée, débordante, qui sera versée dans le pan de votre vêtement ; car la mesure dont vous vous servez pour les autres servira de mesure aussi pour vous. »
${}^{39}Il leur dit encore en parabole :
      « Un aveugle peut-il guider un autre aveugle ? Ne vont-ils pas tomber tous les deux dans un trou ? 
${}^{40}Le disciple n’est pas au-dessus du maître ; mais une fois bien formé, chacun sera comme son maître.
${}^{41}Qu’as-tu à regarder la paille dans l’œil de ton frère, alors que la poutre qui est dans ton œil à toi, tu ne la remarques pas ? 
${}^{42}Comment peux-tu dire à ton frère : “Frère, laisse-moi enlever la paille qui est dans ton œil”, alors que toi-même ne vois pas la poutre qui est dans le tien ? Hypocrite ! Enlève d’abord la poutre de ton œil ; alors tu verras clair pour enlever la paille qui est dans l’œil de ton frère.
${}^{43}Un bon arbre ne donne pas de fruit pourri ; jamais non plus un arbre qui pourrit ne donne de bon fruit. 
${}^{44}Chaque arbre, en effet, se reconnaît à son fruit : on ne cueille pas des figues sur des épines ; on ne vendange pas non plus du raisin sur des ronces. 
${}^{45}L’homme bon tire le bien du trésor de son cœur qui est bon ; et l’homme mauvais tire le mal de son cœur qui est mauvais : car ce que dit la bouche, c’est ce qui déborde du cœur.
${}^{46}Et pourquoi m’appelez-vous en disant : “Seigneur ! Seigneur !” et ne faites-vous pas ce que je dis ? 
${}^{47}Quiconque vient à moi, écoute mes paroles et les met en pratique, je vais vous montrer à qui il ressemble. 
${}^{48}Il ressemble à celui qui construit une maison. Il a creusé très profond et il a posé les fondations sur le roc. Quand est venue l’inondation, le torrent s’est précipité sur cette maison, mais il n’a pas pu l’ébranler parce qu’elle était bien construite. 
${}^{49}Mais celui qui a écouté et n’a pas mis en pratique ressemble à celui qui a construit sa maison à même le sol, sans fondations. Le torrent s’est précipité sur elle, et aussitôt elle s’est effondrée ; la destruction de cette maison a été complète. »
      
         
      \bchapter{}
      \begin{verse}
${}^{1}Lorsque Jésus eut achevé de faire entendre au peuple toutes ses paroles, il entra dans Capharnaüm. 
${}^{2}Il y avait un centurion dont un esclave était malade et sur le point de mourir ; or le centurion tenait beaucoup à lui. 
${}^{3}Ayant entendu parler de Jésus, il lui envoya des notables juifs pour lui demander de venir sauver son esclave. 
${}^{4}Arrivés près de Jésus, ceux-ci le suppliaient instamment : « Il mérite que tu lui accordes cela. 
${}^{5}Il aime notre nation : c’est lui qui nous a construit la synagogue. » 
${}^{6}Jésus était en route avec eux, et déjà il n’était plus loin de la maison, quand le centurion envoya des amis lui dire : « Seigneur, ne prends pas cette peine, car je ne suis pas digne que tu entres sous mon toit. 
${}^{7}C’est pourquoi je ne me suis pas autorisé, moi-même, à venir te trouver. Mais dis une parole, et que mon serviteur soit guéri ! 
${}^{8}Moi, je suis quelqu’un de subordonné à une autorité, mais j’ai des soldats sous mes ordres ; à l’un, je dis : “Va”, et il va ; à un autre : “Viens”, et il vient ; et à mon esclave : “Fais ceci”, et il le fait. » 
${}^{9}Entendant cela, Jésus fut en admiration devant lui. Il se retourna et dit à la foule qui le suivait : « Je vous le déclare, même en Israël, je n’ai pas trouvé une telle foi ! » 
${}^{10}Revenus à la maison, les envoyés trouvèrent l’esclave en bonne santé.
      
         
${}^{11}Par la suite, Jésus se rendit dans une ville appelée Naïm. Ses disciples faisaient route avec lui, ainsi qu’une grande foule. 
${}^{12}Il arriva près de la porte de la ville au moment où l’on emportait un mort pour l’enterrer ; c’était un fils unique, et sa mère était veuve. Une foule importante de la ville accompagnait cette femme. 
${}^{13}Voyant celle-ci, le Seigneur fut saisi de compassion pour elle et lui dit : « Ne pleure pas. » 
${}^{14}Il s’approcha et toucha le cercueil ; les porteurs s’arrêtèrent, et Jésus dit : « Jeune homme, je te l’ordonne, lève-toi. » 
${}^{15}Alors le mort se redressa et se mit à parler. Et Jésus le rendit à sa mère. 
${}^{16}La crainte s’empara de tous, et ils rendaient gloire à Dieu en disant : « Un grand prophète s’est levé parmi nous, et Dieu a visité son peuple. » 
${}^{17}Et cette parole sur Jésus se répandit dans la Judée entière et dans toute la région.
${}^{18}Les disciples de Jean le Baptiste annoncèrent tout cela à leur maître. <a class="anchor verset_lettre" id="bib_lc_7_18_b"/>Alors Jean appela deux d’entre eux 
${}^{19}et les envoya demander au Seigneur : « Es-tu celui qui doit venir, ou devons-nous en attendre un autre ? » 
${}^{20}Arrivés près de Jésus, ils lui dirent : « Jean le Baptiste nous a envoyés te demander : Es-tu celui qui doit venir, ou devons-nous en attendre un autre ? » 
${}^{21}À cette heure-là, Jésus guérit beaucoup de gens de leurs maladies, de leurs infirmités et des esprits mauvais dont ils étaient affligés, et à beaucoup d’aveugles, il accorda de voir. 
${}^{22}Puis il répondit aux envoyés : « Allez annoncer à Jean ce que vous avez vu et entendu : les aveugles retrouvent la vue, les boiteux marchent, les lépreux sont purifiés, les sourds entendent, les morts ressuscitent, les pauvres reçoivent la Bonne Nouvelle. 
${}^{23}Heureux celui qui ne trébuchera pas à cause de moi ! »
${}^{24}Après le départ des messagers de Jean, Jésus se mit à dire aux foules à propos de Jean : « Qu’êtes-vous allés regarder au désert ? un roseau agité par le vent ? 
${}^{25}Alors, qu’êtes-vous allés voir ? un homme habillé de vêtements raffinés ? Mais ceux qui portent des vêtements somptueux et qui vivent dans le luxe sont dans les palais royaux. 
${}^{26}Alors, qu’êtes-vous allés voir ? un prophète ? Oui, je vous le dis ; et bien plus qu’un prophète ! 
${}^{27}C’est de lui qu’il est écrit :
        \\Voici que j’envoie mon messager en avant de toi,
        \\pour préparer le chemin devant toi.
${}^{28}Je vous le dis : Parmi ceux qui sont nés d’une femme, personne n’est plus grand que Jean ; et cependant le plus petit dans le royaume de Dieu est plus grand que lui.
${}^{29}Tout le peuple qui a écouté Jean, y compris les publicains, en recevant de lui le baptême, a reconnu que Dieu était juste. 
${}^{30}Mais les pharisiens et les docteurs de la Loi, en ne recevant pas son baptême, ont rejeté le dessein que Dieu avait sur eux.
${}^{31}À qui donc vais-je comparer les gens de cette génération ? À qui ressemblent-ils ? 
${}^{32}Ils ressemblent à des gamins assis sur la place, qui s’interpellent en disant :
        \\“Nous vous avons joué de la flûte,
        \\et vous n’avez pas dansé.
        \\Nous avons chanté des lamentations,
        \\et vous n’avez pas pleuré.”
${}^{33}Jean le Baptiste est venu, en effet ; il ne mange pas de pain, il ne boit pas de vin, et vous dites : “C’est un possédé !” 
${}^{34}Le Fils de l’homme est venu ; il mange et il boit, et vous dites : “Voilà un glouton et un ivrogne, un ami des publicains et des pécheurs.” 
${}^{35}Mais, par tous ses enfants, la sagesse de Dieu a été reconnue juste. »
${}^{36}Un pharisien avait invité Jésus à manger avec lui. Jésus entra chez lui et prit place à table. 
${}^{37}Survint une femme de la ville, une pécheresse. Ayant appris que Jésus était attablé dans la maison du pharisien, elle avait apporté un flacon d’albâtre contenant un parfum. 
${}^{38}Tout en pleurs, elle se tenait derrière lui, près de ses pieds, et elle se mit à mouiller de ses larmes les pieds de Jésus. Elle les essuyait avec ses cheveux, les couvrait de baisers et répandait sur eux le parfum. 
${}^{39}En voyant cela, le pharisien qui avait invité Jésus se dit en lui-même : « Si cet homme était prophète, il saurait qui est cette femme qui le touche, et ce qu’elle est : une pécheresse. » 
${}^{40}Jésus, prenant la parole, lui dit : « Simon, j’ai quelque chose à te dire. – Parle, Maître. » 
${}^{41}Jésus reprit : « Un créancier avait deux débiteurs ; le premier lui devait cinq cents pièces d’argent, l’autre cinquante. 
${}^{42}Comme ni l’un ni l’autre ne pouvait les lui rembourser, il en fit grâce à tous deux. Lequel des deux l’aimera davantage ? » 
${}^{43}Simon répondit : « Je suppose que c’est celui à qui on a fait grâce de la plus grande dette. – Tu as raison », lui dit Jésus. 
${}^{44}Il se tourna vers la femme et dit à Simon : « Tu vois cette femme ? Je suis entré dans ta maison, et tu ne m’as pas versé de l’eau sur les pieds ; elle, elle les a mouillés de ses larmes et essuyés avec ses cheveux. 
${}^{45}Tu ne m’as pas embrassé ; elle, depuis qu’elle est entrée, n’a pas cessé d’embrasser mes pieds. 
${}^{46}Tu n’as pas fait d’onction sur ma tête ; elle, elle a répandu du parfum sur mes pieds. 
${}^{47}Voilà pourquoi je te le dis : ses péchés, ses nombreux péchés, sont pardonnés, puisqu’elle a montré beaucoup d’amour. Mais celui à qui on pardonne peu montre peu d’amour. » 
${}^{48}Il dit alors à la femme : « Tes péchés sont pardonnés. » 
${}^{49}Les convives se mirent à dire en eux-mêmes : « Qui est cet homme, qui va jusqu’à pardonner les péchés ? » 
${}^{50}Jésus dit alors à la femme : « Ta foi t’a sauvée. Va en paix ! »
      
         
      \bchapter{}
      \begin{verse}
${}^{1}Ensuite, il arriva que Jésus, passant à travers villes et villages, proclamait et annonçait la Bonne Nouvelle du règne de Dieu. Les Douze l’accompagnaient, 
${}^{2}ainsi que des femmes qui avaient été guéries de maladies et d’esprits mauvais : Marie, appelée Madeleine, de laquelle étaient sortis sept démons, 
${}^{3}Jeanne, femme de Kouza, intendant d’Hérode, Suzanne, et beaucoup d’autres, qui les servaient en prenant sur leurs ressources.
      
         
${}^{4}Comme une grande foule se rassemblait, et que de chaque ville on venait vers Jésus, il dit dans une parabole : 
${}^{5}« Le semeur sortit pour semer la semence, et comme il semait, il en tomba au bord du chemin. Les passants la piétinèrent, et les oiseaux du ciel mangèrent tout. 
${}^{6}Il en tomba aussi dans les pierres, elle poussa et elle sécha parce qu’elle n’avait pas d’humidité. 
${}^{7}Il en tomba aussi au milieu des ronces, et les ronces, en poussant avec elle, l’étouffèrent. 
${}^{8}Il en tomba enfin dans la bonne terre, elle poussa et elle donna du fruit au centuple. » Disant cela, il éleva la voix : « Celui qui a des oreilles pour entendre, qu’il entende ! »
${}^{9}Ses disciples lui demandaient ce que signifiait cette parabole. 
${}^{10}Il leur déclara : « À vous il est donné de connaître les mystères du royaume de Dieu, mais les autres n’ont que les paraboles. Ainsi, comme il est écrit :
        \\Ils regardent sans regarder,
        \\ils entendent sans comprendre.
${}^{11}Voici ce que signifie la parabole. La semence, c’est la parole de Dieu. 
${}^{12}Il y a ceux qui sont au bord du chemin : ceux-là ont entendu ; puis le diable survient et il enlève de leur cœur la Parole, pour les empêcher de croire et d’être sauvés. 
${}^{13}Il y a ceux qui sont dans les pierres : lorsqu’ils entendent, ils accueillent la Parole avec joie ; mais ils n’ont pas de racines, ils croient pour un moment et, au moment de l’épreuve, ils abandonnent. 
${}^{14}Ce qui est tombé dans les ronces, ce sont les gens qui ont entendu, mais qui sont étouffés, chemin faisant, par les soucis, la richesse et les plaisirs de la vie, et ne parviennent pas à maturité. 
${}^{15}Et ce qui est tombé dans la bonne terre, ce sont les gens qui ont entendu la Parole dans un cœur bon et généreux, qui la retiennent et portent du fruit par leur persévérance.
${}^{16}Personne, après avoir allumé une lampe, ne la couvre d’un vase ou ne la met sous le lit ; on la met sur le lampadaire pour que ceux qui entrent voient la lumière. 
${}^{17}Car rien n’est caché qui ne doive paraître au grand jour ; rien n’est secret qui ne doive être connu et venir au grand jour.
${}^{18}Faites attention à la manière dont vous écoutez. Car à celui qui a, on donnera ; et à celui qui n’a pas, même ce qu’il croit avoir sera enlevé. »
${}^{19}La mère et les frères de Jésus vinrent le trouver, mais ils ne pouvaient pas arriver jusqu’à lui à cause de la foule.
${}^{20}On le lui fit savoir : « Ta mère et tes frères sont là dehors, qui veulent te voir. » 
${}^{21}Il leur répondit : « Ma mère et mes frères sont ceux qui écoutent la parole de Dieu et la mettent en pratique. »
${}^{22}Un jour, Jésus monta dans une barque avec ses disciples et il leur dit : « Passons sur l’autre rive du lac. » Et ils gagnèrent le large. 
${}^{23}Pendant qu’ils naviguaient, Jésus s’endormit. Une tempête s’abattit sur le lac. Ils étaient submergés et en grand péril. 
${}^{24}Les disciples s’approchèrent et le réveillèrent en disant : « Maître, maître ! Nous sommes perdus ! » Et lui, se réveillant, menaça le vent et les flots agités. Ils s’apaisèrent et le calme se fit. 
${}^{25}Alors Jésus leur dit : « Où est votre foi ? » Remplis de crainte, ils furent saisis d’étonnement et se disaient entre eux : « Qui est-il donc, celui-ci, pour qu’il commande même aux vents et aux flots, et que ceux-ci lui obéissent ? »
${}^{26}Ils abordèrent au pays des Géraséniens, qui est en face de la Galilée. 
${}^{27}Comme Jésus descendait à terre, un homme de la ville, qui était possédé par des démons, vint à sa rencontre. Depuis assez longtemps il ne mettait pas de vêtement et n’habitait pas dans une maison, mais dans les tombeaux. 
${}^{28}Voyant Jésus, il poussa des cris, tomba à ses pieds et dit d’une voix forte : « Que me veux-tu, Jésus, Fils du Dieu Très-Haut ? Je t’en prie, ne me tourmente pas. » 
${}^{29}En effet, Jésus commandait à l’esprit impur de sortir de cet homme, car l’esprit s’était emparé de lui bien des fois. On le gardait alors lié par des chaînes, avec des entraves aux pieds, mais il rompait ses liens et le démon l’entraînait vers les endroits déserts. 
${}^{30}Jésus lui demanda : « Quel est ton nom ? » Il répondit : « Légion ». En effet, beaucoup de démons étaient entrés en lui. 
${}^{31}Et ces démons suppliaient Jésus de ne pas leur ordonner de s’en aller dans l’abîme. 
${}^{32}Or, il y avait là un troupeau de porcs assez important qui cherchait sa nourriture sur la colline. Les démons supplièrent Jésus de leur permettre d’entrer dans ces porcs, et il le leur permit. 
${}^{33}Ils sortirent de l’homme et ils entrèrent dans les porcs. Du haut de la falaise, le troupeau se précipita dans le lac et s’y noya. 
${}^{34}Voyant ce qui s’était passé, les gardiens du troupeau prirent la fuite ; ils annoncèrent la nouvelle dans la ville et dans la campagne, 
${}^{35}et les gens sortirent pour voir ce qui s’était passé. Arrivés auprès de Jésus, ils trouvèrent l’homme que les démons avaient quitté ; il était assis, habillé, et revenu à la raison, aux pieds de Jésus. Et ils furent saisis de crainte. 
${}^{36}Ceux qui avaient vu leur rapportèrent comment le possédé avait été sauvé. 
${}^{37}Alors toute la population du territoire des Géraséniens demanda à Jésus de partir de chez eux, parce qu’ils étaient en proie à une grande crainte. Jésus remonta dans la barque et s’en retourna. 
${}^{38}L’homme que les démons avaient quitté lui demandait de pouvoir être avec lui. Mais Jésus le renvoya en disant : 
${}^{39}« Retourne chez toi et raconte tout ce que Dieu a fait pour toi. » Alors cet homme partit proclamer dans la ville entière tout ce que Jésus avait fait pour lui.
${}^{40}Quand Jésus revint en Galilée, il fut accueilli par la foule, car tous l’attendaient. 
${}^{41}Et voici qu’arriva un homme du nom de Jaïre ; c’était le chef de la synagogue. Tombant aux pieds de Jésus, il le suppliait de venir dans sa maison, 
${}^{42}parce qu’il avait une fille unique, d’environ douze ans, qui se mourait. Et tandis que Jésus s’y rendait, les foules le pressaient au point de l’étouffer.
${}^{43}Or, une femme qui avait des pertes de sang depuis douze ans, et qui avait dépensé tous ses biens chez les médecins sans que personne n’ait pu la guérir, 
${}^{44}s’approcha de lui par-derrière et toucha la frange de son vêtement. À l’instant même, sa perte de sang s’arrêta. 
${}^{45}Mais Jésus dit : « Qui m’a touché ? » Comme ils s’en défendaient tous, Pierre lui dit : « Maître, les foules te bousculent et t’écrasent. » 
${}^{46}Mais Jésus reprit : « Quelqu’un m’a touché, car j’ai reconnu qu’une force était sortie de moi. » 
${}^{47}La femme, se voyant découverte, vint, toute tremblante, se jeter à ses pieds ; elle raconta devant tout le peuple pourquoi elle l’avait touché, et comment elle avait été guérie à l’instant même. 
${}^{48}Jésus lui dit : « Ma fille, ta foi t’a sauvée. Va en paix. »
${}^{49}Comme il parlait encore, quelqu’un arrive de la maison de Jaïre, le chef de synagogue, pour dire à celui-ci : « Ta fille est morte. Ne dérange plus le maître. » 
${}^{50}Jésus, qui avait entendu, lui déclara : « Ne crains pas. Crois seulement, et elle sera sauvée. » 
${}^{51}En arrivant à la maison, il ne laissa personne entrer avec lui, sauf Pierre, Jean et Jacques, ainsi que le père de l’enfant et sa mère. 
${}^{52}Tous la pleuraient en se frappant la poitrine. Mais Jésus dit : « Ne pleurez pas ; elle n’est pas morte : elle dort. » 
${}^{53}Mais on se moquait de lui, sachant qu’elle venait de mourir. 
${}^{54}Alors il lui saisit la main et dit d’une voix forte : « Mon enfant, éveille-toi ! » 
${}^{55}L’esprit lui revint et, à l’instant même, elle se leva. Alors Jésus ordonna de lui donner à manger. 
${}^{56}Ses parents furent frappés de stupeur ; quant à Jésus, il leur commanda de ne dire à personne ce qui était arrivé.
      
         
      \bchapter{}
      \begin{verse}
${}^{1}Jésus rassembla les Douze ; il leur donna pouvoir et autorité sur tous les démons, et de même pour faire des guérisons ; 
${}^{2}il les envoya proclamer le règne de Dieu et guérir les malades. 
${}^{3}Il leur dit : « Ne prenez rien pour la route, ni bâton, ni sac, ni pain, ni argent ; n’ayez pas chacun une tunique de rechange. 
${}^{4}Quand vous serez reçus dans une maison, restez-y ; c’est de là que vous repartirez. 
${}^{5}Et si les gens ne vous accueillent pas, sortez de la ville et secouez la poussière de vos pieds : ce sera un témoignage contre eux. » 
${}^{6}Ils partirent et ils allaient de village en village, annonçant la Bonne Nouvelle et faisant partout des guérisons.
      
         
${}^{7}Hérode, qui était au pouvoir en Galilée, entendit parler de tout ce qui se passait et il ne savait que penser. En effet, certains disaient que Jean le Baptiste était ressuscité d’entre les morts. 
${}^{8}D’autres disaient : « C’est le prophète Élie qui est apparu. » D’autres encore : « C’est un prophète d’autrefois qui est ressuscité. » 
${}^{9}Quant à Hérode, il disait : « Jean, je l’ai fait décapiter. Mais qui est cet homme dont j’entends dire de telles choses ? » Et il cherchait à le voir.
${}^{10}Quand les Apôtres revinrent, ils racontèrent à Jésus tout ce qu’ils avaient fait. Alors Jésus, les prenant avec lui, partit à l’écart, vers une ville appelée Bethsaïde. 
${}^{11}Les foules s’en aperçurent et le suivirent. Il leur fit bon accueil ; il leur parlait du règne de Dieu et guérissait ceux qui en avaient besoin.
${}^{12}Le jour commençait à baisser. Alors les Douze s’approchèrent de lui et lui dirent : « Renvoie cette foule : qu’ils aillent dans les villages et les campagnes des environs afin d’y loger et de trouver des vivres ; ici nous sommes dans un endroit désert. » 
${}^{13}Mais il leur dit : « Donnez-leur vous-mêmes à manger. » Ils répondirent : « Nous n’avons pas plus de cinq pains et deux poissons. À moins peut-être d’aller nous-mêmes acheter de la nourriture pour tout ce peuple. » 
${}^{14}Il y avait environ cinq mille hommes. Jésus dit à ses disciples : « Faites-les asseoir par groupes de cinquante environ. » 
${}^{15}Ils exécutèrent cette demande et firent asseoir tout le monde. 
${}^{16}Jésus prit les cinq pains et les deux poissons, et, levant les yeux au ciel, il prononça la bénédiction sur eux, les rompit et les donna à ses disciples pour qu’ils les distribuent à la foule. 
${}^{17}Ils mangèrent et ils furent tous rassasiés ; puis on ramassa les morceaux qui leur restaient : cela faisait douze paniers.
${}^{18}En ce jour-là, Jésus était en prière à l’écart. Comme ses disciples étaient là, il les interrogea : « Au dire des foules, qui suis-je ? » 
${}^{19}Ils répondirent : « Jean le Baptiste ; mais pour d’autres, Élie ; et pour d’autres, un prophète d’autrefois qui serait ressuscité. » 
${}^{20}Jésus leur demanda : « Et vous, que dites-vous ? Pour vous, qui suis-je ? » Alors Pierre prit la parole et dit : « Le Christ, le Messie de Dieu. » 
${}^{21}Mais Jésus, avec autorité, leur défendit vivement de le dire à personne, 
${}^{22}et déclara : « Il faut que le Fils de l’homme souffre beaucoup, qu’il soit rejeté par les anciens, les grands prêtres et les scribes, qu’il soit tué, et que, le troisième jour, il ressuscite. »
${}^{23}Il leur disait à tous : « Celui qui veut marcher à ma suite, qu’il renonce à lui-même, qu’il prenne sa croix chaque jour et qu’il me suive. 
${}^{24}Car celui qui veut sauver sa vie la perdra ; mais celui qui perdra sa vie à cause de moi la sauvera. 
${}^{25}Quel avantage un homme aura-t-il à gagner le monde entier, s’il se perd ou se ruine lui-même ? 
${}^{26}Celui qui a honte de moi et de mes paroles, le Fils de l’homme aura honte de lui, quand il viendra dans la gloire, la sienne, celle du Père et des saints anges. 
${}^{27}Je vous le dis en vérité : parmi ceux qui sont ici présents, certains ne connaîtront pas la mort avant d’avoir vu le règne de Dieu. »
${}^{28}Environ huit jours après avoir prononcé ces paroles, Jésus prit avec lui Pierre, Jean et Jacques, et il gravit la montagne pour prier. 
${}^{29}Pendant qu’il priait, l’aspect de son visage devint autre, et son vêtement devint d’une blancheur éblouissante. 
${}^{30}Voici que deux hommes s’entretenaient avec lui : c’étaient Moïse et Élie, 
${}^{31}apparus dans la gloire. Ils parlaient de son départ qui allait s’accomplir à Jérusalem. 
${}^{32}Pierre et ses compagnons étaient accablés de sommeil ; mais, restant éveillés, ils virent la gloire de Jésus, et les deux hommes à ses côtés. 
${}^{33}Ces derniers s’éloignaient de lui, quand Pierre dit à Jésus : « Maître, il est bon que nous soyons ici ! Faisons trois tentes : une pour toi, une pour Moïse, et une pour Élie. » Il ne savait pas ce qu’il disait. 
${}^{34}Pierre n’avait pas fini de parler, qu’une nuée survint et les couvrit de son ombre ; ils furent saisis de frayeur lorsqu’ils y pénétrèrent. 
${}^{35}Et, de la nuée, une voix se fit entendre : « Celui-ci est mon Fils, celui que j’ai choisi : écoutez-le ! » 
${}^{36}Et pendant que la voix se faisait entendre, il n’y avait plus que Jésus, seul. Les disciples gardèrent le silence et, en ces jours-là, ils ne rapportèrent à personne rien de ce qu’ils avaient vu.
${}^{37}Le lendemain, quand ils descendirent de la montagne, une grande foule vint à la rencontre de Jésus. 
${}^{38}Et voilà qu’un homme, dans la foule, se mit à crier : « Maître, je t’en prie, regarde mon fils, car c’est mon unique enfant, 
${}^{39}et il arrive qu’un esprit s’empare de lui, pousse tout à coup des cris, le secoue de convulsions et le fait écumer ; il ne s’éloigne de lui qu’à grand-peine en le laissant tout brisé. 
${}^{40}J’ai prié tes disciples d’expulser cet esprit, mais ils n’ont pas pu le faire. » 
${}^{41}Prenant la parole, Jésus dit : « Génération incroyante et dévoyée, combien de temps vais-je rester près de vous et vous supporter ? Fais avancer ici ton fils. » 
${}^{42}À peine l’enfant s’était-il approché que le démon le terrassa et le fit entrer en convulsions. Jésus menaça l’esprit impur, guérit l’enfant et le rendit à son père. 
${}^{43}Et tous étaient frappés d’étonnement devant la grandeur de Dieu.
      <a class="anchor verset_lettre" id="bib_lc_9_43_b"/>Comme tout le monde était dans l’admiration devant tout ce qu’il faisait, Jésus dit à ses disciples : 
${}^{44}« Ouvrez bien vos oreilles à ce que je vous dis maintenant : le Fils de l’homme va être livré aux mains des hommes. » 
${}^{45}Mais les disciples ne comprenaient pas cette parole, elle leur était voilée, si bien qu’ils n’en percevaient pas le sens, et ils avaient peur de l’interroger sur cette parole.
${}^{46}Une discussion survint entre les disciples pour savoir qui, parmi eux, était le plus grand. 
${}^{47}Mais Jésus, sachant quelle discussion occupait leur cœur, prit un enfant, le plaça à côté de lui 
${}^{48}et leur dit : « Celui qui accueille en mon nom cet enfant, il m’accueille, moi. Et celui qui m’accueille accueille celui qui m’a envoyé. En effet, le plus petit d’entre vous tous, c’est celui-là qui est grand. »
${}^{49}Jean, l’un des Douze, dit à Jésus : « Maître, nous avons vu quelqu’un expulser des démons en ton nom ; nous l’en avons empêché, car il ne marche pas à ta suite avec nous. » 
${}^{50}Jésus lui répondit : « Ne l’en empêchez pas : qui n’est pas contre vous est pour vous. »
