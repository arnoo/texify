  
  
      
         
      \bchapter{}
      \begin{verse}
${}^{1}L’an quinze du règne de l’empereur Tibère, Ponce Pilate étant gouverneur de la Judée, Hérode étant alors au pouvoir en Galilée, son frère Philippe dans le pays d’Iturée et de Traconitide, Lysanias en Abilène, 
${}^{2}les grands prêtres étant Hanne et Caïphe, la parole de Dieu fut adressée dans le désert à Jean, le fils de Zacharie. 
${}^{3}Il parcourut toute la région du Jourdain, en proclamant un baptême de conversion pour le pardon des péchés, 
${}^{4}comme il est écrit dans le livre des oracles d’Isaïe, le prophète :
        \\Voix de celui qui crie dans le désert :
        \\Préparez le chemin du Seigneur,
        \\rendez droits ses sentiers.
        ${}^{5}Tout ravin sera comblé,
        \\toute montagne et toute colline seront abaissées ;
        \\les passages tortueux deviendront droits,
        \\les chemins rocailleux seront aplanis ;
        ${}^{6}et tout être vivant verra le salut de Dieu.
${}^{7}Jean disait aux foules qui arrivaient pour être baptisées par lui : « Engeance de vipères ! Qui vous a appris à fuir la colère qui vient ? 
${}^{8}Produisez donc des fruits qui expriment votre conversion. Ne commencez pas à vous dire : “Nous avons Abraham pour père”, car je vous dis que, de ces pierres, Dieu peut faire surgir des enfants à Abraham. 
${}^{9}Déjà la cognée se trouve à la racine des arbres : tout arbre qui ne produit pas de bons fruits va être coupé et jeté au feu. »
${}^{10}Les foules lui demandaient : « Que devons-nous donc faire ? » 
${}^{11}Jean leur répondait : « Celui qui a deux vêtements, qu’il partage avec celui qui n’en a pas ; et celui qui a de quoi manger, qu’il fasse de même ! » 
${}^{12}Des publicains (c’est-à-dire des collecteurs d’impôts) vinrent aussi pour être baptisés ; ils lui dirent : « Maître, que devons-nous faire ? » 
${}^{13}Il leur répondit : « N’exigez rien de plus que ce qui vous est fixé. » 
${}^{14}Des soldats lui demandèrent à leur tour : « Et nous, que devons-nous faire ? » Il leur répondit : « Ne faites violence à personne, n’accusez personne à tort ; et contentez-vous de votre solde. »
${}^{15}Or le peuple était en attente, et tous se demandaient en eux-mêmes si Jean n’était pas le Christ. 
${}^{16}Jean s’adressa alors à tous : « Moi, je vous baptise avec de l’eau ; mais il vient, celui qui est plus fort que moi. Je ne suis pas digne de dénouer la courroie de ses sandales. Lui vous baptisera dans l’Esprit Saint et le feu. 
${}^{17}Il tient à la main la pelle à vanner pour nettoyer son aire à battre le blé, et il amassera le grain dans son grenier ; quant à la paille, il la brûlera au feu qui ne s’éteint pas. »
${}^{18}Par beaucoup d’autres exhortations encore, il annonçait au peuple la Bonne Nouvelle. 
${}^{19}Hérode, qui était au pouvoir en Galilée, avait reçu des reproches de Jean au sujet d’Hérodiade, la femme de son frère, et au sujet de tous les méfaits qu’il avait commis. 
${}^{20}À tout cela il ajouta encore ceci : il fit enfermer Jean dans une prison.
${}^{21}Comme tout le peuple se faisait baptiser et qu’après avoir été baptisé lui aussi, Jésus priait, le ciel s’ouvrit. 
${}^{22}L’Esprit Saint, sous une apparence corporelle, comme une colombe, descendit sur Jésus, et il y eut une voix venant du ciel : « Toi, tu es mon Fils bien-aimé ; en toi, je trouve ma joie. »
${}^{23}Quand il commença, Jésus avait environ trente ans ; il était, à ce que l’on pensait, fils de Joseph, fils d’Éli, 
${}^{24}fils de Matthate, fils de Lévi, fils de Melki, fils de Jannaï, fils de Joseph, 
${}^{25}fils de Mattathias, fils d’Amos, fils de Nahoum, fils de Hesli, fils de Naggaï, 
${}^{26}fils de Maath, fils de Mattathias, fils de Séméine, fils de Josek, fils de Joda, 
${}^{27}fils de Joanane, fils de Résa,
      fils de Zorobabel, fils de Salathiel, fils de Néri, 
${}^{28}fils de Melki, fils d’Addi, fils de Kosam, fils d’Elmadam, fils d’Er, 
${}^{29}fils de Jésus, fils d’Éliézer, fils de Jorim, fils de Matthate, fils de Lévi, 
${}^{30}fils de Syméon, fils de Juda, fils de Joseph, fils de Jonam, fils d’Éliakim, 
${}^{31}fils de Méléa, fils de Menna, fils de Mattatha, fils de Natham,
      fils de David, 
${}^{32}fils de Jessé, fils de Jobed, fils de Booz, fils de Sala, fils de Naassone, 
${}^{33}fils d’Aminadab, fils d’Admine, fils d’Arni, fils d’Esrom, fils de Pharès, fils de Juda, 
${}^{34}fils de Jacob, fils d’Isaac,
      fils d’Abraham, fils de Thara, fils de Nakor, 
${}^{35}fils de Sérouk, fils de Ragaou, fils de Phalek, fils d’Éber, fils de Sala, 
${}^{36}fils de Kaïnam, fils d’Arphaxad, fils de Sem, fils de Noé, fils de Lamek, 
${}^{37}fils de Mathusalem, fils de Hénok, fils de Jareth, fils de Maléléel, fils de Kaïnam, 
${}^{38}fils d’Énos, fils de Seth, fils d’Adam,
      fils de Dieu.
      
         
      \bchapter{}
      \begin{verse}
${}^{1}Jésus, rempli d’Esprit Saint, quitta les bords du Jourdain ; dans l’Esprit, il fut conduit à travers le désert 
${}^{2}où, pendant quarante jours, il fut tenté par le diable. Il ne mangea rien durant ces jours-là, et, quand ce temps fut écoulé, il eut faim. 
${}^{3}Le diable lui dit alors : « Si tu es Fils de Dieu, ordonne à cette pierre de devenir du pain. » 
${}^{4}Jésus répondit : « Il est écrit : L’homme ne vit pas seulement de pain. »
${}^{5}Alors le diable l’emmena plus haut et lui montra en un instant tous les royaumes de la terre. 
${}^{6}Il lui dit : « Je te donnerai tout ce pouvoir et la gloire de ces royaumes, car cela m’a été remis et je le donne à qui je veux. 
${}^{7}Toi donc, si tu te prosternes devant moi, tu auras tout cela. » 
${}^{8}Jésus lui répondit : « Il est écrit : C’est devant le Seigneur ton Dieu que tu te prosterneras, à lui seul tu rendras un culte. »
${}^{9}Puis le diable le conduisit à Jérusalem, il le plaça au sommet du Temple et lui dit : « Si tu es Fils de Dieu, d’ici jette-toi en bas ; 
${}^{10}car il est écrit :
        \\Il donnera pour toi, à ses anges, l’ordre de te garder ;
${}^{11}et encore :
        \\Ils te porteront sur leurs mains,
        \\de peur que ton pied ne heurte une pierre. »
${}^{12}Jésus lui fit cette réponse : « Il est dit : Tu ne mettras pas à l’épreuve le Seigneur ton Dieu. » 
${}^{13}Ayant ainsi épuisé toutes les formes de tentations, le diable s’éloigna de Jésus jusqu’au moment fixé.
