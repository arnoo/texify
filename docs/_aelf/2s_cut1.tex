  
  
    
    \bbook{DEUXIÈME LIVRE DE SAMUEL}{DEUXIÈME LIVRE DE SAMUEL}
      
         
      \bchapter{}
      \begin{verse}
${}^{1}C’était après la mort de Saül. David, après avoir battu les Amalécites, revint à Ciqlag et y demeura deux jours. 
${}^{2}Or, le troisième jour, un homme arriva du camp de Saül, les vêtements déchirés et la tête couverte de poussière. En arrivant auprès de David, il se jeta à terre et se prosterna. 
${}^{3}David lui demanda : « D’où viens-tu donc ? » Il lui répondit : « Je me suis échappé du camp d’Israël. » 
${}^{4}David lui dit : « Que s’est-il passé ? Raconte-le-moi ! » L’homme répondit : « Le peuple s’est enfui du champ de bataille ; beaucoup d’entre eux\\sont tombés et sont morts. Et même Saül et son fils Jonathan sont morts ! » 
${}^{5}David dit au jeune homme qui lui apportait la nouvelle : « Comment sais-tu que Saül et son fils Jonathan sont morts ? » 
${}^{6}Le jeune homme lui dit : « Je me trouvais, par hasard, sur le mont Gelboé. Et j’ai vu Saül appuyé sur sa lance, et les chars et les cavaliers qui le serraient de près. 
${}^{7}Il s’est retourné, il m’a vu et m’a appelé. J’ai dit : “Me voici.” 
${}^{8}Il m’a dit : “Qui es-tu ?” et je lui ai dit : “Je suis un Amalécite.” 
${}^{9}Il m’a dit : “Tiens-toi près de moi, je t’en prie, et donne-moi la mort, car je suis saisi de vertige alors que je suis encore plein de vie.” 
${}^{10}Je me suis tenu près de lui et je lui ai donné la mort car je savais qu’il n’aurait pas survécu à sa chute. Ensuite, j’ai pris le diadème qui était sur sa tête et la chaînette qu’il avait au bras. Je les ai apportés ici à mon seigneur. »
${}^{11}Alors David arracha et déchira ses vêtements, et tous les hommes qui étaient avec lui firent de même.  
${}^{12} Ils se lamentèrent, pleurèrent et jeûnèrent jusqu’au soir, à cause de Saül et de son fils Jonathan, à cause du peuple du Seigneur et de la maison d’Israël, parce qu’ils étaient tombés par l’épée.
${}^{13}David demanda au jeune homme qui lui apportait la nouvelle : « D’où es-tu ? » Il répondit : « Je suis le fils d’un immigré amalécite. » 
${}^{14}David lui dit : « Comment ! Tu n’as pas craint d’étendre la main pour supprimer le messie du Seigneur ? » 
${}^{15}David appela l’un des serviteurs et dit : « Approche, frappe-le ! » Celui-ci le frappa à mort, 
${}^{16}tandis que David lui disait : « Que ton sang retombe sur ta tête ! Car ta bouche a parlé contre toi quand tu as dit : Moi, j’ai donné la mort au messie du Seigneur. »
${}^{17}Alors David chanta cette lamentation sur Saül et sur son fils Jonathan, 
${}^{18}et il demanda qu’on l’enseigne aux fils de Juda : c’est le chant de « L’Arc ». Voici comment il est écrit dans le livre du Juste :
       
        ${}^{19}« Ta fierté, Israël, transpercée sur tes hauteurs !
        \\Comment sont-ils tombés, les héros ?
         
${}^{20}Ne l’annoncez pas dans la ville de Gath,
        ne portez pas la nouvelle dans les rues d’Ascalon,
        \\de peur que les filles des Philistins ne se réjouissent,
        et les filles des incirconcis ne bondissent de joie.
         
${}^{21}Montagnes de Gelboé, qu’il n’y ait pour vous
        ni rosée ni pluie ni champs fertiles :
        \\c’est là que fut souillé le bouclier des héros,
        le bouclier de Saül qui n’était pas frotté d’huile.
         
${}^{22}Devant le sang des transpercés
        et la blessure des héros,
        \\l’arc de Jonathan ne reculait pas,
        l’épée de Saül ne revenait pas sans effet.
         
        ${}^{23}Saül et Jonathan, aimables, pleins de charme,
        ni dans la vie ni dans la mort ne furent séparés,
        \\plus rapides que les aigles,
        plus vaillants que les lions.
         
        ${}^{24}Filles d’Israël, pleurez sur Saül :
        \\il vous revêtait de pourpre somptueuse
        et rehaussait de joyaux d’or vos vêtements.
         
        ${}^{25}Comment sont-ils tombés, les héros,
        au milieu du combat ?
        \\Jonathan, transpercé sur les\\hauteurs !
         
        ${}^{26}J’ai le cœur serré à cause de toi,
        mon frère Jonathan.
        \\Tu étais plein d’affection pour moi,
        \\et ton amitié pour moi était merveille
        plus grande que l’amour des femmes\\ !
         
        ${}^{27}Comment sont-ils tombés, les héros ?
        \\Comment ont-elles disparu, les armes du combat ? »
      <h2 class="intertitle" id="d85e69003">1. David roi de Juda (<a class="unitex_link" href="#bib_2s_2">2 S 2 – 4</a>)</h2>
      
         
      \bchapter{}
      \begin{verse}
${}^{1}Voici ce qui arriva ensuite. David consulta le Seigneur : « Dois-je monter dans l’une des villes de Juda ? » Le Seigneur lui dit : « Monte. » David demanda : « Où dois-je monter ? » Et le Seigneur dit : « À Hébron. » 
${}^{2}David y monta, ainsi que ses deux femmes, Ahinoam de Yizréel, et Abigaïl, femme de Nabal de Carmel. 
${}^{3}Quant aux hommes qui étaient avec lui, David les fit aussi monter, chacun avec sa famille ; et ils habitèrent dans les villes qui dépendaient d’Hébron. 
${}^{4}Alors les hommes de Juda vinrent à Hébron où ils donnèrent l’onction à David, comme roi sur la maison de Juda. On fit savoir à David : « Les hommes de Yabesh-de-Galaad ont enseveli Saül. » 
${}^{5}David envoya des messagers aux hommes de Yabesh-de-Galaad pour leur dire : « Soyez bénis du Seigneur, vous qui avez agi avec fidélité envers Saül, votre maître, et l’avez enseveli. 
${}^{6}Que le Seigneur agisse maintenant envers vous avec fidélité et loyauté. Moi aussi, j’agirai envers vous avec la même bonté, puisque vous avez fait cela. 
${}^{7}Et maintenant, reprenez courage, soyez vaillants, car votre maître Saül est mort, mais c’est à moi que la maison de Juda a donné l’onction pour que je règne sur elle. »
      
         
${}^{8}Abner, fils de Ner, chef de l’armée de Saül, avait pris Ishbosheth, fils de Saül, et l’avait fait passer à Mahanaïm. 
${}^{9}Il le fit roi sur le pays de Galaad, les Ashourites et Yizréel, ainsi que sur Éphraïm, Benjamin et Israël tout entier. 
${}^{10}Ishbosheth, fils de Saül, avait quarante ans lorsqu’il devint roi sur Israël, et il régna deux ans. Mais la maison de Juda s’était ralliée à David. 
${}^{11}Le temps que David passa à Hébron, comme roi de la maison de Juda, fut de sept ans et six mois.
${}^{12}Abner, fils de Ner, et les serviteurs d’Ishbosheth, fils de Saül, sortirent de Mahanaïm en direction de Gabaon. 
${}^{13}Joab, fils de Cerouya, et les serviteurs de David sortirent d’Hébron. Ils se rencontrèrent au réservoir de Gabaon. Ils s’assirent de part et d’autre du réservoir. 
${}^{14}Abner dit à Joab : « Que les jeunes gens se lèvent donc et qu’ils s’affrontent en une joute devant nous. » Joab dit : « Qu’ils se lèvent ! » 
${}^{15}Ils se levèrent ; on en compta douze pour Benjamin et Ishbosheth, fils de Saül, et douze parmi les serviteurs de David. 
${}^{16}Chacun empoigna son vis-à-vis par la tête et lui planta son épée dans le flanc, si bien qu’ils tombèrent ensemble. On appela ce lieu « le Champ des Flancs ». Il se trouve à Gabaon.
${}^{17}Le combat fut très dur ce jour-là. Abner et les hommes d’Israël furent battus devant les serviteurs de David. 
${}^{18}Il y avait là les trois fils de Cerouya : Joab, Abishaï et Asahel. Or, Asahel avait le pied aussi léger qu’une gazelle dans la campagne. 
${}^{19}Asahel se lança à la poursuite d’Abner et le talonnait sans dévier ni à droite ni à gauche. 
${}^{20}Alors Abner se retourna et dit : « Est-ce toi, Asahel ? » Il répondit : « C’est moi. » 
${}^{21}Alors Abner lui dit : « Dévie à ta droite ou à ta gauche, saisis l’un des jeunes gens et empare-toi de ses dépouilles. » Mais Asahel ne voulut pas s’écarter ni cesser de le poursuivre. 
${}^{22}Abner redit encore à Asahel : « Écarte-toi, cesse de me poursuivre ! Pourquoi faudrait-il que je t’abatte ? Comment pourrais-je alors regarder en face ton frère Joab ? » 
${}^{23}Mais Asahel refusa de s’écarter. Alors Abner le frappa au ventre, avec le talon de sa lance, et la lance ressortit par- derrière. Il tomba là et mourut sur place. Et tous ceux qui arrivaient à l’endroit où Asahel était tombé mort, s’arrêtaient.
${}^{24}Joab et Abishaï se lancèrent à la poursuite d’Abner. Le soleil se couchait quand ils arrivèrent à Guibeat-Amma, qui se trouve à l’est de Guiah, en direction du désert de Gabaon. 
${}^{25}Les fils de Benjamin se rassemblèrent derrière Abner et, ne formant qu’un seul bloc, se postèrent au sommet d’une colline. 
${}^{26}Abner interpella Joab en disant : « Est-ce que l’épée ne s’arrêtera jamais de dévorer ? Ne sais-tu pas qu’à la fin il n’y aura qu’amertume ? Qu’attends-tu pour dire aux gens d’abandonner la poursuite de leurs frères ? » 
${}^{27}Joab répondit : « Par le Dieu vivant, si tu n’avais pas parlé, c’est seulement au matin que ces gens auraient cessé de poursuivre chacun son frère ! » 
${}^{28}Joab sonna du cor et toute sa troupe fit halte. Ils cessèrent de poursuivre Israël et de combattre. 
${}^{29}Abner et ses hommes marchèrent dans la Araba, toute la nuit. Ils passèrent le Jourdain, marchèrent toute la matinée et arrivèrent à Mahanaïm. 
${}^{30}Joab, ayant abandonné la poursuite d’Abner, rassembla toute sa troupe. Parmi les serviteurs de David, il manquait à l’appel dix-neuf hommes et Asahel. 
${}^{31}Mais parmi les gens de Benjamin et ceux d’Abner, les serviteurs de David avaient frappé à mort trois cent soixante hommes. 
${}^{32}On emporta Asahel et on l’ensevelit dans le tombeau de son père à Bethléem. Joab et ses hommes marchèrent toute la nuit, et le jour se leva sur eux à Hébron.
      
         
      \bchapter{}
      \begin{verse}
${}^{1}La guerre fut longue entre la maison de Saül et la maison de David. Mais David allait se fortifiant, tandis que la maison de Saül allait s’affaiblissant.
      
         
${}^{2}Dans la ville d’Hébron, des fils naquirent à David. Son premier-né fut Amnone, d’Ahinoam de Yizréel ; 
${}^{3}le deuxième, Kiléab, d’Abigaïl, femme de Nabal de Carmel ; le troisième, Absalom, fils de Maaka, fille de Talmaï, roi de Gueshour ; 
${}^{4}le quatrième, Adonias, fils de Hagguith ; le cinquième, Shefatya, fils d’Abital ; 
${}^{5}le sixième, Yitréam, d’Égla, femme de David. Ceux-là naquirent à David, dans la ville d’Hébron.
${}^{6}Au cours de cette guerre entre la maison de Saül et la maison de David, Abner renforçait sa position dans la maison de Saül. 
${}^{7}Or, Saül avait eu une concubine nommée Rispa, fille de Ayya. Ishbosheth dit à Abner : « Pourquoi es-tu allé vers la concubine de mon père ? » 
${}^{8}À ces paroles d’Ishbosheth, Abner fut pris d’une violente colère et dit : « Suis-je une tête de chien au service de Juda ? Aujourd’hui, j’agis en toute fidélité envers la maison de ton père Saül, envers ses frères et ses amis, je t’empêche de tomber aux mains de David, et toi, tu viens me reprocher aujourd’hui une faute avec cette femme ! 
${}^{9}Que Dieu amène le malheur sur moi, Abner, et pire encore, si je ne fais pas pour David ce que le Seigneur lui a promis par serment : 
${}^{10}enlever la royauté à la maison de Saül, ériger le trône de David sur Israël et sur Juda, depuis Dane jusqu’à Bershéba ! » 
${}^{11}Ishbosheth ne put répliquer un mot à Abner, tant il avait peur de lui.
${}^{12}Abner envoya, en son propre nom, des messagers dire à David : « À qui appartient le pays ? » Et il ajouta : « Conclus donc une alliance avec moi, et je te prêterai main-forte pour que tout Israël se tourne vers toi. » 
${}^{13}David répondit : « Bien ! Je vais conclure une alliance avec toi. Seulement, je te demande une chose : tu ne seras admis en ma présence que si tu amènes d’abord Mikal, fille de Saül. À cette condition, tu pourras venir en ma présence. » 
${}^{14}Alors David envoya des messagers à Ishbosheth, fils de Saül, pour lui dire : « Donne-moi donc ma femme Mikal : je l’ai acquise pour épouse contre cent prépuces de Philistins. » 
${}^{15}Ishbosheth envoya prendre Mikal chez son mari, Paltiel fils de Laïsh. 
${}^{16}Celui-ci partit avec elle et la suivit en pleurant jusqu’à Bahourim. Abner lui dit alors : « Va-t’en, retourne ! » Et il s’en retourna.
${}^{17}Abner engagea des pourparlers avec les anciens d’Israël. Il leur dit : « Depuis longtemps déjà, vous désirez que David soit votre roi. 
${}^{18}C’est le moment d’agir ! En effet, le Seigneur a déclaré au sujet de David : “Par la main de mon serviteur David, je sauverai mon peuple Israël de la main des Philistins et de la main de tous ses ennemis”. » 
${}^{19}Abner parla aussi avec les Benjaminites ; puis il alla lui-même à Hébron parler avec David de tout ce qui paraissait bon à Israël et à toute la maison de Benjamin. 
${}^{20}Abner arriva donc chez David à Hébron ; vingt hommes l’accompagnaient. David fit un festin pour Abner et ses compagnons. 
${}^{21}Abner dit à David : « Je vais me mettre en route et rassembler tout Israël auprès de mon seigneur le roi. Ils concluront une alliance avec toi, et tu régneras partout où tu le désires. » Puis David congédia Abner qui s’en alla en paix.
${}^{22}Mais voici que Joab et les serviteurs de David revenaient d’une razzia, ramenant un énorme butin. Abner n’était plus à Hébron auprès de David, puisque celui-ci avait congédié Abner qui était parti en paix. 
${}^{23}Quand Joab et toute son armée furent arrivés, on vint annoncer à Joab : « Abner, fils de Ner, est venu chez le roi. Celui-ci l’a congédié et il est parti en paix. » 
${}^{24}Alors Joab entra chez le roi et lui dit : « Qu’as-tu fait ? Voilà donc qu’Abner est venu chez toi ! Pourquoi l’as-tu congédié et a-t-il pu repartir ? 
${}^{25}Tu connais Abner, fils de Ner : c’est pour te séduire qu’il est venu, pour observer tes allées et venues et savoir tout ce que tu fais. »
${}^{26}Joab sortit de chez David et envoya à la poursuite d’Abner des messagers qui le ramenèrent depuis la citerne de Sira ; mais David n’en savait rien. 
${}^{27}Quand Abner fut ramené à Hébron, Joab l’attira à l’intérieur de la Porte comme pour lui parler tranquillement. Là, il le frappa au ventre et le fit mourir, pour venger le sang d’Asahel, son frère. 
${}^{28}Lorsque, par la suite, David apprit cela, il déclara : « Je suis à jamais innocent devant le Seigneur, moi et mon royaume, du sang d’Abner, fils de Ner. 
${}^{29}Qu’il retombe sur la tête de Joab et sur toute la maison de son père ! Que la maison de Joab ne manque jamais d’hommes atteints de maladies purulentes ou de lèpre, ou réduits aux occupations domestiques, ou victimes de l’épée ou privés de pain ! » 
${}^{30}Si Joab et son frère Abishaï avaient assassiné Abner, c’est qu’il avait fait mourir leur frère Asahel, à la bataille de Gabaon.
${}^{31}David dit à Joab et à tout le peuple qui était avec lui : « Déchirez vos vêtements, revêtez-vous de toile à sac, prenez le deuil pour Abner ! » Et le roi David marchait derrière la litière. 
${}^{32}On ensevelit Abner à Hébron. Le roi éclata en sanglots sur la tombe d’Abner, et tout le peuple se mit à pleurer. 
${}^{33}Puis le roi entonna cette lamentation sur Abner :
        \\« Abner devait-il mourir
        comme meurt un insensé ?
${}^{34}Tes mains n’ont pas été liées,
        ni tes pieds, mis aux fers.
        \\Comme on tombe sous les coups de criminels,
        tu es tombé ! »
      Et tout le peuple se remit à pleurer sur lui.
${}^{35}Comme tout le peuple approchait pour faire prendre à David quelque nourriture, alors qu’il faisait encore jour, David fit ce serment : « Que Dieu amène le malheur sur moi, et pire encore, si je goûte au pain ou à quoi que ce soit, avant le coucher du soleil ! » 
${}^{36}Tout le peuple s’en rendit compte et le trouva bon, de même qu’il trouvait bon tout ce que faisait le roi. 
${}^{37}Ainsi, tout le peuple et tout Israël comprirent, ce jour-là, que le roi n’était pour rien dans la mort d’Abner, fils de Ner. 
${}^{38}Le roi dit à ses serviteurs : « Ne comprenez-vous pas qu’en ce jour, un prince – un grand – est tombé en Israël ? 
${}^{39}Et moi, aujourd’hui je suis faible, bien que roi par l’onction, alors que ces hommes-là, les fils de Cerouya, sont plus durs que moi. Mais que le Seigneur rende au méchant selon sa méchanceté ! »
      
         
      \bchapter{}
      \begin{verse}
${}^{1}Ishbosheth, le fils de Saül, apprit qu’Abner était mort à Hébron. Les bras lui en tombèrent, et tout Israël fut bouleversé.
${}^{2}Il y avait deux hommes – des chefs de bandes – au service du fils de Saül. L’un s’appelait Baana, l’autre Récab. Ils étaient fils de Rimmone, de Beéroth, de la tribu de Benjamin – en effet, même Beéroth est considérée comme benjaminite. 
${}^{3}Les gens de Beéroth s’étaient enfuis à Guittaïm où ils sont restés jusqu’à ce jour comme des immigrés.
${}^{4}D’autre part, Jonathan, fils de Saül, avait un fils perclus des deux pieds. Ce dernier était âgé de cinq ans lorsque parvint de Yizréel la nouvelle concernant Saül et Jonathan. Sa nourrice l’avait emporté et s’était enfuie, mais, dans la précipitation de cette fuite, l’enfant était tombé. Il resta boiteux. Il s’appelait Mefibosheth.
       
${}^{5}Donc Récab et Baana, les fils de Rimmone, de Beéroth, s’étant mis en marche, arrivèrent à l’heure la plus chaude du jour à la maison d’Ishbosheth. Celui-ci était couché pour la sieste de midi. 
${}^{6}Ils pénétrèrent à l’intérieur de la maison comme pour prendre du blé. Ils le frappèrent au ventre et s’échappèrent. Récab et son frère Baana 
${}^{7}étaient donc entrés dans la maison alors qu’Ishbosheth était couché sur son lit, dans sa chambre. Ils l’avaient frappé à mort, décapité, et ils avaient pris sa tête. Ils marchèrent toute la nuit par le chemin de la Araba 
${}^{8}et apportèrent la tête d’Ishbosheth chez David à Hébron. Ils dirent au roi : « Voici la tête d’Ishbosheth, fils de Saül, ton ennemi, qui en voulait à ta vie. Le Seigneur a accordé aujourd’hui même à mon seigneur le roi une revanche complète sur Saül et sur sa descendance. »
${}^{9}Mais David répondit à Récab et à son frère Baana, les fils de Rimmone, de Beéroth, en leur disant : « Par le Seigneur vivant qui m’a racheté de toute détresse, 
${}^{10}celui qui m’annonçait “Saül est mort !”, celui-là se prenait pour un porteur de bonne nouvelle. Et cependant, je l’ai fait saisir et tuer à Ciqlag, en guise de récompense pour sa bonne nouvelle ! 
${}^{11}À plus forte raison, si des hommes mauvais ont tué un homme juste dans sa maison et sur son lit ! Ne faut-il pas maintenant que je réclame son sang qui est sur vos mains et que je vous balaie de la terre ? » 
${}^{12}Alors David donna un ordre aux serviteurs. Ils les tuèrent et leur coupèrent les mains et les pieds, que l’on suspendit au-dessus du réservoir, à Hébron. Quant à la tête d’Ishbosheth, on la prit pour l’ensevelir dans la tombe d’Abner, à Hébron.
      <h2 class="intertitle" id="d85e69626">2. David roi de Juda et d’Israël (<a class="unitex_link" href="#bib_2s_5">2 S 5 – 8</a>)</h2>
      
         
      \bchapter{}
      \begin{verse}
${}^{1}Alors toutes les tribus d’Israël vinrent trouver David à Hébron et lui dirent : « Vois ! Nous sommes de tes os et de ta chair. 
${}^{2} Dans le passé déjà, quand Saül était notre roi, c’est toi qui menais Israël en campagne et le ramenais, et le Seigneur t’a dit : “Tu seras le berger d’Israël mon peuple, tu seras le chef d’Israël.” » 
${}^{3} Ainsi, tous les anciens d’Israël vinrent trouver le roi à Hébron. Le roi David fit alliance avec eux, à Hébron, devant le Seigneur. Ils donnèrent l’onction à David pour le faire roi sur Israël.
${}^{4}Il avait trente ans quand il devint roi, et il régna quarante ans : 
${}^{5} à Hébron il régna sur Juda pendant sept ans et demi ; et à Jérusalem il régna trente-trois ans, à la fois sur\\Israël et sur Juda.
${}^{6}Le roi avec ses hommes marcha sur Jérusalem contre les habitants de la région, les Jébuséens. On lui dit : « Tu n’entreras pas ici : des aveugles et des boiteux\\suffiraient à te repousser\\. » Autrement dit : David n’entrera pas ici. 
${}^{7}Mais David s’empara de la forteresse de Sion – c’est la Cité de David. 
${}^{8}Il dit ce jour-là : « Quiconque veut frapper les Jébuséens, qu’il passe par le souterrain pour atteindre ces boiteux et ces aveugles que David déteste de toute son âme ! » C’est pourquoi l’on dit : « Ni aveugle, ni boiteux n’entrera dans la Maison ».
${}^{9}David s’établit dans la forteresse et l’appela « Cité de David ». Ensuite, tout autour, depuis le Terre-Plein vers l’intérieur de la ville, il fit des constructions. 
${}^{10}David devint de plus en plus puissant. Le Seigneur, Dieu des armées, était avec lui. 
${}^{11}Hiram, le roi de Tyr, envoya des messagers à David, des charpentiers avec du bois de cèdre, des tailleurs de pierre pour les murs ; et ils bâtirent pour lui une maison. 
${}^{12}Alors David comprit que le Seigneur l’avait établi comme roi sur Israël et qu’il avait exalté sa royauté, à cause d’Israël son peuple.
${}^{13}David prit encore des concubines et des femmes à Jérusalem après son arrivée d’Hébron ; il eut encore des fils et des filles. 
${}^{14}Voici les noms des enfants qu’il eut à Jérusalem : Shammoua, Shobab, Nathan, Salomon, 
${}^{15}Yibhar, Élishoua, Nèfeg, Yafia, 
${}^{16}Élishama, Élyada et Élifèleth.
${}^{17}Les Philistins apprirent qu’on avait donné l’onction à David comme roi sur Israël, et ils montèrent tous à sa recherche. Mais David l’apprit et se réfugia dans la forteresse. 
${}^{18}Les Philistins arrivèrent et se déployèrent dans le Val des Refaïtes. 
${}^{19}Alors David consulta le Seigneur ; il demanda : « Dois-je monter au-devant des Philistins ? Les livreras-tu entre mes mains ? » Le Seigneur lui répondit : « Monte ! Oui, je vais livrer les Philistins entre tes mains. » 
${}^{20}Alors David partit pour Baal-Peracim, où il battit les Philistins. Et David déclara :
        \\« C’est une brèche que le Seigneur a ouverte
        \\devant moi chez l’ennemi
        \\comme une brèche ouverte par les eaux. »
      C’est pourquoi on a donné à ce lieu le nom de Baal-Peracim (c’est-à-dire : Maître des brèches). 
${}^{21}Les Philistins avaient abandonné sur place leurs idoles, et David et ses hommes les emportèrent.
${}^{22}De nouveau, les Philistins montèrent à l’attaque et se déployèrent dans le Val des Refaïtes. 
${}^{23}David consulta le Seigneur qui lui répondit : « Ne monte pas. Contourne-les par leurs arrières : tu les aborderas devant les micocouliers. 
${}^{24}Quand tu entendras un bruit de pas à la cime des micocouliers, dépêche-toi : c’est que le Seigneur sera sorti devant toi pour frapper dans le camp des Philistins ! » 
${}^{25}David agit comme le Seigneur le lui avait ordonné, et il frappa les Philistins depuis Guéba jusqu’à l’entrée de Guèzer.
      
         
      \bchapter{}
      \begin{verse}
${}^{1}David rassembla encore toute l’élite d’Israël : trente mille hommes. 
${}^{2}Puis il se mit en route ; avec tout le peuple qui l’accompagnait, il partit de Baalé-de-Juda pour en faire monter l’arche de Dieu sur laquelle est invoqué un nom : le nom du Seigneur des armées qui siège sur les Kéroubim. 
${}^{3}On chargea l’arche de Dieu sur un chariot neuf et on l’emmena depuis la maison d’Abinadab située sur la colline. Ouzza et Ahyo, les fils d’Abinadab, conduisaient le chariot 
${}^{4}avec l’arche de Dieu. Or Ahyo marchait devant l’Arche. 
${}^{5}David et toute la maison d’Israël dansaient devant le Seigneur, au son des instruments en bois de cyprès, cithares et harpes, des tambourins, des sistres et des cymbales. 
${}^{6}Comme on arrivait à l’aire de Nakone, Ouzza étendit la main vers l’arche de Dieu, et la retint car les bœufs la faisaient verser. 
${}^{7}Alors la colère du Seigneur s’enflamma contre Ouzza ; Dieu le frappa sur place pour ce comportement. Ouzza mourut là, près de l’arche de Dieu. 
${}^{8}David fut irrité de ce que le Seigneur avait ouvert une brèche parmi les siens en frappant Ouzza, et on appela ce lieu Pèrès-Ouzza (c’est-à-dire : Brèche-d’Ouzza), nom qu’il a gardé jusqu’à ce jour.
${}^{9}David eut peur du Seigneur, ce jour-là, et dit : « Comment l’arche du Seigneur pourrait-elle entrer chez moi ? » 
${}^{10}David renonça donc à transférer chez lui, dans la Cité de David, l’arche du Seigneur, mais il la dévia vers la maison d’Obed-Édom, le Guittite. 
${}^{11}L’arche du Seigneur resta pendant trois mois dans la maison d’Obed-Édom, le Guittite, et le Seigneur bénit Obed-Édom ainsi que toute sa maison.
${}^{12}On rapporta au roi David : « Le Seigneur a béni la maison d’Obed-Édom et tout ce qui lui appartient, à cause de l’arche de Dieu. » <a class="anchor verset_lettre" id="bib_2s_6_12_b"/>David partit alors et fit monter l’arche de Dieu de la maison d’Obed-Édom jusqu’à la Cité de David, au milieu des cris de joie. 
${}^{13}Quand les porteurs de l’Arche eurent avancé de six pas, il offrit en sacrifice un taureau et un veau gras. 
${}^{14}David, vêtu d’un pagne de lin, dansait devant le Seigneur, en tournoyant de toutes ses forces. 
${}^{15}David et tout le peuple d’Israël firent monter l’arche du Seigneur parmi les ovations, au son du cor. 
${}^{16}Or, comme l’arche du Seigneur entrait dans la Cité de David, Mikal, fille de Saül, se pencha par la fenêtre : elle vit le roi David qui sautait et tournoyait devant le Seigneur. Dans son cœur, elle le méprisa. 
${}^{17}Ils amenèrent donc l’arche du Seigneur et l’installèrent à sa place, au milieu de la tente que David avait dressée pour elle. Puis il offrit devant le Seigneur des holocaustes et des sacrifices de paix. 
${}^{18}Quand David eut achevé d’offrir les holocaustes\\et les sacrifices de paix, il bénit le peuple au nom du Seigneur des armées\\. 
${}^{19}Il fit une distribution à tout le peuple, à la foule entière des Israélites, hommes et femmes : pour chacun une galette de pain, un morceau de rôti et un gâteau de raisins. Ensuite tout le monde s’en retourna chacun chez soi.
${}^{20}Alors que David revenait pour bénir sa maisonnée, Mikal, fille de Saül, sortit à sa rencontre et dit : « Comme il s’est honoré aujourd’hui, le roi d’Israël ! Lui qui s’est découvert aux yeux des servantes de ses esclaves comme se découvrirait un homme de rien ! » 
${}^{21}David dit à Mikal : « Devant le Seigneur, lui qui m’a choisi de préférence à ton père et à toute sa maison pour m’instituer chef sur Israël, sur le peuple du Seigneur, oui, je danserai devant le Seigneur. 
${}^{22}Je me déshonorerai encore plus que cela, et je serai abaissé à mes propres yeux, mais auprès des servantes dont tu parles, auprès d’elles je serai honoré. » 
${}^{23}Et, jusqu’au jour de sa mort, Mikal, fille de Saül, n’eut pas d’enfant.
      
         
      \bchapter{}
      \begin{verse}
${}^{1}Le roi habitait enfin\\dans sa maison. Le Seigneur lui avait accordé la tranquillité en le délivrant\\de tous les ennemis qui l’entouraient. 
${}^{2} Le roi dit alors au prophète Nathan : « Regarde ! J’habite dans une maison de cèdre, et l’arche de Dieu habite sous un abri de toile ! » 
${}^{3} Nathan répondit au roi : « Tout ce que tu as l’intention de faire, fais-le, car le Seigneur est avec toi. » 
${}^{4} Mais, cette nuit-là, la parole du Seigneur fut adressée à Nathan : 
${}^{5} « Va dire à mon serviteur David : Ainsi parle le Seigneur : Est-ce toi qui me bâtiras une maison pour que j’y habite ? 
${}^{6} Depuis le jour où j’ai fait monter d’Égypte les fils d’Israël et jusqu’à ce jour, je n’ai jamais habité dans une maison ; j’ai été comme un voyageur, sous la tente qui était ma demeure. 
${}^{7} Pendant tout le temps où j’étais comme un voyageur parmi tous les fils d’Israël, ai-je demandé à un seul des juges\\que j’avais institués pasteurs de mon peuple Israël : “Pourquoi ne m’avez-vous pas bâti une maison de cèdre ?” 
${}^{8} Tu diras donc à mon serviteur David : Ainsi parle le Seigneur de l’univers : C’est moi qui t’ai pris au pâturage, derrière le troupeau, pour que tu sois le chef de mon peuple Israël. 
${}^{9} J’ai été avec toi partout où tu es allé, j’ai abattu devant toi tous tes ennemis. Je t’ai fait un nom aussi grand que celui des plus grands de la terre. 
${}^{10} Je fixerai en ce lieu mon peuple Israël, je l’y planterai, il s’y établira et ne tremblera plus, et les méchants ne viendront plus l’humilier, comme ils l’ont fait autrefois, 
${}^{11} depuis le jour où j’ai institué des juges pour conduire mon peuple Israël. Oui, je t’ai accordé la tranquillité en te délivrant\\de tous tes ennemis. Le Seigneur t’annonce qu’il te fera lui-même une maison. 
${}^{12} Quand tes jours seront accomplis et que tu reposeras auprès de tes pères, je te susciterai\\dans ta descendance un successeur, qui naîtra de toi\\, et je rendrai stable sa royauté. 
${}^{13} C’est lui qui bâtira une maison pour mon nom, et je rendrai stable pour toujours son trône royal\\. 
${}^{14} Moi, je serai pour lui un père ; et lui sera pour moi un fils. S’il fait le mal, je le corrigerai avec le bâton, à la manière humaine, je le frapperai comme font les hommes. 
${}^{15} Mais ma fidélité ne lui sera pas retirée, comme je l’ai retirée à Saül que j’ai écarté de devant toi. 
${}^{16} Ta maison et ta royauté subsisteront\\toujours devant moi\\, ton trône sera stable pour toujours. » 
${}^{17} Toutes ces paroles, toute cette vision, Nathan les rapporta fidèlement à David.
      
         
${}^{18}Le roi David vint s’asseoir en présence du Seigneur. Il dit : « Qui suis-je donc, Seigneur, et qu’est-ce que ma maison, pour que tu m’aies conduit jusqu’ici ? 
${}^{19}Mais cela ne te paraît pas encore suffisant, Seigneur, et tu adresses une parole à la maison de ton serviteur pour un avenir lointain. Est-ce là, Seigneur Dieu, la destinée de l’homme ? 
${}^{20}Qu’est-ce que David pourrait encore ajouter par ses paroles ? Toi, Seigneur Dieu, tu connais ton serviteur. 
${}^{21}À cause de ta parole et selon ton cœur, tu as accompli toute cette grande action pour instruire ton serviteur. 
${}^{22}Ainsi, tu es grand, Seigneur Dieu. Oui, tu es sans égal et il n’y a pas de Dieu en dehors de toi, d’après tout ce que nous avons entendu de nos oreilles. 
${}^{23}Est-il sur la terre une seule nation comme ton peuple, comme Israël ? Ce peuple, Dieu est allé le libérer pour qu’il devienne son peuple, et pour lui faire un nom. Il a accompli pour vous cette grande action. Tu as fait pour ton pays des choses redoutables, et tu l’as fait à cause de ton peuple que tu as libéré d’Égypte, de cette nation et de ses dieux. 
${}^{24}Pour toi, tu as établi à jamais ton peuple Israël, et toi, Seigneur, tu es devenu son Dieu. 
${}^{25}Maintenant donc, Seigneur Dieu, la parole que tu as dite au sujet de ton serviteur et de sa maison, tiens-la pour toujours, et agis selon ce que tu as dit. 
${}^{26}Que ton nom soit exalté pour toujours\\ ! Que l’on dise : “Le Seigneur de l’univers est le Dieu d’Israël”, et la maison de ton serviteur David sera stable en ta présence. 
${}^{27}Oui, c’est toi, Seigneur de l’univers, Dieu d’Israël, qui as fait cette révélation à ton serviteur : “Je te bâtirai une maison.” C’est pourquoi ton serviteur ose t’adresser cette prière\\ : 
${}^{28}Seigneur\\, c’est toi qui es Dieu, tes paroles sont vérité, et tu as fait cette magnifique promesse\\à ton serviteur. 
${}^{29}Daigne\\bénir la maison de ton serviteur, afin qu’elle soit pour toujours en ta présence. Car toi, Seigneur Dieu, tu as parlé, et par ta bénédiction la maison de ton serviteur sera bénie pour toujours. »
      
         
      \bchapter{}
      \begin{verse}
${}^{1}Voici ce qui arriva ensuite. David battit les Philistins, les soumit et les priva de leur hégémonie. 
${}^{2}Il battit les gens de Moab et, les ayant couchés à terre, il les mesura au cordeau : deux cordeaux à destiner à la mort et un plein cordeau à laisser en vie. Alors Moab fut asservi à David et paya tribut.
${}^{3}David battit Hadadèzer, fils de Rehob, roi de Soba, lorsqu’il s’apprêtait à rétablir son pouvoir sur l’Euphrate. 
${}^{4}David lui prit mille sept cents cavaliers et vingt mille fantassins. Il fit couper les jarrets de tous les attelages et n’en laissa qu’une centaine. 
${}^{5}Les Araméens de Damas vinrent au secours de Hadadèzer, roi de Soba, mais David battit vingt-deux mille hommes parmi les Araméens. 
${}^{6}Puis David établit des postes de garde chez les Araméens de Damas. Aram fut asservi à David et paya tribut. En tout lieu où allait David, le Seigneur lui donnait la victoire. 
${}^{7}David prit les carquois en or appartenant aux serviteurs de Hadadèzer et les emporta à Jérusalem. 
${}^{8}À Bétah et à Bérotaï, villes de Hadadèzer, le roi David prit du bronze en grande quantité.
${}^{9}Lorsque Toyi, roi de Hamath, apprit que David avait battu toute l’armée de Hadadèzer, 
${}^{10}il envoya son fils Joram auprès du roi David pour le saluer et le féliciter d’avoir fait la guerre à Hadadèzer et de l’avoir battu – Hadadèzer, en effet, était constamment en guerre avec Toyi. Joram apporta des objets en argent, en or et en bronze. 
${}^{11}Le roi David les consacra au Seigneur, comme il avait consacré l’argent et l’or venant de toutes les nations qu’il avait assujetties : 
${}^{12}Aram, Moab, les fils d’Ammone, les Philistins, Amalec – sans compter ce qui provenait du butin de Hadadèzer, fils de Rehob, roi de Soba.
${}^{13}David se fit un nom en revenant d’avoir abattu Édom et ses dix-huit mille hommes, dans la vallée du Sel. 
${}^{14}Il établit des postes de garde en Édom ; il en établit dans tout le pays. Tous les gens d’Édom furent asservis à David. En tout lieu où allait David, le Seigneur lui donnait la victoire.
${}^{15}David régna sur tout Israël, faisant droit et justice à tout son peuple. 
${}^{16}Joab, fils de Cerouya, commandait l’armée ; Josaphat, fils d’Ahiloud, était archiviste ; 
${}^{17}Sadoc, fils d’Ahitoub, et Ahimélek, fils d’Abiatar, étaient prêtres ; Seraya était secrétaire ; 
${}^{18}Benaya, fils de Joad, commandait les Kerétiens et les Pelétiens. Les fils de David étaient prêtres.
      <h2 class="intertitle" id="d85e70630">3. La famille de David et ses intrigues (<a class="unitex_link" href="#bib_2s_9">2 S 9 – 20</a>)</h2>
      
         
      \bchapter{}
      \begin{verse}
${}^{1}David demanda : « Existe-t-il encore un survivant de la maison de Saül que je puisse traiter avec fidélité en souvenir de Jonathan ? » 
${}^{2}Or, la maison de Saül avait un serviteur nommé Ciba. On le convoqua chez David, et le roi lui demanda : « Est-ce bien toi, Ciba ? » Il répondit : « C’est moi, ton serviteur. » 
${}^{3}Le roi lui dit : « N’y a-t-il plus personne de la maison de Saül que je puisse traiter avec la fidélité de Dieu ? » Ciba dit au roi : « Il y a encore un fils de Jonathan, perclus des deux pieds. » 
${}^{4}« Où est-il donc ? », lui demanda le roi. Ciba répondit au roi : « Dans la maison de Makir, fils d’Ammiel, à Lo-Debar. » 
${}^{5}Le roi David l’envoya chercher et on le ramena de la maison de Makir, fils d’Ammiel, de Lo-Debar.
${}^{6}Mefibosheth, fils de Jonathan, fils de Saül, arriva auprès de David. Tombant face contre terre, il se prosterna. David lui dit : « Mefibosheth ! » Il répondit : « Voici ton serviteur. » 
${}^{7}David lui déclara : « N’aie aucune crainte : je veux te traiter avec fidélité, en souvenir de ton père Jonathan. Je te restituerai toutes les terres de Saül, ton aïeul. Et toi, désormais, tu prendras tous tes repas à ma table. » 
${}^{8}Mefibosheth se prosterna et dit : « Qu’est-ce donc que ton serviteur, pour que tu tournes ton visage vers un chien crevé tel que moi ! » 
${}^{9}Le roi convoqua Ciba, serviteur de Saül, et lui dit : « Tout ce qui appartenait à Saül et à l’ensemble de sa maison, je le donne à Mefibosheth, le fils de ton maître. 
${}^{10}Pour lui, tu cultiveras la terre, toi, tes fils et tes serviteurs. Tu en apporteras le produit qui servira de nourriture pour la maison de ton maître, et ils en mangeront. Quant à Mefibosheth, le fils de ton maître, il prendra tous ses repas à ma table. » Or Ciba avait quinze fils et vingt serviteurs. 
${}^{11}Ciba dit au roi : « Ton serviteur agira selon tout ce que mon seigneur le roi lui a ordonné, en disant : “Mefibosheth mange à ma table comme l’un des fils du roi” ! »
${}^{12}Mefibosheth avait un jeune fils appelé Mika. Tous ceux qui habitaient la maison de Ciba étaient au service de Mefibosheth. 
${}^{13}Mefibosheth résidait à Jérusalem puisqu’il prenait tous ses repas à la table du roi. Il boitait des deux pieds.
      
         
      \bchapter{}
      \begin{verse}
${}^{1}Voici ce qui arriva ensuite. Le roi des fils d’Ammone mourut, et son fils Hanoun régna à sa place. 
${}^{2}David dit alors : « Je traiterai Hanoun, fils de Nahash, avec la fidélité que son père a montrée envers moi. » Et David, par l’intermédiaire de serviteurs, lui envoya des condoléances au sujet de son père. Les serviteurs de David arrivèrent au pays des fils d’Ammone. 
${}^{3}Mais les princes des Ammonites dirent à Hanoun, leur seigneur : « Penses-tu que c’est bien pour honorer ton père que David t’envoie des porteurs de condoléances ? N’est-ce pas plutôt en vue d’explorer la ville, pour l’espionner et la détruire, que David a envoyé ses serviteurs auprès de toi ? » 
${}^{4}Alors Hanoun se saisit des serviteurs de David, leur fit raser la moitié de la barbe et couper les vêtements à mi-hauteur jusqu’aux fesses, puis il les renvoya. 
${}^{5}On en informa David qui envoya quelqu’un à leur rencontre, car ces hommes étaient pleins de confusion. Le roi leur fit dire : « Restez à Jéricho jusqu’à ce que votre barbe ait repoussé. Ensuite, vous reviendrez. »
${}^{6}Les fils d’Ammone virent bien qu’ils s’étaient rendus odieux à David. Alors ils envoyèrent prendre à leur solde les Araméens de Beth-Rehob et ceux de Soba, soit vingt mille fantassins, avec mille hommes du roi de Maaka et douze mille hommes de Tob. 
${}^{7}À cette nouvelle, David envoya Joab avec toute l’armée des guerriers. 
${}^{8}Les fils d’Ammone firent une sortie et se rangèrent en ordre de bataille devant la porte de la ville, tandis que les Araméens de Soba et de Rehob, les hommes de Tob et de Maaka, se tenaient à l’écart, en rase campagne.
${}^{9}Lorsque Joab vit que le front de combat était à la fois devant et derrière lui, il choisit, parmi toute l’élite d’Israël, des hommes qu’il rangea face aux Araméens. 
${}^{10}Il confia le reste de la troupe à son frère Abishaï qui se rangea face aux fils d’Ammone. 
${}^{11}Joab lui dit : « Si les Araméens sont plus forts que moi, tu viendras à mon secours. Et si les fils d’Ammone sont plus forts que toi, j’irai te secourir. 
${}^{12}Sois fort, montrons-nous forts pour notre peuple, pour les villes de notre Dieu. Et le Seigneur fera ce qui est bon à ses yeux ! » 
${}^{13}Alors, Joab, avec la troupe qui l’accompagnait, s’avança pour combattre les Araméens, qui s’enfuirent devant lui. 
${}^{14}Quand les fils d’Ammone virent que les Araméens s’étaient enfuis, ils prirent la fuite devant Abishaï et rentrèrent dans la ville. Joab s’en revint de la guerre contre les fils d’Ammone et rentra à Jérusalem.
${}^{15}Les Araméens, se voyant battus par Israël, regroupèrent toutes leurs forces. 
${}^{16}Hadadèzer envoya des messagers pour faire venir les Araméens d’au-delà de l’Euphrate. Ceux-ci arrivèrent à Hélam, avec à leur tête Shobak, chef de l’armée de Hadadèzer. 
${}^{17}David en fut informé. Il rassembla tout Israël, passa le Jourdain et parvint à Hélam. Les Araméens se rangèrent en face de David et engagèrent la bataille contre lui. 
${}^{18}Mais les Araméens s’enfuirent devant Israël. David massacra parmi les Araméens sept cents attelages et quarante mille cavaliers. Il frappa Shobak, le chef de leur armée ; c’est là que celui-ci mourut. 
${}^{19}Quand tous les rois qui servaient Hadadèzer se virent battus par Israël, ils firent la paix avec Israël et passèrent à son service. Et désormais, les Araméens eurent peur de porter secours aux fils d’Ammone.
      
         
      \bchapter{}
      \begin{verse}
${}^{1}Au retour du printemps, à l’époque où les rois se mettent en campagne\\, David envoya Joab en expédition, avec ses officiers et toute l’armée\\d’Israël ; ils massacrèrent les fils d’Ammone et mirent le siège devant Rabba. David était resté à Jérusalem.
${}^{2}Un soir, il se leva de sa couche pour se promener sur la terrasse du palais\\. De là\\, il aperçut une femme en train de se baigner. Cette femme était très belle. 
${}^{3}David fit demander qui elle était, et on lui répondit : « Mais c’est Bethsabée\\, fille d’Éliam, la femme d’Ourias le Hittite ! » 
${}^{4}Alors David envoya des gens la chercher. Elle vint chez lui ; il coucha avec elle, alors qu’elle s’était purifiée de ses règles. Après quoi, elle retourna chez elle.
${}^{5}La femme devint enceinte, et elle fit savoir à David : « Je suis enceinte ! » 
${}^{6}Alors David expédia ce message à Joab : « Envoie-moi Ourias le Hittite. » Et Joab l’envoya à David. 
${}^{7}Lorsque Ourias fut arrivé auprès de lui, David lui demanda comment allaient Joab, et l’armée, et la guerre. 
${}^{8}Puis il lui dit : « Descends chez toi, prends du repos\\. » Ourias sortit du palais, et l’on portait derrière lui une portion de la table\\du roi. 
${}^{9}Mais Ourias se coucha à l’entrée du palais avec les serviteurs de son maître ; il ne descendit pas chez lui. 
${}^{10}On annonça à David : « Ourias n’est pas descendu chez lui. » David dit à Ourias : « N’arrives-tu pas de voyage ? Pourquoi n’es-tu pas descendu dans ta maison ? » 
${}^{11}Ourias dit à David : « L’Arche ainsi qu’Israël et Juda habitent sous des huttes. Joab, mon seigneur, et les serviteurs de mon seigneur le roi campent en rase campagne. Et moi, j’irais dans ma maison manger, boire et coucher avec ma femme ! Par ta vie, par ta propre vie, je ne ferai pas une chose pareille ! » 
${}^{12}David dit à Ourias : « Reste ici aujourd’hui encore, et demain je te renverrai. » Ourias resta donc à Jérusalem ce jour-là et le lendemain. 
${}^{13}David l’invita à manger et à boire à sa table, et il l’enivra. Le soir, Ourias sortit et alla se coucher à nouveau\\avec les serviteurs de son maître ; mais il ne descendit pas chez lui. 
${}^{14}Le matin suivant, David écrivit une lettre pour Joab, et la fit porter par Ourias. 
${}^{15}Il disait dans cette lettre : « Mettez Ourias en première ligne, au plus fort de la mêlée, puis repliez-vous derrière lui ; qu’il soit frappé et qu’il meure ! »
${}^{16}Joab, qui assiégeait la ville, plaça Ourias à un endroit où il savait que les ennemis\\étaient en force. 
${}^{17}Les assiégés\\firent une sortie contre Joab. Il y eut des tués dans l’armée, parmi les serviteurs de David, et Ourias le Hittite mourut aussi. 
${}^{18}Joab envoya raconter à David tous les détails du combat. 
${}^{19}Il donna cet ordre au messager : « Quand tu en auras fini de rapporter au roi tous les détails du combat, 
${}^{20}il se peut que le roi entre en fureur et te dise : “Pourquoi vous êtes-vous tant approchés de la ville en livrant bataille ? Ne saviez-vous pas qu’on tire du haut du rempart ? 
${}^{21}Qui donc autrefois a frappé Abimélek, fils de Yeroubbèsheth ? N’est-ce pas une femme qui, du haut d’un rempart, a jeté une meule sur lui ? Et il en est mort, à Tébès. Pourquoi vous êtes-vous approchés du rempart ?” Alors, tu lui diras : “Ton serviteur Ourias le Hittite est mort, lui aussi !” » 
${}^{22}Le messager partit et vint raconter à David tout ce pour quoi Joab l’avait envoyé. 
${}^{23}Le messager dit à David : « Ces gens-là étaient plus forts que nous. Ils ont fait une sortie contre nous, en rase campagne. Mais nous les avons poursuivis jusqu’à l’entrée de la porte. 
${}^{24}C’est là que les tireurs ont tiré sur tes serviteurs, du haut du rempart, et qu’il y a eu des morts parmi les serviteurs du roi. Ton serviteur Ourias le Hittite est mort, lui aussi. » 
${}^{25}David répondit au messager : « Voici ce que tu diras à Joab : “Ne prends pas en mal cette affaire : l’épée dévore tantôt ici, tantôt là. Renforce ton assaut contre la ville et renverse-la !” C’est ainsi que tu réconforteras Joab. » 
${}^{26}La femme d’Ourias, apprenant que son mari était mort, le pleura. 
${}^{27}Le deuil passé, David l’envoya chercher pour la recueillir chez lui. Elle devint sa femme et lui donna un fils. Mais ce que David venait de faire était mal aux yeux du Seigneur.
      
         
      \bchapter{}
      \begin{verse}
${}^{1}Le Seigneur envoya vers David le prophète\\Nathan qui alla le trouver et lui dit :
        \\« Dans une même ville, il y avait deux hommes ;
        l’un était riche, l’autre était pauvre.
        ${}^{2}Le riche avait des moutons et des bœufs
        en très grand nombre.
        ${}^{3}Le pauvre n’avait rien qu’une brebis,
        une toute petite, qu’il avait achetée.
        \\Il la nourrissait, et elle grandissait
        chez lui au milieu de ses fils ;
        \\elle mangeait de son pain\\, buvait de sa coupe,
        elle dormait dans ses bras : elle était comme sa fille.
        ${}^{4}Un voyageur arriva chez l’homme riche.
        \\Pour préparer le repas de son hôte,
        celui-ci épargna ses moutons et ses bœufs.
        \\Il alla prendre la brebis du pauvre,
        et la prépara pour l’homme qui était arrivé chez lui. »
       
${}^{5}Alors, David s’enflamma d’une grande colère contre cet homme, et dit à Nathan : « Par le Seigneur vivant, l’homme qui a fait cela mérite la mort\\ ! 
${}^{6} Et il remboursera la brebis au quadruple, pour avoir commis une telle action et n’avoir pas épargné le pauvre\\. » 
${}^{7} Alors Nathan dit à David\\ : « Cet homme, c’est toi ! Ainsi parle le Seigneur Dieu d’Israël : Je t’ai consacré comme roi d’Israël, je t’ai délivré de la main de Saül, 
${}^{8} puis je t’ai donné la maison de ton maître, j’ai mis dans tes bras les femmes de ton maître ; je t’ai donné la maison d’Israël et de Juda et, si ce n’est pas assez, j’ajouterai encore autant. 
${}^{9} Pourquoi donc as-tu méprisé le Seigneur en faisant ce qui est mal à ses yeux ? Tu as frappé par l’épée Ourias le Hittite ; sa femme, tu l’as prise pour femme ; lui, tu l’as fait périr par l’épée des fils d’Ammone. 
${}^{10} Désormais, l’épée ne s’écartera plus jamais de ta maison, parce que tu m’as méprisé et que tu as pris la femme d’Ourias le Hittite pour qu’elle devienne ta femme. 
${}^{11} Ainsi parle le Seigneur : De ta propre maison, je ferai surgir contre toi le malheur. Je t’enlèverai tes femmes sous tes yeux et je les donnerai à l’un de tes proches, qui les prendra\\sous les yeux du soleil. 
${}^{12} Toi, tu as agi en cachette, mais moi, j’agirai à la face de tout Israël, et à la face du soleil ! »
${}^{13}David dit à Nathan : « J’ai péché contre le Seigneur\\ ! » Nathan lui répondit : « Le Seigneur a passé\\sur ton péché, tu ne mourras pas. 
${}^{14} Cependant, parce que tu as bafoué le Seigneur\\, le fils que tu viens d’avoir mourra. » 
${}^{15} Et Nathan retourna chez lui.
      Le Seigneur frappa l’enfant que la femme d’Ourias avait donné à David, et il tomba gravement malade. 
${}^{16}David implora Dieu pour le petit enfant : il jeûna strictement, et, quand il rentrait chez lui, il passait la nuit couché par terre. 
${}^{17}Les anciens de sa maison insistaient auprès de lui pour qu’il se relève, mais il refusa, et ne prit avec eux aucune nourriture. 
${}^{18}Le septième jour, l’enfant mourut. Les serviteurs de David n’osaient pas lui annoncer que l’enfant était mort. Ils se disaient en effet : « Lorsque l’enfant était vivant, nous lui avons parlé, et il ne nous a pas écoutés. Maintenant, comment lui dire que l’enfant est mort ? Il ferait un malheur ! » 
${}^{19}Voyant ses serviteurs chuchoter entre eux, David comprit que l’enfant était mort. Il demanda aux serviteurs : « L’enfant est-il mort ? » Ils répondirent : « Il est mort. »
${}^{20}Alors David se releva de terre, se baigna, se parfuma et changea de vêtement. Il entra dans la maison du Seigneur et se prosterna. Puis il rentra chez lui ; il demanda qu’on lui serve de la nourriture et il mangea. 
${}^{21}Ses serviteurs lui dirent : « Mais que fais-tu ? Pour l’enfant, quand il était en vie, tu as jeûné et pleuré, et maintenant qu’il est mort, tu te relèves et tu prends de la nourriture ! » 
${}^{22}Il répondit : « Tant que l’enfant était encore en vie, j’ai jeûné et j’ai prié en me disant : Qui sait ? Le Seigneur aura peut-être pitié de moi, et l’enfant vivra ! 
${}^{23}Mais maintenant qu’il est mort, à quoi bon jeûner ? Pourrais-je encore le faire revenir ? C’est moi qui m’en irai le rejoindre, mais lui ne reviendra pas vers moi. »
       
${}^{24}David consola Bethsabée sa femme : il la retrouva et coucha avec elle. Elle lui donna un fils qu’il nomma Salomon. Le Seigneur l’aima, 
${}^{25}et il le fit savoir par le prophète Nathan qui lui donna, à cause du Seigneur, le nom de Yedidya : Aimé-du-Seigneur.
${}^{26}Joab attaqua la ville de Rabba, celle des fils d’Ammone. Comme il allait s’emparer de la ville royale, 
${}^{27}il envoya des messagers à David pour lui dire : « J’ai attaqué Rabba et j’ai même pris la ville basse. 
${}^{28}Rassemble maintenant le reste des troupes et viens dresser ton camp devant la ville pour t’en emparer ; sinon c’est moi qui le ferai, et la ville porterait mon nom ! » 
${}^{29}Ayant donc rassemblé toutes les troupes, David marcha sur Rabba, lui donna l’assaut et s’en empara. 
${}^{30}David enleva la couronne qui était sur la tête de leur roi ; elle pesait un talent d’or et était ornée d’une pierre précieuse. On la mit sur la tête de David. Puis celui-ci emporta de la ville une très grande quantité de butin. 
${}^{31}Quant à sa population, il l’emmena, lui imposa le travail de la scie, des pics de fer et des haches, et l’affecta aux fours à briques. Il traita ainsi toutes les villes des fils d’Ammone. Puis David et toutes ses troupes revinrent à Jérusalem.
      
         
      \bchapter{}
      \begin{verse}
${}^{1}Voici ce qui arriva ensuite. Absalom, fils de David, avait une sœur d’une grande beauté, nommée Tamar. Et Amnone, fils de David, en devint amoureux. 
${}^{2}Il était tourmenté, au point de se rendre malade à cause de sa sœur Tamar. Celle-ci était vierge, et pour Amnone, lui faire quoi que ce soit semblait impossible. 
${}^{3}Mais Amnone avait un ami nommé Yonadab, fils de Shiméa, un frère de David. Yonadab était un homme très astucieux. 
${}^{4}Il dit à Amnone : « Pourquoi donc, fils du roi, es-tu si déprimé, matin après matin ? Ne veux-tu pas me l’expliquer ? » Et Amnone lui répondit : « C’est Tamar, la sœur de mon frère Absalom. J’en suis amoureux ! » 
${}^{5}Yonadab lui dit alors : « Couche-toi sur ton lit et fais le malade. Quand ton père viendra te voir, tu lui diras : “Permets que ma sœur Tamar vienne me faire prendre quelque nourriture. Qu’elle prépare cette nourriture sous mes yeux pour que je la voie, et je la mangerai de sa main !” » 
${}^{6}Amnone se coucha et fit le malade. Le roi vint le voir et Amnone lui dit : « Permets que ma sœur Tamar vienne faire sous mes yeux deux galettes, et je prendrai cette nourriture de sa main. » 
${}^{7}David envoya dire à Tamar, chez elle : « Va donc chez ton frère Amnone et prépare-lui sa nourriture. » 
${}^{8}Tamar se rendit chez son frère Amnone qu’elle trouva couché. Elle prit la pâte et la pétrit, fit les galettes sous ses yeux et les mit à cuire. 
${}^{9}Quand elle prit la poële et en fit glisser les galettes devant lui, il refusa de manger. Amnone dit alors : « Faites sortir tout le monde de chez moi ! » Et tout le monde sortit de chez lui. 
${}^{10}Amnone dit à Tamar : « Apporte la nourriture dans la chambre ; que je la prenne de ta main. » Tamar prit les galettes qu’elle avait préparées et les apporta à son frère Amnone, dans la chambre. 
${}^{11}Comme elle s’approchait de lui pour qu’il mange, il la saisit en disant : « Viens, ma sœur, couche avec moi ! » 
${}^{12}Elle lui dit : « Non ! mon frère, ne me fais pas violence : on n’agit pas ainsi en Israël. Ne commets pas cette infamie ! 
${}^{13}Et moi, où irais-je porter mon déshonneur ? Et toi, tu serais considéré comme un infâme en Israël ! Va plutôt parler au roi : il n’empêchera pas que je t’appartienne. » 
${}^{14}Mais lui ne voulut pas l’écouter. Il la maîtrisa, lui fit violence et coucha avec elle. 
${}^{15}Après quoi, Amnone se mit à la haïr d’une très grande haine. Oui, la haine qu’il lui porta fut plus grande que l’amour dont il l’avait aimée. Amnone lui dit alors : « Lève-toi, va-t’en ! » 
${}^{16}Elle lui dit : « Non ! Car me renvoyer serait un autre mal, plus grand que celui que tu viens de me faire ! » Mais il ne voulut pas l’écouter. 
${}^{17}Il appela le garçon qui le servait et lui dit : « Qu’on la jette hors de chez moi, celle-là ! Et verrouille la porte derrière elle ! » 
${}^{18}Elle avait une tunique de grand prix ; c’était le vêtement que portaient les filles de roi quand elles étaient vierges. Le serviteur la mit dehors et verrouilla la porte derrière elle. 
${}^{19}Alors Tamar répandit de la cendre sur sa tête et déchira la tunique à longues manches qu’elle portait. Elle mit la main sur sa tête et s’en alla en criant. 
${}^{20}Son frère Absalom lui dit : « Ton frère Amnone a donc été avec toi ? Maintenant, ma sœur, calme-toi ! C’est ton frère. Ne prends pas trop à cœur cette affaire. » Et Tamar, abandonnée, habita dans la maison de son frère Absalom. 
${}^{21}Le roi David apprit tous ces événements et il en fut très irrité. 
${}^{22}Quant à Absalom, il n’adressait plus la parole à son frère Amnone, ni en mal ni en bien, car il s’était mis à le haïr en raison de la violence qu’Amnone avait faite à sa sœur Tamar.
      
         
${}^{23}Deux ans plus tard, lorsque les tondeurs de moutons étaient à Baal-Haçor, près d’Éphraïm, chez Absalom, celui-ci invita tous les fils du roi. 
${}^{24}Absalom alla trouver le roi et lui dit : « Les tondeurs sont arrivés chez ton serviteur : que le roi daigne donc venir chez moi avec tous ses serviteurs ! » 
${}^{25}Le roi répondit à Absalom : « Non, mon fils, nous n’irons pas tous. Nous ne voulons pas être à ta charge ! » Comme l’autre insistait, le roi refusa de venir, tout en lui donnant sa bénédiction. 
${}^{26}Absalom reprit : « Soit ! Permets au moins que mon frère Amnone vienne avec nous. » Le roi lui dit : « Et pourquoi irait-il avec toi ? » 
${}^{27}Mais devant l’insistance d’Absalom, il laissa partir avec lui Amnone et tous ses autres fils. 
${}^{28}Absalom donna cet ordre à ses serviteurs : « Faites bien attention ! Dès qu’Amnone aura le cœur en joie sous l’effet du vin, et que je vous dirai : “Frappez Amnone !”, vous le ferez mourir. N’ayez pas peur : n’est-ce pas moi qui vous donne cet ordre ? Soyez forts, soyez vaillants ! » 
${}^{29}Les serviteurs traitèrent Amnone comme Absalom l’avait ordonné. Alors tous les fils du roi se levèrent et, montant chacun sur sa mule, ils s’enfuirent.
${}^{30}Ils étaient encore en chemin quand la rumeur parvint à David qu’Absalom avait abattu tous les fils du roi et qu’il n’en restait pas un seul. 
${}^{31}Le roi se leva, déchira ses vêtements et se coucha par terre. Tous ses serviteurs se tenaient debout, les vêtements déchirés. 
${}^{32}Mais Yonadab, fils de Shiméa un frère de David, intervint en disant : « Que mon seigneur ne se dise pas qu’on a fait mourir tous les jeunes gens, les fils du roi, car seul Amnone est mort. La bouche d’Absalom l’avait décrété depuis le jour où Amnone avait fait violence à sa sœur Tamar. 
${}^{33}Maintenant, que mon seigneur le roi ne s’imagine donc pas que tous les fils du roi sont morts. Non, seul Amnone est mort. »
${}^{34}Et Absalom prit la fuite.
      Un jeune homme, le guetteur, leva les yeux et vit derrière lui une troupe nombreuse qui débouchait au flanc de la montagne. 
${}^{35}Alors Yonadab dit au roi : « Voici les fils du roi qui arrivent. Tout s’est donc passé comme ton serviteur l’avait dit. » 
${}^{36}À peine eut-il achevé ces paroles que les fils du roi étaient arrivés, criant et pleurant. Alors le roi et tous ses serviteurs se mirent aussi à pleurer et fondirent en larmes. 
${}^{37}Absalom avait pris la fuite. Il était allé chez Talmaï, fils d’Ammihour, roi de Gueshour. Et pendant tout ce temps, David fut en deuil de son fils.
${}^{38}Absalom avait donc pris la fuite et il était allé à Gueshour où il resta trois ans. 
${}^{39}L’esprit du roi cessa de s’emporter contre Absalom car il s’était consolé de la mort d’Amnone.
      
         
      \bchapter{}
      \begin{verse}
${}^{1}Joab, fils de Cerouya, sachant que le cœur du roi se préoccupait d’Absalom, 
${}^{2}envoya chercher à Teqoa une femme avisée et lui dit : « Mets-toi en deuil, je te prie ! Revêts des habits de deuil, ne te parfume pas, comporte-toi comme une femme depuis longtemps en deuil d’un mort. 
${}^{3}Puis, va trouver le roi et parle-lui comme je vais te le dire. » Et Joab lui indiqua les paroles qu’elle devait dire.
${}^{4}La femme de Teqoa s’adressa donc au roi. Se jetant face contre terre, elle se prosterna et dit : « Que le roi vienne à mon secours ! » 
${}^{5}Le roi lui demanda : « Que veux-tu ? » Elle répondit : « Hélas, je suis veuve, mon mari est mort. 
${}^{6}Ta servante avait deux fils. Ils se sont querellés tous deux dans les champs, où il n’y avait personne pour s’interposer. L’un a frappé l’autre d’un coup mortel. 
${}^{7}Et voilà que tout le clan s’est dressé contre ta servante en disant : “Livre-nous celui qui a frappé son frère ! Nous le mettrons à mort pour la vie de son frère qu’il a assassiné et nous ferons, par le fait même, disparaître l’héritier !” Ils vont éteindre la seule braise qui reste à mon foyer, de sorte que mon mari n’ait plus de nom ni de postérité sur la terre. » 
${}^{8}Le roi dit à la femme : « Rentre chez toi. Je vais donner des ordres à ton sujet. » 
${}^{9}La femme de Teqoa dit au roi : « Mon seigneur le roi, si l’on épargne le meurtrier, cette faute sera sur moi et sur la maison de mon père ! Mais le roi et son trône en seront exempts. » 
${}^{10}Le roi lui dit : « Celui qui te parle de supprimer le meurtrier, tu me l’amèneras. Il cessera de s’en prendre à toi. » 
${}^{11}Elle poursuivit : « Que le roi daigne donc en appeler au Seigneur son Dieu, de peur que le vengeur du sang n’augmente les ravages et qu’on ne fasse disparaître mon fils. » Il répondit : « Par le Seigneur vivant, pas un seul cheveu de ton fils ne tombera à terre. » 
${}^{12}La femme reprit : « Que ta servante puisse encore dire un mot à mon seigneur le roi. » Il lui dit : « Parle ! » 
${}^{13}Alors la femme poursuivit : « Pourquoi nourris-tu un tel projet à l’encontre du peuple de Dieu ? Avec le serment qu’il vient de prononcer, le roi se déclare lui-même coupable en ne laissant pas revenir celui qu’il a banni. 
${}^{14}À coup sûr, nous mourrons, et nous sommes comme l’eau qui s’écoule sur la terre sans être recueillie. Mais Dieu n’enlève pas la vie, et il forme des projets pour que le banni ne soit plus banni loin de sa présence. 
${}^{15}Maintenant, je suis venue dire cette parole à mon seigneur le roi, car les gens m’ont fait peur. Alors, je me suis dit : Je vais parler au roi ; peut-être fera-t-il ce que lui dit sa servante. 
${}^{16}Puisque le roi accepte d’arracher sa servante à la poigne de l’homme qui veut nous faire disparaître de l’héritage de Dieu, moi et mon fils, 
${}^{17}alors je me suis dit : Que la parole de mon seigneur le roi apporte donc l’apaisement ! En effet, mon seigneur le roi est comme l’ange de Dieu qui distingue le bien et le mal. Que le Seigneur ton Dieu soit avec toi ! »
${}^{18}Le roi reprit la parole et dit à la femme : « Ne me cache rien, je t’en prie, quand tu répondras à la question que je vais te poser. » La femme répondit : « Que mon seigneur le roi daigne parler. » 
${}^{19}Le roi demanda : « Ne serait-ce pas la main de Joab qui t’a guidée dans toute cette affaire ? » La femme reprit la parole et dit : « Par ta vie, mon seigneur le roi, que nul ne s’écarte à droite ou à gauche de tout ce qu’a dit mon seigneur le roi ! Oui, c’est bien ton serviteur Joab qui m’a donné l’ordre et qui a mis toutes ces paroles dans la bouche de ta servante. 
${}^{20}Il a fait cela afin de retourner la situation. Mais mon seigneur le roi est d’une sagesse pareille à celle de l’ange de Dieu : il sait tout ce qui se passe dans le pays. »
${}^{21}David, alors, dit à Joab : « Soit ! Je fais comme il a été dit. Va et ramène le jeune Absalom. » 
${}^{22}Joab, se jetant face contre terre, se prosterna et bénit le roi. Puis il dit : « Mon seigneur le roi, aujourd’hui ton serviteur a trouvé grâce à tes yeux, puisque le roi accomplit la parole de son serviteur. » 
${}^{23}Joab se releva, partit pour Gueshour et ramena Absalom à Jérusalem. 
${}^{24}Le roi déclara : « Qu’il retourne chez lui, mais qu’il ne paraisse pas en ma présence ! » Absalom retourna donc chez lui, mais il ne parut pas en présence du roi.
${}^{25}Il n’y avait pas dans tout Israël un homme aussi beau qu’Absalom, ni plus admiré que lui. De la plante des pieds au sommet de la tête, on ne trouvait en lui aucun défaut. 
${}^{26}Quand il se faisait raser la tête – à la fin de chaque année, sa chevelure devenant trop lourde, il devait la faire raser –, il pesait sa chevelure. Son poids atteignait deux cents sicles, selon la mesure royale ! 
${}^{27}Il naquit à Absalom trois fils et une fille nommée Tamar qui devint une femme très belle.
${}^{28}Absalom resta deux ans à Jérusalem sans paraître en présence du roi. 
${}^{29}Il envoya chercher Joab pour le faire intervenir auprès du roi, mais Joab refusa de venir chez lui. Absalom l’envoya chercher une seconde fois, mais Joab refusa encore de venir. 
${}^{30}Alors Absalom dit à ses serviteurs : « Vous voyez le champ de Joab, à côté du mien, là où il a son orge : allez-y, incendiez-le ! » Les serviteurs d’Absalom incendièrent donc le champ. 
${}^{31}Alors Joab se décida à venir chez Absalom et lui demanda : « Pourquoi tes serviteurs ont-ils incendié le champ qui m’appartient ? » 
${}^{32}Absalom répondit à Joab : « Eh bien, je t’avais fait dire : Viens ici, afin que je t’envoie, toi Joab, chez le roi et que tu lui dises de ma part : “Pourquoi fallait-il que je revienne de Gueshour ? Il vaudrait mieux pour moi que j’y sois encore. Maintenant, je veux paraître en présence du roi, et, s’il y a en moi une faute, il pourra me mettre à mort !” » 
${}^{33}Alors Joab se rendit auprès du roi et lui rapporta les paroles d’Absalom. Le roi fit appeler Absalom qui vint auprès de lui et se prosterna devant lui face contre terre. Et le roi embrassa Absalom.
      
         
      \bchapter{}
      \begin{verse}
${}^{1}Voici ce qui arriva par la suite. Absalom se fit préparer un char, des chevaux, ainsi que cinquante hommes pour courir devant lui. 
${}^{2}Tôt levé, il se postait au bord du chemin qui mène à la porte de la ville. Chaque fois qu’un homme ayant un procès se rendait auprès du roi pour obtenir un jugement, Absalom l’interpellait en disant : « De quelle ville es-tu ? » Et l’autre répondait : « Ton serviteur est de telle tribu d’Israël. » 
${}^{3}Alors Absalom lui disait : « Vois ! Ta cause est bonne et légitime, mais il n’y aura personne de chez le roi pour t’écouter. » 
${}^{4}Il disait encore : « Ah, si l’on m’établissait juge sur le pays ! C’est à moi que viendraient tous ceux qui ont un procès en attente de jugement, et je leur rendrais justice ! » 
${}^{5}Et si l’homme s’approchait pour se prosterner devant lui, il lui tendait la main, le saisissait et l’embrassait. 
${}^{6}Absalom agissait de la sorte envers tous ceux qui, en Israël, venaient auprès du roi pour obtenir un jugement. C’est ainsi qu’il ravissait le cœur des gens d’Israël.
${}^{7}Au bout de quatre ans, Absalom dit au roi : « Permets que j’aille acquitter à Hébron un vœu que j’ai fait au Seigneur. 
${}^{8}Oui, ton serviteur, pendant son séjour en Aram, à Gueshour, a fait un vœu en disant : “Si jamais le Seigneur me laisse revenir à Jérusalem, je lui rendrai un culte.” » 
${}^{9}Le roi lui dit : « Va en paix. » Absalom se mit donc en route et partit à Hébron.
${}^{10}Puis il envoya des espions dans toutes les tribus d’Israël avec cette consigne : « Dès que vous entendrez le son du cor, vous direz : “Absalom est devenu roi à Hébron !” » 
${}^{11}Avec Absalom étaient partis de Jérusalem deux cents hommes, des invités, venus en toute innocence et qui ne savaient rien de l’affaire. 
${}^{12}Or, pendant qu’il offrait les sacrifices, Absalom envoya chercher Ahitofel le Guilonite, conseiller de David, dans sa ville de Guilo. La conjuration devint puissante, et la foule de ceux qui se ralliaient à Absalom, de plus en plus nombreuse.
${}^{13}Un messager vint annoncer à David : « Le cœur des hommes d’Israël a pris parti pour Absalom. » 
${}^{14}Alors David dit à tous ses serviteurs, qui étaient avec lui à Jérusalem : « Debout, fuyons ! Autrement nous n’échapperons pas à Absalom. Vite, partez ! Sans quoi, il nous gagnera de vitesse, il nous précipitera dans le malheur et passera la ville au fil de l’épée. » 
${}^{15}Les serviteurs du roi lui dirent : « Quoi que tu choisisses, mon seigneur le roi, nous sommes tes serviteurs. » 
${}^{16}Le roi sortit, avec toute sa famille sur ses pas, laissant dix femmes, des concubines, pour garder la maison. 
${}^{17}Le roi sortit, avec tout le peuple sur ses pas, et l’on fit halte à la dernière maison. 
${}^{18}Tous ceux qui le servaient marchaient à ses côtés ; tous les Kerétiens et les Pelétiens ainsi que les Guittites, six cents hommes qui l’avaient suivi depuis Gath, passaient devant le roi. 
${}^{19}Le roi dit alors à Ittaï le Guittite : « Pourquoi viens-tu, toi aussi, avec nous ? Retourne et reste avec ce roi, puisque tu es un étranger et même un exilé, où que tu sois. 
${}^{20}Tu es arrivé hier, et aujourd’hui je t’obligerais à errer avec nous, alors que moi-même, je ne sais pas où je vais ! Retourne, et remmène tes frères avec toi. Fidélité et loyauté ! » 
${}^{21}Mais Ittaï répondit au roi : « Par la vie du Seigneur et par la vie de mon seigneur le roi, en tout lieu où tu seras, là aussi sera ton serviteur – à la mort, à la vie. » 
${}^{22}Alors David lui dit : « Va, passe. » Ittaï le Guittite passa donc avec ses hommes et toutes leurs familles. 
${}^{23}Tout le monde pleurait à grands sanglots, tandis que tout le peuple passait. Le roi traversa le torrent du Cédron, et tout le peuple passa en face du chemin qui longe le désert.
${}^{24}Voici que Sadoc, lui aussi, était là, accompagné de tous les Lévites portant l’arche de l’Alliance de Dieu. Ils déposèrent l’arche de Dieu, alors qu’Abiatar offrait des holocaustes, jusqu’à ce que tout le peuple qui sortait de la ville ait fini de passer. 
${}^{25}Le roi dit à Sadoc : « Ramène l’arche de Dieu dans la ville. Si je trouve grâce aux yeux du Seigneur, il me ramènera et me permettra de la revoir, ainsi que son domaine. 
${}^{26}Mais s’il dit : “Tu n’as plus ma faveur”, alors me voici : qu’il me traite comme bon lui semblera ! » 
${}^{27}Le roi dit encore au prêtre Sadoc : « Tu vois ce qu’il en est ? Retourne en paix à la ville avec ton fils Ahimaas et avec Jonathan, le fils d’Abiatar ; vos deux fils seront avec vous. 
${}^{28}Voyez, je vais m’attarder dans les passes du désert jusqu’à ce qu’un mot de votre part m’apporte des nouvelles. » 
${}^{29}Sadoc et Abiatar ramenèrent l’arche de Dieu à Jérusalem où ils restèrent.
${}^{30}David montait par la montée des Oliviers\\ ; il montait en pleurant, la tête voilée ; il marchait pieds nus. Tous ceux qui l’accompagnaient avaient la tête voilée ; et ils montaient en pleurant. 
${}^{31}Comme on avait annoncé à David qu’Ahitofel était parmi les conjurés avec Absalom, il dit : « Je te prie, Seigneur, frappe de folie les conseils d’Ahitofel ! » 
${}^{32}David arrivait au sommet, là où l’on se prosterne devant Dieu, lorsque Houshaï l’Arkite vint à sa rencontre, la tunique déchirée et la tête couverte de terre. 
${}^{33}David lui dit : « Si tu passes avec moi, tu me seras à charge. 
${}^{34}Mais si tu retournes en ville et que tu dises à Absalom : “Je serai ton serviteur, ô roi ; j’étais autrefois au service de ton père, mais maintenant, moi, je suis ton serviteur”, alors tu pourras faire échouer à mon profit les conseils d’Ahitofel. 
${}^{35}N’y aura-t-il pas là-bas, avec toi, les prêtres Sadoc et Abiatar ? Tu pourras les avertir de toute parole entendue chez le roi. 
${}^{36}Là-bas, il y aura aussi avec eux leurs deux fils : Ahimaas pour Sadoc, et Jonathan pour Abiatar. Vous me transmettrez par leur intermédiaire tout ce que vous apprendrez. » 
${}^{37}Houshaï, l’ami de David, arriva dans la ville comme Absalom arrivait lui-même à Jérusalem.
      
         
      \bchapter{}
      \begin{verse}
${}^{1}À peine David avait-il passé le sommet que Ciba, le serviteur de Mefibosheth, vint à sa rencontre avec une paire d’ânes bâtés, portant deux cents pains, cent gâteaux de raisins secs, cent fruits de l’été et une outre de vin. 
${}^{2}Le roi dit à Ciba : « Que vas-tu faire de tout cela ? » Ciba répondit : « Les ânes serviront de monture à la famille du roi ; les pains et les fruits seront la nourriture des plus jeunes, et le vin, la boisson pour qui sera épuisé dans le désert. » 
${}^{3}Le roi lui demanda : « Mais où est donc Mefibosheth, le fils de ton maître ? » Ciba lui dit : « Eh bien, il est resté à Jérusalem car il s’est dit : Aujourd’hui la maison d’Israël va me rendre la royauté de mon père ! » 
${}^{4}Alors le roi lui déclara : « Tout ce qui est à Mefibosheth t’appartient. » Ciba lui dit : « Me voici prosterné devant toi. Que je trouve grâce à tes yeux, mon seigneur le roi ! »
      
         
${}^{5}Comme le roi David atteignait Bahourim, il en sortit un homme du même clan que la maison de Saül. Il s’appelait Shiméï, fils de Guéra. Tout en sortant, il proférait des malédictions. 
${}^{6}Il lançait des pierres à David et à tous les serviteurs du roi\\, tandis que la foule et les guerriers entouraient le roi à droite et à gauche. 
${}^{7}Shiméï maudissait le roi en lui criant : « Va-t’en, va-t’en, homme de sang, vaurien ! 
${}^{8}Le Seigneur a fait retomber sur toi tout le sang de la maison de Saül dont tu as usurpé la royauté ; c’est pourquoi le Seigneur a remis la royauté entre les mains de ton fils Absalom. Et te voilà dans le malheur, car tu es un homme de sang. » 
${}^{9}Abishaï, fils de Cerouya, dit au roi : « Comment ce chien crevé peut-il maudire mon seigneur le roi ? Laisse-moi passer, que je lui tranche la tête. » 
${}^{10}Mais le roi répondit : « Que me voulez-vous, fils de Cerouya ? S’il maudit, c’est peut-être parce que le Seigneur lui a ordonné de maudire David. Alors, qui donc pourrait le lui reprocher ? » 
${}^{11}David dit à Abishaï et à tous ses serviteurs : « Même celui qui est mon propre fils\\s’attaque à ma vie : à plus forte raison ce descendant de Benjamin ! Laissez-le maudire, si le Seigneur le lui a ordonné. 
${}^{12}Peut-être que le Seigneur considérera ma misère et me rendra le bonheur au lieu de sa malédiction d’aujourd’hui. » 
${}^{13}David et ses hommes continuèrent leur chemin, tandis que Shiméï avançait, tout près de lui, sur le flanc de la montagne. Il proférait ses malédictions et lançait des pierres tout près du roi, en faisant voler la poussière. 
${}^{14}Le roi et toute la troupe qui était avec lui arrivèrent, épuisés, au Jourdain où ils purent enfin reprendre haleine.
${}^{15}Absalom ainsi que toute la troupe des hommes d’Israël étaient donc arrivés à Jérusalem. Ahitofel se trouvait avec lui. 
${}^{16}Or, quand Houshaï l’Arkite, l’ami de David, arriva près d’Absalom, il lui dit : « Vive le roi ! Vive le roi ! » 
${}^{17}Absalom dit à Houshaï : « Voilà comment tu es fidèle à ton ami ! Pourquoi n’es-tu pas allé avec lui ? » 
${}^{18}Houshaï lui répondit : « Je n’y suis pas allé, car celui qui a été choisi par le Seigneur, par cette troupe et tous les hommes d’Israël, c’est bien à lui que j’appartiens et avec lui que je reste. 
${}^{19}Et d’ailleurs, qui vais-je servir ? N’es-tu pas son fils ? Comme j’ai servi ton père, ainsi ferai-je pour toi. »
${}^{20}Puis, Absalom dit à Ahitofel : « Tenez conseil entre vous. Qu’allons-nous faire ? » 
${}^{21}Ahitofel dit à Absalom : « Va vers les concubines de ton père, celles qu’il a laissées pour qu’elles gardent la maison. Tout Israël apprendra que tu t’es rendu odieux à ton père, et le pouvoir de tes partisans en sera affermi. » 
${}^{22}On installa donc pour Absalom une tente sur la terrasse et il alla vers les concubines de son père à la vue de tout Israël. 
${}^{23}En ce temps-là, le conseil donné par Ahitofel avait même valeur qu’une consultation de la parole de Dieu. Il en allait ainsi de tout conseil d’Ahitofel, aussi bien pour David que pour Absalom.
      
         
      \bchapter{}
      \begin{verse}
${}^{1}Ahitofel dit à Absalom : « Laisse-moi choisir douze mille hommes et partir cette nuit à la poursuite de David. 
${}^{2}J’arriverai alors sur lui au moment où il sera fatigué et affaibli, et je le ferai trembler. Toute sa troupe prendra la fuite et je frapperai le roi quand il sera isolé. 
${}^{3}C’est ainsi que je ramènerai à toi tout le peuple : frapper l’homme que tu recherches équivaut à te rallier tout le monde, et le peuple entier sera en paix. » 
${}^{4}Cet avis parut judicieux à Absalom et à tous les anciens d’Israël. 
${}^{5}Absalom poursuivit : « Appelle donc aussi Houshaï l’Arkite, pour que nous entendions ce qu’il a à dire, lui aussi. » 
${}^{6}Houshaï se rendit auprès d’Absalom qui lui déclara : « Voilà comment Ahitofel a parlé. Devons-nous faire ce qu’il a dit ? Sinon, à toi de parler ! » 
${}^{7}Houshaï répondit à Absalom : « Cette fois-ci, le conseil qu’a donné Ahitofel n’est pas bon. » 
${}^{8}Puis il ajouta : « Tu connais toi-même ton père et ses hommes : de vaillants guerriers. Ils sont exaspérés comme une ourse sauvage privée de ses petits. Ton père est un homme de guerre : il ne passera pas la nuit avec le peuple. 
${}^{9}Voici maintenant qu’il se cache dans quelque trou ou dans un autre endroit. S’il tombe sur les nôtres dès le début, il y aura toujours quelqu’un pour l’apprendre et dire : “C’est une défaite pour le peuple qui suit Absalom !” 
${}^{10}Et alors, même un homme de valeur au cœur de lion sentirait fondre son courage, car tout Israël sait que ton père est un vaillant guerrier, et ses compagnons, des hommes de valeur. 
${}^{11}Si j’ai un conseil à donner, le voici : que se rassemble auprès de toi tout Israël, depuis Dane jusqu’à Bershéba, en aussi grand nombre que les grains de sable au bord de la mer, et tu mèneras toi-même le combat. 
${}^{12}Nous arriverons sur lui, quel que soit le lieu où il se trouve, nous nous poserons sur lui comme la rosée tombe sur la terre, et rien ne subsistera de lui, ni d’aucun de ses hommes. 
${}^{13}S’il se retire dans une ville, tout Israël fera porter des cordes à cette ville, et nous l’entraînerons dans le torrent jusqu’à ce qu’on ne puisse plus y trouver le moindre caillou ! » 
${}^{14}Alors Absalom et tous les hommes d’Israël déclarèrent : « Le conseil de Houshaï l’Arkite est meilleur que celui d’Ahitofel. » En effet, le Seigneur avait décrété de faire échouer le conseil d’Ahitofel, un bon conseil, pour amener le malheur sur Absalom.
${}^{15}Houshaï rapporta ceci aux prêtres Sadoc et Abiatar : « Voilà comment Ahitofel a conseillé Absalom et les anciens d’Israël, et voilà comment moi, je les ai conseillés. 
${}^{16}Et maintenant, envoyez vite quelqu’un pour informer David et lui dire : “Ne passe pas cette nuit-ci dans les steppes du désert ; mais il faut absolument que tu traverses le Jourdain.” Sinon, ils ne feront qu’une bouchée du roi et de toute la troupe qui est avec lui. »
${}^{17}Or Jonathan et Ahimaas, les fils des prêtres Sadoc et Abiatar, étaient postés à Ein-Roguel. Une servante alla les informer pour qu’ils aillent à leur tour avertir le roi David. Eux-mêmes, en effet, ne pouvaient entrer en ville sans être repérés. 
${}^{18}Toutefois, un jeune homme les aperçut et alla prévenir Absalom. Partis en grande hâte, ils arrivèrent tous deux à la maison d’un homme de Bahourim qui avait un puits dans sa cour. Ils y descendirent. 
${}^{19}Sa femme prit une couverture qu’elle étendit au-dessus du puits ; elle y répandit ensuite du grain pilé, de sorte qu’on ne remarquait rien. 
${}^{20}Les serviteurs d’Absalom entrèrent chez cette femme, dans la maison, et demandèrent : « Où sont Ahimaas et Jonathan ? » La femme leur répondit : « Ils ont dépassé le réservoir des eaux. » Les serviteurs cherchèrent sans les trouver et retournèrent à Jérusalem.
${}^{21}Après leur départ, les deux autres remontèrent du puits et s’en allèrent informer le roi David. Ils lui dirent : « Levez-vous et passez l’eau en toute hâte car Ahitofel a conseillé de vous attaquer. » 
${}^{22}David se leva donc, ainsi que toute la troupe qui était avec lui, et ils passèrent le Jourdain ; au lever du jour, il n’en restait plus un seul qui n’ait passé le Jourdain.
${}^{23}Quant à Ahitofel, voyant que son conseil n’était pas suivi d’effet, il sella son âne et se mit en route pour regagner sa maison, dans sa ville. Il donna ses instructions aux gens de sa maison. Puis il se pendit, mourut et fut enseveli dans le tombeau de son père.
${}^{24}David arrivait à Mahanaïm, quand Absalom traversait le Jourdain avec tous les hommes d’Israël. 
${}^{25}Absalom avait mis à la tête de l’armée Amasa pour remplacer Joab. Amasa était le fils d’un certain Yitra l’Israélite, qui s’était uni à Abigal, fille de Nahash et sœur de Cerouya, mère de Joab. 
${}^{26}Israël et Absalom établirent leur camp au pays de Galaad.
${}^{27}Dès que David fut arrivé à Mahanaïm, se présentèrent : Shobi, fils de Nahash – de Rabba-des-fils-d’Amone –, Makir, fils d’Ammiel – de Lo-Debar – et Barzillaï le Galaadite – de Roguelime. 
${}^{28}Ils apportaient de la literie, des récipients, de la vaisselle, ainsi que du blé et de l’orge, de la farine et des épis grillés, des fèves et des lentilles, 
${}^{29}du miel et de la crème, des moutons et des quartiers de bœuf. Ils apportaient ces provisions pour David et la troupe qui était avec lui, car ils s’étaient dit : « Dans le désert, toute la troupe a souffert de la faim, de la fatigue et de la soif. »
      
         
      \bchapter{}
      \begin{verse}
${}^{1}David passa en revue ses troupes et mit à leur tête des officiers de millier et des officiers de centaine. 
${}^{2}Puis il leur donna le signal du départ : un tiers était mené par Joab, un tiers par Abishaï, fils de Cerouya, frère de Joab, un tiers par Ittaï le Guittite. Le roi dit à la troupe : « Il faut que je sorte avec vous, moi aussi, pour le combat ! » 
${}^{3}Mais la troupe répliqua : « Tu ne dois pas sortir. S’il nous fallait prendre la fuite, on ne ferait pas attention à nous, et si la moitié d’entre nous venait à mourir, on n’y ferait pas attention non plus ! Toi, tu es comme dix mille d’entre nous, et mieux vaut que tu puisses nous secourir depuis la ville. » 
${}^{4}Le roi leur dit : « Je ferai ce que bon vous semble. » Et il se tint près de la porte de la ville, pendant que toute la troupe en sortait, regroupée par centaines et par milliers. 
${}^{5}À Joab, Abishaï et Ittaï, le roi donna alors cet ordre : « Par égard pour moi, ménagez le jeune Absalom ! » Et toute la troupe entendit quand le roi donna cet ordre aux chefs à propos d’Absalom. 
${}^{6}Les troupes sortirent dans la campagne à la rencontre d’Israël, et le combat eut lieu dans la forêt d’Éphraïm. 
${}^{7}C’est là que les troupes d’Israël furent battues par les serviteurs de David et qu’il y eut de grandes pertes : vingt mille hommes, ce jour-là ! 
${}^{8}Le combat s’éparpilla ensuite dans tout le pays, et la forêt dévora encore plus d’hommes parmi le peuple que l’épée n’en avait dévoré ce jour-là.
${}^{9}Absalom se retrouva par hasard en face des serviteurs de David. Il montait un mulet, et le mulet s’engagea sous la ramure d’un grand térébinthe. La tête d’Absalom se prit dans les branches\\, et il resta entre ciel et terre, tandis que le mulet qui était sous lui continuait d’avancer. 
${}^{10}Quelqu’un l’aperçut et avertit Joab : « Je viens de voir Absalom suspendu dans un térébinthe. » 
${}^{11}Joab dit à l’homme qui l’avait averti : « Tu l’as vu ! Pourquoi donc ne l’as-tu pas frappé et abattu sur place ? J’aurais dû alors te donner dix pièces d’argent et une ceinture. » 
${}^{12}L’homme répondit à Joab : « Même si je soupesais maintenant, dans la paume de mes mains, mille pièces d’argent, je ne porterais pas la main sur le fils du roi, car nous avons entendu de nos oreilles l’ordre que le roi vous a donné à toi, à Abishaï et à Ittaï : “Par égard pour moi, veillez sur le jeune Absalom !” 
${}^{13}Et si j’avais commis cette trahison au péril de ma vie, comme rien n’échappe au roi, tu te serais toi-même tenu à l’écart. »
${}^{14}Joab lui dit : « Je ne vais pas perdre mon temps avec toi ! » Et il se saisit de trois épieux qu’il planta dans le cœur d’Absalom, encore vivant au milieu du térébinthe. 
${}^{15}Alors, dix jeunes écuyers au service de Joab entourèrent Absalom pour le frapper à mort.
${}^{16}Joab sonna du cor. La troupe, faisant demi-tour, cessa de poursuivre Israël, car Joab l’en empêcha. 
${}^{17}On prit Absalom, on le jeta dans la grande fosse en pleine forêt, et l’on érigea par-dessus un monceau de pierres très imposant. Tout Israël s’était enfui, chacun à ses tentes.
${}^{18}De son vivant, Absalom avait entrepris de se faire ériger une stèle, qui se trouve dans la vallée du Roi. Il se disait : « Je n’ai pas de fils pour faire mémoire de mon nom. » Il donna son nom à la stèle. Aujourd’hui encore, on l’appelle « Monument d’Absalom ».
${}^{19}Ahimaas, fils de Sadoc, dit à Joab : « Permets que je coure porter au roi la bonne nouvelle : le Seigneur lui a rendu justice en l’arrachant aux mains de ses ennemis ! » 
${}^{20}Mais Joab lui répondit : « Non ! Aujourd’hui, tu ne serais pas un porteur de bonne nouvelle. Tu le seras un autre jour. Mais aujourd’hui tu ne peux porter une bonne nouvelle, car c’est le fils du roi qui est mort. » 
${}^{21}Et Joab dit à l’Éthiopien : « Va rapporter au roi ce que tu as vu ! » L’Éthiopien se prosterna devant Joab, puis il partit en courant. 
${}^{22}Mais Ahimaas, fils de Sadoc, insista et dit à Joab : « Quoi qu’il arrive, permets que je coure aussi, derrière l’Éthiopien. » Joab lui dit : « Pourquoi veux-tu courir, toi aussi, mon fils, alors qu’il n’y a pas de bonne nouvelle qui te vaudrait une récompense ? » 
${}^{23}Ahimaas répondit : « Quoi qu’il arrive, je veux courir. » Et Joab lui dit : « Cours ! » Ahimaas prit en courant le chemin de la région du Jourdain. Il dépassa l’Éthiopien.
${}^{24}David était assis à l’intérieur de la double porte de la ville\\. Un guetteur allait et venait sur la terrasse de la porte, au-dessus du rempart ; comme il regardait au loin, il aperçut un homme seul qui courait. 
${}^{25}Le guetteur cria pour avertir le roi, et le roi dit : « S’il est seul, c’est qu’il a une bonne nouvelle à nous annoncer. » Tandis que l’homme continuait d’approcher, 
${}^{26}le guetteur en vit accourir un autre. Il cria au portier : « Voici encore un homme en train de courir seul ! » Le roi dit alors : « Celui-là aussi apporte une bonne nouvelle. » 
${}^{27}Le guetteur ajouta : « Je reconnais la façon de courir du premier : c’est celle d’Ahimaas, fils de Sadoc. » Le roi dit alors : « C’est un homme de bien. Il vient sûrement porter une bonne nouvelle. » 
${}^{28}Ahimaas s’approcha et dit au roi : « C’est la paix ! » Il se prosterna, face contre terre, devant le roi et ajouta : « Béni soit le Seigneur ton Dieu : il a livré les hommes qui s’en étaient pris à mon seigneur le roi. » 
${}^{29}Le roi demanda : « Le jeune Absalom est-il en bonne santé ? » Ahimaas répondit : « J’ai bien remarqué une grande agitation au moment où Joab a envoyé l’Éthiopien, serviteur du roi, et ton serviteur, mais je ne sais pas ce qu’il y avait. » 
${}^{30}Le roi lui dit : « Écarte-toi et tiens-toi là. » Il s’écarta et attendit. 
${}^{31}Alors arriva l’Éthiopien, qui déclara : « Bonne nouvelle pour mon seigneur le roi ! Le Seigneur t’a rendu justice aujourd’hui, en t’arrachant aux mains\\de tous ceux qui se dressaient contre toi. » 
${}^{32}Le roi demanda : « Le jeune Absalom est-il en bonne santé ? » Et l’Éthiopien répondit : « Qu’ils aient le sort de ce jeune homme, les ennemis de mon seigneur le roi, et tous ceux qui se sont dressés contre toi pour le mal ! »
      
         
      \bchapter{}
      \begin{verse}
${}^{1}Alors le roi fut bouleversé, il monta dans la salle au-dessus de la porte, et il se mit à pleurer\\. Tout en marchant, il disait : « Mon fils Absalom ! mon fils ! mon fils Absalom ! Pourquoi ne suis-je pas mort à ta place ? Absalom, mon fils ! mon fils ! » 
${}^{2}On alla prévenir Joab : « Voici que le roi pleure : il est en deuil d’Absalom. » 
${}^{3}La victoire, ce jour-là, se changea en deuil pour toute l’armée\\, car elle apprit ce jour-là que le roi était dans l’affliction à cause de son fils. 
${}^{4}Et ce jour-là, l’armée rentra dans la ville à la dérobée, comme se dérobe une armée qui s’est couverte de honte en fuyant durant la bataille. 
${}^{5}Le roi s’était voilé le visage et criait à pleine voix : « Mon fils Absalom ! Absalom, mon fils ! mon fils ! »
${}^{6}Alors Joab alla trouver le roi dans sa maison et lui dit : « Aujourd’hui, tu couvres de confusion le visage de tous tes serviteurs, eux qui, aujourd’hui, t’ont sauvé la vie à toi, à tes fils et à tes filles, à tes femmes et à tes concubines. 
${}^{7}En manifestant de l’amour envers ceux qui te haïssent et de la haine envers ceux qui t’aiment, tu montres aujourd’hui que chefs et serviteurs ne sont rien pour toi. Oui, aujourd’hui je comprends : si Absalom était vivant et que nous soyons tous morts aujourd’hui, tu trouverais cela très bien. 
${}^{8}Mais maintenant, lève-toi, sors et va parler au cœur de tes serviteurs ! Sinon, par le Seigneur j’en fais le serment : si tu ne sors pas, personne, ce soir, ne restera avec toi pour passer la nuit, et ce serait pour toi un malheur pire que tous ceux qui te sont arrivés, depuis ta jeunesse jusqu’à maintenant. » 
${}^{9}Alors le roi se leva et vint s’asseoir à la porte de la ville. On informa tout le peuple en disant : « Voici que le roi est assis à la porte ! » Et le peuple se présenta devant le roi.
      Israël s’était enfui, chacun à ses tentes. 
${}^{10}Or, dans toutes les tribus d’Israël, les gens se mirent à discuter. On disait : « Le roi nous avait délivrés de la main de nos ennemis et fait échapper à la main des Philistins. Or maintenant, il a dû fuir le pays à cause d’Absalom. 
${}^{11}Mais Absalom, lui à qui nous avons donné l’onction royale, il est mort au combat ! Alors, comment ne faites-vous rien pour que le roi revienne ? » 
${}^{12}De son côté, le roi David envoya dire aux prêtres Sadoc et Abiatar : « Vous parlerez ainsi aux anciens de Juda : “Pourquoi seriez-vous les derniers à faire revenir le roi dans sa maison ? La question que se pose tout Israël est déjà parvenue auprès du roi, chez lui ! 
${}^{13}Vous êtes mes frères, vous êtes mes os et ma chair. Alors, pourquoi seriez-vous les derniers à faire revenir le roi ?” 
${}^{14}Vous direz aussi à Amasa : “N’es-tu pas mes os et ma chair ? Que Dieu amène le malheur sur moi, et pire encore, si tu ne deviens pas, pour toujours, le chef de mon armée à la place de Joab.” »
${}^{15}C’est ainsi qu’il inclina vers lui le cœur de tous les hommes de Juda comme le cœur d’un seul homme, si bien qu’ils envoyèrent dire au roi : « Reviens, toi et tous tes serviteurs ! » 
${}^{16}Le roi revint donc et arriva au Jourdain tandis que les hommes de Juda arrivaient à Guilgal, pour aller à la rencontre du roi et lui faire passer le Jourdain.
${}^{17}Shiméï, fils de Guéra, le Benjaminite de Bahourim, se hâta de descendre avec les hommes de Juda, à la rencontre du roi David. 
${}^{18}Mille hommes de Benjamin l’accompagnaient, ainsi que Ciba, serviteur de la maison de Saül, avec ses quinze fils et ses vingt serviteurs. Ils se précipitèrent vers le Jourdain, au devant du roi, 
${}^{19}et le bac traversa pour faire passer la maison du roi ; ils agirent ainsi pour se rendre agréables à ses yeux.
      Lors de ce passage, Shiméï, fils de Guéra, se jeta aux pieds du roi 
${}^{20}et lui dit : « Que mon seigneur le roi ne retienne pas l’offense ! Ne te souviens plus que ton serviteur t’a offensé le jour où mon seigneur le roi est sorti de Jérusalem ! Que le roi ne prenne pas cela à cœur ! 
${}^{21}Oui, ton serviteur le sait : j’ai péché. Mais aujourd’hui je suis venu, précédant toute la maison de Joseph, pour descendre à la rencontre de mon seigneur le roi. »
${}^{22}Abishaï, fils de Cerouya, prit alors la parole : « Serait-ce là un motif pour ne pas mettre à mort Shiméï, lui qui a maudit le messie du Seigneur ? » 
${}^{23}Mais David lui répliqua : « Que me voulez-vous, les fils de Cerouya, en vous faisant aujourd’hui mes adversaires ? Aujourd’hui, on mettrait quelqu’un à mort ? En effet, ne suis-je pas certain d’être aujourd’hui roi sur Israël ? » 
${}^{24}Et le roi dit à Shiméï : « Tu ne mourras pas. » Le roi lui en fit le serment.
${}^{25}Mefibosheth, fils de Saül, descendit à la rencontre du roi. Il n’avait pris aucun soin de ses pieds, ni de sa barbe, ni lavé ses vêtements, depuis le jour où le roi était parti jusqu’à ce jour où il revenait en paix. 
${}^{26}Or, comme il arrivait de Jérusalem pour rencontrer le roi, celui-ci lui demanda : « Pourquoi n’es-tu pas venu avec moi, Mefibosheth ? » 
${}^{27}Il répondit : « Mon seigneur le roi, c’est que mon serviteur Ciba m’a trompé ! En effet, je m’étais dit : “Je vais seller mon ânesse et la monter pour m’en aller avec le roi” – puisque ton serviteur est boiteux. 
${}^{28}Ciba a bien calomnié ton serviteur auprès de mon seigneur le roi ! Mais mon seigneur le roi est comme un ange de Dieu : alors, agis comme bon te semble. 
${}^{29}Pour mon seigneur le roi, en effet, la maison de mon père ne comptait que des hommes qui méritaient la mort, et pourtant tu as admis ton serviteur parmi ceux qui mangent à ta table. Ai-je encore un droit ? Que puis-je encore réclamer au roi ? » 
${}^{30}Le roi lui dit : « Pourquoi continuer à te répandre en paroles ? Je le déclare : Toi et Ciba, vous vous partagerez les terres. » 
${}^{31}Mefibosheth dit alors au roi : « Qu’il prenne même le tout, du moment que mon seigneur le roi rentre chez lui en paix ! »
${}^{32}Barzillaï le Galaadite était descendu de Roguelim. Il devait passer le Jourdain avec le roi, pour ensuite prendre congé de lui près du Jourdain. 
${}^{33}Or, Barzillaï était très vieux : il avait quatre-vingts ans. C’est lui qui avait pourvu à l’entretien du roi lors de son séjour à Mahanaïm : c’était un personnage important. 
${}^{34}Le roi dit à Barzillaï : « Continue avec moi ! À Jérusalem, quand tu seras près de moi, j’assurerai ton entretien. » 
${}^{35}Barzillaï dit au roi : « Combien d’années me reste-t-il à vivre, pour que je monte avec le roi à Jérusalem ? 
${}^{36}J’ai aujourd’hui quatre-vingts ans. Est-ce que je peux discerner entre le bon et le mauvais ? Ton serviteur peut-il apprécier ce qu’il mange et ce qu’il boit ? entendre encore la voix des chanteurs et des chanteuses ? Pourquoi ton serviteur devrait-il encore être une charge pour mon seigneur le roi ? 
${}^{37}C’est à peine si ton serviteur pourrait passer le Jourdain avec le roi ! Alors pourquoi le roi m’accorderait-il une telle récompense ? 
${}^{38}Permets que ton serviteur s’en retourne et que je meure dans ma ville, près de la tombe de mon père et de ma mère. Mais voici ton serviteur Kimham : c’est lui qui passera avec mon seigneur le roi. Fais donc pour lui ce qui est bon à tes yeux ! » 
${}^{39}Le roi dit : « Kimham passera avec moi, et moi, je ferai pour lui ce qui est bon à tes yeux. Quoi que tu choisisses, je le ferai pour toi. » 
${}^{40}Quand tout le peuple, ainsi que le roi, allait passer le Jourdain, le roi embrassa et bénit Barzillaï qui s’en retourna chez lui. 
${}^{41}Le roi continua vers Guilgal, accompagné de Kimham.
      Tout le peuple de Juda, ainsi que la moitié du peuple d’Israël, avaient fait passer au roi le Jourdain. 
${}^{42}Alors tous les hommes d’Israël, venant auprès du roi, lui dirent : « Pourquoi donc nos frères, ceux de Juda, t’ont-ils accaparé pour faire passer le Jourdain au roi et à sa maison, alors que tous les hommes de David étaient avec lui ? » 
${}^{43}Tous les hommes de Juda répliquèrent aux hommes d’Israël : « C’est que le roi est plus proche de nous. Pourquoi vous irriter de cela ? Avons-nous mangé quelque chose aux dépens du roi ? Nous a-t-il fait des cadeaux ? » 
${}^{44}Israël répondit à Juda : « J’ai dix fois des droits sur le roi, si bien que, même pour David, je suis plus que toi. Pourquoi m’as-tu méprisé ? N’ai-je pas été le premier à demander que revienne mon roi ? » Mais la parole des hommes de Juda l’emporta sur celle des hommes d’Israël.
      
         
      \bchapter{}
      \begin{verse}
${}^{1}Un vaurien se trouvait là, un dénommé Shéba, fils de Bikri, un Benjaminite. Il se mit à sonner du cor et déclara :
        \\« Pour nous, aucune part avec David,
        \\pas d’héritage avec le fils de Jessé.
        \\Chacun à ses tentes, Israël ! »
${}^{2}Alors tous les hommes d’Israël quittèrent David et montèrent à la suite de Shéba, fils de Bikri, tandis que les hommes de Juda, depuis le Jourdain jusqu’à Jérusalem, restèrent attachés aux pas de leur roi.
${}^{3}David rentra chez lui, à Jérusalem. Il prit les dix concubines qu’il avait laissées pour garder sa maison et les mit dans une maison sous bonne garde. Il pourvut à leur entretien mais ne s’en approcha plus. Elles furent séquestrées jusqu’au jour de leur mort, comme des veuves d’un vivant.
${}^{4}Le roi dit à Amasa : « Convoque-moi les hommes de Juda dans les trois jours. Puis, tiens-toi ici ! » 
${}^{5}Amasa s’en alla convoquer Juda, mais il prit du retard sur le délai que David lui avait fixé. 
${}^{6}David dit à Abishaï : « Maintenant, c’est Shéba, fils de Bikri, qui va être pour nous pire qu’Absalom. Toi, donc, prends les serviteurs de ton maître et poursuis-le, avant qu’il ne trouve des villes fortifiées et qu’il n’échappe à nos regards ! » 
${}^{7}Alors partirent en campagne derrière Abishaï les hommes de Joab, les Kerétiens et les Pelétiens ainsi que toute l’élite des guerriers. Ils sortirent de Jérusalem à la poursuite de Shéba, fils de Bikri. 
${}^{8}Ils se trouvaient près de la Grande Pierre qui est à Gabaon, quand Amasa arriva en face d’eux. Joab était équipé de sa tenue et, par-dessus, d’un ceinturon autour des reins, avec une épée dans son fourreau. Elle tomba tandis qu’il s’avançait. 
${}^{9}Joab dit à Amasa : « Vas-tu bien, mon frère ? » Et, de sa main droite, il saisit Amasa par la barbe pour l’embrasser. 
${}^{10}Mais Amasa n’avait pas fait attention à l’épée que Joab avait reprise en main. Celui-ci l’en frappa au ventre, et ses entrailles se répandirent à terre. Il n’eut pas à s’y prendre à deux fois, et Amasa mourut.
      Joab et son frère Abishaï repartirent à la poursuite de Shéba, fils de Bikri. 
${}^{11}L’un des serviteurs de Joab était resté près du corps et disait : « Quiconque est en faveur de Joab, quiconque est pour David, qu’il suive Joab ! » 
${}^{12}Cependant, Amasa avait roulé dans son sang, au beau milieu de la route, et l’homme s’aperçut que tout le peuple s’arrêtait là. Il tira le corps d’Amasa à l’écart de la route dans un champ et jeta sur lui un manteau, s’étant aperçu que tous ceux qui arrivaient près de lui s’arrêtaient. 
${}^{13}Dès qu’il l’eut enlevé de la route, tout le monde se lança derrière Joab à la poursuite de Shéba, fils de Bikri. 
${}^{14}Joab parcourut toutes les tribus d’Israël jusqu’à Abel-Beth-Maaka. Tous les alliés se rassemblèrent et partirent eux-aussi derrière lui. 
${}^{15}Ils arrivèrent pour assiéger Shéba dans Abel-Beth-Maaka. Ils amoncelèrent contre la ville un remblai qui s’adossait à l’avant-mur, et toute cette troupe qui était avec Joab se mit à saper le rempart pour le faire tomber. 
${}^{16}Mais une femme avisée s’écria depuis la ville : « Écoutez, écoutez donc ! Veuillez dire à Joab : “Approche-toi jusqu’ici, je veux te parler”. » 
${}^{17}Joab s’approcha d’elle, et la femme lui demanda : « Est-ce bien toi, Joab ? – C’est moi », répondit-il. Elle lui dit : « Écoute les paroles de ta servante ! » Il répondit : « J’écoute. » 
${}^{18}Elle poursuivit : « Jadis on disait : “Interroge Abel, l’affaire sera close.” 
${}^{19}Moi, je suis en Israël des plus pacifiques et des plus sûres, et toi, tu cherches à faire périr cette ville, qui est une mère en Israël ! Pourquoi veux-tu engloutir l’héritage du Seigneur ? » 
${}^{20}Joab lui répondit : « Quelle horreur, quelle horreur pour moi, que d’engloutir et de saccager ! 
${}^{21}Il ne s’agit pas de cela, mais il y a un homme de la montagne d’Éphraïm, un dénommé Shéba, fils de Bikri, qui a levé la main contre le roi David. Livrez-le, lui seul, et je m’éloignerai de la ville. » La femme dit à Joab : « Eh bien, sa tête, on va te la jeter par-dessus le rempart ! » 
${}^{22}Cette femme avisée vint faire part à tout le peuple de sa proposition. Ils coupèrent alors la tête de Shéba, fils de Bikri, et la jetèrent à Joab. Celui-ci sonna du cor, et l’on se dispersa loin de la ville, chacun à ses tentes. Joab, quant à lui, revint à Jérusalem auprès du roi.
${}^{23}Joab commandait toute l’armée d’Israël. Benaya, fils de Joad, dirigeait les Kerétiens et les Pelétiens. 
${}^{24}Adoram était chef de la corvée. Josaphat, fils d’Ahiloud, était archiviste. 
${}^{25}Shéwa était scribe. Sadoc et Abiatar étaient prêtres. 
${}^{26}Il y avait aussi Ira, descendant de Yaïr, qui était prêtre au service de David.
      <h2 class="intertitle" id="d85e73289">4. Documents divers (<a class="unitex_link" href="#bib_2s_21">2 S 21 – 24</a>)</h2>
      
         
      \bchapter{}
      \begin{verse}
${}^{1}Il y eut une famine au temps de David, pendant trois années de suite, et David consulta le Seigneur. Et le Seigneur répondit : « Sur Saül et sur sa maison, il y a du sang, parce qu’il a fait mourir les Gabaonites. » 
${}^{2}Alors le roi convoqua les Gabaonites et leur parla. Les Gabaonites n’étaient pas des fils d’Israël mais se rattachaient à quelques survivants des Amorites. Les fils d’Israël s’étaient engagés envers eux par serment, mais Saül, dans son zèle pour Israël et Juda, avait cherché à les abattre. 
${}^{3}David déclara donc aux Gabaonites : « Que faire pour vous et comment réparer, afin que vous bénissiez l’héritage du Seigneur ? » 
${}^{4}Ils lui dirent : « À propos de Saül et de sa maison, il ne s’agit pas pour nous d’or ou d’argent, ni de quelqu’un d’autre à faire mourir en Israël. » David déclara : « Ce que vous direz, je le ferai pour vous. » 
${}^{5}Ils dirent au roi : « Cet homme qui voulait nous exterminer, qui pensait nous anéantir et nous éliminer de tout le territoire d’Israël, 
${}^{6}qu’on nous livre sept de ses descendants : pour le Seigneur, nous les écartèlerons à Guibéa de Saül, sur la montagne du Seigneur. » Le roi dit : « Moi, je vous les livre. » 
${}^{7}Cependant, le roi épargna Mefibosheth, fils de Jonathan, fils de Saül, à cause du serment au nom du Seigneur qui liait entre eux David et Jonathan, fils de Saül. 
${}^{8}Le roi prit donc les deux fils que Rispa, fille d’Ayya, avait donnés à Saül, Armoni et Mefibosheth, et les cinq fils que Mérab, fille de Saül, avait donnés à Adriel, fils de Barzillaï, de Mehola, 
${}^{9}et il les livra aux mains des Gabaonites qui les écartelèrent sur la montagne, devant le Seigneur. Ils succombèrent tous les sept ensemble. C’était le temps de la moisson. Ils furent mis à mort dans les premiers jours, au commencement de la moisson des orges. 
${}^{10}Rispa, fille d’Ayya, prit un sac qu’elle étendit pour elle sur le rocher ; elle y resta depuis le commencement de la moisson des orges jusqu’à ce que l’eau du ciel se répandît sur les corps ; elle ne laissa pas les oiseaux du ciel venir sur eux pendant le jour, ni les bêtes sauvages pendant la nuit. 
${}^{11}On informa David de ce qu’avait fait Rispa, fille d’Ayya, la concubine de Saül. 
${}^{12}Alors David alla reprendre les ossements de Saül et de son fils Jonathan aux notables de Yabesh-de-Galaad : ceux-ci les avaient dérobés sur la place de Beth-Shéane où les Philistins les avaient suspendus, quand Saül était tombé sous leurs coups à Gelboé. 
${}^{13}David rapporta donc de Yabesh les ossements de Saül et de son fils Jonathan ; on recueillit aussi les ossements de ceux qu’on avait écartelés. 
${}^{14}Les ossements de Saül avec ceux de son fils Jonathan furent ensevelis au pays de Benjamin, à Céla, dans la tombe de Kish, son père. On fit tout ce que le roi avait ordonné. Après quoi, Dieu se montra favorable envers le pays.
      
         
${}^{15}Il y eut encore une bataille des Philistins contre Israël, et David descendit avec ses serviteurs. Ils combattirent les Philistins, et David en fut épuisé. 
${}^{16}Yishbi-Benob, un descendant du géant Rafa, qui avait une lance d’un poids de trois cents sicles, poids du bronze, et une épée neuve à la ceinture, parlait de frapper David. 
${}^{17}Mais Abishaï, fils de Cerouya, vint au secours de David et frappa le Philistin à mort. Alors les hommes de David se mirent à l’adjurer en disant : « Tu ne sortiras plus avec nous au combat, tu n’éteindras pas la lampe d’Israël ! »
       
${}^{18}Après quoi, il y eut encore une bataille contre les Philistins à Gob. C’est alors que Sibbekaï de Housha abattit Saf, l’un des descendants du géantRafa.
${}^{19}Il y eut encore une autre bataille contre les Philistins à Gob. Elhanane, fils de Yaaré-Oreguim, de Bethléem, abattit Goliath de Gath, celui dont le bois de la lance était comme le rouleau d’un métier à tisser.
${}^{20}Il y eut encore une bataille à Gath. Là se trouvait un homme de haute taille qui avait six doigts aux mains et six doigts aux pieds, vingt-quatre en tout. Lui aussi était un descendant du géant Rafa. 
${}^{21}Comme il défiait Israël, Jonathan, fils de Shiméa, un frère de David, l’abattit.
${}^{22}Ces quatre-là étaient des descendants de Rafa, à Gath. Ils tombèrent sous les coups de David et de ses serviteurs.
      
         
      \bchapter{}
      \begin{verse}
${}^{1}David prononça pour le Seigneur les paroles de ce chant, le jour où le Seigneur l’eut délivré de la main de tous ses ennemis et de la main de Saül. 
${}^{2}Il dit :
      
         
       
        \\Le Seigneur est mon roc, ma forteresse,
        \\il est mon libérateur !
${}^{3}Dieu, le rocher qui m’abrite,
        \\mon bouclier, la force qui me sauve,
        \\ma citadelle, mon refuge,
        \\mon sauveur, tu me sauves de la violence !
         
${}^{4}Louange à Dieu !
        \\Quand je fais appel au Seigneur,
        \\je suis sauvé de mes ennemis.
         
${}^{5}Les flots de la mort m’entouraient,
        \\le torrent fatal m’épouvantait ;
${}^{6}des liens infernaux m’étreignaient :
        \\j’étais pris aux pièges de la mort.
         
${}^{7}Dans mon angoisse, j’appelai le Seigneur ;
        \\vers mon Dieu, je lançai un appel ;
        \\de son temple il entend ma voix :
        \\mon cri parvient à ses oreilles.
         
${}^{8}La terre titube et tremble,
        \\les fondements du ciel frémissent,
        \\secoués par l’explosion de sa colère.
         
${}^{9}Une fumée sort de ses narines,
        \\de sa bouche, un feu qui dévore,
        \\une gerbe de charbons embrasés.
         
${}^{10}Il incline les cieux et descend,
        \\une sombre nuée sous ses pieds :
${}^{11}d’un Kéroub, il fait sa monture
        \\il vole sur les ailes du vent.
         
${}^{12}Il s’entoure de ténèbres comme d’une tente,
        \\de masses d’eau, d’épaisses nuées.
${}^{13}Une lueur le précède,
        \\allumant des gerbes de feu.
         
${}^{14}Le Seigneur tonne du haut du ciel,
        \\le Très-Haut fait entendre sa voix.
${}^{15}De tous côtés, il tire des flèches,
        \\il décoche un éclair, il répand la terreur.
         
${}^{16}Alors le fond de la mer se découvrit,
        \\les assises du monde apparurent,
        \\sous la voix menaçante du Seigneur,
        \\au souffle qu’exhalait sa colère.
         
${}^{17}Des hauteurs il tend la main pour me saisir,
        \\il me retire du gouffre des eaux ;
${}^{18}il me délivre d’un puissant ennemi,
        \\d’adversaires plus forts que moi.
         
${}^{19}Au jour de ma défaite ils m’attendaient,
        \\mais j’avais le Seigneur pour appui.
${}^{20}Et lui m’a dégagé, mis au large,
        \\il m’a libéré, car il m’aime.
         
${}^{21}Le Seigneur me traite selon ma justice,
        \\il me donne le salaire des mains pures,
${}^{22}car j’ai gardé les chemins du Seigneur,
        \\jamais je n’ai trahi mon Dieu.
         
${}^{23}Ses ordres sont tous devant moi,
        \\jamais je ne m’écarte de ses lois.
${}^{24}Je suis sans reproche envers lui,
        \\je me garde loin du péché.
${}^{25}Le Seigneur me donne selon ma justice,
        \\selon ma pureté qu’il voit de ses yeux.
         
${}^{26}Tu es fidèle envers l’homme fidèle,
        \\sans reproche avec l’homme sans reproche ;
${}^{27}envers qui est loyal, tu es loyal,
        \\tu ruses avec le pervers.
         
${}^{28}Tu sauves le peuple des humbles ;
        \\les regards hautains, tu les rabaisses.
${}^{29}Toi, Seigneur, tu es ma lampe.
        \\Le Seigneur éclaire mes ténèbres.
${}^{30}Grâce à toi, je saute le fossé,
        \\grâce à mon Dieu, je franchis la muraille.
         
${}^{31}Ce Dieu a des chemins sans reproche,
        \\la parole du Seigneur est sans alliage,
        \\il est un bouclier pour qui s’abrite en lui.
         
${}^{32}Qui est Dieu, hormis le Seigneur ?
        \\le Rocher, sinon notre Dieu ?
${}^{33}C’est le Dieu qui est ma place forte
        \\et me fraie un chemin sans reproche.
         
${}^{34}Il me donne l’agilité du chamois,
        \\il me tient debout sur les hauteurs,
${}^{35}il exerce mes mains à combattre
        \\et mon bras, à tendre l’arc.
         
${}^{36}Par ton bouclier tu m’assures la victoire,
        \\ta patience m’élève.
${}^{37}C’est toi qui allonges ma foulée
        \\sans que faiblissent mes chevilles.
         
${}^{38}Je poursuis mes ennemis, je les détruis,
        \\je ne reviens qu’après leur défaite ;
${}^{39}je les ai achevés, brisés :
        \\ils ne se relèveront pas ;
        \\ils sont tombés : les voilà sous mes pieds.
         
${}^{40}Pour le combat tu m’emplis de vaillance ;
        \\devant moi tu fais plier mes agresseurs.
${}^{41}Tu me livres des ennemis en déroute ;
        \\j’anéantis mes adversaires.
         
${}^{42}Ils appellent ? pas de sauveur !
        \\le Seigneur ? pas de réponse !
${}^{43}J’en fais de la poussière,
        \\comme la boue des rues, je les écrase et les piétine.
         
${}^{44}Tu me libères des querelles du peuple,
        \\tu me gardes à la tête des nations.
        \\Un peuple d’inconnus m’est asservi.
         
${}^{45}Des fils d’étrangers se soumettent ;
        \\au premier mot, ils m’obéissent.
${}^{46}Ces fils d’étrangers capitulent :
        \\quittant leurs bastions, ils seront capturés.
         
${}^{47}Vive le Seigneur ! Béni soit mon Rocher !
        \\Qu’il triomphe, Dieu, le roc de mon salut,
${}^{48}ce Dieu qui m’accorde la revanche,
        \\qui fait tomber des peuples en mon pouvoir !
         
${}^{49}Tu me fais échapper à mes ennemis
        \\et triompher de l’agresseur,
        \\tu m’arraches à la violence de l’homme.
         
${}^{50}Aussi, je te rendrai grâce parmi les peuples,
        \\Seigneur, je fêterai ton nom.
${}^{51}Il donne à son roi de grandes victoires,
        \\il se montre fidèle à son messie,
        \\à David et sa descendance, pour toujours.
      
         
      \bchapter{}
       
      \begin{verse}
${}^{1}Voici les dernières paroles de David :
       
        \\Oracle de David, fils de Jessé,
        oracle de l’homme hautement exalté,
        \\le messie du Dieu de Jacob,
        le charme des hymnes d’Israël.
${}^{2}L’Esprit du Seigneur parle par ma bouche,
        ses mots viennent sur ma langue.
${}^{3}Le Dieu d’Israël a parlé,
        le Rocher d’Israël m’a dit :
         
        \\« Le juste qui gouverne les hommes,
        celui qui les gouverne dans la crainte de Dieu,
${}^{4}il est comme la lumière du matin
        quand se lève le soleil par un matin sans nuages :
        \\à cet éclat, après la pluie,
        l’herbe sort de la terre. »
         
${}^{5}N’en est-il pas ainsi de ma maison avec Dieu ?
        \\Puisqu’il a établi pour moi une alliance éternelle,
        qui est ordonnée en tout et respectée,
        \\ne fera-t-il pas germer pour moi
        tout mon salut et tout mon désir ?
         
${}^{6}Mais les vauriens sont tous comme des épines que l’on évite :
        on ne les prend pas avec la main.
${}^{7}On ne les touche qu’avec une arme de fer
        ou un bois de lance.
        \\On y met le feu
        pour les brûler sur place.
${}^{8}Voici les noms des guerriers de David : Yosheb-ba-Shèbeth, un Tahkmonite, le chef des Trois : c’est lui qui brandit sa lance et frappa à mort huit cents hommes en une seule fois. 
${}^{9}Après lui, Éléazar, fils de Dodo, fils d’un Ahohite, l’un des Trois Guerriers. Il était avec David lorsqu’ils défièrent les Philistins rassemblés pour les combattre. Alors que les hommes d’Israël se retiraient, 
${}^{10}lui se dressa et frappa les Philistins jusqu’à ce que sa main, fatiguée, reste engourdie sur l’épée. Ce jour-là, le Seigneur remporta une grande victoire et la troupe revint derrière Éléazar, mais seulement pour dépouiller les morts. 
${}^{11}Après lui, Shamma, fils d’Agué, le Hararite. Les Philistins s’étaient rassemblés à Lèhi. Il y avait là une parcelle de champ pleine de lentilles, et la troupe fuyait devant les Philistins. 
${}^{12}Mais lui, se postant au milieu du champ qu’il dégagea, battit les Philistins. Ainsi le Seigneur remporta une grande victoire.
       
${}^{13}Trois guerriers, parmi l’élite des Trente, descendirent au temps de la moisson rejoindre David à la grotte d’Adoullam. Un corps de Philistins campait dans le Val des Refaïtes. 
${}^{14}David était alors dans son refuge fortifié, et il y avait encore un poste de Philistins à Bethléem. 
${}^{15}David exprima un désir : « Qui me fera boire de l’eau du puits qui est à la porte de Bethléem ? » 
${}^{16}Les Trois Guerriers s’ouvrirent un passage à travers le camp des Philistins, tirèrent de l’eau du puits qui est à la porte de Bethléem, puis ils l’emportèrent pour l’offrir à David. Mais il refusa d’en boire et la répandit en libation devant le Seigneur, 
${}^{17}en disant : « Que le Seigneur me garde de faire cela ! Boirais-je le sang des hommes qui sont allés là-bas en risquant leur vie ? » Il refusa donc de boire. Voilà ce que firent les Trois Guerriers.
       
${}^{18}Abishaï, frère de Joab, fils de Cerouya, était le chef des Trente : c’est lui qui brandit sa lance et frappa à mort trois cents hommes ; il se fit un nom parmi les Trente. 
${}^{19}Certes, il fut honoré plus que les Trente et devint leur chef, mais il ne parvint pas au rang des Trois.
${}^{20}Benaya, fils d’un homme de valeur, Joad, fut prodigue en exploits. Il était originaire de Qabcéel. C’est lui qui frappa les deux Ariel de Moab, et c’est lui qui descendit tuer le lion dans la citerne, un jour de neige. 
${}^{21}C’est lui aussi qui frappa un Égyptien, un homme de fière allure, qui avait en main une lance. Il descendit contre l’Égyptien avec un bâton, lui arracha la lance de la main et le tua avec sa propre lance. 
${}^{22}Voilà ce qu’accomplit Benaya, fils de Joad, et il se fit un nom parmi les Trente Guerriers. 
${}^{23}Il fut plus honoré que les Trente, mais ne parvint pas au rang des Trois. David le mit à la tête de sa garde personnelle.
       
${}^{24}Asahel, frère de Joab, faisait partie des Trente, ainsi que :
      <p class="retrait1">Elhanane, fils de Dodo, de Bethléem,
${}^{25}Shamma de Harod, Eliqa de Harod,
${}^{26}Hèlès de Pèlèt, Ira, fils d’Iqqesh, de Teqoa,
${}^{27}Abièzer d’Anatoth, Mebounaï de Housha,
${}^{28}Salmone d’Ahoh, Mahraï de Netofa,
${}^{29}Héleb, fils de Baana, de Netofa,
      <p class="retrait1">Ittaï, fils de Ribaï, de Guibéa des fils de Benjamin,
${}^{30}Benaya de Piréatone, Hiddaï des Torrents de Gaash,
${}^{31}Abi-Albone d’Araba, Azmaweth de Bahourim,
${}^{32}Élyahba de Shaalbone,
      <p class="retrait1">les fils de Yashène, Jonathan,
${}^{33}Shamma de Harar, Ahiam, fils de Sharar, de Harar,
${}^{34}Élifèleth, fils d’Ahashbaï, fils du Maakatite,
      <p class="retrait1">Éliam, fils d’Ahitofel de Guilo,
${}^{35}Hesraï de Carmel, Paaraï d’Arab,
${}^{36}Yiguéal, fils de Nathan, de Soba,
      <p class="retrait1">Bani de Gad,
${}^{37}Cèleq d’Ammone,
      <p class="retrait1">Naharaï de Beéroth, écuyer de Joab, un fils de Cerouya,
${}^{38}Ira de Yattir, Gareb de Yattir,
${}^{39}Ourias le Hittite.
      <p class="retrait1">Au total : trente-sept.
      
         
      \bchapter{}
      \begin{verse}
${}^{1}La colère du Seigneur s’enflamma de nouveau contre Israël. Le Seigneur incita David à nuire au peuple. Il lui dit : « Va, dénombre Israël et Juda ! »
${}^{2}Le roi dit à Joab, le chef de l’armée, qui était près de lui : « Parcourez\\toutes les tribus d’Israël, de Dane à Bershéba, et faites le recensement du peuple, afin que je connaisse le chiffre de la population. » 
${}^{3}Joab dit au roi : « Que le Seigneur ton Dieu accroisse le peuple au centuple, et que mon seigneur le roi le voie de ses yeux ! Mais pourquoi mon seigneur le roi veut-il une chose pareille ? » 
${}^{4}Néanmoins, l’ordre du roi s’imposa à Joab et aux chefs de l’armée. Ils sortirent de chez le roi pour faire le recensement du peuple d’Israël.
${}^{5}Ils passèrent le Jourdain et campèrent à Aroër, au sud de la ville qui est au milieu de la vallée, dans le territoire de Gad, puis partirent vers Yazèr. 
${}^{6}Ils arrivèrent en Galaad et dans le pays d’en bas à Hodshi. Ils poursuivirent jusqu’à Dane-Yaân et jusqu’aux alentours de Sidon. 
${}^{7}Ils entrèrent dans la ville forte de Tyr et dans toutes les villes des Hivvites et des Cananéens. Puis, ils partirent pour le Néguev de Juda, vers Bershéba. 
${}^{8}Ils parcoururent ainsi tout le pays et rentrèrent à Jérusalem au bout de neuf mois et vingt jours. 
${}^{9}Joab donna au roi les chiffres du recensement : Israël comptait huit cent mille hommes capables de combattre\\, et Juda cinq cent mille hommes.
${}^{10}Mais après cela, le cœur de David lui battit\\d’avoir recensé le peuple, et il dit au Seigneur : « C’est un grand péché que j’ai commis\\ ! Maintenant, Seigneur, daigne passer sur la faute\\de ton serviteur, car je me suis vraiment conduit comme un insensé ! » 
${}^{11} Le lendemain matin, David se leva. Or la parole du Seigneur avait été adressée au prophète Gad, le voyant attaché à David : 
${}^{12} « Va dire à David : Ainsi parle le Seigneur : Je vais te présenter trois châtiments ; choisis l’un d’entre eux, et je te l’infligerai. » 
${}^{13} Gad se rendit alors chez David et lui transmit ce message : « Préfères-tu qu’il y ait la famine dans ton pays pendant sept ans ? Ou bien fuir devant tes adversaires lancés à ta poursuite, pendant trois mois ? Ou bien la peste dans ton pays pendant trois jours ? Réfléchis donc, et vois ce que je dois répondre à celui qui m’a envoyé. » 
${}^{14} David répondit au prophète\\Gad : « Je suis dans une grande angoisse… Eh bien ! tombons plutôt entre les mains du Seigneur, car sa compassion est grande, mais que je ne tombe pas entre les mains des hommes ! » 
${}^{15} Le Seigneur envoya donc la peste en Israël dès le lendemain jusqu’à la fin des trois jours\\. Depuis Dane jusqu’à Bershéba, il mourut soixante-dix mille hommes. 
${}^{16} Mais lorsque l’ange du Seigneur\\étendit la main vers Jérusalem pour l’exterminer, le Seigneur renonça à ce mal, et il dit à l’ange exterminateur : « Assez ! Maintenant, retire ta main. » L’ange du Seigneur se trouvait alors près de l’aire à grain d’Arauna le Jébuséen. 
${}^{17} David, en voyant l’ange frapper le peuple, avait dit au Seigneur : « C’est moi qui ai péché, c’est moi qui suis coupable ; mais ceux-là, le troupeau, qu’ont-ils fait ? Que ta main s’appesantisse\\donc sur moi et sur la maison de mon père ! »
       
${}^{18}Ce jour-là, Gad alla trouver David et lui dit : « Monte, élève un autel au Seigneur sur l’aire d’Arauna le Jébuséen ! » 
${}^{19}David monta donc suivant la parole de Gad, comme le Seigneur l’avait ordonné. 
${}^{20}Arauna regarda et aperçut le roi et ses serviteurs qui se dirigeaient vers lui. Il sortit et se prosterna, face contre terre, devant le roi. 
${}^{21}Arauna demanda alors : « Pourquoi mon seigneur le roi vient-il chez son serviteur ? » David lui répondit : « Pour acheter ton aire à grain, afin d’y bâtir un autel. Et le fléau s’écartera du peuple. » 
${}^{22}Arauna dit à David : « Que mon seigneur le roi prenne et offre en holocauste ce qui lui semblera bon : voici les bœufs pour l’holocauste et, pour le bois du feu, les traîneaux à battre le grain et les pièces de l’attelage. 
${}^{23}Ô roi, tout cela, Arauna te le donne. » Puis il ajouta : « Que le Seigneur ton Dieu agrée ton sacrifice ! » 
${}^{24}Mais le roi dit à Arauna : « Non ! je veux te l’acheter et t’en payer le prix : je n’offrirai pas au Seigneur mon Dieu des holocaustes qui ne me coûteraient rien ! » David acheta donc l’aire et les bœufs pour cinquante pièces d’argent. 
${}^{25}Là, il bâtit un autel pour le Seigneur, puis il offrit des holocaustes et des sacrifices de paix. Le Seigneur redevint favorable au pays, et le fléau s’écarta d’Israël.
