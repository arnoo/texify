  
  
    
    \bbook{ÉVANGILE SELON SAINT MATTHIEU}{ÉVANGILE SELON SAINT MATTHIEU}
      
         
      \bchapter{}
        ${}^{1}Généalogie de Jésus, Christ,
        fils de David, fils d’Abraham.
        ${}^{2}Abraham engendra Isaac,
        \\Isaac engendra Jacob,
        \\Jacob engendra Juda et ses frères,
        ${}^{3}Juda, de son union avec Thamar, engendra Pharès et Zara,
        \\Pharès engendra Esrom,
        \\Esrom engendra Aram,
        ${}^{4}Aram engendra Aminadab,
        \\Aminadab engendra Naassone,
        \\Naassone engendra Salmone,
        ${}^{5}Salmone, de son union avec Rahab, engendra Booz,
        \\Booz, de son union avec Ruth, engendra Jobed,
        \\Jobed engendra Jessé,
        ${}^{6}Jessé engendra le roi David.
        
           
       
        \\David, de son union avec la femme d’Ourias, engendra Salomon,
        ${}^{7}Salomon engendra Roboam,
        \\Roboam engendra Abia,
        \\Abia engendra Asa,
        ${}^{8}Asa engendra Josaphat,
        \\Josaphat engendra Joram,
        \\Joram engendra Ozias,
        ${}^{9}Ozias engendra Joatham,
        \\Joatham engendra Acaz,
        \\Acaz engendra Ézékias,
        ${}^{10}Ézékias engendra Manassé,
        \\Manassé engendra Amone,
        \\Amone engendra Josias,
        ${}^{11}Josias engendra Jékonias et ses frères à l’époque de l’exil à Babylone.
       
        ${}^{12}Après l’exil à Babylone, Jékonias engendra Salathiel,
        \\Salathiel engendra Zorobabel,
        ${}^{13}Zorobabel engendra Abioud,
        \\Abioud engendra Éliakim,
        \\Éliakim engendra Azor,
        ${}^{14}Azor engendra Sadok,
        \\Sadok engendra Akim,
        \\Akim engendra Élioud,
        ${}^{15}Élioud engendra Éléazar,
        \\Éléazar engendra Mattane,
        \\Mattane engendra Jacob,
        ${}^{16}Jacob engendra Joseph, l’époux de Marie,
        \\de laquelle fut engendré Jésus,
        \\que l’on appelle Christ.
       
${}^{17}Le nombre total des générations est donc : depuis Abraham jusqu’à David, quatorze générations ; depuis David jusqu’à l’exil à Babylone, quatorze générations ; depuis l’exil à Babylone jusqu’au Christ, quatorze générations.
${}^{18}Or, voici comment fut engendré Jésus Christ : Marie, sa mère, avait été accordée en mariage à Joseph ; avant qu’ils aient habité ensemble, elle fut enceinte par l’action de l’Esprit Saint. 
${}^{19}Joseph, son époux, qui était un homme juste, et ne voulait pas la dénoncer publiquement, décida de la renvoyer en secret. 
${}^{20}Comme il avait formé ce projet, voici que l’ange du Seigneur lui apparut en songe et lui dit : « Joseph, fils de David, ne crains pas de prendre chez toi Marie, ton épouse, puisque l’enfant qui est engendré en elle vient de l’Esprit Saint ; 
${}^{21}elle enfantera un fils, et tu lui donneras le nom de Jésus (c’est-à-dire : Le-Seigneur-sauve), car c’est lui qui sauvera son peuple de ses péchés. » 
${}^{22}Tout cela est arrivé pour que soit accomplie la parole du Seigneur prononcée par le prophète :
        ${}^{23}Voici que la Vierge concevra,
        \\et elle enfantera un fils ;
        \\on lui donnera le nom d’Emmanuel,
        \\qui se traduit : « Dieu-avec-nous ».
${}^{24}Quand Joseph se réveilla, il fit ce que l’ange du Seigneur lui avait prescrit : il prit chez lui son épouse, 
${}^{25}mais il ne s’unit pas à elle, jusqu’à ce qu’elle enfante un fils, auquel il donna le nom de Jésus.
      
         
      \bchapter{}
      \begin{verse}
${}^{1}Jésus était né à Bethléem en Judée, au temps du roi Hérode le Grand. Or, voici que des mages venus d’Orient arrivèrent à Jérusalem 
${}^{2}et demandèrent : « Où est le roi des Juifs qui vient de naître ? Nous avons vu son étoile à l’orient et nous sommes venus nous prosterner devant lui. » 
${}^{3}En apprenant cela, le roi Hérode fut bouleversé, et tout Jérusalem avec lui. 
${}^{4}Il réunit tous les grands prêtres et les scribes du peuple, pour leur demander où devait naître le Christ. 
${}^{5}Ils lui répondirent : « À Bethléem en Judée, car voici ce qui est écrit par le prophète :
        ${}^{6}Et toi, Bethléem, terre de Juda,
        \\tu n’es certes pas le dernier parmi les chefs-lieux de Juda,
        \\car de toi sortira un chef,
        \\qui sera le berger de mon peuple Israël. »
${}^{7}Alors Hérode convoqua les mages en secret pour leur faire préciser à quelle date l’étoile était apparue ; 
${}^{8}puis il les envoya à Bethléem, en leur disant : « Allez vous renseigner avec précision sur l’enfant. Et quand vous l’aurez trouvé, venez me l’annoncer pour que j’aille, moi aussi, me prosterner devant lui. » 
${}^{9}Après avoir entendu le roi, ils partirent.
      Et voici que l’étoile qu’ils avaient vue à l’orient les précédait, jusqu’à ce qu’elle vienne s’arrêter au-dessus de l’endroit où se trouvait l’enfant. 
${}^{10}Quand ils virent l’étoile, ils se réjouirent d’une très grande joie. 
${}^{11}Ils entrèrent dans la maison, ils virent l’enfant avec Marie sa mère ; et, tombant à ses pieds, ils se prosternèrent devant lui. Ils ouvrirent leurs coffrets, et lui offrirent leurs présents : de l’or, de l’encens et de la myrrhe. 
${}^{12}Mais, avertis en songe de ne pas retourner chez Hérode, ils regagnèrent leur pays par un autre chemin.
${}^{13}Après leur départ, voici que l’ange du Seigneur apparaît en songe à Joseph et lui dit : « Lève-toi ; prends l’enfant et sa mère, et fuis en Égypte. Reste là-bas jusqu’à ce que je t’avertisse, car Hérode va rechercher l’enfant pour le faire périr. » 
${}^{14}Joseph se leva ; dans la nuit, il prit l’enfant et sa mère, et se retira en Égypte, 
${}^{15}où il resta jusqu’à la mort d’Hérode, pour que soit accomplie la parole du Seigneur prononcée par le prophète :
        \\D’Égypte, j’ai appelé mon fils.
${}^{16}Alors Hérode, voyant que les mages s’étaient moqués de lui, entra dans une violente fureur. Il envoya tuer tous les enfants jusqu’à l’âge de deux ans à Bethléem et dans toute la région, d’après la date qu’il s’était fait préciser par les mages. 
${}^{17}Alors fut accomplie la parole prononcée par le prophète Jérémie :
        ${}^{18}Un cri s’élève dans Rama,
        \\pleurs et longue plainte :
        \\c’est Rachel qui pleure ses enfants
        \\et ne veut pas être consolée,
        \\car ils ne sont plus.
${}^{19}Après la mort d’Hérode, voici que l’ange du Seigneur apparaît en songe à Joseph en Égypte 
${}^{20}et lui dit : « Lève-toi ; prends l’enfant et sa mère, et pars pour le pays d’Israël, car ils sont morts, ceux qui en voulaient à la vie de l’enfant. » 
${}^{21}Joseph se leva, prit l’enfant et sa mère, et il entra dans le pays d’Israël. 
${}^{22}Mais, apprenant qu’Arkélaüs régnait sur la Judée à la place de son père Hérode, il eut peur de s’y rendre. Averti en songe, il se retira dans la région de Galilée 
${}^{23}et vint habiter dans une ville appelée Nazareth, pour que soit accomplie la parole dite par les prophètes :
        \\Il sera appelé Nazaréen.
      <h2 class="intertitle" id="d85e334966">1. Préparation de la mission de Jésus (3 – 4,11)</h2>
      
         
      \bchapter{}
      \begin{verse}
${}^{1}En ces jours-là, paraît Jean le Baptiste, qui proclame dans le désert de Judée : 
${}^{2}« Convertissez-vous, car le royaume des Cieux est tout proche. »
${}^{3}Jean est celui que désignait la parole prononcée par le prophète Isaïe :
        \\Voix de celui qui crie dans le désert :
        \\Préparez le chemin du Seigneur,
        \\rendez droits ses sentiers.
${}^{4}Lui, Jean, portait un vêtement de poils de chameau, et une ceinture de cuir autour des reins ; il avait pour nourriture des sauterelles et du miel sauvage. 
${}^{5}Alors Jérusalem, toute la Judée et toute la région du Jourdain se rendaient auprès de lui, 
${}^{6}et ils étaient baptisés par lui dans le Jourdain en reconnaissant leurs péchés.
${}^{7}Voyant beaucoup de pharisiens et de sadducéens se présenter à son baptême, il leur dit : « Engeance de vipères ! Qui vous a appris à fuir la colère qui vient ? 
${}^{8}Produisez donc un fruit digne de la conversion. 
${}^{9}N’allez pas dire en vous-mêmes : “Nous avons Abraham pour père” ; car, je vous le dis : des pierres que voici, Dieu peut faire surgir des enfants à Abraham. 
${}^{10}Déjà la cognée se trouve à la racine des arbres : tout arbre qui ne produit pas de bons fruits va être coupé et jeté au feu.
${}^{11}Moi, je vous baptise dans l’eau, en vue de la conversion. Mais celui qui vient derrière moi est plus fort que moi, et je ne suis pas digne de lui retirer ses sandales. Lui vous baptisera dans l’Esprit Saint et le feu. 
${}^{12}Il tient dans sa main la pelle à vanner, il va nettoyer son aire à battre le blé, et il amassera son grain dans le grenier ; quant à la paille, il la brûlera au feu qui ne s’éteint pas. »
${}^{13}Alors paraît Jésus. Il était venu de Galilée jusqu’au Jourdain auprès de Jean, pour être baptisé par lui. 
${}^{14}Jean voulait l’en empêcher et disait : « C’est moi qui ai besoin d’être baptisé par toi, et c’est toi qui viens à moi ! » 
${}^{15}Mais Jésus lui répondit : « Laisse faire pour le moment, car il convient que nous accomplissions ainsi toute justice. » Alors Jean le laisse faire. 
${}^{16}Dès que Jésus fut baptisé, il remonta de l’eau, et voici que les cieux s’ouvrirent : il vit l’Esprit de Dieu descendre comme une colombe et venir sur lui. 
${}^{17}Et des cieux, une voix disait : « Celui-ci est mon Fils bien-aimé, en qui je trouve ma joie. »
      
         
      \bchapter{}
      \begin{verse}
${}^{1}Alors Jésus fut conduit au désert par l’Esprit pour être tenté par le diable. 
${}^{2}Après avoir jeûné quarante jours et quarante nuits, il eut faim. 
${}^{3}Le tentateur s’approcha et lui dit : « Si tu es Fils de Dieu, ordonne que ces pierres deviennent des pains. » 
${}^{4}Mais Jésus répondit : « Il est écrit :
        \\L’homme ne vit pas seulement de pain,
        \\mais de toute parole qui sort de la bouche de Dieu. »
${}^{5}Alors le diable l’emmène à la Ville sainte, le place au sommet du Temple 
${}^{6}et lui dit : « Si tu es Fils de Dieu, jette-toi en bas ; car il est écrit :
        \\Il donnera pour toi des ordres à ses anges,
        \\et : Ils te porteront sur leurs mains,
        \\de peur que ton pied ne heurte une pierre. »
${}^{7}Jésus lui déclara : « Il est encore écrit :
        \\Tu ne mettras pas à l’épreuve le Seigneur ton Dieu. »
${}^{8}Le diable l’emmène encore sur une très haute montagne et lui montre tous les royaumes du monde et leur gloire. 
${}^{9}Il lui dit : « Tout cela, je te le donnerai, si, tombant à mes pieds, tu te prosternes devant moi. » 
${}^{10}Alors, Jésus lui dit : « Arrière, Satan ! car il est écrit :
        \\C’est le Seigneur ton Dieu que tu adoreras,
        \\à lui seul tu rendras un culte. »
${}^{11}Alors le diable le quitte. Et voici que des anges s’approchèrent, et ils le servaient.
      <h2 class="intertitle" id="d85e335314">2. Les débuts en Galilée (4,12-25)</h2>
${}^{12}Quand Jésus apprit l’arrestation de Jean le Baptiste, il se retira en Galilée. 
${}^{13}Il quitta Nazareth et vint habiter à Capharnaüm, ville située au bord de la mer de Galilée, dans les territoires de Zabulon et de Nephtali. 
${}^{14}C’était pour que soit accomplie la parole prononcée par le prophète Isaïe :
        ${}^{15}Pays de Zabulon et pays de Nephtali,
        \\route de la mer et pays au-delà du Jourdain,
        \\Galilée des nations !
        ${}^{16}Le peuple qui habitait dans les ténèbres
        \\a vu une grande lumière.
        \\Sur ceux qui habitaient dans le pays et l’ombre de la mort,
        \\une lumière s’est levée.
${}^{17}À partir de ce moment, Jésus commença à proclamer : « Convertissez-vous, car le royaume des Cieux est tout proche. »
${}^{18}Comme il marchait le long de la mer de Galilée, il vit deux frères, Simon, appelé Pierre, et son frère André, qui jetaient leurs filets dans la mer ; car c’étaient des pêcheurs. 
${}^{19}Jésus leur dit : « Venez à ma suite, et je vous ferai pêcheurs d’hommes. » 
${}^{20}Aussitôt, laissant leurs filets, ils le suivirent. 
${}^{21}De là, il avança et il vit deux autres frères, Jacques, fils de Zébédée, et son frère Jean, qui étaient dans la barque avec leur père, en train de réparer leurs filets. Il les appela. 
${}^{22}Aussitôt, laissant la barque et leur père, ils le suivirent.
${}^{23}Jésus parcourait toute la Galilée ; il enseignait dans leurs synagogues, proclamait l’Évangile du Royaume, guérissait toute maladie et toute infirmité dans le peuple.
${}^{24}Sa renommée se répandit dans toute la Syrie. On lui amena tous ceux qui souffraient, atteints de maladies et de tourments de toutes sortes : possédés, épileptiques, paralysés. Et il les guérit. 
${}^{25}De grandes foules le suivirent, venues de la Galilée, de la Décapole, de Jérusalem, de la Judée, et de l’autre côté du Jourdain.
      <h2 class="intertitle" id="d85e335490">3. Premier discours : Sermon sur la montagne (5 – 7)</h2>
      
         
      \bchapter{}
      \begin{verse}
${}^{1}Voyant les foules, Jésus gravit la montagne. Il s’assit, et ses disciples s’approchèrent de lui. 
${}^{2}Alors, ouvrant la bouche, il les enseignait. Il disait :
        ${}^{3}« Heureux les pauvres de cœur,
        \\car le royaume des Cieux est à eux.
        ${}^{4}Heureux ceux qui pleurent,
        \\car ils seront consolés.
        ${}^{5}Heureux les doux,
        \\car ils recevront la terre en héritage.
        ${}^{6}Heureux ceux qui ont faim et soif de la justice,
        \\car ils seront rassasiés.
        ${}^{7}Heureux les miséricordieux,
        \\car ils obtiendront miséricorde.
        ${}^{8}Heureux les cœurs purs,
        \\car ils verront Dieu.
        ${}^{9}Heureux les artisans de paix,
        \\car ils seront appelés fils de Dieu.
        ${}^{10}Heureux ceux qui sont persécutés pour la justice,
        \\car le royaume des Cieux est à eux.
${}^{11}Heureux êtes-vous si l’on vous insulte, si l’on vous persécute et si l’on dit faussement toute sorte de mal contre vous, à cause de moi. 
${}^{12}Réjouissez-vous, soyez dans l’allégresse, car votre récompense est grande dans les cieux ! C’est ainsi qu’on a persécuté les prophètes qui vous ont précédés.
${}^{13}« Vous êtes le sel de la terre. Mais si le sel devient fade, comment lui rendre de la saveur ? Il ne vaut plus rien : on le jette dehors et il est piétiné par les gens.
${}^{14}Vous êtes la lumière du monde. Une ville située sur une montagne ne peut être cachée. 
${}^{15}Et l’on n’allume pas une lampe pour la mettre sous le boisseau ; on la met sur le lampadaire, et elle brille pour tous ceux qui sont dans la maison. 
${}^{16}De même, que votre lumière brille devant les hommes : alors, voyant ce que vous faites de bien, ils rendront gloire à votre Père qui est aux cieux.
${}^{17}« Ne pensez pas que je sois venu abolir la Loi ou les Prophètes : je ne suis pas venu abolir, mais accomplir. 
${}^{18}Amen, je vous le dis : Avant que le ciel et la terre disparaissent, pas un seul iota, pas un seul trait ne disparaîtra de la Loi jusqu’à ce que tout se réalise. 
${}^{19}Donc, celui qui rejettera un seul de ces plus petits commandements, et qui enseignera aux hommes à faire ainsi, sera déclaré le plus petit dans le royaume des Cieux. Mais celui qui les observera et les enseignera, celui-là sera déclaré grand dans le royaume des Cieux. 
${}^{20}Je vous le dis en effet : Si votre justice ne surpasse pas celle des scribes et des pharisiens, vous n’entrerez pas dans le royaume des Cieux.
${}^{21}« Vous avez appris qu’il a été dit aux anciens : Tu ne commettras pas de meurtre, et si quelqu’un commet un meurtre, il devra passer en jugement. 
${}^{22}Eh bien ! moi, je vous dis : Tout homme qui se met en colère contre son frère devra passer en jugement. Si quelqu’un insulte son frère, il devra passer devant le tribunal. Si quelqu’un le traite de fou, il sera passible de la géhenne de feu. 
${}^{23}Donc, lorsque tu vas présenter ton offrande à l’autel, si, là, tu te souviens que ton frère a quelque chose contre toi, 
${}^{24}laisse ton offrande, là, devant l’autel, va d’abord te réconcilier avec ton frère, et ensuite viens présenter ton offrande. 
${}^{25}Mets-toi vite d’accord avec ton adversaire pendant que tu es en chemin avec lui, pour éviter que ton adversaire ne te livre au juge, le juge au garde, et qu’on ne te jette en prison. 
${}^{26}Amen, je te le dis : tu n’en sortiras pas avant d’avoir payé jusqu’au dernier sou.
${}^{27}Vous avez appris qu’il a été dit : Tu ne commettras pas d’adultère. 
${}^{28}Eh bien ! moi, je vous dis : Tout homme qui regarde une femme avec convoitise a déjà commis l’adultère avec elle dans son cœur. 
${}^{29}Si ton œil droit entraîne ta chute, arrache-le et jette-le loin de toi, car mieux vaut pour toi perdre un de tes membres que d’avoir ton corps tout entier jeté dans la géhenne. 
${}^{30}Et si ta main droite entraîne ta chute, coupe-la et jette-la loin de toi, car mieux vaut pour toi perdre un de tes membres que d’avoir ton corps tout entier qui s’en aille dans la géhenne.
${}^{31}Il a été dit également : Si quelqu’un renvoie sa femme, qu’il lui donne un acte de répudiation. 
${}^{32}Eh bien ! moi, je vous dis : Tout homme qui renvoie sa femme, sauf en cas d’union illégitime, la pousse à l’adultère ; et si quelqu’un épouse une femme renvoyée, il est adultère.
${}^{33}Vous avez encore appris qu’il a été dit aux anciens : Tu ne manqueras pas à tes serments, mais tu t’acquitteras de tes serments envers le Seigneur. 
${}^{34}Eh bien ! moi, je vous dis de ne pas jurer du tout, ni par le ciel, car c’est le trône de Dieu, 
${}^{35}ni par la terre, car elle est son marchepied, ni par Jérusalem, car elle est la Ville du grand Roi. 
${}^{36}Et ne jure pas non plus sur ta tête, parce que tu ne peux pas rendre un seul de tes cheveux blanc ou noir. 
${}^{37}Que votre parole soit “oui”, si c’est “oui”, “non”, si c’est “non”. Ce qui est en plus vient du Mauvais.
${}^{38}Vous avez appris qu’il a été dit : Œil pour œil, et dent pour dent. 
${}^{39}Eh bien ! moi, je vous dis de ne pas riposter au méchant ; mais si quelqu’un te gifle sur la joue droite, tends-lui encore l’autre. 
${}^{40}Et si quelqu’un veut te poursuivre en justice et prendre ta tunique, laisse-lui encore ton manteau. 
${}^{41}Et si quelqu’un te réquisitionne pour faire mille pas, fais-en deux mille avec lui. 
${}^{42}À qui te demande, donne ; à qui veut t’emprunter, ne tourne pas le dos !
${}^{43}Vous avez appris qu’il a été dit : Tu aimeras ton prochain et tu haïras ton ennemi. 
${}^{44}Eh bien ! moi, je vous dis : Aimez vos ennemis, et priez pour ceux qui vous persécutent, 
${}^{45}afin d’être vraiment les fils de votre Père qui est aux cieux ; car il fait lever son soleil sur les méchants et sur les bons, il fait tomber la pluie sur les justes et sur les injustes. 
${}^{46}En effet, si vous aimez ceux qui vous aiment, quelle récompense méritez-vous ? Les publicains eux-mêmes n’en font-ils pas autant ? 
${}^{47}Et si vous ne saluez que vos frères, que faites-vous d’extraordinaire ? Les païens eux-mêmes n’en font-ils pas autant ? 
${}^{48}Vous donc, vous serez parfaits comme votre Père céleste est parfait.
      
         
      \bchapter{}
      \begin{verse}
${}^{1} « Ce que vous faites pour devenir des justes, évitez de l’accomplir devant les hommes pour vous faire remarquer. Sinon, il n’y a pas de récompense pour vous auprès de votre Père qui est aux cieux.
${}^{2}Ainsi, quand tu fais l’aumône, ne fais pas sonner la trompette devant toi, comme les hypocrites qui se donnent en spectacle dans les synagogues et dans les rues, pour obtenir la gloire qui vient des hommes. Amen, je vous le déclare : ceux-là ont reçu leur récompense. 
${}^{3}Mais toi, quand tu fais l’aumône, que ta main gauche ignore ce que fait ta main droite, 
${}^{4}afin que ton aumône reste dans le secret ; ton Père qui voit dans le secret te le rendra.
${}^{5}Et quand vous priez, ne soyez pas comme les hypocrites : ils aiment à se tenir debout dans les synagogues et aux carrefours pour bien se montrer aux hommes quand ils prient. Amen, je vous le déclare : ceux-là ont reçu leur récompense. 
${}^{6}Mais toi, quand tu pries, retire-toi dans ta pièce la plus retirée, ferme la porte, et prie ton Père qui est présent dans le secret ; ton Père qui voit dans le secret te le rendra.
${}^{7}Lorsque vous priez, ne rabâchez pas comme les païens : ils s’imaginent qu’à force de paroles ils seront exaucés. 
${}^{8}Ne les imitez donc pas, car votre Père sait de quoi vous avez besoin, avant même que vous l’ayez demandé. 
${}^{9}Vous donc, priez ainsi :
        \\Notre Père, qui es aux cieux,
        \\que ton nom soit sanctifié,
        ${}^{10}que ton règne vienne,
        \\que ta volonté soit faite
        \\sur la terre comme au ciel.
        ${}^{11}Donne-nous aujourd’hui notre pain de ce jour.
        ${}^{12}Remets-nous nos dettes,
        \\comme nous-mêmes nous remettons leurs dettes
        \\à nos débiteurs.
        ${}^{13}Et ne nous laisse pas entrer en tentation,
        \\mais délivre-nous du Mal.
${}^{14}Car, si vous pardonnez aux hommes leurs fautes, votre Père céleste vous pardonnera aussi. 
${}^{15}Mais si vous ne pardonnez pas aux hommes, votre Père non plus ne pardonnera pas vos fautes.
${}^{16}Et quand vous jeûnez, ne prenez pas un air abattu, comme les hypocrites : ils prennent une mine défaite pour bien montrer aux hommes qu’ils jeûnent. Amen, je vous le déclare : ceux-là ont reçu leur récompense. 
${}^{17}Mais toi, quand tu jeûnes, parfume-toi la tête et lave-toi le visage ; 
${}^{18}ainsi, ton jeûne ne sera pas connu des hommes, mais seulement de ton Père qui est présent au plus secret ; ton Père qui voit au plus secret te le rendra.
${}^{19}« Ne vous faites pas de trésors sur la terre, là où les mites et les vers les dévorent, où les voleurs percent les murs pour voler. 
${}^{20}Mais faites-vous des trésors dans le ciel, là où il n’y a pas de mites ni de vers qui dévorent, pas de voleurs qui percent les murs pour voler. 
${}^{21}Car là où est ton trésor, là aussi sera ton cœur.
${}^{22}La lampe du corps, c’est l’œil. Donc, si ton œil est limpide, ton corps tout entier sera dans la lumière ; 
${}^{23}mais si ton œil est mauvais, ton corps tout entier sera dans les ténèbres. Si donc la lumière qui est en toi est ténèbres, comme elles seront grandes, les ténèbres !
${}^{24}Nul ne peut servir deux maîtres : ou bien il haïra l’un et aimera l’autre, ou bien il s’attachera à l’un et méprisera l’autre. Vous ne pouvez pas servir à la fois Dieu et l’Argent.
${}^{25}C’est pourquoi je vous dis : Ne vous souciez pas, pour votre vie, de ce que vous mangerez, ni, pour votre corps, de quoi vous le vêtirez. La vie ne vaut-elle pas plus que la nourriture, et le corps plus que les vêtements ? 
${}^{26}Regardez les oiseaux du ciel : ils ne font ni semailles ni moisson, ils n’amassent pas dans des greniers, et votre Père céleste les nourrit. Vous-mêmes, ne valez-vous pas beaucoup plus qu’eux ? 
${}^{27}Qui d’entre vous, en se faisant du souci, peut ajouter une coudée à la longueur de sa vie ?
${}^{28}Et au sujet des vêtements, pourquoi se faire tant de souci ? Observez comment poussent les lis des champs : ils ne travaillent pas, ils ne filent pas. 
${}^{29}Or je vous dis que Salomon lui-même, dans toute sa gloire, n’était pas habillé comme l’un d’entre eux. 
${}^{30}Si Dieu donne un tel vêtement à l’herbe des champs, qui est là aujourd’hui, et qui demain sera jetée au feu, ne fera-t-il pas bien davantage pour vous, hommes de peu de foi ?
${}^{31}Ne vous faites donc pas tant de souci ; ne dites pas : “Qu’allons-nous manger ?” ou bien : “Qu’allons-nous boire ?” ou encore : “Avec quoi nous habiller ?” 
${}^{32}Tout cela, les païens le recherchent. Mais votre Père céleste sait que vous en avez besoin. 
${}^{33}Cherchez d’abord le royaume de Dieu et sa justice, et tout cela vous sera donné par surcroît. 
${}^{34}Ne vous faites pas de souci pour demain : demain aura souci de lui-même ; à chaque jour suffit sa peine.
      
         
      \bchapter{}
      \begin{verse}
${}^{1}« Ne jugez pas, pour ne pas être jugés ; 
${}^{2}de la manière dont vous jugez, vous serez jugés ; de la mesure dont vous mesurez, on vous mesurera. 
${}^{3}Quoi ! tu regardes la paille dans l’œil de ton frère ; et la poutre qui est dans ton œil, tu ne la remarques pas ? 
${}^{4}Ou encore : Comment vas-tu dire à ton frère : “Laisse-moi enlever la paille de ton œil”, alors qu’il y a une poutre dans ton œil à toi ? 
${}^{5}Hypocrite ! Enlève d’abord la poutre de ton œil ; alors tu verras clair pour enlever la paille qui est dans l’œil de ton frère.
      
         
${}^{6}« Ne donnez pas aux chiens ce qui est sacré ; ne jetez pas vos perles aux pourceaux, de peur qu’ils ne les piétinent, puis se retournent pour vous déchirer.
${}^{7}« Demandez, on vous donnera ; cherchez, vous trouverez ; frappez, on vous ouvrira. 
${}^{8}En effet, quiconque demande reçoit ; qui cherche trouve ; à qui frappe, on ouvrira. 
${}^{9}Ou encore : lequel d’entre vous donnera une pierre à son fils quand il lui demande du pain ? 
${}^{10}ou bien lui donnera un serpent, quand il lui demande un poisson ? 
${}^{11}Si donc vous, qui êtes mauvais, vous savez donner de bonnes choses à vos enfants, combien plus votre Père qui est aux cieux donnera-t-il de bonnes choses à ceux qui les lui demandent !
${}^{12}« Donc, tout ce que vous voudriez que les autres fassent pour vous, faites-le pour eux, vous aussi : voilà ce que disent la Loi et les Prophètes.
${}^{13}« Entrez par la porte étroite. Elle est grande, la porte, il est large, le chemin qui conduit à la perdition ; et ils sont nombreux, ceux qui s’y engagent. 
${}^{14}Mais elle est étroite, la porte, il est resserré, le chemin qui conduit à la vie ; et ils sont peu nombreux, ceux qui le trouvent.
${}^{15}Méfiez-vous des faux prophètes qui viennent à vous déguisés en brebis, alors qu’au-dedans ce sont des loups voraces. 
${}^{16}C’est à leurs fruits que vous les reconnaîtrez. Va-t-on cueillir du raisin sur des épines, ou des figues sur des chardons ? 
${}^{17}C’est ainsi que tout arbre bon donne de beaux fruits, et que l’arbre qui pourrit donne des fruits mauvais. 
${}^{18}Un arbre bon ne peut pas donner des fruits mauvais, ni un arbre qui pourrit donner de beaux fruits. 
${}^{19}Tout arbre qui ne donne pas de beaux fruits est coupé et jeté au feu. 
${}^{20}Donc, c’est à leurs fruits que vous les reconnaîtrez.
${}^{21}Ce n’est pas en me disant : “Seigneur, Seigneur !” qu’on entrera dans le royaume des Cieux, mais c’est en faisant la volonté de mon Père qui est aux cieux. 
${}^{22}Ce jour-là, beaucoup me diront : “Seigneur, Seigneur, n’est-ce pas en ton nom que nous avons prophétisé, en ton nom que nous avons expulsé les démons, en ton nom que nous avons fait beaucoup de miracles ?” 
${}^{23}Alors je leur déclarerai : “Je ne vous ai jamais connus. Écartez-vous de moi, vous qui commettez le mal !”
${}^{24}Ainsi, celui qui entend les paroles que je dis là et les met en pratique est comparable à un homme prévoyant qui a construit sa maison sur le roc. 
${}^{25}La pluie est tombée, les torrents ont dévalé, les vents ont soufflé et se sont abattus sur cette maison ; la maison ne s’est pas écroulée, car elle était fondée sur le roc. 
${}^{26}Et celui qui entend de moi ces paroles sans les mettre en pratique est comparable à un homme insensé qui a construit sa maison sur le sable. 
${}^{27}La pluie est tombée, les torrents ont dévalé, les vents ont soufflé, ils sont venus battre cette maison ; la maison s’est écroulée, et son écroulement a été complet. »
${}^{28}Lorsque Jésus eut terminé ce discours, les foules restèrent frappées de son enseignement, 
${}^{29}car il les enseignait en homme qui a autorité, et non pas comme leurs scribes.
      <h2 class="intertitle" id="d85e336445">4. Les miracles, signes de l’avènement du Royaume (8 – 9,35)</h2>
      
         
      \bchapter{}
      \begin{verse}
${}^{1}Lorsque Jésus descendit de la montagne, des foules nombreuses le suivirent. 
${}^{2}Et voici qu’un lépreux s’approcha, se prosterna devant lui et dit : « Seigneur, si tu le veux, tu peux me purifier. » 
${}^{3}Jésus étendit la main, le toucha et lui dit : « Je le veux, sois purifié. » Et aussitôt il fut purifié de sa lèpre. 
${}^{4}Jésus lui dit : « Attention, ne dis rien à personne, mais va te montrer au prêtre. Et donne l’offrande que Moïse a prescrite : ce sera pour les gens un témoignage. »
      
         
${}^{5}Comme Jésus était entré à Capharnaüm, un centurion s’approcha de lui et le supplia : 
${}^{6}« Seigneur, mon serviteur est couché, à la maison, paralysé, et il souffre terriblement. » 
${}^{7}Jésus lui dit : « Je vais aller moi-même le guérir. » 
${}^{8}Le centurion reprit : « Seigneur, je ne suis pas digne que tu entres sous mon toit, mais dis seulement une parole et mon serviteur sera guéri. 
${}^{9}Moi-même qui suis soumis à une autorité, j’ai des soldats sous mes ordres ; à l’un, je dis : “Va”, et il va ; à un autre : “Viens”, et il vient, et à mon esclave : “Fais ceci”, et il le fait. » 
${}^{10}À ces mots, Jésus fut dans l’admiration et dit à ceux qui le suivaient : « Amen, je vous le déclare, chez personne en Israël, je n’ai trouvé une telle foi. 
${}^{11}Aussi je vous le dis : Beaucoup viendront de l’orient et de l’occident et prendront place avec Abraham, Isaac et Jacob au festin du royaume des Cieux, 
${}^{12}mais les fils du Royaume seront jetés dans les ténèbres du dehors ; là, il y aura des pleurs et des grincements de dents. » 
${}^{13}Et Jésus dit au centurion : « Rentre chez toi, que tout se passe pour toi selon ta foi. » Et, à l’heure même, le serviteur fut guéri.
${}^{14}Comme Jésus entrait chez Pierre, dans sa maison, il vit sa belle-mère couchée avec de la fièvre. 
${}^{15}Il lui toucha la main, et la fièvre la quitta. Elle se leva, et elle le servait.
${}^{16}Le soir venu, on présenta à Jésus beaucoup de possédés. D’une parole, il expulsa les esprits et, tous ceux qui étaient atteints d’un mal, il les guérit, 
${}^{17}pour que soit accomplie la parole prononcée par le prophète Isaïe :
        \\Il a pris nos souffrances,
        \\il a porté nos maladies.
${}^{18}Jésus, voyant une foule autour de lui, donna l’ordre de partir vers l’autre rive. 
${}^{19}Un scribe s’approcha et lui dit : « Maître, je te suivrai partout où tu iras. » 
${}^{20}Mais Jésus lui déclara : « Les renards ont des terriers, les oiseaux du ciel ont des nids ; mais le Fils de l’homme n’a pas d’endroit où reposer la tête. » 
${}^{21}Un autre de ses disciples lui dit : « Seigneur, permets-moi d’aller d’abord enterrer mon père. » 
${}^{22}Jésus lui dit : « Suis-moi, et laisse les morts enterrer leurs morts. »
${}^{23}Comme Jésus montait dans la barque, ses disciples le suivirent. 
${}^{24}Et voici que la mer devint tellement agitée que la barque était recouverte par les vagues. Mais lui dormait. 
${}^{25}Les disciples s’approchèrent et le réveillèrent en disant : « Seigneur, sauve-nous ! Nous sommes perdus. » 
${}^{26}Mais il leur dit : « Pourquoi êtes-vous si craintifs, hommes de peu de foi ? » Alors, Jésus, debout, menaça les vents et la mer, et il se fit un grand calme. 
${}^{27}Les gens furent saisis d’étonnement et disaient : « Quel est donc celui-ci, pour que même les vents et la mer lui obéissent ? »
${}^{28}Comme Jésus arrivait sur l’autre rive, dans le pays des Gadaréniens, deux possédés sortirent d’entre les tombes à sa rencontre ; ils étaient si agressifs que personne ne pouvait passer par ce chemin. 
${}^{29}Et voilà qu’ils se mirent à crier : « Que nous veux-tu, Fils de Dieu ? Es-tu venu pour nous tourmenter avant le moment fixé ? » 
${}^{30}Or, il y avait au loin un grand troupeau de porcs qui cherchait sa nourriture. 
${}^{31}Les démons suppliaient Jésus : « Si tu nous expulses, envoie-nous dans le troupeau de porcs. » 
${}^{32}Il leur répondit : « Allez. » Ils sortirent et ils s’en allèrent dans les porcs ; et voilà que, du haut de la falaise, tout le troupeau se précipita dans la mer, et les porcs moururent dans les flots. 
${}^{33}Les gardiens prirent la fuite et s’en allèrent dans la ville annoncer tout cela, et en particulier ce qui était arrivé aux possédés. 
${}^{34}Et voilà que toute la ville sortit à la rencontre de Jésus ; et lorsqu’ils le virent, les gens le supplièrent de partir de leur territoire.
      
         
      \bchapter{}
      \begin{verse}
${}^{1}Jésus monta en barque, refit la traversée, et alla dans sa ville de Capharnaüm. 
${}^{2}Et voici qu’on lui présenta un paralysé, couché sur une civière. Voyant leur foi, Jésus dit au paralysé : « Confiance, mon enfant, tes péchés sont pardonnés. » 
${}^{3}Et voici que certains parmi les scribes se disaient : « Celui-là blasphème. » 
${}^{4}Mais Jésus, connaissant leurs pensées, demanda : « Pourquoi avez-vous des pensées mauvaises ? 
${}^{5}En effet, qu’est-ce qui est le plus facile ? Dire : “Tes péchés sont pardonnés”, ou bien dire : “Lève-toi et marche” ? 
${}^{6}Eh bien ! pour que vous sachiez que le Fils de l’homme a le pouvoir, sur la terre, de pardonner les péchés… – Jésus s’adressa alors au paralysé – lève-toi, prends ta civière, et rentre dans ta maison. » 
${}^{7}Il se leva et rentra dans sa maison. 
${}^{8}Voyant cela, les foules furent saisies de crainte, et rendirent gloire à Dieu qui a donné un tel pouvoir aux hommes.
      
         
${}^{9}Jésus partit de là et vit, en passant, un homme, du nom de Matthieu, assis à son bureau de collecteur d’impôts. Il lui dit : « Suis-moi. » L’homme se leva et le suivit.
${}^{10}Comme Jésus était à table à la maison, voici que beaucoup de publicains (c’est-à-dire des collecteurs d’impôts) et beaucoup de pécheurs vinrent prendre place avec lui et ses disciples. 
${}^{11}Voyant cela, les pharisiens disaient à ses disciples : « Pourquoi votre maître mange-t-il avec les publicains et les pécheurs ? » 
${}^{12}Jésus, qui avait entendu, déclara : « Ce ne sont pas les gens bien portants qui ont besoin du médecin, mais les malades. 
${}^{13}Allez apprendre ce que signifie : Je veux la miséricorde, non le sacrifice. En effet, je ne suis pas venu appeler des justes, mais des pécheurs. »
${}^{14}Alors les disciples de Jean le Baptiste s’approchent de Jésus en disant : « Pourquoi, alors que nous et les pharisiens, nous jeûnons, tes disciples ne jeûnent-ils pas ? » 
${}^{15}Jésus leur répondit : « Les invités de la noce pourraient-ils donc être en deuil pendant le temps où l’Époux est avec eux ? Mais des jours viendront où l’Époux leur sera enlevé ; alors ils jeûneront. 
${}^{16}Et personne ne pose une pièce d’étoffe neuve sur un vieux vêtement, car le morceau ajouté tire sur le vêtement, et la déchirure s’agrandit. 
${}^{17}Et on ne met pas du vin nouveau dans de vieilles outres ; autrement, les outres éclatent, le vin se répand, et les outres sont perdues. Mais on met le vin nouveau dans des outres neuves, et le tout se conserve. »
${}^{18}Tandis que Jésus leur parlait ainsi, voilà qu’un notable s’approcha. Il se prosternait devant lui en disant : « Ma fille est morte à l’instant ; mais viens lui imposer la main, et elle vivra. » 
${}^{19}Jésus se leva et le suivit, ainsi que ses disciples.
${}^{20}Et voici qu’une femme souffrant d’hémorragies depuis douze ans s’approcha par-derrière et toucha la frange de son vêtement. 
${}^{21}Car elle se disait en elle-même : « Si je parviens seulement à toucher son vêtement, je serai sauvée. » 
${}^{22}Jésus se retourna et, la voyant, lui dit : « Confiance, ma fille ! Ta foi t’a sauvée. » Et, à l’heure même, la femme fut sauvée.
${}^{23}Jésus, arrivé à la maison du notable, vit les joueurs de flûte et la foule qui s’agitait bruyamment. Il dit alors : 
${}^{24}« Retirez-vous. La jeune fille n’est pas morte : elle dort. » Mais on se moquait de lui. 
${}^{25}Quand la foule fut mise dehors, il entra, lui saisit la main, et la jeune fille se leva. 
${}^{26}Et la nouvelle se répandit dans toute la région.
${}^{27}Tandis que Jésus s’en allait, deux aveugles le suivirent, en criant : « Prends pitié de nous, fils de David ! » 
${}^{28}Quand il fut entré dans la maison, les aveugles s’approchèrent de lui, et Jésus leur dit : « Croyez-vous que je peux faire cela ? » Ils lui répondirent : « Oui, Seigneur. » 
${}^{29}Alors il leur toucha les yeux, en disant : « Que tout se passe pour vous selon votre foi ! » 
${}^{30}Leurs yeux s’ouvrirent, et Jésus leur dit avec fermeté : « Attention ! que personne ne le sache ! » 
${}^{31}Mais, une fois sortis, ils parlèrent de lui dans toute la région.
${}^{32}Ils sortirent donc, et voici qu’on présenta à Jésus un possédé qui était sourd-muet. 
${}^{33}Lorsque le démon eut été expulsé, le sourd-muet se mit à parler. Les foules furent dans l’admiration, et elles disaient : « Jamais rien de pareil ne s’est vu en Israël ! » 
${}^{34}Mais les pharisiens disaient : « C’est par le chef des démons qu’il expulse les démons. »
${}^{35}Jésus parcourait toutes les villes et tous les villages, enseignant dans leurs synagogues, proclamant l’Évangile du Royaume et guérissant toute maladie et toute infirmité.
      <h2 class="intertitle" id="d85e336997">1. Les foules et leurs pasteurs (9,36 – 10,4)</h2>
${}^{36}Voyant les foules, Jésus fut saisi de compassion envers elles parce qu’elles étaient désemparées et abattues comme des brebis sans berger. 
${}^{37}Il dit alors à ses disciples : « La moisson est abondante, mais les ouvriers sont peu nombreux. 
${}^{38}Priez donc le maître de la moisson d’envoyer des ouvriers pour sa moisson. »
      
         
      \bchapter{}
      \begin{verse}
${}^{1}Alors Jésus appela ses douze disciples et leur donna le pouvoir d’expulser les esprits impurs et de guérir toute maladie et toute infirmité. 
${}^{2}Voici les noms des douze Apôtres : le premier, Simon, nommé Pierre ; André son frère ; Jacques, fils de Zébédée, et Jean son frère ; 
${}^{3}Philippe et Barthélemy ; Thomas et Matthieu le publicain ; Jacques, fils d’Alphée, et Thaddée ; 
${}^{4}Simon le Zélote et Judas l’Iscariote, celui-là même qui le livra.
      
         
      <h2 class="intertitle" id="d85e337057">2. Deuxième discours : envoi en mission (10,5 – 11,1)</h2>
${}^{5}Ces douze, Jésus les envoya en mission avec les instructions suivantes : « Ne prenez pas le chemin qui mène vers les nations païennes et n’entrez dans aucune ville des Samaritains. 
${}^{6}Allez plutôt vers les brebis perdues de la maison d’Israël. 
${}^{7}Sur votre route, proclamez que le royaume des Cieux est tout proche. 
${}^{8}Guérissez les malades, ressuscitez les morts, purifiez les lépreux, expulsez les démons. Vous avez reçu gratuitement : donnez gratuitement.
${}^{9}Ne vous procurez ni or ni argent, ni monnaie de cuivre à mettre dans vos ceintures, 
${}^{10}ni sac pour la route, ni tunique de rechange, ni sandales, ni bâton. L’ouvrier, en effet, mérite sa nourriture.
${}^{11}Dans chaque ville ou village où vous entrerez, informez-vous pour savoir qui est digne de vous accueillir, et restez là jusqu’à votre départ. 
${}^{12}En entrant dans la maison, saluez ceux qui l’habitent. 
${}^{13}Si cette maison en est digne, que votre paix vienne sur elle. Si elle n’en est pas digne, que votre paix retourne vers vous. 
${}^{14}Si l’on ne vous accueille pas et si l’on n’écoute pas vos paroles, sortez de cette maison ou de cette ville, et secouez la poussière de vos pieds. 
${}^{15}Amen, je vous le dis : au jour du Jugement, le pays de Sodome et de Gomorrhe sera traité moins sévèrement que cette ville.
${}^{16}« Voici que moi, je vous envoie comme des brebis au milieu des loups. Soyez donc prudents comme les serpents, et candides comme les colombes.
${}^{17}Méfiez-vous des hommes : ils vous livreront aux tribunaux et vous flagelleront dans leurs synagogues. 
${}^{18}Vous serez conduits devant des gouverneurs et des rois à cause de moi : il y aura là un témoignage pour eux et pour les païens. 
${}^{19}Quand on vous livrera, ne vous inquiétez pas de savoir ce que vous direz ni comment vous le direz : ce que vous aurez à dire vous sera donné à cette heure-là. 
${}^{20}Car ce n’est pas vous qui parlerez, c’est l’Esprit de votre Père qui parlera en vous. 
${}^{21}Le frère livrera son frère à la mort, et le père, son enfant ; les enfants se dresseront contre leurs parents et les feront mettre à mort. 
${}^{22}Vous serez détestés de tous à cause de mon nom ; mais celui qui aura persévéré jusqu’à la fin, celui-là sera sauvé. 
${}^{23}Quand on vous persécutera dans une ville, fuyez dans une autre. Amen, je vous le dis : vous n’aurez pas fini de passer dans toutes les villes d’Israël quand le Fils de l’homme viendra.
${}^{24}Le disciple n’est pas au-dessus de son maître, ni le serviteur au-dessus de son seigneur. 
${}^{25}Il suffit que le disciple soit comme son maître, et le serviteur, comme son seigneur. Si les gens ont traité de Béelzéboul le maître de maison, ce sera bien pire pour ceux de sa maison.
${}^{26}Ne craignez donc pas ces gens-là ; rien n’est voilé qui ne sera dévoilé, rien n’est caché qui ne sera connu. 
${}^{27}Ce que je vous dis dans les ténèbres, dites-le en pleine lumière ; ce que vous entendez au creux de l’oreille, proclamez-le sur les toits. 
${}^{28}Ne craignez pas ceux qui tuent le corps sans pouvoir tuer l’âme ; craignez plutôt celui qui peut faire périr dans la géhenne l’âme aussi bien que le corps. 
${}^{29}Deux moineaux ne sont-ils pas vendus pour un sou ? Or, pas un seul ne tombe à terre sans que votre Père le veuille. 
${}^{30}Quant à vous, même les cheveux de votre tête sont tous comptés. 
${}^{31}Soyez donc sans crainte : vous valez bien plus qu’une multitude de moineaux.
${}^{32}Quiconque se déclarera pour moi devant les hommes, moi aussi je me déclarerai pour lui devant mon Père qui est aux cieux. 
${}^{33}Mais celui qui me reniera devant les hommes, moi aussi je le renierai devant mon Père qui est aux cieux.
${}^{34}Ne pensez pas que je sois venu apporter la paix sur la terre : je ne suis pas venu apporter la paix, mais le glaive. 
${}^{35}Oui, je suis venu séparer l’homme de son père, la fille de sa mère, la belle-fille de sa belle-mère : 
${}^{36}on aura pour ennemis les gens de sa propre maison. 
${}^{37}Celui qui aime son père ou sa mère plus que moi n’est pas digne de moi ; celui qui aime son fils ou sa fille plus que moi n’est pas digne de moi ; 
${}^{38}celui qui ne prend pas sa croix et ne me suit pas n’est pas digne de moi. 
${}^{39}Qui a trouvé sa vie la perdra ; qui a perdu sa vie à cause de moi la gardera.
${}^{40}Qui vous accueille m’accueille ; et qui m’accueille accueille Celui qui m’a envoyé. 
${}^{41}Qui accueille un prophète en sa qualité de prophète recevra une récompense de prophète ; qui accueille un homme juste en sa qualité de juste recevra une récompense de juste. 
${}^{42}Et celui qui donnera à boire, même un simple verre d’eau fraîche, à l’un de ces petits en sa qualité de disciple, amen, je vous le dis : non, il ne perdra pas sa récompense. »
      
         
      \bchapter{}
      \begin{verse}
${}^{1}Lorsque Jésus eut terminé les instructions qu’il donnait à ses douze disciples, il partit de là pour enseigner et proclamer la Parole dans les villes du pays.
      
         
      <h2 class="intertitle" id="d85e337330">1. L’identité de Jésus et son action (11,2 – 12)</h2>
${}^{2}Jean le Baptiste entendit parler, dans sa prison, des œuvres réalisées par le Christ. Il lui envoya ses disciples et, par eux, 
${}^{3}lui demanda : « Es-tu celui qui doit venir, ou devons-nous en attendre un autre ? » 
${}^{4}Jésus leur répondit : « Allez annoncer à Jean ce que vous entendez et voyez : 
${}^{5}Les aveugles retrouvent la vue, et les boiteux marchent, les lépreux sont purifiés, et les sourds entendent, les morts ressuscitent, et les pauvres reçoivent la Bonne Nouvelle. 
${}^{6}Heureux celui pour qui je ne suis pas une occasion de chute ! »
${}^{7}Tandis que les envoyés de Jean s’en allaient, Jésus se mit à dire aux foules à propos de Jean : « Qu’êtes-vous allés regarder au désert ? un roseau agité par le vent ? 
${}^{8}Alors, qu’êtes-vous donc allés voir ? un homme habillé de façon raffinée ? Mais ceux qui portent de tels vêtements vivent dans les palais des rois. 
${}^{9}Alors, qu’êtes-vous allés voir ? un prophète ? Oui, je vous le dis, et bien plus qu’un prophète. 
${}^{10}C’est de lui qu’il est écrit :
        \\Voici que j’envoie mon messager en avant de toi,
        \\pour préparer le chemin devant toi.
${}^{11}Amen, je vous le dis : Parmi ceux qui sont nés d’une femme, personne ne s’est levé de plus grand que Jean le Baptiste ; et cependant le plus petit dans le royaume des Cieux est plus grand que lui. 
${}^{12}Depuis les jours de Jean le Baptiste jusqu’à présent, le royaume des Cieux subit la violence, et des violents cherchent à s’en emparer. 
${}^{13}Tous les Prophètes, ainsi que la Loi, ont prophétisé jusqu’à Jean. 
${}^{14}Et, si vous voulez bien comprendre, c’est lui, le prophète Élie qui doit venir. 
${}^{15}Celui qui a des oreilles, qu’il entende !
${}^{16}À qui vais-je comparer cette génération ? Elle ressemble à des gamins assis sur les places, qui en interpellent d’autres en disant :
        ${}^{17}“Nous vous avons joué de la flûte,
        \\et vous n’avez pas dansé.
        \\Nous avons chanté des lamentations,
        \\et vous ne vous êtes pas frappé la poitrine.”
${}^{18}Jean Baptiste est venu, en effet ; il ne mange pas, il ne boit pas, et l’on dit : “C’est un possédé !” 
${}^{19}Le Fils de l’homme est venu ; il mange et il boit, et l’on dit : “Voilà un glouton et un ivrogne, un ami des publicains et des pécheurs.” Mais la sagesse de Dieu a été reconnue juste à travers ce qu’elle fait. »
${}^{20}Alors Jésus se mit à faire des reproches aux villes où avaient eu lieu la plupart de ses miracles, parce qu’elles ne s’étaient pas converties : 
${}^{21}« Malheureuse es-tu, Corazine ! Malheureuse es-tu, Bethsaïde ! Car, si les miracles qui ont eu lieu chez vous avaient eu lieu à Tyr et à Sidon, ces villes, autrefois, se seraient converties sous le sac et la cendre. 
${}^{22}Aussi, je vous le déclare : au jour du Jugement, Tyr et Sidon seront traitées moins sévèrement que vous. 
${}^{23}Et toi, Capharnaüm, seras-tu donc élevée jusqu’au ciel ? Non, tu descendras jusqu’au séjour des morts ! Car, si les miracles qui ont eu lieu chez toi avaient eu lieu à Sodome, cette ville serait encore là aujourd’hui. 
${}^{24}Aussi, je vous le déclare : au jour du Jugement, le pays de Sodome sera traité moins sévèrement que toi. »
${}^{25}En ce temps-là, Jésus prit la parole et dit : « Père, Seigneur du ciel et de la terre, je proclame ta louange : ce que tu as caché aux sages et aux savants, tu l’as révélé aux tout-petits. 
${}^{26}Oui, Père, tu l’as voulu ainsi dans ta bienveillance. 
${}^{27}Tout m’a été remis par mon Père ; personne ne connaît le Fils, sinon le Père, et personne ne connaît le Père, sinon le Fils, et celui à qui le Fils veut le révéler.
${}^{28}« Venez à moi, vous tous qui peinez sous le poids du fardeau, et moi, je vous procurerai le repos. 
${}^{29}Prenez sur vous mon joug, devenez mes disciples, car je suis doux et humble de cœur, et vous trouverez le repos pour votre âme. 
${}^{30}Oui, mon joug est facile à porter, et mon fardeau, léger. »
      
         
      \bchapter{}
      \begin{verse}
${}^{1}En ce temps-là, un jour de sabbat, Jésus vint à passer à travers les champs de blé ; ses disciples eurent faim et ils se mirent à arracher des épis et à les manger. 
${}^{2}Voyant cela, les pharisiens lui dirent : « Voilà que tes disciples font ce qu’il n’est pas permis de faire le jour du sabbat ! » 
${}^{3}Mais il leur dit : « N’avez-vous pas lu ce que fit David, quand il eut faim, lui et ceux qui l’accompagnaient ? 
${}^{4}Il entra dans la maison de Dieu, et ils mangèrent les pains de l’offrande ; or, ni lui ni les autres n’avaient le droit d’en manger, mais seulement les prêtres. 
${}^{5}Ou bien encore, n’avez-vous pas lu dans la Loi que le jour du sabbat, les prêtres, dans le Temple, manquent au repos du sabbat sans commettre de faute ? 
${}^{6}Or, je vous le dis : il y a ici plus grand que le Temple. 
${}^{7}Si vous aviez compris ce que signifie : Je veux la miséricorde, non le sacrifice, 
${}^{8}vous n’auriez pas condamné ceux qui n’ont pas commis de faute. En effet, le Fils de l’homme est maître du sabbat. »
${}^{9}Il partit de là et entra dans leur synagogue. 
${}^{10}Or il s’y trouvait un homme qui avait une main atrophiée. Et l’on demanda à Jésus : « Est-il permis de faire une guérison le jour du sabbat ? » C’était afin de pouvoir l’accuser. 
${}^{11}Mais il leur dit : « Si l’un d’entre vous possède une seule brebis, et qu’elle tombe dans un trou le jour du sabbat, ne va-t-il pas la saisir pour la faire remonter ? 
${}^{12}Or, un homme vaut tellement plus qu’une brebis ! Il est donc permis de faire le bien le jour du sabbat. » 
${}^{13}Alors Jésus dit à l’homme : « Étends la main. » L’homme l’étendit, et elle redevint normale et saine comme l’autre. 
${}^{14}Une fois sortis, les pharisiens se réunirent en conseil contre Jésus pour voir comment le faire périr.
${}^{15}Jésus, l’ayant appris, se retira de là ; beaucoup de gens le suivirent, et il les guérit tous. 
${}^{16}Mais il leur défendit vivement de parler de lui. 
${}^{17}Ainsi devait s’accomplir la parole prononcée par le prophète Isaïe :
        ${}^{18}Voici mon serviteur que j’ai choisi,
        \\mon bien-aimé en qui je trouve mon bonheur.
        \\Je ferai reposer sur lui mon Esprit,
        \\aux nations il fera connaître le jugement.
        ${}^{19}Il ne cherchera pas querelle, il ne criera pas,
        \\on n’entendra pas sa voix sur les places publiques.
        ${}^{20}Il n’écrasera pas le roseau froissé,
        \\il n’éteindra pas la mèche qui faiblit,
        \\jusqu’à ce qu’il ait fait triompher le jugement.
        ${}^{21}Les nations mettront en son nom leur espérance.
${}^{22}Alors on lui présenta un possédé qui était aveugle et muet. Jésus le guérit, de sorte que le muet parlait et qu’il voyait.
${}^{23}Toutes les foules étaient dans la stupéfaction et disaient : « Cet homme ne serait-il pas le fils de David ? »
${}^{24}En entendant cela, les pharisiens disaient : « Il n’expulse les démons que par Béelzéboul, le chef des démons. »
${}^{25}Connaissant leurs pensées, Jésus leur dit : « Tout royaume divisé contre lui-même devient un désert ; toute ville ou maison divisée contre elle-même sera incapable de tenir. 
${}^{26}Si Satan expulse Satan, c’est donc qu’il est divisé contre lui-même ; comment son royaume tiendra-t-il ?
${}^{27}Et si c’est par Béelzéboul que moi, j’expulse les démons, vos disciples, par qui les expulsent-ils ? C’est pourquoi ils seront eux-mêmes vos juges. 
${}^{28}Mais, si c’est par l’Esprit de Dieu que moi, j’expulse les démons, c’est donc que le règne de Dieu est venu jusqu’à vous.
${}^{29}Ou encore, comment quelqu’un peut-il entrer dans la maison de l’homme fort et piller ses biens, sans avoir d’abord ligoté cet homme fort ? Alors seulement il pillera sa maison.
${}^{30}Celui qui n’est pas avec moi est contre moi ; celui qui ne rassemble pas avec moi disperse. 
${}^{31}C’est pourquoi, je vous le dis : Tout péché, tout blasphème, sera pardonné aux hommes, mais le blasphème contre l’Esprit ne sera pas pardonné. 
${}^{32}Et si quelqu’un dit une parole contre le Fils de l’homme, cela lui sera pardonné ; mais si quelqu’un parle contre l’Esprit Saint, cela ne lui sera pas pardonné, ni en ce monde-ci, ni dans le monde à venir.
${}^{33}Prenez un bel arbre, son fruit sera beau ; prenez un arbre qui pourrit, son fruit sera pourri, car c’est à son fruit qu’on reconnaît l’arbre. 
${}^{34}Engeance de vipères ! comment pouvez-vous dire des paroles bonnes, vous qui êtes mauvais ? Car ce que dit la bouche, c’est ce qui déborde du cœur. 
${}^{35}L’homme bon, de son trésor qui est bon, tire de bonnes choses ; l’homme mauvais, de son trésor qui est mauvais, tire de mauvaises choses. 
${}^{36}Je vous le dis : toute parole creuse que prononceront les hommes, ils devront en rendre compte au jour du Jugement. 
${}^{37}D’après tes paroles, en effet, tu seras reconnu juste ; d’après tes paroles tu seras condamné. »
${}^{38}Quelques-uns des scribes et des pharisiens lui adressèrent la parole : « Maître, nous voudrions voir un signe venant de toi. » 
${}^{39}Il leur répondit : « Cette génération mauvaise et adultère réclame un signe, mais, en fait de signe, il ne lui sera donné que le signe du prophète Jonas. 
${}^{40}En effet, comme Jonas est resté dans le ventre du monstre marin trois jours et trois nuits, le Fils de l’homme restera de même au cœur de la terre trois jours et trois nuits. 
${}^{41}Lors du Jugement, les habitants de Ninive se lèveront en même temps que cette génération, et ils la condamneront ; en effet, ils se sont convertis en réponse à la proclamation faite par Jonas, et il y a ici bien plus que Jonas. 
${}^{42}Lors du Jugement, la reine de Saba se dressera en même temps que cette génération, et elle la condamnera ; en effet, elle est venue des extrémités de la terre pour écouter la sagesse de Salomon, et il y a ici bien plus que Salomon.
${}^{43}Quand l’esprit impur est sorti de l’homme, il parcourt des lieux arides en cherchant où se reposer, et il ne trouve pas. 
${}^{44}Alors il se dit : “Je vais retourner dans ma maison, d’où je suis sorti.” En arrivant, il la trouve inoccupée, balayée et bien rangée. 
${}^{45}Alors il s’en va, il prend avec lui sept autres esprits, encore plus mauvais que lui ; ils y entrent et s’y installent. Ainsi, l’état de cet homme-là est pire à la fin qu’au début. Voilà ce qui arrivera à cette génération mauvaise. »
${}^{46}Comme Jésus parlait encore aux foules, voici que sa mère et ses frères se tenaient au-dehors, cherchant à lui parler. 
${}^{47}Quelqu’un lui dit : « Ta mère et tes frères sont là, dehors, qui cherchent à te parler. » 
${}^{48}Jésus lui répondit : « Qui est ma mère, et qui sont mes frères ? » 
${}^{49}Puis, étendant la main vers ses disciples, il dit : « Voici ma mère et mes frères. 
${}^{50}Car celui qui fait la volonté de mon Père qui est aux cieux, celui-là est pour moi un frère, une sœur, une mère. »
      <h2 class="intertitle hmbot" id="d85e338104">2. Troisième discours : Les paraboles du Royaume (13,1-53)</h2>
      
         
      \bchapter{}
      \begin{verse}
${}^{1}Ce jour-là, Jésus était sorti de la maison, et il était assis au bord de la mer. 
${}^{2}Auprès de lui se rassemblèrent des foules si grandes qu’il monta dans une barque où il s’assit ; toute la foule se tenait sur le rivage. 
${}^{3}Il leur dit beaucoup de choses en paraboles :
      
         
      « Voici que le semeur sortit pour semer. 
${}^{4}Comme il semait, des grains sont tombés au bord du chemin, et les oiseaux sont venus tout manger. 
${}^{5}D’autres sont tombés sur le sol pierreux, où ils n’avaient pas beaucoup de terre ; ils ont levé aussitôt, parce que la terre était peu profonde. 
${}^{6}Le soleil s’étant levé, ils ont brûlé et, faute de racines, ils ont séché. 
${}^{7}D’autres sont tombés dans les ronces ; les ronces ont poussé et les ont étouffés. 
${}^{8}D’autres sont tombés dans la bonne terre, et ils ont donné du fruit à raison de cent, ou soixante, ou trente pour un. 
${}^{9}Celui qui a des oreilles, qu’il entende ! »
${}^{10}Les disciples s’approchèrent de Jésus et lui dirent : « Pourquoi leur parles-tu en paraboles ? » 
${}^{11}Il leur répondit : « À vous il est donné de connaître les mystères du royaume des Cieux, mais ce n’est pas donné à ceux-là. 
${}^{12}À celui qui a, on donnera, et il sera dans l’abondance ; à celui qui n’a pas, on enlèvera même ce qu’il a. 
${}^{13}Si je leur parle en paraboles, c’est parce qu’ils regardent sans regarder, et qu’ils écoutent sans écouter ni comprendre. 
${}^{14}Ainsi s’accomplit pour eux la prophétie d’Isaïe :
        \\Vous aurez beau écouter, vous ne comprendrez pas.
        \\Vous aurez beau regarder, vous ne verrez pas.
        ${}^{15}Le cœur de ce peuple s’est alourdi :
        \\ils sont devenus durs d’oreille,
        \\ils se sont bouché les yeux,
        \\de peur que leurs yeux ne voient,
        \\que leurs oreilles n’entendent,
        \\que leur cœur ne comprenne,
        \\qu’ils ne se convertissent,
        \\– et moi, je les guérirai.
${}^{16}Mais vous, heureux vos yeux puisqu’ils voient, et vos oreilles puisqu’elles entendent ! 
${}^{17}Amen, je vous le dis : beaucoup de prophètes et de justes ont désiré voir ce que vous voyez, et ne l’ont pas vu, entendre ce que vous entendez, et ne l’ont pas entendu.
${}^{18}Vous donc, écoutez ce que veut dire la parabole du semeur. 
${}^{19}Quand quelqu’un entend la parole du Royaume sans la comprendre, le Mauvais survient et s’empare de ce qui est semé dans son cœur : celui-là, c’est le terrain ensemencé au bord du chemin. 
${}^{20}Celui qui a reçu la semence sur un sol pierreux, c’est celui qui entend la Parole et la reçoit aussitôt avec joie ; 
${}^{21}mais il n’a pas de racines en lui, il est l’homme d’un moment : quand vient la détresse ou la persécution à cause de la Parole, il trébuche aussitôt. 
${}^{22}Celui qui a reçu la semence dans les ronces, c’est celui qui entend la Parole ; mais le souci du monde et la séduction de la richesse étouffent la Parole, qui ne donne pas de fruit. 
${}^{23}Celui qui a reçu la semence dans la bonne terre, c’est celui qui entend la Parole et la comprend : il porte du fruit à raison de cent, ou soixante, ou trente pour un. »
${}^{24}Il leur proposa une autre parabole : « Le royaume des Cieux est comparable à un homme qui a semé du bon grain dans son champ. 
${}^{25}Or, pendant que les gens dormaient, son ennemi survint ; il sema de l’ivraie au milieu du blé et s’en alla. 
${}^{26}Quand la tige poussa et produisit l’épi, alors l’ivraie apparut aussi.
${}^{27}Les serviteurs du maître vinrent lui dire : “Seigneur, n’est-ce pas du bon grain que tu as semé dans ton champ ? D’où vient donc qu’il y a de l’ivraie ?” 
${}^{28}Il leur dit : “C’est un ennemi qui a fait cela.” Les serviteurs lui disent : “Veux-tu donc que nous allions l’enlever ?”
${}^{29}Il répond : “Non, en enlevant l’ivraie, vous risquez d’arracher le blé en même temps. 
${}^{30}Laissez-les pousser ensemble jusqu’à la moisson ; et, au temps de la moisson, je dirai aux moissonneurs : Enlevez d’abord l’ivraie, liez-la en bottes pour la brûler ; quant au blé, ramassez-le pour le rentrer dans mon grenier.” »
${}^{31}Il leur proposa une autre parabole : « Le royaume des Cieux est comparable à une graine de moutarde qu’un homme a prise et qu’il a semée dans son champ. 
${}^{32}C’est la plus petite de toutes les semences, mais, quand elle a poussé, elle dépasse les autres plantes potagères et devient un arbre, si bien que les oiseaux du ciel viennent et font leurs nids dans ses branches. »
${}^{33}Il leur dit une autre parabole : « Le royaume des Cieux est comparable au levain qu’une femme a pris et qu’elle a enfoui dans trois mesures de farine, jusqu’à ce que toute la pâte ait levé. »
${}^{34}Tout cela, Jésus le dit aux foules en paraboles, et il ne leur disait rien sans parabole, 
${}^{35}accomplissant ainsi la parole du prophète :
      J’ouvrirai la bouche pour des paraboles,
      je publierai ce qui fut caché depuis la fondation du monde.
${}^{36}Alors, laissant les foules, il vint à la maison. Ses disciples s’approchèrent et lui dirent : « Explique-nous clairement la parabole de l’ivraie dans le champ. » 
${}^{37}Il leur répondit : « Celui qui sème le bon grain, c’est le Fils de l’homme ; 
${}^{38}le champ, c’est le monde ; le bon grain, ce sont les fils du Royaume ; l’ivraie, ce sont les fils du Mauvais. 
${}^{39}L’ennemi qui l’a semée, c’est le diable ; la moisson, c’est la fin du monde ; les moissonneurs, ce sont les anges. 
${}^{40}De même que l’on enlève l’ivraie pour la jeter au feu, ainsi en sera-t-il à la fin du monde. 
${}^{41}Le Fils de l’homme enverra ses anges, et ils enlèveront de son Royaume toutes les causes de chute et ceux qui font le mal ; 
${}^{42}ils les jetteront dans la fournaise : là, il y aura des pleurs et des grincements de dents. 
${}^{43}Alors les justes resplendiront comme le soleil dans le royaume de leur Père. Celui qui a des oreilles, qu’il entende !
${}^{44}Le royaume des Cieux est comparable à un trésor caché dans un champ ; l’homme qui l’a découvert le cache de nouveau. Dans sa joie, il va vendre tout ce qu’il possède, et il achète ce champ.
${}^{45}Ou encore : Le royaume des Cieux est comparable à un négociant qui recherche des perles fines. 
${}^{46}Ayant trouvé une perle de grande valeur, il va vendre tout ce qu’il possède, et il achète la perle.
${}^{47}Le royaume des Cieux est encore comparable à un filet que l’on jette dans la mer, et qui ramène toutes sortes de poissons. 
${}^{48}Quand il est plein, on le tire sur le rivage, on s’assied, on ramasse dans des paniers ce qui est bon, et on rejette ce qui ne vaut rien. 
${}^{49}Ainsi en sera-t-il à la fin du monde : les anges sortiront pour séparer les méchants du milieu des justes 
${}^{50}et les jetteront dans la fournaise : là, il y aura des pleurs et des grincements de dents. »
${}^{51}« Avez-vous compris tout cela ? » Ils lui répondent : « Oui ». 
${}^{52}Jésus ajouta : « C’est pourquoi tout scribe devenu disciple du royaume des Cieux est comparable à un maître de maison qui tire de son trésor du neuf et de l’ancien. » 
${}^{53}Lorsque Jésus eut terminé ces paraboles, il s’éloigna de là.
      <h2 class="intertitle" id="d85e338581">3. Vers la reconnaissance de Jésus, Messie, Fils de Dieu (13,54 – 16,20)</h2>
${}^{54}Il se rendit dans son lieu d’origine, et il enseignait les gens dans leur synagogue, de telle manière qu’ils étaient frappés d’étonnement et disaient : « D’où lui viennent cette sagesse et ces miracles ? 
${}^{55}N’est-il pas le fils du charpentier ? Sa mère ne s’appelle-t-elle pas Marie, et ses frères : Jacques, Joseph, Simon et Jude ? 
${}^{56}Et ses sœurs ne sont-elles pas toutes chez nous ? Alors, d’où lui vient tout cela ? » 
${}^{57}Et ils étaient profondément choqués à son sujet. Jésus leur dit : « Un prophète n’est méprisé que dans son pays et dans sa propre maison. » 
${}^{58}Et il ne fit pas beaucoup de miracles à cet endroit-là, à cause de leur manque de foi.
      
         
      \bchapter{}
      \begin{verse}
${}^{1}En ce temps-là, Hérode, qui était au pouvoir en Galilée, apprit la renommée de Jésus 
${}^{2}et dit à ses serviteurs : « Celui-là, c’est Jean le Baptiste, il est ressuscité d’entre les morts, et voilà pourquoi des miracles se réalisent par lui. »
${}^{3}Car Hérode avait fait arrêter Jean, l’avait fait enchaîner et mettre en prison. C’était à cause d’Hérodiade, la femme de son frère Philippe. 
${}^{4}En effet, Jean lui avait dit : « Tu n’as pas le droit de l’avoir pour femme. » 
${}^{5}Hérode cherchait à le faire mourir, mais il eut peur de la foule qui le tenait pour un prophète.
${}^{6}Lorsque arriva l’anniversaire d’Hérode, la fille d’Hérodiade dansa au milieu des convives, et elle plut à Hérode. 
${}^{7}Alors il s’engagea par serment à lui donner ce qu’elle demanderait. 
${}^{8}Poussée par sa mère, elle dit : « Donne-moi ici, sur un plat, la tête de Jean le Baptiste. » 
${}^{9}Le roi fut contrarié ; mais à cause de son serment et des convives, il commanda de la lui donner. 
${}^{10}Il envoya décapiter Jean dans la prison. 
${}^{11}La tête de celui-ci fut apportée sur un plat et donnée à la jeune fille, qui l’apporta à sa mère. 
${}^{12}Les disciples de Jean arrivèrent pour prendre son corps, qu’ils ensevelirent ; puis ils allèrent l’annoncer à Jésus.
${}^{13}Quand Jésus apprit cela, il se retira et partit en barque pour un endroit désert, à l’écart. Les foules l’apprirent et, quittant leurs villes, elles suivirent à pied. 
${}^{14}En débarquant, il vit une grande foule de gens ; il fut saisi de compassion envers eux et guérit leurs malades.
${}^{15}Le soir venu, les disciples s’approchèrent et lui dirent : « L’endroit est désert et l’heure est déjà avancée. Renvoie donc la foule : qu’ils aillent dans les villages s’acheter de la nourriture ! » 
${}^{16}Mais Jésus leur dit : « Ils n’ont pas besoin de s’en aller. Donnez-leur vous-mêmes à manger. » 
${}^{17}Alors ils lui disent : « Nous n’avons là que cinq pains et deux poissons. » 
${}^{18}Jésus dit : « Apportez-les moi. » 
${}^{19}Puis, ordonnant à la foule de s’asseoir sur l’herbe, il prit les cinq pains et les deux poissons, et, levant les yeux au ciel, il prononça la bénédiction ; il rompit les pains, il les donna aux disciples, et les disciples les donnèrent à la foule. 
${}^{20}Ils mangèrent tous et ils furent rassasiés. On ramassa les morceaux qui restaient : cela faisait douze paniers pleins. 
${}^{21}Ceux qui avaient mangé étaient environ cinq mille, sans compter les femmes et les enfants.
${}^{22}Aussitôt Jésus obligea les disciples à monter dans la barque et à le précéder sur l’autre rive, pendant qu’il renverrait les foules. 
${}^{23}Quand il les eut renvoyées, il gravit la montagne, à l’écart, pour prier. Le soir venu, il était là, seul.
${}^{24}La barque était déjà à une bonne distance de la terre, elle était battue par les vagues, car le vent était contraire. 
${}^{25}Vers la fin de la nuit, Jésus vint vers eux en marchant sur la mer. 
${}^{26}En le voyant marcher sur la mer, les disciples furent bouleversés. Ils dirent : « C’est un fantôme. » Pris de peur, ils se mirent à crier. 
${}^{27}Mais aussitôt Jésus leur parla : « Confiance ! c’est moi ; n’ayez plus peur ! » 
${}^{28}Pierre prit alors la parole : « Seigneur, si c’est bien toi, ordonne-moi de venir vers toi sur les eaux. » 
${}^{29}Jésus lui dit : « Viens ! » Pierre descendit de la barque et marcha sur les eaux pour aller vers Jésus. 
${}^{30}Mais, voyant la force du vent, il eut peur et, comme il commençait à enfoncer, il cria : « Seigneur, sauve-moi ! » 
${}^{31}Aussitôt, Jésus étendit la main, le saisit et lui dit : « Homme de peu de foi, pourquoi as-tu douté ? » 
${}^{32}Et quand ils furent montés dans la barque, le vent tomba. 
${}^{33}Alors ceux qui étaient dans la barque se prosternèrent devant lui, et ils lui dirent : « Vraiment, tu es le Fils de Dieu ! »
${}^{34}Après la traversée, ils abordèrent à Génésareth. 
${}^{35}Les gens de cet endroit reconnurent Jésus ; ils firent avertir toute la région, et on lui amena tous les malades. 
${}^{36}Ils le suppliaient de leur laisser seulement toucher la frange de son manteau, et tous ceux qui le faisaient furent sauvés.
      
         
      \bchapter{}
      \begin{verse}
${}^{1}Alors des pharisiens et des scribes venus de Jérusalem s’approchent de Jésus et lui disent : 
${}^{2}« Pourquoi tes disciples transgressent-ils la tradition des anciens ? En effet, ils ne se lavent pas les mains avant de manger. »
${}^{3}Jésus leur répondit : « Et vous, pourquoi transgressez-vous le commandement de Dieu au nom de votre tradition ? 
${}^{4}Car Dieu a dit : Honore ton père et ta mère. Et encore : Celui qui maudit son père ou sa mère sera mis à mort. 
${}^{5}Et vous, vous dites : “Supposons que quelqu’un déclare à son père ou à sa mère : “Les ressources qui m’auraient permis de t’aider sont un don réservé à Dieu.” 
${}^{6}Dans ce cas, il n’aura plus à honorer son père ou sa mère.” Ainsi, vous avez annulé la parole de Dieu au nom de votre tradition ! Hypocrites ! 
${}^{7}Isaïe a bien prophétisé à votre sujet quand il a dit :
${}^{8}Ce peuple m’honore des lèvres,
        \\mais son cœur est loin de moi.
${}^{9}C’est en vain qu’ils me rendent un culte ;
        \\les doctrines qu’ils enseignent ne sont que des préceptes humains. »
${}^{10}Jésus appela la foule et lui dit : « Écoutez et comprenez bien ! 
${}^{11}Ce n’est pas ce qui entre dans la bouche qui rend l’homme impur ; mais ce qui sort de la bouche, voilà ce qui rend l’homme impur. »
${}^{12}Alors les disciples s’approchèrent et lui dirent : « Sais-tu que les pharisiens ont été scandalisés en entendant cette parole ? » 
${}^{13}Il répondit : « Toute plante que mon Père du ciel n’a pas plantée sera arrachée. 
${}^{14}Laissez-les ! Ce sont des aveugles qui guident des aveugles. Si un aveugle guide un aveugle, tous les deux tomberont dans un trou. » 
${}^{15}Prenant la parole, Pierre lui dit : « Explique-nous cette parabole. » 
${}^{16}Jésus répliqua : « Êtes-vous encore sans intelligence, vous aussi ? 
${}^{17}Ne comprenez-vous pas que tout ce qui entre dans la bouche passe dans le ventre pour être éliminé ? 
${}^{18}Mais ce qui sort de la bouche provient du cœur, et c’est cela qui rend l’homme impur. 
${}^{19}Car c’est du cœur que proviennent les pensées mauvaises : meurtres, adultères, inconduite, vols, faux témoignages, diffamations. 
${}^{20}C’est cela qui rend l’homme impur, mais manger sans se laver les mains ne rend pas l’homme impur. »
${}^{21}Partant de là, Jésus se retira dans la région de Tyr et de Sidon. 
${}^{22}Voici qu’une Cananéenne, venue de ces territoires, disait en criant : « Prends pitié de moi, Seigneur, fils de David ! Ma fille est tourmentée par un démon. » 
${}^{23}Mais il ne lui répondit pas un mot. Les disciples s’approchèrent pour lui demander : « Renvoie-la, car elle nous poursuit de ses cris ! » 
${}^{24}Jésus répondit : « Je n’ai été envoyé qu’aux brebis perdues de la maison d’Israël. » 
${}^{25}Mais elle vint se prosterner devant lui en disant : « Seigneur, viens à mon secours ! » 
${}^{26}Il répondit : « Il n’est pas bien de prendre le pain des enfants et de le jeter aux petits chiens. » 
${}^{27}Elle reprit : « Oui, Seigneur ; mais justement, les petits chiens mangent les miettes qui tombent de la table de leurs maîtres. » 
${}^{28}Jésus répondit : « Femme, grande est ta foi, que tout se passe pour toi comme tu le veux ! » Et, à l’heure même, sa fille fut guérie.
${}^{29}Jésus partit de là et arriva près de la mer de Galilée. Il gravit la montagne et là, il s’assit. 
${}^{30}De grandes foules s’approchèrent de lui, avec des boiteux, des aveugles, des estropiés, des muets, et beaucoup d’autres encore ; on les déposa à ses pieds et il les guérit. 
${}^{31}Alors la foule était dans l’admiration en voyant des muets qui parlaient, des estropiés rétablis, des boiteux qui marchaient, des aveugles qui voyaient ; et ils rendirent gloire au Dieu d’Israël.
${}^{32}Jésus appela ses disciples et leur dit : « Je suis saisi de compassion pour cette foule, car depuis trois jours déjà ils restent auprès de moi, et n’ont rien à manger. Je ne veux pas les renvoyer à jeun, ils pourraient défaillir en chemin. » 
${}^{33}Les disciples lui disent : « Où trouverons-nous dans un désert assez de pain pour rassasier une telle foule ? » 
${}^{34}Jésus leur demanda : « Combien de pains avez-vous ? » Ils dirent : « Sept, et quelques petits poissons. » 
${}^{35}Alors il ordonna à la foule de s’asseoir par terre. 
${}^{36}Il prit les sept pains et les poissons ; rendant grâce, il les rompit, et il les donnait aux disciples, et les disciples aux foules. 
${}^{37}Tous mangèrent et furent rassasiés. On ramassa les morceaux qui restaient : cela faisait sept corbeilles pleines. 
${}^{38}Or, ceux qui avaient mangé étaient quatre mille, sans compter les femmes et les enfants. 
${}^{39}Après avoir renvoyé la foule, Jésus monta dans la barque et alla dans le territoire de Magadane.
      
         
      \bchapter{}
      \begin{verse}
${}^{1}Les pharisiens et les sadducéens s’approchèrent pour mettre Jésus à l’épreuve ; ils lui demandèrent de leur montrer un signe venant du ciel. 
${}^{2}Il leur répondit : « Quand vient le soir, vous dites : “Voici le beau temps, car le ciel est rouge.” 
${}^{3}Et le matin, vous dites : “Aujourd’hui, il fera mauvais, car le ciel est d’un rouge menaçant.” Ainsi l’aspect du ciel, vous savez en juger ; mais pour les signes des temps, vous n’en êtes pas capables. 
${}^{4}Cette génération mauvaise et adultère réclame un signe, mais, en fait de signe, il ne lui sera donné que le signe de Jonas. » Alors il les abandonna et partit.
      
         
${}^{5}En se rendant sur l’autre rive, les disciples avaient oublié d’emporter des pains. 
${}^{6}Jésus leur dit : « Attention ! Méfiez-vous du levain des pharisiens et des sadducéens. » 
${}^{7}Ils discutaient entre eux en disant : « C’est parce que nous n’avons pas pris de pains. » 
${}^{8}Mais Jésus s’en rendit compte et leur dit : « Hommes de peu de foi, pourquoi discutez-vous entre vous sur ce manque de pains ? 
${}^{9}Vous ne comprenez pas encore ? Ne vous rappelez-vous pas les cinq pains pour cinq mille personnes et combien de paniers vous avez emportés ? 
${}^{10}Les sept pains pour quatre mille personnes et combien de corbeilles vous avez emportées ? 
${}^{11}Comment ne comprenez-vous pas que je ne parlais pas du pain ? Méfiez-vous donc du levain des pharisiens et des sadducéens. » 
${}^{12}Alors ils comprirent qu’il ne leur avait pas dit de se méfier du levain pour le pain, mais de l’enseignement des pharisiens et des sadducéens.
${}^{13}Jésus, arrivé dans la région de Césarée-de-Philippe, demandait à ses disciples : « Au dire des gens, qui est le Fils de l’homme ? » 
${}^{14}Ils répondirent : « Pour les uns, Jean le Baptiste ; pour d’autres, Élie ; pour d’autres encore, Jérémie ou l’un des prophètes. » 
${}^{15}Jésus leur demanda : « Et vous, que dites-vous ? Pour vous, qui suis-je ? » 
${}^{16}Alors Simon-Pierre prit la parole et dit : « Tu es le Christ, le Fils du Dieu vivant ! » 
${}^{17}Prenant la parole à son tour, Jésus lui dit : « Heureux es-tu, Simon fils de Yonas : ce n’est pas la chair et le sang qui t’ont révélé cela, mais mon Père qui est aux cieux. 
${}^{18}Et moi, je te le déclare : Tu es Pierre, et sur cette pierre je bâtirai mon Église ; et la puissance de la Mort ne l’emportera pas sur elle. 
${}^{19}Je te donnerai les clés du royaume des Cieux : tout ce que tu auras lié sur la terre sera lié dans les cieux, et tout ce que tu auras délié sur la terre sera délié dans les cieux. » 
${}^{20}Alors, il ordonna aux disciples de ne dire à personne que c’était lui le Christ.
      <h2 class="intertitle" id="d85e339368">1. Appel à suivre le Christ souffrant (16,21 – 17)</h2>
${}^{21}À partir de ce moment, Jésus commença à montrer à ses disciples qu’il lui fallait partir pour Jérusalem, souffrir beaucoup de la part des anciens, des grands prêtres et des scribes, être tué, et le troisième jour ressusciter. 
${}^{22}Pierre, le prenant à part, se mit à lui faire de vifs reproches : « Dieu t’en garde, Seigneur ! cela ne t’arrivera pas. » 
${}^{23}Mais lui, se retournant, dit à Pierre : « Passe derrière moi, Satan ! Tu es pour moi une occasion de chute : tes pensées ne sont pas celles de Dieu, mais celles des hommes. »
${}^{24}Alors Jésus dit à ses disciples : « Si quelqu’un veut marcher à ma suite, qu’il renonce à lui-même, qu’il prenne sa croix et qu’il me suive. 
${}^{25}Car celui qui veut sauver sa vie la perdra, mais qui perd sa vie à cause de moi la gardera. 
${}^{26}Quel avantage, en effet, un homme aura-t-il à gagner le monde entier, si c’est au prix de sa vie ? Et que pourra-t-il donner en échange de sa vie ? 
${}^{27}Car le Fils de l’homme va venir avec ses anges dans la gloire de son Père ; alors il rendra à chacun selon sa conduite. 
${}^{28}Amen, je vous le dis : parmi ceux qui sont ici, certains ne connaîtront pas la mort avant d’avoir vu le Fils de l’homme venir dans son Règne. »
      
         
      \bchapter{}
      \begin{verse}
${}^{1}Six jours après, Jésus prend avec lui Pierre, Jacques et Jean son frère, et il les emmène à l’écart, sur une haute montagne. 
${}^{2}Il fut transfiguré devant eux ; son visage devint brillant comme le soleil, et ses vêtements, blancs comme la lumière. 
${}^{3}Voici que leur apparurent Moïse et Élie, qui s’entretenaient avec lui. 
${}^{4}Pierre alors prit la parole et dit à Jésus : « Seigneur, il est bon que nous soyons ici ! Si tu le veux, je vais dresser ici trois tentes, une pour toi, une pour Moïse, et une pour Élie. » 
${}^{5}Il parlait encore, lorsqu’une nuée lumineuse les couvrit de son ombre, et voici que, de la nuée, une voix disait : « Celui-ci est mon Fils bien-aimé, en qui je trouve ma joie : écoutez-le ! » 
${}^{6}Quand ils entendirent cela, les disciples tombèrent face contre terre et furent saisis d’une grande crainte. 
${}^{7}Jésus s’approcha, les toucha et leur dit : « Relevez-vous et soyez sans crainte ! » 
${}^{8}Levant les yeux, ils ne virent plus personne, sinon lui, Jésus, seul.
${}^{9}En descendant de la montagne, Jésus leur donna cet ordre : « Ne parlez de cette vision à personne, avant que le Fils de l’homme soit ressuscité d’entre les morts. »
${}^{10}Les disciples interrogèrent Jésus : « Pourquoi donc les scribes disent-ils que le prophète Élie doit venir d’abord ? » 
${}^{11}Jésus leur répondit : « Élie va venir pour remettre toute chose à sa place. 
${}^{12}Mais, je vous le déclare : Élie est déjà venu ; au lieu de le reconnaître, ils lui ont fait tout ce qu’ils ont voulu. Et de même, le Fils de l’homme va souffrir par eux. » 
${}^{13}Alors les disciples comprirent qu’il leur parlait de Jean le Baptiste.
${}^{14}Quand ils eurent rejoint la foule, un homme s’approcha de lui, et tombant à ses genoux, 
${}^{15}il dit : « Seigneur, prends pitié de mon fils. Il est épileptique et il souffre beaucoup. Souvent il tombe dans le feu et, souvent aussi, dans l’eau. 
${}^{16}Je l’ai amené à tes disciples, mais ils n’ont pas pu le guérir. » 
${}^{17}Prenant la parole, Jésus dit : « Génération incroyante et dévoyée, combien de temps devrai-je rester avec vous ? Combien de temps devrai-je vous supporter ? Amenez-le-moi. » 
${}^{18}Jésus menaça le démon, et il sortit de lui. À l’heure même, l’enfant fut guéri.
${}^{19}Alors les disciples s’approchèrent de Jésus et lui dirent en particulier : « Pour quelle raison est-ce que nous, nous n’avons pas réussi à l’expulser ? » 
${}^{20}Jésus leur répond : « En raison de votre peu de foi. Amen, je vous le dis : si vous avez de la foi gros comme une graine de moutarde, vous direz à cette montagne : “Transporte-toi d’ici jusque là-bas”, et elle se transportera ; rien ne vous sera impossible. »
${}^{22}Comme ils étaient réunis en Galilée, Jésus leur dit : « Le Fils de l’homme va être livré aux mains des hommes ; 
${}^{23}ils le tueront et, le troisième jour, il ressuscitera. » Et ils furent profondément attristés.
${}^{24}Comme ils arrivaient à Capharnaüm, ceux qui perçoivent la redevance des deux drachmes pour le Temple vinrent trouver Pierre et lui dirent : « Votre maître paye bien les deux drachmes, n’est-ce pas ? » 
${}^{25}Il répondit : « Oui. » Quand Pierre entra dans la maison, Jésus prit la parole le premier : « Simon, quel est ton avis ? Les rois de la terre, de qui perçoivent-ils les taxes ou l’impôt ? De leurs fils, ou des autres personnes ? » 
${}^{26}Pierre lui répondit : « Des autres. » Et Jésus reprit : « Donc, les fils sont libres. 
${}^{27}Mais, pour ne pas scandaliser les gens, va donc jusqu’à la mer, jette l’hameçon, et saisis le premier poisson qui mordra ; ouvre-lui la bouche, et tu y trouveras une pièce de quatre drachmes. Prends-la, tu la donneras pour moi et pour toi. »
      <h2 class="intertitle" id="d85e339666">2. Quatrième discours : la communauté des disciples (18 – 19,2)</h2>
      
         
      \bchapter{}
      \begin{verse}
${}^{1}À ce moment-là, les disciples s’approchèrent de Jésus et lui dirent : « Qui donc est le plus grand dans le royaume des Cieux ? » 
${}^{2}Alors Jésus appela un petit enfant ; il le plaça au milieu d’eux, 
${}^{3}et il déclara : « Amen, je vous le dis : si vous ne changez pas pour devenir comme les enfants, vous n’entrerez pas dans le royaume des Cieux. 
${}^{4}Mais celui qui se fera petit comme cet enfant, celui-là est le plus grand dans le royaume des Cieux. 
${}^{5}Et celui qui accueille un enfant comme celui-ci en mon nom, il m’accueille, moi.
      
         
${}^{6}Celui qui est un scandale, une occasion de chute, pour un seul de ces petits qui croient en moi, il est préférable pour lui qu’on lui accroche au cou une de ces meules que tournent les ânes, et qu’il soit englouti en pleine mer. 
${}^{7}Malheureux le monde à cause des scandales ! Il est inévitable qu’arrivent les scandales ; cependant, malheureux celui par qui le scandale arrive ! 
${}^{8}Si ta main ou ton pied est pour toi une occasion de chute, coupe-le et jette-le loin de toi. Mieux vaut pour toi entrer dans la vie éternelle manchot ou estropié, que d’être jeté avec tes deux mains ou tes deux pieds dans le feu éternel. 
${}^{9}Et si ton œil est pour toi une occasion de chute, arrache-le et jette-le loin de toi. Mieux vaut pour toi entrer borgne dans la vie éternelle, que d’être jeté avec tes deux yeux dans la géhenne de feu.
${}^{10}Gardez-vous de mépriser un seul de ces petits, car, je vous le dis, leurs anges dans les cieux voient sans cesse la face de mon Père qui est aux cieux. 
${}^{12}Quel est votre avis ? Si un homme possède cent brebis et que l’une d’entre elles s’égare, ne va-t-il pas laisser les quatre-vingt-dix-neuf autres dans la montagne pour partir à la recherche de la brebis égarée ? 
${}^{13}Et, s’il arrive à la retrouver, amen, je vous le dis : il se réjouit pour elle plus que pour les quatre-vingt-dix-neuf qui ne se sont pas égarées. 
${}^{14}Ainsi, votre Père qui est aux cieux ne veut pas qu’un seul de ces petits soit perdu.
${}^{15}Si ton frère a commis un péché contre toi, va lui faire des reproches seul à seul. S’il t’écoute, tu as gagné ton frère. 
${}^{16}S’il ne t’écoute pas, prends en plus avec toi une ou deux personnes afin que toute l’affaire soit réglée sur la parole de deux ou trois témoins. 
${}^{17}S’il refuse de les écouter, dis-le à l’assemblée de l’Église ; s’il refuse encore d’écouter l’Église, considère-le comme un païen et un publicain. 
${}^{18}Amen, je vous le dis : tout ce que vous aurez lié sur la terre sera lié dans le ciel, et tout ce que vous aurez délié sur la terre sera délié dans le ciel.
${}^{19}Et pareillement, amen, je vous le dis, si deux d’entre vous sur la terre se mettent d’accord pour demander quoi que ce soit, ils l’obtiendront de mon Père qui est aux cieux. 
${}^{20}En effet, quand deux ou trois sont réunis en mon nom, je suis là, au milieu d’eux. »
${}^{21}Alors Pierre s’approcha de Jésus pour lui demander : « Seigneur, lorsque mon frère commettra des fautes contre moi, combien de fois dois-je lui pardonner ? Jusqu’à sept fois ? » 
${}^{22}Jésus lui répondit : « Je ne te dis pas jusqu’à sept fois, mais jusqu’à soixante-dix fois sept fois.
${}^{23}Ainsi, le royaume des Cieux est comparable à un roi qui voulut régler ses comptes avec ses serviteurs. 
${}^{24}Il commençait, quand on lui amena quelqu’un qui lui devait dix mille talents (c’est-à-dire soixante millions de pièces d’argent). 
${}^{25}Comme cet homme n’avait pas de quoi rembourser, le maître ordonna de le vendre, avec sa femme, ses enfants et tous ses biens, en remboursement de sa dette. 
${}^{26}Alors, tombant à ses pieds, le serviteur demeurait prosterné et disait : “Prends patience envers moi, et je te rembourserai tout.” 
${}^{27}Saisi de compassion, le maître de ce serviteur le laissa partir et lui remit sa dette. 
${}^{28}Mais, en sortant, ce serviteur trouva un de ses compagnons qui lui devait cent pièces d’argent. Il se jeta sur lui pour l’étrangler, en disant : “Rembourse ta dette !” 
${}^{29}Alors, tombant à ses pieds, son compagnon le suppliait : “Prends patience envers moi, et je te rembourserai.” 
${}^{30}Mais l’autre refusa et le fit jeter en prison jusqu’à ce qu’il ait remboursé ce qu’il devait.
${}^{31}Ses compagnons, voyant cela, furent profondément attristés et allèrent raconter à leur maître tout ce qui s’était passé. 
${}^{32}Alors celui-ci le fit appeler et lui dit : “Serviteur mauvais ! je t’avais remis toute cette dette parce que tu m’avais supplié. 
${}^{33}Ne devais-tu pas, à ton tour, avoir pitié de ton compagnon, comme moi-même j’avais eu pitié de toi ?” 
${}^{34}Dans sa colère, son maître le livra aux bourreaux jusqu’à ce qu’il eût remboursé tout ce qu’il devait. 
${}^{35}C’est ainsi que mon Père du ciel vous traitera, si chacun de vous ne pardonne pas à son frère du fond du cœur. »
      
         
      \bchapter{}
      \begin{verse}
${}^{1}Lorsque Jésus eut terminé ce discours, il s’éloigna de la Galilée et se rendit dans le territoire de la Judée, au-delà du Jourdain. 
${}^{2}De grandes foules le suivirent, et là il les guérit.
      
         
      <h2 class="intertitle" id="d85e339951">3. Exigences et paradoxes de l’Évangile (19,3 – 20)</h2>
${}^{3}Des pharisiens s’approchèrent de lui pour le mettre à l’épreuve ; ils lui demandèrent : « Est-il permis à un homme de renvoyer sa femme pour n’importe quel motif ? » 
${}^{4}Il répondit : « N’avez-vous pas lu ceci ? Dès le commencement, le Créateur les fit homme et femme, 
${}^{5}et dit :
      À cause de cela, l’homme quittera son père et sa mère,
      il s’attachera à sa femme,
      et tous deux deviendront une seule chair.
${}^{6}Ainsi, ils ne sont plus deux, mais une seule chair. Donc, ce que Dieu a uni, que l’homme ne le sépare pas ! »
${}^{7}Les pharisiens lui répliquent : « Pourquoi donc Moïse a-t-il prescrit la remise d’un acte de divorce avant la répudiation ? » 
${}^{8}Jésus leur répond : « C’est en raison de la dureté de votre cœur que Moïse vous a permis de renvoyer vos femmes. Mais au commencement, il n’en était pas ainsi. 
${}^{9}Or je vous le dis : si quelqu’un renvoie sa femme – sauf en cas d’union illégitime – et qu’il en épouse une autre, il est adultère. »
${}^{10}Ses disciples lui disent : « Si telle est la situation de l’homme par rapport à sa femme, mieux vaut ne pas se marier. » 
${}^{11}Il leur répondit : « Tous ne comprennent pas cette parole, mais seulement ceux à qui cela est donné. 
${}^{12}Il y a des gens qui ne se marient pas car, de naissance, ils en sont incapables ; il y en a qui ne peuvent pas se marier car ils ont été mutilés par les hommes ; il y en a qui ont choisi de ne pas se marier à cause du royaume des Cieux. Celui qui peut comprendre, qu’il comprenne ! »
${}^{13}Ensuite, on présenta des enfants à Jésus pour qu’il leur impose les mains en priant. Mais les disciples les écartèrent vivement. 
${}^{14}Jésus leur dit : « Laissez les enfants, ne les empêchez pas de venir à moi, car le royaume des Cieux est à ceux qui leur ressemblent. » 
${}^{15}Il leur imposa les mains, puis il partit de là.
${}^{16}Et voici que quelqu’un s’approcha de Jésus et lui dit : « Maître, que dois-je faire de bon pour avoir la vie éternelle ? » 
${}^{17}Jésus lui dit : « Pourquoi m’interroges-tu sur ce qui est bon ? Celui qui est bon, c’est Dieu, et lui seul ! Si tu veux entrer dans la vie, observe les commandements. » 
${}^{18}Il lui dit : « Lesquels ? » Jésus reprit :
        \\« Tu ne commettras pas de meurtre.
        \\Tu ne commettras pas d’adultère.
        \\Tu ne commettras pas de vol.
        \\Tu ne porteras pas de faux témoignage.
        ${}^{19}Honore ton père et ta mère.
      Et aussi :
        \\Tu aimeras ton prochain comme toi-même. »
${}^{20}Le jeune homme lui dit : « Tout cela, je l’ai observé : que me manque-t-il encore ? » 
${}^{21}Jésus lui répondit : « Si tu veux être parfait, va, vends ce que tu possèdes, donne-le aux pauvres, et tu auras un trésor dans les cieux. Puis viens, suis-moi. » 
${}^{22}À ces mots, le jeune homme s’en alla tout triste, car il avait de grands biens.
${}^{23}Et Jésus dit à ses disciples : « Amen, je vous le dis : un riche entrera difficilement dans le royaume des Cieux. 
${}^{24}Je vous le répète : il est plus facile à un chameau de passer par un trou d’aiguille qu’à un riche d’entrer dans le royaume des Cieux. » 
${}^{25}Entendant ces paroles, les disciples furent profondément déconcertés, et ils disaient : « Qui donc peut être sauvé ? » 
${}^{26}Jésus posa sur eux son regard et dit : « Pour les hommes, c’est impossible, mais pour Dieu tout est possible. »
${}^{27}Alors Pierre prit la parole et dit à Jésus : « Voici que nous avons tout quitté pour te suivre : quelle sera donc notre part ? » 
${}^{28}Jésus leur déclara : « Amen, je vous le dis : lors du renouvellement du monde, lorsque le Fils de l’homme siégera sur son trône de gloire, vous qui m’avez suivi, vous siégerez vous aussi sur douze trônes pour juger les douze tribus d’Israël. 
${}^{29}Et celui qui aura quitté, à cause de mon nom, des maisons, des frères, des sœurs, un père, une mère, des enfants, ou une terre, recevra le centuple, et il aura en héritage la vie éternelle. 
${}^{30}Beaucoup de premiers seront derniers, beaucoup de derniers seront premiers.
      
         
      \bchapter{}
      \begin{verse}
${}^{1} « En effet, le royaume des Cieux est comparable au maître d’un domaine qui sortit dès le matin afin d’embaucher des ouvriers pour sa vigne. 
${}^{2}Il se mit d’accord avec eux sur le salaire de la journée : un denier, c’est-à-dire une pièce d’argent, et il les envoya à sa vigne. 
${}^{3}Sorti vers neuf heures, il en vit d’autres qui étaient là, sur la place, sans rien faire. 
${}^{4}Et à ceux-là, il dit : “Allez à ma vigne, vous aussi, et je vous donnerai ce qui est juste.” 
${}^{5}Ils y allèrent. Il sortit de nouveau vers midi, puis vers trois heures, et fit de même. 
${}^{6}Vers cinq heures, il sortit encore, en trouva d’autres qui étaient là et leur dit : “Pourquoi êtes-vous restés là, toute la journée, sans rien faire ?” 
${}^{7}Ils lui répondirent : “Parce que personne ne nous a embauchés.” Il leur dit : “Allez à ma vigne, vous aussi.”
${}^{8}Le soir venu, le maître de la vigne dit à son intendant : “Appelle les ouvriers et distribue le salaire, en commençant par les derniers pour finir par les premiers.” 
${}^{9}Ceux qui avaient commencé à cinq heures s’avancèrent et reçurent chacun une pièce d’un denier. 
${}^{10}Quand vint le tour des premiers, ils pensaient recevoir davantage, mais ils reçurent, eux aussi, chacun une pièce d’un denier. 
${}^{11}En la recevant, ils récriminaient contre le maître du domaine : 
${}^{12}“Ceux-là, les derniers venus, n’ont fait qu’une heure, et tu les traites à l’égal de nous, qui avons enduré le poids du jour et la chaleur !”
${}^{13}Mais le maître répondit à l’un d’entre eux : “Mon ami, je ne suis pas injuste envers toi. N’as-tu pas été d’accord avec moi pour un denier ? 
${}^{14}Prends ce qui te revient, et va-t’en. Je veux donner au dernier venu autant qu’à toi : 
${}^{15}n’ai-je pas le droit de faire ce que je veux de mes biens ? Ou alors ton regard est-il mauvais parce que moi, je suis bon ?” 
${}^{16}C’est ainsi que les derniers seront premiers, et les premiers seront derniers. »
${}^{17}Montant alors à Jérusalem, Jésus prit à part les Douze disciples et, en chemin, il leur dit : 
${}^{18}« Voici que nous montons à Jérusalem. Le Fils de l’homme sera livré aux grands prêtres et aux scribes, ils le condamneront à mort 
${}^{19}et le livreront aux nations païennes pour qu’elles se moquent de lui, le flagellent et le crucifient ; le troisième jour, il ressuscitera. »
${}^{20}Alors la mère des fils de Zébédée s’approcha de Jésus avec ses fils Jacques et Jean, et elle se prosterna pour lui faire une demande. 
${}^{21}Jésus lui dit : « Que veux-tu ? » Elle répondit : « Ordonne que mes deux fils que voici siègent, l’un à ta droite et l’autre à ta gauche, dans ton Royaume. » 
${}^{22}Jésus répondit : « Vous ne savez pas ce que vous demandez. Pouvez-vous boire la coupe que je vais boire ? » Ils lui disent : « Nous le pouvons. » 
${}^{23}Il leur dit : « Ma coupe, vous la boirez ; quant à siéger à ma droite et à ma gauche, ce n’est pas à moi de l’accorder ; il y a ceux pour qui cela est préparé par mon Père. »
${}^{24}Les dix autres, qui avaient entendu, s’indignèrent contre les deux frères. 
${}^{25}Jésus les appela et dit : « Vous le savez : les chefs des nations les commandent en maîtres, et les grands font sentir leur pouvoir. 
${}^{26}Parmi vous, il ne devra pas en être ainsi : celui qui veut devenir grand parmi vous sera votre serviteur ; 
${}^{27}et celui qui veut être parmi vous le premier sera votre esclave. 
${}^{28}Ainsi, le Fils de l’homme n’est pas venu pour être servi, mais pour servir, et donner sa vie en rançon pour la multitude. »
${}^{29}Tandis que Jésus avec ses disciples sortait de Jéricho, une foule nombreuse se mit à le suivre. 
${}^{30}Et voilà que deux aveugles, assis au bord de la route, apprenant que Jésus passait, crièrent : « Prends pitié de nous, Seigneur, fils de David ! » 
${}^{31}La foule les rabroua pour les faire taire. Mais ils criaient encore plus fort : « Prends pitié de nous, Seigneur, fils de David ! » 
${}^{32}Jésus s’arrêta et les appela : « Que voulez-vous que je fasse pour vous ? » 
${}^{33}Ils répondent : « Seigneur, que nos yeux s’ouvrent ! » 
${}^{34}Saisi de compassion, Jésus leur toucha les yeux ; aussitôt ils retrouvèrent la vue, et ils le suivirent.
      
         
      \bchapter{}
      \begin{verse}
${}^{1}Jésus et ses disciples, approchant de Jérusalem, arrivèrent en vue de Bethphagé, sur les pentes du mont des Oliviers. Alors Jésus envoya deux disciples 
${}^{2}en leur disant : « Allez au village qui est en face de vous ; vous trouverez aussitôt une ânesse attachée et son petit avec elle. Détachez-les et amenez-les moi. 
${}^{3}Et si l’on vous dit quelque chose, vous répondrez : “Le Seigneur en a besoin”. Et aussitôt on les laissera partir. » 
${}^{4}Cela est arrivé pour que soit accomplie la parole prononcée par le prophète :
        ${}^{5}Dites à la fille de Sion :
        \\Voici ton roi qui vient vers toi,
        \\plein de douceur,
        \\monté sur une ânesse et un petit âne,
        \\le petit d’une bête de somme.
${}^{6}Les disciples partirent et firent ce que Jésus leur avait ordonné. 
${}^{7}Ils amenèrent l’ânesse et son petit, disposèrent sur eux leurs manteaux, et Jésus s’assit dessus. 
${}^{8}Dans la foule, la plupart étendirent leurs manteaux sur le chemin ; d’autres coupaient des branches aux arbres et en jonchaient la route. 
${}^{9}Les foules qui marchaient devant Jésus et celles qui suivaient criaient :
        \\« Hosanna au fils de David !
        \\Béni soit celui qui vient au nom du Seigneur !
        \\Hosanna au plus haut des cieux ! »
${}^{10}Comme Jésus entrait à Jérusalem, toute la ville fut en proie à l’agitation, et disait : « Qui est cet homme ? » 
${}^{11}Et les foules répondaient : « C’est le prophète Jésus, de Nazareth en Galilée. »
      <h2 class="intertitle" id="d85e340621">1. Controverses de Jésus à Jérusalem (21,12 – 23)</h2>
${}^{12}Jésus entra dans le Temple, et il expulsa tous ceux qui vendaient et achetaient dans le Temple ; il renversa les comptoirs des changeurs et les sièges des marchands de colombes. 
${}^{13}Il leur dit : « Il est écrit : Ma maison sera appelée maison de prière. Or vous, vous en faites une caverne de bandits. »
${}^{14}Des aveugles et des boiteux s’approchèrent de lui dans le Temple, et il les guérit. 
${}^{15}Les grands prêtres et les scribes s’indignèrent quand ils virent les actions étonnantes qu’il avait faites, et les enfants qui criaient dans le Temple : « Hosanna au fils de David ! » 
${}^{16}Ils dirent à Jésus : « Tu entends ce qu’ils disent ? » Jésus leur répond : « Oui. Vous n’avez donc jamais lu dans l’Écriture : De la bouche des enfants, des tout-petits, tu as fait monter une louange ? » 
${}^{17}Alors il les quitta et sortit de la ville en direction de Béthanie, où il passa la nuit.
${}^{18}Le matin, en revenant vers la ville, il eut faim. 
${}^{19}Voyant un figuier au bord du chemin, il s’en approcha, mais il n’y trouva rien d’autre que des feuilles, et il lui dit : « Que plus jamais aucun fruit ne vienne de toi. » Et à l’instant même, le figuier se dessécha. 
${}^{20}En voyant cela, les disciples s’étonnèrent et dirent : « Comment se fait-il que le figuier s’est desséché à l’instant même ? » 
${}^{21}Alors Jésus leur déclara : « Amen, je vous le dis : si vous avez la foi et si vous ne doutez pas, vous ne ferez pas seulement ce que j’ai fait au figuier ; vous pourrez même dire à cette montagne : “Enlève-toi de là, et va te jeter dans la mer”, et cela se produira. 
${}^{22}Tout ce que vous demanderez dans votre prière avec foi, vous l’obtiendrez. »
${}^{23}Jésus était entré dans le Temple, et, pendant qu’il enseignait, les grands prêtres et les anciens du peuple s’approchèrent de lui et demandèrent : « Par quelle autorité fais-tu cela, et qui t’a donné cette autorité ? » 
${}^{24}Jésus leur répliqua : « À mon tour, je vais vous poser une question, une seule ; et si vous me répondez, je vous dirai, moi aussi, par quelle autorité je fais cela : 
${}^{25}Le baptême de Jean, d’où venait-il ? du ciel ou des hommes ? » Ils faisaient en eux-mêmes ce raisonnement : « Si nous disons : “Du ciel”, il va nous dire : “Pourquoi donc n’avez-vous pas cru à sa parole ?” 
${}^{26}Si nous disons : “Des hommes”, nous devons redouter la foule, car tous tiennent Jean pour un prophète. » 
${}^{27}Ils répondirent donc à Jésus : « Nous ne savons pas ! » Il leur dit à son tour : « Moi, je ne vous dis pas non plus par quelle autorité je fais cela.
${}^{28}Quel est votre avis ? Un homme avait deux fils. Il vint trouver le premier et lui dit : “Mon enfant, va travailler aujourd’hui à la vigne.” 
${}^{29}Celui-ci répondit : “Je ne veux pas.” Mais ensuite, s’étant repenti, il y alla. 
${}^{30}Puis le père alla trouver le second et lui parla de la même manière. Celui-ci répondit : “Oui, Seigneur !” et il n’y alla pas. 
${}^{31}Lequel des deux a fait la volonté du père ? » Ils lui répondent : « Le premier. » Jésus leur dit : « Amen, je vous le déclare : les publicains et les prostituées vous précèdent dans le royaume de Dieu. 
${}^{32}Car Jean le Baptiste est venu à vous sur le chemin de la justice, et vous n’avez pas cru à sa parole ; mais les publicains et les prostituées y ont cru. Tandis que vous, après avoir vu cela, vous ne vous êtes même pas repentis plus tard pour croire à sa parole.
${}^{33}« Écoutez une autre parabole : Un homme était propriétaire d’un domaine ; il planta une vigne, l’entoura d’une clôture, y creusa un pressoir et bâtit une tour de garde. Puis il loua cette vigne à des vignerons, et partit en voyage. 
${}^{34}Quand arriva le temps des fruits, il envoya ses serviteurs auprès des vignerons pour se faire remettre le produit de sa vigne. 
${}^{35}Mais les vignerons se saisirent des serviteurs, frappèrent l’un, tuèrent l’autre, lapidèrent le troisième. 
${}^{36}De nouveau, le propriétaire envoya d’autres serviteurs plus nombreux que les premiers ; mais on les traita de la même façon. 
${}^{37}Finalement, il leur envoya son fils, en se disant : “Ils respecteront mon fils.” 
${}^{38}Mais, voyant le fils, les vignerons se dirent entre eux : “Voici l’héritier : venez ! tuons-le, nous aurons son héritage !” 
${}^{39}Ils se saisirent de lui, le jetèrent hors de la vigne et le tuèrent. 
${}^{40}Eh bien ! quand le maître de la vigne viendra, que fera-t-il à ces vignerons ? » 
${}^{41}On lui répond : « Ces misérables, il les fera périr misérablement. Il louera la vigne à d’autres vignerons, qui lui en remettront le produit en temps voulu. »
${}^{42}Jésus leur dit : « N’avez-vous jamais lu dans les Écritures :
        \\La pierre qu’ont rejetée les bâtisseurs
        \\est devenue la pierre d’angle :
        \\c’est là l’œuvre du Seigneur,
        \\la merveille devant nos yeux !
${}^{43}Aussi, je vous le dis : Le royaume de Dieu vous sera enlevé pour être donné à une nation qui lui fera produire ses fruits. 
${}^{44}Et tout homme qui tombera sur cette pierre s’y brisera ; celui sur qui elle tombera, elle le réduira en poussière ! »
${}^{45}En entendant les paraboles de Jésus, les grands prêtres et les pharisiens avaient bien compris qu’il parlait d’eux. 
${}^{46}Tout en cherchant à l’arrêter, ils eurent peur des foules, parce qu’elles le tenaient pour un prophète.
      
         
      \bchapter{}
      \begin{verse}
${}^{1}Jésus se mit de nouveau à leur parler et leur dit en paraboles : 
${}^{2}« Le royaume des Cieux est comparable à un roi qui célébra les noces de son fils. 
${}^{3}Il envoya ses serviteurs appeler à la noce les invités, mais ceux-ci ne voulaient pas venir. 
${}^{4}Il envoya encore d’autres serviteurs dire aux invités : “Voilà : j’ai préparé mon banquet, mes bœufs et mes bêtes grasses sont égorgés ; tout est prêt : venez à la noce.” 
${}^{5}Mais ils n’en tinrent aucun compte et s’en allèrent, l’un à son champ, l’autre à son commerce ; 
${}^{6}les autres empoignèrent les serviteurs, les maltraitèrent et les tuèrent. 
${}^{7}Le roi se mit en colère, il envoya ses troupes, fit périr les meurtriers et incendia leur ville.
${}^{8}Alors il dit à ses serviteurs : “Le repas de noce est prêt, mais les invités n’en étaient pas dignes. 
${}^{9}Allez donc aux croisées des chemins : tous ceux que vous trouverez, invitez-les à la noce.” 
${}^{10}Les serviteurs allèrent sur les chemins, rassemblèrent tous ceux qu’ils trouvèrent, les mauvais comme les bons, et la salle de noce fut remplie de convives. 
${}^{11}Le roi entra pour examiner les convives, et là il vit un homme qui ne portait pas le vêtement de noce. 
${}^{12}Il lui dit : “Mon ami, comment es-tu entré ici, sans avoir le vêtement de noce ?” L’autre garda le silence. 
${}^{13}Alors le roi dit aux serviteurs : “Jetez-le, pieds et poings liés, dans les ténèbres du dehors ; là, il y aura des pleurs et des grincements de dents.” 
${}^{14}Car beaucoup sont appelés, mais peu sont élus. »
${}^{15}Alors les pharisiens allèrent tenir conseil pour prendre Jésus au piège en le faisant parler. 
${}^{16}Ils lui envoient leurs disciples, accompagnés des partisans d’Hérode : « Maître, lui disent-ils, nous le savons : tu es toujours vrai et tu enseignes le chemin de Dieu en vérité ; tu ne te laisses influencer par personne, car ce n’est pas selon l’apparence que tu considères les gens. 
${}^{17}Alors, donne-nous ton avis : Est-il permis, oui ou non, de payer l’impôt à César, l’empereur ? »
${}^{18}Connaissant leur perversité, Jésus dit : « Hypocrites ! pourquoi voulez-vous me mettre à l’épreuve ? 
${}^{19}Montrez-moi la monnaie de l’impôt. » Ils lui présentèrent une pièce d’un denier. 
${}^{20}Il leur dit : « Cette effigie et cette inscription, de qui sont-elles ? » 
${}^{21}Ils répondirent : « De César. » Alors il leur dit :
        \\« Rendez donc à César ce qui est à César,
        \\et à Dieu ce qui est à Dieu. »
${}^{22}À ces mots, ils furent tout étonnés. Ils le laissèrent et s’en allèrent.
${}^{23}Ce jour-là, des sadducéens – ceux qui affirment qu’il n’y a pas de résurrection – s’approchèrent de Jésus et l’interrogèrent : 
${}^{24}« Maître, Moïse a dit : Si un homme meurt sans avoir d’enfants, le frère de cet homme épousera sa belle-sœur pour susciter une descendance à son frère. 
${}^{25}Il y avait chez nous sept frères : le premier, qui s’était marié, mourut ; et, comme il n’avait pas de descendance, il laissa sa femme à son frère. 
${}^{26}Pareillement, le deuxième, puis le troisième, jusqu’au septième, 
${}^{27}et finalement, après eux tous, la femme mourut. 
${}^{28}Alors, à la résurrection, duquel des sept sera-t-elle l’épouse, puisque chacun l’a eue pour épouse ? »
${}^{29}Jésus leur répondit : « Vous vous égarez, en méconnaissant les Écritures et la puissance de Dieu. 
${}^{30}À la résurrection, en effet, on ne prend ni femme ni mari, mais on est comme les anges dans le ciel. 
${}^{31}Et au sujet de la résurrection des morts, n’avez-vous pas lu ce qui vous a été dit par Dieu : 
${}^{32}Moi, je suis le Dieu d’Abraham, le Dieu d’Isaac, le Dieu de Jacob ? Il n’est pas le Dieu des morts, mais des vivants. »
${}^{33}Les foules qui l’avaient entendu étaient frappées par son enseignement.
${}^{34}Les pharisiens, apprenant qu’il avait fermé la bouche aux sadducéens, se réunirent, 
${}^{35}et l’un d’entre eux, un docteur de la Loi, posa une question à Jésus pour le mettre à l’épreuve : 
${}^{36}« Maître, dans la Loi, quel est le grand commandement ? » 
${}^{37}Jésus lui répondit : « Tu aimeras le Seigneur ton Dieu de tout ton cœur, de toute ton âme et de tout ton esprit. 
${}^{38}Voilà le grand, le premier commandement. 
${}^{39}Et le second lui est semblable : Tu aimeras ton prochain comme toi-même. 
${}^{40}De ces deux commandements dépend toute la Loi, ainsi que les Prophètes. »
${}^{41}Comme les pharisiens se trouvaient réunis, Jésus les interrogea : 
${}^{42}« Quel est votre avis au sujet du Christ ? de qui est-il le fils ? » Ils lui répondent : « De David. » 
${}^{43}Jésus leur réplique : « Comment donc David, inspiré par l’Esprit, peut-il l’appeler “Seigneur”, en disant :
${}^{44}Le Seigneur a dit à mon Seigneur :
        “Siège à ma droite
        \\jusqu’à ce que j’aie placé tes ennemis
        sous tes pieds” ?
${}^{45}Si donc David l’appelle Seigneur, comment peut-il être son fils ? »
${}^{46}Personne n’était capable de lui répondre un mot et, à partir de ce jour-là, nul n’osa plus l’interroger.
      
         
      \bchapter{}
      \begin{verse}
${}^{1}Alors Jésus s’adressa aux foules et à ses disciples, 
${}^{2}et il déclara : « Les scribes et les pharisiens enseignent dans la chaire de Moïse. 
${}^{3}Donc, tout ce qu’ils peuvent vous dire, faites-le et observez-le. Mais n’agissez pas d’après leurs actes, car ils disent et ne font pas. 
${}^{4}Ils attachent de pesants fardeaux, difficiles à porter, et ils en chargent les épaules des gens ; mais eux-mêmes ne veulent pas les remuer du doigt. 
${}^{5}Toutes leurs actions, ils les font pour être remarqués des gens : ils élargissent leurs phylactères et rallongent leurs franges ; 
${}^{6}ils aiment les places d’honneur dans les dîners, les sièges d’honneur dans les synagogues 
${}^{7}et les salutations sur les places publiques ; ils aiment recevoir des gens le titre de Rabbi.
${}^{8}Pour vous, ne vous faites pas donner le titre de Rabbi, car vous n’avez qu’un seul maître pour vous enseigner, et vous êtes tous frères. 
${}^{9}Ne donnez à personne sur terre le nom de père, car vous n’avez qu’un seul Père, celui qui est aux cieux. 
${}^{10}Ne vous faites pas non plus donner le titre de maîtres, car vous n’avez qu’un seul maître, le Christ. 
${}^{11}Le plus grand parmi vous sera votre serviteur. 
${}^{12}Qui s’élèvera sera abaissé, qui s’abaissera sera élevé.
${}^{13}Malheureux êtes-vous, scribes et pharisiens hypocrites, parce que vous fermez à clé le royaume des Cieux devant les hommes ; vous-mêmes, en effet, n’y entrez pas, et vous ne laissez pas entrer ceux qui veulent entrer !
${}^{15}Malheureux êtes-vous, scribes et pharisiens hypocrites, parce que vous parcourez la mer et la terre pour faire un seul converti, et quand c’est arrivé, vous faites de lui un homme voué à la géhenne, deux fois pire que vous !
${}^{16}Malheureux êtes-vous, guides aveugles, vous qui dites : “Si l’on fait un serment par le Sanctuaire, il est nul ; mais si l’on fait un serment par l’or du Sanctuaire, on doit s’en acquitter.” 
${}^{17}Insensés et aveugles ! Qu’est-ce qui est le plus important : l’or ? ou bien le Sanctuaire qui consacre cet or ? 
${}^{18}Vous dites encore : “Si l’on fait un serment par l’autel, il est nul ; mais si l’on fait un serment par l’offrande posée sur l’autel, on doit s’en acquitter.” 
${}^{19}Aveugles ! Qu’est-ce qui est le plus important : l’offrande ? ou bien l’autel qui consacre cette offrande ? 
${}^{20}Celui donc qui fait un serment par l’autel fait un serment par l’autel et par tout ce qui est posé dessus ; 
${}^{21}celui qui fait un serment par le Sanctuaire fait un serment par le Sanctuaire et par Celui qui l’habite ; 
${}^{22}et celui qui fait un serment par le ciel fait un serment par le trône de Dieu et par Celui qui siège sur ce trône.
${}^{23}Malheureux êtes-vous, scribes et pharisiens hypocrites, parce que vous payez la dîme sur la menthe, le fenouil et le cumin, mais vous avez négligé ce qui est le plus important dans la Loi : la justice, la miséricorde et la fidélité. Voilà ce qu’il fallait pratiquer sans négliger le reste. 
${}^{24}Guides aveugles ! Vous filtrez le moucheron, et vous avalez le chameau !
${}^{25}Malheureux êtes-vous, scribes et pharisiens hypocrites, parce que vous purifiez l’extérieur de la coupe et de l’assiette, mais l’intérieur est rempli de cupidité et d’intempérance ! 
${}^{26}Pharisien aveugle, purifie d’abord l’intérieur de la coupe, afin que l’extérieur aussi devienne pur.
${}^{27}Malheureux êtes-vous, scribes et pharisiens hypocrites, parce que vous ressemblez à des sépulcres blanchis à la chaux : à l’extérieur ils ont une belle apparence, mais l’intérieur est rempli d’ossements et de toutes sortes de choses impures. 
${}^{28}C’est ainsi que vous, à l’extérieur, pour les gens, vous avez l’apparence d’hommes justes, mais à l’intérieur vous êtes pleins d’hypocrisie et de mal.
${}^{29}Malheureux êtes-vous, scribes et pharisiens hypocrites, parce que vous bâtissez les sépulcres des prophètes, vous décorez les tombeaux des justes, 
${}^{30}et vous dites : “Si nous avions vécu à l’époque de nos pères, nous n’aurions pas été leurs complices pour verser le sang des prophètes.” 
${}^{31}Ainsi, vous témoignez contre vous-mêmes : vous êtes bien les fils de ceux qui ont assassiné les prophètes. 
${}^{32}Vous donc, mettez le comble à la mesure de vos pères ! 
${}^{33}Serpents, engeance de vipères, comment éviteriez-vous d’être condamnés à la géhenne ?
${}^{34}C’est pourquoi, voici que moi, j’envoie vers vous des prophètes, des sages et des scribes ; vous tuerez et crucifierez les uns, vous en flagellerez d’autres dans vos synagogues, vous les poursuivrez de ville en ville ; 
${}^{35}ainsi, sur vous retombera tout le sang des justes qui a été versé sur la terre, depuis le sang d’Abel le juste jusqu’au sang de Zacharie, fils de Barachie, que vous avez assassiné entre le sanctuaire et l’autel. 
${}^{36}Amen, je vous le dis : tout cela viendra sur cette génération.
${}^{37}Jérusalem, Jérusalem, toi qui tues les prophètes et qui lapides ceux qui te sont envoyés, combien de fois ai-je voulu rassembler tes enfants comme la poule rassemble ses poussins sous ses ailes, et vous n’avez pas voulu ! 
${}^{38}Voici que votre temple vous est laissé : il est désert. 
${}^{39}En effet, je vous le déclare : vous ne me verrez plus désormais jusqu’à ce que vous disiez : Béni soit celui qui vient au nom du Seigneur ! »
      <h2 class="intertitle" id="d85e341634">2. Cinquième discours : la venue du Fils de l’homme (24 – 25)</h2>
      
         
      \bchapter{}
      \begin{verse}
${}^{1}Jésus était sorti du Temple et s’en allait, lorsque ses disciples s’approchèrent pour lui faire remarquer les constructions du Temple. 
${}^{2}Alors, prenant la parole, il leur dit : « Vous voyez tout cela, n’est-ce pas ? Amen, je vous le dis : il ne restera pas ici pierre sur pierre ; tout sera détruit. » 
${}^{3}Puis, comme il s’était assis au mont des Oliviers, les disciples s’approchèrent de lui à l’écart pour lui demander : « Dis-nous quand cela arrivera, et quel sera le signe de ta venue et de la fin du monde. »
${}^{4}Jésus leur répondit : « Prenez garde que personne ne vous égare. 
${}^{5}Car beaucoup viendront sous mon nom, et diront : “C’est moi le Christ” ; alors ils égareront bien des gens.
${}^{6}Vous allez entendre parler de guerres et de rumeurs de guerre. Faites attention ! ne vous laissez pas effrayer, car il faut que cela arrive, mais ce n’est pas encore la fin. 
${}^{7}On se dressera nation contre nation, royaume contre royaume ; il y aura, en divers lieux, des famines et des tremblements de terre. 
${}^{8}Or tout cela n’est que le commencement des douleurs de l’enfantement.
${}^{9}Alors, vous serez livrés à la détresse, on vous tuera, vous serez détestés de toutes les nations à cause de mon nom. 
${}^{10}Alors ce sera pour beaucoup une occasion de chute ; ils se livreront les uns les autres, se détesteront les uns les autres. 
${}^{11}Beaucoup de faux prophètes se lèveront, et ils égareront bien des gens. 
${}^{12}À cause de l’ampleur du mal, la charité de la plupart des hommes se refroidira. 
${}^{13}Mais celui qui aura persévéré jusqu’à la fin, celui-là sera sauvé. 
${}^{14}Et cet Évangile du Royaume sera proclamé dans le monde entier ; il y aura là un témoignage pour toutes les nations. Alors viendra la fin.
${}^{15}Lorsque vous verrez l’Abomination de la désolation, installée dans le Lieu saint comme l’a dit le prophète Daniel – que le lecteur comprenne ! – 
${}^{16}alors, ceux qui seront en Judée, qu’ils s’enfuient dans les montagnes ; 
${}^{17}celui qui sera sur sa terrasse, qu’il ne descende pas pour emporter ce qu’il y a dans sa maison ; 
${}^{18}celui qui sera dans son champ, qu’il ne retourne pas en arrière pour emporter son manteau. 
${}^{19}Malheureuses les femmes qui seront enceintes et celles qui allaiteront en ces jours-là ! 
${}^{20}Priez pour que votre fuite n’arrive pas en hiver ni un jour de sabbat. 
${}^{21}Alors, en effet, il y aura une grande détresse, telle qu’il n’y en a jamais eu depuis le commencement du monde jusqu’à maintenant, et telle qu’il n’y en aura jamais plus. 
${}^{22}Et si le nombre de ces jours-là n’était pas abrégé, personne n’aurait la vie sauve ; mais à cause des élus, ces jours-là seront abrégés.
${}^{23}Alors si quelqu’un vous dit : “Voilà le Messie ! Il est là !” ou bien encore : “Il est là !”, n’en croyez rien. 
${}^{24}Il surgira des faux messies et des faux prophètes, ils produiront des signes grandioses et des prodiges, au point d’égarer, si c’était possible, même les élus. 
${}^{25}Voilà : je vous l’ai dit à l’avance. 
${}^{26}Si l’on vous dit : “Le voilà dans le désert”, ne sortez pas. Si l’on vous dit : “Le voilà dans le fond de la maison”, n’en croyez rien. 
${}^{27}En effet, comme l’éclair part de l’orient et brille jusqu’à l’occident, ainsi sera la venue du Fils de l’homme. 
${}^{28}Selon le proverbe : Là où se trouve le cadavre, là se rassembleront les vautours.
${}^{29}Aussitôt après la détresse de ces jours-là,
      <p class="retrait1">le soleil s’obscurcira
      <p class="retrait1">et la lune ne donnera plus sa clarté ;
      <p class="retrait1">les étoiles tomberont du ciel
      <p class="retrait1">et les puissances célestes seront ébranlées.
${}^{30}Alors paraîtra dans le ciel le signe du Fils de l’homme ; alors toutes les tribus de la terre se frapperont la poitrine et verront le Fils de l’homme venir sur les nuées du ciel, avec puissance et grande gloire. 
${}^{31}Il enverra ses anges avec une trompette retentissante, et ils rassembleront ses élus des quatre coins du monde, d’une extrémité des cieux jusqu’à l’autre.
${}^{32}Laissez-vous instruire par la parabole du figuier : dès que ses branches deviennent tendres et que ses feuilles sortent, vous savez que l’été est proche. 
${}^{33}De même, vous aussi, lorsque vous verrez tout cela, sachez que le Fils de l’homme est proche, à votre porte. 
${}^{34}Amen, je vous le dis : cette génération ne passera pas avant que tout cela n’arrive. 
${}^{35}Le ciel et la terre passeront, mes paroles ne passeront pas.
${}^{36}Quant à ce jour et à cette heure-là, nul ne les connaît, pas même les anges des cieux, pas même le Fils, mais seulement le Père, et lui seul.
${}^{37}Comme il en fut aux jours de Noé, ainsi en sera-t-il lors de la venue du Fils de l’homme. 
${}^{38}En ces jours-là, avant le déluge, on mangeait et on buvait, on prenait femme et on prenait mari, jusqu’au jour où Noé entra dans l’arche ; 
${}^{39}les gens ne se sont doutés de rien, jusqu’à ce que survienne le déluge qui les a tous engloutis : telle sera aussi la venue du Fils de l’homme. 
${}^{40}Alors deux hommes seront aux champs : l’un sera pris, l’autre laissé. 
${}^{41}Deux femmes seront au moulin en train de moudre : l’une sera prise, l’autre laissée. 
${}^{42}Veillez donc, car vous ne savez pas quel jour votre Seigneur vient. 
${}^{43}Comprenez-le bien : si le maître de maison avait su à quelle heure de la nuit le voleur viendrait, il aurait veillé et n’aurait pas laissé percer le mur de sa maison. 
${}^{44}Tenez-vous donc prêts, vous aussi : c’est à l’heure où vous n’y penserez pas que le Fils de l’homme viendra.
${}^{45}Que dire du serviteur fidèle et sensé à qui le maître a confié la charge des gens de sa maison, pour leur donner la nourriture en temps voulu ? 
${}^{46}Heureux ce serviteur que son maître, en arrivant, trouvera en train d’agir ainsi ! 
${}^{47}Amen, je vous le déclare : il l’établira sur tous ses biens.
${}^{48}Mais si ce mauvais serviteur se dit en lui-même : “Mon maître tarde”, 
${}^{49}et s’il se met à frapper ses compagnons, s’il mange et boit avec les ivrognes, 
${}^{50}alors quand le maître viendra, le jour où son serviteur ne s’y attend pas et à l’heure qu’il ne connaît pas, 
${}^{51}il l’écartera et lui fera partager le sort des hypocrites ; là, il y aura des pleurs et des grincements de dents.
      
         
      \bchapter{}
      \begin{verse}
${}^{1}« Alors, le royaume des Cieux sera comparable à dix jeunes filles invitées à des noces, qui prirent leur lampe pour sortir à la rencontre de l’époux. 
${}^{2}Cinq d’entre elles étaient insouciantes, et cinq étaient prévoyantes : 
${}^{3}les insouciantes avaient pris leur lampe sans emporter d’huile, 
${}^{4}tandis que les prévoyantes avaient pris, avec leurs lampes, des flacons d’huile. 
${}^{5}Comme l’époux tardait, elles s’assoupirent toutes et s’endormirent.
${}^{6}Au milieu de la nuit, il y eut un cri : “Voici l’époux ! Sortez à sa rencontre.” 
${}^{7}Alors toutes ces jeunes filles se réveillèrent et se mirent à préparer leur lampe. 
${}^{8}Les insouciantes demandèrent aux prévoyantes : “Donnez-nous de votre huile, car nos lampes s’éteignent.” 
${}^{9}Les prévoyantes leur répondirent : “Jamais cela ne suffira pour nous et pour vous, allez plutôt chez les marchands vous en acheter.”
${}^{10}Pendant qu’elles allaient en acheter, l’époux arriva. Celles qui étaient prêtes entrèrent avec lui dans la salle des noces, et la porte fut fermée. 
${}^{11}Plus tard, les autres jeunes filles arrivèrent à leur tour et dirent : “Seigneur, Seigneur, ouvre-nous !” 
${}^{12}Il leur répondit : “Amen, je vous le dis : je ne vous connais pas.”
${}^{13}Veillez donc, car vous ne savez ni le jour ni l’heure.
${}^{14}« C’est comme un homme qui partait en voyage : il appela ses serviteurs et leur confia ses biens. 
${}^{15}À l’un il remit une somme de cinq talents, à un autre deux talents, au troisième un seul talent, à chacun selon ses capacités. Puis il partit. Aussitôt, 
${}^{16}celui qui avait reçu les cinq talents s’en alla pour les faire valoir et en gagna cinq autres. 
${}^{17}De même, celui qui avait reçu deux talents en gagna deux autres. 
${}^{18}Mais celui qui n’en avait reçu qu’un alla creuser la terre et cacha l’argent de son maître.
${}^{19}Longtemps après, le maître de ces serviteurs revint et il leur demanda des comptes. 
${}^{20}Celui qui avait reçu cinq talents s’approcha, présenta cinq autres talents et dit : “Seigneur, tu m’as confié cinq talents ; voilà, j’en ai gagné cinq autres.” 
${}^{21}Son maître lui déclara : “Très bien, serviteur bon et fidèle, tu as été fidèle pour peu de choses, je t’en confierai beaucoup ; entre dans la joie de ton seigneur.”
${}^{22}Celui qui avait reçu deux talents s’approcha aussi et dit : “Seigneur, tu m’as confié deux talents ; voilà, j’en ai gagné deux autres.” 
${}^{23}Son maître lui déclara : “Très bien, serviteur bon et fidèle, tu as été fidèle pour peu de choses, je t’en confierai beaucoup ; entre dans la joie de ton seigneur.”
${}^{24}Celui qui avait reçu un seul talent s’approcha aussi et dit : “Seigneur, je savais que tu es un homme dur : tu moissonnes là où tu n’as pas semé, tu ramasses là où tu n’as pas répandu le grain. 
${}^{25}J’ai eu peur, et je suis allé cacher ton talent dans la terre. Le voici. Tu as ce qui t’appartient.” 
${}^{26}Son maître lui répliqua : “Serviteur mauvais et paresseux, tu savais que je moissonne là où je n’ai pas semé, que je ramasse le grain là où je ne l’ai pas répandu. 
${}^{27}Alors, il fallait placer mon argent à la banque ; et, à mon retour, je l’aurais retrouvé avec les intérêts. 
${}^{28}Enlevez-lui donc son talent et donnez-le à celui qui en a dix. 
${}^{29}À celui qui a, on donnera encore, et il sera dans l’abondance ; mais celui qui n’a rien se verra enlever même ce qu’il a. 
${}^{30}Quant à ce serviteur bon à rien, jetez-le dans les ténèbres extérieures ; là, il y aura des pleurs et des grincements de dents !”
${}^{31}« Quand le Fils de l’homme viendra dans sa gloire, et tous les anges avec lui, alors il siégera sur son trône de gloire. 
${}^{32}Toutes les nations seront rassemblées devant lui ; il séparera les hommes les uns des autres, comme le berger sépare les brebis des boucs : 
${}^{33}il placera les brebis à sa droite, et les boucs à gauche.
${}^{34}Alors le Roi dira à ceux qui seront à sa droite : “Venez, les bénis de mon Père, recevez en héritage le Royaume préparé pour vous depuis la fondation du monde. 
${}^{35}Car j’avais faim, et vous m’avez donné à manger ; j’avais soif, et vous m’avez donné à boire ; j’étais un étranger, et vous m’avez accueilli ; 
${}^{36}j’étais nu, et vous m’avez habillé ; j’étais malade, et vous m’avez visité ; j’étais en prison, et vous êtes venus jusqu’à moi !” 
${}^{37}Alors les justes lui répondront : “Seigneur, quand est-ce que nous t’avons vu… ? tu avais donc faim, et nous t’avons nourri ? tu avais soif, et nous t’avons donné à boire ? 
${}^{38}tu étais un étranger, et nous t’avons accueilli ? tu étais nu, et nous t’avons habillé ? 
${}^{39}tu étais malade ou en prison… Quand sommes-nous venus jusqu’à toi ?” 
${}^{40}Et le Roi leur répondra : “Amen, je vous le dis : chaque fois que vous l’avez fait à l’un de ces plus petits de mes frères, c’est à moi que vous l’avez fait.”
${}^{41}Alors il dira à ceux qui seront à sa gauche : “Allez-vous-en loin de moi, vous les maudits, dans le feu éternel préparé pour le diable et ses anges. 
${}^{42}Car j’avais faim, et vous ne m’avez pas donné à manger ; j’avais soif, et vous ne m’avez pas donné à boire ; 
${}^{43}j’étais un étranger, et vous ne m’avez pas accueilli ; j’étais nu, et vous ne m’avez pas habillé ; j’étais malade et en prison, et vous ne m’avez pas visité.” 
${}^{44}Alors ils répondront, eux aussi : “Seigneur, quand t’avons-nous vu avoir faim, avoir soif, être nu, étranger, malade ou en prison, sans nous mettre à ton service ?” 
${}^{45}Il leur répondra : “Amen, je vous le dis : chaque fois que vous ne l’avez pas fait à l’un de ces plus petits, c’est à moi que vous ne l’avez pas fait.”
${}^{46}Et ils s’en iront, ceux-ci au châtiment éternel, et les justes, à la vie éternelle. »
      <h2 class="intertitle" id="d85e342351">1. La Passion et la mort (26 – 27)</h2>
      
         
      \bchapter{}
      \begin{verse}
${}^{1}Lorsque Jésus eut terminé tout ce discours, il s’adressa à ses disciples : 
${}^{2}« Vous savez que la Pâque a lieu dans deux jours, et que le Fils de l’homme va être livré pour être crucifié. »
${}^{3}Alors les grands prêtres et les anciens du peuple se réunirent dans le palais du grand prêtre, qui s’appelait Caïphe ; 
${}^{4}ils tinrent conseil pour arrêter Jésus par ruse et le faire mourir. 
${}^{5}Mais ils se disaient : « Pas en pleine fête, afin qu’il n’y ait pas de troubles dans le peuple. »
${}^{6}Comme Jésus se trouvait à Béthanie dans la maison de Simon le lépreux, 
${}^{7}une femme s’approcha, portant un flacon d’albâtre contenant un parfum de grand prix. Elle le versa sur la tête de Jésus, qui était à table. 
${}^{8}Voyant cela, les disciples s’indignèrent en disant : « À quoi bon ce gaspillage ? 
${}^{9}On aurait pu, en effet, vendre ce parfum pour beaucoup d’argent, que l’on aurait donné à des pauvres. » 
${}^{10}Jésus s’en aperçut et leur dit : « Pourquoi tourmenter cette femme ? Il est beau, le geste qu’elle a fait à mon égard. 
${}^{11}Des pauvres, vous en aurez toujours avec vous, mais moi, vous ne m’aurez pas toujours. 
${}^{12}Si elle a fait cela, si elle a versé ce parfum sur mon corps, c’est en vue de mon ensevelissement. 
${}^{13}Amen, je vous le dis : partout où cet Évangile sera proclamé – dans le monde entier –, on racontera aussi, en souvenir d’elle, ce qu’elle vient de faire. »
${}^{14}Alors, l’un des Douze, nommé Judas Iscariote, se rendit chez les grands prêtres 
${}^{15}et leur dit : « Que voulez-vous me donner, si je vous le livre ? » Ils lui remirent trente pièces d’argent. 
${}^{16}Et depuis, Judas cherchait une occasion favorable pour le livrer.
${}^{17}Le premier jour de la fête des pains sans levain, les disciples s’approchèrent et dirent à Jésus : « Où veux-tu que nous te fassions les préparatifs pour manger la Pâque ? » 
${}^{18}Il leur dit : « Allez à la ville, chez un tel, et dites-lui : “Le Maître te fait dire : Mon temps est proche ; c’est chez toi que je veux célébrer la Pâque avec mes disciples.” » 
${}^{19}Les disciples firent ce que Jésus leur avait prescrit et ils préparèrent la Pâque.
${}^{20}Le soir venu, Jésus se trouvait à table avec les Douze. 
${}^{21}Pendant le repas, il déclara : « Amen, je vous le dis : l’un de vous va me livrer. » 
${}^{22}Profondément attristés, ils se mirent à lui demander, chacun son tour : « Serait-ce moi, Seigneur ? » 
${}^{23}Prenant la parole, il dit : « Celui qui s’est servi au plat en même temps que moi, celui-là va me livrer. 
${}^{24}Le Fils de l’homme s’en va, comme il est écrit à son sujet ; mais malheureux celui par qui le Fils de l’homme est livré ! Il vaudrait mieux pour lui qu’il ne soit pas né, cet homme-là ! » 
${}^{25}Judas, celui qui le livrait, prit la parole : « Rabbi, serait-ce moi ? » Jésus lui répond : « C’est toi-même qui l’as dit ! »
${}^{26}Pendant le repas, Jésus, ayant pris du pain et prononcé la bénédiction, le rompit et, le donnant aux disciples, il dit : « Prenez, mangez : ceci est mon corps. » 
${}^{27}Puis, ayant pris une coupe et ayant rendu grâce, il la leur donna, en disant : « Buvez-en tous, 
${}^{28}car ceci est mon sang, le sang de l’Alliance, versé pour la multitude en rémission des péchés. 
${}^{29}Je vous le dis : désormais je ne boirai plus de ce fruit de la vigne, jusqu’au jour où je le boirai, nouveau, avec vous dans le royaume de mon Père. »
${}^{30}Après avoir chanté les psaumes, ils partirent pour le mont des Oliviers. 
${}^{31}Alors Jésus leur dit : « Cette nuit, je serai pour vous tous une occasion de chute ; car il est écrit :
        \\Je frapperai le berger,
        \\et les brebis du troupeau seront dispersées.
${}^{32}Mais, une fois ressuscité, je vous précéderai en Galilée. » 
${}^{33}Prenant la parole, Pierre lui dit : « Si tous viennent à tomber à cause de toi, moi, je ne tomberai jamais. » 
${}^{34}Jésus lui répondit : « Amen, je te le dis : cette nuit même, avant que le coq chante, tu m’auras renié trois fois. » 
${}^{35}Pierre lui dit : « Même si je dois mourir avec toi, je ne te renierai pas. » Et tous les disciples dirent de même.
${}^{36}Alors Jésus parvient avec eux à un domaine appelé Gethsémani et leur dit : « Asseyez-vous ici, pendant que je vais là-bas pour prier. » 
${}^{37}Il emmena Pierre, ainsi que Jacques et Jean, les deux fils de Zébédée, et il commença à ressentir tristesse et angoisse. 
${}^{38}Il leur dit alors : « Mon âme est triste à en mourir. Restez ici et veillez avec moi. » 
${}^{39}Allant un peu plus loin, il tomba face contre terre en priant, et il disait : « Mon Père, s’il est possible, que cette coupe passe loin de moi ! Cependant, non pas comme moi, je veux, mais comme toi, tu veux. » 
${}^{40}Puis il revient vers ses disciples et les trouve endormis ; il dit à Pierre : « Ainsi, vous n’avez pas eu la force de veiller seulement une heure avec moi ? 
${}^{41}Veillez et priez, pour ne pas entrer en tentation ; l’esprit est ardent, mais la chair est faible. » 
${}^{42}De nouveau, il s’éloigna et pria, pour la deuxième fois ; il disait : « Mon Père, si cette coupe ne peut passer sans que je la boive, que ta volonté soit faite ! » 
${}^{43}Revenu près des disciples, de nouveau il les trouva endormis, car leurs yeux étaient lourds de sommeil. 
${}^{44}Les laissant, de nouveau il s’éloigna et pria pour la troisième fois, en répétant les mêmes paroles. 
${}^{45}Alors il revient vers les disciples et leur dit : « Désormais, vous pouvez dormir et vous reposer. Voici qu’elle est proche, l’heure où le Fils de l’homme est livré aux mains des pécheurs. 
${}^{46}Levez-vous ! Allons ! Voici qu’il est proche, celui qui me livre. »
${}^{47}Jésus parlait encore, lorsque Judas, l’un des Douze, arriva, et avec lui une grande foule armée d’épées et de bâtons, envoyée par les grands prêtres et les anciens du peuple. 
${}^{48}Celui qui le livrait leur avait donné un signe : « Celui que j’embrasserai, c’est lui : arrêtez-le. » 
${}^{49}Aussitôt, s’approchant de Jésus, il lui dit : « Salut, Rabbi ! » Et il l’embrassa. 
${}^{50}Jésus lui dit : « Mon ami, ce que tu es venu faire, fais-le ! » Alors ils s’approchèrent, mirent la main sur Jésus et l’arrêtèrent.
${}^{51}L’un de ceux qui étaient avec Jésus, portant la main à son épée, la tira, frappa le serviteur du grand prêtre, et lui trancha l’oreille. 
${}^{52}Alors Jésus lui dit : « Rentre ton épée, car tous ceux qui prennent l’épée périront par l’épée. 
${}^{53}Crois-tu que je ne puisse pas faire appel à mon Père ? Il mettrait aussitôt à ma disposition plus de douze légions d’anges. 
${}^{54}Mais alors, comment s’accompliraient les Écritures selon lesquelles il faut qu’il en soit ainsi ? » 
${}^{55}À ce moment-là, Jésus dit aux foules : « Suis-je donc un bandit, pour que vous soyez venus vous saisir de moi, avec des épées et des bâtons ? Chaque jour, dans le Temple, j’étais assis en train d’enseigner, et vous ne m’avez pas arrêté. » 
${}^{56}Mais tout cela est arrivé pour que s’accomplissent les écrits des prophètes. Alors tous les disciples l’abandonnèrent et s’enfuirent.
${}^{57}Ceux qui avaient arrêté Jésus l’amenèrent devant Caïphe, le grand prêtre, chez qui s’étaient réunis les scribes et les anciens. 
${}^{58}Quant à Pierre, il le suivait à distance, jusqu’au palais du grand prêtre ; il entra dans la cour et s’assit avec les serviteurs pour voir comment cela finirait.
${}^{59}Les grands prêtres et tout le Conseil suprême cherchaient un faux témoignage contre Jésus pour le faire mettre à mort. 
${}^{60}Ils n’en trouvèrent pas ; pourtant beaucoup de faux témoins s’étaient présentés. Finalement il s’en présenta deux, 
${}^{61}qui déclarèrent : « Celui-là a dit : “Je peux détruire le Sanctuaire de Dieu et, en trois jours, le rebâtir.” » 
${}^{62}Alors le grand prêtre se leva et lui dit : « Tu ne réponds rien ? Que dis-tu des témoignages qu’ils portent contre toi ? » 
${}^{63}Mais Jésus gardait le silence. Le grand prêtre lui dit : « Je t’adjure, par le Dieu vivant, de nous dire si c’est toi qui es le Christ, le Fils de Dieu. » 
${}^{64}Jésus lui répond : « C’est toi-même qui l’as dit ! En tout cas, je vous le déclare : désormais vous verrez le Fils de l’homme siéger à la droite du Tout-Puissant et venir sur les nuées du ciel. »
${}^{65}Alors le grand prêtre déchira ses vêtements, en disant : « Il a blasphémé ! Pourquoi nous faut-il encore des témoins ? Vous venez d’entendre le blasphème ! 
${}^{66}Quel est votre avis ? » Ils répondirent : « Il mérite la mort. » 
${}^{67}Alors ils lui crachèrent au visage et le giflèrent ; d’autres le rouèrent de coups 
${}^{68}en disant : « Fais-nous le prophète, ô Christ ! Qui t’a frappé ? »
${}^{69}Cependant Pierre était assis dehors dans la cour. Une jeune servante s’approcha de lui et lui dit : « Toi aussi, tu étais avec Jésus, le Galiléen ! » 
${}^{70}Mais il le nia devant tout le monde et dit : « Je ne sais pas de quoi tu parles. » 
${}^{71}Une autre servante le vit sortir en direction du portail et elle dit à ceux qui étaient là : « Celui-ci était avec Jésus, le Nazaréen. » 
${}^{72}De nouveau, Pierre le nia en faisant ce serment : « Je ne connais pas cet homme. » 
${}^{73}Peu après, ceux qui se tenaient là s’approchèrent et dirent à Pierre : « Sûrement, toi aussi, tu es l’un d’entre eux ! D’ailleurs, ta façon de parler te trahit. » 
${}^{74}Alors, il se mit à protester violemment et à jurer : « Je ne connais pas cet homme. » Et aussitôt un coq chanta. 
${}^{75}Alors Pierre se souvint de la parole que Jésus lui avait dite : « Avant que le coq chante, tu m’auras renié trois fois. » Il sortit et, dehors, pleura amèrement.
      
         
      \bchapter{}
      \begin{verse}
${}^{1}Le matin venu, tous les grands prêtres et les anciens du peuple tinrent conseil contre Jésus pour le faire mettre à mort. 
${}^{2}Après l’avoir ligoté, ils l’emmenèrent et le livrèrent à Pilate, le gouverneur.
      
         
${}^{3}Alors, en voyant que Jésus était condamné, Judas, qui l’avait livré, fut pris de remords ; il rendit les trente pièces d’argent aux grands prêtres et aux anciens. 
${}^{4}Il leur dit : « J’ai péché en livrant à la mort un innocent. » Ils répliquèrent : « Que nous importe ? Cela te regarde ! » 
${}^{5}Jetant alors les pièces d’argent dans le Temple, il se retira et alla se pendre. 
${}^{6}Les grands prêtres ramassèrent l’argent et dirent : « Il n’est pas permis de le verser dans le trésor, puisque c’est le prix du sang. » 
${}^{7}Après avoir tenu conseil, ils achetèrent avec cette somme le champ du potier pour y enterrer les étrangers. 
${}^{8}Voilà pourquoi ce champ est appelé jusqu’à ce jour le Champ-du-Sang. 
${}^{9}Alors fut accomplie la parole prononcée par le prophète Jérémie :
        \\Ils ramassèrent les trente pièces d’argent,
        \\le prix de celui qui fut mis à prix,
        \\le prix fixé par les fils d’Israël,
        ${}^{10}et ils les donnèrent pour le champ du potier,
        \\comme le Seigneur me l’avait ordonné.
${}^{11}On fit comparaître Jésus devant Pilate, le gouverneur, qui l’interrogea : « Es-tu le roi des Juifs ? » Jésus déclara : « C’est toi-même qui le dis. » 
${}^{12}Mais, tandis que les grands prêtres et les anciens l’accusaient, il ne répondit rien. 
${}^{13}Alors Pilate lui dit : « Tu n’entends pas tous les témoignages portés contre toi ? » 
${}^{14}Mais Jésus ne lui répondit plus un mot, si bien que le gouverneur fut très étonné.
${}^{15}Or, à chaque fête, celui-ci avait coutume de relâcher un prisonnier, celui que la foule demandait. 
${}^{16}Il y avait alors un prisonnier bien connu, nommé Barabbas. 
${}^{17}Les foules s’étant donc rassemblées, Pilate leur dit : « Qui voulez-vous que je vous relâche : Barabbas ? ou Jésus, appelé le Christ ? » 
${}^{18}Il savait en effet que c’était par jalousie qu’on avait livré Jésus.
${}^{19}Tandis qu’il siégeait au tribunal, sa femme lui fit dire : « Ne te mêle pas de l’affaire de ce juste, car aujourd’hui j’ai beaucoup souffert en songe à cause de lui. » 
${}^{20}Les grands prêtres et les anciens poussèrent les foules à réclamer Barabbas et à faire périr Jésus. 
${}^{21}Le gouverneur reprit : « Lequel des deux voulez-vous que je vous relâche ? » Ils répondirent : « Barabbas ! » 
${}^{22}Pilate leur dit : « Que ferai-je donc de Jésus appelé le Christ ? » Ils répondirent tous : « Qu’il soit crucifié ! » 
${}^{23}Pilate demanda : « Quel mal a-t-il donc fait ? » Ils criaient encore plus fort : « Qu’il soit crucifié ! » 
${}^{24}Pilate, voyant que ses efforts ne servaient à rien, sinon à augmenter le tumulte, prit de l’eau et se lava les mains devant la foule, en disant : « Je suis innocent du sang de cet homme : cela vous regarde ! » 
${}^{25}Tout le peuple répondit : « Son sang, qu’il soit sur nous et sur nos enfants ! » 
${}^{26}Alors, il leur relâcha Barabbas ; quant à Jésus, il le fit flageller, et il le livra pour qu’il soit crucifié.
${}^{27}Alors les soldats du gouverneur emmenèrent Jésus dans la salle du Prétoire et rassemblèrent autour de lui toute la garde. 
${}^{28}Ils lui enlevèrent ses vêtements et le couvrirent d’un manteau rouge. 
${}^{29}Puis, avec des épines, ils tressèrent une couronne, et la posèrent sur sa tête ; ils lui mirent un roseau dans la main droite et, pour se moquer de lui, ils s’agenouillaient devant lui en disant : « Salut, roi des Juifs ! » 
${}^{30}Et, après avoir craché sur lui, ils prirent le roseau, et ils le frappaient à la tête. 
${}^{31}Quand ils se furent bien moqués de lui, ils lui enlevèrent le manteau, lui remirent ses vêtements, et l’emmenèrent pour le crucifier.
${}^{32}En sortant, ils trouvèrent un nommé Simon, originaire de Cyrène, et ils le réquisitionnèrent pour porter la croix de Jésus. 
${}^{33}Arrivés en un lieu dit Golgotha, c’est-à-dire : Lieu-du-Crâne (ou Calvaire), 
${}^{34}ils donnèrent à boire à Jésus du vin mêlé de fiel ; il en goûta, mais ne voulut pas boire.
${}^{35}Après l’avoir crucifié, ils se partagèrent ses vêtements en tirant au sort ; 
${}^{36}et ils restaient là, assis, à le garder. 
${}^{37}Au-dessus de sa tête ils placèrent une inscription indiquant le motif de sa condamnation : « Celui-ci est Jésus, le roi des Juifs. » 
${}^{38}Alors on crucifia avec lui deux bandits, l’un à droite et l’autre à gauche.
${}^{39}Les passants l’injuriaient en hochant la tête ; 
${}^{40}ils disaient : « Toi qui détruis le Sanctuaire et le rebâtis en trois jours, sauve-toi toi-même, si tu es Fils de Dieu, et descends de la croix ! » 
${}^{41}De même, les grands prêtres se moquaient de lui avec les scribes et les anciens, en disant : 
${}^{42}« Il en a sauvé d’autres, et il ne peut pas se sauver lui-même ! Il est roi d’Israël : qu’il descende maintenant de la croix, et nous croirons en lui ! 
${}^{43}Il a mis sa confiance en Dieu. Que Dieu le délivre maintenant, s’il l’aime ! Car il a dit : “Je suis Fils de Dieu.” » 
${}^{44}Les bandits crucifiés avec lui l’insultaient de la même manière.
${}^{45}À partir de la sixième heure (c’est-à-dire : midi), l’obscurité se fit sur toute la terre jusqu’à la neuvième heure.
${}^{46}Vers la neuvième heure, Jésus cria d’une voix forte :
      « Éli, Éli, lema sabactani ? »,
      ce qui veut dire : « Mon Dieu, mon Dieu, pourquoi m’as-tu abandonné ? » 
${}^{47}L’ayant entendu, quelques-uns de ceux qui étaient là disaient : « Le voilà qui appelle le prophète Élie ! » 
${}^{48}Aussitôt l’un d’eux courut prendre une éponge qu’il trempa dans une boisson vinaigrée ; il la mit au bout d’un roseau, et il lui donnait à boire. 
${}^{49}Les autres disaient : « Attends ! Nous verrons bien si Élie vient le sauver. » 
${}^{50}Mais Jésus, poussant de nouveau un grand cri, rendit l’esprit.
${}^{51}Et voici que le rideau du Sanctuaire se déchira en deux, depuis le haut jusqu’en bas ; la terre trembla et les rochers se fendirent. 
${}^{52}Les tombeaux s’ouvrirent ; les corps de nombreux saints qui étaient morts ressuscitèrent, 
${}^{53}et, sortant des tombeaux après la résurrection de Jésus, ils entrèrent dans la Ville sainte, et se montrèrent à un grand nombre de gens. 
${}^{54}À la vue du tremblement de terre et de ces événements, le centurion et ceux qui, avec lui, gardaient Jésus, furent saisis d’une grande crainte et dirent :
      « Vraiment, celui-ci était Fils de Dieu ! »
${}^{55}Il y avait là de nombreuses femmes qui observaient de loin. Elles avaient suivi Jésus depuis la Galilée pour le servir. 
${}^{56}Parmi elles se trouvaient Marie Madeleine, Marie, mère de Jacques et de Joseph, et la mère des fils de Zébédée.
${}^{57}Comme il se faisait tard, arriva un homme riche, originaire d’Arimathie, qui s’appelait Joseph, et qui était devenu, lui aussi, disciple de Jésus. 
${}^{58}Il alla trouver Pilate pour demander le corps de Jésus. Alors Pilate ordonna qu’on le lui remette. 
${}^{59}Prenant le corps, Joseph l’enveloppa dans un linceul immaculé, 
${}^{60}et le déposa dans le tombeau neuf qu’il s’était fait creuser dans le roc. Puis il roula une grande pierre à l’entrée du tombeau et s’en alla.
${}^{61}Or Marie Madeleine et l’autre Marie étaient là, assises en face du sépulcre.
${}^{62}Le lendemain, après le jour de la Préparation, les grands prêtres et les pharisiens s’assemblèrent chez Pilate, 
${}^{63}en disant : « Seigneur, nous nous sommes rappelé que cet imposteur a dit, de son vivant : “Trois jours après, je ressusciterai.” 
${}^{64}Alors, donne l’ordre que le sépulcre soit surveillé jusqu’au troisième jour, de peur que ses disciples ne viennent voler le corps et ne disent au peuple : “Il est ressuscité d’entre les morts.” Cette dernière imposture serait pire que la première. » 
${}^{65}Pilate leur déclara : « Vous avez une garde. Allez, organisez la surveillance comme vous l’entendez ! » 
${}^{66}Ils partirent donc et assurèrent la surveillance du sépulcre en mettant les scellés sur la pierre et en y plaçant la garde.
      <h2 class="intertitle" id="d85e343485">2. Le tombeau vide et les apparitions (28)</h2>
      
         
      \bchapter{}
      \begin{verse}
${}^{1}Après le sabbat, à l’heure où commençait à poindre le premier jour de la semaine, Marie Madeleine et l’autre Marie vinrent pour regarder le sépulcre. 
${}^{2}Et voilà qu’il y eut un grand tremblement de terre ; l’ange du Seigneur descendit du ciel, vint rouler la pierre et s’assit dessus. 
${}^{3}Il avait l’aspect de l’éclair, et son vêtement était blanc comme neige. 
${}^{4}Les gardes, dans la crainte qu’ils éprouvèrent, se mirent à trembler et devinrent comme morts.
${}^{5}L’ange prit la parole et dit aux femmes : « Vous, soyez sans crainte ! Je sais que vous cherchez Jésus le Crucifié. 
${}^{6}Il n’est pas ici, car il est ressuscité, comme il l’avait dit. Venez voir l’endroit où il reposait. 
${}^{7}Puis, vite, allez dire à ses disciples :
        \\“Il est ressuscité d’entre les morts,
        \\et voici qu’il vous précède en Galilée ;
        \\là, vous le verrez.”
        \\Voilà ce que j’avais à vous dire. »
${}^{8}Vite, elles quittèrent le tombeau, remplies à la fois de crainte et d’une grande joie, et elles coururent porter la nouvelle à ses disciples.
${}^{9}Et voici que Jésus vint à leur rencontre et leur dit : « Je vous salue. » Elles s’approchèrent, lui saisirent les pieds et se prosternèrent devant lui. 
${}^{10}Alors Jésus leur dit : « Soyez sans crainte, allez annoncer à mes frères qu’ils doivent se rendre en Galilée : c’est là qu’ils me verront. »
${}^{11}Tandis qu’elles étaient en chemin, quelques-uns des gardes allèrent en ville annoncer aux grands prêtres tout ce qui s’était passé. 
${}^{12}Ceux-ci, après s’être réunis avec les anciens et avoir tenu conseil, donnèrent aux soldats une forte somme 
${}^{13}en disant : « Voici ce que vous direz : “Ses disciples sont venus voler le corps, la nuit pendant que nous dormions.” 
${}^{14}Et si tout cela vient aux oreilles du gouverneur, nous lui expliquerons la chose, et nous vous éviterons tout ennui. » 
${}^{15}Les soldats prirent l’argent et suivirent les instructions. Et cette explication s’est propagée chez les Juifs jusqu’à aujourd’hui.
${}^{16}Les onze disciples s’en allèrent en Galilée, à la montagne où Jésus leur avait ordonné de se rendre. 
${}^{17}Quand ils le virent, ils se prosternèrent, mais certains eurent des doutes. 
${}^{18}Jésus s’approcha d’eux et leur adressa ces paroles : « Tout pouvoir m’a été donné au ciel et sur la terre. 
${}^{19}Allez ! De toutes les nations faites des disciples : baptisez-les au nom du Père, et du Fils, et du Saint-Esprit, 
${}^{20}apprenez-leur à observer tout ce que je vous ai commandé. Et moi, je suis avec vous tous les jours jusqu’à la fin du monde. »
