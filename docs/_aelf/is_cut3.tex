  
  
      <h2 class="intertitle" id="d85e245672">4. Apocalypse (24 – 27)</h2>
      
         
      \bchapter{}
${}^{1}Voici que le Seigneur saccage la terre,
        qu’il la ravage,
        \\qu’il en bouleverse la face,
        qu’il en disperse les habitants.
${}^{2}Il en sera du prêtre comme du peuple,
        du maître comme de l’esclave,
        \\de la maîtresse comme de la servante,
        du vendeur comme de l’acheteur,
        \\du prêteur comme de l’emprunteur,
        du créancier comme du débiteur.
${}^{3}Saccagée, elle est saccagée, la terre ;
        pillée, elle est pillée.
        \\Car le Seigneur a proféré cette parole.
        
           
         
${}^{4}La terre est en deuil, elle s’épuise,
        le monde dépérit, il s’épuise,
        \\et le ciel dépérit en même temps que la terre.
${}^{5}La terre est profanée par ses habitants :
        ils ont transgressé les lois,
        \\ils ont changé les décrets,
        ils ont rompu l’alliance éternelle.
${}^{6}C’est pourquoi la malédiction dévore la terre :
        ses habitants en subissent la peine ;
        \\c’est pourquoi les habitants de la terre diminuent :
        il n’en reste qu’un petit nombre.
${}^{7}Deuil pour le vin nouveau : la vigne a dépéri !
        Tous ceux qui avaient le cœur en fête se lamentent.
${}^{8}Elle a cessé, l’allégresse des tambourins ;
        il a pris fin, le joyeux vacarme ;
        \\elle a cessé, l’allégresse des cithares !
${}^{9}Ils ne boiront plus de vin en chantant ;
        la boisson forte est amère aux buveurs.
${}^{10}La cité-du-néant est en ruine,
        chaque maison est fermée,
        nul ne peut y entrer.
${}^{11}Dans la rue, on réclame du vin ;
        toute joie a disparu ;
        l’allégresse est bannie du pays.
${}^{12}Il ne reste de la ville que désolation :
        sa porte est brisée, fracassée.
${}^{13}Au cœur du pays,
        au milieu des populations,
        \\il en sera comme à la cueillette des olives,
        comme au grappillage après la vendange.
        
           
         
${}^{14}Ceux qui restent élèvent la voix, ils crient de joie ;
        du côté de la mer, on célèbre la grandeur du Seigneur ;
${}^{15}au pays de la lumière, on glorifie le Seigneur
        et, dans les îles de la mer,
        le nom du Seigneur, Dieu d’Israël.
${}^{16}Depuis les limites de la terre nous entendons des hymnes :
        « Honneur à Dieu le juste ! »
        \\Mais je dis : « Quelle épreuve pour moi !
        Quelle épreuve pour moi ! Malheur à moi ! »
        \\Ils ont ravagé, les ravageurs !
        Ravage : les ravageurs ont fait des ravages !
${}^{17}La frayeur, la fosse et le filet
        sont pour toi, habitant de la terre.
${}^{18}Celui qui fuit devant des cris de frayeur
        tombe dans la fosse ;
        \\celui qui remonte de la fosse
        est pris dans le filet !
        \\Oui, les vannes d’en-haut s’ouvriront,
        les fondements de la terre trembleront.
${}^{19}La terre se brise, se brise en morceaux !
        La terre éclate, elle vole en éclats !
        \\La terre frémit, frémit tout entière !
${}^{20}La terre vacille, vacille comme un ivrogne,
        comme une cabane branlante ;
        \\son forfait pèse sur elle,
        elle tombe sans pouvoir se relever.
        
           
         
${}^{21}Ce jour-là, il arrivera
        \\que le Seigneur viendra sévir
        là-haut, contre l’armée d’en haut,
        \\et sur la terre
        contre les rois de la terre.
${}^{22}Ils seront entassés,
        enchaînés dans un cachot,
        prisonniers d’une prison.
        \\Après de nombreux jours,
        on sévira contre eux.
${}^{23}La lune rougira,
        le soleil se couvrira de honte.
        
           
         
        \\Car, sur le mont Sion et à Jérusalem,
        le Seigneur de l’univers régnera :
        devant les anciens resplendira sa gloire.
        
           
      
         
      \bchapter{}
${}^{1}Seigneur, tu es mon Dieu, je t’exalte,
        je rends grâce à ton nom,
        \\car tu as accompli projets et merveilles,
        sûrs et stables depuis longtemps.
${}^{2}Tu as changé la ville en tas de pierres,
        la cité fortifiée, en champ de ruines ;
        \\la citadelle des étrangers n’est plus une ville,
        jamais elle ne sera rebâtie :
${}^{3}voilà pourquoi un peuple fort reconnaît ta gloire,
        les cités des nations tyranniques te craignent.
${}^{4}Tu es devenu forteresse pour le faible,
        forteresse pour le malheureux en sa détresse,
        \\un abri contre l’orage,
        une ombre contre la chaleur :
        \\le souffle des tyrans
        n’est que pluie d’orage sur un mur.
${}^{5}Comme une chaleur étouffante sur la terre desséchée,
        tu étouffes le vacarme des étrangers ;
        \\comme faiblit la chaleur à l’ombre d’un nuage,
        ainsi faiblit le chant de victoire des tyrans.
        
           
        ${}^{6}Le Seigneur de l’univers
        préparera pour tous les peuples, sur sa\\montagne,
        \\un festin de viandes grasses et de vins capiteux,
        un festin de viandes succulentes et de vins décantés.
        ${}^{7}Sur cette montagne, il fera disparaître
        le voile de deuil qui enveloppe tous les peuples
        et le linceul qui couvre toutes les nations.
        ${}^{8}Il fera disparaître la mort pour toujours.
        Le Seigneur Dieu essuiera les larmes sur tous les visages,
        \\et par toute la terre il effacera l’humiliation de son peuple.
        Le Seigneur a parlé.
         
        ${}^{9}Et ce jour-là, on dira :
        « Voici notre Dieu,
        en lui nous espérions, et il nous a sauvés ;
        c’est lui le Seigneur,
        en lui nous espérions ;
        exultons, réjouissons-nous :
        il nous a sauvés ! »
        ${}^{10}Car la main du Seigneur
        reposera sur cette montagne.
         
        \\Mais Moab sera piétiné sur place,
        comme la paille est piétinée dans le fumier.
${}^{11}Là, il étendra les mains,
        comme un nageur les étend pour nager ;
        \\malgré ses mouvements habiles,
        Dieu rabattra son arrogance.
${}^{12}Moab, les bastions inaccessibles de tes murailles,
        il les renverse, il les abat,
        les jette à terre, dans la poussière.
      <p class="cantique" id="bib_ct-at_20"><span class="cantique_label">Cantique AT 20</span> = <span class="cantique_ref"><a class="unitex_link" href="#bib_is_26_1">Is 26, 1b-4.7-9.12</a></span>
      
         
      \bchapter{}
        ${}^{1}En ce jour-là, ce cantique sera chanté
        \\dans le pays de Juda :
        
           
       
        \\Nous avons une ville forte !
        \\Le Seigneur\\a mis pour sauvegarde
        muraille et avant-mur.
        ${}^{2}Ouvrez les portes !
        \\Elle entrera, la nation juste,
        qui se garde fidèle.
        ${}^{3}Immuable en ton dessein, tu préserves la paix,
        la paix de qui s’appuie sur toi.
        ${}^{4}Prenez appui sur le Seigneur, à jamais,
        sur lui, le Seigneur, le Roc éternel.
        ${}^{5}\[Il a rabaissé ceux qui siégeaient dans les hauteurs,
        il a humilié la cité inaccessible,
        \\l’a humiliée jusqu’à terre,
        et lui a fait mordre la poussière.
        ${}^{6}Elle sera foulée aux pieds,
        sous le pied des pauvres,
        les pas des faibles.\]
        ${}^{7}Il est droit, le chemin du juste ;
        toi qui es droit, tu aplanis le sentier du juste.
        ${}^{8}Oui, sur le chemin de tes jugements,
        Seigneur, nous t’espérons.
        \\Dire ton nom, faire mémoire de toi\\,
        c’est le désir de l’âme.
        ${}^{9}Mon âme, la nuit, te désire,
        et mon esprit, au fond de moi, te guette dès l’aurore\\.
        \\Quand s’exercent tes jugements sur la terre,
        les habitants du monde apprennent la justice.
${}^{10}\[Si l’on fait grâce au méchant,
        il n’apprend pas la justice ;
        \\perfide sur la terre de probité,
        il ne voit pas la majesté du Seigneur.
${}^{11}Seigneur, ta main est levée :
        ils ne l’aperçoivent pas ;
        \\mais ils percevront, pleins de honte,
        ta passion pour le peuple.
        \\En vérité, le feu les dévorera,
        celui que tu destines à tes ennemis !\]
         
        ${}^{12}Seigneur, tu nous assures la paix :
        dans toutes nos œuvres, toi-même agis pour nous.
       
${}^{13}Seigneur notre Dieu,
        d’autres maîtres que toi ont dominé sur nous,
        \\mais de toi seul nous faisons mémoire,
        de ton seul nom.
${}^{14}Ceux-là sont morts, ils ne revivront pas ;
        ce sont des ombres, ils ne se relèveront pas :
        \\voilà pourquoi tu interviens, tu les extermines,
        tu effaces jusqu’à leur mémoire.
         
${}^{15}Tu as fait grandir la nation, Seigneur,
        tu as fait grandir notre nation, tu en es glorifié,
        tu as repoussé toutes les limites du pays.
         
        ${}^{16}Seigneur, dans la détresse on a recours à toi ;
        \\quand tu envoies un châtiment,
        on s’efforce de le conjurer.
        ${}^{17}Nous étions devant toi, Seigneur,
        comme la femme enceinte sur le point d’enfanter,
        qui se tord et crie dans les douleurs.
        ${}^{18}Nous avons conçu, nous avons été dans les douleurs,
        mais nous n’avons enfanté que du vent :
        \\nous n’apportons pas le salut à la terre,
        nul habitant du monde ne vient\\à la vie.
        ${}^{19}Tes morts revivront,
        leurs cadavres\\se lèveront.
        \\Ils se réveilleront, crieront de joie,
        ceux qui demeurent dans la poussière\\,
        \\car ta rosée, Seigneur\\, est rosée de lumière\\,
        et le pays des ombres redonnera la vie.
${}^{20}Va, mon peuple, rentre dans tes maisons,
        ferme sur toi les portes ;
        \\cache-toi un court instant,
        pendant que passe la colère.
${}^{21}Car voici le Seigneur qui sort de son lieu saint
        pour châtier la faute des habitants de la terre ;
        \\la terre laissera voir le sang versé
        et ne recouvrira plus ses victimes.
       
      
         
      \bchapter{}
${}^{1}Ce jour-là, le Seigneur châtiera
        \\de son épée dure et grande et forte,
        \\Léviathan, le serpent fuyard,
        Léviathan, le serpent tortueux ;
        \\il tuera le dragon de la mer.
        
           
         
${}^{2}Ce jour-là, chantez la vigne exquise !
${}^{3}Moi, le Seigneur, j’en suis le gardien ;
        je l’arrose en temps voulu.
        \\De peur qu’on ne la visite,
        je la garde nuit et jour.
${}^{4}Je ne suis plus en fureur,
        mais si je trouve des épines et des ronces,
        \\je leur ferai la guerre,
        je les brûlerai toutes,
${}^{5}à moins que l’on cherche ma protection,
        que l’on fasse la paix avec moi,
        oui, qu’avec moi on fasse la paix.
${}^{6}À l’avenir,
        Jacob s’enracinera,
        \\Israël fleurira et poussera ses bourgeons,
        la face du monde sera couverte de fruits.
        
           
         
${}^{7}Israël a-t-il été frappé,
        comme le Seigneur avait frappé ceux qui le frappaient ?
        \\A-t-il été égorgé,
        comme il avait égorgé ses égorgeurs ?
${}^{8}Il l’a condamné au bannissement,
        à la déportation,
        \\il l’a chassé par le souffle violent
        d’un jour de vent d’est.
${}^{9}C’est ainsi que sera expiée la faute de Jacob ;
        et tel sera le fruit de la rémission de son péché :
        \\il traitera toutes les pierres des autels
        comme les pierres à chaux que l’on pulvérise,
        \\poteaux sacrés et colonnes à encens ne se dresseront plus.
        
           
         
${}^{10}La ville fortifiée, à l’écart,
        est un lieu dépeuplé, abandonné comme un désert.
        \\Le veau y viendra paître ; là, il se couchera,
        il saccagera ses arbustes.
${}^{11}Ses rameaux qui ont séché seront brisés :
        des femmes viennent qui les brûlent.
        \\Oui, ce peuple est sans discernement,
        \\voilà pourquoi Celui qui l’a fait n’en a pas compassion :
        Celui qui l’a façonné ne lui fera pas grâce.
        
           
         
${}^{12}Il arrivera, en ce jour-là,
        \\que le Seigneur fera le battage des épis
        depuis l’Euphrate jusqu’au Torrent d’Égypte ;
        \\quant à vous, fils d’Israël,
        vous serez glanés un par un.
        
           
         
${}^{13}Il arrivera, en ce jour-là,
        que l’on sonnera de la grande trompe ;
        \\ils viendront, ceux qui étaient perdus au pays d’Assour,
        et ceux qui étaient dispersés au pays d’Égypte ;
        \\ils se prosterneront devant le Seigneur,
        sur sa montagne sainte, à Jérusalem.
        
           
      <h2 class="intertitle" id="d85e247178">5. Sur Israël et Juda (28 – 33)</h2>
      
         
      \bchapter{}
${}^{1}Malheur ! Samarie,
        insolente couronne des buveurs d’Éphraïm,
        \\fleur qui se fane, splendide parure,
        dominant une vallée plantureuse !
        \\Vous êtes assommés par le vin !
${}^{2}Voici au service du Seigneur un homme fort et vigoureux,
        comme un orage de grêle, une tempête dévastatrice,
        \\un orage d’eaux puissantes, torrentielles :
        de sa main il va tout mettre à terre.
${}^{3}Elle sera foulée aux pieds,
        l’insolente couronne des buveurs d’Éphraïm.
${}^{4}La fleur qui se fane, splendide parure,
        dominant la vallée plantureuse,
        sera comme une figue précoce avant l’été :
        \\sitôt vue, sitôt cueillie,
        sitôt avalée.
        
           
         
${}^{5}Ce jour-là, le Seigneur de l’univers
        deviendra prestigieuse couronne,
        \\diadème de splendeur,
        pour le reste de son peuple ;
${}^{6}il deviendra esprit de jugement
        pour qui siège au jugement,
        \\et vaillance de ceux qui repoussent l’assaut
        devant la porte.
        
           
${}^{7}En voici d’autres que le vin égare,
        que la boisson forte fait divaguer :
        \\prêtre et prophète, la boisson les égare,
        ils sont troublés par le vin,
        ils divaguent sous l’effet de la boisson,
        s’égarent dans leurs visions
        et s’embrouillent dans leurs sentences.
${}^{8}Leurs tables sont couvertes d’infectes vomissures :
        plus un endroit propre.
${}^{9}Ils disent : « À qui veut-il faire la leçon ?
        À qui veut-il faire entendre le message ?
        \\À des enfants à peine sevrés,
        qui ne prennent plus le sein ?
${}^{10}Écoutez-le : “Fais-ci, fais-ça ;
        par-ci, par-là ;
        un peu ci, un peu là !” »
${}^{11}Eh bien, oui : c’est dans une langue ridicule,
        dans un jargon étrange,
        qu’il parlera à ce peuple.
${}^{12}Il leur avait dit : « C’est le repos !
        Laissez l’accablé se reposer !
        C’est le répit ! »,
        \\mais ils n’ont pas voulu écouter.
${}^{13}Voici donc ce que leur dira le Seigneur :
        « Fais-ci, fais-ça ;
        par-ci, par-là ;
        un peu ci, un peu là ! »,
        \\afin qu’en marchant ils trébuchent et tombent à la renverse,
        soient brisés, pris au piège et capturés.
${}^{14}C’est pourquoi, écoutez la parole du Seigneur,
        vous, les moqueurs,
        \\vous qui gouvernez ce peuple
        qui est à Jérusalem.
${}^{15}Vous dites : « Nous avons conclu
        une alliance avec la mort ;
        \\avec le séjour des morts
        nous avons fait un pacte ;
        \\quand passera le flot torrentiel,
        il ne nous atteindra pas ;
        \\car nous faisons du mensonge notre abri,
        dans la tromperie nous sommes cachés. »
         
        ${}^{16}Voilà pourquoi, ainsi parle le Seigneur Dieu :
        \\Moi, dans Sion, je pose une pierre,
        une pierre à toute épreuve,
        \\choisie pour être une pierre d’angle,
        une véritable pierre de fondement.
        \\Celui qui croit ne s’inquiétera pas.
        ${}^{17}Je prendrai le droit comme cordeau,
        et la justice comme fil à plomb.
        \\Mais la grêle balaiera l’abri de mensonge
        et les eaux submergeront le refuge caché.
${}^{18}Votre alliance avec la mort se rompra,
        votre pacte avec le séjour des morts ne tiendra pas.
        \\Quand passera le flot torrentiel,
        vous serez broyés.
${}^{19}Chaque fois qu’il passera,
        il vous prendra ;
        \\il passera matin après matin,
        et le jour et la nuit ;
        \\ne restera que l’épouvante
        d’en apprendre la nouvelle.
${}^{20}Car : « Trop court est le lit pour se coucher,
        et trop étroite, la couverture pour s’y blottir. »
${}^{21}Oui, comme sur le mont Peracim,
        le Seigneur se dressera ;
        \\comme dans la vallée de Gabaon,
        il frémira d’indignation,
        \\pour accomplir son œuvre,
        – étrange est son œuvre –,
        \\pour se mettre à son travail,
        – insolite est son travail.
${}^{22}Et maintenant, ne vous moquez pas
        de peur que vos liens ne se resserrent :
        \\c’est la destruction,
        j’ai entendu qu’elle a été décidée
        \\par le Seigneur, Dieu de l’univers,
        contre tout le pays.
${}^{23}Tendez l’oreille, écoutez ma voix ;
        soyez attentifs, écoutez ma parole :
${}^{24}pour semer, faut-il que, tout le jour,
        le laboureur laboure sa terre, la retourne et la herse ?
${}^{25}Ne va-t-il pas aplanir le sol,
        pour répandre la nigelle, jeter le cumin,
        \\semer en pleine terre le blé, l’orge, le millet,
        et l’épeautre en bordure ?
${}^{26}Son Dieu lui enseigne cette pratique ;
        il l’a instruit :
${}^{27}on n’écrase pas la nigelle au traîneau,
        la roue du chariot ne passe pas sur le cumin ;
        \\mais on bat la nigelle avec le fléau,
        et le cumin avec le bâton.
${}^{28}Va-t-on broyer le froment ?
        On ne l’écrase pas sans fin :
        \\on le foule sous les roues du chariot,
        mais il n’est pas broyé.
${}^{29}Même cela vient du Seigneur de l’univers :
        il conçoit des merveilles, il réussit de grandes choses.
      
         
      \bchapter{}
${}^{1}Malheur ! Ariel, Ariel,
        cité où campa David !
        \\Ajoutez une année à l’année,
        que s’accomplisse le cycle des fêtes,
${}^{2}et j’opprimerai Ariel :
        il ne sera que plaintes et complaintes,
        il ne sera pour moi qu’un ariel, un brasier d’autel.
${}^{3}Contre toi, tout autour, je dresserai le camp,
        j’érigerai contre toi des remblais,
        j’élèverai contre toi des fortifications.
${}^{4}Tu seras si abaissée que ta parole semblera venir de terre,
        de la poussière elle montera assourdie,
        \\comme celle d’un revenant ta voix viendra de la terre,
        et de la poussière ta parole ne sera que chuchotement.
${}^{5}Mais la foule de tes ennemis n’est que poudre fine,
        la foule des tyrans n’est que paille au vent.
        \\Tout à coup, en un instant,
${}^{6}le Seigneur de l’univers interviendra
        dans le tonnerre, avec tremblement et fracas,
        dans l’ouragan et la tempête,
        dans la flamme d’un feu dévorant.
${}^{7}Comme disparaît un songe, une vision de nuit,
        telle est la foule de toutes les nations
        mobilisées contre Ariel,
        \\tous ceux qui l’assiègent dans sa forteresse,
        et qui l’oppriment.
${}^{8}Comme un affamé rêve qu’il mange
        et s’éveille le ventre creux,
        \\comme un assoiffé rêve qu’il boit
        et s’éveille épuisé, le gosier sec,
        \\ainsi en sera-t-il de la foule de toutes les nations
        mobilisées contre le mont Sion.
        
           
${}^{9}Soyez stupéfiés, stupéfaits,
        aveuglés, et aveugles ;
        \\enivrés sans vin,
        titubants sans avoir bu.
${}^{10}Car le Seigneur a répandu sur vous
        un esprit de torpeur ;
        \\il a fermé les yeux, que sont vos prophètes ;
        \\il a voilé les têtes, que sont vos voyants.
${}^{11}Toute vision est devenue pour vous
        comme les mots d’un livre scellé.
        \\On le donne à qui sait lire,
        en lui disant : « Lis donc ceci » ;
        \\mais il répond : « Je ne peux pas : le livre est scellé ! »
${}^{12}On le donne alors à qui ne sait pas lire,
        en lui disant : « Lis donc ceci » ;
        \\mais il répond : « Je ne sais pas lire. »
         
${}^{13}Le Seigneur dit :
        \\Parce que ce peuple s’approche de moi
        en me glorifiant de la bouche et des lèvres,
        alors que son cœur est loin de moi,
        \\parce que la crainte qu’ils ont de moi
        n’est que précepte enseigné par les hommes,
${}^{14}eh bien ! j’émerveillerai encore ce peuple
        par des merveilles de merveilles,
        \\et la sagesse de leurs sages se perdra
        et l’intelligence des intelligents disparaîtra.
         
${}^{15}Malheur ! Dans un profond secret, loin du Seigneur
        ils cachent leur projet,
        \\et leur ouvrage est fait dans l’obscurité ;
        ils se disent : « Qui nous voit ? Qui nous reconnaît ? »
${}^{16}C’est le monde à l’envers !
        L’argile se prend-elle pour le potier ?
        \\L’ouvrage va-t-il dire de son fabricant :
        « Il ne m’a pas fabriqué »,
        \\et le pot va-t-il dire du potier :
        « Il n’y connaît rien » ?
        ${}^{17}Ne le savez-vous pas ?
        \\Encore un peu, très peu de temps,
        et le Liban se changera en verger,
        et le verger sera pareil à une forêt.
        ${}^{18}Les sourds, en ce jour-là,
        entendront les paroles du livre.
        \\Quant aux aveugles, sortant de l’obscurité et des ténèbres,
        leurs yeux verront.
        ${}^{19}Les humbles se réjouiront de plus en plus
        dans le Seigneur,
        \\les malheureux exulteront
        en Dieu\\, le Saint d’Israël.
        ${}^{20}Car ce sera la fin des tyrans,
        l’extermination des moqueurs,
        \\et seront supprimés tous ceux qui s’empressent à mal faire,
        ${}^{21}ceux qui font condamner quelqu’un par leur témoignage,
        qui faussent les débats du tribunal
        et sans raison font débouter l’innocent.
        ${}^{22}C’est pourquoi le Seigneur, lui qui a libéré Abraham,
        \\parle ainsi à la maison de Jacob :
        « Désormais Jacob n’aura plus de honte,
        son visage ne pâlira plus ;
        ${}^{23}car, quand il verra chez lui ses enfants,
        l’œuvre de mes mains,
        il sanctifiera\\mon nom,
        il sanctifiera\\le Dieu\\Saint de Jacob,
        il tremblera\\devant le Dieu d’Israël.
        ${}^{24}Les esprits égarés découvriront l’intelligence,
        et les récalcitrants accepteront qu’on les instruise. »
      
         
      \bchapter{}
${}^{1}Malheur aux fils rebelles,
        – oracle du Seigneur –,
        \\qui font un projet, mais sans moi,
        qui concluent un traité, mais sans mon esprit,
        \\accumulant ainsi péché sur péché.
${}^{2}Ils descendent en Égypte,
        sans m’avoir consulté,
        \\pour trouver refuge auprès de Pharaon,
        pour s’abriter à l’ombre de l’Égypte.
${}^{3}Mais le refuge de Pharaon sera votre honte,
        et l’abri que vous cherchez à l’ombre de l’Égypte
        sera votre confusion.
${}^{4}Même si vos princes sont déjà à Tanis,
        si vos ambassadeurs sont parvenus à Hanès,
${}^{5}ils seront tous couverts de honte
        par un peuple qui leur sera inutile,
        \\qui ne leur sera d’aucun secours, d’aucune utilité,
        sinon pour la honte et même l’infamie.
        
           
         
${}^{6}Proclamation des bêtes du Néguev.
        
           
         
        \\Au pays de la détresse et de l’angoisse,
        de la lionne et du lion rugissant,
        de la vipère et du dragon volant,
        \\ils transportent sur le dos des ânes leurs richesses,
        et leurs trésors sur la bosse des chameaux
        \\vers un peuple qui leur sera inutile,
${}^{7}l’Égypte, dont le secours n’est que vide et vent.
        C’est pourquoi je l’ai nommée « Rahab-la-paresse ».
        
           
        ${}^{8}Maintenant, viens, écris ceci pour eux sur une tablette,
        inscris-le sur un document,
        \\et que ce soit dans l’avenir
        un témoignage\\à tout jamais :
        ${}^{9}« C’est un peuple rebelle,
        ce sont des fils menteurs,
        \\des fils qui n’acceptent pas d’écouter
        la loi du Seigneur,
        ${}^{10}eux qui disent aux voyants :
        “Ne voyez pas !”
        \\et aux prophètes\\ :
        “Ne prophétisez pas pour nous des choses vraies,
        dites-nous des choses agréables,
        prophétisez des chimères.
        ${}^{11}Quittez donc le chemin,
        écartez-vous de la route,
        laissez-nous tranquilles avec le Dieu\\Saint d’Israël !” »
        ${}^{12}Mais voici ce qu’il déclare, le Saint d’Israël :
        Vous avez rejeté ce que j’ai dit,
        vous avez mis votre confiance dans la violence et la ruse
        et vous en avez fait votre appui ;
        ${}^{13}ce péché-là sera pour vous
        comme une lézarde qui se creuse :
        un renflement apparaît sur une haute muraille,
        elle s’effondre brusquement, d’un seul coup.
        ${}^{14}Elle s’effondre comme une poterie
        que l’on brise sans ménagement :
        impossible de trouver dans ses débris un tesson
        pour prendre du feu dans le foyer
        ou puiser de l’eau à la citerne.
         
        ${}^{15}Le Seigneur, le Dieu saint d’Israël, avait parlé ainsi :
        Par la conversion et le calme, vous serez sauvés ;
        dans la tranquillité, dans la confiance sera votre force ;
        mais vous n’avez pas accepté !
        ${}^{16}Vous avez dit :
        « Pas du tout ! Nous fuirons à cheval ! »
        – Eh bien, oui, vous fuirez !
        \\Vous avez dit encore\\ :
        « Nos chars sont rapides ! »
        – Eh bien, rapides seront vos poursuivants !
        ${}^{17}Vous serez un millier sous la menace d’un seul,
        et sous la menace de cinq vous prendrez la fuite :
        \\il ne restera de vous
        qu’un mât au sommet de la montagne,
        un étendard sur la hauteur.
        ${}^{18}Cependant\\le Seigneur attend de vous faire grâce,
        il\\se dressera pour vous montrer sa tendresse,
        \\car le Seigneur est le Dieu juste :
        heureux tous ceux qui l’attendent !
        ${}^{19}Peuple de Sion,
        toi qui habites Jérusalem,
        tu ne pleureras jamais plus.
        \\À l’appel de ton cri, le Seigneur te fera grâce.
        Dès qu’il t’aura entendu, il te répondra.
        ${}^{20}Le Seigneur te\\donnera du pain dans la détresse,
        et de l’eau dans l’épreuve\\.
        \\Celui qui t’instruit ne se dérobera plus
        et tes yeux le\\verront.
        ${}^{21}Tes oreilles entendront derrière toi une parole\\ :
        « Voici le chemin, prends-le ! »,
        et cela, que tu ailles à droite ou à gauche.
${}^{22}Tu déclareras impur le placage d’argent de tes statues
        et le revêtement d’or de tes idoles de métal :
        \\tu les jetteras comme des immondices
        et tu diras : « Dehors ! »
         
        ${}^{23}Le Seigneur\\te donnera la pluie
        pour la semence que tu auras jetée en terre,
        \\et le pain que produira la terre
        sera riche et nourrissant.
        \\Ton bétail ira paître, ce jour-là,
        sur de vastes pâturages.
        ${}^{24}Les bœufs et les ânes qui travaillent dans les champs
        mangeront un fourrage salé,
        étalé avec la pelle et la fourche.
        ${}^{25}Sur toute haute montagne, sur toute colline élevée
        couleront des ruisseaux,
        \\au jour du grand massacre,
        quand tomberont les tours de défense.
        ${}^{26}La lune brillera comme le soleil,
        le soleil brillera sept fois plus,
        – autant que sept jours de lumière –
        \\le jour où le Seigneur pansera les plaies de son peuple
        et guérira ses meurtrissures.
${}^{27}Voici venir de loin le nom du Seigneur ;
        brûlante est sa colère, lourde, écrasante ;
        \\ses lèvres sont gonflées d’indignation,
        sa langue est un feu dévorant,
${}^{28}son souffle, un torrent qui déborde
        et monte jusqu’à la gorge ;
        \\il va secouer les nations d’une secousse fatale,
        mettre aux mâchoires des peuples
        un mors qui les fasse divaguer.
${}^{29}Alors vous pourrez chanter
        \\comme dans la nuit où l’on célèbre la fête
        avec la joie au cœur,
        \\comme on va, au son des flûtes,
        à la montagne du Seigneur, vers le rocher d’Israël.
${}^{30}Et le Seigneur fera entendre
        sa voix majestueuse ;
        \\il fera sentir le poids de son bras
        dans la fureur de sa colère,
        \\par la flamme d’un feu dévorant,
        la tornade, l’orage et la grêle.
${}^{31}À la voix du Seigneur qui frappera de son sceptre,
        Assour tremblera.
${}^{32}Chaque coup de bâton sera un châtiment
        que le Seigneur lui assénera
        \\au son des tambourins et des cithares ;
        par le geste de sa main il les combattra.
${}^{33}D’avance, la fournaise est préparée,
        y compris pour le roi ;
        \\elle est prête, profonde et large ;
        en son foyer, un grand feu, beaucoup de bois ;
        \\comme un torrent de soufre,
        le souffle du Seigneur l’embrasera.
      
         
      \bchapter{}
${}^{1}Malheur ! Ceux qui descendent en Égypte
        pour y trouver secours :
        ils comptent sur des chevaux,
        \\ils s’appuient sur le nombre des chars,
        sur la grande puissance d’une cavalerie,
        \\mais ils ne regardent pas vers le Saint d’Israël :
        le Seigneur, ils ne le consultent pas.
${}^{2}Pourtant c’est lui, le sage :
        quand il fait venir le malheur,
        il ne reprend pas sa parole ;
        \\il se dresse contre le parti des malfaisants
        contre le secours venant des ouvriers du mal.
${}^{3}Les Égyptiens sont des hommes,
        pas des dieux :
        \\leurs chevaux sont de chair,
        pas d’esprit !
        \\Quand le Seigneur étend la main,
        \\celui qui secourt trébuche,
        qui est secouru s’écroule :
        \\tous deux disparaîtront.
        
           
${}^{4}Car le Seigneur m’a dit ceci :
        \\Quand rugit vers sa proie
        le lion, le jeune lion,
        \\et que la foule des bergers
        est appelée contre lui,
        \\il n’a pas peur de leurs cris,
        ne répond pas à leur tapage.
        \\Ainsi le Seigneur de l’univers descendra
        pour combattre sur la montagne de Sion, sur sa colline.
${}^{5}Comme les oiseaux qui étendent leurs ailes,
        ainsi le Seigneur de l’univers protégera Jérusalem :
        \\il protégera et libérera,
        il épargnera et délivrera.
${}^{6}Revenez donc, fils d’Israël,
        vers celui que vous avez gravement trahi.
${}^{7}Ce jour-là, chacun de vous rejettera
        ses idoles d’argent, ses idoles d’or,
        \\celles que vous vous êtes fabriquées de vos mains
        – c’est un péché !
${}^{8}Assour tombera sous une épée
        qui n’est pas celle d’un homme,
        une épée surhumaine qui le dévorera :
        \\il fuira devant cette épée ;
        ses jeunes gens seront soumis à la corvée.
${}^{9}Celui qui est son rocher, s’en ira, pris de terreur,
        et ses princes, effrayés, abandonneront l’étendard,
        \\– oracle du Seigneur
        dont le feu brûle à Sion,
        et la fournaise, à Jérusalem.
      
         
      \bchapter{}
${}^{1}Voici un roi qui règne avec justice,
        des princes qui gouvernent selon le droit :
${}^{2}chacun sera comme un abri contre le vent,
        un refuge contre l’orage,
        \\comme un ruisseau sur une terre desséchée,
        l’ombre d’un grand rocher dans un pays torride.
${}^{3}Les yeux qui regardent ne seront plus aveuglés,
        les oreilles qui écoutent seront attentives ;
${}^{4}le cœur frivole réfléchira pour comprendre
        et la langue des bègues parlera vite et clairement.
${}^{5}Le fou ne sera plus déclaré noble,
        l’escroc ne sera pas dit honorable.
        
           
         
${}^{6}Qui est fou ne dit que des folies
        et son cœur fait le mal :
        \\il commet l’impiété,
        il blasphème le Seigneur ;
        \\il laisse l’affamé le ventre creux
        et l’assoiffé sans rien à boire.
${}^{7}Quant à l’escroc, elles sont odieuses, ses escroqueries :
        il conçoit des mauvais coups
        \\pour perdre les humbles par des mensonges,
        quand le malheureux plaide son bon droit.
${}^{8}Qui est noble conçoit de nobles projets :
        il se lève, lui, pour de nobles causes.
        
           
${}^{9}Femmes insouciantes, debout !
        Écoutez ma voix !
        \\Filles présomptueuses,
        prêtez l’oreille à ma parole :
${}^{10}Dans un an révolu,
        vous tremblerez, présomptueuses,
        \\car la vendange sera perdue,
        on ne rentrera pas de récolte.
${}^{11}Alarmez-vous, insouciantes !
        tremblez, présomptueuses !
        \\Dévêtez-vous, dénudez-vous
        avec un pagne autour des reins.
${}^{12}Frappez-vous la poitrine :
        \\faites le deuil sur la campagne riante
        sur les vignes fertiles,
${}^{13}sur la terre de mon peuple,
        où poussent la broussaille et l’épine,
        \\et sur les maisons joyeuses
        de la cité en liesse !
${}^{14}Oui, le palais sera abandonné,
        la ville bruyante sera désertée.
        \\L’Ophel et la Tour de guet
        deviendront à jamais des repaires,
        \\joie des ânes sauvages
        et pâture des troupeaux,
        ${}^{15}jusqu’à ce que soit répandu sur nous
        l’esprit qui vient d’en haut.
        \\Alors le désert deviendra un verger,
        et le verger sera pareil à une forêt.
        ${}^{16}Le droit habitera le désert,
        la justice résidera dans le verger.
        ${}^{17}L’œuvre de la justice sera la paix,
        et la pratique de la justice, le calme et la sécurité
        pour toujours.
        ${}^{18}Mon peuple habitera un séjour de paix,
        des demeures protégées,
        des lieux sûrs de repos.
${}^{19}– Mais la forêt s’écroulera sous la grêle
        et la ville sera entièrement démolie.
${}^{20}Heureux vous qui sèmerez près de tous les cours d’eau,
        et laisserez aller le bœuf et l’âne.
      <p class="cantique" id="bib_ct-at_21"><span class="cantique_label">Cantique AT 21</span> = <span class="cantique_ref"><a class="unitex_link" href="#bib_is_33_2">Is 33, 2-10</a></span>
      <p class="cantique" id="bib_ct-at_22"><span class="cantique_label">Cantique AT 22</span> = <span class="cantique_ref"><a class="unitex_link" href="#bib_is_33_13">Is 33, 13-16</a></span>
      
         
      \bchapter{}
${}^{1}Malheur ! Toi, dévastateur qui n’as pas été dévasté,
        ravageur qui n’as pas subi de ravage,
        \\quand tu auras fini de dévaster,
        tu seras dévasté !
        \\quand tu auras cessé de ravager,
        on te ravagera !
        
           
       
        ${}^{2}Seigneur, fais-nous grâce\\ :
        c’est toi que nous attendons !
        \\Chaque matin, sois notre bras\\,
        notre salut aux jours de détresse.
         
        ${}^{3}À la voix qui tonne, les peuples s’enfuient ;
        quand tu te lèves, les nations se dispersent.
        ${}^{4}Votre butin s’entasse comme s’entassent des insectes ;
        c’est la ruée, une ruée de sauterelles.
         
        ${}^{5}Le Seigneur domine,
        il habite les hauteurs ;
        \\il emplit Sion de droit et de justice :
        ${}^{6}il sera la sécurité de tes jours.
         
        \\Sagesse et connaissance :
        des biens pour le salut ;
        \\la crainte du Seigneur :
        un trésor qu’il te donne.
         
        ${}^{7}Voici que les voyants\\se lamentent sur les places ;
        les messagers de paix pleurent amèrement.
         
        ${}^{8}Les routes sont désolées ;
        sur les chemins, le passant a disparu.
        \\L’alliance est rompue : on méprise les témoins\\ ;
        un homme ne compte plus.
         
        ${}^{9}La terre, en deuil, languit ;
        le Liban, honteux, s’assombrit\\.
        \\Le Sarone ressemble au désert\\ ;
        le Bashane et le Carmel se fanent.
        ${}^{10}« Maintenant, je surgis – dit le Seigneur ;
        maintenant, je me dresse ;
        maintenant, je m’élève !
         
${}^{11}Vous concevez du foin : vous enfantez de la paille !
        Votre souffle est le feu qui vous dévorera.
${}^{12}Les peuples seront brûlés à la chaux,
        épines coupées que l’on enflamme.
       
        ${}^{13}Écoutez ce que j’ai fait,
        vous qui êtes loin\\ ;
        \\et vous qui êtes proches,
        reconnaissez ma vaillance !
         
        ${}^{14}Dans Sion, les pécheurs sont terrifiés ;
        un tremblement saisit les pervers :
        \\“Qui de nous résistera ? c’est un feu dévorant !
        Qui de nous résistera ? c’est une fournaise sans fin !”
         
        ${}^{15}Celui qui va selon la justice et parle avec droiture,
        \\qui méprise un gain frauduleux,
        détourne sa main d’un profit malhonnête,
        \\qui ferme son oreille aux propos sanguinaires
        et baisse les yeux pour ne pas voir le mal,
         
        ${}^{16}celui-là habitera les hauteurs,
        hors d’atteinte, à l’abri des rochers.
        \\Le pain lui sera donné ;
        les eaux lui seront assurées. »
${}^{17}Tes yeux verront le roi dans sa beauté ;
        ils découvriront les lointains du pays.
${}^{18}Tu repenseras aux terreurs passées :
        « Celui qui comptait, où est-il ?
        \\Celui qui contrôlait, où est-il ?
        Où est celui qui comptait les tours ? »
${}^{19}Tu ne verras plus le peuple brutal,
        ce peuple au langage impénétrable,
        à la langue ridicule et incompréhensible.
${}^{20}Contemple Sion, la cité de nos fêtes,
        tes yeux verront Jérusalem :
        \\c’est une résidence sûre,
        la tente qu’on ne déplacera plus,
        \\dont les piquets ne seront jamais arrachés,
        dont aucune corde ne sera rompue.
${}^{21}Et même, c’est là que le Seigneur nous montre sa grandeur :
        c’est un lieu de fleuves, de larges canaux,
        \\qu’aucune galère ne traverse,
        qu’aucun grand navire ne sillonne.
${}^{22}Oui, le Seigneur est notre juge,
        le Seigneur nous donne des lois,
        \\le Seigneur est notre roi :
        c’est lui qui nous sauve.
         
${}^{23}Tes cordes sont relâchées :
        elles n’assurent pas la stabilité du mât
        et ne tiennent pas l’étendard déployé.
         
        \\Alors les aveugles se partageront quantité de butin ;
        les boiteux prendront part au pillage.
${}^{24}Aucun de ceux qui demeurent là ne dira plus :
        “Je suis malade.”
        \\Le peuple qui habite Jérusalem
        sera déchargé de sa faute.
      <h2 class="intertitle" id="d85e250469">6. Petite Apocalypse (34 – 35)</h2>
      
         
      \bchapter{}
${}^{1}Approchez, nations, pour entendre !
        Peuples, soyez attentifs !
        \\Que la terre entende, avec sa richesse,
        et le monde, avec tout ce qu’il produit.
${}^{2}Car le Seigneur s’irrite contre toutes les nations,
        il est en fureur contre toute leur armée :
        \\il les a vouées à l’anathème
        et les a livrées au carnage.
${}^{3}Leurs morts sont abandonnés sur le sol,
        de leurs cadavres monte une puanteur,
        les montagnes ruissellent de leur sang.
${}^{4}Toute l’armée des cieux se liquéfie,
        les cieux s’enroulent comme un livre ;
        \\toute leur armée se flétrit
        comme se flétrissent les feuilles de la vigne
        ou les fruits avortés du figuier.
${}^{5}Car mon épée apparaît dans les cieux,
        et voici qu’elle descend sur Édom,
        sur le peuple que j’ai condamné à l’anathème.
${}^{6}L’épée du Seigneur est pleine de sang,
        elle est gluante de graisse,
        du sang des agneaux et des boucs,
        de la graisse des rognons de béliers ;
        \\c’est un sacrifice pour le Seigneur à Bosra,
        un grand carnage au pays d’Édom.
${}^{7}Avec eux tombent des buffles,
        des taureaux et des bœufs,
        \\leur terre s’enivre de sang
        et leur poussière est gluante de graisse.
${}^{8}C’est pour le Seigneur un jour de vengeance,
        l’année des représailles pour la cause de Sion.
${}^{9}Les torrents d’Édom se changent en goudron,
        sa poussière, en soufre ;
        \\et sa terre devient du goudron brûlant,
${}^{10}qui ne s’éteindra ni de jour ni de nuit :
        sa fumée montera sans fin ;
        \\pour toutes les générations, Édom sera un désert
        où plus personne jamais ne passera.
${}^{11}La hulotte et le hérisson vont l’occuper,
        la chouette et le corbeau y demeurer.
        \\Le Seigneur tendra sur Édom cordeau et fil à plomb
        pour en faire un chaos.
${}^{12}Ses notables n’y seront plus pour proclamer la royauté,
        tous ses princes auront disparu.
${}^{13}Dans ses citadelles pousseront des épines,
        dans ses forteresses, orties et chardons.
        \\Ce sera le séjour des chacals,
        la pâture des autruches.
${}^{14}Les chats sauvages y côtoieront les hyènes,
        les boucs s’appelleront l’un l’autre.
        \\C’est là que le démon de la nuit se tapira
        pour y prendre son repos ;
${}^{15}là que le serpent nichera et pondra,
        qu’il fera éclore et protégera de son ombre ;
        \\là aussi se rassembleront les vautours,
        l’un avec l’autre.
        
           
         
${}^{16}Cherchez dans le livre du Seigneur et lisez :
        \\Aucun d’entre eux ne manque,
        pas un n’aura à chercher l’autre.
        \\Car c’est la bouche du Seigneur qui ordonne
        et son souffle qui les rassemble.
${}^{17}Lui-même a tiré au sort pour eux,
        sa main leur a partagé le territoire au cordeau.
        \\Pour toujours ils le posséderont,
        ils y demeureront de génération en génération.
        
           
      
         
      \bchapter{}
        ${}^{1}Le désert et la terre de la soif,
        qu’ils se réjouissent !
        \\Le pays aride, qu’il exulte
        et fleurisse comme la rose\\,
        ${}^{2}qu’il se couvre de fleurs des champs,
        qu’il exulte et crie de joie !
        \\La gloire du Liban lui est donnée,
        la splendeur du Carmel et du Sarone.
        \\On verra\\la gloire du Seigneur,
        la splendeur de notre Dieu.
        ${}^{3}Fortifiez les mains défaillantes,
        affermissez les genoux qui fléchissent\\,
        ${}^{4}dites aux gens qui s’affolent :
        « Soyez forts, ne craignez pas.
        Voici votre Dieu :
        c’est la vengeance qui vient,
        la revanche de Dieu\\.
        Il vient lui-même
        et va vous sauver. »
        ${}^{5}Alors se dessilleront les yeux des aveugles,
        et s’ouvriront les oreilles des sourds.
        ${}^{6}Alors le boiteux bondira comme un cerf,
        et la bouche\\du muet criera de joie ;
        \\car l’eau jaillira dans le désert,
        des torrents dans le pays aride.
        ${}^{7}La terre brûlante se changera en lac,
        la région de la soif, en eaux jaillissantes.
        \\Dans le séjour où gîtent\\les chacals,
        l’herbe deviendra des roseaux et des joncs\\.
        ${}^{8}Là, il y aura une chaussée, une voie
        qu’on appellera la Voie sacrée.
        \\L’homme impur n’y passera pas
        \\– il suit sa propre voie\\ –
        et les insensés ne viendront pas s’y égarer.
        ${}^{9}Là, il n’y aura pas de lion,
        aucune bête féroce ne surgira,
        \\il ne s’en trouvera pas ;
        mais les rachetés y marcheront.
        ${}^{10}Ceux qu’a libérés le Seigneur reviennent,
        ils entrent dans Sion avec des cris de fête,
        couronnés de l’éternelle joie\\.
        \\Allégresse et joie les rejoindront,
        douleur et plainte s’enfuient\\.
        
           
      <h2 class="intertitle" id="d85e251187">7. Interventions d’Isaïe (36 – 39)</h2>
      
         
      \bchapter{}
      \begin{verse}
${}^{1}La quatorzième année du roi Ézékias, Sennakérib, roi d’Assour, monta contre toutes les villes fortifiées de Juda et s’en empara. 
${}^{2}Depuis la ville de Lakish, le roi d’Assour envoya au roi Ézékias, à Jérusalem, le grand échanson, avec une armée considérable. Il prit position près du canal du réservoir supérieur, sur la route du Champ-du-Foulon. 
${}^{3}Le maître du palais, Éliakim, fils d’Helcias, le secrétaire Shebna, et l’archiviste Joah, fils d’Assaf, sortirent à sa rencontre.
${}^{4}Le grand échanson leur dit : « Je vous en prie, dites à Ézékias : Ainsi parle le grand roi, le roi d’Assour : Quelle est cette confiance en laquelle tu te reposes ? 
${}^{5}Tu te dis : “Parole des lèvres vaut conseil et vaillance pour la guerre !” En qui donc as-tu mis ta confiance pour te révolter contre moi ? 
${}^{6}Voici que tu as mis ta confiance dans le soutien de ce roseau brisé, l’Égypte, qui pénètre et perce la main de quiconque s’appuie sur lui : tel est Pharaon, roi d’Égypte, pour tous ceux qui mettent leur confiance en lui !
${}^{7}Tu me diras peut-être : “C’est dans le Seigneur notre Dieu que nous mettons notre confiance…” Mais n’est-ce pas ce Dieu dont Ézékias a fait supprimer les lieux sacrés et les autels, en disant aux gens de Juda et de Jérusalem : “C’est devant cet autel, à Jérusalem, que vous vous prosternerez” ? 
${}^{8}Eh bien ! lance donc un défi à mon seigneur le roi d’Assour, et je te donnerai deux mille chevaux si tu peux te procurer des cavaliers pour les monter ! 
${}^{9}Comment ferais-tu reculer un seul gouverneur, le moindre des serviteurs de mon seigneur ? Et tu mets ta confiance dans l’Égypte pour avoir chars et cavaliers ! 
${}^{10}Et puis, est-ce indépendamment du Seigneur que je suis monté contre ce pays pour le détruire ? C’est le Seigneur qui m’a dit : “Monte vers ce pays et détruis-le !” »
${}^{11}Éliakim, Shebna et Joah dirent au grand échanson : « Je t’en prie, parle en araméen à tes serviteurs, car nous le comprenons ; mais ne nous parle pas en judéen, près des oreilles du peuple qui est sur le rempart. » 
${}^{12}Le grand échanson répondit : « Est-ce à ton maître et à toi que mon seigneur m’a envoyé dire ces paroles ? N’est-ce pas aux hommes qui se tiennent sur le rempart, réduits, comme vous, à manger leurs excréments et à boire leur urine ? »
${}^{13}Le grand échanson se tint debout et cria d’une voix forte en judéen ; il prononça ces mots : « Écoutez les paroles du grand roi, le roi d’Assour. 
${}^{14}Ainsi parle le roi : Qu’Ézékias ne vous trompe pas car il ne pourra vous délivrer. 
${}^{15}Et qu’Ézékias ne vous persuade pas de mettre votre confiance dans le Seigneur, en disant : “Sûrement le Seigneur nous délivrera ; cette ville ne sera pas livrée aux mains du roi d’Assour.” 
${}^{16}N’écoutez pas Ézékias, car ainsi parle le roi d’Assour : “Faites la paix avec moi, et rendez-vous à moi. Que chacun de vous mange les fruits de sa vigne et de son figuier, et qu’il boive l’eau de sa citerne, 
${}^{17}jusqu’à ce que je vienne vous prendre pour vous emmener dans un pays comme le vôtre, un pays de froment et de vin nouveau, un pays de pain et de vignobles.” 
${}^{18}Il ne faudrait pas qu’Ézékias vous abuse en disant : “Le Seigneur nous délivrera.” Les dieux des nations ont-ils délivré chacun son pays de la main du roi d’Assour ? 
${}^{19}Où sont les dieux de Hamath et d’Arpad ? Où sont les dieux de Sefarwaïm ? Ont-ils délivré Samarie de ma main ? 
${}^{20}Parmi tous les dieux de ces pays, lesquels ont délivré leur pays de ma main, pour que le Seigneur délivre de ma main Jérusalem ? »
${}^{21}Le peuple garda le silence et ne lui répondit pas un mot, car tel était l’ordre du roi : « Vous ne lui répondrez pas. » 
${}^{22}Éliakim, fils d’Helcias, maître du palais, le secrétaire Shebna et l’archiviste Joah, fils d’Assaf, remontèrent vers Ézékias, les vêtements déchirés, et lui rapportèrent les paroles du grand échanson.
      
         
      \bchapter{}
      \begin{verse}
${}^{1}Quand le roi Ézékias entendit cela, il déchira ses vêtements, se couvrit d’une toile à sac et se rendit à la maison du Seigneur. 
${}^{2}Et il envoya le maître du palais Éliakim, le secrétaire Shebna et les plus anciens des prêtres, couverts de toile à sac, vers Isaïe le prophète, fils d’Amots. 
${}^{3}Ils lui dirent : « Ainsi parle Ézékias : Jour d’angoisse, de châtiment et de honte, que ce jour-ci ! Car des fils arrivent à terme, et la force manque pour enfanter. 
${}^{4}Peut-être le Seigneur ton Dieu entendra-t-il les paroles du grand échanson, lui que le roi d’Assour, son seigneur, a envoyé pour insulter le Dieu vivant. Peut-être le punira-t-il pour les paroles que le Seigneur ton Dieu aura entendues. Fais monter une prière en faveur du reste qui subsiste. »
${}^{5}Les serviteurs du roi Ézékias se rendirent auprès d’Isaïe, 
${}^{6}qui leur dit : « Vous parlerez ainsi à votre maître : Ainsi parle le Seigneur : Ne crains pas les paroles que tu as entendues, les insultes proférées contre moi par les valets du roi d’Assour. 
${}^{7}Voici que, sur une nouvelle qu’il apprendra, je vais lui inspirer de retourner dans son pays et, dans son pays, je le ferai tomber par l’épée. »
${}^{8}Le grand échanson s’en retourna. Ayant appris que le roi d’Assour avait quitté la ville de Lakish, il le trouva qui attaquait la ville de Libna. 
${}^{9}Le roi d’Assour avait appris cette nouvelle, au sujet de Tirhaqa, roi d’Éthiopie : « Il s’est mis en campagne pour passer à l’attaque contre toi. »
      Quand le roi d’Assour l’apprit, il envoya des messagers dire à Ézékias : 
${}^{10}« Vous parlerez à Ézékias, roi de Juda, en ces termes : Ne te laisse pas tromper par ton Dieu, en qui tu mets ta confiance, et ne dis pas : “Jérusalem ne sera pas livrée aux mains du roi d’Assour !” 
${}^{11}Tu sais bien ce que les rois d’Assour ont fait à tous les pays : ils les ont voués à l’anathème. Et toi seul, tu serais délivré ? 
${}^{12}Les dieux des nations les ont-ils délivrées, elles que mes pères ont fait détruire : Gozane, Harrane, Récef, et les gens d’Éden qui sont à Telassar ? 
${}^{13}Où sont le roi de Hamath, le roi d’Arpad, le roi de Lahir, de Sefarwaïm, de Héna et de Iwwa ? »
${}^{14}Ézékias prit la lettre de la main des messagers ; il la lut. Puis il monta à la maison du Seigneur, déplia la lettre devant le Seigneur, 
${}^{15}et le pria en disant : 
${}^{16}« Seigneur de l’univers, Dieu d’Israël, toi qui sièges sur les Kéroubim, tu es le seul Dieu de tous les royaumes de la terre, c’est toi qui as fait le ciel et la terre. 
${}^{17}Prête l’oreille, Seigneur, et entends, ouvre les yeux, Seigneur, et vois ! Écoute le message envoyé par Sennakérib pour insulter le Dieu vivant. 
${}^{18}Il est vrai, Seigneur, que les rois d’Assour ont ravagé tous les pays et leur territoire, 
${}^{19}et brûlé leurs dieux : en réalité, ce n’étaient pas des dieux, mais un ouvrage de mains d’hommes, fait avec du bois et de la pierre ; c’est pourquoi ils ont pu les faire disparaître. 
${}^{20}Maintenant, Seigneur notre Dieu, sauve-nous de la main de Sennakérib, et tous les royaumes de la terre sauront que tu es Seigneur, toi le seul ! »
${}^{21}Alors le prophète Isaïe, fils d’Amots, envoya dire à Ézékias : « Ainsi parle le Seigneur, Dieu d’Israël : Tu m’as adressé une prière au sujet de Sennakérib, roi d’Assour. 
${}^{22}Voici la parole que le Seigneur a prononcée contre lui :
       
        \\Elle te méprise, elle te nargue,
        la vierge, la fille de Sion.
        \\Elle hoche la tête pour se moquer de toi,
        la fille de Jérusalem.
         
${}^{23}Qui as-tu insulté, outragé,
        contre qui as-tu élevé la voix ?
        \\Sur qui, avec hauteur, as-tu porté les yeux ?
        Sur le Saint d’Israël !
         
${}^{24}Par tes serviteurs tu as insulté mon Seigneur.
        \\Tu as dit : “Avec mes nombreux chars,
        \\moi, j’ai gravi le sommet des montagnes,
        les cimes du Liban ;
        \\j’ai abattu ses cèdres les plus fiers,
        ses cyprès les plus beaux ;
        \\j’ai atteint sa plus lointaine hauteur,
        son parc forestier.
${}^{25}Moi, j’ai creusé, et j’ai bu
        des eaux étrangères,
        \\j’ai asséché sous mes pas
        tous les canaux de l’Égypte.”
         
${}^{26}N’entends-tu pas ? Depuis longtemps
        j’avais fait ce projet,
        \\depuis les temps anciens je l’ai formé ;
        maintenant je le réalise.
         
        \\Toi, tu étais destiné à réduire en tas de ruines
        les villes fortifiées.
${}^{27}Leurs habitants ont la main trop courte,
        ils sont effrayés, confondus ;
        \\ils ressemblent à l’herbe des champs,
        à la verdure des prés,
        \\à l’herbe des toits,
        et aux cultures avant qu’elles aient levé.
         
${}^{28}Mais je sais quand tu t’assieds,
        quand tu sors et quand tu rentres,
        quand tu t’emportes contre moi.
${}^{29}Parce que tu t’es emporté contre moi,
        que tes insolences sont montées à mes oreilles,
        \\je passerai un crochet à ton nez,
        un mors à ta bouche ;
        \\je te ferai retourner par le chemin
        par lequel tu es venu.
         
${}^{30}Voici pour toi un signe, Ézékias :
        \\Cette année on mangera le grain tombé,
        l’an prochain, ce qui aura poussé tout seul ;
        \\mais la troisième année, semez et moissonnez,
        plantez des vignes et mangez-en le fruit.
${}^{31}Le reste, survivant de la maison de Juda,
        fera de nouveau des racines par en bas,
        et par en haut donnera des fruits.
${}^{32}Oui, un reste sortira de Jérusalem,
        et des survivants, de la montagne de Sion.
        Il fera cela, l’amour jaloux du Seigneur de l’univers !
         
${}^{33}Et voici ce que dit le Seigneur au sujet du roi d’Assour :
        \\Il n’entrera pas dans cette ville,
        il ne lui lancera pas une seule flèche,
        \\il ne lui opposera pas un seul bouclier,
        il n’élèvera pas un seul remblai :
${}^{34}il retournera par le chemin
        par lequel il est venu.
        \\Non, il n’entrera pas dans cette ville,
        – oracle du Seigneur.
${}^{35}Je protégerai cette ville, je la sauverai
        à cause de moi-même et de David mon serviteur. »
${}^{36}L’ange du Seigneur sortit et frappa cent quatre-vingt-cinq mille hommes dans le camp assyrien. Le matin, quand on se leva, ce n’était que des cadavres. 
${}^{37}Sennakérib, roi d’Assour, plia bagage et s’en alla. Il revint à Ninive et y demeura. 
${}^{38}Or, comme il se prosternait dans la maison de Nisrok, son dieu, ses fils Adrammélek et Sarècer le frappèrent de l’épée et s’enfuirent au pays d’Ararat. Son fils Asarhaddone régna à sa place.
      <p class="cantique" id="bib_ct-at_23"><span class="cantique_label">Cantique AT 23</span> = <span class="cantique_ref"><a class="unitex_link" href="#bib_is_38_10">Is 38, 10-14.17b-20</a></span>
      
         
      \bchapter{}
      \begin{verse}
${}^{1}En ces jours-là, le roi\\Ézékias souffrait d’une maladie mortelle. Le prophète Isaïe, fils d’Amots, vint lui dire : « Ainsi parle le Seigneur : Prends des dispositions pour ta maison, car tu vas mourir, tu ne guériras pas. » 
${}^{2} Ézékias se tourna vers le mur et fit cette prière au Seigneur : 
${}^{3} « Ah ! Seigneur, souviens-toi ! J’ai marché en ta présence, dans la loyauté et d’un cœur sans partage, et j’ai fait ce qui est bien à tes yeux. » Puis le roi Ézékias fondit en larmes.
${}^{4}La parole du Seigneur fut adressée à Isaïe : 
${}^{5} « Va dire à Ézékias : Ainsi parle le Seigneur, Dieu de David ton ancêtre : J’ai entendu ta prière, j’ai vu tes larmes. Je vais ajouter quinze années à ta vie\\. 
${}^{6} Je te délivrerai, toi et cette ville, de la main du roi d’Assour, je protégerai cette ville. 
${}^{7} Voici le signe que le Seigneur te donne pour montrer qu’il accomplira sa promesse : 
${}^{8} Je vais faire reculer de dix degrés l’ombre qui est déjà descendue sur le cadran solaire d’Acaz. » Et le soleil remonta sur le cadran les dix degrés qu’il avait déjà descendus.
${}^{9}Cantique d’Ézékias, roi de Juda, lorsqu’il tomba malade et survécut à sa maladie.
       
        ${}^{10}Je disais : Au milieu de mes jours,
        je m’en vais ;
        \\j’ai ma place entre les morts
        pour la fin de mes années\\.
         
        ${}^{11}Je disais : Je ne verrai pas le Seigneur
        sur la terre des vivants,
        \\plus un visage d’homme
        parmi les habitants du monde\\ !
         
        ${}^{12}Ma demeure m’est enlevée, arrachée,
        comme une tente de berger.
        \\Tel un tisserand, j’ai dévidé ma vie :
        le fil est tranché\\.
         
        \\Du jour à la nuit, tu m’achèves ;
        ${}^{13}j’ai crié\\jusqu’au matin.
        \\Comme un lion, il a broyé tous mes os.
        Du jour à la nuit, tu m’achèves.
         
        ${}^{14}Comme l’hirondelle\\, je crie ;
        je gémis comme la colombe.
        \\À regarder là-haut, mes yeux faiblissent :
        Seigneur, je défaille ! Sois mon soutien !
         
${}^{15}\[Que lui dirai-je pour qu’il me réponde,
        à lui qui agit ?
        \\J’irais, errant au long de mes années
        avec mon amertume ?
         
${}^{16}« Le Seigneur est auprès d’eux : ils vivront !
        \\Tout ce qui vit en eux vit de son\\esprit ! »\]
        \\Oui, tu me guériras, tu me feras vivre :
        ${}^{17}voici que mon amertume se change en paix.
         
        \\Et toi, tu t’es attaché à mon âme,
        tu me tires du néant de l’abîme.
        \\Tu as jeté, loin derrière toi,
        tous mes péchés.
         
        ${}^{18}La mort ne peut te rendre grâce,
        ni le séjour des morts, te louer.
        \\Ils n’espèrent plus ta fidélité,
        ceux qui descendent dans la fosse.
         
        ${}^{19}Le vivant, le vivant, lui, te rend grâce,
        comme moi, aujourd’hui.
        \\Et le père à ses enfants
        montrera ta fidélité.
         
        ${}^{20}Seigneur, viens me sauver !
        \\Et nous jouerons sur nos cithares,
        tous les jours de notre vie,
        \\auprès de la maison du Seigneur.
       
${}^{21}Puis Isaïe dit : « Qu’on apporte un gâteau de figues ; qu’on l’applique sur l’ulcère, et le roi vivra. » 
${}^{22}Ézékias dit : « À quel signe reconnaîtrai-je que je pourrai monter à la maison du Seigneur ? »
      
         
      \bchapter{}
      \begin{verse}
${}^{1}En ce temps-là, Mérodak-Baladane, fils de Baladane, roi de Babylone, envoya des lettres et un présent à Ézékias ; il avait appris qu’Ézékias avait été malade et avait retrouvé des forces. 
${}^{2}Ézékias s’en réjouit et montra aux envoyés ses entrepôts, l’argent et l’or, les aromates et l’huile parfumée, ainsi que tout son arsenal et tout ce qui se trouvait dans ses trésors. Il n’y eut rien qu’Ézékias ne leur ait montré dans sa maison et dans tout son royaume.
${}^{3}Le prophète Isaïe vint alors trouver le roi Ézékias et lui demanda : « Qu’ont-ils dit, ces gens-là, et d’où venaient-ils ? » Ézékias répondit : « Ils venaient d’un pays lointain, de Babylone. » 
${}^{4}Il demanda : « Et qu’ont-ils vu dans ta maison ? » Ézékias dit : « Tout ce qui se trouve dans ma maison, ils l’ont vu. Il n’y a rien, dans mes trésors, que je ne leur aie montré. »
${}^{5}Alors Isaïe dit à Ézékias : « Écoute la parole du Seigneur de l’univers : 
${}^{6}Voici venir des jours, où tout ce qui est dans ta maison, ce que tes pères ont amassé jusqu’à aujourd’hui, sera emporté à Babylone ; il n’en restera rien – dit le Seigneur. 
${}^{7}On prendra plusieurs de tes fils, issus de toi, engendrés par toi ; ils seront eunuques dans le palais du roi de Babylone. » 
${}^{8}Ézékias dit à Isaïe : « C’est une bonne chose que tu me dis de la part du Seigneur. » Il se disait : « Il y aura la paix et la stabilité pendant ma vie ! »
