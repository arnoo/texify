  
  
    
    \bbook{NOMBRES}{NOMBRES}
      
         
      \bchapter{}
      \begin{verse}
${}^{1}Au désert du Sinaï, le Seigneur parla à Moïse, dans la tente de la Rencontre, le premier jour du deuxième mois de la deuxième année après la sortie du pays d’Égypte. Il dit : 
${}^{2}« Faites le dénombrement de toute la communauté des fils d’Israël par clans, par familles, en comptant nommément tous les hommes, un par un. 
${}^{3}Tous ceux qui ont vingt ans et plus, ceux qui, en Israël, sont aptes à rejoindre l’armée, toi et Aaron, vous les recenserez par formations de combat. 
${}^{4}Prenez avec vous un homme par tribu, un homme qui soit chef de famille.
${}^{5}Voici les noms des hommes qui vous assisteront :
      <p class="retrait1">pour Roubène : Éliçour, fils de Shedéour ;
      <p class="retrait1char">
${}^{6}pour Siméon : Sheloumiel, fils de Sourishaddaï ;
      <p class="retrait1char">
${}^{7}pour Juda : Nahshone, fils d’Amminadab ;
      <p class="retrait1char">
${}^{8}pour Issakar : Netanel, fils de Souar ;
      <p class="retrait1char">
${}^{9}pour Zabulon : Éliab, fils de Hélone ;
${}^{10}et quant aux fils de Joseph :
      <p class="retrait1">pour Éphraïm : Élishama, fils d’Ammihoud ;
      <p class="retrait1">pour Manassé : Gameliël, fils de Pedahçour ;
${}^{11}pour Benjamin : Abidane, fils de Guidéoni ;
${}^{12}pour Dane : Ahiézer, fils d’Ammishaddaï ;
${}^{13}pour Asher : Paguiël, fils d’Okrane ;
${}^{14}pour Gad : Élyasaf, fils de Déouël ;
${}^{15}pour Nephtali : Ahira, fils d’Einane. »
${}^{16}Tels furent les délégués de la communauté, les responsables des tribus patriarcales, les chefs des clans en Israël. 
${}^{17}Moïse et Aaron prirent ces hommes désignés par leurs noms, 
${}^{18}ils rassemblèrent toute la communauté, le premier jour du deuxième mois, et les hommes furent enregistrés par clans, par familles, en comptant nommément, un par un, ceux qui avaient vingt ans et plus. 
${}^{19}Comme le Seigneur l’avait ordonné à Moïse, on les recensa au désert du Sinaï.
${}^{20}Voici le résultat.
      Descendance par clans, par familles, des fils de Roubène, le premier-né d’Israël, en comptant nommément, un par un, tous les hommes âgés de vingt ans et plus, tous ceux qui pourraient aller au combat : 
${}^{21}pour la tribu de Roubène, on en recensa 46 500.
${}^{22}Descendance par clans, par familles, des fils de Siméon, recensés en comptant nommément, un par un, tous les hommes âgés de vingt ans et plus, tous ceux qui pourraient aller au combat : 
${}^{23}pour la tribu de Siméon, on en recensa 59 300.
${}^{24}Descendance par clans, par familles, des fils de Gad, en comptant nommément ceux qui avaient vingt ans et plus, tous ceux qui pourraient aller au combat : 
${}^{25}pour la tribu de Gad, on en recensa 45 650.
${}^{26}Descendance par clans, par familles, des fils de Juda, en comptant nommément ceux qui avaient vingt ans et plus, tous ceux qui pourraient aller au combat : 
${}^{27}pour la tribu de Juda, on en recensa 74 600.
${}^{28}Descendance par clans, par familles, des fils d’Issakar, en comptant nommément ceux qui avaient vingt ans et plus, tous ceux qui pourraient aller au combat : 
${}^{29}pour la tribu d’Issakar, on en recensa 54 400.
${}^{30}Descendance par clans, par familles, des fils de Zabulon, en comptant nommément ceux qui avaient vingt ans et plus, tous ceux qui pourraient aller au combat : 
${}^{31}pour la tribu de Zabulon, on en recensa 57 400.
${}^{32}Descendance par clans, par familles, des fils de Joseph, en ce qui concerne les fils d’Éphraïm, en comptant nommément ceux qui avaient vingt ans et plus, tous ceux qui pourraient aller au combat : 
${}^{33}pour la tribu d’Éphraïm, on en recensa 40 500.
${}^{34}En ce qui concerne les fils de Manassé, descendance par clans, par familles, en comptant nommément ceux qui avaient vingt ans et plus, tous ceux qui pourraient aller au combat : 
${}^{35}pour la tribu de Manassé, on en recensa 32 200.
${}^{36}Descendance par clans, par familles, des fils de Benjamin, en comptant nommément ceux qui avaient vingt ans et plus, tous ceux qui pourraient aller au combat : 
${}^{37}pour la tribu de Benjamin, on en recensa 35 400.
${}^{38}Descendance par clans, par familles, des fils de Dane, en comptant nommément ceux qui avaient vingt ans et plus, tous ceux qui pourraient aller au combat : 
${}^{39}pour la tribu de Dane, on en recensa 62 700.
${}^{40}Descendance par clans, par familles, des fils d’Asher, en comptant nommément ceux qui avaient vingt ans et plus, tous ceux qui pourraient aller au combat : 
${}^{41}pour la tribu d’Asher, on en recensa 41 500.
${}^{42}Descendance par clans, par familles, des fils de Nephtali, en comptant nommément ceux qui avaient vingt ans et plus, tous ceux qui pourraient aller au combat : 
${}^{43}pour la tribu de Nephtali, on en recensa 53 400.
${}^{44}Voilà ceux que recensèrent Moïse et Aaron avec les douze chefs d’Israël, soit un homme par tribu. 
${}^{45}On obtint le total des fils d’Israël recensés par familles, âgés de vingt ans et plus, tous ceux qui, en Israël, pourraient aller au combat. 
${}^{46}Total des recensés : 603 550. 
${}^{47}Les Lévites, en tant que tribu patriarcale, ne faisaient pas partie du recensement.
${}^{48}Le Seigneur parla à Moïse. Il dit : 
${}^{49}« Seule la tribu de Lévi, tu ne la recenseras pas, tu n’en feras pas le dénombrement parmi les fils d’Israël. 
${}^{50}Mais toi, confie aux Lévites la Demeure du Témoignage, tous ses objets, tout ce qu’elle contient ; ce sont les Lévites qui porteront la Demeure et tous ses objets ; ils en assureront le service. Ils camperont autour de la Demeure. 
${}^{51}Quand on déplacera la Demeure, les Lévites la démonteront, et quand on installera la Demeure, les Lévites la dresseront. Le profane qui approchera sera mis à mort.
${}^{52}Les fils d’Israël camperont chacun dans son camp, chacun sous son étendard, par formations de combat. 
${}^{53}Mais les Lévites camperont autour de la Demeure du Témoignage ; ainsi Dieu ne s’irritera pas contre la communauté des fils d’Israël. Les Lévites prendront soin de la Demeure du Témoignage. »
${}^{54}Les fils d’Israël firent tout ce que le Seigneur avait ordonné à Moïse. Ainsi firent-ils.
      
         
      \bchapter{}
      \begin{verse}
${}^{1}Le Seigneur parla à Moïse et Aaron. Il dit : 
${}^{2}« Les fils d’Israël camperont chacun sous son étendard, sous les enseignes de leurs familles, ils camperont tout autour de la tente de la Rencontre.
${}^{3}Camperont à l’est, à l’orient, par formations de combat, ceux qui appartiennent à l’étendard du camp de Juda. Le chef des fils de Juda est Nahshone, fils d’Amminadab ; 
${}^{4}selon le recensement, son armée est de 74 600 hommes.
${}^{5}Campera près de lui la tribu d’Issakar. Le chef des fils d’Issakar est Netanel, fils de Souar ; 
${}^{6}selon le recensement, son armée est de 54 400 hommes.
${}^{7}Campera aussi la tribu de Zabulon. Le chef des fils de Zabulon est Éliab, fils de Hélone ; 
${}^{8}selon le recensement, son armée est de 57 400 hommes. 
${}^{9}Total des recensés du camp de Juda : 186 400 hommes, groupés par formations de combat. Ils lèveront le camp en premier.
${}^{10}Au sud, l’étendard du camp de Roubène, les hommes étant par formations de combat. Le chef des fils de Roubène est Éliçour, fils de Shedéour ; 
${}^{11}selon le recensement, son armée est de 46 500 hommes.
${}^{12}Campera près de lui la tribu de Siméon. Le chef des fils de Siméon est Sheloumiel, fils de Sourishaddaï ; 
${}^{13}selon le recensement, son armée est de 59 300 hommes.
${}^{14}Campera aussi la tribu de Gad. Le chef des fils de Gad est Élyasaf, fils de Réouël ; 
${}^{15}selon le recensement, son armée est de 45 650 hommes. 
${}^{16}Total des recensés du camp de Roubène : 151 450 hommes, groupés par formations de combat. Ils lèveront le camp les deuxièmes.
${}^{17}Alors partira la tente de la Rencontre, le camp des Lévites étant au milieu des autres camps. Dans l’ordre où ils auront campé, ils lèveront le camp, chacun à leur tour, par étendards.
${}^{18}À l’ouest, l’étendard du camp d’Éphraïm, les hommes étant par formations de combat. Le chef des fils d’Éphraïm est Élishama, fils d’Ammihoud ; 
${}^{19}selon le recensement, son armée est de 40 500 hommes.
${}^{20}Campera près de lui la tribu de Manassé. Le chef des fils de Manassé est Gameliël, fils de Pedahçour ; 
${}^{21}selon le recensement, son armée est de 32 200 hommes.
${}^{22}Campera aussi la tribu de Benjamin. Le chef des fils de Benjamin est Abidane, fils de Guidoni ; 
${}^{23}selon le recensement, son armée est de 35 400 hommes. 
${}^{24}Total des recensés du camp d’Éphraïm : 108 100 hommes, groupés par formations de combat. Ils lèveront le camp les troisièmes.
${}^{25}Au nord, l’étendard du camp de Dane, les hommes étant par formations de combat. Le chef des fils de Dane est Ahiézer, fils d’Ammishaddaï ; 
${}^{26}selon le recensement, son armée est de 62 700 hommes.
${}^{27}Campera près de lui la tribu d’Asher. Le chef des fils d’Asher est Paguiël, fils d’Okrane ; 
${}^{28}selon le recensement, son armée est de 41 500 hommes.
${}^{29}Campera aussi la tribu de Nephtali. Le chef des fils de Nephtali est Ahira, fils d’Einane ; 
${}^{30}selon le recensement, son armée est de 53 400 hommes. 
${}^{31}Total des recensés du camp de Dane : 157 600 hommes. Ils lèveront le camp en dernier, par étendards. »
${}^{32}Tels étaient les recensés des fils d’Israël, par familles. Total des recensés des camps par formations de combat : 603 550 hommes. 
${}^{33}Mais les Lévites ne furent pas recensés parmi les fils d’Israël, comme le Seigneur l’avait ordonné à Moïse. 
${}^{34}Les fils d’Israël firent tout ce que le Seigneur avait ordonné à Moïse : c’est ainsi qu’ils campaient par étendards et levaient le camp, par clans, par familles.
      
         
      \bchapter{}
      \begin{verse}
${}^{1}Voici la descendance d’Aaron et de Moïse, le jour où le Seigneur parla à Moïse sur le mont Sinaï.
${}^{2}Voici les noms des fils d’Aaron : l’aîné, Nadab, puis Abihou, Éléazar et Itamar. 
${}^{3}Tels sont les noms des fils d’Aaron, les prêtres consacrés par l’onction, qui étaient investis de la fonction sacerdotale. 
${}^{4}Nadab et Abihou moururent devant le Seigneur, car, devant le Seigneur, ils avaient présenté un feu profane, au désert du Sinaï. Ils n’avaient pas eu de fils. Quant à Éléazar et Itamar, ils exercèrent le sacerdoce en présence de leur père Aaron.
${}^{5}Le Seigneur parla à Moïse. Il dit : 
${}^{6}« Fais approcher la tribu de Lévi : qu’elle soit à la disposition du prêtre Aaron pour l’assister. 
${}^{7}Les Lévites prendront soin de tout ce qui est confié à sa garde, et à celle de toute la communauté ; devant la tente de la Rencontre, ils accompliront le service de la Demeure. 
${}^{8}Ils prendront soin de tous les objets se trouvant dans la tente de la Rencontre et de tout ce qui est confié aux fils d’Israël : ils accompliront le service de la Demeure. 
${}^{9}Tu donneras les Lévites à Aaron et à ses fils ; c’est eux, les Lévites\\, qui, parmi les fils d’Israël, leur seront entièrement donnés. 
${}^{10}Tu établiras Aaron et ses fils dans leur charge pour qu’ils exercent le sacerdoce. Le profane qui approchera sera mis à mort. »
${}^{11}Le Seigneur parla à Moïse. Il dit : 
${}^{12}« Moi, j’ai pris les Lévites au milieu des fils d’Israël, à la place de tout premier-né parmi les fils d’Israël. Ainsi les Lévites sont à moi. 
${}^{13}Car c’est à moi qu’appartient tout premier-né : le jour où j’ai frappé tout premier-né au pays d’Égypte, je me suis consacré tout premier-né en Israël ; depuis l’homme jusqu’au bétail, ils sont à moi. Je suis le Seigneur. »
${}^{14}Au désert du Sinaï, le Seigneur parla à Moïse. Il dit : 
${}^{15}« Recense les fils de Lévi, par familles, par clans. Tu recenseras tous ceux de sexe masculin, âgés d’un mois et plus. » 
${}^{16}Moïse les recensa sur l’ordre du Seigneur et comme le Seigneur le lui avait ordonné.
${}^{17}Voici les noms des fils de Lévi : Guershone, Qehath et Merari. 
${}^{18}Et voici les noms des fils de Guershone, par clans : Libni et Shiméï. 
${}^{19}Les fils de Qehath, par clans : Amrane et Yicehar, Hébrone et Ouzziël. 
${}^{20}Les fils de Merari, par clans : Mahli et Moushi. Tels sont les clans de Lévi, par familles.
${}^{21}Appartiennent à Guershone le clan des Libnites et le clan des Shiméites : tels sont les clans des Guershonites. 
${}^{22}Le nombre de tous les recensés de sexe masculin, âgés d’un mois et plus, est de 7 500. 
${}^{23}Les clans des Guershonites campaient derrière la Demeure, à l’ouest. 
${}^{24}Le chef de famille des Guershonites était Élyasaf, fils de Laël. 
${}^{25}Le service des fils de Guershone, dans la tente de la Rencontre, concernait la Demeure et la tente, sa couverture et le voile d’entrée de la tente de la Rencontre, 
${}^{26}ainsi que les toiles du parvis et le voile d’entrée du parvis entourant la Demeure et l’autel, et les cordes pour tous les travaux.
${}^{27}Appartiennent à Qehath le clan des Amramites, le clan des Yiceharites, le clan des Hébronites et le clan des Ouzziélites ; tels sont les clans des Qehatites. 
${}^{28}Le nombre de tous ceux de sexe masculin, âgés d’un mois et plus, est de 8 600 ; ils prennent soin du sanctuaire. 
${}^{29}Les clans des fils de Qehath campaient sur le côté sud de la Demeure. 
${}^{30}Le chef de famille des clans des Qehatites était Élisafane, fils d’Ouzziël. 
${}^{31}Leur service concernait l’Arche, la Table, le chandelier, les autels, les objets du sanctuaire qu’ils utilisent pour le culte, le voile ainsi que tous les travaux. 
${}^{32}Le responsable des chefs des Lévites était Éléazar, le fils du prêtre Aaron ; il surveillait ceux qui prenaient soin du sanctuaire.
${}^{33}Appartiennent à Merari le clan des Mahlites et le clan des Moushites : tels sont les clans des Merarites. 
${}^{34}Le nombre de tous les recensés de sexe masculin, âgés d’un mois et plus, est de 6 200. 
${}^{35}Le chef de famille des clans des Merarites était Souriël, fils d’Abihaïl. Les Merarites campaient sur le côté nord de la Demeure. 
${}^{36}La fonction confiée aux fils de Merari concernait les cadres de la Demeure, ses traverses, ses colonnes, ses socles, tous ses objets ainsi que tous les travaux, 
${}^{37}et aussi les colonnes autour du parvis, leurs socles, leurs piquets et leurs cordes. 
${}^{38}Campaient devant la Demeure, à l’est, face à la tente de la Rencontre, à l’orient, Moïse, Aaron et ses fils ; ils prenaient soin du sanctuaire, ils en prenaient soin au nom des fils d’Israël. Le profane qui approchera sera mis à mort.
${}^{39}Le total des Lévites que recensèrent Moïse et Aaron sur l’ordre du Seigneur, par clans, tous ceux de sexe masculin, âgés d’un mois et plus, était de 22 000.
${}^{40}Le Seigneur dit à Moïse : « Recense tous les premiers-nés de sexe masculin des fils d’Israël, âgés d’un mois et plus, et fais le relevé de leurs noms. 
${}^{41}Tu prendras pour moi les Lévites – Je suis le Seigneur – en échange de tous les premiers-nés des fils d’Israël, et aussi le bétail des Lévites en échange de tous les premiers-nés du bétail des fils d’Israël. » 
${}^{42}Moïse recensa donc tous les premiers-nés des fils d’Israël, comme le lui avait ordonné le Seigneur. 
${}^{43}Le total de tous les premiers-nés de sexe masculin, en comptant nommément les recensés âgés d’un mois et plus, était de 22 273.
${}^{44}Le Seigneur parla à Moïse. Il dit : 
${}^{45}« Prends les Lévites en échange de tous les premiers-nés des fils d’Israël et le bétail des Lévites en échange de leur bétail. Les Lévites seront à moi. Je suis le Seigneur. 
${}^{46}Pour le rachat des 273 premiers-nés des fils d’Israël en surnombre par rapport aux Lévites, 
${}^{47}tu prendras cinq sicles par tête ; tu les prendras en monnaie du sanctuaire où le sicle fait vingt guéras. 
${}^{48}Tu donneras l’argent à Aaron et à ses fils pour le rachat des premiers-nés qui sont en surnombre. » 
${}^{49}Moïse prit l’argent du rachat provenant de ceux qui étaient en surnombre par rapport à ceux déjà rachetés par les Lévites. 
${}^{50}Il prit l’argent des premiers-nés des fils d’Israël, soit 1 365 sicles en monnaie du sanctuaire. 
${}^{51}Puis Moïse remit à Aaron et à ses fils l’argent du rachat, sur l’ordre du Seigneur et comme le Seigneur l’avait ordonné à Moïse.
      
         
      \bchapter{}
      \begin{verse}
${}^{1}Le Seigneur parla à Moïse et Aaron. Il dit : 
${}^{2}« Parmi les fils de Lévi, dénombrez les fils de Qehath, par clans, par familles, 
${}^{3}tous ceux qui ont de trente à cinquante ans et sont destinés au service du culte, pour travailler dans la tente de la Rencontre. 
${}^{4}Voici la tâche des fils de Qehath dans la tente de la Rencontre : s’occuper du Saint des Saints. 
${}^{5}Quand on lèvera le camp, Aaron et ses fils viendront descendre le rideau qui sert de voile et en couvriront l’arche du Témoignage. 
${}^{6}Ils poseront dessus une couverture en peau de dauphin, puis déploieront par-dessus un drap tout de pourpre violette ; enfin ils mettront les barres de l’Arche. 
${}^{7}Sur la table en face de moi, ils déploieront un drap de pourpre violette et poseront sur elle les plats, les gobelets, les timbales et les aiguières pour les libations ; là sera le pain perpétuel. 
${}^{8}Sur tout cela, ils déploieront un drap de cramoisi éclatant et le recouvriront d’une couverture en peau de dauphin ; puis ils mettront les barres de la table. 
${}^{9}Ils prendront un drap de pourpre violette et en couvriront le chandelier du luminaire, ses lampes, ses pincettes, ses porte-lampes et tous les vases d’huile qu’ils utilisent pour le culte. 
${}^{10}Ils placeront le chandelier et tous ses accessoires dans une couverture en peau de dauphin et placeront le tout sur le brancard. 
${}^{11}Sur l’autel d’or, ils déploieront un drap de pourpre violette qu’ils recouvriront d’une couverture en peau de dauphin ; puis ils mettront les barres de l’autel. 
${}^{12}Ils prendront tous les objets dont ils se servent pour le culte dans le sanctuaire ; ils les placeront sur un drap de pourpre violette et les couvriront d’une couverture en peau de dauphin, puis ils les placeront sur le brancard. 
${}^{13}Ils enlèveront les cendres de l’autel et déploieront sur lui un drap de pourpre rouge, 
${}^{14}ils y placeront tous les objets dont ils se servent pour le culte : porte-lampes, fourchettes, pelles, bols pour l’aspersion, tous les objets de l’autel. Ils déploieront par-dessus une couverture en peau de dauphin et mettront les barres de l’autel. 
${}^{15}Aaron et ses fils finiront de couvrir le sanctuaire et tous les objets du sanctuaire, au moment de lever le camp. Ensuite viendront les fils de Qehath pour le transport, mais ils ne toucheront pas au sanctuaire car ils mourraient. Telle est donc la charge des fils de Qehath dans la tente de la Rencontre. 
${}^{16}Éléazar, fils du prêtre Aaron, s’occupera de l’huile du luminaire, de l’encens aromatique, de l’offrande perpétuelle et de l’huile de l’onction ; il s’occupera de toute la Demeure et de tout ce qu’elle contient, du sanctuaire aux accessoires. »
${}^{17}Le Seigneur parla à Moïse et Aaron. Il dit : 
${}^{18}« N’exposez pas la branche des clans qehatites à être retranchée du milieu des Lévites. 
${}^{19}Voici comment vous agirez envers eux afin qu’ils vivent, et qu’en s’approchant du Saint des Saints, ils ne meurent pas : Aaron et ses fils viendront et affecteront les uns à leur travail, et les autres à la charge du transport. 
${}^{20}Ainsi, les Qehatites n’entreront pas pour voir le sanctuaire, ne fût-ce qu’un instant, car ils mourraient. »
${}^{21}Le Seigneur parla à Moïse. Il dit : 
${}^{22}« Fais aussi le dénombrement des fils de Guershone, par familles, par clans. 
${}^{23}Ceux qui ont de trente à cinquante ans, tu les recenseras, tous ceux qui sont destinés au service du culte pour remplir une tâche dans la tente de la Rencontre. 
${}^{24}Voici la tâche des clans des Guershonites, leur travail et leur charge : 
${}^{25}transporter les tentures de la Demeure ainsi que la tente de la Rencontre avec sa couverture, la couverture en dauphin qui est au-dessus d’elle, et le voile d’entrée de la tente de la Rencontre ; 
${}^{26}ils transporteront aussi les toiles du parvis et le voile de la porte d’entrée du parvis entourant la Demeure et l’autel, ainsi que leurs cordes et tous les accessoires, tout ce qui leur a été fourni pour leur travail. 
${}^{27}C’est sous les ordres d’Aaron et de ses fils que s’accomplira la tâche des fils des Guershonites, en ce qui concerne toute la charge du transport et tout leur travail ; vous aurez autorité pour surveiller tout le transport. 
${}^{28}Telle est la tâche des clans des fils des Guershonites dans la tente de la Rencontre, et leur service s’effectuera sous la direction d’Itamar, fils du prêtre Aaron.
${}^{29}Les fils de Merari, tu les recenseras par clans, par familles. 
${}^{30}Ceux qui ont de trente à cinquante ans, tu les recenseras, tous ceux qui sont destinés au service du culte pour remplir une tâche dans la tente de la Rencontre. 
${}^{31}Voici ce qu’il leur revient de transporter, pour remplir le service de la tente de la Rencontre : les cadres de la Demeure, ses traverses, ses colonnes et ses socles, 
${}^{32}ainsi que les colonnes autour du parvis, leurs socles, leurs piquets et leurs cordes, et tous les outils de travail. Vous dresserez l’inventaire des objets qu’il leur revient de transporter. 
${}^{33}Telle est la tâche des clans des fils de Merari, leur tâche dans la tente de la Rencontre sous la direction d’Itamar, fils du prêtre Aaron. »
${}^{34}Moïse et Aaron ainsi que les chefs de la communauté recensèrent donc les fils des Qehatites par clans, par familles, 
${}^{35}ceux qui avaient de trente à cinquante ans, tous ceux qui étaient destinés au service du culte et devaient remplir une tâche dans la tente de la Rencontre. 
${}^{36}Ces recensés par clans furent au nombre de 2 750. 
${}^{37}Tels furent les recensés des clans qehatites, tous ceux qui travaillaient dans la tente de la Rencontre ; Moïse et Aaron les avaient recensés sur l’ordre du Seigneur transmis par Moïse. 
${}^{38}On recensa par clans, par familles, les fils de Guershone, 
${}^{39}âgés de trente à cinquante ans, tous ceux qui étaient destinés au service du culte et devaient remplir une tâche dans la tente de la Rencontre ; 
${}^{40}ces recensés par clans, par familles, furent au nombre de 2 630. 
${}^{41}Tels furent les recensés des clans des fils de Guershone, tous ceux qui travaillaient dans la tente de la Rencontre ; Moïse et Aaron les avaient recensés sur l’ordre du Seigneur. 
${}^{42}On recensa par clans, par familles, les clans des fils de Merari, 
${}^{43}âgés de trente à cinquante ans, tous ceux qui étaient destinés au service du culte et devaient remplir une tâche dans la tente de la Rencontre ; 
${}^{44}ces recensés par clans furent au nombre de 3 200. 
${}^{45}Tels furent les recensés des clans des fils de Merari ; Moïse et Aaron les avaient recensés sur l’ordre du Seigneur transmis par Moïse.
${}^{46}Total de ceux que recensèrent, par clans, par familles, Moïse, Aaron et les chefs d’Israël parmi les Lévites, 
${}^{47}ceux qui étaient âgés de trente à cinquante ans, tous ceux qui, dans la tente de la Rencontre, étaient destinés à aider au service du culte, et à prendre en charge le transport 
${}^{48}– total de ces recensés : 8 580. 
${}^{49}C’est sur l’ordre du Seigneur qu’ils furent recensés par Moïse, chacun en vue de son travail et de sa charge. Le recensement eut lieu comme le Seigneur l’avait ordonné à Moïse.
      
         
      \bchapter{}
      \begin{verse}
${}^{1}Le Seigneur parla à Moïse. Il dit : 
${}^{2}« Ordonne aux fils d’Israël de renvoyer du camp tous les lépreux, toute personne atteinte d’un écoulement ou rendue impure par le contact d’un mort. 
${}^{3}Homme ou femme, vous les enverrez à l’extérieur du camp, vous les renverrez pour qu’ils ne rendent pas impur ce camp où je demeure au milieu d’eux. » 
${}^{4}Ainsi firent les fils d’Israël : ils les envoyèrent à l’extérieur du camp. Comme le Seigneur l’avait dit à Moïse, ainsi firent les fils d’Israël.
${}^{5}Le Seigneur parla à Moïse. Il dit : 
${}^{6}« Parle aux fils d’Israël : Lorsque un homme ou une femme commet l’un ou l’autre péché envers autrui, se rendant ainsi infidèle au Seigneur, ces gens-là sont coupables. 
${}^{7}Ils confesseront le péché qu’ils ont fait. Le coupable restituera entièrement l’objet du délit, il ajoutera un cinquième de sa valeur, et donnera le tout à celui envers qui il est coupable. 
${}^{8}Si ce dernier a disparu sans avoir de parent proche à qui restituer l’objet du délit, l’objet du délit restitué au Seigneur ira au prêtre ; il y aura en outre le bélier d’expiation pour le rite d’expiation sur le coupable. 
${}^{9}Toute part prélevée concernant les choses saintes que les fils d’Israël consacreront et apporteront au prêtre lui reviendra. 
${}^{10}À chacun les choses saintes qu’il a consacrées. Mais ce qu’il donne au prêtre est au prêtre. »
${}^{11}Le Seigneur parla à Moïse. Il dit : 
${}^{12}« Parle aux fils d’Israël. Tu leur diras : Soit un homme dont la femme se conduit mal, en lui étant infidèle. 
${}^{13}Un autre homme couche avec elle pour avoir des rapports sexuels avec elle à l’insu du mari, car elle s’est dissimulée tandis qu’elle se rendait impure : il n’y a pas de témoin contre elle, elle n’a pas été surprise. 
${}^{14}Mais un esprit de jalousie s’empare de l’homme et il devient jaloux de sa femme qui s’est rendue impure ; ou bien un esprit de jalousie s’empare de l’homme et il devient jaloux de sa femme alors qu’elle ne s’est pas rendue impure. 
${}^{15}Dans l’un ou l’autre cas, l’homme amènera sa femme au prêtre et apportera pour elle le présent réservé : le dixième d’une mesure de farine d’orge. Il n’y versera pas d’huile et n’y mettra pas d’encens car c’est une offrande de jalousie, une offrande en mémorial rappelant une faute.
${}^{16}Le prêtre fera approcher la femme et la placera devant le Seigneur. 
${}^{17}Il prendra de l’eau sainte dans un vase d’argile, il prendra de la poussière qui se trouve sur le sol de la Demeure, et la mettra dans l’eau. 
${}^{18}Alors il placera la femme devant le Seigneur et lui dénouera les cheveux. Il posera sur ses paumes l’offrande en mémorial – c’est une offrande de jalousie –, et lui-même aura dans ses mains l’eau amère qui porte la malédiction.
${}^{19}Le prêtre fera prêter serment à la femme et lui dira : “Si aucun homme n’a couché avec toi, si tu ne t’es pas mal conduite et rendue impure en trompant ton mari, sois innocentée par cette eau amère qui porte la malédiction ! 
${}^{20}Mais si toi, tu t’es mal conduite en trompant ton mari, si tu t’es rendue impure, si un autre homme que ton mari a eu des rapports sexuels avec toi…” 
${}^{21}– alors le prêtre fera prêter serment à la femme, un serment imprécatoire ; il lui dira : “Que le Seigneur fasse de toi un objet d’imprécation et de serment au milieu de ton peuple, qu’il fasse dépérir ton flanc et gonfler ton ventre ! 
${}^{22}Que cette eau qui porte la malédiction pénètre tes entrailles pour faire gonfler ton ventre et dépérir ton flanc !” Et la femme dira : “Amen ! Amen !”
${}^{23}Puis le prêtre écrira ces imprécations sur un document et les effacera dans l’eau amère. 
${}^{24}Il fera boire à la femme l’eau amère qui porte la malédiction, et l’eau qui porte la malédiction entrera en elle pour la rendre amère. 
${}^{25}Le prêtre prendra des mains de la femme l’offrande de jalousie. Il la présentera au Seigneur, sur l’autel, avec le geste d’élévation. 
${}^{26}Puis le prêtre prélèvera sur l’offrande une poignée de farine comme mémorial et la fera fumer sur l’autel ; c’est alors qu’il fera boire l’eau à la femme. 
${}^{27}S’il se trouve qu’elle s’est rendue impure, si elle a été infidèle à son mari, l’eau qui porte la malédiction entrera en elle pour la rendre amère, son ventre gonflera et son flanc dépérira. Alors la femme deviendra un objet d’imprécation au milieu de son peuple. 
${}^{28}Mais si la femme ne s’est pas rendue impure, si elle est restée pure, elle sera innocentée et pourra être féconde.
${}^{29}Telle est la loi de jalousie concernant la femme qui se conduit mal en trompant son mari et se rend impure, 
${}^{30}ou concernant l’homme dont s’empare un esprit de jalousie et qui devient jaloux de sa femme. Celui-ci placera sa femme devant le Seigneur, et le prêtre accomplira pour elle toute cette loi. 
${}^{31}L’homme sera exempt de faute et cette femme portera le poids de sa propre faute. »
      
         
      \bchapter{}
      \begin{verse}
${}^{1}Le Seigneur parla à Moïse. Il dit : 
${}^{2}« Parle aux fils d’Israël. Tu leur diras : Quand un homme ou une femme fait un vœu particulier, le vœu de naziréat, par lequel il se voue au Seigneur, 
${}^{3}il s’abstiendra de vin et de boisson forte, il ne boira ni vinaigre de vin ni vinaigre d’alcool, il ne boira aucun jus de raisin, il ne mangera ni raisins frais ni raisins secs. 
${}^{4}Tous les jours de son naziréat, il ne mangera aucun produit de la vigne, même pas les pépins ou la peau. 
${}^{5}Tous les jours de son vœu de naziréat, le rasoir ne passera pas sur sa tête. Jusqu’à la fin de cette période de naziréat, il sera saint pour le Seigneur, il laissera pousser librement sa chevelure. 
${}^{6}Tous les jours de son naziréat pour le Seigneur, il n’approchera d’aucun mort. 
${}^{7}Père ou mère, frère ou sœur, pour aucun d’eux, à leur mort, il ne se rendra impur, car il porte sur la tête le signe de la consécration à son Dieu.
${}^{8}Tous les jours de son naziréat, il sera saint pour le Seigneur. 
${}^{9}Si quelqu’un meurt subitement près de lui, rendant ainsi impure sa tête sanctifiée par le naziréat, il se rasera la tête le jour de sa purification : le septième jour il la rasera. 
${}^{10}Et, le huitième jour, il apportera deux tourterelles ou deux jeunes colombes au prêtre, à l’entrée de la tente de la Rencontre. 
${}^{11}Le prêtre offrira l’une en sacrifice pour la faute, l’autre en holocauste. Il fera sur le naziréen le rite d’expiation de la faute que celui-ci aura commise en touchant un mort, et ce jour-là le naziréen sanctifiera de nouveau sa tête. 
${}^{12}Le naziréen reconsacrera au Seigneur le temps de son naziréat ; il amènera un agneau de l’année pour le sacrifice de réparation ; les jours précédents ne comptent pas, puisque son naziréat était devenu impur.
${}^{13}Voici la loi concernant le naziréen : le jour où se termine son naziréat, on le conduira à l’entrée de la tente de la Rencontre. 
${}^{14}Il apportera au Seigneur son présent réservé : un agneau de l’année, sans défaut, pour l’holocauste, et une agnelle de l’année, sans défaut, pour le sacrifice pour la faute, et un bélier, sans défaut, pour le sacrifice de paix, 
${}^{15}ainsi qu’une corbeille de pains sans levain, faits de fleur de farine, des gâteaux pétris à l’huile, et des galettes sans levain frottées d’huile, ainsi que l’offrande de céréales et les libations requises. 
${}^{16}Le prêtre les apportera devant le Seigneur et il accomplira son sacrifice pour la faute et son holocauste. 
${}^{17}Quant au bélier, il l’offrira en sacrifice de paix pour le Seigneur, en plus de la corbeille de pains sans levain ; puis il fera l’offrande de céréales et la libation requises. 
${}^{18}Alors, à l’entrée de la tente de la Rencontre, le naziréen se rasera la tête, la tête de son naziréat ; il prendra la chevelure de son naziréat et la posera sur le feu où se consume le sacrifice de paix. 
${}^{19}Le prêtre prendra l’épaule du bélier, quand elle sera cuite, un gâteau sans levain dans la corbeille et une galette sans levain ; il les posera sur les paumes du naziréen après que celui-ci aura rasé la tête de son naziréat. 
${}^{20}Puis le prêtre les présentera au Seigneur avec le geste d’élévation : c’est une chose sainte qui reviendra au prêtre en plus de la poitrine présentée avec le geste d’élévation et de la cuisse prélevée. Ensuite, le naziréen boira du vin.
${}^{21}Telle est la loi du naziréen qui a fait un vœu ; tel est le présent qu’il réserve au Seigneur à l’occasion de son naziréat, sans tenir compte de ce que ses moyens lui permettraient d’ajouter. Selon le vœu qu’il a prononcé, ainsi fera-t-il selon la loi de son naziréat. »
${}^{22}Le Seigneur parla à Moïse. Il dit : 
${}^{23} « Parle à Aaron et à ses fils. Tu leur diras : Voici en quels termes vous bénirez les fils d’Israël :
        ${}^{24}“Que le Seigneur te bénisse et te garde !
        ${}^{25}Que le Seigneur fasse briller sur toi son visage,
        \\qu’il te prenne en grâce\\ !
        ${}^{26}Que le Seigneur tourne\\vers toi son visage,
        \\qu’il t’apporte\\la paix !”
${}^{27}Ils invoqueront ainsi mon nom sur les fils d’Israël, et moi, je les bénirai. »
      
         
      \bchapter{}
      \begin{verse}
${}^{1}Le jour où Moïse acheva de dresser la Demeure, il fit l’onction sur elle et la sanctifia, ainsi que tous ses objets ; il fit aussi l’onction sur l’autel et tous ses objets, et les sanctifia. 
${}^{2}Alors, les responsables d’Israël, les chefs de famille, ceux qui étaient responsables des tribus, ceux qui avaient présidé au recensement apportèrent leurs présents. 
${}^{3}Ils amenèrent devant le Seigneur les présents qu’ils lui avaient réservés : six chariots couverts et douze bœufs, un chariot pour deux responsables et un bœuf pour chaque responsable. Ils les firent approcher, devant la Demeure. 
${}^{4}Alors le Seigneur dit à Moïse : 
${}^{5}« Accepte leurs présents pour qu’ils servent aux travaux de la tente de la Rencontre. Tu les donneras aux Lévites, à chacun en fonction de son service. »
${}^{6}Moïse reçut donc les chariots et les bœufs et les donna aux Lévites. 
${}^{7}Il donna deux chariots et quatre bœufs aux fils de Guershone, en fonction de leur service. 
${}^{8}Il donna quatre chariots et huit bœufs aux fils de Merari, en fonction de leur service, par l’intermédiaire d’Itamar, fils du prêtre d’Aaron. 
${}^{9}Mais aux fils de Qehath il ne donna rien car, dans le service du sanctuaire qui leur incombait, ils portaient les objets sur leurs épaules.
${}^{10}Les responsables apportèrent leurs présents pour la dédicace de l’autel, le jour où on fit sur lui l’onction ; les responsables apportèrent devant l’autel les présents qu’ils avaient réservés.
${}^{11}Le Seigneur dit à Moïse : « Chaque jour, un des responsables apportera pour la dédicace de l’autel le présent qu’il aura réservé. »
${}^{12}Le premier jour, ce fut Nahshone, fils d’Amminadab, de la tribu de Juda, qui apporta son présent. 
${}^{13}Ce présent comprenait un plat d’argent pesant cent trente sicles, une coupe en argent de soixante-dix sicles – selon le sicle du sanctuaire –, tous deux remplis de fleur de farine pétrie à l’huile, pour l’offrande de céréales ; 
${}^{14}puis un gobelet de dix sicles d’or rempli d’encens ; 
${}^{15}puis, pour l’holocauste, un taureau, un bélier, un agneau de l’année ; 
${}^{16}ensuite un bouc destiné au sacrifice pour la faute 
${}^{17}et, pour le sacrifice de paix, deux bœufs, cinq béliers, cinq boucs, cinq agneaux de l’année. Tel fut le présent de Nahshone, fils d’Amminadab.
${}^{18}Le deuxième jour, Netanel, fils de Souar, chef d’Issakar, apporta son présent. 
${}^{19}Il apporta comme présent un plat d’argent pesant cent trente sicles, une coupe en argent de soixante-dix sicles – selon le sicle du sanctuaire –, tous deux remplis de fleur de farine pétrie à l’huile, pour l’offrande de céréales ; 
${}^{20}puis un gobelet de dix sicles d’or rempli d’encens ; 
${}^{21}puis, pour l’holocauste, un taureau, un bélier, un agneau de l’année ; 
${}^{22}ensuite un bouc destiné au sacrifice pour la faute 
${}^{23}et, pour le sacrifice de paix, deux bœufs, cinq béliers, cinq boucs, cinq agneaux de l’année. Tel fut le présent de Netanel, fils de Souar.
${}^{24}Le troisième jour, ce fut le chef des fils de Zabulon, Éliab, fils de Hélone. 
${}^{25}Son présent comprenait un plat d’argent pesant cent trente sicles, une coupe en argent de soixante-dix sicles – selon le sicle du sanctuaire –, tous deux remplis de fleur de farine pétrie à l’huile, pour l’offrande de céréales ; 
${}^{26}puis un gobelet de dix sicles d’or rempli d’encens ; 
${}^{27}puis, pour l’holocauste, un taureau, un bélier, un agneau de l’année ; 
${}^{28}ensuite un bouc destiné au sacrifice pour la faute 
${}^{29}et, pour le sacrifice de paix, deux bœufs, cinq béliers, cinq boucs, cinq agneaux de l’année. Tel fut le présent d’Éliab, fils de Hélone.
${}^{30}Le quatrième jour, ce fut le chef des fils de Roubène, Éliçour, fils de Shedéour. 
${}^{31}Son présent comprenait un plat d’argent pesant cent trente sicles, une coupe d’argent de soixante-dix sicles – selon le sicle du sanctuaire –, tous deux remplis de fleur de farine pétrie à l’huile, pour l’offrande de céréales ; 
${}^{32}puis un gobelet de dix sicles d’or rempli d’encens ; 
${}^{33}puis, pour l’holocauste, un taureau, un bélier, un agneau de l’année ; 
${}^{34}ensuite un bouc destiné au sacrifice pour la faute 
${}^{35}et, pour le sacrifice de paix, deux bœufs, cinq béliers, cinq boucs, cinq agneaux de l’année. Tel fut le présent d’Éliçour, fils de Shedéour.
${}^{36}Le cinquième jour, ce fut le chef des fils de Siméon, Sheloumiel, fils de Sourishaddaï. 
${}^{37}Son présent comprenait un plat d’argent pesant cent trente sicles, une coupe en argent de soixante-dix sicles – selon le sicle du sanctuaire –, tous deux remplis de fleur de farine pétrie à l’huile, pour l’offrande de céréales ; 
${}^{38}puis un gobelet de dix sicles d’or rempli d’encens ; 
${}^{39}puis, pour l’holocauste, un taureau, un bélier, un agneau de l’année ; 
${}^{40}ensuite un bouc destiné au sacrifice pour la faute 
${}^{41}et, pour le sacrifice de paix, deux bœufs, cinq béliers, cinq boucs, cinq agneaux de l’année. Tel fut le présent de Sheloumiel, fils de Sourishaddaï.
${}^{42}Le sixième jour, ce fut le chef des fils de Gad, Élyasaf, fils de Déouël. 
${}^{43}Son présent comprenait un plat d’argent pesant cent trente sicles, une coupe en argent de soixante-dix sicles – selon le sicle du sanctuaire –, tous deux remplis de fleur de farine pétrie à l’huile, pour l’offrande de céréales ; 
${}^{44}puis un gobelet de dix sicles d’or rempli d’encens ; 
${}^{45}puis, pour l’holocauste, un taureau, un bélier, un agneau de l’année ; 
${}^{46}ensuite un bouc destiné au sacrifice pour la faute 
${}^{47}et, pour le sacrifice de paix, deux bœufs, cinq béliers, cinq boucs, cinq agneaux de l’année. Tel fut le présent d’Élyasaf, fils de Déouël.
${}^{48}Le septième jour, ce fut le chef des fils d’Éphraïm, Élishama, fils d’Ammihoud. 
${}^{49}Son présent comprenait un plat d’argent pesant cent trente sicles, une coupe en argent de soixante-dix sicles – selon le sicle du sanctuaire –, tous deux remplis de fleur de farine pétrie à l’huile, pour l’offrande de céréales ; 
${}^{50}puis un gobelet de dix sicles d’or rempli d’encens ; 
${}^{51}puis, pour l’holocauste, un taureau, un bélier, un agneau de l’année ; 
${}^{52}ensuite un bouc destiné au sacrifice pour la faute 
${}^{53}et, pour le sacrifice de paix, deux bœufs, cinq béliers, cinq boucs, cinq agneaux de l’année. Tel fut le présent d’Élishama, fils d’Ammihoud.
${}^{54}Le huitième jour, ce fut le chef des fils de Manassé, Gamliel, fils de Pedahçour. 
${}^{55}Son présent comprenait un plat d’argent pesant cent trente sicles, une coupe en argent de soixante-dix sicles – selon le sicle du sanctuaire –, tous deux remplis de fleur de farine pétrie à l’huile, pour l’offrande de céréales ; 
${}^{56}puis un gobelet de dix sicles d’or rempli d’encens ; 
${}^{57}puis, pour l’holocauste, un taureau, un bélier, un agneau de l’année ; 
${}^{58}ensuite un bouc destiné au sacrifice pour la faute 
${}^{59}et, pour le sacrifice de paix, deux bœufs, cinq béliers, cinq boucs, cinq agneaux de l’année. Tel fut le présent de Gamliel, fils de Pedahçour.
${}^{60}Le neuvième jour, ce fut le chef des fils de Benjamin, Abidane, fils de Guidéoni. 
${}^{61}Son présent comprenait un plat d’argent pesant cent trente sicles, une coupe en argent de soixante-dix sicles – selon le sicle du sanctuaire –, tous deux remplis de fleur de farine pétrie à l’huile, pour l’offrande de céréales ; 
${}^{62}puis un gobelet de dix sicles d’or rempli d’encens ; 
${}^{63}puis, pour l’holocauste, un taureau, un bélier, un agneau de l’année ; 
${}^{64}ensuite un bouc destiné au sacrifice pour la faute 
${}^{65}et, pour le sacrifice de paix, deux bœufs, cinq béliers, cinq boucs, cinq agneaux de l’année. Tel fut le présent d’Abidane, fils de Guidéoni.
${}^{66}Le dixième jour, ce fut le chef des fils de Dane, Ahiézer, fils d’Ammishaddaï. 
${}^{67}Son présent comprenait un plat d’argent pesant cent trente sicles, une coupe en argent de soixante-dix sicles – selon le sicle du sanctuaire –, tous deux remplis de fleur de farine pétrie à l’huile, pour l’offrande de céréales ; 
${}^{68}puis un gobelet de dix sicles d’or rempli d’encens ; 
${}^{69}puis, pour l’holocauste, un taureau, un bélier, un agneau de l’année ; 
${}^{70}ensuite un bouc destiné au sacrifice pour la faute 
${}^{71}et, pour le sacrifice de paix, deux bœufs, cinq béliers, cinq boucs, cinq agneaux de l’année. Tel fut le présent d’Ahiézer, fils d’Ammishaddaï.
${}^{72}Le onzième jour, ce fut le chef des fils d’Asher, Paguiël, fils d’Okrane. 
${}^{73}Son présent comprenait un plat d’argent pesant cent trente sicles, une coupe en argent de soixante-dix sicles – selon le sicle du sanctuaire –, tous deux remplis de fleur de farine pétrie à l’huile, pour l’offrande de céréales ; 
${}^{74}puis un gobelet de dix sicles d’or rempli d’encens ; 
${}^{75}puis, pour l’holocauste, un taureau, un bélier, un agneau de l’année ; 
${}^{76}ensuite un bouc destiné au sacrifice pour la faute 
${}^{77}et, pour le sacrifice de paix, deux bœufs, cinq béliers, cinq boucs, cinq agneaux de l’année. Tel fut le présent de Paguiël, fils d’Okrane.
${}^{78}Le douzième jour, ce fut le chef des fils de Nephtali, Ahira, fils d’Einane. 
${}^{79}Son présent comprenait un plat d’argent pesant cent trente sicles, une coupe en argent de soixante-dix sicles – selon le sicle du sanctuaire –, tous deux remplis de fleur de farine pétrie à l’huile, pour l’offrande de céréales ; 
${}^{80}puis un gobelet de dix sicles d’or rempli d’encens ; 
${}^{81}puis, pour l’holocauste, un taureau, un bélier, un agneau de l’année ; 
${}^{82}ensuite un bouc destiné au sacrifice pour la faute 
${}^{83}et, pour le sacrifice de paix, deux bœufs, cinq béliers, cinq boucs, cinq agneaux de l’année. Tel fut le présent d’Ahira, fils d’Einane.
${}^{84}Tels furent les présents des chefs d’Israël pour la dédicace de l’autel le jour où on en fit l’onction : douze plats d’argent, douze coupes en argent, douze gobelets en or. 
${}^{85}Chaque plat d’argent pesait cent trente sicles ; chaque coupe, soixante-dix sicles. Les objets d’argent pesaient, au total, 2 400 sicles, selon le sicle du sanctuaire. 
${}^{86}Ils apportèrent aussi douze gobelets en or remplis d’encens, pesant chacun dix sicles, selon le sicle du sanctuaire. Les gobelets en or pesaient, au total, cent vingt sicles. 
${}^{87}Le bétail pour l’holocauste comprenait au total douze taureaux, douze béliers, douze agneaux de l’année, avec les offrandes requises et, pour le sacrifice pour la faute, douze boucs. 
${}^{88}Le bétail pour le sacrifice de paix comprenait au total vingt-quatre taureaux, soixante béliers, soixante boucs, soixante agneaux de l’année. Telle fut la dédicace de l’autel, après qu’on eut fait l’onction sur lui.
${}^{89}Quand Moïse venait dans la tente de la Rencontre pour parler avec le Seigneur, il entendait la voix lui parler d’au-dessus du propitiatoire posé sur l’arche du Témoignage, entre les deux Kéroubim. Et il lui parlait.
      
         
      \bchapter{}
      \begin{verse}
${}^{1}Le Seigneur parla à Moïse. Il dit : 
${}^{2}« Parle à Aaron et dis-lui : Quand tu disposeras les lampes, c’est vers le devant du chandelier que les sept lampes devront éclairer. » 
${}^{3}Ainsi fit Aaron : il disposa les lampes vers le devant du chandelier, comme le Seigneur l’avait ordonné à Moïse.
${}^{4}Voici comment était fait le chandelier : il était forgé en or, forgé depuis la base jusqu’aux fleurs. C’est selon la vision que le Seigneur en avait donnée à Moïse qu’on avait fait le chandelier.
${}^{5}Le Seigneur parla à Moïse. Il dit : 
${}^{6}« Parmi les fils d’Israël, prends les Lévites et purifie-les. 
${}^{7}Voici ce que tu leur feras pour les purifier : Asperge-les avec l’eau qui enlève la faute ; puis ils se passeront tout le corps au rasoir, laveront leurs vêtements, et ils seront purs. 
${}^{8}Ensuite ils prendront un taureau et son offrande de fleur de farine pétrie à l’huile, tandis que tu prendras un second taureau destiné au sacrifice pour la faute. 
${}^{9}Puis tu feras approcher les Lévites devant la tente de la Rencontre et tu rassembleras toute la communauté des fils d’Israël. 
${}^{10}Tu feras approcher les Lévites devant le Seigneur, et les fils d’Israël leur imposeront les mains. 
${}^{11}Alors, de la part des fils d’Israël, Aaron les présentera avec le geste d’élévation devant le Seigneur, et ils seront voués au service du Seigneur. 
${}^{12}Puis les Lévites poseront leurs mains sur la tête des taureaux. Avec l’un, tu feras un sacrifice pour la faute, et avec l’autre un holocauste pour le Seigneur, en rite d’expiation sur les Lévites. 
${}^{13}Tu placeras les Lévites devant Aaron et ses fils, tu les présenteras avec le geste d’élévation au Seigneur. 
${}^{14}Parmi les fils d’Israël, tu mettras à part les Lévites, et les Lévites seront à moi. 
${}^{15}Après quoi les Lévites viendront faire le service de la tente de la Rencontre ; tu les auras purifiés, tu les auras présentés avec le geste d’élévation.
${}^{16}Oui, ils sont donnés, ils me sont donnés, parmi les fils d’Israël ; je les ai pris pour moi en échange de chaque fils aîné, de tout premier-né des fils d’Israël. 
${}^{17}Car c’est à moi qu’appartient, chez les fils d’Israël, tout premier-né, homme ou bétail : le jour où j’ai frappé tout premier-né au pays d’Égypte, c’est pour moi-même que je les ai sanctifiés 
${}^{18}et j’ai pris les Lévites en échange de tout premier-né parmi les fils d’Israël. 
${}^{19}Je donne les Lévites, ils sont donnés à Aaron et à ses fils, parmi les fils d’Israël, pour effectuer le service des fils d’Israël dans la tente de la Rencontre et pour accomplir le rite d’expiation sur les fils d’Israël. Ainsi, aucun fléau ne frappera les fils d’Israël quand ceux-ci approcheront du sanctuaire. »
${}^{20}Moïse, Aaron et toute la communauté des fils d’Israël firent pour les Lévites tout ce que le Seigneur avait ordonné à Moïse au sujet des Lévites. Ainsi firent les fils d’Israël. 
${}^{21}Les Lévites se purifièrent de leurs fautes et lavèrent leurs vêtements. Aaron les présenta avec le geste d’élévation devant le Seigneur, puis il accomplit sur eux le rite d’expiation pour les purifier. 
${}^{22}Après quoi, les Lévites vinrent effectuer leur service dans la tente de la Rencontre devant Aaron et ses fils. Comme le Seigneur l’avait ordonné à Moïse au sujet des Lévites, ainsi fut-il fait à leur égard.
${}^{23}Le Seigneur parla à Moïse. Il dit : 
${}^{24}« Ceci concerne les Lévites : depuis l’âge de vingt-cinq ans et au-delà, chacun sera destiné au service du culte pour remplir une tâche dans la tente de la Rencontre. 
${}^{25}À cinquante ans, il se retirera du service actif, il ne servira plus. 
${}^{26}Toutefois, il aidera ses frères à garder les observances dans la tente de la Rencontre. Mais il ne remplira plus de tâche cultuelle. Ainsi feras-tu pour les Lévites, en ce qui concerne leurs observances. »
      
         
      \bchapter{}
      \begin{verse}
${}^{1}Au désert du Sinaï, le Seigneur parla à Moïse la deuxième année après leur sortie d’Égypte, au cours du premier mois. Il dit : 
${}^{2}« Que les fils d’Israël célèbrent la Pâque à la date fixée ! 
${}^{3}Le quatorzième jour de ce mois, au coucher du soleil, vous la célébrerez à la date fixée. Vous la célébrerez selon tous les rituels et toutes les ordonnances qui la concernent. » 
${}^{4}Et Moïse dit aux fils d’Israël de célébrer la Pâque. 
${}^{5}Alors ils célébrèrent la Pâque, le quatorzième jour du premier mois, au coucher du soleil, dans le désert du Sinaï. Les fils d’Israël la célébrèrent conformément à tout ce que le Seigneur avait ordonné à Moïse.
${}^{6}Or, il y avait des hommes rendus impurs par le contact d’un mort : ils ne pouvaient donc pas célébrer la Pâque ce jour-là. Ils s’approchèrent de Moïse et d’Aaron ce jour-là 
${}^{7}et dirent : « Nous avons été rendus impurs par le contact d’un mort, mais pourquoi sommes-nous exclus et, de ce fait, privés d’apporter notre présent réservé au Seigneur, à la date fixée, au milieu des fils d’Israël ? » 
${}^{8}Moïse leur dit : « Restez là ! Je vais écouter ce que le Seigneur ordonne dans votre cas. » 
${}^{9}Et le Seigneur parla à Moïse. Il dit : 
${}^{10}« Parle aux fils d’Israël. Tu diras : Quiconque, parmi vous ou dans les générations futures, sera rendu impur par le contact d’un mort, ou se trouvera en voyage au loin, célébrera néanmoins la Pâque pour le Seigneur. 
${}^{11}C’est le deuxième mois qu’ils la célébreront, le quatorzième jour, au coucher du soleil. Ils mangeront l’agneau avec des pains sans levain et des herbes amères. 
${}^{12}Ils feront en sorte que rien n’en reste au matin, et ils ne briseront aucun de ses os ; ils célébreront la Pâque selon tout le rituel. 
${}^{13}Mais l’homme qui est pur et qui n’est pas en voyage, s’il néglige de célébrer la Pâque, cet individu sera retranché de sa parenté car il n’aura pas apporté son présent réservé au Seigneur, à la date fixée ; cet homme-là portera le poids de sa faute. 
${}^{14}Si un immigré résidant avec vous célèbre la Pâque pour le Seigneur, il la célébrera selon les rituels et l’ordonnance de celle-ci. Il y aura un seul rituel pour vous, tant pour l’immigré que pour l’Israélite originaire du pays. »
${}^{15}Le jour où l’on dressa la Demeure, la nuée couvrit la Demeure – c’est-à-dire la tente du Témoignage – et, le soir, il y eut sur la Demeure comme l’apparence d’un feu, et cela jusqu’au matin. 
${}^{16}Il en fut toujours ainsi : la nuée couvrait la Demeure et, la nuit, il y avait l’apparence d’un feu. 
${}^{17}Dès que la nuée montait au-dessus de la Tente, les fils d’Israël levaient le camp ; à l’endroit où elle s’arrêtait, les fils d’Israël campaient. 
${}^{18}Sur l’ordre du Seigneur, les fils d’Israël levaient le camp, et sur l’ordre du Seigneur, ils campaient ; tous les jours où la nuée demeurait sur la Demeure, ils campaient. 
${}^{19}Et quand la nuée s’attardait de nombreux jours sur la Demeure, les fils d’Israël observaient l’ordre du Seigneur : ils ne levaient pas le camp. 
${}^{20}Il arrivait que la nuée reste peu de jours sur la Demeure : sur l’ordre du Seigneur, ils campaient et, sur l’ordre du Seigneur, ils levaient le camp. 
${}^{21}Il arrivait que la nuée reste seulement du soir au matin : le matin, la nuée s’élevait et ils levaient le camp ; ou bien elle restait un jour et une nuit : la nuée s’élevait et ils levaient le camp. 
${}^{22}Que ce fût deux jours, un mois, ou plus encore, tant que la nuée s’attardait sur la Demeure – demeurait au-dessus d’elle –, les fils d’Israël campaient et ne levaient pas le camp ; mais dès qu’elle s’élevait, ils levaient le camp. 
${}^{23}C’est sur l’ordre du Seigneur qu’ils campaient, et sur l’ordre du Seigneur qu’ils levaient le camp : ils observaient les dispositions du Seigneur selon son ordre transmis par Moïse.
      
         
      \bchapter{}
      \begin{verse}
${}^{1}Le Seigneur parla à Moïse. Il dit : 
${}^{2} « Fais-toi deux trompettes d’argent, tu les feras en métal repoussé. Elles te serviront à convoquer la communauté et à donner le signal pour lever le camp. 
${}^{3} Lorsqu’elles sonneront toutes les deux\\, toute la communauté se rassemblera auprès de toi, à l’entrée de la tente de la Rencontre. 
${}^{4} Lorsqu’une seule trompette sonnera, les responsables, les chefs des clans\\d’Israël, se rassembleront auprès de toi. 
${}^{5} Lorsque vous accompagnerez d’ovations la sonnerie, ceux qui campent à l’est lèveront le camp. 
${}^{6} À la deuxième sonnerie accompagnée d’ovations, ceux qui campent au sud lèveront le camp. Pour lever le camp, on accompagnera d’ovations la sonnerie. 
${}^{7} Mais pour réunir l’assemblée, on n’accompagnera pas d’ovations la sonnerie. 
${}^{8} Les prêtres, fils d’Aaron, seront chargés de sonner\\de la trompette ; ce sera pour vous et vos descendants un décret perpétuel.
${}^{9}Lorsque, sur votre terre, vous marcherez au combat contre l’ennemi qui vous enserre, vous accompagnerez les cris de guerre de la sonnerie des trompettes. Ainsi il sera fait mémoire de vous devant le Seigneur votre Dieu, et vous serez sauvés de vos ennemis.
${}^{10}Lors de vos jours de fête, de vos solennités et à chaque nouvelle lune, vous sonnerez des trompettes pour accompagner vos holocaustes et vos sacrifices de paix, et elles seront pour vous un mémorial devant votre Dieu. Je suis le Seigneur votre Dieu ! »
${}^{11}La deuxième année après la sortie d’Égypte, le deuxième mois, le vingt du mois, la nuée s’éleva au-dessus de la Demeure du Témoignage. 
${}^{12}Alors, les fils d’Israël levèrent le camp pour partir du désert du Sinaï. La nuée se posa au désert de Parane. 
${}^{13}Pour la première fois, ils levèrent le camp, selon l’ordre du Seigneur transmis par Moïse : 
${}^{14}en premier lieu partit l’étendard du camp des fils de Juda, groupés en formation de combat ; Nahshone, fils d’Amminadab, commandait l’armée de Juda ; 
${}^{15}Netanel, fils de Souar, commandait l’armée de la tribu des fils d’Issakar ; 
${}^{16}Éliab, fils de Hélone, commandait l’armée de la tribu des fils de Zabulon.
${}^{17}La Demeure fut démontée, et alors les fils de Guershone et les fils de Merari, porteurs de la Demeure, levèrent le camp. 
${}^{18}Ensuite partit l’étendard du camp des fils de Roubène, groupés en formation de combat : Éliçour, fils de Shedéour, commandait l’armée de Roubène ; 
${}^{19}Sheloumiel, fils de Sourishaddaï, commandait l’armée de la tribu des fils de Siméon ; 
${}^{20}Éliasaf, fils de Deouël, commandait l’armée de la tribu des fils de Gad.
${}^{21}Alors les Qehatites, porteurs des objets du sanctuaire, levèrent le camp. On dressait la Demeure avant leur arrivée.
${}^{22}Ensuite partit l’étendard du camp des fils d’Éphraïm, groupés par formations de combat : Élishama, fils d’Ammihoud, commandait l’armée d’Éphraïm ; 
${}^{23}Gameliël, fils de Pedahçour, commandait l’armée de la tribu des fils de Manassé ; 
${}^{24}Abidane, fils de Guidéoni, commandait l’armée de la tribu des fils de Benjamin.
${}^{25}Ensuite, comme arrière-garde de tous les camps, partit l’étendard du camp des fils de Dane, groupés en formation de combat : Ahiézer, fils d’Ammishaddaï, commandait l’armée de Dane ; 
${}^{26}Paguiël, fils d’Okrane, commandait l’armée de la tribu des fils d’Asher ; 
${}^{27}Ahira, fils d’Einane, commandait l’armée de la tribu des fils de Nephtali.
${}^{28}Tel était l’ordre des départs des fils d’Israël, groupés en formation de combat. C’est ainsi qu’ils levèrent le camp.
${}^{29}Moïse dit à Hobab, fils de Réouël le Madianite, beau-père de Moïse : « Nous, nous partons vers le lieu dont le Seigneur nous a dit : “Je vous le donnerai !” Viens avec nous, pour que nous te rendions heureux, car le Seigneur a promis du bonheur pour Israël. » 
${}^{30}Mais il répondit : « Non, je n’irai pas. Je n’irai que dans mon pays, chez les miens. » 
${}^{31}Moïse reprit : « Ne nous abandonne pas, je t’en prie, car vraiment tu sais où nous pouvons camper dans le désert, tu seras pour nous comme des yeux. 
${}^{32}Si tu viens avec nous, ce bonheur dont nous fera bénéficier le Seigneur, nous t’en ferons bénéficier. »
${}^{33}Ils partirent donc de la montagne du Seigneur pour une marche de trois jours. Pendant cette marche de trois jours, l’arche de l’Alliance du Seigneur les précédait, afin de chercher pour eux un lieu de repos. 
${}^{34}Et la nuée du Seigneur les couvrait durant le jour, lorsqu’ils levaient le camp.
${}^{35}Au moment où l’Arche partait, Moïse disait :
        \\« Lève-toi, Seigneur, pour que tes ennemis se dispersent,
        et ceux qui te haïssent fuiront loin de ta face ! »
${}^{36}Et quand l’Arche s’arrêtait, il disait :
        \\« Reviens, Seigneur,
        vers la multitude innombrable d’Israël ! »
