  
  
      <h2 class="intertitle" id="d85e7373">1. Histoire d’Abraham (11,27 – 25,18)</h2>
${}^{27}Voici la descendance de Tèrah. Tèrah engendra Abram, Nahor et Harane. Harane engendra Loth. 
${}^{28}Harane mourut avant son père Tèrah dans le pays de sa parenté, à Our des Chaldéens. 
${}^{29}Abram et Nahor prirent femme ; l’épouse d’Abram s’appelait Saraï, et celle de Nahor, Milka, fille de Harane, père de Milka et de Yiska. 
${}^{30}Saraï était stérile, elle n’avait pas d’enfant.
${}^{31}Tèrah prit son fils Abram, son petit-fils Loth, fils de Harane, et sa bru Saraï, femme de son fils Abram, qui sortirent avec eux d’Our des Chaldéens pour aller au pays de Canaan. Ils gagnèrent Harane où ils s’établirent. 
${}^{32}Tèrah vécut deux cent cinq ans ; puis il mourut à Harane.
      
         
      \bchapter{}
      \begin{verse}
${}^{1}Le Seigneur dit à Abram :
      « Quitte ton pays,
        \\ta parenté et la maison de ton père,
        \\et va\\vers le pays que je te montrerai.
        ${}^{2}Je ferai de toi une grande nation, je te bénirai,
        \\je rendrai grand ton nom,
        \\et tu deviendras une bénédiction.
        ${}^{3}Je bénirai ceux qui te béniront ;
        \\celui qui te maudira, je le réprouverai.
        \\En toi seront bénies\\toutes les familles de la terre\\. »
       
${}^{4}Abram s’en alla, comme le Seigneur le lui avait dit, et Loth s’en alla avec lui. Abram avait soixante-quinze ans lorsqu’il sortit de Harane. 
${}^{5} Il prit sa femme Saraï\\, son neveu Loth, tous les biens qu’ils avaient acquis, et les personnes dont ils s’étaient entourés\\à Harane ; ils se mirent en route pour Canaan et ils arrivèrent dans ce pays.
${}^{6}Abram traversa le pays jusqu’au lieu nommé\\Sichem, au chêne de Moré. Les Cananéens étaient alors dans le pays. 
${}^{7} Le Seigneur apparut à Abram et dit : « À ta descendance je donnerai ce pays. » Et là, Abram bâtit un autel au Seigneur qui lui était apparu. 
${}^{8} De là, il se rendit dans la montagne, à l’est de Béthel, et il planta sa tente, ayant Béthel à l’ouest, et Aï à l’est. Là, il bâtit un autel au Seigneur et il invoqua le nom du Seigneur. 
${}^{9} Puis, de campement en campement, Abram s’en alla vers le Néguev.
${}^{10}Il y eut une famine dans le pays et Abram descendit en Égypte pour y séjourner car la famine accablait son pays.
${}^{11}Quand il fut sur le point d’entrer en Égypte, il dit à Saraï, sa femme : « Vois-tu, je le sais, toi, tu es une femme belle à regarder. 
${}^{12}Quand les Égyptiens te verront, ils diront : “C’est sa femme” et ils me tueront, tandis que toi, ils te laisseront vivre. 
${}^{13}S’il te plaît, dis que tu es ma sœur ; alors, à cause de toi ils me traiteront bien et, grâce à toi, je resterai en vie. » 
${}^{14}En effet, quand Abram arriva en Égypte, les Égyptiens virent la femme et la trouvèrent très belle. 
${}^{15}Les officiers de Pharaon la virent, chantèrent ses louanges à Pharaon et elle fut emmenée au palais. 
${}^{16}À cause d’elle, on traita bien Abram qui reçut petit et gros bétail, ânes, esclaves et servantes, ânesses et chameaux. 
${}^{17}Mais le Seigneur frappa de grandes plaies Pharaon et sa maison à cause de Saraï, la femme d’Abram. 
${}^{18}Pharaon convoqua Abram et lui dit : « Que m’as-tu fait là ! Pourquoi ne m’as-tu pas fait savoir qu’elle était ta femme ? 
${}^{19}Pourquoi as-tu dit : “C’est ma sœur” ? Aussi je l’ai prise pour femme. Maintenant, voici ta femme, prends-la et va-t’en ! » 
${}^{20}Pharaon donna ordre à ses gens de le renvoyer, lui, sa femme et tout ce qu’il possédait.
      
         
      \bchapter{}
      \begin{verse}
${}^{1}Abram remonta d’Égypte vers le Néguev, lui, sa femme et tout ce qu’il possédait. Loth l’accompagnait. 
${}^{2}Abram était extrêmement riche en troupeaux, en argent et en or. 
${}^{3}D’étape en étape, il alla du Néguev jusqu’à Béthel, jusqu’au lieu où il avait planté sa tente auparavant, entre Béthel et Aï. 
${}^{4}En ce lieu où naguère il avait fait un autel, en ce lieu même, il invoqua le nom du Seigneur. 
${}^{5}Loth, qui accompagnait Abram, avait également du petit et du gros bétail, et son propre campement. 
${}^{6}Le pays ne leur permettait pas d’habiter ensemble, car leurs biens étaient trop considérables pour qu’ils puissent habiter ensemble. 
${}^{7}Il y eut des disputes entre les bergers d’Abram et ceux de Loth. Les Cananéens et les Perizzites habitaient aussi le pays.
${}^{8}Abram dit à Loth : « Surtout, qu’il n’y ait pas de querelle entre toi et moi, entre tes bergers et les miens, car nous sommes frères ! 
${}^{9} N’as-tu pas tout le pays devant toi ? Sépare-toi donc de moi. Si tu vas à gauche, j’irai à droite, et si tu vas à droite, j’irai à gauche. » 
${}^{10} Loth leva les yeux et il vit que toute la région du Jourdain était bien irriguée. Avant que le Seigneur détruisît Sodome et Gomorrhe, elle était comme le jardin du Seigneur, comme le pays d’Égypte, quand on arrive au delta du Nil\\. 
${}^{11} Loth choisit pour lui toute la région du Jourdain et il partit vers l’est. C’est ainsi qu’ils se séparèrent\\. 
${}^{12} Abram habita dans le pays de Canaan, et Loth habita dans les villes de la région du Jourdain\\ ; il poussa ses campements jusqu’à Sodome. 
${}^{13} Les gens de Sodome se conduisaient mal, et ils péchaient gravement contre le Seigneur.
${}^{14}Après le départ de Loth\\, le Seigneur dit à Abram : « Lève les yeux et regarde, de l’endroit où tu es, vers le nord et le midi, vers l’orient et l’occident. 
${}^{15} Tout le pays que tu vois, je te le donnerai, à toi et à ta descendance, pour toujours. 
${}^{16} Je rendrai nombreuse ta descendance, autant que la poussière de la terre : si l’on pouvait compter les grains de poussière\\, on pourrait compter tes descendants ! 
${}^{17} Lève-toi ! Parcours le pays en long et en large : c’est à toi que je vais le donner. » 
${}^{18} Abram déplaça son campement et alla s’établir aux chênes de Mambré, près d’Hébron ; et là, il bâtit un autel au Seigneur.
      
         
      \bchapter{}
      \begin{verse}
${}^{1}Voici ce qui arriva au temps d’Amrafel, roi de Shinéar, d’Ariok, roi d’Ellasar, de Kedorlahomer, roi d’Élam, et de Tidéal, roi de Goïm. 
${}^{2}Ceux-ci firent la guerre à Béra, roi de Sodome, Birsha, roi de Gomorrhe, Shineab, roi d’Adma, Shémeéber, roi de Seboïm, et au roi de Bèla, c’est-à-dire Soar.
${}^{3}Ces derniers s’étaient tous rejoints dans la vallée de Siddim, c’est-à-dire la vallée de la mer Morte. 
${}^{4}Pendant douze ans, ils avaient servi Kedorlahomer, mais la treizième année, ils s’étaient révoltés. 
${}^{5}Au cours de la quatorzième année, arriva Kedorlahomer et les rois qui l’accompagnaient. Ils battirent les Refaïtes à Ashtaroth-Qarnayim, les Zouzim à Ham, les Émim à Shaveh-Quiriataïm 
${}^{6}et les Horites dans leur montagne de Séïr jusqu’au chêne de Parane qui est au bord du désert. 
${}^{7}Puis ils s’en retournèrent et arrivèrent à la source du Jugement, c’est-à-dire Cadès ; ils ravagèrent tout le territoire des Amalécites et battirent aussi les Amorites qui habitaient à Haceçone-Tamar. 
${}^{8}Sortirent alors le roi de Sodome, le roi de Gomorrhe, le roi d’Adma, le roi de Seboïm et le roi de Bèla, c’est-à-dire Soar. Ils se rangèrent en ordre de bataille dans la vallée de Siddim, 
${}^{9}face à Kedorlahomer, roi d’Élam, Tidéal, roi de Goïm, Amrafel, roi de Shinéar, Ariok, roi d’Ellasar : quatre rois contre cinq !
${}^{10}La vallée de Siddim était creusée de puits de bitume. Dans leur fuite, le roi de Sodome et le roi de Gomorrhe y tombèrent, et les autres s’enfuirent vers la montagne. 
${}^{11}Les ennemis prirent tous les biens de Sodome et de Gomorrhe, ainsi que tous leurs vivres, et ils s’en allèrent. 
${}^{12}Ils prirent aussi Loth et ses biens et s’en allèrent. Loth était le neveu d’Abram et il habitait Sodome.
${}^{13}Un fuyard vint informer Abram l’Hébreu de ces événements. Celui-ci demeurait aux chênes de Mambré l’Amorite, le frère d’Eshkol et d’Aner qui étaient des alliés d’Abram. 
${}^{14}Dès qu’Abram entendit que son frère avait été capturé, il mobilisa trois cent dix-huit hommes de guerre qui appartenaient à sa maison et mena la poursuite jusqu’à Dane. 
${}^{15}Durant la nuit, il se déploya contre ses ennemis, lui et ses serviteurs, il les battit et les poursuivit jusqu’à Hoba, au nord de Damas. 
${}^{16}Il ramena tous les biens, il ramena aussi son frère Loth et ses biens, ainsi que les femmes et tous les gens.
${}^{17}Le roi de Sodome s’avança vers la vallée de Shavé, c’est-à-dire la vallée du Roi, à la rencontre d’Abram. Celui-ci venait de battre Kedorlahomer et les rois qui l’accompagnaient.
${}^{18}Melkisédek, roi de Salem, fit apporter du pain et du vin : il était prêtre du Dieu très-haut. 
${}^{19} Il le bénit en disant :
        \\« Béni soit Abram par le Dieu très-haut,
        \\qui a créé le ciel et la terre ;
        ${}^{20}et béni soit le Dieu très-haut,
        \\qui a livré tes ennemis entre tes mains. »
      Et Abram lui donna le dixième de tout ce qu’il avait pris\\.
${}^{21}Le roi de Sodome dit à Abram : « Donne-moi les personnes et garde pour toi les biens. » 
${}^{22}Abram lui répondit : « J’ai levé la main vers le Seigneur, le Dieu très-haut qui a fait le ciel et la terre, 
${}^{23}et j’ai juré que je ne prendrais rien, pas même un fil, pas même une courroie de sandale, rien de tout ce qui t’appartient. Tu ne pourras pas dire : “C’est moi qui ai enrichi Abram.” 
${}^{24}Rien pour moi ! Seulement ce que les jeunes ont mangé et la part des hommes qui m’accompagnaient, Aner, Eshkol et Mambré. Qu’ils prennent eux-mêmes leur part ! »
      
         
      \bchapter{}
      \begin{verse}
${}^{1}Après ces événements, la parole du Seigneur fut adressée à Abram dans une vision : « Ne crains pas, Abram ! Je suis un bouclier pour toi. Ta récompense sera très grande. »
${}^{2}Abram répondit : « Mon Seigneur Dieu, que pourrais-tu donc me donner ? Je m’en vais sans enfant, et l’héritier\\de ma maison, c’est Élièzer de Damas. » 
${}^{3} Abram dit encore : « Tu ne m’as pas donné de descendance, et c’est un de mes serviteurs qui sera mon héritier. » 
${}^{4} Alors cette parole du Seigneur fut adressée à Abram : « Ce n’est pas lui qui sera ton héritier, mais quelqu’un de ton sang\\. » 
${}^{5} Puis il le fit sortir et lui dit : « Regarde le ciel, et compte les étoiles, si tu le peux… » Et il déclara : « Telle sera ta descendance ! » 
${}^{6} Abram eut foi dans le Seigneur et le Seigneur estima qu’il était juste.
${}^{7}Puis il dit : « Je suis le Seigneur, qui t’ai fait sortir d’Our en Chaldée pour te donner ce pays en héritage. » 
${}^{8}Abram répondit : « Seigneur mon Dieu, comment vais-je savoir que je l’ai en héritage ? » 
${}^{9}Le Seigneur lui dit : « Prends-moi une génisse de trois ans, une chèvre de trois ans, un bélier de trois ans, une tourterelle et une jeune colombe. » 
${}^{10}Abram prit tous ces animaux\\, les partagea en deux, et plaça chaque moitié en face de l’autre ; mais il ne partagea pas les oiseaux. 
${}^{11}Comme les rapaces descendaient sur les cadavres, Abram les chassa. 
${}^{12}Au coucher du soleil, un sommeil mystérieux\\tomba sur Abram, une sombre et profonde frayeur tomba sur lui. 
${}^{13}Dieu dit à Abram : « Sache-le bien : tes descendants seront des immigrés dans un pays qui ne leur appartient pas. On en fera des esclaves, on les opprimera pendant quatre cents ans. 
${}^{14}Mais la nation qu’ils auront servie, je la jugerai à son tour, et ils sortiront ensuite avec de grands biens. 
${}^{15}Quant à toi, tu rejoindras tes pères dans la paix. Tu seras enseveli après une heureuse vieillesse. 
${}^{16}Tes descendants ne reviendront ici qu’à la quatrième génération, car alors seulement, la faute des Amorites aura atteint son comble. » 
${}^{17}Après le coucher du soleil, il y eut des ténèbres épaisses. Alors un brasier fumant et une torche enflammée passèrent entre les morceaux d’animaux\\.
${}^{18}Ce jour-là, le Seigneur conclut une alliance avec Abram en ces termes :
        \\« À ta descendance
        \\je donne le pays que voici,
      depuis le Torrent d’Égypte jusqu’au Grand Fleuve, l’Euphrate, soit le pays des Qénites, des Qenizzites, des Qadmonites, des Hittites, des Perizzites, des Refaïtes, des Amorites, des Cananéens, des Guirgashites et des Jébuséens. »
      
         
      \bchapter{}
      \begin{verse}
${}^{1}Saraï, la femme d’Abram, ne lui avait pas donné d’enfant. Elle avait une servante égyptienne, nommée Agar, 
${}^{2} et elle dit à Abram : « Écoute-moi\\ : le Seigneur ne m’a pas permis d’avoir un enfant. Va donc vers ma servante ; grâce à elle, peut-être aurai-je un fils\\. » Abram écouta\\Saraï. 
${}^{3} Et donc dix ans après qu’Abram se fut établi au pays de Canaan, Saraï, femme d’Abram, prit Agar l’Égyptienne, sa servante, et la donna pour femme à son mari Abram. 
${}^{4} Celui-ci alla vers Agar, et elle devint enceinte. Quand elle se vit enceinte, sa maîtresse ne compta plus à ses yeux. 
${}^{5} Saraï dit à Abram : « Que la violence qui m’est faite retombe sur toi ! C’est moi qui ai mis ma servante dans tes bras, et, depuis qu’elle s’est vue enceinte, je ne compte plus à ses yeux. Que le Seigneur soit juge entre moi et toi ! » 
${}^{6} Abram lui répondit : « Ta servante est entre tes mains, fais-lui ce que bon te semble\\. » Saraï humilia Agar et celle-ci prit la fuite\\.
${}^{7}L’ange du Seigneur la trouva dans le désert, près d’une source, celle qui est sur la route de Shour. 
${}^{8} L’ange lui dit : « Agar, servante de Saraï, d’où viens-tu et où vas-tu ? » Elle répondit : « Je fuis ma maîtresse Saraï. » 
${}^{9} L’ange du Seigneur lui dit : « Retourne chez ta maîtresse, et humilie-toi sous sa main\\. »
${}^{10}L’ange du Seigneur lui dit : « Je te donnerai une descendance tellement nombreuse qu’il sera impossible de la compter. »
${}^{11}L’ange du Seigneur lui dit :
        \\« Tu es enceinte, tu vas enfanter un fils,
        \\et tu lui donneras le nom d’Ismaël (c’est-à-dire : Dieu entend)\\,
        \\car le Seigneur t’a entendue dans ton humiliation.
        ${}^{12}Cet homme sera comme l’âne sauvage :
        \\sa main se dressera\\contre tous,
        \\et la main de tous contre lui ;
        \\il établira sa demeure face à tous ses frères. »
       
${}^{13}Au Seigneur qui lui parlait, Agar donna ce nom : « Tu es El-Roï (c’est-à-dire : le-Dieu-qui-me-voit) », car elle se demandait : « Ai-je bien vu ici, de dos, celui qui me voit ? » 
${}^{14}C’est pourquoi on appela ce puits : Lahaï-Roï (c’est-à-dire : le-Vivant-qui-me-voit). Il se trouve entre Cadès et Béred.
${}^{15}Agar enfanta un fils à Abram, qui lui\\donna le nom d’Ismaël. 
${}^{16} Abram avait quatre-vingt-six ans quand Agar lui enfanta Ismaël.
      
         
      \bchapter{}
      \begin{verse}
${}^{1}Lorsque Abram eut atteint quatre-vingt-dix-neuf ans, le Seigneur lui apparut et lui dit : « Je suis le Dieu-Puissant\\ ; marche en ma présence et sois parfait. 
${}^{2} J’établirai mon alliance entre moi et toi, et je multiplierai ta descendance\\à l’infini\\. »
${}^{3}Abram tomba face contre terre\\et Dieu lui parla ainsi : 
${}^{4} « Moi, voici l’alliance que je fais avec toi : tu deviendras le père d’une multitude de nations. 
${}^{5} Tu ne seras plus appelé du nom d’Abram, ton nom sera Abraham\\, car je fais de toi le père d’une multitude de nations. 
${}^{6} Je te ferai porter des fruits à l’infini\\, de toi je ferai des nations, et des rois sortiront de toi. 
${}^{7} J’établirai mon alliance entre moi et toi, et après toi avec ta descendance, de génération en génération ; ce sera une alliance éternelle ; ainsi je serai ton Dieu et le Dieu de ta descendance après toi. 
${}^{8} À toi et à ta descendance après toi je donnerai le pays où tu résides, tout le pays de Canaan en propriété perpétuelle, et je serai leur Dieu. »
${}^{9}Dieu dit à Abraham : « Toi, tu observeras mon alliance, toi et ta descendance après toi, de génération en génération. 
${}^{10}Et voici l’alliance qui sera observée entre moi et vous, c’est-à-dire toi et ta descendance après toi : tous vos enfants mâles seront circoncis. 
${}^{11}La chair de votre prépuce sera circoncise, et cela deviendra le signe de l’alliance entre moi et vous. 
${}^{12}À chaque génération, tous vos enfants mâles âgés de huit jours seront circoncis, les enfants nés dans la maison, ou les enfants étrangers qui ne sont pas de ta descendance mais sont acquis à prix d’argent. 
${}^{13}Né dans la maison ou acquis à prix d’argent, tout mâle sera circoncis. Inscrite dans votre chair, mon alliance deviendra une alliance éternelle. 
${}^{14}L’incirconcis, le mâle dont la chair du prépuce n’aura pas été circoncise, celui-là sera retranché d’entre les siens : il aura rompu mon alliance. »
${}^{15}Dieu dit encore à Abraham : « Saraï, ta femme, tu ne l’appelleras plus du nom de Saraï ; désormais son nom est Sara (c’est-à-dire : Princesse)\\. 
${}^{16} Je la bénirai : d’elle aussi je te donnerai un fils ; oui, je la bénirai, elle sera à l’origine de nations\\, d’elle proviendront les rois de plusieurs peuples. » 
${}^{17} Abraham tomba face contre terre\\. Il se mit à rire car il se disait : « Un homme de cent ans va-t-il avoir un fils, et Sara va-t-elle enfanter à quatre-vingt-dix ans ? » 
${}^{18} Et il dit à Dieu : « Accorde-moi seulement qu’Ismaël vive sous ton regard ! » 
${}^{19} Mais Dieu reprit : « Oui, vraiment, ta femme Sara va t’enfanter un fils, tu lui donneras le nom d’Isaac. J’établirai mon alliance avec lui, comme une alliance éternelle avec sa descendance après lui. 
${}^{20} Au sujet d’Ismaël, je t’ai bien entendu : je le bénis, je le ferai fructifier et se multiplier à l’infini\\ ; il engendrera douze princes, et je ferai de lui une grande nation. 
${}^{21} Quant à mon alliance, c’est avec Isaac que je l’établirai, avec l’enfant que Sara va te donner l’an prochain à pareille époque. »
${}^{22}Lorsque Dieu eut fini de parler avec Abraham, il s’éleva loin de lui\\.
${}^{23}Abraham prit son fils Ismaël, et tout mâle né dans sa maison ou acquis à prix d’argent ; il circoncit la chair de leur prépuce, en ce jour même, comme Dieu le lui avait dit. 
${}^{24}Abraham avait quatre-vingt-dix-neuf ans quand fut circoncise la chair de son prépuce, 
${}^{25}et Ismaël avait treize ans quand fut circoncise la chair de son prépuce. 
${}^{26}En ce jour même, Abraham et son fils Ismaël furent circoncis. 
${}^{27}Tous les hommes de sa maison, nés dans la maison ou acquis d’un étranger à prix d’argent, furent circoncis avec lui.
      
         
      \bchapter{}
      \begin{verse}
${}^{1}Aux chênes de Mambré, le Seigneur apparut à Abraham, qui était assis à l’entrée de la tente. C’était l’heure la plus chaude du jour. 
${}^{2} Abraham leva les yeux, et il vit trois hommes qui se tenaient debout près de lui. Dès qu’il les vit, il courut à leur rencontre depuis l’entrée de la tente et se prosterna jusqu’à terre. 
${}^{3} Il dit : « Mon seigneur, si j’ai pu trouver grâce à tes yeux, ne passe pas sans t’arrêter près de ton serviteur. 
${}^{4} Permettez que l’on vous apporte un peu d’eau, vous vous laverez les pieds, et vous vous étendrez sous cet arbre. 
${}^{5} Je vais chercher de quoi manger\\, et vous reprendrez des forces avant d’aller plus loin, puisque vous êtes passés près de votre serviteur ! » Ils répondirent : « Fais comme tu l’as dit. »
${}^{6}Abraham se hâta d’aller trouver Sara dans sa tente, et il dit : « Prends\\vite trois grandes mesures de fleur de farine, pétris la pâte et fais des galettes. » 
${}^{7} Puis Abraham courut au troupeau, il prit un veau gras\\et tendre, et le donna à un serviteur, qui se hâta de le préparer. 
${}^{8} Il prit du fromage blanc, du lait, le veau que l’on avait apprêté, et les déposa devant eux ; il se tenait debout près d’eux, sous l’arbre, pendant qu’ils mangeaient.
${}^{9}Ils lui demandèrent : « Où est Sara, ta femme ? » Il répondit : « Elle est à l’intérieur de la tente. » 
${}^{10}Le voyageur\\reprit : « Je reviendrai chez toi au temps fixé pour la naissance\\, et à ce moment-là, Sara, ta femme, aura un fils. » Or, Sara écoutait par-derrière, à l’entrée de la tente. 
${}^{11}– Abraham et Sara étaient très avancés en âge, et Sara avait cessé d’avoir ce qui arrive\\aux femmes. 
${}^{12}Elle se mit à rire en elle-même ; elle se disait : « J’ai pourtant passé l’âge du plaisir, et mon seigneur est un vieillard ! » 
${}^{13}Le Seigneur Dieu\\dit à Abraham : « Pourquoi Sara a-t-elle ri, en disant : “Est-ce que vraiment j’aurais un enfant, vieille comme je suis ?” 
${}^{14}Y a-t-il une merveille\\que le Seigneur ne puisse accomplir ? Au moment où je reviendrai chez toi, au temps fixé pour la naissance, Sara aura un fils. » 
${}^{15}Sara mentit en disant : « Je n’ai pas ri », car elle avait peur. Mais le Seigneur\\répliqua : « Si, tu as ri. »
${}^{16}Les hommes se levèrent pour partir\\et regardèrent du côté de Sodome. Abraham marchait avec eux pour les reconduire. 
${}^{17} Le Seigneur s’était dit : « Est-ce que je vais cacher à Abraham ce que je veux faire ? 
${}^{18} Car Abraham doit devenir une nation grande et puissante, et toutes les nations de la terre doivent être bénies en lui. 
${}^{19} En effet, je l’ai choisi\\pour qu’il ordonne à ses fils et à sa descendance\\de garder le chemin du Seigneur, en pratiquant la justice et le droit ; ainsi, le Seigneur réalisera sa parole à Abraham. »
${}^{20}Alors le Seigneur dit : « Comme elle est grande, la clameur au sujet de Sodome et de Gomorrhe ! Et leur faute, comme elle est lourde ! 
${}^{21} Je veux descendre pour voir si leur conduite correspond à la clameur venue jusqu’à moi. Si c’est faux, je le reconnaîtrai. »
${}^{22}Les hommes se dirigèrent vers Sodome, tandis qu’Abraham demeurait devant le Seigneur. 
${}^{23} Abraham s’approcha et dit : « Vas-tu vraiment faire périr le juste avec le coupable ? 
${}^{24} Peut-être y a-t-il cinquante justes dans la ville. Vas-tu vraiment les faire périr ? Ne pardonneras-tu pas à toute la ville\\à cause des cinquante justes qui s’y trouvent ? 
${}^{25} Loin de toi de faire une chose pareille ! Faire mourir le juste avec le coupable, traiter le juste de la même manière que le coupable, loin de toi d’agir ainsi\\ ! Celui qui juge toute la terre n’agirait-il pas selon le droit ? » 
${}^{26} Le Seigneur déclara : « Si je trouve cinquante justes dans Sodome\\, à cause d’eux je pardonnerai à toute la ville\\. »
${}^{27}Abraham répondit : « J’ose encore parler à mon Seigneur, moi qui suis poussière et cendre. 
${}^{28}Peut-être, sur les cinquante justes, en manquera-t-il cinq : pour ces cinq-là, vas-tu détruire toute la ville ? » Il déclara : « Non, je ne la détruirai pas, si j’en trouve quarante-cinq. »
${}^{29}Abraham insista : « Peut-être s’en trouvera-t-il seulement quarante ? » Le Seigneur déclara : « Pour quarante, je ne le ferai pas. »
${}^{30}Abraham dit : « Que mon Seigneur ne se mette pas en colère, si j’ose parler encore. Peut-être s’en trouvera-t-il seulement trente ? » Il déclara : « Si j’en trouve trente, je ne le ferai pas. »
${}^{31}Abraham dit alors : « J’ose encore parler à mon Seigneur. Peut-être s’en trouvera-t-il seulement vingt ? » Il déclara : « Pour vingt, je ne détruirai pas. »
${}^{32}Il dit : « Que mon Seigneur ne se mette pas en colère : je ne parlerai plus qu’une fois. Peut-être s’en trouvera-t-il seulement dix ? » Et le Seigneur déclara : « Pour dix, je ne détruirai pas. »
${}^{33}Quand le Seigneur eut fini de s’entretenir avec Abraham, il partit, et Abraham retourna chez lui.
      
         
      \bchapter{}
      \begin{verse}
${}^{1}Les deux anges arrivèrent à Sodome, le soir. Loth était assis à la porte de Sodome ; il les aperçut, se leva pour aller à leur rencontre et se prosterna, face contre terre. 
${}^{2}Il dit : « De grâce, mes seigneurs, faites un détour par la maison de votre serviteur ; vous y passerez la nuit, vous vous laverez les pieds et vous vous lèverez de bon matin pour reprendre votre route. » Ils répondirent : « Non ! nous passerons la nuit sur la place. » 
${}^{3}Mais il insista tellement auprès d’eux qu’ils firent le détour et entrèrent dans sa maison. Il leur prépara un festin, fit cuire des pains sans levain, et ils mangèrent.
${}^{4}Ils n’étaient pas encore couchés que les hommes de la ville, ceux de Sodome, cernèrent la maison, des plus jeunes aux plus vieux, toute la population sans exception. 
${}^{5}Ils appelèrent Loth et lui dirent : « Où sont les hommes qui sont venus chez toi cette nuit ? Amène-les : nous voulons nous unir à eux. »
${}^{6}Loth s’avança vers eux, à l’entrée, et ferma la porte derrière lui. 
${}^{7}Il dit : « De grâce, mes frères, ne commettez pas le mal ! 
${}^{8}Voici, j’ai deux filles qui ne se sont unies à aucun homme. Je vais vous les amener, et vous leur ferez ce que bon vous semblera. Mais à ces hommes ne faites rien : ils sont venus s’abriter sous mon toit. » 
${}^{9}Ils répliquèrent : « Ôte-toi de là ! » Et ils ajoutèrent : « Lui, le seul étranger, il voudrait juger ! À toi, nous ferons plus de mal qu’à eux ! » Ils bousculèrent Loth et s’approchèrent pour enfoncer la porte. 
${}^{10}Mais les deux hommes étendirent la main et firent rentrer Loth dans la maison, auprès d’eux. Et ils refermèrent la porte. 
${}^{11}Ils frappèrent de cécité les hommes qui se trouvaient à l’extérieur de la maison, du plus petit au plus grand, si bien que ceux-ci ne purent trouver l’entrée.
${}^{12}Les deux hommes dirent à Loth : « Qui as-tu encore ici avec toi ? Gendre, fils, filles, tous ceux qui sont avec toi dans la ville, fais-les sortir de ce lieu. 
${}^{13}Car nous allons le détruire. Elle est grande à la face du Seigneur, la clameur qui s’est élevée contre ses habitants, et le Seigneur nous a envoyés pour détruire ce lieu. » 
${}^{14}Loth sortit parler à ses gendres, ceux qui allaient épouser ses filles, et dit : « Debout ! Sortez de ce lieu car le Seigneur va détruire la ville ! » Mais, aux yeux de ses gendres, il parut plaisanter.
${}^{15}À l’aurore, les deux anges pressèrent Loth, en disant : « Debout ! Prends ta femme et tes deux filles qui se trouvent ici, et va-t’en\\, de peur que tu ne périsses à cause des crimes de cette ville. » 
${}^{16}Comme il s’attardait, ces hommes le saisirent par la main, ainsi que sa femme et ses deux filles, parce que le Seigneur voulait l’épargner. Ils le firent sortir et le conduisirent hors de la ville.
${}^{17}Une fois sortis\\, ils dirent\\ : « Sauve-toi si tu tiens à la vie ! Ne regarde pas en arrière, ne t’arrête nulle part dans cette région, sauve-toi dans la montagne, si tu ne veux pas périr ! » 
${}^{18} Loth leur dit : « Non, je vous en prie, mes seigneurs ! 
${}^{19} Votre serviteur\\a trouvé grâce à vos yeux, et vous m’avez fait une grande faveur en me laissant la vie. Mais je n’ai pas le temps de\\me sauver dans la montagne : le malheur va me rattraper et je mourrai. 
${}^{20} Voici une ville assez proche pour y fuir – elle est si petite ! – Permettez que je me sauve là-bas – elle est si petite ! – afin de rester en vie ! » 
${}^{21} Ils lui répondirent : « Pour te faire plaisir cette fois encore, je ne détruirai pas la ville dont tu parles. 
${}^{22} Vite, sauve-toi là-bas, car je ne puis rien faire avant que tu y sois arrivé. » C’est pour cela qu’on a donné à cette ville le nom de Soar (ce qui veut dire : Petite)\\.
${}^{23}Le soleil se levait sur le pays et Loth entrait à Soar, 
${}^{24} quand le Seigneur fit tomber du ciel sur Sodome et Gomorrhe une pluie de soufre et de feu venant du Seigneur. 
${}^{25} Dieu détruisit ces villes et toute la région, avec tous leurs habitants et la végétation. 
${}^{26} Or, la femme de Loth avait regardé en arrière, et elle était devenue une colonne de sel.
${}^{27}Abraham se leva de bon matin pour se rendre à l’endroit où il s’était tenu en présence du Seigneur, 
${}^{28} et il regarda du côté de Sodome, de Gomorrhe et de toute la région\\ : il vit monter de la terre une fumée semblable à celle d’une fournaise !
${}^{29}Lorsque Dieu a détruit les villes de cette région, il s’est souvenu d’Abraham ; et il a fait échapper Loth au cataclysme qui a détruit les villes où il habitait.
${}^{30}Loth monta de Soar pour habiter dans la montagne avec ses deux filles. Il craignait d’habiter Soar et il vécut dans une caverne avec ses deux filles.
${}^{31}L’aînée dit à la cadette : « Notre père est vieux, et il n’y a pas d’homme dans le pays pour venir à nous, comme cela se fait partout. 
${}^{32}Allons ! Faisons boire du vin à notre père et couchons avec lui ; ainsi, grâce à lui, nous donnerons la vie à une descendance. » 
${}^{33}Elles firent boire du vin à leur père cette nuit-là, et l’aînée alla coucher avec son père qui ne s’aperçut de rien, ni de son coucher ni de son lever.
${}^{34}Le lendemain, l’aînée dit à la cadette : « Voici ! Hier soir, j’ai couché avec mon père. Faisons-lui boire du vin, cette nuit encore. Et toi, tu iras coucher avec lui. Ainsi, nous donnerons la vie à une descendance issue de notre père. » 
${}^{35}Cette nuit encore, elles firent boire du vin à leur père. La cadette se leva et alla coucher avec lui ; il ne s’aperçut de rien, ni de son coucher ni de son lever. 
${}^{36}Les deux filles de Loth devinrent enceintes de leur père. 
${}^{37}L’aînée donna naissance à un fils qu’elle appela du nom de Moab ; c’est le père des Moabites d’aujourd’hui. 
${}^{38}La cadette, elle aussi, donna naissance à un fils qu’elle appela du nom de Ben-Ammi ; c’est le père des Ammonites d’aujourd’hui.
