\bchapter{Psaume}
            Le chemin des justes
${}^{1}Heure\underline{u}x est l’homme
        qui n’entre pas au cons\underline{e}il des méchants, +
        \\qui ne suit pas le chem\underline{i}n des pécheurs, *
        \\ne siège pas avec ce\underline{u}x qui ricanent,
${}^{2}mais se plaît dans la l\underline{o}i du Seigneur
        \\et murmure sa l\underline{o}i jour et nuit !
         
${}^{3}Il \underline{e}st comme un arbre
        plant\underline{é} près d’un ruisseau, +
        \\qui donne du fr\underline{ui}t en son temps, *
        \\et jamais son feuill\underline{a}ge ne meurt ;
        \\tout ce qu’il entrepr\underline{e}nd réussira,
${}^{4}tel n’est pas le s\underline{o}rt des méchants.
         
        \\Mais ils s\underline{o}nt comme la paille
        balay\underline{é}e par le vent : +
${}^{5}au jugement, les méchants ne se l\underline{è}veront pas, *
        \\ni les pécheurs au rassemblem\underline{e}nt des justes.
${}^{6}Le Seigneur connaît le chem\underline{i}n des justes,
        \\mais le chemin des méch\underline{a}nts se perdra.
      \bchapter{Psaume}
          
\bchapter{Psaume}
            « Tu es mon fils »
${}^{1}Pourquoi ce tum\underline{u}lte des nations,
        \\ce vain murm\underline{u}re des peuples ?
${}^{2}Les rois de la t\underline{e}rre se dressent,
        \\les grands se liguent entre eux
        contre le Seigne\underline{u}r et son messie :
${}^{3}« Faisons saut\underline{e}r nos chaînes,
        \\rejet\underline{o}ns ces entraves ! »
         
${}^{4}Celui qui règne dans les cie\underline{u}x s’en amuse,
        \\le Seigneur les to\underline{u}rne en dérision ;
${}^{5}puis il leur p\underline{a}rle avec fureur
        \\et sa col\underline{è}re les épouvante :
${}^{6}« Moi, j’ai sacr\underline{é} mon roi
        \\sur Sion, ma s\underline{a}inte mont\underline{a}gne. »
         
${}^{7}Je proclame le décr\underline{e}t du Seigneur ! +
         
        \\Il m’a d\underline{i}t : « Tu es mon fils ;
        \\moi, aujourd’hu\underline{i}, je t’ai engendré.
${}^{8}Demande, et je te donne en hérit\underline{a}ge les nations,
        \\pour domaine la t\underline{e}rre tout entière.
${}^{9}Tu les détruiras de ton sc\underline{e}ptre de fer,
        \\tu les briseras comme un v\underline{a}se de potier. »
         
${}^{10}Maintenant, r\underline{o}is, comprenez,
        \\reprenez-vous, j\underline{u}ges de la terre.
${}^{11}Servez le Seigne\underline{u}r avec crainte,
        \\rendez-lui votre homm\underline{a}ge en tremblant.
${}^{12}Qu’il s’irrite et vous \underline{ê}tes perdus :
        \\soudain sa col\underline{è}re éclatera.
         
        \\Heureux qui trouve en lu\underline{i} son refuge !
      \bchapter{Psaume}
          
\bchapter{Psaume}
            Du Seigneur vient le salut
${}^{1}Psaume. De David. Quand il fuyait devant son fils Absalom.
         
${}^{2}Seigneur, qu’ils sont nombre\underline{u}x mes adversaires,
        \\nombreux à se lev\underline{e}r contre moi,
${}^{3}nombreux à déclar\underline{e}r à mon sujet :
        \\« Pour lui, pas de sal\underline{u}t auprès de Dieu ! »
         
${}^{4}Mais toi, Seigne\underline{u}r, mon bouclier,
        \\ma gloire, tu tiens ha\underline{u}te ma tête.
${}^{5}À pleine voix je cr\underline{i}e vers le Seigneur ;
        \\il me répond de sa mont\underline{a}gne sainte.
         
${}^{6}Et moi, je me co\underline{u}che et je dors ;
        \\je m’éveille : le Seigne\underline{u}r est mon soutien.
${}^{7}Je ne crains pas ce pe\underline{u}ple nombreux
        \\qui me cerne et s’av\underline{a}nce contre moi.
         
${}^{8}Lève-t\underline{o}i, Seigneur !
        \\Sauve-m\underline{o}i, mon Dieu !
        \\Tous mes ennemis, tu les fr\underline{a}ppes à la mâchoire ;
        \\les méchants, tu leur br\underline{i}ses les dents.
         
${}^{9}Du Seigneur vi\underline{e}nt le salut ;
        \\vienne ta bénédicti\underline{o}n sur ton peuple !
      \bchapter{Psaume}
          
\bchapter{Psaume}
            Tu me donnes d’habiter dans la confiance
${}^{1}Du maître de chœur. Avec instruments à corde. Psaume.
        \\De David.
         
${}^{2}Quand je cr\underline{i}e, réponds-moi,
        \\Die\underline{u}, ma justice !
         
        \\Toi qui me lib\underline{è}res dans la détresse,
        \\pitié pour moi, éco\underline{u}te ma prière !
         
${}^{3}Fils des hommes,
        jusqu’où irez-vous dans l’ins\underline{u}lte à ma gloire, *
        \\l’amour du néant et la co\underline{u}rse au mensonge ?
         
${}^{4}Sachez que le Seigneur a mis à p\underline{a}rt son fidèle,
        \\le Seigneur entend quand je cr\underline{i}e vers lui.
         
${}^{5}Mais vous, trembl\underline{e}z, ne péchez pas ;
        \\réfléchissez dans le secret, f\underline{a}ites silence.
         
${}^{6}Offrez les offr\underline{a}ndes justes
        \\et faites confi\underline{a}nce au Seigneur.
         
${}^{7}Beaucoup demandent :
        « Qui nous fera v\underline{o}ir le bonheur ? » *
        \\Sur nous, Seigneur, que s’illum\underline{i}ne ton visage !
         
${}^{8}Tu mets dans mon cœ\underline{u}r plus de joie
        \\que toutes leurs vend\underline{a}nges et leurs moissons.
         
${}^{9}Dans la paix moi aussi, je me co\underline{u}che et je dors, *
        \\car tu me donnes d’habiter, Seigneur,
        se\underline{u}l, dans la confiance.
      \bchapter{Psaume}
          
\bchapter{Psaume}
            Au matin, tu écoutes ma voix
${}^{1}Du maître de chœur. Sur les flûtes. Psaume. De David.
         
${}^{2}Écoute mes par\underline{o}les, Seigneur,
        compr\underline{e}nds ma plainte ; *
${}^{3}entends ma v\underline{o}ix qui t’appelle,
        ô mon R\underline{o}i et mon Dieu !
         
${}^{4}Je me tourne vers t\underline{o}i, Seigneur,
        au matin, tu éco\underline{u}tes ma voix ; *
        \\au matin, je me prép\underline{a}re pour toi
        et je r\underline{e}ste en éveil.
         
${}^{5}Tu n’es pas un Die\underline{u} ami du mal,
        chez toi, le méch\underline{a}nt n’est pas reçu. *
${}^{6}Non, l’insens\underline{é} ne tient pas
        dev\underline{a}nt ton regard.
         
        \\Tu détestes to\underline{u}s les malfaisants,
${}^{7}tu exterm\underline{i}nes les menteurs ; *
        \\l’homme de r\underline{u}se et de sang,
        le Seigne\underline{u}r le hait.
         
${}^{8}Pour moi, gr\underline{â}ce à ton amour,
        j’acc\underline{è}de à ta maison ; *
        \\vers ton temple saint, je me prosterne,
        sais\underline{i} de crainte.
         
${}^{9}Seigneur, que ta just\underline{i}ce me conduise ; *
        \\des ennem\underline{i}s me guettent :
        aplanis devant moi ton chemin.
         
${}^{10}Rien n’est vrai dans leur bouche,
        ils sont rempl\underline{i}s de malveillance ; *
        \\leur gosier est un sép\underline{u}lcre béant,
        et leur l\underline{a}ngue, un piège.
         
${}^{11}\[Dieu, traite-l\underline{e}s en coupables :
        qu’ils écho\underline{u}ent dans leurs projets ! *
        \\Pour tant de méf\underline{a}its, disperse-les,
        p\underline{u}isqu’ils te résistent.\]
         
${}^{12}Allégresse pour qui s’abr\underline{i}te en toi,
        j\underline{o}ie éternelle ! *
        \\Tu les protèges, pour t\underline{o}i ils exultent,
        ceux qui \underline{a}iment ton nom.
         
${}^{13}Toi, Seigneur, tu bén\underline{i}s le juste ;
        \\du bouclier de ta fave\underline{u}r, tu le couvres.
      \bchapter{Psaume}
          
\bchapter{Psaume}
            Seigneur, guéris-moi
${}^{1}Du maître de chœur. Avec instruments à corde. À l’octave. Psaume. De David.
         
${}^{2}Seigneur, corrige-m\underline{o}i sans colère,
        \\et reprends-m\underline{o}i sans fureur.
${}^{3}Pitié, Seigne\underline{u}r, je dépéris !
        \\Seigne\underline{u}r, guéris-moi !
        \\Car je tremble de to\underline{u}s mes os,
${}^{4}de toute mon \underline{â}me, je tremble.
         
        \\Et toi, Seigne\underline{u}r, que fais-tu ? +
${}^{5}Reviens, Seigne\underline{u}r, délivre-moi,
        \\sauve-moi en rais\underline{o}n de ton amour !
${}^{6}Personne, dans la mort, n’inv\underline{o}que ton nom ;
        \\au séjour des morts, qu\underline{i} te rend grâce ?
         
${}^{7}Je m’épuise à f\underline{o}rce de gémir ; +
        \\chaque nuit, je ple\underline{u}re sur mon lit :
        \\ma couche est tremp\underline{é}e de mes larmes.
${}^{8}Mes yeux sont rong\underline{é}s de chagrin ;
        \\j’ai vieilli parmi t\underline{a}nt d’adversaires !
         
${}^{9}Loin de moi, vous to\underline{u}s, malfaisants,
        \\car le Seigneur ent\underline{e}nd mes sanglots !
${}^{10}Le Seigneur accu\underline{e}ille ma demande,
        \\le Seigneur ent\underline{e}nd ma prière.
${}^{11}Qu’ils aient honte et qu’ils tremblent, to\underline{u}s mes ennemis,
        \\qu’ils reculent, soud\underline{a}in, couverts de honte !
      \bchapter{Psaume}
          
\bchapter{Psaume}
            Toi qui scrutes les cœurs et les reins
${}^{1}Complainte. De David. Qu’il chanta au Seigneur
        \\à propos de Koush le Benjaminite.
         
${}^{2}Seigneur mon Dieu, tu \underline{e}s mon refuge !
        \\On me poursuit : sauve-m\underline{o}i, délivre-moi !
${}^{3}Sinon ils vont m’égorg\underline{e}r, tous ces fauves,
        \\me déchirer, sans que pers\underline{o}nne me délivre.
         
${}^{4}Seigneur mon Dieu, si j’\underline{a}i fait cela,
        \\si j’ai vraiment un cr\underline{i}me sur les mains,
${}^{5}si j’ai causé du t\underline{o}rt à mon allié
        \\en épargn\underline{a}nt son adversaire,
${}^{6}que l’ennemi me poursu\underline{i}ve, qu’il m’atteigne *
        \\(qu’il foule au sol ma vie)
        \\et livre ma gl\underline{o}ire à la poussière.
         
        *
         
${}^{7}Dans ta colère, Seigne\underline{u}r, lève-toi, +
        \\domine mes advers\underline{a}ires en furie,
        \\réveille-toi pour me défendre et prononc\underline{e}r ta sentence.
${}^{8}Une assemblée de pe\underline{u}ples t’environne : +
        \\reprends ta pl\underline{a}ce au-dessus d’elle,
${}^{9}Seigneur qui arb\underline{i}tres les nations.
         
        \\Juge-moi, Seigne\underline{u}r, sur ma justice :
        \\mon innocence p\underline{a}rle pour moi.
${}^{10}Mets fin à la r\underline{a}ge des impies,
        \\afferm\underline{i}s le juste,
        \\toi qui scrutes les cœ\underline{u}rs et les reins,
        \\Die\underline{u}, le juste.
         
${}^{11}J’aurai mon boucli\underline{e}r auprès de Dieu,
        \\le sauve\underline{u}r des cœurs droits.
${}^{12}Dieu j\underline{u}ge avec justice ;
        \\Dieu menace chaque jour
        l’homme qui ne se r\underline{e}prend pas.
         
${}^{13}Le méchant aff\underline{û}te son épée,
        \\il tend son \underline{a}rc et le tient prêt.
${}^{14}Il se prépare des eng\underline{i}ns de mort ;
        \\de ses flèches, il f\underline{a}it des brandons.
         
${}^{15}Qui conçoit le mal et co\underline{u}ve le crime
        \\enfanter\underline{a} le mensonge.
${}^{16}Qui ouvre une f\underline{o}sse et la creuse
        \\tombera dans le tro\underline{u} qu’il a fait.
${}^{17}Son mauvais coup lui revi\underline{e}nt sur la tête,
        \\sa violence ret\underline{o}mbe sur son crâne.
         
${}^{18}Je rendrai grâce au Seigne\underline{u}r pour sa justice,
        \\je chanterai le nom du Seigne\underline{u}r, le Très-Haut.
      \bchapter{Psaume}
          
\bchapter{Psaume}
            Qu’il est grand ton nom !
${}^{1}Du maître de chœur. Sur la guittith. Psaume. De David.
         
${}^{2}Ô Seigne\underline{u}r, notre Dieu,
        qu’il est gr\underline{a}nd, ton nom,
        par to\underline{u}te la terre !
         
        \\Jusqu’aux cieux, ta splende\underline{u}r est chantée
${}^{3}par la bouche des enf\underline{a}nts, des tout-petits :
        \\rempart que tu opp\underline{o}ses à l’adversaire,
        \\où l’ennemi se br\underline{i}se en sa révolte.
         
${}^{4}À voir ton ciel, ouvr\underline{a}ge de tes doigts,
        \\la lune et les ét\underline{o}iles que tu fixas,
${}^{5}qu’est-ce que l’homme pour que tu p\underline{e}nses à lui,
        \\le fils d’un homme, que tu en pr\underline{e}nnes souci ?
         
${}^{6}Tu l’as voulu un peu m\underline{o}indre qu’un dieu,
        \\le couronnant de gl\underline{o}ire et d’honneur ;
${}^{7}tu l’établis sur les œ\underline{u}vres de tes mains,
        \\tu mets toute ch\underline{o}se à ses pieds :
         
${}^{8}les troupeaux de bœ\underline{u}fs et de brebis,
        \\et même les b\underline{ê}tes sauvages,
${}^{9}les oiseaux du ciel et les poiss\underline{o}ns de la mer,
        \\tout ce qui va son chem\underline{i}n dans les eaux.
         
${}^{10}Ô Seigne\underline{u}r, notre Dieu,
        qu’il est gr\underline{a}nd ton nom
        par to\underline{u}te la terre !
      \bchapter{Psaume}
          
            Tu as jugé avec justice
${}^{1}Du maître de chœur. Sur hautbois et harpe. Psaume.
        \\De David.
         
${}^{2}De tout mon cœur, Seigne\underline{u}r, je rendrai grâce,
        \\je dirai tes innombr\underline{a}bles merveilles ;
${}^{3}pour toi, j’exulter\underline{a}i, je danserai,
        \\je fêterai ton n\underline{o}m, Dieu Très-Haut.
         
${}^{4}Mes ennemis ont batt\underline{u} en retraite,
        \\devant ta face, ils s’écro\underline{u}lent et périssent.
${}^{5}Tu as plaidé mon dr\underline{o}it et ma cause,
        \\tu as siégé, tu as jug\underline{é} avec justice.
         
${}^{6}Tu menaces les nations, tu fais pér\underline{i}r les méchants,
        \\à tout jamais tu eff\underline{a}ces leur nom.
${}^{7}L’ennemi est achevé, ruin\underline{é} pour toujours,
        \\tu as rasé des villes, leur souven\underline{i}r a péri.
         
${}^{8}Mais il siège, le Seigne\underline{u}r, à jamais :
        \\pour juger, il afferm\underline{i}t son trône ;
${}^{9}il juge le m\underline{o}nde avec justice
        \\et gouverne les pe\underline{u}ples avec droiture.
         
${}^{10}Qu’il soit la forter\underline{e}sse de l’opprimé,
        \\sa forteresse aux he\underline{u}res d’angoisse :
${}^{11}ils s’appuieront sur toi, ceux qui conn\underline{a}issent ton nom ;
        \\jamais tu n’abandonnes, Seigneur, ce\underline{u}x qui te cherchent.
         
${}^{12}Fêtez le Seigneur qui si\underline{è}ge dans Sion,
        \\annoncez parmi les pe\underline{u}ples ses exploits !
${}^{13}Attentif au sang vers\underline{é}, il se rappelle,
        \\il n’oublie pas le cr\underline{i} des malheureux.
         
${}^{14}Pitié pour moi, Seigneur,
        vois le mal que m’ont f\underline{a}it mes adversaires, *
        \\toi qui m’arraches aux p\underline{o}rtes de la mort ;
${}^{15}et je dirai tes innombrables louanges
        aux p\underline{o}rtes de Sion, *
        \\je danserai de j\underline{o}ie pour ta victoire.
         
${}^{16}Ils sont tombés, les païens, dans la f\underline{o}sse qu’ils creusaient ;
        \\aux filets qu’ils ont tendus, leurs pi\underline{e}ds se sont pris.
${}^{17}Le Seigneur s’est fait connaître : il a rend\underline{u} le jugement,
        \\il prend les méch\underline{a}nts à leur piège.
         
${}^{18}Que les méchants reto\underline{u}rnent chez les morts,
        \\toutes les nations qui oubl\underline{i}ent le vrai Dieu !
${}^{19}Mais le pauvre n’est pas oubli\underline{é} pour toujours :
        \\jamais ne périt l’esp\underline{o}ir des malheureux.
         
${}^{20}Lève-toi, Seigneur : qu’un mortel ne soit p\underline{a}s le plus fort,
        \\que les nations soient jug\underline{é}es devant ta face !
${}^{21}Frappe-les d’épouv\underline{a}nte, Seigneur :
        \\que les nations se reconn\underline{a}issent mortelles !
      \bchapter{Psaume}
          
            \bchapter{Psaume}
            Tu entends le désir des pauvres
${}^{1}Pourquoi, Seigne\underline{u}r, es-tu si loin ?
        \\Pourquoi te cach\underline{e}r aux jours d’angoisse ?
${}^{2}L’impie, dans son orgueil, poursu\underline{i}t les malheureux :
        \\ils se font prendre aux r\underline{u}ses qu’il invente.
         
${}^{3}L’impie se glorifie du dés\underline{i}r de son âme,
        \\l’arrogant blasphème, il br\underline{a}ve le Seigneur ;
${}^{4}plein de suffisance, l’imp\underline{i}e ne cherche plus :
        \\« Dieu n’est rien », voil\underline{à} toute sa ruse.
         
${}^{5}À tout moment, ce qu’il f\underline{a}it réussit ; +
        \\tes sentences le dom\underline{i}nent de très haut. *
        \\(Tous ses advers\underline{a}ires, il les méprise.)
${}^{6}Il s’est dit : « Rien ne pe\underline{u}t m’ébranler,
        \\je suis pour longtemps à l’abr\underline{i} du malheur. »
         
${}^{7}Sa bouche qui maudit n’est que fra\underline{u}de et violence,
        \\sa langue, mens\underline{o}nge et blessure.
${}^{8}Il se tient à l’aff\underline{û}t près des villages,
        \\il se cache pour tu\underline{e}r l’innocent.
         
        \\Des yeux, il ép\underline{i}e le faible,
${}^{9}il se cache à l’affût, comme un li\underline{o}n dans son fourré ;
        \\il se tient à l’affût pour surpr\underline{e}ndre le pauvre,
        \\il attire le pauvre, il le pr\underline{e}nd dans son filet.
         
${}^{10}Il se b\underline{a}isse, il se tapit ;
        \\de tout son poids, il t\underline{o}mbe sur le faible.
${}^{11}Il dit en lui-même : « Die\underline{u} oublie !
        \\il couvre sa face, jam\underline{a}is il ne verra ! »
         
${}^{12}Lève-toi, Seigneur ! Die\underline{u}, étends la main !
        \\N’oublie p\underline{a}s le pauvre !
${}^{13}Pourquoi l’impie brave-t-\underline{i}l le Seigneur
        \\en lui disant : « Viendras-t\underline{u} me chercher ? »
         
${}^{14}Mais tu as vu : tu regardes le m\underline{a}l et la souffrance,
        \\tu les pr\underline{e}nds dans ta main ;
        \\sur toi rep\underline{o}se le faible,
        \\c’est toi qui viens en \underline{a}ide à l’orphelin.
         
${}^{15}Brise le bras de l’imp\underline{i}e, du méchant ;
        \\alors tu chercheras son impiét\underline{é} sans la trouver.
${}^{16}À tout jamais, le Seigne\underline{u}r est roi :
        \\les païens ont pér\underline{i} sur sa terre.
         
${}^{17}Tu entends, Seigneur, le dés\underline{i}r des pauvres,
        \\tu rassures leur cœ\underline{u}r, tu les écoutes.
${}^{18}Que justice soit rendue à l’orphelin,
        qu’il n’y ait pl\underline{u}s d’opprimé, *
        \\et que tremble le mortel, n\underline{é} de la terre !
      \bchapter{Psaume}
          
            \bchapter{Psaume}
            Il garde les yeux ouverts sur le monde
${}^{1}Du maître de chœur. De David.
         
        \\Auprès du Seigne\underline{u}r j’ai mon refuge.+
        \\Comment pouvez-vo\underline{u}s me dire :
        \\oiseaux, fuy\underline{e}z à la montagne !
         
${}^{2}Voici que les méch\underline{a}nts tendent l’arc : +
        \\ils ajustent leur fl\underline{è}che à la corde
        \\pour viser dans l’ombre l’h\underline{o}mme au cœur droit.
         
${}^{3}Quand sont ruin\underline{é}es les fondations,
        \\que peut f\underline{a}ire le juste ?
         
${}^{4}Mais le Seigneur, dans son t\underline{e}mple saint, +
        \\le Seigneur, dans les cie\underline{u}x où il trône,
        \\garde les yeux ouv\underline{e}rts sur le monde.
         
        \\Il voit, il scr\underline{u}te les hommes ; +
${}^{5}le Seigneur a scruté le j\underline{u}ste et le méchant :
        \\l’ami de la viol\underline{e}nce, il le hait.
         
${}^{6}Il fera pleuvoir ses fléa\underline{u}x sur les méchants, +
        \\feu et soufre et v\underline{e}nt de tempête ;
        \\c’est la coupe qu’ils aur\underline{o}nt en partage.
         
${}^{7}Vraiment, le Seigne\underline{u}r est juste ; +
        \\il aime to\underline{u}te justice :
        \\les hommes droits le verr\underline{o}nt face à face.
      \bchapter{Psaume}
          
            \bchapter{Psaume}
            Toi, tu tiens parole
${}^{1}Du maître de chœur. À l’octave. Psaume. De David.
         
${}^{2}Seigneur, au secours ! Il n’y a pl\underline{u}s de fidèle !
        \\La loyauté a dispar\underline{u} chez les hommes.
${}^{3}Entre eux la par\underline{o}le est mensonge,
        \\cœur double, l\underline{è}vres menteuses.
         
${}^{4}Que le Seigneur supprime ces l\underline{è}vres menteuses,
        \\cette langue qui p\underline{a}rle insolemment,
${}^{5}ceux-là qui disent : « Arm\underline{o}ns notre langue !
        \\À nous la parole ! Qui ser\underline{a} notre maître ? »
         
${}^{6}– « Pour le pauvre qui gémit,
        le malheure\underline{u}x que l’on dépouille, +
        \\maintenant je me lève, d\underline{i}t le Seigneur ; *
        \\à celui qu’on méprise, je p\underline{o}rte secours. »
         
${}^{7}Les paroles du Seigneur sont des par\underline{o}les pures,
        \\argent passé au feu, affin\underline{é} sept fois.
${}^{8}Toi, Seigne\underline{u}r, tu tiens parole,
        \\tu nous gardes pour toujo\underline{u}rs de cette engeance.
         
${}^{9}De tous côtés, s’ag\underline{i}tent les impies :
        \\la corruption g\underline{a}gne chez les hommes.
      \bchapter{Psaume}
          
            \bchapter{Psaume}
            Vas-tu m’oublier ?
${}^{1}Du maître de chœur. Psaume. De David.
         
${}^{2}Combien de temps, Seigneur, vas-t\underline{u} m’oublier,
        \\combien de temps, me cach\underline{e}r ton visage ?
${}^{3}Combien de temps aurai-je l’âme en peine
        et le cœur attrist\underline{é} chaque jour ? *
        \\Combien de temps mon ennemi sera-t-\underline{i}l le plus fort ?
         
${}^{4}Regarde, réponds-moi, Seigne\underline{u}r mon Dieu ! *
        \\Donne la lumière à mes yeux,
        garde-moi du somm\underline{e}il de la mort ;
${}^{5}que l’adversaire ne crie p\underline{a}s : « Victoire ! »,
        \\que l’ennemi n’ait pas la j\underline{o}ie de ma défaite !
         
${}^{6}Moi, je prends appu\underline{i} sur ton amour ; +
        \\que mon cœur ait la j\underline{o}ie de ton salut !
        \\Je chanterai le Seigneur pour le bi\underline{e}n qu’il m’a fait.
      
          
            \bchapter{Psaume}
            Pas un homme de bien !
${}^{1}Du maître de chœur. De David.
         
        \\Dans son cœur le fo\underline{u} déclare :
        « P\underline{a}s de Dieu ! » *
        \\Tout est corromp\underline{u}, abominable,
        pas un h\underline{o}mme de bien !
         
${}^{2}Des cieux, le Seigne\underline{u}r se penche
        v\underline{e}rs les fils d’Adam *
        \\pour voir s’il en est \underline{u}n de sensé,
        \underline{u}n qui cherche Dieu.
         
${}^{3}Tous, ils s\underline{o}nt dévoyés ;
        tous ens\underline{e}mble, pervertis : *
        \\pas un h\underline{o}mme de bien,
        pas m\underline{ê}me un seul !
         
${}^{4}N’ont-ils d\underline{o}nc pas compris,
        ces g\underline{e}ns qui font le mal ? +
        \\Quand ils mangent leur pain,
        ils m\underline{a}ngent mon peuple. *
        \\Jamais ils n’inv\underline{o}quent le Seigneur.
         
${}^{5}Et voilà qu’ils se sont m\underline{i}s à trembler,
        car Dieu accomp\underline{a}gne les justes. *
${}^{6}Vous riez des proj\underline{e}ts du malheureux,
        mais le Seigne\underline{u}r est son refuge.
         
${}^{7}Qui fera ven\underline{i}r de Sion
        la délivr\underline{a}nce d’Israël ? +
        \\Quand le Seigneur ramènera les déport\underline{é}s de son peuple, *
        quelle fête en Jacob, en Isra\underline{ë}l, quelle joie !
      \bchapter{Psaume}
          
            \bchapter{Psaume}
            Qui séjournera sous ta tente ?
${}^{1}Psaume. De David.
         
        \\Seigneur, qui séjourner\underline{a} sous ta tente ?
        \\Qui habitera ta s\underline{a}inte montagne ?
         
${}^{2}Celui qui se condu\underline{i}t parfaitement, +
        \\qui ag\underline{i}t avec justice
        \\et dit la vérit\underline{é} selon son cœur.
         
${}^{3}Il met un fr\underline{e}in à sa langue, +
        \\ne fait pas de t\underline{o}rt à son frère
        \\et n’outrage p\underline{a}s son prochain.
         
${}^{4}À ses yeux, le réprouv\underline{é} est méprisable
        \\mais il honore les fid\underline{è}les du Seigneur.
         
        \\S’il a jur\underline{é} à ses dépens,
        \\il ne reprend p\underline{a}s sa parole.
         
${}^{5}Il prête son arg\underline{e}nt sans intérêt, +
        \\n’accepte rien qui nu\underline{i}se à l’innocent.
        \\Qui fait ainsi deme\underline{u}re inébranlable.
      \bchapter{Psaume}
          
            \bchapter{Psaume}
            Seigneur, mon partage
${}^{1}Miktâm. De David.
         
        \\Garde-m\underline{o}i, mon Dieu :
        \\j’ai fait de t\underline{o}i mon refuge.
${}^{2}J’ai dit au Seigneur : « Tu \underline{e}s mon Dieu !
        \\Je n’ai pas d’autre bonhe\underline{u}r que toi. »
         
${}^{3}Toutes les idoles du pays,
        ces die\underline{u}x que j’aimais, +
        \\ne cessent d’ét\underline{e}ndre leurs ravages, *
        \\et l’on se r\underline{u}e à leur suite.
${}^{4}Je n’irai pas leur offrir le s\underline{a}ng des sacrifices ; *
        \\leur nom ne viendra p\underline{a}s sur mes lèvres !
         
${}^{5}Seigneur, mon part\underline{a}ge et ma coupe :
        \\de toi dép\underline{e}nd mon sort.
${}^{6}La part qui me revi\underline{e}nt fait mes délices ;
        \\j’ai même le plus b\underline{e}l héritage !
         
${}^{7}Je bénis le Seigne\underline{u}r qui me conseille :
        \\même la nuit mon cœ\underline{u}r m’avertit.
${}^{8}Je garde le Seigneur devant m\underline{o}i sans relâche ;
        \\il est à ma droite : je su\underline{i}s inébranlable.
         
${}^{9}Mon cœur exulte, mon \underline{â}me est en fête,
        \\ma chair elle-même rep\underline{o}se en confiance :
${}^{10}tu ne peux m’abandonn\underline{e}r à la mort
        \\ni laisser ton ami v\underline{o}ir la corruption.
         
${}^{11}Tu m’apprends le chem\underline{i}n de la vie : +
        \\devant ta face, débordem\underline{e}nt de joie !
        \\À ta droite, éternit\underline{é} de délices !
      \bchapter{Psaume}
          
            \bchapter{Psaume}
            Garde-moi
${}^{1}Prière. De David.
         
        \\Seigneur, éco\underline{u}te la justice ! +
        \\Entends ma plainte, accu\underline{e}ille ma prière :
        \\mes lèvres ne m\underline{e}ntent pas.
         
${}^{2}De ta face, me viendr\underline{a} la sentence :
        \\tes yeux verr\underline{o}nt où est le droit.
         
${}^{3}Tu sondes mon cœur, tu me vis\underline{i}tes la nuit, +
        \\tu m’éprouves, sans ri\underline{e}n trouver ;
        \\mes pensées n’ont pas franch\underline{i} mes lèvres.
         
${}^{4}Pour me conduire sel\underline{o}n ta parole,
        \\j’ai gardé le chem\underline{i}n prescrit ;
${}^{5}j’ai tenu mes p\underline{a}s sur tes traces :
        \\jamais mon pi\underline{e}d n’a trébuché.
         
${}^{6}Je t’appelle, toi, le Die\underline{u} qui répond :
        \\écoute-moi, ent\underline{e}nds ce que je dis.
         
${}^{7}Montre les merv\underline{e}illes de ta grâce, *
        \\toi qui libères de l’agresseur
        ceux qui se réfug\underline{i}ent sous ta droite.
         
${}^{8}Garde-moi comme la prun\underline{e}lle de l’œil ;
        \\à l’ombre de tes \underline{a}iles, cache-moi,
${}^{9}loin des méch\underline{a}nts qui m’ont ruiné,
        \\des ennemis mort\underline{e}ls qui m’entourent.
         
${}^{10}Ils s’enferment d\underline{a}ns leur suffisance ;
        \\l’arrogance à la bo\underline{u}che, ils parlent.
         
${}^{11}Ils sont sur mes pas : mainten\underline{a}nt ils me cernent,
        \\l’œil sur moi, pour me jet\underline{e}r à terre,
${}^{12}comme des lions pr\underline{ê}ts au carnage,
        \\de jeunes fauves tap\underline{i}s en embuscade.
         
${}^{13}Lève-toi, Seigneur, affronte-l\underline{e}s, renverse-les ;
        \\par ton épée, libère-m\underline{o}i des méchants.
         
${}^{14}Que ta main, Seigneur, les excl\underline{u}e d’entre les hommes, *
        \\hors de l’humanité, hors de ce monde :
        tel soit le s\underline{o}rt de leur vie !
         
        \\Réserve-leur de qu\underline{o}i les rassasier : +
        \\que leurs fils en s\underline{o}ient saturés,
        \\qu’il en reste enc\underline{o}re pour leurs enfants !
         
${}^{15}Et moi, par ta justice, je verr\underline{a}i ta face :
        \\au réveil, je me rassasier\underline{a}i de ton visage.
      \bchapter{Psaume}
          
            \bchapter{Psaume}
            Il m’a libéré, car il m’aime
${}^{1}Du maître de chœur. Du serviteur du Seigneur, de David, qui adressa au Seigneur les paroles de ce cantique, au jour où le Seigneur le délivra de la main de tous ses ennemis et de la main de Saül. Il dit :
         
${}^{2}Je t’aime, Seigne\underline{u}r, ma force :
        \\Seigneur, mon r\underline{o}c, ma forteresse,
${}^{3}Dieu mon libérateur, le roch\underline{e}r qui m’abrite,
        \\mon bouclier, mon fort, mon \underline{a}rme de victoire !
         
${}^{4}Lou\underline{a}nge à Dieu ! +
        \\Quand je fais app\underline{e}l au Seigneur, *
        \\je suis sauvé de to\underline{u}s mes ennemis.
         
        *
         
${}^{5}Les liens de la m\underline{o}rt m’entouraient,
        \\le torrent fat\underline{a}l m’épouvantait ;
${}^{6}des liens inferna\underline{u}x m’étreignaient :
        \\j’étais pris aux pi\underline{è}ges de la mort.
         
${}^{7}Dans mon angoisse, j’appel\underline{a}i le Seigneur ;
        \\vers mon Dieu, je lanç\underline{a}i un cri ;
        \\de son temple il ent\underline{e}nd ma voix :
        \\mon cri parvi\underline{e}nt à ses oreilles.
         
${}^{8}La terre tit\underline{u}be et tremble, +
        \\les assises des mont\underline{a}gnes frémissent,
        \\secouées par l’explosi\underline{o}n de sa colère.
         
${}^{9}Une fumée s\underline{o}rt de ses narines, +
        \\de sa bouche, un fe\underline{u} qui dévore,
        \\une gerbe de charb\underline{o}ns embrasés.
         
${}^{10}Il incline les cie\underline{u}x et descend,
        \\une sombre nu\underline{é}e sous ses pieds :
${}^{11}d’un Kéroub, il f\underline{a}it sa monture,
        \\il vole sur les \underline{a}iles du vent.
         
${}^{12}Il se cache au s\underline{e}in des ténèbres +
        \\et dans leurs repl\underline{i}s se dérobe :
        \\nuées sur nuées, tén\underline{è}bres diluviennes.
         
${}^{13}Une lue\underline{u}r le précède, +
        \\ses nu\underline{a}ges déferlent :
        \\grêle et g\underline{e}rbes de feu.
         
${}^{14}Tonnerre du Seigne\underline{u}r dans le ciel, *
        \\le Très-Haut fait entendre sa voix :
        grêle et g\underline{e}rbes de feu.
${}^{15}De tous côtés, il t\underline{i}re des flèches,
        \\il décoche des éclairs, il rép\underline{a}nd la terreur.
         
${}^{16}Alors le fond des m\underline{e}rs se découvrit,
        \\les assises du m\underline{o}nde apparurent,
        \\sous ta voix menaç\underline{a}nte, Seigneur,
        \\au souffle qu’exhal\underline{a}it ta colère.
         
${}^{17}Des hauteurs il tend la m\underline{a}in pour me saisir,
        \\il me retire du go\underline{u}ffre des eaux ;
${}^{18}il me délivre d’un puiss\underline{a}nt ennemi,
        \\d’adversaires plus f\underline{o}rts que moi.
         
${}^{19}Au jour de ma déf\underline{a}ite ils m’attendaient,
        \\mais j’avais le Seigne\underline{u}r pour appui.
${}^{20}Et lui m’a dégag\underline{é}, mis au large,
        \\il m’a libér\underline{é}, car il m’aime.
         
        *
         
${}^{21}Le Seigneur me traite sel\underline{o}n ma justice,
        \\il me donne le sal\underline{a}ire des mains pures,
${}^{22}car j’ai gardé les chem\underline{i}ns du Seigneur,
        \\jamais je n’ai trah\underline{i} mon Dieu.
         
${}^{23}Ses ordres sont to\underline{u}s devant moi,
        \\jamais je ne m’éc\underline{a}rte de ses lois.
${}^{24}Je suis sans repr\underline{o}che envers lui,
        \\je me garde l\underline{o}in du péché.
${}^{25}Le Seigneur me donne sel\underline{o}n ma justice,
        \\selon la pureté des m\underline{a}ins que je lui tends.
         
${}^{26}Tu es fidèle envers l’h\underline{o}mme fidèle,
        \\sans reproche avec l’h\underline{o}mme sans reproche ;
${}^{27}envers qui est loy\underline{a}l, tu es loyal,
        \\tu ruses av\underline{e}c le pervers.
         
${}^{28}Tu sauves le pe\underline{u}ple des humbles ;
        \\les regards haut\underline{a}ins, tu les rabaisses.
${}^{29}Tu es la lumi\underline{è}re de ma lampe,
        \\Seigneur mon Dieu, tu écl\underline{a}ires ma nuit.
${}^{30}Grâce à toi, je sa\underline{u}te le fossé,
        \\grâce à mon Dieu, je franch\underline{i}s la muraille.
        *
         
${}^{31}Ce Dieu a des chem\underline{i}ns sans reproche, +
        \\la parole du Seigne\underline{u}r est sans alliage,
        \\il est un bouclier pour qui s’abr\underline{i}te en lui.
         
${}^{32}Qui est Dieu, horm\underline{i}s le Seigneur ?
        \\le Rocher, sin\underline{o}n notre Dieu ?
${}^{33}C’est le Dieu qui m’empl\underline{i}t de vaillance
        \\et m’indique un chem\underline{i}n sans reproche.
         
${}^{34}Il me donne l’agilit\underline{é} du chamois,
        \\il me tient debo\underline{u}t sur les hauteurs,
${}^{35}il exerce mes m\underline{a}ins à combattre
        \\et mon bras, à t\underline{e}ndre l’arc.
         
${}^{36}Par ton bouclier tu m’ass\underline{u}res la victoire,
        \\ta droite me soutient, ta pati\underline{e}nce m’élève.
${}^{37}C’est toi qui all\underline{o}nges ma foulée
        \\sans que faibl\underline{i}ssent mes chevilles.
         
${}^{38}Je poursuis mes ennem\underline{i}s, je les rejoins,
        \\je ne reviens qu’apr\underline{è}s leur défaite ;
${}^{39}je les abats : ils ne pourr\underline{o}nt se relever ;
        \\ils tombent : les voil\underline{à} sous mes pieds.
         
${}^{40}Pour le combat tu m’empl\underline{i}s de vaillance ;
        \\devant moi tu fais pli\underline{e}r mes agresseurs.
${}^{41}Tu me livres des ennem\underline{i}s en déroute ;
        \\j’anéant\underline{i}s mes adversaires.
         
${}^{42}Ils appellent ? p\underline{a}s de sauveur !
        \\le Seigneur ? p\underline{a}s de réponse !
${}^{43}J’en fais de la poussi\underline{è}re pour le vent,
        \\de la boue qu’on enl\underline{è}ve des rues.
         
${}^{44}Tu me libères des quer\underline{e}lles du peuple,
        \\tu me places à la t\underline{ê}te des nations.
        \\Un peuple d’inconn\underline{u}s m’est asservi :
${}^{45}au premier m\underline{o}t, ils m’obéissent.
         
        \\Ces fils d’étrang\underline{e}rs se soumettent ; +
${}^{46}ces fils d’étrang\underline{e}rs capitulent :
        \\en tremblant ils qu\underline{i}ttent leurs bastions.
         
        *
         
${}^{47}Vive le Seigneur ! Bén\underline{i} soit mon Rocher !
        \\Qu’il triomphe, le Die\underline{u} de ma victoire,
${}^{48}ce Dieu qui m’acc\underline{o}rde la revanche,
        \\qui soumet à mon pouv\underline{o}ir les nations !
         
${}^{49}Tu me délivres de to\underline{u}s mes ennemis, +
        \\tu me fais triomph\underline{e}r de l’agresseur,
        \\tu m’arraches à la viol\underline{e}nce de l’homme.
         
${}^{50}Aussi, je te rendrai gr\underline{â}ce parmi les peuples,
        \\Seigneur, je fêter\underline{a}i ton nom.
${}^{51}Il donne à son roi de gr\underline{a}ndes victoires, *
        \\il se montre fidèle à son messie,
        à David et sa descend\underline{a}nce, pour toujours.
      \bchapter{Psaume}
          
            \bchapter{Psaume}
            Les cieux proclament la gloire de Dieu
${}^{1}Du maître de chœur. Psaume. De David.
         
${}^{2}Les cieux proclament la gl\underline{o}ire de Dieu,
        \\le firmament raconte l’ouvr\underline{a}ge de ses mains.
${}^{3}Le jour au jour en l\underline{i}vre le récit
        \\et la nuit à la nuit en d\underline{o}nne connaissance.
         
${}^{4}Pas de par\underline{o}les dans ce récit,
        \\pas de v\underline{o}ix qui s’entende ;
${}^{5}mais sur toute la terre en par\underline{a}ît le message
        \\et la nouvelle, aux lim\underline{i}tes du monde.
         
        \\Là, se trouve la deme\underline{u}re du soleil : +
${}^{6}tel un époux, il par\underline{a}ît hors de sa tente,
        \\il s’élance en conquér\underline{a}nt joyeux.
         
${}^{7}Il paraît où comm\underline{e}nce le ciel, +
        \\il s’en va jusqu’où le ci\underline{e}l s’achève :
        \\rien n’éch\underline{a}ppe à son ardeur.
      \bchapter{Psaume}
          
            \bchapter{Psaume}
            La loi qui clarifie le regard
${}^{8}La loi du Seigne\underline{u}r est parfaite,
        qui red\underline{o}nne vie ; *
        \\la charte du Seigne\underline{u}r est sûre,
        qui rend s\underline{a}ges les simples.
         
${}^{9}Les préceptes du Seigne\underline{u}r sont droits,
        ils réjou\underline{i}ssent le cœur ; *
        \\le commandement du Seigne\underline{u}r est limpide,
        il clarif\underline{i}e le regard.
         
${}^{10}La crainte qu’il insp\underline{i}re est pure,
        elle est l\underline{à} pour toujours ; *
        \\les décisions du Seigne\underline{u}r sont justes
        et vraim\underline{e}nt équitables :
         
${}^{11}plus désir\underline{a}bles que l’or,
        qu’une m\underline{a}sse d’or fin, *
        \\plus savoure\underline{u}ses que le miel
        qui co\underline{u}le des rayons.
         
${}^{12}Aussi ton serviteur en \underline{e}st illuminé ; +
        à les garder, il tro\underline{u}ve son profit. *
${}^{13}Qui peut discern\underline{e}r ses erreurs ?
        Purifie-moi de c\underline{e}lles qui m’échappent.
         
${}^{14}Préserve aussi ton servite\underline{u}r de l’orgueil :
        qu’il n’ait sur m\underline{o}i aucune emprise. *
        \\Alors je ser\underline{a}i sans reproche,
        p\underline{u}r d’un grand péché.
         
${}^{15}Accueille les par\underline{o}les de ma bouche,
        le murm\underline{u}re de mon cœur ; *
        \\qu’ils parvi\underline{e}nnent devant toi,
        Seigneur, mon roch\underline{e}r, mon défenseur !
      \bchapter{Psaume}
          
            \bchapter{Psaume}
            Il donne la victoire à son messie
${}^{1}Du maître de chœur. Psaume. De David.
         
${}^{2}Que le Seigneur te réponde au jo\underline{u}r de détresse,
        \\que le nom du Dieu de Jac\underline{o}b te défende.
${}^{3}Du sanctuaire, qu’il t’env\underline{o}ie le secours,
        \\qu’il te soutienne des haute\underline{u}rs de Sion.
         
${}^{4}Qu’il se rappelle to\underline{u}tes tes offrandes ;
        \\ton holocauste, qu’il le tro\underline{u}ve savoureux.
${}^{5}Qu’il te donne à la mes\underline{u}re de ton cœur,
        \\qu’il accomplisse to\underline{u}s tes projets.
         
${}^{6}Nous acclamerons ta victoire
        en arborant le n\underline{o}m de notre Dieu. *
        \\Le Seigneur accomplira
        to\underline{u}tes tes demandes.
         
${}^{7}Maintenant, je le sais :
        le Seigneur donne la vict\underline{o}ire à son messie ; *
        \\du sanctuaire des cieux, il lui répond
        par les exploits de sa m\underline{a}in victorieuse.
         
${}^{8}Aux uns, les chars ; aux a\underline{u}tres, les chevaux ;
        \\à nous, le nom de notre Die\underline{u} : le Seigneur.
${}^{9}Eux, ils pl\underline{i}ent et s’effondrent ;
        \\nous, debo\underline{u}t, nous résistons.
         
${}^{10}Seigneur, donne au r\underline{o}i la victoire !
        \\Réponds-nous au jo\underline{u}r de notre appel.
      \bchapter{Psaume}
          
            \bchapter{Psaume}
            Dresse-toi dans ta force
${}^{1}Du maître de chœur. Psaume. De David.
         
${}^{2}Seigneur, le roi se réjou\underline{i}t de ta force ;
        \\quelle allégresse lui d\underline{o}nne ta victoire !
${}^{3}Tu as répondu au dés\underline{i}r de son cœur,
        \\tu n’as pas rejeté le souh\underline{a}it de ses lèvres.
         
${}^{4}Tu lui destines bénédicti\underline{o}ns et bienfaits,
        \\tu mets sur sa tête une cour\underline{o}nne d’or.
${}^{5}La vie qu’il t’a demand\underline{é}e, tu la lui donnes,
        \\de longs jours, des ann\underline{é}es sans fin.
         
${}^{6}Par ta victoire, grand\underline{i}t son éclat :
        \\tu le revêts de splende\underline{u}r et de gloire.
${}^{7}Tu mets en lui ta bénédicti\underline{o}n pour toujours :
        \\ta présence l’empl\underline{i}t de joie !
         
${}^{8}Oui, le roi s’appu\underline{i}e sur le Seigneur :
        \\la grâce du Très-Haut le r\underline{e}nd inébranlable.
${}^{9}\[Ta main trouver\underline{a} tes ennemis,
        \\ta droite trouver\underline{a} tes adversaires.
         
${}^{10}Tu parais, tu en f\underline{a}is un brasier :
        \\la colère du Seigneur les consume,
        un fe\underline{u} les dévore.
${}^{11}Tu aboliras leur lign\underline{é}e sur la terre
        \\et leur descendance parm\underline{i} les hommes.
         
${}^{12}S’ils trament le mal contre toi,
        s’ils prép\underline{a}rent un complot, *
        \\ils ir\underline{o}nt à l’échec.
${}^{13}Oui, tu les renv\underline{e}rses et les terrasses ;
        \\ton arc les v\underline{i}se en plein cœur.\]
         
${}^{14}Dresse-toi, Seigne\underline{u}r, dans ta force :
        \\nous fêterons, nous chanter\underline{o}ns ta vaillance.
      \bchapter{Psaume}
          
            \bchapter{Psaume}
            Mon Dieu, pourquoi m’as-tu abandonné ?
${}^{1}Du maître de chœur. Sur l’air de « La biche de l’aurore ». Psaume. De David.
         
${}^{2}Mon Die\underline{u}, mon Dieu,
        pourquoi m’as-t\underline{u} abandonné ? *
        \\Le sal\underline{u}t est loin de moi,
        loin des m\underline{o}ts que je rugis.
         
${}^{3}Mon Dieu, j’app\underline{e}lle tout le jour,
        et tu ne r\underline{é}ponds pas ; *
        \\m\underline{ê}me la nuit,
        je n’ai p\underline{a}s de repos.
         
${}^{4}Toi, pourt\underline{a}nt, tu es saint,
        \\toi qui habites les h\underline{y}mnes d’Israël !
${}^{5}C’est en toi que nos p\underline{è}res espéraient,
        \\ils espéraient et tu les d\underline{é}livrais.
${}^{6}Quand ils criaient vers t\underline{o}i, ils échappaient ;
        \\en toi ils espéraient et n’étaient p\underline{a}s déçus.
         
${}^{7}Et moi, je suis un v\underline{e}r, pas un homme,
        \\raillé par les gens, rejet\underline{é} par le peuple.
${}^{8}Tous ceux qui me v\underline{o}ient me bafouent,
        \\ils ricanent et h\underline{o}chent la tête :
${}^{9}« Il comptait sur le Seigne\underline{u}r : qu’il le délivre !
        \\Qu’il le sauve, puisqu’il \underline{e}st son ami ! »
         
${}^{10}C’est toi qui m’as tiré du v\underline{e}ntre de ma mère,
        \\qui m’a mis en sûret\underline{é} entre ses bras.
${}^{11}À toi je fus confi\underline{é} dès ma naissance ;
        \\dès le ventre de ma m\underline{è}re, tu es mon Dieu.
         
${}^{12}Ne sois pas loin : l’ang\underline{o}isse est proche,
        \\je n’ai pers\underline{o}nne pour m’aider.
${}^{13}Des fauves nombre\underline{u}x me cernent,
        \\des taureaux de Bash\underline{a}ne m’encerclent.
${}^{14}Des lions qui déch\underline{i}rent et rugissent
        \\ouvrent leur gue\underline{u}le contre moi.
         
${}^{15}Je suis comme l’ea\underline{u} qui se répand,
        \\tous mes m\underline{e}mbres se disloquent.
        \\Mon cœur est c\underline{o}mme la cire,
        \\il fond au milie\underline{u} de mes entrailles.
${}^{16}Ma vigueur a séch\underline{é} comme l’argile,
        \\ma langue c\underline{o}lle à mon palais.
         
        \\Tu me mènes à la poussi\underline{è}re de la mort. +
         
${}^{17}Oui, des chi\underline{e}ns me cernent,
        \\une bande de vauri\underline{e}ns m’entoure.
        \\Ils me percent les m\underline{a}ins et les pieds ;
${}^{18}je peux compt\underline{e}r tous mes os.
         
        \\Ces gens me v\underline{o}ient, ils me regardent. +
${}^{19}Ils partagent entre e\underline{u}x mes habits
        \\et tirent au s\underline{o}rt mon vêtement.
         
${}^{20}Mais toi, Seigne\underline{u}r, ne sois pas loin :
        \\ô ma force, viens v\underline{i}te à mon aide !
${}^{21}Préserve ma v\underline{i}e de l’épée,
        \\arrache-moi aux gr\underline{i}ffes du chien ;
${}^{22}sauve-moi de la gue\underline{u}le du lion
        \\et de la c\underline{o}rne des buffles.
         
        *
         
        \\Tu m’\underline{a}s répondu ! +
${}^{23}Et je proclame ton n\underline{o}m devant mes frères,
        \\je te loue en pl\underline{e}ine assemblée.
         
${}^{24}Vous qui le craignez, lou\underline{e}z le Seigneur, +
        \\glorifiez-le, vous tous, descend\underline{a}nts de Jacob,
        \\vous tous, redoutez-le, descend\underline{a}nts d’Israël.
         
${}^{25}Car il n’a p\underline{a}s rejeté,
        \\il n’a pas réprouvé le malheure\underline{u}x dans sa misère ;
        \\il ne s’est pas voilé la f\underline{a}ce devant lui,
        \\mais il ent\underline{e}nd sa plainte.
         
${}^{26}Tu seras ma louange dans la gr\underline{a}nde assemblée ;
        \\devant ceux qui te craignent, je tiendr\underline{a}i mes promesses.
${}^{27}Les pauvres mangeront : ils ser\underline{o}nt rassasiés ;
        \\ils loueront le Seigneur, ceux qui le cherchent :
        « À vous, toujours, la v\underline{i}e et la joie ! »
         
${}^{28}La terre entière se souviendra
        et reviendr\underline{a} vers le Seigneur,
        \\chaque famille de nations se prosterner\underline{a} devant lui :
${}^{29}« Oui, au Seigne\underline{u}r la royauté,
        \\le pouv\underline{o}ir sur les nations ! »
         
${}^{30}Tous ceux qui festoy\underline{a}ient s’inclinent ;
        \\promis à la mort, ils pl\underline{i}ent en sa présence.
         
${}^{31}Et moi, je vis pour lui : ma descend\underline{a}nce le servira ;
        \\on annoncera le Seigneur aux générati\underline{o}ns à venir.
${}^{32}On proclamera sa justice au pe\underline{u}ple qui va naître :
        \\Voil\underline{à} son œuvre !
