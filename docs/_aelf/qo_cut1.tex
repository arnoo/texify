  
  
    
    \bbook{QOHÈLETH}{QOHÈLETH}
      
         
      \bchapter{}
${}^{1}Paroles de Qohèleth,
        \\fils de David, roi de Jérusalem.
        
           
        ${}^{2}Vanité\\des vanités disait Qohèleth.
        Vanité des vanités, tout est vanité !
        ${}^{3}Quel profit l’homme retire-t-il
        de toute la peine qu’il se donne sous le soleil ?
        ${}^{4}Une génération s’en va, une génération s’en vient,
        et la terre subsiste toujours.
        ${}^{5}Le soleil se lève, le soleil se couche ;
        il se hâte de retourner à sa place,
        et de nouveau il se lèvera.
        ${}^{6}Le vent part vers le sud, il tourne vers le nord ;
        il tourne et il tourne,
        et recommence à tournoyer.
        ${}^{7}Tous les fleuves vont à la mer,
        et la mer n’est pas remplie ;
        \\dans le sens où vont les fleuves,
        les fleuves continuent de couler.
        ${}^{8}Tout discours est fatigant,
        on ne peut jamais tout dire.
        \\L’œil n’a jamais fini de voir,
        ni l’oreille d’entendre.
        ${}^{9}Ce qui a existé, c’est cela qui existera ;
        ce qui s’est fait, c’est cela qui se fera ;
        rien de nouveau sous le soleil.
        ${}^{10}Y a-t-il une seule chose dont on dise :
        « Voilà enfin du nouveau ! »
        \\– Non\\, cela existait déjà dans les siècles passés.
        ${}^{11}Mais, il ne reste pas de souvenir d’autrefois\\ ;
        \\de même, les événements futurs
        ne laisseront pas de souvenir après eux.
${}^{12}Moi, Qohèleth,
        j’étais roi d’Israël à Jérusalem.
${}^{13}J’ai pris à cœur de rechercher et d’explorer,
        grâce à la sagesse,
        tout ce qui se fait sous le ciel ;
        \\c’est là une rude besogne
        que Dieu donne aux fils d’Adam
        pour les tenir en haleine.
${}^{14}J’ai vu tout ce qui se fait et se refait sous le soleil.
        \\Eh bien ! Tout cela n’est que vanité et poursuite du vent.
${}^{15}Ce qui est courbé ne se redresse pas
        et ce qui manque ne peut être compté.
         
${}^{16}J’ai réfléchi et je me disais :
        \\C’est moi qui ai fait grandir et progresser la sagesse
        plus que tous mes prédécesseurs à Jérusalem.
        \\J’ai approfondi la sagesse et le savoir.
         
${}^{17}J’avais à cœur de connaître la sagesse,
        de connaître aussi la sottise et la folie,
        et j’ai su que cela encore était tourment de l’esprit.
${}^{18}Beaucoup de sagesse, c’est beaucoup de chagrin.
        Qui augmente son savoir augmente sa douleur.
      
         
      \bchapter{}
${}^{1}Je me suis dit :
        « Va, essaie la joie
        et goûte au bonheur. »
        \\Eh bien, cela aussi n’était que vanité :
${}^{2}Au rire, j’ai dit : « Tu es sot ! »
        et à la joie : « À quoi sers-tu ? »
${}^{3}Je résolus de m’adonner au vin,
        tout en poursuivant la sagesse,
        \\et je me livrai à la démesure,
        le temps de voir ce qu’il est bon, pour les fils d’Adam,
        \\de faire sous le ciel
        pendant le peu de jours qu’ils ont à vivre.
        
           
         
${}^{4}J’ai entrepris de grands travaux :
        je me suis bâti des maisons
        et planté des vignes.
${}^{5}Je me suis aménagé des jardins et des vergers ;
        j’y ai planté toutes sortes d’arbres fruitiers.
${}^{6}J’ai creusé pour moi des bassins
        dont les eaux irriguent des pépinières.
${}^{7}J’ai eu des serviteurs et des servantes,
        leurs enfants nés dans ma maison,
        \\ainsi qu’une abondance de gros et petit bétail,
        plus que tous mes prédécesseurs à Jérusalem.
${}^{8}J’ai encore amassé de l’argent et de l’or,
        la fortune des rois et des États.
        \\J’ai eu des chanteurs et des chanteuses
        et ce plaisir des fils d’Adam :
        une compagne, des compagnes…
${}^{9}Je me suis agrandi, j’ai surpassé
        tous mes prédécesseurs à Jérusalem,
        \\et ma sagesse me restait.
${}^{10}Rien de ce que mes yeux convoitaient,
        je ne l’ai refusé.
        \\Je n’ai privé mon cœur d’aucune joie ;
        je me suis réjoui de tous mes travaux,
        \\et ce fut ma part pour tant de labeur.
${}^{11}Mais quand j’ai regardé
        tous les travaux accomplis par mes mains
        et ce qu’ils m’avaient coûté d’efforts,
        \\voilà : tout n’était que vanité et poursuite de vent ;
        rien à gagner sous le soleil !
        
           
${}^{12}Alors j’ai tourné mes regards vers la sagesse,
        vers la sottise et la folie :
        \\« Voyons, que fera le successeur du roi ?
        – Ce que déjà on a fait ! »
${}^{13}Voici donc ce que j’ai constaté :
        autant la lumière l’emporte sur les ténèbres,
        autant la sagesse l’emporte sur la folie.
${}^{14}Le sage a les yeux où il faut ;
        le fou marche dans l’obscurité.
        \\Mais je sais aussi que tous deux
        auront le même sort.
${}^{15}Alors je me suis dit :
        \\« Si le sort du fou et le mien sont les mêmes,
        à quoi bon avoir été si sage ? »
        \\Et j’ai pensé en moi-même :
        Cela aussi n’est que vanité !
${}^{16}Car le sage ne laisse aucun souvenir,
        pas plus que le fou, et cela pour toujours,
        \\puisque, dès les jours suivants, tout est oublié.
        Comment se fait-il que le sage meure aussi bien que le fou ?
${}^{17}Oui, je déteste la vie ;
        je trouve mauvais ce qui se fait sous le soleil :
        tout n’est que vanité et poursuite de vent.
${}^{18}Je déteste tout ce travail que j’accomplis sous le soleil
        et que je vais laisser à mon successeur.
${}^{19}Qui sait s’il sera sage ou insensé ?
        \\Ce sera lui le maître de tous ces travaux
        accomplis par ma sagesse sous le soleil.
        \\Cela aussi n’est que vanité !
${}^{20}J’ai fini par me dégoûter
        de toute la peine que je m’étais donnée sous le soleil.
         
        ${}^{21}Un homme s’est donné de la peine ;
        il est avisé, il s’y connaissait, il a réussi.
        \\Et voilà qu’il doit laisser son bien
        à quelqu’un qui ne s’est donné aucune peine.
        \\Cela aussi n’est que vanité,
        c’est un grand mal !
        ${}^{22}En effet, que reste-t-il à l’homme
        de toute la peine et de tous les calculs
        pour lesquels il se fatigue sous le soleil ?
        ${}^{23}Tous ses jours sont autant de souffrances,
        ses occupations sont autant de tourments :
        même la nuit, son cœur n’a pas de repos.
        \\Cela aussi n’est que vanité.
         
${}^{24}Rien de bon pour l’homme,
        sinon manger et boire,
        et trouver le bonheur dans son travail.
        \\J’ai vu que cela aussi
        vient de la main de Dieu.
${}^{25}Et qui donc pourrait manger
        et prendre du plaisir à ma place ?
${}^{26}À l’homme qui lui est agréable,
        Dieu donne sagesse, savoir et joie.
        \\Quant au pécheur, il le charge
        de recueillir et d’amasser
        pour donner à qui lui plaît.
        \\Cela aussi n’est que vanité et poursuite de vent.
      
         
      \bchapter{}
        ${}^{1}Il y a un moment pour tout,
        et un temps pour chaque chose sous le ciel :
        ${}^{2}un temps pour donner la vie,
        et un temps pour mourir ;
        \\un temps pour planter,
        et un temps pour arracher.
        ${}^{3}Un temps pour tuer,
        et un temps pour guérir ;
        \\un temps pour détruire
        et un temps pour construire.
        ${}^{4}Un temps pour pleurer,
        et un temps pour rire ;
        \\un temps pour gémir,
        et un temps pour danser.
        ${}^{5}Un temps pour jeter des pierres,
        et un temps pour les amasser ;
        \\un temps pour s’étreindre,
        et un temps pour s’abstenir.
        ${}^{6}Un temps pour chercher,
        et un temps pour perdre ;
        \\un temps pour garder,
        et un temps pour jeter.
        ${}^{7}Un temps pour déchirer,
        et un temps pour coudre ;
        \\un temps pour se taire,
        et un temps pour parler.
        ${}^{8}Un temps pour aimer,
        et un temps pour ne pas aimer\\ ;
        \\un temps pour la guerre,
        et un temps pour la paix.
        
           
         
        ${}^{9}Quel profit le travailleur retire-t-il
        de toute la peine qu’il prend ?
        ${}^{10}J’ai vu la besogne que Dieu impose aux fils d’Adam
        pour les tenir en haleine.
        ${}^{11}Toutes les choses que Dieu a faites
        sont bonnes en leur temps.
        \\Dieu a mis toute la durée du temps dans l’esprit de l’homme,
        mais celui-ci est incapable
        \\d’embrasser l’œuvre que Dieu a faite
        du début jusqu’à la fin.
${}^{12}J’ai compris qu’il n’y a rien de bon pour les humains,
        sinon se réjouir et prendre du bon temps durant leur vie.
${}^{13}Bien plus, pour chacun, manger et boire
        et trouver le bonheur dans son travail,
        c’est un don de Dieu.
${}^{14}Je le sais : tout ce que Dieu fait,
        à jamais, demeurera.
        \\À cela, il n’y a rien à ajouter,
        rien à retrancher.
        \\Dieu fait en sorte
        que l’on craigne en sa présence.
${}^{15}Ce qui est a déjà été,
        ce qui sera a déjà existé.
        \\Dieu fera revenir
        ce qui a passé.
        
           
${}^{16}J’ai vu encore sous le soleil
        la corruption sur le siège du droit,
        la corruption sur le siège de la justice.
${}^{17}Je me suis dit :
        \\le juste et l’injuste,
        Dieu les jugera,
        \\car il y a un temps pour chaque chose
        et un jugement pour chaque action.
${}^{18}Je me suis dit
        à propos des fils d’Adam :
        \\Dieu les met à l’épreuve
        pour leur montrer qu’ils sont comme les bêtes.
${}^{19}Car le sort des fils d’Adam et celui de la bête
        sont un seul et même sort.
        \\Comme est la mort de l’un,
        ainsi la mort de l’autre :
        ils ont tous un seul et même souffle.
        \\L’homme n’a rien de plus que la bête :
        tout est vanité.
${}^{20}Tout va vers un même lieu :
        tout est tiré de la poussière,
        et tout retourne à la poussière.
${}^{21}Qui sait où va le souffle des fils d’Adam ?
        Monte-t-il vers le haut,
        \\tandis que le souffle de la bête
        descendrait vers la terre ?
${}^{22}Je ne vois rien de mieux pour l’homme
        que de jouir de son ouvrage, car tel est son lot.
        \\Qui donc l’emmènera voir
        ce qui, après lui, sera ?
       
      
         
      \bchapter{}
${}^{1}J’ai regardé encore et j’ai vu
        toutes les oppressions pratiquées sous le soleil.
        \\Voyez les pleurs des opprimés :
        ils n’ont pas de consolateur ;
        \\des oppresseurs leur font violence :
        ils n’ont pas de consolateur.
${}^{2}Les morts qui sont déjà morts,
        je les déclare plus heureux
        que les vivants encore en vie,
${}^{3}et plus heureux que ceux-là
        celui qui n’existe pas encore,
        \\car il n’a pas connu le mal
        qui se fait sous le soleil.
        
           
${}^{4}J’ai vu aussi que toute la peine,
        tout le succès d’un travail,
        \\n’est que jalousie des uns envers les autres.
        C’est encore vanité et poursuite de vent.
         
${}^{5}Le fou se croise les bras :
        il consume sa propre vie.
${}^{6}Mieux vaut une pleine main de repos
        que deux pleines poignées d’efforts
        à la poursuite du vent.
${}^{7}J’ai regardé encore et j’ai vu
        une autre vanité sous le soleil :
${}^{8}voici un homme seul,
        sans personne, ni frère ni fils,
        \\qui travaille à n’en plus finir,
        toujours avide de plus de richesses.
        \\Il ne se demande pas :
        \\« Mais pour qui travailler ainsi
        en me privant de bonheur ? »
        \\C’est encore de la vanité,
        une besogne de malheur.
         
${}^{9}Mieux vaut être deux qu’un seul :
        le salaire de leur peine sera meilleur.
${}^{10}S’ils tombent, l’un relève l’autre.
        \\Malheur à l’homme seul :
        s’il tombe, personne ne le relève.
${}^{11}De même, si l’on dort à deux,
        on se tient chaud.
        \\Mais tout seul,
        comment se réchauffer ?
${}^{12}L’agresseur terrasse un homme seul :
        à deux, on lui résiste.
        \\Une corde à trois brins
        n’est pas facile à rompre.
${}^{13}Mieux vaut un gamin pauvre et sage
        qu’un vieux roi débile, refusant tout conseil,
${}^{14}car il peut sortir de prison pour régner,
        bien que né pauvre dans son royaume.
${}^{15}J’ai vu tous les vivants qui vont sous le soleil
        se joindre à ce gamin
        prétendant à la succession du roi :
${}^{16}innombrable était la foule
        de ceux qu’il conduisait.
        \\Mais ses futurs sujets n’en seront pas heureux.
        \\Car cela aussi n’est que vanité,
        tourment de l’esprit.
${}^{17}Surveille tes pas
        quand tu vas à la maison de Dieu ;
        \\approche-toi pour écouter
        plutôt que pour offrir le sacrifice des sots :
        ils ignorent le mal qu’ils font.
         
      
         
      \bchapter{}
${}^{1}Ne te presse pas d’ouvrir la bouche,
        que ton cœur ne se hâte pas de parler à Dieu,
        \\car Dieu est au ciel,
        et toi, sur la terre.
        \\Donc, que tes paroles soient rares.
        
           
         
${}^{2}Trop de tracas fait délirer,
        trop de discours fait divaguer.
${}^{3}Quand tu fais à Dieu une promesse,
        ne tarde pas à l’accomplir.
        \\Dieu n’aime pas les insensés :
        ce que tu as promis, tiens-le.
${}^{4}Mieux vaut ne rien promettre
        que promettre sans tenir.
        
           
         
${}^{5}Évite les mots qui conduisent au péché
        et font dire devant le messager de Dieu :
        « C’est une erreur ! »
        \\Faudrait-il que Dieu s’irrite de tes propos
        et ruine le travail de tes mains ?
${}^{6}Quand foisonnent les délires
        et prolifèrent les paroles vaines,
        \\alors, crains Dieu.
        
           
${}^{7}Si tu vois, dans le pays, l’oppression du pauvre,
        le droit et la justice violés,
        ne t’étonne pas de tels agissements ;
        \\car un grand personnage est couvert par un plus grand,
        et ceux-là le sont par de plus grands encore.
${}^{8}Mais la terre profite à tous :
        le roi lui-même en dépend.
${}^{9}Qui aime l’argent
        n’a jamais assez d’argent,
        \\et qui aime l’abondance
        ne récolte rien.
        \\Cela aussi n’est que vanité.
         
${}^{10}Plus il y a de richesses,
        plus il y a de profiteurs.
        \\Que va en retirer celui qui les possède,
        sinon un spectacle pour ses yeux ?
${}^{11}Le travailleur dormira en paix,
        qu’il ait peu ou beaucoup à manger,
        \\alors que, rassasié,
        le riche ne parvient pas à dormir.
         
${}^{12}Voici un triste cas que j’ai vu sous le soleil :
        une fortune amassée pour le malheur de son maître.
${}^{13}Il perd son avoir dans une mauvaise affaire,
        et quand lui naît un fils, celui-ci n’a rien en main.
${}^{14}Sorti nu du sein de sa mère,
        il s’en ira comme il est venu.
        \\Il n’emportera rien de son travail,
        rien que sa main puisse tenir.
${}^{15}C’est aussi une triste chose
        qu’il s’en aille comme il était venu.
        \\Qu’a-t-il gagné en peinant pour du vent ?
${}^{16}Il ronge ses jours dans le noir,
        la tristesse profonde, la souffrance et l’irritation.
         
${}^{17}Voilà donc ce que moi j’ai vu :
        \\c’est chose belle et bonne, pour quelqu’un,
        de manger et de boire,
        \\de trouver son bonheur
        dans toute la peine qu’il se donne sous le soleil
        pendant les jours que Dieu lui accorde.
        \\Telle est la part qui lui revient.
${}^{18}Si Dieu donne à quelqu’un biens et richesses
        avec pouvoir d’en profiter, d’en prendre sa part
        et de jouir ainsi de son travail,
        \\c’est là un don de Dieu.
${}^{19}Il ne s’inquiète guère pour sa vie
        tant que Dieu emplit de joie son cœur.
         
      
         
      \bchapter{}
${}^{1}Il est un autre mal que j’ai vu sous le soleil,
        \\un grand mal pour la race humaine.
${}^{2}Voilà un homme auquel Dieu a donné
        d’être riche, nanti, considéré :
        rien ne lui manque de tout ce qu’il souhaite.
        \\Mais Dieu ne lui a pas laissé le temps d’en profiter :
        un autre, un étranger, en profite.
        \\Cela aussi n’est que vanité, mal cruel.
        
           
${}^{3}Un homme peut avoir eu une centaine d’enfants
        et avoir vécu de longues années :
        \\aussi nombreux qu’aient été les jours de sa vie,
        s’il n’a pas été heureux et comblé,
        \\s’il n’a même pas eu de sépulture,
        je dis que l’avorton a plus de chance ;
${}^{4}lui qui est venu dans la vanité,
        il a passé comme une ombre ;
        son nom reste enfoui dans les ténèbres ;
${}^{5}il n’a même pas vu le soleil,
        il ne l’a pas connu ;
        il est plus tranquille que l’autre.
${}^{6}Même si un homme devait vivre deux fois mille ans,
        sans connaître le bonheur,
        tout ne va-t-il pas au même lieu ?
${}^{7}Tout le travail de l’être humain est pour la bouche,
        et pourtant son appétit n’est jamais comblé.
${}^{8}Qu’est-ce qu’un sage a de plus qu’un fou ?
        \\Qu’est-ce qu’un indigent a de plus
        quand il se tire d’affaire ?
${}^{9}Mieux vaut ce que l’on voit de ses yeux
        qu’une bouffée de désirs.
        \\Cela aussi n’est que vanité et poursuite de vent.
         
${}^{10}Tout ce qui existe a déjà reçu son nom ;
        on sait ce qu’est un homme :
        \\il ne peut entrer en procès
        contre un plus fort que lui.
${}^{11}Beaucoup de paroles, c’est beaucoup de vanité :
        et quel profit pour l’homme ?
${}^{12}Qui sait ce qui est bon pour l’homme durant sa vie,
        durant le peu de jours de cette vie de vanité
        qu’il traverse comme une ombre ?
        \\Qui donc peut lui révéler
        ce qui, après lui, sera sous le soleil ?
      
         
      \bchapter{}
${}^{1}Mieux vaut bonne renommée
        que parfum de grand prix,
        \\et le jour de la mort
        plutôt que le jour de la naissance.
        
           
         
${}^{2}Mieux vaut aller à la maison du deuil
        qu’à la maison du banquet :
        \\telle est la fin de tous les humains ;
        que les vivants s’en souviennent !
${}^{3}La tristesse vaut mieux que le rire :
        à mine sombre, cœur content !
${}^{4}Le cœur du sage habite la maison du deuil,
        et le cœur du fou, la maison du plaisir.
        
           
         
${}^{5}Mieux vaut prêter l’oreille aux reproches d’un sage
        que d’écouter les louanges d’un fou.
${}^{6}Crépitement d’épines sous la marmite,
        voilà le rire des fous.
        Cela aussi n’est que vanité !
${}^{7}Le pouvoir tourne la tête du sage,
        les cadeaux corrompent son cœur.
        
           
         
${}^{8}Mieux vaut la fin d’une chose
        que son commencement.
        \\Mieux vaut un esprit patient
        qu’un esprit arrogant.
${}^{9}Ne cède pas trop vite à la colère :
        la colère couve au cœur du fou.
${}^{10}Ne dis pas :
        \\D’où vient que les jours d’autrefois
        étaient meilleurs que ceux d’aujourd’hui ?
        \\Ce n’est pas la sagesse qui t’inspire cette question.
        
           
         
${}^{11}La sagesse est bonne autant qu’un héritage,
        et bénéfique pour ceux qui voient le soleil.
${}^{12}Oui, la sagesse met à l’abri,
        comme le fait l’argent,
        \\mais l’avantage de la sagesse
        est de faire vivre qui la possède.
${}^{13}Regarde l’ouvrage de Dieu :
        qui peut redresser ce que lui a courbé ?
${}^{14}Au jour de bonheur, réjouis-toi ;
        au jour de malheur, réfléchis :
        \\c’est Dieu qui les a faits l’un et l’autre,
        pour que l’homme ne sache rien
        de ce qui viendra après lui.
        
           
         
${}^{15}J’aurai tout vu au long de mes jours incertains :
        et le juste périr malgré sa justice,
        et le méchant tenir malgré sa méchanceté.
${}^{16}Ne sois pas juste à l’extrême,
        ne fais pas le sage à l’excès :
        pourquoi te détruire ?
${}^{17}Ne sois pas méchant à l’extrême,
        ne sois pas insensé :
        pourquoi mourir avant l’heure ?
        
           
         
${}^{18}Il est bon que tu tiennes à ceci
        sans pour autant lâcher cela :
        car celui qui craint Dieu réussit l’un et l’autre.
${}^{19}La sagesse rend le sage plus fort
        que dix gouverneurs dans la cité.
        
           
         
${}^{20}Certes, aucun homme, sur terre, n’est assez juste
        pour faire le bien sans jamais pécher.
${}^{21}Ne t’attache pas non plus à tout ce que l’on dit :
        ainsi tu n’entendras pas ton serviteur te maudire.
${}^{22}Car, bien des fois, tu as pris conscience
        d’avoir, toi aussi, maudit les autres.
        
           
         
${}^{23}Avec la sagesse, j’ai expérimenté toutes ces choses.
        J’ai dit : « Je veux être un sage ! »
        Mais c’était loin de ma portée.
${}^{24}Lointain, tout ce qui existe,
        profond, profond…
        qui le découvrira ?
        
           
         
${}^{25}J’ai concentré ma réflexion
        \\pour connaître, explorer, rechercher
        la sagesse et la raison des choses,
        \\pour comprendre que la méchanceté est folie,
        et la démesure, sottise.
        
           
${}^{26}Voici ce que je trouve :
        il y a plus amer que la mort,
        \\c’est la femme quand elle est un piège,
        quand son cœur est un filet,
        et ses bras, des chaînes.
        \\Celui qui plaît à Dieu lui échappe,
        mais elle a prise sur le pécheur.
         
${}^{27}Voilà ce que j’ai trouvé, dit Qohèleth,
        en prenant les choses une par une
        pour en découvrir la raison.
${}^{28}Ce que j’ai cherché jusqu’à maintenant,
        je ne l’ai pas trouvé :
        \\un homme entre mille ? je le trouve ;
        une femme entre toutes : je ne la trouve pas.
${}^{29}La seule chose que j’ai trouvée, la voici :
        Dieu a fait l’homme droit,
        mais les humains ont inventé tant de détours…
      
         
      \bchapter{}
${}^{1}Qui donc est comparable au sage ?
        Qui sait expliquer le sens des choses ?
        \\La sagesse d’un homme fait briller son visage ;
        la dureté du visage en est changée.
        
           
         
${}^{2}Je dis : Obéis aux ordres du roi
        selon le serment fait à Dieu.
${}^{3}Ne sois pas pressé de t’écarter de lui,
        ne t’obstine pas dans un mauvais cas :
        il ne fera que ce qui lui plaît,
${}^{4}car la parole du roi est souveraine.
        Qui pourrait lui dire : « Que fais-tu là ? »
${}^{5}Celui qui obéit au commandement
        ne se mettra pas dans un mauvais cas.
        
           
         
        \\Le temps du jugement, le cœur du sage le connaît,
${}^{6}car il y a un temps et un jugement pour tout,
        et le mal retombe sur l’homme.
${}^{7}Il ignore ce qui arrivera :
        qui pourrait lui en révéler le moment ?
        
           
         
${}^{8}Nul homme n’est maître de son souffle
        au point de retenir le souffle de sa vie.
        \\Nulle maîtrise sur le jour de la mort,
        nulle délégation pour ce combat :
        le crime ne sauve pas le criminel.
        
           
         
${}^{9}Tout cela, je l’ai vu,
        \\ayant à cœur de comprendre
        tout ce qui se fait sous le soleil,
        \\en ce temps où l’homme domine l’homme
        pour son malheur.
        
           
${}^{10}J’ai vu ainsi des criminels conduits en terre
        depuis la cité sainte ;
        \\on avait déjà oublié dans la ville
        comment ils avaient vécu.
        \\Cela aussi n’est que vanité !
         
${}^{11}Lorsque la sanction d’un méfait
        n’est pas immédiatement exécutée,
        \\l’envie de faire le mal
        monte au cœur des fils d’Adam.
${}^{12}Même si un pécheur commet cent fois le mal
        et continue de vivre,
        \\je sais, moi, que le bonheur
        sera pour ceux qui craignent Dieu,
        car ils craignent devant sa face.
${}^{13}Le bonheur ne sera pas pour le méchant ;
        il ne vivra pas de longs jours :
        \\il sera comme une ombre,
        lui qui ne craint pas devant la face de Dieu.
${}^{14}Encore un fait, une autre vanité sur la terre :
        \\des justes sont traités
        comme s’ils avaient agi en méchants,
        \\et des méchants sont traités
        comme s’ils avaient agi en justes.
        \\Je dis qu’il n’y a là que vanité.
         
${}^{15}Alors j’ai célébré la joie
        \\car il n’y a de bonheur pour l’homme sous le soleil
        que manger, boire et se réjouir,
        \\de quoi l’accompagner dans sa peine
        tous les jours de sa vie,
        les jours que Dieu lui donne sous le soleil.
${}^{16}Quand je m’appliquai à connaître la sagesse,
        considérant les travaux sur la terre
        qui empêchent de fermer l’œil jour et nuit,
${}^{17}alors j’ai vu : face à toute l’œuvre de Dieu,
        l’homme ne peut pas comprendre les œuvres
        qui se font sous le soleil.
        \\Plus l’homme se fatigue à chercher,
        moins il trouve.
        \\Même si le sage affirme savoir,
        il ne pourra pas trouver.
         
      
         
      \bchapter{}
${}^{1}Oui, j’ai réfléchi à tout cela,
        afin de tout tirer au clair,
        \\que les justes, les sages et leurs actions
        sont dans la main de Dieu.
        \\Même l’amour, même la haine,
        l’homme ne les comprend pas :
        pourtant tout est devant lui.
        
           
${}^{2}Tout est pareil pour tous ;
        il y a un même sort
        \\pour le juste et l’injuste,
        le bon et le mauvais,
        \\le pur et l’impur,
        pour qui sacrifie ou ne sacrifie pas.
        \\Il en va du bon comme du pécheur,
        de celui qui prête serment
        et de celui qui craint de le faire.
         
${}^{3}Le pire de ce qui advient sous le soleil
        est que tous ont le même sort.
        \\Alors le cœur des fils d’Adam
        s’emplit de méchanceté,
        \\la folie mène leur vie ;
        puis, ils s’en vont chez les morts.
${}^{4}Pour celui qui reste avec tous les vivants
        il y a de l’espoir :
        chien vivant vaut mieux que lion mort.
         
${}^{5}Car les vivants savent qu’ils mourront,
        mais les morts ne savent plus rien.
        \\Pour eux, plus de récompense,
        ils sont tombés dans l’oubli.
${}^{6}Leurs amours, leurs haines, leurs jalousies
        ont déjà disparu.
        \\Plus jamais ils n’auront part
        à tout ce qui se fait sous le soleil.
${}^{7}Va, mange avec plaisir ton pain
        et bois d’un cœur joyeux ton vin,
        car Dieu, déjà, prend plaisir à ce que tu fais.
${}^{8}Porte tes habits de fête en tout temps,
        n’oublie pas de te parfumer la tête.
${}^{9}Savoure la vie avec la femme que tu aimes,
        chaque jour de cette vie de vanité
        \\qui t’est donnée sous le soleil,
        tous ces jours de vanité…
        \\Voilà ton lot dans la vie
        et dans la peine que tu prends sous le soleil.
         
${}^{10}Tout ce que ta main trouve à faire,
        fais-le avec la force dont tu disposes,
        \\car il n’y a ni travaux, ni projets,
        ni science, ni sagesse
        au séjour des morts où tu vas.
${}^{11}J’ai regardé encore,
        \\et j’ai vu sous le soleil
        que la course n’est pas donnée aux plus rapides,
        \\ni aux braves la victoire,
        ni aux sages le pain,
        \\ni aux avisés la richesse,
        ni aux intelligents la faveur,
        \\car il y a pour chacun d’eux
        des temps et des contretemps.
         
${}^{12}L’homme ne connaît même pas son heure :
        \\comme le poisson pris au filet fatal,
        comme l’oiseau pris au piège,
        \\ainsi en est-il des fils d’Adam
        surpris par le moment fatal
        qui tombe sur eux à l’improviste.
         
${}^{13}J’ai vu aussi sous le soleil
        un exemple de sagesse qui m’a frappé.
${}^{14}Il y avait une petite ville, de peu d’habitants ;
        \\un roi puissant marcha contre elle,
        l’assiégea et l’encercla d’ouvrages fortifiés.
${}^{15}Or, se trouvait là un homme pauvre et sage
        qui sauva la ville par sa sagesse.
        \\Mais ensuite, plus personne ne pensa à cet homme pauvre.
         
${}^{16}Alors j’ai dit :
        Mieux vaut la sagesse que la force.
        \\Et pourtant, la sagesse du pauvre est méprisée,
        sa parole n’est pas écoutée.
         
${}^{17}La parole tranquille du sage est mieux écoutée
        que les éclats de voix d’un chef parmi les sots.
         
${}^{18}Mieux vaut la sagesse que les armes de guerre,
        mais un seul incapable peut ruiner des fortunes.
         
      
         
      \bchapter{}
${}^{1}Une seule mouche morte
        infeste et gâte l’huile du parfumeur.
        \\Un petit grain de folie
        pèse plus que sagesse et plus que gloire.
        
           
         
${}^{2}Le raisonnement du sage est droit,
        le raisonnement du fou va de travers.
${}^{3}Quand le fou marche sur la route,
        le bon sens lui fait défaut,
        et il dit aux autres qu’ils sont fous.
        
           
         
${}^{4}Si la colère du chef s’allume contre toi,
        ne quitte pas ton poste :
        le sang-froid fait éviter de grandes fautes.
        
           
         
${}^{5}Il est un mal que j’ai vu sous le soleil,
        comme une erreur échappée au souverain :
${}^{6}la folie placée au plus haut rang
        et des riches assis au plus bas.
${}^{7}J’ai vu des esclaves sur des chevaux,
        et des princes aller à pied comme des esclaves.
        
           
         
${}^{8}Qui creuse un trou tombera dedans ;
        qui perce un mur, un serpent le mordra.
${}^{9}Qui extrait des pierres s’y blessera ;
        qui fend du bois court un danger.
${}^{10}Si la lame est émoussée,
        que son tranchant n’est pas affûté,
        \\il faut redoubler d’efforts.
        Y remédier est sagesse.
${}^{11}Si le serpent mord, faute d’être charmé,
        il ne rapporte rien au charmeur.
        
           
         
${}^{12}Les paroles du sage le gratifient,
        les lèvres du fou le dévorent.
${}^{13}Il commence par débiter des sottises
        et termine en proférant des insanités.
${}^{14}Le fou multiplie ses prédictions.
        Mais l’homme ignore l’avenir :
        \\ce qui arrivera après lui,
        qui peut le lui révéler ?
${}^{15}Le travail du fou le fatigue,
        lui qui n’est même pas capable d’aller en ville.
        
           
${}^{16}Malheur à toi,
        pays dont le roi n’est qu’un enfant
        et dont les princes festoient dès l’aurore !
${}^{17}Heureux es-tu,
        pays dont le roi est de noble race
        \\et dont les princes mangent au moment voulu
        pour reprendre des forces et non pour s’enivrer !
         
${}^{18}Entre des mains paresseuses, la charpente s’écroule ;
        entre des mains nonchalantes, la maison prend l’eau.
${}^{19}Pour s’amuser, on fait un bon repas,
        le vin égaie la vie,
        l’argent permet tout.
         
${}^{20}Même en pensée, ne maudis pas le roi ;
        même en privé, ne maudis pas le riche,
        \\car un oiseau du ciel colporterait ta voix,
        tout ce qui vole divulguerait tes propos.
      
         
      \bchapter{}
${}^{1}Risque ta fortune sur les mers :
        après de longs jours, tu la retrouveras.
${}^{2}Divise ton bien en sept ou même huit parts :
        tu ne sais quel malheur peut arriver dans le pays.
        
           
         
${}^{3}Quand les nuages sont gorgés d’eau,
        ils déversent leur pluie sur la terre.
        \\Qu’un arbre tombe au nord ou au midi,
        là où il est tombé, il restera.
${}^{4}Qui attend le bon vent, jamais ne sèmera ;
        qui scrute les nuages, jamais ne moissonnera.
        
           
         
${}^{5}De même que tu ignores les routes du vent
        et comment se forment les os de l’enfant dans le ventre de la mère,
        \\de même tu ne comprends pas l’œuvre de Dieu qui fait toute chose.
        
           
         
${}^{6}Dès le matin, sème tes semences,
        et jusqu’au soir n’arrête pas ton ouvrage,
        \\car tu ignores ce qui réussira,
        ceci ou cela,
        \\ou si les deux ensemble seront bons.
        
           
${}^{7}Oui, douce est la lumière !
        Quel bonheur pour les yeux de voir le soleil !
         
${}^{8}L’homme vivrait-il de longues années,
        qu’il se réjouisse de chacune d’elles !
        \\Qu’il songe aussi aux jours de ténèbres,
        car ils seront nombreux.
        \\Tout ce qui arrive n’est que vanité.
         
        ${}^{9}Réjouis-toi, jeune homme, dans ton adolescence,
        et sois heureux aux jours de ta jeunesse.
        \\Suis les sentiers de ton cœur
        et les désirs de tes yeux !
        \\Mais sache que pour tout cela
        Dieu t’appellera en jugement.
         
        ${}^{10}Éloigne de ton cœur le chagrin,
        écarte de ta chair la souffrance
        \\car l’adolescence et le printemps de la vie
        ne sont que vanité.
      
         
      \bchapter{}
        ${}^{1}Souviens-toi de ton Créateur,
        aux jours de ta jeunesse,
        \\avant que viennent les jours mauvais,
        et qu’approchent les années dont tu diras :
        « Je ne les aime pas » ;
        ${}^{2}avant que s’obscurcissent le soleil et la lumière,
        la lune et les étoiles,
        \\et que reviennent les nuages après la pluie ;
        ${}^{3}au jour où tremblent les gardiens de la maison,
        où se courbent les hommes vigoureux ;
        \\où les femmes, l’une après l’autre, cessent de moudre,
        où le jour baisse aux fenêtres\\ ;
        ${}^{4}quand la porte se ferme sur la rue,
        quand s’éteint la voix de la meule,
        \\quand s’arrête\\le chant de l’oiseau,
        et quand se taisent les chansons ;
        ${}^{5}lorsqu’on redoute la montée
        et qu’on a des frayeurs en chemin ;
        \\l’amandier est en fleurs,
        la sauterelle s’alourdit,
        \\et la câpre ne produit aucun effet ;
        lorsque l’homme s’en va vers sa maison d’éternité,
        \\et que les pleureurs sont déjà au coin de la rue ;
        ${}^{6}avant que le fil d’argent se détache,
        que la lampe d’or se brise,
        \\que la cruche se casse à la fontaine,
        que la poulie se fende sur le puits ;
        ${}^{7}et que la poussière retourne à la terre
        comme elle en vint,
        \\et le souffle de vie\\, à Dieu qui l’a donné.
        
           
        ${}^{8}Vanité des vanités, disait Qohèleth,
        tout est vanité !
${}^{9}Qohèleth ne fut pas seulement un sage,
        il a encore enseigné son savoir au peuple.
        \\Il a pesé, scruté et serti lui-même
        un grand nombre de proverbes.
${}^{10}Qohèleth a cherché et trouvé de belles sentences,
        il a écrit des mots justes et vrais.
         
${}^{11}Les paroles des sages sont des aiguillons
        et les recueils de dictons, des clous bien plantés
        – présents d’un incomparable Berger.
         
${}^{12}Prends garde, mon fils, à n’y rien ajouter.
        Écrire de nombreux livres est une tâche sans fin.
        À trop étudier, le corps s’épuise.
         
${}^{13}Pour conclure ces paroles, et tout bien considéré,
        crains Dieu et observe ses commandements.
        Tout est là pour l’homme.
         
${}^{14}Dieu mettra en jugement toutes les actions,
        tout ce qui est caché, bon ou mauvais.
