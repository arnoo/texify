  
  
${}^{31}Le Seigneur vit que Léa n’était pas aimée et il la rendit féconde tandis que Rachel était stérile. 
${}^{32}Léa devint enceinte et enfanta un fils qu’elle appela Roubène car, dit-elle, « le Seigneur a vu ma détresse et maintenant mon mari m’aimera ». 
${}^{33}Elle devint encore enceinte et enfanta un fils. Elle dit : « Le Seigneur a compris que je n’étais pas aimée et il m’a encore donné cet enfant ! » Elle l’appela Siméon. 
${}^{34}Elle devint encore enceinte et enfanta un fils. Elle dit : « Maintenant, cette fois-ci, mon mari va s’attacher à moi car je lui ai donné trois fils ! » C’est pourquoi on l’appela Lévi. 
${}^{35}Elle devint encore enceinte et enfanta un fils. Elle dit : « Cette fois-ci, je louerai le Seigneur ! » C’est pourquoi elle l’appela Juda. Ensuite, elle n’eut plus d’enfant.
      
         
      \bchapter{}
      \begin{verse}
${}^{1}Voyant qu’elle n’avait pas donné d’enfant à Jacob, Rachel devint jalouse de sa sœur. Elle dit à Jacob : « Donne-moi des fils, sinon je vais mourir ! » 
${}^{2}Jacob s’enflamma de colère contre Rachel et dit : « Suis-je à la place de Dieu, moi ? C’est lui qui t’a empêchée d’avoir des enfants. » 
${}^{3}Rachel reprit : « Voici ma servante Bilha, unis-toi à elle pour qu’elle enfante sur mes genoux ; ainsi, grâce à elle, j’aurai un fils, moi aussi. » 
${}^{4}Elle lui donna donc pour femme sa servante Bilha, et Jacob s’unit à elle. 
${}^{5}Bilha devint enceinte et enfanta un fils à Jacob. 
${}^{6}Rachel dit : « Dieu m’a rendu justice, il a écouté ma voix, il m’a donné un fils. » C’est pourquoi elle l’appela Dane. 
${}^{7}Bilha, la servante de Rachel, devint encore enceinte et elle enfanta un second fils à Jacob. 
${}^{8}Rachel dit : « J’ai livré contre ma sœur des combats de Dieu et je l’ai emporté ! » Elle appela donc l’enfant Nephtali.
${}^{9}Quand Léa s’aperçut qu’elle avait cessé d’enfanter, elle prit sa servante Zilpa et la donna pour femme à Jacob. 
${}^{10}Zilpa, la servante de Léa, enfanta un fils à Jacob. 
${}^{11}Léa dit : « Quelle chance ! » Et elle l’appela Gad. 
${}^{12}Zilpa, la servante de Léa, enfanta un second fils à Jacob. 
${}^{13}Léa dit : « Quel bonheur pour moi ! Les filles me proclament bienheureuse ! » Et elle appela l’enfant Asher.
${}^{14}Au temps de la moisson des blés, Roubène s’en alla dans les champs pour y chercher des mandragores. Il les apporta à Léa, sa mère, et Rachel dit à Léa : « Donne-moi donc les mandragores de ton fils. » 
${}^{15}Léa répondit : « Ne te suffit-il pas de m’avoir pris mon mari que tu veuilles aussi les mandragores de mon fils ? » Alors Rachel dit : « Eh bien ! Que Jacob couche avec toi, cette nuit, en échange des mandragores de ton fils. » 
${}^{16}Le soir, quand Jacob revint des champs, Léa sortit à sa rencontre et dit : « Viens donc, car c’est toi mon cadeau en échange des mandragores de mon fils. » Il coucha donc avec elle, cette nuit-là. 
${}^{17}Dieu exauça Léa : elle devint enceinte et enfanta un cinquième fils à Jacob. 
${}^{18}Léa dit : « Dieu m’a donné un cadeau, parce que j’ai donné ma servante à mon mari. » Et elle appela l’enfant Issakar. 
${}^{19}Léa devint encore enceinte et enfanta un sixième fils à Jacob. 
${}^{20}Léa dit : « Dieu m’a fait un beau présent ! Cette fois-ci, mon mari m’estimera car je lui ai donné six fils ! » Et elle appela l’enfant Zabulon. 
${}^{21}Ensuite, elle enfanta une fille qu’elle appela Dina.
${}^{22}Dieu se souvint de Rachel, il l’exauça et la rendit féconde. 
${}^{23}Elle devint enceinte et enfanta un fils. Elle dit : « Dieu a enlevé ma honte. » 
${}^{24}Elle appela l’enfant Joseph, en disant : « Que le Seigneur m’ajoute un autre fils ! »
${}^{25}Lorsque Rachel eut enfanté Joseph, Jacob dit à Laban : « Laisse-moi partir ; je vais retourner chez moi, dans mon pays. 
${}^{26}Donne-moi mes femmes, pour lesquelles je t’ai servi, ainsi que mes enfants ; je vais partir. Tu sais combien j’ai travaillé à ton service. » 
${}^{27}Laban lui dit : « Que je puisse trouver grâce à tes yeux ! J’ai appris par divination que le Seigneur m’a béni à cause de toi. » 
${}^{28}Puis il ajouta : « Fixe-moi ton salaire et je te le donnerai. » 
${}^{29}Jacob dit : « Tu sais combien j’ai travaillé à ton service et ce que ton bétail est devenu grâce à moi. 
${}^{30}Il représentait peu de chose avant moi, mais il a considérablement augmenté : le Seigneur t’a béni depuis que je suis là. Et moi, maintenant, quand vais-je travailler pour ma maison ? » 
${}^{31}Laban dit : « Que vais-je te donner ? » Jacob répondit : « Tu ne me donneras rien, et si tu fais pour moi ce que je vais te dire, je recommencerai à faire paître ton petit bétail et à le garder. 
${}^{32}Aujourd’hui, je vais passer au milieu de tout ton petit bétail et je mettrai à l’écart tout mouton tacheté ou moucheté, tout mouton brun parmi les jeunes béliers, et toute bête mouchetée ou tachetée parmi les chèvres : ce sera mon salaire. 
${}^{33}Plus tard, mon honnêteté répondra pour moi quand tu viendras vérifier mon salaire : tout ce qui, en ma possession, ne sera pas moucheté ou tacheté parmi les chèvres et brun parmi les jeunes béliers sera considéré comme un vol de ma part. » 
${}^{34}Et Laban conclut : « C’est bien ! Qu’il en soit comme tu l’as dit. »
${}^{35}Le jour même, Laban mit à l’écart les boucs rayés ou mouchetés, ainsi que toutes les chèvres tachetées ou mouchetées, tout ce qui portait des taches blanches ou était brun parmi les jeunes béliers, et il les confia à ses fils. 
${}^{36}Puis il mit une distance de trois jours de marche entre lui et Jacob, tandis que Jacob faisait paître le reste du troupeau de Laban.
${}^{37}Jacob se procura de fraîches baguettes de peuplier, d’amandier, de platane. Il les écorça, faisant apparaître des raies blanches, mettant à nu le blanc qui était sous l’écorce des baguettes. 
${}^{38}Puis il plaça les baguettes ainsi écorcées dans les auges, dans les abreuvoirs où le petit bétail venait boire ; les baguettes étaient devant les yeux des bêtes qui entraient en chaleur quand elles venaient boire. 
${}^{39}Les bêtes qui entraient en chaleur devant les baguettes mettaient bas des petits rayés, tachetés, mouchetés. 
${}^{40}Quant aux jeunes béliers, Jacob les mit à part et il tourna les bêtes vers ce qui était rayé ou tout ce qui était brun dans le troupeau de Laban. Ainsi il se constitua des troupeaux pour lui seul et ne les ajouta pas au petit bétail de Laban. 
${}^{41}Chaque fois que les bêtes les plus vigoureuses s’accouplaient, Jacob mettait les baguettes dans les auges, sous leurs yeux, pour qu’elles s’accouplent devant les baguettes. 
${}^{42}Mais quand les bêtes étaient chétives, il ne mettait pas de baguettes, si bien que les chétives étaient pour Laban et les vigoureuses pour Jacob.
${}^{43}Ainsi l’homme déborda de richesses : il posséda du petit bétail en grand nombre, et des servantes, des serviteurs, des chameaux, des ânes.
      
         
      \bchapter{}
      \begin{verse}
${}^{1}Jacob entendit parler les fils de Laban qui disaient : « Jacob a pris tout ce qui appartenait à notre père, c’est à partir des biens de notre père qu’il a bâti toute sa fortune. » 
${}^{2}Jacob observa le visage de Laban et constata qu’il ne se comportait plus vis-à-vis de lui comme auparavant.
${}^{3}Le Seigneur dit à Jacob : « Retourne au pays de tes pères, dans ta parenté. Je serai avec toi. » 
${}^{4}Jacob fit appeler Rachel et Léa pour qu’elles le rejoignent dans le champ où il se trouvait avec le bétail. 
${}^{5}Et il leur dit : « J’ai vu sur le visage de votre père qu’il ne se comportait plus avec moi comme auparavant. Mais le Dieu de mon père a été avec moi. 
${}^{6}Vous, vous le savez bien, j’ai servi votre père de toutes mes forces, 
${}^{7}et votre père s’est joué de moi, il a changé dix fois mon salaire, mais Dieu ne lui a pas permis de me faire du mal. 
${}^{8}Ainsi, quand il disait : “Tu auras pour salaire les bêtes tachetées”, toutes les brebis mettaient bas des tachetées, et quand il disait : “Tu auras pour salaire les rayées”, toutes les brebis mettaient bas des rayées. 
${}^{9}C’est ainsi que Dieu a enlevé son troupeau à votre père et me l’a donné. 
${}^{10}Au temps où les brebis entraient en chaleur, je levai les yeux et je vis en songe des béliers qui s’accouplaient aux brebis, tous rayés, tachetés, tavelés. 
${}^{11}Et l’ange de Dieu me dit en songe : “Jacob !” Je répondis : “Me voici”. 
${}^{12}Il reprit : “Lève donc les yeux et regarde les béliers qui s’accouplent aux brebis, tous rayés, tachetés, tavelés. C’est parce que j’ai vu tout ce que t’a fait subir Laban. 
${}^{13}Je suis le Dieu de Béthel, là où tu as fait l’onction sur une stèle et où tu t’es engagé envers moi par un vœu. Maintenant, lève-toi, quitte ce pays et retourne dans le pays de ta parenté.” »
${}^{14}Rachel et Léa lui firent cette réponse : « Avons-nous encore une part, un héritage, dans la maison de notre père ? 
${}^{15}Ne nous a-t-il pas traitées comme des étrangères puisqu’il nous a vendues et qu’il a bel et bien mangé notre argent ? 
${}^{16}Maintenant, puisque toutes les richesses que Dieu a enlevées à notre père sont à nous et à nos fils, fais donc tout ce que Dieu t’a dit. »
${}^{17}Alors, Jacob se leva et fit monter ses fils et ses femmes sur les chameaux. 
${}^{18}Il emmena aussi tous ses troupeaux et tous les biens qu’il avait acquis – le troupeau qu’il avait acquis en Paddane-Aram – pour retourner chez son père Isaac, au pays de Canaan. 
${}^{19}Tandis que Laban était allé tondre son petit bétail, Rachel déroba les idoles domestiques qui appartenaient à son père. 
${}^{20}Et Jacob se déroba à la vigilance de Laban l’Araméen, en s’enfuyant sans le prévenir. 
${}^{21}Il s’enfuit donc avec tout ce qu’il possédait, il se leva, traversa l’Euphrate et se dirigea vers la montagne de Galaad.
${}^{22}Le troisième jour, on avertit Laban de la fuite de Jacob. 
${}^{23}Laban prit avec lui ses frères et poursuivit Jacob pendant sept jours de marche. Il le rejoignit à la montagne de Galaad. 
${}^{24}Pendant la nuit, Dieu vint trouver Laban l’Araméen dans un songe et lui dit : « Garde-toi de dire le moindre mot à Jacob, en bien ou en mal. »
${}^{25}Laban rattrapa Jacob qui avait planté sa tente dans la montagne ; Laban et ses frères plantèrent la leur dans la montagne de Galaad. 
${}^{26}Laban dit à Jacob : « Qu’as-tu fait ? Tu t’es dérobé à ma vigilance, tu as emmené mes filles comme des captives de guerre ! 
${}^{27}Pourquoi t’es-tu caché pour fuir ? Tu m’as volé ! Tu ne m’as pas prévenu ! Je t’aurais laissé partir dans la joie et les chants, au son du tambourin et de la cithare. 
${}^{28}Mais tu ne m’as pas laissé embrasser mes fils et mes filles ! Tu te comportes vraiment comme un fou ! 
${}^{29}J’ai entre les mains le pouvoir de vous faire du mal, mais, hier soir, le Dieu de votre père m’a adressé cette parole : “Garde-toi de dire le moindre mot à Jacob, en bien ou en mal.” 
${}^{30}Maintenant, si tu es parti car tu désirais ardemment retrouver la maison de ton père, pourquoi as-tu dérobé mes dieux ? »
${}^{31}Jacob répondit à Laban : « C’est que j’ai eu peur ; je me disais que tu voudrais peut-être me reprendre de force tes filles ! 
${}^{32}Quant à celui chez qui tu trouveras tes dieux, il perdra la vie. Devant nos frères, inspecte mes affaires et reprends ce qui est à toi. » Jacob, en effet, ne savait pas que Rachel avait dérobé les idoles domestiques. 
${}^{33}Laban entra dans la tente de Jacob, la tente de Léa et celle des deux servantes. Il ne trouva rien. Il sortit de la tente de Léa et entra dans la tente de Rachel. 
${}^{34}Or, Rachel avait pris les idoles, les avait mises dans le sac dont on charge le chameau et s’était assise dessus. Laban fouilla toute la tente mais il ne trouva rien. 
${}^{35}Rachel dit à son père : « Que les yeux de mon seigneur ne s’enflamment pas de colère ! En effet, je ne peux pas me lever devant toi car j’ai ce qui arrive aux femmes. » Il chercha partout mais ne trouva pas les idoles.
${}^{36}Jacob s’enflamma de colère et prit Laban à partie. Il s’écria : « Quel crime ai-je fait ? Quelle faute ai-je commise pour que tu t’acharnes contre moi ? 
${}^{37}Tu as fouillé toutes mes affaires : as-tu trouvé un seul objet qui provienne de ta maison ? Expose-le ici même, devant mes frères et tes frères, et qu’ils nous départagent ! 
${}^{38}Voici vingt ans que je suis avec toi : tes brebis et tes chèvres n’ont pas avorté ; les béliers de ton troupeau, je ne les ai pas mangés ! 
${}^{39}La bête déchirée, je ne te la rapportais pas, c’est moi qui en subissais le dommage ; et la bête volée le jour ou la nuit, tu me la réclamais ! 
${}^{40}J’étais là, le jour, quand la chaleur me dévorait et la nuit, quand le froid me glaçait. Le sommeil me fuyait ! 
${}^{41}Voici vingt ans que je suis dans ta maison. Je t’ai servi quatorze ans pour tes deux filles, six ans pour ton troupeau. Et, dix fois, tu as changé mon salaire ! 
${}^{42}Si le Dieu de mon père, le Dieu d’Abraham, l’Effroi d’Isaac, n’avait pas été avec moi, tu m’aurais, maintenant, renvoyé les mains vides. Mais Dieu a vu ma misère et la fatigue de mes mains. Hier soir, il s’est prononcé ! »
${}^{43}Laban répondit à Jacob : « Ces filles ? Ce sont mes filles ! Ces fils ? mes fils ! Ces troupeaux ? mes troupeaux ! Tout ce que tu vois, c’est à moi ! Que ne ferai-je, aujourd’hui, pour mes filles et pour les fils qu’elles ont enfantés ? 
${}^{44}Maintenant donc, allons, concluons une alliance, moi et toi, et qu’il y ait un témoin entre moi et toi ! »
${}^{45}Jacob prit une pierre et l’érigea en stèle. 
${}^{46}Puis il dit à ses frères : « Ramassez des pierres ! » Ils prirent des pierres et en firent un monticule. Ils mangèrent là, sur le monticule. 
${}^{47}Laban l’appela Yegar-Sahadouta et Jacob, Galéed.
${}^{48}Laban déclara : « Ce monticule est aujourd’hui témoin entre moi et toi. » C’est pourquoi on l’a appelé Galéed. 
${}^{49}On l’appela aussi Mispa (c’est-à-dire : Poste de guet), car Laban avait ajouté : « Que le Seigneur fasse le guet entre moi et toi quand nous serons hors de vue l’un de l’autre. 
${}^{50}Si tu humilies mes filles, si tu prends des femmes en plus de mes filles, ce ne sera pas un homme comme nous mais bien Dieu lui-même qui sera témoin entre moi et toi ! »
${}^{51}Laban déclara encore à Jacob : « Voici ce monticule et voici la stèle que j’ai élevée entre moi et toi. 
${}^{52}Témoin sera ce monticule, et témoin la stèle : je ne passerai pas de ton côté au-delà de ce monticule, tu ne passeras pas de mon côté au-delà de ce monticule et de la stèle, avec de mauvaises intentions ! 
${}^{53}Au Dieu d’Abraham et au Dieu de Nahor – le Dieu de leur père – de juger entre nous ! » Alors Jacob prêta serment par l’Effroi d’Isaac, son père. 
${}^{54}Puis Jacob offrit un sacrifice sur la montagne et invita ses frères à manger le pain. Ils mangèrent le pain et passèrent la nuit sur la montagne.
      
         
      \bchapter{}
      \begin{verse}
${}^{1}Laban se leva de bon matin, embrassa ses fils et ses filles, il les bénit, puis s’en retourna chez lui. 
${}^{2}Jacob se mit en route, et des anges de Dieu vinrent à sa rencontre. 
${}^{3}À leur vue, Jacob déclara : « C’est le camp de Dieu ! » Et il appela ce lieu « Mahanaïm » (c’est-à-dire : Les deux camps).
      
         
${}^{4}Jacob envoya des messagers devant lui vers son frère Ésaü, au pays de Séïr, dans la campagne d’Édom. 
${}^{5}Il leur donna cet ordre : « Vous parlerez ainsi à mon seigneur, à Ésaü : “Ainsi parle ton serviteur Jacob : J’ai séjourné chez Laban comme un immigré et j’y suis resté jusqu’à présent. 
${}^{6}Je possède des bœufs et des ânes, du petit bétail, des serviteurs et des servantes. J’en ai fait informer mon seigneur, afin de trouver grâce à tes yeux.” »
${}^{7}Les messagers revinrent vers Jacob et dirent : « Nous avons trouvé ton frère Ésaü. Lui aussi marche à ta rencontre et, avec lui, quatre cents hommes. »
${}^{8}Jacob eut très peur et l’angoisse le saisit. Il partagea en deux camps les gens qui étaient avec lui, le petit et le gros bétail ainsi que les chameaux. 
${}^{9}Il se disait : « Si Ésaü attaque un camp et le saccage, l’autre camp pourra en réchapper. »
${}^{10}Jacob dit alors : « Dieu de mon père Abraham, Dieu de mon père Isaac, ô Seigneur, toi qui m’as dit : “Retourne dans ton pays, dans ta parenté, et je te ferai du bien”, 
${}^{11}je suis trop petit pour toutes les faveurs et toute la fidélité que tu as prodiguées à ton serviteur, car je n’avais que mon bâton quand j’ai traversé ce Jourdain, et maintenant je suis à la tête de deux camps. 
${}^{12}Délivre-moi donc de la main de mon frère, de la main d’Ésaü, car j’ai peur de lui, j’ai peur qu’il ne vienne, me frappe et frappe la mère avec les fils. 
${}^{13}Or, toi, tu m’as dit : “Je te comblerai de bienfaits, je rendrai ta descendance comme le sable de la mer qu’on ne peut dénombrer, tant il y en a !” »
${}^{14}Jacob passa la nuit à cet endroit, puis, sur ce qu’il avait acquis, il préleva un présent pour son frère Ésaü : 
${}^{15}deux cents chèvres, vingt boucs, deux cents brebis et vingt béliers, 
${}^{16}trente chamelles laitières avec leurs petits, quarante vaches et dix taureaux, vingt ânesses et dix ânons. 
${}^{17}Il confia les bêtes aux mains de ses serviteurs, troupeau par troupeau, et leur dit : « Passez devant moi, et vous laisserez un espace entre chaque troupeau. » 
${}^{18}Puis il donna cet ordre au premier serviteur : « Quand mon frère Ésaü te rencontrera et te demandera : “À qui appartiens-tu ? Où vas-tu ? À qui appartient ce troupeau qui est là devant toi ?”, 
${}^{19}tu répondras : “À ton serviteur, à Jacob. C’est un présent qu’il envoie à mon seigneur, à Ésaü ; et le voici, lui-même, derrière nous.” » 
${}^{20}Il donna le même ordre au deuxième, puis au troisième, bref à tous ceux qui marchaient derrière les troupeaux. Il disait : « C’est en ces termes que vous parlerez à Ésaü quand vous le trouverez. 
${}^{21}Vous direz : “Voici que ton serviteur Jacob arrive derrière nous”. » En effet, Jacob disait : « Je l’apaiserai avec le présent qui m’aura précédé ; ensuite je pourrai paraître devant sa face. Peut-être me fera-t-il bon accueil. » 
${}^{22}Ainsi le présent précéda Jacob. Et lui-même passa la nuit au camp.
${}^{23}Cette nuit-là, Jacob se leva, il prit ses deux femmes, ses deux servantes, ses onze enfants, et passa le gué du Yabboq. 
${}^{24}Il leur fit passer le torrent et fit aussi passer ce qui lui appartenait. 
${}^{25}Jacob resta seul. Or, quelqu’un lutta avec lui jusqu’au lever de l’aurore. 
${}^{26}L’homme, voyant qu’il ne pouvait rien contre lui, le frappa au creux de la hanche, et la hanche de Jacob se démit pendant ce combat. 
${}^{27}L’homme dit : « Lâche-moi, car l’aurore s’est levée. » Jacob répondit : « Je ne te lâcherai que si tu me bénis. » 
${}^{28}L’homme demanda : « Quel est ton nom ? » Il répondit : « Jacob. » 
${}^{29}Il reprit : « Ton nom ne sera plus Jacob, mais Israël (c’est-à-dire : Dieu lutte)\\, parce que tu as lutté avec Dieu et avec des hommes, et tu l’as emporté. » 
${}^{30}Jacob demanda : « Fais-moi connaître ton nom, je t’en prie. » Mais il répondit : « Pourquoi me demandes-tu mon nom ? » Et là il le bénit.
${}^{31}Jacob appela ce lieu Penouël (c’est-à-dire : Face de Dieu)\\, car, disait-il, « j’ai vu Dieu face à face, et j’ai eu la vie sauve ». 
${}^{32}Au lever du soleil, il passa le torrent\\à Penouël. Il resta boiteux de la hanche. 
${}^{33}C’est pourquoi, aujourd’hui encore, les fils d’Israël ne mangent pas le muscle qui est au creux de la hanche, car c’est là que Jacob avait été touché.
      
         
      \bchapter{}
      \begin{verse}
${}^{1}Jacob leva les yeux. Il vit qu’Ésaü arrivait, et avec lui quatre cents hommes. Il répartit alors les enfants entre Léa, Rachel et les deux servantes. 
${}^{2}En tête, il mit les servantes et leurs enfants, puis Léa et ses enfants, et derrière, Rachel et Joseph. 
${}^{3}Quant à lui, il passa devant eux et il se prosterna sept fois, face contre terre, avant d’aborder son frère. 
${}^{4}Ésaü courut à sa rencontre, l’étreignit, se jeta à son cou, l’embrassa, et tous deux pleurèrent. 
${}^{5}Ésaü leva les yeux, vit les femmes et les enfants, et dit : « Qui sont ceux-là pour toi ? » Jacob répondit : « Ce sont les enfants que Dieu a accordés à ton serviteur. » 
${}^{6}Alors les servantes s’avancèrent avec leurs enfants et se prosternèrent. 
${}^{7}Puis s’avança Léa avec ses enfants, et ils se prosternèrent. Enfin s’avancèrent Joseph et Rachel, et ils se prosternèrent.
${}^{8}Ésaü reprit : « Qu’est-ce que toute cette troupe que j’ai rencontrée ? » Jacob répondit : « C’est pour trouver grâce aux yeux de mon seigneur. »
${}^{9}Ésaü dit : « J’ai largement ce qu’il me faut, mon frère. Garde pour toi ce qui est à toi. » 
${}^{10}Jacob répondit : « Oh que non ! Si j’ai trouvé grâce à tes yeux, de ma main tu accepteras mon présent. En effet, j’ai pu paraître devant ta face comme on paraît devant la face de Dieu, et tu t’es montré bienveillant envers moi. 
${}^{11}Accepte donc le présent que je t’ai apporté. Car Dieu m’a fait grâce et j’ai tout ce qu’il me faut. » Il insista auprès de lui et celui-ci accepta.
${}^{12}Ésaü dit : « Levons le camp ! En route ! Je marcherai en tête. » 
${}^{13}Mais Jacob répondit : « Mon seigneur sait que les enfants sont fragiles, et j’ai à ma charge des brebis et des vaches qui allaitent. Si on presse l’allure un seul jour, tout le petit bétail meurt ! 
${}^{14}Que mon seigneur passe donc devant son serviteur, mais moi, je cheminerai tranquillement, au pas du convoi qui me précédera et au pas des enfants, jusqu’à ce que j’arrive chez mon seigneur, en Séïr. » 
${}^{15}Ésaü dit : « Je vais laisser auprès de toi quelques-uns des gens qui m’accompagnent. » Jacob répondit : « À quoi bon ? Pourvu que je trouve grâce aux yeux de mon seigneur ! » 
${}^{16}Ce jour même, Ésaü reprit son chemin vers Séïr. 
${}^{17}Jacob, lui, partit pour Souccoth où il se bâtit une maison et fit des huttes pour son troupeau. C’est pourquoi on appela cet endroit « Souccoth » (c’est-à-dire : Huttes).
${}^{18}Venant de Paddane-Aram, Jacob arriva sain et sauf à la ville de Sichem, au pays de Canaan, et il campa en face de la ville. 
${}^{19}Pour cent pièces d’argent, il acheta aux fils de Hamor, père de Sichem, la parcelle de champ où il avait dressé sa tente. 
${}^{20}Là, il érigea un autel qu’il appela « El, Dieu d’Israël ».
      
         
      \bchapter{}
      \begin{verse}
${}^{1}Dina, la fille que Léa avait enfantée à Jacob, sortit pour aller voir les filles du pays. 
${}^{2}Sichem, fils de Hamor le Hivvite, chef du pays, la vit, l’enleva, coucha avec elle et la viola. 
${}^{3}Alors, de tout son être, il s’attacha à Dina, la fille de Jacob, il aima la jeune fille et ses paroles touchèrent le cœur de celle-ci. 
${}^{4}Sichem dit à Hamor, son père : « Demande pour moi cette enfant, qu’elle devienne ma femme. » 
${}^{5}Or, Jacob apprit que Sichem avait souillé sa fille Dina ; mais ses fils étaient aux champs avec le troupeau, et il garda le silence jusqu’à leur retour.
${}^{6}Hamor, le père de Sichem, sortit parler à Jacob. 
${}^{7}Revenus des champs, les fils de Jacob apprirent ce qui s’était passé. Ces hommes se sentirent outragés et furent pris d’une grande colère car Sichem avait commis une infamie envers Israël en couchant avec la fille de Jacob : c’est une chose qui ne se fait pas. 
${}^{8}Hamor leur parla en ces termes : « Mon fils Sichem est épris de votre fille de tout son être, veuillez donc la lui donner pour femme. 
${}^{9}Alliez-vous avec nous par mariage : vous nous donnerez vos filles et vous prendrez pour vous les nôtres. 
${}^{10}Vous habiterez avec nous ; le pays est ouvert devant vous : vous pouvez y habiter, le parcourir et y avoir des propriétés. »
${}^{11}Sichem s’adressa au père et aux frères de la jeune fille : « Que je trouve grâce à vos yeux ! Ce que vous me demanderez, je le donnerai. 
${}^{12}Imposez-moi une grosse somme pour prix de la jeune fille et de nombreux cadeaux ; je paierai ce que vous me demanderez, mais donnez-moi la jeune fille pour femme. »
${}^{13}Les fils de Jacob répondirent à Sichem et à son père Hamor, mais ils parlèrent avec ruse car Sichem avait souillé leur sœur Dina. 
${}^{14}Ils dirent : « Nous ne pouvons accepter cette proposition. Donner notre sœur à un homme incirconcis, ce serait pour nous une honte ! 
${}^{15}Nous ne serons d’accord avec vous que si vous devenez comme nous en faisant circoncire tous vos mâles. 
${}^{16}Alors nous vous donnerons nos filles et nous prendrons pour nous les vôtres, nous habiterons avec vous et nous formerons un seul peuple. 
${}^{17}Mais si vous refusez la circoncision, nous reprendrons notre fille et nous partirons. » 
${}^{18}Leurs paroles plurent à Hamor et à son fils Sichem. 
${}^{19}Le jeune homme ne tarda pas à réaliser ce qui avait été proposé, tant il désirait la fille de Jacob. Il était l’homme le plus influent dans la maison de son père.
${}^{20}Hamor et son fils Sichem allèrent à la porte de leur ville et parlèrent ainsi à leurs concitoyens : 
${}^{21}« Ces gens ont des intentions pacifiques envers nous ! Qu’ils habitent donc le pays, qu’ils le parcourent, que le pays soit largement ouvert devant eux. Nous, nous prendrons leurs filles pour femmes et nous leur donnerons les nôtres. 
${}^{22}Toutefois, ces gens ne consentiront à habiter avec nous pour former un seul peuple que si tous nos mâles sont circoncis comme ils le sont eux-mêmes. 
${}^{23}Leurs troupeaux, leurs biens, tout leur bétail ne seront-ils pas à nous, si nous consentons seulement à cela, pour qu’ils puissent habiter avec nous ? » 
${}^{24}Tous ceux qui sortaient par la porte de la ville entendirent Hamor et son fils Sichem ; et tous les mâles furent circoncis.
${}^{25}Or, le troisième jour, alors que les hommes étaient souffrants, deux des fils de Jacob, Siméon et Lévi, qui étaient frères de Dina, prirent chacun leur épée, entrèrent dans la ville en toute sécurité et tuèrent tous les mâles. 
${}^{26}Ils passèrent au fil de l’épée Hamor et son fils Sichem, ils reprirent Dina dans la maison de Sichem et ressortirent. 
${}^{27}Les fils de Jacob se jetèrent sur les victimes et pillèrent la ville, parce qu’on avait souillé leur sœur. 
${}^{28}Ils s’emparèrent de leur petit bétail, de leur gros bétail, de leurs ânes, de tout ce qui se trouvait dans la ville et dans les champs. 
${}^{29}Ils ramenèrent au camp toutes leurs richesses, tous leurs jeunes enfants et leurs femmes, et ils pillèrent toutes les maisons de fond en comble. 
${}^{30}Jacob dit à Simon et à Lévi : « Vous avez fait mon malheur en me rendant odieux aux habitants du pays, Cananéens et Perizzites. Moi, je n’ai qu’un petit nombre d’hommes. Ils vont s’unir contre moi et m’abattre. Nous serons exterminés, moi et ma maison. » 
${}^{31}Mais ils répliquèrent : « Devions-nous laisser traiter notre sœur comme une prostituée ? »
      
         
      \bchapter{}
      \begin{verse}
${}^{1}Dieu dit à Jacob : « Debout ! Monte à Béthel et fixe-toi là-bas. Tu y feras un autel au Dieu qui t’est apparu lorsque tu fuyais devant ton frère Ésaü. » 
${}^{2}Jacob dit à sa famille et à tous ceux qui l’accompagnaient : « Enlevez les dieux étrangers qui sont au milieu de vous, purifiez-vous et changez de vêtements. 
${}^{3}Debout ! Montons à Béthel. J’y ferai un autel au Dieu qui m’a répondu au jour de ma détresse, qui a été avec moi sur le chemin où je marchais. » 
${}^{4}Ils remirent à Jacob tous les dieux étrangers qu’ils possédaient et les anneaux qu’ils portaient aux oreilles. Jacob les enfouit sous le térébinthe qui est près de Sichem. 
${}^{5}Ils levèrent le camp, et une épouvante divine s’abattit sur les villes d’alentour. Ainsi personne ne poursuivit les fils de Jacob.
${}^{6}Jacob arriva avec tous les gens qui l’accompagnaient à Louz, au pays de Canaan : c’est Béthel (c’est-à-dire : Maison de Dieu). 
${}^{7}Là, il bâtit un autel et il appela cet endroit « El-Béthel » (c’est-à-dire : Dieu-de-Béthel). Car c’est là que Dieu s’était révélé à lui quand il fuyait devant son frère. 
${}^{8}Alors mourut Débora, la nourrice de Rébecca, et on l’enterra en dessous de Béthel, au pied d’un chêne qu’on appela « Alone-Bakouth » (c’est-à-dire : le Chêne-des-pleurs).
${}^{9}Dieu apparut encore à Jacob quand celui-ci arriva de Paddane-Aram, et il le bénit. 
${}^{10}Dieu lui dit : « Ton nom est Jacob, mais on ne t’appellera plus du nom de Jacob, ton nom sera Israël. » Et il l’appela du nom d’Israël.
${}^{11}Dieu lui dit encore : « Je suis le Dieu-Puissant. Sois fécond, multiplie-toi ! Une nation – et même une assemblée de nations – sera issue de toi, des rois sortiront de toi. 
${}^{12}La terre que j’ai donnée à Abraham et à Isaac, je te la donne, et je la donnerai à ta descendance après toi. »
${}^{13}Alors, au lieu même où il lui avait parlé, Dieu s’éleva loin de Jacob.
${}^{14}Jacob érigea une stèle en ce lieu où Dieu avait parlé avec lui : une stèle de pierre. Sur elle, il versa une libation et répandit de l’huile. 
${}^{15}Et Jacob appela « Béthel » ce lieu où Dieu avait parlé avec lui.
${}^{16}Ils levèrent le camp et quittèrent Béthel. Il restait à parcourir une certaine distance pour arriver à Éphrata, quand Rachel accoucha. Et ses couches furent pénibles. 
${}^{17}Au cours de cet accouchement difficile, la sage-femme lui dit : « N’aie pas peur ! Tu as encore un fils ! » 
${}^{18}Dans son dernier souffle, au moment de mourir, Rachel l’appela Ben-Oni (c’est-à-dire : Fils-du-deuil) ; mais son père l’appela Benjamin (c’est-à-dire : Fils-de-la-droite). 
${}^{19}Rachel mourut et on l’enterra sur la route d’Éphrata, c’est-à-dire Bethléem. 
${}^{20}Jacob érigea une stèle sur sa tombe. Aujourd’hui encore, c’est la stèle de la tombe de Rachel.
${}^{21}Israël leva le camp et planta sa tente au-delà de Migdal-Éder. 
${}^{22}Or, tandis qu’Israël demeurait dans cette région, Roubène alla coucher avec Bilha, concubine de son père, et Israël l’apprit.
${}^{23}Les fils de Jacob furent au nombre de douze. Les fils de Léa : Roubène, le premier-né de Jacob, Siméon, Lévi, Juda, Issakar et Zabulon. 
${}^{24}Les fils de Rachel : Joseph et Benjamin. 
${}^{25}Les fils de Bilha, la servante de Rachel : Dane et Nephtali. 
${}^{26}Les fils de Zilpa, la servante de Léa : Gad et Asher. Tels sont les fils de Jacob qui lui naquirent en Paddane-Aram.
${}^{27}Jacob arriva chez son père Isaac à Mambré, à Qiryath-ha-Arba, c’est-à-dire Hébron, là où Abraham et Isaac avaient séjourné comme des immigrés. 
${}^{28}Isaac vécut cent quatre-vingts ans, 
${}^{29}puis il expira. Il mourut et fut réuni aux siens, âgé et rassasié de jours. Ésaü et Jacob, ses fils, l’ensevelirent.
      
         
      \bchapter{}
      \begin{verse}
${}^{1}Voici la descendance d’Ésaü, autrement dit Édom. 
${}^{2}Ésaü choisit ses femmes parmi les filles de Canaan : Ada, fille d’Élone le Hittite, Oholibama, fille d’Ana, fils de Sibéone le Hivvite, 
${}^{3}et Basmath, fille d’Ismaël et sœur de Nebayoth. 
${}^{4}Ada enfanta à Ésaü Élifaz, et Basmath enfanta Réouël, 
${}^{5}Oholibama enfanta Yéoush, Yahelam et Coré. Tels sont les fils d’Ésaü, qui lui naquirent au pays de Canaan.
${}^{6}Ésaü prit avec lui ses femmes, ses fils, ses filles, tous les gens de sa maison, son troupeau, tout son bétail et tous les biens qu’il avait acquis au pays de Canaan ; puis il partit vers un autre pays, loin de son frère Jacob. 
${}^{7}En effet, ils avaient trop de biens pour habiter ensemble : le pays où ils étaient venus en immigrés ne pouvait leur suffire, leurs troupeaux étant trop nombreux. 
${}^{8}Ésaü – autrement dit Édom – alla donc habiter dans la montagne de Séïr.
${}^{9}Voici la descendance d’Ésaü, père d’Édom, dans la montagne de Séïr.
${}^{10}Voici les noms des fils d’Ésaü : Élifaz, fils de Ada, femme d’Ésaü, et Réouël, fils de Basmath, femme d’Ésaü. 
${}^{11}Les fils d’Élifaz furent Témane, Omar, Cefo, Gahetam et Qenaz. 
${}^{12}Timna fut la concubine d’Élifaz, fils d’Ésaü, et lui donna un fils, Amalec. Ce sont les fils de Ada, femme d’Ésaü.
${}^{13}Voici les fils de Réouël : Nahath, Zèrah, Shamma et Mizza. Ce sont les fils de Basmath, femme d’Ésaü. 
${}^{14}Voici quels furent les fils d’Oholibama, fille d’Ana, fille de Sibéone et femme d’Ésaü ; elle lui enfanta Yéoush, Yahelam et Coré.
${}^{15}Voici les chefs des fils d’Ésaü. Les fils d’Élifaz, premier-né d’Ésaü, sont les chefs Témane, Omar, Séfo, Qenaz, 
${}^{16}Coré, Gahetam et Amalec. Ce sont les chefs d’Élifaz dans le pays d’Édom, ce sont les fils de Ada.
${}^{17}Voici les fils de Réouël, fils d’Ésaü : les chefs Nahath, Zèrah, Shamma et Mizza. Ce sont les chefs de Réouël dans le pays d’Édom ; ce sont les fils de Basmath, femme d’Ésaü.
${}^{18}Voici les fils d’Oholibama, femme d’Ésaü : les chefs Yéoush, Yahelam et Coré. Ce sont les chefs d’Oholibama, fille d’Ana, femme d’Ésaü.
${}^{19}Tels sont les fils d’Ésaü, autrement dit Édom, et tels sont leurs chefs.
${}^{20}Voici les fils de Séïr le Horite, habitant le pays : Lotane, Shobal, Sibéone, Ana, 
${}^{21}Dishone, Écer et Dishane. Ce sont les chefs horites, fils de Séïr, dans le pays d’Édom. 
${}^{22}Les fils de Lotane furent Hori et Hémam, la sœur de Lotane était Timna. 
${}^{23}Voici les fils de Shobal : Alvane, Manahath, Ébal, Shefo et Onam. 
${}^{24}Voici les fils de Sibéone : Ayya et Ana. Ce fut Ana qui trouva les sources d’eau chaude dans le désert, en faisant paître les ânes pour son père Sibéone. 
${}^{25}Voici les enfants d’Ana : Dishone et Oholibama, fille d’Ana. 
${}^{26}Voici les fils de Dishane : Hèmdane, Eshbane, Yitrane et Kerane. 
${}^{27}Voici les fils de Écer : Bilhane, Zaavane, Aqane. 
${}^{28}Voici les fils de Dishane : Ouç et Arane.
${}^{29}Voici les chefs horites : Lotane, Shobal, Sibéone, Ana, 
${}^{30}Dishone, Écer et Dishane. Ce sont les chefs horites selon leurs clans, dans le pays de Séïr.
${}^{31}Voici les rois qui ont régné au pays d’Édom avant que ne règne un roi sur les fils d’Israël : 
${}^{32}Bèla, fils de Béor, régna sur Édom, et le nom de sa ville était Dinhaba. 
${}^{33}Bèla mourut, et Yobab, fils de Zèrah de Bosra, régna à sa place. 
${}^{34}Yobab mourut, et Housham, du pays des Témanites, régna à sa place. 
${}^{35}Housham mourut, et Hadad, fils de Bedad, régna à sa place. Il l’emporta sur Madiane aux Champs-de-Moab ; le nom de sa ville était Avith. 
${}^{36}Hadad mourut, et Samla de Masréqa régna à sa place. 
${}^{37}Samla mourut, et Saül, de Rehoboth-sur-le-Fleuve, régna à sa place. 
${}^{38}Saül mourut, et Baal-Hanane, fils d’Akbor, régna à sa place. 
${}^{39}Baal-Hanane, fils d’Akbor, mourut, et Hadar régna à sa place ; le nom de sa ville était Paou. Le nom de sa femme était Mehétabéel, fille de Matred, fille de Mé-Zahab.
${}^{40}Voici la liste des chefs d’Ésaü selon leurs clans et leurs localités. Leurs noms étaient : Timna, Alva, Yeteth, 
${}^{41}Oholibama, Éla, Pinone, 
${}^{42}Qenaz, Témane, Mibsar, 
${}^{43}Magdiël et Iram. Ce sont les chefs d’Édom selon l’endroit où ils habitaient au pays dont ils avaient la propriété. C’est Ésaü, le père d’Édom.
      <h2 class="intertitle hmbot" id="d85e13909">3. Histoire de Joseph et de ses frères (37 – 50)</h2>
      
         
      \bchapter{}
      \begin{verse}
${}^{1}Jacob habita la terre où son père était venu en immigré : la terre de Canaan. 
${}^{2}Voici l’histoire de la descendance de Jacob.
      Joseph, âgé de dix-sept ans, faisait paître le petit bétail avec ses frères. Le jeune homme accompagnait les fils de Bilha et les fils de Zilpa, femmes de son père. Il fit part à leur père de la mauvaise réputation de ses frères. 
${}^{3}Israël, c’est-à-dire Jacob\\, aimait Joseph plus que tous ses autres enfants, parce qu’il était le fils de sa vieillesse, et il lui fit faire une tunique de grand prix\\. 
${}^{4}En voyant qu’il leur préférait Joseph, ses autres fils se mirent à détester celui-ci, et ils ne pouvaient plus lui parler sans hostilité.
${}^{5}Joseph eut un songe et le raconta à ses frères qui l’en détestèrent d’autant plus. 
${}^{6}« Écoutez donc, leur dit-il, le songe que j’ai eu. 
${}^{7}Nous étions en train de lier des gerbes au milieu des champs, et voici que ma gerbe se dressa et resta debout. Alors vos gerbes l’ont entourée et se sont prosternées devant ma gerbe. » 
${}^{8}Ses frères lui répliquèrent : « Voudrais-tu donc régner sur nous ? nous dominer ? » Ils le détestèrent encore plus, à cause de ses songes et de ses paroles.
${}^{9}Il eut encore un autre songe et le raconta à ses frères. Il leur dit : « Écoutez, j’ai encore eu un songe : voici que le soleil, la lune et onze étoiles se prosternaient devant moi. » 
${}^{10}Il le raconta également à son père qui le réprimanda et lui dit : « Qu’est-ce que c’est que ce songe que tu as eu ? Nous faudra-t-il venir, moi, ta mère et tes frères, nous prosterner jusqu’à terre devant toi ? » 
${}^{11}Ses frères furent jaloux de lui, mais son père retint la chose.
${}^{12}Les frères de Joseph étaient allés à Sichem faire paître le troupeau de leur père. 
${}^{13}Israël dit à Joseph : « Tes frères ne gardent-ils pas le troupeau à Sichem ? Va donc les trouver de ma part ! » Il répondit : « Me voici. » 
${}^{14}Jacob reprit : « Va voir comment se portent tes frères et comment va le troupeau, et rapporte-moi des nouvelles. » C’est de la vallée d’Hébron qu’il l’envoya, et Joseph parvint à Sichem. 
${}^{15}Un homme le rencontra alors qu’il était perdu en pleine campagne, et lui demanda : « Que cherches-tu ? » 
${}^{16}Il répondit : « Je cherche mes frères. Indique-moi donc où ils font paître le troupeau. » 
${}^{17}L’homme dit : « Ils sont partis d’ici, et je les ai entendu dire : “Allons à Dotane !” » Joseph continua donc à chercher ses frères et les trouva à Dotane. 
${}^{18}Ceux-ci l’aperçurent de loin et, avant qu’il arrive près d’eux, ils complotèrent de le faire mourir. 
${}^{19}Ils se dirent l’un à l’autre : « Voici l’expert\\en songes qui arrive ! 
${}^{20}C’est le moment, allons-y, tuons-le, et jetons-le dans une de ces citernes. Nous dirons qu’une bête féroce l’a dévoré, et on verra ce que voulaient dire ses songes ! »
${}^{21}Mais Roubène les entendit, et voulut le sauver de leurs mains. Il leur dit : « Ne touchons pas à sa vie. » 
${}^{22}Et il ajouta : « Ne répandez pas son sang : jetez-le dans cette citerne du désert, mais ne portez pas la main sur lui. » Il voulait le sauver de leurs mains et le ramener à son père.
${}^{23}Dès que Joseph eut rejoint ses frères, ils le dépouillèrent de sa tunique, la tunique de grand prix qu’il portait, 
${}^{24} ils se saisirent de lui et le jetèrent dans la citerne, qui était vide et sans eau. 
${}^{25} Ils s’assirent ensuite pour manger. En levant les yeux, ils virent une caravane d’Ismaélites qui venait de Galaad. Leurs chameaux étaient chargés d’aromates\\, de baume\\et de myrrhe\\qu’ils allaient livrer en Égypte. 
${}^{26} Alors Juda dit à ses frères : « Quel profit aurions-nous à tuer notre frère et à dissimuler sa mort\\ ? 
${}^{27} Vendons-le plutôt aux Ismaélites et ne portons pas la main sur lui, car il est notre frère, notre propre chair. » Ses frères l’écoutèrent.
${}^{28}Des marchands madianites qui passaient par là retirèrent\\Joseph de la citerne, ils le vendirent pour vingt pièces d’argent aux Ismaélites, et ceux-ci l’emmenèrent en Égypte.
${}^{29}Quand Roubène revint à la citerne, Joseph n’y était plus. Il déchira ses vêtements, 
${}^{30}revint vers ses frères et dit : « L’enfant n’est plus là ! Et moi, où vais-je donc aller, moi ? »
${}^{31}Ils prirent alors la tunique de Joseph, égorgèrent un bouc et trempèrent la tunique dans le sang. 
${}^{32}Puis ils firent porter à leur père la tunique de grand prix, avec ce message : « Nous avons trouvé ceci. Regarde bien : est-ce ou n’est-ce pas la tunique de ton fils ? » 
${}^{33}Il la reconnut et s’écria : « La tunique de mon fils ! Une bête féroce a dévoré Joseph ! Il a été mis en pièces ! » 
${}^{34}Jacob déchira ses vêtements, mit un sac sur ses reins et porta le deuil de son fils pendant de longs jours. 
${}^{35}Ses fils et ses filles se mirent tous à le consoler, mais il refusait les consolations, en disant : « C’est en deuil que je descendrai vers mon fils, au séjour des morts. » Et son père le pleura.
${}^{36}Quant aux Madianites, ils le vendirent en Égypte à Putiphar, dignitaire de Pharaon et grand intendant.
      
         
      \bchapter{}
      \begin{verse}
${}^{1}En ce temps-là, Juda quitta ses frères et se rendit chez un homme d’Adoullam appelé Hira. 
${}^{2}Là, Juda aperçut la fille d’un Cananéen appelé Shoua. Il la prit et s’unit à elle. 
${}^{3}Elle devint enceinte et enfanta un fils qu’on appela Er. 
${}^{4}Elle devint encore enceinte et enfanta un fils qu’elle appela Onane. 
${}^{5}Elle devint enceinte une troisième fois et enfanta un fils qu’elle appela Shéla. Juda était à Késib lors de cette naissance.
${}^{6}Juda prit une femme pour Er, son premier-né. Elle s’appelait Tamar. 
${}^{7}Mais Er, le premier-né de Juda, déplut au Seigneur, et le Seigneur le fit mourir. 
${}^{8}Alors Juda dit à Onane : « Unis-toi à la femme de ton frère, pour remplir envers elle ton devoir de beau-frère : suscite une descendance à ton frère. » 
${}^{9}Mais Onane savait que la descendance ne serait pas à lui. Aussi, quand il s’unissait à la femme de son frère, il laissait la semence se perdre à terre, pour ne pas donner de descendance à son frère. 
${}^{10}Ce qu’il faisait déplut au Seigneur qui le fit mourir, lui aussi. 
${}^{11}Juda dit alors à Tamar, sa bru : « Habite comme une veuve dans la maison de ton père, jusqu’à ce que mon fils Shéla ait grandi. » Il se disait, en effet : « Il ne faudrait pas que celui-ci meure aussi, comme ses frères. » Tamar s’en alla donc habiter dans la maison de son père.
${}^{12}Bien des jours passèrent, et la fille de Shoua, la femme de Juda, mourut. Quand Juda fut consolé, il monta à Timna chez les tondeurs de son troupeau avec son ami Hira, qui était d’Adoullam. 
${}^{13}On informa Tamar : « Voici que ton beau-père monte à Timna pour la tonte de son troupeau. » 
${}^{14}Alors elle ôta ses vêtements de veuve, se couvrit d’un voile, se rendit méconnaissable et s’assit à l’entrée d’Énaïm, sur le chemin de Timna. En effet, elle voyait bien que Shéla avait grandi et qu’elle ne lui était toujours pas donnée pour femme.
${}^{15}Juda l’aperçut et la prit pour une prostituée, puisqu’elle avait couvert son visage. 
${}^{16}Il se dirigea vers elle, au bord du chemin, et dit :« Permets donc que j’aille avec toi ». En effet, il n’avait pas reconnu sa bru. Elle répondit : « Que me donneras-tu pour aller avec moi ? » 
${}^{17}Il dit : « Je t’enverrai un chevreau de mon troupeau. » Elle reprit : « Oui, si tu me donnes un gage jusqu’à ce que tu l’envoies. » 
${}^{18}Et lui : « Quel gage vais-je te donner ? » Elle répondit : « Ton sceau à cacheter, ton cordon et le bâton que tu tiens en main. » Il les lui donna et s’unit à elle. Et elle devint enceinte de lui. 
${}^{19}Elle se leva, s’en retourna, ôta son voile et reprit ses vêtements de veuve.
${}^{20}Juda envoya le chevreau par l’intermédiaire de son ami d’Adoullam, pour reprendre le gage des mains de la femme. Celui-ci ne la trouva pas. 
${}^{21}Il interrogea les gens de l’endroit : « Où est la prostituée qui se trouvait à Énaïm, au bord de la route ? » Ils répondirent : « Il n’y a jamais eu là de prostituée. » 
${}^{22}Il retourna donc chez Juda et dit : « Je ne l’ai pas trouvée, et les gens de l’endroit m’ont même déclaré qu’il n’y avait jamais eu là de prostituée. » 
${}^{23}Juda répondit : « Qu’elle garde tout pour elle ! Ne nous couvrons pas de ridicule, moi qui lui ai envoyé un chevreau, et toi qui ne l’as pas trouvée ! »
${}^{24}Or, trois mois plus tard, on informa Juda : « Ta bru Tamar s’est prostituée et voilà même qu’elle est enceinte ! » Juda déclara : « Qu’on la jette dehors et qu’on la brûle ! » 
${}^{25}Tandis qu’on la jetait dehors, elle envoya dire à son beau-père : « C’est de l’homme à qui appartiennent ces objets que je suis enceinte. » Et elle ajouta : « Regarde donc bien à qui appartiennent le sceau à cacheter, le cordon et le bâton que voici ! » 
${}^{26}Juda les reconnut et dit : « Elle est plus juste que moi car, de fait, je ne l’ai pas donnée à mon fils Shéla. » Et désormais il ne s’unit plus à elle.
${}^{27}Or, quand elle accoucha, on s’aperçut qu’elle portait des jumeaux. 
${}^{28}Pendant l’accouchement, l’un d’eux présenta une main que la sage-femme saisit : elle y attacha un fil écarlate, en disant : « Celui-ci est sorti le premier ! » 
${}^{29}Mais il retira sa main et c’est son frère qui sortit. La sage-femme dit : « Quelle brèche tu as ouverte ! » Et on l’appela Pérès (c’est-à-dire : Brèche). 
${}^{30}Son frère sortit ensuite, lui qui avait à la main le fil écarlate. On l’appela Zèrah.
      
         
      \bchapter{}
      \begin{verse}
${}^{1}Joseph fut emmené en Égypte. Putiphar, dignitaire de Pharaon et grand intendant, un Égyptien, l’acheta aux Ismaélites qui l’avaient emmené là-bas. 
${}^{2}Le Seigneur était avec Joseph, et tout lui réussissait ; il vivait dans la maison de son maître, l’Égyptien. 
${}^{3}Ce dernier vit que le Seigneur était avec Joseph et faisait réussir tout ce qu’il entreprenait. 
${}^{4}Joseph trouva grâce aux yeux de son maître qui l’attacha à son service : il lui donna autorité sur sa maison et remit entre ses mains tout ce qu’il possédait. 
${}^{5}Dès que l’Égyptien eut confié cette charge à Joseph, le Seigneur bénit sa maison, à cause de Joseph, et la bénédiction du Seigneur s’étendit sur tout ce que possédait l’Égyptien, sa maison et ses champs. 
${}^{6}Il abandonna entre les mains de Joseph tout ce qu’il possédait et ne s’occupa plus de rien, sinon de la nourriture qu’il prenait.
      Joseph avait belle allure et il était agréable à regarder. 
${}^{7}À quelque temps de là, la femme de son maître leva les yeux sur Joseph et dit : « Couche avec moi ! » 
${}^{8}Mais il refusa et répondit à la femme de son maître : « Voici que mon maître ne s’occupe plus de rien dans la maison. Tout ce qu’il possède, il l’a remis entre mes mains. 
${}^{9}Dans cette maison, il ne m’est pas supérieur et il ne me refuse rien, sinon toi, car tu es sa femme. Comment donc pourrai-je commettre ce grand mal et pécher contre Dieu ? » 
${}^{10}Chaque jour, elle insistait auprès de Joseph. Mais lui n’acceptait pas de partager sa couche et d’être à elle.
${}^{11}Vint le jour où Joseph entra dans la maison pour faire son travail, alors qu’aucun domestique n’était là. 
${}^{12}La femme l’attrapa par son vêtement, en disant : « Couche avec moi ! » Mais il abandonna le vêtement entre ses mains et s’enfuitau-dehors.
${}^{13}Lorsqu’elle réalisa que, dans sa fuite, il avait abandonné son vêtement entre ses mains, 
${}^{14}elle appela ses domestiques et leur dit : « Voyez ça ! On nous a amené un Hébreu pour se jouer de nous ! Il est venu vers moi pour coucher avec moi, et j’ai appelé à grands cris. 
${}^{15}Alors, quand il m’a entendu élever la voix pour appeler, il a abandonné son vêtement à côté de moi et s’est enfui au-dehors. »
${}^{16}Elle garda près d’elle le vêtement de Joseph, jusqu’à ce que le maître rentre chez lui. 
${}^{17}Elle lui tint alors le même langage : « Le serviteur hébreu que tu nous as amené est venu vers moi pour se jouer de moi. 
${}^{18}Mais j’ai appelé à grands cris, et il a abandonné son vêtement à côté de moi et s’est enfui au-dehors. » 
${}^{19}Quand le maître entendit sa femme lui dire : « Voilà comment ton serviteur a agi envers moi ! », il s’enflamma de colère. 
${}^{20}Le maître de Joseph se saisit de lui et le jeta dans la prison où étaient enfermés les prisonniers du roi.
      Joseph était en prison, 
${}^{21}mais le Seigneur était avec lui ; il lui accorda sa faveur et lui fit trouver grâce aux yeux du chef de la prison. 
${}^{22}Le chef de la prison remit entre les mains de Joseph tous les prisonniers : tout ce qui se faisait, c’est Joseph qui le faisait faire. 
${}^{23}Le chef de la prison ne s’occupait en rien de ce qui était confié à Joseph car le Seigneur était avec lui, et ce qu’il entreprenait, le Seigneur le faisait réussir.
      
         
      \bchapter{}
      \begin{verse}
${}^{1}À quelque temps de là, l’échanson du roi d’Égypte ainsi que le panetier commirent une faute envers leur maître, le roi d’Égypte. 
${}^{2}Pharaon s’irrita contre ses deux dignitaires, le grand échanson et le grand panetier, 
${}^{3}et il les fit mettre au poste de garde, dans la maison du grand intendant, au lieu même où Joseph était prisonnier. 
${}^{4}Le grand intendant les confia aux soins de Joseph qui fut attaché à leur service. Ils demeurèrent un certain temps au poste de garde. 
${}^{5}Une même nuit, l’échanson et le panetier du roi d’Égypte firent tous deux un songe, alors qu’ils étaient prisonniers dans la prison. Et chacun des songes avait sa propre signification. 
${}^{6}Au matin, quand Joseph entra chez eux, il vit qu’ils avaient la mine défaite. 
${}^{7}Il demanda donc aux dignitaires de Pharaon qui étaient avec lui au poste de garde, dans la maison de son maître : « Pourquoi vos visages sont-ils si sombres aujourd’hui ? » 
${}^{8}Ils lui répondirent : « Nous avons eu un songe, et il n’y a personne pour l’interpréter. » Joseph leur dit : « N’est-ce pas à Dieu qu’il appartient d’interpréter ? Racontez-moi donc ! »
${}^{9}Le grand échanson raconta à Joseph le songe qu’il avait fait : « J’ai rêvé qu’une vigne était devant moi. 
${}^{10}Elle portait trois sarments. Elle bourgeonnait, fleurissait, puis ses grappes donnaient des raisins mûrs. 
${}^{11}J’avais entre les mains la coupe de Pharaon. Je saisissais les grappes, je les pressais au-dessus de la coupe de Pharaon et je lui remettais la coupe entre les mains. »
${}^{12}Joseph lui dit : « Voici l’interprétation : les trois sarments représentent trois jours. 
${}^{13}Encore trois jours et Pharaon t’élèvera la tête, il te rétablira dans ta charge, et tu placeras la coupe entre ses mains, comme tu avais coutume de le faire précédemment quand tu étais son échanson. 
${}^{14}Mais quand tout ira bien pour toi, pour autant que tu te souviennes d’avoir été avec moi, montre ta faveur à mon égard : rappelle-moi au souvenir de Pharaon et fais-moi sortir de cette maison ! 
${}^{15}En effet, j’ai été enlevé au pays des Hébreux, et là non plus je n’avais rien fait pour qu’on me jette dans la citerne. »
${}^{16}Voyant que Joseph avait fait une interprétation favorable, le grand panetier lui dit : « Moi, j’ai rêvé que je portais sur la tête trois corbeilles de gâteaux. 
${}^{17}Et dans la corbeille d’au-dessus, il y avait tous les aliments que le panetier fabrique pour la nourriture de Pharaon, et les oiseaux picoraient dans la corbeille au-dessus de ma tête. » 
${}^{18}Joseph répondit : « Voici l’interprétation : les trois corbeilles représentent trois jours. 
${}^{19}Encore trois jours et Pharaon t’élèvera la tête, il te pendra à un arbre, et les oiseaux picoreront ta chair. » 
${}^{20}Le troisième jour, jour anniversaire de Pharaon, celui-ci fit un festin pour tous ses serviteurs. Il éleva la tête du grand échanson et celle du grand panetier en présence de ses serviteurs : 
${}^{21}il rétablit dans sa charge le grand échanson, et celui-ci plaça la coupe entre les mains de Pharaon ; 
${}^{22}mais le grand panetier, il le pendit, comme l’avait annoncé Joseph. 
${}^{23}Toutefois le grand échanson ne se souvint pas de Joseph ; il l’oublia.
      
         
      \bchapter{}
      \begin{verse}
${}^{1}Deux ans plus tard, Pharaon eut un songe. Il se tenait debout près du Nil, 
${}^{2}et voici que montaient du Nil sept vaches, belles et bien grasses, qui broutaient dans les roseaux. 
${}^{3}Puis, derrière elles, montaient du Nil sept autres vaches, laides et très maigres. Elles se tenaient à côté des premières, sur la rive du Nil. 
${}^{4}Et les vaches laides et très maigres mangeaient les sept vaches belles et bien grasses. Alors Pharaon s’éveilla. 
${}^{5}Il se rendormit et fit encore un songe : sept épis montaient sur une seule tige ; ils étaient gros et beaux. 
${}^{6}Puis, après eux, germaient sept épis maigres et desséchés par le vent d’est. 
${}^{7}Et les épis maigres avalaient les sept épis gros et pleins. Alors Pharaon s’éveilla : c’était un songe ! 
${}^{8}Mais le matin, son esprit était troublé ; il fit convoquer tous les magiciens et tous les sages d’Égypte. Pharaon leur raconta les songes, mais personne ne pouvait les interpréter.
${}^{9}Alors le grand échanson parla à Pharaon en ces termes : « Aujourd’hui, je me rappelle mes fautes. 
${}^{10}Pharaon s’était irrité contre ses serviteurs et il m’avait mis au poste de garde, dans la maison du grand intendant, et avec moi, le grand panetier. 
${}^{11}Une même nuit, nous avons fait un songe, moi et lui. Et chacun des songes avait sa propre signification. 
${}^{12}Il y avait là, avec nous, un jeune Hébreu, serviteur du grand intendant. Nous lui avons raconté nos songes et il a donné à chacun l’interprétation du songe qu’il avait fait. 
${}^{13}Et ses interprétations s’avérèrent exactes : moi, on m’a rétabli dans ma charge, et l’autre, on l’a pendu. »
${}^{14}Pharaon fit appeler Joseph. En toute hâte, on le tira de son cachot. Il se rasa, changea de vêtements et se rendit chez Pharaon. 
${}^{15}Pharaon dit à Joseph : « J’ai fait un songe et personne ne peut l’interpréter. Mais j’ai entendu dire de toi, qu’il te suffit d’entendre raconter un songe pour en donner l’interprétation. » 
${}^{16}Joseph répondit à Pharaon : « Ce n’est pas moi, c’est Dieu qui donnera à Pharaon la réponse qui lui rendra la paix. »
${}^{17}Alors, Pharaon dit à Joseph : « Dans le songe, j’étais debout au bord du Nil, 
${}^{18}et voici que montaient du Nil sept vaches, bien grasses et de belle allure, qui broutaient dans les roseaux. 
${}^{19}Puis, derrière elles, montaient sept autres vaches, chétives, très laides et décharnées. Je n’en avais jamais vu d’une telle laideur dans tout le pays d’Égypte. 
${}^{20}Les vaches décharnées et laides mangeaient les premières vaches, les grasses, 
${}^{21}qui entraient dans leur panse. Mais on ne s’apercevait pas que les grasses étaient entrées dans leur panse : elles restaient aussi laides qu’avant. Alors je me suis réveillé. 
${}^{22}Mais j’ai encore vu, en songe, sept épis qui montaient sur une seule tige ; ils étaient pleins et beaux. 
${}^{23}Puis, après eux, germaient sept épis durcis, maigres et desséchés par le vent d’est. 
${}^{24}Et les épis maigres avalaient les sept beaux épis. J’en ai parlé aux magiciens, mais personne n’a pu me fournir d’explication. »
${}^{25}Joseph répondit à Pharaon : « Pharaon n’a eu qu’un seul et même songe. Ce que Dieu va faire, il l’a indiqué à Pharaon. 
${}^{26}Les sept belles vaches représentent sept années, et les sept beaux épis, sept années : c’est un seul et même songe ! 
${}^{27}Les sept vaches décharnées et laides qui montaient derrière les autres représentent sept années ; de même, les sept épis vides et desséchés par le vent d’est. Ce seront sept années de famine. 
${}^{28}C’est bien ce que j’ai dit à Pharaon : ce que Dieu va faire, il l’a montré à Pharaon. 
${}^{29}Voici qu’arrivent sept années de grande abondance dans tout le pays d’Égypte. 
${}^{30}Mais après elles viendront sept années de famine : alors on oubliera toute abondance dans le pays d’Égypte, la famine épuisera le pays. 
${}^{31}On ne saura plus ce que pouvait être l’abondance dans le pays, tant la famine qui suivra pèsera lourdement. 
${}^{32}Si le songe de Pharaon s’est répété une seconde fois, c’est que la décision de Dieu est bien arrêtée et qu’il va se hâter de l’exécuter. 
${}^{33}Maintenant donc, que Pharaon voie s’il y a un homme intelligent et sage pour l’établir sur le pays d’Égypte. 
${}^{34}Que Pharaon agisse en instituant des fonctionnaires sur le pays d’Égypte, afin de prélever le cinquième des récoltes pendant les sept années d’abondance. 
${}^{35}Ils recueilleront toute la nourriture de ces bonnes années qui viennent et, sous l’autorité de Pharaon, ils entasseront dans les villes du froment comme nourriture : ils le garderont en réserve. 
${}^{36}Ainsi, il y aura une réserve de nourriture pour le pays en vue des sept années de famine qui suivront dans le pays d’Égypte, et la famine ne détruira pas le pays. »
${}^{37}Cette proposition plut à Pharaon et à tous ses serviteurs. 
${}^{38}Pharaon leur dit : « Trouverons-nous un homme comme celui-ci, qui a l’esprit de Dieu en lui ? » 
${}^{39}Alors, Pharaon dit à Joseph : « Dès lors que Dieu t’a fait connaître tout cela, personne ne peut être aussi intelligent et aussi sage que toi. 
${}^{40}C’est toi qui auras autorité sur ma maison ; tout mon peuple se soumettra à tes ordres ; par le trône seulement, je serai plus grand que toi. » 
${}^{41}Pharaon dit à Joseph : « Vois ! Je t’établis sur tout le pays d’Égypte. » 
${}^{42}Il ôta l’anneau de son doigt et le passa au doigt de Joseph ; il le revêtit d’habits de lin fin et lui mit autour du cou le collier d’or. 
${}^{43}Il le fit monter sur son deuxième char et on criait devant lui : « À genoux ! » Et ainsi il l’établit sur tout le pays d’Égypte.
${}^{44}Pharaon dit encore à Joseph : « Je suis Pharaon. Mais sans ta permission, personne ne lèvera le petit doigt dans tout le pays d’Égypte. » 
${}^{45}Pharaon appela Joseph Safnath-Panéah et lui donna pour femme Asnath, fille de Poti-Phéra, prêtre de One. Alors Joseph partit inspecter le pays d’Égypte.
${}^{46}Joseph avait trente ans quand il se tint en présence de Pharaon, le roi d’Égypte. Il prit congé de lui et parcourut tout le pays d’Égypte. 
${}^{47}Pendant les sept années d’abondance, la terre produisit à plein. 
${}^{48}Pendant les sept années d’abondance au pays d’Égypte, Joseph recueillit toute la nourriture et l’entreposa dans les villes. Il entreposait au centre de la ville la nourriture produite dans la campagne environnante. 
${}^{49}Joseph accumula tellement de froment, qu’on cessa d’en faire le compte ; on ne pouvait pas plus le mesurer que le sable de la mer.
${}^{50}Avant l’année où survint la famine, il naquit à Joseph deux fils que lui enfanta Asnath, fille de Poti-Phéra, prêtre de One. 
${}^{51}Joseph appela l’aîné Manassé car, disait-il, « Dieu m’a fait oublier toute ma peine et toute celle de la maison de mon père ». 
${}^{52}Le second, il l’appela Éphraïm car, disait-il, « Dieu m’a fait fructifier dans le pays de ma misère ».
${}^{53}Les sept années d’abondance dans le pays d’Égypte prirent fin. 
${}^{54}Alors commencèrent les sept années de famine, ainsi que Joseph l’avait annoncé. La famine sévissait partout, mais dans tout le pays d’Égypte il y avait du pain. 
${}^{55}Puis, tout le pays d’Égypte souffrit, lui aussi, de la faim, et le peuple, à grands cris, réclama du pain à Pharaon. Mais Pharaon dit à tous les Égyptiens : « Allez trouver Joseph, et faites ce qu’il vous dira. » 
${}^{56}La famine s’étendait à tout le pays. Alors Joseph ouvrit toutes les réserves et vendit du blé aux Égyptiens, tandis que la famine s’aggravait encore dans le pays\\. 
${}^{57}De partout on vint en Égypte pour acheter du blé à Joseph, car la famine s’aggravait partout.
      
         
      \bchapter{}
      \begin{verse}
${}^{1}Apprenant qu’il y avait du blé en Égypte, Jacob dit à ses fils : « Pourquoi restez-vous là à vous regarder ? » 
${}^{2}Il ajouta : « J’ai entendu dire qu’il y avait du blé en Égypte. Descendez là-bas et achetez-y du blé pour nous : ainsi nous ne mourrons pas, nous vivrons. » 
${}^{3}Dix des frères de Joseph descendirent acheter du froment en Égypte. 
${}^{4}Mais Benjamin, frère de Joseph, Jacob ne l’envoya pas avec ses frères, car il se disait : « J’ai peur qu’il lui arrive malheur ! »
${}^{5}Les fils d’Israël, c’est-à-dire de Jacob\\, parmi beaucoup d’autres gens, vinrent donc pour acheter du blé, car la famine sévissait au pays de Canaan. 
${}^{6}C’était Joseph qui organisait la vente du blé pour tout le peuple du pays, car il avait pleins pouvoirs dans le pays. En arrivant, les frères de Joseph se prosternèrent devant lui, face contre terre. 
${}^{7}Dès qu’il les vit, il les reconnut, mais il se comporta comme un étranger à leur égard et il leur parla avec dureté. Il leur dit : « D’où venez-vous ? » Ils répondirent : « Du pays de Canaan, pour acheter du blé en nourriture. »
${}^{8}Joseph avait reconnu ses frères, mais eux ne l’avaient pas reconnu. 
${}^{9}Joseph se rappela les songes qu’il avait eus à leur sujet et leur dit : « Vous êtes des espions ! C’est pour découvrir les points faibles du pays que vous êtes venus ! » 
${}^{10}Ils répondirent : « Non, mon seigneur, tes serviteurs sont venus pour acheter du blé en nourriture. 
${}^{11}Nous sommes tous fils du même homme. Nous sommes de bonne foi : tes serviteurs ne sont pas des espions. » 
${}^{12}Joseph leur répéta : « Non ! C’est pour découvrir les points faibles du pays que vous êtes venus. »
${}^{13}Alors ils ajoutèrent : « Tes serviteurs étaient douze frères. Nous sommes fils d’un même homme, au pays de Canaan. Aujourd’hui le plus jeune est resté avec notre père, et l’un de nous n’est plus. » 
${}^{14}Joseph leur déclara : « Je maintiens ce que je vous ai dit, vous êtes des espions ! 
${}^{15}Voici l’épreuve que vous devrez subir : Par la vie de Pharaon, vous ne sortirez de ce pays que si votre plus jeune frère vient ici ! 
${}^{16}Envoyez l’un de vous chercher votre frère. Mais vous, vous resterez prisonniers. On va vérifier vos paroles : est-ce la vérité ? Si c’est non, par la vie de Pharaon, vous êtes vraiment des espions ! » 
${}^{17}Il les retint au poste de garde pendant trois jours.
${}^{18}Le troisième jour, il leur dit : « Faites ce que je vais vous dire, et vous resterez en vie, car je crains Dieu. 
${}^{19} Si vous êtes de bonne foi, que l’un d’entre vous\\reste prisonnier au poste de garde. Vous autres, partez en emportant ce qu’il faut de blé pour éviter la famine à votre clan\\. 
${}^{20} Puis vous m’amènerez votre plus jeune frère : ainsi vos paroles seront vérifiées, et vous ne serez pas mis à mort. »
      Ils acceptèrent, 
${}^{21} et ils se disaient l’un à l’autre : « Hélas ! nous sommes coupables envers Joseph\\notre frère : nous avons vu dans quelle détresse il se trouvait quand il nous suppliait, et nous ne l’avons pas écouté. C’est pourquoi nous sommes maintenant dans une telle détresse. »
${}^{22}Roubène, alors, prit la parole : « Je vous l’avais bien dit : “Ne commettez pas ce crime contre notre jeune frère !” Mais vous ne m’avez pas écouté, et maintenant il faut répondre de son sang. »
${}^{23}Comme il y avait un interprète, ils ne se rendaient pas compte que Joseph les comprenait. 
${}^{24} Alors Joseph se retira pour pleurer.
      Ensuite il revint près d’eux et leur parla. Parmi eux, il choisit Siméon et le fit enchaîner sous leurs yeux. 
${}^{25}Alors Joseph ordonna de remplir de froment leurs bagages, de replacer l’argent de chacun dans son sac et de leur donner des provisions pour la route. C’est ainsi qu’il agit envers eux. 
${}^{26}Ils chargèrent le blé sur leurs ânes et partirent.
${}^{27}À l’étape, l’un d’eux ouvrit son sac pour donner du fourrage à son âne et il découvrit son argent : il était sur le dessus de la besace ! 
${}^{28}Il dit à ses frères : « On m’a rendu mon argent, il est là dans ma besace. » Le cœur leur manqua, ils tressaillirent, se regardant l’un l’autre, et dirent : « Qu’est-ce que Dieu nous a fait ? »
${}^{29}De retour au pays de Canaan chez Jacob, leur père, ils lui rapportèrent tout ce qui leur était arrivé. 
${}^{30}Ils dirent : « L’homme qui est le maître du pays nous a parlé avec dureté, il nous a pris pour des espions du pays. 
${}^{31}Nous lui avons dit : “Nous sommes de bonne foi, nous ne sommes pas des espions. 
${}^{32}Nous étions douze frères, fils d’un même père : l’un de nous n’est plus, et aujourd’hui le plus jeune est resté avec notre père, au pays de Canaan.” 
${}^{33}Alors, l’homme qui est le maître du pays nous a dit : “Voici comment je saurai si vous êtes de bonne foi. Laissez avec moi l’un de vos frères, prenez de quoi éviter la famine à votre clan et partez ! 
${}^{34}Puis amenez-moi votre plus jeune frère pour que je sache que vous n’êtes pas des espions mais que vous êtes de bonne foi. Je vous rendrai votre autre frère, et vous pourrez aller et venir dans le pays.” »
${}^{35}Ils se mirent à vider leurs sacs, et voici que chacun trouvait, dans son sac, la bourse avec son argent ! Quand eux-mêmes et leur père virent les bourses avec leur argent, ils eurent peur.
${}^{36}Jacob, leur père, dit alors : « Vous me privez de mes enfants ! Joseph n’est plus ! Siméon n’est plus ! Et vous voulez me prendre Benjamin ! Tout est contre moi. » 
${}^{37}Roubène dit à son père : « Tu pourras faire mourir mes deux fils, si je ne te ramène pas Benjamin. Remets-le entre mes mains et je te le rendrai. » 
${}^{38}Mais Jacob reprit : « Mon fils ne descendra pas avec vous. Son frère est mort, il ne me reste que lui. S’il lui arrivait malheur sur la route que vous allez prendre, c’est dans la douleur que vous feriez descendre mes cheveux blancs au séjour des morts. »
      
         
      \bchapter{}
      \begin{verse}
${}^{1}La famine continuait à peser sur le pays. 
${}^{2}Aussi, quand ils eurent fini de manger le blé rapporté d’Égypte, leur père leur dit : « Retournez nous acheter un peu de nourriture. » 
${}^{3}Juda lui répondit : « L’homme nous a déclaré expressément : “Vous ne serez pas admis en ma présence si votre frère n’est pas avec vous.” 
${}^{4}Si tu laisses notre frère partir avec nous, nous descendrons acheter de la nourriture. 
${}^{5}Mais si tu ne le laisses pas partir, nous ne descendrons pas, puisque l’homme nous a dit : “Vous ne serez pas admis en ma présence si votre frère n’est pas avec vous.” »
${}^{6}Israël dit alors : « Pourquoi m’avoir fait du mal en apprenant à l’homme que vous aviez encore un frère ? » 
${}^{7}Ils répondirent : « L’homme nous a pressés de questions sur nous et notre parenté : “Votre père est-il encore en vie ?” disait-il. “Avez-vous un frère ?” Nous avons répondu à ces questions. Est-ce que nous pouvions savoir qu’il dirait : “Amenez ici votre frère” ? » 
${}^{8}Juda dit alors à son père Israël : « Laisse partir le jeune homme avec moi. Debout ! Allons, si nous voulons vivre et non pas mourir, nous, toi et nos jeunes enfants ! 
${}^{9}Moi, je me porte garant de lui, tu pourras m’en demander compte. Si je ne le ramène pas auprès de toi, si je ne le présente pas devant toi, j’aurai commis une faute envers toi pour toujours ! 
${}^{10}Si nous n’avions pas tellement hésité, nous serions déjà revenus deux fois ! »
${}^{11}Leur père Israël reprit : « Si c’est le cas, eh bien ! faites ceci : prenez dans vos bagages des produits du pays pour en faire présent à cet homme, un peu de baume, un peu de miel, des aromates et de la myrrhe, des pistaches et des amandes. 
${}^{12}Prenez avec vous deux fois la somme d’argent ; ainsi l’argent remis sur le dessus de vos besaces, vous pourrez le restituer. C’était peut-être une erreur. 
${}^{13}Emmenez votre frère ! Debout, retournez chez cet homme ! 
${}^{14}Que le Dieu-Puissant vous donne de susciter la compassion de cet homme : que celui-ci vous laisse ramener votre autre frère, et aussi Benjamin. Pour moi, si je dois être privé d’enfants, que j’en sois privé ! »
${}^{15}Les hommes prirent avec eux le présent et la double somme d’argent ; ils emmenèrent aussi Benjamin. Ils se levèrent, descendirent en Égypte et se présentèrent devant Joseph. 
${}^{16}Apercevant Benjamin avec eux, Joseph dit à son intendant : « Fais entrer ces hommes dans la maison. Tue une bête et apprête-la, car ces hommes mangeront avec moi ce midi. » 
${}^{17}L’intendant exécuta les ordres de Joseph et fit entrer les hommes dans la maison. 
${}^{18}Mais ceux-ci eurent peur car on les faisait entrer dans la maison de Joseph. Ils se disaient : « C’est à cause de l’argent remis dans nos besaces la fois passée, c’est pour cela qu’on nous amène ici. Ils vont se ruer sur nous, tomber sur nous, nous garder comme esclaves, avec nos ânes. »
${}^{19}Ils s’approchèrent de l’intendant de Joseph et lui parlèrent à l’entrée de la maison, 
${}^{20}en disant : « Pardon, mon seigneur. Nous sommes déjà descendus, une première fois, pour acheter de la nourriture. 
${}^{21}Or, quand nous sommes arrivés à l’étape et avons ouvert nos besaces, chacun a retrouvé son argent sur le dessus de sa besace. La somme exacte, nous la rapportons avec nous. 
${}^{22}Et nous sommes descendus avec une autre somme d’argent pour acheter de la nourriture. Nous ne savons pas qui avait remis notre argent dans nos besaces. » 
${}^{23}L’intendant répondit : « Soyez en paix ! N’ayez pas peur ! C’est votre Dieu, le Dieu de votre père, qui a caché un trésor dans vos besaces. Votre argent m’était bien parvenu. » Et il leur relâcha Siméon.
${}^{24}L’homme les fit entrer dans la maison de Joseph. Il leur apporta de l’eau et ils se lavèrent les pieds. Puis il donna du fourrage à leurs ânes. 
${}^{25}Ils préparèrent le présent en attendant l’arrivée de Joseph pour midi, car ils avaient appris qu’ils prendraient là leur repas. 
${}^{26}Joseph entra dans la maison, et ils lui offrirent le présent qu’ils tenaient entre les mains. Puis ils se prosternèrent devant lui jusqu’à terre. 
${}^{27}Il leur demanda comment ils allaient et ajouta : « Comment va votre vieux père dont vous m’aviez parlé ? Est-il toujours en vie ? » 
${}^{28}Ils répondirent : « Ton serviteur, notre père, se porte bien. Il est toujours en vie. » Puis ils s’inclinèrent et se prosternèrent. 
${}^{29}Joseph leva les yeux et aperçut son frère Benjamin, le fils de sa mère. Il dit : « Est-ce lui, votre plus jeune frère, celui dont vous m’aviez parlé ? » Puis il ajouta : « Dieu te prenne en grâce, mon fils. » 
${}^{30}Ému jusqu’aux entrailles à la vue de son frère, Joseph chercha en toute hâte un endroit pour pleurer. Il entra dans sa chambre et là, il pleura. 
${}^{31}Il se lava le visage et ressortit. Il se domina et dit : « Servez le repas. » 
${}^{32}On le servit à part ; on les servit à part, eux aussi, et on servit à part les Égyptiens qui mangeaient chez lui, car les Égyptiens ne peuvent prendre un repas avec les Hébreux : ce serait une abomination pour les Égyptiens ! 
${}^{33}Les Hébreux se placèrent devant Joseph par rang d’âge, depuis l’aîné selon son droit d’aînesse jusqu’au plus jeune ; et ils se regardaient l’un l’autre avec étonnement. 
${}^{34}Puis Joseph leur fit servir des portions de ce qui était devant lui. Et la portion de Benjamin était cinq fois plus copieuse que celle de tous les autres. Ils burent et s’enivrèrent avec lui.
      
         
      \bchapter{}
      \begin{verse}
${}^{1}Joseph donna ses ordres à son intendant : « Remplis de nourriture les besaces de ces hommes, dit-il, autant qu’ils pourront en porter, et remets l’argent de chacun sur le dessus de la besace. 
${}^{2}Puis, ma coupe, la coupe d’argent, tu la mettras sur le dessus de la besace du plus jeune, avec l’argent de son blé. » Il fit ce que Joseph lui avait dit.
${}^{3}Aux premières lueurs du matin, on renvoya ces hommes avec leurs ânes. 
${}^{4}Comme ils étaient sortis de la ville mais n’étaient pas encore loin, Joseph dit à son intendant : « Debout ! Poursuis ces hommes, rattrape-les, et tu leur diras : “Pourquoi avez-vous rendu le mal pour le bien ? 
${}^{5}N’y a-t-il pas ici cet objet dont mon maître se sert pour boire et pratiquer la divination ? C’est très mal, ce que vous avez fait.” »
${}^{6}L’intendant les rattrapa et leur répéta ces paroles. 
${}^{7}Ils répondirent : « Pourquoi mon seigneur parle-t-il ainsi ? Loin de tes serviteurs d’avoir agi de cette façon ! 
${}^{8}L’argent que nous avions trouvé sur le dessus de nos besaces, nous te l’avons rapporté du pays de Canaan. Comment donc aurions-nous pu voler de l’or ou de l’argent dans la maison de ton maître ? 
${}^{9}Celui de tes serviteurs que l’on trouvera en possession de cet objet, il mourra, et nous-mêmes, nous deviendrons esclaves de mon seigneur. » 
${}^{10}Il répondit : « Eh bien, qu’il en soit comme vous avez dit ! Celui que l’on trouvera en possession de l’objet deviendra mon esclave, et vous, vous serez quittes ! » 
${}^{11}Vite, chacun déposa sa besace à terre et l’ouvrit. 
${}^{12}L’intendant se mit à fouiller, en commençant par l’aîné et en terminant par le plus jeune. Et l’on trouva la coupe dans la besace de Benjamin. 
${}^{13}Ils déchirèrent leurs vêtements, chacun rechargea son âne et ils retournèrent en ville. 
${}^{14}Juda et ses frères arrivèrent à la maison de Joseph. Il y était encore. Ils se jetèrent devant lui, face contre terre. 
${}^{15}Joseph leur dit : « Qu’avez-vous donc fait ! Ne saviez-vous pas qu’un homme comme moi pratique la divination ? » 
${}^{16}Juda répondit : « Qu’allons-nous pouvoir dire à mon seigneur ? Quels mots prononcer ? Quelles justifications avancer ? Dieu a trouvé que tes serviteurs étaient en faute. Nous serons donc les esclaves de mon seigneur, nous et celui qui a été trouvé en possession de la coupe. » 
${}^{17}Joseph répliqua : « Loin de moi d’agir ainsi, c’est l’homme trouvé en possession de la coupe qui sera mon esclave. Vous autres, retournez en paix chez votre père ! »
${}^{18}Alors Juda s’approcha de lui et dit : « De grâce, mon seigneur, permets que ton serviteur t’adresse une parole\\sans que la colère de mon seigneur s’enflamme contre ton serviteur, car tu es aussi grand que Pharaon ! 
${}^{19}Mon seigneur avait demandé à ses serviteurs : “Avez-vous encore votre père ou un autre frère ?” 
${}^{20}Et nous avons répondu à mon seigneur : “Nous avons encore notre vieux père et un petit frère\\, l’enfant qu’il a eu dans sa vieillesse ; celui-ci avait un frère qui est mort, il reste donc le seul enfant de sa mère, et notre père l’aime !” 
${}^{21}Alors tu as dit à tes serviteurs : “Amenez-le-moi\\ : je veux m’occuper de lui\\.” 
${}^{22}Nous avons dit à mon seigneur : “Le garçon ne peut pas quitter son père ; s’il quittait son père, celui-ci mourrait.” 
${}^{23}Alors tu as dit à tes serviteurs : “Si votre plus jeune frère ne revient pas avec vous, vous ne serez plus admis en ma présence.” 
${}^{24}Donc, lorsque nous sommes retournés auprès de notre\\père, ton serviteur, nous lui avons rapporté les paroles de mon seigneur. 
${}^{25}Et, lorsque notre père a dit : “Repartez pour nous acheter un peu de nourriture”, 
${}^{26}nous lui avons répondu : “Nous ne pourrons pas repartir si notre plus jeune frère n’est pas avec nous, car nous ne pourrons pas être admis en présence de cet homme si notre plus jeune frère n’est pas avec nous.” 
${}^{27}Alors notre père, ton serviteur, nous a dit : “Vous savez bien que ma femme Rachel\\ne m’a donné que deux fils. 
${}^{28}Le premier a disparu\\. Sûrement, une bête féroce\\l’aura mis en pièces, et je ne l’ai jamais revu. 
${}^{29}Si vous emmenez encore celui-ci loin de moi et qu’il lui arrive malheur, vous ferez descendre misérablement mes cheveux blancs au séjour des morts.” 
${}^{30}Maintenant, si je retourne, sans le garçon, chez mon père, ton serviteur, ils sont tellement attachés l’un à l’autre 
${}^{31}que mon père mourra quand il s’apercevra de son absence ; et c’est dans la douleur que tes serviteurs auront fait descendre les cheveux blancs de leur père au séjour des morts. 
${}^{32}Or, ton serviteur s’est porté garant du garçon auprès de son père, en disant : “Si je ne le ramène pas auprès de toi, j’aurai commis une faute envers toi, mon père, pour toujours !” 
${}^{33}Maintenant donc, que ton serviteur reste à la place du garçon comme esclave de mon seigneur et que le garçon retourne avec ses frères ! 
${}^{34}Comment retournerai-je vers mon père sans que le garçon soit avec moi ? Je ne veux pas voir le malheur atteindre mon père ! »
      
         
      \bchapter{}
      \begin{verse}
${}^{1}Joseph ne put se contenir devant tous les gens de sa suite, et il s’écria : « Faites sortir tout le monde. » Quand il n’y eut plus personne auprès de lui, il se fit reconnaître de ses frères. 
${}^{2} Il pleura si fort\\que les Égyptiens l’entendirent, et même la maison de Pharaon. 
${}^{3} Il dit à ses frères : « Je suis Joseph ! Est-ce que mon père vit encore ? » Mais ses frères étaient incapables de lui répondre, tant ils étaient bouleversés de se trouver en face de lui.
${}^{4}Alors Joseph dit à ses frères : « Approchez-vous de moi ». Ils s’approchèrent, et il leur dit : « Je suis Joseph, votre frère, que vous avez vendu pour qu’il soit emmené\\en Égypte. 
${}^{5}Mais maintenant ne vous affligez pas, et ne soyez pas tourmentés\\de m’avoir vendu, car c’est pour vous conserver la vie que Dieu m’a envoyé ici avant vous. 
${}^{6}Voici déjà deux ans que la famine sévit dans le pays, et cinq années passeront encore sans labour ni moisson. 
${}^{7}Dieu m’a envoyé ici avant vous, afin de vous assurer un reste dans le pays et ainsi vous maintenir en vie en prévision d’une grande délivrance. 
${}^{8}Non, ce n’est pas vous qui m’avez envoyé ici, mais Dieu. C’est lui qui m’a élevé au rang de Père de Pharaon, maître de toute sa maison, gouverneur de tout le pays d’Égypte. 
${}^{9}Dépêchez-vous de retourner chez mon père pour lui dire : Ainsi parle ton fils Joseph : “Dieu m’a élevé au rang de maître de toute l’Égypte. Rejoins-moi. Ne t’arrête pas ! 
${}^{10}Tu habiteras le pays de Goshèn et tu seras près de moi, toi, tes fils, les fils de tes fils, ton petit et ton gros bétail, tout ce qui t’appartient. 
${}^{11}Là, je veillerai à ta subsistance – car il y aura encore cinq années de famine –, afin que tu ne manques de rien, toi, ta famille et tout ce qui t’appartient.” 
${}^{12}Vous le voyez de vos yeux, et mon frère Benjamin aussi le voit : c’est bien ma bouche qui vous parle. 
${}^{13}Vous rapporterez à mon père tout le prestige que j’ai en Égypte et tout ce que vous avez vu. Dépêchez-vous d’amener mon père ici. »
${}^{14}Il se jeta au cou de son frère Benjamin et pleura, et Benjamin pleura dans ses bras. 
${}^{15}Il embrassa tous ses frères, en les couvrant de larmes. Puis tous ses frères se mirent à converser avec lui.
${}^{16}La rumeur se répandit dans la maison de Pharaon. On disait : « Les frères de Joseph sont arrivés ! » Pharaon et ses serviteurs virent cela d’un bon œil. 
${}^{17}Pharaon dit à Joseph : « Dis à tes frères : “Faites ceci : chargez vos bêtes et partez ; rentrez au pays de Canaan ! 
${}^{18}Puis, prenez votre père et vos familles, et revenez chez moi pour que je vous offre ce qu’il y a de mieux au pays d’Égypte et que vous mangiez les meilleurs produits du pays. 
${}^{19}Quant à toi, transmets-leur cet ordre : Faites ceci : au pays d’Égypte, procurez-vous des chariots pour vos jeunes enfants et vos femmes ; amenez votre père et revenez ! 
${}^{20}Ne jetez pas un regard désolé sur vos affaires, car ce qu’il y a de mieux dans tout le pays d’Égypte vous appartiendra.” »
${}^{21}Ainsi firent les fils d’Israël. Sur l’ordre de Pharaon, Joseph leur donna des chariots et des provisions de route. 
${}^{22}Il distribua à chacun des vêtements de rechange, mais à Benjamin, il donna trois cents pièces d’argent et cinq vêtements de rechange. 
${}^{23}Il envoya également à son père dix ânes chargés de ce qu’il y a de mieux en Égypte, et dix ânesses chargées de froment, de pain, de vivres, pour le voyage de son père. 
${}^{24}Puis il renvoya ses frères qui se mirent en route. Joseph leur avait dit : « Ne vous disputez pas en chemin ! »
${}^{25}Ils remontèrent donc d’Égypte et arrivèrent au pays de Canaan chez leur père Jacob. 
${}^{26}Ils lui annoncèrent la nouvelle : « Joseph est encore vivant, et c’est lui qui est gouverneur de tout le pays d’Égypte ! » Mais le cœur de Jacob demeurait insensible, car il ne les croyait pas. 
${}^{27}Alors ils lui répétèrent toutes les paroles que Joseph leur avait dites, et Jacob vit les chariots que Joseph avait envoyés pour le transporter. Alors l’esprit de leur père Jacob reprit vie. 
${}^{28}Israël s’écria : « Il ne m’en faut pas plus, mon fils Joseph est encore vivant ! Je veux partir et le revoir avant de mourir. »
      
         
      \bchapter{}
      \begin{verse}
${}^{1}Israël, c’est-à-dire Jacob\\, se mit en route avec tout ce qui lui appartenait. Arrivé à Bershéba, il offrit des sacrifices au Dieu de son père Isaac, 
${}^{2} et Dieu parla à Israël dans une vision nocturne. Il dit : « Jacob ! Jacob ! » Il répondit : « Me voici. » 
${}^{3} Dieu reprit : « Je suis Dieu\\, le Dieu de ton père. Ne crains pas de descendre en Égypte, car là-bas je ferai de toi une grande nation. 
${}^{4} Moi, je descendrai avec toi en Égypte. Moi-même, je t’en ferai aussi remonter, et Joseph te fermera les yeux de sa propre main. »
${}^{5}Jacob partit de\\Bershéba. Ses fils l’installèrent, avec leurs jeunes enfants et leurs femmes, sur les chariots que Pharaon avait envoyés pour le transporter. 
${}^{6} Ils prirent aussi leurs troupeaux et les biens qu’ils avaient acquis au pays de Canaan. Jacob arriva en Égypte avec toute sa descendance. 
${}^{7} Ainsi donc, ses fils et ses petits-fils, ses filles et ses petites-filles, bref toute sa descendance, il les emmena avec lui en Égypte.
${}^{8}Voici les noms des fils d’Israël venus en Égypte : Jacob et ses fils. 
${}^{9}Premier-né de Jacob : Roubène. Fils de Roubène : Hanok, Pallou, Hesrone, Karmi.
${}^{10}Fils de Siméon : Yemouël, Yamine, Ohad, Yakine, Sohar, Saül, le fils de la Cananéenne.
${}^{11}Fils de Lévi : Guershone, Qehath et Merari.
${}^{12}Fils de Juda : Er, Onane, Shéla, Pérès, Zèrah. Er et Onane étaient morts au pays de Canaan. Les fils de Pérès furent Hesrone et Hamoul.
${}^{13}Fils d’Issakar : Tola, Poua, Job, Shimrone.
${}^{14}Fils de Zabulon : Sèred, Élone, Yahleël.
${}^{15}Ce furent les fils que Léa donna à Jacob en Paddane-Aram. Elle lui donna aussi sa fille Dina. Ses fils et ses filles comptaient au total trente-trois personnes.
${}^{16}Fils de Gad : Sifeyone et Haggui, Shouni et Esbone, Éri, Arodi et Aréli.
${}^{17}Fils d’Asher : Yimna, Yishva, Yishvi, Beria, et leur sœur Sérah. Fils de Beria : Hèber et Malkiël.
${}^{18}Ce furent les fils que Zilpa, la servante donnée par Laban à sa fille Léa, enfanta à Jacob : seize personnes.
${}^{19}Fils de Rachel, femme de Jacob : Joseph et Benjamin.
${}^{20}Au pays d’Égypte, il naquit à Joseph deux fils, Manassé et Éphraïm, que lui avait enfantés Asnath, fille de Poti-Phéra, prêtre de One.
${}^{21}Fils de Benjamin : Bèla, Bèker et Ashbel, Guéra et Naamane, Éhi et Rosh, Mouppim, Houppim et Arde.
${}^{22}Ce furent les fils de Rachel qu’elle enfanta à Jacob. Au total : quatorze personnes.
${}^{23}Fils de Dane : Houshim.
${}^{24}Fils de Nephtali : Yahçeël, Gouni, Yécèr, Shillem.
${}^{25}Ce furent les fils que Bilha, la servante donnée par Laban à sa fille Rachel, enfanta à Jacob. Au total : sept personnes.
${}^{26}Total des personnes appartenant à Jacob et issues de lui, qui vinrent en Égypte, sans compter les femmes de ses fils : soixante-six en tout.
${}^{27}Fils de Joseph qui lui furent enfantés en Égypte : deux personnes. Le total des personnes de la maison de Jacob qui vinrent en Égypte fut de soixante-dix.
${}^{28}Jacob avait envoyé Juda en avant vers Joseph, pour préparer son arrivée\\dans le pays de Goshèn. Quand ils furent arrivés dans le pays de Goshèn, 
${}^{29} Joseph fit atteler son char et monta à la rencontre de son père Israël\\. Dès qu’il le vit, il se jeta à son cou et pleura longuement dans ses bras. 
${}^{30} Israël dit à Joseph : « Maintenant que j’ai revu ton visage, je peux mourir, puisque tu es encore vivant ! »
${}^{31}Joseph dit à ses frères et à la famille de son père : « Je vais monter prévenir Pharaon. Je lui dirai : “Mes frères et la famille de mon père, qui étaient au pays de Canaan, sont venus me rejoindre. 
${}^{32}Ces hommes sont des bergers ! Ils élèvent des troupeaux et ont amené leur petit et leur gros bétail, ainsi que tout ce qui leur appartient.” 
${}^{33}Donc, quand Pharaon vous convoquera pour vous demander quel est votre métier, 
${}^{34}vous répondrez : “Tes serviteurs élèvent des troupeaux, depuis leur jeunesse jusqu’à ce jour. Nous faisons ce que faisaient nos pères.” Ainsi vous pourrez demeurer au pays de Goshèn. Tout berger, en effet, est une abomination pour les Égyptiens. »
      
         
      \bchapter{}
      \begin{verse}
${}^{1}Joseph alla donc prévenir Pharaon. Il lui dit : « Mon père et mes frères sont arrivés du pays de Canaan avec leur petit et leur gros bétail, ainsi que tout ce qui leur appartient ; les voici au pays de Goshèn. » 
${}^{2}Puis, parmi ses frères, il en choisit cinq, qu’il présenta à Pharaon. 
${}^{3}Pharaon leur demanda : « Quel est votre métier ? » Ils lui répondirent : « Tes serviteurs sont des bergers. Nous le sommes comme l’étaient nos pères. » 
${}^{4}Et ils ajoutèrent : « Nous sommes venus séjourner comme des immigrés dans le pays, car il n’y a plus de pâturage pour le petit bétail de tes serviteurs : la famine pèse sur le pays de Canaan. Permets que tes serviteurs habitent maintenant au pays de Goshèn. »
${}^{5}Pharaon s’adressa à Joseph et lui dit : « Ton père et tes frères sont venus te rejoindre. 
${}^{6}Le pays d’Égypte est à ta disposition : installe ton père et tes frères au meilleur endroit du pays. Qu’ils habitent au pays de Goshèn, et si tu connais parmi eux des hommes de valeur, nomme-les chefs des troupeaux qui m’appartiennent. »
${}^{7}Alors, Joseph fit venir son père Jacob et le présenta à Pharaon. Jacob salua Pharaon 
${}^{8}qui lui demanda : « Quel âge as-tu ? » 
${}^{9}Jacob lui répondit : « Il y a cent trente ans que je vis en immigré. Ma vie a été courte et malheureuse. Je n’ai pas atteint l’âge de mes pères, au temps où ils vivaient en immigrés. » 
${}^{10}Puis Jacob salua Pharaon et sortit de chez lui.
${}^{11}Joseph installa donc son père et ses frères, il leur donna une propriété au pays d’Égypte, au meilleur endroit du pays, sur la terre de Ramsès, comme l’avait ordonné Pharaon. 
${}^{12}Joseph pourvut aux besoins de son père, de ses frères et de toute la maison de son père, en leur procurant du pain selon le nombre de jeunes enfants.
${}^{13}Or, il n’y avait plus de nourriture, nulle part dans le pays, tant la famine pesait lourdement. Le pays d’Égypte et le pays de Canaan étaient épuisés par la famine. 
${}^{14}Joseph ramassa tout l’argent qui se trouvait au pays d’Égypte et au pays de Canaan, en échange du grain qu’ils achetaient. Ainsi Joseph fit rentrer l’argent dans la maison de Pharaon. 
${}^{15}Quand il n’y eut plus d’argent au pays d’Égypte et au pays de Canaan, tous les Égyptiens vinrent trouver Joseph pour lui dire : « Donne-nous du pain. Pourquoi devrions-nous mourir devant toi, faute d’argent ? » 
${}^{16}Joseph répondit : « Livrez vos troupeaux ; et, en échange de vos troupeaux, je vous donnerai du pain, si vous n’avez plus d’argent. » 
${}^{17}Ils amenèrent leurs troupeaux à Joseph qui leur donna du pain en échange des chevaux, des troupeaux de petit et de gros bétail, et des ânes. En échange de tous leurs troupeaux, il leur procura du pain, cette année-là.
${}^{18}Cette année s’acheva et, l’année suivante, ils vinrent le trouver pour lui dire : « Nous ne le cacherons pas à mon seigneur : puisqu’il n’y a plus d’argent et que les troupeaux de bêtes appartiennent à mon seigneur, il ne reste à la disposition de mon seigneur que nos corps et nos terres. 
${}^{19}Pourquoi devrions-nous mourir sous tes yeux, nous et nos terres avec nous ? Achète-nous donc, nous et nos terres, en échange de pain ; nous serons, nous et nos terres, esclaves de Pharaon. Donne-nous des semences pour que nous vivions, que nous ne mourrions pas et que la terre ne soit pas désolée ! »
${}^{20}Joseph acheta toute la terre de l’Égypte pour Pharaon. En effet, chaque Égyptien vendait son champ, tant la famine les accablait. Et tout le pays appartint à Pharaon. 
${}^{21}Quant au peuple, Joseph le réduisit en esclavage, d’un bout à l’autre du territoire de l’Égypte. 
${}^{22}C’est uniquement la terre appartenant aux prêtres qu’il ne put acheter, car il y avait un décret de Pharaon en faveur des prêtres : ceux-ci vivaient de ce que leur attribuait le décret de Pharaon. Aussi, ils n’eurent pas à vendre la terre qui leur appartenait.
${}^{23}Joseph dit au peuple : « Voici qu’aujourd’hui je vous ai achetés pour Pharaon, vous et votre terre. Et voici de la semence pour vous : ensemencez la terre ! 
${}^{24}Au moment de la récolte, vous en donnerez un cinquième à Pharaon. Les quatre autres seront pour vous : pour ensemencer le champ, pour vous nourrir, pour nourrir les gens de votre maison et pour nourrir vos jeunes enfants. » 
${}^{25}Ils répondirent : « Nous te devons la vie. Puissions-nous trouver grâce aux yeux de mon seigneur et être les esclaves de Pharaon ! » 
${}^{26}En conséquence, Joseph prit un décret concernant la terre de l’Égypte, décret toujours en vigueur aujourd’hui : le cinquième des récoltes appartient à Pharaon ; seule la terre des prêtres ne lui appartint pas.
${}^{27}Les fils d’Israël habitaient en Égypte, au pays de Goshèn. Ils y furent propriétaires, ils étaient féconds et se multiplièrent énormément. 
${}^{28}Jacob vécut dix-sept ans au pays d’Égypte. La durée de sa vie fut de cent quarante-sept ans. 
${}^{29}Quand approcha le jour de sa mort, Israël (c’est-à-dire : Jacob) appela son fils Joseph et lui dit : « Si j’ai trouvé grâce à tes yeux, jure-moi de ne pas m’enterrer en Égypte ; ainsi tu me montreras ta fidélité et ta loyauté. 
${}^{30}Je reposerai avec mes pères : tu m’emporteras hors d’Égypte et tu m’enterreras dans leur tombeau. » Joseph répondit : « Oui, je ferai comme tu as dit. » 
${}^{31}Et Jacob reprit : « Prête-moi serment. » Joseph lui en fit le serment, et Israël se prosterna au chevet de son lit.
      
         
      \bchapter{}
      \begin{verse}
${}^{1}Or, après ces événements, on dit à Joseph : « Voici que ton père est malade ! » Il prit avec lui ses deux fils, Manassé et Éphraïm. 
${}^{2}On l’annonça à Jacob en disant : « Voici venir vers toi ton fils Joseph ! » Israël rassembla ses forces et s’assit sur le lit.
${}^{3}Puis Jacob dit à Joseph : « Le Dieu-Puissant m’est apparu à Louz au pays de Canaan et il m’a béni. 
${}^{4}Il m’a dit : “Voici que je te rendrai fécond et multiplierai ta descendance, je ferai de toi une assemblée de peuples et je donnerai ce pays à ta descendance, en propriété perpétuelle.” 
${}^{5}Et maintenant, tes deux fils – ceux qui te sont nés au pays d’Égypte avant que je t’y rejoigne – ils sont à moi. Éphraïm et Manassé sont à moi comme Roubène et Siméon. 
${}^{6}Mais les enfants que tu auras engendrés après eux seront à toi. C’est au nom de leurs frères qu’on les convoquera pour leur part d’héritage. 
${}^{7}Quant à moi, alors que j’arrivais de Paddane, Rachel est morte dans mes bras, au pays de Canaan, sur la route, à une certaine distance d’Éphrata. C’est là que je l’ai enterrée, sur la route d’Éphrata, c’est-à-dire Bethléem. »
${}^{8}À la vue des fils de Joseph, Israël dit : « Qui sont ceux-là ? » 
${}^{9}Joseph répondit à son père : « Ce sont les fils que Dieu m’a donnés ici. » Jacob dit : « Amène-les moi, je vais les bénir. »
${}^{10}Les yeux affaiblis par l’âge, Israël n’y voyait plus bien. Joseph fit approcher ses fils. Israël les embrassa et les étreignit. 
${}^{11}Puis il dit à Joseph : « Je ne pensais plus revoir ton visage, et voici que Dieu m’a fait voir même ta descendance ! » 
${}^{12}Joseph retira ses fils des genoux de son père et se prosterna face contre terre. 
${}^{13}Joseph prit ses deux fils, Éphraïm à sa droite, soit à la gauche d’Israël, et Manassé à sa gauche, soit à la droite d’Israël. Il les fit s’approcher de celui-ci. 
${}^{14}Israël posa sa main droite sur la tête d’Éphraïm qui était le cadet, et sa main gauche sur la tête de Manassé : il avait croisé ses mains ; or Manassé était l’aîné. 
${}^{15}Ensuite, il bénit Joseph en disant :
        \\« Que le Dieu en présence de qui ont marché
        mes pères Abraham et Isaac,
        \\que le Dieu qui fut mon berger depuis que j’existe
        et jusqu’à ce jour,
${}^{16}l’Ange qui m’a libéré de tout mal,
        qu’il bénisse ces garçons !
        \\Qu’en eux survive mon nom,
        et le nom de mes pères, Abraham et Isaac !
        \\Qu’ils surabondent dans le pays ! »
       
${}^{17}Or Joseph vit que son père avait posé sa main droite sur la tête d’Éphraïm. À ses yeux, cela ne convenait pas. Il saisit donc la main de son père pour la déplacer de la tête d’Éphraïm et la mettre sur la tête de Manassé. 
${}^{18}Il dit à son père : « Pas ainsi, mon père, c’est celui-ci l’aîné. Mets ta main droite sur sa tête ! » 
${}^{19}Mais son père refusa : « Je sais, mon fils, je sais : lui aussi deviendra un peuple, lui aussi grandira ; toutefois, son frère cadet sera plus grand que lui, il aura pour descendance une foule de nations. »
${}^{20}Il les bénit, ce jour-là, en disant :
        \\« Par toi, Israël prononcera cette bénédiction :
        \\Que Dieu te rende comme Éphraïm et comme Manassé ! »
      Ainsi, il plaça Éphraïm avant Manassé.
       
${}^{21}Alors Israël dit à Joseph : « Voici que je vais mourir, mais Dieu sera avec vous, il vous fera retourner au pays de vos pères. 
${}^{22}Et moi, je te donne une colline de plus qu’à tes frères : Sichem, que j’ai conquise des mains des Amorites par mon arc et mon épée. »
      
         
      \bchapter{}
      \begin{verse}
${}^{1}Jacob appela ses fils et dit :
        « Assemblez-vous ! Je veux vous dévoiler
        \\ce qui vous arrivera dans les temps à venir.
         
        ${}^{2}Rassemblez-vous, écoutez, fils de Jacob,
        écoutez Israël, votre père.
         
${}^{3}Toi, Roubène, mon premier-né,
        ma force, les prémices de ma virilité,
        débordant de fierté, débordant d’énergie,
${}^{4}torrent impétueux, ne déborde plus,
        toi qui es monté sur le lit de ton père
        et, en y montant, tu l’as profané.
         
${}^{5}Siméon et Lévi sont bien frères :
        leurs couteaux sont des instruments de violence !
${}^{6}Que je ne participe pas à leur conseil,
        que je ne rejoigne pas leur assemblée !
        \\Car, dans leur colère, ils ont massacré des hommes,
        dans leur frénésie, ils ont mutilé des taureaux.
${}^{7}Maudite soit leur colère, car elle est violente,
        et leur fureur, car elle est dure !
        \\Je les démembrerai en Jacob,
        je les disperserai en Israël.
         
        ${}^{8}Juda, à toi, tes frères rendront hommage,
        ta main fera plier la nuque de tes ennemis
        et les fils de ton père se prosterneront devant toi.
        ${}^{9}Juda est un jeune lion.
        Tu remontes du carnage, mon fils.
        \\Il s’est accroupi, il s’est couché comme un lion ;
        ce fauve\\, qui le fera lever ?
        ${}^{10}Le sceptre royal\\n’échappera pas à Juda,
        ni le bâton de commandement, à sa descendance,
        \\jusqu’à ce que vienne celui à qui le pouvoir appartient,
        à qui les peuples obéiront.
${}^{11}Il attache à la vigne son ânon,
        au cep, le petit de son ânesse.
        \\Il foule dans le vin son vêtement,
        dans le sang des raisins, son manteau.
${}^{12}Ses yeux brillent plus que le vin,
        ses dents sont plus blanches que le lait.
         
${}^{13}Zabulon habitera au bord de la mer.
        \\Il voyagera à bord des vaisseaux
        et ses confins toucheront à Sidon.
${}^{14}Issakar est un âne robuste,
        accroupi entre deux enclos.
${}^{15}Il constate que le repos est agréable
        et le pays, plaisant.
        \\Il tend l’échine au fardeau :
        il est bon pour la corvée d’esclave.
${}^{16}Dane jugera son peuple
        comme l’une des tribus d’Israël.
${}^{17}Que Dane soit un serpent sur la route,
        une vipère sur le sentier,
        \\qui mord le cheval au talon,
        et son cavalier tombe à la renverse !
         
${}^{18}En ton salut, j’espère, Seigneur !
         
${}^{19}Gad, des attaquants l’attaquent,
        et lui, il porte l’attaque au talon.
         
${}^{20}Asher : son pain est savoureux,
        il fournit des mets de roi.
         
${}^{21}Nephtali est une biche en liberté
        \\qui donne de beaux petits faons.
         
${}^{22}C’est une plante fertile, que Joseph,
        une plante fertile près d’une source.
        Ses branches franchissent le mur.
${}^{23}Ils l’ont exaspéré, ils l’ont pris pour cible,
        ils l’ont persécuté, ceux qui lancent des flèches.
${}^{24}Mais son arc est demeuré ferme ;
        ses bras et ses mains ont gardé leur agilité
        \\grâce à Celui qui est Force de Jacob,
        grâce au nom du Berger, la Pierre d’Israël,
${}^{25}grâce au Dieu de ton père – qu’il te vienne en aide !
        grâce au Puissant – qu’il te bénisse !
        \\D’en haut, bénédictions des cieux !
        Bénédictions de l’abîme tout en bas !
        Bénédictions des mamelles et du sein !
${}^{26}Les bénédictions de ton père ont surpassé
        les bénédictions des montagnes antiques,
        le désir des collines éternelles :
        \\qu’elles viennent sur la tête de Joseph,
        sur la chevelure du consacré parmi ses frères.
         
${}^{27}Benjamin est un loup qui déchire ;
        le matin, il dévore la proie ;
        le soir, il partage le butin. »
       
${}^{28}Ce sont là toutes les tribus d’Israël, les douze tribus ! Et voilà ce que leur a dit leur père, en les bénissant. Il les a bénies en donnant à chacun de ses fils sa bénédiction.
${}^{29}Jacob donna cet ordre à ses fils\\ : « Je vais être réuni aux miens\\. Enterrez-moi auprès de mes pères, dans la caverne qui est dans le champ d’Éphrone le Hittite, 
${}^{30} dans la caverne du champ de Macpéla, en face de Mambré, au pays de Canaan, le champ qu’Abraham a acheté à Éphrone le Hittite comme propriété funéraire\\. 
${}^{31} C’est là que furent enterrés Abraham et son épouse Sara ; c’est là que furent enterrés Isaac et son épouse Rébecca ; c’est là que j’ai enterré Léa. 
${}^{32} C’est le champ qui fut acheté aux Hittites, avec la caverne qui s’y trouve. »
${}^{33}Lorsque Jacob eut achevé de donner ses instructions\\à ses fils, il s’allongea sur son lit\\, il expira et fut réuni aux siens\\.
      
         
      \bchapter{}
      \begin{verse}
${}^{1}Joseph se pencha sur le visage de son père, le couvrit de larmes et l’embrassa. 
${}^{2}Puis il ordonna aux médecins qui étaient à son service d’embaumer son père, et ceux-ci embaumèrent Israël. 
${}^{3}Cela dura quarante jours, le temps qu’il faut pour l’embaumement. Ensuite, les Égyptiens le pleurèrent soixante-dix jours. 
${}^{4}Quand fut écoulé le temps des pleurs, Joseph parla ainsi aux gens de la maison de Pharaon : « Si j’ai trouvé grâce à vos yeux, allez donc rapporter à Pharaon 
${}^{5}que mon père m’a fait prêter serment, en me disant : “Voici que je vais mourir. C’est dans le tombeau que je me suis creusé au pays de Canaan, que tu m’enterreras.” Maintenant, laisse-moi donc monter en Canaan et enterrer mon père. Ensuite, je reviendrai. » 
${}^{6}Pharaon répondit : « Monte donc enterrer ton père, comme il t’a fait prêter serment. »
${}^{7}Et Joseph monta enterrer son père. Tous les serviteurs de Pharaon, les anciens de sa maison et tous les anciens du pays d’Égypte montèrent avec lui, 
${}^{8}ainsi que tous les gens de la maison de Joseph, ses frères et les gens de la maison de son père. Ils ne laissèrent au pays de Goshèn que leurs jeunes enfants, leur petit et leur gros bétail. 
${}^{9}Même les chars et les cavaliers montèrent avec lui. C’était une caravane imposante. 
${}^{10}Ils arrivèrent à l’Aire-de-l’Épine, au-delà du Jourdain, et là, ils célébrèrent des funérailles solennelles et imposantes. Joseph observa pour son père un deuil de sept jours.
${}^{11}À la vue du deuil à l’Aire-de-l’Épine, les habitants du pays, les Cananéens, s’écrièrent : « C’est un deuil important pour l’Égypte ! » C’est pourquoi on appela cet endroit au-delà du Jourdain « Deuil-de-l’Égypte ».
${}^{12}Les fils de Jacob agirent pour lui comme il l’avait ordonné. 
${}^{13}Ils le transportèrent au pays de Canaan et l’enterrèrent dans la caverne du champ de Macpéla, en face de Mambré, ce champ qu’Abraham avait acheté à Éphrone le Hittite, comme propriété funéraire. 
${}^{14}Après avoir enterré son père, Joseph retourna en Égypte avec ses frères et tous ceux qui étaient montés avec lui pour enterrer son père.
${}^{15}Voyant que leur père était mort, les frères de Joseph se dirent : « Si jamais Joseph nous prenait en haine, s’il allait nous rendre tout le mal que nous lui avons fait… » 
${}^{16} Ils firent dire\\à Joseph : « Avant de mourir, ton père a exprimé cette volonté\\ : 
${}^{17} “Vous demanderez ceci à Joseph : De grâce, pardonne à tes frères leur crime et leur péché. Oui, ils t’ont fait du mal, mais toi, maintenant, pardonne donc le crime des serviteurs du Dieu de ton père !” » En entendant ce message, Joseph pleura.
${}^{18}Puis ses frères vinrent eux-mêmes se jeter à ses pieds\\et lui dire : « Voici que nous sommes tes esclaves. » 
${}^{19} Mais Joseph leur répondit : « Soyez sans crainte ! Vais-je prendre la place de Dieu ? 
${}^{20} Vous aviez voulu me faire du mal, Dieu a voulu le changer en bien, afin d’accomplir ce qui se réalise aujourd’hui : préserver la vie d’un peuple nombreux. 
${}^{21} Soyez donc sans crainte : moi, je prendrai soin de vous et de vos jeunes enfants. » Il les réconforta par des paroles qui leur allaient au cœur.
${}^{22}Joseph demeura en Égypte avec la famille de son père, et il vécut cent dix ans. 
${}^{23} Il vit les petits-enfants de son fils Éphraïm ; quant aux enfants de Makir, fils de Manassé son autre fils\\, il les reçut sur ses genoux à leur naissance.
${}^{24}Joseph dit à ses frères : « Je vais mourir. Dieu vous visitera et vous fera remonter de ce pays dans le pays qu’il a fait serment de donner\\à Abraham, Isaac et Jacob. » 
${}^{25} Joseph fit prêter serment aux fils d’Israël, en disant : « Quand Dieu vous visitera, vous ferez monter d’ici mes ossements. »
${}^{26}Joseph mourut à cent dix ans. On l’embauma et on le mit dans un cercueil en Égypte.
